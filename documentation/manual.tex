% User manual for the Eigen Compiler Suite
% Copyright (C) Florian Negele

% This file is part of the Eigen Compiler Suite.

% Permission is granted to copy, distribute and/or modify this document
% under the terms of the GNU Free Documentation License, Version 1.3
% or any later version published by the Free Software Foundation.

% You should have received a copy of the GNU Free Documentation License
% along with the ECS.  If not, see <https://www.gnu.org/licenses/>.

% Generic documentation utilities
% Copyright (C) Florian Negele

% This file is part of the Eigen Compiler Suite.

% Permission is granted to copy, distribute and/or modify this document
% under the terms of the GNU Free Documentation License, Version 1.3
% or any later version published by the Free Software Foundation.

% You should have received a copy of the GNU Free Documentation License
% along with the ECS.  If not, see <https://www.gnu.org/licenses/>.

\providecommand{\cpp}{C\texttt{++}}
\providecommand{\opt}{_\mathit{opt}}
\providecommand{\tool}[1]{\texttt{#1}}
\providecommand{\version}{Version 0.0.40}
\providecommand{\resource}[1]{*++\txt{#1}}
\providecommand{\ecs}{Eigen Compiler Suite}
\providecommand{\changed}[1]{\underline{#1}}
\providecommand{\toolbox}[1]{\converter{#1}}
\providecommand{\file}{}\renewcommand{\file}[1]{\texttt{#1}}
\providecommand{\alignright}{\hfill\linebreak[0]\hspace*{\fill}}
\providecommand{\converter}[1]{*++[F][F*:white][F,:gray]\txt{#1}}
\providecommand{\documentation}{\ifbook chapter\else document\fi}
\providecommand{\Documentation}{\ifbook Chapter\else Document\fi}
\providecommand{\variable}[1]{\resource{\texttt{\small#1}\\variable}}
\providecommand{\documentationref}[2]{\ifbook\ref{#1}\else``\href{#1}{#2}''~\cite{#1}\fi}
\providecommand{\objfile}[1]{\texttt{#1}\index[runtime]{#1 object file@\texttt{#1} object file}}
\providecommand{\libfile}[1]{\texttt{#1}\index[runtime]{#1 library file@\texttt{#1} library file}}
\providecommand{\epigraph}[2]{\ifbook\begin{quote}\flushright\textit{#1}\par--- #2\end{quote}\fi}
\providecommand{\environmentvariable}[1]{\texttt{#1}\index{Environment variables!#1@\texttt{#1}}}
\providecommand{\environment}[1]{\texttt{#1}\index[environment]{#1 environment@\texttt{#1} environment}}
\providecommand{\toolsection}{}\renewcommand{\toolsection}[1]{\subsection{#1}\label{\prefix:#1}\tool{#1}}
\providecommand{\instruction}{}\renewcommand{\instruction}[2]{\noindent\qquad\pdftooltip{\texttt{#1}}{#2}\refstepcounter{instruction}\par}
\providecommand{\flowgraph}{}\renewcommand{\flowgraph}[1]{\par\sffamily\begin{displaymath}\xymatrix@=4ex{#1}\end{displaymath}\normalfont\par}
\providecommand{\instructionset}{}\renewcommand{\instructionset}[4]{\setcounter{instruction}{0}\begin{multicols}{\ifbook#3\else#4\fi}[{\captionof{table}[#2]{#2 (\ref*{#1:instructions}~instructions)}\label{tab:#1set}\vspace{-2ex}}]\footnotesize\raggedcolumns\input{#1.set}\label{#1:instructions}\end{multicols}}

\providecommand{\gpl}{GNU General Public License}
\providecommand{\rse}{ECS Runtime Support Exception}
\providecommand{\fdl}{\href{https://www.gnu.org/licenses/fdl.html}{GNU Free Documentation License}}

\providecommand{\docbegin}{}
\providecommand{\docend}{}
\providecommand{\doclabel}[1]{\hypertarget{#1}}
\providecommand{\doclink}[2]{\hyperlink{#1}{#2}}
\providecommand{\docsection}[3]{\hypertarget{#1}{\subsection{#2}}\label{sec:#1}\index[library]{#2@#3}}
\providecommand{\docsectionstar}[1]{}
\providecommand{\docsubbegin}{\begin{description}}
\providecommand{\docsubend}{\end{description}}
\providecommand{\docsubsection}[3]{\item[\hypertarget{#1}{#2}]\index[library]{#2@#3}}
\providecommand{\docsubsectionstar}[1]{\smallskip}
\providecommand{\docsubsubsection}[3]{\docsubsection{#1}{#2}{#3}}
\providecommand{\docsubsubsectionstar}[1]{}
\providecommand{\docsubsubsubsection}[3]{}
\providecommand{\docsubsubsubsectionstar}[1]{}
\providecommand{\doctable}{}

\providecommand{\debuggingtool}{}\renewcommand{\debuggingtool}{This tool is provided for debugging purposes.
It allows exposing and modifying an internal data structure that is usually not accessible.
}

\providecommand{\interface}{All tools accept command-line arguments which are taken as names of plain text files containing the source code.
If no arguments are provided, the standard input stream is used instead.
Output files are generated in the current working directory and have the same name as the input file being processed whereas the filename extension gets replaced by an appropriate suffix.
\seeinterface
}

\providecommand{\license}{\noindent Copyright \copyright{} Florian Negele\par\medskip\noindent
Permission is granted to copy, distribute and/or modify this document under the terms of the
\fdl{}, Version 1.3 or any later version published by the \href{https://fsf.org/}{Free Software Foundation}.
}

\providecommand{\ecslogosurface}{
\fill[darkgray] (0,0,0) -- (0,0,3) -- (0,3,3) -- (0,3,1) -- (0,4,1) -- (0,4,3) -- (0,5,3) -- (0,5,0) -- (0,2,0) -- (0,2,2) -- (0,1,2) -- (0,1,0) -- cycle;
\fill[gray] (0,5,0) -- (0,5,3) -- (1,5,3) -- (1,5,1) -- (2,5,1) -- (2,5,3) -- (3,5,3) -- (3,5,0) -- cycle;
\fill[lightgray] (0,0,0) -- (0,1,0) -- (2,1,0) -- (2,4,0) -- (1,4,0) -- (1,3,0) -- (2,3,0) -- (2,2,0) -- (0,2,0) -- (0,5,0) -- (3,5,0) -- (3,0,0) -- cycle;
\begin{scope}[line width=0.5]
\begin{scope}[gray]
\draw (0,0,0) -- (0,1,0);
\draw (2,1,0) -- (2,2,0);
\draw (0,1,2) -- (0,2,2);
\draw (0,2,0) -- (0,5,0);
\draw (2,3,0) -- (2,4,0);
\end{scope}
\begin{scope}[lightgray]
\draw (0,1,0) -- (0,1,2);
\draw (0,3,1) -- (0,3,3);
\draw (0,5,0) -- (0,5,3);
\draw (2,5,1) -- (2,5,3);
\end{scope}
\begin{scope}[white]
\draw (0,1,0) -- (2,1,0);
\draw (1,3,0) -- (2,3,0);
\draw (0,5,0) -- (3,5,0);
\end{scope}
\end{scope}
}

\providecommand{\ecslogo}[1]{
\begin{tikzpicture}[scale={(#1)/((sin(45)+cos(45))*3cm)},x={({-cos(45)*1cm},{sin(45)*sin(30)*1cm})},y={({0cm},{(cos(30)*1cm})},z={({sin(45)*1cm},{cos(45)*sin(30)*1cm})}]
\begin{scope}[darkgray,line width=1]
\draw (0,0,0) -- (0,0,3) -- (0,3,3) -- (2,3,3) -- (2,5,3) -- (3,5,3) -- (3,5,0) -- (3,0,0) -- cycle;
\draw (0,3,1) -- (0,4,1) -- (0,4,3) -- (0,5,3) -- (1,5,3) -- (1,5,1) -- (2,5,1);
\draw (1,3,0) -- (1,4,0) -- (2,4,0);
\end{scope}
\fill[darkgray] (2,0,0) -- (2,0,3) -- (2,5,3) -- (2,5,1) -- (2,4,1) -- (2,4,0) -- cycle;
\fill[lightgray] (2,0,2) -- (0,0,2) -- (0,2,2) -- (2,2,2) -- cycle;
\fill[gray] (0,1,0) -- (2,1,0) -- (2,1,2) -- (0,1,2) -- cycle;
\fill[gray] (0,3,1) -- (0,3,3) -- (2,3,3) -- (2,3,0) -- (1,3,0) -- (1,3,1) -- cycle;
\ecslogosurface
\end{tikzpicture}
}

\providecommand{\shadowedecslogo}[3]{
\begin{tikzpicture}[scale={(#1)/((sin(#2)+cos(#2))*3cm)},x={({-cos(#2)*1cm},{sin(#2)*sin(#3)*1cm})},y={({0cm},{(cos(#3)*1cm})},z={({sin(#2)*1cm},{cos(#2)*sin(#3)*1cm})}]
\shade[top color=lightgray!50!white,bottom color=white,middle color=lightgray!50!white] (0,0,0) -- (3,0,0) -- (3,{-0.5-3*sin(#2)*sin(#3)/cos(#3)},0) -- (0,-0.5,0) -- cycle;
\shade[top color=darkgray!50!gray,bottom color=white,middle color=darkgray!50!white] (0,0,0) -- (0,0,3) -- (0,{-0.5-3*cos(#2)*sin(#3)/cos(#3)},3) -- (0,-0.5,0) -- cycle;
\begin{scope}[y={({(cos(#2)+sin(#2))*0.5cm},{(cos(#2)*sin(#3)-sin(#2)*sin(#3))*0.5cm})}]
\useasboundingbox (3,0,0) -- (0,0,0) -- (0,0,3);
\shade[left color=darkgray!80!black,right color=lightgray,middle color=gray] (0,0,0) -- (0,1,0) -- (0,1,0.5) -- (0,2,0) -- (0,5,0) -- (0,5,3) -- (1,5,3) -- (1,4,3) -- (1,4,2.5) -- (1,3,3) -- (2,5,3) -- (3,5,3) -- (3,0,3) -- cycle;
\clip (0,0,0) -- (0,0,3) -- ({-3*sin(#2)/cos(#2)},0,0) -- cycle;
\shade[left color=darkgray,right color=lightgray!50!gray] (0,0,0) -- (0,1,0) -- (0,1,0.5) -- (0,2,0) -- (0,5,0) -- (0,5,3) -- (1,5,3) -- (1,4,3) -- (1,4,2.5) -- (1,3,3) -- (2,5,3) -- (3,5,3) -- (3,0,3) -- cycle;
\end{scope}
\shade[left color=darkgray,right color=darkgray!80!black] (2,0,0) -- (2,0,3) -- (2,5,3) -- (2,5,1) -- (2,4,1) -- (2,4,0) -- cycle;
\shade[left color=darkgray!90!black,right color=gray!80!darkgray] (2,0,2) -- (0,0,2) -- (0,2,2) -- (2,2,2) -- cycle;
\shade[top color=darkgray!90!black,bottom color=gray!80!darkgray] (0,1,0) -- (2,1,0) -- (2,1,2) -- (0,1,2) -- cycle;
\shade[top color=darkgray!90!black,bottom color=gray!80!darkgray] (0,3,1) -- (0,3,3) -- (2,3,3) -- (2,3,0) -- (1,3,0) -- (1,3,1) -- cycle;
\fill[gray] (2,1,0) -- (1.5,1,0.5) -- (0,1,0.5) -- (0,1,0) -- cycle;
\fill[gray] (1,3,2) -- (0.5,3,2) -- (0.5,3,3) -- (1,3,3) -- cycle;
\fill[gray] (2,3,0) -- (1.5,3,0.5) -- (1,3,0.5) -- (1,3,0) -- cycle;
\ecslogosurface
\end{tikzpicture}
}

\providecommand{\cpplogo}[1]{
\begin{tikzpicture}[scale=(#1)/512em]
\fill[gray] (435.2794,398.7159) -- (247.1911,507.3075) .. controls (236.3563,513.5642) and (218.6240,513.5642) .. (207.7892,507.3075) -- (19.7009,398.7159) .. controls (8.8646,392.4606) and (0.0000,377.1043) .. (0.0000,364.5924) -- (0.0000,147.4076) .. controls (0.8430,132.8363) and (8.2856,120.7683) .. (19.7009,113.2842) -- (207.7892,4.6926) .. controls (218.6240,-1.5642) and (236.3564,-1.5642) .. (247.1911,4.6926) -- (435.2794,113.2842) .. controls (447.5273,121.4304) and (454.4987,133.6918) .. (454.9803,147.4076) -- (454.9803,364.5924) .. controls (454.5404,377.7571) and (446.6566,391.0351) .. (435.2794,398.7159) -- cycle(75.8301,255.9993) .. controls (74.9389,404.0881) and (273.2892,469.4783) .. (358.8263,331.8769) -- (293.1917,293.8965) .. controls (253.5702,359.4301) and (155.1909,335.9977) .. (151.6601,255.9993) .. controls (152.7204,182.2703) and (249.4137,148.0211) .. (293.1961,218.1065) -- (358.8308,180.1276) .. controls (283.4477,49.2645) and (79.6318,96.3470) .. (75.8301,255.9993) -- cycle(379.1503,247.5747) -- (362.2982,247.5747) -- (362.2982,230.7226) -- (345.4490,230.7226) -- (345.4490,247.5747) -- (328.5969,247.5747) -- (328.5969,264.4254) -- (345.4490,264.4254) -- (345.4490,281.2759) -- (362.2982,281.2759) -- (362.2982,264.4254) -- (379.1503,264.4254) -- cycle(442.3420,247.5747) -- (425.4899,247.5747) -- (425.4899,230.7226) -- (408.6408,230.7226) -- (408.6408,247.5747) -- (391.7886,247.5747) -- (391.7886,264.4254) -- (408.6408,264.4254) -- (408.6408,281.2759) -- (425.4899,281.2759) -- (425.4899,264.4254) -- (442.3420,264.4254) -- cycle;
\end{tikzpicture}
}

\providecommand{\fallogo}[1]{
\begin{tikzpicture}[scale=(#1)/512em]
\fill[gray] (185.7774,0.0000) .. controls (200.4486,15.9798) and (226.8966,8.7148) .. (235.0426,31.5836) .. controls (249.5297,58.0598) and (247.9581,97.9161) .. (280.3335,110.9762) .. controls (309.1690,120.3496) and (337.8406,104.2727) .. (366.5753,103.9379) .. controls (373.4449,111.5171) and (379.2885,128.2574) .. (383.9755,108.9744) .. controls (396.6979,102.5615) and (437.2808,107.6681) .. (426.9652,124.3252) .. controls (408.9822,121.0785) and (412.4742,146.0729) .. (426.5192,131.4996) .. controls (433.8413,120.8489) and (465.1541,126.5522) .. (441.9067,135.7950) .. controls (396.1879,157.7478) and (344.1112,161.5079) .. (298.5528,183.5702) .. controls (277.7471,193.5198) and (284.6941,218.7163) .. (285.2127,236.9640) .. controls (292.3599,316.2826) and (307.3929,394.6311) .. (317.1198,473.6154) .. controls (329.0637,505.4736) and (292.1195,528.5004) .. (265.9183,511.2761) .. controls (237.9284,499.2462) and (237.3684,465.2681) .. (230.9102,439.9421) .. controls (218.6692,374.3397) and (215.6307,306.9662) .. (198.1732,242.3977) .. controls (183.1379,232.7444) and (164.4245,256.0298) .. (149.0430,261.4799) .. controls (116.9328,279.2585) and (87.1822,308.5851) .. (48.2293,307.8914) .. controls (21.3220,306.9037) and (-15.9107,281.8761) .. (7.2921,252.7908) .. controls (29.7799,220.6177) and (67.5177,204.3028) .. (100.9287,185.9449) .. controls (130.8217,170.8906) and (161.1548,156.5903) .. (191.0278,141.5847) .. controls (196.1738,120.0520) and (186.6049,95.2409) .. (186.8382,72.4353) .. controls (185.5234,48.4204) and (183.1700,23.9341) .. (185.7774,0.0000) -- cycle;
\end{tikzpicture}
}

\providecommand{\oblogo}[1]{
\begin{tikzpicture}[scale=(#1)/512em]
\fill[gray] (160.3865,208.9117) .. controls (154.0879,214.6478) and (149.0735,221.2409) .. (145.4125,228.5384) .. controls (184.8790,248.4273) and (234.7122,269.8787) .. (297.5493,291.8782) .. controls (300.3943,281.4769) and (300.9552,268.7619) .. (300.4023,255.2389) .. controls (248.9909,244.7891) and (200.0310,225.9279) .. (160.3865,208.9117) -- cycle(225.7398,392.6996) .. controls (308.0209,392.1716) and (359.3326,345.9277) .. (368.7203,285.2098) .. controls (376.6742,197.1784) and (311.7194,141.3342) .. (205.4287,142.1456) .. controls (139.9485,141.4804) and (88.7155,166.1957) .. (73.5775,228.0086) .. controls (52.0297,320.3408) and (123.4078,391.0103) .. (225.7398,392.6996) -- cycle(216.0739,176.4733) .. controls (268.9183,179.2424) and (315.8292,206.5488) .. (312.7454,265.1139) .. controls (313.2769,315.6384) and (286.5993,353.4946) .. (216.6040,355.7934) .. controls (162.4657,355.7934) and (126.0914,317.5023) .. (126.0914,260.5103) .. controls (126.1733,214.2900) and (163.3363,176.2849) .. (216.0739,176.4733) -- cycle(76.4897,189.1754) .. controls (13.1586,147.5631) and (0.0000,119.4207) .. (0.0000,119.4207) -- (90.6499,170.1632) .. controls (85.3004,175.8497) and (80.5994,182.1633) .. (76.4897,189.1754) -- cycle(353.9486,119.3004) -- (402.9482,119.3004) .. controls (427.0025,137.0797) and (450.9893,162.7034) .. (474.9529,191.0213) .. controls (509.3540,228.5339) and (531.3391,294.2091) .. (487.8149,312.1206) .. controls (462.8165,324.7652) and (394.3874,316.8943) .. (373.8912,313.6651) .. controls (379.9291,297.7449) and (383.2899,278.4204) .. (381.4989,257.7214) .. controls (420.3069,248.0321) and (421.9610,218.3461) .. (407.7867,192.6417) .. controls (391.1113,162.4018) and (370.1114,132.9097) .. (353.9486,119.3004) -- cycle;
\end{tikzpicture}
}

\providecommand{\markuptable}{
\begin{table}
\sffamily\centering
\begin{tabular}{@{}lcl@{}}
\toprule
\texttt{//italics//} & $\rightarrow$ & \textit{italics} \\
\midrule
\texttt{**bold**} & $\rightarrow$ & \textbf{bold} \\
\midrule
\texttt{\# ordered list} & & 1 ordered list \\
\texttt{\# second item} & $\rightarrow$ & 2 second item \\
\texttt{\#\# sub item} & & \hspace{1em} 1 sub item \\
\midrule
\texttt{* unordered list} & & $\bullet$ unordered list \\
\texttt{* second item} & $\rightarrow$ & $\bullet$ second item \\
\texttt{** sub item} & & \hspace{1em} $\bullet$ sub item \\
\midrule
\texttt{link to [[label]]} & $\rightarrow$ & link to \underline{label} \\
\midrule
\texttt{<{}<label>{}> definition } & $\rightarrow$ & definition \\
\midrule
\texttt{[[url|link name]]} & $\rightarrow$ & \underline{link name} \\
\midrule\addlinespace
\texttt{= large heading} & & {\Large large heading} \smallskip \\
\texttt{== medium heading} & $\rightarrow$ & {\large medium heading} \\
\texttt{=== small heading} & & small heading \\
\midrule
\texttt{no line break} & & no line break for paragraphs \\
\texttt{for paragraphs} & $\rightarrow$ \\
& & use empty line \\
\texttt{use empty line} \\
\midrule
\texttt{force\textbackslash\textbackslash line break} & $\rightarrow$ & force \\
& & line break \\
\midrule
\texttt{horizontal line} & $\rightarrow$ & horizontal line \\
\texttt{----} & & \hrulefill \\
\midrule
\texttt{|=a|=table|=header} & & \underline{a \enspace table \enspace header} \\
\texttt{|a|table|row} & $\rightarrow$ & a \enspace table \enspace row \\
\texttt{|b|table|row} & & b \enspace table \enspace row \\
\midrule
\texttt{\{\{\{} \\
\texttt{unformatted} & $\rightarrow$ & \texttt{unformatted} \\
\texttt{code} & & \texttt{code} \\
\texttt{\}\}\}} \\
\midrule\addlinespace
\texttt{@ new article} & & {\Large 1.\ new article} \smallskip \\
\texttt{@ second article} & $\rightarrow$ & {\Large 2.\ second article} \smallskip \\
\texttt{@@ sub article} & & {\large 2.1.\ sub article} \\
\bottomrule
\end{tabular}
\normalfont\caption{Elements of the generic documentation markup language}
\label{tab:docmarkup}
\end{table}
}

\providecommand{\startchapter}[4]{
\documentclass[11pt,a4paper]{article}
\usepackage{booktabs}
\usepackage[format=hang,labelfont=bf]{caption}
\usepackage{changepage}
\usepackage[T1]{fontenc}
\usepackage[margin=2cm]{geometry}
\usepackage{hyperref}
\usepackage[american]{isodate}
\usepackage{lmodern}
\usepackage{longtable}
\usepackage{mathptmx}
\usepackage{microtype}
\usepackage[toc]{multitoc}
\usepackage{multirow}
\usepackage[all]{nowidow}
\usepackage{pdfcomment}
\usepackage{syntax}
\usepackage{tikz}
\usepackage[all]{xy}
\hypersetup{pdfborder={0 0 0},bookmarksnumbered=true,pdftitle={\ecs{}: #2},pdfauthor={Florian Negele},pdfsubject={\ecs{}},pdfkeywords={#1}}
\setlength{\grammarindent}{8em}\setlength{\grammarparsep}{0.2ex}
\setlength{\columnsep}{2em}
\newcommand{\prefix}{}
\newcounter{instruction}
\bibliographystyle{unsrt}
\renewcommand{\index}[2][]{}
\renewcommand{\arraystretch}{1.05}
\renewcommand{\floatpagefraction}{0.7}
\renewcommand{\syntleft}{\itshape}\renewcommand{\syntright}{}
\title{\vspace{-5ex}\Huge{\ecs{}}\medskip\hrule}
\author{\huge{#2}}
\date{\medskip\version}
\newif\ifbook\bookfalse
\pagestyle{headings}
\frenchspacing
\begin{document}
\maketitle\thispagestyle{empty}\noindent#4\setlength{\columnseprule}{0.4pt}\tableofcontents\setlength{\columnseprule}{0pt}\vfill\pagebreak[3]\null\vfill\bigskip\noindent
\parbox{\textwidth-4em}{\license The contents of this \documentation{} are part of the \href{manual}{\ecs{} User Manual}~\cite{manual} and correspond to Chapter ``\href{manual\##3}{#1}''.\alignright\mbox{\today}}
\parbox{4em}{\flushright\ecslogo{3em}}
\clearpage
}

\providecommand{\concludechapter}{
\vfill\pagebreak[3]\null\vfill
\thispagestyle{myheadings}\markright{REFERENCES}
\noindent\begin{minipage}{\textwidth}\begin{multicols}{2}[\section*{References}]
\renewcommand{\section}[2]{}\small\bibliography{references}
\end{multicols}\end{minipage}\end{document}
}

\providecommand{\startpresentation}[2]{
\documentclass[14pt,aspectratio=43,usepdftitle=false]{beamer}
\usepackage{booktabs}
\usepackage{etex}
\usepackage{multicol}
\usepackage{tikz}
\usepackage[all]{xy}
\bibliographystyle{unsrt}
\setlength{\columnsep}{1em}
\setlength{\leftmargini}{1em}
\setbeamercolor{title}{fg=black}
\setbeamercolor{structure}{fg=darkgray}
\setbeamercolor{bibliography item}{fg=darkgray}
\setbeamerfont{title}{series=\bfseries}
\setbeamerfont{subtitle}{series=\normalfont}
\setbeamerfont*{frametitle}{parent=title}
\setbeamerfont{block title}{series=\bfseries}
\setbeamerfont*{framesubtitle}{parent=subtitle}
\setbeamersize{text margin left=1em,text margin right=1em}
\setbeamertemplate{navigation symbols}{}
\setbeamertemplate{itemize item}[circle]{}
\setbeamertemplate{bibliography item}[triangle]{}
\setbeamertemplate{bibliography entry author}{\usebeamercolor[fg]{bibliography item}}
\setbeamertemplate{frametitle}{\medskip\usebeamerfont{frametitle}\color{gray}\raisebox{-2.5ex}[0ex][0ex]{\rule{0.1em}{4.5ex}}}
\addtobeamertemplate{frametitle}{}{\hspace{0.4em}\usebeamercolor[fg]{title}\insertframetitle\par\vspace{0.2ex}\hspace{0.5em}\usebeamerfont{framesubtitle}\insertframesubtitle}
\hypersetup{pdfborder={0 0 0},bookmarksnumbered=true,bookmarksopen=true,bookmarksopenlevel=0,pdftitle={\ecs{}: #1},pdfauthor={Florian Negele},pdfsubject={\ecs{}},pdfkeywords={#1}}
\renewcommand{\flowgraph}[1]{\resizebox{\textwidth}{!}{$$\xymatrix{##1}$$}}
\title{\ecs{}\medskip\hrule\medskip}
\institute{\shadowedecslogo{5em}{30}{15}}
\date{\version}
\subtitle{#1}
\begin{document}
\begin{frame}[plain]\titlepage\nocite{manual}\end{frame}
\begin{frame}{Contents}{#1}\begin{center}\tableofcontents\end{center}\end{frame}
}

\providecommand{\concludepresentation}{
\begin{frame}{References}\begin{footnotesize}\setlength{\columnseprule}{0.4pt}\begin{multicols}{2}\bibliography{references}\end{multicols}\end{footnotesize}\end{frame}
\end{document}
}

\providecommand{\startbook}[1]{
\documentclass[10pt,paper=17cm:24cm,DIV=13,twoside=semi,headings=normal,numbers=noendperiod,cleardoublepage=plain]{scrbook}
\usepackage{atveryend}
\usepackage{booktabs}
\usepackage{caption}
\usepackage{changepage}
\usepackage[T1]{fontenc}
\usepackage{imakeidx}
\usepackage{hyperref}
\usepackage[american]{isodate}
\usepackage{lmodern}
\usepackage{longtable}
\usepackage{mathptmx}
\usepackage[final]{microtype}
\usepackage{multicol}
\usepackage{multirow}
\usepackage[all]{nowidow}
\usepackage{pdfcomment}
\usepackage{scrlayer-scrpage}
\usepackage{setspace}
\usepackage{syntax}
\usepackage[eventxtindent=4pt,oddtxtexdent=4pt]{thumbs}
\usepackage{tikz}
\usepackage[all]{xy}
\hyphenation{Micro-Blaze Open-Cores Open-RISC Power-PC}
\hypersetup{pdfborder={0 0 0},bookmarksnumbered=true,bookmarksopen=true,bookmarksopenlevel=0,pdftitle={\ecs{}: #1},pdfauthor={Florian Negele},pdfsubject={\ecs{}},pdfkeywords={#1}}
\setlength{\grammarindent}{8em}\setlength{\grammarparsep}{0.7ex}
\setkomafont{captionlabel}{\usekomafont{descriptionlabel}}
\renewcommand{\arraystretch}{1.05}\setstretch{1.1}
\renewcommand{\chapterformat}{\thechapter\autodot\enskip\raisebox{-1ex}[0ex][0ex]{\color{gray}\rule{0.1em}{3.5ex}}\enskip}
\renewcommand{\startchapter}[4]{\hypertarget{##3}{\chapter{##1}}\label{##3}##4\addthumb{##1}{\LARGE\sffamily\bfseries\thechapter}{white}{gray}\renewcommand{\prefix}{##3}}
\renewcommand{\concludechapter}{\clearpage{\stopthumb\cleardoublepage}}
\renewcommand{\syntleft}{\itshape}\renewcommand{\syntright}{}
\renewcommand{\floatpagefraction}{0.7}
\renewcommand{\partheademptypage}{}
\DeclareMicrotypeAlias{lmss}{cmr}
\newcommand{\prefix}{}
\newcounter{instruction}
\bibliographystyle{unsrt}
\newif\ifbook\booktrue
\makeindex[intoc,title=Index]
\makeindex[intoc,name=tools,title=Index of Tools,columns=3]
\makeindex[intoc,name=library,title=Index of Library Names]
\makeindex[intoc,name=runtime,title=Index of Runtime Support]
\makeindex[intoc,name=environment,title=Index of Target Environments]
\indexsetup{toclevel=chapter,headers={\indexname}{\indexname}}
\frenchspacing
\begin{document}
\pagenumbering{alph}
\begin{titlepage}\centering
\huge\sffamily\null\vfill\textbf{\ecs{}}\bigskip\hrule\bigskip#1
\normalsize\normalfont\vfill\vfill\shadowedecslogo{10em}{30}{15}
\large\vfill\vfill\version
\end{titlepage}
\null\vfill
\thispagestyle{empty}
\noindent\today\par\medskip
\license A copy of this license is included in Appendix~\ref{fdl} on page~\pageref{fdl}.
All product names used herein are for identification purposes only and may be trademarks of their respective companies.
\concludechapter
\frontmatter
\setcounter{tocdepth}{1}
\tableofcontents
\setcounter{tocdepth}{2}
\concludechapter
\listoffigures
\concludechapter
\listoftables
\concludechapter
}

\providecommand{\concludebook}{
\backmatter
\addtocontents{toc}{\protect\setcounter{tocdepth}{-1}}
\phantomsection\addcontentsline{toc}{part}{Bibliography}
\bibliography{references}
\concludechapter
\phantomsection\addcontentsline{toc}{part}{Indexes}
\printindex
\concludechapter
\indexprologue{\label{idx:tools}}
\printindex[tools]
\concludechapter
\printindex[library]
\concludechapter
\indexprologue{\label{idx:runtime}}
\printindex[runtime]
\concludechapter
\indexprologue{\label{idx:environment}}
\printindex[environment]
\concludechapter
\pagestyle{empty}\pagenumbering{Alph}\null\clearpage
\null\vfill\centering\ecslogo{4em}\par\medskip\license
\end{document}
}

% chapter references

\providecommand{\seedocumentationref}{}\renewcommand{\seedocumentationref}[3]{#1, see \Documentation{}~\documentationref{#2}{#3}. }
\providecommand{\seeinterface}{}\renewcommand{\seeinterface}{\ifbook See \Documentation{}~\documentationref{interface}{User Interface} for more information about the common user interface of all of these tools. \fi}
\providecommand{\seeguide}{}\renewcommand{\seeguide}{\seedocumentationref{For basic examples of using some of these tools in practice}{guide}{User Guide}}
\providecommand{\seecpp}{}\renewcommand{\seecpp}{\seedocumentationref{For more information about the \cpp{} programming language and its implementation by the \ecs{}}{cpp}{User Manual for \cpp{}}}
\providecommand{\seefalse}{}\renewcommand{\seefalse}{\seedocumentationref{For more information about the FALSE programming language and its implementation by the \ecs{}}{false}{User Manual for FALSE}}
\providecommand{\seeoberon}{}\renewcommand{\seeoberon}{\seedocumentationref{For more information about the Oberon programming language and its implementation by the \ecs{}}{oberon}{User Manual for Oberon}}
\providecommand{\seeassembly}{}\renewcommand{\seeassembly}{\seedocumentationref{For more information about the generic assembly language and how to use it}{assembly}{Generic Assembly Language Specification}}
\providecommand{\seeamd}{}\renewcommand{\seeamd}{\seedocumentationref{For more information about how the \ecs{} supports the AMD64 hardware architecture}{amd64}{AMD64 Hardware Architecture Support}}
\providecommand{\seearm}{}\renewcommand{\seearm}{\seedocumentationref{For more information about how the \ecs{} supports the ARM hardware architecture}{arm}{ARM Hardware Architecture Support}}
\providecommand{\seeavr}{}\renewcommand{\seeavr}{\seedocumentationref{For more information about how the \ecs{} supports the AVR hardware architecture}{avr}{AVR Hardware Architecture Support}}
\providecommand{\seeavrtt}{}\renewcommand{\seeavrtt}{\seedocumentationref{For more information about how the \ecs{} supports the AVR32 hardware architecture}{avr32}{AVR32 Hardware Architecture Support}}
\providecommand{\seemabk}{}\renewcommand{\seemabk}{\seedocumentationref{For more information about how the \ecs{} supports the M68000 hardware architecture}{m68k}{M68000 Hardware Architecture Support}}
\providecommand{\seemibl}{}\renewcommand{\seemibl}{\seedocumentationref{For more information about how the \ecs{} supports the MicroBlaze hardware architecture}{mibl}{MicroBlaze Hardware Architecture Support}}
\providecommand{\seemips}{}\renewcommand{\seemips}{\seedocumentationref{For more information about how the \ecs{} supports the MIPS32 and MIPS64 hardware architectures}{mips}{MIPS Hardware Architecture Support}}
\providecommand{\seemmix}{}\renewcommand{\seemmix}{\seedocumentationref{For more information about how the \ecs{} supports the MMIX hardware architecture}{mmix}{MMIX Hardware Architecture Support}}
\providecommand{\seeorok}{}\renewcommand{\seeorok}{\seedocumentationref{For more information about how the \ecs{} supports the OpenRISC 1000 hardware architecture}{or1k}{OpenRISC 1000 Hardware Architecture Support}}
\providecommand{\seeppc}{}\renewcommand{\seeppc}{\seedocumentationref{For more information about how the \ecs{} supports the PowerPC hardware architecture}{ppc}{PowerPC Hardware Architecture Support}}
\providecommand{\seerisc}{}\renewcommand{\seerisc}{\seedocumentationref{For more information about how the \ecs{} supports the RISC hardware architecture}{risc}{RISC Hardware Architecture Support}}
\providecommand{\seewasm}{}\renewcommand{\seewasm}{\seedocumentationref{For more information about how the \ecs{} supports the WebAssembly architecture}{wasm}{WebAssembly Architecture Support}}
\providecommand{\seedocumentation}{}\renewcommand{\seedocumentation}{\seedocumentationref{For more information about generic documentations and their generation by the \ecs{}}{documentation}{Generic Documentation Generation}}
\providecommand{\seedebugging}{}\renewcommand{\seedebugging}{\seedocumentationref{For more information about debugging information and its representation}{debugging}{Debugging Information Representation}}
\providecommand{\seecode}{}\renewcommand{\seecode}{\seedocumentationref{For more information about intermediate code and its purpose}{code}{Intermediate Code Representation}}
\providecommand{\seeobject}{}\renewcommand{\seeobject}{\seedocumentationref{For more information about object files and their purpose}{object}{Object File Representation}}

% generic documentation tools

\providecommand{\docprint}{
\toolsection{docprint} is a pretty printer for generic documentations.
It reformats generic documentations and writes it to the standard output stream.
\debuggingtool
\flowgraph{\resource{generic\\documentation} \ar[r] & \toolbox{docprint} \ar[r] & \resource{generic\\documentation}}
\seedocumentation
}

\providecommand{\doccheck}{
\toolsection{doccheck} is a syntactic and semantic checker for generic documentations.
It just performs syntactic and semantic checks on generic documentations and writes its diagnostic messages to the standard error stream.
\debuggingtool
\flowgraph{\resource{generic\\documentation} \ar[r] & \toolbox{doccheck} \ar[r] & \resource{diagnostic\\messages}}
\seedocumentation
}

\providecommand{\dochtml}{
\toolsection{dochtml} is an HTML documentation generator for generic documentations.
It processes several generic documentations and assembles all information therein into an HTML document.
\debuggingtool
\flowgraph{\resource{generic\\documentation} \ar[r] & \toolbox{dochtml} \ar[r] & \resource{HTML\\document}}
\seedocumentation
}

\providecommand{\doclatex}{
\toolsection{doclatex} is a Latex documentation generator for generic documentations.
It processes several generic documentations and assembles all information therein into a Latex document.
\debuggingtool
\flowgraph{\resource{generic\\documentation} \ar[r] & \toolbox{doclatex} \ar[r] & \resource{Latex\\document}}
\seedocumentation
}

% intermediate code tools

\providecommand{\cdcheck}{
\toolsection{cdcheck} is a syntactic and semantic checker for intermediate code.
It just performs syntactic and semantic checks on programs written in intermediate code and writes its diagnostic messages to the standard error stream.
\debuggingtool
\flowgraph{\resource{intermediate\\code} \ar[r] & \toolbox{cdcheck} \ar[r] & \resource{diagnostic\\messages}}
\seeassembly\seecode
}

\providecommand{\cdopt}{
\toolsection{cdopt} is an optimizer for intermediate code.
It performs various optimizations on programs written in intermediate code and writes the result to the standard output stream.
\debuggingtool
\flowgraph{\resource{intermediate\\code} \ar[r] & \toolbox{cdopt} \ar[r] & \resource{optimized\\code}}
\seeassembly\seecode
}

\providecommand{\cdrun}{
\toolsection{cdrun} is an interpreter for intermediate code.
It processes and executes programs written in intermediate code.
The following code sections are predefined and have the usual semantics:
\texttt{abort}, \texttt{\_Exit}, \texttt{fflush}, \texttt{floor}, \texttt{fputc}, \texttt{free}, \texttt{getchar}, \texttt{malloc}, and \texttt{putchar}.
Diagnostic messages about invalid operations include the name of the executed code section and the index of the erroneous instruction.
\debuggingtool
\flowgraph{\resource{intermediate\\code} \ar[r] & \toolbox{cdrun} \ar@/u/[r] & \resource{input/\\output} \ar@/d/[l]}
\seeassembly\seecode
}

\providecommand{\cdamda}{
\toolsection{cdamd16} is a compiler for intermediate code targeting the AMD64 hardware architecture.
It generates machine code for AMD64 processors from programs written in intermediate code and stores it in corresponding object files.
The compiler generates machine code for the 16-bit operating mode defined by the AMD64 architecture.
It also creates a debugging information file as well as an assembly file containing a listing of the generated machine code.
\debuggingtool
\flowgraph{\resource{intermediate\\code} \ar[r] & \toolbox{cdamd16} \ar[r] \ar[d] \ar[rd] & \resource{object file} \\ & \resource{assembly\\listing} & \resource{debugging\\information}}
\seeassembly\seeamd\seeobject\seecode\seedebugging
}

\providecommand{\cdamdb}{
\toolsection{cdamd32} is a compiler for intermediate code targeting the AMD64 hardware architecture.
It generates machine code for AMD64 processors from programs written in intermediate code and stores it in corresponding object files.
The compiler generates machine code for the 32-bit operating mode defined by the AMD64 architecture.
It also creates a debugging information file as well as an assembly file containing a listing of the generated machine code.
\debuggingtool
\flowgraph{\resource{intermediate\\code} \ar[r] & \toolbox{cdamd32} \ar[r] \ar[d] \ar[rd] & \resource{object file} \\ & \resource{assembly\\listing} & \resource{debugging\\information}}
\seeassembly\seeamd\seeobject\seecode\seedebugging
}

\providecommand{\cdamdc}{
\toolsection{cdamd64} is a compiler for intermediate code targeting the AMD64 hardware architecture.
It generates machine code for AMD64 processors from programs written in intermediate code and stores it in corresponding object files.
The compiler generates machine code for the 64-bit operating mode defined by the AMD64 architecture.
It also creates a debugging information file as well as an assembly file containing a listing of the generated machine code.
\debuggingtool
\flowgraph{\resource{intermediate\\code} \ar[r] & \toolbox{cdamd64} \ar[r] \ar[d] \ar[rd] & \resource{object file} \\ & \resource{assembly\\listing} & \resource{debugging\\information}}
\seeassembly\seeamd\seeobject\seecode\seedebugging
}

\providecommand{\cdarma}{
\toolsection{cdarma32} is a compiler for intermediate code targeting the ARM hardware architecture.
It generates machine code for ARM processors executing A32 instructions from programs written in intermediate code and stores it in corresponding object files.
It also creates a debugging information file as well as an assembly file containing a listing of the generated machine code.
\debuggingtool
\flowgraph{\resource{intermediate\\code} \ar[r] & \toolbox{cdarma32} \ar[r] \ar[d] \ar[rd] & \resource{object file} \\ & \resource{assembly\\listing} & \resource{debugging\\information}}
\seeassembly\seearm\seeobject\seecode\seedebugging
}

\providecommand{\cdarmb}{
\toolsection{cdarma64} is a compiler for intermediate code targeting the ARM hardware architecture.
It generates machine code for ARM processors executing A64 instructions from programs written in intermediate code and stores it in corresponding object files.
It also creates a debugging information file as well as an assembly file containing a listing of the generated machine code.
\debuggingtool
\flowgraph{\resource{intermediate\\code} \ar[r] & \toolbox{cdarma64} \ar[r] \ar[d] \ar[rd] & \resource{object file} \\ & \resource{assembly\\listing} & \resource{debugging\\information}}
\seeassembly\seearm\seeobject\seecode\seedebugging
}

\providecommand{\cdarmc}{
\toolsection{cdarmt32} is a compiler for intermediate code targeting the ARM hardware architecture.
It generates machine code for ARM processors without floating-point extension executing T32 instructions from programs written in intermediate code and stores it in corresponding object files.
It also creates a debugging information file as well as an assembly file containing a listing of the generated machine code.
\debuggingtool
\flowgraph{\resource{intermediate\\code} \ar[r] & \toolbox{cdarmt32} \ar[r] \ar[d] \ar[rd] & \resource{object file} \\ & \resource{assembly\\listing} & \resource{debugging\\information}}
\seeassembly\seearm\seeobject\seecode\seedebugging
}

\providecommand{\cdarmcfpe}{
\toolsection{cdarmt32fpe} is a compiler for intermediate code targeting the ARM hardware architecture.
It generates machine code for ARM processors with floating-point extension executing T32 instructions from programs written in intermediate code and stores it in corresponding object files.
It also creates a debugging information file as well as an assembly file containing a listing of the generated machine code.
\debuggingtool
\flowgraph{\resource{intermediate\\code} \ar[r] & \toolbox{cdarmt32fpe} \ar[r] \ar[d] \ar[rd] & \resource{object file} \\ & \resource{assembly\\listing} & \resource{debugging\\information}}
\seeassembly\seearm\seeobject\seecode\seedebugging
}

\providecommand{\cdavr}{
\toolsection{cdavr} is a compiler for intermediate code targeting the AVR hardware architecture.
It generates machine code for AVR processors from programs written in intermediate code and stores it in corresponding object files.
It also creates a debugging information file as well as an assembly file containing a listing of the generated machine code.
\debuggingtool
\flowgraph{\resource{intermediate\\code} \ar[r] & \toolbox{cdavr} \ar[r] \ar[d] \ar[rd] & \resource{object file} \\ & \resource{assembly\\listing} & \resource{debugging\\information}}
\seeassembly\seeavr\seeobject\seecode\seedebugging
}

\providecommand{\cdavrtt}{
\toolsection{cdavr32} is a compiler for intermediate code targeting the AVR32 hardware architecture.
It generates machine code for AVR32 processors from programs written in intermediate code and stores it in corresponding object files.
It also creates a debugging information file as well as an assembly file containing a listing of the generated machine code.
\debuggingtool
\flowgraph{\resource{intermediate\\code} \ar[r] & \toolbox{cdavr32} \ar[r] \ar[d] \ar[rd] & \resource{object file} \\ & \resource{assembly\\listing} & \resource{debugging\\information}}
\seeassembly\seeavrtt\seeobject\seecode\seedebugging
}

\providecommand{\cdmabk}{
\toolsection{cdm68k} is a compiler for intermediate code targeting the M68000 hardware architecture.
It generates machine code for M68000 processors from programs written in intermediate code and stores it in corresponding object files.
It also creates a debugging information file as well as an assembly file containing a listing of the generated machine code.
\debuggingtool
\flowgraph{\resource{intermediate\\code} \ar[r] & \toolbox{cdm68k} \ar[r] \ar[d] \ar[rd] & \resource{object file} \\ & \resource{assembly\\listing} & \resource{debugging\\information}}
\seeassembly\seemabk\seeobject\seecode\seedebugging
}

\providecommand{\cdmibl}{
\toolsection{cdmibl} is a compiler for intermediate code targeting the MicroBlaze hardware architecture.
It generates machine code for MicroBlaze processors from programs written in intermediate code and stores it in corresponding object files.
It also creates a debugging information file as well as an assembly file containing a listing of the generated machine code.
\debuggingtool
\flowgraph{\resource{intermediate\\code} \ar[r] & \toolbox{cdmibl} \ar[r] \ar[d] \ar[rd] & \resource{object file} \\ & \resource{assembly\\listing} & \resource{debugging\\information}}
\seeassembly\seemibl\seeobject\seecode\seedebugging
}

\providecommand{\cdmipsa}{
\toolsection{cdmips32} is a compiler for intermediate code targeting the MIPS32 hardware architecture.
It generates machine code for MIPS32 processors from programs written in intermediate code and stores it in corresponding object files.
It also creates a debugging information file as well as an assembly file containing a listing of the generated machine code.
\debuggingtool
\flowgraph{\resource{intermediate\\code} \ar[r] & \toolbox{cdmips32} \ar[r] \ar[d] \ar[rd] & \resource{object file} \\ & \resource{assembly\\listing} & \resource{debugging\\information}}
\seeassembly\seemips\seeobject\seecode\seedebugging
}

\providecommand{\cdmipsb}{
\toolsection{cdmips64} is a compiler for intermediate code targeting the MIPS64 hardware architecture.
It generates machine code for MIPS64 processors from programs written in intermediate code and stores it in corresponding object files.
It also creates a debugging information file as well as an assembly file containing a listing of the generated machine code.
\debuggingtool
\flowgraph{\resource{intermediate\\code} \ar[r] & \toolbox{cdmips64} \ar[r] \ar[d] \ar[rd] & \resource{object file} \\ & \resource{assembly\\listing} & \resource{debugging\\information}}
\seeassembly\seemips\seeobject\seecode\seedebugging
}

\providecommand{\cdmmix}{
\toolsection{cdmmix} is a compiler for intermediate code targeting the MMIX hardware architecture.
It generates machine code for MMIX processors from programs written in intermediate code and stores it in corresponding object files.
It also creates a debugging information file as well as an assembly file containing a listing of the generated machine code.
\debuggingtool
\flowgraph{\resource{intermediate\\code} \ar[r] & \toolbox{cdmmix} \ar[r] \ar[d] \ar[rd] & \resource{object file} \\ & \resource{assembly\\listing} & \resource{debugging\\information}}
\seeassembly\seemmix\seeobject\seecode\seedebugging
}

\providecommand{\cdorok}{
\toolsection{cdor1k} is a compiler for intermediate code targeting the OpenRISC 1000 hardware architecture.
It generates machine code for OpenRISC 1000 processors from programs written in intermediate code and stores it in corresponding object files.
It also creates a debugging information file as well as an assembly file containing a listing of the generated machine code.
\debuggingtool
\flowgraph{\resource{intermediate\\code} \ar[r] & \toolbox{cdor1k} \ar[r] \ar[d] \ar[rd] & \resource{object file} \\ & \resource{assembly\\listing} & \resource{debugging\\information}}
\seeassembly\seeorok\seeobject\seecode\seedebugging
}

\providecommand{\cdppca}{
\toolsection{cdppc32} is a compiler for intermediate code targeting the PowerPC hardware architecture.
It generates machine code for PowerPC processors from programs written in intermediate code and stores it in corresponding object files.
The compiler generates machine code for the 32-bit operating mode defined by the PowerPC architecture.
It also creates a debugging information file as well as an assembly file containing a listing of the generated machine code.
\debuggingtool
\flowgraph{\resource{intermediate\\code} \ar[r] & \toolbox{cdppc32} \ar[r] \ar[d] \ar[rd] & \resource{object file} \\ & \resource{assembly\\listing} & \resource{debugging\\information}}
\seeassembly\seeppc\seeobject\seecode\seedebugging
}

\providecommand{\cdppcb}{
\toolsection{cdppc64} is a compiler for intermediate code targeting the PowerPC hardware architecture.
It generates machine code for PowerPC processors from programs written in intermediate code and stores it in corresponding object files.
The compiler generates machine code for the 64-bit operating mode defined by the PowerPC architecture.
It also creates a debugging information file as well as an assembly file containing a listing of the generated machine code.
\debuggingtool
\flowgraph{\resource{intermediate\\code} \ar[r] & \toolbox{cdppc64} \ar[r] \ar[d] \ar[rd] & \resource{object file} \\ & \resource{assembly\\listing} & \resource{debugging\\information}}
\seeassembly\seeppc\seeobject\seecode\seedebugging
}

\providecommand{\cdrisc}{
\toolsection{cdrisc} is a compiler for intermediate code targeting the RISC hardware architecture.
It generates machine code for RISC processors from programs written in intermediate code and stores it in corresponding object files.
It also creates a debugging information file as well as an assembly file containing a listing of the generated machine code.
\debuggingtool
\flowgraph{\resource{intermediate\\code} \ar[r] & \toolbox{cdrisc} \ar[r] \ar[d] \ar[rd] & \resource{object file} \\ & \resource{assembly\\listing} & \resource{debugging\\information}}
\seeassembly\seerisc\seeobject\seecode\seedebugging
}

\providecommand{\cdwasm}{
\toolsection{cdwasm} is a compiler for intermediate code targeting the WebAssembly architecture.
It generates machine code for WebAssembly targets from programs written in intermediate code and stores it in corresponding object files.
It also creates a debugging information file as well as an assembly file containing a listing of the generated machine code.
\debuggingtool
\flowgraph{\resource{intermediate\\code} \ar[r] & \toolbox{cdwasm} \ar[r] \ar[d] \ar[rd] & \resource{object file} \\ & \resource{assembly\\listing} & \resource{debugging\\information}}
\seeassembly\seewasm\seeobject\seecode\seedebugging
}

% C++ tools

\providecommand{\cppprep}{
\toolsection{cppprep} is a preprocessor for the \cpp{} programming language.
It preprocesses source code according to the rules of \cpp{} and writes it to the standard output stream.
Only the macro names \texttt{\_\_DATE\_\_}, \texttt{\_\_FILE\_\_}, \texttt{\_\_LINE\_\_}, and \texttt{\_\_TIME\_\_} are predefined.
\flowgraph{\resource{\cpp{} or other\\source code} \ar[r] & \toolbox{cppprep} \ar[r] & \resource{preprocessed\\source code} \\ & \variable{ECSINCLUDE} \ar[u]}
\seecpp
}

\providecommand{\cppprint}{
\toolsection{cppprint} is a pretty printer for the \cpp{} programming language.
It reformats the source code of \cpp{} programs and writes it to the standard output stream.
\flowgraph{\resource{\cpp{}\\source code} \ar[r] & \toolbox{cppprint} \ar[r] & \resource{reformatted\\source code} \\ & \variable{ECSINCLUDE} \ar[u]}
\seecpp
}

\providecommand{\cppcheck}{
\toolsection{cppcheck} is a syntactic and semantic checker for the \cpp{} programming language.
It just performs syntactic and semantic checks on \cpp{} programs and writes its diagnostic messages to the standard error stream.
\flowgraph{\resource{\cpp{}\\source code} \ar[r] & \toolbox{cppcheck} \ar[r] & \resource{diagnostic\\messages} \\ & \variable{ECSINCLUDE} \ar[u]}
\seecpp
}

\providecommand{\cppdump}{
\toolsection{cppdump} is a serializer for the \cpp{} programming language.
It dumps the complete internal representation of programs written in \cpp{} into an XML document.
\debuggingtool
\flowgraph{\resource{\cpp{}\\source code} \ar[r] & \toolbox{cppdump} \ar[r] & \resource{internal\\representation} \\ & \variable{ECSINCLUDE} \ar[u]}
\seecpp
}

\providecommand{\cpprun}{
\toolsection{cpprun} is an interpreter for the \cpp{} programming language.
It processes and executes programs written in \cpp{}.
The macro \texttt{\_\_run\_\_} is predefined in order to enable programmers to identify this tool while interpreting.
\flowgraph{\resource{\cpp{}\\source code} \ar[r] & \toolbox{cpprun} \ar@/u/[r] & \resource{input/\\output} \ar@/d/[l] \\ & \variable{ECSINCLUDE} \ar[u]}
\seecpp
}

\providecommand{\cppdoc}{
\toolsection{cppdoc} is a generic documentation generator for the \cpp{} programming language.
It processes several \cpp{} source files and assembles all information therein into a generic documentation.
\debuggingtool
\flowgraph{\resource{\cpp{}\\source code} \ar[r] & \toolbox{cppdoc} \ar[r] & \resource{generic\\documentation} \\ & \variable{ECSINCLUDE} \ar[u]}
\seecpp\seedocumentation
}

\providecommand{\cpphtml}{
\toolsection{cpphtml} is an HTML documentation generator for the \cpp{} programming language.
It processes several \cpp{} source files and assembles all information therein into an HTML document.
\flowgraph{\resource{\cpp{}\\source code} \ar[r] & \toolbox{cpphtml} \ar[r] & \resource{HTML\\document} \\ & \variable{ECSINCLUDE} \ar[u]}
\seecpp\seedocumentation
}

\providecommand{\cpplatex}{
\toolsection{cpplatex} is a Latex documentation generator for the \cpp{} programming language.
It processes several \cpp{} source files and assembles all information therein into a Latex document.
\flowgraph{\resource{\cpp{}\\source code} \ar[r] & \toolbox{cpplatex} \ar[r] & \resource{Latex\\document} \\ & \variable{ECSINCLUDE} \ar[u]}
\seecpp\seedocumentation
}

\providecommand{\cppcode}{
\toolsection{cppcode} is an intermediate code generator for the \cpp{} programming language.
It generates intermediate code from programs written in \cpp{} and stores it in corresponding assembly files.
The macro \texttt{\_\_code\_\_} is predefined in order to enable programmers to identify this tool while generating intermediate code.
Programs generated with this tool require additional runtime support that is stored in the \file{cpp\-code\-run} library file.
\debuggingtool
\flowgraph{\resource{\cpp{}\\source code} \ar[r] & \toolbox{cppcode} \ar[r] & \resource{intermediate\\code} \\ & \variable{ECSINCLUDE} \ar[u]}
\seecpp\seeassembly\seecode
}

\providecommand{\cppamda}{
\toolsection{cppamd16} is a compiler for the \cpp{} programming language targeting the AMD64 hardware architecture.
It generates machine code for AMD64 processors from programs written in \cpp{} and stores it in corresponding object files.
The compiler generates machine code for the 16-bit operating mode defined by the AMD64 architecture.
For debugging purposes, it also creates a debugging information file as well as an assembly file containing a listing of the generated machine code.
The macro \texttt{\_\_amd16\_\_} is predefined in order to enable programmers to identify this tool and its target architecture while compiling.
Programs generated with this compiler require additional runtime support that is stored in the \file{cpp\-amd16\-run} library file.
\flowgraph{\resource{\cpp{}\\source code} \ar[r] & \toolbox{cppamd16} \ar[r] \ar[d] \ar[rd] & \resource{object file} \\ \variable{ECSINCLUDE} \ar[ru] & \resource{debugging\\information} & \resource{assembly\\listing}}
\seecpp\seeassembly\seeamd\seeobject\seedebugging
}

\providecommand{\cppamdb}{
\toolsection{cppamd32} is a compiler for the \cpp{} programming language targeting the AMD64 hardware architecture.
It generates machine code for AMD64 processors from programs written in \cpp{} and stores it in corresponding object files.
The compiler generates machine code for the 32-bit operating mode defined by the AMD64 architecture.
For debugging purposes, it also creates a debugging information file as well as an assembly file containing a listing of the generated machine code.
The macro \texttt{\_\_amd32\_\_} is predefined in order to enable programmers to identify this tool and its target architecture while compiling.
Programs generated with this compiler require additional runtime support that is stored in the \file{cpp\-amd32\-run} library file.
\flowgraph{\resource{\cpp{}\\source code} \ar[r] & \toolbox{cppamd32} \ar[r] \ar[d] \ar[rd] & \resource{object file} \\ \variable{ECSINCLUDE} \ar[ru] & \resource{debugging\\information} & \resource{assembly\\listing}}
\seecpp\seeassembly\seeamd\seeobject\seedebugging
}

\providecommand{\cppamdc}{
\toolsection{cppamd64} is a compiler for the \cpp{} programming language targeting the AMD64 hardware architecture.
It generates machine code for AMD64 processors from programs written in \cpp{} and stores it in corresponding object files.
The compiler generates machine code for the 64-bit operating mode defined by the AMD64 architecture.
For debugging purposes, it also creates a debugging information file as well as an assembly file containing a listing of the generated machine code.
The macro \texttt{\_\_amd64\_\_} is predefined in order to enable programmers to identify this tool and its target architecture while compiling.
Programs generated with this compiler require additional runtime support that is stored in the \file{cpp\-amd64\-run} library file.
\flowgraph{\resource{\cpp{}\\source code} \ar[r] & \toolbox{cppamd64} \ar[r] \ar[d] \ar[rd] & \resource{object file} \\ \variable{ECSINCLUDE} \ar[ru] & \resource{debugging\\information} & \resource{assembly\\listing}}
\seecpp\seeassembly\seeamd\seeobject\seedebugging
}

\providecommand{\cpparma}{
\toolsection{cpparma32} is a compiler for the \cpp{} programming language targeting the ARM hardware architecture.
It generates machine code for ARM processors executing A32 instructions from programs written in \cpp{} and stores it in corresponding object files.
For debugging purposes, it also creates a debugging information file as well as an assembly file containing a listing of the generated machine code.
The macro \texttt{\_\_arma32\_\_} is predefined in order to enable programmers to identify this tool and its target architecture while compiling.
Programs generated with this compiler require additional runtime support that is stored in the \file{cpp\-arma32\-run} library file.
\flowgraph{\resource{\cpp{}\\source code} \ar[r] & \toolbox{cpparma32} \ar[r] \ar[d] \ar[rd] & \resource{object file} \\ \variable{ECSINCLUDE} \ar[ru] & \resource{debugging\\information} & \resource{assembly\\listing}}
\seecpp\seeassembly\seearm\seeobject\seedebugging
}

\providecommand{\cpparmb}{
\toolsection{cpparma64} is a compiler for the \cpp{} programming language targeting the ARM hardware architecture.
It generates machine code for ARM processors executing A64 instructions from programs written in \cpp{} and stores it in corresponding object files.
For debugging purposes, it also creates a debugging information file as well as an assembly file containing a listing of the generated machine code.
The macro \texttt{\_\_arma64\_\_} is predefined in order to enable programmers to identify this tool and its target architecture while compiling.
Programs generated with this compiler require additional runtime support that is stored in the \file{cpp\-arma64\-run} library file.
\flowgraph{\resource{\cpp{}\\source code} \ar[r] & \toolbox{cpparma64} \ar[r] \ar[d] \ar[rd] & \resource{object file} \\ \variable{ECSINCLUDE} \ar[ru] & \resource{debugging\\information} & \resource{assembly\\listing}}
\seecpp\seeassembly\seearm\seeobject\seedebugging
}

\providecommand{\cpparmc}{
\toolsection{cpparmt32} is a compiler for the \cpp{} programming language targeting the ARM hardware architecture.
It generates machine code for ARM processors without floating-point extension executing T32 instructions from programs written in \cpp{} and stores it in corresponding object files.
For debugging purposes, it also creates a debugging information file as well as an assembly file containing a listing of the generated machine code.
The macro \texttt{\_\_armt32\_\_} is predefined in order to enable programmers to identify this tool and its target architecture while compiling.
Programs generated with this compiler require additional runtime support that is stored in the \file{cpp\-armt32\-run} library file.
\flowgraph{\resource{\cpp{}\\source code} \ar[r] & \toolbox{cpparmt32} \ar[r] \ar[d] \ar[rd] & \resource{object file} \\ \variable{ECSINCLUDE} \ar[ru] & \resource{debugging\\information} & \resource{assembly\\listing}}
\seecpp\seeassembly\seearm\seeobject\seedebugging
}

\providecommand{\cpparmcfpe}{
\toolsection{cpparmt32fpe} is a compiler for the \cpp{} programming language targeting the ARM hardware architecture.
It generates machine code for ARM processors with floating-point extension executing T32 instructions from programs written in \cpp{} and stores it in corresponding object files.
For debugging purposes, it also creates a debugging information file as well as an assembly file containing a listing of the generated machine code.
The macro \texttt{\_\_armt32fpe\_\_} is predefined in order to enable programmers to identify this tool and its target architecture while compiling.
Programs generated with this compiler require additional runtime support that is stored in the \file{cpp\-armt32\-fpe\-run} library file.
\flowgraph{\resource{\cpp{}\\source code} \ar[r] & \toolbox{cpparmt32fpe} \ar[r] \ar[d] \ar[rd] & \resource{object file} \\ \variable{ECSINCLUDE} \ar[ru] & \resource{debugging\\information} & \resource{assembly\\listing}}
\seecpp\seeassembly\seearm\seeobject\seedebugging
}

\providecommand{\cppavr}{
\toolsection{cppavr} is a compiler for the \cpp{} programming language targeting the AVR hardware architecture.
It generates machine code for AVR processors from programs written in \cpp{} and stores it in corresponding object files.
For debugging purposes, it also creates a debugging information file as well as an assembly file containing a listing of the generated machine code.
The macro \texttt{\_\_avr\_\_} is predefined in order to enable programmers to identify this tool and its target architecture while compiling.
Programs generated with this compiler require additional runtime support that is stored in the \file{cpp\-avr\-run} library file.
\flowgraph{\resource{\cpp{}\\source code} \ar[r] & \toolbox{cppavr} \ar[r] \ar[d] \ar[rd] & \resource{object file} \\ \variable{ECSINCLUDE} \ar[ru] & \resource{debugging\\information} & \resource{assembly\\listing}}
\seecpp\seeassembly\seeavr\seeobject\seedebugging
}

\providecommand{\cppavrtt}{
\toolsection{cppavr32} is a compiler for the \cpp{} programming language targeting the AVR32 hardware architecture.
It generates machine code for AVR32 processors from programs written in \cpp{} and stores it in corresponding object files.
For debugging purposes, it also creates a debugging information file as well as an assembly file containing a listing of the generated machine code.
The macro \texttt{\_\_avr32\_\_} is predefined in order to enable programmers to identify this tool and its target architecture while compiling.
Programs generated with this compiler require additional runtime support that is stored in the \file{cpp\-avr32\-run} library file.
\flowgraph{\resource{\cpp{}\\source code} \ar[r] & \toolbox{cppavr32} \ar[r] \ar[d] \ar[rd] & \resource{object file} \\ \variable{ECSINCLUDE} \ar[ru] & \resource{debugging\\information} & \resource{assembly\\listing}}
\seecpp\seeassembly\seeavrtt\seeobject\seedebugging
}

\providecommand{\cppmabk}{
\toolsection{cppm68k} is a compiler for the \cpp{} programming language targeting the M68000 hardware architecture.
It generates machine code for M68000 processors from programs written in \cpp{} and stores it in corresponding object files.
For debugging purposes, it also creates a debugging information file as well as an assembly file containing a listing of the generated machine code.
The macro \texttt{\_\_m68k\_\_} is predefined in order to enable programmers to identify this tool and its target architecture while compiling.
Programs generated with this compiler require additional runtime support that is stored in the \file{cpp\-m68k\-run} library file.
\flowgraph{\resource{\cpp{}\\source code} \ar[r] & \toolbox{cppm68k} \ar[r] \ar[d] \ar[rd] & \resource{object file} \\ \variable{ECSINCLUDE} \ar[ru] & \resource{debugging\\information} & \resource{assembly\\listing}}
\seecpp\seeassembly\seemabk\seeobject\seedebugging
}

\providecommand{\cppmibl}{
\toolsection{cppmibl} is a compiler for the \cpp{} programming language targeting the MicroBlaze hardware architecture.
It generates machine code for MicroBlaze processors from programs written in \cpp{} and stores it in corresponding object files.
For debugging purposes, it also creates a debugging information file as well as an assembly file containing a listing of the generated machine code.
The macro \texttt{\_\_mibl\_\_} is predefined in order to enable programmers to identify this tool and its target architecture while compiling.
Programs generated with this compiler require additional runtime support that is stored in the \file{cpp\-mibl\-run} library file.
\flowgraph{\resource{\cpp{}\\source code} \ar[r] & \toolbox{cppmibl} \ar[r] \ar[d] \ar[rd] & \resource{object file} \\ \variable{ECSINCLUDE} \ar[ru] & \resource{debugging\\information} & \resource{assembly\\listing}}
\seecpp\seeassembly\seemibl\seeobject\seedebugging
}

\providecommand{\cppmipsa}{
\toolsection{cppmips32} is a compiler for the \cpp{} programming language targeting the MIPS32 hardware architecture.
It generates machine code for MIPS32 processors from programs written in \cpp{} and stores it in corresponding object files.
For debugging purposes, it also creates a debugging information file as well as an assembly file containing a listing of the generated machine code.
The macro \texttt{\_\_mips32\_\_} is predefined in order to enable programmers to identify this tool and its target architecture while compiling.
Programs generated with this compiler require additional runtime support that is stored in the \file{cpp\-mips32\-run} library file.
\flowgraph{\resource{\cpp{}\\source code} \ar[r] & \toolbox{cppmips32} \ar[r] \ar[d] \ar[rd] & \resource{object file} \\ \variable{ECSINCLUDE} \ar[ru] & \resource{debugging\\information} & \resource{assembly\\listing}}
\seecpp\seeassembly\seemips\seeobject\seedebugging
}

\providecommand{\cppmipsb}{
\toolsection{cppmips64} is a compiler for the \cpp{} programming language targeting the MIPS64 hardware architecture.
It generates machine code for MIPS64 processors from programs written in \cpp{} and stores it in corresponding object files.
For debugging purposes, it also creates a debugging information file as well as an assembly file containing a listing of the generated machine code.
The macro \texttt{\_\_mips64\_\_} is predefined in order to enable programmers to identify this tool and its target architecture while compiling.
Programs generated with this compiler require additional runtime support that is stored in the \file{cpp\-mips64\-run} library file.
\flowgraph{\resource{\cpp{}\\source code} \ar[r] & \toolbox{cppmips64} \ar[r] \ar[d] \ar[rd] & \resource{object file} \\ \variable{ECSINCLUDE} \ar[ru] & \resource{debugging\\information} & \resource{assembly\\listing}}
\seecpp\seeassembly\seemips\seeobject\seedebugging
}

\providecommand{\cppmmix}{
\toolsection{cppmmix} is a compiler for the \cpp{} programming language targeting the MMIX hardware architecture.
It generates machine code for MMIX processors from programs written in \cpp{} and stores it in corresponding object files.
For debugging purposes, it also creates a debugging information file as well as an assembly file containing a listing of the generated machine code.
The macro \texttt{\_\_mmix\_\_} is predefined in order to enable programmers to identify this tool and its target architecture while compiling.
Programs generated with this compiler require additional runtime support that is stored in the \file{cpp\-mmix\-run} library file.
\flowgraph{\resource{\cpp{}\\source code} \ar[r] & \toolbox{cppmmix} \ar[r] \ar[d] \ar[rd] & \resource{object file} \\ \variable{ECSINCLUDE} \ar[ru] & \resource{debugging\\information} & \resource{assembly\\listing}}
\seecpp\seeassembly\seemmix\seeobject\seedebugging
}

\providecommand{\cpporok}{
\toolsection{cppor1k} is a compiler for the \cpp{} programming language targeting the OpenRISC 1000 hardware architecture.
It generates machine code for OpenRISC 1000 processors from programs written in \cpp{} and stores it in corresponding object files.
For debugging purposes, it also creates a debugging information file as well as an assembly file containing a listing of the generated machine code.
The macro \texttt{\_\_or1k\_\_} is predefined in order to enable programmers to identify this tool and its target architecture while compiling.
Programs generated with this compiler require additional runtime support that is stored in the \file{cpp\-or1k\-run} library file.
\flowgraph{\resource{\cpp{}\\source code} \ar[r] & \toolbox{cppor1k} \ar[r] \ar[d] \ar[rd] & \resource{object file} \\ \variable{ECSINCLUDE} \ar[ru] & \resource{debugging\\information} & \resource{assembly\\listing}}
\seecpp\seeassembly\seeorok\seeobject\seedebugging
}

\providecommand{\cppppca}{
\toolsection{cppppc32} is a compiler for the \cpp{} programming language targeting the PowerPC hardware architecture.
It generates machine code for PowerPC processors from programs written in \cpp{} and stores it in corresponding object files.
The compiler generates machine code for the 32-bit operating mode defined by the PowerPC architecture.
For debugging purposes, it also creates a debugging information file as well as an assembly file containing a listing of the generated machine code.
The macro \texttt{\_\_ppc32\_\_} is predefined in order to enable programmers to identify this tool and its target architecture while compiling.
Programs generated with this compiler require additional runtime support that is stored in the \file{cpp\-ppc32\-run} library file.
\flowgraph{\resource{\cpp{}\\source code} \ar[r] & \toolbox{cppppc32} \ar[r] \ar[d] \ar[rd] & \resource{object file} \\ \variable{ECSINCLUDE} \ar[ru] & \resource{debugging\\information} & \resource{assembly\\listing}}
\seecpp\seeassembly\seeppc\seeobject\seedebugging
}

\providecommand{\cppppcb}{
\toolsection{cppppc64} is a compiler for the \cpp{} programming language targeting the PowerPC hardware architecture.
It generates machine code for PowerPC processors from programs written in \cpp{} and stores it in corresponding object files.
The compiler generates machine code for the 64-bit operating mode defined by the PowerPC architecture.
For debugging purposes, it also creates a debugging information file as well as an assembly file containing a listing of the generated machine code.
The macro \texttt{\_\_ppc64\_\_} is predefined in order to enable programmers to identify this tool and its target architecture while compiling.
Programs generated with this compiler require additional runtime support that is stored in the \file{cpp\-ppc64\-run} library file.
\flowgraph{\resource{\cpp{}\\source code} \ar[r] & \toolbox{cppppc64} \ar[r] \ar[d] \ar[rd] & \resource{object file} \\ \variable{ECSINCLUDE} \ar[ru] & \resource{debugging\\information} & \resource{assembly\\listing}}
\seecpp\seeassembly\seeppc\seeobject\seedebugging
}

\providecommand{\cpprisc}{
\toolsection{cpprisc} is a compiler for the \cpp{} programming language targeting the RISC hardware architecture.
It generates machine code for RISC processors from programs written in \cpp{} and stores it in corresponding object files.
For debugging purposes, it also creates a debugging information file as well as an assembly file containing a listing of the generated machine code.
The macro \texttt{\_\_risc\_\_} is predefined in order to enable programmers to identify this tool and its target architecture while compiling.
Programs generated with this compiler require additional runtime support that is stored in the \file{cpp\-risc\-run} library file.
\flowgraph{\resource{\cpp{}\\source code} \ar[r] & \toolbox{cpprisc} \ar[r] \ar[d] \ar[rd] & \resource{object file} \\ \variable{ECSINCLUDE} \ar[ru] & \resource{debugging\\information} & \resource{assembly\\listing}}
\seecpp\seeassembly\seerisc\seeobject\seedebugging
}

\providecommand{\cppwasm}{
\toolsection{cppwasm} is a compiler for the \cpp{} programming language targeting the WebAssembly architecture.
It generates machine code for WebAssembly targets from programs written in \cpp{} and stores it in corresponding object files.
For debugging purposes, it also creates a debugging information file as well as an assembly file containing a listing of the generated machine code.
The macro \texttt{\_\_wasm\_\_} is predefined in order to enable programmers to identify this tool and its target architecture while compiling.
Programs generated with this compiler require additional runtime support that is stored in the \file{cpp\-wasm\-run} library file.
\flowgraph{\resource{\cpp{}\\source code} \ar[r] & \toolbox{cppwasm} \ar[r] \ar[d] \ar[rd] & \resource{object file} \\ \variable{ECSINCLUDE} \ar[ru] & \resource{debugging\\information} & \resource{assembly\\listing}}
\seecpp\seeassembly\seewasm\seeobject\seedebugging
}

% FALSE tools

\providecommand{\falprint}{
\toolsection{falprint} is a pretty printer for the FALSE programming language.
It reformats the source code of FALSE programs and writes it to the standard output stream.
\flowgraph{\resource{FALSE\\source code} \ar[r] & \toolbox{falprint} \ar[r] & \resource{reformatted\\source code}}
\seefalse
}

\providecommand{\falcheck}{
\toolsection{falcheck} is a syntactic and semantic checker for the FALSE programming language.
It just performs syntactic and semantic checks on FALSE programs and writes its diagnostic messages to the standard error stream.
\flowgraph{\resource{FALSE\\source code} \ar[r] & \toolbox{falcheck} \ar[r] & \resource{diagnostic\\messages}}
\seefalse
}

\providecommand{\faldump}{
\toolsection{faldump} is a serializer for the FALSE programming language.
It dumps the complete internal representation of programs written in FALSE into an XML document.
\debuggingtool
\flowgraph{\resource{FALSE\\source code} \ar[r] & \toolbox{faldump} \ar[r] & \resource{internal\\representation}}
\seefalse
}

\providecommand{\falrun}{
\toolsection{falrun} is an interpreter for the FALSE programming language.
It processes and executes programs written in FALSE\@.
\flowgraph{\resource{FALSE\\source code} \ar[r] & \toolbox{falrun} \ar@/u/[r] & \resource{input/\\output} \ar@/d/[l]}
\seefalse
}

\providecommand{\falcpp}{
\toolsection{falcpp} is a transpiler for the FALSE programming language.
It translates programs written in FALSE into \cpp{} programs and stores them in corresponding source files.
\flowgraph{\resource{FALSE\\source code} \ar[r] & \toolbox{falcpp} \ar[r] & \resource{\cpp{}\\source file}}
\seefalse\seecpp
}

\providecommand{\falcode}{
\toolsection{falcode} is an intermediate code generator for the FALSE programming language.
It generates intermediate code from programs written in FALSE and stores it in corresponding assembly files.
\debuggingtool
\flowgraph{\resource{FALSE\\source code} \ar[r] & \toolbox{falcode} \ar[r] & \resource{intermediate\\code}}
\seefalse\seeassembly\seecode
}

\providecommand{\falamda}{
\toolsection{falamd16} is a compiler for the FALSE programming language targeting the AMD64 hardware architecture.
It generates machine code for AMD64 processors from programs written in FALSE and stores it in corresponding object files.
The compiler generates machine code for the 16-bit operating mode defined by the AMD64 architecture.
\flowgraph{\resource{FALSE\\source code} \ar[r] & \toolbox{falamd16} \ar[r] & \resource{object file}}
\seefalse\seeamd\seeobject
}

\providecommand{\falamdb}{
\toolsection{falamd32} is a compiler for the FALSE programming language targeting the AMD64 hardware architecture.
It generates machine code for AMD64 processors from programs written in FALSE and stores it in corresponding object files.
The compiler generates machine code for the 32-bit operating mode defined by the AMD64 architecture.
\flowgraph{\resource{FALSE\\source code} \ar[r] & \toolbox{falamd32} \ar[r] & \resource{object file}}
\seefalse\seeamd\seeobject
}

\providecommand{\falamdc}{
\toolsection{falamd64} is a compiler for the FALSE programming language targeting the AMD64 hardware architecture.
It generates machine code for AMD64 processors from programs written in FALSE and stores it in corresponding object files.
The compiler generates machine code for the 64-bit operating mode defined by the AMD64 architecture.
\flowgraph{\resource{FALSE\\source code} \ar[r] & \toolbox{falamd64} \ar[r] & \resource{object file}}
\seefalse\seeamd\seeobject
}

\providecommand{\falarma}{
\toolsection{falarma32} is a compiler for the FALSE programming language targeting the ARM hardware architecture.
It generates machine code for ARM processors executing A32 instructions from programs written in FALSE and stores it in corresponding object files.
\flowgraph{\resource{FALSE\\source code} \ar[r] & \toolbox{falarma32} \ar[r] & \resource{object file}}
\seefalse\seearm\seeobject
}

\providecommand{\falarmb}{
\toolsection{falarma64} is a compiler for the FALSE programming language targeting the ARM hardware architecture.
It generates machine code for ARM processors executing A64 instructions from programs written in FALSE and stores it in corresponding object files.
\flowgraph{\resource{FALSE\\source code} \ar[r] & \toolbox{falarma64} \ar[r] & \resource{object file}}
\seefalse\seearm\seeobject
}

\providecommand{\falarmc}{
\toolsection{falarmt32} is a compiler for the FALSE programming language targeting the ARM hardware architecture.
It generates machine code for ARM processors without floating-point extension executing T32 instructions from programs written in FALSE and stores it in corresponding object files.
\flowgraph{\resource{FALSE\\source code} \ar[r] & \toolbox{falarmt32} \ar[r] & \resource{object file}}
\seefalse\seearm\seeobject
}

\providecommand{\falarmcfpe}{
\toolsection{falarmt32fpe} is a compiler for the FALSE programming language targeting the ARM hardware architecture.
It generates machine code for ARM processors with floating-point extension executing T32 instructions from programs written in FALSE and stores it in corresponding object files.
\flowgraph{\resource{FALSE\\source code} \ar[r] & \toolbox{falarmt32fpe} \ar[r] & \resource{object file}}
\seefalse\seearm\seeobject
}

\providecommand{\falavr}{
\toolsection{falavr} is a compiler for the FALSE programming language targeting the AVR hardware architecture.
It generates machine code for AVR processors from programs written in FALSE and stores it in corresponding object files.
\flowgraph{\resource{FALSE\\source code} \ar[r] & \toolbox{falavr} \ar[r] & \resource{object file}}
\seefalse\seeavr\seeobject
}

\providecommand{\falavrtt}{
\toolsection{falavr32} is a compiler for the FALSE programming language targeting the AVR32 hardware architecture.
It generates machine code for AVR32 processors from programs written in FALSE and stores it in corresponding object files.
\flowgraph{\resource{FALSE\\source code} \ar[r] & \toolbox{falavr32} \ar[r] & \resource{object file}}
\seefalse\seeavrtt\seeobject
}

\providecommand{\falmabk}{
\toolsection{falm68k} is a compiler for the FALSE programming language targeting the M68000 hardware architecture.
It generates machine code for M68000 processors from programs written in FALSE and stores it in corresponding object files.
\flowgraph{\resource{FALSE\\source code} \ar[r] & \toolbox{falm68k} \ar[r] & \resource{object file}}
\seefalse\seemabk\seeobject
}

\providecommand{\falmibl}{
\toolsection{falmibl} is a compiler for the FALSE programming language targeting the MicroBlaze hardware architecture.
It generates machine code for MicroBlaze processors from programs written in FALSE and stores it in corresponding object files.
\flowgraph{\resource{FALSE\\source code} \ar[r] & \toolbox{falmibl} \ar[r] & \resource{object file}}
\seefalse\seemibl\seeobject
}

\providecommand{\falmipsa}{
\toolsection{falmips32} is a compiler for the FALSE programming language targeting the MIPS32 hardware architecture.
It generates machine code for MIPS32 processors from programs written in FALSE and stores it in corresponding object files.
\flowgraph{\resource{FALSE\\source code} \ar[r] & \toolbox{falmips32} \ar[r] & \resource{object file}}
\seefalse\seemips\seeobject
}

\providecommand{\falmipsb}{
\toolsection{falmips64} is a compiler for the FALSE programming language targeting the MIPS64 hardware architecture.
It generates machine code for MIPS64 processors from programs written in FALSE and stores it in corresponding object files.
\flowgraph{\resource{FALSE\\source code} \ar[r] & \toolbox{falmips64} \ar[r] & \resource{object file}}
\seefalse\seemips\seeobject
}

\providecommand{\falmmix}{
\toolsection{falmmix} is a compiler for the FALSE programming language targeting the MMIX hardware architecture.
It generates machine code for MMIX processors from programs written in FALSE and stores it in corresponding object files.
\flowgraph{\resource{FALSE\\source code} \ar[r] & \toolbox{falmmix} \ar[r] & \resource{object file}}
\seefalse\seemmix\seeobject
}

\providecommand{\falorok}{
\toolsection{falor1k} is a compiler for the FALSE programming language targeting the OpenRISC 1000 hardware architecture.
It generates machine code for OpenRISC 1000 processors from programs written in FALSE and stores it in corresponding object files.
\flowgraph{\resource{FALSE\\source code} \ar[r] & \toolbox{falor1k} \ar[r] & \resource{object file}}
\seefalse\seeorok\seeobject
}

\providecommand{\falppca}{
\toolsection{falppc32} is a compiler for the FALSE programming language targeting the PowerPC hardware architecture.
It generates machine code for PowerPC processors from programs written in FALSE and stores it in corresponding object files.
The compiler generates machine code for the 32-bit operating mode defined by the PowerPC architecture.
\flowgraph{\resource{FALSE\\source code} \ar[r] & \toolbox{falppc32} \ar[r] & \resource{object file}}
\seefalse\seeppc\seeobject
}

\providecommand{\falppcb}{
\toolsection{falppc64} is a compiler for the FALSE programming language targeting the PowerPC hardware architecture.
It generates machine code for PowerPC processors from programs written in FALSE and stores it in corresponding object files.
The compiler generates machine code for the 64-bit operating mode defined by the PowerPC architecture.
\flowgraph{\resource{FALSE\\source code} \ar[r] & \toolbox{falppc64} \ar[r] & \resource{object file}}
\seefalse\seeppc\seeobject
}

\providecommand{\falrisc}{
\toolsection{falrisc} is a compiler for the FALSE programming language targeting the RISC hardware architecture.
It generates machine code for RISC processors from programs written in FALSE and stores it in corresponding object files.
\flowgraph{\resource{FALSE\\source code} \ar[r] & \toolbox{falrisc} \ar[r] & \resource{object file}}
\seefalse\seerisc\seeobject
}

\providecommand{\falwasm}{
\toolsection{falwasm} is a compiler for the FALSE programming language targeting the WebAssembly architecture.
It generates machine code for WebAssembly targets from programs written in FALSE and stores it in corresponding object files.
\flowgraph{\resource{FALSE\\source code} \ar[r] & \toolbox{falwasm} \ar[r] & \resource{object file}}
\seefalse\seewasm\seeobject
}

% Oberon tools

\providecommand{\obprint}{
\toolsection{obprint} is a pretty printer for the Oberon programming language.
It reformats the source code of Oberon modules and writes it to the standard output stream.
\flowgraph{\resource{Oberon\\source code} \ar[r] & \toolbox{obprint} \ar[r] & \resource{reformatted\\source code}}
\seeoberon
}

\providecommand{\obcheck}{
\toolsection{obcheck} is a syntactic and semantic checker for the Oberon programming language.
It just performs syntactic and semantic checks on Oberon modules and writes its diagnostic messages to the standard error stream.
In addition, it stores the interface of each module in a symbol file which is required when other modules import the module.
\flowgraph{\resource{Oberon\\source code} \ar[r] & \toolbox{obcheck} \ar[r] \ar@/l/[d] & \resource{diagnostic\\messages} \\ \variable{ECSIMPORT} \ar[ru] & \resource{symbol\\files} \ar@/r/[u]}
\seeoberon
}

\providecommand{\obdump}{
\toolsection{obdump} is a serializer for the Oberon programming language.
It dumps the complete internal representation of modules written in Oberon into an XML document.
\debuggingtool
\flowgraph{\resource{Oberon\\source code} \ar[r] & \toolbox{obdump} \ar[r] \ar@/l/[d] & \resource{internal\\representation} \\ \variable{ECSIMPORT} \ar[ru] & \resource{symbol\\files} \ar@/r/[u]}
\seeoberon
}

\providecommand{\obrun}{
\toolsection{obrun} is an interpreter for the Oberon programming language.
It processes and executes modules written in Oberon.
This tool does neither generate nor process symbol files while interpreting modules.
If a module is imported by another one, its filename has to be named before the other one in the list of command-line arguments.
\flowgraph{\resource{Oberon\\source code} \ar[r] & \toolbox{obrun} \ar@/u/[r] & \resource{input/\\output} \ar@/d/[l]}
\seeoberon
}

\providecommand{\obcpp}{
\toolsection{obcpp} is a transpiler for the Oberon programming language.
It translates programs written in Oberon into \cpp{} programs and stores them in corresponding source and header files.
In addition, it stores the interface of each module in a symbol file which is required when other modules import the module.
The same interface is provided by the generated header file which can be used in other parts of the \cpp{} program.
\flowgraph{\resource{Oberon\\source code} \ar[r] & \toolbox{obcpp} \ar[r] \ar@/l/[d] \ar[rd] & \resource{\cpp{}\\source file} \\ \variable{ECSIMPORT} \ar[ru] & \resource{symbol\\files} \ar@/r/[u] & \resource{\cpp{}\\header file}}
\seeoberon\seecpp
}

\providecommand{\obdoc}{
\toolsection{obdoc} is a generic documentation generator for the Oberon programming language.
It processes several Oberon modules and assembles all information therein into a generic documentation.
In addition, it stores the interface of each module in a symbol file which is required when other modules import the module.
\debuggingtool
\flowgraph{\resource{Oberon\\source code} \ar[r] & \toolbox{obdoc} \ar[r] \ar@/l/[d] & \resource{generic\\documentation} \\ \variable{ECSIMPORT} \ar[ru] & \resource{symbol\\files} \ar@/r/[u]}
\seeoberon\seedocumentation
}

\providecommand{\obhtml}{
\toolsection{obhtml} is an HTML documentation generator for the Oberon programming language.
It processes several Oberon modules and assembles all information therein into an HTML document.
In addition, it stores the interface of each module in a symbol file which is required when other modules import the module.
\flowgraph{\resource{Oberon\\source code} \ar[r] & \toolbox{obhtml} \ar[r] \ar@/l/[d] & \resource{HTML\\document} \\ \variable{ECSIMPORT} \ar[ru] & \resource{symbol\\files} \ar@/r/[u]}
\seeoberon\seedocumentation
}

\providecommand{\oblatex}{
\toolsection{oblatex} is a Latex documentation generator for the Oberon programming language.
It processes several Oberon modules and assembles all information therein into a Latex document.
In addition, it stores the interface of each module in a symbol file which is required when other modules import the module.
\flowgraph{\resource{Oberon\\source code} \ar[r] & \toolbox{oblatex} \ar[r] \ar@/l/[d] & \resource{Latex\\document} \\ \variable{ECSIMPORT} \ar[ru] & \resource{symbol\\files} \ar@/r/[u]}
\seeoberon\seedocumentation
}

\providecommand{\obcode}{
\toolsection{obcode} is an intermediate code generator for the Oberon programming language.
It generates intermediate code from modules written in Oberon and stores it in corresponding assembly files.
In addition, it stores the interface of each module in a symbol file which is required when other modules import the module.
Programs generated with this tool require additional runtime support that is stored in the \file{ob\-code\-run} library file.
\debuggingtool
\flowgraph{\resource{Oberon\\source code} \ar[r] & \toolbox{obcode} \ar[r] \ar@/l/[d] & \resource{intermediate\\code} \\ \variable{ECSIMPORT} \ar[ru] & \resource{symbol\\files} \ar@/r/[u]}
\seeoberon\seeassembly\seecode
}

\providecommand{\obamda}{
\toolsection{obamd16} is a compiler for the Oberon programming language targeting the AMD64 hardware architecture.
It generates machine code for AMD64 processors from modules written in Oberon and stores it in corresponding object files.
The compiler generates machine code for the 16-bit operating mode defined by the AMD64 architecture.
For debugging purposes, it also creates a debugging information file as well as an assembly file containing a listing of the generated machine code.
In addition, it stores the interface of each module in a symbol file which is required when other modules import the module.
Programs generated with this compiler require additional runtime support that is stored in the \file{ob\-amd16\-run} library file.
\flowgraph{\resource{Oberon\\source code} \ar[r] & \toolbox{obamd16} \ar[r] \ar@/l/[d] \ar[rd] & \resource{object file} \\ \variable{ECSIMPORT} \ar[ru] & \resource{symbol\\files} \ar@/r/[u] & \resource{debugging\\information}}
\seeoberon\seeassembly\seeamd\seeobject\seedebugging
}

\providecommand{\obamdb}{
\toolsection{obamd32} is a compiler for the Oberon programming language targeting the AMD64 hardware architecture.
It generates machine code for AMD64 processors from modules written in Oberon and stores it in corresponding object files.
The compiler generates machine code for the 32-bit operating mode defined by the AMD64 architecture.
For debugging purposes, it also creates a debugging information file as well as an assembly file containing a listing of the generated machine code.
In addition, it stores the interface of each module in a symbol file which is required when other modules import the module.
Programs generated with this compiler require additional runtime support that is stored in the \file{ob\-amd32\-run} library file.
\flowgraph{\resource{Oberon\\source code} \ar[r] & \toolbox{obamd32} \ar[r] \ar@/l/[d] \ar[rd] & \resource{object file} \\ \variable{ECSIMPORT} \ar[ru] & \resource{symbol\\files} \ar@/r/[u] & \resource{debugging\\information}}
\seeoberon\seeassembly\seeamd\seeobject\seedebugging
}

\providecommand{\obamdc}{
\toolsection{obamd64} is a compiler for the Oberon programming language targeting the AMD64 hardware architecture.
It generates machine code for AMD64 processors from modules written in Oberon and stores it in corresponding object files.
The compiler generates machine code for the 64-bit operating mode defined by the AMD64 architecture.
For debugging purposes, it also creates a debugging information file as well as an assembly file containing a listing of the generated machine code.
In addition, it stores the interface of each module in a symbol file which is required when other modules import the module.
Programs generated with this compiler require additional runtime support that is stored in the \file{ob\-amd64\-run} library file.
\flowgraph{\resource{Oberon\\source code} \ar[r] & \toolbox{obamd64} \ar[r] \ar@/l/[d] \ar[rd] & \resource{object file} \\ \variable{ECSIMPORT} \ar[ru] & \resource{symbol\\files} \ar@/r/[u] & \resource{debugging\\information}}
\seeoberon\seeassembly\seeamd\seeobject\seedebugging
}

\providecommand{\obarma}{
\toolsection{obarma32} is a compiler for the Oberon programming language targeting the ARM hardware architecture.
It generates machine code for ARM processors executing A32 instructions from modules written in Oberon and stores it in corresponding object files.
For debugging purposes, it also creates a debugging information file as well as an assembly file containing a listing of the generated machine code.
In addition, it stores the interface of each module in a symbol file which is required when other modules import the module.
Programs generated with this compiler require additional runtime support that is stored in the \file{ob\-arma32\-run} library file.
\flowgraph{\resource{Oberon\\source code} \ar[r] & \toolbox{obarma32} \ar[r] \ar@/l/[d] \ar[rd] & \resource{object file} \\ \variable{ECSIMPORT} \ar[ru] & \resource{symbol\\files} \ar@/r/[u] & \resource{debugging\\information}}
\seeoberon\seeassembly\seearm\seeobject\seedebugging
}

\providecommand{\obarmb}{
\toolsection{obarma64} is a compiler for the Oberon programming language targeting the ARM hardware architecture.
It generates machine code for ARM processors executing A64 instructions from modules written in Oberon and stores it in corresponding object files.
For debugging purposes, it also creates a debugging information file as well as an assembly file containing a listing of the generated machine code.
In addition, it stores the interface of each module in a symbol file which is required when other modules import the module.
Programs generated with this compiler require additional runtime support that is stored in the \file{ob\-arma64\-run} library file.
\flowgraph{\resource{Oberon\\source code} \ar[r] & \toolbox{obarma64} \ar[r] \ar@/l/[d] \ar[rd] & \resource{object file} \\ \variable{ECSIMPORT} \ar[ru] & \resource{symbol\\files} \ar@/r/[u] & \resource{debugging\\information}}
\seeoberon\seeassembly\seearm\seeobject\seedebugging
}

\providecommand{\obarmc}{
\toolsection{obarmt32} is a compiler for the Oberon programming language targeting the ARM hardware architecture.
It generates machine code for ARM processors without floating-point extension executing T32 instructions from modules written in Oberon and stores it in corresponding object files.
For debugging purposes, it also creates a debugging information file as well as an assembly file containing a listing of the generated machine code.
In addition, it stores the interface of each module in a symbol file which is required when other modules import the module.
Programs generated with this compiler require additional runtime support that is stored in the \file{ob\-armt32\-run} library file.
\flowgraph{\resource{Oberon\\source code} \ar[r] & \toolbox{obarmt32} \ar[r] \ar@/l/[d] \ar[rd] & \resource{object file} \\ \variable{ECSIMPORT} \ar[ru] & \resource{symbol\\files} \ar@/r/[u] & \resource{debugging\\information}}
\seeoberon\seeassembly\seearm\seeobject\seedebugging
}

\providecommand{\obarmcfpe}{
\toolsection{obarmt32fpe} is a compiler for the Oberon programming language targeting the ARM hardware architecture.
It generates machine code for ARM processors with floating-point extension executing T32 instructions from modules written in Oberon and stores it in corresponding object files.
For debugging purposes, it also creates a debugging information file as well as an assembly file containing a listing of the generated machine code.
In addition, it stores the interface of each module in a symbol file which is required when other modules import the module.
Programs generated with this compiler require additional runtime support that is stored in the \file{ob\-armt32\-fpe\-run} library file.
\flowgraph{\resource{Oberon\\source code} \ar[r] & \toolbox{obarmt32fpe} \ar[r] \ar@/l/[d] \ar[rd] & \resource{object file} \\ \variable{ECSIMPORT} \ar[ru] & \resource{symbol\\files} \ar@/r/[u] & \resource{debugging\\information}}
\seeoberon\seeassembly\seearm\seeobject\seedebugging
}

\providecommand{\obavr}{
\toolsection{obavr} is a compiler for the Oberon programming language targeting the AVR hardware architecture.
It generates machine code for AVR processors from modules written in Oberon and stores it in corresponding object files.
For debugging purposes, it also creates a debugging information file as well as an assembly file containing a listing of the generated machine code.
In addition, it stores the interface of each module in a symbol file which is required when other modules import the module.
Programs generated with this compiler require additional runtime support that is stored in the \file{ob\-avr\-run} library file.
\flowgraph{\resource{Oberon\\source code} \ar[r] & \toolbox{obavr} \ar[r] \ar@/l/[d] \ar[rd] & \resource{object file} \\ \variable{ECSIMPORT} \ar[ru] & \resource{symbol\\files} \ar@/r/[u] & \resource{debugging\\information}}
\seeoberon\seeassembly\seeavr\seeobject\seedebugging
}

\providecommand{\obavrtt}{
\toolsection{obavr32} is a compiler for the Oberon programming language targeting the AVR32 hardware architecture.
It generates machine code for AVR32 processors from modules written in Oberon and stores it in corresponding object files.
For debugging purposes, it also creates a debugging information file as well as an assembly file containing a listing of the generated machine code.
In addition, it stores the interface of each module in a symbol file which is required when other modules import the module.
Programs generated with this compiler require additional runtime support that is stored in the \file{ob\-avr32\-run} library file.
\flowgraph{\resource{Oberon\\source code} \ar[r] & \toolbox{obavr32} \ar[r] \ar@/l/[d] \ar[rd] & \resource{object file} \\ \variable{ECSIMPORT} \ar[ru] & \resource{symbol\\files} \ar@/r/[u] & \resource{debugging\\information}}
\seeoberon\seeassembly\seeavrtt\seeobject\seedebugging
}

\providecommand{\obmabk}{
\toolsection{obm68k} is a compiler for the Oberon programming language targeting the M68000 hardware architecture.
It generates machine code for M68000 processors from modules written in Oberon and stores it in corresponding object files.
For debugging purposes, it also creates a debugging information file as well as an assembly file containing a listing of the generated machine code.
In addition, it stores the interface of each module in a symbol file which is required when other modules import the module.
Programs generated with this compiler require additional runtime support that is stored in the \file{ob\-m68k\-run} library file.
\flowgraph{\resource{Oberon\\source code} \ar[r] & \toolbox{obm68k} \ar[r] \ar@/l/[d] \ar[rd] & \resource{object file} \\ \variable{ECSIMPORT} \ar[ru] & \resource{symbol\\files} \ar@/r/[u] & \resource{debugging\\information}}
\seeoberon\seeassembly\seemabk\seeobject\seedebugging
}

\providecommand{\obmibl}{
\toolsection{obmibl} is a compiler for the Oberon programming language targeting the MicroBlaze hardware architecture.
It generates machine code for MicroBlaze processors from modules written in Oberon and stores it in corresponding object files.
For debugging purposes, it also creates a debugging information file as well as an assembly file containing a listing of the generated machine code.
In addition, it stores the interface of each module in a symbol file which is required when other modules import the module.
Programs generated with this compiler require additional runtime support that is stored in the \file{ob\-mibl\-run} library file.
\flowgraph{\resource{Oberon\\source code} \ar[r] & \toolbox{obmibl} \ar[r] \ar@/l/[d] \ar[rd] & \resource{object file} \\ \variable{ECSIMPORT} \ar[ru] & \resource{symbol\\files} \ar@/r/[u] & \resource{debugging\\information}}
\seeoberon\seeassembly\seemibl\seeobject\seedebugging
}

\providecommand{\obmipsa}{
\toolsection{obmips32} is a compiler for the Oberon programming language targeting the MIPS32 hardware architecture.
It generates machine code for MIPS32 processors from modules written in Oberon and stores it in corresponding object files.
For debugging purposes, it also creates a debugging information file as well as an assembly file containing a listing of the generated machine code.
In addition, it stores the interface of each module in a symbol file which is required when other modules import the module.
Programs generated with this compiler require additional runtime support that is stored in the \file{ob\-mips32\-run} library file.
\flowgraph{\resource{Oberon\\source code} \ar[r] & \toolbox{obmips32} \ar[r] \ar@/l/[d] \ar[rd] & \resource{object file} \\ \variable{ECSIMPORT} \ar[ru] & \resource{symbol\\files} \ar@/r/[u] & \resource{debugging\\information}}
\seeoberon\seeassembly\seemips\seeobject\seedebugging
}

\providecommand{\obmipsb}{
\toolsection{obmips64} is a compiler for the Oberon programming language targeting the MIPS64 hardware architecture.
It generates machine code for MIPS64 processors from modules written in Oberon and stores it in corresponding object files.
For debugging purposes, it also creates a debugging information file as well as an assembly file containing a listing of the generated machine code.
In addition, it stores the interface of each module in a symbol file which is required when other modules import the module.
Programs generated with this compiler require additional runtime support that is stored in the \file{ob\-mips64\-run} library file.
\flowgraph{\resource{Oberon\\source code} \ar[r] & \toolbox{obmips64} \ar[r] \ar@/l/[d] \ar[rd] & \resource{object file} \\ \variable{ECSIMPORT} \ar[ru] & \resource{symbol\\files} \ar@/r/[u] & \resource{debugging\\information}}
\seeoberon\seeassembly\seemips\seeobject\seedebugging
}

\providecommand{\obmmix}{
\toolsection{obmmix} is a compiler for the Oberon programming language targeting the MMIX hardware architecture.
It generates machine code for MMIX processors from modules written in Oberon and stores it in corresponding object files.
For debugging purposes, it also creates a debugging information file as well as an assembly file containing a listing of the generated machine code.
In addition, it stores the interface of each module in a symbol file which is required when other modules import the module.
Programs generated with this compiler require additional runtime support that is stored in the \file{ob\-mmix\-run} library file.
\flowgraph{\resource{Oberon\\source code} \ar[r] & \toolbox{obmmix} \ar[r] \ar@/l/[d] \ar[rd] & \resource{object file} \\ \variable{ECSIMPORT} \ar[ru] & \resource{symbol\\files} \ar@/r/[u] & \resource{debugging\\information}}
\seeoberon\seeassembly\seemmix\seeobject\seedebugging
}

\providecommand{\oborok}{
\toolsection{obor1k} is a compiler for the Oberon programming language targeting the OpenRISC 1000 hardware architecture.
It generates machine code for OpenRISC 1000 processors from modules written in Oberon and stores it in corresponding object files.
For debugging purposes, it also creates a debugging information file as well as an assembly file containing a listing of the generated machine code.
In addition, it stores the interface of each module in a symbol file which is required when other modules import the module.
Programs generated with this compiler require additional runtime support that is stored in the \file{ob\-or1k\-run} library file.
\flowgraph{\resource{Oberon\\source code} \ar[r] & \toolbox{obor1k} \ar[r] \ar@/l/[d] \ar[rd] & \resource{object file} \\ \variable{ECSIMPORT} \ar[ru] & \resource{symbol\\files} \ar@/r/[u] & \resource{debugging\\information}}
\seeoberon\seeassembly\seeorok\seeobject\seedebugging
}

\providecommand{\obppca}{
\toolsection{obppc32} is a compiler for the Oberon programming language targeting the PowerPC hardware architecture.
It generates machine code for PowerPC processors from modules written in Oberon and stores it in corresponding object files.
The compiler generates machine code for the 32-bit operating mode defined by the PowerPC architecture.
For debugging purposes, it also creates a debugging information file as well as an assembly file containing a listing of the generated machine code.
In addition, it stores the interface of each module in a symbol file which is required when other modules import the module.
Programs generated with this compiler require additional runtime support that is stored in the \file{ob\-ppc32\-run} library file.
\flowgraph{\resource{Oberon\\source code} \ar[r] & \toolbox{obppc32} \ar[r] \ar@/l/[d] \ar[rd] & \resource{object file} \\ \variable{ECSIMPORT} \ar[ru] & \resource{symbol\\files} \ar@/r/[u] & \resource{debugging\\information}}
\seeoberon\seeassembly\seeppc\seeobject\seedebugging
}

\providecommand{\obppcb}{
\toolsection{obppc64} is a compiler for the Oberon programming language targeting the PowerPC hardware architecture.
It generates machine code for PowerPC processors from modules written in Oberon and stores it in corresponding object files.
The compiler generates machine code for the 64-bit operating mode defined by the PowerPC architecture.
For debugging purposes, it also creates a debugging information file as well as an assembly file containing a listing of the generated machine code.
In addition, it stores the interface of each module in a symbol file which is required when other modules import the module.
Programs generated with this compiler require additional runtime support that is stored in the \file{ob\-ppc64\-run} library file.
\flowgraph{\resource{Oberon\\source code} \ar[r] & \toolbox{obppc64} \ar[r] \ar@/l/[d] \ar[rd] & \resource{object file} \\ \variable{ECSIMPORT} \ar[ru] & \resource{symbol\\files} \ar@/r/[u] & \resource{debugging\\information}}
\seeoberon\seeassembly\seeppc\seeobject\seedebugging
}

\providecommand{\obrisc}{
\toolsection{obrisc} is a compiler for the Oberon programming language targeting the RISC hardware architecture.
It generates machine code for RISC processors from modules written in Oberon and stores it in corresponding object files.
For debugging purposes, it also creates a debugging information file as well as an assembly file containing a listing of the generated machine code.
In addition, it stores the interface of each module in a symbol file which is required when other modules import the module.
Programs generated with this compiler require additional runtime support that is stored in the \file{ob\-risc\-run} library file.
\flowgraph{\resource{Oberon\\source code} \ar[r] & \toolbox{obrisc} \ar[r] \ar@/l/[d] \ar[rd] & \resource{object file} \\ \variable{ECSIMPORT} \ar[ru] & \resource{symbol\\files} \ar@/r/[u] & \resource{debugging\\information}}
\seeoberon\seeassembly\seerisc\seeobject\seedebugging
}

\providecommand{\obwasm}{
\toolsection{obwasm} is a compiler for the Oberon programming language targeting the WebAssembly architecture.
It generates machine code for WebAssembly targets from modules written in Oberon and stores it in corresponding object files.
For debugging purposes, it also creates a debugging information file as well as an assembly file containing a listing of the generated machine code.
In addition, it stores the interface of each module in a symbol file which is required when other modules import the module.
Programs generated with this compiler require additional runtime support that is stored in the \file{ob\-wasm\-run} library file.
\flowgraph{\resource{Oberon\\source code} \ar[r] & \toolbox{obwasm} \ar[r] \ar@/l/[d] \ar[rd] & \resource{object file} \\ \variable{ECSIMPORT} \ar[ru] & \resource{symbol\\files} \ar@/r/[u] & \resource{debugging\\information}}
\seeoberon\seeassembly\seewasm\seeobject\seedebugging
}

% converter tools

\providecommand{\dbgdwarf}{
\toolsection{dbgdwarf} is a DWARF debugging information converter tool.
It converts debugging information into the DWARF debugging data format and stores it in corresponding object files~\cite{dwarffile}.
The resulting debugging object files can be combined with runtime support that creates Executable and Linking Format (ELF) files~\cite{elffile}.
\flowgraph{\resource{debugging\\information} \ar[r] & \toolbox{dbgdwarf} \ar[r] & \resource{debugging\\object file}}
\seeobject\seedebugging
}

% assembler tools

\providecommand{\asmprint}{
\toolsection{asmprint} is a pretty printer for generic assembly code.
It reformats generic assembly code and writes it to the standard output stream.
\flowgraph{\resource{generic assembly\\source code} \ar[r] & \toolbox{asmprint} \ar[r] & \resource{reformatted\\source code}}
\seeassembly
}

\providecommand{\amdaasm}{
\toolsection{amd16asm} is an assembler for the AMD64 hardware architecture.
It translates assembly code into machine code for AMD64 processors and stores it in corresponding object files.
By default, the assembler generates machine code for the 16-bit operating mode defined by the AMD64 architecture.
\flowgraph{\resource{AMD16 assembly\\source code} \ar[r] & \toolbox{amd16asm} \ar[r] & \resource{object file}}
\seeassembly\seeamd\seeobject
}

\providecommand{\amdadism}{
\toolsection{amd16dism} is a disassembler for the AMD64 hardware architecture.
It translates machine code from object files targeting AMD64 processors into assembly code and writes it to the standard output stream.
It assumes that the machine code was generated for the 16-bit operating mode defined by the AMD64 architecture.
\flowgraph{\resource{object file} \ar[r] & \toolbox{amd16dism} \ar[r] & \resource{disassembly\\listing}}
\seeassembly\seeamd\seeobject
}

\providecommand{\amdbasm}{
\toolsection{amd32asm} is an assembler for the AMD64 hardware architecture.
It translates assembly code into machine code for AMD64 processors and stores it in corresponding object files.
By default, the assembler generates machine code for the 32-bit operating mode defined by the AMD64 architecture.
\flowgraph{\resource{AMD32 assembly\\source code} \ar[r] & \toolbox{amd32asm} \ar[r] & \resource{object file}}
\seeassembly\seeamd\seeobject
}

\providecommand{\amdbdism}{
\toolsection{amd32dism} is a disassembler for the AMD64 hardware architecture.
It translates machine code from object files targeting AMD64 processors into assembly code and writes it to the standard output stream.
It assumes that the machine code was generated for the 32-bit operating mode defined by the AMD64 architecture.
\flowgraph{\resource{object file} \ar[r] & \toolbox{amd32dism} \ar[r] & \resource{disassembly\\listing}}
\seeassembly\seeamd\seeobject
}

\providecommand{\amdcasm}{
\toolsection{amd64asm} is an assembler for the AMD64 hardware architecture.
It translates assembly code into machine code for AMD64 processors and stores it in corresponding object files.
By default, the assembler generates machine code for the 64-bit operating mode defined by the AMD64 architecture.
\flowgraph{\resource{AMD64 assembly\\source code} \ar[r] & \toolbox{amd64asm} \ar[r] & \resource{object file}}
\seeassembly\seeamd\seeobject
}

\providecommand{\amdcdism}{
\toolsection{amd64dism} is a disassembler for the AMD64 hardware architecture.
It translates machine code from object files targeting AMD64 processors into assembly code and writes it to the standard output stream.
It assumes that the machine code was generated for the 64-bit operating mode defined by the AMD64 architecture.
\flowgraph{\resource{object file} \ar[r] & \toolbox{amd64dism} \ar[r] & \resource{disassembly\\listing}}
\seeassembly\seeamd\seeobject
}

\providecommand{\armaasm}{
\toolsection{arma32asm} is an assembler for the ARM hardware architecture.
It translates assembly code into machine code for ARM processors executing A32 instructions and stores it in corresponding object files.
\flowgraph{\resource{ARM A32 assembly\\source code} \ar[r] & \toolbox{arma32asm} \ar[r] & \resource{object file}}
\seeassembly\seearm\seeobject
}

\providecommand{\armadism}{
\toolsection{arma32dism} is a disassembler for the ARM hardware architecture.
It translates machine code from object files targeting ARM processors executing A32 instructions into assembly code and writes it to the standard output stream.
\flowgraph{\resource{object file} \ar[r] & \toolbox{arma32dism} \ar[r] & \resource{disassembly\\listing}}
\seeassembly\seearm\seeobject
}

\providecommand{\armbasm}{
\toolsection{arma64asm} is an assembler for the ARM hardware architecture.
It translates assembly code into machine code for ARM processors executing A64 instructions and stores it in corresponding object files.
\flowgraph{\resource{ARM A64 assembly\\source code} \ar[r] & \toolbox{arma64asm} \ar[r] & \resource{object file}}
\seeassembly\seearm\seeobject
}

\providecommand{\armbdism}{
\toolsection{arma64dism} is a disassembler for the ARM hardware architecture.
It translates machine code from object files targeting ARM processors executing A64 instructions into assembly code and writes it to the standard output stream.
\flowgraph{\resource{object file} \ar[r] & \toolbox{arma64dism} \ar[r] & \resource{disassembly\\listing}}
\seeassembly\seearm\seeobject
}

\providecommand{\armcasm}{
\toolsection{armt32asm} is an assembler for the ARM hardware architecture.
It translates assembly code into machine code for ARM processors executing T32 instructions and stores it in corresponding object files.
\flowgraph{\resource{ARM T32 assembly\\source code} \ar[r] & \toolbox{armt32asm} \ar[r] & \resource{object file}}
\seeassembly\seearm\seeobject
}

\providecommand{\armcdism}{
\toolsection{armt32dism} is a disassembler for the ARM hardware architecture.
It translates machine code from object files targeting ARM processors executing T32 instructions into assembly code and writes it to the standard output stream.
\flowgraph{\resource{object file} \ar[r] & \toolbox{armt32dism} \ar[r] & \resource{disassembly\\listing}}
\seeassembly\seearm\seeobject
}

\providecommand{\avrasm}{
\toolsection{avrasm} is an assembler for the AVR hardware architecture.
It translates assembly code into machine code for AVR processors and stores it in corresponding object files.
The identifiers \texttt{RXL}, \texttt{RXH}, \texttt{RYL}, \texttt{RYH}, \texttt{RZL}, and \texttt{RZH} are predefined and name the corresponding registers.
The identifiers \texttt{SPL} and \texttt{SPH} are also predefined and evaluate to the address of the corresponding registers.
\flowgraph{\resource{AVR assembly\\source code} \ar[r] & \toolbox{avrasm} \ar[r] & \resource{object file}}
\seeassembly\seeavr\seeobject
}

\providecommand{\avrdism}{
\toolsection{avrdism} is a disassembler for the AVR hardware architecture.
It translates machine code from object files targeting AVR processors into assembly code and writes it to the standard output stream.
\flowgraph{\resource{object file} \ar[r] & \toolbox{avrdism} \ar[r] & \resource{disassembly\\listing}}
\seeassembly\seeavr\seeobject
}

\providecommand{\avrttasm}{
\toolsection{avr32asm} is an assembler for the AVR32 hardware architecture.
It translates assembly code into machine code for AVR32 processors and stores it in corresponding object files.
\flowgraph{\resource{AVR32 assembly\\source code} \ar[r] & \toolbox{avr32asm} \ar[r] & \resource{object file}}
\seeassembly\seeavrtt\seeobject
}

\providecommand{\avrttdism}{
\toolsection{avr32dism} is a disassembler for the AVR32 hardware architecture.
It translates machine code from object files targeting AVR32 processors into assembly code and writes it to the standard output stream.
\flowgraph{\resource{object file} \ar[r] & \toolbox{avr32dism} \ar[r] & \resource{disassembly\\listing}}
\seeassembly\seeavrtt\seeobject
}

\providecommand{\mabkasm}{
\toolsection{m68kasm} is an assembler for the M68000 hardware architecture.
It translates assembly code into machine code for M68000 processors and stores it in corresponding object files.
\flowgraph{\resource{68000 assembly\\source code} \ar[r] & \toolbox{m68kasm} \ar[r] & \resource{object file}}
\seeassembly\seemabk\seeobject
}

\providecommand{\mabkdism}{
\toolsection{m68kdism} is a disassembler for the M68000 hardware architecture.
It translates machine code from object files targeting M68000 processors into assembly code and writes it to the standard output stream.
\flowgraph{\resource{object file} \ar[r] & \toolbox{m68kdism} \ar[r] & \resource{disassembly\\listing}}
\seeassembly\seemabk\seeobject
}

\providecommand{\miblasm}{
\toolsection{miblasm} is an assembler for the MicroBlaze hardware architecture.
It translates assembly code into machine code for MicroBlaze processors and stores it in corresponding object files.
\flowgraph{\resource{MicroBlaze assembly\\source code} \ar[r] & \toolbox{miblasm} \ar[r] & \resource{object file}}
\seeassembly\seemibl\seeobject
}

\providecommand{\mibldism}{
\toolsection{mibldism} is a disassembler for the MicroBlaze hardware architecture.
It translates machine code from object files targeting MicroBlaze processors into assembly code and writes it to the standard output stream.
\flowgraph{\resource{object file} \ar[r] & \toolbox{mibldism} \ar[r] & \resource{disassembly\\listing}}
\seeassembly\seemibl\seeobject
}

\providecommand{\mipsaasm}{
\toolsection{mips32asm} is an assembler for the MIPS32 hardware architecture.
It translates assembly code into machine code for MIPS32 processors and stores it in corresponding object files.
\flowgraph{\resource{MIPS32 assembly\\source code} \ar[r] & \toolbox{mips32asm} \ar[r] & \resource{object file}}
\seeassembly\seemips\seeobject
}

\providecommand{\mipsadism}{
\toolsection{mips32dism} is a disassembler for the MIPS32 hardware architecture.
It translates machine code from object files targeting MIPS32 processors into assembly code and writes it to the standard output stream.
\flowgraph{\resource{object file} \ar[r] & \toolbox{mips32dism} \ar[r] & \resource{disassembly\\listing}}
\seeassembly\seemips\seeobject
}

\providecommand{\mipsbasm}{
\toolsection{mips64asm} is an assembler for the MIPS64 hardware architecture.
It translates assembly code into machine code for MIPS64 processors and stores it in corresponding object files.
\flowgraph{\resource{MIPS64 assembly\\source code} \ar[r] & \toolbox{mips64asm} \ar[r] & \resource{object file}}
\seeassembly\seemips\seeobject
}

\providecommand{\mipsbdism}{
\toolsection{mips64dism} is a disassembler for the MIPS64 hardware architecture.
It translates machine code from object files targeting MIPS64 processors into assembly code and writes it to the standard output stream.
\flowgraph{\resource{object file} \ar[r] & \toolbox{mips64dism} \ar[r] & \resource{disassembly\\listing}}
\seeassembly\seemips\seeobject
}

\providecommand{\mmixasm}{
\toolsection{mmixasm} is an assembler for the MMIX hardware architecture.
It translates assembly code into machine code for MMIX processors and stores it in corresponding object files.
The names of all special registers are predefined and evaluate to the corresponding number.
\flowgraph{\resource{MMIX assembly\\source code} \ar[r] & \toolbox{mmixasm} \ar[r] & \resource{object file}}
\seeassembly\seemmix\seeobject
}

\providecommand{\mmixdism}{
\toolsection{mmixdism} is a disassembler for the MMIX hardware architecture.
It translates machine code from object files targeting MMIX processors into assembly code and writes it to the standard output stream.
\flowgraph{\resource{object file} \ar[r] & \toolbox{mmixdism} \ar[r] & \resource{disassembly\\listing}}
\seeassembly\seemmix\seeobject
}

\providecommand{\orokasm}{
\toolsection{or1kasm} is an assembler for the OpenRISC 1000 hardware architecture.
It translates assembly code into machine code for OpenRISC 1000 processors and stores it in corresponding object files.
\flowgraph{\resource{OpenRISC 1000 assembly\\source code} \ar[r] & \toolbox{or1kasm} \ar[r] & \resource{object file}}
\seeassembly\seeorok\seeobject
}

\providecommand{\orokdism}{
\toolsection{or1kdism} is a disassembler for the OpenRISC 1000 hardware architecture.
It translates machine code from object files targeting OpenRISC 1000 processors into assembly code and writes it to the standard output stream.
\flowgraph{\resource{object file} \ar[r] & \toolbox{or1kdism} \ar[r] & \resource{disassembly\\listing}}
\seeassembly\seeorok\seeobject
}

\providecommand{\ppcaasm}{
\toolsection{ppc32asm} is an assembler for the PowerPC hardware architecture.
It translates assembly code into machine code for PowerPC processors and stores it in corresponding object files.
By default, the assembler generates machine code for the 32-bit operating mode defined by the PowerPC architecture.
\flowgraph{\resource{PowerPC assembly\\source code} \ar[r] & \toolbox{ppc32asm} \ar[r] & \resource{object file}}
\seeassembly\seeppc\seeobject
}

\providecommand{\ppcadism}{
\toolsection{ppc32dism} is a disassembler for the PowerPC hardware architecture.
It translates machine code from object files targeting PowerPC processors into assembly code and writes it to the standard output stream.
It assumes that the machine code was generated for the 32-bit operating mode defined by the PowerPC architecture.
\flowgraph{\resource{object file} \ar[r] & \toolbox{ppc32dism} \ar[r] & \resource{disassembly\\listing}}
\seeassembly\seeppc\seeobject
}

\providecommand{\ppcbasm}{
\toolsection{ppc64asm} is an assembler for the PowerPC hardware architecture.
It translates assembly code into machine code for PowerPC processors and stores it in corresponding object files.
By default, the assembler generates machine code for the 64-bit operating mode defined by the PowerPC architecture.
\flowgraph{\resource{PowerPC assembly\\source code} \ar[r] & \toolbox{ppc64asm} \ar[r] & \resource{object file}}
\seeassembly\seeppc\seeobject
}

\providecommand{\ppcbdism}{
\toolsection{ppc64dism} is a disassembler for the PowerPC hardware architecture.
It translates machine code from object files targeting PowerPC processors into assembly code and writes it to the standard output stream.
It assumes that the machine code was generated for the 64-bit operating mode defined by the PowerPC architecture.
\flowgraph{\resource{object file} \ar[r] & \toolbox{ppc64dism} \ar[r] & \resource{disassembly\\listing}}
\seeassembly\seeppc\seeobject
}

\providecommand{\riscasm}{
\toolsection{riscasm} is an assembler for the RISC hardware architecture.
It translates assembly code into machine code for RISC processors and stores it in corresponding object files.
The names of all special registers are predefined and evaluate to the corresponding number.
\flowgraph{\resource{RISC assembly\\source code} \ar[r] & \toolbox{riscasm} \ar[r] & \resource{object file}}
\seeassembly\seerisc\seeobject
}

\providecommand{\riscdism}{
\toolsection{riscdism} is a disassembler for the RISC hardware architecture.
It translates machine code from object files targeting RISC processors into assembly code and writes it to the standard output stream.
\flowgraph{\resource{object file} \ar[r] & \toolbox{riscdism} \ar[r] & \resource{disassembly\\listing}}
\seeassembly\seerisc\seeobject
}

\providecommand{\wasmasm}{
\toolsection{wasmasm} is an assembler for the WebAssembly architecture.
It translates assembly code into machine code for WebAssembly targets and stores it in corresponding object files.
The names of all special registers are predefined and evaluate to the corresponding number.
\flowgraph{\resource{WebAssembly assembly\\source code} \ar[r] & \toolbox{wasmasm} \ar[r] & \resource{object file}}
\seeassembly\seewasm\seeobject
}

\providecommand{\wasmdism}{
\toolsection{wasmdism} is a disassembler for the WebAssembly architecture.
It translates machine code from object files targeting WebAssembly targets into assembly code and writes it to the standard output stream.
\flowgraph{\resource{object file} \ar[r] & \toolbox{wasmdism} \ar[r] & \resource{disassembly\\listing}}
\seeassembly\seewasm\seeobject
}

% linker tools

\providecommand{\linklib}{
\toolsection{linklib} is an object file combiner.
It creates a static library file by combining all object files given to it into a single one.
\flowgraph{\resource{object files} \ar[r] & \toolbox{linklib} \ar[r] & \resource{library file}}
\seeobject
}

\providecommand{\linkbin}{
\toolsection{linkbin} is a linker for plain binary files.
It links all object files given to it into a single image and stores it in a binary file that begins with the first linked section.
It also creates a map file that lists the address, type, name and size of all used sections.
The filename extension of the resulting binary file can be specified by putting it into a constant data section called \texttt{\_extension}.
\flowgraph{\resource{object files} \ar[r] & \toolbox{linkbin} \ar[r] \ar[d] & \resource{binary file} \\ & \resource{map file}}
\seeobject
}

\providecommand{\linkmem}{
\toolsection{linkmem} is a linker for plain binary files partitioned into random-access and read-only memory.
It links all object files given to it into two distinct images, one for data sections and one for code and constant data sections, and stores each image in a binary file that begins with the first linked section of the corresponding type.
It also creates a map file that lists the address, type, name and size of all used sections.
\flowgraph{\resource{object files} \ar[r] & \toolbox{linkmem} \ar[r] \ar[d] & \resource{RAM file/\\ROM file} \\ & \resource{map file}}
\seeobject
}

\providecommand{\linkprg}{
\toolsection{linkprg} is a linker for GEMDOS executable files.
It links all object files given to it into a single image and stores the image in an Atari GEMDOS executable file~\cite{gemdosfile}.
It also creates a map file that lists the address relative to the text segment, type, name and size of all used sections.
The filename extension of the resulting executable file can be specified by putting it into a constant data section called \texttt{\_extension}.
The GEMDOS executable file format requires all patch patterns of absolute link patches to consist of four full bitmasks with descending offsets.
\flowgraph{\resource{object files} \ar[r] & \toolbox{linkprg} \ar[r] \ar[d] & \resource{executable file} \\ & \resource{map file}}
\seeobject
}

\providecommand{\linkhex}{
\toolsection{linkhex} is a linker for Intel HEX files.
It links all code sections of the object files given to it into single image and stores the image in an Intel HEX file~\cite{hexfile} that begins with the first linked section.
It also creates a map file that lists the address, type, name and size of all used sections.
\flowgraph{\resource{object files} \ar[r] & \toolbox{linkhex} \ar[r] \ar[d] & \resource{HEX file} \\ & \resource{map file}}
\seeobject
}

\providecommand{\mapsearch}{
\toolsection{mapsearch} is a debugging tool.
It searches map files generated by linker tools for the name of a binary section that encompasses a memory address read from the standard input stream.
If additionally provided with one or more object files, it also stores an excerpt thereof in a separate object file called map search result which only contains the identified binary section for disassembling purposes.
\flowgraph{& \resource{map files/\\object files} \ar[d] \\ \resource{memory\\address} \ar[r] & \toolbox{mapsearch} \ar[r] \ar[d] & \resource{section name/\\relative offset} \\ & \resource{object file\\excerpt}}
\seeobject
}


\startbook{User Manual}

\chapter*{Preface}\markboth{Preface}{Preface}

This user manual describes all components of a free and self-hosted software development toolchain called the \emph{\ecs{}} and explains how to use them.
It is partitioned into four parts.
The first part describes the overall mission, design, and the common user interface of the \ecs{}.
The second part specifies the programming languages supported by the \ecs{}, while the third part summarizes the hardware architectures and how they are supported.
The last part of this manual finally describes some internal functionality and explains how the \ecs{} can be extended in order to support additional programming languages, hardware architectures, and runtime environments.
The addendum elaborates on the implementation of the \ecs{} and includes its licenses.

\epigraph{Il semble que la perfection soit atteinte \\ non quand il n'y a plus rien \`a ajouter, \\ mais quand il n'y a plus rien \`a retrancher.}{Antoine de Saint-Exup\'ery}

\concludechapter
\mainmatter

\part{Using the \ecs{}}
% Introduction to the Eigen Compiler Suite
% Copyright (C) Florian Negele

% This file is part of the Eigen Compiler Suite.

% Permission is granted to copy, distribute and/or modify this document
% under the terms of the GNU Free Documentation License, Version 1.3
% or any later version published by the Free Software Foundation.

% You should have received a copy of the GNU Free Documentation License
% along with the ECS.  If not, see <https://www.gnu.org/licenses/>.

% Generic documentation utilities
% Copyright (C) Florian Negele

% This file is part of the Eigen Compiler Suite.

% Permission is granted to copy, distribute and/or modify this document
% under the terms of the GNU Free Documentation License, Version 1.3
% or any later version published by the Free Software Foundation.

% You should have received a copy of the GNU Free Documentation License
% along with the ECS.  If not, see <https://www.gnu.org/licenses/>.

\providecommand{\cpp}{C\texttt{++}}
\providecommand{\opt}{_\mathit{opt}}
\providecommand{\tool}[1]{\texttt{#1}}
\providecommand{\version}{Version 0.0.40}
\providecommand{\resource}[1]{*++\txt{#1}}
\providecommand{\ecs}{Eigen Compiler Suite}
\providecommand{\changed}[1]{\underline{#1}}
\providecommand{\toolbox}[1]{\converter{#1}}
\providecommand{\file}{}\renewcommand{\file}[1]{\texttt{#1}}
\providecommand{\alignright}{\hfill\linebreak[0]\hspace*{\fill}}
\providecommand{\converter}[1]{*++[F][F*:white][F,:gray]\txt{#1}}
\providecommand{\documentation}{\ifbook chapter\else document\fi}
\providecommand{\Documentation}{\ifbook Chapter\else Document\fi}
\providecommand{\variable}[1]{\resource{\texttt{\small#1}\\variable}}
\providecommand{\documentationref}[2]{\ifbook\ref{#1}\else``\href{#1}{#2}''~\cite{#1}\fi}
\providecommand{\objfile}[1]{\texttt{#1}\index[runtime]{#1 object file@\texttt{#1} object file}}
\providecommand{\libfile}[1]{\texttt{#1}\index[runtime]{#1 library file@\texttt{#1} library file}}
\providecommand{\epigraph}[2]{\ifbook\begin{quote}\flushright\textit{#1}\par--- #2\end{quote}\fi}
\providecommand{\environmentvariable}[1]{\texttt{#1}\index{Environment variables!#1@\texttt{#1}}}
\providecommand{\environment}[1]{\texttt{#1}\index[environment]{#1 environment@\texttt{#1} environment}}
\providecommand{\toolsection}{}\renewcommand{\toolsection}[1]{\subsection{#1}\label{\prefix:#1}\tool{#1}}
\providecommand{\instruction}{}\renewcommand{\instruction}[2]{\noindent\qquad\pdftooltip{\texttt{#1}}{#2}\refstepcounter{instruction}\par}
\providecommand{\flowgraph}{}\renewcommand{\flowgraph}[1]{\par\sffamily\begin{displaymath}\xymatrix@=4ex{#1}\end{displaymath}\normalfont\par}
\providecommand{\instructionset}{}\renewcommand{\instructionset}[4]{\setcounter{instruction}{0}\begin{multicols}{\ifbook#3\else#4\fi}[{\captionof{table}[#2]{#2 (\ref*{#1:instructions}~instructions)}\label{tab:#1set}\vspace{-2ex}}]\footnotesize\raggedcolumns\input{#1.set}\label{#1:instructions}\end{multicols}}

\providecommand{\gpl}{GNU General Public License}
\providecommand{\rse}{ECS Runtime Support Exception}
\providecommand{\fdl}{\href{https://www.gnu.org/licenses/fdl.html}{GNU Free Documentation License}}

\providecommand{\docbegin}{}
\providecommand{\docend}{}
\providecommand{\doclabel}[1]{\hypertarget{#1}}
\providecommand{\doclink}[2]{\hyperlink{#1}{#2}}
\providecommand{\docsection}[3]{\hypertarget{#1}{\subsection{#2}}\label{sec:#1}\index[library]{#2@#3}}
\providecommand{\docsectionstar}[1]{}
\providecommand{\docsubbegin}{\begin{description}}
\providecommand{\docsubend}{\end{description}}
\providecommand{\docsubsection}[3]{\item[\hypertarget{#1}{#2}]\index[library]{#2@#3}}
\providecommand{\docsubsectionstar}[1]{\smallskip}
\providecommand{\docsubsubsection}[3]{\docsubsection{#1}{#2}{#3}}
\providecommand{\docsubsubsectionstar}[1]{}
\providecommand{\docsubsubsubsection}[3]{}
\providecommand{\docsubsubsubsectionstar}[1]{}
\providecommand{\doctable}{}

\providecommand{\debuggingtool}{}\renewcommand{\debuggingtool}{This tool is provided for debugging purposes.
It allows exposing and modifying an internal data structure that is usually not accessible.
}

\providecommand{\interface}{All tools accept command-line arguments which are taken as names of plain text files containing the source code.
If no arguments are provided, the standard input stream is used instead.
Output files are generated in the current working directory and have the same name as the input file being processed whereas the filename extension gets replaced by an appropriate suffix.
\seeinterface
}

\providecommand{\license}{\noindent Copyright \copyright{} Florian Negele\par\medskip\noindent
Permission is granted to copy, distribute and/or modify this document under the terms of the
\fdl{}, Version 1.3 or any later version published by the \href{https://fsf.org/}{Free Software Foundation}.
}

\providecommand{\ecslogosurface}{
\fill[darkgray] (0,0,0) -- (0,0,3) -- (0,3,3) -- (0,3,1) -- (0,4,1) -- (0,4,3) -- (0,5,3) -- (0,5,0) -- (0,2,0) -- (0,2,2) -- (0,1,2) -- (0,1,0) -- cycle;
\fill[gray] (0,5,0) -- (0,5,3) -- (1,5,3) -- (1,5,1) -- (2,5,1) -- (2,5,3) -- (3,5,3) -- (3,5,0) -- cycle;
\fill[lightgray] (0,0,0) -- (0,1,0) -- (2,1,0) -- (2,4,0) -- (1,4,0) -- (1,3,0) -- (2,3,0) -- (2,2,0) -- (0,2,0) -- (0,5,0) -- (3,5,0) -- (3,0,0) -- cycle;
\begin{scope}[line width=0.5]
\begin{scope}[gray]
\draw (0,0,0) -- (0,1,0);
\draw (2,1,0) -- (2,2,0);
\draw (0,1,2) -- (0,2,2);
\draw (0,2,0) -- (0,5,0);
\draw (2,3,0) -- (2,4,0);
\end{scope}
\begin{scope}[lightgray]
\draw (0,1,0) -- (0,1,2);
\draw (0,3,1) -- (0,3,3);
\draw (0,5,0) -- (0,5,3);
\draw (2,5,1) -- (2,5,3);
\end{scope}
\begin{scope}[white]
\draw (0,1,0) -- (2,1,0);
\draw (1,3,0) -- (2,3,0);
\draw (0,5,0) -- (3,5,0);
\end{scope}
\end{scope}
}

\providecommand{\ecslogo}[1]{
\begin{tikzpicture}[scale={(#1)/((sin(45)+cos(45))*3cm)},x={({-cos(45)*1cm},{sin(45)*sin(30)*1cm})},y={({0cm},{(cos(30)*1cm})},z={({sin(45)*1cm},{cos(45)*sin(30)*1cm})}]
\begin{scope}[darkgray,line width=1]
\draw (0,0,0) -- (0,0,3) -- (0,3,3) -- (2,3,3) -- (2,5,3) -- (3,5,3) -- (3,5,0) -- (3,0,0) -- cycle;
\draw (0,3,1) -- (0,4,1) -- (0,4,3) -- (0,5,3) -- (1,5,3) -- (1,5,1) -- (2,5,1);
\draw (1,3,0) -- (1,4,0) -- (2,4,0);
\end{scope}
\fill[darkgray] (2,0,0) -- (2,0,3) -- (2,5,3) -- (2,5,1) -- (2,4,1) -- (2,4,0) -- cycle;
\fill[lightgray] (2,0,2) -- (0,0,2) -- (0,2,2) -- (2,2,2) -- cycle;
\fill[gray] (0,1,0) -- (2,1,0) -- (2,1,2) -- (0,1,2) -- cycle;
\fill[gray] (0,3,1) -- (0,3,3) -- (2,3,3) -- (2,3,0) -- (1,3,0) -- (1,3,1) -- cycle;
\ecslogosurface
\end{tikzpicture}
}

\providecommand{\shadowedecslogo}[3]{
\begin{tikzpicture}[scale={(#1)/((sin(#2)+cos(#2))*3cm)},x={({-cos(#2)*1cm},{sin(#2)*sin(#3)*1cm})},y={({0cm},{(cos(#3)*1cm})},z={({sin(#2)*1cm},{cos(#2)*sin(#3)*1cm})}]
\shade[top color=lightgray!50!white,bottom color=white,middle color=lightgray!50!white] (0,0,0) -- (3,0,0) -- (3,{-0.5-3*sin(#2)*sin(#3)/cos(#3)},0) -- (0,-0.5,0) -- cycle;
\shade[top color=darkgray!50!gray,bottom color=white,middle color=darkgray!50!white] (0,0,0) -- (0,0,3) -- (0,{-0.5-3*cos(#2)*sin(#3)/cos(#3)},3) -- (0,-0.5,0) -- cycle;
\begin{scope}[y={({(cos(#2)+sin(#2))*0.5cm},{(cos(#2)*sin(#3)-sin(#2)*sin(#3))*0.5cm})}]
\useasboundingbox (3,0,0) -- (0,0,0) -- (0,0,3);
\shade[left color=darkgray!80!black,right color=lightgray,middle color=gray] (0,0,0) -- (0,1,0) -- (0,1,0.5) -- (0,2,0) -- (0,5,0) -- (0,5,3) -- (1,5,3) -- (1,4,3) -- (1,4,2.5) -- (1,3,3) -- (2,5,3) -- (3,5,3) -- (3,0,3) -- cycle;
\clip (0,0,0) -- (0,0,3) -- ({-3*sin(#2)/cos(#2)},0,0) -- cycle;
\shade[left color=darkgray,right color=lightgray!50!gray] (0,0,0) -- (0,1,0) -- (0,1,0.5) -- (0,2,0) -- (0,5,0) -- (0,5,3) -- (1,5,3) -- (1,4,3) -- (1,4,2.5) -- (1,3,3) -- (2,5,3) -- (3,5,3) -- (3,0,3) -- cycle;
\end{scope}
\shade[left color=darkgray,right color=darkgray!80!black] (2,0,0) -- (2,0,3) -- (2,5,3) -- (2,5,1) -- (2,4,1) -- (2,4,0) -- cycle;
\shade[left color=darkgray!90!black,right color=gray!80!darkgray] (2,0,2) -- (0,0,2) -- (0,2,2) -- (2,2,2) -- cycle;
\shade[top color=darkgray!90!black,bottom color=gray!80!darkgray] (0,1,0) -- (2,1,0) -- (2,1,2) -- (0,1,2) -- cycle;
\shade[top color=darkgray!90!black,bottom color=gray!80!darkgray] (0,3,1) -- (0,3,3) -- (2,3,3) -- (2,3,0) -- (1,3,0) -- (1,3,1) -- cycle;
\fill[gray] (2,1,0) -- (1.5,1,0.5) -- (0,1,0.5) -- (0,1,0) -- cycle;
\fill[gray] (1,3,2) -- (0.5,3,2) -- (0.5,3,3) -- (1,3,3) -- cycle;
\fill[gray] (2,3,0) -- (1.5,3,0.5) -- (1,3,0.5) -- (1,3,0) -- cycle;
\ecslogosurface
\end{tikzpicture}
}

\providecommand{\cpplogo}[1]{
\begin{tikzpicture}[scale=(#1)/512em]
\fill[gray] (435.2794,398.7159) -- (247.1911,507.3075) .. controls (236.3563,513.5642) and (218.6240,513.5642) .. (207.7892,507.3075) -- (19.7009,398.7159) .. controls (8.8646,392.4606) and (0.0000,377.1043) .. (0.0000,364.5924) -- (0.0000,147.4076) .. controls (0.8430,132.8363) and (8.2856,120.7683) .. (19.7009,113.2842) -- (207.7892,4.6926) .. controls (218.6240,-1.5642) and (236.3564,-1.5642) .. (247.1911,4.6926) -- (435.2794,113.2842) .. controls (447.5273,121.4304) and (454.4987,133.6918) .. (454.9803,147.4076) -- (454.9803,364.5924) .. controls (454.5404,377.7571) and (446.6566,391.0351) .. (435.2794,398.7159) -- cycle(75.8301,255.9993) .. controls (74.9389,404.0881) and (273.2892,469.4783) .. (358.8263,331.8769) -- (293.1917,293.8965) .. controls (253.5702,359.4301) and (155.1909,335.9977) .. (151.6601,255.9993) .. controls (152.7204,182.2703) and (249.4137,148.0211) .. (293.1961,218.1065) -- (358.8308,180.1276) .. controls (283.4477,49.2645) and (79.6318,96.3470) .. (75.8301,255.9993) -- cycle(379.1503,247.5747) -- (362.2982,247.5747) -- (362.2982,230.7226) -- (345.4490,230.7226) -- (345.4490,247.5747) -- (328.5969,247.5747) -- (328.5969,264.4254) -- (345.4490,264.4254) -- (345.4490,281.2759) -- (362.2982,281.2759) -- (362.2982,264.4254) -- (379.1503,264.4254) -- cycle(442.3420,247.5747) -- (425.4899,247.5747) -- (425.4899,230.7226) -- (408.6408,230.7226) -- (408.6408,247.5747) -- (391.7886,247.5747) -- (391.7886,264.4254) -- (408.6408,264.4254) -- (408.6408,281.2759) -- (425.4899,281.2759) -- (425.4899,264.4254) -- (442.3420,264.4254) -- cycle;
\end{tikzpicture}
}

\providecommand{\fallogo}[1]{
\begin{tikzpicture}[scale=(#1)/512em]
\fill[gray] (185.7774,0.0000) .. controls (200.4486,15.9798) and (226.8966,8.7148) .. (235.0426,31.5836) .. controls (249.5297,58.0598) and (247.9581,97.9161) .. (280.3335,110.9762) .. controls (309.1690,120.3496) and (337.8406,104.2727) .. (366.5753,103.9379) .. controls (373.4449,111.5171) and (379.2885,128.2574) .. (383.9755,108.9744) .. controls (396.6979,102.5615) and (437.2808,107.6681) .. (426.9652,124.3252) .. controls (408.9822,121.0785) and (412.4742,146.0729) .. (426.5192,131.4996) .. controls (433.8413,120.8489) and (465.1541,126.5522) .. (441.9067,135.7950) .. controls (396.1879,157.7478) and (344.1112,161.5079) .. (298.5528,183.5702) .. controls (277.7471,193.5198) and (284.6941,218.7163) .. (285.2127,236.9640) .. controls (292.3599,316.2826) and (307.3929,394.6311) .. (317.1198,473.6154) .. controls (329.0637,505.4736) and (292.1195,528.5004) .. (265.9183,511.2761) .. controls (237.9284,499.2462) and (237.3684,465.2681) .. (230.9102,439.9421) .. controls (218.6692,374.3397) and (215.6307,306.9662) .. (198.1732,242.3977) .. controls (183.1379,232.7444) and (164.4245,256.0298) .. (149.0430,261.4799) .. controls (116.9328,279.2585) and (87.1822,308.5851) .. (48.2293,307.8914) .. controls (21.3220,306.9037) and (-15.9107,281.8761) .. (7.2921,252.7908) .. controls (29.7799,220.6177) and (67.5177,204.3028) .. (100.9287,185.9449) .. controls (130.8217,170.8906) and (161.1548,156.5903) .. (191.0278,141.5847) .. controls (196.1738,120.0520) and (186.6049,95.2409) .. (186.8382,72.4353) .. controls (185.5234,48.4204) and (183.1700,23.9341) .. (185.7774,0.0000) -- cycle;
\end{tikzpicture}
}

\providecommand{\oblogo}[1]{
\begin{tikzpicture}[scale=(#1)/512em]
\fill[gray] (160.3865,208.9117) .. controls (154.0879,214.6478) and (149.0735,221.2409) .. (145.4125,228.5384) .. controls (184.8790,248.4273) and (234.7122,269.8787) .. (297.5493,291.8782) .. controls (300.3943,281.4769) and (300.9552,268.7619) .. (300.4023,255.2389) .. controls (248.9909,244.7891) and (200.0310,225.9279) .. (160.3865,208.9117) -- cycle(225.7398,392.6996) .. controls (308.0209,392.1716) and (359.3326,345.9277) .. (368.7203,285.2098) .. controls (376.6742,197.1784) and (311.7194,141.3342) .. (205.4287,142.1456) .. controls (139.9485,141.4804) and (88.7155,166.1957) .. (73.5775,228.0086) .. controls (52.0297,320.3408) and (123.4078,391.0103) .. (225.7398,392.6996) -- cycle(216.0739,176.4733) .. controls (268.9183,179.2424) and (315.8292,206.5488) .. (312.7454,265.1139) .. controls (313.2769,315.6384) and (286.5993,353.4946) .. (216.6040,355.7934) .. controls (162.4657,355.7934) and (126.0914,317.5023) .. (126.0914,260.5103) .. controls (126.1733,214.2900) and (163.3363,176.2849) .. (216.0739,176.4733) -- cycle(76.4897,189.1754) .. controls (13.1586,147.5631) and (0.0000,119.4207) .. (0.0000,119.4207) -- (90.6499,170.1632) .. controls (85.3004,175.8497) and (80.5994,182.1633) .. (76.4897,189.1754) -- cycle(353.9486,119.3004) -- (402.9482,119.3004) .. controls (427.0025,137.0797) and (450.9893,162.7034) .. (474.9529,191.0213) .. controls (509.3540,228.5339) and (531.3391,294.2091) .. (487.8149,312.1206) .. controls (462.8165,324.7652) and (394.3874,316.8943) .. (373.8912,313.6651) .. controls (379.9291,297.7449) and (383.2899,278.4204) .. (381.4989,257.7214) .. controls (420.3069,248.0321) and (421.9610,218.3461) .. (407.7867,192.6417) .. controls (391.1113,162.4018) and (370.1114,132.9097) .. (353.9486,119.3004) -- cycle;
\end{tikzpicture}
}

\providecommand{\markuptable}{
\begin{table}
\sffamily\centering
\begin{tabular}{@{}lcl@{}}
\toprule
\texttt{//italics//} & $\rightarrow$ & \textit{italics} \\
\midrule
\texttt{**bold**} & $\rightarrow$ & \textbf{bold} \\
\midrule
\texttt{\# ordered list} & & 1 ordered list \\
\texttt{\# second item} & $\rightarrow$ & 2 second item \\
\texttt{\#\# sub item} & & \hspace{1em} 1 sub item \\
\midrule
\texttt{* unordered list} & & $\bullet$ unordered list \\
\texttt{* second item} & $\rightarrow$ & $\bullet$ second item \\
\texttt{** sub item} & & \hspace{1em} $\bullet$ sub item \\
\midrule
\texttt{link to [[label]]} & $\rightarrow$ & link to \underline{label} \\
\midrule
\texttt{<{}<label>{}> definition } & $\rightarrow$ & definition \\
\midrule
\texttt{[[url|link name]]} & $\rightarrow$ & \underline{link name} \\
\midrule\addlinespace
\texttt{= large heading} & & {\Large large heading} \smallskip \\
\texttt{== medium heading} & $\rightarrow$ & {\large medium heading} \\
\texttt{=== small heading} & & small heading \\
\midrule
\texttt{no line break} & & no line break for paragraphs \\
\texttt{for paragraphs} & $\rightarrow$ \\
& & use empty line \\
\texttt{use empty line} \\
\midrule
\texttt{force\textbackslash\textbackslash line break} & $\rightarrow$ & force \\
& & line break \\
\midrule
\texttt{horizontal line} & $\rightarrow$ & horizontal line \\
\texttt{----} & & \hrulefill \\
\midrule
\texttt{|=a|=table|=header} & & \underline{a \enspace table \enspace header} \\
\texttt{|a|table|row} & $\rightarrow$ & a \enspace table \enspace row \\
\texttt{|b|table|row} & & b \enspace table \enspace row \\
\midrule
\texttt{\{\{\{} \\
\texttt{unformatted} & $\rightarrow$ & \texttt{unformatted} \\
\texttt{code} & & \texttt{code} \\
\texttt{\}\}\}} \\
\midrule\addlinespace
\texttt{@ new article} & & {\Large 1.\ new article} \smallskip \\
\texttt{@ second article} & $\rightarrow$ & {\Large 2.\ second article} \smallskip \\
\texttt{@@ sub article} & & {\large 2.1.\ sub article} \\
\bottomrule
\end{tabular}
\normalfont\caption{Elements of the generic documentation markup language}
\label{tab:docmarkup}
\end{table}
}

\providecommand{\startchapter}[4]{
\documentclass[11pt,a4paper]{article}
\usepackage{booktabs}
\usepackage[format=hang,labelfont=bf]{caption}
\usepackage{changepage}
\usepackage[T1]{fontenc}
\usepackage[margin=2cm]{geometry}
\usepackage{hyperref}
\usepackage[american]{isodate}
\usepackage{lmodern}
\usepackage{longtable}
\usepackage{mathptmx}
\usepackage{microtype}
\usepackage[toc]{multitoc}
\usepackage{multirow}
\usepackage[all]{nowidow}
\usepackage{pdfcomment}
\usepackage{syntax}
\usepackage{tikz}
\usepackage[all]{xy}
\hypersetup{pdfborder={0 0 0},bookmarksnumbered=true,pdftitle={\ecs{}: #2},pdfauthor={Florian Negele},pdfsubject={\ecs{}},pdfkeywords={#1}}
\setlength{\grammarindent}{8em}\setlength{\grammarparsep}{0.2ex}
\setlength{\columnsep}{2em}
\newcommand{\prefix}{}
\newcounter{instruction}
\bibliographystyle{unsrt}
\renewcommand{\index}[2][]{}
\renewcommand{\arraystretch}{1.05}
\renewcommand{\floatpagefraction}{0.7}
\renewcommand{\syntleft}{\itshape}\renewcommand{\syntright}{}
\title{\vspace{-5ex}\Huge{\ecs{}}\medskip\hrule}
\author{\huge{#2}}
\date{\medskip\version}
\newif\ifbook\bookfalse
\pagestyle{headings}
\frenchspacing
\begin{document}
\maketitle\thispagestyle{empty}\noindent#4\setlength{\columnseprule}{0.4pt}\tableofcontents\setlength{\columnseprule}{0pt}\vfill\pagebreak[3]\null\vfill\bigskip\noindent
\parbox{\textwidth-4em}{\license The contents of this \documentation{} are part of the \href{manual}{\ecs{} User Manual}~\cite{manual} and correspond to Chapter ``\href{manual\##3}{#1}''.\alignright\mbox{\today}}
\parbox{4em}{\flushright\ecslogo{3em}}
\clearpage
}

\providecommand{\concludechapter}{
\vfill\pagebreak[3]\null\vfill
\thispagestyle{myheadings}\markright{REFERENCES}
\noindent\begin{minipage}{\textwidth}\begin{multicols}{2}[\section*{References}]
\renewcommand{\section}[2]{}\small\bibliography{references}
\end{multicols}\end{minipage}\end{document}
}

\providecommand{\startpresentation}[2]{
\documentclass[14pt,aspectratio=43,usepdftitle=false]{beamer}
\usepackage{booktabs}
\usepackage{etex}
\usepackage{multicol}
\usepackage{tikz}
\usepackage[all]{xy}
\bibliographystyle{unsrt}
\setlength{\columnsep}{1em}
\setlength{\leftmargini}{1em}
\setbeamercolor{title}{fg=black}
\setbeamercolor{structure}{fg=darkgray}
\setbeamercolor{bibliography item}{fg=darkgray}
\setbeamerfont{title}{series=\bfseries}
\setbeamerfont{subtitle}{series=\normalfont}
\setbeamerfont*{frametitle}{parent=title}
\setbeamerfont{block title}{series=\bfseries}
\setbeamerfont*{framesubtitle}{parent=subtitle}
\setbeamersize{text margin left=1em,text margin right=1em}
\setbeamertemplate{navigation symbols}{}
\setbeamertemplate{itemize item}[circle]{}
\setbeamertemplate{bibliography item}[triangle]{}
\setbeamertemplate{bibliography entry author}{\usebeamercolor[fg]{bibliography item}}
\setbeamertemplate{frametitle}{\medskip\usebeamerfont{frametitle}\color{gray}\raisebox{-2.5ex}[0ex][0ex]{\rule{0.1em}{4.5ex}}}
\addtobeamertemplate{frametitle}{}{\hspace{0.4em}\usebeamercolor[fg]{title}\insertframetitle\par\vspace{0.2ex}\hspace{0.5em}\usebeamerfont{framesubtitle}\insertframesubtitle}
\hypersetup{pdfborder={0 0 0},bookmarksnumbered=true,bookmarksopen=true,bookmarksopenlevel=0,pdftitle={\ecs{}: #1},pdfauthor={Florian Negele},pdfsubject={\ecs{}},pdfkeywords={#1}}
\renewcommand{\flowgraph}[1]{\resizebox{\textwidth}{!}{$$\xymatrix{##1}$$}}
\title{\ecs{}\medskip\hrule\medskip}
\institute{\shadowedecslogo{5em}{30}{15}}
\date{\version}
\subtitle{#1}
\begin{document}
\begin{frame}[plain]\titlepage\nocite{manual}\end{frame}
\begin{frame}{Contents}{#1}\begin{center}\tableofcontents\end{center}\end{frame}
}

\providecommand{\concludepresentation}{
\begin{frame}{References}\begin{footnotesize}\setlength{\columnseprule}{0.4pt}\begin{multicols}{2}\bibliography{references}\end{multicols}\end{footnotesize}\end{frame}
\end{document}
}

\providecommand{\startbook}[1]{
\documentclass[10pt,paper=17cm:24cm,DIV=13,twoside=semi,headings=normal,numbers=noendperiod,cleardoublepage=plain]{scrbook}
\usepackage{atveryend}
\usepackage{booktabs}
\usepackage{caption}
\usepackage{changepage}
\usepackage[T1]{fontenc}
\usepackage{imakeidx}
\usepackage{hyperref}
\usepackage[american]{isodate}
\usepackage{lmodern}
\usepackage{longtable}
\usepackage{mathptmx}
\usepackage[final]{microtype}
\usepackage{multicol}
\usepackage{multirow}
\usepackage[all]{nowidow}
\usepackage{pdfcomment}
\usepackage{scrlayer-scrpage}
\usepackage{setspace}
\usepackage{syntax}
\usepackage[eventxtindent=4pt,oddtxtexdent=4pt]{thumbs}
\usepackage{tikz}
\usepackage[all]{xy}
\hyphenation{Micro-Blaze Open-Cores Open-RISC Power-PC}
\hypersetup{pdfborder={0 0 0},bookmarksnumbered=true,bookmarksopen=true,bookmarksopenlevel=0,pdftitle={\ecs{}: #1},pdfauthor={Florian Negele},pdfsubject={\ecs{}},pdfkeywords={#1}}
\setlength{\grammarindent}{8em}\setlength{\grammarparsep}{0.7ex}
\setkomafont{captionlabel}{\usekomafont{descriptionlabel}}
\renewcommand{\arraystretch}{1.05}\setstretch{1.1}
\renewcommand{\chapterformat}{\thechapter\autodot\enskip\raisebox{-1ex}[0ex][0ex]{\color{gray}\rule{0.1em}{3.5ex}}\enskip}
\renewcommand{\startchapter}[4]{\hypertarget{##3}{\chapter{##1}}\label{##3}##4\addthumb{##1}{\LARGE\sffamily\bfseries\thechapter}{white}{gray}\renewcommand{\prefix}{##3}}
\renewcommand{\concludechapter}{\clearpage{\stopthumb\cleardoublepage}}
\renewcommand{\syntleft}{\itshape}\renewcommand{\syntright}{}
\renewcommand{\floatpagefraction}{0.7}
\renewcommand{\partheademptypage}{}
\DeclareMicrotypeAlias{lmss}{cmr}
\newcommand{\prefix}{}
\newcounter{instruction}
\bibliographystyle{unsrt}
\newif\ifbook\booktrue
\makeindex[intoc,title=Index]
\makeindex[intoc,name=tools,title=Index of Tools,columns=3]
\makeindex[intoc,name=library,title=Index of Library Names]
\makeindex[intoc,name=runtime,title=Index of Runtime Support]
\makeindex[intoc,name=environment,title=Index of Target Environments]
\indexsetup{toclevel=chapter,headers={\indexname}{\indexname}}
\frenchspacing
\begin{document}
\pagenumbering{alph}
\begin{titlepage}\centering
\huge\sffamily\null\vfill\textbf{\ecs{}}\bigskip\hrule\bigskip#1
\normalsize\normalfont\vfill\vfill\shadowedecslogo{10em}{30}{15}
\large\vfill\vfill\version
\end{titlepage}
\null\vfill
\thispagestyle{empty}
\noindent\today\par\medskip
\license A copy of this license is included in Appendix~\ref{fdl} on page~\pageref{fdl}.
All product names used herein are for identification purposes only and may be trademarks of their respective companies.
\concludechapter
\frontmatter
\setcounter{tocdepth}{1}
\tableofcontents
\setcounter{tocdepth}{2}
\concludechapter
\listoffigures
\concludechapter
\listoftables
\concludechapter
}

\providecommand{\concludebook}{
\backmatter
\addtocontents{toc}{\protect\setcounter{tocdepth}{-1}}
\phantomsection\addcontentsline{toc}{part}{Bibliography}
\bibliography{references}
\concludechapter
\phantomsection\addcontentsline{toc}{part}{Indexes}
\printindex
\concludechapter
\indexprologue{\label{idx:tools}}
\printindex[tools]
\concludechapter
\printindex[library]
\concludechapter
\indexprologue{\label{idx:runtime}}
\printindex[runtime]
\concludechapter
\indexprologue{\label{idx:environment}}
\printindex[environment]
\concludechapter
\pagestyle{empty}\pagenumbering{Alph}\null\clearpage
\null\vfill\centering\ecslogo{4em}\par\medskip\license
\end{document}
}

% chapter references

\providecommand{\seedocumentationref}{}\renewcommand{\seedocumentationref}[3]{#1, see \Documentation{}~\documentationref{#2}{#3}. }
\providecommand{\seeinterface}{}\renewcommand{\seeinterface}{\ifbook See \Documentation{}~\documentationref{interface}{User Interface} for more information about the common user interface of all of these tools. \fi}
\providecommand{\seeguide}{}\renewcommand{\seeguide}{\seedocumentationref{For basic examples of using some of these tools in practice}{guide}{User Guide}}
\providecommand{\seecpp}{}\renewcommand{\seecpp}{\seedocumentationref{For more information about the \cpp{} programming language and its implementation by the \ecs{}}{cpp}{User Manual for \cpp{}}}
\providecommand{\seefalse}{}\renewcommand{\seefalse}{\seedocumentationref{For more information about the FALSE programming language and its implementation by the \ecs{}}{false}{User Manual for FALSE}}
\providecommand{\seeoberon}{}\renewcommand{\seeoberon}{\seedocumentationref{For more information about the Oberon programming language and its implementation by the \ecs{}}{oberon}{User Manual for Oberon}}
\providecommand{\seeassembly}{}\renewcommand{\seeassembly}{\seedocumentationref{For more information about the generic assembly language and how to use it}{assembly}{Generic Assembly Language Specification}}
\providecommand{\seeamd}{}\renewcommand{\seeamd}{\seedocumentationref{For more information about how the \ecs{} supports the AMD64 hardware architecture}{amd64}{AMD64 Hardware Architecture Support}}
\providecommand{\seearm}{}\renewcommand{\seearm}{\seedocumentationref{For more information about how the \ecs{} supports the ARM hardware architecture}{arm}{ARM Hardware Architecture Support}}
\providecommand{\seeavr}{}\renewcommand{\seeavr}{\seedocumentationref{For more information about how the \ecs{} supports the AVR hardware architecture}{avr}{AVR Hardware Architecture Support}}
\providecommand{\seeavrtt}{}\renewcommand{\seeavrtt}{\seedocumentationref{For more information about how the \ecs{} supports the AVR32 hardware architecture}{avr32}{AVR32 Hardware Architecture Support}}
\providecommand{\seemabk}{}\renewcommand{\seemabk}{\seedocumentationref{For more information about how the \ecs{} supports the M68000 hardware architecture}{m68k}{M68000 Hardware Architecture Support}}
\providecommand{\seemibl}{}\renewcommand{\seemibl}{\seedocumentationref{For more information about how the \ecs{} supports the MicroBlaze hardware architecture}{mibl}{MicroBlaze Hardware Architecture Support}}
\providecommand{\seemips}{}\renewcommand{\seemips}{\seedocumentationref{For more information about how the \ecs{} supports the MIPS32 and MIPS64 hardware architectures}{mips}{MIPS Hardware Architecture Support}}
\providecommand{\seemmix}{}\renewcommand{\seemmix}{\seedocumentationref{For more information about how the \ecs{} supports the MMIX hardware architecture}{mmix}{MMIX Hardware Architecture Support}}
\providecommand{\seeorok}{}\renewcommand{\seeorok}{\seedocumentationref{For more information about how the \ecs{} supports the OpenRISC 1000 hardware architecture}{or1k}{OpenRISC 1000 Hardware Architecture Support}}
\providecommand{\seeppc}{}\renewcommand{\seeppc}{\seedocumentationref{For more information about how the \ecs{} supports the PowerPC hardware architecture}{ppc}{PowerPC Hardware Architecture Support}}
\providecommand{\seerisc}{}\renewcommand{\seerisc}{\seedocumentationref{For more information about how the \ecs{} supports the RISC hardware architecture}{risc}{RISC Hardware Architecture Support}}
\providecommand{\seewasm}{}\renewcommand{\seewasm}{\seedocumentationref{For more information about how the \ecs{} supports the WebAssembly architecture}{wasm}{WebAssembly Architecture Support}}
\providecommand{\seedocumentation}{}\renewcommand{\seedocumentation}{\seedocumentationref{For more information about generic documentations and their generation by the \ecs{}}{documentation}{Generic Documentation Generation}}
\providecommand{\seedebugging}{}\renewcommand{\seedebugging}{\seedocumentationref{For more information about debugging information and its representation}{debugging}{Debugging Information Representation}}
\providecommand{\seecode}{}\renewcommand{\seecode}{\seedocumentationref{For more information about intermediate code and its purpose}{code}{Intermediate Code Representation}}
\providecommand{\seeobject}{}\renewcommand{\seeobject}{\seedocumentationref{For more information about object files and their purpose}{object}{Object File Representation}}

% generic documentation tools

\providecommand{\docprint}{
\toolsection{docprint} is a pretty printer for generic documentations.
It reformats generic documentations and writes it to the standard output stream.
\debuggingtool
\flowgraph{\resource{generic\\documentation} \ar[r] & \toolbox{docprint} \ar[r] & \resource{generic\\documentation}}
\seedocumentation
}

\providecommand{\doccheck}{
\toolsection{doccheck} is a syntactic and semantic checker for generic documentations.
It just performs syntactic and semantic checks on generic documentations and writes its diagnostic messages to the standard error stream.
\debuggingtool
\flowgraph{\resource{generic\\documentation} \ar[r] & \toolbox{doccheck} \ar[r] & \resource{diagnostic\\messages}}
\seedocumentation
}

\providecommand{\dochtml}{
\toolsection{dochtml} is an HTML documentation generator for generic documentations.
It processes several generic documentations and assembles all information therein into an HTML document.
\debuggingtool
\flowgraph{\resource{generic\\documentation} \ar[r] & \toolbox{dochtml} \ar[r] & \resource{HTML\\document}}
\seedocumentation
}

\providecommand{\doclatex}{
\toolsection{doclatex} is a Latex documentation generator for generic documentations.
It processes several generic documentations and assembles all information therein into a Latex document.
\debuggingtool
\flowgraph{\resource{generic\\documentation} \ar[r] & \toolbox{doclatex} \ar[r] & \resource{Latex\\document}}
\seedocumentation
}

% intermediate code tools

\providecommand{\cdcheck}{
\toolsection{cdcheck} is a syntactic and semantic checker for intermediate code.
It just performs syntactic and semantic checks on programs written in intermediate code and writes its diagnostic messages to the standard error stream.
\debuggingtool
\flowgraph{\resource{intermediate\\code} \ar[r] & \toolbox{cdcheck} \ar[r] & \resource{diagnostic\\messages}}
\seeassembly\seecode
}

\providecommand{\cdopt}{
\toolsection{cdopt} is an optimizer for intermediate code.
It performs various optimizations on programs written in intermediate code and writes the result to the standard output stream.
\debuggingtool
\flowgraph{\resource{intermediate\\code} \ar[r] & \toolbox{cdopt} \ar[r] & \resource{optimized\\code}}
\seeassembly\seecode
}

\providecommand{\cdrun}{
\toolsection{cdrun} is an interpreter for intermediate code.
It processes and executes programs written in intermediate code.
The following code sections are predefined and have the usual semantics:
\texttt{abort}, \texttt{\_Exit}, \texttt{fflush}, \texttt{floor}, \texttt{fputc}, \texttt{free}, \texttt{getchar}, \texttt{malloc}, and \texttt{putchar}.
Diagnostic messages about invalid operations include the name of the executed code section and the index of the erroneous instruction.
\debuggingtool
\flowgraph{\resource{intermediate\\code} \ar[r] & \toolbox{cdrun} \ar@/u/[r] & \resource{input/\\output} \ar@/d/[l]}
\seeassembly\seecode
}

\providecommand{\cdamda}{
\toolsection{cdamd16} is a compiler for intermediate code targeting the AMD64 hardware architecture.
It generates machine code for AMD64 processors from programs written in intermediate code and stores it in corresponding object files.
The compiler generates machine code for the 16-bit operating mode defined by the AMD64 architecture.
It also creates a debugging information file as well as an assembly file containing a listing of the generated machine code.
\debuggingtool
\flowgraph{\resource{intermediate\\code} \ar[r] & \toolbox{cdamd16} \ar[r] \ar[d] \ar[rd] & \resource{object file} \\ & \resource{assembly\\listing} & \resource{debugging\\information}}
\seeassembly\seeamd\seeobject\seecode\seedebugging
}

\providecommand{\cdamdb}{
\toolsection{cdamd32} is a compiler for intermediate code targeting the AMD64 hardware architecture.
It generates machine code for AMD64 processors from programs written in intermediate code and stores it in corresponding object files.
The compiler generates machine code for the 32-bit operating mode defined by the AMD64 architecture.
It also creates a debugging information file as well as an assembly file containing a listing of the generated machine code.
\debuggingtool
\flowgraph{\resource{intermediate\\code} \ar[r] & \toolbox{cdamd32} \ar[r] \ar[d] \ar[rd] & \resource{object file} \\ & \resource{assembly\\listing} & \resource{debugging\\information}}
\seeassembly\seeamd\seeobject\seecode\seedebugging
}

\providecommand{\cdamdc}{
\toolsection{cdamd64} is a compiler for intermediate code targeting the AMD64 hardware architecture.
It generates machine code for AMD64 processors from programs written in intermediate code and stores it in corresponding object files.
The compiler generates machine code for the 64-bit operating mode defined by the AMD64 architecture.
It also creates a debugging information file as well as an assembly file containing a listing of the generated machine code.
\debuggingtool
\flowgraph{\resource{intermediate\\code} \ar[r] & \toolbox{cdamd64} \ar[r] \ar[d] \ar[rd] & \resource{object file} \\ & \resource{assembly\\listing} & \resource{debugging\\information}}
\seeassembly\seeamd\seeobject\seecode\seedebugging
}

\providecommand{\cdarma}{
\toolsection{cdarma32} is a compiler for intermediate code targeting the ARM hardware architecture.
It generates machine code for ARM processors executing A32 instructions from programs written in intermediate code and stores it in corresponding object files.
It also creates a debugging information file as well as an assembly file containing a listing of the generated machine code.
\debuggingtool
\flowgraph{\resource{intermediate\\code} \ar[r] & \toolbox{cdarma32} \ar[r] \ar[d] \ar[rd] & \resource{object file} \\ & \resource{assembly\\listing} & \resource{debugging\\information}}
\seeassembly\seearm\seeobject\seecode\seedebugging
}

\providecommand{\cdarmb}{
\toolsection{cdarma64} is a compiler for intermediate code targeting the ARM hardware architecture.
It generates machine code for ARM processors executing A64 instructions from programs written in intermediate code and stores it in corresponding object files.
It also creates a debugging information file as well as an assembly file containing a listing of the generated machine code.
\debuggingtool
\flowgraph{\resource{intermediate\\code} \ar[r] & \toolbox{cdarma64} \ar[r] \ar[d] \ar[rd] & \resource{object file} \\ & \resource{assembly\\listing} & \resource{debugging\\information}}
\seeassembly\seearm\seeobject\seecode\seedebugging
}

\providecommand{\cdarmc}{
\toolsection{cdarmt32} is a compiler for intermediate code targeting the ARM hardware architecture.
It generates machine code for ARM processors without floating-point extension executing T32 instructions from programs written in intermediate code and stores it in corresponding object files.
It also creates a debugging information file as well as an assembly file containing a listing of the generated machine code.
\debuggingtool
\flowgraph{\resource{intermediate\\code} \ar[r] & \toolbox{cdarmt32} \ar[r] \ar[d] \ar[rd] & \resource{object file} \\ & \resource{assembly\\listing} & \resource{debugging\\information}}
\seeassembly\seearm\seeobject\seecode\seedebugging
}

\providecommand{\cdarmcfpe}{
\toolsection{cdarmt32fpe} is a compiler for intermediate code targeting the ARM hardware architecture.
It generates machine code for ARM processors with floating-point extension executing T32 instructions from programs written in intermediate code and stores it in corresponding object files.
It also creates a debugging information file as well as an assembly file containing a listing of the generated machine code.
\debuggingtool
\flowgraph{\resource{intermediate\\code} \ar[r] & \toolbox{cdarmt32fpe} \ar[r] \ar[d] \ar[rd] & \resource{object file} \\ & \resource{assembly\\listing} & \resource{debugging\\information}}
\seeassembly\seearm\seeobject\seecode\seedebugging
}

\providecommand{\cdavr}{
\toolsection{cdavr} is a compiler for intermediate code targeting the AVR hardware architecture.
It generates machine code for AVR processors from programs written in intermediate code and stores it in corresponding object files.
It also creates a debugging information file as well as an assembly file containing a listing of the generated machine code.
\debuggingtool
\flowgraph{\resource{intermediate\\code} \ar[r] & \toolbox{cdavr} \ar[r] \ar[d] \ar[rd] & \resource{object file} \\ & \resource{assembly\\listing} & \resource{debugging\\information}}
\seeassembly\seeavr\seeobject\seecode\seedebugging
}

\providecommand{\cdavrtt}{
\toolsection{cdavr32} is a compiler for intermediate code targeting the AVR32 hardware architecture.
It generates machine code for AVR32 processors from programs written in intermediate code and stores it in corresponding object files.
It also creates a debugging information file as well as an assembly file containing a listing of the generated machine code.
\debuggingtool
\flowgraph{\resource{intermediate\\code} \ar[r] & \toolbox{cdavr32} \ar[r] \ar[d] \ar[rd] & \resource{object file} \\ & \resource{assembly\\listing} & \resource{debugging\\information}}
\seeassembly\seeavrtt\seeobject\seecode\seedebugging
}

\providecommand{\cdmabk}{
\toolsection{cdm68k} is a compiler for intermediate code targeting the M68000 hardware architecture.
It generates machine code for M68000 processors from programs written in intermediate code and stores it in corresponding object files.
It also creates a debugging information file as well as an assembly file containing a listing of the generated machine code.
\debuggingtool
\flowgraph{\resource{intermediate\\code} \ar[r] & \toolbox{cdm68k} \ar[r] \ar[d] \ar[rd] & \resource{object file} \\ & \resource{assembly\\listing} & \resource{debugging\\information}}
\seeassembly\seemabk\seeobject\seecode\seedebugging
}

\providecommand{\cdmibl}{
\toolsection{cdmibl} is a compiler for intermediate code targeting the MicroBlaze hardware architecture.
It generates machine code for MicroBlaze processors from programs written in intermediate code and stores it in corresponding object files.
It also creates a debugging information file as well as an assembly file containing a listing of the generated machine code.
\debuggingtool
\flowgraph{\resource{intermediate\\code} \ar[r] & \toolbox{cdmibl} \ar[r] \ar[d] \ar[rd] & \resource{object file} \\ & \resource{assembly\\listing} & \resource{debugging\\information}}
\seeassembly\seemibl\seeobject\seecode\seedebugging
}

\providecommand{\cdmipsa}{
\toolsection{cdmips32} is a compiler for intermediate code targeting the MIPS32 hardware architecture.
It generates machine code for MIPS32 processors from programs written in intermediate code and stores it in corresponding object files.
It also creates a debugging information file as well as an assembly file containing a listing of the generated machine code.
\debuggingtool
\flowgraph{\resource{intermediate\\code} \ar[r] & \toolbox{cdmips32} \ar[r] \ar[d] \ar[rd] & \resource{object file} \\ & \resource{assembly\\listing} & \resource{debugging\\information}}
\seeassembly\seemips\seeobject\seecode\seedebugging
}

\providecommand{\cdmipsb}{
\toolsection{cdmips64} is a compiler for intermediate code targeting the MIPS64 hardware architecture.
It generates machine code for MIPS64 processors from programs written in intermediate code and stores it in corresponding object files.
It also creates a debugging information file as well as an assembly file containing a listing of the generated machine code.
\debuggingtool
\flowgraph{\resource{intermediate\\code} \ar[r] & \toolbox{cdmips64} \ar[r] \ar[d] \ar[rd] & \resource{object file} \\ & \resource{assembly\\listing} & \resource{debugging\\information}}
\seeassembly\seemips\seeobject\seecode\seedebugging
}

\providecommand{\cdmmix}{
\toolsection{cdmmix} is a compiler for intermediate code targeting the MMIX hardware architecture.
It generates machine code for MMIX processors from programs written in intermediate code and stores it in corresponding object files.
It also creates a debugging information file as well as an assembly file containing a listing of the generated machine code.
\debuggingtool
\flowgraph{\resource{intermediate\\code} \ar[r] & \toolbox{cdmmix} \ar[r] \ar[d] \ar[rd] & \resource{object file} \\ & \resource{assembly\\listing} & \resource{debugging\\information}}
\seeassembly\seemmix\seeobject\seecode\seedebugging
}

\providecommand{\cdorok}{
\toolsection{cdor1k} is a compiler for intermediate code targeting the OpenRISC 1000 hardware architecture.
It generates machine code for OpenRISC 1000 processors from programs written in intermediate code and stores it in corresponding object files.
It also creates a debugging information file as well as an assembly file containing a listing of the generated machine code.
\debuggingtool
\flowgraph{\resource{intermediate\\code} \ar[r] & \toolbox{cdor1k} \ar[r] \ar[d] \ar[rd] & \resource{object file} \\ & \resource{assembly\\listing} & \resource{debugging\\information}}
\seeassembly\seeorok\seeobject\seecode\seedebugging
}

\providecommand{\cdppca}{
\toolsection{cdppc32} is a compiler for intermediate code targeting the PowerPC hardware architecture.
It generates machine code for PowerPC processors from programs written in intermediate code and stores it in corresponding object files.
The compiler generates machine code for the 32-bit operating mode defined by the PowerPC architecture.
It also creates a debugging information file as well as an assembly file containing a listing of the generated machine code.
\debuggingtool
\flowgraph{\resource{intermediate\\code} \ar[r] & \toolbox{cdppc32} \ar[r] \ar[d] \ar[rd] & \resource{object file} \\ & \resource{assembly\\listing} & \resource{debugging\\information}}
\seeassembly\seeppc\seeobject\seecode\seedebugging
}

\providecommand{\cdppcb}{
\toolsection{cdppc64} is a compiler for intermediate code targeting the PowerPC hardware architecture.
It generates machine code for PowerPC processors from programs written in intermediate code and stores it in corresponding object files.
The compiler generates machine code for the 64-bit operating mode defined by the PowerPC architecture.
It also creates a debugging information file as well as an assembly file containing a listing of the generated machine code.
\debuggingtool
\flowgraph{\resource{intermediate\\code} \ar[r] & \toolbox{cdppc64} \ar[r] \ar[d] \ar[rd] & \resource{object file} \\ & \resource{assembly\\listing} & \resource{debugging\\information}}
\seeassembly\seeppc\seeobject\seecode\seedebugging
}

\providecommand{\cdrisc}{
\toolsection{cdrisc} is a compiler for intermediate code targeting the RISC hardware architecture.
It generates machine code for RISC processors from programs written in intermediate code and stores it in corresponding object files.
It also creates a debugging information file as well as an assembly file containing a listing of the generated machine code.
\debuggingtool
\flowgraph{\resource{intermediate\\code} \ar[r] & \toolbox{cdrisc} \ar[r] \ar[d] \ar[rd] & \resource{object file} \\ & \resource{assembly\\listing} & \resource{debugging\\information}}
\seeassembly\seerisc\seeobject\seecode\seedebugging
}

\providecommand{\cdwasm}{
\toolsection{cdwasm} is a compiler for intermediate code targeting the WebAssembly architecture.
It generates machine code for WebAssembly targets from programs written in intermediate code and stores it in corresponding object files.
It also creates a debugging information file as well as an assembly file containing a listing of the generated machine code.
\debuggingtool
\flowgraph{\resource{intermediate\\code} \ar[r] & \toolbox{cdwasm} \ar[r] \ar[d] \ar[rd] & \resource{object file} \\ & \resource{assembly\\listing} & \resource{debugging\\information}}
\seeassembly\seewasm\seeobject\seecode\seedebugging
}

% C++ tools

\providecommand{\cppprep}{
\toolsection{cppprep} is a preprocessor for the \cpp{} programming language.
It preprocesses source code according to the rules of \cpp{} and writes it to the standard output stream.
Only the macro names \texttt{\_\_DATE\_\_}, \texttt{\_\_FILE\_\_}, \texttt{\_\_LINE\_\_}, and \texttt{\_\_TIME\_\_} are predefined.
\flowgraph{\resource{\cpp{} or other\\source code} \ar[r] & \toolbox{cppprep} \ar[r] & \resource{preprocessed\\source code} \\ & \variable{ECSINCLUDE} \ar[u]}
\seecpp
}

\providecommand{\cppprint}{
\toolsection{cppprint} is a pretty printer for the \cpp{} programming language.
It reformats the source code of \cpp{} programs and writes it to the standard output stream.
\flowgraph{\resource{\cpp{}\\source code} \ar[r] & \toolbox{cppprint} \ar[r] & \resource{reformatted\\source code} \\ & \variable{ECSINCLUDE} \ar[u]}
\seecpp
}

\providecommand{\cppcheck}{
\toolsection{cppcheck} is a syntactic and semantic checker for the \cpp{} programming language.
It just performs syntactic and semantic checks on \cpp{} programs and writes its diagnostic messages to the standard error stream.
\flowgraph{\resource{\cpp{}\\source code} \ar[r] & \toolbox{cppcheck} \ar[r] & \resource{diagnostic\\messages} \\ & \variable{ECSINCLUDE} \ar[u]}
\seecpp
}

\providecommand{\cppdump}{
\toolsection{cppdump} is a serializer for the \cpp{} programming language.
It dumps the complete internal representation of programs written in \cpp{} into an XML document.
\debuggingtool
\flowgraph{\resource{\cpp{}\\source code} \ar[r] & \toolbox{cppdump} \ar[r] & \resource{internal\\representation} \\ & \variable{ECSINCLUDE} \ar[u]}
\seecpp
}

\providecommand{\cpprun}{
\toolsection{cpprun} is an interpreter for the \cpp{} programming language.
It processes and executes programs written in \cpp{}.
The macro \texttt{\_\_run\_\_} is predefined in order to enable programmers to identify this tool while interpreting.
\flowgraph{\resource{\cpp{}\\source code} \ar[r] & \toolbox{cpprun} \ar@/u/[r] & \resource{input/\\output} \ar@/d/[l] \\ & \variable{ECSINCLUDE} \ar[u]}
\seecpp
}

\providecommand{\cppdoc}{
\toolsection{cppdoc} is a generic documentation generator for the \cpp{} programming language.
It processes several \cpp{} source files and assembles all information therein into a generic documentation.
\debuggingtool
\flowgraph{\resource{\cpp{}\\source code} \ar[r] & \toolbox{cppdoc} \ar[r] & \resource{generic\\documentation} \\ & \variable{ECSINCLUDE} \ar[u]}
\seecpp\seedocumentation
}

\providecommand{\cpphtml}{
\toolsection{cpphtml} is an HTML documentation generator for the \cpp{} programming language.
It processes several \cpp{} source files and assembles all information therein into an HTML document.
\flowgraph{\resource{\cpp{}\\source code} \ar[r] & \toolbox{cpphtml} \ar[r] & \resource{HTML\\document} \\ & \variable{ECSINCLUDE} \ar[u]}
\seecpp\seedocumentation
}

\providecommand{\cpplatex}{
\toolsection{cpplatex} is a Latex documentation generator for the \cpp{} programming language.
It processes several \cpp{} source files and assembles all information therein into a Latex document.
\flowgraph{\resource{\cpp{}\\source code} \ar[r] & \toolbox{cpplatex} \ar[r] & \resource{Latex\\document} \\ & \variable{ECSINCLUDE} \ar[u]}
\seecpp\seedocumentation
}

\providecommand{\cppcode}{
\toolsection{cppcode} is an intermediate code generator for the \cpp{} programming language.
It generates intermediate code from programs written in \cpp{} and stores it in corresponding assembly files.
The macro \texttt{\_\_code\_\_} is predefined in order to enable programmers to identify this tool while generating intermediate code.
Programs generated with this tool require additional runtime support that is stored in the \file{cpp\-code\-run} library file.
\debuggingtool
\flowgraph{\resource{\cpp{}\\source code} \ar[r] & \toolbox{cppcode} \ar[r] & \resource{intermediate\\code} \\ & \variable{ECSINCLUDE} \ar[u]}
\seecpp\seeassembly\seecode
}

\providecommand{\cppamda}{
\toolsection{cppamd16} is a compiler for the \cpp{} programming language targeting the AMD64 hardware architecture.
It generates machine code for AMD64 processors from programs written in \cpp{} and stores it in corresponding object files.
The compiler generates machine code for the 16-bit operating mode defined by the AMD64 architecture.
For debugging purposes, it also creates a debugging information file as well as an assembly file containing a listing of the generated machine code.
The macro \texttt{\_\_amd16\_\_} is predefined in order to enable programmers to identify this tool and its target architecture while compiling.
Programs generated with this compiler require additional runtime support that is stored in the \file{cpp\-amd16\-run} library file.
\flowgraph{\resource{\cpp{}\\source code} \ar[r] & \toolbox{cppamd16} \ar[r] \ar[d] \ar[rd] & \resource{object file} \\ \variable{ECSINCLUDE} \ar[ru] & \resource{debugging\\information} & \resource{assembly\\listing}}
\seecpp\seeassembly\seeamd\seeobject\seedebugging
}

\providecommand{\cppamdb}{
\toolsection{cppamd32} is a compiler for the \cpp{} programming language targeting the AMD64 hardware architecture.
It generates machine code for AMD64 processors from programs written in \cpp{} and stores it in corresponding object files.
The compiler generates machine code for the 32-bit operating mode defined by the AMD64 architecture.
For debugging purposes, it also creates a debugging information file as well as an assembly file containing a listing of the generated machine code.
The macro \texttt{\_\_amd32\_\_} is predefined in order to enable programmers to identify this tool and its target architecture while compiling.
Programs generated with this compiler require additional runtime support that is stored in the \file{cpp\-amd32\-run} library file.
\flowgraph{\resource{\cpp{}\\source code} \ar[r] & \toolbox{cppamd32} \ar[r] \ar[d] \ar[rd] & \resource{object file} \\ \variable{ECSINCLUDE} \ar[ru] & \resource{debugging\\information} & \resource{assembly\\listing}}
\seecpp\seeassembly\seeamd\seeobject\seedebugging
}

\providecommand{\cppamdc}{
\toolsection{cppamd64} is a compiler for the \cpp{} programming language targeting the AMD64 hardware architecture.
It generates machine code for AMD64 processors from programs written in \cpp{} and stores it in corresponding object files.
The compiler generates machine code for the 64-bit operating mode defined by the AMD64 architecture.
For debugging purposes, it also creates a debugging information file as well as an assembly file containing a listing of the generated machine code.
The macro \texttt{\_\_amd64\_\_} is predefined in order to enable programmers to identify this tool and its target architecture while compiling.
Programs generated with this compiler require additional runtime support that is stored in the \file{cpp\-amd64\-run} library file.
\flowgraph{\resource{\cpp{}\\source code} \ar[r] & \toolbox{cppamd64} \ar[r] \ar[d] \ar[rd] & \resource{object file} \\ \variable{ECSINCLUDE} \ar[ru] & \resource{debugging\\information} & \resource{assembly\\listing}}
\seecpp\seeassembly\seeamd\seeobject\seedebugging
}

\providecommand{\cpparma}{
\toolsection{cpparma32} is a compiler for the \cpp{} programming language targeting the ARM hardware architecture.
It generates machine code for ARM processors executing A32 instructions from programs written in \cpp{} and stores it in corresponding object files.
For debugging purposes, it also creates a debugging information file as well as an assembly file containing a listing of the generated machine code.
The macro \texttt{\_\_arma32\_\_} is predefined in order to enable programmers to identify this tool and its target architecture while compiling.
Programs generated with this compiler require additional runtime support that is stored in the \file{cpp\-arma32\-run} library file.
\flowgraph{\resource{\cpp{}\\source code} \ar[r] & \toolbox{cpparma32} \ar[r] \ar[d] \ar[rd] & \resource{object file} \\ \variable{ECSINCLUDE} \ar[ru] & \resource{debugging\\information} & \resource{assembly\\listing}}
\seecpp\seeassembly\seearm\seeobject\seedebugging
}

\providecommand{\cpparmb}{
\toolsection{cpparma64} is a compiler for the \cpp{} programming language targeting the ARM hardware architecture.
It generates machine code for ARM processors executing A64 instructions from programs written in \cpp{} and stores it in corresponding object files.
For debugging purposes, it also creates a debugging information file as well as an assembly file containing a listing of the generated machine code.
The macro \texttt{\_\_arma64\_\_} is predefined in order to enable programmers to identify this tool and its target architecture while compiling.
Programs generated with this compiler require additional runtime support that is stored in the \file{cpp\-arma64\-run} library file.
\flowgraph{\resource{\cpp{}\\source code} \ar[r] & \toolbox{cpparma64} \ar[r] \ar[d] \ar[rd] & \resource{object file} \\ \variable{ECSINCLUDE} \ar[ru] & \resource{debugging\\information} & \resource{assembly\\listing}}
\seecpp\seeassembly\seearm\seeobject\seedebugging
}

\providecommand{\cpparmc}{
\toolsection{cpparmt32} is a compiler for the \cpp{} programming language targeting the ARM hardware architecture.
It generates machine code for ARM processors without floating-point extension executing T32 instructions from programs written in \cpp{} and stores it in corresponding object files.
For debugging purposes, it also creates a debugging information file as well as an assembly file containing a listing of the generated machine code.
The macro \texttt{\_\_armt32\_\_} is predefined in order to enable programmers to identify this tool and its target architecture while compiling.
Programs generated with this compiler require additional runtime support that is stored in the \file{cpp\-armt32\-run} library file.
\flowgraph{\resource{\cpp{}\\source code} \ar[r] & \toolbox{cpparmt32} \ar[r] \ar[d] \ar[rd] & \resource{object file} \\ \variable{ECSINCLUDE} \ar[ru] & \resource{debugging\\information} & \resource{assembly\\listing}}
\seecpp\seeassembly\seearm\seeobject\seedebugging
}

\providecommand{\cpparmcfpe}{
\toolsection{cpparmt32fpe} is a compiler for the \cpp{} programming language targeting the ARM hardware architecture.
It generates machine code for ARM processors with floating-point extension executing T32 instructions from programs written in \cpp{} and stores it in corresponding object files.
For debugging purposes, it also creates a debugging information file as well as an assembly file containing a listing of the generated machine code.
The macro \texttt{\_\_armt32fpe\_\_} is predefined in order to enable programmers to identify this tool and its target architecture while compiling.
Programs generated with this compiler require additional runtime support that is stored in the \file{cpp\-armt32\-fpe\-run} library file.
\flowgraph{\resource{\cpp{}\\source code} \ar[r] & \toolbox{cpparmt32fpe} \ar[r] \ar[d] \ar[rd] & \resource{object file} \\ \variable{ECSINCLUDE} \ar[ru] & \resource{debugging\\information} & \resource{assembly\\listing}}
\seecpp\seeassembly\seearm\seeobject\seedebugging
}

\providecommand{\cppavr}{
\toolsection{cppavr} is a compiler for the \cpp{} programming language targeting the AVR hardware architecture.
It generates machine code for AVR processors from programs written in \cpp{} and stores it in corresponding object files.
For debugging purposes, it also creates a debugging information file as well as an assembly file containing a listing of the generated machine code.
The macro \texttt{\_\_avr\_\_} is predefined in order to enable programmers to identify this tool and its target architecture while compiling.
Programs generated with this compiler require additional runtime support that is stored in the \file{cpp\-avr\-run} library file.
\flowgraph{\resource{\cpp{}\\source code} \ar[r] & \toolbox{cppavr} \ar[r] \ar[d] \ar[rd] & \resource{object file} \\ \variable{ECSINCLUDE} \ar[ru] & \resource{debugging\\information} & \resource{assembly\\listing}}
\seecpp\seeassembly\seeavr\seeobject\seedebugging
}

\providecommand{\cppavrtt}{
\toolsection{cppavr32} is a compiler for the \cpp{} programming language targeting the AVR32 hardware architecture.
It generates machine code for AVR32 processors from programs written in \cpp{} and stores it in corresponding object files.
For debugging purposes, it also creates a debugging information file as well as an assembly file containing a listing of the generated machine code.
The macro \texttt{\_\_avr32\_\_} is predefined in order to enable programmers to identify this tool and its target architecture while compiling.
Programs generated with this compiler require additional runtime support that is stored in the \file{cpp\-avr32\-run} library file.
\flowgraph{\resource{\cpp{}\\source code} \ar[r] & \toolbox{cppavr32} \ar[r] \ar[d] \ar[rd] & \resource{object file} \\ \variable{ECSINCLUDE} \ar[ru] & \resource{debugging\\information} & \resource{assembly\\listing}}
\seecpp\seeassembly\seeavrtt\seeobject\seedebugging
}

\providecommand{\cppmabk}{
\toolsection{cppm68k} is a compiler for the \cpp{} programming language targeting the M68000 hardware architecture.
It generates machine code for M68000 processors from programs written in \cpp{} and stores it in corresponding object files.
For debugging purposes, it also creates a debugging information file as well as an assembly file containing a listing of the generated machine code.
The macro \texttt{\_\_m68k\_\_} is predefined in order to enable programmers to identify this tool and its target architecture while compiling.
Programs generated with this compiler require additional runtime support that is stored in the \file{cpp\-m68k\-run} library file.
\flowgraph{\resource{\cpp{}\\source code} \ar[r] & \toolbox{cppm68k} \ar[r] \ar[d] \ar[rd] & \resource{object file} \\ \variable{ECSINCLUDE} \ar[ru] & \resource{debugging\\information} & \resource{assembly\\listing}}
\seecpp\seeassembly\seemabk\seeobject\seedebugging
}

\providecommand{\cppmibl}{
\toolsection{cppmibl} is a compiler for the \cpp{} programming language targeting the MicroBlaze hardware architecture.
It generates machine code for MicroBlaze processors from programs written in \cpp{} and stores it in corresponding object files.
For debugging purposes, it also creates a debugging information file as well as an assembly file containing a listing of the generated machine code.
The macro \texttt{\_\_mibl\_\_} is predefined in order to enable programmers to identify this tool and its target architecture while compiling.
Programs generated with this compiler require additional runtime support that is stored in the \file{cpp\-mibl\-run} library file.
\flowgraph{\resource{\cpp{}\\source code} \ar[r] & \toolbox{cppmibl} \ar[r] \ar[d] \ar[rd] & \resource{object file} \\ \variable{ECSINCLUDE} \ar[ru] & \resource{debugging\\information} & \resource{assembly\\listing}}
\seecpp\seeassembly\seemibl\seeobject\seedebugging
}

\providecommand{\cppmipsa}{
\toolsection{cppmips32} is a compiler for the \cpp{} programming language targeting the MIPS32 hardware architecture.
It generates machine code for MIPS32 processors from programs written in \cpp{} and stores it in corresponding object files.
For debugging purposes, it also creates a debugging information file as well as an assembly file containing a listing of the generated machine code.
The macro \texttt{\_\_mips32\_\_} is predefined in order to enable programmers to identify this tool and its target architecture while compiling.
Programs generated with this compiler require additional runtime support that is stored in the \file{cpp\-mips32\-run} library file.
\flowgraph{\resource{\cpp{}\\source code} \ar[r] & \toolbox{cppmips32} \ar[r] \ar[d] \ar[rd] & \resource{object file} \\ \variable{ECSINCLUDE} \ar[ru] & \resource{debugging\\information} & \resource{assembly\\listing}}
\seecpp\seeassembly\seemips\seeobject\seedebugging
}

\providecommand{\cppmipsb}{
\toolsection{cppmips64} is a compiler for the \cpp{} programming language targeting the MIPS64 hardware architecture.
It generates machine code for MIPS64 processors from programs written in \cpp{} and stores it in corresponding object files.
For debugging purposes, it also creates a debugging information file as well as an assembly file containing a listing of the generated machine code.
The macro \texttt{\_\_mips64\_\_} is predefined in order to enable programmers to identify this tool and its target architecture while compiling.
Programs generated with this compiler require additional runtime support that is stored in the \file{cpp\-mips64\-run} library file.
\flowgraph{\resource{\cpp{}\\source code} \ar[r] & \toolbox{cppmips64} \ar[r] \ar[d] \ar[rd] & \resource{object file} \\ \variable{ECSINCLUDE} \ar[ru] & \resource{debugging\\information} & \resource{assembly\\listing}}
\seecpp\seeassembly\seemips\seeobject\seedebugging
}

\providecommand{\cppmmix}{
\toolsection{cppmmix} is a compiler for the \cpp{} programming language targeting the MMIX hardware architecture.
It generates machine code for MMIX processors from programs written in \cpp{} and stores it in corresponding object files.
For debugging purposes, it also creates a debugging information file as well as an assembly file containing a listing of the generated machine code.
The macro \texttt{\_\_mmix\_\_} is predefined in order to enable programmers to identify this tool and its target architecture while compiling.
Programs generated with this compiler require additional runtime support that is stored in the \file{cpp\-mmix\-run} library file.
\flowgraph{\resource{\cpp{}\\source code} \ar[r] & \toolbox{cppmmix} \ar[r] \ar[d] \ar[rd] & \resource{object file} \\ \variable{ECSINCLUDE} \ar[ru] & \resource{debugging\\information} & \resource{assembly\\listing}}
\seecpp\seeassembly\seemmix\seeobject\seedebugging
}

\providecommand{\cpporok}{
\toolsection{cppor1k} is a compiler for the \cpp{} programming language targeting the OpenRISC 1000 hardware architecture.
It generates machine code for OpenRISC 1000 processors from programs written in \cpp{} and stores it in corresponding object files.
For debugging purposes, it also creates a debugging information file as well as an assembly file containing a listing of the generated machine code.
The macro \texttt{\_\_or1k\_\_} is predefined in order to enable programmers to identify this tool and its target architecture while compiling.
Programs generated with this compiler require additional runtime support that is stored in the \file{cpp\-or1k\-run} library file.
\flowgraph{\resource{\cpp{}\\source code} \ar[r] & \toolbox{cppor1k} \ar[r] \ar[d] \ar[rd] & \resource{object file} \\ \variable{ECSINCLUDE} \ar[ru] & \resource{debugging\\information} & \resource{assembly\\listing}}
\seecpp\seeassembly\seeorok\seeobject\seedebugging
}

\providecommand{\cppppca}{
\toolsection{cppppc32} is a compiler for the \cpp{} programming language targeting the PowerPC hardware architecture.
It generates machine code for PowerPC processors from programs written in \cpp{} and stores it in corresponding object files.
The compiler generates machine code for the 32-bit operating mode defined by the PowerPC architecture.
For debugging purposes, it also creates a debugging information file as well as an assembly file containing a listing of the generated machine code.
The macro \texttt{\_\_ppc32\_\_} is predefined in order to enable programmers to identify this tool and its target architecture while compiling.
Programs generated with this compiler require additional runtime support that is stored in the \file{cpp\-ppc32\-run} library file.
\flowgraph{\resource{\cpp{}\\source code} \ar[r] & \toolbox{cppppc32} \ar[r] \ar[d] \ar[rd] & \resource{object file} \\ \variable{ECSINCLUDE} \ar[ru] & \resource{debugging\\information} & \resource{assembly\\listing}}
\seecpp\seeassembly\seeppc\seeobject\seedebugging
}

\providecommand{\cppppcb}{
\toolsection{cppppc64} is a compiler for the \cpp{} programming language targeting the PowerPC hardware architecture.
It generates machine code for PowerPC processors from programs written in \cpp{} and stores it in corresponding object files.
The compiler generates machine code for the 64-bit operating mode defined by the PowerPC architecture.
For debugging purposes, it also creates a debugging information file as well as an assembly file containing a listing of the generated machine code.
The macro \texttt{\_\_ppc64\_\_} is predefined in order to enable programmers to identify this tool and its target architecture while compiling.
Programs generated with this compiler require additional runtime support that is stored in the \file{cpp\-ppc64\-run} library file.
\flowgraph{\resource{\cpp{}\\source code} \ar[r] & \toolbox{cppppc64} \ar[r] \ar[d] \ar[rd] & \resource{object file} \\ \variable{ECSINCLUDE} \ar[ru] & \resource{debugging\\information} & \resource{assembly\\listing}}
\seecpp\seeassembly\seeppc\seeobject\seedebugging
}

\providecommand{\cpprisc}{
\toolsection{cpprisc} is a compiler for the \cpp{} programming language targeting the RISC hardware architecture.
It generates machine code for RISC processors from programs written in \cpp{} and stores it in corresponding object files.
For debugging purposes, it also creates a debugging information file as well as an assembly file containing a listing of the generated machine code.
The macro \texttt{\_\_risc\_\_} is predefined in order to enable programmers to identify this tool and its target architecture while compiling.
Programs generated with this compiler require additional runtime support that is stored in the \file{cpp\-risc\-run} library file.
\flowgraph{\resource{\cpp{}\\source code} \ar[r] & \toolbox{cpprisc} \ar[r] \ar[d] \ar[rd] & \resource{object file} \\ \variable{ECSINCLUDE} \ar[ru] & \resource{debugging\\information} & \resource{assembly\\listing}}
\seecpp\seeassembly\seerisc\seeobject\seedebugging
}

\providecommand{\cppwasm}{
\toolsection{cppwasm} is a compiler for the \cpp{} programming language targeting the WebAssembly architecture.
It generates machine code for WebAssembly targets from programs written in \cpp{} and stores it in corresponding object files.
For debugging purposes, it also creates a debugging information file as well as an assembly file containing a listing of the generated machine code.
The macro \texttt{\_\_wasm\_\_} is predefined in order to enable programmers to identify this tool and its target architecture while compiling.
Programs generated with this compiler require additional runtime support that is stored in the \file{cpp\-wasm\-run} library file.
\flowgraph{\resource{\cpp{}\\source code} \ar[r] & \toolbox{cppwasm} \ar[r] \ar[d] \ar[rd] & \resource{object file} \\ \variable{ECSINCLUDE} \ar[ru] & \resource{debugging\\information} & \resource{assembly\\listing}}
\seecpp\seeassembly\seewasm\seeobject\seedebugging
}

% FALSE tools

\providecommand{\falprint}{
\toolsection{falprint} is a pretty printer for the FALSE programming language.
It reformats the source code of FALSE programs and writes it to the standard output stream.
\flowgraph{\resource{FALSE\\source code} \ar[r] & \toolbox{falprint} \ar[r] & \resource{reformatted\\source code}}
\seefalse
}

\providecommand{\falcheck}{
\toolsection{falcheck} is a syntactic and semantic checker for the FALSE programming language.
It just performs syntactic and semantic checks on FALSE programs and writes its diagnostic messages to the standard error stream.
\flowgraph{\resource{FALSE\\source code} \ar[r] & \toolbox{falcheck} \ar[r] & \resource{diagnostic\\messages}}
\seefalse
}

\providecommand{\faldump}{
\toolsection{faldump} is a serializer for the FALSE programming language.
It dumps the complete internal representation of programs written in FALSE into an XML document.
\debuggingtool
\flowgraph{\resource{FALSE\\source code} \ar[r] & \toolbox{faldump} \ar[r] & \resource{internal\\representation}}
\seefalse
}

\providecommand{\falrun}{
\toolsection{falrun} is an interpreter for the FALSE programming language.
It processes and executes programs written in FALSE\@.
\flowgraph{\resource{FALSE\\source code} \ar[r] & \toolbox{falrun} \ar@/u/[r] & \resource{input/\\output} \ar@/d/[l]}
\seefalse
}

\providecommand{\falcpp}{
\toolsection{falcpp} is a transpiler for the FALSE programming language.
It translates programs written in FALSE into \cpp{} programs and stores them in corresponding source files.
\flowgraph{\resource{FALSE\\source code} \ar[r] & \toolbox{falcpp} \ar[r] & \resource{\cpp{}\\source file}}
\seefalse\seecpp
}

\providecommand{\falcode}{
\toolsection{falcode} is an intermediate code generator for the FALSE programming language.
It generates intermediate code from programs written in FALSE and stores it in corresponding assembly files.
\debuggingtool
\flowgraph{\resource{FALSE\\source code} \ar[r] & \toolbox{falcode} \ar[r] & \resource{intermediate\\code}}
\seefalse\seeassembly\seecode
}

\providecommand{\falamda}{
\toolsection{falamd16} is a compiler for the FALSE programming language targeting the AMD64 hardware architecture.
It generates machine code for AMD64 processors from programs written in FALSE and stores it in corresponding object files.
The compiler generates machine code for the 16-bit operating mode defined by the AMD64 architecture.
\flowgraph{\resource{FALSE\\source code} \ar[r] & \toolbox{falamd16} \ar[r] & \resource{object file}}
\seefalse\seeamd\seeobject
}

\providecommand{\falamdb}{
\toolsection{falamd32} is a compiler for the FALSE programming language targeting the AMD64 hardware architecture.
It generates machine code for AMD64 processors from programs written in FALSE and stores it in corresponding object files.
The compiler generates machine code for the 32-bit operating mode defined by the AMD64 architecture.
\flowgraph{\resource{FALSE\\source code} \ar[r] & \toolbox{falamd32} \ar[r] & \resource{object file}}
\seefalse\seeamd\seeobject
}

\providecommand{\falamdc}{
\toolsection{falamd64} is a compiler for the FALSE programming language targeting the AMD64 hardware architecture.
It generates machine code for AMD64 processors from programs written in FALSE and stores it in corresponding object files.
The compiler generates machine code for the 64-bit operating mode defined by the AMD64 architecture.
\flowgraph{\resource{FALSE\\source code} \ar[r] & \toolbox{falamd64} \ar[r] & \resource{object file}}
\seefalse\seeamd\seeobject
}

\providecommand{\falarma}{
\toolsection{falarma32} is a compiler for the FALSE programming language targeting the ARM hardware architecture.
It generates machine code for ARM processors executing A32 instructions from programs written in FALSE and stores it in corresponding object files.
\flowgraph{\resource{FALSE\\source code} \ar[r] & \toolbox{falarma32} \ar[r] & \resource{object file}}
\seefalse\seearm\seeobject
}

\providecommand{\falarmb}{
\toolsection{falarma64} is a compiler for the FALSE programming language targeting the ARM hardware architecture.
It generates machine code for ARM processors executing A64 instructions from programs written in FALSE and stores it in corresponding object files.
\flowgraph{\resource{FALSE\\source code} \ar[r] & \toolbox{falarma64} \ar[r] & \resource{object file}}
\seefalse\seearm\seeobject
}

\providecommand{\falarmc}{
\toolsection{falarmt32} is a compiler for the FALSE programming language targeting the ARM hardware architecture.
It generates machine code for ARM processors without floating-point extension executing T32 instructions from programs written in FALSE and stores it in corresponding object files.
\flowgraph{\resource{FALSE\\source code} \ar[r] & \toolbox{falarmt32} \ar[r] & \resource{object file}}
\seefalse\seearm\seeobject
}

\providecommand{\falarmcfpe}{
\toolsection{falarmt32fpe} is a compiler for the FALSE programming language targeting the ARM hardware architecture.
It generates machine code for ARM processors with floating-point extension executing T32 instructions from programs written in FALSE and stores it in corresponding object files.
\flowgraph{\resource{FALSE\\source code} \ar[r] & \toolbox{falarmt32fpe} \ar[r] & \resource{object file}}
\seefalse\seearm\seeobject
}

\providecommand{\falavr}{
\toolsection{falavr} is a compiler for the FALSE programming language targeting the AVR hardware architecture.
It generates machine code for AVR processors from programs written in FALSE and stores it in corresponding object files.
\flowgraph{\resource{FALSE\\source code} \ar[r] & \toolbox{falavr} \ar[r] & \resource{object file}}
\seefalse\seeavr\seeobject
}

\providecommand{\falavrtt}{
\toolsection{falavr32} is a compiler for the FALSE programming language targeting the AVR32 hardware architecture.
It generates machine code for AVR32 processors from programs written in FALSE and stores it in corresponding object files.
\flowgraph{\resource{FALSE\\source code} \ar[r] & \toolbox{falavr32} \ar[r] & \resource{object file}}
\seefalse\seeavrtt\seeobject
}

\providecommand{\falmabk}{
\toolsection{falm68k} is a compiler for the FALSE programming language targeting the M68000 hardware architecture.
It generates machine code for M68000 processors from programs written in FALSE and stores it in corresponding object files.
\flowgraph{\resource{FALSE\\source code} \ar[r] & \toolbox{falm68k} \ar[r] & \resource{object file}}
\seefalse\seemabk\seeobject
}

\providecommand{\falmibl}{
\toolsection{falmibl} is a compiler for the FALSE programming language targeting the MicroBlaze hardware architecture.
It generates machine code for MicroBlaze processors from programs written in FALSE and stores it in corresponding object files.
\flowgraph{\resource{FALSE\\source code} \ar[r] & \toolbox{falmibl} \ar[r] & \resource{object file}}
\seefalse\seemibl\seeobject
}

\providecommand{\falmipsa}{
\toolsection{falmips32} is a compiler for the FALSE programming language targeting the MIPS32 hardware architecture.
It generates machine code for MIPS32 processors from programs written in FALSE and stores it in corresponding object files.
\flowgraph{\resource{FALSE\\source code} \ar[r] & \toolbox{falmips32} \ar[r] & \resource{object file}}
\seefalse\seemips\seeobject
}

\providecommand{\falmipsb}{
\toolsection{falmips64} is a compiler for the FALSE programming language targeting the MIPS64 hardware architecture.
It generates machine code for MIPS64 processors from programs written in FALSE and stores it in corresponding object files.
\flowgraph{\resource{FALSE\\source code} \ar[r] & \toolbox{falmips64} \ar[r] & \resource{object file}}
\seefalse\seemips\seeobject
}

\providecommand{\falmmix}{
\toolsection{falmmix} is a compiler for the FALSE programming language targeting the MMIX hardware architecture.
It generates machine code for MMIX processors from programs written in FALSE and stores it in corresponding object files.
\flowgraph{\resource{FALSE\\source code} \ar[r] & \toolbox{falmmix} \ar[r] & \resource{object file}}
\seefalse\seemmix\seeobject
}

\providecommand{\falorok}{
\toolsection{falor1k} is a compiler for the FALSE programming language targeting the OpenRISC 1000 hardware architecture.
It generates machine code for OpenRISC 1000 processors from programs written in FALSE and stores it in corresponding object files.
\flowgraph{\resource{FALSE\\source code} \ar[r] & \toolbox{falor1k} \ar[r] & \resource{object file}}
\seefalse\seeorok\seeobject
}

\providecommand{\falppca}{
\toolsection{falppc32} is a compiler for the FALSE programming language targeting the PowerPC hardware architecture.
It generates machine code for PowerPC processors from programs written in FALSE and stores it in corresponding object files.
The compiler generates machine code for the 32-bit operating mode defined by the PowerPC architecture.
\flowgraph{\resource{FALSE\\source code} \ar[r] & \toolbox{falppc32} \ar[r] & \resource{object file}}
\seefalse\seeppc\seeobject
}

\providecommand{\falppcb}{
\toolsection{falppc64} is a compiler for the FALSE programming language targeting the PowerPC hardware architecture.
It generates machine code for PowerPC processors from programs written in FALSE and stores it in corresponding object files.
The compiler generates machine code for the 64-bit operating mode defined by the PowerPC architecture.
\flowgraph{\resource{FALSE\\source code} \ar[r] & \toolbox{falppc64} \ar[r] & \resource{object file}}
\seefalse\seeppc\seeobject
}

\providecommand{\falrisc}{
\toolsection{falrisc} is a compiler for the FALSE programming language targeting the RISC hardware architecture.
It generates machine code for RISC processors from programs written in FALSE and stores it in corresponding object files.
\flowgraph{\resource{FALSE\\source code} \ar[r] & \toolbox{falrisc} \ar[r] & \resource{object file}}
\seefalse\seerisc\seeobject
}

\providecommand{\falwasm}{
\toolsection{falwasm} is a compiler for the FALSE programming language targeting the WebAssembly architecture.
It generates machine code for WebAssembly targets from programs written in FALSE and stores it in corresponding object files.
\flowgraph{\resource{FALSE\\source code} \ar[r] & \toolbox{falwasm} \ar[r] & \resource{object file}}
\seefalse\seewasm\seeobject
}

% Oberon tools

\providecommand{\obprint}{
\toolsection{obprint} is a pretty printer for the Oberon programming language.
It reformats the source code of Oberon modules and writes it to the standard output stream.
\flowgraph{\resource{Oberon\\source code} \ar[r] & \toolbox{obprint} \ar[r] & \resource{reformatted\\source code}}
\seeoberon
}

\providecommand{\obcheck}{
\toolsection{obcheck} is a syntactic and semantic checker for the Oberon programming language.
It just performs syntactic and semantic checks on Oberon modules and writes its diagnostic messages to the standard error stream.
In addition, it stores the interface of each module in a symbol file which is required when other modules import the module.
\flowgraph{\resource{Oberon\\source code} \ar[r] & \toolbox{obcheck} \ar[r] \ar@/l/[d] & \resource{diagnostic\\messages} \\ \variable{ECSIMPORT} \ar[ru] & \resource{symbol\\files} \ar@/r/[u]}
\seeoberon
}

\providecommand{\obdump}{
\toolsection{obdump} is a serializer for the Oberon programming language.
It dumps the complete internal representation of modules written in Oberon into an XML document.
\debuggingtool
\flowgraph{\resource{Oberon\\source code} \ar[r] & \toolbox{obdump} \ar[r] \ar@/l/[d] & \resource{internal\\representation} \\ \variable{ECSIMPORT} \ar[ru] & \resource{symbol\\files} \ar@/r/[u]}
\seeoberon
}

\providecommand{\obrun}{
\toolsection{obrun} is an interpreter for the Oberon programming language.
It processes and executes modules written in Oberon.
This tool does neither generate nor process symbol files while interpreting modules.
If a module is imported by another one, its filename has to be named before the other one in the list of command-line arguments.
\flowgraph{\resource{Oberon\\source code} \ar[r] & \toolbox{obrun} \ar@/u/[r] & \resource{input/\\output} \ar@/d/[l]}
\seeoberon
}

\providecommand{\obcpp}{
\toolsection{obcpp} is a transpiler for the Oberon programming language.
It translates programs written in Oberon into \cpp{} programs and stores them in corresponding source and header files.
In addition, it stores the interface of each module in a symbol file which is required when other modules import the module.
The same interface is provided by the generated header file which can be used in other parts of the \cpp{} program.
\flowgraph{\resource{Oberon\\source code} \ar[r] & \toolbox{obcpp} \ar[r] \ar@/l/[d] \ar[rd] & \resource{\cpp{}\\source file} \\ \variable{ECSIMPORT} \ar[ru] & \resource{symbol\\files} \ar@/r/[u] & \resource{\cpp{}\\header file}}
\seeoberon\seecpp
}

\providecommand{\obdoc}{
\toolsection{obdoc} is a generic documentation generator for the Oberon programming language.
It processes several Oberon modules and assembles all information therein into a generic documentation.
In addition, it stores the interface of each module in a symbol file which is required when other modules import the module.
\debuggingtool
\flowgraph{\resource{Oberon\\source code} \ar[r] & \toolbox{obdoc} \ar[r] \ar@/l/[d] & \resource{generic\\documentation} \\ \variable{ECSIMPORT} \ar[ru] & \resource{symbol\\files} \ar@/r/[u]}
\seeoberon\seedocumentation
}

\providecommand{\obhtml}{
\toolsection{obhtml} is an HTML documentation generator for the Oberon programming language.
It processes several Oberon modules and assembles all information therein into an HTML document.
In addition, it stores the interface of each module in a symbol file which is required when other modules import the module.
\flowgraph{\resource{Oberon\\source code} \ar[r] & \toolbox{obhtml} \ar[r] \ar@/l/[d] & \resource{HTML\\document} \\ \variable{ECSIMPORT} \ar[ru] & \resource{symbol\\files} \ar@/r/[u]}
\seeoberon\seedocumentation
}

\providecommand{\oblatex}{
\toolsection{oblatex} is a Latex documentation generator for the Oberon programming language.
It processes several Oberon modules and assembles all information therein into a Latex document.
In addition, it stores the interface of each module in a symbol file which is required when other modules import the module.
\flowgraph{\resource{Oberon\\source code} \ar[r] & \toolbox{oblatex} \ar[r] \ar@/l/[d] & \resource{Latex\\document} \\ \variable{ECSIMPORT} \ar[ru] & \resource{symbol\\files} \ar@/r/[u]}
\seeoberon\seedocumentation
}

\providecommand{\obcode}{
\toolsection{obcode} is an intermediate code generator for the Oberon programming language.
It generates intermediate code from modules written in Oberon and stores it in corresponding assembly files.
In addition, it stores the interface of each module in a symbol file which is required when other modules import the module.
Programs generated with this tool require additional runtime support that is stored in the \file{ob\-code\-run} library file.
\debuggingtool
\flowgraph{\resource{Oberon\\source code} \ar[r] & \toolbox{obcode} \ar[r] \ar@/l/[d] & \resource{intermediate\\code} \\ \variable{ECSIMPORT} \ar[ru] & \resource{symbol\\files} \ar@/r/[u]}
\seeoberon\seeassembly\seecode
}

\providecommand{\obamda}{
\toolsection{obamd16} is a compiler for the Oberon programming language targeting the AMD64 hardware architecture.
It generates machine code for AMD64 processors from modules written in Oberon and stores it in corresponding object files.
The compiler generates machine code for the 16-bit operating mode defined by the AMD64 architecture.
For debugging purposes, it also creates a debugging information file as well as an assembly file containing a listing of the generated machine code.
In addition, it stores the interface of each module in a symbol file which is required when other modules import the module.
Programs generated with this compiler require additional runtime support that is stored in the \file{ob\-amd16\-run} library file.
\flowgraph{\resource{Oberon\\source code} \ar[r] & \toolbox{obamd16} \ar[r] \ar@/l/[d] \ar[rd] & \resource{object file} \\ \variable{ECSIMPORT} \ar[ru] & \resource{symbol\\files} \ar@/r/[u] & \resource{debugging\\information}}
\seeoberon\seeassembly\seeamd\seeobject\seedebugging
}

\providecommand{\obamdb}{
\toolsection{obamd32} is a compiler for the Oberon programming language targeting the AMD64 hardware architecture.
It generates machine code for AMD64 processors from modules written in Oberon and stores it in corresponding object files.
The compiler generates machine code for the 32-bit operating mode defined by the AMD64 architecture.
For debugging purposes, it also creates a debugging information file as well as an assembly file containing a listing of the generated machine code.
In addition, it stores the interface of each module in a symbol file which is required when other modules import the module.
Programs generated with this compiler require additional runtime support that is stored in the \file{ob\-amd32\-run} library file.
\flowgraph{\resource{Oberon\\source code} \ar[r] & \toolbox{obamd32} \ar[r] \ar@/l/[d] \ar[rd] & \resource{object file} \\ \variable{ECSIMPORT} \ar[ru] & \resource{symbol\\files} \ar@/r/[u] & \resource{debugging\\information}}
\seeoberon\seeassembly\seeamd\seeobject\seedebugging
}

\providecommand{\obamdc}{
\toolsection{obamd64} is a compiler for the Oberon programming language targeting the AMD64 hardware architecture.
It generates machine code for AMD64 processors from modules written in Oberon and stores it in corresponding object files.
The compiler generates machine code for the 64-bit operating mode defined by the AMD64 architecture.
For debugging purposes, it also creates a debugging information file as well as an assembly file containing a listing of the generated machine code.
In addition, it stores the interface of each module in a symbol file which is required when other modules import the module.
Programs generated with this compiler require additional runtime support that is stored in the \file{ob\-amd64\-run} library file.
\flowgraph{\resource{Oberon\\source code} \ar[r] & \toolbox{obamd64} \ar[r] \ar@/l/[d] \ar[rd] & \resource{object file} \\ \variable{ECSIMPORT} \ar[ru] & \resource{symbol\\files} \ar@/r/[u] & \resource{debugging\\information}}
\seeoberon\seeassembly\seeamd\seeobject\seedebugging
}

\providecommand{\obarma}{
\toolsection{obarma32} is a compiler for the Oberon programming language targeting the ARM hardware architecture.
It generates machine code for ARM processors executing A32 instructions from modules written in Oberon and stores it in corresponding object files.
For debugging purposes, it also creates a debugging information file as well as an assembly file containing a listing of the generated machine code.
In addition, it stores the interface of each module in a symbol file which is required when other modules import the module.
Programs generated with this compiler require additional runtime support that is stored in the \file{ob\-arma32\-run} library file.
\flowgraph{\resource{Oberon\\source code} \ar[r] & \toolbox{obarma32} \ar[r] \ar@/l/[d] \ar[rd] & \resource{object file} \\ \variable{ECSIMPORT} \ar[ru] & \resource{symbol\\files} \ar@/r/[u] & \resource{debugging\\information}}
\seeoberon\seeassembly\seearm\seeobject\seedebugging
}

\providecommand{\obarmb}{
\toolsection{obarma64} is a compiler for the Oberon programming language targeting the ARM hardware architecture.
It generates machine code for ARM processors executing A64 instructions from modules written in Oberon and stores it in corresponding object files.
For debugging purposes, it also creates a debugging information file as well as an assembly file containing a listing of the generated machine code.
In addition, it stores the interface of each module in a symbol file which is required when other modules import the module.
Programs generated with this compiler require additional runtime support that is stored in the \file{ob\-arma64\-run} library file.
\flowgraph{\resource{Oberon\\source code} \ar[r] & \toolbox{obarma64} \ar[r] \ar@/l/[d] \ar[rd] & \resource{object file} \\ \variable{ECSIMPORT} \ar[ru] & \resource{symbol\\files} \ar@/r/[u] & \resource{debugging\\information}}
\seeoberon\seeassembly\seearm\seeobject\seedebugging
}

\providecommand{\obarmc}{
\toolsection{obarmt32} is a compiler for the Oberon programming language targeting the ARM hardware architecture.
It generates machine code for ARM processors without floating-point extension executing T32 instructions from modules written in Oberon and stores it in corresponding object files.
For debugging purposes, it also creates a debugging information file as well as an assembly file containing a listing of the generated machine code.
In addition, it stores the interface of each module in a symbol file which is required when other modules import the module.
Programs generated with this compiler require additional runtime support that is stored in the \file{ob\-armt32\-run} library file.
\flowgraph{\resource{Oberon\\source code} \ar[r] & \toolbox{obarmt32} \ar[r] \ar@/l/[d] \ar[rd] & \resource{object file} \\ \variable{ECSIMPORT} \ar[ru] & \resource{symbol\\files} \ar@/r/[u] & \resource{debugging\\information}}
\seeoberon\seeassembly\seearm\seeobject\seedebugging
}

\providecommand{\obarmcfpe}{
\toolsection{obarmt32fpe} is a compiler for the Oberon programming language targeting the ARM hardware architecture.
It generates machine code for ARM processors with floating-point extension executing T32 instructions from modules written in Oberon and stores it in corresponding object files.
For debugging purposes, it also creates a debugging information file as well as an assembly file containing a listing of the generated machine code.
In addition, it stores the interface of each module in a symbol file which is required when other modules import the module.
Programs generated with this compiler require additional runtime support that is stored in the \file{ob\-armt32\-fpe\-run} library file.
\flowgraph{\resource{Oberon\\source code} \ar[r] & \toolbox{obarmt32fpe} \ar[r] \ar@/l/[d] \ar[rd] & \resource{object file} \\ \variable{ECSIMPORT} \ar[ru] & \resource{symbol\\files} \ar@/r/[u] & \resource{debugging\\information}}
\seeoberon\seeassembly\seearm\seeobject\seedebugging
}

\providecommand{\obavr}{
\toolsection{obavr} is a compiler for the Oberon programming language targeting the AVR hardware architecture.
It generates machine code for AVR processors from modules written in Oberon and stores it in corresponding object files.
For debugging purposes, it also creates a debugging information file as well as an assembly file containing a listing of the generated machine code.
In addition, it stores the interface of each module in a symbol file which is required when other modules import the module.
Programs generated with this compiler require additional runtime support that is stored in the \file{ob\-avr\-run} library file.
\flowgraph{\resource{Oberon\\source code} \ar[r] & \toolbox{obavr} \ar[r] \ar@/l/[d] \ar[rd] & \resource{object file} \\ \variable{ECSIMPORT} \ar[ru] & \resource{symbol\\files} \ar@/r/[u] & \resource{debugging\\information}}
\seeoberon\seeassembly\seeavr\seeobject\seedebugging
}

\providecommand{\obavrtt}{
\toolsection{obavr32} is a compiler for the Oberon programming language targeting the AVR32 hardware architecture.
It generates machine code for AVR32 processors from modules written in Oberon and stores it in corresponding object files.
For debugging purposes, it also creates a debugging information file as well as an assembly file containing a listing of the generated machine code.
In addition, it stores the interface of each module in a symbol file which is required when other modules import the module.
Programs generated with this compiler require additional runtime support that is stored in the \file{ob\-avr32\-run} library file.
\flowgraph{\resource{Oberon\\source code} \ar[r] & \toolbox{obavr32} \ar[r] \ar@/l/[d] \ar[rd] & \resource{object file} \\ \variable{ECSIMPORT} \ar[ru] & \resource{symbol\\files} \ar@/r/[u] & \resource{debugging\\information}}
\seeoberon\seeassembly\seeavrtt\seeobject\seedebugging
}

\providecommand{\obmabk}{
\toolsection{obm68k} is a compiler for the Oberon programming language targeting the M68000 hardware architecture.
It generates machine code for M68000 processors from modules written in Oberon and stores it in corresponding object files.
For debugging purposes, it also creates a debugging information file as well as an assembly file containing a listing of the generated machine code.
In addition, it stores the interface of each module in a symbol file which is required when other modules import the module.
Programs generated with this compiler require additional runtime support that is stored in the \file{ob\-m68k\-run} library file.
\flowgraph{\resource{Oberon\\source code} \ar[r] & \toolbox{obm68k} \ar[r] \ar@/l/[d] \ar[rd] & \resource{object file} \\ \variable{ECSIMPORT} \ar[ru] & \resource{symbol\\files} \ar@/r/[u] & \resource{debugging\\information}}
\seeoberon\seeassembly\seemabk\seeobject\seedebugging
}

\providecommand{\obmibl}{
\toolsection{obmibl} is a compiler for the Oberon programming language targeting the MicroBlaze hardware architecture.
It generates machine code for MicroBlaze processors from modules written in Oberon and stores it in corresponding object files.
For debugging purposes, it also creates a debugging information file as well as an assembly file containing a listing of the generated machine code.
In addition, it stores the interface of each module in a symbol file which is required when other modules import the module.
Programs generated with this compiler require additional runtime support that is stored in the \file{ob\-mibl\-run} library file.
\flowgraph{\resource{Oberon\\source code} \ar[r] & \toolbox{obmibl} \ar[r] \ar@/l/[d] \ar[rd] & \resource{object file} \\ \variable{ECSIMPORT} \ar[ru] & \resource{symbol\\files} \ar@/r/[u] & \resource{debugging\\information}}
\seeoberon\seeassembly\seemibl\seeobject\seedebugging
}

\providecommand{\obmipsa}{
\toolsection{obmips32} is a compiler for the Oberon programming language targeting the MIPS32 hardware architecture.
It generates machine code for MIPS32 processors from modules written in Oberon and stores it in corresponding object files.
For debugging purposes, it also creates a debugging information file as well as an assembly file containing a listing of the generated machine code.
In addition, it stores the interface of each module in a symbol file which is required when other modules import the module.
Programs generated with this compiler require additional runtime support that is stored in the \file{ob\-mips32\-run} library file.
\flowgraph{\resource{Oberon\\source code} \ar[r] & \toolbox{obmips32} \ar[r] \ar@/l/[d] \ar[rd] & \resource{object file} \\ \variable{ECSIMPORT} \ar[ru] & \resource{symbol\\files} \ar@/r/[u] & \resource{debugging\\information}}
\seeoberon\seeassembly\seemips\seeobject\seedebugging
}

\providecommand{\obmipsb}{
\toolsection{obmips64} is a compiler for the Oberon programming language targeting the MIPS64 hardware architecture.
It generates machine code for MIPS64 processors from modules written in Oberon and stores it in corresponding object files.
For debugging purposes, it also creates a debugging information file as well as an assembly file containing a listing of the generated machine code.
In addition, it stores the interface of each module in a symbol file which is required when other modules import the module.
Programs generated with this compiler require additional runtime support that is stored in the \file{ob\-mips64\-run} library file.
\flowgraph{\resource{Oberon\\source code} \ar[r] & \toolbox{obmips64} \ar[r] \ar@/l/[d] \ar[rd] & \resource{object file} \\ \variable{ECSIMPORT} \ar[ru] & \resource{symbol\\files} \ar@/r/[u] & \resource{debugging\\information}}
\seeoberon\seeassembly\seemips\seeobject\seedebugging
}

\providecommand{\obmmix}{
\toolsection{obmmix} is a compiler for the Oberon programming language targeting the MMIX hardware architecture.
It generates machine code for MMIX processors from modules written in Oberon and stores it in corresponding object files.
For debugging purposes, it also creates a debugging information file as well as an assembly file containing a listing of the generated machine code.
In addition, it stores the interface of each module in a symbol file which is required when other modules import the module.
Programs generated with this compiler require additional runtime support that is stored in the \file{ob\-mmix\-run} library file.
\flowgraph{\resource{Oberon\\source code} \ar[r] & \toolbox{obmmix} \ar[r] \ar@/l/[d] \ar[rd] & \resource{object file} \\ \variable{ECSIMPORT} \ar[ru] & \resource{symbol\\files} \ar@/r/[u] & \resource{debugging\\information}}
\seeoberon\seeassembly\seemmix\seeobject\seedebugging
}

\providecommand{\oborok}{
\toolsection{obor1k} is a compiler for the Oberon programming language targeting the OpenRISC 1000 hardware architecture.
It generates machine code for OpenRISC 1000 processors from modules written in Oberon and stores it in corresponding object files.
For debugging purposes, it also creates a debugging information file as well as an assembly file containing a listing of the generated machine code.
In addition, it stores the interface of each module in a symbol file which is required when other modules import the module.
Programs generated with this compiler require additional runtime support that is stored in the \file{ob\-or1k\-run} library file.
\flowgraph{\resource{Oberon\\source code} \ar[r] & \toolbox{obor1k} \ar[r] \ar@/l/[d] \ar[rd] & \resource{object file} \\ \variable{ECSIMPORT} \ar[ru] & \resource{symbol\\files} \ar@/r/[u] & \resource{debugging\\information}}
\seeoberon\seeassembly\seeorok\seeobject\seedebugging
}

\providecommand{\obppca}{
\toolsection{obppc32} is a compiler for the Oberon programming language targeting the PowerPC hardware architecture.
It generates machine code for PowerPC processors from modules written in Oberon and stores it in corresponding object files.
The compiler generates machine code for the 32-bit operating mode defined by the PowerPC architecture.
For debugging purposes, it also creates a debugging information file as well as an assembly file containing a listing of the generated machine code.
In addition, it stores the interface of each module in a symbol file which is required when other modules import the module.
Programs generated with this compiler require additional runtime support that is stored in the \file{ob\-ppc32\-run} library file.
\flowgraph{\resource{Oberon\\source code} \ar[r] & \toolbox{obppc32} \ar[r] \ar@/l/[d] \ar[rd] & \resource{object file} \\ \variable{ECSIMPORT} \ar[ru] & \resource{symbol\\files} \ar@/r/[u] & \resource{debugging\\information}}
\seeoberon\seeassembly\seeppc\seeobject\seedebugging
}

\providecommand{\obppcb}{
\toolsection{obppc64} is a compiler for the Oberon programming language targeting the PowerPC hardware architecture.
It generates machine code for PowerPC processors from modules written in Oberon and stores it in corresponding object files.
The compiler generates machine code for the 64-bit operating mode defined by the PowerPC architecture.
For debugging purposes, it also creates a debugging information file as well as an assembly file containing a listing of the generated machine code.
In addition, it stores the interface of each module in a symbol file which is required when other modules import the module.
Programs generated with this compiler require additional runtime support that is stored in the \file{ob\-ppc64\-run} library file.
\flowgraph{\resource{Oberon\\source code} \ar[r] & \toolbox{obppc64} \ar[r] \ar@/l/[d] \ar[rd] & \resource{object file} \\ \variable{ECSIMPORT} \ar[ru] & \resource{symbol\\files} \ar@/r/[u] & \resource{debugging\\information}}
\seeoberon\seeassembly\seeppc\seeobject\seedebugging
}

\providecommand{\obrisc}{
\toolsection{obrisc} is a compiler for the Oberon programming language targeting the RISC hardware architecture.
It generates machine code for RISC processors from modules written in Oberon and stores it in corresponding object files.
For debugging purposes, it also creates a debugging information file as well as an assembly file containing a listing of the generated machine code.
In addition, it stores the interface of each module in a symbol file which is required when other modules import the module.
Programs generated with this compiler require additional runtime support that is stored in the \file{ob\-risc\-run} library file.
\flowgraph{\resource{Oberon\\source code} \ar[r] & \toolbox{obrisc} \ar[r] \ar@/l/[d] \ar[rd] & \resource{object file} \\ \variable{ECSIMPORT} \ar[ru] & \resource{symbol\\files} \ar@/r/[u] & \resource{debugging\\information}}
\seeoberon\seeassembly\seerisc\seeobject\seedebugging
}

\providecommand{\obwasm}{
\toolsection{obwasm} is a compiler for the Oberon programming language targeting the WebAssembly architecture.
It generates machine code for WebAssembly targets from modules written in Oberon and stores it in corresponding object files.
For debugging purposes, it also creates a debugging information file as well as an assembly file containing a listing of the generated machine code.
In addition, it stores the interface of each module in a symbol file which is required when other modules import the module.
Programs generated with this compiler require additional runtime support that is stored in the \file{ob\-wasm\-run} library file.
\flowgraph{\resource{Oberon\\source code} \ar[r] & \toolbox{obwasm} \ar[r] \ar@/l/[d] \ar[rd] & \resource{object file} \\ \variable{ECSIMPORT} \ar[ru] & \resource{symbol\\files} \ar@/r/[u] & \resource{debugging\\information}}
\seeoberon\seeassembly\seewasm\seeobject\seedebugging
}

% converter tools

\providecommand{\dbgdwarf}{
\toolsection{dbgdwarf} is a DWARF debugging information converter tool.
It converts debugging information into the DWARF debugging data format and stores it in corresponding object files~\cite{dwarffile}.
The resulting debugging object files can be combined with runtime support that creates Executable and Linking Format (ELF) files~\cite{elffile}.
\flowgraph{\resource{debugging\\information} \ar[r] & \toolbox{dbgdwarf} \ar[r] & \resource{debugging\\object file}}
\seeobject\seedebugging
}

% assembler tools

\providecommand{\asmprint}{
\toolsection{asmprint} is a pretty printer for generic assembly code.
It reformats generic assembly code and writes it to the standard output stream.
\flowgraph{\resource{generic assembly\\source code} \ar[r] & \toolbox{asmprint} \ar[r] & \resource{reformatted\\source code}}
\seeassembly
}

\providecommand{\amdaasm}{
\toolsection{amd16asm} is an assembler for the AMD64 hardware architecture.
It translates assembly code into machine code for AMD64 processors and stores it in corresponding object files.
By default, the assembler generates machine code for the 16-bit operating mode defined by the AMD64 architecture.
\flowgraph{\resource{AMD16 assembly\\source code} \ar[r] & \toolbox{amd16asm} \ar[r] & \resource{object file}}
\seeassembly\seeamd\seeobject
}

\providecommand{\amdadism}{
\toolsection{amd16dism} is a disassembler for the AMD64 hardware architecture.
It translates machine code from object files targeting AMD64 processors into assembly code and writes it to the standard output stream.
It assumes that the machine code was generated for the 16-bit operating mode defined by the AMD64 architecture.
\flowgraph{\resource{object file} \ar[r] & \toolbox{amd16dism} \ar[r] & \resource{disassembly\\listing}}
\seeassembly\seeamd\seeobject
}

\providecommand{\amdbasm}{
\toolsection{amd32asm} is an assembler for the AMD64 hardware architecture.
It translates assembly code into machine code for AMD64 processors and stores it in corresponding object files.
By default, the assembler generates machine code for the 32-bit operating mode defined by the AMD64 architecture.
\flowgraph{\resource{AMD32 assembly\\source code} \ar[r] & \toolbox{amd32asm} \ar[r] & \resource{object file}}
\seeassembly\seeamd\seeobject
}

\providecommand{\amdbdism}{
\toolsection{amd32dism} is a disassembler for the AMD64 hardware architecture.
It translates machine code from object files targeting AMD64 processors into assembly code and writes it to the standard output stream.
It assumes that the machine code was generated for the 32-bit operating mode defined by the AMD64 architecture.
\flowgraph{\resource{object file} \ar[r] & \toolbox{amd32dism} \ar[r] & \resource{disassembly\\listing}}
\seeassembly\seeamd\seeobject
}

\providecommand{\amdcasm}{
\toolsection{amd64asm} is an assembler for the AMD64 hardware architecture.
It translates assembly code into machine code for AMD64 processors and stores it in corresponding object files.
By default, the assembler generates machine code for the 64-bit operating mode defined by the AMD64 architecture.
\flowgraph{\resource{AMD64 assembly\\source code} \ar[r] & \toolbox{amd64asm} \ar[r] & \resource{object file}}
\seeassembly\seeamd\seeobject
}

\providecommand{\amdcdism}{
\toolsection{amd64dism} is a disassembler for the AMD64 hardware architecture.
It translates machine code from object files targeting AMD64 processors into assembly code and writes it to the standard output stream.
It assumes that the machine code was generated for the 64-bit operating mode defined by the AMD64 architecture.
\flowgraph{\resource{object file} \ar[r] & \toolbox{amd64dism} \ar[r] & \resource{disassembly\\listing}}
\seeassembly\seeamd\seeobject
}

\providecommand{\armaasm}{
\toolsection{arma32asm} is an assembler for the ARM hardware architecture.
It translates assembly code into machine code for ARM processors executing A32 instructions and stores it in corresponding object files.
\flowgraph{\resource{ARM A32 assembly\\source code} \ar[r] & \toolbox{arma32asm} \ar[r] & \resource{object file}}
\seeassembly\seearm\seeobject
}

\providecommand{\armadism}{
\toolsection{arma32dism} is a disassembler for the ARM hardware architecture.
It translates machine code from object files targeting ARM processors executing A32 instructions into assembly code and writes it to the standard output stream.
\flowgraph{\resource{object file} \ar[r] & \toolbox{arma32dism} \ar[r] & \resource{disassembly\\listing}}
\seeassembly\seearm\seeobject
}

\providecommand{\armbasm}{
\toolsection{arma64asm} is an assembler for the ARM hardware architecture.
It translates assembly code into machine code for ARM processors executing A64 instructions and stores it in corresponding object files.
\flowgraph{\resource{ARM A64 assembly\\source code} \ar[r] & \toolbox{arma64asm} \ar[r] & \resource{object file}}
\seeassembly\seearm\seeobject
}

\providecommand{\armbdism}{
\toolsection{arma64dism} is a disassembler for the ARM hardware architecture.
It translates machine code from object files targeting ARM processors executing A64 instructions into assembly code and writes it to the standard output stream.
\flowgraph{\resource{object file} \ar[r] & \toolbox{arma64dism} \ar[r] & \resource{disassembly\\listing}}
\seeassembly\seearm\seeobject
}

\providecommand{\armcasm}{
\toolsection{armt32asm} is an assembler for the ARM hardware architecture.
It translates assembly code into machine code for ARM processors executing T32 instructions and stores it in corresponding object files.
\flowgraph{\resource{ARM T32 assembly\\source code} \ar[r] & \toolbox{armt32asm} \ar[r] & \resource{object file}}
\seeassembly\seearm\seeobject
}

\providecommand{\armcdism}{
\toolsection{armt32dism} is a disassembler for the ARM hardware architecture.
It translates machine code from object files targeting ARM processors executing T32 instructions into assembly code and writes it to the standard output stream.
\flowgraph{\resource{object file} \ar[r] & \toolbox{armt32dism} \ar[r] & \resource{disassembly\\listing}}
\seeassembly\seearm\seeobject
}

\providecommand{\avrasm}{
\toolsection{avrasm} is an assembler for the AVR hardware architecture.
It translates assembly code into machine code for AVR processors and stores it in corresponding object files.
The identifiers \texttt{RXL}, \texttt{RXH}, \texttt{RYL}, \texttt{RYH}, \texttt{RZL}, and \texttt{RZH} are predefined and name the corresponding registers.
The identifiers \texttt{SPL} and \texttt{SPH} are also predefined and evaluate to the address of the corresponding registers.
\flowgraph{\resource{AVR assembly\\source code} \ar[r] & \toolbox{avrasm} \ar[r] & \resource{object file}}
\seeassembly\seeavr\seeobject
}

\providecommand{\avrdism}{
\toolsection{avrdism} is a disassembler for the AVR hardware architecture.
It translates machine code from object files targeting AVR processors into assembly code and writes it to the standard output stream.
\flowgraph{\resource{object file} \ar[r] & \toolbox{avrdism} \ar[r] & \resource{disassembly\\listing}}
\seeassembly\seeavr\seeobject
}

\providecommand{\avrttasm}{
\toolsection{avr32asm} is an assembler for the AVR32 hardware architecture.
It translates assembly code into machine code for AVR32 processors and stores it in corresponding object files.
\flowgraph{\resource{AVR32 assembly\\source code} \ar[r] & \toolbox{avr32asm} \ar[r] & \resource{object file}}
\seeassembly\seeavrtt\seeobject
}

\providecommand{\avrttdism}{
\toolsection{avr32dism} is a disassembler for the AVR32 hardware architecture.
It translates machine code from object files targeting AVR32 processors into assembly code and writes it to the standard output stream.
\flowgraph{\resource{object file} \ar[r] & \toolbox{avr32dism} \ar[r] & \resource{disassembly\\listing}}
\seeassembly\seeavrtt\seeobject
}

\providecommand{\mabkasm}{
\toolsection{m68kasm} is an assembler for the M68000 hardware architecture.
It translates assembly code into machine code for M68000 processors and stores it in corresponding object files.
\flowgraph{\resource{68000 assembly\\source code} \ar[r] & \toolbox{m68kasm} \ar[r] & \resource{object file}}
\seeassembly\seemabk\seeobject
}

\providecommand{\mabkdism}{
\toolsection{m68kdism} is a disassembler for the M68000 hardware architecture.
It translates machine code from object files targeting M68000 processors into assembly code and writes it to the standard output stream.
\flowgraph{\resource{object file} \ar[r] & \toolbox{m68kdism} \ar[r] & \resource{disassembly\\listing}}
\seeassembly\seemabk\seeobject
}

\providecommand{\miblasm}{
\toolsection{miblasm} is an assembler for the MicroBlaze hardware architecture.
It translates assembly code into machine code for MicroBlaze processors and stores it in corresponding object files.
\flowgraph{\resource{MicroBlaze assembly\\source code} \ar[r] & \toolbox{miblasm} \ar[r] & \resource{object file}}
\seeassembly\seemibl\seeobject
}

\providecommand{\mibldism}{
\toolsection{mibldism} is a disassembler for the MicroBlaze hardware architecture.
It translates machine code from object files targeting MicroBlaze processors into assembly code and writes it to the standard output stream.
\flowgraph{\resource{object file} \ar[r] & \toolbox{mibldism} \ar[r] & \resource{disassembly\\listing}}
\seeassembly\seemibl\seeobject
}

\providecommand{\mipsaasm}{
\toolsection{mips32asm} is an assembler for the MIPS32 hardware architecture.
It translates assembly code into machine code for MIPS32 processors and stores it in corresponding object files.
\flowgraph{\resource{MIPS32 assembly\\source code} \ar[r] & \toolbox{mips32asm} \ar[r] & \resource{object file}}
\seeassembly\seemips\seeobject
}

\providecommand{\mipsadism}{
\toolsection{mips32dism} is a disassembler for the MIPS32 hardware architecture.
It translates machine code from object files targeting MIPS32 processors into assembly code and writes it to the standard output stream.
\flowgraph{\resource{object file} \ar[r] & \toolbox{mips32dism} \ar[r] & \resource{disassembly\\listing}}
\seeassembly\seemips\seeobject
}

\providecommand{\mipsbasm}{
\toolsection{mips64asm} is an assembler for the MIPS64 hardware architecture.
It translates assembly code into machine code for MIPS64 processors and stores it in corresponding object files.
\flowgraph{\resource{MIPS64 assembly\\source code} \ar[r] & \toolbox{mips64asm} \ar[r] & \resource{object file}}
\seeassembly\seemips\seeobject
}

\providecommand{\mipsbdism}{
\toolsection{mips64dism} is a disassembler for the MIPS64 hardware architecture.
It translates machine code from object files targeting MIPS64 processors into assembly code and writes it to the standard output stream.
\flowgraph{\resource{object file} \ar[r] & \toolbox{mips64dism} \ar[r] & \resource{disassembly\\listing}}
\seeassembly\seemips\seeobject
}

\providecommand{\mmixasm}{
\toolsection{mmixasm} is an assembler for the MMIX hardware architecture.
It translates assembly code into machine code for MMIX processors and stores it in corresponding object files.
The names of all special registers are predefined and evaluate to the corresponding number.
\flowgraph{\resource{MMIX assembly\\source code} \ar[r] & \toolbox{mmixasm} \ar[r] & \resource{object file}}
\seeassembly\seemmix\seeobject
}

\providecommand{\mmixdism}{
\toolsection{mmixdism} is a disassembler for the MMIX hardware architecture.
It translates machine code from object files targeting MMIX processors into assembly code and writes it to the standard output stream.
\flowgraph{\resource{object file} \ar[r] & \toolbox{mmixdism} \ar[r] & \resource{disassembly\\listing}}
\seeassembly\seemmix\seeobject
}

\providecommand{\orokasm}{
\toolsection{or1kasm} is an assembler for the OpenRISC 1000 hardware architecture.
It translates assembly code into machine code for OpenRISC 1000 processors and stores it in corresponding object files.
\flowgraph{\resource{OpenRISC 1000 assembly\\source code} \ar[r] & \toolbox{or1kasm} \ar[r] & \resource{object file}}
\seeassembly\seeorok\seeobject
}

\providecommand{\orokdism}{
\toolsection{or1kdism} is a disassembler for the OpenRISC 1000 hardware architecture.
It translates machine code from object files targeting OpenRISC 1000 processors into assembly code and writes it to the standard output stream.
\flowgraph{\resource{object file} \ar[r] & \toolbox{or1kdism} \ar[r] & \resource{disassembly\\listing}}
\seeassembly\seeorok\seeobject
}

\providecommand{\ppcaasm}{
\toolsection{ppc32asm} is an assembler for the PowerPC hardware architecture.
It translates assembly code into machine code for PowerPC processors and stores it in corresponding object files.
By default, the assembler generates machine code for the 32-bit operating mode defined by the PowerPC architecture.
\flowgraph{\resource{PowerPC assembly\\source code} \ar[r] & \toolbox{ppc32asm} \ar[r] & \resource{object file}}
\seeassembly\seeppc\seeobject
}

\providecommand{\ppcadism}{
\toolsection{ppc32dism} is a disassembler for the PowerPC hardware architecture.
It translates machine code from object files targeting PowerPC processors into assembly code and writes it to the standard output stream.
It assumes that the machine code was generated for the 32-bit operating mode defined by the PowerPC architecture.
\flowgraph{\resource{object file} \ar[r] & \toolbox{ppc32dism} \ar[r] & \resource{disassembly\\listing}}
\seeassembly\seeppc\seeobject
}

\providecommand{\ppcbasm}{
\toolsection{ppc64asm} is an assembler for the PowerPC hardware architecture.
It translates assembly code into machine code for PowerPC processors and stores it in corresponding object files.
By default, the assembler generates machine code for the 64-bit operating mode defined by the PowerPC architecture.
\flowgraph{\resource{PowerPC assembly\\source code} \ar[r] & \toolbox{ppc64asm} \ar[r] & \resource{object file}}
\seeassembly\seeppc\seeobject
}

\providecommand{\ppcbdism}{
\toolsection{ppc64dism} is a disassembler for the PowerPC hardware architecture.
It translates machine code from object files targeting PowerPC processors into assembly code and writes it to the standard output stream.
It assumes that the machine code was generated for the 64-bit operating mode defined by the PowerPC architecture.
\flowgraph{\resource{object file} \ar[r] & \toolbox{ppc64dism} \ar[r] & \resource{disassembly\\listing}}
\seeassembly\seeppc\seeobject
}

\providecommand{\riscasm}{
\toolsection{riscasm} is an assembler for the RISC hardware architecture.
It translates assembly code into machine code for RISC processors and stores it in corresponding object files.
The names of all special registers are predefined and evaluate to the corresponding number.
\flowgraph{\resource{RISC assembly\\source code} \ar[r] & \toolbox{riscasm} \ar[r] & \resource{object file}}
\seeassembly\seerisc\seeobject
}

\providecommand{\riscdism}{
\toolsection{riscdism} is a disassembler for the RISC hardware architecture.
It translates machine code from object files targeting RISC processors into assembly code and writes it to the standard output stream.
\flowgraph{\resource{object file} \ar[r] & \toolbox{riscdism} \ar[r] & \resource{disassembly\\listing}}
\seeassembly\seerisc\seeobject
}

\providecommand{\wasmasm}{
\toolsection{wasmasm} is an assembler for the WebAssembly architecture.
It translates assembly code into machine code for WebAssembly targets and stores it in corresponding object files.
The names of all special registers are predefined and evaluate to the corresponding number.
\flowgraph{\resource{WebAssembly assembly\\source code} \ar[r] & \toolbox{wasmasm} \ar[r] & \resource{object file}}
\seeassembly\seewasm\seeobject
}

\providecommand{\wasmdism}{
\toolsection{wasmdism} is a disassembler for the WebAssembly architecture.
It translates machine code from object files targeting WebAssembly targets into assembly code and writes it to the standard output stream.
\flowgraph{\resource{object file} \ar[r] & \toolbox{wasmdism} \ar[r] & \resource{disassembly\\listing}}
\seeassembly\seewasm\seeobject
}

% linker tools

\providecommand{\linklib}{
\toolsection{linklib} is an object file combiner.
It creates a static library file by combining all object files given to it into a single one.
\flowgraph{\resource{object files} \ar[r] & \toolbox{linklib} \ar[r] & \resource{library file}}
\seeobject
}

\providecommand{\linkbin}{
\toolsection{linkbin} is a linker for plain binary files.
It links all object files given to it into a single image and stores it in a binary file that begins with the first linked section.
It also creates a map file that lists the address, type, name and size of all used sections.
The filename extension of the resulting binary file can be specified by putting it into a constant data section called \texttt{\_extension}.
\flowgraph{\resource{object files} \ar[r] & \toolbox{linkbin} \ar[r] \ar[d] & \resource{binary file} \\ & \resource{map file}}
\seeobject
}

\providecommand{\linkmem}{
\toolsection{linkmem} is a linker for plain binary files partitioned into random-access and read-only memory.
It links all object files given to it into two distinct images, one for data sections and one for code and constant data sections, and stores each image in a binary file that begins with the first linked section of the corresponding type.
It also creates a map file that lists the address, type, name and size of all used sections.
\flowgraph{\resource{object files} \ar[r] & \toolbox{linkmem} \ar[r] \ar[d] & \resource{RAM file/\\ROM file} \\ & \resource{map file}}
\seeobject
}

\providecommand{\linkprg}{
\toolsection{linkprg} is a linker for GEMDOS executable files.
It links all object files given to it into a single image and stores the image in an Atari GEMDOS executable file~\cite{gemdosfile}.
It also creates a map file that lists the address relative to the text segment, type, name and size of all used sections.
The filename extension of the resulting executable file can be specified by putting it into a constant data section called \texttt{\_extension}.
The GEMDOS executable file format requires all patch patterns of absolute link patches to consist of four full bitmasks with descending offsets.
\flowgraph{\resource{object files} \ar[r] & \toolbox{linkprg} \ar[r] \ar[d] & \resource{executable file} \\ & \resource{map file}}
\seeobject
}

\providecommand{\linkhex}{
\toolsection{linkhex} is a linker for Intel HEX files.
It links all code sections of the object files given to it into single image and stores the image in an Intel HEX file~\cite{hexfile} that begins with the first linked section.
It also creates a map file that lists the address, type, name and size of all used sections.
\flowgraph{\resource{object files} \ar[r] & \toolbox{linkhex} \ar[r] \ar[d] & \resource{HEX file} \\ & \resource{map file}}
\seeobject
}

\providecommand{\mapsearch}{
\toolsection{mapsearch} is a debugging tool.
It searches map files generated by linker tools for the name of a binary section that encompasses a memory address read from the standard input stream.
If additionally provided with one or more object files, it also stores an excerpt thereof in a separate object file called map search result which only contains the identified binary section for disassembling purposes.
\flowgraph{& \resource{map files/\\object files} \ar[d] \\ \resource{memory\\address} \ar[r] & \toolbox{mapsearch} \ar[r] \ar[d] & \resource{section name/\\relative offset} \\ & \resource{object file\\excerpt}}
\seeobject
}


\startchapter{Introduction}{Introduction to the \ecs{}}{introduction}
{\emph{\ecs{}}\index{Eigen Compiler Suite} is the name of a free software collection of development tools like
compilers, assemblers, and linkers targeting different programming languages and different hardware architectures.
This \documentation{} gives a general overview over the \ecs{} and describes its features and design in detail.}

\epigraph{I will not follow where the path may lead, \\ instead I will go where there is no path \\ and leave a trail.}{Muriel Strode}

\section{Features}\index{Features, of the Eigen Compiler Suite}\index{Eigen Compiler Suite!Features}

The \ecs{} is a software development toolchain.
It contains tools like compilers, pretty printers, interpreters, assemblers, and linkers targeting a variety of programming languages and hardware architectures.
The \ecs{} features pretty printers, interpreters, and compilers for the following programming languages:

\begin{center}\cpplogo{1em}\fallogo{1em}\oblogo{2em}\end{center}

\begin{itemize}

\item \cpp{}\nopagebreak

\cpp{} is a general-purpose programming language with a bias toward systems programming.
\seecpp

\item FALSE\nopagebreak

FALSE is an esoteric stack-oriented programming language that provides lambda abstractions and is quite powerful for its size.
\seefalse

\item Oberon\nopagebreak

Oberon is a general-purpose programming language that supports type extension with type-bound procedures which makes it an object-oriented language.
\seeoberon

\end{itemize}

Besides the support for these programming languages, the \ecs{} also features assemblers, disassemblers, and compilers targeting the following hardware architectures:

\begin{itemize}

\item AMD64\nopagebreak

AMD64 is a 64-bit instruction set architecture developed by AMD\@.
\seeamd

\item ARM\nopagebreak

ARM is a 32-bit and 64-bit instruction set architecture developed by ARM Holdings.
\seearm

\item AVR\nopagebreak

AVR is the name of an 8-bit microcontroller architecture developed by Atmel.
\seeavr

\item AVR32\nopagebreak

AVR32 is the name of an 32-bit microcontroller architecture developed by Atmel.
\seeavrtt

\item M68000\nopagebreak

M68000 is a 16/32-bit instruction set architecture developed by Motorola.
\seemabk

\item MicroBlaze\nopagebreak

MicroBlaze is a 32-bit instruction set architecture developed by Xilinx.
\seemibl

\item MIPS\nopagebreak

MIPS32 and MIPS64 are 32-bit and 64-bit instruction set architectures developed by MIPS Technologies.
\seemips

\item MMIX\nopagebreak

MMIX is a 64-bit instruction set architecture designed by Donald~E.\ Knuth.
\seemmix

\item OpenRISC 1000\nopagebreak

OpenRISC 1000 is a 32/64-bit instruction set architecture developed by OpenCores.
\seeorok

\item PowerPC\nopagebreak

PowerPC is 32-bit and 64-bit instruction set architecture developed by AIM\@.
\seeppc

\item RISC\nopagebreak

RISC is a 32-bit instruction set architecture designed by Niklaus Wirth.
\seerisc

\item WebAssembly\nopagebreak

WebAssembly is a 32-bit instruction set architecture designed by the World Wide Web Consortium (W3C).
\seewasm

\end{itemize}

Finally, the \ecs{} features various linkers and runtime support\index{Runtime support} for the following runtime environments:

\begin{itemize}

\item Atari TOS\nopagebreak

The \ecs{} provides runtime support and a linker for programs that can be executed under Atari TOS\@.

\item AVR Microcontrollers\nopagebreak

The \ecs{} provides runtime support for AVR microcontrollers.
Additionally, it features a linker that generates Intel HEX files that can be used to program these microcontrollers.

\item BIOS\nopagebreak

The \ecs{} provides runtime support and a linker for bootloaders executed by the BIOS\@.

\item DOS\nopagebreak

The \ecs{} features runtime support and a linker for programs that are executed under DOS\@.

\item EFI\nopagebreak

The \ecs{} provides runtime support and a linker for 32-bit as well as 64-bit EFI applications that can be executed in an EFI boot console.

\item Linux\nopagebreak

The \ecs{} features runtime support and a linker for programs that can be executed under Linux-based operating systems.

\item MMIX Simulator\nopagebreak

The \ecs{} provides runtime support and a linker for object files that are executed within the MMIX simulator.

\item OpenRISC 1000 Simulator\nopagebreak

The \ecs{} provides runtime support and a linker for object files that are executed within the OpenRISC 1000 simulator.

\item OS~X\nopagebreak

The \ecs{} provides runtime support and a linker for 32-bit as well as 64-bit programs that can be executed under OS~X\@.

\item Raspberry Pi\nopagebreak

The \ecs{} provides runtime support and a linker for bootloaders running on the Raspberry Pi~2 Model~B.

\item RISC Microcontrollers\nopagebreak

The \ecs{} provides runtime support for RISC microcontrollers.

\item WebAssembly Environments\nopagebreak

The \ecs{} provides runtime support and a linker for WebAssembly modules that can be executed in corresponding web environments.

\item Windows\nopagebreak

The \ecs{} provides runtime support and a linker for 32-bit as well as 64-bit programs that can be executed under Windows.

\end{itemize}

The \ecs{} features additional linkers that are not mentioned above.
See \Documentation{}~\documentationref{object}{Object File Representation} for a list of all linkers provided by the \ecs{}.

\section{Design}\index{Design, of the Eigen Compiler Suite}\index{Eigen Compiler Suite!Design}

The \ecs{} was designed to be a simple and minimalistic but complete and self-contained toolchain for several different programming languages and hardware architectures.
While the main objective of the \ecs{} is to be self-hosting, its design strives for the following goals:

\begin{itemize}

\item Usability\index{Usability}\nopagebreak

All tools featured by the \ecs{} shall be easy to use and should not require complex installations or other prerequisites.
Their user-friendliness shall be enabled by a common basic user interface which does not demand any options or configurations of the user.
This design enforces stability and reproducibility because the result of executing any tool depends solely on the contents of its input files.

\item Reliability\index{Reliability}\nopagebreak

All programming tools of the \ecs{} shall be reliable and produce correct results as well as comprehensive diagnostic messages.
For this purpose, the \ecs{} contains several test and validation suites that enable automatic regression testing of its various tools.
In general, the implementation of the \ecs{} is mainly driven by correctness, robustness, and simplicity.
Therefore, optimizations and performance are explicitly secondary.

\item Portability\index{Portability}\nopagebreak

The \ecs{} is completely written using standard and portable programming language features in order to guarantee the portability of its source code.
All tools of the \ecs{} are therefore compilable and executable within different runtime environments.
Thus, all compilers featured by the \ecs{} are cross compilers by design.

\item Interoperability\index{Interoperability}\nopagebreak

Tools like compilers and assemblers shall enable interoperability in-between all programming languages supported by the \ecs{}.
Usually, programs are written using a single programming language.
Interoperability in this case means, that some parts of the program can also be implemented using a different programming language or with the help of an assembler.
In the end, all these parts work seamlessly together and constitute the complete program.

\item Textual Intermediate Representations\index{Intermediate representations}\nopagebreak

The various tools of the \ecs{} are typically used in a chain, such that the output of one tool becomes the input of the next one.
However, all data that is transported this way should be represented using a human-readable and machine-independent text format.
This design enables programmers and maintainers to view, modify and even manually create all kinds of intermediate data.
In addition to the tools themselves, also the temporary data they generate is therefore portable across different runtime environments.

\item Reuse of Generic Abstractions\index{Generic abstractions}\nopagebreak

The implementation of the \ecs{} shall provide generic abstractions that help to achieve all goals mentioned above.
Additionally, the abstractions shall enable a good code reuse within the implementation itself.
The most important abstractions provided by the \ecs{} are described in Section~\ref{sec:introabstractions}.

\item Complete and Consistent Documentation\nopagebreak

All features of the \ecs{} and especially the usage of its toolchain shall be documented in detail.
For all major components like implementations of a programming language or supported hardware architectures, there are consistent documentations available which describe the usage of the corresponding component and its implementation by the \ecs{}.
This user manual merges all of these documentations into a single document.

\end{itemize}

Some design guidelines like reliability and interoperability have also been incoorporated into the logo of the \ecs{} which combines the three self-contained letters of its abbreviation into a robust three-dimensional structure as shown in Figure~\ref{fig:intrologo}.

\begin{figure}
\centering\ecslogo{5em}
\caption{The logo of the \ecs{}}
\label{fig:intrologo}
\end{figure}

\section{Abstractions}\index{Abstractions}\label{sec:introabstractions}

The \ecs{} supports several different programming languages as well as several different target hardware architectures.
One goal of the \ecs{} is to provide compilers, assemblers, and linkers for all possible combinations of programming languages and target architectures.
For this reason, the \ecs{} defines the following generic abstractions:

\begin{itemize}

\item Object Files\nopagebreak

Object files are the key abstraction in order to enable interoperability between all compilers and assemblers of the \ecs{}.
All these tools generate the same kind of output called object files which can be processed by all linkers and disassemblers of the \ecs{}.
Figure~\ref{fig:introobject} shows how object files are used in-between some compilers, assemblers, linkers, and disassemblers for the AMD64 hardware architecture.
\seeobject

\begin{figure}
\flowgraph{
\resource{\cpp{}\\source code} \ar[d] & \resource{assembly\\source code} \ar[d] & \resource{Oberon\\source code} \ar[d] \\
\converter{cppamd64} \ar[rd] & \converter{amd64asm} \ar[d] & \converter{obamd64} \ar[ld] \\
& \resource{object files} \ar[ld] \ar[d] \ar[rd] \\
\converter{linkbin} \ar[d] & \converter{amd64dism} \ar[d] & \converter{linklib} \ar[d] \\
\resource{binary file} & \resource{disassembly\\listing} & \resource{library file} \\
}\caption{Some tools of the \ecs{} that process object files}
\label{fig:introobject}
\end{figure}

\item Intermediate Code\nopagebreak

Intermediate code is the generic abstraction in-between \emph{front-ends} implementing programming languages and \emph{back-ends} targeting hardware architectures.
Front-ends implement the actual translation of source code into intermediate code, whereas back-ends translate intermediate code into machine code for the respective architecture.
Figure~\ref{fig:introcode} shows some front-ends and back-ends supported by the \ecs{}.
Each data flow within this diagram depicts the input and output of a single compiler for a particular programming language targeting a particular hardware architecture.
As a consequence, programmers only have to provide a single front-end or back-end in order to establish all possible combinations with existing implementations of programming languages and target architectures.
\seecode

\begin{figure}
\flowgraph{
\resource{\cpp{}\\source code} \ar[d] & \resource{Oberon\\source code} \ar[d] & \resource{FALSE\\source code} \ar[d] \\
\converter{Front-End\\for \cpp{}} \ar[rd] & \converter{Front-End for\\Oberon} \ar[d] & \converter{Front-End\\for FALSE} \ar[ld] \\
& \resource{intermediate\\code} \ar[ld] \ar[d] \ar[rd] \\
\converter{Back-End\\for AVR} \ar[d] & \converter{Back-End\\for M68000} \ar[d] & \converter{Back-End\\for AMD64} \ar[d] \\
\resource{AVR\\machine code} & \resource{M68000\\machine code} & \resource{AMD64\\machine code} \\
}\caption{Some front-ends and back-ends provided by the \ecs{}}
\label{fig:introcode}
\end{figure}

\item Generic Assembly Language\nopagebreak

The generic assembly language is an abstraction for the common features of all assemblers provided by the \ecs{}.
It is capable of representing complete assembly programs including all the different instruction sets supported by the \ecs{}.
Concrete assembler tools only have to implement the translation of architecture specific instructions into their binary representation.
Everything else that is actually not dependent on the actual hardware architecture can therefore be translated separately.
\seeassembly

\item Debugging Information\nopagebreak

Debugging information is a generic abstraction of symbolic metadata about a program generated by compilers.
It represents all programming language constructs like functions, variables, data types, and statements compiled into an object file.
Converter tools convert debugging information into a binary debugging data format which decouples compilers from the debugger of the target runtime environment.
\seedebugging

\end{itemize}

Concrete information about how programmers can make use of these abstractions in order to extend the \ecs{} with additional support
for programming languages and hardware architectures is given in \Documentation{}~\documentationref{extensions}{Extensions to the \ecs{}}.

\concludechapter

% Common user interface
% Copyright (C) Florian Negele

% This file is part of the Eigen Compiler Suite.

% Permission is granted to copy, distribute and/or modify this document
% under the terms of the GNU Free Documentation License, Version 1.3
% or any later version published by the Free Software Foundation.

% You should have received a copy of the GNU Free Documentation License
% along with the ECS.  If not, see <https://www.gnu.org/licenses/>.

% Generic documentation utilities
% Copyright (C) Florian Negele

% This file is part of the Eigen Compiler Suite.

% Permission is granted to copy, distribute and/or modify this document
% under the terms of the GNU Free Documentation License, Version 1.3
% or any later version published by the Free Software Foundation.

% You should have received a copy of the GNU Free Documentation License
% along with the ECS.  If not, see <https://www.gnu.org/licenses/>.

\providecommand{\cpp}{C\texttt{++}}
\providecommand{\opt}{_\mathit{opt}}
\providecommand{\tool}[1]{\texttt{#1}}
\providecommand{\version}{Version 0.0.40}
\providecommand{\resource}[1]{*++\txt{#1}}
\providecommand{\ecs}{Eigen Compiler Suite}
\providecommand{\changed}[1]{\underline{#1}}
\providecommand{\toolbox}[1]{\converter{#1}}
\providecommand{\file}{}\renewcommand{\file}[1]{\texttt{#1}}
\providecommand{\alignright}{\hfill\linebreak[0]\hspace*{\fill}}
\providecommand{\converter}[1]{*++[F][F*:white][F,:gray]\txt{#1}}
\providecommand{\documentation}{\ifbook chapter\else document\fi}
\providecommand{\Documentation}{\ifbook Chapter\else Document\fi}
\providecommand{\variable}[1]{\resource{\texttt{\small#1}\\variable}}
\providecommand{\documentationref}[2]{\ifbook\ref{#1}\else``\href{#1}{#2}''~\cite{#1}\fi}
\providecommand{\objfile}[1]{\texttt{#1}\index[runtime]{#1 object file@\texttt{#1} object file}}
\providecommand{\libfile}[1]{\texttt{#1}\index[runtime]{#1 library file@\texttt{#1} library file}}
\providecommand{\epigraph}[2]{\ifbook\begin{quote}\flushright\textit{#1}\par--- #2\end{quote}\fi}
\providecommand{\environmentvariable}[1]{\texttt{#1}\index{Environment variables!#1@\texttt{#1}}}
\providecommand{\environment}[1]{\texttt{#1}\index[environment]{#1 environment@\texttt{#1} environment}}
\providecommand{\toolsection}{}\renewcommand{\toolsection}[1]{\subsection{#1}\label{\prefix:#1}\tool{#1}}
\providecommand{\instruction}{}\renewcommand{\instruction}[2]{\noindent\qquad\pdftooltip{\texttt{#1}}{#2}\refstepcounter{instruction}\par}
\providecommand{\flowgraph}{}\renewcommand{\flowgraph}[1]{\par\sffamily\begin{displaymath}\xymatrix@=4ex{#1}\end{displaymath}\normalfont\par}
\providecommand{\instructionset}{}\renewcommand{\instructionset}[4]{\setcounter{instruction}{0}\begin{multicols}{\ifbook#3\else#4\fi}[{\captionof{table}[#2]{#2 (\ref*{#1:instructions}~instructions)}\label{tab:#1set}\vspace{-2ex}}]\footnotesize\raggedcolumns\input{#1.set}\label{#1:instructions}\end{multicols}}

\providecommand{\gpl}{GNU General Public License}
\providecommand{\rse}{ECS Runtime Support Exception}
\providecommand{\fdl}{\href{https://www.gnu.org/licenses/fdl.html}{GNU Free Documentation License}}

\providecommand{\docbegin}{}
\providecommand{\docend}{}
\providecommand{\doclabel}[1]{\hypertarget{#1}}
\providecommand{\doclink}[2]{\hyperlink{#1}{#2}}
\providecommand{\docsection}[3]{\hypertarget{#1}{\subsection{#2}}\label{sec:#1}\index[library]{#2@#3}}
\providecommand{\docsectionstar}[1]{}
\providecommand{\docsubbegin}{\begin{description}}
\providecommand{\docsubend}{\end{description}}
\providecommand{\docsubsection}[3]{\item[\hypertarget{#1}{#2}]\index[library]{#2@#3}}
\providecommand{\docsubsectionstar}[1]{\smallskip}
\providecommand{\docsubsubsection}[3]{\docsubsection{#1}{#2}{#3}}
\providecommand{\docsubsubsectionstar}[1]{}
\providecommand{\docsubsubsubsection}[3]{}
\providecommand{\docsubsubsubsectionstar}[1]{}
\providecommand{\doctable}{}

\providecommand{\debuggingtool}{}\renewcommand{\debuggingtool}{This tool is provided for debugging purposes.
It allows exposing and modifying an internal data structure that is usually not accessible.
}

\providecommand{\interface}{All tools accept command-line arguments which are taken as names of plain text files containing the source code.
If no arguments are provided, the standard input stream is used instead.
Output files are generated in the current working directory and have the same name as the input file being processed whereas the filename extension gets replaced by an appropriate suffix.
\seeinterface
}

\providecommand{\license}{\noindent Copyright \copyright{} Florian Negele\par\medskip\noindent
Permission is granted to copy, distribute and/or modify this document under the terms of the
\fdl{}, Version 1.3 or any later version published by the \href{https://fsf.org/}{Free Software Foundation}.
}

\providecommand{\ecslogosurface}{
\fill[darkgray] (0,0,0) -- (0,0,3) -- (0,3,3) -- (0,3,1) -- (0,4,1) -- (0,4,3) -- (0,5,3) -- (0,5,0) -- (0,2,0) -- (0,2,2) -- (0,1,2) -- (0,1,0) -- cycle;
\fill[gray] (0,5,0) -- (0,5,3) -- (1,5,3) -- (1,5,1) -- (2,5,1) -- (2,5,3) -- (3,5,3) -- (3,5,0) -- cycle;
\fill[lightgray] (0,0,0) -- (0,1,0) -- (2,1,0) -- (2,4,0) -- (1,4,0) -- (1,3,0) -- (2,3,0) -- (2,2,0) -- (0,2,0) -- (0,5,0) -- (3,5,0) -- (3,0,0) -- cycle;
\begin{scope}[line width=0.5]
\begin{scope}[gray]
\draw (0,0,0) -- (0,1,0);
\draw (2,1,0) -- (2,2,0);
\draw (0,1,2) -- (0,2,2);
\draw (0,2,0) -- (0,5,0);
\draw (2,3,0) -- (2,4,0);
\end{scope}
\begin{scope}[lightgray]
\draw (0,1,0) -- (0,1,2);
\draw (0,3,1) -- (0,3,3);
\draw (0,5,0) -- (0,5,3);
\draw (2,5,1) -- (2,5,3);
\end{scope}
\begin{scope}[white]
\draw (0,1,0) -- (2,1,0);
\draw (1,3,0) -- (2,3,0);
\draw (0,5,0) -- (3,5,0);
\end{scope}
\end{scope}
}

\providecommand{\ecslogo}[1]{
\begin{tikzpicture}[scale={(#1)/((sin(45)+cos(45))*3cm)},x={({-cos(45)*1cm},{sin(45)*sin(30)*1cm})},y={({0cm},{(cos(30)*1cm})},z={({sin(45)*1cm},{cos(45)*sin(30)*1cm})}]
\begin{scope}[darkgray,line width=1]
\draw (0,0,0) -- (0,0,3) -- (0,3,3) -- (2,3,3) -- (2,5,3) -- (3,5,3) -- (3,5,0) -- (3,0,0) -- cycle;
\draw (0,3,1) -- (0,4,1) -- (0,4,3) -- (0,5,3) -- (1,5,3) -- (1,5,1) -- (2,5,1);
\draw (1,3,0) -- (1,4,0) -- (2,4,0);
\end{scope}
\fill[darkgray] (2,0,0) -- (2,0,3) -- (2,5,3) -- (2,5,1) -- (2,4,1) -- (2,4,0) -- cycle;
\fill[lightgray] (2,0,2) -- (0,0,2) -- (0,2,2) -- (2,2,2) -- cycle;
\fill[gray] (0,1,0) -- (2,1,0) -- (2,1,2) -- (0,1,2) -- cycle;
\fill[gray] (0,3,1) -- (0,3,3) -- (2,3,3) -- (2,3,0) -- (1,3,0) -- (1,3,1) -- cycle;
\ecslogosurface
\end{tikzpicture}
}

\providecommand{\shadowedecslogo}[3]{
\begin{tikzpicture}[scale={(#1)/((sin(#2)+cos(#2))*3cm)},x={({-cos(#2)*1cm},{sin(#2)*sin(#3)*1cm})},y={({0cm},{(cos(#3)*1cm})},z={({sin(#2)*1cm},{cos(#2)*sin(#3)*1cm})}]
\shade[top color=lightgray!50!white,bottom color=white,middle color=lightgray!50!white] (0,0,0) -- (3,0,0) -- (3,{-0.5-3*sin(#2)*sin(#3)/cos(#3)},0) -- (0,-0.5,0) -- cycle;
\shade[top color=darkgray!50!gray,bottom color=white,middle color=darkgray!50!white] (0,0,0) -- (0,0,3) -- (0,{-0.5-3*cos(#2)*sin(#3)/cos(#3)},3) -- (0,-0.5,0) -- cycle;
\begin{scope}[y={({(cos(#2)+sin(#2))*0.5cm},{(cos(#2)*sin(#3)-sin(#2)*sin(#3))*0.5cm})}]
\useasboundingbox (3,0,0) -- (0,0,0) -- (0,0,3);
\shade[left color=darkgray!80!black,right color=lightgray,middle color=gray] (0,0,0) -- (0,1,0) -- (0,1,0.5) -- (0,2,0) -- (0,5,0) -- (0,5,3) -- (1,5,3) -- (1,4,3) -- (1,4,2.5) -- (1,3,3) -- (2,5,3) -- (3,5,3) -- (3,0,3) -- cycle;
\clip (0,0,0) -- (0,0,3) -- ({-3*sin(#2)/cos(#2)},0,0) -- cycle;
\shade[left color=darkgray,right color=lightgray!50!gray] (0,0,0) -- (0,1,0) -- (0,1,0.5) -- (0,2,0) -- (0,5,0) -- (0,5,3) -- (1,5,3) -- (1,4,3) -- (1,4,2.5) -- (1,3,3) -- (2,5,3) -- (3,5,3) -- (3,0,3) -- cycle;
\end{scope}
\shade[left color=darkgray,right color=darkgray!80!black] (2,0,0) -- (2,0,3) -- (2,5,3) -- (2,5,1) -- (2,4,1) -- (2,4,0) -- cycle;
\shade[left color=darkgray!90!black,right color=gray!80!darkgray] (2,0,2) -- (0,0,2) -- (0,2,2) -- (2,2,2) -- cycle;
\shade[top color=darkgray!90!black,bottom color=gray!80!darkgray] (0,1,0) -- (2,1,0) -- (2,1,2) -- (0,1,2) -- cycle;
\shade[top color=darkgray!90!black,bottom color=gray!80!darkgray] (0,3,1) -- (0,3,3) -- (2,3,3) -- (2,3,0) -- (1,3,0) -- (1,3,1) -- cycle;
\fill[gray] (2,1,0) -- (1.5,1,0.5) -- (0,1,0.5) -- (0,1,0) -- cycle;
\fill[gray] (1,3,2) -- (0.5,3,2) -- (0.5,3,3) -- (1,3,3) -- cycle;
\fill[gray] (2,3,0) -- (1.5,3,0.5) -- (1,3,0.5) -- (1,3,0) -- cycle;
\ecslogosurface
\end{tikzpicture}
}

\providecommand{\cpplogo}[1]{
\begin{tikzpicture}[scale=(#1)/512em]
\fill[gray] (435.2794,398.7159) -- (247.1911,507.3075) .. controls (236.3563,513.5642) and (218.6240,513.5642) .. (207.7892,507.3075) -- (19.7009,398.7159) .. controls (8.8646,392.4606) and (0.0000,377.1043) .. (0.0000,364.5924) -- (0.0000,147.4076) .. controls (0.8430,132.8363) and (8.2856,120.7683) .. (19.7009,113.2842) -- (207.7892,4.6926) .. controls (218.6240,-1.5642) and (236.3564,-1.5642) .. (247.1911,4.6926) -- (435.2794,113.2842) .. controls (447.5273,121.4304) and (454.4987,133.6918) .. (454.9803,147.4076) -- (454.9803,364.5924) .. controls (454.5404,377.7571) and (446.6566,391.0351) .. (435.2794,398.7159) -- cycle(75.8301,255.9993) .. controls (74.9389,404.0881) and (273.2892,469.4783) .. (358.8263,331.8769) -- (293.1917,293.8965) .. controls (253.5702,359.4301) and (155.1909,335.9977) .. (151.6601,255.9993) .. controls (152.7204,182.2703) and (249.4137,148.0211) .. (293.1961,218.1065) -- (358.8308,180.1276) .. controls (283.4477,49.2645) and (79.6318,96.3470) .. (75.8301,255.9993) -- cycle(379.1503,247.5747) -- (362.2982,247.5747) -- (362.2982,230.7226) -- (345.4490,230.7226) -- (345.4490,247.5747) -- (328.5969,247.5747) -- (328.5969,264.4254) -- (345.4490,264.4254) -- (345.4490,281.2759) -- (362.2982,281.2759) -- (362.2982,264.4254) -- (379.1503,264.4254) -- cycle(442.3420,247.5747) -- (425.4899,247.5747) -- (425.4899,230.7226) -- (408.6408,230.7226) -- (408.6408,247.5747) -- (391.7886,247.5747) -- (391.7886,264.4254) -- (408.6408,264.4254) -- (408.6408,281.2759) -- (425.4899,281.2759) -- (425.4899,264.4254) -- (442.3420,264.4254) -- cycle;
\end{tikzpicture}
}

\providecommand{\fallogo}[1]{
\begin{tikzpicture}[scale=(#1)/512em]
\fill[gray] (185.7774,0.0000) .. controls (200.4486,15.9798) and (226.8966,8.7148) .. (235.0426,31.5836) .. controls (249.5297,58.0598) and (247.9581,97.9161) .. (280.3335,110.9762) .. controls (309.1690,120.3496) and (337.8406,104.2727) .. (366.5753,103.9379) .. controls (373.4449,111.5171) and (379.2885,128.2574) .. (383.9755,108.9744) .. controls (396.6979,102.5615) and (437.2808,107.6681) .. (426.9652,124.3252) .. controls (408.9822,121.0785) and (412.4742,146.0729) .. (426.5192,131.4996) .. controls (433.8413,120.8489) and (465.1541,126.5522) .. (441.9067,135.7950) .. controls (396.1879,157.7478) and (344.1112,161.5079) .. (298.5528,183.5702) .. controls (277.7471,193.5198) and (284.6941,218.7163) .. (285.2127,236.9640) .. controls (292.3599,316.2826) and (307.3929,394.6311) .. (317.1198,473.6154) .. controls (329.0637,505.4736) and (292.1195,528.5004) .. (265.9183,511.2761) .. controls (237.9284,499.2462) and (237.3684,465.2681) .. (230.9102,439.9421) .. controls (218.6692,374.3397) and (215.6307,306.9662) .. (198.1732,242.3977) .. controls (183.1379,232.7444) and (164.4245,256.0298) .. (149.0430,261.4799) .. controls (116.9328,279.2585) and (87.1822,308.5851) .. (48.2293,307.8914) .. controls (21.3220,306.9037) and (-15.9107,281.8761) .. (7.2921,252.7908) .. controls (29.7799,220.6177) and (67.5177,204.3028) .. (100.9287,185.9449) .. controls (130.8217,170.8906) and (161.1548,156.5903) .. (191.0278,141.5847) .. controls (196.1738,120.0520) and (186.6049,95.2409) .. (186.8382,72.4353) .. controls (185.5234,48.4204) and (183.1700,23.9341) .. (185.7774,0.0000) -- cycle;
\end{tikzpicture}
}

\providecommand{\oblogo}[1]{
\begin{tikzpicture}[scale=(#1)/512em]
\fill[gray] (160.3865,208.9117) .. controls (154.0879,214.6478) and (149.0735,221.2409) .. (145.4125,228.5384) .. controls (184.8790,248.4273) and (234.7122,269.8787) .. (297.5493,291.8782) .. controls (300.3943,281.4769) and (300.9552,268.7619) .. (300.4023,255.2389) .. controls (248.9909,244.7891) and (200.0310,225.9279) .. (160.3865,208.9117) -- cycle(225.7398,392.6996) .. controls (308.0209,392.1716) and (359.3326,345.9277) .. (368.7203,285.2098) .. controls (376.6742,197.1784) and (311.7194,141.3342) .. (205.4287,142.1456) .. controls (139.9485,141.4804) and (88.7155,166.1957) .. (73.5775,228.0086) .. controls (52.0297,320.3408) and (123.4078,391.0103) .. (225.7398,392.6996) -- cycle(216.0739,176.4733) .. controls (268.9183,179.2424) and (315.8292,206.5488) .. (312.7454,265.1139) .. controls (313.2769,315.6384) and (286.5993,353.4946) .. (216.6040,355.7934) .. controls (162.4657,355.7934) and (126.0914,317.5023) .. (126.0914,260.5103) .. controls (126.1733,214.2900) and (163.3363,176.2849) .. (216.0739,176.4733) -- cycle(76.4897,189.1754) .. controls (13.1586,147.5631) and (0.0000,119.4207) .. (0.0000,119.4207) -- (90.6499,170.1632) .. controls (85.3004,175.8497) and (80.5994,182.1633) .. (76.4897,189.1754) -- cycle(353.9486,119.3004) -- (402.9482,119.3004) .. controls (427.0025,137.0797) and (450.9893,162.7034) .. (474.9529,191.0213) .. controls (509.3540,228.5339) and (531.3391,294.2091) .. (487.8149,312.1206) .. controls (462.8165,324.7652) and (394.3874,316.8943) .. (373.8912,313.6651) .. controls (379.9291,297.7449) and (383.2899,278.4204) .. (381.4989,257.7214) .. controls (420.3069,248.0321) and (421.9610,218.3461) .. (407.7867,192.6417) .. controls (391.1113,162.4018) and (370.1114,132.9097) .. (353.9486,119.3004) -- cycle;
\end{tikzpicture}
}

\providecommand{\markuptable}{
\begin{table}
\sffamily\centering
\begin{tabular}{@{}lcl@{}}
\toprule
\texttt{//italics//} & $\rightarrow$ & \textit{italics} \\
\midrule
\texttt{**bold**} & $\rightarrow$ & \textbf{bold} \\
\midrule
\texttt{\# ordered list} & & 1 ordered list \\
\texttt{\# second item} & $\rightarrow$ & 2 second item \\
\texttt{\#\# sub item} & & \hspace{1em} 1 sub item \\
\midrule
\texttt{* unordered list} & & $\bullet$ unordered list \\
\texttt{* second item} & $\rightarrow$ & $\bullet$ second item \\
\texttt{** sub item} & & \hspace{1em} $\bullet$ sub item \\
\midrule
\texttt{link to [[label]]} & $\rightarrow$ & link to \underline{label} \\
\midrule
\texttt{<{}<label>{}> definition } & $\rightarrow$ & definition \\
\midrule
\texttt{[[url|link name]]} & $\rightarrow$ & \underline{link name} \\
\midrule\addlinespace
\texttt{= large heading} & & {\Large large heading} \smallskip \\
\texttt{== medium heading} & $\rightarrow$ & {\large medium heading} \\
\texttt{=== small heading} & & small heading \\
\midrule
\texttt{no line break} & & no line break for paragraphs \\
\texttt{for paragraphs} & $\rightarrow$ \\
& & use empty line \\
\texttt{use empty line} \\
\midrule
\texttt{force\textbackslash\textbackslash line break} & $\rightarrow$ & force \\
& & line break \\
\midrule
\texttt{horizontal line} & $\rightarrow$ & horizontal line \\
\texttt{----} & & \hrulefill \\
\midrule
\texttt{|=a|=table|=header} & & \underline{a \enspace table \enspace header} \\
\texttt{|a|table|row} & $\rightarrow$ & a \enspace table \enspace row \\
\texttt{|b|table|row} & & b \enspace table \enspace row \\
\midrule
\texttt{\{\{\{} \\
\texttt{unformatted} & $\rightarrow$ & \texttt{unformatted} \\
\texttt{code} & & \texttt{code} \\
\texttt{\}\}\}} \\
\midrule\addlinespace
\texttt{@ new article} & & {\Large 1.\ new article} \smallskip \\
\texttt{@ second article} & $\rightarrow$ & {\Large 2.\ second article} \smallskip \\
\texttt{@@ sub article} & & {\large 2.1.\ sub article} \\
\bottomrule
\end{tabular}
\normalfont\caption{Elements of the generic documentation markup language}
\label{tab:docmarkup}
\end{table}
}

\providecommand{\startchapter}[4]{
\documentclass[11pt,a4paper]{article}
\usepackage{booktabs}
\usepackage[format=hang,labelfont=bf]{caption}
\usepackage{changepage}
\usepackage[T1]{fontenc}
\usepackage[margin=2cm]{geometry}
\usepackage{hyperref}
\usepackage[american]{isodate}
\usepackage{lmodern}
\usepackage{longtable}
\usepackage{mathptmx}
\usepackage{microtype}
\usepackage[toc]{multitoc}
\usepackage{multirow}
\usepackage[all]{nowidow}
\usepackage{pdfcomment}
\usepackage{syntax}
\usepackage{tikz}
\usepackage[all]{xy}
\hypersetup{pdfborder={0 0 0},bookmarksnumbered=true,pdftitle={\ecs{}: #2},pdfauthor={Florian Negele},pdfsubject={\ecs{}},pdfkeywords={#1}}
\setlength{\grammarindent}{8em}\setlength{\grammarparsep}{0.2ex}
\setlength{\columnsep}{2em}
\newcommand{\prefix}{}
\newcounter{instruction}
\bibliographystyle{unsrt}
\renewcommand{\index}[2][]{}
\renewcommand{\arraystretch}{1.05}
\renewcommand{\floatpagefraction}{0.7}
\renewcommand{\syntleft}{\itshape}\renewcommand{\syntright}{}
\title{\vspace{-5ex}\Huge{\ecs{}}\medskip\hrule}
\author{\huge{#2}}
\date{\medskip\version}
\newif\ifbook\bookfalse
\pagestyle{headings}
\frenchspacing
\begin{document}
\maketitle\thispagestyle{empty}\noindent#4\setlength{\columnseprule}{0.4pt}\tableofcontents\setlength{\columnseprule}{0pt}\vfill\pagebreak[3]\null\vfill\bigskip\noindent
\parbox{\textwidth-4em}{\license The contents of this \documentation{} are part of the \href{manual}{\ecs{} User Manual}~\cite{manual} and correspond to Chapter ``\href{manual\##3}{#1}''.\alignright\mbox{\today}}
\parbox{4em}{\flushright\ecslogo{3em}}
\clearpage
}

\providecommand{\concludechapter}{
\vfill\pagebreak[3]\null\vfill
\thispagestyle{myheadings}\markright{REFERENCES}
\noindent\begin{minipage}{\textwidth}\begin{multicols}{2}[\section*{References}]
\renewcommand{\section}[2]{}\small\bibliography{references}
\end{multicols}\end{minipage}\end{document}
}

\providecommand{\startpresentation}[2]{
\documentclass[14pt,aspectratio=43,usepdftitle=false]{beamer}
\usepackage{booktabs}
\usepackage{etex}
\usepackage{multicol}
\usepackage{tikz}
\usepackage[all]{xy}
\bibliographystyle{unsrt}
\setlength{\columnsep}{1em}
\setlength{\leftmargini}{1em}
\setbeamercolor{title}{fg=black}
\setbeamercolor{structure}{fg=darkgray}
\setbeamercolor{bibliography item}{fg=darkgray}
\setbeamerfont{title}{series=\bfseries}
\setbeamerfont{subtitle}{series=\normalfont}
\setbeamerfont*{frametitle}{parent=title}
\setbeamerfont{block title}{series=\bfseries}
\setbeamerfont*{framesubtitle}{parent=subtitle}
\setbeamersize{text margin left=1em,text margin right=1em}
\setbeamertemplate{navigation symbols}{}
\setbeamertemplate{itemize item}[circle]{}
\setbeamertemplate{bibliography item}[triangle]{}
\setbeamertemplate{bibliography entry author}{\usebeamercolor[fg]{bibliography item}}
\setbeamertemplate{frametitle}{\medskip\usebeamerfont{frametitle}\color{gray}\raisebox{-2.5ex}[0ex][0ex]{\rule{0.1em}{4.5ex}}}
\addtobeamertemplate{frametitle}{}{\hspace{0.4em}\usebeamercolor[fg]{title}\insertframetitle\par\vspace{0.2ex}\hspace{0.5em}\usebeamerfont{framesubtitle}\insertframesubtitle}
\hypersetup{pdfborder={0 0 0},bookmarksnumbered=true,bookmarksopen=true,bookmarksopenlevel=0,pdftitle={\ecs{}: #1},pdfauthor={Florian Negele},pdfsubject={\ecs{}},pdfkeywords={#1}}
\renewcommand{\flowgraph}[1]{\resizebox{\textwidth}{!}{$$\xymatrix{##1}$$}}
\title{\ecs{}\medskip\hrule\medskip}
\institute{\shadowedecslogo{5em}{30}{15}}
\date{\version}
\subtitle{#1}
\begin{document}
\begin{frame}[plain]\titlepage\nocite{manual}\end{frame}
\begin{frame}{Contents}{#1}\begin{center}\tableofcontents\end{center}\end{frame}
}

\providecommand{\concludepresentation}{
\begin{frame}{References}\begin{footnotesize}\setlength{\columnseprule}{0.4pt}\begin{multicols}{2}\bibliography{references}\end{multicols}\end{footnotesize}\end{frame}
\end{document}
}

\providecommand{\startbook}[1]{
\documentclass[10pt,paper=17cm:24cm,DIV=13,twoside=semi,headings=normal,numbers=noendperiod,cleardoublepage=plain]{scrbook}
\usepackage{atveryend}
\usepackage{booktabs}
\usepackage{caption}
\usepackage{changepage}
\usepackage[T1]{fontenc}
\usepackage{imakeidx}
\usepackage{hyperref}
\usepackage[american]{isodate}
\usepackage{lmodern}
\usepackage{longtable}
\usepackage{mathptmx}
\usepackage[final]{microtype}
\usepackage{multicol}
\usepackage{multirow}
\usepackage[all]{nowidow}
\usepackage{pdfcomment}
\usepackage{scrlayer-scrpage}
\usepackage{setspace}
\usepackage{syntax}
\usepackage[eventxtindent=4pt,oddtxtexdent=4pt]{thumbs}
\usepackage{tikz}
\usepackage[all]{xy}
\hyphenation{Micro-Blaze Open-Cores Open-RISC Power-PC}
\hypersetup{pdfborder={0 0 0},bookmarksnumbered=true,bookmarksopen=true,bookmarksopenlevel=0,pdftitle={\ecs{}: #1},pdfauthor={Florian Negele},pdfsubject={\ecs{}},pdfkeywords={#1}}
\setlength{\grammarindent}{8em}\setlength{\grammarparsep}{0.7ex}
\setkomafont{captionlabel}{\usekomafont{descriptionlabel}}
\renewcommand{\arraystretch}{1.05}\setstretch{1.1}
\renewcommand{\chapterformat}{\thechapter\autodot\enskip\raisebox{-1ex}[0ex][0ex]{\color{gray}\rule{0.1em}{3.5ex}}\enskip}
\renewcommand{\startchapter}[4]{\hypertarget{##3}{\chapter{##1}}\label{##3}##4\addthumb{##1}{\LARGE\sffamily\bfseries\thechapter}{white}{gray}\renewcommand{\prefix}{##3}}
\renewcommand{\concludechapter}{\clearpage{\stopthumb\cleardoublepage}}
\renewcommand{\syntleft}{\itshape}\renewcommand{\syntright}{}
\renewcommand{\floatpagefraction}{0.7}
\renewcommand{\partheademptypage}{}
\DeclareMicrotypeAlias{lmss}{cmr}
\newcommand{\prefix}{}
\newcounter{instruction}
\bibliographystyle{unsrt}
\newif\ifbook\booktrue
\makeindex[intoc,title=Index]
\makeindex[intoc,name=tools,title=Index of Tools,columns=3]
\makeindex[intoc,name=library,title=Index of Library Names]
\makeindex[intoc,name=runtime,title=Index of Runtime Support]
\makeindex[intoc,name=environment,title=Index of Target Environments]
\indexsetup{toclevel=chapter,headers={\indexname}{\indexname}}
\frenchspacing
\begin{document}
\pagenumbering{alph}
\begin{titlepage}\centering
\huge\sffamily\null\vfill\textbf{\ecs{}}\bigskip\hrule\bigskip#1
\normalsize\normalfont\vfill\vfill\shadowedecslogo{10em}{30}{15}
\large\vfill\vfill\version
\end{titlepage}
\null\vfill
\thispagestyle{empty}
\noindent\today\par\medskip
\license A copy of this license is included in Appendix~\ref{fdl} on page~\pageref{fdl}.
All product names used herein are for identification purposes only and may be trademarks of their respective companies.
\concludechapter
\frontmatter
\setcounter{tocdepth}{1}
\tableofcontents
\setcounter{tocdepth}{2}
\concludechapter
\listoffigures
\concludechapter
\listoftables
\concludechapter
}

\providecommand{\concludebook}{
\backmatter
\addtocontents{toc}{\protect\setcounter{tocdepth}{-1}}
\phantomsection\addcontentsline{toc}{part}{Bibliography}
\bibliography{references}
\concludechapter
\phantomsection\addcontentsline{toc}{part}{Indexes}
\printindex
\concludechapter
\indexprologue{\label{idx:tools}}
\printindex[tools]
\concludechapter
\printindex[library]
\concludechapter
\indexprologue{\label{idx:runtime}}
\printindex[runtime]
\concludechapter
\indexprologue{\label{idx:environment}}
\printindex[environment]
\concludechapter
\pagestyle{empty}\pagenumbering{Alph}\null\clearpage
\null\vfill\centering\ecslogo{4em}\par\medskip\license
\end{document}
}

% chapter references

\providecommand{\seedocumentationref}{}\renewcommand{\seedocumentationref}[3]{#1, see \Documentation{}~\documentationref{#2}{#3}. }
\providecommand{\seeinterface}{}\renewcommand{\seeinterface}{\ifbook See \Documentation{}~\documentationref{interface}{User Interface} for more information about the common user interface of all of these tools. \fi}
\providecommand{\seeguide}{}\renewcommand{\seeguide}{\seedocumentationref{For basic examples of using some of these tools in practice}{guide}{User Guide}}
\providecommand{\seecpp}{}\renewcommand{\seecpp}{\seedocumentationref{For more information about the \cpp{} programming language and its implementation by the \ecs{}}{cpp}{User Manual for \cpp{}}}
\providecommand{\seefalse}{}\renewcommand{\seefalse}{\seedocumentationref{For more information about the FALSE programming language and its implementation by the \ecs{}}{false}{User Manual for FALSE}}
\providecommand{\seeoberon}{}\renewcommand{\seeoberon}{\seedocumentationref{For more information about the Oberon programming language and its implementation by the \ecs{}}{oberon}{User Manual for Oberon}}
\providecommand{\seeassembly}{}\renewcommand{\seeassembly}{\seedocumentationref{For more information about the generic assembly language and how to use it}{assembly}{Generic Assembly Language Specification}}
\providecommand{\seeamd}{}\renewcommand{\seeamd}{\seedocumentationref{For more information about how the \ecs{} supports the AMD64 hardware architecture}{amd64}{AMD64 Hardware Architecture Support}}
\providecommand{\seearm}{}\renewcommand{\seearm}{\seedocumentationref{For more information about how the \ecs{} supports the ARM hardware architecture}{arm}{ARM Hardware Architecture Support}}
\providecommand{\seeavr}{}\renewcommand{\seeavr}{\seedocumentationref{For more information about how the \ecs{} supports the AVR hardware architecture}{avr}{AVR Hardware Architecture Support}}
\providecommand{\seeavrtt}{}\renewcommand{\seeavrtt}{\seedocumentationref{For more information about how the \ecs{} supports the AVR32 hardware architecture}{avr32}{AVR32 Hardware Architecture Support}}
\providecommand{\seemabk}{}\renewcommand{\seemabk}{\seedocumentationref{For more information about how the \ecs{} supports the M68000 hardware architecture}{m68k}{M68000 Hardware Architecture Support}}
\providecommand{\seemibl}{}\renewcommand{\seemibl}{\seedocumentationref{For more information about how the \ecs{} supports the MicroBlaze hardware architecture}{mibl}{MicroBlaze Hardware Architecture Support}}
\providecommand{\seemips}{}\renewcommand{\seemips}{\seedocumentationref{For more information about how the \ecs{} supports the MIPS32 and MIPS64 hardware architectures}{mips}{MIPS Hardware Architecture Support}}
\providecommand{\seemmix}{}\renewcommand{\seemmix}{\seedocumentationref{For more information about how the \ecs{} supports the MMIX hardware architecture}{mmix}{MMIX Hardware Architecture Support}}
\providecommand{\seeorok}{}\renewcommand{\seeorok}{\seedocumentationref{For more information about how the \ecs{} supports the OpenRISC 1000 hardware architecture}{or1k}{OpenRISC 1000 Hardware Architecture Support}}
\providecommand{\seeppc}{}\renewcommand{\seeppc}{\seedocumentationref{For more information about how the \ecs{} supports the PowerPC hardware architecture}{ppc}{PowerPC Hardware Architecture Support}}
\providecommand{\seerisc}{}\renewcommand{\seerisc}{\seedocumentationref{For more information about how the \ecs{} supports the RISC hardware architecture}{risc}{RISC Hardware Architecture Support}}
\providecommand{\seewasm}{}\renewcommand{\seewasm}{\seedocumentationref{For more information about how the \ecs{} supports the WebAssembly architecture}{wasm}{WebAssembly Architecture Support}}
\providecommand{\seedocumentation}{}\renewcommand{\seedocumentation}{\seedocumentationref{For more information about generic documentations and their generation by the \ecs{}}{documentation}{Generic Documentation Generation}}
\providecommand{\seedebugging}{}\renewcommand{\seedebugging}{\seedocumentationref{For more information about debugging information and its representation}{debugging}{Debugging Information Representation}}
\providecommand{\seecode}{}\renewcommand{\seecode}{\seedocumentationref{For more information about intermediate code and its purpose}{code}{Intermediate Code Representation}}
\providecommand{\seeobject}{}\renewcommand{\seeobject}{\seedocumentationref{For more information about object files and their purpose}{object}{Object File Representation}}

% generic documentation tools

\providecommand{\docprint}{
\toolsection{docprint} is a pretty printer for generic documentations.
It reformats generic documentations and writes it to the standard output stream.
\debuggingtool
\flowgraph{\resource{generic\\documentation} \ar[r] & \toolbox{docprint} \ar[r] & \resource{generic\\documentation}}
\seedocumentation
}

\providecommand{\doccheck}{
\toolsection{doccheck} is a syntactic and semantic checker for generic documentations.
It just performs syntactic and semantic checks on generic documentations and writes its diagnostic messages to the standard error stream.
\debuggingtool
\flowgraph{\resource{generic\\documentation} \ar[r] & \toolbox{doccheck} \ar[r] & \resource{diagnostic\\messages}}
\seedocumentation
}

\providecommand{\dochtml}{
\toolsection{dochtml} is an HTML documentation generator for generic documentations.
It processes several generic documentations and assembles all information therein into an HTML document.
\debuggingtool
\flowgraph{\resource{generic\\documentation} \ar[r] & \toolbox{dochtml} \ar[r] & \resource{HTML\\document}}
\seedocumentation
}

\providecommand{\doclatex}{
\toolsection{doclatex} is a Latex documentation generator for generic documentations.
It processes several generic documentations and assembles all information therein into a Latex document.
\debuggingtool
\flowgraph{\resource{generic\\documentation} \ar[r] & \toolbox{doclatex} \ar[r] & \resource{Latex\\document}}
\seedocumentation
}

% intermediate code tools

\providecommand{\cdcheck}{
\toolsection{cdcheck} is a syntactic and semantic checker for intermediate code.
It just performs syntactic and semantic checks on programs written in intermediate code and writes its diagnostic messages to the standard error stream.
\debuggingtool
\flowgraph{\resource{intermediate\\code} \ar[r] & \toolbox{cdcheck} \ar[r] & \resource{diagnostic\\messages}}
\seeassembly\seecode
}

\providecommand{\cdopt}{
\toolsection{cdopt} is an optimizer for intermediate code.
It performs various optimizations on programs written in intermediate code and writes the result to the standard output stream.
\debuggingtool
\flowgraph{\resource{intermediate\\code} \ar[r] & \toolbox{cdopt} \ar[r] & \resource{optimized\\code}}
\seeassembly\seecode
}

\providecommand{\cdrun}{
\toolsection{cdrun} is an interpreter for intermediate code.
It processes and executes programs written in intermediate code.
The following code sections are predefined and have the usual semantics:
\texttt{abort}, \texttt{\_Exit}, \texttt{fflush}, \texttt{floor}, \texttt{fputc}, \texttt{free}, \texttt{getchar}, \texttt{malloc}, and \texttt{putchar}.
Diagnostic messages about invalid operations include the name of the executed code section and the index of the erroneous instruction.
\debuggingtool
\flowgraph{\resource{intermediate\\code} \ar[r] & \toolbox{cdrun} \ar@/u/[r] & \resource{input/\\output} \ar@/d/[l]}
\seeassembly\seecode
}

\providecommand{\cdamda}{
\toolsection{cdamd16} is a compiler for intermediate code targeting the AMD64 hardware architecture.
It generates machine code for AMD64 processors from programs written in intermediate code and stores it in corresponding object files.
The compiler generates machine code for the 16-bit operating mode defined by the AMD64 architecture.
It also creates a debugging information file as well as an assembly file containing a listing of the generated machine code.
\debuggingtool
\flowgraph{\resource{intermediate\\code} \ar[r] & \toolbox{cdamd16} \ar[r] \ar[d] \ar[rd] & \resource{object file} \\ & \resource{assembly\\listing} & \resource{debugging\\information}}
\seeassembly\seeamd\seeobject\seecode\seedebugging
}

\providecommand{\cdamdb}{
\toolsection{cdamd32} is a compiler for intermediate code targeting the AMD64 hardware architecture.
It generates machine code for AMD64 processors from programs written in intermediate code and stores it in corresponding object files.
The compiler generates machine code for the 32-bit operating mode defined by the AMD64 architecture.
It also creates a debugging information file as well as an assembly file containing a listing of the generated machine code.
\debuggingtool
\flowgraph{\resource{intermediate\\code} \ar[r] & \toolbox{cdamd32} \ar[r] \ar[d] \ar[rd] & \resource{object file} \\ & \resource{assembly\\listing} & \resource{debugging\\information}}
\seeassembly\seeamd\seeobject\seecode\seedebugging
}

\providecommand{\cdamdc}{
\toolsection{cdamd64} is a compiler for intermediate code targeting the AMD64 hardware architecture.
It generates machine code for AMD64 processors from programs written in intermediate code and stores it in corresponding object files.
The compiler generates machine code for the 64-bit operating mode defined by the AMD64 architecture.
It also creates a debugging information file as well as an assembly file containing a listing of the generated machine code.
\debuggingtool
\flowgraph{\resource{intermediate\\code} \ar[r] & \toolbox{cdamd64} \ar[r] \ar[d] \ar[rd] & \resource{object file} \\ & \resource{assembly\\listing} & \resource{debugging\\information}}
\seeassembly\seeamd\seeobject\seecode\seedebugging
}

\providecommand{\cdarma}{
\toolsection{cdarma32} is a compiler for intermediate code targeting the ARM hardware architecture.
It generates machine code for ARM processors executing A32 instructions from programs written in intermediate code and stores it in corresponding object files.
It also creates a debugging information file as well as an assembly file containing a listing of the generated machine code.
\debuggingtool
\flowgraph{\resource{intermediate\\code} \ar[r] & \toolbox{cdarma32} \ar[r] \ar[d] \ar[rd] & \resource{object file} \\ & \resource{assembly\\listing} & \resource{debugging\\information}}
\seeassembly\seearm\seeobject\seecode\seedebugging
}

\providecommand{\cdarmb}{
\toolsection{cdarma64} is a compiler for intermediate code targeting the ARM hardware architecture.
It generates machine code for ARM processors executing A64 instructions from programs written in intermediate code and stores it in corresponding object files.
It also creates a debugging information file as well as an assembly file containing a listing of the generated machine code.
\debuggingtool
\flowgraph{\resource{intermediate\\code} \ar[r] & \toolbox{cdarma64} \ar[r] \ar[d] \ar[rd] & \resource{object file} \\ & \resource{assembly\\listing} & \resource{debugging\\information}}
\seeassembly\seearm\seeobject\seecode\seedebugging
}

\providecommand{\cdarmc}{
\toolsection{cdarmt32} is a compiler for intermediate code targeting the ARM hardware architecture.
It generates machine code for ARM processors without floating-point extension executing T32 instructions from programs written in intermediate code and stores it in corresponding object files.
It also creates a debugging information file as well as an assembly file containing a listing of the generated machine code.
\debuggingtool
\flowgraph{\resource{intermediate\\code} \ar[r] & \toolbox{cdarmt32} \ar[r] \ar[d] \ar[rd] & \resource{object file} \\ & \resource{assembly\\listing} & \resource{debugging\\information}}
\seeassembly\seearm\seeobject\seecode\seedebugging
}

\providecommand{\cdarmcfpe}{
\toolsection{cdarmt32fpe} is a compiler for intermediate code targeting the ARM hardware architecture.
It generates machine code for ARM processors with floating-point extension executing T32 instructions from programs written in intermediate code and stores it in corresponding object files.
It also creates a debugging information file as well as an assembly file containing a listing of the generated machine code.
\debuggingtool
\flowgraph{\resource{intermediate\\code} \ar[r] & \toolbox{cdarmt32fpe} \ar[r] \ar[d] \ar[rd] & \resource{object file} \\ & \resource{assembly\\listing} & \resource{debugging\\information}}
\seeassembly\seearm\seeobject\seecode\seedebugging
}

\providecommand{\cdavr}{
\toolsection{cdavr} is a compiler for intermediate code targeting the AVR hardware architecture.
It generates machine code for AVR processors from programs written in intermediate code and stores it in corresponding object files.
It also creates a debugging information file as well as an assembly file containing a listing of the generated machine code.
\debuggingtool
\flowgraph{\resource{intermediate\\code} \ar[r] & \toolbox{cdavr} \ar[r] \ar[d] \ar[rd] & \resource{object file} \\ & \resource{assembly\\listing} & \resource{debugging\\information}}
\seeassembly\seeavr\seeobject\seecode\seedebugging
}

\providecommand{\cdavrtt}{
\toolsection{cdavr32} is a compiler for intermediate code targeting the AVR32 hardware architecture.
It generates machine code for AVR32 processors from programs written in intermediate code and stores it in corresponding object files.
It also creates a debugging information file as well as an assembly file containing a listing of the generated machine code.
\debuggingtool
\flowgraph{\resource{intermediate\\code} \ar[r] & \toolbox{cdavr32} \ar[r] \ar[d] \ar[rd] & \resource{object file} \\ & \resource{assembly\\listing} & \resource{debugging\\information}}
\seeassembly\seeavrtt\seeobject\seecode\seedebugging
}

\providecommand{\cdmabk}{
\toolsection{cdm68k} is a compiler for intermediate code targeting the M68000 hardware architecture.
It generates machine code for M68000 processors from programs written in intermediate code and stores it in corresponding object files.
It also creates a debugging information file as well as an assembly file containing a listing of the generated machine code.
\debuggingtool
\flowgraph{\resource{intermediate\\code} \ar[r] & \toolbox{cdm68k} \ar[r] \ar[d] \ar[rd] & \resource{object file} \\ & \resource{assembly\\listing} & \resource{debugging\\information}}
\seeassembly\seemabk\seeobject\seecode\seedebugging
}

\providecommand{\cdmibl}{
\toolsection{cdmibl} is a compiler for intermediate code targeting the MicroBlaze hardware architecture.
It generates machine code for MicroBlaze processors from programs written in intermediate code and stores it in corresponding object files.
It also creates a debugging information file as well as an assembly file containing a listing of the generated machine code.
\debuggingtool
\flowgraph{\resource{intermediate\\code} \ar[r] & \toolbox{cdmibl} \ar[r] \ar[d] \ar[rd] & \resource{object file} \\ & \resource{assembly\\listing} & \resource{debugging\\information}}
\seeassembly\seemibl\seeobject\seecode\seedebugging
}

\providecommand{\cdmipsa}{
\toolsection{cdmips32} is a compiler for intermediate code targeting the MIPS32 hardware architecture.
It generates machine code for MIPS32 processors from programs written in intermediate code and stores it in corresponding object files.
It also creates a debugging information file as well as an assembly file containing a listing of the generated machine code.
\debuggingtool
\flowgraph{\resource{intermediate\\code} \ar[r] & \toolbox{cdmips32} \ar[r] \ar[d] \ar[rd] & \resource{object file} \\ & \resource{assembly\\listing} & \resource{debugging\\information}}
\seeassembly\seemips\seeobject\seecode\seedebugging
}

\providecommand{\cdmipsb}{
\toolsection{cdmips64} is a compiler for intermediate code targeting the MIPS64 hardware architecture.
It generates machine code for MIPS64 processors from programs written in intermediate code and stores it in corresponding object files.
It also creates a debugging information file as well as an assembly file containing a listing of the generated machine code.
\debuggingtool
\flowgraph{\resource{intermediate\\code} \ar[r] & \toolbox{cdmips64} \ar[r] \ar[d] \ar[rd] & \resource{object file} \\ & \resource{assembly\\listing} & \resource{debugging\\information}}
\seeassembly\seemips\seeobject\seecode\seedebugging
}

\providecommand{\cdmmix}{
\toolsection{cdmmix} is a compiler for intermediate code targeting the MMIX hardware architecture.
It generates machine code for MMIX processors from programs written in intermediate code and stores it in corresponding object files.
It also creates a debugging information file as well as an assembly file containing a listing of the generated machine code.
\debuggingtool
\flowgraph{\resource{intermediate\\code} \ar[r] & \toolbox{cdmmix} \ar[r] \ar[d] \ar[rd] & \resource{object file} \\ & \resource{assembly\\listing} & \resource{debugging\\information}}
\seeassembly\seemmix\seeobject\seecode\seedebugging
}

\providecommand{\cdorok}{
\toolsection{cdor1k} is a compiler for intermediate code targeting the OpenRISC 1000 hardware architecture.
It generates machine code for OpenRISC 1000 processors from programs written in intermediate code and stores it in corresponding object files.
It also creates a debugging information file as well as an assembly file containing a listing of the generated machine code.
\debuggingtool
\flowgraph{\resource{intermediate\\code} \ar[r] & \toolbox{cdor1k} \ar[r] \ar[d] \ar[rd] & \resource{object file} \\ & \resource{assembly\\listing} & \resource{debugging\\information}}
\seeassembly\seeorok\seeobject\seecode\seedebugging
}

\providecommand{\cdppca}{
\toolsection{cdppc32} is a compiler for intermediate code targeting the PowerPC hardware architecture.
It generates machine code for PowerPC processors from programs written in intermediate code and stores it in corresponding object files.
The compiler generates machine code for the 32-bit operating mode defined by the PowerPC architecture.
It also creates a debugging information file as well as an assembly file containing a listing of the generated machine code.
\debuggingtool
\flowgraph{\resource{intermediate\\code} \ar[r] & \toolbox{cdppc32} \ar[r] \ar[d] \ar[rd] & \resource{object file} \\ & \resource{assembly\\listing} & \resource{debugging\\information}}
\seeassembly\seeppc\seeobject\seecode\seedebugging
}

\providecommand{\cdppcb}{
\toolsection{cdppc64} is a compiler for intermediate code targeting the PowerPC hardware architecture.
It generates machine code for PowerPC processors from programs written in intermediate code and stores it in corresponding object files.
The compiler generates machine code for the 64-bit operating mode defined by the PowerPC architecture.
It also creates a debugging information file as well as an assembly file containing a listing of the generated machine code.
\debuggingtool
\flowgraph{\resource{intermediate\\code} \ar[r] & \toolbox{cdppc64} \ar[r] \ar[d] \ar[rd] & \resource{object file} \\ & \resource{assembly\\listing} & \resource{debugging\\information}}
\seeassembly\seeppc\seeobject\seecode\seedebugging
}

\providecommand{\cdrisc}{
\toolsection{cdrisc} is a compiler for intermediate code targeting the RISC hardware architecture.
It generates machine code for RISC processors from programs written in intermediate code and stores it in corresponding object files.
It also creates a debugging information file as well as an assembly file containing a listing of the generated machine code.
\debuggingtool
\flowgraph{\resource{intermediate\\code} \ar[r] & \toolbox{cdrisc} \ar[r] \ar[d] \ar[rd] & \resource{object file} \\ & \resource{assembly\\listing} & \resource{debugging\\information}}
\seeassembly\seerisc\seeobject\seecode\seedebugging
}

\providecommand{\cdwasm}{
\toolsection{cdwasm} is a compiler for intermediate code targeting the WebAssembly architecture.
It generates machine code for WebAssembly targets from programs written in intermediate code and stores it in corresponding object files.
It also creates a debugging information file as well as an assembly file containing a listing of the generated machine code.
\debuggingtool
\flowgraph{\resource{intermediate\\code} \ar[r] & \toolbox{cdwasm} \ar[r] \ar[d] \ar[rd] & \resource{object file} \\ & \resource{assembly\\listing} & \resource{debugging\\information}}
\seeassembly\seewasm\seeobject\seecode\seedebugging
}

% C++ tools

\providecommand{\cppprep}{
\toolsection{cppprep} is a preprocessor for the \cpp{} programming language.
It preprocesses source code according to the rules of \cpp{} and writes it to the standard output stream.
Only the macro names \texttt{\_\_DATE\_\_}, \texttt{\_\_FILE\_\_}, \texttt{\_\_LINE\_\_}, and \texttt{\_\_TIME\_\_} are predefined.
\flowgraph{\resource{\cpp{} or other\\source code} \ar[r] & \toolbox{cppprep} \ar[r] & \resource{preprocessed\\source code} \\ & \variable{ECSINCLUDE} \ar[u]}
\seecpp
}

\providecommand{\cppprint}{
\toolsection{cppprint} is a pretty printer for the \cpp{} programming language.
It reformats the source code of \cpp{} programs and writes it to the standard output stream.
\flowgraph{\resource{\cpp{}\\source code} \ar[r] & \toolbox{cppprint} \ar[r] & \resource{reformatted\\source code} \\ & \variable{ECSINCLUDE} \ar[u]}
\seecpp
}

\providecommand{\cppcheck}{
\toolsection{cppcheck} is a syntactic and semantic checker for the \cpp{} programming language.
It just performs syntactic and semantic checks on \cpp{} programs and writes its diagnostic messages to the standard error stream.
\flowgraph{\resource{\cpp{}\\source code} \ar[r] & \toolbox{cppcheck} \ar[r] & \resource{diagnostic\\messages} \\ & \variable{ECSINCLUDE} \ar[u]}
\seecpp
}

\providecommand{\cppdump}{
\toolsection{cppdump} is a serializer for the \cpp{} programming language.
It dumps the complete internal representation of programs written in \cpp{} into an XML document.
\debuggingtool
\flowgraph{\resource{\cpp{}\\source code} \ar[r] & \toolbox{cppdump} \ar[r] & \resource{internal\\representation} \\ & \variable{ECSINCLUDE} \ar[u]}
\seecpp
}

\providecommand{\cpprun}{
\toolsection{cpprun} is an interpreter for the \cpp{} programming language.
It processes and executes programs written in \cpp{}.
The macro \texttt{\_\_run\_\_} is predefined in order to enable programmers to identify this tool while interpreting.
\flowgraph{\resource{\cpp{}\\source code} \ar[r] & \toolbox{cpprun} \ar@/u/[r] & \resource{input/\\output} \ar@/d/[l] \\ & \variable{ECSINCLUDE} \ar[u]}
\seecpp
}

\providecommand{\cppdoc}{
\toolsection{cppdoc} is a generic documentation generator for the \cpp{} programming language.
It processes several \cpp{} source files and assembles all information therein into a generic documentation.
\debuggingtool
\flowgraph{\resource{\cpp{}\\source code} \ar[r] & \toolbox{cppdoc} \ar[r] & \resource{generic\\documentation} \\ & \variable{ECSINCLUDE} \ar[u]}
\seecpp\seedocumentation
}

\providecommand{\cpphtml}{
\toolsection{cpphtml} is an HTML documentation generator for the \cpp{} programming language.
It processes several \cpp{} source files and assembles all information therein into an HTML document.
\flowgraph{\resource{\cpp{}\\source code} \ar[r] & \toolbox{cpphtml} \ar[r] & \resource{HTML\\document} \\ & \variable{ECSINCLUDE} \ar[u]}
\seecpp\seedocumentation
}

\providecommand{\cpplatex}{
\toolsection{cpplatex} is a Latex documentation generator for the \cpp{} programming language.
It processes several \cpp{} source files and assembles all information therein into a Latex document.
\flowgraph{\resource{\cpp{}\\source code} \ar[r] & \toolbox{cpplatex} \ar[r] & \resource{Latex\\document} \\ & \variable{ECSINCLUDE} \ar[u]}
\seecpp\seedocumentation
}

\providecommand{\cppcode}{
\toolsection{cppcode} is an intermediate code generator for the \cpp{} programming language.
It generates intermediate code from programs written in \cpp{} and stores it in corresponding assembly files.
The macro \texttt{\_\_code\_\_} is predefined in order to enable programmers to identify this tool while generating intermediate code.
Programs generated with this tool require additional runtime support that is stored in the \file{cpp\-code\-run} library file.
\debuggingtool
\flowgraph{\resource{\cpp{}\\source code} \ar[r] & \toolbox{cppcode} \ar[r] & \resource{intermediate\\code} \\ & \variable{ECSINCLUDE} \ar[u]}
\seecpp\seeassembly\seecode
}

\providecommand{\cppamda}{
\toolsection{cppamd16} is a compiler for the \cpp{} programming language targeting the AMD64 hardware architecture.
It generates machine code for AMD64 processors from programs written in \cpp{} and stores it in corresponding object files.
The compiler generates machine code for the 16-bit operating mode defined by the AMD64 architecture.
For debugging purposes, it also creates a debugging information file as well as an assembly file containing a listing of the generated machine code.
The macro \texttt{\_\_amd16\_\_} is predefined in order to enable programmers to identify this tool and its target architecture while compiling.
Programs generated with this compiler require additional runtime support that is stored in the \file{cpp\-amd16\-run} library file.
\flowgraph{\resource{\cpp{}\\source code} \ar[r] & \toolbox{cppamd16} \ar[r] \ar[d] \ar[rd] & \resource{object file} \\ \variable{ECSINCLUDE} \ar[ru] & \resource{debugging\\information} & \resource{assembly\\listing}}
\seecpp\seeassembly\seeamd\seeobject\seedebugging
}

\providecommand{\cppamdb}{
\toolsection{cppamd32} is a compiler for the \cpp{} programming language targeting the AMD64 hardware architecture.
It generates machine code for AMD64 processors from programs written in \cpp{} and stores it in corresponding object files.
The compiler generates machine code for the 32-bit operating mode defined by the AMD64 architecture.
For debugging purposes, it also creates a debugging information file as well as an assembly file containing a listing of the generated machine code.
The macro \texttt{\_\_amd32\_\_} is predefined in order to enable programmers to identify this tool and its target architecture while compiling.
Programs generated with this compiler require additional runtime support that is stored in the \file{cpp\-amd32\-run} library file.
\flowgraph{\resource{\cpp{}\\source code} \ar[r] & \toolbox{cppamd32} \ar[r] \ar[d] \ar[rd] & \resource{object file} \\ \variable{ECSINCLUDE} \ar[ru] & \resource{debugging\\information} & \resource{assembly\\listing}}
\seecpp\seeassembly\seeamd\seeobject\seedebugging
}

\providecommand{\cppamdc}{
\toolsection{cppamd64} is a compiler for the \cpp{} programming language targeting the AMD64 hardware architecture.
It generates machine code for AMD64 processors from programs written in \cpp{} and stores it in corresponding object files.
The compiler generates machine code for the 64-bit operating mode defined by the AMD64 architecture.
For debugging purposes, it also creates a debugging information file as well as an assembly file containing a listing of the generated machine code.
The macro \texttt{\_\_amd64\_\_} is predefined in order to enable programmers to identify this tool and its target architecture while compiling.
Programs generated with this compiler require additional runtime support that is stored in the \file{cpp\-amd64\-run} library file.
\flowgraph{\resource{\cpp{}\\source code} \ar[r] & \toolbox{cppamd64} \ar[r] \ar[d] \ar[rd] & \resource{object file} \\ \variable{ECSINCLUDE} \ar[ru] & \resource{debugging\\information} & \resource{assembly\\listing}}
\seecpp\seeassembly\seeamd\seeobject\seedebugging
}

\providecommand{\cpparma}{
\toolsection{cpparma32} is a compiler for the \cpp{} programming language targeting the ARM hardware architecture.
It generates machine code for ARM processors executing A32 instructions from programs written in \cpp{} and stores it in corresponding object files.
For debugging purposes, it also creates a debugging information file as well as an assembly file containing a listing of the generated machine code.
The macro \texttt{\_\_arma32\_\_} is predefined in order to enable programmers to identify this tool and its target architecture while compiling.
Programs generated with this compiler require additional runtime support that is stored in the \file{cpp\-arma32\-run} library file.
\flowgraph{\resource{\cpp{}\\source code} \ar[r] & \toolbox{cpparma32} \ar[r] \ar[d] \ar[rd] & \resource{object file} \\ \variable{ECSINCLUDE} \ar[ru] & \resource{debugging\\information} & \resource{assembly\\listing}}
\seecpp\seeassembly\seearm\seeobject\seedebugging
}

\providecommand{\cpparmb}{
\toolsection{cpparma64} is a compiler for the \cpp{} programming language targeting the ARM hardware architecture.
It generates machine code for ARM processors executing A64 instructions from programs written in \cpp{} and stores it in corresponding object files.
For debugging purposes, it also creates a debugging information file as well as an assembly file containing a listing of the generated machine code.
The macro \texttt{\_\_arma64\_\_} is predefined in order to enable programmers to identify this tool and its target architecture while compiling.
Programs generated with this compiler require additional runtime support that is stored in the \file{cpp\-arma64\-run} library file.
\flowgraph{\resource{\cpp{}\\source code} \ar[r] & \toolbox{cpparma64} \ar[r] \ar[d] \ar[rd] & \resource{object file} \\ \variable{ECSINCLUDE} \ar[ru] & \resource{debugging\\information} & \resource{assembly\\listing}}
\seecpp\seeassembly\seearm\seeobject\seedebugging
}

\providecommand{\cpparmc}{
\toolsection{cpparmt32} is a compiler for the \cpp{} programming language targeting the ARM hardware architecture.
It generates machine code for ARM processors without floating-point extension executing T32 instructions from programs written in \cpp{} and stores it in corresponding object files.
For debugging purposes, it also creates a debugging information file as well as an assembly file containing a listing of the generated machine code.
The macro \texttt{\_\_armt32\_\_} is predefined in order to enable programmers to identify this tool and its target architecture while compiling.
Programs generated with this compiler require additional runtime support that is stored in the \file{cpp\-armt32\-run} library file.
\flowgraph{\resource{\cpp{}\\source code} \ar[r] & \toolbox{cpparmt32} \ar[r] \ar[d] \ar[rd] & \resource{object file} \\ \variable{ECSINCLUDE} \ar[ru] & \resource{debugging\\information} & \resource{assembly\\listing}}
\seecpp\seeassembly\seearm\seeobject\seedebugging
}

\providecommand{\cpparmcfpe}{
\toolsection{cpparmt32fpe} is a compiler for the \cpp{} programming language targeting the ARM hardware architecture.
It generates machine code for ARM processors with floating-point extension executing T32 instructions from programs written in \cpp{} and stores it in corresponding object files.
For debugging purposes, it also creates a debugging information file as well as an assembly file containing a listing of the generated machine code.
The macro \texttt{\_\_armt32fpe\_\_} is predefined in order to enable programmers to identify this tool and its target architecture while compiling.
Programs generated with this compiler require additional runtime support that is stored in the \file{cpp\-armt32\-fpe\-run} library file.
\flowgraph{\resource{\cpp{}\\source code} \ar[r] & \toolbox{cpparmt32fpe} \ar[r] \ar[d] \ar[rd] & \resource{object file} \\ \variable{ECSINCLUDE} \ar[ru] & \resource{debugging\\information} & \resource{assembly\\listing}}
\seecpp\seeassembly\seearm\seeobject\seedebugging
}

\providecommand{\cppavr}{
\toolsection{cppavr} is a compiler for the \cpp{} programming language targeting the AVR hardware architecture.
It generates machine code for AVR processors from programs written in \cpp{} and stores it in corresponding object files.
For debugging purposes, it also creates a debugging information file as well as an assembly file containing a listing of the generated machine code.
The macro \texttt{\_\_avr\_\_} is predefined in order to enable programmers to identify this tool and its target architecture while compiling.
Programs generated with this compiler require additional runtime support that is stored in the \file{cpp\-avr\-run} library file.
\flowgraph{\resource{\cpp{}\\source code} \ar[r] & \toolbox{cppavr} \ar[r] \ar[d] \ar[rd] & \resource{object file} \\ \variable{ECSINCLUDE} \ar[ru] & \resource{debugging\\information} & \resource{assembly\\listing}}
\seecpp\seeassembly\seeavr\seeobject\seedebugging
}

\providecommand{\cppavrtt}{
\toolsection{cppavr32} is a compiler for the \cpp{} programming language targeting the AVR32 hardware architecture.
It generates machine code for AVR32 processors from programs written in \cpp{} and stores it in corresponding object files.
For debugging purposes, it also creates a debugging information file as well as an assembly file containing a listing of the generated machine code.
The macro \texttt{\_\_avr32\_\_} is predefined in order to enable programmers to identify this tool and its target architecture while compiling.
Programs generated with this compiler require additional runtime support that is stored in the \file{cpp\-avr32\-run} library file.
\flowgraph{\resource{\cpp{}\\source code} \ar[r] & \toolbox{cppavr32} \ar[r] \ar[d] \ar[rd] & \resource{object file} \\ \variable{ECSINCLUDE} \ar[ru] & \resource{debugging\\information} & \resource{assembly\\listing}}
\seecpp\seeassembly\seeavrtt\seeobject\seedebugging
}

\providecommand{\cppmabk}{
\toolsection{cppm68k} is a compiler for the \cpp{} programming language targeting the M68000 hardware architecture.
It generates machine code for M68000 processors from programs written in \cpp{} and stores it in corresponding object files.
For debugging purposes, it also creates a debugging information file as well as an assembly file containing a listing of the generated machine code.
The macro \texttt{\_\_m68k\_\_} is predefined in order to enable programmers to identify this tool and its target architecture while compiling.
Programs generated with this compiler require additional runtime support that is stored in the \file{cpp\-m68k\-run} library file.
\flowgraph{\resource{\cpp{}\\source code} \ar[r] & \toolbox{cppm68k} \ar[r] \ar[d] \ar[rd] & \resource{object file} \\ \variable{ECSINCLUDE} \ar[ru] & \resource{debugging\\information} & \resource{assembly\\listing}}
\seecpp\seeassembly\seemabk\seeobject\seedebugging
}

\providecommand{\cppmibl}{
\toolsection{cppmibl} is a compiler for the \cpp{} programming language targeting the MicroBlaze hardware architecture.
It generates machine code for MicroBlaze processors from programs written in \cpp{} and stores it in corresponding object files.
For debugging purposes, it also creates a debugging information file as well as an assembly file containing a listing of the generated machine code.
The macro \texttt{\_\_mibl\_\_} is predefined in order to enable programmers to identify this tool and its target architecture while compiling.
Programs generated with this compiler require additional runtime support that is stored in the \file{cpp\-mibl\-run} library file.
\flowgraph{\resource{\cpp{}\\source code} \ar[r] & \toolbox{cppmibl} \ar[r] \ar[d] \ar[rd] & \resource{object file} \\ \variable{ECSINCLUDE} \ar[ru] & \resource{debugging\\information} & \resource{assembly\\listing}}
\seecpp\seeassembly\seemibl\seeobject\seedebugging
}

\providecommand{\cppmipsa}{
\toolsection{cppmips32} is a compiler for the \cpp{} programming language targeting the MIPS32 hardware architecture.
It generates machine code for MIPS32 processors from programs written in \cpp{} and stores it in corresponding object files.
For debugging purposes, it also creates a debugging information file as well as an assembly file containing a listing of the generated machine code.
The macro \texttt{\_\_mips32\_\_} is predefined in order to enable programmers to identify this tool and its target architecture while compiling.
Programs generated with this compiler require additional runtime support that is stored in the \file{cpp\-mips32\-run} library file.
\flowgraph{\resource{\cpp{}\\source code} \ar[r] & \toolbox{cppmips32} \ar[r] \ar[d] \ar[rd] & \resource{object file} \\ \variable{ECSINCLUDE} \ar[ru] & \resource{debugging\\information} & \resource{assembly\\listing}}
\seecpp\seeassembly\seemips\seeobject\seedebugging
}

\providecommand{\cppmipsb}{
\toolsection{cppmips64} is a compiler for the \cpp{} programming language targeting the MIPS64 hardware architecture.
It generates machine code for MIPS64 processors from programs written in \cpp{} and stores it in corresponding object files.
For debugging purposes, it also creates a debugging information file as well as an assembly file containing a listing of the generated machine code.
The macro \texttt{\_\_mips64\_\_} is predefined in order to enable programmers to identify this tool and its target architecture while compiling.
Programs generated with this compiler require additional runtime support that is stored in the \file{cpp\-mips64\-run} library file.
\flowgraph{\resource{\cpp{}\\source code} \ar[r] & \toolbox{cppmips64} \ar[r] \ar[d] \ar[rd] & \resource{object file} \\ \variable{ECSINCLUDE} \ar[ru] & \resource{debugging\\information} & \resource{assembly\\listing}}
\seecpp\seeassembly\seemips\seeobject\seedebugging
}

\providecommand{\cppmmix}{
\toolsection{cppmmix} is a compiler for the \cpp{} programming language targeting the MMIX hardware architecture.
It generates machine code for MMIX processors from programs written in \cpp{} and stores it in corresponding object files.
For debugging purposes, it also creates a debugging information file as well as an assembly file containing a listing of the generated machine code.
The macro \texttt{\_\_mmix\_\_} is predefined in order to enable programmers to identify this tool and its target architecture while compiling.
Programs generated with this compiler require additional runtime support that is stored in the \file{cpp\-mmix\-run} library file.
\flowgraph{\resource{\cpp{}\\source code} \ar[r] & \toolbox{cppmmix} \ar[r] \ar[d] \ar[rd] & \resource{object file} \\ \variable{ECSINCLUDE} \ar[ru] & \resource{debugging\\information} & \resource{assembly\\listing}}
\seecpp\seeassembly\seemmix\seeobject\seedebugging
}

\providecommand{\cpporok}{
\toolsection{cppor1k} is a compiler for the \cpp{} programming language targeting the OpenRISC 1000 hardware architecture.
It generates machine code for OpenRISC 1000 processors from programs written in \cpp{} and stores it in corresponding object files.
For debugging purposes, it also creates a debugging information file as well as an assembly file containing a listing of the generated machine code.
The macro \texttt{\_\_or1k\_\_} is predefined in order to enable programmers to identify this tool and its target architecture while compiling.
Programs generated with this compiler require additional runtime support that is stored in the \file{cpp\-or1k\-run} library file.
\flowgraph{\resource{\cpp{}\\source code} \ar[r] & \toolbox{cppor1k} \ar[r] \ar[d] \ar[rd] & \resource{object file} \\ \variable{ECSINCLUDE} \ar[ru] & \resource{debugging\\information} & \resource{assembly\\listing}}
\seecpp\seeassembly\seeorok\seeobject\seedebugging
}

\providecommand{\cppppca}{
\toolsection{cppppc32} is a compiler for the \cpp{} programming language targeting the PowerPC hardware architecture.
It generates machine code for PowerPC processors from programs written in \cpp{} and stores it in corresponding object files.
The compiler generates machine code for the 32-bit operating mode defined by the PowerPC architecture.
For debugging purposes, it also creates a debugging information file as well as an assembly file containing a listing of the generated machine code.
The macro \texttt{\_\_ppc32\_\_} is predefined in order to enable programmers to identify this tool and its target architecture while compiling.
Programs generated with this compiler require additional runtime support that is stored in the \file{cpp\-ppc32\-run} library file.
\flowgraph{\resource{\cpp{}\\source code} \ar[r] & \toolbox{cppppc32} \ar[r] \ar[d] \ar[rd] & \resource{object file} \\ \variable{ECSINCLUDE} \ar[ru] & \resource{debugging\\information} & \resource{assembly\\listing}}
\seecpp\seeassembly\seeppc\seeobject\seedebugging
}

\providecommand{\cppppcb}{
\toolsection{cppppc64} is a compiler for the \cpp{} programming language targeting the PowerPC hardware architecture.
It generates machine code for PowerPC processors from programs written in \cpp{} and stores it in corresponding object files.
The compiler generates machine code for the 64-bit operating mode defined by the PowerPC architecture.
For debugging purposes, it also creates a debugging information file as well as an assembly file containing a listing of the generated machine code.
The macro \texttt{\_\_ppc64\_\_} is predefined in order to enable programmers to identify this tool and its target architecture while compiling.
Programs generated with this compiler require additional runtime support that is stored in the \file{cpp\-ppc64\-run} library file.
\flowgraph{\resource{\cpp{}\\source code} \ar[r] & \toolbox{cppppc64} \ar[r] \ar[d] \ar[rd] & \resource{object file} \\ \variable{ECSINCLUDE} \ar[ru] & \resource{debugging\\information} & \resource{assembly\\listing}}
\seecpp\seeassembly\seeppc\seeobject\seedebugging
}

\providecommand{\cpprisc}{
\toolsection{cpprisc} is a compiler for the \cpp{} programming language targeting the RISC hardware architecture.
It generates machine code for RISC processors from programs written in \cpp{} and stores it in corresponding object files.
For debugging purposes, it also creates a debugging information file as well as an assembly file containing a listing of the generated machine code.
The macro \texttt{\_\_risc\_\_} is predefined in order to enable programmers to identify this tool and its target architecture while compiling.
Programs generated with this compiler require additional runtime support that is stored in the \file{cpp\-risc\-run} library file.
\flowgraph{\resource{\cpp{}\\source code} \ar[r] & \toolbox{cpprisc} \ar[r] \ar[d] \ar[rd] & \resource{object file} \\ \variable{ECSINCLUDE} \ar[ru] & \resource{debugging\\information} & \resource{assembly\\listing}}
\seecpp\seeassembly\seerisc\seeobject\seedebugging
}

\providecommand{\cppwasm}{
\toolsection{cppwasm} is a compiler for the \cpp{} programming language targeting the WebAssembly architecture.
It generates machine code for WebAssembly targets from programs written in \cpp{} and stores it in corresponding object files.
For debugging purposes, it also creates a debugging information file as well as an assembly file containing a listing of the generated machine code.
The macro \texttt{\_\_wasm\_\_} is predefined in order to enable programmers to identify this tool and its target architecture while compiling.
Programs generated with this compiler require additional runtime support that is stored in the \file{cpp\-wasm\-run} library file.
\flowgraph{\resource{\cpp{}\\source code} \ar[r] & \toolbox{cppwasm} \ar[r] \ar[d] \ar[rd] & \resource{object file} \\ \variable{ECSINCLUDE} \ar[ru] & \resource{debugging\\information} & \resource{assembly\\listing}}
\seecpp\seeassembly\seewasm\seeobject\seedebugging
}

% FALSE tools

\providecommand{\falprint}{
\toolsection{falprint} is a pretty printer for the FALSE programming language.
It reformats the source code of FALSE programs and writes it to the standard output stream.
\flowgraph{\resource{FALSE\\source code} \ar[r] & \toolbox{falprint} \ar[r] & \resource{reformatted\\source code}}
\seefalse
}

\providecommand{\falcheck}{
\toolsection{falcheck} is a syntactic and semantic checker for the FALSE programming language.
It just performs syntactic and semantic checks on FALSE programs and writes its diagnostic messages to the standard error stream.
\flowgraph{\resource{FALSE\\source code} \ar[r] & \toolbox{falcheck} \ar[r] & \resource{diagnostic\\messages}}
\seefalse
}

\providecommand{\faldump}{
\toolsection{faldump} is a serializer for the FALSE programming language.
It dumps the complete internal representation of programs written in FALSE into an XML document.
\debuggingtool
\flowgraph{\resource{FALSE\\source code} \ar[r] & \toolbox{faldump} \ar[r] & \resource{internal\\representation}}
\seefalse
}

\providecommand{\falrun}{
\toolsection{falrun} is an interpreter for the FALSE programming language.
It processes and executes programs written in FALSE\@.
\flowgraph{\resource{FALSE\\source code} \ar[r] & \toolbox{falrun} \ar@/u/[r] & \resource{input/\\output} \ar@/d/[l]}
\seefalse
}

\providecommand{\falcpp}{
\toolsection{falcpp} is a transpiler for the FALSE programming language.
It translates programs written in FALSE into \cpp{} programs and stores them in corresponding source files.
\flowgraph{\resource{FALSE\\source code} \ar[r] & \toolbox{falcpp} \ar[r] & \resource{\cpp{}\\source file}}
\seefalse\seecpp
}

\providecommand{\falcode}{
\toolsection{falcode} is an intermediate code generator for the FALSE programming language.
It generates intermediate code from programs written in FALSE and stores it in corresponding assembly files.
\debuggingtool
\flowgraph{\resource{FALSE\\source code} \ar[r] & \toolbox{falcode} \ar[r] & \resource{intermediate\\code}}
\seefalse\seeassembly\seecode
}

\providecommand{\falamda}{
\toolsection{falamd16} is a compiler for the FALSE programming language targeting the AMD64 hardware architecture.
It generates machine code for AMD64 processors from programs written in FALSE and stores it in corresponding object files.
The compiler generates machine code for the 16-bit operating mode defined by the AMD64 architecture.
\flowgraph{\resource{FALSE\\source code} \ar[r] & \toolbox{falamd16} \ar[r] & \resource{object file}}
\seefalse\seeamd\seeobject
}

\providecommand{\falamdb}{
\toolsection{falamd32} is a compiler for the FALSE programming language targeting the AMD64 hardware architecture.
It generates machine code for AMD64 processors from programs written in FALSE and stores it in corresponding object files.
The compiler generates machine code for the 32-bit operating mode defined by the AMD64 architecture.
\flowgraph{\resource{FALSE\\source code} \ar[r] & \toolbox{falamd32} \ar[r] & \resource{object file}}
\seefalse\seeamd\seeobject
}

\providecommand{\falamdc}{
\toolsection{falamd64} is a compiler for the FALSE programming language targeting the AMD64 hardware architecture.
It generates machine code for AMD64 processors from programs written in FALSE and stores it in corresponding object files.
The compiler generates machine code for the 64-bit operating mode defined by the AMD64 architecture.
\flowgraph{\resource{FALSE\\source code} \ar[r] & \toolbox{falamd64} \ar[r] & \resource{object file}}
\seefalse\seeamd\seeobject
}

\providecommand{\falarma}{
\toolsection{falarma32} is a compiler for the FALSE programming language targeting the ARM hardware architecture.
It generates machine code for ARM processors executing A32 instructions from programs written in FALSE and stores it in corresponding object files.
\flowgraph{\resource{FALSE\\source code} \ar[r] & \toolbox{falarma32} \ar[r] & \resource{object file}}
\seefalse\seearm\seeobject
}

\providecommand{\falarmb}{
\toolsection{falarma64} is a compiler for the FALSE programming language targeting the ARM hardware architecture.
It generates machine code for ARM processors executing A64 instructions from programs written in FALSE and stores it in corresponding object files.
\flowgraph{\resource{FALSE\\source code} \ar[r] & \toolbox{falarma64} \ar[r] & \resource{object file}}
\seefalse\seearm\seeobject
}

\providecommand{\falarmc}{
\toolsection{falarmt32} is a compiler for the FALSE programming language targeting the ARM hardware architecture.
It generates machine code for ARM processors without floating-point extension executing T32 instructions from programs written in FALSE and stores it in corresponding object files.
\flowgraph{\resource{FALSE\\source code} \ar[r] & \toolbox{falarmt32} \ar[r] & \resource{object file}}
\seefalse\seearm\seeobject
}

\providecommand{\falarmcfpe}{
\toolsection{falarmt32fpe} is a compiler for the FALSE programming language targeting the ARM hardware architecture.
It generates machine code for ARM processors with floating-point extension executing T32 instructions from programs written in FALSE and stores it in corresponding object files.
\flowgraph{\resource{FALSE\\source code} \ar[r] & \toolbox{falarmt32fpe} \ar[r] & \resource{object file}}
\seefalse\seearm\seeobject
}

\providecommand{\falavr}{
\toolsection{falavr} is a compiler for the FALSE programming language targeting the AVR hardware architecture.
It generates machine code for AVR processors from programs written in FALSE and stores it in corresponding object files.
\flowgraph{\resource{FALSE\\source code} \ar[r] & \toolbox{falavr} \ar[r] & \resource{object file}}
\seefalse\seeavr\seeobject
}

\providecommand{\falavrtt}{
\toolsection{falavr32} is a compiler for the FALSE programming language targeting the AVR32 hardware architecture.
It generates machine code for AVR32 processors from programs written in FALSE and stores it in corresponding object files.
\flowgraph{\resource{FALSE\\source code} \ar[r] & \toolbox{falavr32} \ar[r] & \resource{object file}}
\seefalse\seeavrtt\seeobject
}

\providecommand{\falmabk}{
\toolsection{falm68k} is a compiler for the FALSE programming language targeting the M68000 hardware architecture.
It generates machine code for M68000 processors from programs written in FALSE and stores it in corresponding object files.
\flowgraph{\resource{FALSE\\source code} \ar[r] & \toolbox{falm68k} \ar[r] & \resource{object file}}
\seefalse\seemabk\seeobject
}

\providecommand{\falmibl}{
\toolsection{falmibl} is a compiler for the FALSE programming language targeting the MicroBlaze hardware architecture.
It generates machine code for MicroBlaze processors from programs written in FALSE and stores it in corresponding object files.
\flowgraph{\resource{FALSE\\source code} \ar[r] & \toolbox{falmibl} \ar[r] & \resource{object file}}
\seefalse\seemibl\seeobject
}

\providecommand{\falmipsa}{
\toolsection{falmips32} is a compiler for the FALSE programming language targeting the MIPS32 hardware architecture.
It generates machine code for MIPS32 processors from programs written in FALSE and stores it in corresponding object files.
\flowgraph{\resource{FALSE\\source code} \ar[r] & \toolbox{falmips32} \ar[r] & \resource{object file}}
\seefalse\seemips\seeobject
}

\providecommand{\falmipsb}{
\toolsection{falmips64} is a compiler for the FALSE programming language targeting the MIPS64 hardware architecture.
It generates machine code for MIPS64 processors from programs written in FALSE and stores it in corresponding object files.
\flowgraph{\resource{FALSE\\source code} \ar[r] & \toolbox{falmips64} \ar[r] & \resource{object file}}
\seefalse\seemips\seeobject
}

\providecommand{\falmmix}{
\toolsection{falmmix} is a compiler for the FALSE programming language targeting the MMIX hardware architecture.
It generates machine code for MMIX processors from programs written in FALSE and stores it in corresponding object files.
\flowgraph{\resource{FALSE\\source code} \ar[r] & \toolbox{falmmix} \ar[r] & \resource{object file}}
\seefalse\seemmix\seeobject
}

\providecommand{\falorok}{
\toolsection{falor1k} is a compiler for the FALSE programming language targeting the OpenRISC 1000 hardware architecture.
It generates machine code for OpenRISC 1000 processors from programs written in FALSE and stores it in corresponding object files.
\flowgraph{\resource{FALSE\\source code} \ar[r] & \toolbox{falor1k} \ar[r] & \resource{object file}}
\seefalse\seeorok\seeobject
}

\providecommand{\falppca}{
\toolsection{falppc32} is a compiler for the FALSE programming language targeting the PowerPC hardware architecture.
It generates machine code for PowerPC processors from programs written in FALSE and stores it in corresponding object files.
The compiler generates machine code for the 32-bit operating mode defined by the PowerPC architecture.
\flowgraph{\resource{FALSE\\source code} \ar[r] & \toolbox{falppc32} \ar[r] & \resource{object file}}
\seefalse\seeppc\seeobject
}

\providecommand{\falppcb}{
\toolsection{falppc64} is a compiler for the FALSE programming language targeting the PowerPC hardware architecture.
It generates machine code for PowerPC processors from programs written in FALSE and stores it in corresponding object files.
The compiler generates machine code for the 64-bit operating mode defined by the PowerPC architecture.
\flowgraph{\resource{FALSE\\source code} \ar[r] & \toolbox{falppc64} \ar[r] & \resource{object file}}
\seefalse\seeppc\seeobject
}

\providecommand{\falrisc}{
\toolsection{falrisc} is a compiler for the FALSE programming language targeting the RISC hardware architecture.
It generates machine code for RISC processors from programs written in FALSE and stores it in corresponding object files.
\flowgraph{\resource{FALSE\\source code} \ar[r] & \toolbox{falrisc} \ar[r] & \resource{object file}}
\seefalse\seerisc\seeobject
}

\providecommand{\falwasm}{
\toolsection{falwasm} is a compiler for the FALSE programming language targeting the WebAssembly architecture.
It generates machine code for WebAssembly targets from programs written in FALSE and stores it in corresponding object files.
\flowgraph{\resource{FALSE\\source code} \ar[r] & \toolbox{falwasm} \ar[r] & \resource{object file}}
\seefalse\seewasm\seeobject
}

% Oberon tools

\providecommand{\obprint}{
\toolsection{obprint} is a pretty printer for the Oberon programming language.
It reformats the source code of Oberon modules and writes it to the standard output stream.
\flowgraph{\resource{Oberon\\source code} \ar[r] & \toolbox{obprint} \ar[r] & \resource{reformatted\\source code}}
\seeoberon
}

\providecommand{\obcheck}{
\toolsection{obcheck} is a syntactic and semantic checker for the Oberon programming language.
It just performs syntactic and semantic checks on Oberon modules and writes its diagnostic messages to the standard error stream.
In addition, it stores the interface of each module in a symbol file which is required when other modules import the module.
\flowgraph{\resource{Oberon\\source code} \ar[r] & \toolbox{obcheck} \ar[r] \ar@/l/[d] & \resource{diagnostic\\messages} \\ \variable{ECSIMPORT} \ar[ru] & \resource{symbol\\files} \ar@/r/[u]}
\seeoberon
}

\providecommand{\obdump}{
\toolsection{obdump} is a serializer for the Oberon programming language.
It dumps the complete internal representation of modules written in Oberon into an XML document.
\debuggingtool
\flowgraph{\resource{Oberon\\source code} \ar[r] & \toolbox{obdump} \ar[r] \ar@/l/[d] & \resource{internal\\representation} \\ \variable{ECSIMPORT} \ar[ru] & \resource{symbol\\files} \ar@/r/[u]}
\seeoberon
}

\providecommand{\obrun}{
\toolsection{obrun} is an interpreter for the Oberon programming language.
It processes and executes modules written in Oberon.
This tool does neither generate nor process symbol files while interpreting modules.
If a module is imported by another one, its filename has to be named before the other one in the list of command-line arguments.
\flowgraph{\resource{Oberon\\source code} \ar[r] & \toolbox{obrun} \ar@/u/[r] & \resource{input/\\output} \ar@/d/[l]}
\seeoberon
}

\providecommand{\obcpp}{
\toolsection{obcpp} is a transpiler for the Oberon programming language.
It translates programs written in Oberon into \cpp{} programs and stores them in corresponding source and header files.
In addition, it stores the interface of each module in a symbol file which is required when other modules import the module.
The same interface is provided by the generated header file which can be used in other parts of the \cpp{} program.
\flowgraph{\resource{Oberon\\source code} \ar[r] & \toolbox{obcpp} \ar[r] \ar@/l/[d] \ar[rd] & \resource{\cpp{}\\source file} \\ \variable{ECSIMPORT} \ar[ru] & \resource{symbol\\files} \ar@/r/[u] & \resource{\cpp{}\\header file}}
\seeoberon\seecpp
}

\providecommand{\obdoc}{
\toolsection{obdoc} is a generic documentation generator for the Oberon programming language.
It processes several Oberon modules and assembles all information therein into a generic documentation.
In addition, it stores the interface of each module in a symbol file which is required when other modules import the module.
\debuggingtool
\flowgraph{\resource{Oberon\\source code} \ar[r] & \toolbox{obdoc} \ar[r] \ar@/l/[d] & \resource{generic\\documentation} \\ \variable{ECSIMPORT} \ar[ru] & \resource{symbol\\files} \ar@/r/[u]}
\seeoberon\seedocumentation
}

\providecommand{\obhtml}{
\toolsection{obhtml} is an HTML documentation generator for the Oberon programming language.
It processes several Oberon modules and assembles all information therein into an HTML document.
In addition, it stores the interface of each module in a symbol file which is required when other modules import the module.
\flowgraph{\resource{Oberon\\source code} \ar[r] & \toolbox{obhtml} \ar[r] \ar@/l/[d] & \resource{HTML\\document} \\ \variable{ECSIMPORT} \ar[ru] & \resource{symbol\\files} \ar@/r/[u]}
\seeoberon\seedocumentation
}

\providecommand{\oblatex}{
\toolsection{oblatex} is a Latex documentation generator for the Oberon programming language.
It processes several Oberon modules and assembles all information therein into a Latex document.
In addition, it stores the interface of each module in a symbol file which is required when other modules import the module.
\flowgraph{\resource{Oberon\\source code} \ar[r] & \toolbox{oblatex} \ar[r] \ar@/l/[d] & \resource{Latex\\document} \\ \variable{ECSIMPORT} \ar[ru] & \resource{symbol\\files} \ar@/r/[u]}
\seeoberon\seedocumentation
}

\providecommand{\obcode}{
\toolsection{obcode} is an intermediate code generator for the Oberon programming language.
It generates intermediate code from modules written in Oberon and stores it in corresponding assembly files.
In addition, it stores the interface of each module in a symbol file which is required when other modules import the module.
Programs generated with this tool require additional runtime support that is stored in the \file{ob\-code\-run} library file.
\debuggingtool
\flowgraph{\resource{Oberon\\source code} \ar[r] & \toolbox{obcode} \ar[r] \ar@/l/[d] & \resource{intermediate\\code} \\ \variable{ECSIMPORT} \ar[ru] & \resource{symbol\\files} \ar@/r/[u]}
\seeoberon\seeassembly\seecode
}

\providecommand{\obamda}{
\toolsection{obamd16} is a compiler for the Oberon programming language targeting the AMD64 hardware architecture.
It generates machine code for AMD64 processors from modules written in Oberon and stores it in corresponding object files.
The compiler generates machine code for the 16-bit operating mode defined by the AMD64 architecture.
For debugging purposes, it also creates a debugging information file as well as an assembly file containing a listing of the generated machine code.
In addition, it stores the interface of each module in a symbol file which is required when other modules import the module.
Programs generated with this compiler require additional runtime support that is stored in the \file{ob\-amd16\-run} library file.
\flowgraph{\resource{Oberon\\source code} \ar[r] & \toolbox{obamd16} \ar[r] \ar@/l/[d] \ar[rd] & \resource{object file} \\ \variable{ECSIMPORT} \ar[ru] & \resource{symbol\\files} \ar@/r/[u] & \resource{debugging\\information}}
\seeoberon\seeassembly\seeamd\seeobject\seedebugging
}

\providecommand{\obamdb}{
\toolsection{obamd32} is a compiler for the Oberon programming language targeting the AMD64 hardware architecture.
It generates machine code for AMD64 processors from modules written in Oberon and stores it in corresponding object files.
The compiler generates machine code for the 32-bit operating mode defined by the AMD64 architecture.
For debugging purposes, it also creates a debugging information file as well as an assembly file containing a listing of the generated machine code.
In addition, it stores the interface of each module in a symbol file which is required when other modules import the module.
Programs generated with this compiler require additional runtime support that is stored in the \file{ob\-amd32\-run} library file.
\flowgraph{\resource{Oberon\\source code} \ar[r] & \toolbox{obamd32} \ar[r] \ar@/l/[d] \ar[rd] & \resource{object file} \\ \variable{ECSIMPORT} \ar[ru] & \resource{symbol\\files} \ar@/r/[u] & \resource{debugging\\information}}
\seeoberon\seeassembly\seeamd\seeobject\seedebugging
}

\providecommand{\obamdc}{
\toolsection{obamd64} is a compiler for the Oberon programming language targeting the AMD64 hardware architecture.
It generates machine code for AMD64 processors from modules written in Oberon and stores it in corresponding object files.
The compiler generates machine code for the 64-bit operating mode defined by the AMD64 architecture.
For debugging purposes, it also creates a debugging information file as well as an assembly file containing a listing of the generated machine code.
In addition, it stores the interface of each module in a symbol file which is required when other modules import the module.
Programs generated with this compiler require additional runtime support that is stored in the \file{ob\-amd64\-run} library file.
\flowgraph{\resource{Oberon\\source code} \ar[r] & \toolbox{obamd64} \ar[r] \ar@/l/[d] \ar[rd] & \resource{object file} \\ \variable{ECSIMPORT} \ar[ru] & \resource{symbol\\files} \ar@/r/[u] & \resource{debugging\\information}}
\seeoberon\seeassembly\seeamd\seeobject\seedebugging
}

\providecommand{\obarma}{
\toolsection{obarma32} is a compiler for the Oberon programming language targeting the ARM hardware architecture.
It generates machine code for ARM processors executing A32 instructions from modules written in Oberon and stores it in corresponding object files.
For debugging purposes, it also creates a debugging information file as well as an assembly file containing a listing of the generated machine code.
In addition, it stores the interface of each module in a symbol file which is required when other modules import the module.
Programs generated with this compiler require additional runtime support that is stored in the \file{ob\-arma32\-run} library file.
\flowgraph{\resource{Oberon\\source code} \ar[r] & \toolbox{obarma32} \ar[r] \ar@/l/[d] \ar[rd] & \resource{object file} \\ \variable{ECSIMPORT} \ar[ru] & \resource{symbol\\files} \ar@/r/[u] & \resource{debugging\\information}}
\seeoberon\seeassembly\seearm\seeobject\seedebugging
}

\providecommand{\obarmb}{
\toolsection{obarma64} is a compiler for the Oberon programming language targeting the ARM hardware architecture.
It generates machine code for ARM processors executing A64 instructions from modules written in Oberon and stores it in corresponding object files.
For debugging purposes, it also creates a debugging information file as well as an assembly file containing a listing of the generated machine code.
In addition, it stores the interface of each module in a symbol file which is required when other modules import the module.
Programs generated with this compiler require additional runtime support that is stored in the \file{ob\-arma64\-run} library file.
\flowgraph{\resource{Oberon\\source code} \ar[r] & \toolbox{obarma64} \ar[r] \ar@/l/[d] \ar[rd] & \resource{object file} \\ \variable{ECSIMPORT} \ar[ru] & \resource{symbol\\files} \ar@/r/[u] & \resource{debugging\\information}}
\seeoberon\seeassembly\seearm\seeobject\seedebugging
}

\providecommand{\obarmc}{
\toolsection{obarmt32} is a compiler for the Oberon programming language targeting the ARM hardware architecture.
It generates machine code for ARM processors without floating-point extension executing T32 instructions from modules written in Oberon and stores it in corresponding object files.
For debugging purposes, it also creates a debugging information file as well as an assembly file containing a listing of the generated machine code.
In addition, it stores the interface of each module in a symbol file which is required when other modules import the module.
Programs generated with this compiler require additional runtime support that is stored in the \file{ob\-armt32\-run} library file.
\flowgraph{\resource{Oberon\\source code} \ar[r] & \toolbox{obarmt32} \ar[r] \ar@/l/[d] \ar[rd] & \resource{object file} \\ \variable{ECSIMPORT} \ar[ru] & \resource{symbol\\files} \ar@/r/[u] & \resource{debugging\\information}}
\seeoberon\seeassembly\seearm\seeobject\seedebugging
}

\providecommand{\obarmcfpe}{
\toolsection{obarmt32fpe} is a compiler for the Oberon programming language targeting the ARM hardware architecture.
It generates machine code for ARM processors with floating-point extension executing T32 instructions from modules written in Oberon and stores it in corresponding object files.
For debugging purposes, it also creates a debugging information file as well as an assembly file containing a listing of the generated machine code.
In addition, it stores the interface of each module in a symbol file which is required when other modules import the module.
Programs generated with this compiler require additional runtime support that is stored in the \file{ob\-armt32\-fpe\-run} library file.
\flowgraph{\resource{Oberon\\source code} \ar[r] & \toolbox{obarmt32fpe} \ar[r] \ar@/l/[d] \ar[rd] & \resource{object file} \\ \variable{ECSIMPORT} \ar[ru] & \resource{symbol\\files} \ar@/r/[u] & \resource{debugging\\information}}
\seeoberon\seeassembly\seearm\seeobject\seedebugging
}

\providecommand{\obavr}{
\toolsection{obavr} is a compiler for the Oberon programming language targeting the AVR hardware architecture.
It generates machine code for AVR processors from modules written in Oberon and stores it in corresponding object files.
For debugging purposes, it also creates a debugging information file as well as an assembly file containing a listing of the generated machine code.
In addition, it stores the interface of each module in a symbol file which is required when other modules import the module.
Programs generated with this compiler require additional runtime support that is stored in the \file{ob\-avr\-run} library file.
\flowgraph{\resource{Oberon\\source code} \ar[r] & \toolbox{obavr} \ar[r] \ar@/l/[d] \ar[rd] & \resource{object file} \\ \variable{ECSIMPORT} \ar[ru] & \resource{symbol\\files} \ar@/r/[u] & \resource{debugging\\information}}
\seeoberon\seeassembly\seeavr\seeobject\seedebugging
}

\providecommand{\obavrtt}{
\toolsection{obavr32} is a compiler for the Oberon programming language targeting the AVR32 hardware architecture.
It generates machine code for AVR32 processors from modules written in Oberon and stores it in corresponding object files.
For debugging purposes, it also creates a debugging information file as well as an assembly file containing a listing of the generated machine code.
In addition, it stores the interface of each module in a symbol file which is required when other modules import the module.
Programs generated with this compiler require additional runtime support that is stored in the \file{ob\-avr32\-run} library file.
\flowgraph{\resource{Oberon\\source code} \ar[r] & \toolbox{obavr32} \ar[r] \ar@/l/[d] \ar[rd] & \resource{object file} \\ \variable{ECSIMPORT} \ar[ru] & \resource{symbol\\files} \ar@/r/[u] & \resource{debugging\\information}}
\seeoberon\seeassembly\seeavrtt\seeobject\seedebugging
}

\providecommand{\obmabk}{
\toolsection{obm68k} is a compiler for the Oberon programming language targeting the M68000 hardware architecture.
It generates machine code for M68000 processors from modules written in Oberon and stores it in corresponding object files.
For debugging purposes, it also creates a debugging information file as well as an assembly file containing a listing of the generated machine code.
In addition, it stores the interface of each module in a symbol file which is required when other modules import the module.
Programs generated with this compiler require additional runtime support that is stored in the \file{ob\-m68k\-run} library file.
\flowgraph{\resource{Oberon\\source code} \ar[r] & \toolbox{obm68k} \ar[r] \ar@/l/[d] \ar[rd] & \resource{object file} \\ \variable{ECSIMPORT} \ar[ru] & \resource{symbol\\files} \ar@/r/[u] & \resource{debugging\\information}}
\seeoberon\seeassembly\seemabk\seeobject\seedebugging
}

\providecommand{\obmibl}{
\toolsection{obmibl} is a compiler for the Oberon programming language targeting the MicroBlaze hardware architecture.
It generates machine code for MicroBlaze processors from modules written in Oberon and stores it in corresponding object files.
For debugging purposes, it also creates a debugging information file as well as an assembly file containing a listing of the generated machine code.
In addition, it stores the interface of each module in a symbol file which is required when other modules import the module.
Programs generated with this compiler require additional runtime support that is stored in the \file{ob\-mibl\-run} library file.
\flowgraph{\resource{Oberon\\source code} \ar[r] & \toolbox{obmibl} \ar[r] \ar@/l/[d] \ar[rd] & \resource{object file} \\ \variable{ECSIMPORT} \ar[ru] & \resource{symbol\\files} \ar@/r/[u] & \resource{debugging\\information}}
\seeoberon\seeassembly\seemibl\seeobject\seedebugging
}

\providecommand{\obmipsa}{
\toolsection{obmips32} is a compiler for the Oberon programming language targeting the MIPS32 hardware architecture.
It generates machine code for MIPS32 processors from modules written in Oberon and stores it in corresponding object files.
For debugging purposes, it also creates a debugging information file as well as an assembly file containing a listing of the generated machine code.
In addition, it stores the interface of each module in a symbol file which is required when other modules import the module.
Programs generated with this compiler require additional runtime support that is stored in the \file{ob\-mips32\-run} library file.
\flowgraph{\resource{Oberon\\source code} \ar[r] & \toolbox{obmips32} \ar[r] \ar@/l/[d] \ar[rd] & \resource{object file} \\ \variable{ECSIMPORT} \ar[ru] & \resource{symbol\\files} \ar@/r/[u] & \resource{debugging\\information}}
\seeoberon\seeassembly\seemips\seeobject\seedebugging
}

\providecommand{\obmipsb}{
\toolsection{obmips64} is a compiler for the Oberon programming language targeting the MIPS64 hardware architecture.
It generates machine code for MIPS64 processors from modules written in Oberon and stores it in corresponding object files.
For debugging purposes, it also creates a debugging information file as well as an assembly file containing a listing of the generated machine code.
In addition, it stores the interface of each module in a symbol file which is required when other modules import the module.
Programs generated with this compiler require additional runtime support that is stored in the \file{ob\-mips64\-run} library file.
\flowgraph{\resource{Oberon\\source code} \ar[r] & \toolbox{obmips64} \ar[r] \ar@/l/[d] \ar[rd] & \resource{object file} \\ \variable{ECSIMPORT} \ar[ru] & \resource{symbol\\files} \ar@/r/[u] & \resource{debugging\\information}}
\seeoberon\seeassembly\seemips\seeobject\seedebugging
}

\providecommand{\obmmix}{
\toolsection{obmmix} is a compiler for the Oberon programming language targeting the MMIX hardware architecture.
It generates machine code for MMIX processors from modules written in Oberon and stores it in corresponding object files.
For debugging purposes, it also creates a debugging information file as well as an assembly file containing a listing of the generated machine code.
In addition, it stores the interface of each module in a symbol file which is required when other modules import the module.
Programs generated with this compiler require additional runtime support that is stored in the \file{ob\-mmix\-run} library file.
\flowgraph{\resource{Oberon\\source code} \ar[r] & \toolbox{obmmix} \ar[r] \ar@/l/[d] \ar[rd] & \resource{object file} \\ \variable{ECSIMPORT} \ar[ru] & \resource{symbol\\files} \ar@/r/[u] & \resource{debugging\\information}}
\seeoberon\seeassembly\seemmix\seeobject\seedebugging
}

\providecommand{\oborok}{
\toolsection{obor1k} is a compiler for the Oberon programming language targeting the OpenRISC 1000 hardware architecture.
It generates machine code for OpenRISC 1000 processors from modules written in Oberon and stores it in corresponding object files.
For debugging purposes, it also creates a debugging information file as well as an assembly file containing a listing of the generated machine code.
In addition, it stores the interface of each module in a symbol file which is required when other modules import the module.
Programs generated with this compiler require additional runtime support that is stored in the \file{ob\-or1k\-run} library file.
\flowgraph{\resource{Oberon\\source code} \ar[r] & \toolbox{obor1k} \ar[r] \ar@/l/[d] \ar[rd] & \resource{object file} \\ \variable{ECSIMPORT} \ar[ru] & \resource{symbol\\files} \ar@/r/[u] & \resource{debugging\\information}}
\seeoberon\seeassembly\seeorok\seeobject\seedebugging
}

\providecommand{\obppca}{
\toolsection{obppc32} is a compiler for the Oberon programming language targeting the PowerPC hardware architecture.
It generates machine code for PowerPC processors from modules written in Oberon and stores it in corresponding object files.
The compiler generates machine code for the 32-bit operating mode defined by the PowerPC architecture.
For debugging purposes, it also creates a debugging information file as well as an assembly file containing a listing of the generated machine code.
In addition, it stores the interface of each module in a symbol file which is required when other modules import the module.
Programs generated with this compiler require additional runtime support that is stored in the \file{ob\-ppc32\-run} library file.
\flowgraph{\resource{Oberon\\source code} \ar[r] & \toolbox{obppc32} \ar[r] \ar@/l/[d] \ar[rd] & \resource{object file} \\ \variable{ECSIMPORT} \ar[ru] & \resource{symbol\\files} \ar@/r/[u] & \resource{debugging\\information}}
\seeoberon\seeassembly\seeppc\seeobject\seedebugging
}

\providecommand{\obppcb}{
\toolsection{obppc64} is a compiler for the Oberon programming language targeting the PowerPC hardware architecture.
It generates machine code for PowerPC processors from modules written in Oberon and stores it in corresponding object files.
The compiler generates machine code for the 64-bit operating mode defined by the PowerPC architecture.
For debugging purposes, it also creates a debugging information file as well as an assembly file containing a listing of the generated machine code.
In addition, it stores the interface of each module in a symbol file which is required when other modules import the module.
Programs generated with this compiler require additional runtime support that is stored in the \file{ob\-ppc64\-run} library file.
\flowgraph{\resource{Oberon\\source code} \ar[r] & \toolbox{obppc64} \ar[r] \ar@/l/[d] \ar[rd] & \resource{object file} \\ \variable{ECSIMPORT} \ar[ru] & \resource{symbol\\files} \ar@/r/[u] & \resource{debugging\\information}}
\seeoberon\seeassembly\seeppc\seeobject\seedebugging
}

\providecommand{\obrisc}{
\toolsection{obrisc} is a compiler for the Oberon programming language targeting the RISC hardware architecture.
It generates machine code for RISC processors from modules written in Oberon and stores it in corresponding object files.
For debugging purposes, it also creates a debugging information file as well as an assembly file containing a listing of the generated machine code.
In addition, it stores the interface of each module in a symbol file which is required when other modules import the module.
Programs generated with this compiler require additional runtime support that is stored in the \file{ob\-risc\-run} library file.
\flowgraph{\resource{Oberon\\source code} \ar[r] & \toolbox{obrisc} \ar[r] \ar@/l/[d] \ar[rd] & \resource{object file} \\ \variable{ECSIMPORT} \ar[ru] & \resource{symbol\\files} \ar@/r/[u] & \resource{debugging\\information}}
\seeoberon\seeassembly\seerisc\seeobject\seedebugging
}

\providecommand{\obwasm}{
\toolsection{obwasm} is a compiler for the Oberon programming language targeting the WebAssembly architecture.
It generates machine code for WebAssembly targets from modules written in Oberon and stores it in corresponding object files.
For debugging purposes, it also creates a debugging information file as well as an assembly file containing a listing of the generated machine code.
In addition, it stores the interface of each module in a symbol file which is required when other modules import the module.
Programs generated with this compiler require additional runtime support that is stored in the \file{ob\-wasm\-run} library file.
\flowgraph{\resource{Oberon\\source code} \ar[r] & \toolbox{obwasm} \ar[r] \ar@/l/[d] \ar[rd] & \resource{object file} \\ \variable{ECSIMPORT} \ar[ru] & \resource{symbol\\files} \ar@/r/[u] & \resource{debugging\\information}}
\seeoberon\seeassembly\seewasm\seeobject\seedebugging
}

% converter tools

\providecommand{\dbgdwarf}{
\toolsection{dbgdwarf} is a DWARF debugging information converter tool.
It converts debugging information into the DWARF debugging data format and stores it in corresponding object files~\cite{dwarffile}.
The resulting debugging object files can be combined with runtime support that creates Executable and Linking Format (ELF) files~\cite{elffile}.
\flowgraph{\resource{debugging\\information} \ar[r] & \toolbox{dbgdwarf} \ar[r] & \resource{debugging\\object file}}
\seeobject\seedebugging
}

% assembler tools

\providecommand{\asmprint}{
\toolsection{asmprint} is a pretty printer for generic assembly code.
It reformats generic assembly code and writes it to the standard output stream.
\flowgraph{\resource{generic assembly\\source code} \ar[r] & \toolbox{asmprint} \ar[r] & \resource{reformatted\\source code}}
\seeassembly
}

\providecommand{\amdaasm}{
\toolsection{amd16asm} is an assembler for the AMD64 hardware architecture.
It translates assembly code into machine code for AMD64 processors and stores it in corresponding object files.
By default, the assembler generates machine code for the 16-bit operating mode defined by the AMD64 architecture.
\flowgraph{\resource{AMD16 assembly\\source code} \ar[r] & \toolbox{amd16asm} \ar[r] & \resource{object file}}
\seeassembly\seeamd\seeobject
}

\providecommand{\amdadism}{
\toolsection{amd16dism} is a disassembler for the AMD64 hardware architecture.
It translates machine code from object files targeting AMD64 processors into assembly code and writes it to the standard output stream.
It assumes that the machine code was generated for the 16-bit operating mode defined by the AMD64 architecture.
\flowgraph{\resource{object file} \ar[r] & \toolbox{amd16dism} \ar[r] & \resource{disassembly\\listing}}
\seeassembly\seeamd\seeobject
}

\providecommand{\amdbasm}{
\toolsection{amd32asm} is an assembler for the AMD64 hardware architecture.
It translates assembly code into machine code for AMD64 processors and stores it in corresponding object files.
By default, the assembler generates machine code for the 32-bit operating mode defined by the AMD64 architecture.
\flowgraph{\resource{AMD32 assembly\\source code} \ar[r] & \toolbox{amd32asm} \ar[r] & \resource{object file}}
\seeassembly\seeamd\seeobject
}

\providecommand{\amdbdism}{
\toolsection{amd32dism} is a disassembler for the AMD64 hardware architecture.
It translates machine code from object files targeting AMD64 processors into assembly code and writes it to the standard output stream.
It assumes that the machine code was generated for the 32-bit operating mode defined by the AMD64 architecture.
\flowgraph{\resource{object file} \ar[r] & \toolbox{amd32dism} \ar[r] & \resource{disassembly\\listing}}
\seeassembly\seeamd\seeobject
}

\providecommand{\amdcasm}{
\toolsection{amd64asm} is an assembler for the AMD64 hardware architecture.
It translates assembly code into machine code for AMD64 processors and stores it in corresponding object files.
By default, the assembler generates machine code for the 64-bit operating mode defined by the AMD64 architecture.
\flowgraph{\resource{AMD64 assembly\\source code} \ar[r] & \toolbox{amd64asm} \ar[r] & \resource{object file}}
\seeassembly\seeamd\seeobject
}

\providecommand{\amdcdism}{
\toolsection{amd64dism} is a disassembler for the AMD64 hardware architecture.
It translates machine code from object files targeting AMD64 processors into assembly code and writes it to the standard output stream.
It assumes that the machine code was generated for the 64-bit operating mode defined by the AMD64 architecture.
\flowgraph{\resource{object file} \ar[r] & \toolbox{amd64dism} \ar[r] & \resource{disassembly\\listing}}
\seeassembly\seeamd\seeobject
}

\providecommand{\armaasm}{
\toolsection{arma32asm} is an assembler for the ARM hardware architecture.
It translates assembly code into machine code for ARM processors executing A32 instructions and stores it in corresponding object files.
\flowgraph{\resource{ARM A32 assembly\\source code} \ar[r] & \toolbox{arma32asm} \ar[r] & \resource{object file}}
\seeassembly\seearm\seeobject
}

\providecommand{\armadism}{
\toolsection{arma32dism} is a disassembler for the ARM hardware architecture.
It translates machine code from object files targeting ARM processors executing A32 instructions into assembly code and writes it to the standard output stream.
\flowgraph{\resource{object file} \ar[r] & \toolbox{arma32dism} \ar[r] & \resource{disassembly\\listing}}
\seeassembly\seearm\seeobject
}

\providecommand{\armbasm}{
\toolsection{arma64asm} is an assembler for the ARM hardware architecture.
It translates assembly code into machine code for ARM processors executing A64 instructions and stores it in corresponding object files.
\flowgraph{\resource{ARM A64 assembly\\source code} \ar[r] & \toolbox{arma64asm} \ar[r] & \resource{object file}}
\seeassembly\seearm\seeobject
}

\providecommand{\armbdism}{
\toolsection{arma64dism} is a disassembler for the ARM hardware architecture.
It translates machine code from object files targeting ARM processors executing A64 instructions into assembly code and writes it to the standard output stream.
\flowgraph{\resource{object file} \ar[r] & \toolbox{arma64dism} \ar[r] & \resource{disassembly\\listing}}
\seeassembly\seearm\seeobject
}

\providecommand{\armcasm}{
\toolsection{armt32asm} is an assembler for the ARM hardware architecture.
It translates assembly code into machine code for ARM processors executing T32 instructions and stores it in corresponding object files.
\flowgraph{\resource{ARM T32 assembly\\source code} \ar[r] & \toolbox{armt32asm} \ar[r] & \resource{object file}}
\seeassembly\seearm\seeobject
}

\providecommand{\armcdism}{
\toolsection{armt32dism} is a disassembler for the ARM hardware architecture.
It translates machine code from object files targeting ARM processors executing T32 instructions into assembly code and writes it to the standard output stream.
\flowgraph{\resource{object file} \ar[r] & \toolbox{armt32dism} \ar[r] & \resource{disassembly\\listing}}
\seeassembly\seearm\seeobject
}

\providecommand{\avrasm}{
\toolsection{avrasm} is an assembler for the AVR hardware architecture.
It translates assembly code into machine code for AVR processors and stores it in corresponding object files.
The identifiers \texttt{RXL}, \texttt{RXH}, \texttt{RYL}, \texttt{RYH}, \texttt{RZL}, and \texttt{RZH} are predefined and name the corresponding registers.
The identifiers \texttt{SPL} and \texttt{SPH} are also predefined and evaluate to the address of the corresponding registers.
\flowgraph{\resource{AVR assembly\\source code} \ar[r] & \toolbox{avrasm} \ar[r] & \resource{object file}}
\seeassembly\seeavr\seeobject
}

\providecommand{\avrdism}{
\toolsection{avrdism} is a disassembler for the AVR hardware architecture.
It translates machine code from object files targeting AVR processors into assembly code and writes it to the standard output stream.
\flowgraph{\resource{object file} \ar[r] & \toolbox{avrdism} \ar[r] & \resource{disassembly\\listing}}
\seeassembly\seeavr\seeobject
}

\providecommand{\avrttasm}{
\toolsection{avr32asm} is an assembler for the AVR32 hardware architecture.
It translates assembly code into machine code for AVR32 processors and stores it in corresponding object files.
\flowgraph{\resource{AVR32 assembly\\source code} \ar[r] & \toolbox{avr32asm} \ar[r] & \resource{object file}}
\seeassembly\seeavrtt\seeobject
}

\providecommand{\avrttdism}{
\toolsection{avr32dism} is a disassembler for the AVR32 hardware architecture.
It translates machine code from object files targeting AVR32 processors into assembly code and writes it to the standard output stream.
\flowgraph{\resource{object file} \ar[r] & \toolbox{avr32dism} \ar[r] & \resource{disassembly\\listing}}
\seeassembly\seeavrtt\seeobject
}

\providecommand{\mabkasm}{
\toolsection{m68kasm} is an assembler for the M68000 hardware architecture.
It translates assembly code into machine code for M68000 processors and stores it in corresponding object files.
\flowgraph{\resource{68000 assembly\\source code} \ar[r] & \toolbox{m68kasm} \ar[r] & \resource{object file}}
\seeassembly\seemabk\seeobject
}

\providecommand{\mabkdism}{
\toolsection{m68kdism} is a disassembler for the M68000 hardware architecture.
It translates machine code from object files targeting M68000 processors into assembly code and writes it to the standard output stream.
\flowgraph{\resource{object file} \ar[r] & \toolbox{m68kdism} \ar[r] & \resource{disassembly\\listing}}
\seeassembly\seemabk\seeobject
}

\providecommand{\miblasm}{
\toolsection{miblasm} is an assembler for the MicroBlaze hardware architecture.
It translates assembly code into machine code for MicroBlaze processors and stores it in corresponding object files.
\flowgraph{\resource{MicroBlaze assembly\\source code} \ar[r] & \toolbox{miblasm} \ar[r] & \resource{object file}}
\seeassembly\seemibl\seeobject
}

\providecommand{\mibldism}{
\toolsection{mibldism} is a disassembler for the MicroBlaze hardware architecture.
It translates machine code from object files targeting MicroBlaze processors into assembly code and writes it to the standard output stream.
\flowgraph{\resource{object file} \ar[r] & \toolbox{mibldism} \ar[r] & \resource{disassembly\\listing}}
\seeassembly\seemibl\seeobject
}

\providecommand{\mipsaasm}{
\toolsection{mips32asm} is an assembler for the MIPS32 hardware architecture.
It translates assembly code into machine code for MIPS32 processors and stores it in corresponding object files.
\flowgraph{\resource{MIPS32 assembly\\source code} \ar[r] & \toolbox{mips32asm} \ar[r] & \resource{object file}}
\seeassembly\seemips\seeobject
}

\providecommand{\mipsadism}{
\toolsection{mips32dism} is a disassembler for the MIPS32 hardware architecture.
It translates machine code from object files targeting MIPS32 processors into assembly code and writes it to the standard output stream.
\flowgraph{\resource{object file} \ar[r] & \toolbox{mips32dism} \ar[r] & \resource{disassembly\\listing}}
\seeassembly\seemips\seeobject
}

\providecommand{\mipsbasm}{
\toolsection{mips64asm} is an assembler for the MIPS64 hardware architecture.
It translates assembly code into machine code for MIPS64 processors and stores it in corresponding object files.
\flowgraph{\resource{MIPS64 assembly\\source code} \ar[r] & \toolbox{mips64asm} \ar[r] & \resource{object file}}
\seeassembly\seemips\seeobject
}

\providecommand{\mipsbdism}{
\toolsection{mips64dism} is a disassembler for the MIPS64 hardware architecture.
It translates machine code from object files targeting MIPS64 processors into assembly code and writes it to the standard output stream.
\flowgraph{\resource{object file} \ar[r] & \toolbox{mips64dism} \ar[r] & \resource{disassembly\\listing}}
\seeassembly\seemips\seeobject
}

\providecommand{\mmixasm}{
\toolsection{mmixasm} is an assembler for the MMIX hardware architecture.
It translates assembly code into machine code for MMIX processors and stores it in corresponding object files.
The names of all special registers are predefined and evaluate to the corresponding number.
\flowgraph{\resource{MMIX assembly\\source code} \ar[r] & \toolbox{mmixasm} \ar[r] & \resource{object file}}
\seeassembly\seemmix\seeobject
}

\providecommand{\mmixdism}{
\toolsection{mmixdism} is a disassembler for the MMIX hardware architecture.
It translates machine code from object files targeting MMIX processors into assembly code and writes it to the standard output stream.
\flowgraph{\resource{object file} \ar[r] & \toolbox{mmixdism} \ar[r] & \resource{disassembly\\listing}}
\seeassembly\seemmix\seeobject
}

\providecommand{\orokasm}{
\toolsection{or1kasm} is an assembler for the OpenRISC 1000 hardware architecture.
It translates assembly code into machine code for OpenRISC 1000 processors and stores it in corresponding object files.
\flowgraph{\resource{OpenRISC 1000 assembly\\source code} \ar[r] & \toolbox{or1kasm} \ar[r] & \resource{object file}}
\seeassembly\seeorok\seeobject
}

\providecommand{\orokdism}{
\toolsection{or1kdism} is a disassembler for the OpenRISC 1000 hardware architecture.
It translates machine code from object files targeting OpenRISC 1000 processors into assembly code and writes it to the standard output stream.
\flowgraph{\resource{object file} \ar[r] & \toolbox{or1kdism} \ar[r] & \resource{disassembly\\listing}}
\seeassembly\seeorok\seeobject
}

\providecommand{\ppcaasm}{
\toolsection{ppc32asm} is an assembler for the PowerPC hardware architecture.
It translates assembly code into machine code for PowerPC processors and stores it in corresponding object files.
By default, the assembler generates machine code for the 32-bit operating mode defined by the PowerPC architecture.
\flowgraph{\resource{PowerPC assembly\\source code} \ar[r] & \toolbox{ppc32asm} \ar[r] & \resource{object file}}
\seeassembly\seeppc\seeobject
}

\providecommand{\ppcadism}{
\toolsection{ppc32dism} is a disassembler for the PowerPC hardware architecture.
It translates machine code from object files targeting PowerPC processors into assembly code and writes it to the standard output stream.
It assumes that the machine code was generated for the 32-bit operating mode defined by the PowerPC architecture.
\flowgraph{\resource{object file} \ar[r] & \toolbox{ppc32dism} \ar[r] & \resource{disassembly\\listing}}
\seeassembly\seeppc\seeobject
}

\providecommand{\ppcbasm}{
\toolsection{ppc64asm} is an assembler for the PowerPC hardware architecture.
It translates assembly code into machine code for PowerPC processors and stores it in corresponding object files.
By default, the assembler generates machine code for the 64-bit operating mode defined by the PowerPC architecture.
\flowgraph{\resource{PowerPC assembly\\source code} \ar[r] & \toolbox{ppc64asm} \ar[r] & \resource{object file}}
\seeassembly\seeppc\seeobject
}

\providecommand{\ppcbdism}{
\toolsection{ppc64dism} is a disassembler for the PowerPC hardware architecture.
It translates machine code from object files targeting PowerPC processors into assembly code and writes it to the standard output stream.
It assumes that the machine code was generated for the 64-bit operating mode defined by the PowerPC architecture.
\flowgraph{\resource{object file} \ar[r] & \toolbox{ppc64dism} \ar[r] & \resource{disassembly\\listing}}
\seeassembly\seeppc\seeobject
}

\providecommand{\riscasm}{
\toolsection{riscasm} is an assembler for the RISC hardware architecture.
It translates assembly code into machine code for RISC processors and stores it in corresponding object files.
The names of all special registers are predefined and evaluate to the corresponding number.
\flowgraph{\resource{RISC assembly\\source code} \ar[r] & \toolbox{riscasm} \ar[r] & \resource{object file}}
\seeassembly\seerisc\seeobject
}

\providecommand{\riscdism}{
\toolsection{riscdism} is a disassembler for the RISC hardware architecture.
It translates machine code from object files targeting RISC processors into assembly code and writes it to the standard output stream.
\flowgraph{\resource{object file} \ar[r] & \toolbox{riscdism} \ar[r] & \resource{disassembly\\listing}}
\seeassembly\seerisc\seeobject
}

\providecommand{\wasmasm}{
\toolsection{wasmasm} is an assembler for the WebAssembly architecture.
It translates assembly code into machine code for WebAssembly targets and stores it in corresponding object files.
The names of all special registers are predefined and evaluate to the corresponding number.
\flowgraph{\resource{WebAssembly assembly\\source code} \ar[r] & \toolbox{wasmasm} \ar[r] & \resource{object file}}
\seeassembly\seewasm\seeobject
}

\providecommand{\wasmdism}{
\toolsection{wasmdism} is a disassembler for the WebAssembly architecture.
It translates machine code from object files targeting WebAssembly targets into assembly code and writes it to the standard output stream.
\flowgraph{\resource{object file} \ar[r] & \toolbox{wasmdism} \ar[r] & \resource{disassembly\\listing}}
\seeassembly\seewasm\seeobject
}

% linker tools

\providecommand{\linklib}{
\toolsection{linklib} is an object file combiner.
It creates a static library file by combining all object files given to it into a single one.
\flowgraph{\resource{object files} \ar[r] & \toolbox{linklib} \ar[r] & \resource{library file}}
\seeobject
}

\providecommand{\linkbin}{
\toolsection{linkbin} is a linker for plain binary files.
It links all object files given to it into a single image and stores it in a binary file that begins with the first linked section.
It also creates a map file that lists the address, type, name and size of all used sections.
The filename extension of the resulting binary file can be specified by putting it into a constant data section called \texttt{\_extension}.
\flowgraph{\resource{object files} \ar[r] & \toolbox{linkbin} \ar[r] \ar[d] & \resource{binary file} \\ & \resource{map file}}
\seeobject
}

\providecommand{\linkmem}{
\toolsection{linkmem} is a linker for plain binary files partitioned into random-access and read-only memory.
It links all object files given to it into two distinct images, one for data sections and one for code and constant data sections, and stores each image in a binary file that begins with the first linked section of the corresponding type.
It also creates a map file that lists the address, type, name and size of all used sections.
\flowgraph{\resource{object files} \ar[r] & \toolbox{linkmem} \ar[r] \ar[d] & \resource{RAM file/\\ROM file} \\ & \resource{map file}}
\seeobject
}

\providecommand{\linkprg}{
\toolsection{linkprg} is a linker for GEMDOS executable files.
It links all object files given to it into a single image and stores the image in an Atari GEMDOS executable file~\cite{gemdosfile}.
It also creates a map file that lists the address relative to the text segment, type, name and size of all used sections.
The filename extension of the resulting executable file can be specified by putting it into a constant data section called \texttt{\_extension}.
The GEMDOS executable file format requires all patch patterns of absolute link patches to consist of four full bitmasks with descending offsets.
\flowgraph{\resource{object files} \ar[r] & \toolbox{linkprg} \ar[r] \ar[d] & \resource{executable file} \\ & \resource{map file}}
\seeobject
}

\providecommand{\linkhex}{
\toolsection{linkhex} is a linker for Intel HEX files.
It links all code sections of the object files given to it into single image and stores the image in an Intel HEX file~\cite{hexfile} that begins with the first linked section.
It also creates a map file that lists the address, type, name and size of all used sections.
\flowgraph{\resource{object files} \ar[r] & \toolbox{linkhex} \ar[r] \ar[d] & \resource{HEX file} \\ & \resource{map file}}
\seeobject
}

\providecommand{\mapsearch}{
\toolsection{mapsearch} is a debugging tool.
It searches map files generated by linker tools for the name of a binary section that encompasses a memory address read from the standard input stream.
If additionally provided with one or more object files, it also stores an excerpt thereof in a separate object file called map search result which only contains the identified binary section for disassembling purposes.
\flowgraph{& \resource{map files/\\object files} \ar[d] \\ \resource{memory\\address} \ar[r] & \toolbox{mapsearch} \ar[r] \ar[d] & \resource{section name/\\relative offset} \\ & \resource{object file\\excerpt}}
\seeobject
}


\startchapter{User Interface}{User Interface}{interface}
{The \ecs{} features a variety of different tools like compilers, assemblers, and linkers.
This \documentation{} describes their input and output and explains the common user interface that is shared across all of these tools.}

\epigraph{The trivial round, the common task, \\ would furnish all we ought to ask.}{John Keble}

\section{Introduction}

The \ecs{} is a toolchain that consists of several different programming tools targeting a variety of programming languages, hardware architectures, and runtime environments.
The following list categorizes all tools provided by the \ecs{} and visualizes their input and output:

\begin{itemize}

\item Preprocessors\index{Preprocessors}\nopagebreak

Preprocessors are tools that perform simple lexical substitutions like macro expansions and conditional inclusion on source code written in a programming language.
This very first phase of translating the source code is also part of all subsequent tools if the corresponding programming language supports preprocessing.
An example of a preprocessor tool is \tool{cpp\-prep}.

\flowgraph{\resource{source code} \ar[r] & \converter{Preprocessor} \ar[r] & \resource{preprocessed\\source code}}

\item Pretty Printers\index{Pretty printers}\nopagebreak

Pretty printers are tools that reformat source code written in a programming language by printing it again using a consistent layout.
Examples of pretty printer tools are \tool{asm\-print} and \tool{fal\-print}.

\flowgraph{\resource{source code} \ar[r] & \converter{Pretty Printer} \ar[r] & \resource{reformatted\\source code}}

\item Semantic Checkers\index{Semantic checkers}\nopagebreak

Semantic checkers perform syntactic and semantic checks on source code written in a programming language and print diagnostic messages.
Some checkers may have to process additional semantic information stored in separate files.
This phase of translating the source code is also part of all subsequent tools.
Examples of semantic checker tools are \tool{cpp\-check} and \tool{cd\-check}.

\flowgraph{\resource{source code} \ar[r] & \converter{Semantic\\Checker} \ar@/l/[d] \ar[r] & \resource{diagnostic\\messages} \\ & \resource{external\\information} \ar@/r/[u]}

\item Serializers\index{Serializers}\nopagebreak

Serializers are debugging tools that dump the complete internal representation of a program in a human readable format.
This potentially also allows external development tools to make use of the same internal program representation.
Examples of serialization tools are \tool{ob\-dump} and \tool{fal\-dump}.

\flowgraph{\resource{source code} \ar[r] & \converter{Serializer} \ar[r] & \resource{internal\\representation}}

\item Interpreters\index{Interpreters}\nopagebreak

Interpreters read and process source code for a programming language by emulating a runtime environment for this language and executing the program accordingly.
The actual interpreted program defines the input and output the interpreter processes and produces.
Examples of interpreter tools are \tool{cpp\-run} and \tool{cd\-run}.

\flowgraph{\resource{source code} \ar[r] & \converter{Interpreter} \ar@/u/[r] & \resource{input/\\output} \ar@/d/[l]}

\item Transpilers\index{Transpilers}\nopagebreak

Transpilers translate source code written in one programming language into source code for another programming language.
They store the resulting code in corresponding source code files.
Examples of transpiler tools are \tool{fal\-cpp} and \tool{ob\-cpp}.

\flowgraph{\resource{source code} \ar[r] & \converter{Transpiler} \ar[r] & \resource{translated\\source code}}

\item Documentation Generators\index{Documentation generators}\nopagebreak

Documentation generators extract the structure and comments of the source code in order to generate documentations about the program.
They store the resulting documentation in document files of various formats. \seedocumentation
Examples of documentation generator tools are \tool{cpp\-html} and \tool{ob\-latex}.

\flowgraph{\resource{source code} \ar[r] & \converter{Documentation\\Generator} \ar[r] & \resource{formatted\\document}}

\item Compilers\index{Compilers}\nopagebreak

Compilers translate programs written in a programming language into machine code for some hardware processors.
They store the resulting binary representation of the source code in object files. \seeobject
In addition, they also create debugging information and assembly code listings of the generated machine code.
Examples of compiler tools are \tool{cpp\-amd64} and \tool{ob\-avr}.

\flowgraph{\resource{source code} \ar[r] & \converter{Compiler} \ar[r] \ar[d] \ar[rd] & \resource{object file} \\ & \resource{debugging\\information} & \resource{assembly\\listing}}

\item Debugging Information Converters\index{Debugging information converters}\nopagebreak

Debugging information converters process debugging information files and generate debugging object files storing a binary representation thereof in a format suitable for the debugger of a specific runtime environment. \seedebugging
An example of a debugging information converter tool is \tool{dbg\-dwarf}.

\flowgraph{\resource{debugging\\information} \ar[r] & \toolbox{Converter} \ar[r] & \resource{debugging\\object file}}

\item Assemblers\index{Assemblers}\nopagebreak

Assemblers translate assembly code for some specific hardware architecture into equivalent binary machine code.
They store the resulting binary representation of the source assembly code in object files. \seeassembly
Examples of assembler tools are \tool{arma32\-asm} and \tool{ppc\-asm}.

\flowgraph{\resource{assembly\\source code} \ar[r] & \converter{Assembler} \ar[r] & \resource{object file}}

\item Disassemblers\index{Disassemblers}\nopagebreak

Disassemblers translate machine code for some specific hardware architecture stored in object files by printing it using a human-readable disassembly listing.
Examples of disassembler tools are \tool{mibl\-dism} and \tool{arma64\-dism}.

\flowgraph{\resource{object file} \ar[r] & \converter{Disassembler} \ar[r] & \resource{disassembly\\listing}}

\item Linkers\index{Linkers}\nopagebreak

Linkers assemble the binary representation of a program stored in object files to generate output files that are executable on some target platforms.
Examples of linker tools are \tool{link\-bin} and \tool{link\-prg}.

\flowgraph{\resource{object files} \ar[r] & \converter{Linker} \ar[r] & \resource{executable\\binary image}}

\end{itemize}

Although all of these tools process and produce different kinds of input and output, all of them present the same command-line interface to the user which is described in the following sections.

\section{Command-Line Arguments}\index{Command-line arguments}

Each tool of the \ecs{} provides a command-line user interface and accepts zero, one or more command-line arguments.
If there are one or more command-line arguments, each of them is treated as the name of an input file relative to the current working directory.
The notion of command-line options is not supported in order to make the contents of output files independent from the tool invocation.

Input files are plain text files containing the source code or other contents to be processed.
Output files are generated in the current working directory and have the same name as the input file being processed whereas the filename extension gets replaced by an appropriate suffix.
If the user does not provide any command-line argument, the tool reads the source code from the standard input stream.
Beforehand, it prints its version and a short copyright notice.

If there are more than one command-line argument, each input file is processed in sequence according to the given order of the arguments.
If there are errors during the processing of an input file, subsequent command-line arguments are ignored.
Some tools like assemblers and disassemblers do not depend on other input except for the actual source code.
In this case, the order of the command-line arguments is irrelevant and it would be equivalent to execute the tool once for every input file.

However, there are tools that depend on the actual order of the command-line arguments.
They even may behave differently if executed several times with the same command-line argument.
This may especially be the case for compilers which may store additional semantic information in separate files for every input file they compile.
In addition, interpreters may also depend on the actual order of the input files they have processed so far.

On the other hand, it is crucial for some tools to process several input files during the same execution of the tool.
Particularly linkers have to collect the information stored in several object files to generate an executable binary image.
In this case, the order of the arguments is not important, since all information is collected first before the linking process is started.
However, linkers generate output files named after the very first argument with the filename extension being replaced by an appropriate suffix.

\section{Diagnostic Messages}\index{Diagnostic messages}\index{Messages|see{Diagnostic messages}}

All tools of the \ecs{} use the same scheme for the diagnostic messages they generate.
The following list describes the different types of messages used by the \ecs{}:

\begin{itemize}

\item Errors\index{Error messages}\nopagebreak

Error messages indicate a problem within the contents of an input file that prohibits the tool from proceeding successfully.
Compilers for example use error messages to output syntactic or semantic errors that have to be fixed by the programmer.
In general, tools diagnose only the first few of the encountered problems but always yield a return code indicating the failure.

\item Fatal Errors\index{Fatal error messages}\nopagebreak

Fatal errors indicate a general failure of the tool that cannot be fixed by changing the contents of the input file.
This includes for example internal program errors, failures to open files, or critical system conditions like out of memory.

\item Warnings\index{Warnings}\nopagebreak

Warning messages indicate potential flaws within the contents of an input file.
They do not cause the tool to fail but they identify issues that may lead to unexpected behavior.
Therefore, warnings may be prevented by changing the contents of an input file without changing its semantics.

\item Notes\index{Notes}\nopagebreak

Notes give additional information about the result of processing an input file.
They are often used for debugging purposes or to diagnose violations of coding conventions.

\end{itemize}

All diagnostic messages are written to the standard error stream.
Each message consists of a short line of text and the name of the input file or the tool that caused the diagnostic message.
The message also contains the position\index{Position}\index{Diagnostic messages!Position} within the input file, if the problem can be located.
Since input files are text files in general, the position is usually given as the number of the text line containing the problem followed by the column therein.
For convenience, some tools like compilers also indicate the problematic position by reproducing the respective line of text from the input file.

\section{The \ecs{} Driver}\index{Eigen Compiler Suite!Driver}\index{Driver, Eigen Compiler Suite}

The \ecs{} provides a utility tool called \tool{ecsd} which autonomously drives its toolchain in order to build complete executable files for a specific runtime environment.
It infers the set of required tools like compilers, assemblers, and linkers from the type of its input files and automatically invokes one tool after the other with appropriate command-line arguments and environment variables\index{Environment variables}.

\flowgraph{\resource{input\\files} \ar[d] \ar[rd]|\hole \ar[rrd]|!{[r];[d]}\hole|\hole \ar@{-->}[r] & \converter{ecsd\vphantom{Compilers}} \ar@{~}[ld] \ar@{~}[d] \ar@{~}[rd] \ar@{-->}[r] & \resource{executable\\file} \\
\converter{Compilers} \ar[d] \ar[rd] & \converter{Assemblers\vphantom{Compilers}} \ar[d] & \converter{Linkers\vphantom{Compilers}} \ar[u] \\
\resource{debugging\\information\vphantom{support}} & \resource{object\vphantom{debugging}\\files\vphantom{support}} \ar[ru] & \resource{runtime\vphantom{debugging}\\support} \ar[u]}

In contrast to all other tools of the \ecs{}, this utility tool does support command-line options which influence how tools are identified and invoked.
The following list describes the form and behavior of the most important options:

\begin{description}

\item\syntax{"-b" <directory> $\mid$ "-""-base" <directory>}\alignright Use Specified Base Directory\nopagebreak

Allows overriding the base directory provided by the \environmentvariable{ECSBASE} environment variable which is used to locate tools and the necessary runtime support.
If omitted, the tool infers the base directory from its own location.

\item\syntax{"-c" $\mid$ "-""-compile"}\alignright Compile and Assemble Only\nopagebreak

Compiles and assembles input files without invoking the linker at the end.
The output is a set of object files rather than an executable file.

\item\syntax{"-d" $\mid$ "-""-disassemble"}\alignright Disassemble Object Files\nopagebreak

Invokes the disassembler instead of the linker at the end.
The output is a set of disassembly listings rather than an executable file.

\item\syntax{"-g" $\mid$ "-""-generate"}\alignright Generate Debugging Information\nopagebreak

Generates and includes debugging information during linking.
The resulting executable file can be executed and tested using a symbolic debugger.

\item\syntax{"-h" $\mid$ "-""-help"}\alignright Print Command-Line Help\nopagebreak

Prints information about all supported command-line arguments.
This especially also includes all supported source types and available target environments.

\item\syntax{"-i" <tool> $\mid$ "-""-invoke" <tool>}\alignright Invoke Specified Tool Only\nopagebreak

Processes either input files or the standard input stream using the specified tool.
The output is whatever that tool generates rather than an executable file.
\ifbook See page~\pageref{idx:tools} for an index of all available tools. \fi

\item\syntax{"-l" $\mid$ "-""-library"}\alignright Create Library File\nopagebreak

Combines input files into a single object file.
The output is a library file rather than an executable file.

\item\syntax{"-o" <format> $\mid$ "-""-output" <format>}\alignright Output Specified File Format\nopagebreak

Specifies the output file format of the linker.
The available file formats correspond to the suffixes of the linker tools listed in \Documentation{}~\documentationref{object}{Object File Representation}.
If omitted, the tool uses the default file format of the target environment.

\item\syntax{"-p" $\mid$ "-""-protect"}\alignright Protect Environment Variables\nopagebreak

Prevents the environment from being automatically modified when invoking tools that depend on environment variables.

\item\syntax{"-r" <support> $\mid$ "-""-runtime" <support>}\alignright Include Specified Runtime Support\nopagebreak

Includes additional runtime support stored in the specified object or library file during linking.
\ifbook See page~\pageref{idx:runtime} for an index of all available runtime support. \fi

\item\syntax{"-s" <type> $\mid$ "-""-source" <type>}\alignright Use Specified Source Type\nopagebreak

Allows defining the source type of input files for which the corresponding tool could not be inferred from the filename extension.

\item\syntax{"-t" <environment> $\mid$ "-""-target" <environment>}\alignright Target Specified Environment\nopagebreak

Specifies the target environment for cross compilations.
\ifbook See page~\pageref{idx:environment} for an index of all available target environments. \fi
If omitted, the tool tries to target its own runtime environment.

\item\syntax{"-v" $\mid$ "-""-verbose"}\alignright Print Command Line\nopagebreak

Enables verbose mode which prints the actual command line before invoking any tool.
This is helpful for examining how and which tools are invoked and what runtime support is provided.

\end{description}

The \tool{ecsd} driver tool is most useful when it is accessible via the \environmentvariable{PATH} environment variable of the runtime environment and is part of an appropriate installation of the \ecs{}.
See \Documentation{}~\documentationref{guide}{User Guide} for some examples of using this utility tool in practice.

\concludechapter

% User guide for the Eigen Compiler Suite
% Copyright (C) Florian Negele

% This file is part of the Eigen Compiler Suite.

% Permission is granted to copy, distribute and/or modify this document
% under the terms of the GNU Free Documentation License, Version 1.3
% or any later version published by the Free Software Foundation.

% You should have received a copy of the GNU Free Documentation License
% along with the ECS.  If not, see <https://www.gnu.org/licenses/>.

% Generic documentation utilities
% Copyright (C) Florian Negele

% This file is part of the Eigen Compiler Suite.

% Permission is granted to copy, distribute and/or modify this document
% under the terms of the GNU Free Documentation License, Version 1.3
% or any later version published by the Free Software Foundation.

% You should have received a copy of the GNU Free Documentation License
% along with the ECS.  If not, see <https://www.gnu.org/licenses/>.

\providecommand{\cpp}{C\texttt{++}}
\providecommand{\opt}{_\mathit{opt}}
\providecommand{\tool}[1]{\texttt{#1}}
\providecommand{\version}{Version 0.0.40}
\providecommand{\resource}[1]{*++\txt{#1}}
\providecommand{\ecs}{Eigen Compiler Suite}
\providecommand{\changed}[1]{\underline{#1}}
\providecommand{\toolbox}[1]{\converter{#1}}
\providecommand{\file}{}\renewcommand{\file}[1]{\texttt{#1}}
\providecommand{\alignright}{\hfill\linebreak[0]\hspace*{\fill}}
\providecommand{\converter}[1]{*++[F][F*:white][F,:gray]\txt{#1}}
\providecommand{\documentation}{\ifbook chapter\else document\fi}
\providecommand{\Documentation}{\ifbook Chapter\else Document\fi}
\providecommand{\variable}[1]{\resource{\texttt{\small#1}\\variable}}
\providecommand{\documentationref}[2]{\ifbook\ref{#1}\else``\href{#1}{#2}''~\cite{#1}\fi}
\providecommand{\objfile}[1]{\texttt{#1}\index[runtime]{#1 object file@\texttt{#1} object file}}
\providecommand{\libfile}[1]{\texttt{#1}\index[runtime]{#1 library file@\texttt{#1} library file}}
\providecommand{\epigraph}[2]{\ifbook\begin{quote}\flushright\textit{#1}\par--- #2\end{quote}\fi}
\providecommand{\environmentvariable}[1]{\texttt{#1}\index{Environment variables!#1@\texttt{#1}}}
\providecommand{\environment}[1]{\texttt{#1}\index[environment]{#1 environment@\texttt{#1} environment}}
\providecommand{\toolsection}{}\renewcommand{\toolsection}[1]{\subsection{#1}\label{\prefix:#1}\tool{#1}}
\providecommand{\instruction}{}\renewcommand{\instruction}[2]{\noindent\qquad\pdftooltip{\texttt{#1}}{#2}\refstepcounter{instruction}\par}
\providecommand{\flowgraph}{}\renewcommand{\flowgraph}[1]{\par\sffamily\begin{displaymath}\xymatrix@=4ex{#1}\end{displaymath}\normalfont\par}
\providecommand{\instructionset}{}\renewcommand{\instructionset}[4]{\setcounter{instruction}{0}\begin{multicols}{\ifbook#3\else#4\fi}[{\captionof{table}[#2]{#2 (\ref*{#1:instructions}~instructions)}\label{tab:#1set}\vspace{-2ex}}]\footnotesize\raggedcolumns\input{#1.set}\label{#1:instructions}\end{multicols}}

\providecommand{\gpl}{GNU General Public License}
\providecommand{\rse}{ECS Runtime Support Exception}
\providecommand{\fdl}{\href{https://www.gnu.org/licenses/fdl.html}{GNU Free Documentation License}}

\providecommand{\docbegin}{}
\providecommand{\docend}{}
\providecommand{\doclabel}[1]{\hypertarget{#1}}
\providecommand{\doclink}[2]{\hyperlink{#1}{#2}}
\providecommand{\docsection}[3]{\hypertarget{#1}{\subsection{#2}}\label{sec:#1}\index[library]{#2@#3}}
\providecommand{\docsectionstar}[1]{}
\providecommand{\docsubbegin}{\begin{description}}
\providecommand{\docsubend}{\end{description}}
\providecommand{\docsubsection}[3]{\item[\hypertarget{#1}{#2}]\index[library]{#2@#3}}
\providecommand{\docsubsectionstar}[1]{\smallskip}
\providecommand{\docsubsubsection}[3]{\docsubsection{#1}{#2}{#3}}
\providecommand{\docsubsubsectionstar}[1]{}
\providecommand{\docsubsubsubsection}[3]{}
\providecommand{\docsubsubsubsectionstar}[1]{}
\providecommand{\doctable}{}

\providecommand{\debuggingtool}{}\renewcommand{\debuggingtool}{This tool is provided for debugging purposes.
It allows exposing and modifying an internal data structure that is usually not accessible.
}

\providecommand{\interface}{All tools accept command-line arguments which are taken as names of plain text files containing the source code.
If no arguments are provided, the standard input stream is used instead.
Output files are generated in the current working directory and have the same name as the input file being processed whereas the filename extension gets replaced by an appropriate suffix.
\seeinterface
}

\providecommand{\license}{\noindent Copyright \copyright{} Florian Negele\par\medskip\noindent
Permission is granted to copy, distribute and/or modify this document under the terms of the
\fdl{}, Version 1.3 or any later version published by the \href{https://fsf.org/}{Free Software Foundation}.
}

\providecommand{\ecslogosurface}{
\fill[darkgray] (0,0,0) -- (0,0,3) -- (0,3,3) -- (0,3,1) -- (0,4,1) -- (0,4,3) -- (0,5,3) -- (0,5,0) -- (0,2,0) -- (0,2,2) -- (0,1,2) -- (0,1,0) -- cycle;
\fill[gray] (0,5,0) -- (0,5,3) -- (1,5,3) -- (1,5,1) -- (2,5,1) -- (2,5,3) -- (3,5,3) -- (3,5,0) -- cycle;
\fill[lightgray] (0,0,0) -- (0,1,0) -- (2,1,0) -- (2,4,0) -- (1,4,0) -- (1,3,0) -- (2,3,0) -- (2,2,0) -- (0,2,0) -- (0,5,0) -- (3,5,0) -- (3,0,0) -- cycle;
\begin{scope}[line width=0.5]
\begin{scope}[gray]
\draw (0,0,0) -- (0,1,0);
\draw (2,1,0) -- (2,2,0);
\draw (0,1,2) -- (0,2,2);
\draw (0,2,0) -- (0,5,0);
\draw (2,3,0) -- (2,4,0);
\end{scope}
\begin{scope}[lightgray]
\draw (0,1,0) -- (0,1,2);
\draw (0,3,1) -- (0,3,3);
\draw (0,5,0) -- (0,5,3);
\draw (2,5,1) -- (2,5,3);
\end{scope}
\begin{scope}[white]
\draw (0,1,0) -- (2,1,0);
\draw (1,3,0) -- (2,3,0);
\draw (0,5,0) -- (3,5,0);
\end{scope}
\end{scope}
}

\providecommand{\ecslogo}[1]{
\begin{tikzpicture}[scale={(#1)/((sin(45)+cos(45))*3cm)},x={({-cos(45)*1cm},{sin(45)*sin(30)*1cm})},y={({0cm},{(cos(30)*1cm})},z={({sin(45)*1cm},{cos(45)*sin(30)*1cm})}]
\begin{scope}[darkgray,line width=1]
\draw (0,0,0) -- (0,0,3) -- (0,3,3) -- (2,3,3) -- (2,5,3) -- (3,5,3) -- (3,5,0) -- (3,0,0) -- cycle;
\draw (0,3,1) -- (0,4,1) -- (0,4,3) -- (0,5,3) -- (1,5,3) -- (1,5,1) -- (2,5,1);
\draw (1,3,0) -- (1,4,0) -- (2,4,0);
\end{scope}
\fill[darkgray] (2,0,0) -- (2,0,3) -- (2,5,3) -- (2,5,1) -- (2,4,1) -- (2,4,0) -- cycle;
\fill[lightgray] (2,0,2) -- (0,0,2) -- (0,2,2) -- (2,2,2) -- cycle;
\fill[gray] (0,1,0) -- (2,1,0) -- (2,1,2) -- (0,1,2) -- cycle;
\fill[gray] (0,3,1) -- (0,3,3) -- (2,3,3) -- (2,3,0) -- (1,3,0) -- (1,3,1) -- cycle;
\ecslogosurface
\end{tikzpicture}
}

\providecommand{\shadowedecslogo}[3]{
\begin{tikzpicture}[scale={(#1)/((sin(#2)+cos(#2))*3cm)},x={({-cos(#2)*1cm},{sin(#2)*sin(#3)*1cm})},y={({0cm},{(cos(#3)*1cm})},z={({sin(#2)*1cm},{cos(#2)*sin(#3)*1cm})}]
\shade[top color=lightgray!50!white,bottom color=white,middle color=lightgray!50!white] (0,0,0) -- (3,0,0) -- (3,{-0.5-3*sin(#2)*sin(#3)/cos(#3)},0) -- (0,-0.5,0) -- cycle;
\shade[top color=darkgray!50!gray,bottom color=white,middle color=darkgray!50!white] (0,0,0) -- (0,0,3) -- (0,{-0.5-3*cos(#2)*sin(#3)/cos(#3)},3) -- (0,-0.5,0) -- cycle;
\begin{scope}[y={({(cos(#2)+sin(#2))*0.5cm},{(cos(#2)*sin(#3)-sin(#2)*sin(#3))*0.5cm})}]
\useasboundingbox (3,0,0) -- (0,0,0) -- (0,0,3);
\shade[left color=darkgray!80!black,right color=lightgray,middle color=gray] (0,0,0) -- (0,1,0) -- (0,1,0.5) -- (0,2,0) -- (0,5,0) -- (0,5,3) -- (1,5,3) -- (1,4,3) -- (1,4,2.5) -- (1,3,3) -- (2,5,3) -- (3,5,3) -- (3,0,3) -- cycle;
\clip (0,0,0) -- (0,0,3) -- ({-3*sin(#2)/cos(#2)},0,0) -- cycle;
\shade[left color=darkgray,right color=lightgray!50!gray] (0,0,0) -- (0,1,0) -- (0,1,0.5) -- (0,2,0) -- (0,5,0) -- (0,5,3) -- (1,5,3) -- (1,4,3) -- (1,4,2.5) -- (1,3,3) -- (2,5,3) -- (3,5,3) -- (3,0,3) -- cycle;
\end{scope}
\shade[left color=darkgray,right color=darkgray!80!black] (2,0,0) -- (2,0,3) -- (2,5,3) -- (2,5,1) -- (2,4,1) -- (2,4,0) -- cycle;
\shade[left color=darkgray!90!black,right color=gray!80!darkgray] (2,0,2) -- (0,0,2) -- (0,2,2) -- (2,2,2) -- cycle;
\shade[top color=darkgray!90!black,bottom color=gray!80!darkgray] (0,1,0) -- (2,1,0) -- (2,1,2) -- (0,1,2) -- cycle;
\shade[top color=darkgray!90!black,bottom color=gray!80!darkgray] (0,3,1) -- (0,3,3) -- (2,3,3) -- (2,3,0) -- (1,3,0) -- (1,3,1) -- cycle;
\fill[gray] (2,1,0) -- (1.5,1,0.5) -- (0,1,0.5) -- (0,1,0) -- cycle;
\fill[gray] (1,3,2) -- (0.5,3,2) -- (0.5,3,3) -- (1,3,3) -- cycle;
\fill[gray] (2,3,0) -- (1.5,3,0.5) -- (1,3,0.5) -- (1,3,0) -- cycle;
\ecslogosurface
\end{tikzpicture}
}

\providecommand{\cpplogo}[1]{
\begin{tikzpicture}[scale=(#1)/512em]
\fill[gray] (435.2794,398.7159) -- (247.1911,507.3075) .. controls (236.3563,513.5642) and (218.6240,513.5642) .. (207.7892,507.3075) -- (19.7009,398.7159) .. controls (8.8646,392.4606) and (0.0000,377.1043) .. (0.0000,364.5924) -- (0.0000,147.4076) .. controls (0.8430,132.8363) and (8.2856,120.7683) .. (19.7009,113.2842) -- (207.7892,4.6926) .. controls (218.6240,-1.5642) and (236.3564,-1.5642) .. (247.1911,4.6926) -- (435.2794,113.2842) .. controls (447.5273,121.4304) and (454.4987,133.6918) .. (454.9803,147.4076) -- (454.9803,364.5924) .. controls (454.5404,377.7571) and (446.6566,391.0351) .. (435.2794,398.7159) -- cycle(75.8301,255.9993) .. controls (74.9389,404.0881) and (273.2892,469.4783) .. (358.8263,331.8769) -- (293.1917,293.8965) .. controls (253.5702,359.4301) and (155.1909,335.9977) .. (151.6601,255.9993) .. controls (152.7204,182.2703) and (249.4137,148.0211) .. (293.1961,218.1065) -- (358.8308,180.1276) .. controls (283.4477,49.2645) and (79.6318,96.3470) .. (75.8301,255.9993) -- cycle(379.1503,247.5747) -- (362.2982,247.5747) -- (362.2982,230.7226) -- (345.4490,230.7226) -- (345.4490,247.5747) -- (328.5969,247.5747) -- (328.5969,264.4254) -- (345.4490,264.4254) -- (345.4490,281.2759) -- (362.2982,281.2759) -- (362.2982,264.4254) -- (379.1503,264.4254) -- cycle(442.3420,247.5747) -- (425.4899,247.5747) -- (425.4899,230.7226) -- (408.6408,230.7226) -- (408.6408,247.5747) -- (391.7886,247.5747) -- (391.7886,264.4254) -- (408.6408,264.4254) -- (408.6408,281.2759) -- (425.4899,281.2759) -- (425.4899,264.4254) -- (442.3420,264.4254) -- cycle;
\end{tikzpicture}
}

\providecommand{\fallogo}[1]{
\begin{tikzpicture}[scale=(#1)/512em]
\fill[gray] (185.7774,0.0000) .. controls (200.4486,15.9798) and (226.8966,8.7148) .. (235.0426,31.5836) .. controls (249.5297,58.0598) and (247.9581,97.9161) .. (280.3335,110.9762) .. controls (309.1690,120.3496) and (337.8406,104.2727) .. (366.5753,103.9379) .. controls (373.4449,111.5171) and (379.2885,128.2574) .. (383.9755,108.9744) .. controls (396.6979,102.5615) and (437.2808,107.6681) .. (426.9652,124.3252) .. controls (408.9822,121.0785) and (412.4742,146.0729) .. (426.5192,131.4996) .. controls (433.8413,120.8489) and (465.1541,126.5522) .. (441.9067,135.7950) .. controls (396.1879,157.7478) and (344.1112,161.5079) .. (298.5528,183.5702) .. controls (277.7471,193.5198) and (284.6941,218.7163) .. (285.2127,236.9640) .. controls (292.3599,316.2826) and (307.3929,394.6311) .. (317.1198,473.6154) .. controls (329.0637,505.4736) and (292.1195,528.5004) .. (265.9183,511.2761) .. controls (237.9284,499.2462) and (237.3684,465.2681) .. (230.9102,439.9421) .. controls (218.6692,374.3397) and (215.6307,306.9662) .. (198.1732,242.3977) .. controls (183.1379,232.7444) and (164.4245,256.0298) .. (149.0430,261.4799) .. controls (116.9328,279.2585) and (87.1822,308.5851) .. (48.2293,307.8914) .. controls (21.3220,306.9037) and (-15.9107,281.8761) .. (7.2921,252.7908) .. controls (29.7799,220.6177) and (67.5177,204.3028) .. (100.9287,185.9449) .. controls (130.8217,170.8906) and (161.1548,156.5903) .. (191.0278,141.5847) .. controls (196.1738,120.0520) and (186.6049,95.2409) .. (186.8382,72.4353) .. controls (185.5234,48.4204) and (183.1700,23.9341) .. (185.7774,0.0000) -- cycle;
\end{tikzpicture}
}

\providecommand{\oblogo}[1]{
\begin{tikzpicture}[scale=(#1)/512em]
\fill[gray] (160.3865,208.9117) .. controls (154.0879,214.6478) and (149.0735,221.2409) .. (145.4125,228.5384) .. controls (184.8790,248.4273) and (234.7122,269.8787) .. (297.5493,291.8782) .. controls (300.3943,281.4769) and (300.9552,268.7619) .. (300.4023,255.2389) .. controls (248.9909,244.7891) and (200.0310,225.9279) .. (160.3865,208.9117) -- cycle(225.7398,392.6996) .. controls (308.0209,392.1716) and (359.3326,345.9277) .. (368.7203,285.2098) .. controls (376.6742,197.1784) and (311.7194,141.3342) .. (205.4287,142.1456) .. controls (139.9485,141.4804) and (88.7155,166.1957) .. (73.5775,228.0086) .. controls (52.0297,320.3408) and (123.4078,391.0103) .. (225.7398,392.6996) -- cycle(216.0739,176.4733) .. controls (268.9183,179.2424) and (315.8292,206.5488) .. (312.7454,265.1139) .. controls (313.2769,315.6384) and (286.5993,353.4946) .. (216.6040,355.7934) .. controls (162.4657,355.7934) and (126.0914,317.5023) .. (126.0914,260.5103) .. controls (126.1733,214.2900) and (163.3363,176.2849) .. (216.0739,176.4733) -- cycle(76.4897,189.1754) .. controls (13.1586,147.5631) and (0.0000,119.4207) .. (0.0000,119.4207) -- (90.6499,170.1632) .. controls (85.3004,175.8497) and (80.5994,182.1633) .. (76.4897,189.1754) -- cycle(353.9486,119.3004) -- (402.9482,119.3004) .. controls (427.0025,137.0797) and (450.9893,162.7034) .. (474.9529,191.0213) .. controls (509.3540,228.5339) and (531.3391,294.2091) .. (487.8149,312.1206) .. controls (462.8165,324.7652) and (394.3874,316.8943) .. (373.8912,313.6651) .. controls (379.9291,297.7449) and (383.2899,278.4204) .. (381.4989,257.7214) .. controls (420.3069,248.0321) and (421.9610,218.3461) .. (407.7867,192.6417) .. controls (391.1113,162.4018) and (370.1114,132.9097) .. (353.9486,119.3004) -- cycle;
\end{tikzpicture}
}

\providecommand{\markuptable}{
\begin{table}
\sffamily\centering
\begin{tabular}{@{}lcl@{}}
\toprule
\texttt{//italics//} & $\rightarrow$ & \textit{italics} \\
\midrule
\texttt{**bold**} & $\rightarrow$ & \textbf{bold} \\
\midrule
\texttt{\# ordered list} & & 1 ordered list \\
\texttt{\# second item} & $\rightarrow$ & 2 second item \\
\texttt{\#\# sub item} & & \hspace{1em} 1 sub item \\
\midrule
\texttt{* unordered list} & & $\bullet$ unordered list \\
\texttt{* second item} & $\rightarrow$ & $\bullet$ second item \\
\texttt{** sub item} & & \hspace{1em} $\bullet$ sub item \\
\midrule
\texttt{link to [[label]]} & $\rightarrow$ & link to \underline{label} \\
\midrule
\texttt{<{}<label>{}> definition } & $\rightarrow$ & definition \\
\midrule
\texttt{[[url|link name]]} & $\rightarrow$ & \underline{link name} \\
\midrule\addlinespace
\texttt{= large heading} & & {\Large large heading} \smallskip \\
\texttt{== medium heading} & $\rightarrow$ & {\large medium heading} \\
\texttt{=== small heading} & & small heading \\
\midrule
\texttt{no line break} & & no line break for paragraphs \\
\texttt{for paragraphs} & $\rightarrow$ \\
& & use empty line \\
\texttt{use empty line} \\
\midrule
\texttt{force\textbackslash\textbackslash line break} & $\rightarrow$ & force \\
& & line break \\
\midrule
\texttt{horizontal line} & $\rightarrow$ & horizontal line \\
\texttt{----} & & \hrulefill \\
\midrule
\texttt{|=a|=table|=header} & & \underline{a \enspace table \enspace header} \\
\texttt{|a|table|row} & $\rightarrow$ & a \enspace table \enspace row \\
\texttt{|b|table|row} & & b \enspace table \enspace row \\
\midrule
\texttt{\{\{\{} \\
\texttt{unformatted} & $\rightarrow$ & \texttt{unformatted} \\
\texttt{code} & & \texttt{code} \\
\texttt{\}\}\}} \\
\midrule\addlinespace
\texttt{@ new article} & & {\Large 1.\ new article} \smallskip \\
\texttt{@ second article} & $\rightarrow$ & {\Large 2.\ second article} \smallskip \\
\texttt{@@ sub article} & & {\large 2.1.\ sub article} \\
\bottomrule
\end{tabular}
\normalfont\caption{Elements of the generic documentation markup language}
\label{tab:docmarkup}
\end{table}
}

\providecommand{\startchapter}[4]{
\documentclass[11pt,a4paper]{article}
\usepackage{booktabs}
\usepackage[format=hang,labelfont=bf]{caption}
\usepackage{changepage}
\usepackage[T1]{fontenc}
\usepackage[margin=2cm]{geometry}
\usepackage{hyperref}
\usepackage[american]{isodate}
\usepackage{lmodern}
\usepackage{longtable}
\usepackage{mathptmx}
\usepackage{microtype}
\usepackage[toc]{multitoc}
\usepackage{multirow}
\usepackage[all]{nowidow}
\usepackage{pdfcomment}
\usepackage{syntax}
\usepackage{tikz}
\usepackage[all]{xy}
\hypersetup{pdfborder={0 0 0},bookmarksnumbered=true,pdftitle={\ecs{}: #2},pdfauthor={Florian Negele},pdfsubject={\ecs{}},pdfkeywords={#1}}
\setlength{\grammarindent}{8em}\setlength{\grammarparsep}{0.2ex}
\setlength{\columnsep}{2em}
\newcommand{\prefix}{}
\newcounter{instruction}
\bibliographystyle{unsrt}
\renewcommand{\index}[2][]{}
\renewcommand{\arraystretch}{1.05}
\renewcommand{\floatpagefraction}{0.7}
\renewcommand{\syntleft}{\itshape}\renewcommand{\syntright}{}
\title{\vspace{-5ex}\Huge{\ecs{}}\medskip\hrule}
\author{\huge{#2}}
\date{\medskip\version}
\newif\ifbook\bookfalse
\pagestyle{headings}
\frenchspacing
\begin{document}
\maketitle\thispagestyle{empty}\noindent#4\setlength{\columnseprule}{0.4pt}\tableofcontents\setlength{\columnseprule}{0pt}\vfill\pagebreak[3]\null\vfill\bigskip\noindent
\parbox{\textwidth-4em}{\license The contents of this \documentation{} are part of the \href{manual}{\ecs{} User Manual}~\cite{manual} and correspond to Chapter ``\href{manual\##3}{#1}''.\alignright\mbox{\today}}
\parbox{4em}{\flushright\ecslogo{3em}}
\clearpage
}

\providecommand{\concludechapter}{
\vfill\pagebreak[3]\null\vfill
\thispagestyle{myheadings}\markright{REFERENCES}
\noindent\begin{minipage}{\textwidth}\begin{multicols}{2}[\section*{References}]
\renewcommand{\section}[2]{}\small\bibliography{references}
\end{multicols}\end{minipage}\end{document}
}

\providecommand{\startpresentation}[2]{
\documentclass[14pt,aspectratio=43,usepdftitle=false]{beamer}
\usepackage{booktabs}
\usepackage{etex}
\usepackage{multicol}
\usepackage{tikz}
\usepackage[all]{xy}
\bibliographystyle{unsrt}
\setlength{\columnsep}{1em}
\setlength{\leftmargini}{1em}
\setbeamercolor{title}{fg=black}
\setbeamercolor{structure}{fg=darkgray}
\setbeamercolor{bibliography item}{fg=darkgray}
\setbeamerfont{title}{series=\bfseries}
\setbeamerfont{subtitle}{series=\normalfont}
\setbeamerfont*{frametitle}{parent=title}
\setbeamerfont{block title}{series=\bfseries}
\setbeamerfont*{framesubtitle}{parent=subtitle}
\setbeamersize{text margin left=1em,text margin right=1em}
\setbeamertemplate{navigation symbols}{}
\setbeamertemplate{itemize item}[circle]{}
\setbeamertemplate{bibliography item}[triangle]{}
\setbeamertemplate{bibliography entry author}{\usebeamercolor[fg]{bibliography item}}
\setbeamertemplate{frametitle}{\medskip\usebeamerfont{frametitle}\color{gray}\raisebox{-2.5ex}[0ex][0ex]{\rule{0.1em}{4.5ex}}}
\addtobeamertemplate{frametitle}{}{\hspace{0.4em}\usebeamercolor[fg]{title}\insertframetitle\par\vspace{0.2ex}\hspace{0.5em}\usebeamerfont{framesubtitle}\insertframesubtitle}
\hypersetup{pdfborder={0 0 0},bookmarksnumbered=true,bookmarksopen=true,bookmarksopenlevel=0,pdftitle={\ecs{}: #1},pdfauthor={Florian Negele},pdfsubject={\ecs{}},pdfkeywords={#1}}
\renewcommand{\flowgraph}[1]{\resizebox{\textwidth}{!}{$$\xymatrix{##1}$$}}
\title{\ecs{}\medskip\hrule\medskip}
\institute{\shadowedecslogo{5em}{30}{15}}
\date{\version}
\subtitle{#1}
\begin{document}
\begin{frame}[plain]\titlepage\nocite{manual}\end{frame}
\begin{frame}{Contents}{#1}\begin{center}\tableofcontents\end{center}\end{frame}
}

\providecommand{\concludepresentation}{
\begin{frame}{References}\begin{footnotesize}\setlength{\columnseprule}{0.4pt}\begin{multicols}{2}\bibliography{references}\end{multicols}\end{footnotesize}\end{frame}
\end{document}
}

\providecommand{\startbook}[1]{
\documentclass[10pt,paper=17cm:24cm,DIV=13,twoside=semi,headings=normal,numbers=noendperiod,cleardoublepage=plain]{scrbook}
\usepackage{atveryend}
\usepackage{booktabs}
\usepackage{caption}
\usepackage{changepage}
\usepackage[T1]{fontenc}
\usepackage{imakeidx}
\usepackage{hyperref}
\usepackage[american]{isodate}
\usepackage{lmodern}
\usepackage{longtable}
\usepackage{mathptmx}
\usepackage[final]{microtype}
\usepackage{multicol}
\usepackage{multirow}
\usepackage[all]{nowidow}
\usepackage{pdfcomment}
\usepackage{scrlayer-scrpage}
\usepackage{setspace}
\usepackage{syntax}
\usepackage[eventxtindent=4pt,oddtxtexdent=4pt]{thumbs}
\usepackage{tikz}
\usepackage[all]{xy}
\hyphenation{Micro-Blaze Open-Cores Open-RISC Power-PC}
\hypersetup{pdfborder={0 0 0},bookmarksnumbered=true,bookmarksopen=true,bookmarksopenlevel=0,pdftitle={\ecs{}: #1},pdfauthor={Florian Negele},pdfsubject={\ecs{}},pdfkeywords={#1}}
\setlength{\grammarindent}{8em}\setlength{\grammarparsep}{0.7ex}
\setkomafont{captionlabel}{\usekomafont{descriptionlabel}}
\renewcommand{\arraystretch}{1.05}\setstretch{1.1}
\renewcommand{\chapterformat}{\thechapter\autodot\enskip\raisebox{-1ex}[0ex][0ex]{\color{gray}\rule{0.1em}{3.5ex}}\enskip}
\renewcommand{\startchapter}[4]{\hypertarget{##3}{\chapter{##1}}\label{##3}##4\addthumb{##1}{\LARGE\sffamily\bfseries\thechapter}{white}{gray}\renewcommand{\prefix}{##3}}
\renewcommand{\concludechapter}{\clearpage{\stopthumb\cleardoublepage}}
\renewcommand{\syntleft}{\itshape}\renewcommand{\syntright}{}
\renewcommand{\floatpagefraction}{0.7}
\renewcommand{\partheademptypage}{}
\DeclareMicrotypeAlias{lmss}{cmr}
\newcommand{\prefix}{}
\newcounter{instruction}
\bibliographystyle{unsrt}
\newif\ifbook\booktrue
\makeindex[intoc,title=Index]
\makeindex[intoc,name=tools,title=Index of Tools,columns=3]
\makeindex[intoc,name=library,title=Index of Library Names]
\makeindex[intoc,name=runtime,title=Index of Runtime Support]
\makeindex[intoc,name=environment,title=Index of Target Environments]
\indexsetup{toclevel=chapter,headers={\indexname}{\indexname}}
\frenchspacing
\begin{document}
\pagenumbering{alph}
\begin{titlepage}\centering
\huge\sffamily\null\vfill\textbf{\ecs{}}\bigskip\hrule\bigskip#1
\normalsize\normalfont\vfill\vfill\shadowedecslogo{10em}{30}{15}
\large\vfill\vfill\version
\end{titlepage}
\null\vfill
\thispagestyle{empty}
\noindent\today\par\medskip
\license A copy of this license is included in Appendix~\ref{fdl} on page~\pageref{fdl}.
All product names used herein are for identification purposes only and may be trademarks of their respective companies.
\concludechapter
\frontmatter
\setcounter{tocdepth}{1}
\tableofcontents
\setcounter{tocdepth}{2}
\concludechapter
\listoffigures
\concludechapter
\listoftables
\concludechapter
}

\providecommand{\concludebook}{
\backmatter
\addtocontents{toc}{\protect\setcounter{tocdepth}{-1}}
\phantomsection\addcontentsline{toc}{part}{Bibliography}
\bibliography{references}
\concludechapter
\phantomsection\addcontentsline{toc}{part}{Indexes}
\printindex
\concludechapter
\indexprologue{\label{idx:tools}}
\printindex[tools]
\concludechapter
\printindex[library]
\concludechapter
\indexprologue{\label{idx:runtime}}
\printindex[runtime]
\concludechapter
\indexprologue{\label{idx:environment}}
\printindex[environment]
\concludechapter
\pagestyle{empty}\pagenumbering{Alph}\null\clearpage
\null\vfill\centering\ecslogo{4em}\par\medskip\license
\end{document}
}

% chapter references

\providecommand{\seedocumentationref}{}\renewcommand{\seedocumentationref}[3]{#1, see \Documentation{}~\documentationref{#2}{#3}. }
\providecommand{\seeinterface}{}\renewcommand{\seeinterface}{\ifbook See \Documentation{}~\documentationref{interface}{User Interface} for more information about the common user interface of all of these tools. \fi}
\providecommand{\seeguide}{}\renewcommand{\seeguide}{\seedocumentationref{For basic examples of using some of these tools in practice}{guide}{User Guide}}
\providecommand{\seecpp}{}\renewcommand{\seecpp}{\seedocumentationref{For more information about the \cpp{} programming language and its implementation by the \ecs{}}{cpp}{User Manual for \cpp{}}}
\providecommand{\seefalse}{}\renewcommand{\seefalse}{\seedocumentationref{For more information about the FALSE programming language and its implementation by the \ecs{}}{false}{User Manual for FALSE}}
\providecommand{\seeoberon}{}\renewcommand{\seeoberon}{\seedocumentationref{For more information about the Oberon programming language and its implementation by the \ecs{}}{oberon}{User Manual for Oberon}}
\providecommand{\seeassembly}{}\renewcommand{\seeassembly}{\seedocumentationref{For more information about the generic assembly language and how to use it}{assembly}{Generic Assembly Language Specification}}
\providecommand{\seeamd}{}\renewcommand{\seeamd}{\seedocumentationref{For more information about how the \ecs{} supports the AMD64 hardware architecture}{amd64}{AMD64 Hardware Architecture Support}}
\providecommand{\seearm}{}\renewcommand{\seearm}{\seedocumentationref{For more information about how the \ecs{} supports the ARM hardware architecture}{arm}{ARM Hardware Architecture Support}}
\providecommand{\seeavr}{}\renewcommand{\seeavr}{\seedocumentationref{For more information about how the \ecs{} supports the AVR hardware architecture}{avr}{AVR Hardware Architecture Support}}
\providecommand{\seeavrtt}{}\renewcommand{\seeavrtt}{\seedocumentationref{For more information about how the \ecs{} supports the AVR32 hardware architecture}{avr32}{AVR32 Hardware Architecture Support}}
\providecommand{\seemabk}{}\renewcommand{\seemabk}{\seedocumentationref{For more information about how the \ecs{} supports the M68000 hardware architecture}{m68k}{M68000 Hardware Architecture Support}}
\providecommand{\seemibl}{}\renewcommand{\seemibl}{\seedocumentationref{For more information about how the \ecs{} supports the MicroBlaze hardware architecture}{mibl}{MicroBlaze Hardware Architecture Support}}
\providecommand{\seemips}{}\renewcommand{\seemips}{\seedocumentationref{For more information about how the \ecs{} supports the MIPS32 and MIPS64 hardware architectures}{mips}{MIPS Hardware Architecture Support}}
\providecommand{\seemmix}{}\renewcommand{\seemmix}{\seedocumentationref{For more information about how the \ecs{} supports the MMIX hardware architecture}{mmix}{MMIX Hardware Architecture Support}}
\providecommand{\seeorok}{}\renewcommand{\seeorok}{\seedocumentationref{For more information about how the \ecs{} supports the OpenRISC 1000 hardware architecture}{or1k}{OpenRISC 1000 Hardware Architecture Support}}
\providecommand{\seeppc}{}\renewcommand{\seeppc}{\seedocumentationref{For more information about how the \ecs{} supports the PowerPC hardware architecture}{ppc}{PowerPC Hardware Architecture Support}}
\providecommand{\seerisc}{}\renewcommand{\seerisc}{\seedocumentationref{For more information about how the \ecs{} supports the RISC hardware architecture}{risc}{RISC Hardware Architecture Support}}
\providecommand{\seewasm}{}\renewcommand{\seewasm}{\seedocumentationref{For more information about how the \ecs{} supports the WebAssembly architecture}{wasm}{WebAssembly Architecture Support}}
\providecommand{\seedocumentation}{}\renewcommand{\seedocumentation}{\seedocumentationref{For more information about generic documentations and their generation by the \ecs{}}{documentation}{Generic Documentation Generation}}
\providecommand{\seedebugging}{}\renewcommand{\seedebugging}{\seedocumentationref{For more information about debugging information and its representation}{debugging}{Debugging Information Representation}}
\providecommand{\seecode}{}\renewcommand{\seecode}{\seedocumentationref{For more information about intermediate code and its purpose}{code}{Intermediate Code Representation}}
\providecommand{\seeobject}{}\renewcommand{\seeobject}{\seedocumentationref{For more information about object files and their purpose}{object}{Object File Representation}}

% generic documentation tools

\providecommand{\docprint}{
\toolsection{docprint} is a pretty printer for generic documentations.
It reformats generic documentations and writes it to the standard output stream.
\debuggingtool
\flowgraph{\resource{generic\\documentation} \ar[r] & \toolbox{docprint} \ar[r] & \resource{generic\\documentation}}
\seedocumentation
}

\providecommand{\doccheck}{
\toolsection{doccheck} is a syntactic and semantic checker for generic documentations.
It just performs syntactic and semantic checks on generic documentations and writes its diagnostic messages to the standard error stream.
\debuggingtool
\flowgraph{\resource{generic\\documentation} \ar[r] & \toolbox{doccheck} \ar[r] & \resource{diagnostic\\messages}}
\seedocumentation
}

\providecommand{\dochtml}{
\toolsection{dochtml} is an HTML documentation generator for generic documentations.
It processes several generic documentations and assembles all information therein into an HTML document.
\debuggingtool
\flowgraph{\resource{generic\\documentation} \ar[r] & \toolbox{dochtml} \ar[r] & \resource{HTML\\document}}
\seedocumentation
}

\providecommand{\doclatex}{
\toolsection{doclatex} is a Latex documentation generator for generic documentations.
It processes several generic documentations and assembles all information therein into a Latex document.
\debuggingtool
\flowgraph{\resource{generic\\documentation} \ar[r] & \toolbox{doclatex} \ar[r] & \resource{Latex\\document}}
\seedocumentation
}

% intermediate code tools

\providecommand{\cdcheck}{
\toolsection{cdcheck} is a syntactic and semantic checker for intermediate code.
It just performs syntactic and semantic checks on programs written in intermediate code and writes its diagnostic messages to the standard error stream.
\debuggingtool
\flowgraph{\resource{intermediate\\code} \ar[r] & \toolbox{cdcheck} \ar[r] & \resource{diagnostic\\messages}}
\seeassembly\seecode
}

\providecommand{\cdopt}{
\toolsection{cdopt} is an optimizer for intermediate code.
It performs various optimizations on programs written in intermediate code and writes the result to the standard output stream.
\debuggingtool
\flowgraph{\resource{intermediate\\code} \ar[r] & \toolbox{cdopt} \ar[r] & \resource{optimized\\code}}
\seeassembly\seecode
}

\providecommand{\cdrun}{
\toolsection{cdrun} is an interpreter for intermediate code.
It processes and executes programs written in intermediate code.
The following code sections are predefined and have the usual semantics:
\texttt{abort}, \texttt{\_Exit}, \texttt{fflush}, \texttt{floor}, \texttt{fputc}, \texttt{free}, \texttt{getchar}, \texttt{malloc}, and \texttt{putchar}.
Diagnostic messages about invalid operations include the name of the executed code section and the index of the erroneous instruction.
\debuggingtool
\flowgraph{\resource{intermediate\\code} \ar[r] & \toolbox{cdrun} \ar@/u/[r] & \resource{input/\\output} \ar@/d/[l]}
\seeassembly\seecode
}

\providecommand{\cdamda}{
\toolsection{cdamd16} is a compiler for intermediate code targeting the AMD64 hardware architecture.
It generates machine code for AMD64 processors from programs written in intermediate code and stores it in corresponding object files.
The compiler generates machine code for the 16-bit operating mode defined by the AMD64 architecture.
It also creates a debugging information file as well as an assembly file containing a listing of the generated machine code.
\debuggingtool
\flowgraph{\resource{intermediate\\code} \ar[r] & \toolbox{cdamd16} \ar[r] \ar[d] \ar[rd] & \resource{object file} \\ & \resource{assembly\\listing} & \resource{debugging\\information}}
\seeassembly\seeamd\seeobject\seecode\seedebugging
}

\providecommand{\cdamdb}{
\toolsection{cdamd32} is a compiler for intermediate code targeting the AMD64 hardware architecture.
It generates machine code for AMD64 processors from programs written in intermediate code and stores it in corresponding object files.
The compiler generates machine code for the 32-bit operating mode defined by the AMD64 architecture.
It also creates a debugging information file as well as an assembly file containing a listing of the generated machine code.
\debuggingtool
\flowgraph{\resource{intermediate\\code} \ar[r] & \toolbox{cdamd32} \ar[r] \ar[d] \ar[rd] & \resource{object file} \\ & \resource{assembly\\listing} & \resource{debugging\\information}}
\seeassembly\seeamd\seeobject\seecode\seedebugging
}

\providecommand{\cdamdc}{
\toolsection{cdamd64} is a compiler for intermediate code targeting the AMD64 hardware architecture.
It generates machine code for AMD64 processors from programs written in intermediate code and stores it in corresponding object files.
The compiler generates machine code for the 64-bit operating mode defined by the AMD64 architecture.
It also creates a debugging information file as well as an assembly file containing a listing of the generated machine code.
\debuggingtool
\flowgraph{\resource{intermediate\\code} \ar[r] & \toolbox{cdamd64} \ar[r] \ar[d] \ar[rd] & \resource{object file} \\ & \resource{assembly\\listing} & \resource{debugging\\information}}
\seeassembly\seeamd\seeobject\seecode\seedebugging
}

\providecommand{\cdarma}{
\toolsection{cdarma32} is a compiler for intermediate code targeting the ARM hardware architecture.
It generates machine code for ARM processors executing A32 instructions from programs written in intermediate code and stores it in corresponding object files.
It also creates a debugging information file as well as an assembly file containing a listing of the generated machine code.
\debuggingtool
\flowgraph{\resource{intermediate\\code} \ar[r] & \toolbox{cdarma32} \ar[r] \ar[d] \ar[rd] & \resource{object file} \\ & \resource{assembly\\listing} & \resource{debugging\\information}}
\seeassembly\seearm\seeobject\seecode\seedebugging
}

\providecommand{\cdarmb}{
\toolsection{cdarma64} is a compiler for intermediate code targeting the ARM hardware architecture.
It generates machine code for ARM processors executing A64 instructions from programs written in intermediate code and stores it in corresponding object files.
It also creates a debugging information file as well as an assembly file containing a listing of the generated machine code.
\debuggingtool
\flowgraph{\resource{intermediate\\code} \ar[r] & \toolbox{cdarma64} \ar[r] \ar[d] \ar[rd] & \resource{object file} \\ & \resource{assembly\\listing} & \resource{debugging\\information}}
\seeassembly\seearm\seeobject\seecode\seedebugging
}

\providecommand{\cdarmc}{
\toolsection{cdarmt32} is a compiler for intermediate code targeting the ARM hardware architecture.
It generates machine code for ARM processors without floating-point extension executing T32 instructions from programs written in intermediate code and stores it in corresponding object files.
It also creates a debugging information file as well as an assembly file containing a listing of the generated machine code.
\debuggingtool
\flowgraph{\resource{intermediate\\code} \ar[r] & \toolbox{cdarmt32} \ar[r] \ar[d] \ar[rd] & \resource{object file} \\ & \resource{assembly\\listing} & \resource{debugging\\information}}
\seeassembly\seearm\seeobject\seecode\seedebugging
}

\providecommand{\cdarmcfpe}{
\toolsection{cdarmt32fpe} is a compiler for intermediate code targeting the ARM hardware architecture.
It generates machine code for ARM processors with floating-point extension executing T32 instructions from programs written in intermediate code and stores it in corresponding object files.
It also creates a debugging information file as well as an assembly file containing a listing of the generated machine code.
\debuggingtool
\flowgraph{\resource{intermediate\\code} \ar[r] & \toolbox{cdarmt32fpe} \ar[r] \ar[d] \ar[rd] & \resource{object file} \\ & \resource{assembly\\listing} & \resource{debugging\\information}}
\seeassembly\seearm\seeobject\seecode\seedebugging
}

\providecommand{\cdavr}{
\toolsection{cdavr} is a compiler for intermediate code targeting the AVR hardware architecture.
It generates machine code for AVR processors from programs written in intermediate code and stores it in corresponding object files.
It also creates a debugging information file as well as an assembly file containing a listing of the generated machine code.
\debuggingtool
\flowgraph{\resource{intermediate\\code} \ar[r] & \toolbox{cdavr} \ar[r] \ar[d] \ar[rd] & \resource{object file} \\ & \resource{assembly\\listing} & \resource{debugging\\information}}
\seeassembly\seeavr\seeobject\seecode\seedebugging
}

\providecommand{\cdavrtt}{
\toolsection{cdavr32} is a compiler for intermediate code targeting the AVR32 hardware architecture.
It generates machine code for AVR32 processors from programs written in intermediate code and stores it in corresponding object files.
It also creates a debugging information file as well as an assembly file containing a listing of the generated machine code.
\debuggingtool
\flowgraph{\resource{intermediate\\code} \ar[r] & \toolbox{cdavr32} \ar[r] \ar[d] \ar[rd] & \resource{object file} \\ & \resource{assembly\\listing} & \resource{debugging\\information}}
\seeassembly\seeavrtt\seeobject\seecode\seedebugging
}

\providecommand{\cdmabk}{
\toolsection{cdm68k} is a compiler for intermediate code targeting the M68000 hardware architecture.
It generates machine code for M68000 processors from programs written in intermediate code and stores it in corresponding object files.
It also creates a debugging information file as well as an assembly file containing a listing of the generated machine code.
\debuggingtool
\flowgraph{\resource{intermediate\\code} \ar[r] & \toolbox{cdm68k} \ar[r] \ar[d] \ar[rd] & \resource{object file} \\ & \resource{assembly\\listing} & \resource{debugging\\information}}
\seeassembly\seemabk\seeobject\seecode\seedebugging
}

\providecommand{\cdmibl}{
\toolsection{cdmibl} is a compiler for intermediate code targeting the MicroBlaze hardware architecture.
It generates machine code for MicroBlaze processors from programs written in intermediate code and stores it in corresponding object files.
It also creates a debugging information file as well as an assembly file containing a listing of the generated machine code.
\debuggingtool
\flowgraph{\resource{intermediate\\code} \ar[r] & \toolbox{cdmibl} \ar[r] \ar[d] \ar[rd] & \resource{object file} \\ & \resource{assembly\\listing} & \resource{debugging\\information}}
\seeassembly\seemibl\seeobject\seecode\seedebugging
}

\providecommand{\cdmipsa}{
\toolsection{cdmips32} is a compiler for intermediate code targeting the MIPS32 hardware architecture.
It generates machine code for MIPS32 processors from programs written in intermediate code and stores it in corresponding object files.
It also creates a debugging information file as well as an assembly file containing a listing of the generated machine code.
\debuggingtool
\flowgraph{\resource{intermediate\\code} \ar[r] & \toolbox{cdmips32} \ar[r] \ar[d] \ar[rd] & \resource{object file} \\ & \resource{assembly\\listing} & \resource{debugging\\information}}
\seeassembly\seemips\seeobject\seecode\seedebugging
}

\providecommand{\cdmipsb}{
\toolsection{cdmips64} is a compiler for intermediate code targeting the MIPS64 hardware architecture.
It generates machine code for MIPS64 processors from programs written in intermediate code and stores it in corresponding object files.
It also creates a debugging information file as well as an assembly file containing a listing of the generated machine code.
\debuggingtool
\flowgraph{\resource{intermediate\\code} \ar[r] & \toolbox{cdmips64} \ar[r] \ar[d] \ar[rd] & \resource{object file} \\ & \resource{assembly\\listing} & \resource{debugging\\information}}
\seeassembly\seemips\seeobject\seecode\seedebugging
}

\providecommand{\cdmmix}{
\toolsection{cdmmix} is a compiler for intermediate code targeting the MMIX hardware architecture.
It generates machine code for MMIX processors from programs written in intermediate code and stores it in corresponding object files.
It also creates a debugging information file as well as an assembly file containing a listing of the generated machine code.
\debuggingtool
\flowgraph{\resource{intermediate\\code} \ar[r] & \toolbox{cdmmix} \ar[r] \ar[d] \ar[rd] & \resource{object file} \\ & \resource{assembly\\listing} & \resource{debugging\\information}}
\seeassembly\seemmix\seeobject\seecode\seedebugging
}

\providecommand{\cdorok}{
\toolsection{cdor1k} is a compiler for intermediate code targeting the OpenRISC 1000 hardware architecture.
It generates machine code for OpenRISC 1000 processors from programs written in intermediate code and stores it in corresponding object files.
It also creates a debugging information file as well as an assembly file containing a listing of the generated machine code.
\debuggingtool
\flowgraph{\resource{intermediate\\code} \ar[r] & \toolbox{cdor1k} \ar[r] \ar[d] \ar[rd] & \resource{object file} \\ & \resource{assembly\\listing} & \resource{debugging\\information}}
\seeassembly\seeorok\seeobject\seecode\seedebugging
}

\providecommand{\cdppca}{
\toolsection{cdppc32} is a compiler for intermediate code targeting the PowerPC hardware architecture.
It generates machine code for PowerPC processors from programs written in intermediate code and stores it in corresponding object files.
The compiler generates machine code for the 32-bit operating mode defined by the PowerPC architecture.
It also creates a debugging information file as well as an assembly file containing a listing of the generated machine code.
\debuggingtool
\flowgraph{\resource{intermediate\\code} \ar[r] & \toolbox{cdppc32} \ar[r] \ar[d] \ar[rd] & \resource{object file} \\ & \resource{assembly\\listing} & \resource{debugging\\information}}
\seeassembly\seeppc\seeobject\seecode\seedebugging
}

\providecommand{\cdppcb}{
\toolsection{cdppc64} is a compiler for intermediate code targeting the PowerPC hardware architecture.
It generates machine code for PowerPC processors from programs written in intermediate code and stores it in corresponding object files.
The compiler generates machine code for the 64-bit operating mode defined by the PowerPC architecture.
It also creates a debugging information file as well as an assembly file containing a listing of the generated machine code.
\debuggingtool
\flowgraph{\resource{intermediate\\code} \ar[r] & \toolbox{cdppc64} \ar[r] \ar[d] \ar[rd] & \resource{object file} \\ & \resource{assembly\\listing} & \resource{debugging\\information}}
\seeassembly\seeppc\seeobject\seecode\seedebugging
}

\providecommand{\cdrisc}{
\toolsection{cdrisc} is a compiler for intermediate code targeting the RISC hardware architecture.
It generates machine code for RISC processors from programs written in intermediate code and stores it in corresponding object files.
It also creates a debugging information file as well as an assembly file containing a listing of the generated machine code.
\debuggingtool
\flowgraph{\resource{intermediate\\code} \ar[r] & \toolbox{cdrisc} \ar[r] \ar[d] \ar[rd] & \resource{object file} \\ & \resource{assembly\\listing} & \resource{debugging\\information}}
\seeassembly\seerisc\seeobject\seecode\seedebugging
}

\providecommand{\cdwasm}{
\toolsection{cdwasm} is a compiler for intermediate code targeting the WebAssembly architecture.
It generates machine code for WebAssembly targets from programs written in intermediate code and stores it in corresponding object files.
It also creates a debugging information file as well as an assembly file containing a listing of the generated machine code.
\debuggingtool
\flowgraph{\resource{intermediate\\code} \ar[r] & \toolbox{cdwasm} \ar[r] \ar[d] \ar[rd] & \resource{object file} \\ & \resource{assembly\\listing} & \resource{debugging\\information}}
\seeassembly\seewasm\seeobject\seecode\seedebugging
}

% C++ tools

\providecommand{\cppprep}{
\toolsection{cppprep} is a preprocessor for the \cpp{} programming language.
It preprocesses source code according to the rules of \cpp{} and writes it to the standard output stream.
Only the macro names \texttt{\_\_DATE\_\_}, \texttt{\_\_FILE\_\_}, \texttt{\_\_LINE\_\_}, and \texttt{\_\_TIME\_\_} are predefined.
\flowgraph{\resource{\cpp{} or other\\source code} \ar[r] & \toolbox{cppprep} \ar[r] & \resource{preprocessed\\source code} \\ & \variable{ECSINCLUDE} \ar[u]}
\seecpp
}

\providecommand{\cppprint}{
\toolsection{cppprint} is a pretty printer for the \cpp{} programming language.
It reformats the source code of \cpp{} programs and writes it to the standard output stream.
\flowgraph{\resource{\cpp{}\\source code} \ar[r] & \toolbox{cppprint} \ar[r] & \resource{reformatted\\source code} \\ & \variable{ECSINCLUDE} \ar[u]}
\seecpp
}

\providecommand{\cppcheck}{
\toolsection{cppcheck} is a syntactic and semantic checker for the \cpp{} programming language.
It just performs syntactic and semantic checks on \cpp{} programs and writes its diagnostic messages to the standard error stream.
\flowgraph{\resource{\cpp{}\\source code} \ar[r] & \toolbox{cppcheck} \ar[r] & \resource{diagnostic\\messages} \\ & \variable{ECSINCLUDE} \ar[u]}
\seecpp
}

\providecommand{\cppdump}{
\toolsection{cppdump} is a serializer for the \cpp{} programming language.
It dumps the complete internal representation of programs written in \cpp{} into an XML document.
\debuggingtool
\flowgraph{\resource{\cpp{}\\source code} \ar[r] & \toolbox{cppdump} \ar[r] & \resource{internal\\representation} \\ & \variable{ECSINCLUDE} \ar[u]}
\seecpp
}

\providecommand{\cpprun}{
\toolsection{cpprun} is an interpreter for the \cpp{} programming language.
It processes and executes programs written in \cpp{}.
The macro \texttt{\_\_run\_\_} is predefined in order to enable programmers to identify this tool while interpreting.
\flowgraph{\resource{\cpp{}\\source code} \ar[r] & \toolbox{cpprun} \ar@/u/[r] & \resource{input/\\output} \ar@/d/[l] \\ & \variable{ECSINCLUDE} \ar[u]}
\seecpp
}

\providecommand{\cppdoc}{
\toolsection{cppdoc} is a generic documentation generator for the \cpp{} programming language.
It processes several \cpp{} source files and assembles all information therein into a generic documentation.
\debuggingtool
\flowgraph{\resource{\cpp{}\\source code} \ar[r] & \toolbox{cppdoc} \ar[r] & \resource{generic\\documentation} \\ & \variable{ECSINCLUDE} \ar[u]}
\seecpp\seedocumentation
}

\providecommand{\cpphtml}{
\toolsection{cpphtml} is an HTML documentation generator for the \cpp{} programming language.
It processes several \cpp{} source files and assembles all information therein into an HTML document.
\flowgraph{\resource{\cpp{}\\source code} \ar[r] & \toolbox{cpphtml} \ar[r] & \resource{HTML\\document} \\ & \variable{ECSINCLUDE} \ar[u]}
\seecpp\seedocumentation
}

\providecommand{\cpplatex}{
\toolsection{cpplatex} is a Latex documentation generator for the \cpp{} programming language.
It processes several \cpp{} source files and assembles all information therein into a Latex document.
\flowgraph{\resource{\cpp{}\\source code} \ar[r] & \toolbox{cpplatex} \ar[r] & \resource{Latex\\document} \\ & \variable{ECSINCLUDE} \ar[u]}
\seecpp\seedocumentation
}

\providecommand{\cppcode}{
\toolsection{cppcode} is an intermediate code generator for the \cpp{} programming language.
It generates intermediate code from programs written in \cpp{} and stores it in corresponding assembly files.
The macro \texttt{\_\_code\_\_} is predefined in order to enable programmers to identify this tool while generating intermediate code.
Programs generated with this tool require additional runtime support that is stored in the \file{cpp\-code\-run} library file.
\debuggingtool
\flowgraph{\resource{\cpp{}\\source code} \ar[r] & \toolbox{cppcode} \ar[r] & \resource{intermediate\\code} \\ & \variable{ECSINCLUDE} \ar[u]}
\seecpp\seeassembly\seecode
}

\providecommand{\cppamda}{
\toolsection{cppamd16} is a compiler for the \cpp{} programming language targeting the AMD64 hardware architecture.
It generates machine code for AMD64 processors from programs written in \cpp{} and stores it in corresponding object files.
The compiler generates machine code for the 16-bit operating mode defined by the AMD64 architecture.
For debugging purposes, it also creates a debugging information file as well as an assembly file containing a listing of the generated machine code.
The macro \texttt{\_\_amd16\_\_} is predefined in order to enable programmers to identify this tool and its target architecture while compiling.
Programs generated with this compiler require additional runtime support that is stored in the \file{cpp\-amd16\-run} library file.
\flowgraph{\resource{\cpp{}\\source code} \ar[r] & \toolbox{cppamd16} \ar[r] \ar[d] \ar[rd] & \resource{object file} \\ \variable{ECSINCLUDE} \ar[ru] & \resource{debugging\\information} & \resource{assembly\\listing}}
\seecpp\seeassembly\seeamd\seeobject\seedebugging
}

\providecommand{\cppamdb}{
\toolsection{cppamd32} is a compiler for the \cpp{} programming language targeting the AMD64 hardware architecture.
It generates machine code for AMD64 processors from programs written in \cpp{} and stores it in corresponding object files.
The compiler generates machine code for the 32-bit operating mode defined by the AMD64 architecture.
For debugging purposes, it also creates a debugging information file as well as an assembly file containing a listing of the generated machine code.
The macro \texttt{\_\_amd32\_\_} is predefined in order to enable programmers to identify this tool and its target architecture while compiling.
Programs generated with this compiler require additional runtime support that is stored in the \file{cpp\-amd32\-run} library file.
\flowgraph{\resource{\cpp{}\\source code} \ar[r] & \toolbox{cppamd32} \ar[r] \ar[d] \ar[rd] & \resource{object file} \\ \variable{ECSINCLUDE} \ar[ru] & \resource{debugging\\information} & \resource{assembly\\listing}}
\seecpp\seeassembly\seeamd\seeobject\seedebugging
}

\providecommand{\cppamdc}{
\toolsection{cppamd64} is a compiler for the \cpp{} programming language targeting the AMD64 hardware architecture.
It generates machine code for AMD64 processors from programs written in \cpp{} and stores it in corresponding object files.
The compiler generates machine code for the 64-bit operating mode defined by the AMD64 architecture.
For debugging purposes, it also creates a debugging information file as well as an assembly file containing a listing of the generated machine code.
The macro \texttt{\_\_amd64\_\_} is predefined in order to enable programmers to identify this tool and its target architecture while compiling.
Programs generated with this compiler require additional runtime support that is stored in the \file{cpp\-amd64\-run} library file.
\flowgraph{\resource{\cpp{}\\source code} \ar[r] & \toolbox{cppamd64} \ar[r] \ar[d] \ar[rd] & \resource{object file} \\ \variable{ECSINCLUDE} \ar[ru] & \resource{debugging\\information} & \resource{assembly\\listing}}
\seecpp\seeassembly\seeamd\seeobject\seedebugging
}

\providecommand{\cpparma}{
\toolsection{cpparma32} is a compiler for the \cpp{} programming language targeting the ARM hardware architecture.
It generates machine code for ARM processors executing A32 instructions from programs written in \cpp{} and stores it in corresponding object files.
For debugging purposes, it also creates a debugging information file as well as an assembly file containing a listing of the generated machine code.
The macro \texttt{\_\_arma32\_\_} is predefined in order to enable programmers to identify this tool and its target architecture while compiling.
Programs generated with this compiler require additional runtime support that is stored in the \file{cpp\-arma32\-run} library file.
\flowgraph{\resource{\cpp{}\\source code} \ar[r] & \toolbox{cpparma32} \ar[r] \ar[d] \ar[rd] & \resource{object file} \\ \variable{ECSINCLUDE} \ar[ru] & \resource{debugging\\information} & \resource{assembly\\listing}}
\seecpp\seeassembly\seearm\seeobject\seedebugging
}

\providecommand{\cpparmb}{
\toolsection{cpparma64} is a compiler for the \cpp{} programming language targeting the ARM hardware architecture.
It generates machine code for ARM processors executing A64 instructions from programs written in \cpp{} and stores it in corresponding object files.
For debugging purposes, it also creates a debugging information file as well as an assembly file containing a listing of the generated machine code.
The macro \texttt{\_\_arma64\_\_} is predefined in order to enable programmers to identify this tool and its target architecture while compiling.
Programs generated with this compiler require additional runtime support that is stored in the \file{cpp\-arma64\-run} library file.
\flowgraph{\resource{\cpp{}\\source code} \ar[r] & \toolbox{cpparma64} \ar[r] \ar[d] \ar[rd] & \resource{object file} \\ \variable{ECSINCLUDE} \ar[ru] & \resource{debugging\\information} & \resource{assembly\\listing}}
\seecpp\seeassembly\seearm\seeobject\seedebugging
}

\providecommand{\cpparmc}{
\toolsection{cpparmt32} is a compiler for the \cpp{} programming language targeting the ARM hardware architecture.
It generates machine code for ARM processors without floating-point extension executing T32 instructions from programs written in \cpp{} and stores it in corresponding object files.
For debugging purposes, it also creates a debugging information file as well as an assembly file containing a listing of the generated machine code.
The macro \texttt{\_\_armt32\_\_} is predefined in order to enable programmers to identify this tool and its target architecture while compiling.
Programs generated with this compiler require additional runtime support that is stored in the \file{cpp\-armt32\-run} library file.
\flowgraph{\resource{\cpp{}\\source code} \ar[r] & \toolbox{cpparmt32} \ar[r] \ar[d] \ar[rd] & \resource{object file} \\ \variable{ECSINCLUDE} \ar[ru] & \resource{debugging\\information} & \resource{assembly\\listing}}
\seecpp\seeassembly\seearm\seeobject\seedebugging
}

\providecommand{\cpparmcfpe}{
\toolsection{cpparmt32fpe} is a compiler for the \cpp{} programming language targeting the ARM hardware architecture.
It generates machine code for ARM processors with floating-point extension executing T32 instructions from programs written in \cpp{} and stores it in corresponding object files.
For debugging purposes, it also creates a debugging information file as well as an assembly file containing a listing of the generated machine code.
The macro \texttt{\_\_armt32fpe\_\_} is predefined in order to enable programmers to identify this tool and its target architecture while compiling.
Programs generated with this compiler require additional runtime support that is stored in the \file{cpp\-armt32\-fpe\-run} library file.
\flowgraph{\resource{\cpp{}\\source code} \ar[r] & \toolbox{cpparmt32fpe} \ar[r] \ar[d] \ar[rd] & \resource{object file} \\ \variable{ECSINCLUDE} \ar[ru] & \resource{debugging\\information} & \resource{assembly\\listing}}
\seecpp\seeassembly\seearm\seeobject\seedebugging
}

\providecommand{\cppavr}{
\toolsection{cppavr} is a compiler for the \cpp{} programming language targeting the AVR hardware architecture.
It generates machine code for AVR processors from programs written in \cpp{} and stores it in corresponding object files.
For debugging purposes, it also creates a debugging information file as well as an assembly file containing a listing of the generated machine code.
The macro \texttt{\_\_avr\_\_} is predefined in order to enable programmers to identify this tool and its target architecture while compiling.
Programs generated with this compiler require additional runtime support that is stored in the \file{cpp\-avr\-run} library file.
\flowgraph{\resource{\cpp{}\\source code} \ar[r] & \toolbox{cppavr} \ar[r] \ar[d] \ar[rd] & \resource{object file} \\ \variable{ECSINCLUDE} \ar[ru] & \resource{debugging\\information} & \resource{assembly\\listing}}
\seecpp\seeassembly\seeavr\seeobject\seedebugging
}

\providecommand{\cppavrtt}{
\toolsection{cppavr32} is a compiler for the \cpp{} programming language targeting the AVR32 hardware architecture.
It generates machine code for AVR32 processors from programs written in \cpp{} and stores it in corresponding object files.
For debugging purposes, it also creates a debugging information file as well as an assembly file containing a listing of the generated machine code.
The macro \texttt{\_\_avr32\_\_} is predefined in order to enable programmers to identify this tool and its target architecture while compiling.
Programs generated with this compiler require additional runtime support that is stored in the \file{cpp\-avr32\-run} library file.
\flowgraph{\resource{\cpp{}\\source code} \ar[r] & \toolbox{cppavr32} \ar[r] \ar[d] \ar[rd] & \resource{object file} \\ \variable{ECSINCLUDE} \ar[ru] & \resource{debugging\\information} & \resource{assembly\\listing}}
\seecpp\seeassembly\seeavrtt\seeobject\seedebugging
}

\providecommand{\cppmabk}{
\toolsection{cppm68k} is a compiler for the \cpp{} programming language targeting the M68000 hardware architecture.
It generates machine code for M68000 processors from programs written in \cpp{} and stores it in corresponding object files.
For debugging purposes, it also creates a debugging information file as well as an assembly file containing a listing of the generated machine code.
The macro \texttt{\_\_m68k\_\_} is predefined in order to enable programmers to identify this tool and its target architecture while compiling.
Programs generated with this compiler require additional runtime support that is stored in the \file{cpp\-m68k\-run} library file.
\flowgraph{\resource{\cpp{}\\source code} \ar[r] & \toolbox{cppm68k} \ar[r] \ar[d] \ar[rd] & \resource{object file} \\ \variable{ECSINCLUDE} \ar[ru] & \resource{debugging\\information} & \resource{assembly\\listing}}
\seecpp\seeassembly\seemabk\seeobject\seedebugging
}

\providecommand{\cppmibl}{
\toolsection{cppmibl} is a compiler for the \cpp{} programming language targeting the MicroBlaze hardware architecture.
It generates machine code for MicroBlaze processors from programs written in \cpp{} and stores it in corresponding object files.
For debugging purposes, it also creates a debugging information file as well as an assembly file containing a listing of the generated machine code.
The macro \texttt{\_\_mibl\_\_} is predefined in order to enable programmers to identify this tool and its target architecture while compiling.
Programs generated with this compiler require additional runtime support that is stored in the \file{cpp\-mibl\-run} library file.
\flowgraph{\resource{\cpp{}\\source code} \ar[r] & \toolbox{cppmibl} \ar[r] \ar[d] \ar[rd] & \resource{object file} \\ \variable{ECSINCLUDE} \ar[ru] & \resource{debugging\\information} & \resource{assembly\\listing}}
\seecpp\seeassembly\seemibl\seeobject\seedebugging
}

\providecommand{\cppmipsa}{
\toolsection{cppmips32} is a compiler for the \cpp{} programming language targeting the MIPS32 hardware architecture.
It generates machine code for MIPS32 processors from programs written in \cpp{} and stores it in corresponding object files.
For debugging purposes, it also creates a debugging information file as well as an assembly file containing a listing of the generated machine code.
The macro \texttt{\_\_mips32\_\_} is predefined in order to enable programmers to identify this tool and its target architecture while compiling.
Programs generated with this compiler require additional runtime support that is stored in the \file{cpp\-mips32\-run} library file.
\flowgraph{\resource{\cpp{}\\source code} \ar[r] & \toolbox{cppmips32} \ar[r] \ar[d] \ar[rd] & \resource{object file} \\ \variable{ECSINCLUDE} \ar[ru] & \resource{debugging\\information} & \resource{assembly\\listing}}
\seecpp\seeassembly\seemips\seeobject\seedebugging
}

\providecommand{\cppmipsb}{
\toolsection{cppmips64} is a compiler for the \cpp{} programming language targeting the MIPS64 hardware architecture.
It generates machine code for MIPS64 processors from programs written in \cpp{} and stores it in corresponding object files.
For debugging purposes, it also creates a debugging information file as well as an assembly file containing a listing of the generated machine code.
The macro \texttt{\_\_mips64\_\_} is predefined in order to enable programmers to identify this tool and its target architecture while compiling.
Programs generated with this compiler require additional runtime support that is stored in the \file{cpp\-mips64\-run} library file.
\flowgraph{\resource{\cpp{}\\source code} \ar[r] & \toolbox{cppmips64} \ar[r] \ar[d] \ar[rd] & \resource{object file} \\ \variable{ECSINCLUDE} \ar[ru] & \resource{debugging\\information} & \resource{assembly\\listing}}
\seecpp\seeassembly\seemips\seeobject\seedebugging
}

\providecommand{\cppmmix}{
\toolsection{cppmmix} is a compiler for the \cpp{} programming language targeting the MMIX hardware architecture.
It generates machine code for MMIX processors from programs written in \cpp{} and stores it in corresponding object files.
For debugging purposes, it also creates a debugging information file as well as an assembly file containing a listing of the generated machine code.
The macro \texttt{\_\_mmix\_\_} is predefined in order to enable programmers to identify this tool and its target architecture while compiling.
Programs generated with this compiler require additional runtime support that is stored in the \file{cpp\-mmix\-run} library file.
\flowgraph{\resource{\cpp{}\\source code} \ar[r] & \toolbox{cppmmix} \ar[r] \ar[d] \ar[rd] & \resource{object file} \\ \variable{ECSINCLUDE} \ar[ru] & \resource{debugging\\information} & \resource{assembly\\listing}}
\seecpp\seeassembly\seemmix\seeobject\seedebugging
}

\providecommand{\cpporok}{
\toolsection{cppor1k} is a compiler for the \cpp{} programming language targeting the OpenRISC 1000 hardware architecture.
It generates machine code for OpenRISC 1000 processors from programs written in \cpp{} and stores it in corresponding object files.
For debugging purposes, it also creates a debugging information file as well as an assembly file containing a listing of the generated machine code.
The macro \texttt{\_\_or1k\_\_} is predefined in order to enable programmers to identify this tool and its target architecture while compiling.
Programs generated with this compiler require additional runtime support that is stored in the \file{cpp\-or1k\-run} library file.
\flowgraph{\resource{\cpp{}\\source code} \ar[r] & \toolbox{cppor1k} \ar[r] \ar[d] \ar[rd] & \resource{object file} \\ \variable{ECSINCLUDE} \ar[ru] & \resource{debugging\\information} & \resource{assembly\\listing}}
\seecpp\seeassembly\seeorok\seeobject\seedebugging
}

\providecommand{\cppppca}{
\toolsection{cppppc32} is a compiler for the \cpp{} programming language targeting the PowerPC hardware architecture.
It generates machine code for PowerPC processors from programs written in \cpp{} and stores it in corresponding object files.
The compiler generates machine code for the 32-bit operating mode defined by the PowerPC architecture.
For debugging purposes, it also creates a debugging information file as well as an assembly file containing a listing of the generated machine code.
The macro \texttt{\_\_ppc32\_\_} is predefined in order to enable programmers to identify this tool and its target architecture while compiling.
Programs generated with this compiler require additional runtime support that is stored in the \file{cpp\-ppc32\-run} library file.
\flowgraph{\resource{\cpp{}\\source code} \ar[r] & \toolbox{cppppc32} \ar[r] \ar[d] \ar[rd] & \resource{object file} \\ \variable{ECSINCLUDE} \ar[ru] & \resource{debugging\\information} & \resource{assembly\\listing}}
\seecpp\seeassembly\seeppc\seeobject\seedebugging
}

\providecommand{\cppppcb}{
\toolsection{cppppc64} is a compiler for the \cpp{} programming language targeting the PowerPC hardware architecture.
It generates machine code for PowerPC processors from programs written in \cpp{} and stores it in corresponding object files.
The compiler generates machine code for the 64-bit operating mode defined by the PowerPC architecture.
For debugging purposes, it also creates a debugging information file as well as an assembly file containing a listing of the generated machine code.
The macro \texttt{\_\_ppc64\_\_} is predefined in order to enable programmers to identify this tool and its target architecture while compiling.
Programs generated with this compiler require additional runtime support that is stored in the \file{cpp\-ppc64\-run} library file.
\flowgraph{\resource{\cpp{}\\source code} \ar[r] & \toolbox{cppppc64} \ar[r] \ar[d] \ar[rd] & \resource{object file} \\ \variable{ECSINCLUDE} \ar[ru] & \resource{debugging\\information} & \resource{assembly\\listing}}
\seecpp\seeassembly\seeppc\seeobject\seedebugging
}

\providecommand{\cpprisc}{
\toolsection{cpprisc} is a compiler for the \cpp{} programming language targeting the RISC hardware architecture.
It generates machine code for RISC processors from programs written in \cpp{} and stores it in corresponding object files.
For debugging purposes, it also creates a debugging information file as well as an assembly file containing a listing of the generated machine code.
The macro \texttt{\_\_risc\_\_} is predefined in order to enable programmers to identify this tool and its target architecture while compiling.
Programs generated with this compiler require additional runtime support that is stored in the \file{cpp\-risc\-run} library file.
\flowgraph{\resource{\cpp{}\\source code} \ar[r] & \toolbox{cpprisc} \ar[r] \ar[d] \ar[rd] & \resource{object file} \\ \variable{ECSINCLUDE} \ar[ru] & \resource{debugging\\information} & \resource{assembly\\listing}}
\seecpp\seeassembly\seerisc\seeobject\seedebugging
}

\providecommand{\cppwasm}{
\toolsection{cppwasm} is a compiler for the \cpp{} programming language targeting the WebAssembly architecture.
It generates machine code for WebAssembly targets from programs written in \cpp{} and stores it in corresponding object files.
For debugging purposes, it also creates a debugging information file as well as an assembly file containing a listing of the generated machine code.
The macro \texttt{\_\_wasm\_\_} is predefined in order to enable programmers to identify this tool and its target architecture while compiling.
Programs generated with this compiler require additional runtime support that is stored in the \file{cpp\-wasm\-run} library file.
\flowgraph{\resource{\cpp{}\\source code} \ar[r] & \toolbox{cppwasm} \ar[r] \ar[d] \ar[rd] & \resource{object file} \\ \variable{ECSINCLUDE} \ar[ru] & \resource{debugging\\information} & \resource{assembly\\listing}}
\seecpp\seeassembly\seewasm\seeobject\seedebugging
}

% FALSE tools

\providecommand{\falprint}{
\toolsection{falprint} is a pretty printer for the FALSE programming language.
It reformats the source code of FALSE programs and writes it to the standard output stream.
\flowgraph{\resource{FALSE\\source code} \ar[r] & \toolbox{falprint} \ar[r] & \resource{reformatted\\source code}}
\seefalse
}

\providecommand{\falcheck}{
\toolsection{falcheck} is a syntactic and semantic checker for the FALSE programming language.
It just performs syntactic and semantic checks on FALSE programs and writes its diagnostic messages to the standard error stream.
\flowgraph{\resource{FALSE\\source code} \ar[r] & \toolbox{falcheck} \ar[r] & \resource{diagnostic\\messages}}
\seefalse
}

\providecommand{\faldump}{
\toolsection{faldump} is a serializer for the FALSE programming language.
It dumps the complete internal representation of programs written in FALSE into an XML document.
\debuggingtool
\flowgraph{\resource{FALSE\\source code} \ar[r] & \toolbox{faldump} \ar[r] & \resource{internal\\representation}}
\seefalse
}

\providecommand{\falrun}{
\toolsection{falrun} is an interpreter for the FALSE programming language.
It processes and executes programs written in FALSE\@.
\flowgraph{\resource{FALSE\\source code} \ar[r] & \toolbox{falrun} \ar@/u/[r] & \resource{input/\\output} \ar@/d/[l]}
\seefalse
}

\providecommand{\falcpp}{
\toolsection{falcpp} is a transpiler for the FALSE programming language.
It translates programs written in FALSE into \cpp{} programs and stores them in corresponding source files.
\flowgraph{\resource{FALSE\\source code} \ar[r] & \toolbox{falcpp} \ar[r] & \resource{\cpp{}\\source file}}
\seefalse\seecpp
}

\providecommand{\falcode}{
\toolsection{falcode} is an intermediate code generator for the FALSE programming language.
It generates intermediate code from programs written in FALSE and stores it in corresponding assembly files.
\debuggingtool
\flowgraph{\resource{FALSE\\source code} \ar[r] & \toolbox{falcode} \ar[r] & \resource{intermediate\\code}}
\seefalse\seeassembly\seecode
}

\providecommand{\falamda}{
\toolsection{falamd16} is a compiler for the FALSE programming language targeting the AMD64 hardware architecture.
It generates machine code for AMD64 processors from programs written in FALSE and stores it in corresponding object files.
The compiler generates machine code for the 16-bit operating mode defined by the AMD64 architecture.
\flowgraph{\resource{FALSE\\source code} \ar[r] & \toolbox{falamd16} \ar[r] & \resource{object file}}
\seefalse\seeamd\seeobject
}

\providecommand{\falamdb}{
\toolsection{falamd32} is a compiler for the FALSE programming language targeting the AMD64 hardware architecture.
It generates machine code for AMD64 processors from programs written in FALSE and stores it in corresponding object files.
The compiler generates machine code for the 32-bit operating mode defined by the AMD64 architecture.
\flowgraph{\resource{FALSE\\source code} \ar[r] & \toolbox{falamd32} \ar[r] & \resource{object file}}
\seefalse\seeamd\seeobject
}

\providecommand{\falamdc}{
\toolsection{falamd64} is a compiler for the FALSE programming language targeting the AMD64 hardware architecture.
It generates machine code for AMD64 processors from programs written in FALSE and stores it in corresponding object files.
The compiler generates machine code for the 64-bit operating mode defined by the AMD64 architecture.
\flowgraph{\resource{FALSE\\source code} \ar[r] & \toolbox{falamd64} \ar[r] & \resource{object file}}
\seefalse\seeamd\seeobject
}

\providecommand{\falarma}{
\toolsection{falarma32} is a compiler for the FALSE programming language targeting the ARM hardware architecture.
It generates machine code for ARM processors executing A32 instructions from programs written in FALSE and stores it in corresponding object files.
\flowgraph{\resource{FALSE\\source code} \ar[r] & \toolbox{falarma32} \ar[r] & \resource{object file}}
\seefalse\seearm\seeobject
}

\providecommand{\falarmb}{
\toolsection{falarma64} is a compiler for the FALSE programming language targeting the ARM hardware architecture.
It generates machine code for ARM processors executing A64 instructions from programs written in FALSE and stores it in corresponding object files.
\flowgraph{\resource{FALSE\\source code} \ar[r] & \toolbox{falarma64} \ar[r] & \resource{object file}}
\seefalse\seearm\seeobject
}

\providecommand{\falarmc}{
\toolsection{falarmt32} is a compiler for the FALSE programming language targeting the ARM hardware architecture.
It generates machine code for ARM processors without floating-point extension executing T32 instructions from programs written in FALSE and stores it in corresponding object files.
\flowgraph{\resource{FALSE\\source code} \ar[r] & \toolbox{falarmt32} \ar[r] & \resource{object file}}
\seefalse\seearm\seeobject
}

\providecommand{\falarmcfpe}{
\toolsection{falarmt32fpe} is a compiler for the FALSE programming language targeting the ARM hardware architecture.
It generates machine code for ARM processors with floating-point extension executing T32 instructions from programs written in FALSE and stores it in corresponding object files.
\flowgraph{\resource{FALSE\\source code} \ar[r] & \toolbox{falarmt32fpe} \ar[r] & \resource{object file}}
\seefalse\seearm\seeobject
}

\providecommand{\falavr}{
\toolsection{falavr} is a compiler for the FALSE programming language targeting the AVR hardware architecture.
It generates machine code for AVR processors from programs written in FALSE and stores it in corresponding object files.
\flowgraph{\resource{FALSE\\source code} \ar[r] & \toolbox{falavr} \ar[r] & \resource{object file}}
\seefalse\seeavr\seeobject
}

\providecommand{\falavrtt}{
\toolsection{falavr32} is a compiler for the FALSE programming language targeting the AVR32 hardware architecture.
It generates machine code for AVR32 processors from programs written in FALSE and stores it in corresponding object files.
\flowgraph{\resource{FALSE\\source code} \ar[r] & \toolbox{falavr32} \ar[r] & \resource{object file}}
\seefalse\seeavrtt\seeobject
}

\providecommand{\falmabk}{
\toolsection{falm68k} is a compiler for the FALSE programming language targeting the M68000 hardware architecture.
It generates machine code for M68000 processors from programs written in FALSE and stores it in corresponding object files.
\flowgraph{\resource{FALSE\\source code} \ar[r] & \toolbox{falm68k} \ar[r] & \resource{object file}}
\seefalse\seemabk\seeobject
}

\providecommand{\falmibl}{
\toolsection{falmibl} is a compiler for the FALSE programming language targeting the MicroBlaze hardware architecture.
It generates machine code for MicroBlaze processors from programs written in FALSE and stores it in corresponding object files.
\flowgraph{\resource{FALSE\\source code} \ar[r] & \toolbox{falmibl} \ar[r] & \resource{object file}}
\seefalse\seemibl\seeobject
}

\providecommand{\falmipsa}{
\toolsection{falmips32} is a compiler for the FALSE programming language targeting the MIPS32 hardware architecture.
It generates machine code for MIPS32 processors from programs written in FALSE and stores it in corresponding object files.
\flowgraph{\resource{FALSE\\source code} \ar[r] & \toolbox{falmips32} \ar[r] & \resource{object file}}
\seefalse\seemips\seeobject
}

\providecommand{\falmipsb}{
\toolsection{falmips64} is a compiler for the FALSE programming language targeting the MIPS64 hardware architecture.
It generates machine code for MIPS64 processors from programs written in FALSE and stores it in corresponding object files.
\flowgraph{\resource{FALSE\\source code} \ar[r] & \toolbox{falmips64} \ar[r] & \resource{object file}}
\seefalse\seemips\seeobject
}

\providecommand{\falmmix}{
\toolsection{falmmix} is a compiler for the FALSE programming language targeting the MMIX hardware architecture.
It generates machine code for MMIX processors from programs written in FALSE and stores it in corresponding object files.
\flowgraph{\resource{FALSE\\source code} \ar[r] & \toolbox{falmmix} \ar[r] & \resource{object file}}
\seefalse\seemmix\seeobject
}

\providecommand{\falorok}{
\toolsection{falor1k} is a compiler for the FALSE programming language targeting the OpenRISC 1000 hardware architecture.
It generates machine code for OpenRISC 1000 processors from programs written in FALSE and stores it in corresponding object files.
\flowgraph{\resource{FALSE\\source code} \ar[r] & \toolbox{falor1k} \ar[r] & \resource{object file}}
\seefalse\seeorok\seeobject
}

\providecommand{\falppca}{
\toolsection{falppc32} is a compiler for the FALSE programming language targeting the PowerPC hardware architecture.
It generates machine code for PowerPC processors from programs written in FALSE and stores it in corresponding object files.
The compiler generates machine code for the 32-bit operating mode defined by the PowerPC architecture.
\flowgraph{\resource{FALSE\\source code} \ar[r] & \toolbox{falppc32} \ar[r] & \resource{object file}}
\seefalse\seeppc\seeobject
}

\providecommand{\falppcb}{
\toolsection{falppc64} is a compiler for the FALSE programming language targeting the PowerPC hardware architecture.
It generates machine code for PowerPC processors from programs written in FALSE and stores it in corresponding object files.
The compiler generates machine code for the 64-bit operating mode defined by the PowerPC architecture.
\flowgraph{\resource{FALSE\\source code} \ar[r] & \toolbox{falppc64} \ar[r] & \resource{object file}}
\seefalse\seeppc\seeobject
}

\providecommand{\falrisc}{
\toolsection{falrisc} is a compiler for the FALSE programming language targeting the RISC hardware architecture.
It generates machine code for RISC processors from programs written in FALSE and stores it in corresponding object files.
\flowgraph{\resource{FALSE\\source code} \ar[r] & \toolbox{falrisc} \ar[r] & \resource{object file}}
\seefalse\seerisc\seeobject
}

\providecommand{\falwasm}{
\toolsection{falwasm} is a compiler for the FALSE programming language targeting the WebAssembly architecture.
It generates machine code for WebAssembly targets from programs written in FALSE and stores it in corresponding object files.
\flowgraph{\resource{FALSE\\source code} \ar[r] & \toolbox{falwasm} \ar[r] & \resource{object file}}
\seefalse\seewasm\seeobject
}

% Oberon tools

\providecommand{\obprint}{
\toolsection{obprint} is a pretty printer for the Oberon programming language.
It reformats the source code of Oberon modules and writes it to the standard output stream.
\flowgraph{\resource{Oberon\\source code} \ar[r] & \toolbox{obprint} \ar[r] & \resource{reformatted\\source code}}
\seeoberon
}

\providecommand{\obcheck}{
\toolsection{obcheck} is a syntactic and semantic checker for the Oberon programming language.
It just performs syntactic and semantic checks on Oberon modules and writes its diagnostic messages to the standard error stream.
In addition, it stores the interface of each module in a symbol file which is required when other modules import the module.
\flowgraph{\resource{Oberon\\source code} \ar[r] & \toolbox{obcheck} \ar[r] \ar@/l/[d] & \resource{diagnostic\\messages} \\ \variable{ECSIMPORT} \ar[ru] & \resource{symbol\\files} \ar@/r/[u]}
\seeoberon
}

\providecommand{\obdump}{
\toolsection{obdump} is a serializer for the Oberon programming language.
It dumps the complete internal representation of modules written in Oberon into an XML document.
\debuggingtool
\flowgraph{\resource{Oberon\\source code} \ar[r] & \toolbox{obdump} \ar[r] \ar@/l/[d] & \resource{internal\\representation} \\ \variable{ECSIMPORT} \ar[ru] & \resource{symbol\\files} \ar@/r/[u]}
\seeoberon
}

\providecommand{\obrun}{
\toolsection{obrun} is an interpreter for the Oberon programming language.
It processes and executes modules written in Oberon.
This tool does neither generate nor process symbol files while interpreting modules.
If a module is imported by another one, its filename has to be named before the other one in the list of command-line arguments.
\flowgraph{\resource{Oberon\\source code} \ar[r] & \toolbox{obrun} \ar@/u/[r] & \resource{input/\\output} \ar@/d/[l]}
\seeoberon
}

\providecommand{\obcpp}{
\toolsection{obcpp} is a transpiler for the Oberon programming language.
It translates programs written in Oberon into \cpp{} programs and stores them in corresponding source and header files.
In addition, it stores the interface of each module in a symbol file which is required when other modules import the module.
The same interface is provided by the generated header file which can be used in other parts of the \cpp{} program.
\flowgraph{\resource{Oberon\\source code} \ar[r] & \toolbox{obcpp} \ar[r] \ar@/l/[d] \ar[rd] & \resource{\cpp{}\\source file} \\ \variable{ECSIMPORT} \ar[ru] & \resource{symbol\\files} \ar@/r/[u] & \resource{\cpp{}\\header file}}
\seeoberon\seecpp
}

\providecommand{\obdoc}{
\toolsection{obdoc} is a generic documentation generator for the Oberon programming language.
It processes several Oberon modules and assembles all information therein into a generic documentation.
In addition, it stores the interface of each module in a symbol file which is required when other modules import the module.
\debuggingtool
\flowgraph{\resource{Oberon\\source code} \ar[r] & \toolbox{obdoc} \ar[r] \ar@/l/[d] & \resource{generic\\documentation} \\ \variable{ECSIMPORT} \ar[ru] & \resource{symbol\\files} \ar@/r/[u]}
\seeoberon\seedocumentation
}

\providecommand{\obhtml}{
\toolsection{obhtml} is an HTML documentation generator for the Oberon programming language.
It processes several Oberon modules and assembles all information therein into an HTML document.
In addition, it stores the interface of each module in a symbol file which is required when other modules import the module.
\flowgraph{\resource{Oberon\\source code} \ar[r] & \toolbox{obhtml} \ar[r] \ar@/l/[d] & \resource{HTML\\document} \\ \variable{ECSIMPORT} \ar[ru] & \resource{symbol\\files} \ar@/r/[u]}
\seeoberon\seedocumentation
}

\providecommand{\oblatex}{
\toolsection{oblatex} is a Latex documentation generator for the Oberon programming language.
It processes several Oberon modules and assembles all information therein into a Latex document.
In addition, it stores the interface of each module in a symbol file which is required when other modules import the module.
\flowgraph{\resource{Oberon\\source code} \ar[r] & \toolbox{oblatex} \ar[r] \ar@/l/[d] & \resource{Latex\\document} \\ \variable{ECSIMPORT} \ar[ru] & \resource{symbol\\files} \ar@/r/[u]}
\seeoberon\seedocumentation
}

\providecommand{\obcode}{
\toolsection{obcode} is an intermediate code generator for the Oberon programming language.
It generates intermediate code from modules written in Oberon and stores it in corresponding assembly files.
In addition, it stores the interface of each module in a symbol file which is required when other modules import the module.
Programs generated with this tool require additional runtime support that is stored in the \file{ob\-code\-run} library file.
\debuggingtool
\flowgraph{\resource{Oberon\\source code} \ar[r] & \toolbox{obcode} \ar[r] \ar@/l/[d] & \resource{intermediate\\code} \\ \variable{ECSIMPORT} \ar[ru] & \resource{symbol\\files} \ar@/r/[u]}
\seeoberon\seeassembly\seecode
}

\providecommand{\obamda}{
\toolsection{obamd16} is a compiler for the Oberon programming language targeting the AMD64 hardware architecture.
It generates machine code for AMD64 processors from modules written in Oberon and stores it in corresponding object files.
The compiler generates machine code for the 16-bit operating mode defined by the AMD64 architecture.
For debugging purposes, it also creates a debugging information file as well as an assembly file containing a listing of the generated machine code.
In addition, it stores the interface of each module in a symbol file which is required when other modules import the module.
Programs generated with this compiler require additional runtime support that is stored in the \file{ob\-amd16\-run} library file.
\flowgraph{\resource{Oberon\\source code} \ar[r] & \toolbox{obamd16} \ar[r] \ar@/l/[d] \ar[rd] & \resource{object file} \\ \variable{ECSIMPORT} \ar[ru] & \resource{symbol\\files} \ar@/r/[u] & \resource{debugging\\information}}
\seeoberon\seeassembly\seeamd\seeobject\seedebugging
}

\providecommand{\obamdb}{
\toolsection{obamd32} is a compiler for the Oberon programming language targeting the AMD64 hardware architecture.
It generates machine code for AMD64 processors from modules written in Oberon and stores it in corresponding object files.
The compiler generates machine code for the 32-bit operating mode defined by the AMD64 architecture.
For debugging purposes, it also creates a debugging information file as well as an assembly file containing a listing of the generated machine code.
In addition, it stores the interface of each module in a symbol file which is required when other modules import the module.
Programs generated with this compiler require additional runtime support that is stored in the \file{ob\-amd32\-run} library file.
\flowgraph{\resource{Oberon\\source code} \ar[r] & \toolbox{obamd32} \ar[r] \ar@/l/[d] \ar[rd] & \resource{object file} \\ \variable{ECSIMPORT} \ar[ru] & \resource{symbol\\files} \ar@/r/[u] & \resource{debugging\\information}}
\seeoberon\seeassembly\seeamd\seeobject\seedebugging
}

\providecommand{\obamdc}{
\toolsection{obamd64} is a compiler for the Oberon programming language targeting the AMD64 hardware architecture.
It generates machine code for AMD64 processors from modules written in Oberon and stores it in corresponding object files.
The compiler generates machine code for the 64-bit operating mode defined by the AMD64 architecture.
For debugging purposes, it also creates a debugging information file as well as an assembly file containing a listing of the generated machine code.
In addition, it stores the interface of each module in a symbol file which is required when other modules import the module.
Programs generated with this compiler require additional runtime support that is stored in the \file{ob\-amd64\-run} library file.
\flowgraph{\resource{Oberon\\source code} \ar[r] & \toolbox{obamd64} \ar[r] \ar@/l/[d] \ar[rd] & \resource{object file} \\ \variable{ECSIMPORT} \ar[ru] & \resource{symbol\\files} \ar@/r/[u] & \resource{debugging\\information}}
\seeoberon\seeassembly\seeamd\seeobject\seedebugging
}

\providecommand{\obarma}{
\toolsection{obarma32} is a compiler for the Oberon programming language targeting the ARM hardware architecture.
It generates machine code for ARM processors executing A32 instructions from modules written in Oberon and stores it in corresponding object files.
For debugging purposes, it also creates a debugging information file as well as an assembly file containing a listing of the generated machine code.
In addition, it stores the interface of each module in a symbol file which is required when other modules import the module.
Programs generated with this compiler require additional runtime support that is stored in the \file{ob\-arma32\-run} library file.
\flowgraph{\resource{Oberon\\source code} \ar[r] & \toolbox{obarma32} \ar[r] \ar@/l/[d] \ar[rd] & \resource{object file} \\ \variable{ECSIMPORT} \ar[ru] & \resource{symbol\\files} \ar@/r/[u] & \resource{debugging\\information}}
\seeoberon\seeassembly\seearm\seeobject\seedebugging
}

\providecommand{\obarmb}{
\toolsection{obarma64} is a compiler for the Oberon programming language targeting the ARM hardware architecture.
It generates machine code for ARM processors executing A64 instructions from modules written in Oberon and stores it in corresponding object files.
For debugging purposes, it also creates a debugging information file as well as an assembly file containing a listing of the generated machine code.
In addition, it stores the interface of each module in a symbol file which is required when other modules import the module.
Programs generated with this compiler require additional runtime support that is stored in the \file{ob\-arma64\-run} library file.
\flowgraph{\resource{Oberon\\source code} \ar[r] & \toolbox{obarma64} \ar[r] \ar@/l/[d] \ar[rd] & \resource{object file} \\ \variable{ECSIMPORT} \ar[ru] & \resource{symbol\\files} \ar@/r/[u] & \resource{debugging\\information}}
\seeoberon\seeassembly\seearm\seeobject\seedebugging
}

\providecommand{\obarmc}{
\toolsection{obarmt32} is a compiler for the Oberon programming language targeting the ARM hardware architecture.
It generates machine code for ARM processors without floating-point extension executing T32 instructions from modules written in Oberon and stores it in corresponding object files.
For debugging purposes, it also creates a debugging information file as well as an assembly file containing a listing of the generated machine code.
In addition, it stores the interface of each module in a symbol file which is required when other modules import the module.
Programs generated with this compiler require additional runtime support that is stored in the \file{ob\-armt32\-run} library file.
\flowgraph{\resource{Oberon\\source code} \ar[r] & \toolbox{obarmt32} \ar[r] \ar@/l/[d] \ar[rd] & \resource{object file} \\ \variable{ECSIMPORT} \ar[ru] & \resource{symbol\\files} \ar@/r/[u] & \resource{debugging\\information}}
\seeoberon\seeassembly\seearm\seeobject\seedebugging
}

\providecommand{\obarmcfpe}{
\toolsection{obarmt32fpe} is a compiler for the Oberon programming language targeting the ARM hardware architecture.
It generates machine code for ARM processors with floating-point extension executing T32 instructions from modules written in Oberon and stores it in corresponding object files.
For debugging purposes, it also creates a debugging information file as well as an assembly file containing a listing of the generated machine code.
In addition, it stores the interface of each module in a symbol file which is required when other modules import the module.
Programs generated with this compiler require additional runtime support that is stored in the \file{ob\-armt32\-fpe\-run} library file.
\flowgraph{\resource{Oberon\\source code} \ar[r] & \toolbox{obarmt32fpe} \ar[r] \ar@/l/[d] \ar[rd] & \resource{object file} \\ \variable{ECSIMPORT} \ar[ru] & \resource{symbol\\files} \ar@/r/[u] & \resource{debugging\\information}}
\seeoberon\seeassembly\seearm\seeobject\seedebugging
}

\providecommand{\obavr}{
\toolsection{obavr} is a compiler for the Oberon programming language targeting the AVR hardware architecture.
It generates machine code for AVR processors from modules written in Oberon and stores it in corresponding object files.
For debugging purposes, it also creates a debugging information file as well as an assembly file containing a listing of the generated machine code.
In addition, it stores the interface of each module in a symbol file which is required when other modules import the module.
Programs generated with this compiler require additional runtime support that is stored in the \file{ob\-avr\-run} library file.
\flowgraph{\resource{Oberon\\source code} \ar[r] & \toolbox{obavr} \ar[r] \ar@/l/[d] \ar[rd] & \resource{object file} \\ \variable{ECSIMPORT} \ar[ru] & \resource{symbol\\files} \ar@/r/[u] & \resource{debugging\\information}}
\seeoberon\seeassembly\seeavr\seeobject\seedebugging
}

\providecommand{\obavrtt}{
\toolsection{obavr32} is a compiler for the Oberon programming language targeting the AVR32 hardware architecture.
It generates machine code for AVR32 processors from modules written in Oberon and stores it in corresponding object files.
For debugging purposes, it also creates a debugging information file as well as an assembly file containing a listing of the generated machine code.
In addition, it stores the interface of each module in a symbol file which is required when other modules import the module.
Programs generated with this compiler require additional runtime support that is stored in the \file{ob\-avr32\-run} library file.
\flowgraph{\resource{Oberon\\source code} \ar[r] & \toolbox{obavr32} \ar[r] \ar@/l/[d] \ar[rd] & \resource{object file} \\ \variable{ECSIMPORT} \ar[ru] & \resource{symbol\\files} \ar@/r/[u] & \resource{debugging\\information}}
\seeoberon\seeassembly\seeavrtt\seeobject\seedebugging
}

\providecommand{\obmabk}{
\toolsection{obm68k} is a compiler for the Oberon programming language targeting the M68000 hardware architecture.
It generates machine code for M68000 processors from modules written in Oberon and stores it in corresponding object files.
For debugging purposes, it also creates a debugging information file as well as an assembly file containing a listing of the generated machine code.
In addition, it stores the interface of each module in a symbol file which is required when other modules import the module.
Programs generated with this compiler require additional runtime support that is stored in the \file{ob\-m68k\-run} library file.
\flowgraph{\resource{Oberon\\source code} \ar[r] & \toolbox{obm68k} \ar[r] \ar@/l/[d] \ar[rd] & \resource{object file} \\ \variable{ECSIMPORT} \ar[ru] & \resource{symbol\\files} \ar@/r/[u] & \resource{debugging\\information}}
\seeoberon\seeassembly\seemabk\seeobject\seedebugging
}

\providecommand{\obmibl}{
\toolsection{obmibl} is a compiler for the Oberon programming language targeting the MicroBlaze hardware architecture.
It generates machine code for MicroBlaze processors from modules written in Oberon and stores it in corresponding object files.
For debugging purposes, it also creates a debugging information file as well as an assembly file containing a listing of the generated machine code.
In addition, it stores the interface of each module in a symbol file which is required when other modules import the module.
Programs generated with this compiler require additional runtime support that is stored in the \file{ob\-mibl\-run} library file.
\flowgraph{\resource{Oberon\\source code} \ar[r] & \toolbox{obmibl} \ar[r] \ar@/l/[d] \ar[rd] & \resource{object file} \\ \variable{ECSIMPORT} \ar[ru] & \resource{symbol\\files} \ar@/r/[u] & \resource{debugging\\information}}
\seeoberon\seeassembly\seemibl\seeobject\seedebugging
}

\providecommand{\obmipsa}{
\toolsection{obmips32} is a compiler for the Oberon programming language targeting the MIPS32 hardware architecture.
It generates machine code for MIPS32 processors from modules written in Oberon and stores it in corresponding object files.
For debugging purposes, it also creates a debugging information file as well as an assembly file containing a listing of the generated machine code.
In addition, it stores the interface of each module in a symbol file which is required when other modules import the module.
Programs generated with this compiler require additional runtime support that is stored in the \file{ob\-mips32\-run} library file.
\flowgraph{\resource{Oberon\\source code} \ar[r] & \toolbox{obmips32} \ar[r] \ar@/l/[d] \ar[rd] & \resource{object file} \\ \variable{ECSIMPORT} \ar[ru] & \resource{symbol\\files} \ar@/r/[u] & \resource{debugging\\information}}
\seeoberon\seeassembly\seemips\seeobject\seedebugging
}

\providecommand{\obmipsb}{
\toolsection{obmips64} is a compiler for the Oberon programming language targeting the MIPS64 hardware architecture.
It generates machine code for MIPS64 processors from modules written in Oberon and stores it in corresponding object files.
For debugging purposes, it also creates a debugging information file as well as an assembly file containing a listing of the generated machine code.
In addition, it stores the interface of each module in a symbol file which is required when other modules import the module.
Programs generated with this compiler require additional runtime support that is stored in the \file{ob\-mips64\-run} library file.
\flowgraph{\resource{Oberon\\source code} \ar[r] & \toolbox{obmips64} \ar[r] \ar@/l/[d] \ar[rd] & \resource{object file} \\ \variable{ECSIMPORT} \ar[ru] & \resource{symbol\\files} \ar@/r/[u] & \resource{debugging\\information}}
\seeoberon\seeassembly\seemips\seeobject\seedebugging
}

\providecommand{\obmmix}{
\toolsection{obmmix} is a compiler for the Oberon programming language targeting the MMIX hardware architecture.
It generates machine code for MMIX processors from modules written in Oberon and stores it in corresponding object files.
For debugging purposes, it also creates a debugging information file as well as an assembly file containing a listing of the generated machine code.
In addition, it stores the interface of each module in a symbol file which is required when other modules import the module.
Programs generated with this compiler require additional runtime support that is stored in the \file{ob\-mmix\-run} library file.
\flowgraph{\resource{Oberon\\source code} \ar[r] & \toolbox{obmmix} \ar[r] \ar@/l/[d] \ar[rd] & \resource{object file} \\ \variable{ECSIMPORT} \ar[ru] & \resource{symbol\\files} \ar@/r/[u] & \resource{debugging\\information}}
\seeoberon\seeassembly\seemmix\seeobject\seedebugging
}

\providecommand{\oborok}{
\toolsection{obor1k} is a compiler for the Oberon programming language targeting the OpenRISC 1000 hardware architecture.
It generates machine code for OpenRISC 1000 processors from modules written in Oberon and stores it in corresponding object files.
For debugging purposes, it also creates a debugging information file as well as an assembly file containing a listing of the generated machine code.
In addition, it stores the interface of each module in a symbol file which is required when other modules import the module.
Programs generated with this compiler require additional runtime support that is stored in the \file{ob\-or1k\-run} library file.
\flowgraph{\resource{Oberon\\source code} \ar[r] & \toolbox{obor1k} \ar[r] \ar@/l/[d] \ar[rd] & \resource{object file} \\ \variable{ECSIMPORT} \ar[ru] & \resource{symbol\\files} \ar@/r/[u] & \resource{debugging\\information}}
\seeoberon\seeassembly\seeorok\seeobject\seedebugging
}

\providecommand{\obppca}{
\toolsection{obppc32} is a compiler for the Oberon programming language targeting the PowerPC hardware architecture.
It generates machine code for PowerPC processors from modules written in Oberon and stores it in corresponding object files.
The compiler generates machine code for the 32-bit operating mode defined by the PowerPC architecture.
For debugging purposes, it also creates a debugging information file as well as an assembly file containing a listing of the generated machine code.
In addition, it stores the interface of each module in a symbol file which is required when other modules import the module.
Programs generated with this compiler require additional runtime support that is stored in the \file{ob\-ppc32\-run} library file.
\flowgraph{\resource{Oberon\\source code} \ar[r] & \toolbox{obppc32} \ar[r] \ar@/l/[d] \ar[rd] & \resource{object file} \\ \variable{ECSIMPORT} \ar[ru] & \resource{symbol\\files} \ar@/r/[u] & \resource{debugging\\information}}
\seeoberon\seeassembly\seeppc\seeobject\seedebugging
}

\providecommand{\obppcb}{
\toolsection{obppc64} is a compiler for the Oberon programming language targeting the PowerPC hardware architecture.
It generates machine code for PowerPC processors from modules written in Oberon and stores it in corresponding object files.
The compiler generates machine code for the 64-bit operating mode defined by the PowerPC architecture.
For debugging purposes, it also creates a debugging information file as well as an assembly file containing a listing of the generated machine code.
In addition, it stores the interface of each module in a symbol file which is required when other modules import the module.
Programs generated with this compiler require additional runtime support that is stored in the \file{ob\-ppc64\-run} library file.
\flowgraph{\resource{Oberon\\source code} \ar[r] & \toolbox{obppc64} \ar[r] \ar@/l/[d] \ar[rd] & \resource{object file} \\ \variable{ECSIMPORT} \ar[ru] & \resource{symbol\\files} \ar@/r/[u] & \resource{debugging\\information}}
\seeoberon\seeassembly\seeppc\seeobject\seedebugging
}

\providecommand{\obrisc}{
\toolsection{obrisc} is a compiler for the Oberon programming language targeting the RISC hardware architecture.
It generates machine code for RISC processors from modules written in Oberon and stores it in corresponding object files.
For debugging purposes, it also creates a debugging information file as well as an assembly file containing a listing of the generated machine code.
In addition, it stores the interface of each module in a symbol file which is required when other modules import the module.
Programs generated with this compiler require additional runtime support that is stored in the \file{ob\-risc\-run} library file.
\flowgraph{\resource{Oberon\\source code} \ar[r] & \toolbox{obrisc} \ar[r] \ar@/l/[d] \ar[rd] & \resource{object file} \\ \variable{ECSIMPORT} \ar[ru] & \resource{symbol\\files} \ar@/r/[u] & \resource{debugging\\information}}
\seeoberon\seeassembly\seerisc\seeobject\seedebugging
}

\providecommand{\obwasm}{
\toolsection{obwasm} is a compiler for the Oberon programming language targeting the WebAssembly architecture.
It generates machine code for WebAssembly targets from modules written in Oberon and stores it in corresponding object files.
For debugging purposes, it also creates a debugging information file as well as an assembly file containing a listing of the generated machine code.
In addition, it stores the interface of each module in a symbol file which is required when other modules import the module.
Programs generated with this compiler require additional runtime support that is stored in the \file{ob\-wasm\-run} library file.
\flowgraph{\resource{Oberon\\source code} \ar[r] & \toolbox{obwasm} \ar[r] \ar@/l/[d] \ar[rd] & \resource{object file} \\ \variable{ECSIMPORT} \ar[ru] & \resource{symbol\\files} \ar@/r/[u] & \resource{debugging\\information}}
\seeoberon\seeassembly\seewasm\seeobject\seedebugging
}

% converter tools

\providecommand{\dbgdwarf}{
\toolsection{dbgdwarf} is a DWARF debugging information converter tool.
It converts debugging information into the DWARF debugging data format and stores it in corresponding object files~\cite{dwarffile}.
The resulting debugging object files can be combined with runtime support that creates Executable and Linking Format (ELF) files~\cite{elffile}.
\flowgraph{\resource{debugging\\information} \ar[r] & \toolbox{dbgdwarf} \ar[r] & \resource{debugging\\object file}}
\seeobject\seedebugging
}

% assembler tools

\providecommand{\asmprint}{
\toolsection{asmprint} is a pretty printer for generic assembly code.
It reformats generic assembly code and writes it to the standard output stream.
\flowgraph{\resource{generic assembly\\source code} \ar[r] & \toolbox{asmprint} \ar[r] & \resource{reformatted\\source code}}
\seeassembly
}

\providecommand{\amdaasm}{
\toolsection{amd16asm} is an assembler for the AMD64 hardware architecture.
It translates assembly code into machine code for AMD64 processors and stores it in corresponding object files.
By default, the assembler generates machine code for the 16-bit operating mode defined by the AMD64 architecture.
\flowgraph{\resource{AMD16 assembly\\source code} \ar[r] & \toolbox{amd16asm} \ar[r] & \resource{object file}}
\seeassembly\seeamd\seeobject
}

\providecommand{\amdadism}{
\toolsection{amd16dism} is a disassembler for the AMD64 hardware architecture.
It translates machine code from object files targeting AMD64 processors into assembly code and writes it to the standard output stream.
It assumes that the machine code was generated for the 16-bit operating mode defined by the AMD64 architecture.
\flowgraph{\resource{object file} \ar[r] & \toolbox{amd16dism} \ar[r] & \resource{disassembly\\listing}}
\seeassembly\seeamd\seeobject
}

\providecommand{\amdbasm}{
\toolsection{amd32asm} is an assembler for the AMD64 hardware architecture.
It translates assembly code into machine code for AMD64 processors and stores it in corresponding object files.
By default, the assembler generates machine code for the 32-bit operating mode defined by the AMD64 architecture.
\flowgraph{\resource{AMD32 assembly\\source code} \ar[r] & \toolbox{amd32asm} \ar[r] & \resource{object file}}
\seeassembly\seeamd\seeobject
}

\providecommand{\amdbdism}{
\toolsection{amd32dism} is a disassembler for the AMD64 hardware architecture.
It translates machine code from object files targeting AMD64 processors into assembly code and writes it to the standard output stream.
It assumes that the machine code was generated for the 32-bit operating mode defined by the AMD64 architecture.
\flowgraph{\resource{object file} \ar[r] & \toolbox{amd32dism} \ar[r] & \resource{disassembly\\listing}}
\seeassembly\seeamd\seeobject
}

\providecommand{\amdcasm}{
\toolsection{amd64asm} is an assembler for the AMD64 hardware architecture.
It translates assembly code into machine code for AMD64 processors and stores it in corresponding object files.
By default, the assembler generates machine code for the 64-bit operating mode defined by the AMD64 architecture.
\flowgraph{\resource{AMD64 assembly\\source code} \ar[r] & \toolbox{amd64asm} \ar[r] & \resource{object file}}
\seeassembly\seeamd\seeobject
}

\providecommand{\amdcdism}{
\toolsection{amd64dism} is a disassembler for the AMD64 hardware architecture.
It translates machine code from object files targeting AMD64 processors into assembly code and writes it to the standard output stream.
It assumes that the machine code was generated for the 64-bit operating mode defined by the AMD64 architecture.
\flowgraph{\resource{object file} \ar[r] & \toolbox{amd64dism} \ar[r] & \resource{disassembly\\listing}}
\seeassembly\seeamd\seeobject
}

\providecommand{\armaasm}{
\toolsection{arma32asm} is an assembler for the ARM hardware architecture.
It translates assembly code into machine code for ARM processors executing A32 instructions and stores it in corresponding object files.
\flowgraph{\resource{ARM A32 assembly\\source code} \ar[r] & \toolbox{arma32asm} \ar[r] & \resource{object file}}
\seeassembly\seearm\seeobject
}

\providecommand{\armadism}{
\toolsection{arma32dism} is a disassembler for the ARM hardware architecture.
It translates machine code from object files targeting ARM processors executing A32 instructions into assembly code and writes it to the standard output stream.
\flowgraph{\resource{object file} \ar[r] & \toolbox{arma32dism} \ar[r] & \resource{disassembly\\listing}}
\seeassembly\seearm\seeobject
}

\providecommand{\armbasm}{
\toolsection{arma64asm} is an assembler for the ARM hardware architecture.
It translates assembly code into machine code for ARM processors executing A64 instructions and stores it in corresponding object files.
\flowgraph{\resource{ARM A64 assembly\\source code} \ar[r] & \toolbox{arma64asm} \ar[r] & \resource{object file}}
\seeassembly\seearm\seeobject
}

\providecommand{\armbdism}{
\toolsection{arma64dism} is a disassembler for the ARM hardware architecture.
It translates machine code from object files targeting ARM processors executing A64 instructions into assembly code and writes it to the standard output stream.
\flowgraph{\resource{object file} \ar[r] & \toolbox{arma64dism} \ar[r] & \resource{disassembly\\listing}}
\seeassembly\seearm\seeobject
}

\providecommand{\armcasm}{
\toolsection{armt32asm} is an assembler for the ARM hardware architecture.
It translates assembly code into machine code for ARM processors executing T32 instructions and stores it in corresponding object files.
\flowgraph{\resource{ARM T32 assembly\\source code} \ar[r] & \toolbox{armt32asm} \ar[r] & \resource{object file}}
\seeassembly\seearm\seeobject
}

\providecommand{\armcdism}{
\toolsection{armt32dism} is a disassembler for the ARM hardware architecture.
It translates machine code from object files targeting ARM processors executing T32 instructions into assembly code and writes it to the standard output stream.
\flowgraph{\resource{object file} \ar[r] & \toolbox{armt32dism} \ar[r] & \resource{disassembly\\listing}}
\seeassembly\seearm\seeobject
}

\providecommand{\avrasm}{
\toolsection{avrasm} is an assembler for the AVR hardware architecture.
It translates assembly code into machine code for AVR processors and stores it in corresponding object files.
The identifiers \texttt{RXL}, \texttt{RXH}, \texttt{RYL}, \texttt{RYH}, \texttt{RZL}, and \texttt{RZH} are predefined and name the corresponding registers.
The identifiers \texttt{SPL} and \texttt{SPH} are also predefined and evaluate to the address of the corresponding registers.
\flowgraph{\resource{AVR assembly\\source code} \ar[r] & \toolbox{avrasm} \ar[r] & \resource{object file}}
\seeassembly\seeavr\seeobject
}

\providecommand{\avrdism}{
\toolsection{avrdism} is a disassembler for the AVR hardware architecture.
It translates machine code from object files targeting AVR processors into assembly code and writes it to the standard output stream.
\flowgraph{\resource{object file} \ar[r] & \toolbox{avrdism} \ar[r] & \resource{disassembly\\listing}}
\seeassembly\seeavr\seeobject
}

\providecommand{\avrttasm}{
\toolsection{avr32asm} is an assembler for the AVR32 hardware architecture.
It translates assembly code into machine code for AVR32 processors and stores it in corresponding object files.
\flowgraph{\resource{AVR32 assembly\\source code} \ar[r] & \toolbox{avr32asm} \ar[r] & \resource{object file}}
\seeassembly\seeavrtt\seeobject
}

\providecommand{\avrttdism}{
\toolsection{avr32dism} is a disassembler for the AVR32 hardware architecture.
It translates machine code from object files targeting AVR32 processors into assembly code and writes it to the standard output stream.
\flowgraph{\resource{object file} \ar[r] & \toolbox{avr32dism} \ar[r] & \resource{disassembly\\listing}}
\seeassembly\seeavrtt\seeobject
}

\providecommand{\mabkasm}{
\toolsection{m68kasm} is an assembler for the M68000 hardware architecture.
It translates assembly code into machine code for M68000 processors and stores it in corresponding object files.
\flowgraph{\resource{68000 assembly\\source code} \ar[r] & \toolbox{m68kasm} \ar[r] & \resource{object file}}
\seeassembly\seemabk\seeobject
}

\providecommand{\mabkdism}{
\toolsection{m68kdism} is a disassembler for the M68000 hardware architecture.
It translates machine code from object files targeting M68000 processors into assembly code and writes it to the standard output stream.
\flowgraph{\resource{object file} \ar[r] & \toolbox{m68kdism} \ar[r] & \resource{disassembly\\listing}}
\seeassembly\seemabk\seeobject
}

\providecommand{\miblasm}{
\toolsection{miblasm} is an assembler for the MicroBlaze hardware architecture.
It translates assembly code into machine code for MicroBlaze processors and stores it in corresponding object files.
\flowgraph{\resource{MicroBlaze assembly\\source code} \ar[r] & \toolbox{miblasm} \ar[r] & \resource{object file}}
\seeassembly\seemibl\seeobject
}

\providecommand{\mibldism}{
\toolsection{mibldism} is a disassembler for the MicroBlaze hardware architecture.
It translates machine code from object files targeting MicroBlaze processors into assembly code and writes it to the standard output stream.
\flowgraph{\resource{object file} \ar[r] & \toolbox{mibldism} \ar[r] & \resource{disassembly\\listing}}
\seeassembly\seemibl\seeobject
}

\providecommand{\mipsaasm}{
\toolsection{mips32asm} is an assembler for the MIPS32 hardware architecture.
It translates assembly code into machine code for MIPS32 processors and stores it in corresponding object files.
\flowgraph{\resource{MIPS32 assembly\\source code} \ar[r] & \toolbox{mips32asm} \ar[r] & \resource{object file}}
\seeassembly\seemips\seeobject
}

\providecommand{\mipsadism}{
\toolsection{mips32dism} is a disassembler for the MIPS32 hardware architecture.
It translates machine code from object files targeting MIPS32 processors into assembly code and writes it to the standard output stream.
\flowgraph{\resource{object file} \ar[r] & \toolbox{mips32dism} \ar[r] & \resource{disassembly\\listing}}
\seeassembly\seemips\seeobject
}

\providecommand{\mipsbasm}{
\toolsection{mips64asm} is an assembler for the MIPS64 hardware architecture.
It translates assembly code into machine code for MIPS64 processors and stores it in corresponding object files.
\flowgraph{\resource{MIPS64 assembly\\source code} \ar[r] & \toolbox{mips64asm} \ar[r] & \resource{object file}}
\seeassembly\seemips\seeobject
}

\providecommand{\mipsbdism}{
\toolsection{mips64dism} is a disassembler for the MIPS64 hardware architecture.
It translates machine code from object files targeting MIPS64 processors into assembly code and writes it to the standard output stream.
\flowgraph{\resource{object file} \ar[r] & \toolbox{mips64dism} \ar[r] & \resource{disassembly\\listing}}
\seeassembly\seemips\seeobject
}

\providecommand{\mmixasm}{
\toolsection{mmixasm} is an assembler for the MMIX hardware architecture.
It translates assembly code into machine code for MMIX processors and stores it in corresponding object files.
The names of all special registers are predefined and evaluate to the corresponding number.
\flowgraph{\resource{MMIX assembly\\source code} \ar[r] & \toolbox{mmixasm} \ar[r] & \resource{object file}}
\seeassembly\seemmix\seeobject
}

\providecommand{\mmixdism}{
\toolsection{mmixdism} is a disassembler for the MMIX hardware architecture.
It translates machine code from object files targeting MMIX processors into assembly code and writes it to the standard output stream.
\flowgraph{\resource{object file} \ar[r] & \toolbox{mmixdism} \ar[r] & \resource{disassembly\\listing}}
\seeassembly\seemmix\seeobject
}

\providecommand{\orokasm}{
\toolsection{or1kasm} is an assembler for the OpenRISC 1000 hardware architecture.
It translates assembly code into machine code for OpenRISC 1000 processors and stores it in corresponding object files.
\flowgraph{\resource{OpenRISC 1000 assembly\\source code} \ar[r] & \toolbox{or1kasm} \ar[r] & \resource{object file}}
\seeassembly\seeorok\seeobject
}

\providecommand{\orokdism}{
\toolsection{or1kdism} is a disassembler for the OpenRISC 1000 hardware architecture.
It translates machine code from object files targeting OpenRISC 1000 processors into assembly code and writes it to the standard output stream.
\flowgraph{\resource{object file} \ar[r] & \toolbox{or1kdism} \ar[r] & \resource{disassembly\\listing}}
\seeassembly\seeorok\seeobject
}

\providecommand{\ppcaasm}{
\toolsection{ppc32asm} is an assembler for the PowerPC hardware architecture.
It translates assembly code into machine code for PowerPC processors and stores it in corresponding object files.
By default, the assembler generates machine code for the 32-bit operating mode defined by the PowerPC architecture.
\flowgraph{\resource{PowerPC assembly\\source code} \ar[r] & \toolbox{ppc32asm} \ar[r] & \resource{object file}}
\seeassembly\seeppc\seeobject
}

\providecommand{\ppcadism}{
\toolsection{ppc32dism} is a disassembler for the PowerPC hardware architecture.
It translates machine code from object files targeting PowerPC processors into assembly code and writes it to the standard output stream.
It assumes that the machine code was generated for the 32-bit operating mode defined by the PowerPC architecture.
\flowgraph{\resource{object file} \ar[r] & \toolbox{ppc32dism} \ar[r] & \resource{disassembly\\listing}}
\seeassembly\seeppc\seeobject
}

\providecommand{\ppcbasm}{
\toolsection{ppc64asm} is an assembler for the PowerPC hardware architecture.
It translates assembly code into machine code for PowerPC processors and stores it in corresponding object files.
By default, the assembler generates machine code for the 64-bit operating mode defined by the PowerPC architecture.
\flowgraph{\resource{PowerPC assembly\\source code} \ar[r] & \toolbox{ppc64asm} \ar[r] & \resource{object file}}
\seeassembly\seeppc\seeobject
}

\providecommand{\ppcbdism}{
\toolsection{ppc64dism} is a disassembler for the PowerPC hardware architecture.
It translates machine code from object files targeting PowerPC processors into assembly code and writes it to the standard output stream.
It assumes that the machine code was generated for the 64-bit operating mode defined by the PowerPC architecture.
\flowgraph{\resource{object file} \ar[r] & \toolbox{ppc64dism} \ar[r] & \resource{disassembly\\listing}}
\seeassembly\seeppc\seeobject
}

\providecommand{\riscasm}{
\toolsection{riscasm} is an assembler for the RISC hardware architecture.
It translates assembly code into machine code for RISC processors and stores it in corresponding object files.
The names of all special registers are predefined and evaluate to the corresponding number.
\flowgraph{\resource{RISC assembly\\source code} \ar[r] & \toolbox{riscasm} \ar[r] & \resource{object file}}
\seeassembly\seerisc\seeobject
}

\providecommand{\riscdism}{
\toolsection{riscdism} is a disassembler for the RISC hardware architecture.
It translates machine code from object files targeting RISC processors into assembly code and writes it to the standard output stream.
\flowgraph{\resource{object file} \ar[r] & \toolbox{riscdism} \ar[r] & \resource{disassembly\\listing}}
\seeassembly\seerisc\seeobject
}

\providecommand{\wasmasm}{
\toolsection{wasmasm} is an assembler for the WebAssembly architecture.
It translates assembly code into machine code for WebAssembly targets and stores it in corresponding object files.
The names of all special registers are predefined and evaluate to the corresponding number.
\flowgraph{\resource{WebAssembly assembly\\source code} \ar[r] & \toolbox{wasmasm} \ar[r] & \resource{object file}}
\seeassembly\seewasm\seeobject
}

\providecommand{\wasmdism}{
\toolsection{wasmdism} is a disassembler for the WebAssembly architecture.
It translates machine code from object files targeting WebAssembly targets into assembly code and writes it to the standard output stream.
\flowgraph{\resource{object file} \ar[r] & \toolbox{wasmdism} \ar[r] & \resource{disassembly\\listing}}
\seeassembly\seewasm\seeobject
}

% linker tools

\providecommand{\linklib}{
\toolsection{linklib} is an object file combiner.
It creates a static library file by combining all object files given to it into a single one.
\flowgraph{\resource{object files} \ar[r] & \toolbox{linklib} \ar[r] & \resource{library file}}
\seeobject
}

\providecommand{\linkbin}{
\toolsection{linkbin} is a linker for plain binary files.
It links all object files given to it into a single image and stores it in a binary file that begins with the first linked section.
It also creates a map file that lists the address, type, name and size of all used sections.
The filename extension of the resulting binary file can be specified by putting it into a constant data section called \texttt{\_extension}.
\flowgraph{\resource{object files} \ar[r] & \toolbox{linkbin} \ar[r] \ar[d] & \resource{binary file} \\ & \resource{map file}}
\seeobject
}

\providecommand{\linkmem}{
\toolsection{linkmem} is a linker for plain binary files partitioned into random-access and read-only memory.
It links all object files given to it into two distinct images, one for data sections and one for code and constant data sections, and stores each image in a binary file that begins with the first linked section of the corresponding type.
It also creates a map file that lists the address, type, name and size of all used sections.
\flowgraph{\resource{object files} \ar[r] & \toolbox{linkmem} \ar[r] \ar[d] & \resource{RAM file/\\ROM file} \\ & \resource{map file}}
\seeobject
}

\providecommand{\linkprg}{
\toolsection{linkprg} is a linker for GEMDOS executable files.
It links all object files given to it into a single image and stores the image in an Atari GEMDOS executable file~\cite{gemdosfile}.
It also creates a map file that lists the address relative to the text segment, type, name and size of all used sections.
The filename extension of the resulting executable file can be specified by putting it into a constant data section called \texttt{\_extension}.
The GEMDOS executable file format requires all patch patterns of absolute link patches to consist of four full bitmasks with descending offsets.
\flowgraph{\resource{object files} \ar[r] & \toolbox{linkprg} \ar[r] \ar[d] & \resource{executable file} \\ & \resource{map file}}
\seeobject
}

\providecommand{\linkhex}{
\toolsection{linkhex} is a linker for Intel HEX files.
It links all code sections of the object files given to it into single image and stores the image in an Intel HEX file~\cite{hexfile} that begins with the first linked section.
It also creates a map file that lists the address, type, name and size of all used sections.
\flowgraph{\resource{object files} \ar[r] & \toolbox{linkhex} \ar[r] \ar[d] & \resource{HEX file} \\ & \resource{map file}}
\seeobject
}

\providecommand{\mapsearch}{
\toolsection{mapsearch} is a debugging tool.
It searches map files generated by linker tools for the name of a binary section that encompasses a memory address read from the standard input stream.
If additionally provided with one or more object files, it also stores an excerpt thereof in a separate object file called map search result which only contains the identified binary section for disassembling purposes.
\flowgraph{& \resource{map files/\\object files} \ar[d] \\ \resource{memory\\address} \ar[r] & \toolbox{mapsearch} \ar[r] \ar[d] & \resource{section name/\\relative offset} \\ & \resource{object file\\excerpt}}
\seeobject
}

\renewcommand{\seeguide}{}

\startchapter{Getting Started}{User Guide}{guide}
{This \documentation{} provides a basic guide for novice users of the \ecs{}.
It explains how to get started with the toolchain and shows some typical use cases.
In the end, users will understand the functionality of the \ecs{} by building executable programs.}

\epigraph{Man f\"uhlt den Glanz von einer neuen Seite, \\ auf der noch alles werden kann.}{Rainer Maria Rilke}

\section{Introduction}

The \ecs{} features a set of development tools for several programming languages targeting a variety of hardware architectures.
This \documentation{} demonstrates how to use these tools in order to build executable applications.
The assumption is that the user already knows how to program in the programming languages presented in this \documentation{}.
For each programming language supported by the \ecs{} there are the following tools for processing and translating the source code:

\begin{itemize}
\item Pretty printers for reformatting the source code
\item Semantic checkers for analyzing the source code for semantic errors
\item Interpreters for executing the source code line by line
\item Compilers for translating the source code into executable machine code
\end{itemize}

All of these tools have the same functionality regardless of which programming language they actually implement.
Without loss of generality, the remainder of this guide therefore focuses exemplarily on the Oberon programming language.
\seeoberon
Regarding the hardware architectures on the other hand, the \ecs{} provides the following tools for each supported hardware architecture:

\begin{itemize}
\item Assemblers for translating assembly code into binary machine code
\item Disassemblers for translating machine code into human-readable text
\end{itemize}

The \ecs{} provides a couple of assemblers, all of which implement the same generic assembly language.
\seeassembly
In general, users can target any hardware architecture supported by the \ecs{} by just replacing the corresponding tools and their runtime support.
For instructional purposes, the remainder of this \documentation{} therefore assumes that the user targets the AMD64 hardware architecture.
\seeamd
Finally, for building executable programs generated by compilers and assemblers the \ecs{} provides the following tools:

\begin{itemize}
\item Linkers for combining binary data into an executable file
\item Debugging tools for processing debugging information
\end{itemize}

The debugging information generated by compilers and linkers allows debugging problematic programs and consists of a symbolic address mapping.
\seedebugging

\section{Prerequisites}

In order to process and build the examples shown in this guide, the user needs access to the following tools:

\begin{itemize}
\item \tool{obprint} for pretty printing Oberon modules
\item \tool{obcheck} for analyzing Oberon modules
\item \tool{obrun} for interpreting Oberon modules
\item \tool{obamd64} for compiling Oberon modules
\item \tool{amd64asm} for translating assembly code
\item \tool{amd64dism} for translating the generated machine code
\item \tool{linklib} for combining object files
\item \tool{linkbin} for linking an executable program
\item \tool{mapsearch} for debugging programs
\item \tool{dbgdwarf} for converting debugging information
\end{itemize}

The examples shown in Sections~\ref{sec:guidebasic} to~\ref{sec:guidedebugging} assume that all of these tools are executable from the current working directory.
The illustrated commands to invoke these tools may therefore have to be adjusted to their actual location and may additionally require a suffix for executable files.
Furthermore, all executable programs created using the compiler require runtime support which is provided by the following object and library files:

\begin{itemize}
\item \file{obamd64run.lib} required by the programming language
\item \file{amd64run.obf} required by the hardware architecture
\item \file{win64run.obf}, \file{osx64run.obf}, or \file{amd64linuxrun.obf} required by runtime environments like Windows, OS~X, or Linux-based operating systems
\end{itemize}

These files are assumed to be located in the current working directory as well and may therefore also have to be adjusted accordingly.
They store binary data such as the machine code of precompiled runtime support for the respective hardware architecture and runtime environment.
\seeobject

Installations of the \ecs{} typically feature a driver utility tool called \tool{ecsd} which is able to automatically locate and invoke all of the tools discussed in this section.
It also provides the necessary runtime support without requiring users to explicitly name any of the prerequisites listed above.
See Section~\ref{sec:guidedriver} for more information about how to run the examples described in Sections~\ref{sec:guidebasic} to~\ref{sec:guidedebugging} using the \ecs{} driver.
\ifbook For more information about this tool in general, see Chapter~\ref{interface}. \fi

The remainder of this section gives more detailed information about the required tools.
\interface

\renewcommand{\seeoberon}{}
\renewcommand{\seeassembly}{}
\renewcommand{\seeamd}{}
\renewcommand{\seedebugging}{}
\renewcommand{\seeobject}{}

\obprint
\obcheck
\obrun
\obamdc
\amdcasm
\amdcdism
\linklib
\linkbin
\mapsearch
\dbgdwarf

\section{Basic Processing}\label{sec:guidebasic}

The first example is a very simple Oberon module that defines a global variable, assigns a value to it and prints that value.
The corresponding source code looks as follows:

\begin{quote}\begin{verbatim}
MODULE Simple;
VAR variable: INTEGER;
BEGIN
  variable := 10;
  TRACE (variable);
END Simple.
\end{verbatim}\end{quote}

The remainder of this section assumes that these six lines of source code are stored in a plain text file called \file{simple.mod} in the current working directory.

\subsection{Pretty Printing}

Executing the pretty printer should yield the very same code in a slightly different layout and helps to check whether the source was syntactically recognized.
The corresponding command-line call looks like the following and produces the subsequent output:

\begin{quote}\begin{verbatim}
obprint simple.mod
MODULE Simple;

VAR
  variable: INTEGER;

BEGIN
  variable := 10;
  TRACE (variable);
END Simple.
\end{verbatim}\end{quote}

\subsection{Semantic Checking}

For checking whether the source code is also semantically valid one can execute the semantic checker using the following command-line call:

\begin{quote}\begin{verbatim}
obcheck simple.mod
\end{verbatim}\end{quote}

Since the input was semantically valid, the invocation of this tool does not generate any output message and succeeds.
However, if line 5 of the original source code was for example changed to \texttt{TRACE (x);} then the tool would identify a violation of the language rules:

\begin{quote}\begin{verbatim}
obcheck simple.mod
simple.mod:5:10: error: undeclared identifier 'x'
\end{verbatim}\end{quote}

If the semantic checker succeeds, it generates a plain text file called \file{simple.sym} in the current working directory which contains additional semantic information about the source code.
In the case of Oberon, this file is a symbol file containing the interface of the sample module which is required by other modules importing it.

\subsection{Interpreting}

Valid modules can be executed by the interpreter which simulates a runtime environment for the Oberon language.
The corresponding command-line call looks like the following and produces the subsequent output:

\begin{quote}\begin{verbatim}
obrun simple.mod
simple.mod:5:10: note: 'variable' = 10
\end{verbatim}\end{quote}

The interpreter does not generate any output files.
It just executes the source code line by line by first assigning the value 10 to the variable and then printing its value.

\section{Building Simple Programs}\label{sec:guidesimple}

Running code using the interpreter is quite slow compared to executing the same program directly on the machine.
This section shows how the same source code can be translated into an executable program.

\subsection{Compiling}

The compiler translates the source code of the sample module into machine code using the following command-line call:

\begin{quote}\begin{verbatim}
obamd64 simple.mod
\end{verbatim}\end{quote}

If its invocation is successful, the compiler generates four different plain text files for this module.
The first file \file{simple.sym} is the same symbol file as generated by the semantic checker.
The second file \file{simple.obf} is an object file which stores the binary encoding of the machine code translated from the source code.
The third file \file{simple.lst} is a human-readable assembly code listing of the generated machine code.
The fourth file \file{simple.dbg} contains additional symbolic debugging information about the module.
The last two files are actually not required for building the executable program but are generated for debugging purposes.
Invoking the assembler on the assembly code listing using the following command-line call yields exactly the same object file:

\begin{quote}\begin{verbatim}
amd64asm simple.lst
\end{verbatim}\end{quote}

\subsection{Linking}

Although the generated object file contains executable machine code, it is not executable by itself.
The machine code has first to be combined with the required runtime support for the programming language, the hardware architecture, as well as the runtime environment.
This process is called linking and can be executed by invoking the linker using the following command-line call.
The actual order of the command-line arguments is not important but the linker takes the filename of the first object file as a basis for its output files:

\begin{quote}\begin{verbatim}
linkbin simple.obf obamd64run.lib amd64run.obf win64run.obf
\end{verbatim}\end{quote}

If this call succeeds, the linker generates a binary image of an executable program and stores it in a file called \file{simple.exe}.
This file can be executed on a machine running the Windows operating system producing exactly the same output as the interpreter above.
For creating a similar executable file called \file{simple} for the OS~X operating system, only the last command-line argument has to be changed:

\begin{quote}\begin{verbatim}
linkbin simple.obf obamd64run.lib amd64run.obf osx64run.obf
\end{verbatim}\end{quote}

This executable program still needs the same runtime support for the programming language and hardware architecture.
The same holds for programs that can be executed on Linux-based operating systems.
Again only the last command-line argument has to be modified accordingly:

\begin{quote}\begin{verbatim}
linkbin simple.obf obamd64run.lib amd64run.obf amd64linuxrun.obf
\end{verbatim}\end{quote}

In all three cases, the linker also generates a plain text file called \file{simple.map} summarizing the mapping of symbol names in the source code to their linked addresses.
This information is provided for debugging purposes and used in Section~\ref{sec:guidedebugging}.

\section{Building Complex Programs}\label{sec:guidecomplex}

The second example makes use of the interoperability features of the \ecs{}.
It shows how code from one programming language can access code and data defined in another language and vice versa.

\subsection{Compiling}

Here is an extended version of the first example.
Instead of assigning the value to the global variable directly, this version calls a procedure that overwrites the value of the variable.
In the end however, it should generate the same output as the previous example:

\begin{quote}\begin{verbatim}
MODULE Complex;
IMPORT SYSTEM;
VAR variable: INTEGER;
PROCEDURE ^ Assign ["assign"] (value: INTEGER);
BEGIN
  Assign (10);
  TRACE (variable);
END Complex.
\end{verbatim}\end{quote}

The forward declaration of the procedure is marked as external which causes the compiler to refer to the procedure using the given external name instead of defining a new procedure.
If the source code is stored in a plain text file called \file{complex.mod}, the compiler can be invoked using the following command-line call:

\begin{quote}\begin{verbatim}
obamd64 complex.mod
\end{verbatim}\end{quote}

The resulting plain text files are called \file{complex.sym}, \file{complex.obf}, \file{complex.lst}, and \file{complex.dbg}.
However, linking the generated object file together with the runtime support for any runtime environment as before yields the following linker error:

\begin{quote}\begin{verbatim}
linkbin complex.obf obamd64run.lib amd64run.obf amd64linuxrun.obf
Complex._body: error: unresolved symbol 'assign'
\end{verbatim}\end{quote}

The diagnostic message states that the body of the module calls a procedure called \texttt{assign} which is defined nowhere.
This behavior is intended because all external procedures need to be defined by another part of the program.
The following section shows how to implement it using assembly language.

\subsection{Assembling}

The missing \texttt{assign} procedure is defined in the following assembly source code.
It contains instructions which copy the procedure argument into a register and from there into the global variable:

\begin{quote}\begin{verbatim}
.code assign
  mov  ebx, [rsp + 8]
  mov  [@Complex.variable], ebx
  ret
\end{verbatim}\end{quote}

Assuming these four lines of source code are stored in a plain text file called \file{assign.asm}, the assembler can be invoked using the following command-line call:

\begin{quote}\begin{verbatim}
amd64asm assign.asm
\end{verbatim}\end{quote}

This generates an object file called \file{assign.obf} which stores the binary representation of the corresponding machine code of this procedure.
Its contents can be listed by the disassembler using the following command-line call which produces the subsequent output:

\begin{quote}\begin{verbatim}
amd64dism assign.obf
.code assign
   0  8b5c2408        mov  ebx, dword [rsp + 8]
   4  891c2500000000  mov  dword [0], ebx  ; abs @Complex.variable
  11  c3              ret
  12
\end{verbatim}\end{quote}

The disassembler lists all three instructions which require twelve octets of code space in total.
Each instruction is prefixed with its relative offset and binary encoding.

\subsection{Combining}

The following step is optional but shows how the object files generated by the compiler and the assembler can be merged together.
The result is a new object file called a library file which combines the machine code of the Oberon module with the procedure defined in the assembly code.
It is generated using the following command-line call:

\begin{quote}\begin{verbatim}
linklib complex.obf assign.obf
\end{verbatim}\end{quote}

This call generates a new library file called \file{complex.lib} which contains the contents of both object files.
This file can be used to link the executable program in the next step.

\subsection{Linking}

Since the library file contains the machine code of both the module and the \texttt{assign} procedure, the linker should not fail any longer.
The corresponding command-line call looks like the following:

\begin{quote}\begin{verbatim}
linkbin complex.lib obamd64run.lib amd64run.obf amd64linuxrun.obf
\end{verbatim}\end{quote}

This call generates a new binary executable file called \file{complex} that produces the same output as before when executed.
The second generated file \file{complex.map} contains the mapping of symbol names in the source code to their linked addresses and includes the new \texttt{assign} procedure.

The linker could have also been called using the two object files generated by the assembler and the compiler instead of the single library file yielding exactly the same result.
This combining and linking works for arbitrary object files generated by the various development tools of the \ecs{} as long as they target the same hardware architecture.

\section{Debugging Programs}\label{sec:guidedebugging}

Compilers and linkers provided by the \ecs{} generate various files for debugging purposes.
This section shows how this symbolic debugging information can be used to debug problematic programs.

\subsection{Basic Debugging}

If an erroneous program fails, the corresponding error message of the runtime environment or debugger often mentions the address of the first problematic instruction.
The \ecs{} provides a debugging tool called \tool{mapsearch} which allows searching map files generated by linker tools for the section encompassing the reported address.
This is helpful for identifying the executed code section and all other active procedures in a stack trace.
The tool reads a single address from the standard input stream and prints the name of the identified section if it is listed in the specified map file.
For convenience, the address can be provided by a single command-line call as follows:

\begin{quote}\begin{verbatim}
echo 0x80482a7 | mapsearch complex.map
mapsearch: note: offset 4 in code section 'assign'
\end{verbatim}\end{quote}

The address in this particular example corresponds to the instruction at offset four relative to the \texttt{assign} procedure.
If additionally provided with the object file containing the identified section, the tool also generates a separate object file called \file{complex.msr} which can be used to disassemble only that section:

\begin{quote}\begin{verbatim}
echo 0x80482a7 | mapsearch complex.map complex.lib
amd64dism complex.msr
\end{verbatim}\end{quote}

Searching map files generated by linkers for problematic addresses is a rather limited debugging facility but always available.

\subsection{Advanced Debugging}

For runtime environments that support more spohisticated debuggers, the \ecs{} provides tools that convert the symbolic debugging information generated by its compilers alongside an object file into a suitable format.
This debugging information consists of an abstract representation of all programming language constructs compiled into the object file and can be converted as follows for example:

\begin{quote}\begin{verbatim}
dbgdwarf complex.dbg
\end{verbatim}\end{quote}

This generates a debugging object file called \file{complex.dbf} which can optionally be provided together with the corresponding object file when linking for Linux-based operating systems:

\begin{quote}\begin{verbatim}
linkbin complex.obf complex.dbf assign.obf
        obamd64run.lib amd64run.obf amd64linuxrun.obf
\end{verbatim}\end{quote}

The resulting binary executable file behaves as before but additionally contains debugging information that enables interactive program animation and memory inspection when loaded in an external debugger.

\section{Using the \ecs{} Driver}\label{sec:guidedriver}

In addition to invoking tools directly as shown above, users can also call the \tool{ecsd} driver utility tool which is typically provided when installing the \ecs{}.
It allows compiling and linking complete executable files for a specific runtime environment in one step and automatically provides the necessary runtime support.

\flowgraph{\resource{input\\files} \ar[r] & \converter{ecsd} \ar[r] & \resource{executable\\file}}

In contrast to all other tools of the \ecs{}, the driver tool does support the notion of command-line options to allow users to influence how tools are identified and invoked.
The following command-line call prints a list of all supported command-line arguments:

\begin{quote}\begin{verbatim}
ecsd -h
\end{verbatim}\end{quote}

The remainder of this section consistently uses a short notation for options although all of them have also a longer and more descriptive form prefixed by two dashes.
\ifbook For more information about the user interface of this tool, see Chapter~\ref{interface}. \fi

\subsection{Basic Processing}

All of the tools mentioned in Section~\ref{sec:guidebasic} can also be invoked using the \ecs{} driver by putting \texttt{ecsd~-i} in front of the corresponding command-line call.
This particular option prompts the driver to process the input files given as command-line arguments using the specified tool.
Provided that the corresponding tools are available, the command-line calls for pretty printing, semantic checking, and interpreting for example look like the following:

\begin{quote}\begin{verbatim}
ecsd -i obprint simple.mod
ecsd -i obcheck simple.mod
ecsd -i obrun simple.mod
\end{verbatim}\end{quote}

The actual command-line call used by the driver to invoke the respective tool can be shown by adding the \texttt{-v}~flag.
This enables verbose mode which is especially helpful when building programs as shown in the following section.

\subsection{Building Programs}

If the driver is called without command-line options, it builds complete executable files for a specific runtime environment as described in Section~\ref{sec:guidesimple}.
It infers the set of required tools like compilers, assemblers, and linkers from the type of its input files and automatically invokes one tool after the other with appropriate command-line arguments.
By default, it tries to target its own runtime environment:

\begin{quote}\begin{verbatim}
ecsd simple.mod
\end{verbatim}\end{quote}

In order to explicitly target a runtime environment like Windows, OS~X, or Linux-based operating systems for example, use the \texttt{-t}~option as follows.
A list of all available target environments is accessible using the \texttt{-h}~flag:

\begin{quote}\begin{verbatim}
ecsd -t win64 simple.mod
ecsd -t osx64 simple.mod
ecsd -t amd64linux simple.mod
\end{verbatim}\end{quote}

Using the driver to build programs still generates intermediate files like object files and assembly code listings as before.
Use the \texttt{-c}~flag to compile and assemble input files without invoking the linker at the end:

\begin{quote}\begin{verbatim}
ecsd -c simple.mod
\end{verbatim}\end{quote}

The driver can also be called using the generated intermediate files.
However, because the driver cannot infer which compiler was used to generate object files or assembly code listings in the first place, it must be explicitly told which additional runtime support to include this time.
In the case of the Oberon programming language, the required runtime support is made available using the \texttt{-O}~flag:

\begin{quote}\begin{verbatim}
ecsd -O simple.obf
ecsd -O simple.lst
\end{verbatim}\end{quote}

The driver is able to process several different input files at once which allows building complex programs as described in Section~\ref{sec:guidecomplex}:

\begin{quote}\begin{verbatim}
ecsd complex.mod assign.asm
\end{verbatim}\end{quote}

Combining input files into a single object file can be achieved using the \texttt{-l}~flag.
As before, the resulting library file requires additional runtime support to be specified explicitly when building the program:

\begin{quote}\begin{verbatim}
ecsd -l complex.mod assign.obf
ecsd -O complex.lib
\end{verbatim}\end{quote}

In order to disassemble source code or object files instead of linking them, the driver can be invoked using the \texttt{-d}~flag:

\begin{quote}\begin{verbatim}
ecsd -d assign.asm
ecsd -d assign.obf
\end{verbatim}\end{quote}

Generally, the type of an input file is inferred from its filename extension.
The \texttt{-s}~option allows users to explicitly specify the source type of unknown input files.
A list of all supported source types is accessible using the \texttt{-h}~flag.

\subsection{Debugging Programs}

The \ecs{} driver also allows processing map files generated by linker tools.
This is useful for identifying sections that contain problematic instructions in erroneous programs as described in Section~\ref{sec:guidedebugging}.
Building a complex program as described in the previous section for example generates a file called \file{complex.map}:

\begin{quote}\begin{verbatim}
ecsd complex.mod assign.asm
\end{verbatim}\end{quote}

This file can be passed to the driver which then invokes a debugging tool that reads an address from the standard input stream in order to search for a linked section that encompasses that address.
If the address is mapped, the tool prints the name of the corresponding section and the relative offset of the address:

\begin{quote}\begin{verbatim}
echo 0x4011fe | ecsd complex.map
mapsearch: note: offset 4 in code section 'assign'
\end{verbatim}\end{quote}

If additionally provided with the object file or runtime support containing the identified section, the driver subsequently disassembles the latter when using the \texttt{-d}~flag.
In this particular example, the input address points to the second instruction of the \texttt{assign} procedure written in assembly code:

\begin{quote}\begin{verbatim}
echo 0x4011fe | ecsd -d complex.map assign.obf
mapsearch: note: offset 4 in code section 'assign'
.code assign
   0  8b5c2408        mov  ebx, dword [rsp + 8]
   4  891c2500000000  mov  dword [0], ebx  ; abs @Complex.variable
  11  c3              ret
  12
\end{verbatim}\end{quote}

For more advanced debugging using external debuggers on runtime environments that support them, the \ecs{} driver automatically converts and includes the symbolic debugging information generated by compilers using the \texttt{-g} flag:

\begin{quote}\begin{verbatim}
ecsd -g simple.mod
ecsd -g complex.mod assign.asm
\end{verbatim}\end{quote}

This generates executable files that incorporate the address mapping of all symbols and instructions necessary for interactive program animation and memory inspection when loaded in a debugger.

\concludechapter

% Eigen Compiler Suite tool reference
% Copyright (C) Florian Negele

% This file is part of the Eigen Compiler Suite.

% Permission is granted to copy, distribute and/or modify this document
% under the terms of the GNU Free Documentation License, Version 1.3
% or any later version published by the Free Software Foundation.

% You should have received a copy of the GNU Free Documentation License
% along with the ECS.  If not, see <https://www.gnu.org/licenses/>.

% Generic documentation utilities
% Copyright (C) Florian Negele

% This file is part of the Eigen Compiler Suite.

% Permission is granted to copy, distribute and/or modify this document
% under the terms of the GNU Free Documentation License, Version 1.3
% or any later version published by the Free Software Foundation.

% You should have received a copy of the GNU Free Documentation License
% along with the ECS.  If not, see <https://www.gnu.org/licenses/>.

\providecommand{\cpp}{C\texttt{++}}
\providecommand{\opt}{_\mathit{opt}}
\providecommand{\tool}[1]{\texttt{#1}}
\providecommand{\version}{Version 0.0.40}
\providecommand{\resource}[1]{*++\txt{#1}}
\providecommand{\ecs}{Eigen Compiler Suite}
\providecommand{\changed}[1]{\underline{#1}}
\providecommand{\toolbox}[1]{\converter{#1}}
\providecommand{\file}{}\renewcommand{\file}[1]{\texttt{#1}}
\providecommand{\alignright}{\hfill\linebreak[0]\hspace*{\fill}}
\providecommand{\converter}[1]{*++[F][F*:white][F,:gray]\txt{#1}}
\providecommand{\documentation}{\ifbook chapter\else document\fi}
\providecommand{\Documentation}{\ifbook Chapter\else Document\fi}
\providecommand{\variable}[1]{\resource{\texttt{\small#1}\\variable}}
\providecommand{\documentationref}[2]{\ifbook\ref{#1}\else``\href{#1}{#2}''~\cite{#1}\fi}
\providecommand{\objfile}[1]{\texttt{#1}\index[runtime]{#1 object file@\texttt{#1} object file}}
\providecommand{\libfile}[1]{\texttt{#1}\index[runtime]{#1 library file@\texttt{#1} library file}}
\providecommand{\epigraph}[2]{\ifbook\begin{quote}\flushright\textit{#1}\par--- #2\end{quote}\fi}
\providecommand{\environmentvariable}[1]{\texttt{#1}\index{Environment variables!#1@\texttt{#1}}}
\providecommand{\environment}[1]{\texttt{#1}\index[environment]{#1 environment@\texttt{#1} environment}}
\providecommand{\toolsection}{}\renewcommand{\toolsection}[1]{\subsection{#1}\label{\prefix:#1}\tool{#1}}
\providecommand{\instruction}{}\renewcommand{\instruction}[2]{\noindent\qquad\pdftooltip{\texttt{#1}}{#2}\refstepcounter{instruction}\par}
\providecommand{\flowgraph}{}\renewcommand{\flowgraph}[1]{\par\sffamily\begin{displaymath}\xymatrix@=4ex{#1}\end{displaymath}\normalfont\par}
\providecommand{\instructionset}{}\renewcommand{\instructionset}[4]{\setcounter{instruction}{0}\begin{multicols}{\ifbook#3\else#4\fi}[{\captionof{table}[#2]{#2 (\ref*{#1:instructions}~instructions)}\label{tab:#1set}\vspace{-2ex}}]\footnotesize\raggedcolumns\input{#1.set}\label{#1:instructions}\end{multicols}}

\providecommand{\gpl}{GNU General Public License}
\providecommand{\rse}{ECS Runtime Support Exception}
\providecommand{\fdl}{\href{https://www.gnu.org/licenses/fdl.html}{GNU Free Documentation License}}

\providecommand{\docbegin}{}
\providecommand{\docend}{}
\providecommand{\doclabel}[1]{\hypertarget{#1}}
\providecommand{\doclink}[2]{\hyperlink{#1}{#2}}
\providecommand{\docsection}[3]{\hypertarget{#1}{\subsection{#2}}\label{sec:#1}\index[library]{#2@#3}}
\providecommand{\docsectionstar}[1]{}
\providecommand{\docsubbegin}{\begin{description}}
\providecommand{\docsubend}{\end{description}}
\providecommand{\docsubsection}[3]{\item[\hypertarget{#1}{#2}]\index[library]{#2@#3}}
\providecommand{\docsubsectionstar}[1]{\smallskip}
\providecommand{\docsubsubsection}[3]{\docsubsection{#1}{#2}{#3}}
\providecommand{\docsubsubsectionstar}[1]{}
\providecommand{\docsubsubsubsection}[3]{}
\providecommand{\docsubsubsubsectionstar}[1]{}
\providecommand{\doctable}{}

\providecommand{\debuggingtool}{}\renewcommand{\debuggingtool}{This tool is provided for debugging purposes.
It allows exposing and modifying an internal data structure that is usually not accessible.
}

\providecommand{\interface}{All tools accept command-line arguments which are taken as names of plain text files containing the source code.
If no arguments are provided, the standard input stream is used instead.
Output files are generated in the current working directory and have the same name as the input file being processed whereas the filename extension gets replaced by an appropriate suffix.
\seeinterface
}

\providecommand{\license}{\noindent Copyright \copyright{} Florian Negele\par\medskip\noindent
Permission is granted to copy, distribute and/or modify this document under the terms of the
\fdl{}, Version 1.3 or any later version published by the \href{https://fsf.org/}{Free Software Foundation}.
}

\providecommand{\ecslogosurface}{
\fill[darkgray] (0,0,0) -- (0,0,3) -- (0,3,3) -- (0,3,1) -- (0,4,1) -- (0,4,3) -- (0,5,3) -- (0,5,0) -- (0,2,0) -- (0,2,2) -- (0,1,2) -- (0,1,0) -- cycle;
\fill[gray] (0,5,0) -- (0,5,3) -- (1,5,3) -- (1,5,1) -- (2,5,1) -- (2,5,3) -- (3,5,3) -- (3,5,0) -- cycle;
\fill[lightgray] (0,0,0) -- (0,1,0) -- (2,1,0) -- (2,4,0) -- (1,4,0) -- (1,3,0) -- (2,3,0) -- (2,2,0) -- (0,2,0) -- (0,5,0) -- (3,5,0) -- (3,0,0) -- cycle;
\begin{scope}[line width=0.5]
\begin{scope}[gray]
\draw (0,0,0) -- (0,1,0);
\draw (2,1,0) -- (2,2,0);
\draw (0,1,2) -- (0,2,2);
\draw (0,2,0) -- (0,5,0);
\draw (2,3,0) -- (2,4,0);
\end{scope}
\begin{scope}[lightgray]
\draw (0,1,0) -- (0,1,2);
\draw (0,3,1) -- (0,3,3);
\draw (0,5,0) -- (0,5,3);
\draw (2,5,1) -- (2,5,3);
\end{scope}
\begin{scope}[white]
\draw (0,1,0) -- (2,1,0);
\draw (1,3,0) -- (2,3,0);
\draw (0,5,0) -- (3,5,0);
\end{scope}
\end{scope}
}

\providecommand{\ecslogo}[1]{
\begin{tikzpicture}[scale={(#1)/((sin(45)+cos(45))*3cm)},x={({-cos(45)*1cm},{sin(45)*sin(30)*1cm})},y={({0cm},{(cos(30)*1cm})},z={({sin(45)*1cm},{cos(45)*sin(30)*1cm})}]
\begin{scope}[darkgray,line width=1]
\draw (0,0,0) -- (0,0,3) -- (0,3,3) -- (2,3,3) -- (2,5,3) -- (3,5,3) -- (3,5,0) -- (3,0,0) -- cycle;
\draw (0,3,1) -- (0,4,1) -- (0,4,3) -- (0,5,3) -- (1,5,3) -- (1,5,1) -- (2,5,1);
\draw (1,3,0) -- (1,4,0) -- (2,4,0);
\end{scope}
\fill[darkgray] (2,0,0) -- (2,0,3) -- (2,5,3) -- (2,5,1) -- (2,4,1) -- (2,4,0) -- cycle;
\fill[lightgray] (2,0,2) -- (0,0,2) -- (0,2,2) -- (2,2,2) -- cycle;
\fill[gray] (0,1,0) -- (2,1,0) -- (2,1,2) -- (0,1,2) -- cycle;
\fill[gray] (0,3,1) -- (0,3,3) -- (2,3,3) -- (2,3,0) -- (1,3,0) -- (1,3,1) -- cycle;
\ecslogosurface
\end{tikzpicture}
}

\providecommand{\shadowedecslogo}[3]{
\begin{tikzpicture}[scale={(#1)/((sin(#2)+cos(#2))*3cm)},x={({-cos(#2)*1cm},{sin(#2)*sin(#3)*1cm})},y={({0cm},{(cos(#3)*1cm})},z={({sin(#2)*1cm},{cos(#2)*sin(#3)*1cm})}]
\shade[top color=lightgray!50!white,bottom color=white,middle color=lightgray!50!white] (0,0,0) -- (3,0,0) -- (3,{-0.5-3*sin(#2)*sin(#3)/cos(#3)},0) -- (0,-0.5,0) -- cycle;
\shade[top color=darkgray!50!gray,bottom color=white,middle color=darkgray!50!white] (0,0,0) -- (0,0,3) -- (0,{-0.5-3*cos(#2)*sin(#3)/cos(#3)},3) -- (0,-0.5,0) -- cycle;
\begin{scope}[y={({(cos(#2)+sin(#2))*0.5cm},{(cos(#2)*sin(#3)-sin(#2)*sin(#3))*0.5cm})}]
\useasboundingbox (3,0,0) -- (0,0,0) -- (0,0,3);
\shade[left color=darkgray!80!black,right color=lightgray,middle color=gray] (0,0,0) -- (0,1,0) -- (0,1,0.5) -- (0,2,0) -- (0,5,0) -- (0,5,3) -- (1,5,3) -- (1,4,3) -- (1,4,2.5) -- (1,3,3) -- (2,5,3) -- (3,5,3) -- (3,0,3) -- cycle;
\clip (0,0,0) -- (0,0,3) -- ({-3*sin(#2)/cos(#2)},0,0) -- cycle;
\shade[left color=darkgray,right color=lightgray!50!gray] (0,0,0) -- (0,1,0) -- (0,1,0.5) -- (0,2,0) -- (0,5,0) -- (0,5,3) -- (1,5,3) -- (1,4,3) -- (1,4,2.5) -- (1,3,3) -- (2,5,3) -- (3,5,3) -- (3,0,3) -- cycle;
\end{scope}
\shade[left color=darkgray,right color=darkgray!80!black] (2,0,0) -- (2,0,3) -- (2,5,3) -- (2,5,1) -- (2,4,1) -- (2,4,0) -- cycle;
\shade[left color=darkgray!90!black,right color=gray!80!darkgray] (2,0,2) -- (0,0,2) -- (0,2,2) -- (2,2,2) -- cycle;
\shade[top color=darkgray!90!black,bottom color=gray!80!darkgray] (0,1,0) -- (2,1,0) -- (2,1,2) -- (0,1,2) -- cycle;
\shade[top color=darkgray!90!black,bottom color=gray!80!darkgray] (0,3,1) -- (0,3,3) -- (2,3,3) -- (2,3,0) -- (1,3,0) -- (1,3,1) -- cycle;
\fill[gray] (2,1,0) -- (1.5,1,0.5) -- (0,1,0.5) -- (0,1,0) -- cycle;
\fill[gray] (1,3,2) -- (0.5,3,2) -- (0.5,3,3) -- (1,3,3) -- cycle;
\fill[gray] (2,3,0) -- (1.5,3,0.5) -- (1,3,0.5) -- (1,3,0) -- cycle;
\ecslogosurface
\end{tikzpicture}
}

\providecommand{\cpplogo}[1]{
\begin{tikzpicture}[scale=(#1)/512em]
\fill[gray] (435.2794,398.7159) -- (247.1911,507.3075) .. controls (236.3563,513.5642) and (218.6240,513.5642) .. (207.7892,507.3075) -- (19.7009,398.7159) .. controls (8.8646,392.4606) and (0.0000,377.1043) .. (0.0000,364.5924) -- (0.0000,147.4076) .. controls (0.8430,132.8363) and (8.2856,120.7683) .. (19.7009,113.2842) -- (207.7892,4.6926) .. controls (218.6240,-1.5642) and (236.3564,-1.5642) .. (247.1911,4.6926) -- (435.2794,113.2842) .. controls (447.5273,121.4304) and (454.4987,133.6918) .. (454.9803,147.4076) -- (454.9803,364.5924) .. controls (454.5404,377.7571) and (446.6566,391.0351) .. (435.2794,398.7159) -- cycle(75.8301,255.9993) .. controls (74.9389,404.0881) and (273.2892,469.4783) .. (358.8263,331.8769) -- (293.1917,293.8965) .. controls (253.5702,359.4301) and (155.1909,335.9977) .. (151.6601,255.9993) .. controls (152.7204,182.2703) and (249.4137,148.0211) .. (293.1961,218.1065) -- (358.8308,180.1276) .. controls (283.4477,49.2645) and (79.6318,96.3470) .. (75.8301,255.9993) -- cycle(379.1503,247.5747) -- (362.2982,247.5747) -- (362.2982,230.7226) -- (345.4490,230.7226) -- (345.4490,247.5747) -- (328.5969,247.5747) -- (328.5969,264.4254) -- (345.4490,264.4254) -- (345.4490,281.2759) -- (362.2982,281.2759) -- (362.2982,264.4254) -- (379.1503,264.4254) -- cycle(442.3420,247.5747) -- (425.4899,247.5747) -- (425.4899,230.7226) -- (408.6408,230.7226) -- (408.6408,247.5747) -- (391.7886,247.5747) -- (391.7886,264.4254) -- (408.6408,264.4254) -- (408.6408,281.2759) -- (425.4899,281.2759) -- (425.4899,264.4254) -- (442.3420,264.4254) -- cycle;
\end{tikzpicture}
}

\providecommand{\fallogo}[1]{
\begin{tikzpicture}[scale=(#1)/512em]
\fill[gray] (185.7774,0.0000) .. controls (200.4486,15.9798) and (226.8966,8.7148) .. (235.0426,31.5836) .. controls (249.5297,58.0598) and (247.9581,97.9161) .. (280.3335,110.9762) .. controls (309.1690,120.3496) and (337.8406,104.2727) .. (366.5753,103.9379) .. controls (373.4449,111.5171) and (379.2885,128.2574) .. (383.9755,108.9744) .. controls (396.6979,102.5615) and (437.2808,107.6681) .. (426.9652,124.3252) .. controls (408.9822,121.0785) and (412.4742,146.0729) .. (426.5192,131.4996) .. controls (433.8413,120.8489) and (465.1541,126.5522) .. (441.9067,135.7950) .. controls (396.1879,157.7478) and (344.1112,161.5079) .. (298.5528,183.5702) .. controls (277.7471,193.5198) and (284.6941,218.7163) .. (285.2127,236.9640) .. controls (292.3599,316.2826) and (307.3929,394.6311) .. (317.1198,473.6154) .. controls (329.0637,505.4736) and (292.1195,528.5004) .. (265.9183,511.2761) .. controls (237.9284,499.2462) and (237.3684,465.2681) .. (230.9102,439.9421) .. controls (218.6692,374.3397) and (215.6307,306.9662) .. (198.1732,242.3977) .. controls (183.1379,232.7444) and (164.4245,256.0298) .. (149.0430,261.4799) .. controls (116.9328,279.2585) and (87.1822,308.5851) .. (48.2293,307.8914) .. controls (21.3220,306.9037) and (-15.9107,281.8761) .. (7.2921,252.7908) .. controls (29.7799,220.6177) and (67.5177,204.3028) .. (100.9287,185.9449) .. controls (130.8217,170.8906) and (161.1548,156.5903) .. (191.0278,141.5847) .. controls (196.1738,120.0520) and (186.6049,95.2409) .. (186.8382,72.4353) .. controls (185.5234,48.4204) and (183.1700,23.9341) .. (185.7774,0.0000) -- cycle;
\end{tikzpicture}
}

\providecommand{\oblogo}[1]{
\begin{tikzpicture}[scale=(#1)/512em]
\fill[gray] (160.3865,208.9117) .. controls (154.0879,214.6478) and (149.0735,221.2409) .. (145.4125,228.5384) .. controls (184.8790,248.4273) and (234.7122,269.8787) .. (297.5493,291.8782) .. controls (300.3943,281.4769) and (300.9552,268.7619) .. (300.4023,255.2389) .. controls (248.9909,244.7891) and (200.0310,225.9279) .. (160.3865,208.9117) -- cycle(225.7398,392.6996) .. controls (308.0209,392.1716) and (359.3326,345.9277) .. (368.7203,285.2098) .. controls (376.6742,197.1784) and (311.7194,141.3342) .. (205.4287,142.1456) .. controls (139.9485,141.4804) and (88.7155,166.1957) .. (73.5775,228.0086) .. controls (52.0297,320.3408) and (123.4078,391.0103) .. (225.7398,392.6996) -- cycle(216.0739,176.4733) .. controls (268.9183,179.2424) and (315.8292,206.5488) .. (312.7454,265.1139) .. controls (313.2769,315.6384) and (286.5993,353.4946) .. (216.6040,355.7934) .. controls (162.4657,355.7934) and (126.0914,317.5023) .. (126.0914,260.5103) .. controls (126.1733,214.2900) and (163.3363,176.2849) .. (216.0739,176.4733) -- cycle(76.4897,189.1754) .. controls (13.1586,147.5631) and (0.0000,119.4207) .. (0.0000,119.4207) -- (90.6499,170.1632) .. controls (85.3004,175.8497) and (80.5994,182.1633) .. (76.4897,189.1754) -- cycle(353.9486,119.3004) -- (402.9482,119.3004) .. controls (427.0025,137.0797) and (450.9893,162.7034) .. (474.9529,191.0213) .. controls (509.3540,228.5339) and (531.3391,294.2091) .. (487.8149,312.1206) .. controls (462.8165,324.7652) and (394.3874,316.8943) .. (373.8912,313.6651) .. controls (379.9291,297.7449) and (383.2899,278.4204) .. (381.4989,257.7214) .. controls (420.3069,248.0321) and (421.9610,218.3461) .. (407.7867,192.6417) .. controls (391.1113,162.4018) and (370.1114,132.9097) .. (353.9486,119.3004) -- cycle;
\end{tikzpicture}
}

\providecommand{\markuptable}{
\begin{table}
\sffamily\centering
\begin{tabular}{@{}lcl@{}}
\toprule
\texttt{//italics//} & $\rightarrow$ & \textit{italics} \\
\midrule
\texttt{**bold**} & $\rightarrow$ & \textbf{bold} \\
\midrule
\texttt{\# ordered list} & & 1 ordered list \\
\texttt{\# second item} & $\rightarrow$ & 2 second item \\
\texttt{\#\# sub item} & & \hspace{1em} 1 sub item \\
\midrule
\texttt{* unordered list} & & $\bullet$ unordered list \\
\texttt{* second item} & $\rightarrow$ & $\bullet$ second item \\
\texttt{** sub item} & & \hspace{1em} $\bullet$ sub item \\
\midrule
\texttt{link to [[label]]} & $\rightarrow$ & link to \underline{label} \\
\midrule
\texttt{<{}<label>{}> definition } & $\rightarrow$ & definition \\
\midrule
\texttt{[[url|link name]]} & $\rightarrow$ & \underline{link name} \\
\midrule\addlinespace
\texttt{= large heading} & & {\Large large heading} \smallskip \\
\texttt{== medium heading} & $\rightarrow$ & {\large medium heading} \\
\texttt{=== small heading} & & small heading \\
\midrule
\texttt{no line break} & & no line break for paragraphs \\
\texttt{for paragraphs} & $\rightarrow$ \\
& & use empty line \\
\texttt{use empty line} \\
\midrule
\texttt{force\textbackslash\textbackslash line break} & $\rightarrow$ & force \\
& & line break \\
\midrule
\texttt{horizontal line} & $\rightarrow$ & horizontal line \\
\texttt{----} & & \hrulefill \\
\midrule
\texttt{|=a|=table|=header} & & \underline{a \enspace table \enspace header} \\
\texttt{|a|table|row} & $\rightarrow$ & a \enspace table \enspace row \\
\texttt{|b|table|row} & & b \enspace table \enspace row \\
\midrule
\texttt{\{\{\{} \\
\texttt{unformatted} & $\rightarrow$ & \texttt{unformatted} \\
\texttt{code} & & \texttt{code} \\
\texttt{\}\}\}} \\
\midrule\addlinespace
\texttt{@ new article} & & {\Large 1.\ new article} \smallskip \\
\texttt{@ second article} & $\rightarrow$ & {\Large 2.\ second article} \smallskip \\
\texttt{@@ sub article} & & {\large 2.1.\ sub article} \\
\bottomrule
\end{tabular}
\normalfont\caption{Elements of the generic documentation markup language}
\label{tab:docmarkup}
\end{table}
}

\providecommand{\startchapter}[4]{
\documentclass[11pt,a4paper]{article}
\usepackage{booktabs}
\usepackage[format=hang,labelfont=bf]{caption}
\usepackage{changepage}
\usepackage[T1]{fontenc}
\usepackage[margin=2cm]{geometry}
\usepackage{hyperref}
\usepackage[american]{isodate}
\usepackage{lmodern}
\usepackage{longtable}
\usepackage{mathptmx}
\usepackage{microtype}
\usepackage[toc]{multitoc}
\usepackage{multirow}
\usepackage[all]{nowidow}
\usepackage{pdfcomment}
\usepackage{syntax}
\usepackage{tikz}
\usepackage[all]{xy}
\hypersetup{pdfborder={0 0 0},bookmarksnumbered=true,pdftitle={\ecs{}: #2},pdfauthor={Florian Negele},pdfsubject={\ecs{}},pdfkeywords={#1}}
\setlength{\grammarindent}{8em}\setlength{\grammarparsep}{0.2ex}
\setlength{\columnsep}{2em}
\newcommand{\prefix}{}
\newcounter{instruction}
\bibliographystyle{unsrt}
\renewcommand{\index}[2][]{}
\renewcommand{\arraystretch}{1.05}
\renewcommand{\floatpagefraction}{0.7}
\renewcommand{\syntleft}{\itshape}\renewcommand{\syntright}{}
\title{\vspace{-5ex}\Huge{\ecs{}}\medskip\hrule}
\author{\huge{#2}}
\date{\medskip\version}
\newif\ifbook\bookfalse
\pagestyle{headings}
\frenchspacing
\begin{document}
\maketitle\thispagestyle{empty}\noindent#4\setlength{\columnseprule}{0.4pt}\tableofcontents\setlength{\columnseprule}{0pt}\vfill\pagebreak[3]\null\vfill\bigskip\noindent
\parbox{\textwidth-4em}{\license The contents of this \documentation{} are part of the \href{manual}{\ecs{} User Manual}~\cite{manual} and correspond to Chapter ``\href{manual\##3}{#1}''.\alignright\mbox{\today}}
\parbox{4em}{\flushright\ecslogo{3em}}
\clearpage
}

\providecommand{\concludechapter}{
\vfill\pagebreak[3]\null\vfill
\thispagestyle{myheadings}\markright{REFERENCES}
\noindent\begin{minipage}{\textwidth}\begin{multicols}{2}[\section*{References}]
\renewcommand{\section}[2]{}\small\bibliography{references}
\end{multicols}\end{minipage}\end{document}
}

\providecommand{\startpresentation}[2]{
\documentclass[14pt,aspectratio=43,usepdftitle=false]{beamer}
\usepackage{booktabs}
\usepackage{etex}
\usepackage{multicol}
\usepackage{tikz}
\usepackage[all]{xy}
\bibliographystyle{unsrt}
\setlength{\columnsep}{1em}
\setlength{\leftmargini}{1em}
\setbeamercolor{title}{fg=black}
\setbeamercolor{structure}{fg=darkgray}
\setbeamercolor{bibliography item}{fg=darkgray}
\setbeamerfont{title}{series=\bfseries}
\setbeamerfont{subtitle}{series=\normalfont}
\setbeamerfont*{frametitle}{parent=title}
\setbeamerfont{block title}{series=\bfseries}
\setbeamerfont*{framesubtitle}{parent=subtitle}
\setbeamersize{text margin left=1em,text margin right=1em}
\setbeamertemplate{navigation symbols}{}
\setbeamertemplate{itemize item}[circle]{}
\setbeamertemplate{bibliography item}[triangle]{}
\setbeamertemplate{bibliography entry author}{\usebeamercolor[fg]{bibliography item}}
\setbeamertemplate{frametitle}{\medskip\usebeamerfont{frametitle}\color{gray}\raisebox{-2.5ex}[0ex][0ex]{\rule{0.1em}{4.5ex}}}
\addtobeamertemplate{frametitle}{}{\hspace{0.4em}\usebeamercolor[fg]{title}\insertframetitle\par\vspace{0.2ex}\hspace{0.5em}\usebeamerfont{framesubtitle}\insertframesubtitle}
\hypersetup{pdfborder={0 0 0},bookmarksnumbered=true,bookmarksopen=true,bookmarksopenlevel=0,pdftitle={\ecs{}: #1},pdfauthor={Florian Negele},pdfsubject={\ecs{}},pdfkeywords={#1}}
\renewcommand{\flowgraph}[1]{\resizebox{\textwidth}{!}{$$\xymatrix{##1}$$}}
\title{\ecs{}\medskip\hrule\medskip}
\institute{\shadowedecslogo{5em}{30}{15}}
\date{\version}
\subtitle{#1}
\begin{document}
\begin{frame}[plain]\titlepage\nocite{manual}\end{frame}
\begin{frame}{Contents}{#1}\begin{center}\tableofcontents\end{center}\end{frame}
}

\providecommand{\concludepresentation}{
\begin{frame}{References}\begin{footnotesize}\setlength{\columnseprule}{0.4pt}\begin{multicols}{2}\bibliography{references}\end{multicols}\end{footnotesize}\end{frame}
\end{document}
}

\providecommand{\startbook}[1]{
\documentclass[10pt,paper=17cm:24cm,DIV=13,twoside=semi,headings=normal,numbers=noendperiod,cleardoublepage=plain]{scrbook}
\usepackage{atveryend}
\usepackage{booktabs}
\usepackage{caption}
\usepackage{changepage}
\usepackage[T1]{fontenc}
\usepackage{imakeidx}
\usepackage{hyperref}
\usepackage[american]{isodate}
\usepackage{lmodern}
\usepackage{longtable}
\usepackage{mathptmx}
\usepackage[final]{microtype}
\usepackage{multicol}
\usepackage{multirow}
\usepackage[all]{nowidow}
\usepackage{pdfcomment}
\usepackage{scrlayer-scrpage}
\usepackage{setspace}
\usepackage{syntax}
\usepackage[eventxtindent=4pt,oddtxtexdent=4pt]{thumbs}
\usepackage{tikz}
\usepackage[all]{xy}
\hyphenation{Micro-Blaze Open-Cores Open-RISC Power-PC}
\hypersetup{pdfborder={0 0 0},bookmarksnumbered=true,bookmarksopen=true,bookmarksopenlevel=0,pdftitle={\ecs{}: #1},pdfauthor={Florian Negele},pdfsubject={\ecs{}},pdfkeywords={#1}}
\setlength{\grammarindent}{8em}\setlength{\grammarparsep}{0.7ex}
\setkomafont{captionlabel}{\usekomafont{descriptionlabel}}
\renewcommand{\arraystretch}{1.05}\setstretch{1.1}
\renewcommand{\chapterformat}{\thechapter\autodot\enskip\raisebox{-1ex}[0ex][0ex]{\color{gray}\rule{0.1em}{3.5ex}}\enskip}
\renewcommand{\startchapter}[4]{\hypertarget{##3}{\chapter{##1}}\label{##3}##4\addthumb{##1}{\LARGE\sffamily\bfseries\thechapter}{white}{gray}\renewcommand{\prefix}{##3}}
\renewcommand{\concludechapter}{\clearpage{\stopthumb\cleardoublepage}}
\renewcommand{\syntleft}{\itshape}\renewcommand{\syntright}{}
\renewcommand{\floatpagefraction}{0.7}
\renewcommand{\partheademptypage}{}
\DeclareMicrotypeAlias{lmss}{cmr}
\newcommand{\prefix}{}
\newcounter{instruction}
\bibliographystyle{unsrt}
\newif\ifbook\booktrue
\makeindex[intoc,title=Index]
\makeindex[intoc,name=tools,title=Index of Tools,columns=3]
\makeindex[intoc,name=library,title=Index of Library Names]
\makeindex[intoc,name=runtime,title=Index of Runtime Support]
\makeindex[intoc,name=environment,title=Index of Target Environments]
\indexsetup{toclevel=chapter,headers={\indexname}{\indexname}}
\frenchspacing
\begin{document}
\pagenumbering{alph}
\begin{titlepage}\centering
\huge\sffamily\null\vfill\textbf{\ecs{}}\bigskip\hrule\bigskip#1
\normalsize\normalfont\vfill\vfill\shadowedecslogo{10em}{30}{15}
\large\vfill\vfill\version
\end{titlepage}
\null\vfill
\thispagestyle{empty}
\noindent\today\par\medskip
\license A copy of this license is included in Appendix~\ref{fdl} on page~\pageref{fdl}.
All product names used herein are for identification purposes only and may be trademarks of their respective companies.
\concludechapter
\frontmatter
\setcounter{tocdepth}{1}
\tableofcontents
\setcounter{tocdepth}{2}
\concludechapter
\listoffigures
\concludechapter
\listoftables
\concludechapter
}

\providecommand{\concludebook}{
\backmatter
\addtocontents{toc}{\protect\setcounter{tocdepth}{-1}}
\phantomsection\addcontentsline{toc}{part}{Bibliography}
\bibliography{references}
\concludechapter
\phantomsection\addcontentsline{toc}{part}{Indexes}
\printindex
\concludechapter
\indexprologue{\label{idx:tools}}
\printindex[tools]
\concludechapter
\printindex[library]
\concludechapter
\indexprologue{\label{idx:runtime}}
\printindex[runtime]
\concludechapter
\indexprologue{\label{idx:environment}}
\printindex[environment]
\concludechapter
\pagestyle{empty}\pagenumbering{Alph}\null\clearpage
\null\vfill\centering\ecslogo{4em}\par\medskip\license
\end{document}
}

% chapter references

\providecommand{\seedocumentationref}{}\renewcommand{\seedocumentationref}[3]{#1, see \Documentation{}~\documentationref{#2}{#3}. }
\providecommand{\seeinterface}{}\renewcommand{\seeinterface}{\ifbook See \Documentation{}~\documentationref{interface}{User Interface} for more information about the common user interface of all of these tools. \fi}
\providecommand{\seeguide}{}\renewcommand{\seeguide}{\seedocumentationref{For basic examples of using some of these tools in practice}{guide}{User Guide}}
\providecommand{\seecpp}{}\renewcommand{\seecpp}{\seedocumentationref{For more information about the \cpp{} programming language and its implementation by the \ecs{}}{cpp}{User Manual for \cpp{}}}
\providecommand{\seefalse}{}\renewcommand{\seefalse}{\seedocumentationref{For more information about the FALSE programming language and its implementation by the \ecs{}}{false}{User Manual for FALSE}}
\providecommand{\seeoberon}{}\renewcommand{\seeoberon}{\seedocumentationref{For more information about the Oberon programming language and its implementation by the \ecs{}}{oberon}{User Manual for Oberon}}
\providecommand{\seeassembly}{}\renewcommand{\seeassembly}{\seedocumentationref{For more information about the generic assembly language and how to use it}{assembly}{Generic Assembly Language Specification}}
\providecommand{\seeamd}{}\renewcommand{\seeamd}{\seedocumentationref{For more information about how the \ecs{} supports the AMD64 hardware architecture}{amd64}{AMD64 Hardware Architecture Support}}
\providecommand{\seearm}{}\renewcommand{\seearm}{\seedocumentationref{For more information about how the \ecs{} supports the ARM hardware architecture}{arm}{ARM Hardware Architecture Support}}
\providecommand{\seeavr}{}\renewcommand{\seeavr}{\seedocumentationref{For more information about how the \ecs{} supports the AVR hardware architecture}{avr}{AVR Hardware Architecture Support}}
\providecommand{\seeavrtt}{}\renewcommand{\seeavrtt}{\seedocumentationref{For more information about how the \ecs{} supports the AVR32 hardware architecture}{avr32}{AVR32 Hardware Architecture Support}}
\providecommand{\seemabk}{}\renewcommand{\seemabk}{\seedocumentationref{For more information about how the \ecs{} supports the M68000 hardware architecture}{m68k}{M68000 Hardware Architecture Support}}
\providecommand{\seemibl}{}\renewcommand{\seemibl}{\seedocumentationref{For more information about how the \ecs{} supports the MicroBlaze hardware architecture}{mibl}{MicroBlaze Hardware Architecture Support}}
\providecommand{\seemips}{}\renewcommand{\seemips}{\seedocumentationref{For more information about how the \ecs{} supports the MIPS32 and MIPS64 hardware architectures}{mips}{MIPS Hardware Architecture Support}}
\providecommand{\seemmix}{}\renewcommand{\seemmix}{\seedocumentationref{For more information about how the \ecs{} supports the MMIX hardware architecture}{mmix}{MMIX Hardware Architecture Support}}
\providecommand{\seeorok}{}\renewcommand{\seeorok}{\seedocumentationref{For more information about how the \ecs{} supports the OpenRISC 1000 hardware architecture}{or1k}{OpenRISC 1000 Hardware Architecture Support}}
\providecommand{\seeppc}{}\renewcommand{\seeppc}{\seedocumentationref{For more information about how the \ecs{} supports the PowerPC hardware architecture}{ppc}{PowerPC Hardware Architecture Support}}
\providecommand{\seerisc}{}\renewcommand{\seerisc}{\seedocumentationref{For more information about how the \ecs{} supports the RISC hardware architecture}{risc}{RISC Hardware Architecture Support}}
\providecommand{\seewasm}{}\renewcommand{\seewasm}{\seedocumentationref{For more information about how the \ecs{} supports the WebAssembly architecture}{wasm}{WebAssembly Architecture Support}}
\providecommand{\seedocumentation}{}\renewcommand{\seedocumentation}{\seedocumentationref{For more information about generic documentations and their generation by the \ecs{}}{documentation}{Generic Documentation Generation}}
\providecommand{\seedebugging}{}\renewcommand{\seedebugging}{\seedocumentationref{For more information about debugging information and its representation}{debugging}{Debugging Information Representation}}
\providecommand{\seecode}{}\renewcommand{\seecode}{\seedocumentationref{For more information about intermediate code and its purpose}{code}{Intermediate Code Representation}}
\providecommand{\seeobject}{}\renewcommand{\seeobject}{\seedocumentationref{For more information about object files and their purpose}{object}{Object File Representation}}

% generic documentation tools

\providecommand{\docprint}{
\toolsection{docprint} is a pretty printer for generic documentations.
It reformats generic documentations and writes it to the standard output stream.
\debuggingtool
\flowgraph{\resource{generic\\documentation} \ar[r] & \toolbox{docprint} \ar[r] & \resource{generic\\documentation}}
\seedocumentation
}

\providecommand{\doccheck}{
\toolsection{doccheck} is a syntactic and semantic checker for generic documentations.
It just performs syntactic and semantic checks on generic documentations and writes its diagnostic messages to the standard error stream.
\debuggingtool
\flowgraph{\resource{generic\\documentation} \ar[r] & \toolbox{doccheck} \ar[r] & \resource{diagnostic\\messages}}
\seedocumentation
}

\providecommand{\dochtml}{
\toolsection{dochtml} is an HTML documentation generator for generic documentations.
It processes several generic documentations and assembles all information therein into an HTML document.
\debuggingtool
\flowgraph{\resource{generic\\documentation} \ar[r] & \toolbox{dochtml} \ar[r] & \resource{HTML\\document}}
\seedocumentation
}

\providecommand{\doclatex}{
\toolsection{doclatex} is a Latex documentation generator for generic documentations.
It processes several generic documentations and assembles all information therein into a Latex document.
\debuggingtool
\flowgraph{\resource{generic\\documentation} \ar[r] & \toolbox{doclatex} \ar[r] & \resource{Latex\\document}}
\seedocumentation
}

% intermediate code tools

\providecommand{\cdcheck}{
\toolsection{cdcheck} is a syntactic and semantic checker for intermediate code.
It just performs syntactic and semantic checks on programs written in intermediate code and writes its diagnostic messages to the standard error stream.
\debuggingtool
\flowgraph{\resource{intermediate\\code} \ar[r] & \toolbox{cdcheck} \ar[r] & \resource{diagnostic\\messages}}
\seeassembly\seecode
}

\providecommand{\cdopt}{
\toolsection{cdopt} is an optimizer for intermediate code.
It performs various optimizations on programs written in intermediate code and writes the result to the standard output stream.
\debuggingtool
\flowgraph{\resource{intermediate\\code} \ar[r] & \toolbox{cdopt} \ar[r] & \resource{optimized\\code}}
\seeassembly\seecode
}

\providecommand{\cdrun}{
\toolsection{cdrun} is an interpreter for intermediate code.
It processes and executes programs written in intermediate code.
The following code sections are predefined and have the usual semantics:
\texttt{abort}, \texttt{\_Exit}, \texttt{fflush}, \texttt{floor}, \texttt{fputc}, \texttt{free}, \texttt{getchar}, \texttt{malloc}, and \texttt{putchar}.
Diagnostic messages about invalid operations include the name of the executed code section and the index of the erroneous instruction.
\debuggingtool
\flowgraph{\resource{intermediate\\code} \ar[r] & \toolbox{cdrun} \ar@/u/[r] & \resource{input/\\output} \ar@/d/[l]}
\seeassembly\seecode
}

\providecommand{\cdamda}{
\toolsection{cdamd16} is a compiler for intermediate code targeting the AMD64 hardware architecture.
It generates machine code for AMD64 processors from programs written in intermediate code and stores it in corresponding object files.
The compiler generates machine code for the 16-bit operating mode defined by the AMD64 architecture.
It also creates a debugging information file as well as an assembly file containing a listing of the generated machine code.
\debuggingtool
\flowgraph{\resource{intermediate\\code} \ar[r] & \toolbox{cdamd16} \ar[r] \ar[d] \ar[rd] & \resource{object file} \\ & \resource{assembly\\listing} & \resource{debugging\\information}}
\seeassembly\seeamd\seeobject\seecode\seedebugging
}

\providecommand{\cdamdb}{
\toolsection{cdamd32} is a compiler for intermediate code targeting the AMD64 hardware architecture.
It generates machine code for AMD64 processors from programs written in intermediate code and stores it in corresponding object files.
The compiler generates machine code for the 32-bit operating mode defined by the AMD64 architecture.
It also creates a debugging information file as well as an assembly file containing a listing of the generated machine code.
\debuggingtool
\flowgraph{\resource{intermediate\\code} \ar[r] & \toolbox{cdamd32} \ar[r] \ar[d] \ar[rd] & \resource{object file} \\ & \resource{assembly\\listing} & \resource{debugging\\information}}
\seeassembly\seeamd\seeobject\seecode\seedebugging
}

\providecommand{\cdamdc}{
\toolsection{cdamd64} is a compiler for intermediate code targeting the AMD64 hardware architecture.
It generates machine code for AMD64 processors from programs written in intermediate code and stores it in corresponding object files.
The compiler generates machine code for the 64-bit operating mode defined by the AMD64 architecture.
It also creates a debugging information file as well as an assembly file containing a listing of the generated machine code.
\debuggingtool
\flowgraph{\resource{intermediate\\code} \ar[r] & \toolbox{cdamd64} \ar[r] \ar[d] \ar[rd] & \resource{object file} \\ & \resource{assembly\\listing} & \resource{debugging\\information}}
\seeassembly\seeamd\seeobject\seecode\seedebugging
}

\providecommand{\cdarma}{
\toolsection{cdarma32} is a compiler for intermediate code targeting the ARM hardware architecture.
It generates machine code for ARM processors executing A32 instructions from programs written in intermediate code and stores it in corresponding object files.
It also creates a debugging information file as well as an assembly file containing a listing of the generated machine code.
\debuggingtool
\flowgraph{\resource{intermediate\\code} \ar[r] & \toolbox{cdarma32} \ar[r] \ar[d] \ar[rd] & \resource{object file} \\ & \resource{assembly\\listing} & \resource{debugging\\information}}
\seeassembly\seearm\seeobject\seecode\seedebugging
}

\providecommand{\cdarmb}{
\toolsection{cdarma64} is a compiler for intermediate code targeting the ARM hardware architecture.
It generates machine code for ARM processors executing A64 instructions from programs written in intermediate code and stores it in corresponding object files.
It also creates a debugging information file as well as an assembly file containing a listing of the generated machine code.
\debuggingtool
\flowgraph{\resource{intermediate\\code} \ar[r] & \toolbox{cdarma64} \ar[r] \ar[d] \ar[rd] & \resource{object file} \\ & \resource{assembly\\listing} & \resource{debugging\\information}}
\seeassembly\seearm\seeobject\seecode\seedebugging
}

\providecommand{\cdarmc}{
\toolsection{cdarmt32} is a compiler for intermediate code targeting the ARM hardware architecture.
It generates machine code for ARM processors without floating-point extension executing T32 instructions from programs written in intermediate code and stores it in corresponding object files.
It also creates a debugging information file as well as an assembly file containing a listing of the generated machine code.
\debuggingtool
\flowgraph{\resource{intermediate\\code} \ar[r] & \toolbox{cdarmt32} \ar[r] \ar[d] \ar[rd] & \resource{object file} \\ & \resource{assembly\\listing} & \resource{debugging\\information}}
\seeassembly\seearm\seeobject\seecode\seedebugging
}

\providecommand{\cdarmcfpe}{
\toolsection{cdarmt32fpe} is a compiler for intermediate code targeting the ARM hardware architecture.
It generates machine code for ARM processors with floating-point extension executing T32 instructions from programs written in intermediate code and stores it in corresponding object files.
It also creates a debugging information file as well as an assembly file containing a listing of the generated machine code.
\debuggingtool
\flowgraph{\resource{intermediate\\code} \ar[r] & \toolbox{cdarmt32fpe} \ar[r] \ar[d] \ar[rd] & \resource{object file} \\ & \resource{assembly\\listing} & \resource{debugging\\information}}
\seeassembly\seearm\seeobject\seecode\seedebugging
}

\providecommand{\cdavr}{
\toolsection{cdavr} is a compiler for intermediate code targeting the AVR hardware architecture.
It generates machine code for AVR processors from programs written in intermediate code and stores it in corresponding object files.
It also creates a debugging information file as well as an assembly file containing a listing of the generated machine code.
\debuggingtool
\flowgraph{\resource{intermediate\\code} \ar[r] & \toolbox{cdavr} \ar[r] \ar[d] \ar[rd] & \resource{object file} \\ & \resource{assembly\\listing} & \resource{debugging\\information}}
\seeassembly\seeavr\seeobject\seecode\seedebugging
}

\providecommand{\cdavrtt}{
\toolsection{cdavr32} is a compiler for intermediate code targeting the AVR32 hardware architecture.
It generates machine code for AVR32 processors from programs written in intermediate code and stores it in corresponding object files.
It also creates a debugging information file as well as an assembly file containing a listing of the generated machine code.
\debuggingtool
\flowgraph{\resource{intermediate\\code} \ar[r] & \toolbox{cdavr32} \ar[r] \ar[d] \ar[rd] & \resource{object file} \\ & \resource{assembly\\listing} & \resource{debugging\\information}}
\seeassembly\seeavrtt\seeobject\seecode\seedebugging
}

\providecommand{\cdmabk}{
\toolsection{cdm68k} is a compiler for intermediate code targeting the M68000 hardware architecture.
It generates machine code for M68000 processors from programs written in intermediate code and stores it in corresponding object files.
It also creates a debugging information file as well as an assembly file containing a listing of the generated machine code.
\debuggingtool
\flowgraph{\resource{intermediate\\code} \ar[r] & \toolbox{cdm68k} \ar[r] \ar[d] \ar[rd] & \resource{object file} \\ & \resource{assembly\\listing} & \resource{debugging\\information}}
\seeassembly\seemabk\seeobject\seecode\seedebugging
}

\providecommand{\cdmibl}{
\toolsection{cdmibl} is a compiler for intermediate code targeting the MicroBlaze hardware architecture.
It generates machine code for MicroBlaze processors from programs written in intermediate code and stores it in corresponding object files.
It also creates a debugging information file as well as an assembly file containing a listing of the generated machine code.
\debuggingtool
\flowgraph{\resource{intermediate\\code} \ar[r] & \toolbox{cdmibl} \ar[r] \ar[d] \ar[rd] & \resource{object file} \\ & \resource{assembly\\listing} & \resource{debugging\\information}}
\seeassembly\seemibl\seeobject\seecode\seedebugging
}

\providecommand{\cdmipsa}{
\toolsection{cdmips32} is a compiler for intermediate code targeting the MIPS32 hardware architecture.
It generates machine code for MIPS32 processors from programs written in intermediate code and stores it in corresponding object files.
It also creates a debugging information file as well as an assembly file containing a listing of the generated machine code.
\debuggingtool
\flowgraph{\resource{intermediate\\code} \ar[r] & \toolbox{cdmips32} \ar[r] \ar[d] \ar[rd] & \resource{object file} \\ & \resource{assembly\\listing} & \resource{debugging\\information}}
\seeassembly\seemips\seeobject\seecode\seedebugging
}

\providecommand{\cdmipsb}{
\toolsection{cdmips64} is a compiler for intermediate code targeting the MIPS64 hardware architecture.
It generates machine code for MIPS64 processors from programs written in intermediate code and stores it in corresponding object files.
It also creates a debugging information file as well as an assembly file containing a listing of the generated machine code.
\debuggingtool
\flowgraph{\resource{intermediate\\code} \ar[r] & \toolbox{cdmips64} \ar[r] \ar[d] \ar[rd] & \resource{object file} \\ & \resource{assembly\\listing} & \resource{debugging\\information}}
\seeassembly\seemips\seeobject\seecode\seedebugging
}

\providecommand{\cdmmix}{
\toolsection{cdmmix} is a compiler for intermediate code targeting the MMIX hardware architecture.
It generates machine code for MMIX processors from programs written in intermediate code and stores it in corresponding object files.
It also creates a debugging information file as well as an assembly file containing a listing of the generated machine code.
\debuggingtool
\flowgraph{\resource{intermediate\\code} \ar[r] & \toolbox{cdmmix} \ar[r] \ar[d] \ar[rd] & \resource{object file} \\ & \resource{assembly\\listing} & \resource{debugging\\information}}
\seeassembly\seemmix\seeobject\seecode\seedebugging
}

\providecommand{\cdorok}{
\toolsection{cdor1k} is a compiler for intermediate code targeting the OpenRISC 1000 hardware architecture.
It generates machine code for OpenRISC 1000 processors from programs written in intermediate code and stores it in corresponding object files.
It also creates a debugging information file as well as an assembly file containing a listing of the generated machine code.
\debuggingtool
\flowgraph{\resource{intermediate\\code} \ar[r] & \toolbox{cdor1k} \ar[r] \ar[d] \ar[rd] & \resource{object file} \\ & \resource{assembly\\listing} & \resource{debugging\\information}}
\seeassembly\seeorok\seeobject\seecode\seedebugging
}

\providecommand{\cdppca}{
\toolsection{cdppc32} is a compiler for intermediate code targeting the PowerPC hardware architecture.
It generates machine code for PowerPC processors from programs written in intermediate code and stores it in corresponding object files.
The compiler generates machine code for the 32-bit operating mode defined by the PowerPC architecture.
It also creates a debugging information file as well as an assembly file containing a listing of the generated machine code.
\debuggingtool
\flowgraph{\resource{intermediate\\code} \ar[r] & \toolbox{cdppc32} \ar[r] \ar[d] \ar[rd] & \resource{object file} \\ & \resource{assembly\\listing} & \resource{debugging\\information}}
\seeassembly\seeppc\seeobject\seecode\seedebugging
}

\providecommand{\cdppcb}{
\toolsection{cdppc64} is a compiler for intermediate code targeting the PowerPC hardware architecture.
It generates machine code for PowerPC processors from programs written in intermediate code and stores it in corresponding object files.
The compiler generates machine code for the 64-bit operating mode defined by the PowerPC architecture.
It also creates a debugging information file as well as an assembly file containing a listing of the generated machine code.
\debuggingtool
\flowgraph{\resource{intermediate\\code} \ar[r] & \toolbox{cdppc64} \ar[r] \ar[d] \ar[rd] & \resource{object file} \\ & \resource{assembly\\listing} & \resource{debugging\\information}}
\seeassembly\seeppc\seeobject\seecode\seedebugging
}

\providecommand{\cdrisc}{
\toolsection{cdrisc} is a compiler for intermediate code targeting the RISC hardware architecture.
It generates machine code for RISC processors from programs written in intermediate code and stores it in corresponding object files.
It also creates a debugging information file as well as an assembly file containing a listing of the generated machine code.
\debuggingtool
\flowgraph{\resource{intermediate\\code} \ar[r] & \toolbox{cdrisc} \ar[r] \ar[d] \ar[rd] & \resource{object file} \\ & \resource{assembly\\listing} & \resource{debugging\\information}}
\seeassembly\seerisc\seeobject\seecode\seedebugging
}

\providecommand{\cdwasm}{
\toolsection{cdwasm} is a compiler for intermediate code targeting the WebAssembly architecture.
It generates machine code for WebAssembly targets from programs written in intermediate code and stores it in corresponding object files.
It also creates a debugging information file as well as an assembly file containing a listing of the generated machine code.
\debuggingtool
\flowgraph{\resource{intermediate\\code} \ar[r] & \toolbox{cdwasm} \ar[r] \ar[d] \ar[rd] & \resource{object file} \\ & \resource{assembly\\listing} & \resource{debugging\\information}}
\seeassembly\seewasm\seeobject\seecode\seedebugging
}

% C++ tools

\providecommand{\cppprep}{
\toolsection{cppprep} is a preprocessor for the \cpp{} programming language.
It preprocesses source code according to the rules of \cpp{} and writes it to the standard output stream.
Only the macro names \texttt{\_\_DATE\_\_}, \texttt{\_\_FILE\_\_}, \texttt{\_\_LINE\_\_}, and \texttt{\_\_TIME\_\_} are predefined.
\flowgraph{\resource{\cpp{} or other\\source code} \ar[r] & \toolbox{cppprep} \ar[r] & \resource{preprocessed\\source code} \\ & \variable{ECSINCLUDE} \ar[u]}
\seecpp
}

\providecommand{\cppprint}{
\toolsection{cppprint} is a pretty printer for the \cpp{} programming language.
It reformats the source code of \cpp{} programs and writes it to the standard output stream.
\flowgraph{\resource{\cpp{}\\source code} \ar[r] & \toolbox{cppprint} \ar[r] & \resource{reformatted\\source code} \\ & \variable{ECSINCLUDE} \ar[u]}
\seecpp
}

\providecommand{\cppcheck}{
\toolsection{cppcheck} is a syntactic and semantic checker for the \cpp{} programming language.
It just performs syntactic and semantic checks on \cpp{} programs and writes its diagnostic messages to the standard error stream.
\flowgraph{\resource{\cpp{}\\source code} \ar[r] & \toolbox{cppcheck} \ar[r] & \resource{diagnostic\\messages} \\ & \variable{ECSINCLUDE} \ar[u]}
\seecpp
}

\providecommand{\cppdump}{
\toolsection{cppdump} is a serializer for the \cpp{} programming language.
It dumps the complete internal representation of programs written in \cpp{} into an XML document.
\debuggingtool
\flowgraph{\resource{\cpp{}\\source code} \ar[r] & \toolbox{cppdump} \ar[r] & \resource{internal\\representation} \\ & \variable{ECSINCLUDE} \ar[u]}
\seecpp
}

\providecommand{\cpprun}{
\toolsection{cpprun} is an interpreter for the \cpp{} programming language.
It processes and executes programs written in \cpp{}.
The macro \texttt{\_\_run\_\_} is predefined in order to enable programmers to identify this tool while interpreting.
\flowgraph{\resource{\cpp{}\\source code} \ar[r] & \toolbox{cpprun} \ar@/u/[r] & \resource{input/\\output} \ar@/d/[l] \\ & \variable{ECSINCLUDE} \ar[u]}
\seecpp
}

\providecommand{\cppdoc}{
\toolsection{cppdoc} is a generic documentation generator for the \cpp{} programming language.
It processes several \cpp{} source files and assembles all information therein into a generic documentation.
\debuggingtool
\flowgraph{\resource{\cpp{}\\source code} \ar[r] & \toolbox{cppdoc} \ar[r] & \resource{generic\\documentation} \\ & \variable{ECSINCLUDE} \ar[u]}
\seecpp\seedocumentation
}

\providecommand{\cpphtml}{
\toolsection{cpphtml} is an HTML documentation generator for the \cpp{} programming language.
It processes several \cpp{} source files and assembles all information therein into an HTML document.
\flowgraph{\resource{\cpp{}\\source code} \ar[r] & \toolbox{cpphtml} \ar[r] & \resource{HTML\\document} \\ & \variable{ECSINCLUDE} \ar[u]}
\seecpp\seedocumentation
}

\providecommand{\cpplatex}{
\toolsection{cpplatex} is a Latex documentation generator for the \cpp{} programming language.
It processes several \cpp{} source files and assembles all information therein into a Latex document.
\flowgraph{\resource{\cpp{}\\source code} \ar[r] & \toolbox{cpplatex} \ar[r] & \resource{Latex\\document} \\ & \variable{ECSINCLUDE} \ar[u]}
\seecpp\seedocumentation
}

\providecommand{\cppcode}{
\toolsection{cppcode} is an intermediate code generator for the \cpp{} programming language.
It generates intermediate code from programs written in \cpp{} and stores it in corresponding assembly files.
The macro \texttt{\_\_code\_\_} is predefined in order to enable programmers to identify this tool while generating intermediate code.
Programs generated with this tool require additional runtime support that is stored in the \file{cpp\-code\-run} library file.
\debuggingtool
\flowgraph{\resource{\cpp{}\\source code} \ar[r] & \toolbox{cppcode} \ar[r] & \resource{intermediate\\code} \\ & \variable{ECSINCLUDE} \ar[u]}
\seecpp\seeassembly\seecode
}

\providecommand{\cppamda}{
\toolsection{cppamd16} is a compiler for the \cpp{} programming language targeting the AMD64 hardware architecture.
It generates machine code for AMD64 processors from programs written in \cpp{} and stores it in corresponding object files.
The compiler generates machine code for the 16-bit operating mode defined by the AMD64 architecture.
For debugging purposes, it also creates a debugging information file as well as an assembly file containing a listing of the generated machine code.
The macro \texttt{\_\_amd16\_\_} is predefined in order to enable programmers to identify this tool and its target architecture while compiling.
Programs generated with this compiler require additional runtime support that is stored in the \file{cpp\-amd16\-run} library file.
\flowgraph{\resource{\cpp{}\\source code} \ar[r] & \toolbox{cppamd16} \ar[r] \ar[d] \ar[rd] & \resource{object file} \\ \variable{ECSINCLUDE} \ar[ru] & \resource{debugging\\information} & \resource{assembly\\listing}}
\seecpp\seeassembly\seeamd\seeobject\seedebugging
}

\providecommand{\cppamdb}{
\toolsection{cppamd32} is a compiler for the \cpp{} programming language targeting the AMD64 hardware architecture.
It generates machine code for AMD64 processors from programs written in \cpp{} and stores it in corresponding object files.
The compiler generates machine code for the 32-bit operating mode defined by the AMD64 architecture.
For debugging purposes, it also creates a debugging information file as well as an assembly file containing a listing of the generated machine code.
The macro \texttt{\_\_amd32\_\_} is predefined in order to enable programmers to identify this tool and its target architecture while compiling.
Programs generated with this compiler require additional runtime support that is stored in the \file{cpp\-amd32\-run} library file.
\flowgraph{\resource{\cpp{}\\source code} \ar[r] & \toolbox{cppamd32} \ar[r] \ar[d] \ar[rd] & \resource{object file} \\ \variable{ECSINCLUDE} \ar[ru] & \resource{debugging\\information} & \resource{assembly\\listing}}
\seecpp\seeassembly\seeamd\seeobject\seedebugging
}

\providecommand{\cppamdc}{
\toolsection{cppamd64} is a compiler for the \cpp{} programming language targeting the AMD64 hardware architecture.
It generates machine code for AMD64 processors from programs written in \cpp{} and stores it in corresponding object files.
The compiler generates machine code for the 64-bit operating mode defined by the AMD64 architecture.
For debugging purposes, it also creates a debugging information file as well as an assembly file containing a listing of the generated machine code.
The macro \texttt{\_\_amd64\_\_} is predefined in order to enable programmers to identify this tool and its target architecture while compiling.
Programs generated with this compiler require additional runtime support that is stored in the \file{cpp\-amd64\-run} library file.
\flowgraph{\resource{\cpp{}\\source code} \ar[r] & \toolbox{cppamd64} \ar[r] \ar[d] \ar[rd] & \resource{object file} \\ \variable{ECSINCLUDE} \ar[ru] & \resource{debugging\\information} & \resource{assembly\\listing}}
\seecpp\seeassembly\seeamd\seeobject\seedebugging
}

\providecommand{\cpparma}{
\toolsection{cpparma32} is a compiler for the \cpp{} programming language targeting the ARM hardware architecture.
It generates machine code for ARM processors executing A32 instructions from programs written in \cpp{} and stores it in corresponding object files.
For debugging purposes, it also creates a debugging information file as well as an assembly file containing a listing of the generated machine code.
The macro \texttt{\_\_arma32\_\_} is predefined in order to enable programmers to identify this tool and its target architecture while compiling.
Programs generated with this compiler require additional runtime support that is stored in the \file{cpp\-arma32\-run} library file.
\flowgraph{\resource{\cpp{}\\source code} \ar[r] & \toolbox{cpparma32} \ar[r] \ar[d] \ar[rd] & \resource{object file} \\ \variable{ECSINCLUDE} \ar[ru] & \resource{debugging\\information} & \resource{assembly\\listing}}
\seecpp\seeassembly\seearm\seeobject\seedebugging
}

\providecommand{\cpparmb}{
\toolsection{cpparma64} is a compiler for the \cpp{} programming language targeting the ARM hardware architecture.
It generates machine code for ARM processors executing A64 instructions from programs written in \cpp{} and stores it in corresponding object files.
For debugging purposes, it also creates a debugging information file as well as an assembly file containing a listing of the generated machine code.
The macro \texttt{\_\_arma64\_\_} is predefined in order to enable programmers to identify this tool and its target architecture while compiling.
Programs generated with this compiler require additional runtime support that is stored in the \file{cpp\-arma64\-run} library file.
\flowgraph{\resource{\cpp{}\\source code} \ar[r] & \toolbox{cpparma64} \ar[r] \ar[d] \ar[rd] & \resource{object file} \\ \variable{ECSINCLUDE} \ar[ru] & \resource{debugging\\information} & \resource{assembly\\listing}}
\seecpp\seeassembly\seearm\seeobject\seedebugging
}

\providecommand{\cpparmc}{
\toolsection{cpparmt32} is a compiler for the \cpp{} programming language targeting the ARM hardware architecture.
It generates machine code for ARM processors without floating-point extension executing T32 instructions from programs written in \cpp{} and stores it in corresponding object files.
For debugging purposes, it also creates a debugging information file as well as an assembly file containing a listing of the generated machine code.
The macro \texttt{\_\_armt32\_\_} is predefined in order to enable programmers to identify this tool and its target architecture while compiling.
Programs generated with this compiler require additional runtime support that is stored in the \file{cpp\-armt32\-run} library file.
\flowgraph{\resource{\cpp{}\\source code} \ar[r] & \toolbox{cpparmt32} \ar[r] \ar[d] \ar[rd] & \resource{object file} \\ \variable{ECSINCLUDE} \ar[ru] & \resource{debugging\\information} & \resource{assembly\\listing}}
\seecpp\seeassembly\seearm\seeobject\seedebugging
}

\providecommand{\cpparmcfpe}{
\toolsection{cpparmt32fpe} is a compiler for the \cpp{} programming language targeting the ARM hardware architecture.
It generates machine code for ARM processors with floating-point extension executing T32 instructions from programs written in \cpp{} and stores it in corresponding object files.
For debugging purposes, it also creates a debugging information file as well as an assembly file containing a listing of the generated machine code.
The macro \texttt{\_\_armt32fpe\_\_} is predefined in order to enable programmers to identify this tool and its target architecture while compiling.
Programs generated with this compiler require additional runtime support that is stored in the \file{cpp\-armt32\-fpe\-run} library file.
\flowgraph{\resource{\cpp{}\\source code} \ar[r] & \toolbox{cpparmt32fpe} \ar[r] \ar[d] \ar[rd] & \resource{object file} \\ \variable{ECSINCLUDE} \ar[ru] & \resource{debugging\\information} & \resource{assembly\\listing}}
\seecpp\seeassembly\seearm\seeobject\seedebugging
}

\providecommand{\cppavr}{
\toolsection{cppavr} is a compiler for the \cpp{} programming language targeting the AVR hardware architecture.
It generates machine code for AVR processors from programs written in \cpp{} and stores it in corresponding object files.
For debugging purposes, it also creates a debugging information file as well as an assembly file containing a listing of the generated machine code.
The macro \texttt{\_\_avr\_\_} is predefined in order to enable programmers to identify this tool and its target architecture while compiling.
Programs generated with this compiler require additional runtime support that is stored in the \file{cpp\-avr\-run} library file.
\flowgraph{\resource{\cpp{}\\source code} \ar[r] & \toolbox{cppavr} \ar[r] \ar[d] \ar[rd] & \resource{object file} \\ \variable{ECSINCLUDE} \ar[ru] & \resource{debugging\\information} & \resource{assembly\\listing}}
\seecpp\seeassembly\seeavr\seeobject\seedebugging
}

\providecommand{\cppavrtt}{
\toolsection{cppavr32} is a compiler for the \cpp{} programming language targeting the AVR32 hardware architecture.
It generates machine code for AVR32 processors from programs written in \cpp{} and stores it in corresponding object files.
For debugging purposes, it also creates a debugging information file as well as an assembly file containing a listing of the generated machine code.
The macro \texttt{\_\_avr32\_\_} is predefined in order to enable programmers to identify this tool and its target architecture while compiling.
Programs generated with this compiler require additional runtime support that is stored in the \file{cpp\-avr32\-run} library file.
\flowgraph{\resource{\cpp{}\\source code} \ar[r] & \toolbox{cppavr32} \ar[r] \ar[d] \ar[rd] & \resource{object file} \\ \variable{ECSINCLUDE} \ar[ru] & \resource{debugging\\information} & \resource{assembly\\listing}}
\seecpp\seeassembly\seeavrtt\seeobject\seedebugging
}

\providecommand{\cppmabk}{
\toolsection{cppm68k} is a compiler for the \cpp{} programming language targeting the M68000 hardware architecture.
It generates machine code for M68000 processors from programs written in \cpp{} and stores it in corresponding object files.
For debugging purposes, it also creates a debugging information file as well as an assembly file containing a listing of the generated machine code.
The macro \texttt{\_\_m68k\_\_} is predefined in order to enable programmers to identify this tool and its target architecture while compiling.
Programs generated with this compiler require additional runtime support that is stored in the \file{cpp\-m68k\-run} library file.
\flowgraph{\resource{\cpp{}\\source code} \ar[r] & \toolbox{cppm68k} \ar[r] \ar[d] \ar[rd] & \resource{object file} \\ \variable{ECSINCLUDE} \ar[ru] & \resource{debugging\\information} & \resource{assembly\\listing}}
\seecpp\seeassembly\seemabk\seeobject\seedebugging
}

\providecommand{\cppmibl}{
\toolsection{cppmibl} is a compiler for the \cpp{} programming language targeting the MicroBlaze hardware architecture.
It generates machine code for MicroBlaze processors from programs written in \cpp{} and stores it in corresponding object files.
For debugging purposes, it also creates a debugging information file as well as an assembly file containing a listing of the generated machine code.
The macro \texttt{\_\_mibl\_\_} is predefined in order to enable programmers to identify this tool and its target architecture while compiling.
Programs generated with this compiler require additional runtime support that is stored in the \file{cpp\-mibl\-run} library file.
\flowgraph{\resource{\cpp{}\\source code} \ar[r] & \toolbox{cppmibl} \ar[r] \ar[d] \ar[rd] & \resource{object file} \\ \variable{ECSINCLUDE} \ar[ru] & \resource{debugging\\information} & \resource{assembly\\listing}}
\seecpp\seeassembly\seemibl\seeobject\seedebugging
}

\providecommand{\cppmipsa}{
\toolsection{cppmips32} is a compiler for the \cpp{} programming language targeting the MIPS32 hardware architecture.
It generates machine code for MIPS32 processors from programs written in \cpp{} and stores it in corresponding object files.
For debugging purposes, it also creates a debugging information file as well as an assembly file containing a listing of the generated machine code.
The macro \texttt{\_\_mips32\_\_} is predefined in order to enable programmers to identify this tool and its target architecture while compiling.
Programs generated with this compiler require additional runtime support that is stored in the \file{cpp\-mips32\-run} library file.
\flowgraph{\resource{\cpp{}\\source code} \ar[r] & \toolbox{cppmips32} \ar[r] \ar[d] \ar[rd] & \resource{object file} \\ \variable{ECSINCLUDE} \ar[ru] & \resource{debugging\\information} & \resource{assembly\\listing}}
\seecpp\seeassembly\seemips\seeobject\seedebugging
}

\providecommand{\cppmipsb}{
\toolsection{cppmips64} is a compiler for the \cpp{} programming language targeting the MIPS64 hardware architecture.
It generates machine code for MIPS64 processors from programs written in \cpp{} and stores it in corresponding object files.
For debugging purposes, it also creates a debugging information file as well as an assembly file containing a listing of the generated machine code.
The macro \texttt{\_\_mips64\_\_} is predefined in order to enable programmers to identify this tool and its target architecture while compiling.
Programs generated with this compiler require additional runtime support that is stored in the \file{cpp\-mips64\-run} library file.
\flowgraph{\resource{\cpp{}\\source code} \ar[r] & \toolbox{cppmips64} \ar[r] \ar[d] \ar[rd] & \resource{object file} \\ \variable{ECSINCLUDE} \ar[ru] & \resource{debugging\\information} & \resource{assembly\\listing}}
\seecpp\seeassembly\seemips\seeobject\seedebugging
}

\providecommand{\cppmmix}{
\toolsection{cppmmix} is a compiler for the \cpp{} programming language targeting the MMIX hardware architecture.
It generates machine code for MMIX processors from programs written in \cpp{} and stores it in corresponding object files.
For debugging purposes, it also creates a debugging information file as well as an assembly file containing a listing of the generated machine code.
The macro \texttt{\_\_mmix\_\_} is predefined in order to enable programmers to identify this tool and its target architecture while compiling.
Programs generated with this compiler require additional runtime support that is stored in the \file{cpp\-mmix\-run} library file.
\flowgraph{\resource{\cpp{}\\source code} \ar[r] & \toolbox{cppmmix} \ar[r] \ar[d] \ar[rd] & \resource{object file} \\ \variable{ECSINCLUDE} \ar[ru] & \resource{debugging\\information} & \resource{assembly\\listing}}
\seecpp\seeassembly\seemmix\seeobject\seedebugging
}

\providecommand{\cpporok}{
\toolsection{cppor1k} is a compiler for the \cpp{} programming language targeting the OpenRISC 1000 hardware architecture.
It generates machine code for OpenRISC 1000 processors from programs written in \cpp{} and stores it in corresponding object files.
For debugging purposes, it also creates a debugging information file as well as an assembly file containing a listing of the generated machine code.
The macro \texttt{\_\_or1k\_\_} is predefined in order to enable programmers to identify this tool and its target architecture while compiling.
Programs generated with this compiler require additional runtime support that is stored in the \file{cpp\-or1k\-run} library file.
\flowgraph{\resource{\cpp{}\\source code} \ar[r] & \toolbox{cppor1k} \ar[r] \ar[d] \ar[rd] & \resource{object file} \\ \variable{ECSINCLUDE} \ar[ru] & \resource{debugging\\information} & \resource{assembly\\listing}}
\seecpp\seeassembly\seeorok\seeobject\seedebugging
}

\providecommand{\cppppca}{
\toolsection{cppppc32} is a compiler for the \cpp{} programming language targeting the PowerPC hardware architecture.
It generates machine code for PowerPC processors from programs written in \cpp{} and stores it in corresponding object files.
The compiler generates machine code for the 32-bit operating mode defined by the PowerPC architecture.
For debugging purposes, it also creates a debugging information file as well as an assembly file containing a listing of the generated machine code.
The macro \texttt{\_\_ppc32\_\_} is predefined in order to enable programmers to identify this tool and its target architecture while compiling.
Programs generated with this compiler require additional runtime support that is stored in the \file{cpp\-ppc32\-run} library file.
\flowgraph{\resource{\cpp{}\\source code} \ar[r] & \toolbox{cppppc32} \ar[r] \ar[d] \ar[rd] & \resource{object file} \\ \variable{ECSINCLUDE} \ar[ru] & \resource{debugging\\information} & \resource{assembly\\listing}}
\seecpp\seeassembly\seeppc\seeobject\seedebugging
}

\providecommand{\cppppcb}{
\toolsection{cppppc64} is a compiler for the \cpp{} programming language targeting the PowerPC hardware architecture.
It generates machine code for PowerPC processors from programs written in \cpp{} and stores it in corresponding object files.
The compiler generates machine code for the 64-bit operating mode defined by the PowerPC architecture.
For debugging purposes, it also creates a debugging information file as well as an assembly file containing a listing of the generated machine code.
The macro \texttt{\_\_ppc64\_\_} is predefined in order to enable programmers to identify this tool and its target architecture while compiling.
Programs generated with this compiler require additional runtime support that is stored in the \file{cpp\-ppc64\-run} library file.
\flowgraph{\resource{\cpp{}\\source code} \ar[r] & \toolbox{cppppc64} \ar[r] \ar[d] \ar[rd] & \resource{object file} \\ \variable{ECSINCLUDE} \ar[ru] & \resource{debugging\\information} & \resource{assembly\\listing}}
\seecpp\seeassembly\seeppc\seeobject\seedebugging
}

\providecommand{\cpprisc}{
\toolsection{cpprisc} is a compiler for the \cpp{} programming language targeting the RISC hardware architecture.
It generates machine code for RISC processors from programs written in \cpp{} and stores it in corresponding object files.
For debugging purposes, it also creates a debugging information file as well as an assembly file containing a listing of the generated machine code.
The macro \texttt{\_\_risc\_\_} is predefined in order to enable programmers to identify this tool and its target architecture while compiling.
Programs generated with this compiler require additional runtime support that is stored in the \file{cpp\-risc\-run} library file.
\flowgraph{\resource{\cpp{}\\source code} \ar[r] & \toolbox{cpprisc} \ar[r] \ar[d] \ar[rd] & \resource{object file} \\ \variable{ECSINCLUDE} \ar[ru] & \resource{debugging\\information} & \resource{assembly\\listing}}
\seecpp\seeassembly\seerisc\seeobject\seedebugging
}

\providecommand{\cppwasm}{
\toolsection{cppwasm} is a compiler for the \cpp{} programming language targeting the WebAssembly architecture.
It generates machine code for WebAssembly targets from programs written in \cpp{} and stores it in corresponding object files.
For debugging purposes, it also creates a debugging information file as well as an assembly file containing a listing of the generated machine code.
The macro \texttt{\_\_wasm\_\_} is predefined in order to enable programmers to identify this tool and its target architecture while compiling.
Programs generated with this compiler require additional runtime support that is stored in the \file{cpp\-wasm\-run} library file.
\flowgraph{\resource{\cpp{}\\source code} \ar[r] & \toolbox{cppwasm} \ar[r] \ar[d] \ar[rd] & \resource{object file} \\ \variable{ECSINCLUDE} \ar[ru] & \resource{debugging\\information} & \resource{assembly\\listing}}
\seecpp\seeassembly\seewasm\seeobject\seedebugging
}

% FALSE tools

\providecommand{\falprint}{
\toolsection{falprint} is a pretty printer for the FALSE programming language.
It reformats the source code of FALSE programs and writes it to the standard output stream.
\flowgraph{\resource{FALSE\\source code} \ar[r] & \toolbox{falprint} \ar[r] & \resource{reformatted\\source code}}
\seefalse
}

\providecommand{\falcheck}{
\toolsection{falcheck} is a syntactic and semantic checker for the FALSE programming language.
It just performs syntactic and semantic checks on FALSE programs and writes its diagnostic messages to the standard error stream.
\flowgraph{\resource{FALSE\\source code} \ar[r] & \toolbox{falcheck} \ar[r] & \resource{diagnostic\\messages}}
\seefalse
}

\providecommand{\faldump}{
\toolsection{faldump} is a serializer for the FALSE programming language.
It dumps the complete internal representation of programs written in FALSE into an XML document.
\debuggingtool
\flowgraph{\resource{FALSE\\source code} \ar[r] & \toolbox{faldump} \ar[r] & \resource{internal\\representation}}
\seefalse
}

\providecommand{\falrun}{
\toolsection{falrun} is an interpreter for the FALSE programming language.
It processes and executes programs written in FALSE\@.
\flowgraph{\resource{FALSE\\source code} \ar[r] & \toolbox{falrun} \ar@/u/[r] & \resource{input/\\output} \ar@/d/[l]}
\seefalse
}

\providecommand{\falcpp}{
\toolsection{falcpp} is a transpiler for the FALSE programming language.
It translates programs written in FALSE into \cpp{} programs and stores them in corresponding source files.
\flowgraph{\resource{FALSE\\source code} \ar[r] & \toolbox{falcpp} \ar[r] & \resource{\cpp{}\\source file}}
\seefalse\seecpp
}

\providecommand{\falcode}{
\toolsection{falcode} is an intermediate code generator for the FALSE programming language.
It generates intermediate code from programs written in FALSE and stores it in corresponding assembly files.
\debuggingtool
\flowgraph{\resource{FALSE\\source code} \ar[r] & \toolbox{falcode} \ar[r] & \resource{intermediate\\code}}
\seefalse\seeassembly\seecode
}

\providecommand{\falamda}{
\toolsection{falamd16} is a compiler for the FALSE programming language targeting the AMD64 hardware architecture.
It generates machine code for AMD64 processors from programs written in FALSE and stores it in corresponding object files.
The compiler generates machine code for the 16-bit operating mode defined by the AMD64 architecture.
\flowgraph{\resource{FALSE\\source code} \ar[r] & \toolbox{falamd16} \ar[r] & \resource{object file}}
\seefalse\seeamd\seeobject
}

\providecommand{\falamdb}{
\toolsection{falamd32} is a compiler for the FALSE programming language targeting the AMD64 hardware architecture.
It generates machine code for AMD64 processors from programs written in FALSE and stores it in corresponding object files.
The compiler generates machine code for the 32-bit operating mode defined by the AMD64 architecture.
\flowgraph{\resource{FALSE\\source code} \ar[r] & \toolbox{falamd32} \ar[r] & \resource{object file}}
\seefalse\seeamd\seeobject
}

\providecommand{\falamdc}{
\toolsection{falamd64} is a compiler for the FALSE programming language targeting the AMD64 hardware architecture.
It generates machine code for AMD64 processors from programs written in FALSE and stores it in corresponding object files.
The compiler generates machine code for the 64-bit operating mode defined by the AMD64 architecture.
\flowgraph{\resource{FALSE\\source code} \ar[r] & \toolbox{falamd64} \ar[r] & \resource{object file}}
\seefalse\seeamd\seeobject
}

\providecommand{\falarma}{
\toolsection{falarma32} is a compiler for the FALSE programming language targeting the ARM hardware architecture.
It generates machine code for ARM processors executing A32 instructions from programs written in FALSE and stores it in corresponding object files.
\flowgraph{\resource{FALSE\\source code} \ar[r] & \toolbox{falarma32} \ar[r] & \resource{object file}}
\seefalse\seearm\seeobject
}

\providecommand{\falarmb}{
\toolsection{falarma64} is a compiler for the FALSE programming language targeting the ARM hardware architecture.
It generates machine code for ARM processors executing A64 instructions from programs written in FALSE and stores it in corresponding object files.
\flowgraph{\resource{FALSE\\source code} \ar[r] & \toolbox{falarma64} \ar[r] & \resource{object file}}
\seefalse\seearm\seeobject
}

\providecommand{\falarmc}{
\toolsection{falarmt32} is a compiler for the FALSE programming language targeting the ARM hardware architecture.
It generates machine code for ARM processors without floating-point extension executing T32 instructions from programs written in FALSE and stores it in corresponding object files.
\flowgraph{\resource{FALSE\\source code} \ar[r] & \toolbox{falarmt32} \ar[r] & \resource{object file}}
\seefalse\seearm\seeobject
}

\providecommand{\falarmcfpe}{
\toolsection{falarmt32fpe} is a compiler for the FALSE programming language targeting the ARM hardware architecture.
It generates machine code for ARM processors with floating-point extension executing T32 instructions from programs written in FALSE and stores it in corresponding object files.
\flowgraph{\resource{FALSE\\source code} \ar[r] & \toolbox{falarmt32fpe} \ar[r] & \resource{object file}}
\seefalse\seearm\seeobject
}

\providecommand{\falavr}{
\toolsection{falavr} is a compiler for the FALSE programming language targeting the AVR hardware architecture.
It generates machine code for AVR processors from programs written in FALSE and stores it in corresponding object files.
\flowgraph{\resource{FALSE\\source code} \ar[r] & \toolbox{falavr} \ar[r] & \resource{object file}}
\seefalse\seeavr\seeobject
}

\providecommand{\falavrtt}{
\toolsection{falavr32} is a compiler for the FALSE programming language targeting the AVR32 hardware architecture.
It generates machine code for AVR32 processors from programs written in FALSE and stores it in corresponding object files.
\flowgraph{\resource{FALSE\\source code} \ar[r] & \toolbox{falavr32} \ar[r] & \resource{object file}}
\seefalse\seeavrtt\seeobject
}

\providecommand{\falmabk}{
\toolsection{falm68k} is a compiler for the FALSE programming language targeting the M68000 hardware architecture.
It generates machine code for M68000 processors from programs written in FALSE and stores it in corresponding object files.
\flowgraph{\resource{FALSE\\source code} \ar[r] & \toolbox{falm68k} \ar[r] & \resource{object file}}
\seefalse\seemabk\seeobject
}

\providecommand{\falmibl}{
\toolsection{falmibl} is a compiler for the FALSE programming language targeting the MicroBlaze hardware architecture.
It generates machine code for MicroBlaze processors from programs written in FALSE and stores it in corresponding object files.
\flowgraph{\resource{FALSE\\source code} \ar[r] & \toolbox{falmibl} \ar[r] & \resource{object file}}
\seefalse\seemibl\seeobject
}

\providecommand{\falmipsa}{
\toolsection{falmips32} is a compiler for the FALSE programming language targeting the MIPS32 hardware architecture.
It generates machine code for MIPS32 processors from programs written in FALSE and stores it in corresponding object files.
\flowgraph{\resource{FALSE\\source code} \ar[r] & \toolbox{falmips32} \ar[r] & \resource{object file}}
\seefalse\seemips\seeobject
}

\providecommand{\falmipsb}{
\toolsection{falmips64} is a compiler for the FALSE programming language targeting the MIPS64 hardware architecture.
It generates machine code for MIPS64 processors from programs written in FALSE and stores it in corresponding object files.
\flowgraph{\resource{FALSE\\source code} \ar[r] & \toolbox{falmips64} \ar[r] & \resource{object file}}
\seefalse\seemips\seeobject
}

\providecommand{\falmmix}{
\toolsection{falmmix} is a compiler for the FALSE programming language targeting the MMIX hardware architecture.
It generates machine code for MMIX processors from programs written in FALSE and stores it in corresponding object files.
\flowgraph{\resource{FALSE\\source code} \ar[r] & \toolbox{falmmix} \ar[r] & \resource{object file}}
\seefalse\seemmix\seeobject
}

\providecommand{\falorok}{
\toolsection{falor1k} is a compiler for the FALSE programming language targeting the OpenRISC 1000 hardware architecture.
It generates machine code for OpenRISC 1000 processors from programs written in FALSE and stores it in corresponding object files.
\flowgraph{\resource{FALSE\\source code} \ar[r] & \toolbox{falor1k} \ar[r] & \resource{object file}}
\seefalse\seeorok\seeobject
}

\providecommand{\falppca}{
\toolsection{falppc32} is a compiler for the FALSE programming language targeting the PowerPC hardware architecture.
It generates machine code for PowerPC processors from programs written in FALSE and stores it in corresponding object files.
The compiler generates machine code for the 32-bit operating mode defined by the PowerPC architecture.
\flowgraph{\resource{FALSE\\source code} \ar[r] & \toolbox{falppc32} \ar[r] & \resource{object file}}
\seefalse\seeppc\seeobject
}

\providecommand{\falppcb}{
\toolsection{falppc64} is a compiler for the FALSE programming language targeting the PowerPC hardware architecture.
It generates machine code for PowerPC processors from programs written in FALSE and stores it in corresponding object files.
The compiler generates machine code for the 64-bit operating mode defined by the PowerPC architecture.
\flowgraph{\resource{FALSE\\source code} \ar[r] & \toolbox{falppc64} \ar[r] & \resource{object file}}
\seefalse\seeppc\seeobject
}

\providecommand{\falrisc}{
\toolsection{falrisc} is a compiler for the FALSE programming language targeting the RISC hardware architecture.
It generates machine code for RISC processors from programs written in FALSE and stores it in corresponding object files.
\flowgraph{\resource{FALSE\\source code} \ar[r] & \toolbox{falrisc} \ar[r] & \resource{object file}}
\seefalse\seerisc\seeobject
}

\providecommand{\falwasm}{
\toolsection{falwasm} is a compiler for the FALSE programming language targeting the WebAssembly architecture.
It generates machine code for WebAssembly targets from programs written in FALSE and stores it in corresponding object files.
\flowgraph{\resource{FALSE\\source code} \ar[r] & \toolbox{falwasm} \ar[r] & \resource{object file}}
\seefalse\seewasm\seeobject
}

% Oberon tools

\providecommand{\obprint}{
\toolsection{obprint} is a pretty printer for the Oberon programming language.
It reformats the source code of Oberon modules and writes it to the standard output stream.
\flowgraph{\resource{Oberon\\source code} \ar[r] & \toolbox{obprint} \ar[r] & \resource{reformatted\\source code}}
\seeoberon
}

\providecommand{\obcheck}{
\toolsection{obcheck} is a syntactic and semantic checker for the Oberon programming language.
It just performs syntactic and semantic checks on Oberon modules and writes its diagnostic messages to the standard error stream.
In addition, it stores the interface of each module in a symbol file which is required when other modules import the module.
\flowgraph{\resource{Oberon\\source code} \ar[r] & \toolbox{obcheck} \ar[r] \ar@/l/[d] & \resource{diagnostic\\messages} \\ \variable{ECSIMPORT} \ar[ru] & \resource{symbol\\files} \ar@/r/[u]}
\seeoberon
}

\providecommand{\obdump}{
\toolsection{obdump} is a serializer for the Oberon programming language.
It dumps the complete internal representation of modules written in Oberon into an XML document.
\debuggingtool
\flowgraph{\resource{Oberon\\source code} \ar[r] & \toolbox{obdump} \ar[r] \ar@/l/[d] & \resource{internal\\representation} \\ \variable{ECSIMPORT} \ar[ru] & \resource{symbol\\files} \ar@/r/[u]}
\seeoberon
}

\providecommand{\obrun}{
\toolsection{obrun} is an interpreter for the Oberon programming language.
It processes and executes modules written in Oberon.
This tool does neither generate nor process symbol files while interpreting modules.
If a module is imported by another one, its filename has to be named before the other one in the list of command-line arguments.
\flowgraph{\resource{Oberon\\source code} \ar[r] & \toolbox{obrun} \ar@/u/[r] & \resource{input/\\output} \ar@/d/[l]}
\seeoberon
}

\providecommand{\obcpp}{
\toolsection{obcpp} is a transpiler for the Oberon programming language.
It translates programs written in Oberon into \cpp{} programs and stores them in corresponding source and header files.
In addition, it stores the interface of each module in a symbol file which is required when other modules import the module.
The same interface is provided by the generated header file which can be used in other parts of the \cpp{} program.
\flowgraph{\resource{Oberon\\source code} \ar[r] & \toolbox{obcpp} \ar[r] \ar@/l/[d] \ar[rd] & \resource{\cpp{}\\source file} \\ \variable{ECSIMPORT} \ar[ru] & \resource{symbol\\files} \ar@/r/[u] & \resource{\cpp{}\\header file}}
\seeoberon\seecpp
}

\providecommand{\obdoc}{
\toolsection{obdoc} is a generic documentation generator for the Oberon programming language.
It processes several Oberon modules and assembles all information therein into a generic documentation.
In addition, it stores the interface of each module in a symbol file which is required when other modules import the module.
\debuggingtool
\flowgraph{\resource{Oberon\\source code} \ar[r] & \toolbox{obdoc} \ar[r] \ar@/l/[d] & \resource{generic\\documentation} \\ \variable{ECSIMPORT} \ar[ru] & \resource{symbol\\files} \ar@/r/[u]}
\seeoberon\seedocumentation
}

\providecommand{\obhtml}{
\toolsection{obhtml} is an HTML documentation generator for the Oberon programming language.
It processes several Oberon modules and assembles all information therein into an HTML document.
In addition, it stores the interface of each module in a symbol file which is required when other modules import the module.
\flowgraph{\resource{Oberon\\source code} \ar[r] & \toolbox{obhtml} \ar[r] \ar@/l/[d] & \resource{HTML\\document} \\ \variable{ECSIMPORT} \ar[ru] & \resource{symbol\\files} \ar@/r/[u]}
\seeoberon\seedocumentation
}

\providecommand{\oblatex}{
\toolsection{oblatex} is a Latex documentation generator for the Oberon programming language.
It processes several Oberon modules and assembles all information therein into a Latex document.
In addition, it stores the interface of each module in a symbol file which is required when other modules import the module.
\flowgraph{\resource{Oberon\\source code} \ar[r] & \toolbox{oblatex} \ar[r] \ar@/l/[d] & \resource{Latex\\document} \\ \variable{ECSIMPORT} \ar[ru] & \resource{symbol\\files} \ar@/r/[u]}
\seeoberon\seedocumentation
}

\providecommand{\obcode}{
\toolsection{obcode} is an intermediate code generator for the Oberon programming language.
It generates intermediate code from modules written in Oberon and stores it in corresponding assembly files.
In addition, it stores the interface of each module in a symbol file which is required when other modules import the module.
Programs generated with this tool require additional runtime support that is stored in the \file{ob\-code\-run} library file.
\debuggingtool
\flowgraph{\resource{Oberon\\source code} \ar[r] & \toolbox{obcode} \ar[r] \ar@/l/[d] & \resource{intermediate\\code} \\ \variable{ECSIMPORT} \ar[ru] & \resource{symbol\\files} \ar@/r/[u]}
\seeoberon\seeassembly\seecode
}

\providecommand{\obamda}{
\toolsection{obamd16} is a compiler for the Oberon programming language targeting the AMD64 hardware architecture.
It generates machine code for AMD64 processors from modules written in Oberon and stores it in corresponding object files.
The compiler generates machine code for the 16-bit operating mode defined by the AMD64 architecture.
For debugging purposes, it also creates a debugging information file as well as an assembly file containing a listing of the generated machine code.
In addition, it stores the interface of each module in a symbol file which is required when other modules import the module.
Programs generated with this compiler require additional runtime support that is stored in the \file{ob\-amd16\-run} library file.
\flowgraph{\resource{Oberon\\source code} \ar[r] & \toolbox{obamd16} \ar[r] \ar@/l/[d] \ar[rd] & \resource{object file} \\ \variable{ECSIMPORT} \ar[ru] & \resource{symbol\\files} \ar@/r/[u] & \resource{debugging\\information}}
\seeoberon\seeassembly\seeamd\seeobject\seedebugging
}

\providecommand{\obamdb}{
\toolsection{obamd32} is a compiler for the Oberon programming language targeting the AMD64 hardware architecture.
It generates machine code for AMD64 processors from modules written in Oberon and stores it in corresponding object files.
The compiler generates machine code for the 32-bit operating mode defined by the AMD64 architecture.
For debugging purposes, it also creates a debugging information file as well as an assembly file containing a listing of the generated machine code.
In addition, it stores the interface of each module in a symbol file which is required when other modules import the module.
Programs generated with this compiler require additional runtime support that is stored in the \file{ob\-amd32\-run} library file.
\flowgraph{\resource{Oberon\\source code} \ar[r] & \toolbox{obamd32} \ar[r] \ar@/l/[d] \ar[rd] & \resource{object file} \\ \variable{ECSIMPORT} \ar[ru] & \resource{symbol\\files} \ar@/r/[u] & \resource{debugging\\information}}
\seeoberon\seeassembly\seeamd\seeobject\seedebugging
}

\providecommand{\obamdc}{
\toolsection{obamd64} is a compiler for the Oberon programming language targeting the AMD64 hardware architecture.
It generates machine code for AMD64 processors from modules written in Oberon and stores it in corresponding object files.
The compiler generates machine code for the 64-bit operating mode defined by the AMD64 architecture.
For debugging purposes, it also creates a debugging information file as well as an assembly file containing a listing of the generated machine code.
In addition, it stores the interface of each module in a symbol file which is required when other modules import the module.
Programs generated with this compiler require additional runtime support that is stored in the \file{ob\-amd64\-run} library file.
\flowgraph{\resource{Oberon\\source code} \ar[r] & \toolbox{obamd64} \ar[r] \ar@/l/[d] \ar[rd] & \resource{object file} \\ \variable{ECSIMPORT} \ar[ru] & \resource{symbol\\files} \ar@/r/[u] & \resource{debugging\\information}}
\seeoberon\seeassembly\seeamd\seeobject\seedebugging
}

\providecommand{\obarma}{
\toolsection{obarma32} is a compiler for the Oberon programming language targeting the ARM hardware architecture.
It generates machine code for ARM processors executing A32 instructions from modules written in Oberon and stores it in corresponding object files.
For debugging purposes, it also creates a debugging information file as well as an assembly file containing a listing of the generated machine code.
In addition, it stores the interface of each module in a symbol file which is required when other modules import the module.
Programs generated with this compiler require additional runtime support that is stored in the \file{ob\-arma32\-run} library file.
\flowgraph{\resource{Oberon\\source code} \ar[r] & \toolbox{obarma32} \ar[r] \ar@/l/[d] \ar[rd] & \resource{object file} \\ \variable{ECSIMPORT} \ar[ru] & \resource{symbol\\files} \ar@/r/[u] & \resource{debugging\\information}}
\seeoberon\seeassembly\seearm\seeobject\seedebugging
}

\providecommand{\obarmb}{
\toolsection{obarma64} is a compiler for the Oberon programming language targeting the ARM hardware architecture.
It generates machine code for ARM processors executing A64 instructions from modules written in Oberon and stores it in corresponding object files.
For debugging purposes, it also creates a debugging information file as well as an assembly file containing a listing of the generated machine code.
In addition, it stores the interface of each module in a symbol file which is required when other modules import the module.
Programs generated with this compiler require additional runtime support that is stored in the \file{ob\-arma64\-run} library file.
\flowgraph{\resource{Oberon\\source code} \ar[r] & \toolbox{obarma64} \ar[r] \ar@/l/[d] \ar[rd] & \resource{object file} \\ \variable{ECSIMPORT} \ar[ru] & \resource{symbol\\files} \ar@/r/[u] & \resource{debugging\\information}}
\seeoberon\seeassembly\seearm\seeobject\seedebugging
}

\providecommand{\obarmc}{
\toolsection{obarmt32} is a compiler for the Oberon programming language targeting the ARM hardware architecture.
It generates machine code for ARM processors without floating-point extension executing T32 instructions from modules written in Oberon and stores it in corresponding object files.
For debugging purposes, it also creates a debugging information file as well as an assembly file containing a listing of the generated machine code.
In addition, it stores the interface of each module in a symbol file which is required when other modules import the module.
Programs generated with this compiler require additional runtime support that is stored in the \file{ob\-armt32\-run} library file.
\flowgraph{\resource{Oberon\\source code} \ar[r] & \toolbox{obarmt32} \ar[r] \ar@/l/[d] \ar[rd] & \resource{object file} \\ \variable{ECSIMPORT} \ar[ru] & \resource{symbol\\files} \ar@/r/[u] & \resource{debugging\\information}}
\seeoberon\seeassembly\seearm\seeobject\seedebugging
}

\providecommand{\obarmcfpe}{
\toolsection{obarmt32fpe} is a compiler for the Oberon programming language targeting the ARM hardware architecture.
It generates machine code for ARM processors with floating-point extension executing T32 instructions from modules written in Oberon and stores it in corresponding object files.
For debugging purposes, it also creates a debugging information file as well as an assembly file containing a listing of the generated machine code.
In addition, it stores the interface of each module in a symbol file which is required when other modules import the module.
Programs generated with this compiler require additional runtime support that is stored in the \file{ob\-armt32\-fpe\-run} library file.
\flowgraph{\resource{Oberon\\source code} \ar[r] & \toolbox{obarmt32fpe} \ar[r] \ar@/l/[d] \ar[rd] & \resource{object file} \\ \variable{ECSIMPORT} \ar[ru] & \resource{symbol\\files} \ar@/r/[u] & \resource{debugging\\information}}
\seeoberon\seeassembly\seearm\seeobject\seedebugging
}

\providecommand{\obavr}{
\toolsection{obavr} is a compiler for the Oberon programming language targeting the AVR hardware architecture.
It generates machine code for AVR processors from modules written in Oberon and stores it in corresponding object files.
For debugging purposes, it also creates a debugging information file as well as an assembly file containing a listing of the generated machine code.
In addition, it stores the interface of each module in a symbol file which is required when other modules import the module.
Programs generated with this compiler require additional runtime support that is stored in the \file{ob\-avr\-run} library file.
\flowgraph{\resource{Oberon\\source code} \ar[r] & \toolbox{obavr} \ar[r] \ar@/l/[d] \ar[rd] & \resource{object file} \\ \variable{ECSIMPORT} \ar[ru] & \resource{symbol\\files} \ar@/r/[u] & \resource{debugging\\information}}
\seeoberon\seeassembly\seeavr\seeobject\seedebugging
}

\providecommand{\obavrtt}{
\toolsection{obavr32} is a compiler for the Oberon programming language targeting the AVR32 hardware architecture.
It generates machine code for AVR32 processors from modules written in Oberon and stores it in corresponding object files.
For debugging purposes, it also creates a debugging information file as well as an assembly file containing a listing of the generated machine code.
In addition, it stores the interface of each module in a symbol file which is required when other modules import the module.
Programs generated with this compiler require additional runtime support that is stored in the \file{ob\-avr32\-run} library file.
\flowgraph{\resource{Oberon\\source code} \ar[r] & \toolbox{obavr32} \ar[r] \ar@/l/[d] \ar[rd] & \resource{object file} \\ \variable{ECSIMPORT} \ar[ru] & \resource{symbol\\files} \ar@/r/[u] & \resource{debugging\\information}}
\seeoberon\seeassembly\seeavrtt\seeobject\seedebugging
}

\providecommand{\obmabk}{
\toolsection{obm68k} is a compiler for the Oberon programming language targeting the M68000 hardware architecture.
It generates machine code for M68000 processors from modules written in Oberon and stores it in corresponding object files.
For debugging purposes, it also creates a debugging information file as well as an assembly file containing a listing of the generated machine code.
In addition, it stores the interface of each module in a symbol file which is required when other modules import the module.
Programs generated with this compiler require additional runtime support that is stored in the \file{ob\-m68k\-run} library file.
\flowgraph{\resource{Oberon\\source code} \ar[r] & \toolbox{obm68k} \ar[r] \ar@/l/[d] \ar[rd] & \resource{object file} \\ \variable{ECSIMPORT} \ar[ru] & \resource{symbol\\files} \ar@/r/[u] & \resource{debugging\\information}}
\seeoberon\seeassembly\seemabk\seeobject\seedebugging
}

\providecommand{\obmibl}{
\toolsection{obmibl} is a compiler for the Oberon programming language targeting the MicroBlaze hardware architecture.
It generates machine code for MicroBlaze processors from modules written in Oberon and stores it in corresponding object files.
For debugging purposes, it also creates a debugging information file as well as an assembly file containing a listing of the generated machine code.
In addition, it stores the interface of each module in a symbol file which is required when other modules import the module.
Programs generated with this compiler require additional runtime support that is stored in the \file{ob\-mibl\-run} library file.
\flowgraph{\resource{Oberon\\source code} \ar[r] & \toolbox{obmibl} \ar[r] \ar@/l/[d] \ar[rd] & \resource{object file} \\ \variable{ECSIMPORT} \ar[ru] & \resource{symbol\\files} \ar@/r/[u] & \resource{debugging\\information}}
\seeoberon\seeassembly\seemibl\seeobject\seedebugging
}

\providecommand{\obmipsa}{
\toolsection{obmips32} is a compiler for the Oberon programming language targeting the MIPS32 hardware architecture.
It generates machine code for MIPS32 processors from modules written in Oberon and stores it in corresponding object files.
For debugging purposes, it also creates a debugging information file as well as an assembly file containing a listing of the generated machine code.
In addition, it stores the interface of each module in a symbol file which is required when other modules import the module.
Programs generated with this compiler require additional runtime support that is stored in the \file{ob\-mips32\-run} library file.
\flowgraph{\resource{Oberon\\source code} \ar[r] & \toolbox{obmips32} \ar[r] \ar@/l/[d] \ar[rd] & \resource{object file} \\ \variable{ECSIMPORT} \ar[ru] & \resource{symbol\\files} \ar@/r/[u] & \resource{debugging\\information}}
\seeoberon\seeassembly\seemips\seeobject\seedebugging
}

\providecommand{\obmipsb}{
\toolsection{obmips64} is a compiler for the Oberon programming language targeting the MIPS64 hardware architecture.
It generates machine code for MIPS64 processors from modules written in Oberon and stores it in corresponding object files.
For debugging purposes, it also creates a debugging information file as well as an assembly file containing a listing of the generated machine code.
In addition, it stores the interface of each module in a symbol file which is required when other modules import the module.
Programs generated with this compiler require additional runtime support that is stored in the \file{ob\-mips64\-run} library file.
\flowgraph{\resource{Oberon\\source code} \ar[r] & \toolbox{obmips64} \ar[r] \ar@/l/[d] \ar[rd] & \resource{object file} \\ \variable{ECSIMPORT} \ar[ru] & \resource{symbol\\files} \ar@/r/[u] & \resource{debugging\\information}}
\seeoberon\seeassembly\seemips\seeobject\seedebugging
}

\providecommand{\obmmix}{
\toolsection{obmmix} is a compiler for the Oberon programming language targeting the MMIX hardware architecture.
It generates machine code for MMIX processors from modules written in Oberon and stores it in corresponding object files.
For debugging purposes, it also creates a debugging information file as well as an assembly file containing a listing of the generated machine code.
In addition, it stores the interface of each module in a symbol file which is required when other modules import the module.
Programs generated with this compiler require additional runtime support that is stored in the \file{ob\-mmix\-run} library file.
\flowgraph{\resource{Oberon\\source code} \ar[r] & \toolbox{obmmix} \ar[r] \ar@/l/[d] \ar[rd] & \resource{object file} \\ \variable{ECSIMPORT} \ar[ru] & \resource{symbol\\files} \ar@/r/[u] & \resource{debugging\\information}}
\seeoberon\seeassembly\seemmix\seeobject\seedebugging
}

\providecommand{\oborok}{
\toolsection{obor1k} is a compiler for the Oberon programming language targeting the OpenRISC 1000 hardware architecture.
It generates machine code for OpenRISC 1000 processors from modules written in Oberon and stores it in corresponding object files.
For debugging purposes, it also creates a debugging information file as well as an assembly file containing a listing of the generated machine code.
In addition, it stores the interface of each module in a symbol file which is required when other modules import the module.
Programs generated with this compiler require additional runtime support that is stored in the \file{ob\-or1k\-run} library file.
\flowgraph{\resource{Oberon\\source code} \ar[r] & \toolbox{obor1k} \ar[r] \ar@/l/[d] \ar[rd] & \resource{object file} \\ \variable{ECSIMPORT} \ar[ru] & \resource{symbol\\files} \ar@/r/[u] & \resource{debugging\\information}}
\seeoberon\seeassembly\seeorok\seeobject\seedebugging
}

\providecommand{\obppca}{
\toolsection{obppc32} is a compiler for the Oberon programming language targeting the PowerPC hardware architecture.
It generates machine code for PowerPC processors from modules written in Oberon and stores it in corresponding object files.
The compiler generates machine code for the 32-bit operating mode defined by the PowerPC architecture.
For debugging purposes, it also creates a debugging information file as well as an assembly file containing a listing of the generated machine code.
In addition, it stores the interface of each module in a symbol file which is required when other modules import the module.
Programs generated with this compiler require additional runtime support that is stored in the \file{ob\-ppc32\-run} library file.
\flowgraph{\resource{Oberon\\source code} \ar[r] & \toolbox{obppc32} \ar[r] \ar@/l/[d] \ar[rd] & \resource{object file} \\ \variable{ECSIMPORT} \ar[ru] & \resource{symbol\\files} \ar@/r/[u] & \resource{debugging\\information}}
\seeoberon\seeassembly\seeppc\seeobject\seedebugging
}

\providecommand{\obppcb}{
\toolsection{obppc64} is a compiler for the Oberon programming language targeting the PowerPC hardware architecture.
It generates machine code for PowerPC processors from modules written in Oberon and stores it in corresponding object files.
The compiler generates machine code for the 64-bit operating mode defined by the PowerPC architecture.
For debugging purposes, it also creates a debugging information file as well as an assembly file containing a listing of the generated machine code.
In addition, it stores the interface of each module in a symbol file which is required when other modules import the module.
Programs generated with this compiler require additional runtime support that is stored in the \file{ob\-ppc64\-run} library file.
\flowgraph{\resource{Oberon\\source code} \ar[r] & \toolbox{obppc64} \ar[r] \ar@/l/[d] \ar[rd] & \resource{object file} \\ \variable{ECSIMPORT} \ar[ru] & \resource{symbol\\files} \ar@/r/[u] & \resource{debugging\\information}}
\seeoberon\seeassembly\seeppc\seeobject\seedebugging
}

\providecommand{\obrisc}{
\toolsection{obrisc} is a compiler for the Oberon programming language targeting the RISC hardware architecture.
It generates machine code for RISC processors from modules written in Oberon and stores it in corresponding object files.
For debugging purposes, it also creates a debugging information file as well as an assembly file containing a listing of the generated machine code.
In addition, it stores the interface of each module in a symbol file which is required when other modules import the module.
Programs generated with this compiler require additional runtime support that is stored in the \file{ob\-risc\-run} library file.
\flowgraph{\resource{Oberon\\source code} \ar[r] & \toolbox{obrisc} \ar[r] \ar@/l/[d] \ar[rd] & \resource{object file} \\ \variable{ECSIMPORT} \ar[ru] & \resource{symbol\\files} \ar@/r/[u] & \resource{debugging\\information}}
\seeoberon\seeassembly\seerisc\seeobject\seedebugging
}

\providecommand{\obwasm}{
\toolsection{obwasm} is a compiler for the Oberon programming language targeting the WebAssembly architecture.
It generates machine code for WebAssembly targets from modules written in Oberon and stores it in corresponding object files.
For debugging purposes, it also creates a debugging information file as well as an assembly file containing a listing of the generated machine code.
In addition, it stores the interface of each module in a symbol file which is required when other modules import the module.
Programs generated with this compiler require additional runtime support that is stored in the \file{ob\-wasm\-run} library file.
\flowgraph{\resource{Oberon\\source code} \ar[r] & \toolbox{obwasm} \ar[r] \ar@/l/[d] \ar[rd] & \resource{object file} \\ \variable{ECSIMPORT} \ar[ru] & \resource{symbol\\files} \ar@/r/[u] & \resource{debugging\\information}}
\seeoberon\seeassembly\seewasm\seeobject\seedebugging
}

% converter tools

\providecommand{\dbgdwarf}{
\toolsection{dbgdwarf} is a DWARF debugging information converter tool.
It converts debugging information into the DWARF debugging data format and stores it in corresponding object files~\cite{dwarffile}.
The resulting debugging object files can be combined with runtime support that creates Executable and Linking Format (ELF) files~\cite{elffile}.
\flowgraph{\resource{debugging\\information} \ar[r] & \toolbox{dbgdwarf} \ar[r] & \resource{debugging\\object file}}
\seeobject\seedebugging
}

% assembler tools

\providecommand{\asmprint}{
\toolsection{asmprint} is a pretty printer for generic assembly code.
It reformats generic assembly code and writes it to the standard output stream.
\flowgraph{\resource{generic assembly\\source code} \ar[r] & \toolbox{asmprint} \ar[r] & \resource{reformatted\\source code}}
\seeassembly
}

\providecommand{\amdaasm}{
\toolsection{amd16asm} is an assembler for the AMD64 hardware architecture.
It translates assembly code into machine code for AMD64 processors and stores it in corresponding object files.
By default, the assembler generates machine code for the 16-bit operating mode defined by the AMD64 architecture.
\flowgraph{\resource{AMD16 assembly\\source code} \ar[r] & \toolbox{amd16asm} \ar[r] & \resource{object file}}
\seeassembly\seeamd\seeobject
}

\providecommand{\amdadism}{
\toolsection{amd16dism} is a disassembler for the AMD64 hardware architecture.
It translates machine code from object files targeting AMD64 processors into assembly code and writes it to the standard output stream.
It assumes that the machine code was generated for the 16-bit operating mode defined by the AMD64 architecture.
\flowgraph{\resource{object file} \ar[r] & \toolbox{amd16dism} \ar[r] & \resource{disassembly\\listing}}
\seeassembly\seeamd\seeobject
}

\providecommand{\amdbasm}{
\toolsection{amd32asm} is an assembler for the AMD64 hardware architecture.
It translates assembly code into machine code for AMD64 processors and stores it in corresponding object files.
By default, the assembler generates machine code for the 32-bit operating mode defined by the AMD64 architecture.
\flowgraph{\resource{AMD32 assembly\\source code} \ar[r] & \toolbox{amd32asm} \ar[r] & \resource{object file}}
\seeassembly\seeamd\seeobject
}

\providecommand{\amdbdism}{
\toolsection{amd32dism} is a disassembler for the AMD64 hardware architecture.
It translates machine code from object files targeting AMD64 processors into assembly code and writes it to the standard output stream.
It assumes that the machine code was generated for the 32-bit operating mode defined by the AMD64 architecture.
\flowgraph{\resource{object file} \ar[r] & \toolbox{amd32dism} \ar[r] & \resource{disassembly\\listing}}
\seeassembly\seeamd\seeobject
}

\providecommand{\amdcasm}{
\toolsection{amd64asm} is an assembler for the AMD64 hardware architecture.
It translates assembly code into machine code for AMD64 processors and stores it in corresponding object files.
By default, the assembler generates machine code for the 64-bit operating mode defined by the AMD64 architecture.
\flowgraph{\resource{AMD64 assembly\\source code} \ar[r] & \toolbox{amd64asm} \ar[r] & \resource{object file}}
\seeassembly\seeamd\seeobject
}

\providecommand{\amdcdism}{
\toolsection{amd64dism} is a disassembler for the AMD64 hardware architecture.
It translates machine code from object files targeting AMD64 processors into assembly code and writes it to the standard output stream.
It assumes that the machine code was generated for the 64-bit operating mode defined by the AMD64 architecture.
\flowgraph{\resource{object file} \ar[r] & \toolbox{amd64dism} \ar[r] & \resource{disassembly\\listing}}
\seeassembly\seeamd\seeobject
}

\providecommand{\armaasm}{
\toolsection{arma32asm} is an assembler for the ARM hardware architecture.
It translates assembly code into machine code for ARM processors executing A32 instructions and stores it in corresponding object files.
\flowgraph{\resource{ARM A32 assembly\\source code} \ar[r] & \toolbox{arma32asm} \ar[r] & \resource{object file}}
\seeassembly\seearm\seeobject
}

\providecommand{\armadism}{
\toolsection{arma32dism} is a disassembler for the ARM hardware architecture.
It translates machine code from object files targeting ARM processors executing A32 instructions into assembly code and writes it to the standard output stream.
\flowgraph{\resource{object file} \ar[r] & \toolbox{arma32dism} \ar[r] & \resource{disassembly\\listing}}
\seeassembly\seearm\seeobject
}

\providecommand{\armbasm}{
\toolsection{arma64asm} is an assembler for the ARM hardware architecture.
It translates assembly code into machine code for ARM processors executing A64 instructions and stores it in corresponding object files.
\flowgraph{\resource{ARM A64 assembly\\source code} \ar[r] & \toolbox{arma64asm} \ar[r] & \resource{object file}}
\seeassembly\seearm\seeobject
}

\providecommand{\armbdism}{
\toolsection{arma64dism} is a disassembler for the ARM hardware architecture.
It translates machine code from object files targeting ARM processors executing A64 instructions into assembly code and writes it to the standard output stream.
\flowgraph{\resource{object file} \ar[r] & \toolbox{arma64dism} \ar[r] & \resource{disassembly\\listing}}
\seeassembly\seearm\seeobject
}

\providecommand{\armcasm}{
\toolsection{armt32asm} is an assembler for the ARM hardware architecture.
It translates assembly code into machine code for ARM processors executing T32 instructions and stores it in corresponding object files.
\flowgraph{\resource{ARM T32 assembly\\source code} \ar[r] & \toolbox{armt32asm} \ar[r] & \resource{object file}}
\seeassembly\seearm\seeobject
}

\providecommand{\armcdism}{
\toolsection{armt32dism} is a disassembler for the ARM hardware architecture.
It translates machine code from object files targeting ARM processors executing T32 instructions into assembly code and writes it to the standard output stream.
\flowgraph{\resource{object file} \ar[r] & \toolbox{armt32dism} \ar[r] & \resource{disassembly\\listing}}
\seeassembly\seearm\seeobject
}

\providecommand{\avrasm}{
\toolsection{avrasm} is an assembler for the AVR hardware architecture.
It translates assembly code into machine code for AVR processors and stores it in corresponding object files.
The identifiers \texttt{RXL}, \texttt{RXH}, \texttt{RYL}, \texttt{RYH}, \texttt{RZL}, and \texttt{RZH} are predefined and name the corresponding registers.
The identifiers \texttt{SPL} and \texttt{SPH} are also predefined and evaluate to the address of the corresponding registers.
\flowgraph{\resource{AVR assembly\\source code} \ar[r] & \toolbox{avrasm} \ar[r] & \resource{object file}}
\seeassembly\seeavr\seeobject
}

\providecommand{\avrdism}{
\toolsection{avrdism} is a disassembler for the AVR hardware architecture.
It translates machine code from object files targeting AVR processors into assembly code and writes it to the standard output stream.
\flowgraph{\resource{object file} \ar[r] & \toolbox{avrdism} \ar[r] & \resource{disassembly\\listing}}
\seeassembly\seeavr\seeobject
}

\providecommand{\avrttasm}{
\toolsection{avr32asm} is an assembler for the AVR32 hardware architecture.
It translates assembly code into machine code for AVR32 processors and stores it in corresponding object files.
\flowgraph{\resource{AVR32 assembly\\source code} \ar[r] & \toolbox{avr32asm} \ar[r] & \resource{object file}}
\seeassembly\seeavrtt\seeobject
}

\providecommand{\avrttdism}{
\toolsection{avr32dism} is a disassembler for the AVR32 hardware architecture.
It translates machine code from object files targeting AVR32 processors into assembly code and writes it to the standard output stream.
\flowgraph{\resource{object file} \ar[r] & \toolbox{avr32dism} \ar[r] & \resource{disassembly\\listing}}
\seeassembly\seeavrtt\seeobject
}

\providecommand{\mabkasm}{
\toolsection{m68kasm} is an assembler for the M68000 hardware architecture.
It translates assembly code into machine code for M68000 processors and stores it in corresponding object files.
\flowgraph{\resource{68000 assembly\\source code} \ar[r] & \toolbox{m68kasm} \ar[r] & \resource{object file}}
\seeassembly\seemabk\seeobject
}

\providecommand{\mabkdism}{
\toolsection{m68kdism} is a disassembler for the M68000 hardware architecture.
It translates machine code from object files targeting M68000 processors into assembly code and writes it to the standard output stream.
\flowgraph{\resource{object file} \ar[r] & \toolbox{m68kdism} \ar[r] & \resource{disassembly\\listing}}
\seeassembly\seemabk\seeobject
}

\providecommand{\miblasm}{
\toolsection{miblasm} is an assembler for the MicroBlaze hardware architecture.
It translates assembly code into machine code for MicroBlaze processors and stores it in corresponding object files.
\flowgraph{\resource{MicroBlaze assembly\\source code} \ar[r] & \toolbox{miblasm} \ar[r] & \resource{object file}}
\seeassembly\seemibl\seeobject
}

\providecommand{\mibldism}{
\toolsection{mibldism} is a disassembler for the MicroBlaze hardware architecture.
It translates machine code from object files targeting MicroBlaze processors into assembly code and writes it to the standard output stream.
\flowgraph{\resource{object file} \ar[r] & \toolbox{mibldism} \ar[r] & \resource{disassembly\\listing}}
\seeassembly\seemibl\seeobject
}

\providecommand{\mipsaasm}{
\toolsection{mips32asm} is an assembler for the MIPS32 hardware architecture.
It translates assembly code into machine code for MIPS32 processors and stores it in corresponding object files.
\flowgraph{\resource{MIPS32 assembly\\source code} \ar[r] & \toolbox{mips32asm} \ar[r] & \resource{object file}}
\seeassembly\seemips\seeobject
}

\providecommand{\mipsadism}{
\toolsection{mips32dism} is a disassembler for the MIPS32 hardware architecture.
It translates machine code from object files targeting MIPS32 processors into assembly code and writes it to the standard output stream.
\flowgraph{\resource{object file} \ar[r] & \toolbox{mips32dism} \ar[r] & \resource{disassembly\\listing}}
\seeassembly\seemips\seeobject
}

\providecommand{\mipsbasm}{
\toolsection{mips64asm} is an assembler for the MIPS64 hardware architecture.
It translates assembly code into machine code for MIPS64 processors and stores it in corresponding object files.
\flowgraph{\resource{MIPS64 assembly\\source code} \ar[r] & \toolbox{mips64asm} \ar[r] & \resource{object file}}
\seeassembly\seemips\seeobject
}

\providecommand{\mipsbdism}{
\toolsection{mips64dism} is a disassembler for the MIPS64 hardware architecture.
It translates machine code from object files targeting MIPS64 processors into assembly code and writes it to the standard output stream.
\flowgraph{\resource{object file} \ar[r] & \toolbox{mips64dism} \ar[r] & \resource{disassembly\\listing}}
\seeassembly\seemips\seeobject
}

\providecommand{\mmixasm}{
\toolsection{mmixasm} is an assembler for the MMIX hardware architecture.
It translates assembly code into machine code for MMIX processors and stores it in corresponding object files.
The names of all special registers are predefined and evaluate to the corresponding number.
\flowgraph{\resource{MMIX assembly\\source code} \ar[r] & \toolbox{mmixasm} \ar[r] & \resource{object file}}
\seeassembly\seemmix\seeobject
}

\providecommand{\mmixdism}{
\toolsection{mmixdism} is a disassembler for the MMIX hardware architecture.
It translates machine code from object files targeting MMIX processors into assembly code and writes it to the standard output stream.
\flowgraph{\resource{object file} \ar[r] & \toolbox{mmixdism} \ar[r] & \resource{disassembly\\listing}}
\seeassembly\seemmix\seeobject
}

\providecommand{\orokasm}{
\toolsection{or1kasm} is an assembler for the OpenRISC 1000 hardware architecture.
It translates assembly code into machine code for OpenRISC 1000 processors and stores it in corresponding object files.
\flowgraph{\resource{OpenRISC 1000 assembly\\source code} \ar[r] & \toolbox{or1kasm} \ar[r] & \resource{object file}}
\seeassembly\seeorok\seeobject
}

\providecommand{\orokdism}{
\toolsection{or1kdism} is a disassembler for the OpenRISC 1000 hardware architecture.
It translates machine code from object files targeting OpenRISC 1000 processors into assembly code and writes it to the standard output stream.
\flowgraph{\resource{object file} \ar[r] & \toolbox{or1kdism} \ar[r] & \resource{disassembly\\listing}}
\seeassembly\seeorok\seeobject
}

\providecommand{\ppcaasm}{
\toolsection{ppc32asm} is an assembler for the PowerPC hardware architecture.
It translates assembly code into machine code for PowerPC processors and stores it in corresponding object files.
By default, the assembler generates machine code for the 32-bit operating mode defined by the PowerPC architecture.
\flowgraph{\resource{PowerPC assembly\\source code} \ar[r] & \toolbox{ppc32asm} \ar[r] & \resource{object file}}
\seeassembly\seeppc\seeobject
}

\providecommand{\ppcadism}{
\toolsection{ppc32dism} is a disassembler for the PowerPC hardware architecture.
It translates machine code from object files targeting PowerPC processors into assembly code and writes it to the standard output stream.
It assumes that the machine code was generated for the 32-bit operating mode defined by the PowerPC architecture.
\flowgraph{\resource{object file} \ar[r] & \toolbox{ppc32dism} \ar[r] & \resource{disassembly\\listing}}
\seeassembly\seeppc\seeobject
}

\providecommand{\ppcbasm}{
\toolsection{ppc64asm} is an assembler for the PowerPC hardware architecture.
It translates assembly code into machine code for PowerPC processors and stores it in corresponding object files.
By default, the assembler generates machine code for the 64-bit operating mode defined by the PowerPC architecture.
\flowgraph{\resource{PowerPC assembly\\source code} \ar[r] & \toolbox{ppc64asm} \ar[r] & \resource{object file}}
\seeassembly\seeppc\seeobject
}

\providecommand{\ppcbdism}{
\toolsection{ppc64dism} is a disassembler for the PowerPC hardware architecture.
It translates machine code from object files targeting PowerPC processors into assembly code and writes it to the standard output stream.
It assumes that the machine code was generated for the 64-bit operating mode defined by the PowerPC architecture.
\flowgraph{\resource{object file} \ar[r] & \toolbox{ppc64dism} \ar[r] & \resource{disassembly\\listing}}
\seeassembly\seeppc\seeobject
}

\providecommand{\riscasm}{
\toolsection{riscasm} is an assembler for the RISC hardware architecture.
It translates assembly code into machine code for RISC processors and stores it in corresponding object files.
The names of all special registers are predefined and evaluate to the corresponding number.
\flowgraph{\resource{RISC assembly\\source code} \ar[r] & \toolbox{riscasm} \ar[r] & \resource{object file}}
\seeassembly\seerisc\seeobject
}

\providecommand{\riscdism}{
\toolsection{riscdism} is a disassembler for the RISC hardware architecture.
It translates machine code from object files targeting RISC processors into assembly code and writes it to the standard output stream.
\flowgraph{\resource{object file} \ar[r] & \toolbox{riscdism} \ar[r] & \resource{disassembly\\listing}}
\seeassembly\seerisc\seeobject
}

\providecommand{\wasmasm}{
\toolsection{wasmasm} is an assembler for the WebAssembly architecture.
It translates assembly code into machine code for WebAssembly targets and stores it in corresponding object files.
The names of all special registers are predefined and evaluate to the corresponding number.
\flowgraph{\resource{WebAssembly assembly\\source code} \ar[r] & \toolbox{wasmasm} \ar[r] & \resource{object file}}
\seeassembly\seewasm\seeobject
}

\providecommand{\wasmdism}{
\toolsection{wasmdism} is a disassembler for the WebAssembly architecture.
It translates machine code from object files targeting WebAssembly targets into assembly code and writes it to the standard output stream.
\flowgraph{\resource{object file} \ar[r] & \toolbox{wasmdism} \ar[r] & \resource{disassembly\\listing}}
\seeassembly\seewasm\seeobject
}

% linker tools

\providecommand{\linklib}{
\toolsection{linklib} is an object file combiner.
It creates a static library file by combining all object files given to it into a single one.
\flowgraph{\resource{object files} \ar[r] & \toolbox{linklib} \ar[r] & \resource{library file}}
\seeobject
}

\providecommand{\linkbin}{
\toolsection{linkbin} is a linker for plain binary files.
It links all object files given to it into a single image and stores it in a binary file that begins with the first linked section.
It also creates a map file that lists the address, type, name and size of all used sections.
The filename extension of the resulting binary file can be specified by putting it into a constant data section called \texttt{\_extension}.
\flowgraph{\resource{object files} \ar[r] & \toolbox{linkbin} \ar[r] \ar[d] & \resource{binary file} \\ & \resource{map file}}
\seeobject
}

\providecommand{\linkmem}{
\toolsection{linkmem} is a linker for plain binary files partitioned into random-access and read-only memory.
It links all object files given to it into two distinct images, one for data sections and one for code and constant data sections, and stores each image in a binary file that begins with the first linked section of the corresponding type.
It also creates a map file that lists the address, type, name and size of all used sections.
\flowgraph{\resource{object files} \ar[r] & \toolbox{linkmem} \ar[r] \ar[d] & \resource{RAM file/\\ROM file} \\ & \resource{map file}}
\seeobject
}

\providecommand{\linkprg}{
\toolsection{linkprg} is a linker for GEMDOS executable files.
It links all object files given to it into a single image and stores the image in an Atari GEMDOS executable file~\cite{gemdosfile}.
It also creates a map file that lists the address relative to the text segment, type, name and size of all used sections.
The filename extension of the resulting executable file can be specified by putting it into a constant data section called \texttt{\_extension}.
The GEMDOS executable file format requires all patch patterns of absolute link patches to consist of four full bitmasks with descending offsets.
\flowgraph{\resource{object files} \ar[r] & \toolbox{linkprg} \ar[r] \ar[d] & \resource{executable file} \\ & \resource{map file}}
\seeobject
}

\providecommand{\linkhex}{
\toolsection{linkhex} is a linker for Intel HEX files.
It links all code sections of the object files given to it into single image and stores the image in an Intel HEX file~\cite{hexfile} that begins with the first linked section.
It also creates a map file that lists the address, type, name and size of all used sections.
\flowgraph{\resource{object files} \ar[r] & \toolbox{linkhex} \ar[r] \ar[d] & \resource{HEX file} \\ & \resource{map file}}
\seeobject
}

\providecommand{\mapsearch}{
\toolsection{mapsearch} is a debugging tool.
It searches map files generated by linker tools for the name of a binary section that encompasses a memory address read from the standard input stream.
If additionally provided with one or more object files, it also stores an excerpt thereof in a separate object file called map search result which only contains the identified binary section for disassembling purposes.
\flowgraph{& \resource{map files/\\object files} \ar[d] \\ \resource{memory\\address} \ar[r] & \toolbox{mapsearch} \ar[r] \ar[d] & \resource{section name/\\relative offset} \\ & \resource{object file\\excerpt}}
\seeobject
}

\renewcommand{\flowgraph}[1]{}
\newcounter{tool}\newcounter{compiler}
\addtocontents{toc}{\protect\setcounter{tocdepth}{0}}
\lehead{\chaptermarkformat Tool Reference}\rohead{\topmark{ -- }\botmark}
\newcommand{\seeprefix}{}\renewcommand{\seedocumentationref}[3]{\seeprefix{}\documentationref{#2}{#3}\renewcommand{\seeprefix}{/}}
\renewcommand{\toolsection}[1]{\normalfont\par\medskip\noindent\textbf{#1}\refstepcounter{tool}\phantomsection\addcontentsline{toc}{section}{#1}\markboth{}{\ifodd\thepage#1\else\botmark\fi}\index[tools]{#1 tool@\tool{#1} tool}\renewcommand{\seeprefix}{\em\alignright\mbox{See Chapter }}}

\startchapter{Tool Reference}{}{tools}{}

This chapter lists all \ref*{tools:all}~tools provided by the \ecs{} in alphabetical order.
All of them share the same command-line user interface as detailed in Chapter~\ref{interface}.
The implementation-defined behavior of programming language-specific tools like pretty printers, semantic checkers, serializers, interpreters, and compilers is described in Chapters~\ref{cpp} to~\ref{oberon}.
For more information about assembler and disassembler tools, see Chapter~\ref{assembly}.
The hardware architectures targeted by all of these tools are described in Chapters~\ref{amd64} to~\ref{xtensa} whereas Chapter~\ref{object} details linker tools and their functionality.
For more information about debugging tools and documentation generators, see Chapters~\ref{code} to~\ref{documentation}.

\epigraph{This world is but canvas to our imaginations.}{Henry David Thoreau}

\bigskip
\begin{small}
\begin{multicols}{2}
\amdaasm
\amdadism
\amdbasm
\amdbdism
\amdcasm
\amdcdism
\armaasm
\armadism
\armbasm
\armbdism
\armcasm
\armcdism
\asmprint
\avrasm
\avrdism
\avrttasm
\avrttdism
\cdamda
\cdamdb
\cdamdc
\cdarma
\cdarmb
\cdarmc
\cdarmcfpe
\cdavr
\cdavrtt
\cdcheck
\cdmabk
\cdmibl
\cdmipsa
\cdmipsb
\cdmmix
\cdopt
\cdorok
\cdppca
\cdppcb
\cdrisc
\cdrun
\cdwasm
\cdxtensa
\cppamda
\cppamdb
\cppamdc
\cpparma
\cpparmb
\cpparmc
\cpparmcfpe
\cppavr
\cppavrtt
\cppcheck
\cppcode
\cppdoc
\cppdump
\cpphtml
\cpplatex
\cppmabk
\cppmibl
\cppmipsa
\cppmipsb
\cppmmix
\cpporok
\cppppca
\cppppcb
\cppprep
\cppprint
\cpprisc
\cpprun
\cppwasm
\cppxtensa
\dbgdwarf
\doccheck
\dochtml
\doclatex
\docprint
\falamda
\falamdb
\falamdc
\falarma
\falarmb
\falarmc
\falarmcfpe
\falavr
\falavrtt
\falcheck
\falcode
\falcpp
\faldump
\falmabk
\falmibl
\falmipsa
\falmipsb
\falmmix
\falorok
\falppca
\falppcb
\falprint
\falrisc
\falrun
\falwasm
\falxtensa
\linkbin
\linkhex
\linklib
\linkmem
\linkprg
\mabkasm
\mabkdism
\mapsearch
\miblasm
\mibldism
\mipsaasm
\mipsadism
\mipsbasm
\mipsbdism
\mmixasm
\mmixdism
\obamda
\obamdb
\obamdc
\obarma
\obarmb
\obarmc
\obarmcfpe
\obavr
\obavrtt
\obcheck
\obcode
\obcpp
\obdoc
\obdump
\obhtml
\oblatex
\obmabk
\obmibl
\obmipsa
\obmipsb
\obmmix
\oborok
\obppca
\obppcb
\obprint
\obrisc
\obrun
\obwasm
\obxtensa
\orokasm
\orokdism
\ppcaasm
\ppcadism
\ppcbasm
\ppcbdism
\riscasm
\riscdism
\wasmasm
\wasmdism
\xtensaasm
\xtensadism
\label{tools:all}
\end{multicols}
\end{small}

Table~\ref{tab:tools} lists all \ref*{tools:compilers}~compiler and assembler tools and shows their naming scheme consisting of common programming language prefixes and hardware architecture suffixes.

\newcommand{\compilerref}[2]{\emph{\ref{#1:#2}}}
\newcommand{\compiler}[1]{\tool{#1}\refstepcounter{compiler}}
\newcommand{\languageref}[1]{\emph{Chapter~\ref{#1}}}
\newcommand{\architectureref}[1]{\multicolumn{2}{@{}l}{\emph{See Chapter~\ref{#1}}}}
\newcommand{\backends}[1]{\compiler{cpp#1} & \compiler{fal#1} & \compiler{ob#1}}
\newcommand{\backendrefs}[1]{\compilerref{cpp}{cpp#1} & \compilerref{false}{fal#1} & \compilerref{oberon}{ob#1}}
\newcommand{\compilers}[1]{\backends{#1} & \compiler{#1asm}}
\newcommand{\compilerrefs}[1]{\backendrefs{#1} & \compilerref{assembly}{#1asm}}

\begin{table}
\centering\footnotesize
\begin{tabular}{@{}lrcccc@{}}
\toprule \multicolumn{2}{@{}r}{Programming} & & & & Generic \\ \multicolumn{2}{@{}r}{Language} & \cpp{} & FALSE & Oberon & Assembly \\
\multicolumn{2}{@{}l}{Hardware} & \languageref{cpp} & \languageref{false} & \languageref{oberon} & \languageref{assembly} \\ \multicolumn{2}{@{}l}{Architecture} \\
\midrule AMD64 & 16-bit & \compilers{amd16} \\ & & \compilerrefs{amd16} \\ & 32-bit & \compilers{amd32} \\ & & \compilerrefs{amd32} \\ & 64-bit & \compilers{amd64} \\ \architectureref{amd64} & \compilerrefs{amd64} \\
\midrule ARM & A32 & \compilers{arma32} \\ & & \compilerrefs{arma32} \\ & A64 & \compilers{arma64} \\ & & \compilerrefs{arma64} \\ & T32 & \compilers{armt32} \\ & & \compilerrefs{armt32} \\ & & \backends{armt32fpe} \\ \architectureref{arm} & \backendrefs{armt32fpe} \\
\midrule \multicolumn{2}{@{}l}{AVR} & \compilers{avr} \\ \architectureref{avr} & \compilerrefs{avr} \\
\midrule \multicolumn{2}{@{}l}{AVR32} & \compilers{avr32} \\ \architectureref{avr32} & \compilerrefs{avr32} \\
\midrule \multicolumn{2}{@{}l}{M68000} & \compilers{m68k} \\ \architectureref{m68k} & \compilerrefs{m68k} \\
\midrule \multicolumn{2}{@{}l}{MicroBlaze} & \compilers{mibl} \\ \architectureref{mibl} & \compilerrefs{mibl} \\
\midrule MIPS & 32-bit & \compilers{mips32} \\ & & \compilerrefs{mips32} \\ & 64-bit & \compilers{mips64} \\ \architectureref{mips} & \compilerrefs{mips64} \\
\midrule \multicolumn{2}{@{}l}{MMIX} & \compilers{mmix} \\ \architectureref{mmix} & \compilerrefs{mmix} \\
\midrule \multicolumn{2}{@{}l}{OpenRISC 1000} & \compilers{or1k} \\ \architectureref{or1k} & \compilerrefs{or1k} \\
\midrule PowerPC & 32-bit & \compilers{ppc32} \\ & & \compilerrefs{ppc32} \\ & 64-bit & \compilers{ppc64} \\ \architectureref{ppc} & \compilerrefs{ppc64} \\
\midrule \multicolumn{2}{@{}l}{RISC} & \compilers{risc} \\ \architectureref{risc} & \compilerrefs{risc} \\
\midrule \multicolumn{2}{@{}l}{WebAssembly} & \compilers{wasm} \\ \architectureref{wasm} & \compilerrefs{wasm} \\
\midrule \multicolumn{2}{@{}l}{Xtensa} & \compilers{xtensa} \label{tools:compilers} \\ \architectureref{xtensa} & \compilerrefs{xtensa} \\
\bottomrule
\end{tabular}
\caption{References to all \ref*{tools:compilers}~compiler and assembler tools}
\label{tab:tools}
\end{table}

\concludechapter
\lehead{\leftmark}\rohead{\rightmark}
\addtocontents{toc}{\protect\setcounter{tocdepth}{1}}


\part{Supported Programming Languages}
% User manual for C++
% Copyright (C) Florian Negele

% This file is part of the Eigen Compiler Suite.

% Permission is granted to copy, distribute and/or modify this document
% under the terms of the GNU Free Documentation License, Version 1.3
% or any later version published by the Free Software Foundation.

% You should have received a copy of the GNU Free Documentation License
% along with the ECS.  If not, see <https://www.gnu.org/licenses/>.

% Generic documentation utilities
% Copyright (C) Florian Negele

% This file is part of the Eigen Compiler Suite.

% Permission is granted to copy, distribute and/or modify this document
% under the terms of the GNU Free Documentation License, Version 1.3
% or any later version published by the Free Software Foundation.

% You should have received a copy of the GNU Free Documentation License
% along with the ECS.  If not, see <https://www.gnu.org/licenses/>.

\providecommand{\cpp}{C\texttt{++}}
\providecommand{\opt}{_\mathit{opt}}
\providecommand{\tool}[1]{\texttt{#1}}
\providecommand{\version}{Version 0.0.40}
\providecommand{\resource}[1]{*++\txt{#1}}
\providecommand{\ecs}{Eigen Compiler Suite}
\providecommand{\changed}[1]{\underline{#1}}
\providecommand{\toolbox}[1]{\converter{#1}}
\providecommand{\file}{}\renewcommand{\file}[1]{\texttt{#1}}
\providecommand{\alignright}{\hfill\linebreak[0]\hspace*{\fill}}
\providecommand{\converter}[1]{*++[F][F*:white][F,:gray]\txt{#1}}
\providecommand{\documentation}{\ifbook chapter\else document\fi}
\providecommand{\Documentation}{\ifbook Chapter\else Document\fi}
\providecommand{\variable}[1]{\resource{\texttt{\small#1}\\variable}}
\providecommand{\documentationref}[2]{\ifbook\ref{#1}\else``\href{#1}{#2}''~\cite{#1}\fi}
\providecommand{\objfile}[1]{\texttt{#1}\index[runtime]{#1 object file@\texttt{#1} object file}}
\providecommand{\libfile}[1]{\texttt{#1}\index[runtime]{#1 library file@\texttt{#1} library file}}
\providecommand{\epigraph}[2]{\ifbook\begin{quote}\flushright\textit{#1}\par--- #2\end{quote}\fi}
\providecommand{\environmentvariable}[1]{\texttt{#1}\index{Environment variables!#1@\texttt{#1}}}
\providecommand{\environment}[1]{\texttt{#1}\index[environment]{#1 environment@\texttt{#1} environment}}
\providecommand{\toolsection}{}\renewcommand{\toolsection}[1]{\subsection{#1}\label{\prefix:#1}\tool{#1}}
\providecommand{\instruction}{}\renewcommand{\instruction}[2]{\noindent\qquad\pdftooltip{\texttt{#1}}{#2}\refstepcounter{instruction}\par}
\providecommand{\flowgraph}{}\renewcommand{\flowgraph}[1]{\par\sffamily\begin{displaymath}\xymatrix@=4ex{#1}\end{displaymath}\normalfont\par}
\providecommand{\instructionset}{}\renewcommand{\instructionset}[4]{\setcounter{instruction}{0}\begin{multicols}{\ifbook#3\else#4\fi}[{\captionof{table}[#2]{#2 (\ref*{#1:instructions}~instructions)}\label{tab:#1set}\vspace{-2ex}}]\footnotesize\raggedcolumns\input{#1.set}\label{#1:instructions}\end{multicols}}

\providecommand{\gpl}{GNU General Public License}
\providecommand{\rse}{ECS Runtime Support Exception}
\providecommand{\fdl}{\href{https://www.gnu.org/licenses/fdl.html}{GNU Free Documentation License}}

\providecommand{\docbegin}{}
\providecommand{\docend}{}
\providecommand{\doclabel}[1]{\hypertarget{#1}}
\providecommand{\doclink}[2]{\hyperlink{#1}{#2}}
\providecommand{\docsection}[3]{\hypertarget{#1}{\subsection{#2}}\label{sec:#1}\index[library]{#2@#3}}
\providecommand{\docsectionstar}[1]{}
\providecommand{\docsubbegin}{\begin{description}}
\providecommand{\docsubend}{\end{description}}
\providecommand{\docsubsection}[3]{\item[\hypertarget{#1}{#2}]\index[library]{#2@#3}}
\providecommand{\docsubsectionstar}[1]{\smallskip}
\providecommand{\docsubsubsection}[3]{\docsubsection{#1}{#2}{#3}}
\providecommand{\docsubsubsectionstar}[1]{}
\providecommand{\docsubsubsubsection}[3]{}
\providecommand{\docsubsubsubsectionstar}[1]{}
\providecommand{\doctable}{}

\providecommand{\debuggingtool}{}\renewcommand{\debuggingtool}{This tool is provided for debugging purposes.
It allows exposing and modifying an internal data structure that is usually not accessible.
}

\providecommand{\interface}{All tools accept command-line arguments which are taken as names of plain text files containing the source code.
If no arguments are provided, the standard input stream is used instead.
Output files are generated in the current working directory and have the same name as the input file being processed whereas the filename extension gets replaced by an appropriate suffix.
\seeinterface
}

\providecommand{\license}{\noindent Copyright \copyright{} Florian Negele\par\medskip\noindent
Permission is granted to copy, distribute and/or modify this document under the terms of the
\fdl{}, Version 1.3 or any later version published by the \href{https://fsf.org/}{Free Software Foundation}.
}

\providecommand{\ecslogosurface}{
\fill[darkgray] (0,0,0) -- (0,0,3) -- (0,3,3) -- (0,3,1) -- (0,4,1) -- (0,4,3) -- (0,5,3) -- (0,5,0) -- (0,2,0) -- (0,2,2) -- (0,1,2) -- (0,1,0) -- cycle;
\fill[gray] (0,5,0) -- (0,5,3) -- (1,5,3) -- (1,5,1) -- (2,5,1) -- (2,5,3) -- (3,5,3) -- (3,5,0) -- cycle;
\fill[lightgray] (0,0,0) -- (0,1,0) -- (2,1,0) -- (2,4,0) -- (1,4,0) -- (1,3,0) -- (2,3,0) -- (2,2,0) -- (0,2,0) -- (0,5,0) -- (3,5,0) -- (3,0,0) -- cycle;
\begin{scope}[line width=0.5]
\begin{scope}[gray]
\draw (0,0,0) -- (0,1,0);
\draw (2,1,0) -- (2,2,0);
\draw (0,1,2) -- (0,2,2);
\draw (0,2,0) -- (0,5,0);
\draw (2,3,0) -- (2,4,0);
\end{scope}
\begin{scope}[lightgray]
\draw (0,1,0) -- (0,1,2);
\draw (0,3,1) -- (0,3,3);
\draw (0,5,0) -- (0,5,3);
\draw (2,5,1) -- (2,5,3);
\end{scope}
\begin{scope}[white]
\draw (0,1,0) -- (2,1,0);
\draw (1,3,0) -- (2,3,0);
\draw (0,5,0) -- (3,5,0);
\end{scope}
\end{scope}
}

\providecommand{\ecslogo}[1]{
\begin{tikzpicture}[scale={(#1)/((sin(45)+cos(45))*3cm)},x={({-cos(45)*1cm},{sin(45)*sin(30)*1cm})},y={({0cm},{(cos(30)*1cm})},z={({sin(45)*1cm},{cos(45)*sin(30)*1cm})}]
\begin{scope}[darkgray,line width=1]
\draw (0,0,0) -- (0,0,3) -- (0,3,3) -- (2,3,3) -- (2,5,3) -- (3,5,3) -- (3,5,0) -- (3,0,0) -- cycle;
\draw (0,3,1) -- (0,4,1) -- (0,4,3) -- (0,5,3) -- (1,5,3) -- (1,5,1) -- (2,5,1);
\draw (1,3,0) -- (1,4,0) -- (2,4,0);
\end{scope}
\fill[darkgray] (2,0,0) -- (2,0,3) -- (2,5,3) -- (2,5,1) -- (2,4,1) -- (2,4,0) -- cycle;
\fill[lightgray] (2,0,2) -- (0,0,2) -- (0,2,2) -- (2,2,2) -- cycle;
\fill[gray] (0,1,0) -- (2,1,0) -- (2,1,2) -- (0,1,2) -- cycle;
\fill[gray] (0,3,1) -- (0,3,3) -- (2,3,3) -- (2,3,0) -- (1,3,0) -- (1,3,1) -- cycle;
\ecslogosurface
\end{tikzpicture}
}

\providecommand{\shadowedecslogo}[3]{
\begin{tikzpicture}[scale={(#1)/((sin(#2)+cos(#2))*3cm)},x={({-cos(#2)*1cm},{sin(#2)*sin(#3)*1cm})},y={({0cm},{(cos(#3)*1cm})},z={({sin(#2)*1cm},{cos(#2)*sin(#3)*1cm})}]
\shade[top color=lightgray!50!white,bottom color=white,middle color=lightgray!50!white] (0,0,0) -- (3,0,0) -- (3,{-0.5-3*sin(#2)*sin(#3)/cos(#3)},0) -- (0,-0.5,0) -- cycle;
\shade[top color=darkgray!50!gray,bottom color=white,middle color=darkgray!50!white] (0,0,0) -- (0,0,3) -- (0,{-0.5-3*cos(#2)*sin(#3)/cos(#3)},3) -- (0,-0.5,0) -- cycle;
\begin{scope}[y={({(cos(#2)+sin(#2))*0.5cm},{(cos(#2)*sin(#3)-sin(#2)*sin(#3))*0.5cm})}]
\useasboundingbox (3,0,0) -- (0,0,0) -- (0,0,3);
\shade[left color=darkgray!80!black,right color=lightgray,middle color=gray] (0,0,0) -- (0,1,0) -- (0,1,0.5) -- (0,2,0) -- (0,5,0) -- (0,5,3) -- (1,5,3) -- (1,4,3) -- (1,4,2.5) -- (1,3,3) -- (2,5,3) -- (3,5,3) -- (3,0,3) -- cycle;
\clip (0,0,0) -- (0,0,3) -- ({-3*sin(#2)/cos(#2)},0,0) -- cycle;
\shade[left color=darkgray,right color=lightgray!50!gray] (0,0,0) -- (0,1,0) -- (0,1,0.5) -- (0,2,0) -- (0,5,0) -- (0,5,3) -- (1,5,3) -- (1,4,3) -- (1,4,2.5) -- (1,3,3) -- (2,5,3) -- (3,5,3) -- (3,0,3) -- cycle;
\end{scope}
\shade[left color=darkgray,right color=darkgray!80!black] (2,0,0) -- (2,0,3) -- (2,5,3) -- (2,5,1) -- (2,4,1) -- (2,4,0) -- cycle;
\shade[left color=darkgray!90!black,right color=gray!80!darkgray] (2,0,2) -- (0,0,2) -- (0,2,2) -- (2,2,2) -- cycle;
\shade[top color=darkgray!90!black,bottom color=gray!80!darkgray] (0,1,0) -- (2,1,0) -- (2,1,2) -- (0,1,2) -- cycle;
\shade[top color=darkgray!90!black,bottom color=gray!80!darkgray] (0,3,1) -- (0,3,3) -- (2,3,3) -- (2,3,0) -- (1,3,0) -- (1,3,1) -- cycle;
\fill[gray] (2,1,0) -- (1.5,1,0.5) -- (0,1,0.5) -- (0,1,0) -- cycle;
\fill[gray] (1,3,2) -- (0.5,3,2) -- (0.5,3,3) -- (1,3,3) -- cycle;
\fill[gray] (2,3,0) -- (1.5,3,0.5) -- (1,3,0.5) -- (1,3,0) -- cycle;
\ecslogosurface
\end{tikzpicture}
}

\providecommand{\cpplogo}[1]{
\begin{tikzpicture}[scale=(#1)/512em]
\fill[gray] (435.2794,398.7159) -- (247.1911,507.3075) .. controls (236.3563,513.5642) and (218.6240,513.5642) .. (207.7892,507.3075) -- (19.7009,398.7159) .. controls (8.8646,392.4606) and (0.0000,377.1043) .. (0.0000,364.5924) -- (0.0000,147.4076) .. controls (0.8430,132.8363) and (8.2856,120.7683) .. (19.7009,113.2842) -- (207.7892,4.6926) .. controls (218.6240,-1.5642) and (236.3564,-1.5642) .. (247.1911,4.6926) -- (435.2794,113.2842) .. controls (447.5273,121.4304) and (454.4987,133.6918) .. (454.9803,147.4076) -- (454.9803,364.5924) .. controls (454.5404,377.7571) and (446.6566,391.0351) .. (435.2794,398.7159) -- cycle(75.8301,255.9993) .. controls (74.9389,404.0881) and (273.2892,469.4783) .. (358.8263,331.8769) -- (293.1917,293.8965) .. controls (253.5702,359.4301) and (155.1909,335.9977) .. (151.6601,255.9993) .. controls (152.7204,182.2703) and (249.4137,148.0211) .. (293.1961,218.1065) -- (358.8308,180.1276) .. controls (283.4477,49.2645) and (79.6318,96.3470) .. (75.8301,255.9993) -- cycle(379.1503,247.5747) -- (362.2982,247.5747) -- (362.2982,230.7226) -- (345.4490,230.7226) -- (345.4490,247.5747) -- (328.5969,247.5747) -- (328.5969,264.4254) -- (345.4490,264.4254) -- (345.4490,281.2759) -- (362.2982,281.2759) -- (362.2982,264.4254) -- (379.1503,264.4254) -- cycle(442.3420,247.5747) -- (425.4899,247.5747) -- (425.4899,230.7226) -- (408.6408,230.7226) -- (408.6408,247.5747) -- (391.7886,247.5747) -- (391.7886,264.4254) -- (408.6408,264.4254) -- (408.6408,281.2759) -- (425.4899,281.2759) -- (425.4899,264.4254) -- (442.3420,264.4254) -- cycle;
\end{tikzpicture}
}

\providecommand{\fallogo}[1]{
\begin{tikzpicture}[scale=(#1)/512em]
\fill[gray] (185.7774,0.0000) .. controls (200.4486,15.9798) and (226.8966,8.7148) .. (235.0426,31.5836) .. controls (249.5297,58.0598) and (247.9581,97.9161) .. (280.3335,110.9762) .. controls (309.1690,120.3496) and (337.8406,104.2727) .. (366.5753,103.9379) .. controls (373.4449,111.5171) and (379.2885,128.2574) .. (383.9755,108.9744) .. controls (396.6979,102.5615) and (437.2808,107.6681) .. (426.9652,124.3252) .. controls (408.9822,121.0785) and (412.4742,146.0729) .. (426.5192,131.4996) .. controls (433.8413,120.8489) and (465.1541,126.5522) .. (441.9067,135.7950) .. controls (396.1879,157.7478) and (344.1112,161.5079) .. (298.5528,183.5702) .. controls (277.7471,193.5198) and (284.6941,218.7163) .. (285.2127,236.9640) .. controls (292.3599,316.2826) and (307.3929,394.6311) .. (317.1198,473.6154) .. controls (329.0637,505.4736) and (292.1195,528.5004) .. (265.9183,511.2761) .. controls (237.9284,499.2462) and (237.3684,465.2681) .. (230.9102,439.9421) .. controls (218.6692,374.3397) and (215.6307,306.9662) .. (198.1732,242.3977) .. controls (183.1379,232.7444) and (164.4245,256.0298) .. (149.0430,261.4799) .. controls (116.9328,279.2585) and (87.1822,308.5851) .. (48.2293,307.8914) .. controls (21.3220,306.9037) and (-15.9107,281.8761) .. (7.2921,252.7908) .. controls (29.7799,220.6177) and (67.5177,204.3028) .. (100.9287,185.9449) .. controls (130.8217,170.8906) and (161.1548,156.5903) .. (191.0278,141.5847) .. controls (196.1738,120.0520) and (186.6049,95.2409) .. (186.8382,72.4353) .. controls (185.5234,48.4204) and (183.1700,23.9341) .. (185.7774,0.0000) -- cycle;
\end{tikzpicture}
}

\providecommand{\oblogo}[1]{
\begin{tikzpicture}[scale=(#1)/512em]
\fill[gray] (160.3865,208.9117) .. controls (154.0879,214.6478) and (149.0735,221.2409) .. (145.4125,228.5384) .. controls (184.8790,248.4273) and (234.7122,269.8787) .. (297.5493,291.8782) .. controls (300.3943,281.4769) and (300.9552,268.7619) .. (300.4023,255.2389) .. controls (248.9909,244.7891) and (200.0310,225.9279) .. (160.3865,208.9117) -- cycle(225.7398,392.6996) .. controls (308.0209,392.1716) and (359.3326,345.9277) .. (368.7203,285.2098) .. controls (376.6742,197.1784) and (311.7194,141.3342) .. (205.4287,142.1456) .. controls (139.9485,141.4804) and (88.7155,166.1957) .. (73.5775,228.0086) .. controls (52.0297,320.3408) and (123.4078,391.0103) .. (225.7398,392.6996) -- cycle(216.0739,176.4733) .. controls (268.9183,179.2424) and (315.8292,206.5488) .. (312.7454,265.1139) .. controls (313.2769,315.6384) and (286.5993,353.4946) .. (216.6040,355.7934) .. controls (162.4657,355.7934) and (126.0914,317.5023) .. (126.0914,260.5103) .. controls (126.1733,214.2900) and (163.3363,176.2849) .. (216.0739,176.4733) -- cycle(76.4897,189.1754) .. controls (13.1586,147.5631) and (0.0000,119.4207) .. (0.0000,119.4207) -- (90.6499,170.1632) .. controls (85.3004,175.8497) and (80.5994,182.1633) .. (76.4897,189.1754) -- cycle(353.9486,119.3004) -- (402.9482,119.3004) .. controls (427.0025,137.0797) and (450.9893,162.7034) .. (474.9529,191.0213) .. controls (509.3540,228.5339) and (531.3391,294.2091) .. (487.8149,312.1206) .. controls (462.8165,324.7652) and (394.3874,316.8943) .. (373.8912,313.6651) .. controls (379.9291,297.7449) and (383.2899,278.4204) .. (381.4989,257.7214) .. controls (420.3069,248.0321) and (421.9610,218.3461) .. (407.7867,192.6417) .. controls (391.1113,162.4018) and (370.1114,132.9097) .. (353.9486,119.3004) -- cycle;
\end{tikzpicture}
}

\providecommand{\markuptable}{
\begin{table}
\sffamily\centering
\begin{tabular}{@{}lcl@{}}
\toprule
\texttt{//italics//} & $\rightarrow$ & \textit{italics} \\
\midrule
\texttt{**bold**} & $\rightarrow$ & \textbf{bold} \\
\midrule
\texttt{\# ordered list} & & 1 ordered list \\
\texttt{\# second item} & $\rightarrow$ & 2 second item \\
\texttt{\#\# sub item} & & \hspace{1em} 1 sub item \\
\midrule
\texttt{* unordered list} & & $\bullet$ unordered list \\
\texttt{* second item} & $\rightarrow$ & $\bullet$ second item \\
\texttt{** sub item} & & \hspace{1em} $\bullet$ sub item \\
\midrule
\texttt{link to [[label]]} & $\rightarrow$ & link to \underline{label} \\
\midrule
\texttt{<{}<label>{}> definition } & $\rightarrow$ & definition \\
\midrule
\texttt{[[url|link name]]} & $\rightarrow$ & \underline{link name} \\
\midrule\addlinespace
\texttt{= large heading} & & {\Large large heading} \smallskip \\
\texttt{== medium heading} & $\rightarrow$ & {\large medium heading} \\
\texttt{=== small heading} & & small heading \\
\midrule
\texttt{no line break} & & no line break for paragraphs \\
\texttt{for paragraphs} & $\rightarrow$ \\
& & use empty line \\
\texttt{use empty line} \\
\midrule
\texttt{force\textbackslash\textbackslash line break} & $\rightarrow$ & force \\
& & line break \\
\midrule
\texttt{horizontal line} & $\rightarrow$ & horizontal line \\
\texttt{----} & & \hrulefill \\
\midrule
\texttt{|=a|=table|=header} & & \underline{a \enspace table \enspace header} \\
\texttt{|a|table|row} & $\rightarrow$ & a \enspace table \enspace row \\
\texttt{|b|table|row} & & b \enspace table \enspace row \\
\midrule
\texttt{\{\{\{} \\
\texttt{unformatted} & $\rightarrow$ & \texttt{unformatted} \\
\texttt{code} & & \texttt{code} \\
\texttt{\}\}\}} \\
\midrule\addlinespace
\texttt{@ new article} & & {\Large 1.\ new article} \smallskip \\
\texttt{@ second article} & $\rightarrow$ & {\Large 2.\ second article} \smallskip \\
\texttt{@@ sub article} & & {\large 2.1.\ sub article} \\
\bottomrule
\end{tabular}
\normalfont\caption{Elements of the generic documentation markup language}
\label{tab:docmarkup}
\end{table}
}

\providecommand{\startchapter}[4]{
\documentclass[11pt,a4paper]{article}
\usepackage{booktabs}
\usepackage[format=hang,labelfont=bf]{caption}
\usepackage{changepage}
\usepackage[T1]{fontenc}
\usepackage[margin=2cm]{geometry}
\usepackage{hyperref}
\usepackage[american]{isodate}
\usepackage{lmodern}
\usepackage{longtable}
\usepackage{mathptmx}
\usepackage{microtype}
\usepackage[toc]{multitoc}
\usepackage{multirow}
\usepackage[all]{nowidow}
\usepackage{pdfcomment}
\usepackage{syntax}
\usepackage{tikz}
\usepackage[all]{xy}
\hypersetup{pdfborder={0 0 0},bookmarksnumbered=true,pdftitle={\ecs{}: #2},pdfauthor={Florian Negele},pdfsubject={\ecs{}},pdfkeywords={#1}}
\setlength{\grammarindent}{8em}\setlength{\grammarparsep}{0.2ex}
\setlength{\columnsep}{2em}
\newcommand{\prefix}{}
\newcounter{instruction}
\bibliographystyle{unsrt}
\renewcommand{\index}[2][]{}
\renewcommand{\arraystretch}{1.05}
\renewcommand{\floatpagefraction}{0.7}
\renewcommand{\syntleft}{\itshape}\renewcommand{\syntright}{}
\title{\vspace{-5ex}\Huge{\ecs{}}\medskip\hrule}
\author{\huge{#2}}
\date{\medskip\version}
\newif\ifbook\bookfalse
\pagestyle{headings}
\frenchspacing
\begin{document}
\maketitle\thispagestyle{empty}\noindent#4\setlength{\columnseprule}{0.4pt}\tableofcontents\setlength{\columnseprule}{0pt}\vfill\pagebreak[3]\null\vfill\bigskip\noindent
\parbox{\textwidth-4em}{\license The contents of this \documentation{} are part of the \href{manual}{\ecs{} User Manual}~\cite{manual} and correspond to Chapter ``\href{manual\##3}{#1}''.\alignright\mbox{\today}}
\parbox{4em}{\flushright\ecslogo{3em}}
\clearpage
}

\providecommand{\concludechapter}{
\vfill\pagebreak[3]\null\vfill
\thispagestyle{myheadings}\markright{REFERENCES}
\noindent\begin{minipage}{\textwidth}\begin{multicols}{2}[\section*{References}]
\renewcommand{\section}[2]{}\small\bibliography{references}
\end{multicols}\end{minipage}\end{document}
}

\providecommand{\startpresentation}[2]{
\documentclass[14pt,aspectratio=43,usepdftitle=false]{beamer}
\usepackage{booktabs}
\usepackage{etex}
\usepackage{multicol}
\usepackage{tikz}
\usepackage[all]{xy}
\bibliographystyle{unsrt}
\setlength{\columnsep}{1em}
\setlength{\leftmargini}{1em}
\setbeamercolor{title}{fg=black}
\setbeamercolor{structure}{fg=darkgray}
\setbeamercolor{bibliography item}{fg=darkgray}
\setbeamerfont{title}{series=\bfseries}
\setbeamerfont{subtitle}{series=\normalfont}
\setbeamerfont*{frametitle}{parent=title}
\setbeamerfont{block title}{series=\bfseries}
\setbeamerfont*{framesubtitle}{parent=subtitle}
\setbeamersize{text margin left=1em,text margin right=1em}
\setbeamertemplate{navigation symbols}{}
\setbeamertemplate{itemize item}[circle]{}
\setbeamertemplate{bibliography item}[triangle]{}
\setbeamertemplate{bibliography entry author}{\usebeamercolor[fg]{bibliography item}}
\setbeamertemplate{frametitle}{\medskip\usebeamerfont{frametitle}\color{gray}\raisebox{-2.5ex}[0ex][0ex]{\rule{0.1em}{4.5ex}}}
\addtobeamertemplate{frametitle}{}{\hspace{0.4em}\usebeamercolor[fg]{title}\insertframetitle\par\vspace{0.2ex}\hspace{0.5em}\usebeamerfont{framesubtitle}\insertframesubtitle}
\hypersetup{pdfborder={0 0 0},bookmarksnumbered=true,bookmarksopen=true,bookmarksopenlevel=0,pdftitle={\ecs{}: #1},pdfauthor={Florian Negele},pdfsubject={\ecs{}},pdfkeywords={#1}}
\renewcommand{\flowgraph}[1]{\resizebox{\textwidth}{!}{$$\xymatrix{##1}$$}}
\title{\ecs{}\medskip\hrule\medskip}
\institute{\shadowedecslogo{5em}{30}{15}}
\date{\version}
\subtitle{#1}
\begin{document}
\begin{frame}[plain]\titlepage\nocite{manual}\end{frame}
\begin{frame}{Contents}{#1}\begin{center}\tableofcontents\end{center}\end{frame}
}

\providecommand{\concludepresentation}{
\begin{frame}{References}\begin{footnotesize}\setlength{\columnseprule}{0.4pt}\begin{multicols}{2}\bibliography{references}\end{multicols}\end{footnotesize}\end{frame}
\end{document}
}

\providecommand{\startbook}[1]{
\documentclass[10pt,paper=17cm:24cm,DIV=13,twoside=semi,headings=normal,numbers=noendperiod,cleardoublepage=plain]{scrbook}
\usepackage{atveryend}
\usepackage{booktabs}
\usepackage{caption}
\usepackage{changepage}
\usepackage[T1]{fontenc}
\usepackage{imakeidx}
\usepackage{hyperref}
\usepackage[american]{isodate}
\usepackage{lmodern}
\usepackage{longtable}
\usepackage{mathptmx}
\usepackage[final]{microtype}
\usepackage{multicol}
\usepackage{multirow}
\usepackage[all]{nowidow}
\usepackage{pdfcomment}
\usepackage{scrlayer-scrpage}
\usepackage{setspace}
\usepackage{syntax}
\usepackage[eventxtindent=4pt,oddtxtexdent=4pt]{thumbs}
\usepackage{tikz}
\usepackage[all]{xy}
\hyphenation{Micro-Blaze Open-Cores Open-RISC Power-PC}
\hypersetup{pdfborder={0 0 0},bookmarksnumbered=true,bookmarksopen=true,bookmarksopenlevel=0,pdftitle={\ecs{}: #1},pdfauthor={Florian Negele},pdfsubject={\ecs{}},pdfkeywords={#1}}
\setlength{\grammarindent}{8em}\setlength{\grammarparsep}{0.7ex}
\setkomafont{captionlabel}{\usekomafont{descriptionlabel}}
\renewcommand{\arraystretch}{1.05}\setstretch{1.1}
\renewcommand{\chapterformat}{\thechapter\autodot\enskip\raisebox{-1ex}[0ex][0ex]{\color{gray}\rule{0.1em}{3.5ex}}\enskip}
\renewcommand{\startchapter}[4]{\hypertarget{##3}{\chapter{##1}}\label{##3}##4\addthumb{##1}{\LARGE\sffamily\bfseries\thechapter}{white}{gray}\renewcommand{\prefix}{##3}}
\renewcommand{\concludechapter}{\clearpage{\stopthumb\cleardoublepage}}
\renewcommand{\syntleft}{\itshape}\renewcommand{\syntright}{}
\renewcommand{\floatpagefraction}{0.7}
\renewcommand{\partheademptypage}{}
\DeclareMicrotypeAlias{lmss}{cmr}
\newcommand{\prefix}{}
\newcounter{instruction}
\bibliographystyle{unsrt}
\newif\ifbook\booktrue
\makeindex[intoc,title=Index]
\makeindex[intoc,name=tools,title=Index of Tools,columns=3]
\makeindex[intoc,name=library,title=Index of Library Names]
\makeindex[intoc,name=runtime,title=Index of Runtime Support]
\makeindex[intoc,name=environment,title=Index of Target Environments]
\indexsetup{toclevel=chapter,headers={\indexname}{\indexname}}
\frenchspacing
\begin{document}
\pagenumbering{alph}
\begin{titlepage}\centering
\huge\sffamily\null\vfill\textbf{\ecs{}}\bigskip\hrule\bigskip#1
\normalsize\normalfont\vfill\vfill\shadowedecslogo{10em}{30}{15}
\large\vfill\vfill\version
\end{titlepage}
\null\vfill
\thispagestyle{empty}
\noindent\today\par\medskip
\license A copy of this license is included in Appendix~\ref{fdl} on page~\pageref{fdl}.
All product names used herein are for identification purposes only and may be trademarks of their respective companies.
\concludechapter
\frontmatter
\setcounter{tocdepth}{1}
\tableofcontents
\setcounter{tocdepth}{2}
\concludechapter
\listoffigures
\concludechapter
\listoftables
\concludechapter
}

\providecommand{\concludebook}{
\backmatter
\addtocontents{toc}{\protect\setcounter{tocdepth}{-1}}
\phantomsection\addcontentsline{toc}{part}{Bibliography}
\bibliography{references}
\concludechapter
\phantomsection\addcontentsline{toc}{part}{Indexes}
\printindex
\concludechapter
\indexprologue{\label{idx:tools}}
\printindex[tools]
\concludechapter
\printindex[library]
\concludechapter
\indexprologue{\label{idx:runtime}}
\printindex[runtime]
\concludechapter
\indexprologue{\label{idx:environment}}
\printindex[environment]
\concludechapter
\pagestyle{empty}\pagenumbering{Alph}\null\clearpage
\null\vfill\centering\ecslogo{4em}\par\medskip\license
\end{document}
}

% chapter references

\providecommand{\seedocumentationref}{}\renewcommand{\seedocumentationref}[3]{#1, see \Documentation{}~\documentationref{#2}{#3}. }
\providecommand{\seeinterface}{}\renewcommand{\seeinterface}{\ifbook See \Documentation{}~\documentationref{interface}{User Interface} for more information about the common user interface of all of these tools. \fi}
\providecommand{\seeguide}{}\renewcommand{\seeguide}{\seedocumentationref{For basic examples of using some of these tools in practice}{guide}{User Guide}}
\providecommand{\seecpp}{}\renewcommand{\seecpp}{\seedocumentationref{For more information about the \cpp{} programming language and its implementation by the \ecs{}}{cpp}{User Manual for \cpp{}}}
\providecommand{\seefalse}{}\renewcommand{\seefalse}{\seedocumentationref{For more information about the FALSE programming language and its implementation by the \ecs{}}{false}{User Manual for FALSE}}
\providecommand{\seeoberon}{}\renewcommand{\seeoberon}{\seedocumentationref{For more information about the Oberon programming language and its implementation by the \ecs{}}{oberon}{User Manual for Oberon}}
\providecommand{\seeassembly}{}\renewcommand{\seeassembly}{\seedocumentationref{For more information about the generic assembly language and how to use it}{assembly}{Generic Assembly Language Specification}}
\providecommand{\seeamd}{}\renewcommand{\seeamd}{\seedocumentationref{For more information about how the \ecs{} supports the AMD64 hardware architecture}{amd64}{AMD64 Hardware Architecture Support}}
\providecommand{\seearm}{}\renewcommand{\seearm}{\seedocumentationref{For more information about how the \ecs{} supports the ARM hardware architecture}{arm}{ARM Hardware Architecture Support}}
\providecommand{\seeavr}{}\renewcommand{\seeavr}{\seedocumentationref{For more information about how the \ecs{} supports the AVR hardware architecture}{avr}{AVR Hardware Architecture Support}}
\providecommand{\seeavrtt}{}\renewcommand{\seeavrtt}{\seedocumentationref{For more information about how the \ecs{} supports the AVR32 hardware architecture}{avr32}{AVR32 Hardware Architecture Support}}
\providecommand{\seemabk}{}\renewcommand{\seemabk}{\seedocumentationref{For more information about how the \ecs{} supports the M68000 hardware architecture}{m68k}{M68000 Hardware Architecture Support}}
\providecommand{\seemibl}{}\renewcommand{\seemibl}{\seedocumentationref{For more information about how the \ecs{} supports the MicroBlaze hardware architecture}{mibl}{MicroBlaze Hardware Architecture Support}}
\providecommand{\seemips}{}\renewcommand{\seemips}{\seedocumentationref{For more information about how the \ecs{} supports the MIPS32 and MIPS64 hardware architectures}{mips}{MIPS Hardware Architecture Support}}
\providecommand{\seemmix}{}\renewcommand{\seemmix}{\seedocumentationref{For more information about how the \ecs{} supports the MMIX hardware architecture}{mmix}{MMIX Hardware Architecture Support}}
\providecommand{\seeorok}{}\renewcommand{\seeorok}{\seedocumentationref{For more information about how the \ecs{} supports the OpenRISC 1000 hardware architecture}{or1k}{OpenRISC 1000 Hardware Architecture Support}}
\providecommand{\seeppc}{}\renewcommand{\seeppc}{\seedocumentationref{For more information about how the \ecs{} supports the PowerPC hardware architecture}{ppc}{PowerPC Hardware Architecture Support}}
\providecommand{\seerisc}{}\renewcommand{\seerisc}{\seedocumentationref{For more information about how the \ecs{} supports the RISC hardware architecture}{risc}{RISC Hardware Architecture Support}}
\providecommand{\seewasm}{}\renewcommand{\seewasm}{\seedocumentationref{For more information about how the \ecs{} supports the WebAssembly architecture}{wasm}{WebAssembly Architecture Support}}
\providecommand{\seedocumentation}{}\renewcommand{\seedocumentation}{\seedocumentationref{For more information about generic documentations and their generation by the \ecs{}}{documentation}{Generic Documentation Generation}}
\providecommand{\seedebugging}{}\renewcommand{\seedebugging}{\seedocumentationref{For more information about debugging information and its representation}{debugging}{Debugging Information Representation}}
\providecommand{\seecode}{}\renewcommand{\seecode}{\seedocumentationref{For more information about intermediate code and its purpose}{code}{Intermediate Code Representation}}
\providecommand{\seeobject}{}\renewcommand{\seeobject}{\seedocumentationref{For more information about object files and their purpose}{object}{Object File Representation}}

% generic documentation tools

\providecommand{\docprint}{
\toolsection{docprint} is a pretty printer for generic documentations.
It reformats generic documentations and writes it to the standard output stream.
\debuggingtool
\flowgraph{\resource{generic\\documentation} \ar[r] & \toolbox{docprint} \ar[r] & \resource{generic\\documentation}}
\seedocumentation
}

\providecommand{\doccheck}{
\toolsection{doccheck} is a syntactic and semantic checker for generic documentations.
It just performs syntactic and semantic checks on generic documentations and writes its diagnostic messages to the standard error stream.
\debuggingtool
\flowgraph{\resource{generic\\documentation} \ar[r] & \toolbox{doccheck} \ar[r] & \resource{diagnostic\\messages}}
\seedocumentation
}

\providecommand{\dochtml}{
\toolsection{dochtml} is an HTML documentation generator for generic documentations.
It processes several generic documentations and assembles all information therein into an HTML document.
\debuggingtool
\flowgraph{\resource{generic\\documentation} \ar[r] & \toolbox{dochtml} \ar[r] & \resource{HTML\\document}}
\seedocumentation
}

\providecommand{\doclatex}{
\toolsection{doclatex} is a Latex documentation generator for generic documentations.
It processes several generic documentations and assembles all information therein into a Latex document.
\debuggingtool
\flowgraph{\resource{generic\\documentation} \ar[r] & \toolbox{doclatex} \ar[r] & \resource{Latex\\document}}
\seedocumentation
}

% intermediate code tools

\providecommand{\cdcheck}{
\toolsection{cdcheck} is a syntactic and semantic checker for intermediate code.
It just performs syntactic and semantic checks on programs written in intermediate code and writes its diagnostic messages to the standard error stream.
\debuggingtool
\flowgraph{\resource{intermediate\\code} \ar[r] & \toolbox{cdcheck} \ar[r] & \resource{diagnostic\\messages}}
\seeassembly\seecode
}

\providecommand{\cdopt}{
\toolsection{cdopt} is an optimizer for intermediate code.
It performs various optimizations on programs written in intermediate code and writes the result to the standard output stream.
\debuggingtool
\flowgraph{\resource{intermediate\\code} \ar[r] & \toolbox{cdopt} \ar[r] & \resource{optimized\\code}}
\seeassembly\seecode
}

\providecommand{\cdrun}{
\toolsection{cdrun} is an interpreter for intermediate code.
It processes and executes programs written in intermediate code.
The following code sections are predefined and have the usual semantics:
\texttt{abort}, \texttt{\_Exit}, \texttt{fflush}, \texttt{floor}, \texttt{fputc}, \texttt{free}, \texttt{getchar}, \texttt{malloc}, and \texttt{putchar}.
Diagnostic messages about invalid operations include the name of the executed code section and the index of the erroneous instruction.
\debuggingtool
\flowgraph{\resource{intermediate\\code} \ar[r] & \toolbox{cdrun} \ar@/u/[r] & \resource{input/\\output} \ar@/d/[l]}
\seeassembly\seecode
}

\providecommand{\cdamda}{
\toolsection{cdamd16} is a compiler for intermediate code targeting the AMD64 hardware architecture.
It generates machine code for AMD64 processors from programs written in intermediate code and stores it in corresponding object files.
The compiler generates machine code for the 16-bit operating mode defined by the AMD64 architecture.
It also creates a debugging information file as well as an assembly file containing a listing of the generated machine code.
\debuggingtool
\flowgraph{\resource{intermediate\\code} \ar[r] & \toolbox{cdamd16} \ar[r] \ar[d] \ar[rd] & \resource{object file} \\ & \resource{assembly\\listing} & \resource{debugging\\information}}
\seeassembly\seeamd\seeobject\seecode\seedebugging
}

\providecommand{\cdamdb}{
\toolsection{cdamd32} is a compiler for intermediate code targeting the AMD64 hardware architecture.
It generates machine code for AMD64 processors from programs written in intermediate code and stores it in corresponding object files.
The compiler generates machine code for the 32-bit operating mode defined by the AMD64 architecture.
It also creates a debugging information file as well as an assembly file containing a listing of the generated machine code.
\debuggingtool
\flowgraph{\resource{intermediate\\code} \ar[r] & \toolbox{cdamd32} \ar[r] \ar[d] \ar[rd] & \resource{object file} \\ & \resource{assembly\\listing} & \resource{debugging\\information}}
\seeassembly\seeamd\seeobject\seecode\seedebugging
}

\providecommand{\cdamdc}{
\toolsection{cdamd64} is a compiler for intermediate code targeting the AMD64 hardware architecture.
It generates machine code for AMD64 processors from programs written in intermediate code and stores it in corresponding object files.
The compiler generates machine code for the 64-bit operating mode defined by the AMD64 architecture.
It also creates a debugging information file as well as an assembly file containing a listing of the generated machine code.
\debuggingtool
\flowgraph{\resource{intermediate\\code} \ar[r] & \toolbox{cdamd64} \ar[r] \ar[d] \ar[rd] & \resource{object file} \\ & \resource{assembly\\listing} & \resource{debugging\\information}}
\seeassembly\seeamd\seeobject\seecode\seedebugging
}

\providecommand{\cdarma}{
\toolsection{cdarma32} is a compiler for intermediate code targeting the ARM hardware architecture.
It generates machine code for ARM processors executing A32 instructions from programs written in intermediate code and stores it in corresponding object files.
It also creates a debugging information file as well as an assembly file containing a listing of the generated machine code.
\debuggingtool
\flowgraph{\resource{intermediate\\code} \ar[r] & \toolbox{cdarma32} \ar[r] \ar[d] \ar[rd] & \resource{object file} \\ & \resource{assembly\\listing} & \resource{debugging\\information}}
\seeassembly\seearm\seeobject\seecode\seedebugging
}

\providecommand{\cdarmb}{
\toolsection{cdarma64} is a compiler for intermediate code targeting the ARM hardware architecture.
It generates machine code for ARM processors executing A64 instructions from programs written in intermediate code and stores it in corresponding object files.
It also creates a debugging information file as well as an assembly file containing a listing of the generated machine code.
\debuggingtool
\flowgraph{\resource{intermediate\\code} \ar[r] & \toolbox{cdarma64} \ar[r] \ar[d] \ar[rd] & \resource{object file} \\ & \resource{assembly\\listing} & \resource{debugging\\information}}
\seeassembly\seearm\seeobject\seecode\seedebugging
}

\providecommand{\cdarmc}{
\toolsection{cdarmt32} is a compiler for intermediate code targeting the ARM hardware architecture.
It generates machine code for ARM processors without floating-point extension executing T32 instructions from programs written in intermediate code and stores it in corresponding object files.
It also creates a debugging information file as well as an assembly file containing a listing of the generated machine code.
\debuggingtool
\flowgraph{\resource{intermediate\\code} \ar[r] & \toolbox{cdarmt32} \ar[r] \ar[d] \ar[rd] & \resource{object file} \\ & \resource{assembly\\listing} & \resource{debugging\\information}}
\seeassembly\seearm\seeobject\seecode\seedebugging
}

\providecommand{\cdarmcfpe}{
\toolsection{cdarmt32fpe} is a compiler for intermediate code targeting the ARM hardware architecture.
It generates machine code for ARM processors with floating-point extension executing T32 instructions from programs written in intermediate code and stores it in corresponding object files.
It also creates a debugging information file as well as an assembly file containing a listing of the generated machine code.
\debuggingtool
\flowgraph{\resource{intermediate\\code} \ar[r] & \toolbox{cdarmt32fpe} \ar[r] \ar[d] \ar[rd] & \resource{object file} \\ & \resource{assembly\\listing} & \resource{debugging\\information}}
\seeassembly\seearm\seeobject\seecode\seedebugging
}

\providecommand{\cdavr}{
\toolsection{cdavr} is a compiler for intermediate code targeting the AVR hardware architecture.
It generates machine code for AVR processors from programs written in intermediate code and stores it in corresponding object files.
It also creates a debugging information file as well as an assembly file containing a listing of the generated machine code.
\debuggingtool
\flowgraph{\resource{intermediate\\code} \ar[r] & \toolbox{cdavr} \ar[r] \ar[d] \ar[rd] & \resource{object file} \\ & \resource{assembly\\listing} & \resource{debugging\\information}}
\seeassembly\seeavr\seeobject\seecode\seedebugging
}

\providecommand{\cdavrtt}{
\toolsection{cdavr32} is a compiler for intermediate code targeting the AVR32 hardware architecture.
It generates machine code for AVR32 processors from programs written in intermediate code and stores it in corresponding object files.
It also creates a debugging information file as well as an assembly file containing a listing of the generated machine code.
\debuggingtool
\flowgraph{\resource{intermediate\\code} \ar[r] & \toolbox{cdavr32} \ar[r] \ar[d] \ar[rd] & \resource{object file} \\ & \resource{assembly\\listing} & \resource{debugging\\information}}
\seeassembly\seeavrtt\seeobject\seecode\seedebugging
}

\providecommand{\cdmabk}{
\toolsection{cdm68k} is a compiler for intermediate code targeting the M68000 hardware architecture.
It generates machine code for M68000 processors from programs written in intermediate code and stores it in corresponding object files.
It also creates a debugging information file as well as an assembly file containing a listing of the generated machine code.
\debuggingtool
\flowgraph{\resource{intermediate\\code} \ar[r] & \toolbox{cdm68k} \ar[r] \ar[d] \ar[rd] & \resource{object file} \\ & \resource{assembly\\listing} & \resource{debugging\\information}}
\seeassembly\seemabk\seeobject\seecode\seedebugging
}

\providecommand{\cdmibl}{
\toolsection{cdmibl} is a compiler for intermediate code targeting the MicroBlaze hardware architecture.
It generates machine code for MicroBlaze processors from programs written in intermediate code and stores it in corresponding object files.
It also creates a debugging information file as well as an assembly file containing a listing of the generated machine code.
\debuggingtool
\flowgraph{\resource{intermediate\\code} \ar[r] & \toolbox{cdmibl} \ar[r] \ar[d] \ar[rd] & \resource{object file} \\ & \resource{assembly\\listing} & \resource{debugging\\information}}
\seeassembly\seemibl\seeobject\seecode\seedebugging
}

\providecommand{\cdmipsa}{
\toolsection{cdmips32} is a compiler for intermediate code targeting the MIPS32 hardware architecture.
It generates machine code for MIPS32 processors from programs written in intermediate code and stores it in corresponding object files.
It also creates a debugging information file as well as an assembly file containing a listing of the generated machine code.
\debuggingtool
\flowgraph{\resource{intermediate\\code} \ar[r] & \toolbox{cdmips32} \ar[r] \ar[d] \ar[rd] & \resource{object file} \\ & \resource{assembly\\listing} & \resource{debugging\\information}}
\seeassembly\seemips\seeobject\seecode\seedebugging
}

\providecommand{\cdmipsb}{
\toolsection{cdmips64} is a compiler for intermediate code targeting the MIPS64 hardware architecture.
It generates machine code for MIPS64 processors from programs written in intermediate code and stores it in corresponding object files.
It also creates a debugging information file as well as an assembly file containing a listing of the generated machine code.
\debuggingtool
\flowgraph{\resource{intermediate\\code} \ar[r] & \toolbox{cdmips64} \ar[r] \ar[d] \ar[rd] & \resource{object file} \\ & \resource{assembly\\listing} & \resource{debugging\\information}}
\seeassembly\seemips\seeobject\seecode\seedebugging
}

\providecommand{\cdmmix}{
\toolsection{cdmmix} is a compiler for intermediate code targeting the MMIX hardware architecture.
It generates machine code for MMIX processors from programs written in intermediate code and stores it in corresponding object files.
It also creates a debugging information file as well as an assembly file containing a listing of the generated machine code.
\debuggingtool
\flowgraph{\resource{intermediate\\code} \ar[r] & \toolbox{cdmmix} \ar[r] \ar[d] \ar[rd] & \resource{object file} \\ & \resource{assembly\\listing} & \resource{debugging\\information}}
\seeassembly\seemmix\seeobject\seecode\seedebugging
}

\providecommand{\cdorok}{
\toolsection{cdor1k} is a compiler for intermediate code targeting the OpenRISC 1000 hardware architecture.
It generates machine code for OpenRISC 1000 processors from programs written in intermediate code and stores it in corresponding object files.
It also creates a debugging information file as well as an assembly file containing a listing of the generated machine code.
\debuggingtool
\flowgraph{\resource{intermediate\\code} \ar[r] & \toolbox{cdor1k} \ar[r] \ar[d] \ar[rd] & \resource{object file} \\ & \resource{assembly\\listing} & \resource{debugging\\information}}
\seeassembly\seeorok\seeobject\seecode\seedebugging
}

\providecommand{\cdppca}{
\toolsection{cdppc32} is a compiler for intermediate code targeting the PowerPC hardware architecture.
It generates machine code for PowerPC processors from programs written in intermediate code and stores it in corresponding object files.
The compiler generates machine code for the 32-bit operating mode defined by the PowerPC architecture.
It also creates a debugging information file as well as an assembly file containing a listing of the generated machine code.
\debuggingtool
\flowgraph{\resource{intermediate\\code} \ar[r] & \toolbox{cdppc32} \ar[r] \ar[d] \ar[rd] & \resource{object file} \\ & \resource{assembly\\listing} & \resource{debugging\\information}}
\seeassembly\seeppc\seeobject\seecode\seedebugging
}

\providecommand{\cdppcb}{
\toolsection{cdppc64} is a compiler for intermediate code targeting the PowerPC hardware architecture.
It generates machine code for PowerPC processors from programs written in intermediate code and stores it in corresponding object files.
The compiler generates machine code for the 64-bit operating mode defined by the PowerPC architecture.
It also creates a debugging information file as well as an assembly file containing a listing of the generated machine code.
\debuggingtool
\flowgraph{\resource{intermediate\\code} \ar[r] & \toolbox{cdppc64} \ar[r] \ar[d] \ar[rd] & \resource{object file} \\ & \resource{assembly\\listing} & \resource{debugging\\information}}
\seeassembly\seeppc\seeobject\seecode\seedebugging
}

\providecommand{\cdrisc}{
\toolsection{cdrisc} is a compiler for intermediate code targeting the RISC hardware architecture.
It generates machine code for RISC processors from programs written in intermediate code and stores it in corresponding object files.
It also creates a debugging information file as well as an assembly file containing a listing of the generated machine code.
\debuggingtool
\flowgraph{\resource{intermediate\\code} \ar[r] & \toolbox{cdrisc} \ar[r] \ar[d] \ar[rd] & \resource{object file} \\ & \resource{assembly\\listing} & \resource{debugging\\information}}
\seeassembly\seerisc\seeobject\seecode\seedebugging
}

\providecommand{\cdwasm}{
\toolsection{cdwasm} is a compiler for intermediate code targeting the WebAssembly architecture.
It generates machine code for WebAssembly targets from programs written in intermediate code and stores it in corresponding object files.
It also creates a debugging information file as well as an assembly file containing a listing of the generated machine code.
\debuggingtool
\flowgraph{\resource{intermediate\\code} \ar[r] & \toolbox{cdwasm} \ar[r] \ar[d] \ar[rd] & \resource{object file} \\ & \resource{assembly\\listing} & \resource{debugging\\information}}
\seeassembly\seewasm\seeobject\seecode\seedebugging
}

% C++ tools

\providecommand{\cppprep}{
\toolsection{cppprep} is a preprocessor for the \cpp{} programming language.
It preprocesses source code according to the rules of \cpp{} and writes it to the standard output stream.
Only the macro names \texttt{\_\_DATE\_\_}, \texttt{\_\_FILE\_\_}, \texttt{\_\_LINE\_\_}, and \texttt{\_\_TIME\_\_} are predefined.
\flowgraph{\resource{\cpp{} or other\\source code} \ar[r] & \toolbox{cppprep} \ar[r] & \resource{preprocessed\\source code} \\ & \variable{ECSINCLUDE} \ar[u]}
\seecpp
}

\providecommand{\cppprint}{
\toolsection{cppprint} is a pretty printer for the \cpp{} programming language.
It reformats the source code of \cpp{} programs and writes it to the standard output stream.
\flowgraph{\resource{\cpp{}\\source code} \ar[r] & \toolbox{cppprint} \ar[r] & \resource{reformatted\\source code} \\ & \variable{ECSINCLUDE} \ar[u]}
\seecpp
}

\providecommand{\cppcheck}{
\toolsection{cppcheck} is a syntactic and semantic checker for the \cpp{} programming language.
It just performs syntactic and semantic checks on \cpp{} programs and writes its diagnostic messages to the standard error stream.
\flowgraph{\resource{\cpp{}\\source code} \ar[r] & \toolbox{cppcheck} \ar[r] & \resource{diagnostic\\messages} \\ & \variable{ECSINCLUDE} \ar[u]}
\seecpp
}

\providecommand{\cppdump}{
\toolsection{cppdump} is a serializer for the \cpp{} programming language.
It dumps the complete internal representation of programs written in \cpp{} into an XML document.
\debuggingtool
\flowgraph{\resource{\cpp{}\\source code} \ar[r] & \toolbox{cppdump} \ar[r] & \resource{internal\\representation} \\ & \variable{ECSINCLUDE} \ar[u]}
\seecpp
}

\providecommand{\cpprun}{
\toolsection{cpprun} is an interpreter for the \cpp{} programming language.
It processes and executes programs written in \cpp{}.
The macro \texttt{\_\_run\_\_} is predefined in order to enable programmers to identify this tool while interpreting.
\flowgraph{\resource{\cpp{}\\source code} \ar[r] & \toolbox{cpprun} \ar@/u/[r] & \resource{input/\\output} \ar@/d/[l] \\ & \variable{ECSINCLUDE} \ar[u]}
\seecpp
}

\providecommand{\cppdoc}{
\toolsection{cppdoc} is a generic documentation generator for the \cpp{} programming language.
It processes several \cpp{} source files and assembles all information therein into a generic documentation.
\debuggingtool
\flowgraph{\resource{\cpp{}\\source code} \ar[r] & \toolbox{cppdoc} \ar[r] & \resource{generic\\documentation} \\ & \variable{ECSINCLUDE} \ar[u]}
\seecpp\seedocumentation
}

\providecommand{\cpphtml}{
\toolsection{cpphtml} is an HTML documentation generator for the \cpp{} programming language.
It processes several \cpp{} source files and assembles all information therein into an HTML document.
\flowgraph{\resource{\cpp{}\\source code} \ar[r] & \toolbox{cpphtml} \ar[r] & \resource{HTML\\document} \\ & \variable{ECSINCLUDE} \ar[u]}
\seecpp\seedocumentation
}

\providecommand{\cpplatex}{
\toolsection{cpplatex} is a Latex documentation generator for the \cpp{} programming language.
It processes several \cpp{} source files and assembles all information therein into a Latex document.
\flowgraph{\resource{\cpp{}\\source code} \ar[r] & \toolbox{cpplatex} \ar[r] & \resource{Latex\\document} \\ & \variable{ECSINCLUDE} \ar[u]}
\seecpp\seedocumentation
}

\providecommand{\cppcode}{
\toolsection{cppcode} is an intermediate code generator for the \cpp{} programming language.
It generates intermediate code from programs written in \cpp{} and stores it in corresponding assembly files.
The macro \texttt{\_\_code\_\_} is predefined in order to enable programmers to identify this tool while generating intermediate code.
Programs generated with this tool require additional runtime support that is stored in the \file{cpp\-code\-run} library file.
\debuggingtool
\flowgraph{\resource{\cpp{}\\source code} \ar[r] & \toolbox{cppcode} \ar[r] & \resource{intermediate\\code} \\ & \variable{ECSINCLUDE} \ar[u]}
\seecpp\seeassembly\seecode
}

\providecommand{\cppamda}{
\toolsection{cppamd16} is a compiler for the \cpp{} programming language targeting the AMD64 hardware architecture.
It generates machine code for AMD64 processors from programs written in \cpp{} and stores it in corresponding object files.
The compiler generates machine code for the 16-bit operating mode defined by the AMD64 architecture.
For debugging purposes, it also creates a debugging information file as well as an assembly file containing a listing of the generated machine code.
The macro \texttt{\_\_amd16\_\_} is predefined in order to enable programmers to identify this tool and its target architecture while compiling.
Programs generated with this compiler require additional runtime support that is stored in the \file{cpp\-amd16\-run} library file.
\flowgraph{\resource{\cpp{}\\source code} \ar[r] & \toolbox{cppamd16} \ar[r] \ar[d] \ar[rd] & \resource{object file} \\ \variable{ECSINCLUDE} \ar[ru] & \resource{debugging\\information} & \resource{assembly\\listing}}
\seecpp\seeassembly\seeamd\seeobject\seedebugging
}

\providecommand{\cppamdb}{
\toolsection{cppamd32} is a compiler for the \cpp{} programming language targeting the AMD64 hardware architecture.
It generates machine code for AMD64 processors from programs written in \cpp{} and stores it in corresponding object files.
The compiler generates machine code for the 32-bit operating mode defined by the AMD64 architecture.
For debugging purposes, it also creates a debugging information file as well as an assembly file containing a listing of the generated machine code.
The macro \texttt{\_\_amd32\_\_} is predefined in order to enable programmers to identify this tool and its target architecture while compiling.
Programs generated with this compiler require additional runtime support that is stored in the \file{cpp\-amd32\-run} library file.
\flowgraph{\resource{\cpp{}\\source code} \ar[r] & \toolbox{cppamd32} \ar[r] \ar[d] \ar[rd] & \resource{object file} \\ \variable{ECSINCLUDE} \ar[ru] & \resource{debugging\\information} & \resource{assembly\\listing}}
\seecpp\seeassembly\seeamd\seeobject\seedebugging
}

\providecommand{\cppamdc}{
\toolsection{cppamd64} is a compiler for the \cpp{} programming language targeting the AMD64 hardware architecture.
It generates machine code for AMD64 processors from programs written in \cpp{} and stores it in corresponding object files.
The compiler generates machine code for the 64-bit operating mode defined by the AMD64 architecture.
For debugging purposes, it also creates a debugging information file as well as an assembly file containing a listing of the generated machine code.
The macro \texttt{\_\_amd64\_\_} is predefined in order to enable programmers to identify this tool and its target architecture while compiling.
Programs generated with this compiler require additional runtime support that is stored in the \file{cpp\-amd64\-run} library file.
\flowgraph{\resource{\cpp{}\\source code} \ar[r] & \toolbox{cppamd64} \ar[r] \ar[d] \ar[rd] & \resource{object file} \\ \variable{ECSINCLUDE} \ar[ru] & \resource{debugging\\information} & \resource{assembly\\listing}}
\seecpp\seeassembly\seeamd\seeobject\seedebugging
}

\providecommand{\cpparma}{
\toolsection{cpparma32} is a compiler for the \cpp{} programming language targeting the ARM hardware architecture.
It generates machine code for ARM processors executing A32 instructions from programs written in \cpp{} and stores it in corresponding object files.
For debugging purposes, it also creates a debugging information file as well as an assembly file containing a listing of the generated machine code.
The macro \texttt{\_\_arma32\_\_} is predefined in order to enable programmers to identify this tool and its target architecture while compiling.
Programs generated with this compiler require additional runtime support that is stored in the \file{cpp\-arma32\-run} library file.
\flowgraph{\resource{\cpp{}\\source code} \ar[r] & \toolbox{cpparma32} \ar[r] \ar[d] \ar[rd] & \resource{object file} \\ \variable{ECSINCLUDE} \ar[ru] & \resource{debugging\\information} & \resource{assembly\\listing}}
\seecpp\seeassembly\seearm\seeobject\seedebugging
}

\providecommand{\cpparmb}{
\toolsection{cpparma64} is a compiler for the \cpp{} programming language targeting the ARM hardware architecture.
It generates machine code for ARM processors executing A64 instructions from programs written in \cpp{} and stores it in corresponding object files.
For debugging purposes, it also creates a debugging information file as well as an assembly file containing a listing of the generated machine code.
The macro \texttt{\_\_arma64\_\_} is predefined in order to enable programmers to identify this tool and its target architecture while compiling.
Programs generated with this compiler require additional runtime support that is stored in the \file{cpp\-arma64\-run} library file.
\flowgraph{\resource{\cpp{}\\source code} \ar[r] & \toolbox{cpparma64} \ar[r] \ar[d] \ar[rd] & \resource{object file} \\ \variable{ECSINCLUDE} \ar[ru] & \resource{debugging\\information} & \resource{assembly\\listing}}
\seecpp\seeassembly\seearm\seeobject\seedebugging
}

\providecommand{\cpparmc}{
\toolsection{cpparmt32} is a compiler for the \cpp{} programming language targeting the ARM hardware architecture.
It generates machine code for ARM processors without floating-point extension executing T32 instructions from programs written in \cpp{} and stores it in corresponding object files.
For debugging purposes, it also creates a debugging information file as well as an assembly file containing a listing of the generated machine code.
The macro \texttt{\_\_armt32\_\_} is predefined in order to enable programmers to identify this tool and its target architecture while compiling.
Programs generated with this compiler require additional runtime support that is stored in the \file{cpp\-armt32\-run} library file.
\flowgraph{\resource{\cpp{}\\source code} \ar[r] & \toolbox{cpparmt32} \ar[r] \ar[d] \ar[rd] & \resource{object file} \\ \variable{ECSINCLUDE} \ar[ru] & \resource{debugging\\information} & \resource{assembly\\listing}}
\seecpp\seeassembly\seearm\seeobject\seedebugging
}

\providecommand{\cpparmcfpe}{
\toolsection{cpparmt32fpe} is a compiler for the \cpp{} programming language targeting the ARM hardware architecture.
It generates machine code for ARM processors with floating-point extension executing T32 instructions from programs written in \cpp{} and stores it in corresponding object files.
For debugging purposes, it also creates a debugging information file as well as an assembly file containing a listing of the generated machine code.
The macro \texttt{\_\_armt32fpe\_\_} is predefined in order to enable programmers to identify this tool and its target architecture while compiling.
Programs generated with this compiler require additional runtime support that is stored in the \file{cpp\-armt32\-fpe\-run} library file.
\flowgraph{\resource{\cpp{}\\source code} \ar[r] & \toolbox{cpparmt32fpe} \ar[r] \ar[d] \ar[rd] & \resource{object file} \\ \variable{ECSINCLUDE} \ar[ru] & \resource{debugging\\information} & \resource{assembly\\listing}}
\seecpp\seeassembly\seearm\seeobject\seedebugging
}

\providecommand{\cppavr}{
\toolsection{cppavr} is a compiler for the \cpp{} programming language targeting the AVR hardware architecture.
It generates machine code for AVR processors from programs written in \cpp{} and stores it in corresponding object files.
For debugging purposes, it also creates a debugging information file as well as an assembly file containing a listing of the generated machine code.
The macro \texttt{\_\_avr\_\_} is predefined in order to enable programmers to identify this tool and its target architecture while compiling.
Programs generated with this compiler require additional runtime support that is stored in the \file{cpp\-avr\-run} library file.
\flowgraph{\resource{\cpp{}\\source code} \ar[r] & \toolbox{cppavr} \ar[r] \ar[d] \ar[rd] & \resource{object file} \\ \variable{ECSINCLUDE} \ar[ru] & \resource{debugging\\information} & \resource{assembly\\listing}}
\seecpp\seeassembly\seeavr\seeobject\seedebugging
}

\providecommand{\cppavrtt}{
\toolsection{cppavr32} is a compiler for the \cpp{} programming language targeting the AVR32 hardware architecture.
It generates machine code for AVR32 processors from programs written in \cpp{} and stores it in corresponding object files.
For debugging purposes, it also creates a debugging information file as well as an assembly file containing a listing of the generated machine code.
The macro \texttt{\_\_avr32\_\_} is predefined in order to enable programmers to identify this tool and its target architecture while compiling.
Programs generated with this compiler require additional runtime support that is stored in the \file{cpp\-avr32\-run} library file.
\flowgraph{\resource{\cpp{}\\source code} \ar[r] & \toolbox{cppavr32} \ar[r] \ar[d] \ar[rd] & \resource{object file} \\ \variable{ECSINCLUDE} \ar[ru] & \resource{debugging\\information} & \resource{assembly\\listing}}
\seecpp\seeassembly\seeavrtt\seeobject\seedebugging
}

\providecommand{\cppmabk}{
\toolsection{cppm68k} is a compiler for the \cpp{} programming language targeting the M68000 hardware architecture.
It generates machine code for M68000 processors from programs written in \cpp{} and stores it in corresponding object files.
For debugging purposes, it also creates a debugging information file as well as an assembly file containing a listing of the generated machine code.
The macro \texttt{\_\_m68k\_\_} is predefined in order to enable programmers to identify this tool and its target architecture while compiling.
Programs generated with this compiler require additional runtime support that is stored in the \file{cpp\-m68k\-run} library file.
\flowgraph{\resource{\cpp{}\\source code} \ar[r] & \toolbox{cppm68k} \ar[r] \ar[d] \ar[rd] & \resource{object file} \\ \variable{ECSINCLUDE} \ar[ru] & \resource{debugging\\information} & \resource{assembly\\listing}}
\seecpp\seeassembly\seemabk\seeobject\seedebugging
}

\providecommand{\cppmibl}{
\toolsection{cppmibl} is a compiler for the \cpp{} programming language targeting the MicroBlaze hardware architecture.
It generates machine code for MicroBlaze processors from programs written in \cpp{} and stores it in corresponding object files.
For debugging purposes, it also creates a debugging information file as well as an assembly file containing a listing of the generated machine code.
The macro \texttt{\_\_mibl\_\_} is predefined in order to enable programmers to identify this tool and its target architecture while compiling.
Programs generated with this compiler require additional runtime support that is stored in the \file{cpp\-mibl\-run} library file.
\flowgraph{\resource{\cpp{}\\source code} \ar[r] & \toolbox{cppmibl} \ar[r] \ar[d] \ar[rd] & \resource{object file} \\ \variable{ECSINCLUDE} \ar[ru] & \resource{debugging\\information} & \resource{assembly\\listing}}
\seecpp\seeassembly\seemibl\seeobject\seedebugging
}

\providecommand{\cppmipsa}{
\toolsection{cppmips32} is a compiler for the \cpp{} programming language targeting the MIPS32 hardware architecture.
It generates machine code for MIPS32 processors from programs written in \cpp{} and stores it in corresponding object files.
For debugging purposes, it also creates a debugging information file as well as an assembly file containing a listing of the generated machine code.
The macro \texttt{\_\_mips32\_\_} is predefined in order to enable programmers to identify this tool and its target architecture while compiling.
Programs generated with this compiler require additional runtime support that is stored in the \file{cpp\-mips32\-run} library file.
\flowgraph{\resource{\cpp{}\\source code} \ar[r] & \toolbox{cppmips32} \ar[r] \ar[d] \ar[rd] & \resource{object file} \\ \variable{ECSINCLUDE} \ar[ru] & \resource{debugging\\information} & \resource{assembly\\listing}}
\seecpp\seeassembly\seemips\seeobject\seedebugging
}

\providecommand{\cppmipsb}{
\toolsection{cppmips64} is a compiler for the \cpp{} programming language targeting the MIPS64 hardware architecture.
It generates machine code for MIPS64 processors from programs written in \cpp{} and stores it in corresponding object files.
For debugging purposes, it also creates a debugging information file as well as an assembly file containing a listing of the generated machine code.
The macro \texttt{\_\_mips64\_\_} is predefined in order to enable programmers to identify this tool and its target architecture while compiling.
Programs generated with this compiler require additional runtime support that is stored in the \file{cpp\-mips64\-run} library file.
\flowgraph{\resource{\cpp{}\\source code} \ar[r] & \toolbox{cppmips64} \ar[r] \ar[d] \ar[rd] & \resource{object file} \\ \variable{ECSINCLUDE} \ar[ru] & \resource{debugging\\information} & \resource{assembly\\listing}}
\seecpp\seeassembly\seemips\seeobject\seedebugging
}

\providecommand{\cppmmix}{
\toolsection{cppmmix} is a compiler for the \cpp{} programming language targeting the MMIX hardware architecture.
It generates machine code for MMIX processors from programs written in \cpp{} and stores it in corresponding object files.
For debugging purposes, it also creates a debugging information file as well as an assembly file containing a listing of the generated machine code.
The macro \texttt{\_\_mmix\_\_} is predefined in order to enable programmers to identify this tool and its target architecture while compiling.
Programs generated with this compiler require additional runtime support that is stored in the \file{cpp\-mmix\-run} library file.
\flowgraph{\resource{\cpp{}\\source code} \ar[r] & \toolbox{cppmmix} \ar[r] \ar[d] \ar[rd] & \resource{object file} \\ \variable{ECSINCLUDE} \ar[ru] & \resource{debugging\\information} & \resource{assembly\\listing}}
\seecpp\seeassembly\seemmix\seeobject\seedebugging
}

\providecommand{\cpporok}{
\toolsection{cppor1k} is a compiler for the \cpp{} programming language targeting the OpenRISC 1000 hardware architecture.
It generates machine code for OpenRISC 1000 processors from programs written in \cpp{} and stores it in corresponding object files.
For debugging purposes, it also creates a debugging information file as well as an assembly file containing a listing of the generated machine code.
The macro \texttt{\_\_or1k\_\_} is predefined in order to enable programmers to identify this tool and its target architecture while compiling.
Programs generated with this compiler require additional runtime support that is stored in the \file{cpp\-or1k\-run} library file.
\flowgraph{\resource{\cpp{}\\source code} \ar[r] & \toolbox{cppor1k} \ar[r] \ar[d] \ar[rd] & \resource{object file} \\ \variable{ECSINCLUDE} \ar[ru] & \resource{debugging\\information} & \resource{assembly\\listing}}
\seecpp\seeassembly\seeorok\seeobject\seedebugging
}

\providecommand{\cppppca}{
\toolsection{cppppc32} is a compiler for the \cpp{} programming language targeting the PowerPC hardware architecture.
It generates machine code for PowerPC processors from programs written in \cpp{} and stores it in corresponding object files.
The compiler generates machine code for the 32-bit operating mode defined by the PowerPC architecture.
For debugging purposes, it also creates a debugging information file as well as an assembly file containing a listing of the generated machine code.
The macro \texttt{\_\_ppc32\_\_} is predefined in order to enable programmers to identify this tool and its target architecture while compiling.
Programs generated with this compiler require additional runtime support that is stored in the \file{cpp\-ppc32\-run} library file.
\flowgraph{\resource{\cpp{}\\source code} \ar[r] & \toolbox{cppppc32} \ar[r] \ar[d] \ar[rd] & \resource{object file} \\ \variable{ECSINCLUDE} \ar[ru] & \resource{debugging\\information} & \resource{assembly\\listing}}
\seecpp\seeassembly\seeppc\seeobject\seedebugging
}

\providecommand{\cppppcb}{
\toolsection{cppppc64} is a compiler for the \cpp{} programming language targeting the PowerPC hardware architecture.
It generates machine code for PowerPC processors from programs written in \cpp{} and stores it in corresponding object files.
The compiler generates machine code for the 64-bit operating mode defined by the PowerPC architecture.
For debugging purposes, it also creates a debugging information file as well as an assembly file containing a listing of the generated machine code.
The macro \texttt{\_\_ppc64\_\_} is predefined in order to enable programmers to identify this tool and its target architecture while compiling.
Programs generated with this compiler require additional runtime support that is stored in the \file{cpp\-ppc64\-run} library file.
\flowgraph{\resource{\cpp{}\\source code} \ar[r] & \toolbox{cppppc64} \ar[r] \ar[d] \ar[rd] & \resource{object file} \\ \variable{ECSINCLUDE} \ar[ru] & \resource{debugging\\information} & \resource{assembly\\listing}}
\seecpp\seeassembly\seeppc\seeobject\seedebugging
}

\providecommand{\cpprisc}{
\toolsection{cpprisc} is a compiler for the \cpp{} programming language targeting the RISC hardware architecture.
It generates machine code for RISC processors from programs written in \cpp{} and stores it in corresponding object files.
For debugging purposes, it also creates a debugging information file as well as an assembly file containing a listing of the generated machine code.
The macro \texttt{\_\_risc\_\_} is predefined in order to enable programmers to identify this tool and its target architecture while compiling.
Programs generated with this compiler require additional runtime support that is stored in the \file{cpp\-risc\-run} library file.
\flowgraph{\resource{\cpp{}\\source code} \ar[r] & \toolbox{cpprisc} \ar[r] \ar[d] \ar[rd] & \resource{object file} \\ \variable{ECSINCLUDE} \ar[ru] & \resource{debugging\\information} & \resource{assembly\\listing}}
\seecpp\seeassembly\seerisc\seeobject\seedebugging
}

\providecommand{\cppwasm}{
\toolsection{cppwasm} is a compiler for the \cpp{} programming language targeting the WebAssembly architecture.
It generates machine code for WebAssembly targets from programs written in \cpp{} and stores it in corresponding object files.
For debugging purposes, it also creates a debugging information file as well as an assembly file containing a listing of the generated machine code.
The macro \texttt{\_\_wasm\_\_} is predefined in order to enable programmers to identify this tool and its target architecture while compiling.
Programs generated with this compiler require additional runtime support that is stored in the \file{cpp\-wasm\-run} library file.
\flowgraph{\resource{\cpp{}\\source code} \ar[r] & \toolbox{cppwasm} \ar[r] \ar[d] \ar[rd] & \resource{object file} \\ \variable{ECSINCLUDE} \ar[ru] & \resource{debugging\\information} & \resource{assembly\\listing}}
\seecpp\seeassembly\seewasm\seeobject\seedebugging
}

% FALSE tools

\providecommand{\falprint}{
\toolsection{falprint} is a pretty printer for the FALSE programming language.
It reformats the source code of FALSE programs and writes it to the standard output stream.
\flowgraph{\resource{FALSE\\source code} \ar[r] & \toolbox{falprint} \ar[r] & \resource{reformatted\\source code}}
\seefalse
}

\providecommand{\falcheck}{
\toolsection{falcheck} is a syntactic and semantic checker for the FALSE programming language.
It just performs syntactic and semantic checks on FALSE programs and writes its diagnostic messages to the standard error stream.
\flowgraph{\resource{FALSE\\source code} \ar[r] & \toolbox{falcheck} \ar[r] & \resource{diagnostic\\messages}}
\seefalse
}

\providecommand{\faldump}{
\toolsection{faldump} is a serializer for the FALSE programming language.
It dumps the complete internal representation of programs written in FALSE into an XML document.
\debuggingtool
\flowgraph{\resource{FALSE\\source code} \ar[r] & \toolbox{faldump} \ar[r] & \resource{internal\\representation}}
\seefalse
}

\providecommand{\falrun}{
\toolsection{falrun} is an interpreter for the FALSE programming language.
It processes and executes programs written in FALSE\@.
\flowgraph{\resource{FALSE\\source code} \ar[r] & \toolbox{falrun} \ar@/u/[r] & \resource{input/\\output} \ar@/d/[l]}
\seefalse
}

\providecommand{\falcpp}{
\toolsection{falcpp} is a transpiler for the FALSE programming language.
It translates programs written in FALSE into \cpp{} programs and stores them in corresponding source files.
\flowgraph{\resource{FALSE\\source code} \ar[r] & \toolbox{falcpp} \ar[r] & \resource{\cpp{}\\source file}}
\seefalse\seecpp
}

\providecommand{\falcode}{
\toolsection{falcode} is an intermediate code generator for the FALSE programming language.
It generates intermediate code from programs written in FALSE and stores it in corresponding assembly files.
\debuggingtool
\flowgraph{\resource{FALSE\\source code} \ar[r] & \toolbox{falcode} \ar[r] & \resource{intermediate\\code}}
\seefalse\seeassembly\seecode
}

\providecommand{\falamda}{
\toolsection{falamd16} is a compiler for the FALSE programming language targeting the AMD64 hardware architecture.
It generates machine code for AMD64 processors from programs written in FALSE and stores it in corresponding object files.
The compiler generates machine code for the 16-bit operating mode defined by the AMD64 architecture.
\flowgraph{\resource{FALSE\\source code} \ar[r] & \toolbox{falamd16} \ar[r] & \resource{object file}}
\seefalse\seeamd\seeobject
}

\providecommand{\falamdb}{
\toolsection{falamd32} is a compiler for the FALSE programming language targeting the AMD64 hardware architecture.
It generates machine code for AMD64 processors from programs written in FALSE and stores it in corresponding object files.
The compiler generates machine code for the 32-bit operating mode defined by the AMD64 architecture.
\flowgraph{\resource{FALSE\\source code} \ar[r] & \toolbox{falamd32} \ar[r] & \resource{object file}}
\seefalse\seeamd\seeobject
}

\providecommand{\falamdc}{
\toolsection{falamd64} is a compiler for the FALSE programming language targeting the AMD64 hardware architecture.
It generates machine code for AMD64 processors from programs written in FALSE and stores it in corresponding object files.
The compiler generates machine code for the 64-bit operating mode defined by the AMD64 architecture.
\flowgraph{\resource{FALSE\\source code} \ar[r] & \toolbox{falamd64} \ar[r] & \resource{object file}}
\seefalse\seeamd\seeobject
}

\providecommand{\falarma}{
\toolsection{falarma32} is a compiler for the FALSE programming language targeting the ARM hardware architecture.
It generates machine code for ARM processors executing A32 instructions from programs written in FALSE and stores it in corresponding object files.
\flowgraph{\resource{FALSE\\source code} \ar[r] & \toolbox{falarma32} \ar[r] & \resource{object file}}
\seefalse\seearm\seeobject
}

\providecommand{\falarmb}{
\toolsection{falarma64} is a compiler for the FALSE programming language targeting the ARM hardware architecture.
It generates machine code for ARM processors executing A64 instructions from programs written in FALSE and stores it in corresponding object files.
\flowgraph{\resource{FALSE\\source code} \ar[r] & \toolbox{falarma64} \ar[r] & \resource{object file}}
\seefalse\seearm\seeobject
}

\providecommand{\falarmc}{
\toolsection{falarmt32} is a compiler for the FALSE programming language targeting the ARM hardware architecture.
It generates machine code for ARM processors without floating-point extension executing T32 instructions from programs written in FALSE and stores it in corresponding object files.
\flowgraph{\resource{FALSE\\source code} \ar[r] & \toolbox{falarmt32} \ar[r] & \resource{object file}}
\seefalse\seearm\seeobject
}

\providecommand{\falarmcfpe}{
\toolsection{falarmt32fpe} is a compiler for the FALSE programming language targeting the ARM hardware architecture.
It generates machine code for ARM processors with floating-point extension executing T32 instructions from programs written in FALSE and stores it in corresponding object files.
\flowgraph{\resource{FALSE\\source code} \ar[r] & \toolbox{falarmt32fpe} \ar[r] & \resource{object file}}
\seefalse\seearm\seeobject
}

\providecommand{\falavr}{
\toolsection{falavr} is a compiler for the FALSE programming language targeting the AVR hardware architecture.
It generates machine code for AVR processors from programs written in FALSE and stores it in corresponding object files.
\flowgraph{\resource{FALSE\\source code} \ar[r] & \toolbox{falavr} \ar[r] & \resource{object file}}
\seefalse\seeavr\seeobject
}

\providecommand{\falavrtt}{
\toolsection{falavr32} is a compiler for the FALSE programming language targeting the AVR32 hardware architecture.
It generates machine code for AVR32 processors from programs written in FALSE and stores it in corresponding object files.
\flowgraph{\resource{FALSE\\source code} \ar[r] & \toolbox{falavr32} \ar[r] & \resource{object file}}
\seefalse\seeavrtt\seeobject
}

\providecommand{\falmabk}{
\toolsection{falm68k} is a compiler for the FALSE programming language targeting the M68000 hardware architecture.
It generates machine code for M68000 processors from programs written in FALSE and stores it in corresponding object files.
\flowgraph{\resource{FALSE\\source code} \ar[r] & \toolbox{falm68k} \ar[r] & \resource{object file}}
\seefalse\seemabk\seeobject
}

\providecommand{\falmibl}{
\toolsection{falmibl} is a compiler for the FALSE programming language targeting the MicroBlaze hardware architecture.
It generates machine code for MicroBlaze processors from programs written in FALSE and stores it in corresponding object files.
\flowgraph{\resource{FALSE\\source code} \ar[r] & \toolbox{falmibl} \ar[r] & \resource{object file}}
\seefalse\seemibl\seeobject
}

\providecommand{\falmipsa}{
\toolsection{falmips32} is a compiler for the FALSE programming language targeting the MIPS32 hardware architecture.
It generates machine code for MIPS32 processors from programs written in FALSE and stores it in corresponding object files.
\flowgraph{\resource{FALSE\\source code} \ar[r] & \toolbox{falmips32} \ar[r] & \resource{object file}}
\seefalse\seemips\seeobject
}

\providecommand{\falmipsb}{
\toolsection{falmips64} is a compiler for the FALSE programming language targeting the MIPS64 hardware architecture.
It generates machine code for MIPS64 processors from programs written in FALSE and stores it in corresponding object files.
\flowgraph{\resource{FALSE\\source code} \ar[r] & \toolbox{falmips64} \ar[r] & \resource{object file}}
\seefalse\seemips\seeobject
}

\providecommand{\falmmix}{
\toolsection{falmmix} is a compiler for the FALSE programming language targeting the MMIX hardware architecture.
It generates machine code for MMIX processors from programs written in FALSE and stores it in corresponding object files.
\flowgraph{\resource{FALSE\\source code} \ar[r] & \toolbox{falmmix} \ar[r] & \resource{object file}}
\seefalse\seemmix\seeobject
}

\providecommand{\falorok}{
\toolsection{falor1k} is a compiler for the FALSE programming language targeting the OpenRISC 1000 hardware architecture.
It generates machine code for OpenRISC 1000 processors from programs written in FALSE and stores it in corresponding object files.
\flowgraph{\resource{FALSE\\source code} \ar[r] & \toolbox{falor1k} \ar[r] & \resource{object file}}
\seefalse\seeorok\seeobject
}

\providecommand{\falppca}{
\toolsection{falppc32} is a compiler for the FALSE programming language targeting the PowerPC hardware architecture.
It generates machine code for PowerPC processors from programs written in FALSE and stores it in corresponding object files.
The compiler generates machine code for the 32-bit operating mode defined by the PowerPC architecture.
\flowgraph{\resource{FALSE\\source code} \ar[r] & \toolbox{falppc32} \ar[r] & \resource{object file}}
\seefalse\seeppc\seeobject
}

\providecommand{\falppcb}{
\toolsection{falppc64} is a compiler for the FALSE programming language targeting the PowerPC hardware architecture.
It generates machine code for PowerPC processors from programs written in FALSE and stores it in corresponding object files.
The compiler generates machine code for the 64-bit operating mode defined by the PowerPC architecture.
\flowgraph{\resource{FALSE\\source code} \ar[r] & \toolbox{falppc64} \ar[r] & \resource{object file}}
\seefalse\seeppc\seeobject
}

\providecommand{\falrisc}{
\toolsection{falrisc} is a compiler for the FALSE programming language targeting the RISC hardware architecture.
It generates machine code for RISC processors from programs written in FALSE and stores it in corresponding object files.
\flowgraph{\resource{FALSE\\source code} \ar[r] & \toolbox{falrisc} \ar[r] & \resource{object file}}
\seefalse\seerisc\seeobject
}

\providecommand{\falwasm}{
\toolsection{falwasm} is a compiler for the FALSE programming language targeting the WebAssembly architecture.
It generates machine code for WebAssembly targets from programs written in FALSE and stores it in corresponding object files.
\flowgraph{\resource{FALSE\\source code} \ar[r] & \toolbox{falwasm} \ar[r] & \resource{object file}}
\seefalse\seewasm\seeobject
}

% Oberon tools

\providecommand{\obprint}{
\toolsection{obprint} is a pretty printer for the Oberon programming language.
It reformats the source code of Oberon modules and writes it to the standard output stream.
\flowgraph{\resource{Oberon\\source code} \ar[r] & \toolbox{obprint} \ar[r] & \resource{reformatted\\source code}}
\seeoberon
}

\providecommand{\obcheck}{
\toolsection{obcheck} is a syntactic and semantic checker for the Oberon programming language.
It just performs syntactic and semantic checks on Oberon modules and writes its diagnostic messages to the standard error stream.
In addition, it stores the interface of each module in a symbol file which is required when other modules import the module.
\flowgraph{\resource{Oberon\\source code} \ar[r] & \toolbox{obcheck} \ar[r] \ar@/l/[d] & \resource{diagnostic\\messages} \\ \variable{ECSIMPORT} \ar[ru] & \resource{symbol\\files} \ar@/r/[u]}
\seeoberon
}

\providecommand{\obdump}{
\toolsection{obdump} is a serializer for the Oberon programming language.
It dumps the complete internal representation of modules written in Oberon into an XML document.
\debuggingtool
\flowgraph{\resource{Oberon\\source code} \ar[r] & \toolbox{obdump} \ar[r] \ar@/l/[d] & \resource{internal\\representation} \\ \variable{ECSIMPORT} \ar[ru] & \resource{symbol\\files} \ar@/r/[u]}
\seeoberon
}

\providecommand{\obrun}{
\toolsection{obrun} is an interpreter for the Oberon programming language.
It processes and executes modules written in Oberon.
This tool does neither generate nor process symbol files while interpreting modules.
If a module is imported by another one, its filename has to be named before the other one in the list of command-line arguments.
\flowgraph{\resource{Oberon\\source code} \ar[r] & \toolbox{obrun} \ar@/u/[r] & \resource{input/\\output} \ar@/d/[l]}
\seeoberon
}

\providecommand{\obcpp}{
\toolsection{obcpp} is a transpiler for the Oberon programming language.
It translates programs written in Oberon into \cpp{} programs and stores them in corresponding source and header files.
In addition, it stores the interface of each module in a symbol file which is required when other modules import the module.
The same interface is provided by the generated header file which can be used in other parts of the \cpp{} program.
\flowgraph{\resource{Oberon\\source code} \ar[r] & \toolbox{obcpp} \ar[r] \ar@/l/[d] \ar[rd] & \resource{\cpp{}\\source file} \\ \variable{ECSIMPORT} \ar[ru] & \resource{symbol\\files} \ar@/r/[u] & \resource{\cpp{}\\header file}}
\seeoberon\seecpp
}

\providecommand{\obdoc}{
\toolsection{obdoc} is a generic documentation generator for the Oberon programming language.
It processes several Oberon modules and assembles all information therein into a generic documentation.
In addition, it stores the interface of each module in a symbol file which is required when other modules import the module.
\debuggingtool
\flowgraph{\resource{Oberon\\source code} \ar[r] & \toolbox{obdoc} \ar[r] \ar@/l/[d] & \resource{generic\\documentation} \\ \variable{ECSIMPORT} \ar[ru] & \resource{symbol\\files} \ar@/r/[u]}
\seeoberon\seedocumentation
}

\providecommand{\obhtml}{
\toolsection{obhtml} is an HTML documentation generator for the Oberon programming language.
It processes several Oberon modules and assembles all information therein into an HTML document.
In addition, it stores the interface of each module in a symbol file which is required when other modules import the module.
\flowgraph{\resource{Oberon\\source code} \ar[r] & \toolbox{obhtml} \ar[r] \ar@/l/[d] & \resource{HTML\\document} \\ \variable{ECSIMPORT} \ar[ru] & \resource{symbol\\files} \ar@/r/[u]}
\seeoberon\seedocumentation
}

\providecommand{\oblatex}{
\toolsection{oblatex} is a Latex documentation generator for the Oberon programming language.
It processes several Oberon modules and assembles all information therein into a Latex document.
In addition, it stores the interface of each module in a symbol file which is required when other modules import the module.
\flowgraph{\resource{Oberon\\source code} \ar[r] & \toolbox{oblatex} \ar[r] \ar@/l/[d] & \resource{Latex\\document} \\ \variable{ECSIMPORT} \ar[ru] & \resource{symbol\\files} \ar@/r/[u]}
\seeoberon\seedocumentation
}

\providecommand{\obcode}{
\toolsection{obcode} is an intermediate code generator for the Oberon programming language.
It generates intermediate code from modules written in Oberon and stores it in corresponding assembly files.
In addition, it stores the interface of each module in a symbol file which is required when other modules import the module.
Programs generated with this tool require additional runtime support that is stored in the \file{ob\-code\-run} library file.
\debuggingtool
\flowgraph{\resource{Oberon\\source code} \ar[r] & \toolbox{obcode} \ar[r] \ar@/l/[d] & \resource{intermediate\\code} \\ \variable{ECSIMPORT} \ar[ru] & \resource{symbol\\files} \ar@/r/[u]}
\seeoberon\seeassembly\seecode
}

\providecommand{\obamda}{
\toolsection{obamd16} is a compiler for the Oberon programming language targeting the AMD64 hardware architecture.
It generates machine code for AMD64 processors from modules written in Oberon and stores it in corresponding object files.
The compiler generates machine code for the 16-bit operating mode defined by the AMD64 architecture.
For debugging purposes, it also creates a debugging information file as well as an assembly file containing a listing of the generated machine code.
In addition, it stores the interface of each module in a symbol file which is required when other modules import the module.
Programs generated with this compiler require additional runtime support that is stored in the \file{ob\-amd16\-run} library file.
\flowgraph{\resource{Oberon\\source code} \ar[r] & \toolbox{obamd16} \ar[r] \ar@/l/[d] \ar[rd] & \resource{object file} \\ \variable{ECSIMPORT} \ar[ru] & \resource{symbol\\files} \ar@/r/[u] & \resource{debugging\\information}}
\seeoberon\seeassembly\seeamd\seeobject\seedebugging
}

\providecommand{\obamdb}{
\toolsection{obamd32} is a compiler for the Oberon programming language targeting the AMD64 hardware architecture.
It generates machine code for AMD64 processors from modules written in Oberon and stores it in corresponding object files.
The compiler generates machine code for the 32-bit operating mode defined by the AMD64 architecture.
For debugging purposes, it also creates a debugging information file as well as an assembly file containing a listing of the generated machine code.
In addition, it stores the interface of each module in a symbol file which is required when other modules import the module.
Programs generated with this compiler require additional runtime support that is stored in the \file{ob\-amd32\-run} library file.
\flowgraph{\resource{Oberon\\source code} \ar[r] & \toolbox{obamd32} \ar[r] \ar@/l/[d] \ar[rd] & \resource{object file} \\ \variable{ECSIMPORT} \ar[ru] & \resource{symbol\\files} \ar@/r/[u] & \resource{debugging\\information}}
\seeoberon\seeassembly\seeamd\seeobject\seedebugging
}

\providecommand{\obamdc}{
\toolsection{obamd64} is a compiler for the Oberon programming language targeting the AMD64 hardware architecture.
It generates machine code for AMD64 processors from modules written in Oberon and stores it in corresponding object files.
The compiler generates machine code for the 64-bit operating mode defined by the AMD64 architecture.
For debugging purposes, it also creates a debugging information file as well as an assembly file containing a listing of the generated machine code.
In addition, it stores the interface of each module in a symbol file which is required when other modules import the module.
Programs generated with this compiler require additional runtime support that is stored in the \file{ob\-amd64\-run} library file.
\flowgraph{\resource{Oberon\\source code} \ar[r] & \toolbox{obamd64} \ar[r] \ar@/l/[d] \ar[rd] & \resource{object file} \\ \variable{ECSIMPORT} \ar[ru] & \resource{symbol\\files} \ar@/r/[u] & \resource{debugging\\information}}
\seeoberon\seeassembly\seeamd\seeobject\seedebugging
}

\providecommand{\obarma}{
\toolsection{obarma32} is a compiler for the Oberon programming language targeting the ARM hardware architecture.
It generates machine code for ARM processors executing A32 instructions from modules written in Oberon and stores it in corresponding object files.
For debugging purposes, it also creates a debugging information file as well as an assembly file containing a listing of the generated machine code.
In addition, it stores the interface of each module in a symbol file which is required when other modules import the module.
Programs generated with this compiler require additional runtime support that is stored in the \file{ob\-arma32\-run} library file.
\flowgraph{\resource{Oberon\\source code} \ar[r] & \toolbox{obarma32} \ar[r] \ar@/l/[d] \ar[rd] & \resource{object file} \\ \variable{ECSIMPORT} \ar[ru] & \resource{symbol\\files} \ar@/r/[u] & \resource{debugging\\information}}
\seeoberon\seeassembly\seearm\seeobject\seedebugging
}

\providecommand{\obarmb}{
\toolsection{obarma64} is a compiler for the Oberon programming language targeting the ARM hardware architecture.
It generates machine code for ARM processors executing A64 instructions from modules written in Oberon and stores it in corresponding object files.
For debugging purposes, it also creates a debugging information file as well as an assembly file containing a listing of the generated machine code.
In addition, it stores the interface of each module in a symbol file which is required when other modules import the module.
Programs generated with this compiler require additional runtime support that is stored in the \file{ob\-arma64\-run} library file.
\flowgraph{\resource{Oberon\\source code} \ar[r] & \toolbox{obarma64} \ar[r] \ar@/l/[d] \ar[rd] & \resource{object file} \\ \variable{ECSIMPORT} \ar[ru] & \resource{symbol\\files} \ar@/r/[u] & \resource{debugging\\information}}
\seeoberon\seeassembly\seearm\seeobject\seedebugging
}

\providecommand{\obarmc}{
\toolsection{obarmt32} is a compiler for the Oberon programming language targeting the ARM hardware architecture.
It generates machine code for ARM processors without floating-point extension executing T32 instructions from modules written in Oberon and stores it in corresponding object files.
For debugging purposes, it also creates a debugging information file as well as an assembly file containing a listing of the generated machine code.
In addition, it stores the interface of each module in a symbol file which is required when other modules import the module.
Programs generated with this compiler require additional runtime support that is stored in the \file{ob\-armt32\-run} library file.
\flowgraph{\resource{Oberon\\source code} \ar[r] & \toolbox{obarmt32} \ar[r] \ar@/l/[d] \ar[rd] & \resource{object file} \\ \variable{ECSIMPORT} \ar[ru] & \resource{symbol\\files} \ar@/r/[u] & \resource{debugging\\information}}
\seeoberon\seeassembly\seearm\seeobject\seedebugging
}

\providecommand{\obarmcfpe}{
\toolsection{obarmt32fpe} is a compiler for the Oberon programming language targeting the ARM hardware architecture.
It generates machine code for ARM processors with floating-point extension executing T32 instructions from modules written in Oberon and stores it in corresponding object files.
For debugging purposes, it also creates a debugging information file as well as an assembly file containing a listing of the generated machine code.
In addition, it stores the interface of each module in a symbol file which is required when other modules import the module.
Programs generated with this compiler require additional runtime support that is stored in the \file{ob\-armt32\-fpe\-run} library file.
\flowgraph{\resource{Oberon\\source code} \ar[r] & \toolbox{obarmt32fpe} \ar[r] \ar@/l/[d] \ar[rd] & \resource{object file} \\ \variable{ECSIMPORT} \ar[ru] & \resource{symbol\\files} \ar@/r/[u] & \resource{debugging\\information}}
\seeoberon\seeassembly\seearm\seeobject\seedebugging
}

\providecommand{\obavr}{
\toolsection{obavr} is a compiler for the Oberon programming language targeting the AVR hardware architecture.
It generates machine code for AVR processors from modules written in Oberon and stores it in corresponding object files.
For debugging purposes, it also creates a debugging information file as well as an assembly file containing a listing of the generated machine code.
In addition, it stores the interface of each module in a symbol file which is required when other modules import the module.
Programs generated with this compiler require additional runtime support that is stored in the \file{ob\-avr\-run} library file.
\flowgraph{\resource{Oberon\\source code} \ar[r] & \toolbox{obavr} \ar[r] \ar@/l/[d] \ar[rd] & \resource{object file} \\ \variable{ECSIMPORT} \ar[ru] & \resource{symbol\\files} \ar@/r/[u] & \resource{debugging\\information}}
\seeoberon\seeassembly\seeavr\seeobject\seedebugging
}

\providecommand{\obavrtt}{
\toolsection{obavr32} is a compiler for the Oberon programming language targeting the AVR32 hardware architecture.
It generates machine code for AVR32 processors from modules written in Oberon and stores it in corresponding object files.
For debugging purposes, it also creates a debugging information file as well as an assembly file containing a listing of the generated machine code.
In addition, it stores the interface of each module in a symbol file which is required when other modules import the module.
Programs generated with this compiler require additional runtime support that is stored in the \file{ob\-avr32\-run} library file.
\flowgraph{\resource{Oberon\\source code} \ar[r] & \toolbox{obavr32} \ar[r] \ar@/l/[d] \ar[rd] & \resource{object file} \\ \variable{ECSIMPORT} \ar[ru] & \resource{symbol\\files} \ar@/r/[u] & \resource{debugging\\information}}
\seeoberon\seeassembly\seeavrtt\seeobject\seedebugging
}

\providecommand{\obmabk}{
\toolsection{obm68k} is a compiler for the Oberon programming language targeting the M68000 hardware architecture.
It generates machine code for M68000 processors from modules written in Oberon and stores it in corresponding object files.
For debugging purposes, it also creates a debugging information file as well as an assembly file containing a listing of the generated machine code.
In addition, it stores the interface of each module in a symbol file which is required when other modules import the module.
Programs generated with this compiler require additional runtime support that is stored in the \file{ob\-m68k\-run} library file.
\flowgraph{\resource{Oberon\\source code} \ar[r] & \toolbox{obm68k} \ar[r] \ar@/l/[d] \ar[rd] & \resource{object file} \\ \variable{ECSIMPORT} \ar[ru] & \resource{symbol\\files} \ar@/r/[u] & \resource{debugging\\information}}
\seeoberon\seeassembly\seemabk\seeobject\seedebugging
}

\providecommand{\obmibl}{
\toolsection{obmibl} is a compiler for the Oberon programming language targeting the MicroBlaze hardware architecture.
It generates machine code for MicroBlaze processors from modules written in Oberon and stores it in corresponding object files.
For debugging purposes, it also creates a debugging information file as well as an assembly file containing a listing of the generated machine code.
In addition, it stores the interface of each module in a symbol file which is required when other modules import the module.
Programs generated with this compiler require additional runtime support that is stored in the \file{ob\-mibl\-run} library file.
\flowgraph{\resource{Oberon\\source code} \ar[r] & \toolbox{obmibl} \ar[r] \ar@/l/[d] \ar[rd] & \resource{object file} \\ \variable{ECSIMPORT} \ar[ru] & \resource{symbol\\files} \ar@/r/[u] & \resource{debugging\\information}}
\seeoberon\seeassembly\seemibl\seeobject\seedebugging
}

\providecommand{\obmipsa}{
\toolsection{obmips32} is a compiler for the Oberon programming language targeting the MIPS32 hardware architecture.
It generates machine code for MIPS32 processors from modules written in Oberon and stores it in corresponding object files.
For debugging purposes, it also creates a debugging information file as well as an assembly file containing a listing of the generated machine code.
In addition, it stores the interface of each module in a symbol file which is required when other modules import the module.
Programs generated with this compiler require additional runtime support that is stored in the \file{ob\-mips32\-run} library file.
\flowgraph{\resource{Oberon\\source code} \ar[r] & \toolbox{obmips32} \ar[r] \ar@/l/[d] \ar[rd] & \resource{object file} \\ \variable{ECSIMPORT} \ar[ru] & \resource{symbol\\files} \ar@/r/[u] & \resource{debugging\\information}}
\seeoberon\seeassembly\seemips\seeobject\seedebugging
}

\providecommand{\obmipsb}{
\toolsection{obmips64} is a compiler for the Oberon programming language targeting the MIPS64 hardware architecture.
It generates machine code for MIPS64 processors from modules written in Oberon and stores it in corresponding object files.
For debugging purposes, it also creates a debugging information file as well as an assembly file containing a listing of the generated machine code.
In addition, it stores the interface of each module in a symbol file which is required when other modules import the module.
Programs generated with this compiler require additional runtime support that is stored in the \file{ob\-mips64\-run} library file.
\flowgraph{\resource{Oberon\\source code} \ar[r] & \toolbox{obmips64} \ar[r] \ar@/l/[d] \ar[rd] & \resource{object file} \\ \variable{ECSIMPORT} \ar[ru] & \resource{symbol\\files} \ar@/r/[u] & \resource{debugging\\information}}
\seeoberon\seeassembly\seemips\seeobject\seedebugging
}

\providecommand{\obmmix}{
\toolsection{obmmix} is a compiler for the Oberon programming language targeting the MMIX hardware architecture.
It generates machine code for MMIX processors from modules written in Oberon and stores it in corresponding object files.
For debugging purposes, it also creates a debugging information file as well as an assembly file containing a listing of the generated machine code.
In addition, it stores the interface of each module in a symbol file which is required when other modules import the module.
Programs generated with this compiler require additional runtime support that is stored in the \file{ob\-mmix\-run} library file.
\flowgraph{\resource{Oberon\\source code} \ar[r] & \toolbox{obmmix} \ar[r] \ar@/l/[d] \ar[rd] & \resource{object file} \\ \variable{ECSIMPORT} \ar[ru] & \resource{symbol\\files} \ar@/r/[u] & \resource{debugging\\information}}
\seeoberon\seeassembly\seemmix\seeobject\seedebugging
}

\providecommand{\oborok}{
\toolsection{obor1k} is a compiler for the Oberon programming language targeting the OpenRISC 1000 hardware architecture.
It generates machine code for OpenRISC 1000 processors from modules written in Oberon and stores it in corresponding object files.
For debugging purposes, it also creates a debugging information file as well as an assembly file containing a listing of the generated machine code.
In addition, it stores the interface of each module in a symbol file which is required when other modules import the module.
Programs generated with this compiler require additional runtime support that is stored in the \file{ob\-or1k\-run} library file.
\flowgraph{\resource{Oberon\\source code} \ar[r] & \toolbox{obor1k} \ar[r] \ar@/l/[d] \ar[rd] & \resource{object file} \\ \variable{ECSIMPORT} \ar[ru] & \resource{symbol\\files} \ar@/r/[u] & \resource{debugging\\information}}
\seeoberon\seeassembly\seeorok\seeobject\seedebugging
}

\providecommand{\obppca}{
\toolsection{obppc32} is a compiler for the Oberon programming language targeting the PowerPC hardware architecture.
It generates machine code for PowerPC processors from modules written in Oberon and stores it in corresponding object files.
The compiler generates machine code for the 32-bit operating mode defined by the PowerPC architecture.
For debugging purposes, it also creates a debugging information file as well as an assembly file containing a listing of the generated machine code.
In addition, it stores the interface of each module in a symbol file which is required when other modules import the module.
Programs generated with this compiler require additional runtime support that is stored in the \file{ob\-ppc32\-run} library file.
\flowgraph{\resource{Oberon\\source code} \ar[r] & \toolbox{obppc32} \ar[r] \ar@/l/[d] \ar[rd] & \resource{object file} \\ \variable{ECSIMPORT} \ar[ru] & \resource{symbol\\files} \ar@/r/[u] & \resource{debugging\\information}}
\seeoberon\seeassembly\seeppc\seeobject\seedebugging
}

\providecommand{\obppcb}{
\toolsection{obppc64} is a compiler for the Oberon programming language targeting the PowerPC hardware architecture.
It generates machine code for PowerPC processors from modules written in Oberon and stores it in corresponding object files.
The compiler generates machine code for the 64-bit operating mode defined by the PowerPC architecture.
For debugging purposes, it also creates a debugging information file as well as an assembly file containing a listing of the generated machine code.
In addition, it stores the interface of each module in a symbol file which is required when other modules import the module.
Programs generated with this compiler require additional runtime support that is stored in the \file{ob\-ppc64\-run} library file.
\flowgraph{\resource{Oberon\\source code} \ar[r] & \toolbox{obppc64} \ar[r] \ar@/l/[d] \ar[rd] & \resource{object file} \\ \variable{ECSIMPORT} \ar[ru] & \resource{symbol\\files} \ar@/r/[u] & \resource{debugging\\information}}
\seeoberon\seeassembly\seeppc\seeobject\seedebugging
}

\providecommand{\obrisc}{
\toolsection{obrisc} is a compiler for the Oberon programming language targeting the RISC hardware architecture.
It generates machine code for RISC processors from modules written in Oberon and stores it in corresponding object files.
For debugging purposes, it also creates a debugging information file as well as an assembly file containing a listing of the generated machine code.
In addition, it stores the interface of each module in a symbol file which is required when other modules import the module.
Programs generated with this compiler require additional runtime support that is stored in the \file{ob\-risc\-run} library file.
\flowgraph{\resource{Oberon\\source code} \ar[r] & \toolbox{obrisc} \ar[r] \ar@/l/[d] \ar[rd] & \resource{object file} \\ \variable{ECSIMPORT} \ar[ru] & \resource{symbol\\files} \ar@/r/[u] & \resource{debugging\\information}}
\seeoberon\seeassembly\seerisc\seeobject\seedebugging
}

\providecommand{\obwasm}{
\toolsection{obwasm} is a compiler for the Oberon programming language targeting the WebAssembly architecture.
It generates machine code for WebAssembly targets from modules written in Oberon and stores it in corresponding object files.
For debugging purposes, it also creates a debugging information file as well as an assembly file containing a listing of the generated machine code.
In addition, it stores the interface of each module in a symbol file which is required when other modules import the module.
Programs generated with this compiler require additional runtime support that is stored in the \file{ob\-wasm\-run} library file.
\flowgraph{\resource{Oberon\\source code} \ar[r] & \toolbox{obwasm} \ar[r] \ar@/l/[d] \ar[rd] & \resource{object file} \\ \variable{ECSIMPORT} \ar[ru] & \resource{symbol\\files} \ar@/r/[u] & \resource{debugging\\information}}
\seeoberon\seeassembly\seewasm\seeobject\seedebugging
}

% converter tools

\providecommand{\dbgdwarf}{
\toolsection{dbgdwarf} is a DWARF debugging information converter tool.
It converts debugging information into the DWARF debugging data format and stores it in corresponding object files~\cite{dwarffile}.
The resulting debugging object files can be combined with runtime support that creates Executable and Linking Format (ELF) files~\cite{elffile}.
\flowgraph{\resource{debugging\\information} \ar[r] & \toolbox{dbgdwarf} \ar[r] & \resource{debugging\\object file}}
\seeobject\seedebugging
}

% assembler tools

\providecommand{\asmprint}{
\toolsection{asmprint} is a pretty printer for generic assembly code.
It reformats generic assembly code and writes it to the standard output stream.
\flowgraph{\resource{generic assembly\\source code} \ar[r] & \toolbox{asmprint} \ar[r] & \resource{reformatted\\source code}}
\seeassembly
}

\providecommand{\amdaasm}{
\toolsection{amd16asm} is an assembler for the AMD64 hardware architecture.
It translates assembly code into machine code for AMD64 processors and stores it in corresponding object files.
By default, the assembler generates machine code for the 16-bit operating mode defined by the AMD64 architecture.
\flowgraph{\resource{AMD16 assembly\\source code} \ar[r] & \toolbox{amd16asm} \ar[r] & \resource{object file}}
\seeassembly\seeamd\seeobject
}

\providecommand{\amdadism}{
\toolsection{amd16dism} is a disassembler for the AMD64 hardware architecture.
It translates machine code from object files targeting AMD64 processors into assembly code and writes it to the standard output stream.
It assumes that the machine code was generated for the 16-bit operating mode defined by the AMD64 architecture.
\flowgraph{\resource{object file} \ar[r] & \toolbox{amd16dism} \ar[r] & \resource{disassembly\\listing}}
\seeassembly\seeamd\seeobject
}

\providecommand{\amdbasm}{
\toolsection{amd32asm} is an assembler for the AMD64 hardware architecture.
It translates assembly code into machine code for AMD64 processors and stores it in corresponding object files.
By default, the assembler generates machine code for the 32-bit operating mode defined by the AMD64 architecture.
\flowgraph{\resource{AMD32 assembly\\source code} \ar[r] & \toolbox{amd32asm} \ar[r] & \resource{object file}}
\seeassembly\seeamd\seeobject
}

\providecommand{\amdbdism}{
\toolsection{amd32dism} is a disassembler for the AMD64 hardware architecture.
It translates machine code from object files targeting AMD64 processors into assembly code and writes it to the standard output stream.
It assumes that the machine code was generated for the 32-bit operating mode defined by the AMD64 architecture.
\flowgraph{\resource{object file} \ar[r] & \toolbox{amd32dism} \ar[r] & \resource{disassembly\\listing}}
\seeassembly\seeamd\seeobject
}

\providecommand{\amdcasm}{
\toolsection{amd64asm} is an assembler for the AMD64 hardware architecture.
It translates assembly code into machine code for AMD64 processors and stores it in corresponding object files.
By default, the assembler generates machine code for the 64-bit operating mode defined by the AMD64 architecture.
\flowgraph{\resource{AMD64 assembly\\source code} \ar[r] & \toolbox{amd64asm} \ar[r] & \resource{object file}}
\seeassembly\seeamd\seeobject
}

\providecommand{\amdcdism}{
\toolsection{amd64dism} is a disassembler for the AMD64 hardware architecture.
It translates machine code from object files targeting AMD64 processors into assembly code and writes it to the standard output stream.
It assumes that the machine code was generated for the 64-bit operating mode defined by the AMD64 architecture.
\flowgraph{\resource{object file} \ar[r] & \toolbox{amd64dism} \ar[r] & \resource{disassembly\\listing}}
\seeassembly\seeamd\seeobject
}

\providecommand{\armaasm}{
\toolsection{arma32asm} is an assembler for the ARM hardware architecture.
It translates assembly code into machine code for ARM processors executing A32 instructions and stores it in corresponding object files.
\flowgraph{\resource{ARM A32 assembly\\source code} \ar[r] & \toolbox{arma32asm} \ar[r] & \resource{object file}}
\seeassembly\seearm\seeobject
}

\providecommand{\armadism}{
\toolsection{arma32dism} is a disassembler for the ARM hardware architecture.
It translates machine code from object files targeting ARM processors executing A32 instructions into assembly code and writes it to the standard output stream.
\flowgraph{\resource{object file} \ar[r] & \toolbox{arma32dism} \ar[r] & \resource{disassembly\\listing}}
\seeassembly\seearm\seeobject
}

\providecommand{\armbasm}{
\toolsection{arma64asm} is an assembler for the ARM hardware architecture.
It translates assembly code into machine code for ARM processors executing A64 instructions and stores it in corresponding object files.
\flowgraph{\resource{ARM A64 assembly\\source code} \ar[r] & \toolbox{arma64asm} \ar[r] & \resource{object file}}
\seeassembly\seearm\seeobject
}

\providecommand{\armbdism}{
\toolsection{arma64dism} is a disassembler for the ARM hardware architecture.
It translates machine code from object files targeting ARM processors executing A64 instructions into assembly code and writes it to the standard output stream.
\flowgraph{\resource{object file} \ar[r] & \toolbox{arma64dism} \ar[r] & \resource{disassembly\\listing}}
\seeassembly\seearm\seeobject
}

\providecommand{\armcasm}{
\toolsection{armt32asm} is an assembler for the ARM hardware architecture.
It translates assembly code into machine code for ARM processors executing T32 instructions and stores it in corresponding object files.
\flowgraph{\resource{ARM T32 assembly\\source code} \ar[r] & \toolbox{armt32asm} \ar[r] & \resource{object file}}
\seeassembly\seearm\seeobject
}

\providecommand{\armcdism}{
\toolsection{armt32dism} is a disassembler for the ARM hardware architecture.
It translates machine code from object files targeting ARM processors executing T32 instructions into assembly code and writes it to the standard output stream.
\flowgraph{\resource{object file} \ar[r] & \toolbox{armt32dism} \ar[r] & \resource{disassembly\\listing}}
\seeassembly\seearm\seeobject
}

\providecommand{\avrasm}{
\toolsection{avrasm} is an assembler for the AVR hardware architecture.
It translates assembly code into machine code for AVR processors and stores it in corresponding object files.
The identifiers \texttt{RXL}, \texttt{RXH}, \texttt{RYL}, \texttt{RYH}, \texttt{RZL}, and \texttt{RZH} are predefined and name the corresponding registers.
The identifiers \texttt{SPL} and \texttt{SPH} are also predefined and evaluate to the address of the corresponding registers.
\flowgraph{\resource{AVR assembly\\source code} \ar[r] & \toolbox{avrasm} \ar[r] & \resource{object file}}
\seeassembly\seeavr\seeobject
}

\providecommand{\avrdism}{
\toolsection{avrdism} is a disassembler for the AVR hardware architecture.
It translates machine code from object files targeting AVR processors into assembly code and writes it to the standard output stream.
\flowgraph{\resource{object file} \ar[r] & \toolbox{avrdism} \ar[r] & \resource{disassembly\\listing}}
\seeassembly\seeavr\seeobject
}

\providecommand{\avrttasm}{
\toolsection{avr32asm} is an assembler for the AVR32 hardware architecture.
It translates assembly code into machine code for AVR32 processors and stores it in corresponding object files.
\flowgraph{\resource{AVR32 assembly\\source code} \ar[r] & \toolbox{avr32asm} \ar[r] & \resource{object file}}
\seeassembly\seeavrtt\seeobject
}

\providecommand{\avrttdism}{
\toolsection{avr32dism} is a disassembler for the AVR32 hardware architecture.
It translates machine code from object files targeting AVR32 processors into assembly code and writes it to the standard output stream.
\flowgraph{\resource{object file} \ar[r] & \toolbox{avr32dism} \ar[r] & \resource{disassembly\\listing}}
\seeassembly\seeavrtt\seeobject
}

\providecommand{\mabkasm}{
\toolsection{m68kasm} is an assembler for the M68000 hardware architecture.
It translates assembly code into machine code for M68000 processors and stores it in corresponding object files.
\flowgraph{\resource{68000 assembly\\source code} \ar[r] & \toolbox{m68kasm} \ar[r] & \resource{object file}}
\seeassembly\seemabk\seeobject
}

\providecommand{\mabkdism}{
\toolsection{m68kdism} is a disassembler for the M68000 hardware architecture.
It translates machine code from object files targeting M68000 processors into assembly code and writes it to the standard output stream.
\flowgraph{\resource{object file} \ar[r] & \toolbox{m68kdism} \ar[r] & \resource{disassembly\\listing}}
\seeassembly\seemabk\seeobject
}

\providecommand{\miblasm}{
\toolsection{miblasm} is an assembler for the MicroBlaze hardware architecture.
It translates assembly code into machine code for MicroBlaze processors and stores it in corresponding object files.
\flowgraph{\resource{MicroBlaze assembly\\source code} \ar[r] & \toolbox{miblasm} \ar[r] & \resource{object file}}
\seeassembly\seemibl\seeobject
}

\providecommand{\mibldism}{
\toolsection{mibldism} is a disassembler for the MicroBlaze hardware architecture.
It translates machine code from object files targeting MicroBlaze processors into assembly code and writes it to the standard output stream.
\flowgraph{\resource{object file} \ar[r] & \toolbox{mibldism} \ar[r] & \resource{disassembly\\listing}}
\seeassembly\seemibl\seeobject
}

\providecommand{\mipsaasm}{
\toolsection{mips32asm} is an assembler for the MIPS32 hardware architecture.
It translates assembly code into machine code for MIPS32 processors and stores it in corresponding object files.
\flowgraph{\resource{MIPS32 assembly\\source code} \ar[r] & \toolbox{mips32asm} \ar[r] & \resource{object file}}
\seeassembly\seemips\seeobject
}

\providecommand{\mipsadism}{
\toolsection{mips32dism} is a disassembler for the MIPS32 hardware architecture.
It translates machine code from object files targeting MIPS32 processors into assembly code and writes it to the standard output stream.
\flowgraph{\resource{object file} \ar[r] & \toolbox{mips32dism} \ar[r] & \resource{disassembly\\listing}}
\seeassembly\seemips\seeobject
}

\providecommand{\mipsbasm}{
\toolsection{mips64asm} is an assembler for the MIPS64 hardware architecture.
It translates assembly code into machine code for MIPS64 processors and stores it in corresponding object files.
\flowgraph{\resource{MIPS64 assembly\\source code} \ar[r] & \toolbox{mips64asm} \ar[r] & \resource{object file}}
\seeassembly\seemips\seeobject
}

\providecommand{\mipsbdism}{
\toolsection{mips64dism} is a disassembler for the MIPS64 hardware architecture.
It translates machine code from object files targeting MIPS64 processors into assembly code and writes it to the standard output stream.
\flowgraph{\resource{object file} \ar[r] & \toolbox{mips64dism} \ar[r] & \resource{disassembly\\listing}}
\seeassembly\seemips\seeobject
}

\providecommand{\mmixasm}{
\toolsection{mmixasm} is an assembler for the MMIX hardware architecture.
It translates assembly code into machine code for MMIX processors and stores it in corresponding object files.
The names of all special registers are predefined and evaluate to the corresponding number.
\flowgraph{\resource{MMIX assembly\\source code} \ar[r] & \toolbox{mmixasm} \ar[r] & \resource{object file}}
\seeassembly\seemmix\seeobject
}

\providecommand{\mmixdism}{
\toolsection{mmixdism} is a disassembler for the MMIX hardware architecture.
It translates machine code from object files targeting MMIX processors into assembly code and writes it to the standard output stream.
\flowgraph{\resource{object file} \ar[r] & \toolbox{mmixdism} \ar[r] & \resource{disassembly\\listing}}
\seeassembly\seemmix\seeobject
}

\providecommand{\orokasm}{
\toolsection{or1kasm} is an assembler for the OpenRISC 1000 hardware architecture.
It translates assembly code into machine code for OpenRISC 1000 processors and stores it in corresponding object files.
\flowgraph{\resource{OpenRISC 1000 assembly\\source code} \ar[r] & \toolbox{or1kasm} \ar[r] & \resource{object file}}
\seeassembly\seeorok\seeobject
}

\providecommand{\orokdism}{
\toolsection{or1kdism} is a disassembler for the OpenRISC 1000 hardware architecture.
It translates machine code from object files targeting OpenRISC 1000 processors into assembly code and writes it to the standard output stream.
\flowgraph{\resource{object file} \ar[r] & \toolbox{or1kdism} \ar[r] & \resource{disassembly\\listing}}
\seeassembly\seeorok\seeobject
}

\providecommand{\ppcaasm}{
\toolsection{ppc32asm} is an assembler for the PowerPC hardware architecture.
It translates assembly code into machine code for PowerPC processors and stores it in corresponding object files.
By default, the assembler generates machine code for the 32-bit operating mode defined by the PowerPC architecture.
\flowgraph{\resource{PowerPC assembly\\source code} \ar[r] & \toolbox{ppc32asm} \ar[r] & \resource{object file}}
\seeassembly\seeppc\seeobject
}

\providecommand{\ppcadism}{
\toolsection{ppc32dism} is a disassembler for the PowerPC hardware architecture.
It translates machine code from object files targeting PowerPC processors into assembly code and writes it to the standard output stream.
It assumes that the machine code was generated for the 32-bit operating mode defined by the PowerPC architecture.
\flowgraph{\resource{object file} \ar[r] & \toolbox{ppc32dism} \ar[r] & \resource{disassembly\\listing}}
\seeassembly\seeppc\seeobject
}

\providecommand{\ppcbasm}{
\toolsection{ppc64asm} is an assembler for the PowerPC hardware architecture.
It translates assembly code into machine code for PowerPC processors and stores it in corresponding object files.
By default, the assembler generates machine code for the 64-bit operating mode defined by the PowerPC architecture.
\flowgraph{\resource{PowerPC assembly\\source code} \ar[r] & \toolbox{ppc64asm} \ar[r] & \resource{object file}}
\seeassembly\seeppc\seeobject
}

\providecommand{\ppcbdism}{
\toolsection{ppc64dism} is a disassembler for the PowerPC hardware architecture.
It translates machine code from object files targeting PowerPC processors into assembly code and writes it to the standard output stream.
It assumes that the machine code was generated for the 64-bit operating mode defined by the PowerPC architecture.
\flowgraph{\resource{object file} \ar[r] & \toolbox{ppc64dism} \ar[r] & \resource{disassembly\\listing}}
\seeassembly\seeppc\seeobject
}

\providecommand{\riscasm}{
\toolsection{riscasm} is an assembler for the RISC hardware architecture.
It translates assembly code into machine code for RISC processors and stores it in corresponding object files.
The names of all special registers are predefined and evaluate to the corresponding number.
\flowgraph{\resource{RISC assembly\\source code} \ar[r] & \toolbox{riscasm} \ar[r] & \resource{object file}}
\seeassembly\seerisc\seeobject
}

\providecommand{\riscdism}{
\toolsection{riscdism} is a disassembler for the RISC hardware architecture.
It translates machine code from object files targeting RISC processors into assembly code and writes it to the standard output stream.
\flowgraph{\resource{object file} \ar[r] & \toolbox{riscdism} \ar[r] & \resource{disassembly\\listing}}
\seeassembly\seerisc\seeobject
}

\providecommand{\wasmasm}{
\toolsection{wasmasm} is an assembler for the WebAssembly architecture.
It translates assembly code into machine code for WebAssembly targets and stores it in corresponding object files.
The names of all special registers are predefined and evaluate to the corresponding number.
\flowgraph{\resource{WebAssembly assembly\\source code} \ar[r] & \toolbox{wasmasm} \ar[r] & \resource{object file}}
\seeassembly\seewasm\seeobject
}

\providecommand{\wasmdism}{
\toolsection{wasmdism} is a disassembler for the WebAssembly architecture.
It translates machine code from object files targeting WebAssembly targets into assembly code and writes it to the standard output stream.
\flowgraph{\resource{object file} \ar[r] & \toolbox{wasmdism} \ar[r] & \resource{disassembly\\listing}}
\seeassembly\seewasm\seeobject
}

% linker tools

\providecommand{\linklib}{
\toolsection{linklib} is an object file combiner.
It creates a static library file by combining all object files given to it into a single one.
\flowgraph{\resource{object files} \ar[r] & \toolbox{linklib} \ar[r] & \resource{library file}}
\seeobject
}

\providecommand{\linkbin}{
\toolsection{linkbin} is a linker for plain binary files.
It links all object files given to it into a single image and stores it in a binary file that begins with the first linked section.
It also creates a map file that lists the address, type, name and size of all used sections.
The filename extension of the resulting binary file can be specified by putting it into a constant data section called \texttt{\_extension}.
\flowgraph{\resource{object files} \ar[r] & \toolbox{linkbin} \ar[r] \ar[d] & \resource{binary file} \\ & \resource{map file}}
\seeobject
}

\providecommand{\linkmem}{
\toolsection{linkmem} is a linker for plain binary files partitioned into random-access and read-only memory.
It links all object files given to it into two distinct images, one for data sections and one for code and constant data sections, and stores each image in a binary file that begins with the first linked section of the corresponding type.
It also creates a map file that lists the address, type, name and size of all used sections.
\flowgraph{\resource{object files} \ar[r] & \toolbox{linkmem} \ar[r] \ar[d] & \resource{RAM file/\\ROM file} \\ & \resource{map file}}
\seeobject
}

\providecommand{\linkprg}{
\toolsection{linkprg} is a linker for GEMDOS executable files.
It links all object files given to it into a single image and stores the image in an Atari GEMDOS executable file~\cite{gemdosfile}.
It also creates a map file that lists the address relative to the text segment, type, name and size of all used sections.
The filename extension of the resulting executable file can be specified by putting it into a constant data section called \texttt{\_extension}.
The GEMDOS executable file format requires all patch patterns of absolute link patches to consist of four full bitmasks with descending offsets.
\flowgraph{\resource{object files} \ar[r] & \toolbox{linkprg} \ar[r] \ar[d] & \resource{executable file} \\ & \resource{map file}}
\seeobject
}

\providecommand{\linkhex}{
\toolsection{linkhex} is a linker for Intel HEX files.
It links all code sections of the object files given to it into single image and stores the image in an Intel HEX file~\cite{hexfile} that begins with the first linked section.
It also creates a map file that lists the address, type, name and size of all used sections.
\flowgraph{\resource{object files} \ar[r] & \toolbox{linkhex} \ar[r] \ar[d] & \resource{HEX file} \\ & \resource{map file}}
\seeobject
}

\providecommand{\mapsearch}{
\toolsection{mapsearch} is a debugging tool.
It searches map files generated by linker tools for the name of a binary section that encompasses a memory address read from the standard input stream.
If additionally provided with one or more object files, it also stores an excerpt thereof in a separate object file called map search result which only contains the identified binary section for disassembling purposes.
\flowgraph{& \resource{map files/\\object files} \ar[d] \\ \resource{memory\\address} \ar[r] & \toolbox{mapsearch} \ar[r] \ar[d] & \resource{section name/\\relative offset} \\ & \resource{object file\\excerpt}}
\seeobject
}

\renewcommand{\seecpp}{}

\startchapter{\cpp{}}{User Manual for \cpp{}}{cpp}
{\cpp{} is a general-purpose programming language with a bias toward systems programming.
It is based on the programming language C and enhances it by supporting data abstraction, object-oriented programming, and generic programming.
This \documentation{} describes the implementation of \cpp{} by the \ecs{}.}

\epigraph{Greatness lies not in being strong, \\ but in the using of strength.}{Henry Ward Beecher}

\section{Introduction}

The \ecs{} implements the \cpp{} programming language according to the ISO \cpp{} Standard ISO/IEC 14882:2023~\cite{iso2023}.
\cpp{} is a general-purpose programming language based on the C programming language as described in the C Standard ISO/IEC 9899:2018~\cite{iso2018}.

\begin{center}\cpplogo{2em}\end{center}

The remainder of this \documentation{} describes the implementation-defined behavior of the implementation by the \ecs{} as required by both of these international standards.

\section{Implementation-Defined Behavior}

The \cpp{} Standard requires all conforming implementations to include documentation describing its characteristics and behavior in areas designated as implementation-defined.
This section lists all implementation-defined behavior including conditionally-supported constructs and locale-specific characteristics along with the corresponding section and paragraph numbers of the \cpp{} Standard.

\newcommand{\cppref}[3]{\alignright#1~[#2]~#3\nopagebreak}
\newcommand{\cppsection}[3]{\subsection[#1]{#1\alignright#2~[#3]}}

\cppsection{Terms and Definitions}{3}{intro.defs}

\begin{itemize}

\item Diagnostic messages \cppref{3.8}{defns.diagnostic}{1}

The set of diagnostic messages consist of all output messages issued by the tools of the \ecs{}.
Diagnostic messages are either errors, warnings, or notes and are meant to be self-explanatory in the context they occur:
Errors indicate a violation of a diagnosable rule or an occurrence of a unsupported construct described in the \cpp{} Standard as conditionally-supported.
Fatal errors flag internal program errors or environmental problems like failures to open source files or critical system conditions like out of memory.
Warning messages identify issues that may lead to unexpected behavior, whereas notes diagnose violations of common coding conventions.

\end{itemize}

\cppsection{General Principles}{4}{intro}

\begin{itemize}

\item Number of contiguous bits in a byte \cppref{4.4}{intro.memory}{1}

The \ecs{} uniformly uses octets as representation for bytes.

\item Interactive devices \cppref{4.6}{intro.execution}{7}

A device is considered to be interactive when it has to wait for input from the user or some other entity in order to proceed the execution of a program.

\item Multiple threads of execution \cppref{4.7}{intro.multithread}{1}

A program can have more than one thread of execution in a freestanding environment.

\end{itemize}

\cppsection{Lexical Conventions}{5}{lex}

\begin{itemize}

\item Mapping of source file characters to the basic source character set \cppref{5.2}{lex.phases}{1}

The \ecs{} assumes source files to be written using the ASCII character set.
This set allows mapping most of the characters to the basic source character set.
Any other character not in the basic source character set is replaced by the universal character name that designates that character.

\item Nonempty sequences of white-space characters \cppref{5.2}{lex.phases}{1}

Nonempty sequences of white-space characters are replaced by one space character.

\item Conversion of source characters to execution characters \cppref{5.2}{lex.phases}{1}

All members of the source character set have a corresponding member in the execution character set.

\item Source of template definitions \cppref{5.2}{lex.phases}{1}

Instantiations of templates require the source of the translation units containing their definitions.

\item Appearance of special characters in header names \cppref{5.8}{lex.header}{2}

Quotation marks, backslashes, and the character sequences~\texttt{/*} or~\texttt{//} may appear in header names but bear no special meaning and are therefore treated like any other character.

\item Character literals containing multiple characters \cppref{5.13.3}{lex.ccon}{2/6}

Ordinary and wide character literals may contain more than one character.
However, all but the first character are ignored when determining the value of these literals.

\item Additional escape sequences \cppref{5.13.3}{lex.ccon}{7}

The \ecs{} does only support the standard escape sequences.

\item Character values outside valid range \cppref{5.13.3}{lex.ccon}{8/9}

The value of an escape sequence or a universal character name in a character literal must be representable using the underlying type of the literal.

\item String literal concatenations \cppref{5.13.5}{lex.string}{13}

Concatenations of adjacent string literals with different encoding prefixes are not supported.

\end{itemize}

\cppsection{Basic Concepts}{6}{basic}

\begin{itemize}

\item Definition of functions or variables with origin \cppref {6.2}{basic.def.odr}{4}

A non-inline function or variable does not require a definition if it is declared using the origin attribute, see Section~\ref{sec:cppdeclarations}.

\item Definition of \texttt{main} in freestanding environments \cppref{6.6.1}{basic.start.main}{1}

A program in a freestanding environment is required to define a \texttt{main} function.

\item Linkage of \texttt{main} \cppref{6.6.1}{basic.start.main}{3}

The \texttt{main} function has external linkage.

\item Use of invalid pointer values \cppref{6.7}{basic.stc}{4}

There are no restrictions for using an invalid pointer value except for passing it to a deallocation function and accessing the referenced region of storage.

\item Pointer safety \cppref{6.7.4.3}{basic.stc.dynamic.safety}{4}

The \ecs{} has relaxed pointer safety.
A pointer value is valid regardless of whether it is safely derived or not.

\item Extended integer types \cppref{6.9.1}{basic.fundamental}{2}

The \ecs{} does not provide any extended integer types.

\item Fundamental alignments \cppref{6.11}{basic.align}{2}

The alignment of a fundamental type depends on the size of the type as well as the execution environment.
Table~\ref{tab:cppfundamentalalignments} lists the alignment of fundamental types for each \cpp{} compiler and type size supported by the \ecs{}.
The sizes of fundamental types are shown in Table~\ref{tab:cppfundamentaltypes} and Table~\ref{tab:cppdependenttypes}.
The interpreter and all other tools reuse the respective fundamental alignments of their own execution environment instead.

\begin{table}
\centering
\begin{tabular}{@{}lrlc@{\qquad}c@{\qquad}c@{\qquad}c@{}}
\toprule \multicolumn{2}{@{}l}{Hardware} & \cpp{} & \multicolumn{4}{c@{}}{Type Size} \\ \multicolumn{2}{@{}l}{Architecture} & Compiler & 1 & 2 & 4 & 8 \\
\midrule AMD64 & 16-bit & \tool{cppamd16} & 1 & 2 & 4 & 4 \\ & 32-bit & \tool{cppamd32} & 1 & 2 & 4 & 4 \\ & 64-bit & \tool{cppamd64} & 1 & 2 & 4 & 8 \\
\midrule ARM & A32 & \tool{cpparma32} & 1 & 2 & 4 & 8 \\ & A64 & \tool{cpparma64} & 1 & 2 & 4 & 8 \\ & T32 & \tool{cpparmt32} & 1 & 2 & 4 & 8 \\ & & \tool{cpparmt32fpe} & 1 & 2 & 4 & 8 \\
\midrule \multicolumn{2}{@{}l}{AVR} & \tool{cppavr} & 1 & 1 & 1 & 1 \\
\midrule \multicolumn{2}{@{}l}{AVR32} & \tool{cppavr32} & 1 & 2 & 4 & 8 \\
\midrule \multicolumn{2}{@{}l}{M68000} & \tool{cppm68k} & 1 & 2 & 2 & 2 \\
\midrule \multicolumn{2}{@{}l}{MicroBlaze} & \tool{cppmibl} & 1 & 2 & 4 & 8 \\
\midrule MIPS & 32-bit & \tool{cppmips32} & 1 & 2 & 4 & 8 \\ & 64-bit & \tool{cppmips64} & 1 & 2 & 4 & 8 \\
\midrule \multicolumn{2}{@{}l}{MMIX} & \tool{cppmmix} & 1 & 2 & 4 & 8 \\
\midrule \multicolumn{2}{@{}l}{OpenRISC 1000} & \tool{cppor1k} & 1 & 2 & 4 & 4 \\
\midrule PowerPC & 32-bit & \tool{cppppc32} & 1 & 2 & 4 & 8 \\ & 64-bit & \tool{cppppc64} & 1 & 2 & 4 & 8 \\
\midrule \multicolumn{2}{@{}l}{RISC} & \tool{cpprisc} & 1 & 2 & 4 & 4 \\
\midrule \multicolumn{2}{@{}l}{WebAssembly} & \tool{cppwasm} & 1 & 2 & 4 & 8 \\
\midrule \multicolumn{2}{@{}l}{Xtensa} & \tool{cpprisc} & 1 & 2 & 4 & 4 \\
\bottomrule
\end{tabular}
\caption{Alignments of fundamental \cpp{} types}
\label{tab:cppfundamentalalignments}
\end{table}

\item Extended alignments \cppref{6.11}{basic.align}{3}

The \ecs{} supports extended alignments for variables with static storage duration.
The set of valid alignments consists of all non-negative integral powers of two representable in the type \texttt{std::size\_t}.

\end{itemize}

\cppsection{Standard Conversions}{7}{conv}

\begin{itemize}

\item Value of integral conversions \cppref{7.8}{conv.integral}{3}

If the destination type is signed, the resulting value of an integral conversion is the least signed integer congruent to the source integer modulo $2^n$ where $n$ is the number of bits used to represent the signed type.

\item Ranks of extended signed integer types \cppref{7.15}{conv.rank}{1}

The \ecs{} does not provide any extended integer types.

\end{itemize}

\cppsection{Expressions}{8}{expr}

\begin{itemize}

\item Sizes of fundamental types \cppref{8.3.3}{expr.sizeof}{1}

Table~\ref{tab:cppfundamentaltypes} lists the size and value range of each fundamental type as defined by the \ecs{}.
The sizes of the types \texttt{int}, \texttt{unsigned int}, and \texttt{double} depend on the execution environment and are listed in Table~\ref{tab:cppdependenttypes} for each \cpp{} compiler provided by the \ecs{}.
The interpreter and all other tools reuse the respective type sizes of their own execution environment instead.

\begin{table}
\centering
\begin{tabular}{@{}llcl@{}}
\toprule Category & Type & Size & Value Range \\
\midrule Boolean
& \texttt{bool} & 1 & \texttt{false} or \texttt{true} \\
\midrule Character
& \texttt{char} & 1 & $0$ to $255$ \\
& \texttt{signed char} & 1 & $-128$ to $+127$ \\
& \texttt{unsigned char} & 1 & $0$ to $255$ \\
& \texttt{char16_t} & 2 & $0$ to $65\,535$ \\
& \texttt{char32_t} & 4 & $0$ to $4\,294\,967\,295$ \\
& \texttt{wchar_t} & 4 & $0$ to $4\,294\,967\,295$ \\
\midrule Integer
& \texttt{short int} & 2 & $-2^{15}$ to $+2^{15}-1$ \\
& \texttt{unsigned short int} & 2 & $0$ to $2^{16}-1$ \\
& \texttt{int} & $2/4$ & \emph{See Table~\ref{tab:cppdependenttypes}} \\
& \texttt{unsigned int} & $2/4$ & \emph{See Table~\ref{tab:cppdependenttypes}} \\
& \texttt{long int} & 4 & $-2^{31}$ to $+2^{31}-1$ \\
& \texttt{unsigned long int} & 4 & $0$ to $2^{32}-1$ \\
& \texttt{long long int} & 8 & $-2^{63}$ to $+2^{63}-1$ \\
& \texttt{unsigned long long int} & 8 & $0$ to $2^{64}-1$ \\
\midrule Floating-Point
& \texttt{float} & 4 & $\pm 3.4028234 \times 10^{38}$ \\
& \texttt{double} & $4/8$ & \emph{See Table~\ref{tab:cppdependenttypes}} \\
& \texttt{long double} & 8 & $\pm 1.7976931348623157 \times 10^{308}$ \\
\bottomrule
\end{tabular}
\caption{Sizes and value ranges of fundamental \cpp{} types}
\label{tab:cppfundamentaltypes}
\end{table}

\begin{table}
\centering
\begin{tabular}{@{}lrlccc@{}}
\toprule \multicolumn{2}{@{}l}{Hardware} & \cpp{} & \texttt{int} & & pointer types \\ \multicolumn{2}{@{}l}{Architecture} & Compiler & \texttt{unsigned int} & \texttt{double} & \texttt{std::size\_t} \\
\midrule AMD64 & 16-bit & \tool{cppamd16} & 2 & 8 & 2 \\ & 32-bit & \tool{cppamd32} & 4 & 8 & 4 \\ & 64-bit & \tool{cppamd64} & 4 & 8 & 8 \\
\midrule ARM & A32 & \tool{cpparma32} & 4 & 8 & 4 \\ & A64 & \tool{cpparma64} & 4 & 8 & 8 \\ & T32 & \tool{cpparmt32} & 4 & 4 & 4 \\ & & \tool{cpparmt32fpe} & 4 & 8 & 4 \\
\midrule \multicolumn{2}{@{}l}{AVR} & \tool{cppavr} & 2 & 4 & 2 \\
\midrule \multicolumn{2}{@{}l}{AVR32} & \tool{cppavr32} & 4 & 4 & 4 \\
\midrule \multicolumn{2}{@{}l}{M68000} & \tool{cppm68k} & 2 & 4 & 4 \\
\midrule \multicolumn{2}{@{}l}{MicroBlaze} & \tool{cppmibl} & 4 & 4 & 4 \\
\midrule MIPS & 32-bit & \tool{cppmips32} & 4 & 8 & 4 \\ & 64-bit & \tool{cppmips64} & 4 & 8 & 8 \\
\midrule \multicolumn{2}{@{}l}{MMIX} & \tool{cppmmix} & 4 & 8 & 8 \\
\midrule \multicolumn{2}{@{}l}{OpenRISC 1000} & \tool{cppor1k} & 4 & 4 & 4 \\
\midrule PowerPC & 32-bit & \tool{cppppc32} & 4 & 4 & 4 \\ & 64-bit & \tool{cppppc64} & 4 & 4 & 4 \\
\midrule \multicolumn{2}{@{}l}{RISC} & \tool{cpprisc} & 4 & 4 & 4 \\
\midrule \multicolumn{2}{@{}l}{WebAssembly} & \tool{cppwasm} & 4 & 8 & 4 \\
\midrule \multicolumn{2}{@{}l}{Xtensa} & \tool{cppwasm} & 4 & 4 & 4 \\
\bottomrule
\end{tabular}
\caption{Sizes of hardware-dependent \cpp{} types}
\label{tab:cppdependenttypes}
\end{table}

\item Resulting type of pointer subtractions \cppref{8.7}{expr.add}{5}

The type of the result of subtracting two pointer values is the signed counterpart of \texttt{std::size_t}.

\item Result of right-shifting negative values \cppref{8.8}{expr.shift}{3}

Shift operations on signed types are arithmetic.
Shifting a negative value to the right extends its sign.

\end{itemize}

\cppsection{Declarations}{10}{dcl.dcl}\label{sec:cppdeclarations}

\begin{itemize}

\item Meaning of attribute declarations \cppref{10}{dcl.dcl}{3}

Attribute declarations appertain to the current translation unit and may contain the standard \texttt{deprecated} attribute which applies to included headers and source files.

\item The \texttt{asm} declaration \cppref{10.4}{dcl.asm}{1}

The \texttt{asm} declaration is supported and allows writing inline assembly code using one of the various compilers for the \cpp{} programming language.
All of them pass the string literal of the \texttt{asm} declaration to the assembler used to generate the machine code.
The available instruction set therefore depends on the actual compiler used to compile the source code.
\seeassembly
The interpreter does not support executing \texttt{asm} declarations.

The assembly code of an \texttt{asm} declaration appearing at namespace scope has unrestricted access to all features of the assembler.
It must therefore first create a code or data section before any other directive or instruction can be used.
At block scope, the inline assembly code is part of a code section predefined by the compiler which also provides constant definitions for all labels, variables, parameters, enumerations and their enumerators that are accessible from that scope.
The name of a label is predefined as the address of the corresponding labeled statement and can therefore be used as the target of branch instructions.
The names of all parameters and variables with automatic storage duration evaluate to the offset of the corresponding entity relative to the frame pointer.
The name of a variable with static storage duration evaluates to its address.
If a variable or parameter is declared using the register attribute, its name is an alias for the corresponding processor register.
The name of an enumerator or a non-volatile constant variable initialized with a constant expression of fundamental or pointer type is predefined to its value.
For debugging purposes, all names predefined in inline assembly code and their actual values are accessible using the expression evaluation directive.

\item Language linkages \cppref{10.5}{dcl.link}{2}

Besides the two language linkages \texttt{"C"} and \texttt{"C++"} required by the \cpp{} Standard, the \ecs{} also supports the \texttt{"Oberon"} language linkage.
It allows accessing global procedures and variables defined in Oberon modules.
In order to identify the containing module and its package, the corresponding functions and objects declared with this language linkage must be members of named namespaces.
\seeoberon

\item Non-standard attributes \cppref{10.6.1}{dcl.attr.grammar}{6}

Besides the attributes specified in the \cpp{} Standard, the \ecs{} also recognizes the following non-standard attributes.
All of them may only appear once in an attribute list and cannot be used as pack expansions.
Except where otherwise noted, they apply only to functions and variables with static storage duration and do not accept arguments:

\begin{itemize}

\item Alias attribute\alignright\syntax{"ecs::alias" "(" <string-literal> ")"}\nopagebreak

The alias attribute specifies that the corresponding entity may also be accessible in assembly code or other programming languages using the name given in the non-empty string literal.

\item Code attribute\alignright\syntax{"ecs::code"}\nopagebreak

The code attribute applies to \texttt{asm} declarations and causes their string literal to be interpreted as intermediate code.
\seecode

\item Duplicable attribute\alignright\syntax{"ecs::duplicable"}\nopagebreak

The duplicable attribute specifies that linkers merge the definition of the corresponding entity with other entities that have the same definition.

\item Group attribute\alignright\syntax{"ecs::group" "(" <string-literal> ")"}\nopagebreak

The group attribute specifies that linkers place the corresponding entity adjacent to other entities that also belong to the group named in the non-empty string literal.

\item Origin attribute\alignright\syntax{"ecs::origin" "(" <constant-expression> ")"}\nopagebreak

The origin attribute specifies the desired address of the variable or function.
It requires an integral constant expression as argument that is convertible to \texttt{std::size\_t}.
This attribute may not be combined with alignment specifiers.

\item Register attribute\alignright\syntax{"ecs::register"}\nopagebreak

The register attribute applies to up to four non-volatile parameters and variables with automatic storage duration of fundamental or pointer type and specifies that compilers store the corresponding variable or parameter in a register.

\item Replaceable attribute\alignright\syntax{"ecs::replaceable"}\nopagebreak

The replaceable attribute specifies that linkers discard the definition of the corresponding entity if another entity with the same name exists.

\item Required attribute\alignright\syntax{"ecs::required"}\nopagebreak

The required attribute specifies that linkers do not omit the definition of the corresponding entity even if it is never actually used in a program.

\end{itemize}

Any entity declared with one of these attributes can later be redeclared without that attribute and vice versa.
However, redeclarations and declarations in other translation units using different forms of the same attribute are not allowed.

\item Noreturn \texttt{asm} declarations \cppref{9.12.10}{dcl.attr.noreturn}{1}

The noreturn attribute may be applied to \texttt{asm} declarations at block scope and specifies that the corresponding assembly code does not return.

\end{itemize}

\cppsection{Declarators}{11}{dcl.decl}

\begin{itemize}

\item String value of \texttt{\_\_func\_\_} \cppref{11.4.1}{dcl.fct.def.general}{8}

The string resulting from the function-local predefined variable \texttt{\_\_func\_\_} matches the fully qualified section name of the function as described in Section~\ref{sec:cppnamingconventions}.

\end{itemize}

\cppsection{Classes}{12}{class}

\begin{itemize}

\item Allocation and alignment of bit-fields \cppref{12.2.4}{class.bit}{1}

Bit-fields are allocated consecutively and packed adjacent to each other, as long as they do not straddle allocation units and the alignment of their respective types does not require padding bits.

\end{itemize}

\cppsection{Preprocessing Directives}{19}{cpp}

\begin{itemize}

\item Additional preprocessing directives \cppref{19}{cpp}{2}

The \ecs{} does not support additional preprocessing directives.

\item Interpretation of character literals \cppref{19.1}{cpp.cond}{10}

The numeric value for character literals within a \texttt{\#if} or \texttt{\#elif} directive is non-negative and matches the value obtained when an identical character literal occurs in an expression.

\item Source file inclusion \cppref{19.2}{cpp.include}{2/3}

Source files identified between a pair of~\texttt{"} delimiters in \texttt{\#include} directives are searched relative to the directory of the current source file.
Headers identified between the~\texttt{<} and~\texttt{>} delimiters are searched in the relative directory given by the \environmentvariable{ECSINCLUDE} environment variable.
Two or more directories can be separated by semicolons and are searched in order.
Each directory must include a trailing path separator.

\item Combination of preprocessing tokens into one header name token \cppref{19.2}{cpp.include}{4}

A pair of~\texttt{"} delimiters is treated as a string literal token.
All preprocessing tokens between~\texttt{<} and~\texttt{>} delimiters are concatenated to form a single header name token.
Identifiers therein are subject to macro replacement.

\item Pragma directive \cppref{19.6}{cpp.pragma}{1}

The \ecs{} recognizes the \texttt{\#pragma end} pragma directive which simulates the end of a source file.
The remaining part of the source file following this pragma directive is completely ignored.
All other pragma directives are ignored.

\item Predefined macro names \cppref{19.8}{cpp.predefined}{1}

If the date of translation is not available, the macros \texttt{\_\_DATE\_\_} and \texttt{\_\_TIME\_\_} are predefined as \texttt{"May 10 2023"} and \texttt{"00:00:00"} respectively.
The macro \texttt{\_\_STD\_HOSTED\_\_} is defined as~\texttt{1} since the \cpp{} implementation of the \ecs{} is hosted.

\item Conditionally-defined macro names \cppref{16.8}{cpp.predefined}{2}

The macros \texttt{\_\_STDC\_\_} and \texttt{\_\_STDC\_VERSION\_\_} are not predefined.

\item Implementation-defined macro names \cppref{16.8}{cpp.predefined}{4}

The macro \texttt{\_\_ecs\_\_} is predefined in order to enable programmers to detect the \ecs{} while processing compiler-specific source code.
The value of the predefined macro \texttt{\_\_sizeof\_}\textit{type}\texttt{\_\_} where \textit{type} is either \texttt{double}, \texttt{float}, \texttt{int}, \texttt{long}, \texttt{long_double}, \texttt{pointer}, or \texttt{short} is the size of the corresponding type.
The interpreter and all compilers predefine an additional macro name of the form \texttt{\_\_}\textit{target}\texttt{\_\_} for identifying the target environment where \textit{target} is either \texttt{amd16}, \texttt{amd32}, \texttt{amd64}, \texttt{arma32}, \texttt{arma64}, \texttt{armt32}, \texttt{armt32fpe}, \texttt{avr}, \texttt{avr32}, \texttt{code}, \texttt{m68k}, \texttt{mibl}, \texttt{mips32}, \texttt{mips64}, \texttt{mmix}, \texttt{or1k}, \texttt{ppc32}, \texttt{ppc64}, \texttt{risc}, \texttt{run}, \texttt{wasm}, or \texttt{xtensa}.

\end{itemize}

\cppsection{Library Introduction}{20}{library}

\begin{itemize}

\item Declaration of additional functions from the standard C library \cppref{20.5.1.2}{headers}{9}

The functions described in Annex K of the C standard are not declared when including \cpp{} headers.

\item Freestanding implementations \cppref{20.5.1.3}{compliance}{2}

The \ecs{} provides an implementation of \cpp{} that can be used in hosted as well as freestanding environments.
In both cases, the complete set of headers described by the \cpp{} Standard is available.

\item Linkage of names from the standard C library \cppref{20.5.2.3}{using.linkage}{2}

Names from the C standard library declared with external linkage have \texttt{extern "C"} linkage.

\end{itemize}

\cppsection{Language Support Library}{21}{language.support}

\begin{itemize}

\item Definition of macro \texttt{NULL} \cppref{21.2.3}{support.types.nullptr}{2}

The macro \texttt{NULL} is defined as \texttt{nullptr}.

\end{itemize}

\cppsection{Implementation Quantities}{B}{implimits}

According the \cpp{} Standard, each implementation shall document the limitations of the programs they can successfully process.
This section lists all known limitations of the \ecs{}:

\begin{itemize}

\item Size of an object \cppref{6.9.2}{basic.compound}{2}

The size of an object is limited by the maximal value representable in the type \texttt{std::size\_t}.

\item Nesting levels for \texttt{\#include} files \cppref{19.2}{cpp.include}{6}

The nesting level of \texttt{\#include} directives has a limit of 256 in order to detect potentially infinite recursions.

\item Functions registered by \texttt{atexit()} \cppref{21.5}{support.start.term}{6}

The \ecs{} supports the registration of at most 32 functions.

\item Functions registered by \texttt{at_quick_exit()} \cppref{21.5}{support.start.term}{10}

The \ecs{} supports the registration of at most 32 functions.

\item Recursive \texttt{constexpr} function invocations \cppref{8.20}{expr.const}{2}

The total number of nested calls of \texttt{constexpr} functions is limited to 512.

\item Recursively nested template instantiations \cppref{17.7.1}{temp.inst}{15}

The total depth of recursive template instantiations, including substitution during template argument deduction, has a limit of 1024 in order to detect potentially infinite recursions.

\item Number of placeholders \cppref{23.14.11.4}{func.bind.place}{1}

The implementation-defined number of placeholders $M$ is 10.

\end{itemize}

All other quantities listed in Annex~B of the \cpp{} Standard that are not mentioned above have no intrinsic limit and are only restricted by available memory.
The actual limit depends therefore only on the execution environment of the tool used to process the \cpp{} source file.

\section{The Standard \cpp{} Library}

The \ecs{} provides its own implementation of the \cpp{} standard library called the \emph{Standard \cpp{} Library}\index{Standard C++ Library@Standard \cpp{} Library}\index{Libraries!Standard C++ Library@Standard \cpp{} Library}.
Its facilities are made available by including one or more of its headers and providing the necessary runtime support as described in Section~\ref{sec:cppruntimesupport}.
All of these headers are governed by the \rse{} which is an additional permission to the \gpl{} that allows users of the \ecs{} to create proprietary programs.
\ifbook Copies of these licenses are included in Appendices~\ref{gpl} and~\ref{rse} on pages~\pageref{gpl} and~\pageref{rse} respectively. \fi

\input{cpplibrary.doc}

\section{Documentation Generation}

The \ecs{} provides several tools that are able to extract the structure of programs written in \cpp{} and generate documentations for them.
This section describes the contents of the extracted information and explains how programmers can provide a user-defined description of it.

\ifbook\else\markuptable\fi

\section{Runtime Support}\label{sec:cppruntimesupport}\index{Runtime support!for C++@for \cpp{}}

Some language features of \cpp{} such as exceptions require some additional runtime support.
This runtime support is stored in library files which are combinations of object files. \seeobject
The \ecs{} provides the required runtime support in one library file for each hardware architecture it supports.
The name of the corresponding library file consists of a leading \file{cpp}, the name of the target hardware architecture, and a trailing \file{run} as in \file{cpp\-amd64\-run}.

\section{\cpp{} Tools}

The \ecs{} provides several different tools that process source files written in \cpp{}.
\interface

The tools process \cpp{} translation units in several consecutive stages.
In each stage, the internal representation of the translation unit is changed and transformed.
Figure~\ref{fig:cppdataflow} shows all stages and the different representations.

\begin{figure}
\flowgraph{
& \resource{\cpp{}\\source code} \ar[d] \\
\variable{ECSINCLUDE} \ar[r] & \converter{Preprocessor} \ar[d] \ar[r] & \resource{preprocessed\\source code} \\
& \resource{tokens} \ar[d] \\
& \converter{Parser} \ar[d] \\
\converter{Serializer} \ar[d] & \resource{abstract\\syntax tree} \ar[l] \ar[d] \ar[r] & \converter{Pretty Printer} \ar[d] \\
\resource{internal\\representation} & \converter{Semantic\\Checker} \ar[d] & \resource{reformatted\\source code} \\
\converter{Interpreter} \ar@/l/[d] & \resource{attributed\\syntax tree} \ar[l] \ar[d] \ar[r] & \converter{Documentation\\Extractor} \ar[d] \\
\resource{input/\\output} \ar@/r/[u] & \converter{Intermediate\\Code Emitter} \ar[d] & \resource{generic\\documentation} \\
& \resource{intermediate\\code} \ar[d] \ar@/u/[r] & \converter{Optimizer} \ar@/d/[l] \\
\resource{assembly\\listing} & \converter{Machine Code\\Generator} \ar[l] \ar[d] \ar[r] & \resource{debugging\\information} \\
& \resource{object file} \\
}\caption{Data flow within the tools for \cpp{}}
\label{fig:cppdataflow}
\end{figure}

\cppprep
\cppprint
\cppcheck
\cppdump
\cpprun
\cppdoc
\cpphtml
\cpplatex
\cppcode
\cppamda
\cppamdb
\cppamdc
\cpparma
\cpparmb
\cpparmc
\cpparmcfpe
\cppavr
\cppavrtt
\cppmabk
\cppmibl
\cppmipsa
\cppmipsb
\cppmmix
\cpporok
\cppppca
\cppppcb
\cpprisc
\cppwasm
\cppxtensa

\section{Interoperability}

In accordance with the goal of the \ecs{} to enable interoperability between its implemented programming languages,
the compilers for \cpp{} provide different mechanisms to exchange data with programs written with other tools of the \ecs{}.
The interoperability is enabled by a common intermediate code representation and calling convention. \seecode

This section describes the naming conventions used to uniquely identify the intermediate code sections defined by the compilers for \cpp{}
as well as the ways of accessing sections defined by other compilers and assemblers.

\subsection{Naming Conventions}\label{sec:cppnamingconventions}

The compilers for the \cpp{} programming language emit a code section for each function definition, function bodies of lambda expressions, and non-local variable with dynamic initialization.
Additional data sections are defined for non-local variables and type informations required at runtime.

The name of each section equals to the name of the corresponding entity and is prefixed by the name of its containing scope in case the entity does not have C language linkage.
The names of entities and their scope are delimited by scope operators such that names resemble qualified identifiers and no name mangling is necessary.

\subsection{Accessing Sections}

\cpp{} as implemented by the \ecs{} allows two different ways of accessing sections defined by other compilers or assemblers.

\begin{itemize}

\item
The \texttt{asm} declaration allows writing inline assembly code which naturally enables arbitrary access to any section, see Section~\ref{sec:cppdeclarations}.
\seeassembly

\item
The \texttt{extern} specifier allows referring to data and code sections that are defined elsewhere.
In cases where the different language linkages supported by the \ecs{} do not automatically yield the required section name for a suitably declared function or variable,
the alias attribute can be used to provide an arbitrary section name by hand, see Section~\ref{sec:cppdeclarations}.

\end{itemize}

\subsection{Program Execution}

The entry point of a program is represented by the \texttt{main} function.
Unless provided by any other programming language, it must be defined in one of the linked \cpp{} translation units.
Any dynamic initialization of global variables with static storage duration is executed before the \texttt{main} function.
The destruction of variables with static storage duration on the other hand requires an explicit or implicit call of the standard \texttt{exit} function.

\concludechapter

% User manual for FALSE
% Copyright (C) Florian Negele

% This file is part of the Eigen Compiler Suite.

% Permission is granted to copy, distribute and/or modify this document
% under the terms of the GNU Free Documentation License, Version 1.3
% or any later version published by the Free Software Foundation.

% You should have received a copy of the GNU Free Documentation License
% along with the ECS.  If not, see <https://www.gnu.org/licenses/>.

% Generic documentation utilities
% Copyright (C) Florian Negele

% This file is part of the Eigen Compiler Suite.

% Permission is granted to copy, distribute and/or modify this document
% under the terms of the GNU Free Documentation License, Version 1.3
% or any later version published by the Free Software Foundation.

% You should have received a copy of the GNU Free Documentation License
% along with the ECS.  If not, see <https://www.gnu.org/licenses/>.

\providecommand{\cpp}{C\texttt{++}}
\providecommand{\opt}{_\mathit{opt}}
\providecommand{\tool}[1]{\texttt{#1}}
\providecommand{\version}{Version 0.0.40}
\providecommand{\resource}[1]{*++\txt{#1}}
\providecommand{\ecs}{Eigen Compiler Suite}
\providecommand{\changed}[1]{\underline{#1}}
\providecommand{\toolbox}[1]{\converter{#1}}
\providecommand{\file}{}\renewcommand{\file}[1]{\texttt{#1}}
\providecommand{\alignright}{\hfill\linebreak[0]\hspace*{\fill}}
\providecommand{\converter}[1]{*++[F][F*:white][F,:gray]\txt{#1}}
\providecommand{\documentation}{\ifbook chapter\else document\fi}
\providecommand{\Documentation}{\ifbook Chapter\else Document\fi}
\providecommand{\variable}[1]{\resource{\texttt{\small#1}\\variable}}
\providecommand{\documentationref}[2]{\ifbook\ref{#1}\else``\href{#1}{#2}''~\cite{#1}\fi}
\providecommand{\objfile}[1]{\texttt{#1}\index[runtime]{#1 object file@\texttt{#1} object file}}
\providecommand{\libfile}[1]{\texttt{#1}\index[runtime]{#1 library file@\texttt{#1} library file}}
\providecommand{\epigraph}[2]{\ifbook\begin{quote}\flushright\textit{#1}\par--- #2\end{quote}\fi}
\providecommand{\environmentvariable}[1]{\texttt{#1}\index{Environment variables!#1@\texttt{#1}}}
\providecommand{\environment}[1]{\texttt{#1}\index[environment]{#1 environment@\texttt{#1} environment}}
\providecommand{\toolsection}{}\renewcommand{\toolsection}[1]{\subsection{#1}\label{\prefix:#1}\tool{#1}}
\providecommand{\instruction}{}\renewcommand{\instruction}[2]{\noindent\qquad\pdftooltip{\texttt{#1}}{#2}\refstepcounter{instruction}\par}
\providecommand{\flowgraph}{}\renewcommand{\flowgraph}[1]{\par\sffamily\begin{displaymath}\xymatrix@=4ex{#1}\end{displaymath}\normalfont\par}
\providecommand{\instructionset}{}\renewcommand{\instructionset}[4]{\setcounter{instruction}{0}\begin{multicols}{\ifbook#3\else#4\fi}[{\captionof{table}[#2]{#2 (\ref*{#1:instructions}~instructions)}\label{tab:#1set}\vspace{-2ex}}]\footnotesize\raggedcolumns\input{#1.set}\label{#1:instructions}\end{multicols}}

\providecommand{\gpl}{GNU General Public License}
\providecommand{\rse}{ECS Runtime Support Exception}
\providecommand{\fdl}{\href{https://www.gnu.org/licenses/fdl.html}{GNU Free Documentation License}}

\providecommand{\docbegin}{}
\providecommand{\docend}{}
\providecommand{\doclabel}[1]{\hypertarget{#1}}
\providecommand{\doclink}[2]{\hyperlink{#1}{#2}}
\providecommand{\docsection}[3]{\hypertarget{#1}{\subsection{#2}}\label{sec:#1}\index[library]{#2@#3}}
\providecommand{\docsectionstar}[1]{}
\providecommand{\docsubbegin}{\begin{description}}
\providecommand{\docsubend}{\end{description}}
\providecommand{\docsubsection}[3]{\item[\hypertarget{#1}{#2}]\index[library]{#2@#3}}
\providecommand{\docsubsectionstar}[1]{\smallskip}
\providecommand{\docsubsubsection}[3]{\docsubsection{#1}{#2}{#3}}
\providecommand{\docsubsubsectionstar}[1]{}
\providecommand{\docsubsubsubsection}[3]{}
\providecommand{\docsubsubsubsectionstar}[1]{}
\providecommand{\doctable}{}

\providecommand{\debuggingtool}{}\renewcommand{\debuggingtool}{This tool is provided for debugging purposes.
It allows exposing and modifying an internal data structure that is usually not accessible.
}

\providecommand{\interface}{All tools accept command-line arguments which are taken as names of plain text files containing the source code.
If no arguments are provided, the standard input stream is used instead.
Output files are generated in the current working directory and have the same name as the input file being processed whereas the filename extension gets replaced by an appropriate suffix.
\seeinterface
}

\providecommand{\license}{\noindent Copyright \copyright{} Florian Negele\par\medskip\noindent
Permission is granted to copy, distribute and/or modify this document under the terms of the
\fdl{}, Version 1.3 or any later version published by the \href{https://fsf.org/}{Free Software Foundation}.
}

\providecommand{\ecslogosurface}{
\fill[darkgray] (0,0,0) -- (0,0,3) -- (0,3,3) -- (0,3,1) -- (0,4,1) -- (0,4,3) -- (0,5,3) -- (0,5,0) -- (0,2,0) -- (0,2,2) -- (0,1,2) -- (0,1,0) -- cycle;
\fill[gray] (0,5,0) -- (0,5,3) -- (1,5,3) -- (1,5,1) -- (2,5,1) -- (2,5,3) -- (3,5,3) -- (3,5,0) -- cycle;
\fill[lightgray] (0,0,0) -- (0,1,0) -- (2,1,0) -- (2,4,0) -- (1,4,0) -- (1,3,0) -- (2,3,0) -- (2,2,0) -- (0,2,0) -- (0,5,0) -- (3,5,0) -- (3,0,0) -- cycle;
\begin{scope}[line width=0.5]
\begin{scope}[gray]
\draw (0,0,0) -- (0,1,0);
\draw (2,1,0) -- (2,2,0);
\draw (0,1,2) -- (0,2,2);
\draw (0,2,0) -- (0,5,0);
\draw (2,3,0) -- (2,4,0);
\end{scope}
\begin{scope}[lightgray]
\draw (0,1,0) -- (0,1,2);
\draw (0,3,1) -- (0,3,3);
\draw (0,5,0) -- (0,5,3);
\draw (2,5,1) -- (2,5,3);
\end{scope}
\begin{scope}[white]
\draw (0,1,0) -- (2,1,0);
\draw (1,3,0) -- (2,3,0);
\draw (0,5,0) -- (3,5,0);
\end{scope}
\end{scope}
}

\providecommand{\ecslogo}[1]{
\begin{tikzpicture}[scale={(#1)/((sin(45)+cos(45))*3cm)},x={({-cos(45)*1cm},{sin(45)*sin(30)*1cm})},y={({0cm},{(cos(30)*1cm})},z={({sin(45)*1cm},{cos(45)*sin(30)*1cm})}]
\begin{scope}[darkgray,line width=1]
\draw (0,0,0) -- (0,0,3) -- (0,3,3) -- (2,3,3) -- (2,5,3) -- (3,5,3) -- (3,5,0) -- (3,0,0) -- cycle;
\draw (0,3,1) -- (0,4,1) -- (0,4,3) -- (0,5,3) -- (1,5,3) -- (1,5,1) -- (2,5,1);
\draw (1,3,0) -- (1,4,0) -- (2,4,0);
\end{scope}
\fill[darkgray] (2,0,0) -- (2,0,3) -- (2,5,3) -- (2,5,1) -- (2,4,1) -- (2,4,0) -- cycle;
\fill[lightgray] (2,0,2) -- (0,0,2) -- (0,2,2) -- (2,2,2) -- cycle;
\fill[gray] (0,1,0) -- (2,1,0) -- (2,1,2) -- (0,1,2) -- cycle;
\fill[gray] (0,3,1) -- (0,3,3) -- (2,3,3) -- (2,3,0) -- (1,3,0) -- (1,3,1) -- cycle;
\ecslogosurface
\end{tikzpicture}
}

\providecommand{\shadowedecslogo}[3]{
\begin{tikzpicture}[scale={(#1)/((sin(#2)+cos(#2))*3cm)},x={({-cos(#2)*1cm},{sin(#2)*sin(#3)*1cm})},y={({0cm},{(cos(#3)*1cm})},z={({sin(#2)*1cm},{cos(#2)*sin(#3)*1cm})}]
\shade[top color=lightgray!50!white,bottom color=white,middle color=lightgray!50!white] (0,0,0) -- (3,0,0) -- (3,{-0.5-3*sin(#2)*sin(#3)/cos(#3)},0) -- (0,-0.5,0) -- cycle;
\shade[top color=darkgray!50!gray,bottom color=white,middle color=darkgray!50!white] (0,0,0) -- (0,0,3) -- (0,{-0.5-3*cos(#2)*sin(#3)/cos(#3)},3) -- (0,-0.5,0) -- cycle;
\begin{scope}[y={({(cos(#2)+sin(#2))*0.5cm},{(cos(#2)*sin(#3)-sin(#2)*sin(#3))*0.5cm})}]
\useasboundingbox (3,0,0) -- (0,0,0) -- (0,0,3);
\shade[left color=darkgray!80!black,right color=lightgray,middle color=gray] (0,0,0) -- (0,1,0) -- (0,1,0.5) -- (0,2,0) -- (0,5,0) -- (0,5,3) -- (1,5,3) -- (1,4,3) -- (1,4,2.5) -- (1,3,3) -- (2,5,3) -- (3,5,3) -- (3,0,3) -- cycle;
\clip (0,0,0) -- (0,0,3) -- ({-3*sin(#2)/cos(#2)},0,0) -- cycle;
\shade[left color=darkgray,right color=lightgray!50!gray] (0,0,0) -- (0,1,0) -- (0,1,0.5) -- (0,2,0) -- (0,5,0) -- (0,5,3) -- (1,5,3) -- (1,4,3) -- (1,4,2.5) -- (1,3,3) -- (2,5,3) -- (3,5,3) -- (3,0,3) -- cycle;
\end{scope}
\shade[left color=darkgray,right color=darkgray!80!black] (2,0,0) -- (2,0,3) -- (2,5,3) -- (2,5,1) -- (2,4,1) -- (2,4,0) -- cycle;
\shade[left color=darkgray!90!black,right color=gray!80!darkgray] (2,0,2) -- (0,0,2) -- (0,2,2) -- (2,2,2) -- cycle;
\shade[top color=darkgray!90!black,bottom color=gray!80!darkgray] (0,1,0) -- (2,1,0) -- (2,1,2) -- (0,1,2) -- cycle;
\shade[top color=darkgray!90!black,bottom color=gray!80!darkgray] (0,3,1) -- (0,3,3) -- (2,3,3) -- (2,3,0) -- (1,3,0) -- (1,3,1) -- cycle;
\fill[gray] (2,1,0) -- (1.5,1,0.5) -- (0,1,0.5) -- (0,1,0) -- cycle;
\fill[gray] (1,3,2) -- (0.5,3,2) -- (0.5,3,3) -- (1,3,3) -- cycle;
\fill[gray] (2,3,0) -- (1.5,3,0.5) -- (1,3,0.5) -- (1,3,0) -- cycle;
\ecslogosurface
\end{tikzpicture}
}

\providecommand{\cpplogo}[1]{
\begin{tikzpicture}[scale=(#1)/512em]
\fill[gray] (435.2794,398.7159) -- (247.1911,507.3075) .. controls (236.3563,513.5642) and (218.6240,513.5642) .. (207.7892,507.3075) -- (19.7009,398.7159) .. controls (8.8646,392.4606) and (0.0000,377.1043) .. (0.0000,364.5924) -- (0.0000,147.4076) .. controls (0.8430,132.8363) and (8.2856,120.7683) .. (19.7009,113.2842) -- (207.7892,4.6926) .. controls (218.6240,-1.5642) and (236.3564,-1.5642) .. (247.1911,4.6926) -- (435.2794,113.2842) .. controls (447.5273,121.4304) and (454.4987,133.6918) .. (454.9803,147.4076) -- (454.9803,364.5924) .. controls (454.5404,377.7571) and (446.6566,391.0351) .. (435.2794,398.7159) -- cycle(75.8301,255.9993) .. controls (74.9389,404.0881) and (273.2892,469.4783) .. (358.8263,331.8769) -- (293.1917,293.8965) .. controls (253.5702,359.4301) and (155.1909,335.9977) .. (151.6601,255.9993) .. controls (152.7204,182.2703) and (249.4137,148.0211) .. (293.1961,218.1065) -- (358.8308,180.1276) .. controls (283.4477,49.2645) and (79.6318,96.3470) .. (75.8301,255.9993) -- cycle(379.1503,247.5747) -- (362.2982,247.5747) -- (362.2982,230.7226) -- (345.4490,230.7226) -- (345.4490,247.5747) -- (328.5969,247.5747) -- (328.5969,264.4254) -- (345.4490,264.4254) -- (345.4490,281.2759) -- (362.2982,281.2759) -- (362.2982,264.4254) -- (379.1503,264.4254) -- cycle(442.3420,247.5747) -- (425.4899,247.5747) -- (425.4899,230.7226) -- (408.6408,230.7226) -- (408.6408,247.5747) -- (391.7886,247.5747) -- (391.7886,264.4254) -- (408.6408,264.4254) -- (408.6408,281.2759) -- (425.4899,281.2759) -- (425.4899,264.4254) -- (442.3420,264.4254) -- cycle;
\end{tikzpicture}
}

\providecommand{\fallogo}[1]{
\begin{tikzpicture}[scale=(#1)/512em]
\fill[gray] (185.7774,0.0000) .. controls (200.4486,15.9798) and (226.8966,8.7148) .. (235.0426,31.5836) .. controls (249.5297,58.0598) and (247.9581,97.9161) .. (280.3335,110.9762) .. controls (309.1690,120.3496) and (337.8406,104.2727) .. (366.5753,103.9379) .. controls (373.4449,111.5171) and (379.2885,128.2574) .. (383.9755,108.9744) .. controls (396.6979,102.5615) and (437.2808,107.6681) .. (426.9652,124.3252) .. controls (408.9822,121.0785) and (412.4742,146.0729) .. (426.5192,131.4996) .. controls (433.8413,120.8489) and (465.1541,126.5522) .. (441.9067,135.7950) .. controls (396.1879,157.7478) and (344.1112,161.5079) .. (298.5528,183.5702) .. controls (277.7471,193.5198) and (284.6941,218.7163) .. (285.2127,236.9640) .. controls (292.3599,316.2826) and (307.3929,394.6311) .. (317.1198,473.6154) .. controls (329.0637,505.4736) and (292.1195,528.5004) .. (265.9183,511.2761) .. controls (237.9284,499.2462) and (237.3684,465.2681) .. (230.9102,439.9421) .. controls (218.6692,374.3397) and (215.6307,306.9662) .. (198.1732,242.3977) .. controls (183.1379,232.7444) and (164.4245,256.0298) .. (149.0430,261.4799) .. controls (116.9328,279.2585) and (87.1822,308.5851) .. (48.2293,307.8914) .. controls (21.3220,306.9037) and (-15.9107,281.8761) .. (7.2921,252.7908) .. controls (29.7799,220.6177) and (67.5177,204.3028) .. (100.9287,185.9449) .. controls (130.8217,170.8906) and (161.1548,156.5903) .. (191.0278,141.5847) .. controls (196.1738,120.0520) and (186.6049,95.2409) .. (186.8382,72.4353) .. controls (185.5234,48.4204) and (183.1700,23.9341) .. (185.7774,0.0000) -- cycle;
\end{tikzpicture}
}

\providecommand{\oblogo}[1]{
\begin{tikzpicture}[scale=(#1)/512em]
\fill[gray] (160.3865,208.9117) .. controls (154.0879,214.6478) and (149.0735,221.2409) .. (145.4125,228.5384) .. controls (184.8790,248.4273) and (234.7122,269.8787) .. (297.5493,291.8782) .. controls (300.3943,281.4769) and (300.9552,268.7619) .. (300.4023,255.2389) .. controls (248.9909,244.7891) and (200.0310,225.9279) .. (160.3865,208.9117) -- cycle(225.7398,392.6996) .. controls (308.0209,392.1716) and (359.3326,345.9277) .. (368.7203,285.2098) .. controls (376.6742,197.1784) and (311.7194,141.3342) .. (205.4287,142.1456) .. controls (139.9485,141.4804) and (88.7155,166.1957) .. (73.5775,228.0086) .. controls (52.0297,320.3408) and (123.4078,391.0103) .. (225.7398,392.6996) -- cycle(216.0739,176.4733) .. controls (268.9183,179.2424) and (315.8292,206.5488) .. (312.7454,265.1139) .. controls (313.2769,315.6384) and (286.5993,353.4946) .. (216.6040,355.7934) .. controls (162.4657,355.7934) and (126.0914,317.5023) .. (126.0914,260.5103) .. controls (126.1733,214.2900) and (163.3363,176.2849) .. (216.0739,176.4733) -- cycle(76.4897,189.1754) .. controls (13.1586,147.5631) and (0.0000,119.4207) .. (0.0000,119.4207) -- (90.6499,170.1632) .. controls (85.3004,175.8497) and (80.5994,182.1633) .. (76.4897,189.1754) -- cycle(353.9486,119.3004) -- (402.9482,119.3004) .. controls (427.0025,137.0797) and (450.9893,162.7034) .. (474.9529,191.0213) .. controls (509.3540,228.5339) and (531.3391,294.2091) .. (487.8149,312.1206) .. controls (462.8165,324.7652) and (394.3874,316.8943) .. (373.8912,313.6651) .. controls (379.9291,297.7449) and (383.2899,278.4204) .. (381.4989,257.7214) .. controls (420.3069,248.0321) and (421.9610,218.3461) .. (407.7867,192.6417) .. controls (391.1113,162.4018) and (370.1114,132.9097) .. (353.9486,119.3004) -- cycle;
\end{tikzpicture}
}

\providecommand{\markuptable}{
\begin{table}
\sffamily\centering
\begin{tabular}{@{}lcl@{}}
\toprule
\texttt{//italics//} & $\rightarrow$ & \textit{italics} \\
\midrule
\texttt{**bold**} & $\rightarrow$ & \textbf{bold} \\
\midrule
\texttt{\# ordered list} & & 1 ordered list \\
\texttt{\# second item} & $\rightarrow$ & 2 second item \\
\texttt{\#\# sub item} & & \hspace{1em} 1 sub item \\
\midrule
\texttt{* unordered list} & & $\bullet$ unordered list \\
\texttt{* second item} & $\rightarrow$ & $\bullet$ second item \\
\texttt{** sub item} & & \hspace{1em} $\bullet$ sub item \\
\midrule
\texttt{link to [[label]]} & $\rightarrow$ & link to \underline{label} \\
\midrule
\texttt{<{}<label>{}> definition } & $\rightarrow$ & definition \\
\midrule
\texttt{[[url|link name]]} & $\rightarrow$ & \underline{link name} \\
\midrule\addlinespace
\texttt{= large heading} & & {\Large large heading} \smallskip \\
\texttt{== medium heading} & $\rightarrow$ & {\large medium heading} \\
\texttt{=== small heading} & & small heading \\
\midrule
\texttt{no line break} & & no line break for paragraphs \\
\texttt{for paragraphs} & $\rightarrow$ \\
& & use empty line \\
\texttt{use empty line} \\
\midrule
\texttt{force\textbackslash\textbackslash line break} & $\rightarrow$ & force \\
& & line break \\
\midrule
\texttt{horizontal line} & $\rightarrow$ & horizontal line \\
\texttt{----} & & \hrulefill \\
\midrule
\texttt{|=a|=table|=header} & & \underline{a \enspace table \enspace header} \\
\texttt{|a|table|row} & $\rightarrow$ & a \enspace table \enspace row \\
\texttt{|b|table|row} & & b \enspace table \enspace row \\
\midrule
\texttt{\{\{\{} \\
\texttt{unformatted} & $\rightarrow$ & \texttt{unformatted} \\
\texttt{code} & & \texttt{code} \\
\texttt{\}\}\}} \\
\midrule\addlinespace
\texttt{@ new article} & & {\Large 1.\ new article} \smallskip \\
\texttt{@ second article} & $\rightarrow$ & {\Large 2.\ second article} \smallskip \\
\texttt{@@ sub article} & & {\large 2.1.\ sub article} \\
\bottomrule
\end{tabular}
\normalfont\caption{Elements of the generic documentation markup language}
\label{tab:docmarkup}
\end{table}
}

\providecommand{\startchapter}[4]{
\documentclass[11pt,a4paper]{article}
\usepackage{booktabs}
\usepackage[format=hang,labelfont=bf]{caption}
\usepackage{changepage}
\usepackage[T1]{fontenc}
\usepackage[margin=2cm]{geometry}
\usepackage{hyperref}
\usepackage[american]{isodate}
\usepackage{lmodern}
\usepackage{longtable}
\usepackage{mathptmx}
\usepackage{microtype}
\usepackage[toc]{multitoc}
\usepackage{multirow}
\usepackage[all]{nowidow}
\usepackage{pdfcomment}
\usepackage{syntax}
\usepackage{tikz}
\usepackage[all]{xy}
\hypersetup{pdfborder={0 0 0},bookmarksnumbered=true,pdftitle={\ecs{}: #2},pdfauthor={Florian Negele},pdfsubject={\ecs{}},pdfkeywords={#1}}
\setlength{\grammarindent}{8em}\setlength{\grammarparsep}{0.2ex}
\setlength{\columnsep}{2em}
\newcommand{\prefix}{}
\newcounter{instruction}
\bibliographystyle{unsrt}
\renewcommand{\index}[2][]{}
\renewcommand{\arraystretch}{1.05}
\renewcommand{\floatpagefraction}{0.7}
\renewcommand{\syntleft}{\itshape}\renewcommand{\syntright}{}
\title{\vspace{-5ex}\Huge{\ecs{}}\medskip\hrule}
\author{\huge{#2}}
\date{\medskip\version}
\newif\ifbook\bookfalse
\pagestyle{headings}
\frenchspacing
\begin{document}
\maketitle\thispagestyle{empty}\noindent#4\setlength{\columnseprule}{0.4pt}\tableofcontents\setlength{\columnseprule}{0pt}\vfill\pagebreak[3]\null\vfill\bigskip\noindent
\parbox{\textwidth-4em}{\license The contents of this \documentation{} are part of the \href{manual}{\ecs{} User Manual}~\cite{manual} and correspond to Chapter ``\href{manual\##3}{#1}''.\alignright\mbox{\today}}
\parbox{4em}{\flushright\ecslogo{3em}}
\clearpage
}

\providecommand{\concludechapter}{
\vfill\pagebreak[3]\null\vfill
\thispagestyle{myheadings}\markright{REFERENCES}
\noindent\begin{minipage}{\textwidth}\begin{multicols}{2}[\section*{References}]
\renewcommand{\section}[2]{}\small\bibliography{references}
\end{multicols}\end{minipage}\end{document}
}

\providecommand{\startpresentation}[2]{
\documentclass[14pt,aspectratio=43,usepdftitle=false]{beamer}
\usepackage{booktabs}
\usepackage{etex}
\usepackage{multicol}
\usepackage{tikz}
\usepackage[all]{xy}
\bibliographystyle{unsrt}
\setlength{\columnsep}{1em}
\setlength{\leftmargini}{1em}
\setbeamercolor{title}{fg=black}
\setbeamercolor{structure}{fg=darkgray}
\setbeamercolor{bibliography item}{fg=darkgray}
\setbeamerfont{title}{series=\bfseries}
\setbeamerfont{subtitle}{series=\normalfont}
\setbeamerfont*{frametitle}{parent=title}
\setbeamerfont{block title}{series=\bfseries}
\setbeamerfont*{framesubtitle}{parent=subtitle}
\setbeamersize{text margin left=1em,text margin right=1em}
\setbeamertemplate{navigation symbols}{}
\setbeamertemplate{itemize item}[circle]{}
\setbeamertemplate{bibliography item}[triangle]{}
\setbeamertemplate{bibliography entry author}{\usebeamercolor[fg]{bibliography item}}
\setbeamertemplate{frametitle}{\medskip\usebeamerfont{frametitle}\color{gray}\raisebox{-2.5ex}[0ex][0ex]{\rule{0.1em}{4.5ex}}}
\addtobeamertemplate{frametitle}{}{\hspace{0.4em}\usebeamercolor[fg]{title}\insertframetitle\par\vspace{0.2ex}\hspace{0.5em}\usebeamerfont{framesubtitle}\insertframesubtitle}
\hypersetup{pdfborder={0 0 0},bookmarksnumbered=true,bookmarksopen=true,bookmarksopenlevel=0,pdftitle={\ecs{}: #1},pdfauthor={Florian Negele},pdfsubject={\ecs{}},pdfkeywords={#1}}
\renewcommand{\flowgraph}[1]{\resizebox{\textwidth}{!}{$$\xymatrix{##1}$$}}
\title{\ecs{}\medskip\hrule\medskip}
\institute{\shadowedecslogo{5em}{30}{15}}
\date{\version}
\subtitle{#1}
\begin{document}
\begin{frame}[plain]\titlepage\nocite{manual}\end{frame}
\begin{frame}{Contents}{#1}\begin{center}\tableofcontents\end{center}\end{frame}
}

\providecommand{\concludepresentation}{
\begin{frame}{References}\begin{footnotesize}\setlength{\columnseprule}{0.4pt}\begin{multicols}{2}\bibliography{references}\end{multicols}\end{footnotesize}\end{frame}
\end{document}
}

\providecommand{\startbook}[1]{
\documentclass[10pt,paper=17cm:24cm,DIV=13,twoside=semi,headings=normal,numbers=noendperiod,cleardoublepage=plain]{scrbook}
\usepackage{atveryend}
\usepackage{booktabs}
\usepackage{caption}
\usepackage{changepage}
\usepackage[T1]{fontenc}
\usepackage{imakeidx}
\usepackage{hyperref}
\usepackage[american]{isodate}
\usepackage{lmodern}
\usepackage{longtable}
\usepackage{mathptmx}
\usepackage[final]{microtype}
\usepackage{multicol}
\usepackage{multirow}
\usepackage[all]{nowidow}
\usepackage{pdfcomment}
\usepackage{scrlayer-scrpage}
\usepackage{setspace}
\usepackage{syntax}
\usepackage[eventxtindent=4pt,oddtxtexdent=4pt]{thumbs}
\usepackage{tikz}
\usepackage[all]{xy}
\hyphenation{Micro-Blaze Open-Cores Open-RISC Power-PC}
\hypersetup{pdfborder={0 0 0},bookmarksnumbered=true,bookmarksopen=true,bookmarksopenlevel=0,pdftitle={\ecs{}: #1},pdfauthor={Florian Negele},pdfsubject={\ecs{}},pdfkeywords={#1}}
\setlength{\grammarindent}{8em}\setlength{\grammarparsep}{0.7ex}
\setkomafont{captionlabel}{\usekomafont{descriptionlabel}}
\renewcommand{\arraystretch}{1.05}\setstretch{1.1}
\renewcommand{\chapterformat}{\thechapter\autodot\enskip\raisebox{-1ex}[0ex][0ex]{\color{gray}\rule{0.1em}{3.5ex}}\enskip}
\renewcommand{\startchapter}[4]{\hypertarget{##3}{\chapter{##1}}\label{##3}##4\addthumb{##1}{\LARGE\sffamily\bfseries\thechapter}{white}{gray}\renewcommand{\prefix}{##3}}
\renewcommand{\concludechapter}{\clearpage{\stopthumb\cleardoublepage}}
\renewcommand{\syntleft}{\itshape}\renewcommand{\syntright}{}
\renewcommand{\floatpagefraction}{0.7}
\renewcommand{\partheademptypage}{}
\DeclareMicrotypeAlias{lmss}{cmr}
\newcommand{\prefix}{}
\newcounter{instruction}
\bibliographystyle{unsrt}
\newif\ifbook\booktrue
\makeindex[intoc,title=Index]
\makeindex[intoc,name=tools,title=Index of Tools,columns=3]
\makeindex[intoc,name=library,title=Index of Library Names]
\makeindex[intoc,name=runtime,title=Index of Runtime Support]
\makeindex[intoc,name=environment,title=Index of Target Environments]
\indexsetup{toclevel=chapter,headers={\indexname}{\indexname}}
\frenchspacing
\begin{document}
\pagenumbering{alph}
\begin{titlepage}\centering
\huge\sffamily\null\vfill\textbf{\ecs{}}\bigskip\hrule\bigskip#1
\normalsize\normalfont\vfill\vfill\shadowedecslogo{10em}{30}{15}
\large\vfill\vfill\version
\end{titlepage}
\null\vfill
\thispagestyle{empty}
\noindent\today\par\medskip
\license A copy of this license is included in Appendix~\ref{fdl} on page~\pageref{fdl}.
All product names used herein are for identification purposes only and may be trademarks of their respective companies.
\concludechapter
\frontmatter
\setcounter{tocdepth}{1}
\tableofcontents
\setcounter{tocdepth}{2}
\concludechapter
\listoffigures
\concludechapter
\listoftables
\concludechapter
}

\providecommand{\concludebook}{
\backmatter
\addtocontents{toc}{\protect\setcounter{tocdepth}{-1}}
\phantomsection\addcontentsline{toc}{part}{Bibliography}
\bibliography{references}
\concludechapter
\phantomsection\addcontentsline{toc}{part}{Indexes}
\printindex
\concludechapter
\indexprologue{\label{idx:tools}}
\printindex[tools]
\concludechapter
\printindex[library]
\concludechapter
\indexprologue{\label{idx:runtime}}
\printindex[runtime]
\concludechapter
\indexprologue{\label{idx:environment}}
\printindex[environment]
\concludechapter
\pagestyle{empty}\pagenumbering{Alph}\null\clearpage
\null\vfill\centering\ecslogo{4em}\par\medskip\license
\end{document}
}

% chapter references

\providecommand{\seedocumentationref}{}\renewcommand{\seedocumentationref}[3]{#1, see \Documentation{}~\documentationref{#2}{#3}. }
\providecommand{\seeinterface}{}\renewcommand{\seeinterface}{\ifbook See \Documentation{}~\documentationref{interface}{User Interface} for more information about the common user interface of all of these tools. \fi}
\providecommand{\seeguide}{}\renewcommand{\seeguide}{\seedocumentationref{For basic examples of using some of these tools in practice}{guide}{User Guide}}
\providecommand{\seecpp}{}\renewcommand{\seecpp}{\seedocumentationref{For more information about the \cpp{} programming language and its implementation by the \ecs{}}{cpp}{User Manual for \cpp{}}}
\providecommand{\seefalse}{}\renewcommand{\seefalse}{\seedocumentationref{For more information about the FALSE programming language and its implementation by the \ecs{}}{false}{User Manual for FALSE}}
\providecommand{\seeoberon}{}\renewcommand{\seeoberon}{\seedocumentationref{For more information about the Oberon programming language and its implementation by the \ecs{}}{oberon}{User Manual for Oberon}}
\providecommand{\seeassembly}{}\renewcommand{\seeassembly}{\seedocumentationref{For more information about the generic assembly language and how to use it}{assembly}{Generic Assembly Language Specification}}
\providecommand{\seeamd}{}\renewcommand{\seeamd}{\seedocumentationref{For more information about how the \ecs{} supports the AMD64 hardware architecture}{amd64}{AMD64 Hardware Architecture Support}}
\providecommand{\seearm}{}\renewcommand{\seearm}{\seedocumentationref{For more information about how the \ecs{} supports the ARM hardware architecture}{arm}{ARM Hardware Architecture Support}}
\providecommand{\seeavr}{}\renewcommand{\seeavr}{\seedocumentationref{For more information about how the \ecs{} supports the AVR hardware architecture}{avr}{AVR Hardware Architecture Support}}
\providecommand{\seeavrtt}{}\renewcommand{\seeavrtt}{\seedocumentationref{For more information about how the \ecs{} supports the AVR32 hardware architecture}{avr32}{AVR32 Hardware Architecture Support}}
\providecommand{\seemabk}{}\renewcommand{\seemabk}{\seedocumentationref{For more information about how the \ecs{} supports the M68000 hardware architecture}{m68k}{M68000 Hardware Architecture Support}}
\providecommand{\seemibl}{}\renewcommand{\seemibl}{\seedocumentationref{For more information about how the \ecs{} supports the MicroBlaze hardware architecture}{mibl}{MicroBlaze Hardware Architecture Support}}
\providecommand{\seemips}{}\renewcommand{\seemips}{\seedocumentationref{For more information about how the \ecs{} supports the MIPS32 and MIPS64 hardware architectures}{mips}{MIPS Hardware Architecture Support}}
\providecommand{\seemmix}{}\renewcommand{\seemmix}{\seedocumentationref{For more information about how the \ecs{} supports the MMIX hardware architecture}{mmix}{MMIX Hardware Architecture Support}}
\providecommand{\seeorok}{}\renewcommand{\seeorok}{\seedocumentationref{For more information about how the \ecs{} supports the OpenRISC 1000 hardware architecture}{or1k}{OpenRISC 1000 Hardware Architecture Support}}
\providecommand{\seeppc}{}\renewcommand{\seeppc}{\seedocumentationref{For more information about how the \ecs{} supports the PowerPC hardware architecture}{ppc}{PowerPC Hardware Architecture Support}}
\providecommand{\seerisc}{}\renewcommand{\seerisc}{\seedocumentationref{For more information about how the \ecs{} supports the RISC hardware architecture}{risc}{RISC Hardware Architecture Support}}
\providecommand{\seewasm}{}\renewcommand{\seewasm}{\seedocumentationref{For more information about how the \ecs{} supports the WebAssembly architecture}{wasm}{WebAssembly Architecture Support}}
\providecommand{\seedocumentation}{}\renewcommand{\seedocumentation}{\seedocumentationref{For more information about generic documentations and their generation by the \ecs{}}{documentation}{Generic Documentation Generation}}
\providecommand{\seedebugging}{}\renewcommand{\seedebugging}{\seedocumentationref{For more information about debugging information and its representation}{debugging}{Debugging Information Representation}}
\providecommand{\seecode}{}\renewcommand{\seecode}{\seedocumentationref{For more information about intermediate code and its purpose}{code}{Intermediate Code Representation}}
\providecommand{\seeobject}{}\renewcommand{\seeobject}{\seedocumentationref{For more information about object files and their purpose}{object}{Object File Representation}}

% generic documentation tools

\providecommand{\docprint}{
\toolsection{docprint} is a pretty printer for generic documentations.
It reformats generic documentations and writes it to the standard output stream.
\debuggingtool
\flowgraph{\resource{generic\\documentation} \ar[r] & \toolbox{docprint} \ar[r] & \resource{generic\\documentation}}
\seedocumentation
}

\providecommand{\doccheck}{
\toolsection{doccheck} is a syntactic and semantic checker for generic documentations.
It just performs syntactic and semantic checks on generic documentations and writes its diagnostic messages to the standard error stream.
\debuggingtool
\flowgraph{\resource{generic\\documentation} \ar[r] & \toolbox{doccheck} \ar[r] & \resource{diagnostic\\messages}}
\seedocumentation
}

\providecommand{\dochtml}{
\toolsection{dochtml} is an HTML documentation generator for generic documentations.
It processes several generic documentations and assembles all information therein into an HTML document.
\debuggingtool
\flowgraph{\resource{generic\\documentation} \ar[r] & \toolbox{dochtml} \ar[r] & \resource{HTML\\document}}
\seedocumentation
}

\providecommand{\doclatex}{
\toolsection{doclatex} is a Latex documentation generator for generic documentations.
It processes several generic documentations and assembles all information therein into a Latex document.
\debuggingtool
\flowgraph{\resource{generic\\documentation} \ar[r] & \toolbox{doclatex} \ar[r] & \resource{Latex\\document}}
\seedocumentation
}

% intermediate code tools

\providecommand{\cdcheck}{
\toolsection{cdcheck} is a syntactic and semantic checker for intermediate code.
It just performs syntactic and semantic checks on programs written in intermediate code and writes its diagnostic messages to the standard error stream.
\debuggingtool
\flowgraph{\resource{intermediate\\code} \ar[r] & \toolbox{cdcheck} \ar[r] & \resource{diagnostic\\messages}}
\seeassembly\seecode
}

\providecommand{\cdopt}{
\toolsection{cdopt} is an optimizer for intermediate code.
It performs various optimizations on programs written in intermediate code and writes the result to the standard output stream.
\debuggingtool
\flowgraph{\resource{intermediate\\code} \ar[r] & \toolbox{cdopt} \ar[r] & \resource{optimized\\code}}
\seeassembly\seecode
}

\providecommand{\cdrun}{
\toolsection{cdrun} is an interpreter for intermediate code.
It processes and executes programs written in intermediate code.
The following code sections are predefined and have the usual semantics:
\texttt{abort}, \texttt{\_Exit}, \texttt{fflush}, \texttt{floor}, \texttt{fputc}, \texttt{free}, \texttt{getchar}, \texttt{malloc}, and \texttt{putchar}.
Diagnostic messages about invalid operations include the name of the executed code section and the index of the erroneous instruction.
\debuggingtool
\flowgraph{\resource{intermediate\\code} \ar[r] & \toolbox{cdrun} \ar@/u/[r] & \resource{input/\\output} \ar@/d/[l]}
\seeassembly\seecode
}

\providecommand{\cdamda}{
\toolsection{cdamd16} is a compiler for intermediate code targeting the AMD64 hardware architecture.
It generates machine code for AMD64 processors from programs written in intermediate code and stores it in corresponding object files.
The compiler generates machine code for the 16-bit operating mode defined by the AMD64 architecture.
It also creates a debugging information file as well as an assembly file containing a listing of the generated machine code.
\debuggingtool
\flowgraph{\resource{intermediate\\code} \ar[r] & \toolbox{cdamd16} \ar[r] \ar[d] \ar[rd] & \resource{object file} \\ & \resource{assembly\\listing} & \resource{debugging\\information}}
\seeassembly\seeamd\seeobject\seecode\seedebugging
}

\providecommand{\cdamdb}{
\toolsection{cdamd32} is a compiler for intermediate code targeting the AMD64 hardware architecture.
It generates machine code for AMD64 processors from programs written in intermediate code and stores it in corresponding object files.
The compiler generates machine code for the 32-bit operating mode defined by the AMD64 architecture.
It also creates a debugging information file as well as an assembly file containing a listing of the generated machine code.
\debuggingtool
\flowgraph{\resource{intermediate\\code} \ar[r] & \toolbox{cdamd32} \ar[r] \ar[d] \ar[rd] & \resource{object file} \\ & \resource{assembly\\listing} & \resource{debugging\\information}}
\seeassembly\seeamd\seeobject\seecode\seedebugging
}

\providecommand{\cdamdc}{
\toolsection{cdamd64} is a compiler for intermediate code targeting the AMD64 hardware architecture.
It generates machine code for AMD64 processors from programs written in intermediate code and stores it in corresponding object files.
The compiler generates machine code for the 64-bit operating mode defined by the AMD64 architecture.
It also creates a debugging information file as well as an assembly file containing a listing of the generated machine code.
\debuggingtool
\flowgraph{\resource{intermediate\\code} \ar[r] & \toolbox{cdamd64} \ar[r] \ar[d] \ar[rd] & \resource{object file} \\ & \resource{assembly\\listing} & \resource{debugging\\information}}
\seeassembly\seeamd\seeobject\seecode\seedebugging
}

\providecommand{\cdarma}{
\toolsection{cdarma32} is a compiler for intermediate code targeting the ARM hardware architecture.
It generates machine code for ARM processors executing A32 instructions from programs written in intermediate code and stores it in corresponding object files.
It also creates a debugging information file as well as an assembly file containing a listing of the generated machine code.
\debuggingtool
\flowgraph{\resource{intermediate\\code} \ar[r] & \toolbox{cdarma32} \ar[r] \ar[d] \ar[rd] & \resource{object file} \\ & \resource{assembly\\listing} & \resource{debugging\\information}}
\seeassembly\seearm\seeobject\seecode\seedebugging
}

\providecommand{\cdarmb}{
\toolsection{cdarma64} is a compiler for intermediate code targeting the ARM hardware architecture.
It generates machine code for ARM processors executing A64 instructions from programs written in intermediate code and stores it in corresponding object files.
It also creates a debugging information file as well as an assembly file containing a listing of the generated machine code.
\debuggingtool
\flowgraph{\resource{intermediate\\code} \ar[r] & \toolbox{cdarma64} \ar[r] \ar[d] \ar[rd] & \resource{object file} \\ & \resource{assembly\\listing} & \resource{debugging\\information}}
\seeassembly\seearm\seeobject\seecode\seedebugging
}

\providecommand{\cdarmc}{
\toolsection{cdarmt32} is a compiler for intermediate code targeting the ARM hardware architecture.
It generates machine code for ARM processors without floating-point extension executing T32 instructions from programs written in intermediate code and stores it in corresponding object files.
It also creates a debugging information file as well as an assembly file containing a listing of the generated machine code.
\debuggingtool
\flowgraph{\resource{intermediate\\code} \ar[r] & \toolbox{cdarmt32} \ar[r] \ar[d] \ar[rd] & \resource{object file} \\ & \resource{assembly\\listing} & \resource{debugging\\information}}
\seeassembly\seearm\seeobject\seecode\seedebugging
}

\providecommand{\cdarmcfpe}{
\toolsection{cdarmt32fpe} is a compiler for intermediate code targeting the ARM hardware architecture.
It generates machine code for ARM processors with floating-point extension executing T32 instructions from programs written in intermediate code and stores it in corresponding object files.
It also creates a debugging information file as well as an assembly file containing a listing of the generated machine code.
\debuggingtool
\flowgraph{\resource{intermediate\\code} \ar[r] & \toolbox{cdarmt32fpe} \ar[r] \ar[d] \ar[rd] & \resource{object file} \\ & \resource{assembly\\listing} & \resource{debugging\\information}}
\seeassembly\seearm\seeobject\seecode\seedebugging
}

\providecommand{\cdavr}{
\toolsection{cdavr} is a compiler for intermediate code targeting the AVR hardware architecture.
It generates machine code for AVR processors from programs written in intermediate code and stores it in corresponding object files.
It also creates a debugging information file as well as an assembly file containing a listing of the generated machine code.
\debuggingtool
\flowgraph{\resource{intermediate\\code} \ar[r] & \toolbox{cdavr} \ar[r] \ar[d] \ar[rd] & \resource{object file} \\ & \resource{assembly\\listing} & \resource{debugging\\information}}
\seeassembly\seeavr\seeobject\seecode\seedebugging
}

\providecommand{\cdavrtt}{
\toolsection{cdavr32} is a compiler for intermediate code targeting the AVR32 hardware architecture.
It generates machine code for AVR32 processors from programs written in intermediate code and stores it in corresponding object files.
It also creates a debugging information file as well as an assembly file containing a listing of the generated machine code.
\debuggingtool
\flowgraph{\resource{intermediate\\code} \ar[r] & \toolbox{cdavr32} \ar[r] \ar[d] \ar[rd] & \resource{object file} \\ & \resource{assembly\\listing} & \resource{debugging\\information}}
\seeassembly\seeavrtt\seeobject\seecode\seedebugging
}

\providecommand{\cdmabk}{
\toolsection{cdm68k} is a compiler for intermediate code targeting the M68000 hardware architecture.
It generates machine code for M68000 processors from programs written in intermediate code and stores it in corresponding object files.
It also creates a debugging information file as well as an assembly file containing a listing of the generated machine code.
\debuggingtool
\flowgraph{\resource{intermediate\\code} \ar[r] & \toolbox{cdm68k} \ar[r] \ar[d] \ar[rd] & \resource{object file} \\ & \resource{assembly\\listing} & \resource{debugging\\information}}
\seeassembly\seemabk\seeobject\seecode\seedebugging
}

\providecommand{\cdmibl}{
\toolsection{cdmibl} is a compiler for intermediate code targeting the MicroBlaze hardware architecture.
It generates machine code for MicroBlaze processors from programs written in intermediate code and stores it in corresponding object files.
It also creates a debugging information file as well as an assembly file containing a listing of the generated machine code.
\debuggingtool
\flowgraph{\resource{intermediate\\code} \ar[r] & \toolbox{cdmibl} \ar[r] \ar[d] \ar[rd] & \resource{object file} \\ & \resource{assembly\\listing} & \resource{debugging\\information}}
\seeassembly\seemibl\seeobject\seecode\seedebugging
}

\providecommand{\cdmipsa}{
\toolsection{cdmips32} is a compiler for intermediate code targeting the MIPS32 hardware architecture.
It generates machine code for MIPS32 processors from programs written in intermediate code and stores it in corresponding object files.
It also creates a debugging information file as well as an assembly file containing a listing of the generated machine code.
\debuggingtool
\flowgraph{\resource{intermediate\\code} \ar[r] & \toolbox{cdmips32} \ar[r] \ar[d] \ar[rd] & \resource{object file} \\ & \resource{assembly\\listing} & \resource{debugging\\information}}
\seeassembly\seemips\seeobject\seecode\seedebugging
}

\providecommand{\cdmipsb}{
\toolsection{cdmips64} is a compiler for intermediate code targeting the MIPS64 hardware architecture.
It generates machine code for MIPS64 processors from programs written in intermediate code and stores it in corresponding object files.
It also creates a debugging information file as well as an assembly file containing a listing of the generated machine code.
\debuggingtool
\flowgraph{\resource{intermediate\\code} \ar[r] & \toolbox{cdmips64} \ar[r] \ar[d] \ar[rd] & \resource{object file} \\ & \resource{assembly\\listing} & \resource{debugging\\information}}
\seeassembly\seemips\seeobject\seecode\seedebugging
}

\providecommand{\cdmmix}{
\toolsection{cdmmix} is a compiler for intermediate code targeting the MMIX hardware architecture.
It generates machine code for MMIX processors from programs written in intermediate code and stores it in corresponding object files.
It also creates a debugging information file as well as an assembly file containing a listing of the generated machine code.
\debuggingtool
\flowgraph{\resource{intermediate\\code} \ar[r] & \toolbox{cdmmix} \ar[r] \ar[d] \ar[rd] & \resource{object file} \\ & \resource{assembly\\listing} & \resource{debugging\\information}}
\seeassembly\seemmix\seeobject\seecode\seedebugging
}

\providecommand{\cdorok}{
\toolsection{cdor1k} is a compiler for intermediate code targeting the OpenRISC 1000 hardware architecture.
It generates machine code for OpenRISC 1000 processors from programs written in intermediate code and stores it in corresponding object files.
It also creates a debugging information file as well as an assembly file containing a listing of the generated machine code.
\debuggingtool
\flowgraph{\resource{intermediate\\code} \ar[r] & \toolbox{cdor1k} \ar[r] \ar[d] \ar[rd] & \resource{object file} \\ & \resource{assembly\\listing} & \resource{debugging\\information}}
\seeassembly\seeorok\seeobject\seecode\seedebugging
}

\providecommand{\cdppca}{
\toolsection{cdppc32} is a compiler for intermediate code targeting the PowerPC hardware architecture.
It generates machine code for PowerPC processors from programs written in intermediate code and stores it in corresponding object files.
The compiler generates machine code for the 32-bit operating mode defined by the PowerPC architecture.
It also creates a debugging information file as well as an assembly file containing a listing of the generated machine code.
\debuggingtool
\flowgraph{\resource{intermediate\\code} \ar[r] & \toolbox{cdppc32} \ar[r] \ar[d] \ar[rd] & \resource{object file} \\ & \resource{assembly\\listing} & \resource{debugging\\information}}
\seeassembly\seeppc\seeobject\seecode\seedebugging
}

\providecommand{\cdppcb}{
\toolsection{cdppc64} is a compiler for intermediate code targeting the PowerPC hardware architecture.
It generates machine code for PowerPC processors from programs written in intermediate code and stores it in corresponding object files.
The compiler generates machine code for the 64-bit operating mode defined by the PowerPC architecture.
It also creates a debugging information file as well as an assembly file containing a listing of the generated machine code.
\debuggingtool
\flowgraph{\resource{intermediate\\code} \ar[r] & \toolbox{cdppc64} \ar[r] \ar[d] \ar[rd] & \resource{object file} \\ & \resource{assembly\\listing} & \resource{debugging\\information}}
\seeassembly\seeppc\seeobject\seecode\seedebugging
}

\providecommand{\cdrisc}{
\toolsection{cdrisc} is a compiler for intermediate code targeting the RISC hardware architecture.
It generates machine code for RISC processors from programs written in intermediate code and stores it in corresponding object files.
It also creates a debugging information file as well as an assembly file containing a listing of the generated machine code.
\debuggingtool
\flowgraph{\resource{intermediate\\code} \ar[r] & \toolbox{cdrisc} \ar[r] \ar[d] \ar[rd] & \resource{object file} \\ & \resource{assembly\\listing} & \resource{debugging\\information}}
\seeassembly\seerisc\seeobject\seecode\seedebugging
}

\providecommand{\cdwasm}{
\toolsection{cdwasm} is a compiler for intermediate code targeting the WebAssembly architecture.
It generates machine code for WebAssembly targets from programs written in intermediate code and stores it in corresponding object files.
It also creates a debugging information file as well as an assembly file containing a listing of the generated machine code.
\debuggingtool
\flowgraph{\resource{intermediate\\code} \ar[r] & \toolbox{cdwasm} \ar[r] \ar[d] \ar[rd] & \resource{object file} \\ & \resource{assembly\\listing} & \resource{debugging\\information}}
\seeassembly\seewasm\seeobject\seecode\seedebugging
}

% C++ tools

\providecommand{\cppprep}{
\toolsection{cppprep} is a preprocessor for the \cpp{} programming language.
It preprocesses source code according to the rules of \cpp{} and writes it to the standard output stream.
Only the macro names \texttt{\_\_DATE\_\_}, \texttt{\_\_FILE\_\_}, \texttt{\_\_LINE\_\_}, and \texttt{\_\_TIME\_\_} are predefined.
\flowgraph{\resource{\cpp{} or other\\source code} \ar[r] & \toolbox{cppprep} \ar[r] & \resource{preprocessed\\source code} \\ & \variable{ECSINCLUDE} \ar[u]}
\seecpp
}

\providecommand{\cppprint}{
\toolsection{cppprint} is a pretty printer for the \cpp{} programming language.
It reformats the source code of \cpp{} programs and writes it to the standard output stream.
\flowgraph{\resource{\cpp{}\\source code} \ar[r] & \toolbox{cppprint} \ar[r] & \resource{reformatted\\source code} \\ & \variable{ECSINCLUDE} \ar[u]}
\seecpp
}

\providecommand{\cppcheck}{
\toolsection{cppcheck} is a syntactic and semantic checker for the \cpp{} programming language.
It just performs syntactic and semantic checks on \cpp{} programs and writes its diagnostic messages to the standard error stream.
\flowgraph{\resource{\cpp{}\\source code} \ar[r] & \toolbox{cppcheck} \ar[r] & \resource{diagnostic\\messages} \\ & \variable{ECSINCLUDE} \ar[u]}
\seecpp
}

\providecommand{\cppdump}{
\toolsection{cppdump} is a serializer for the \cpp{} programming language.
It dumps the complete internal representation of programs written in \cpp{} into an XML document.
\debuggingtool
\flowgraph{\resource{\cpp{}\\source code} \ar[r] & \toolbox{cppdump} \ar[r] & \resource{internal\\representation} \\ & \variable{ECSINCLUDE} \ar[u]}
\seecpp
}

\providecommand{\cpprun}{
\toolsection{cpprun} is an interpreter for the \cpp{} programming language.
It processes and executes programs written in \cpp{}.
The macro \texttt{\_\_run\_\_} is predefined in order to enable programmers to identify this tool while interpreting.
\flowgraph{\resource{\cpp{}\\source code} \ar[r] & \toolbox{cpprun} \ar@/u/[r] & \resource{input/\\output} \ar@/d/[l] \\ & \variable{ECSINCLUDE} \ar[u]}
\seecpp
}

\providecommand{\cppdoc}{
\toolsection{cppdoc} is a generic documentation generator for the \cpp{} programming language.
It processes several \cpp{} source files and assembles all information therein into a generic documentation.
\debuggingtool
\flowgraph{\resource{\cpp{}\\source code} \ar[r] & \toolbox{cppdoc} \ar[r] & \resource{generic\\documentation} \\ & \variable{ECSINCLUDE} \ar[u]}
\seecpp\seedocumentation
}

\providecommand{\cpphtml}{
\toolsection{cpphtml} is an HTML documentation generator for the \cpp{} programming language.
It processes several \cpp{} source files and assembles all information therein into an HTML document.
\flowgraph{\resource{\cpp{}\\source code} \ar[r] & \toolbox{cpphtml} \ar[r] & \resource{HTML\\document} \\ & \variable{ECSINCLUDE} \ar[u]}
\seecpp\seedocumentation
}

\providecommand{\cpplatex}{
\toolsection{cpplatex} is a Latex documentation generator for the \cpp{} programming language.
It processes several \cpp{} source files and assembles all information therein into a Latex document.
\flowgraph{\resource{\cpp{}\\source code} \ar[r] & \toolbox{cpplatex} \ar[r] & \resource{Latex\\document} \\ & \variable{ECSINCLUDE} \ar[u]}
\seecpp\seedocumentation
}

\providecommand{\cppcode}{
\toolsection{cppcode} is an intermediate code generator for the \cpp{} programming language.
It generates intermediate code from programs written in \cpp{} and stores it in corresponding assembly files.
The macro \texttt{\_\_code\_\_} is predefined in order to enable programmers to identify this tool while generating intermediate code.
Programs generated with this tool require additional runtime support that is stored in the \file{cpp\-code\-run} library file.
\debuggingtool
\flowgraph{\resource{\cpp{}\\source code} \ar[r] & \toolbox{cppcode} \ar[r] & \resource{intermediate\\code} \\ & \variable{ECSINCLUDE} \ar[u]}
\seecpp\seeassembly\seecode
}

\providecommand{\cppamda}{
\toolsection{cppamd16} is a compiler for the \cpp{} programming language targeting the AMD64 hardware architecture.
It generates machine code for AMD64 processors from programs written in \cpp{} and stores it in corresponding object files.
The compiler generates machine code for the 16-bit operating mode defined by the AMD64 architecture.
For debugging purposes, it also creates a debugging information file as well as an assembly file containing a listing of the generated machine code.
The macro \texttt{\_\_amd16\_\_} is predefined in order to enable programmers to identify this tool and its target architecture while compiling.
Programs generated with this compiler require additional runtime support that is stored in the \file{cpp\-amd16\-run} library file.
\flowgraph{\resource{\cpp{}\\source code} \ar[r] & \toolbox{cppamd16} \ar[r] \ar[d] \ar[rd] & \resource{object file} \\ \variable{ECSINCLUDE} \ar[ru] & \resource{debugging\\information} & \resource{assembly\\listing}}
\seecpp\seeassembly\seeamd\seeobject\seedebugging
}

\providecommand{\cppamdb}{
\toolsection{cppamd32} is a compiler for the \cpp{} programming language targeting the AMD64 hardware architecture.
It generates machine code for AMD64 processors from programs written in \cpp{} and stores it in corresponding object files.
The compiler generates machine code for the 32-bit operating mode defined by the AMD64 architecture.
For debugging purposes, it also creates a debugging information file as well as an assembly file containing a listing of the generated machine code.
The macro \texttt{\_\_amd32\_\_} is predefined in order to enable programmers to identify this tool and its target architecture while compiling.
Programs generated with this compiler require additional runtime support that is stored in the \file{cpp\-amd32\-run} library file.
\flowgraph{\resource{\cpp{}\\source code} \ar[r] & \toolbox{cppamd32} \ar[r] \ar[d] \ar[rd] & \resource{object file} \\ \variable{ECSINCLUDE} \ar[ru] & \resource{debugging\\information} & \resource{assembly\\listing}}
\seecpp\seeassembly\seeamd\seeobject\seedebugging
}

\providecommand{\cppamdc}{
\toolsection{cppamd64} is a compiler for the \cpp{} programming language targeting the AMD64 hardware architecture.
It generates machine code for AMD64 processors from programs written in \cpp{} and stores it in corresponding object files.
The compiler generates machine code for the 64-bit operating mode defined by the AMD64 architecture.
For debugging purposes, it also creates a debugging information file as well as an assembly file containing a listing of the generated machine code.
The macro \texttt{\_\_amd64\_\_} is predefined in order to enable programmers to identify this tool and its target architecture while compiling.
Programs generated with this compiler require additional runtime support that is stored in the \file{cpp\-amd64\-run} library file.
\flowgraph{\resource{\cpp{}\\source code} \ar[r] & \toolbox{cppamd64} \ar[r] \ar[d] \ar[rd] & \resource{object file} \\ \variable{ECSINCLUDE} \ar[ru] & \resource{debugging\\information} & \resource{assembly\\listing}}
\seecpp\seeassembly\seeamd\seeobject\seedebugging
}

\providecommand{\cpparma}{
\toolsection{cpparma32} is a compiler for the \cpp{} programming language targeting the ARM hardware architecture.
It generates machine code for ARM processors executing A32 instructions from programs written in \cpp{} and stores it in corresponding object files.
For debugging purposes, it also creates a debugging information file as well as an assembly file containing a listing of the generated machine code.
The macro \texttt{\_\_arma32\_\_} is predefined in order to enable programmers to identify this tool and its target architecture while compiling.
Programs generated with this compiler require additional runtime support that is stored in the \file{cpp\-arma32\-run} library file.
\flowgraph{\resource{\cpp{}\\source code} \ar[r] & \toolbox{cpparma32} \ar[r] \ar[d] \ar[rd] & \resource{object file} \\ \variable{ECSINCLUDE} \ar[ru] & \resource{debugging\\information} & \resource{assembly\\listing}}
\seecpp\seeassembly\seearm\seeobject\seedebugging
}

\providecommand{\cpparmb}{
\toolsection{cpparma64} is a compiler for the \cpp{} programming language targeting the ARM hardware architecture.
It generates machine code for ARM processors executing A64 instructions from programs written in \cpp{} and stores it in corresponding object files.
For debugging purposes, it also creates a debugging information file as well as an assembly file containing a listing of the generated machine code.
The macro \texttt{\_\_arma64\_\_} is predefined in order to enable programmers to identify this tool and its target architecture while compiling.
Programs generated with this compiler require additional runtime support that is stored in the \file{cpp\-arma64\-run} library file.
\flowgraph{\resource{\cpp{}\\source code} \ar[r] & \toolbox{cpparma64} \ar[r] \ar[d] \ar[rd] & \resource{object file} \\ \variable{ECSINCLUDE} \ar[ru] & \resource{debugging\\information} & \resource{assembly\\listing}}
\seecpp\seeassembly\seearm\seeobject\seedebugging
}

\providecommand{\cpparmc}{
\toolsection{cpparmt32} is a compiler for the \cpp{} programming language targeting the ARM hardware architecture.
It generates machine code for ARM processors without floating-point extension executing T32 instructions from programs written in \cpp{} and stores it in corresponding object files.
For debugging purposes, it also creates a debugging information file as well as an assembly file containing a listing of the generated machine code.
The macro \texttt{\_\_armt32\_\_} is predefined in order to enable programmers to identify this tool and its target architecture while compiling.
Programs generated with this compiler require additional runtime support that is stored in the \file{cpp\-armt32\-run} library file.
\flowgraph{\resource{\cpp{}\\source code} \ar[r] & \toolbox{cpparmt32} \ar[r] \ar[d] \ar[rd] & \resource{object file} \\ \variable{ECSINCLUDE} \ar[ru] & \resource{debugging\\information} & \resource{assembly\\listing}}
\seecpp\seeassembly\seearm\seeobject\seedebugging
}

\providecommand{\cpparmcfpe}{
\toolsection{cpparmt32fpe} is a compiler for the \cpp{} programming language targeting the ARM hardware architecture.
It generates machine code for ARM processors with floating-point extension executing T32 instructions from programs written in \cpp{} and stores it in corresponding object files.
For debugging purposes, it also creates a debugging information file as well as an assembly file containing a listing of the generated machine code.
The macro \texttt{\_\_armt32fpe\_\_} is predefined in order to enable programmers to identify this tool and its target architecture while compiling.
Programs generated with this compiler require additional runtime support that is stored in the \file{cpp\-armt32\-fpe\-run} library file.
\flowgraph{\resource{\cpp{}\\source code} \ar[r] & \toolbox{cpparmt32fpe} \ar[r] \ar[d] \ar[rd] & \resource{object file} \\ \variable{ECSINCLUDE} \ar[ru] & \resource{debugging\\information} & \resource{assembly\\listing}}
\seecpp\seeassembly\seearm\seeobject\seedebugging
}

\providecommand{\cppavr}{
\toolsection{cppavr} is a compiler for the \cpp{} programming language targeting the AVR hardware architecture.
It generates machine code for AVR processors from programs written in \cpp{} and stores it in corresponding object files.
For debugging purposes, it also creates a debugging information file as well as an assembly file containing a listing of the generated machine code.
The macro \texttt{\_\_avr\_\_} is predefined in order to enable programmers to identify this tool and its target architecture while compiling.
Programs generated with this compiler require additional runtime support that is stored in the \file{cpp\-avr\-run} library file.
\flowgraph{\resource{\cpp{}\\source code} \ar[r] & \toolbox{cppavr} \ar[r] \ar[d] \ar[rd] & \resource{object file} \\ \variable{ECSINCLUDE} \ar[ru] & \resource{debugging\\information} & \resource{assembly\\listing}}
\seecpp\seeassembly\seeavr\seeobject\seedebugging
}

\providecommand{\cppavrtt}{
\toolsection{cppavr32} is a compiler for the \cpp{} programming language targeting the AVR32 hardware architecture.
It generates machine code for AVR32 processors from programs written in \cpp{} and stores it in corresponding object files.
For debugging purposes, it also creates a debugging information file as well as an assembly file containing a listing of the generated machine code.
The macro \texttt{\_\_avr32\_\_} is predefined in order to enable programmers to identify this tool and its target architecture while compiling.
Programs generated with this compiler require additional runtime support that is stored in the \file{cpp\-avr32\-run} library file.
\flowgraph{\resource{\cpp{}\\source code} \ar[r] & \toolbox{cppavr32} \ar[r] \ar[d] \ar[rd] & \resource{object file} \\ \variable{ECSINCLUDE} \ar[ru] & \resource{debugging\\information} & \resource{assembly\\listing}}
\seecpp\seeassembly\seeavrtt\seeobject\seedebugging
}

\providecommand{\cppmabk}{
\toolsection{cppm68k} is a compiler for the \cpp{} programming language targeting the M68000 hardware architecture.
It generates machine code for M68000 processors from programs written in \cpp{} and stores it in corresponding object files.
For debugging purposes, it also creates a debugging information file as well as an assembly file containing a listing of the generated machine code.
The macro \texttt{\_\_m68k\_\_} is predefined in order to enable programmers to identify this tool and its target architecture while compiling.
Programs generated with this compiler require additional runtime support that is stored in the \file{cpp\-m68k\-run} library file.
\flowgraph{\resource{\cpp{}\\source code} \ar[r] & \toolbox{cppm68k} \ar[r] \ar[d] \ar[rd] & \resource{object file} \\ \variable{ECSINCLUDE} \ar[ru] & \resource{debugging\\information} & \resource{assembly\\listing}}
\seecpp\seeassembly\seemabk\seeobject\seedebugging
}

\providecommand{\cppmibl}{
\toolsection{cppmibl} is a compiler for the \cpp{} programming language targeting the MicroBlaze hardware architecture.
It generates machine code for MicroBlaze processors from programs written in \cpp{} and stores it in corresponding object files.
For debugging purposes, it also creates a debugging information file as well as an assembly file containing a listing of the generated machine code.
The macro \texttt{\_\_mibl\_\_} is predefined in order to enable programmers to identify this tool and its target architecture while compiling.
Programs generated with this compiler require additional runtime support that is stored in the \file{cpp\-mibl\-run} library file.
\flowgraph{\resource{\cpp{}\\source code} \ar[r] & \toolbox{cppmibl} \ar[r] \ar[d] \ar[rd] & \resource{object file} \\ \variable{ECSINCLUDE} \ar[ru] & \resource{debugging\\information} & \resource{assembly\\listing}}
\seecpp\seeassembly\seemibl\seeobject\seedebugging
}

\providecommand{\cppmipsa}{
\toolsection{cppmips32} is a compiler for the \cpp{} programming language targeting the MIPS32 hardware architecture.
It generates machine code for MIPS32 processors from programs written in \cpp{} and stores it in corresponding object files.
For debugging purposes, it also creates a debugging information file as well as an assembly file containing a listing of the generated machine code.
The macro \texttt{\_\_mips32\_\_} is predefined in order to enable programmers to identify this tool and its target architecture while compiling.
Programs generated with this compiler require additional runtime support that is stored in the \file{cpp\-mips32\-run} library file.
\flowgraph{\resource{\cpp{}\\source code} \ar[r] & \toolbox{cppmips32} \ar[r] \ar[d] \ar[rd] & \resource{object file} \\ \variable{ECSINCLUDE} \ar[ru] & \resource{debugging\\information} & \resource{assembly\\listing}}
\seecpp\seeassembly\seemips\seeobject\seedebugging
}

\providecommand{\cppmipsb}{
\toolsection{cppmips64} is a compiler for the \cpp{} programming language targeting the MIPS64 hardware architecture.
It generates machine code for MIPS64 processors from programs written in \cpp{} and stores it in corresponding object files.
For debugging purposes, it also creates a debugging information file as well as an assembly file containing a listing of the generated machine code.
The macro \texttt{\_\_mips64\_\_} is predefined in order to enable programmers to identify this tool and its target architecture while compiling.
Programs generated with this compiler require additional runtime support that is stored in the \file{cpp\-mips64\-run} library file.
\flowgraph{\resource{\cpp{}\\source code} \ar[r] & \toolbox{cppmips64} \ar[r] \ar[d] \ar[rd] & \resource{object file} \\ \variable{ECSINCLUDE} \ar[ru] & \resource{debugging\\information} & \resource{assembly\\listing}}
\seecpp\seeassembly\seemips\seeobject\seedebugging
}

\providecommand{\cppmmix}{
\toolsection{cppmmix} is a compiler for the \cpp{} programming language targeting the MMIX hardware architecture.
It generates machine code for MMIX processors from programs written in \cpp{} and stores it in corresponding object files.
For debugging purposes, it also creates a debugging information file as well as an assembly file containing a listing of the generated machine code.
The macro \texttt{\_\_mmix\_\_} is predefined in order to enable programmers to identify this tool and its target architecture while compiling.
Programs generated with this compiler require additional runtime support that is stored in the \file{cpp\-mmix\-run} library file.
\flowgraph{\resource{\cpp{}\\source code} \ar[r] & \toolbox{cppmmix} \ar[r] \ar[d] \ar[rd] & \resource{object file} \\ \variable{ECSINCLUDE} \ar[ru] & \resource{debugging\\information} & \resource{assembly\\listing}}
\seecpp\seeassembly\seemmix\seeobject\seedebugging
}

\providecommand{\cpporok}{
\toolsection{cppor1k} is a compiler for the \cpp{} programming language targeting the OpenRISC 1000 hardware architecture.
It generates machine code for OpenRISC 1000 processors from programs written in \cpp{} and stores it in corresponding object files.
For debugging purposes, it also creates a debugging information file as well as an assembly file containing a listing of the generated machine code.
The macro \texttt{\_\_or1k\_\_} is predefined in order to enable programmers to identify this tool and its target architecture while compiling.
Programs generated with this compiler require additional runtime support that is stored in the \file{cpp\-or1k\-run} library file.
\flowgraph{\resource{\cpp{}\\source code} \ar[r] & \toolbox{cppor1k} \ar[r] \ar[d] \ar[rd] & \resource{object file} \\ \variable{ECSINCLUDE} \ar[ru] & \resource{debugging\\information} & \resource{assembly\\listing}}
\seecpp\seeassembly\seeorok\seeobject\seedebugging
}

\providecommand{\cppppca}{
\toolsection{cppppc32} is a compiler for the \cpp{} programming language targeting the PowerPC hardware architecture.
It generates machine code for PowerPC processors from programs written in \cpp{} and stores it in corresponding object files.
The compiler generates machine code for the 32-bit operating mode defined by the PowerPC architecture.
For debugging purposes, it also creates a debugging information file as well as an assembly file containing a listing of the generated machine code.
The macro \texttt{\_\_ppc32\_\_} is predefined in order to enable programmers to identify this tool and its target architecture while compiling.
Programs generated with this compiler require additional runtime support that is stored in the \file{cpp\-ppc32\-run} library file.
\flowgraph{\resource{\cpp{}\\source code} \ar[r] & \toolbox{cppppc32} \ar[r] \ar[d] \ar[rd] & \resource{object file} \\ \variable{ECSINCLUDE} \ar[ru] & \resource{debugging\\information} & \resource{assembly\\listing}}
\seecpp\seeassembly\seeppc\seeobject\seedebugging
}

\providecommand{\cppppcb}{
\toolsection{cppppc64} is a compiler for the \cpp{} programming language targeting the PowerPC hardware architecture.
It generates machine code for PowerPC processors from programs written in \cpp{} and stores it in corresponding object files.
The compiler generates machine code for the 64-bit operating mode defined by the PowerPC architecture.
For debugging purposes, it also creates a debugging information file as well as an assembly file containing a listing of the generated machine code.
The macro \texttt{\_\_ppc64\_\_} is predefined in order to enable programmers to identify this tool and its target architecture while compiling.
Programs generated with this compiler require additional runtime support that is stored in the \file{cpp\-ppc64\-run} library file.
\flowgraph{\resource{\cpp{}\\source code} \ar[r] & \toolbox{cppppc64} \ar[r] \ar[d] \ar[rd] & \resource{object file} \\ \variable{ECSINCLUDE} \ar[ru] & \resource{debugging\\information} & \resource{assembly\\listing}}
\seecpp\seeassembly\seeppc\seeobject\seedebugging
}

\providecommand{\cpprisc}{
\toolsection{cpprisc} is a compiler for the \cpp{} programming language targeting the RISC hardware architecture.
It generates machine code for RISC processors from programs written in \cpp{} and stores it in corresponding object files.
For debugging purposes, it also creates a debugging information file as well as an assembly file containing a listing of the generated machine code.
The macro \texttt{\_\_risc\_\_} is predefined in order to enable programmers to identify this tool and its target architecture while compiling.
Programs generated with this compiler require additional runtime support that is stored in the \file{cpp\-risc\-run} library file.
\flowgraph{\resource{\cpp{}\\source code} \ar[r] & \toolbox{cpprisc} \ar[r] \ar[d] \ar[rd] & \resource{object file} \\ \variable{ECSINCLUDE} \ar[ru] & \resource{debugging\\information} & \resource{assembly\\listing}}
\seecpp\seeassembly\seerisc\seeobject\seedebugging
}

\providecommand{\cppwasm}{
\toolsection{cppwasm} is a compiler for the \cpp{} programming language targeting the WebAssembly architecture.
It generates machine code for WebAssembly targets from programs written in \cpp{} and stores it in corresponding object files.
For debugging purposes, it also creates a debugging information file as well as an assembly file containing a listing of the generated machine code.
The macro \texttt{\_\_wasm\_\_} is predefined in order to enable programmers to identify this tool and its target architecture while compiling.
Programs generated with this compiler require additional runtime support that is stored in the \file{cpp\-wasm\-run} library file.
\flowgraph{\resource{\cpp{}\\source code} \ar[r] & \toolbox{cppwasm} \ar[r] \ar[d] \ar[rd] & \resource{object file} \\ \variable{ECSINCLUDE} \ar[ru] & \resource{debugging\\information} & \resource{assembly\\listing}}
\seecpp\seeassembly\seewasm\seeobject\seedebugging
}

% FALSE tools

\providecommand{\falprint}{
\toolsection{falprint} is a pretty printer for the FALSE programming language.
It reformats the source code of FALSE programs and writes it to the standard output stream.
\flowgraph{\resource{FALSE\\source code} \ar[r] & \toolbox{falprint} \ar[r] & \resource{reformatted\\source code}}
\seefalse
}

\providecommand{\falcheck}{
\toolsection{falcheck} is a syntactic and semantic checker for the FALSE programming language.
It just performs syntactic and semantic checks on FALSE programs and writes its diagnostic messages to the standard error stream.
\flowgraph{\resource{FALSE\\source code} \ar[r] & \toolbox{falcheck} \ar[r] & \resource{diagnostic\\messages}}
\seefalse
}

\providecommand{\faldump}{
\toolsection{faldump} is a serializer for the FALSE programming language.
It dumps the complete internal representation of programs written in FALSE into an XML document.
\debuggingtool
\flowgraph{\resource{FALSE\\source code} \ar[r] & \toolbox{faldump} \ar[r] & \resource{internal\\representation}}
\seefalse
}

\providecommand{\falrun}{
\toolsection{falrun} is an interpreter for the FALSE programming language.
It processes and executes programs written in FALSE\@.
\flowgraph{\resource{FALSE\\source code} \ar[r] & \toolbox{falrun} \ar@/u/[r] & \resource{input/\\output} \ar@/d/[l]}
\seefalse
}

\providecommand{\falcpp}{
\toolsection{falcpp} is a transpiler for the FALSE programming language.
It translates programs written in FALSE into \cpp{} programs and stores them in corresponding source files.
\flowgraph{\resource{FALSE\\source code} \ar[r] & \toolbox{falcpp} \ar[r] & \resource{\cpp{}\\source file}}
\seefalse\seecpp
}

\providecommand{\falcode}{
\toolsection{falcode} is an intermediate code generator for the FALSE programming language.
It generates intermediate code from programs written in FALSE and stores it in corresponding assembly files.
\debuggingtool
\flowgraph{\resource{FALSE\\source code} \ar[r] & \toolbox{falcode} \ar[r] & \resource{intermediate\\code}}
\seefalse\seeassembly\seecode
}

\providecommand{\falamda}{
\toolsection{falamd16} is a compiler for the FALSE programming language targeting the AMD64 hardware architecture.
It generates machine code for AMD64 processors from programs written in FALSE and stores it in corresponding object files.
The compiler generates machine code for the 16-bit operating mode defined by the AMD64 architecture.
\flowgraph{\resource{FALSE\\source code} \ar[r] & \toolbox{falamd16} \ar[r] & \resource{object file}}
\seefalse\seeamd\seeobject
}

\providecommand{\falamdb}{
\toolsection{falamd32} is a compiler for the FALSE programming language targeting the AMD64 hardware architecture.
It generates machine code for AMD64 processors from programs written in FALSE and stores it in corresponding object files.
The compiler generates machine code for the 32-bit operating mode defined by the AMD64 architecture.
\flowgraph{\resource{FALSE\\source code} \ar[r] & \toolbox{falamd32} \ar[r] & \resource{object file}}
\seefalse\seeamd\seeobject
}

\providecommand{\falamdc}{
\toolsection{falamd64} is a compiler for the FALSE programming language targeting the AMD64 hardware architecture.
It generates machine code for AMD64 processors from programs written in FALSE and stores it in corresponding object files.
The compiler generates machine code for the 64-bit operating mode defined by the AMD64 architecture.
\flowgraph{\resource{FALSE\\source code} \ar[r] & \toolbox{falamd64} \ar[r] & \resource{object file}}
\seefalse\seeamd\seeobject
}

\providecommand{\falarma}{
\toolsection{falarma32} is a compiler for the FALSE programming language targeting the ARM hardware architecture.
It generates machine code for ARM processors executing A32 instructions from programs written in FALSE and stores it in corresponding object files.
\flowgraph{\resource{FALSE\\source code} \ar[r] & \toolbox{falarma32} \ar[r] & \resource{object file}}
\seefalse\seearm\seeobject
}

\providecommand{\falarmb}{
\toolsection{falarma64} is a compiler for the FALSE programming language targeting the ARM hardware architecture.
It generates machine code for ARM processors executing A64 instructions from programs written in FALSE and stores it in corresponding object files.
\flowgraph{\resource{FALSE\\source code} \ar[r] & \toolbox{falarma64} \ar[r] & \resource{object file}}
\seefalse\seearm\seeobject
}

\providecommand{\falarmc}{
\toolsection{falarmt32} is a compiler for the FALSE programming language targeting the ARM hardware architecture.
It generates machine code for ARM processors without floating-point extension executing T32 instructions from programs written in FALSE and stores it in corresponding object files.
\flowgraph{\resource{FALSE\\source code} \ar[r] & \toolbox{falarmt32} \ar[r] & \resource{object file}}
\seefalse\seearm\seeobject
}

\providecommand{\falarmcfpe}{
\toolsection{falarmt32fpe} is a compiler for the FALSE programming language targeting the ARM hardware architecture.
It generates machine code for ARM processors with floating-point extension executing T32 instructions from programs written in FALSE and stores it in corresponding object files.
\flowgraph{\resource{FALSE\\source code} \ar[r] & \toolbox{falarmt32fpe} \ar[r] & \resource{object file}}
\seefalse\seearm\seeobject
}

\providecommand{\falavr}{
\toolsection{falavr} is a compiler for the FALSE programming language targeting the AVR hardware architecture.
It generates machine code for AVR processors from programs written in FALSE and stores it in corresponding object files.
\flowgraph{\resource{FALSE\\source code} \ar[r] & \toolbox{falavr} \ar[r] & \resource{object file}}
\seefalse\seeavr\seeobject
}

\providecommand{\falavrtt}{
\toolsection{falavr32} is a compiler for the FALSE programming language targeting the AVR32 hardware architecture.
It generates machine code for AVR32 processors from programs written in FALSE and stores it in corresponding object files.
\flowgraph{\resource{FALSE\\source code} \ar[r] & \toolbox{falavr32} \ar[r] & \resource{object file}}
\seefalse\seeavrtt\seeobject
}

\providecommand{\falmabk}{
\toolsection{falm68k} is a compiler for the FALSE programming language targeting the M68000 hardware architecture.
It generates machine code for M68000 processors from programs written in FALSE and stores it in corresponding object files.
\flowgraph{\resource{FALSE\\source code} \ar[r] & \toolbox{falm68k} \ar[r] & \resource{object file}}
\seefalse\seemabk\seeobject
}

\providecommand{\falmibl}{
\toolsection{falmibl} is a compiler for the FALSE programming language targeting the MicroBlaze hardware architecture.
It generates machine code for MicroBlaze processors from programs written in FALSE and stores it in corresponding object files.
\flowgraph{\resource{FALSE\\source code} \ar[r] & \toolbox{falmibl} \ar[r] & \resource{object file}}
\seefalse\seemibl\seeobject
}

\providecommand{\falmipsa}{
\toolsection{falmips32} is a compiler for the FALSE programming language targeting the MIPS32 hardware architecture.
It generates machine code for MIPS32 processors from programs written in FALSE and stores it in corresponding object files.
\flowgraph{\resource{FALSE\\source code} \ar[r] & \toolbox{falmips32} \ar[r] & \resource{object file}}
\seefalse\seemips\seeobject
}

\providecommand{\falmipsb}{
\toolsection{falmips64} is a compiler for the FALSE programming language targeting the MIPS64 hardware architecture.
It generates machine code for MIPS64 processors from programs written in FALSE and stores it in corresponding object files.
\flowgraph{\resource{FALSE\\source code} \ar[r] & \toolbox{falmips64} \ar[r] & \resource{object file}}
\seefalse\seemips\seeobject
}

\providecommand{\falmmix}{
\toolsection{falmmix} is a compiler for the FALSE programming language targeting the MMIX hardware architecture.
It generates machine code for MMIX processors from programs written in FALSE and stores it in corresponding object files.
\flowgraph{\resource{FALSE\\source code} \ar[r] & \toolbox{falmmix} \ar[r] & \resource{object file}}
\seefalse\seemmix\seeobject
}

\providecommand{\falorok}{
\toolsection{falor1k} is a compiler for the FALSE programming language targeting the OpenRISC 1000 hardware architecture.
It generates machine code for OpenRISC 1000 processors from programs written in FALSE and stores it in corresponding object files.
\flowgraph{\resource{FALSE\\source code} \ar[r] & \toolbox{falor1k} \ar[r] & \resource{object file}}
\seefalse\seeorok\seeobject
}

\providecommand{\falppca}{
\toolsection{falppc32} is a compiler for the FALSE programming language targeting the PowerPC hardware architecture.
It generates machine code for PowerPC processors from programs written in FALSE and stores it in corresponding object files.
The compiler generates machine code for the 32-bit operating mode defined by the PowerPC architecture.
\flowgraph{\resource{FALSE\\source code} \ar[r] & \toolbox{falppc32} \ar[r] & \resource{object file}}
\seefalse\seeppc\seeobject
}

\providecommand{\falppcb}{
\toolsection{falppc64} is a compiler for the FALSE programming language targeting the PowerPC hardware architecture.
It generates machine code for PowerPC processors from programs written in FALSE and stores it in corresponding object files.
The compiler generates machine code for the 64-bit operating mode defined by the PowerPC architecture.
\flowgraph{\resource{FALSE\\source code} \ar[r] & \toolbox{falppc64} \ar[r] & \resource{object file}}
\seefalse\seeppc\seeobject
}

\providecommand{\falrisc}{
\toolsection{falrisc} is a compiler for the FALSE programming language targeting the RISC hardware architecture.
It generates machine code for RISC processors from programs written in FALSE and stores it in corresponding object files.
\flowgraph{\resource{FALSE\\source code} \ar[r] & \toolbox{falrisc} \ar[r] & \resource{object file}}
\seefalse\seerisc\seeobject
}

\providecommand{\falwasm}{
\toolsection{falwasm} is a compiler for the FALSE programming language targeting the WebAssembly architecture.
It generates machine code for WebAssembly targets from programs written in FALSE and stores it in corresponding object files.
\flowgraph{\resource{FALSE\\source code} \ar[r] & \toolbox{falwasm} \ar[r] & \resource{object file}}
\seefalse\seewasm\seeobject
}

% Oberon tools

\providecommand{\obprint}{
\toolsection{obprint} is a pretty printer for the Oberon programming language.
It reformats the source code of Oberon modules and writes it to the standard output stream.
\flowgraph{\resource{Oberon\\source code} \ar[r] & \toolbox{obprint} \ar[r] & \resource{reformatted\\source code}}
\seeoberon
}

\providecommand{\obcheck}{
\toolsection{obcheck} is a syntactic and semantic checker for the Oberon programming language.
It just performs syntactic and semantic checks on Oberon modules and writes its diagnostic messages to the standard error stream.
In addition, it stores the interface of each module in a symbol file which is required when other modules import the module.
\flowgraph{\resource{Oberon\\source code} \ar[r] & \toolbox{obcheck} \ar[r] \ar@/l/[d] & \resource{diagnostic\\messages} \\ \variable{ECSIMPORT} \ar[ru] & \resource{symbol\\files} \ar@/r/[u]}
\seeoberon
}

\providecommand{\obdump}{
\toolsection{obdump} is a serializer for the Oberon programming language.
It dumps the complete internal representation of modules written in Oberon into an XML document.
\debuggingtool
\flowgraph{\resource{Oberon\\source code} \ar[r] & \toolbox{obdump} \ar[r] \ar@/l/[d] & \resource{internal\\representation} \\ \variable{ECSIMPORT} \ar[ru] & \resource{symbol\\files} \ar@/r/[u]}
\seeoberon
}

\providecommand{\obrun}{
\toolsection{obrun} is an interpreter for the Oberon programming language.
It processes and executes modules written in Oberon.
This tool does neither generate nor process symbol files while interpreting modules.
If a module is imported by another one, its filename has to be named before the other one in the list of command-line arguments.
\flowgraph{\resource{Oberon\\source code} \ar[r] & \toolbox{obrun} \ar@/u/[r] & \resource{input/\\output} \ar@/d/[l]}
\seeoberon
}

\providecommand{\obcpp}{
\toolsection{obcpp} is a transpiler for the Oberon programming language.
It translates programs written in Oberon into \cpp{} programs and stores them in corresponding source and header files.
In addition, it stores the interface of each module in a symbol file which is required when other modules import the module.
The same interface is provided by the generated header file which can be used in other parts of the \cpp{} program.
\flowgraph{\resource{Oberon\\source code} \ar[r] & \toolbox{obcpp} \ar[r] \ar@/l/[d] \ar[rd] & \resource{\cpp{}\\source file} \\ \variable{ECSIMPORT} \ar[ru] & \resource{symbol\\files} \ar@/r/[u] & \resource{\cpp{}\\header file}}
\seeoberon\seecpp
}

\providecommand{\obdoc}{
\toolsection{obdoc} is a generic documentation generator for the Oberon programming language.
It processes several Oberon modules and assembles all information therein into a generic documentation.
In addition, it stores the interface of each module in a symbol file which is required when other modules import the module.
\debuggingtool
\flowgraph{\resource{Oberon\\source code} \ar[r] & \toolbox{obdoc} \ar[r] \ar@/l/[d] & \resource{generic\\documentation} \\ \variable{ECSIMPORT} \ar[ru] & \resource{symbol\\files} \ar@/r/[u]}
\seeoberon\seedocumentation
}

\providecommand{\obhtml}{
\toolsection{obhtml} is an HTML documentation generator for the Oberon programming language.
It processes several Oberon modules and assembles all information therein into an HTML document.
In addition, it stores the interface of each module in a symbol file which is required when other modules import the module.
\flowgraph{\resource{Oberon\\source code} \ar[r] & \toolbox{obhtml} \ar[r] \ar@/l/[d] & \resource{HTML\\document} \\ \variable{ECSIMPORT} \ar[ru] & \resource{symbol\\files} \ar@/r/[u]}
\seeoberon\seedocumentation
}

\providecommand{\oblatex}{
\toolsection{oblatex} is a Latex documentation generator for the Oberon programming language.
It processes several Oberon modules and assembles all information therein into a Latex document.
In addition, it stores the interface of each module in a symbol file which is required when other modules import the module.
\flowgraph{\resource{Oberon\\source code} \ar[r] & \toolbox{oblatex} \ar[r] \ar@/l/[d] & \resource{Latex\\document} \\ \variable{ECSIMPORT} \ar[ru] & \resource{symbol\\files} \ar@/r/[u]}
\seeoberon\seedocumentation
}

\providecommand{\obcode}{
\toolsection{obcode} is an intermediate code generator for the Oberon programming language.
It generates intermediate code from modules written in Oberon and stores it in corresponding assembly files.
In addition, it stores the interface of each module in a symbol file which is required when other modules import the module.
Programs generated with this tool require additional runtime support that is stored in the \file{ob\-code\-run} library file.
\debuggingtool
\flowgraph{\resource{Oberon\\source code} \ar[r] & \toolbox{obcode} \ar[r] \ar@/l/[d] & \resource{intermediate\\code} \\ \variable{ECSIMPORT} \ar[ru] & \resource{symbol\\files} \ar@/r/[u]}
\seeoberon\seeassembly\seecode
}

\providecommand{\obamda}{
\toolsection{obamd16} is a compiler for the Oberon programming language targeting the AMD64 hardware architecture.
It generates machine code for AMD64 processors from modules written in Oberon and stores it in corresponding object files.
The compiler generates machine code for the 16-bit operating mode defined by the AMD64 architecture.
For debugging purposes, it also creates a debugging information file as well as an assembly file containing a listing of the generated machine code.
In addition, it stores the interface of each module in a symbol file which is required when other modules import the module.
Programs generated with this compiler require additional runtime support that is stored in the \file{ob\-amd16\-run} library file.
\flowgraph{\resource{Oberon\\source code} \ar[r] & \toolbox{obamd16} \ar[r] \ar@/l/[d] \ar[rd] & \resource{object file} \\ \variable{ECSIMPORT} \ar[ru] & \resource{symbol\\files} \ar@/r/[u] & \resource{debugging\\information}}
\seeoberon\seeassembly\seeamd\seeobject\seedebugging
}

\providecommand{\obamdb}{
\toolsection{obamd32} is a compiler for the Oberon programming language targeting the AMD64 hardware architecture.
It generates machine code for AMD64 processors from modules written in Oberon and stores it in corresponding object files.
The compiler generates machine code for the 32-bit operating mode defined by the AMD64 architecture.
For debugging purposes, it also creates a debugging information file as well as an assembly file containing a listing of the generated machine code.
In addition, it stores the interface of each module in a symbol file which is required when other modules import the module.
Programs generated with this compiler require additional runtime support that is stored in the \file{ob\-amd32\-run} library file.
\flowgraph{\resource{Oberon\\source code} \ar[r] & \toolbox{obamd32} \ar[r] \ar@/l/[d] \ar[rd] & \resource{object file} \\ \variable{ECSIMPORT} \ar[ru] & \resource{symbol\\files} \ar@/r/[u] & \resource{debugging\\information}}
\seeoberon\seeassembly\seeamd\seeobject\seedebugging
}

\providecommand{\obamdc}{
\toolsection{obamd64} is a compiler for the Oberon programming language targeting the AMD64 hardware architecture.
It generates machine code for AMD64 processors from modules written in Oberon and stores it in corresponding object files.
The compiler generates machine code for the 64-bit operating mode defined by the AMD64 architecture.
For debugging purposes, it also creates a debugging information file as well as an assembly file containing a listing of the generated machine code.
In addition, it stores the interface of each module in a symbol file which is required when other modules import the module.
Programs generated with this compiler require additional runtime support that is stored in the \file{ob\-amd64\-run} library file.
\flowgraph{\resource{Oberon\\source code} \ar[r] & \toolbox{obamd64} \ar[r] \ar@/l/[d] \ar[rd] & \resource{object file} \\ \variable{ECSIMPORT} \ar[ru] & \resource{symbol\\files} \ar@/r/[u] & \resource{debugging\\information}}
\seeoberon\seeassembly\seeamd\seeobject\seedebugging
}

\providecommand{\obarma}{
\toolsection{obarma32} is a compiler for the Oberon programming language targeting the ARM hardware architecture.
It generates machine code for ARM processors executing A32 instructions from modules written in Oberon and stores it in corresponding object files.
For debugging purposes, it also creates a debugging information file as well as an assembly file containing a listing of the generated machine code.
In addition, it stores the interface of each module in a symbol file which is required when other modules import the module.
Programs generated with this compiler require additional runtime support that is stored in the \file{ob\-arma32\-run} library file.
\flowgraph{\resource{Oberon\\source code} \ar[r] & \toolbox{obarma32} \ar[r] \ar@/l/[d] \ar[rd] & \resource{object file} \\ \variable{ECSIMPORT} \ar[ru] & \resource{symbol\\files} \ar@/r/[u] & \resource{debugging\\information}}
\seeoberon\seeassembly\seearm\seeobject\seedebugging
}

\providecommand{\obarmb}{
\toolsection{obarma64} is a compiler for the Oberon programming language targeting the ARM hardware architecture.
It generates machine code for ARM processors executing A64 instructions from modules written in Oberon and stores it in corresponding object files.
For debugging purposes, it also creates a debugging information file as well as an assembly file containing a listing of the generated machine code.
In addition, it stores the interface of each module in a symbol file which is required when other modules import the module.
Programs generated with this compiler require additional runtime support that is stored in the \file{ob\-arma64\-run} library file.
\flowgraph{\resource{Oberon\\source code} \ar[r] & \toolbox{obarma64} \ar[r] \ar@/l/[d] \ar[rd] & \resource{object file} \\ \variable{ECSIMPORT} \ar[ru] & \resource{symbol\\files} \ar@/r/[u] & \resource{debugging\\information}}
\seeoberon\seeassembly\seearm\seeobject\seedebugging
}

\providecommand{\obarmc}{
\toolsection{obarmt32} is a compiler for the Oberon programming language targeting the ARM hardware architecture.
It generates machine code for ARM processors without floating-point extension executing T32 instructions from modules written in Oberon and stores it in corresponding object files.
For debugging purposes, it also creates a debugging information file as well as an assembly file containing a listing of the generated machine code.
In addition, it stores the interface of each module in a symbol file which is required when other modules import the module.
Programs generated with this compiler require additional runtime support that is stored in the \file{ob\-armt32\-run} library file.
\flowgraph{\resource{Oberon\\source code} \ar[r] & \toolbox{obarmt32} \ar[r] \ar@/l/[d] \ar[rd] & \resource{object file} \\ \variable{ECSIMPORT} \ar[ru] & \resource{symbol\\files} \ar@/r/[u] & \resource{debugging\\information}}
\seeoberon\seeassembly\seearm\seeobject\seedebugging
}

\providecommand{\obarmcfpe}{
\toolsection{obarmt32fpe} is a compiler for the Oberon programming language targeting the ARM hardware architecture.
It generates machine code for ARM processors with floating-point extension executing T32 instructions from modules written in Oberon and stores it in corresponding object files.
For debugging purposes, it also creates a debugging information file as well as an assembly file containing a listing of the generated machine code.
In addition, it stores the interface of each module in a symbol file which is required when other modules import the module.
Programs generated with this compiler require additional runtime support that is stored in the \file{ob\-armt32\-fpe\-run} library file.
\flowgraph{\resource{Oberon\\source code} \ar[r] & \toolbox{obarmt32fpe} \ar[r] \ar@/l/[d] \ar[rd] & \resource{object file} \\ \variable{ECSIMPORT} \ar[ru] & \resource{symbol\\files} \ar@/r/[u] & \resource{debugging\\information}}
\seeoberon\seeassembly\seearm\seeobject\seedebugging
}

\providecommand{\obavr}{
\toolsection{obavr} is a compiler for the Oberon programming language targeting the AVR hardware architecture.
It generates machine code for AVR processors from modules written in Oberon and stores it in corresponding object files.
For debugging purposes, it also creates a debugging information file as well as an assembly file containing a listing of the generated machine code.
In addition, it stores the interface of each module in a symbol file which is required when other modules import the module.
Programs generated with this compiler require additional runtime support that is stored in the \file{ob\-avr\-run} library file.
\flowgraph{\resource{Oberon\\source code} \ar[r] & \toolbox{obavr} \ar[r] \ar@/l/[d] \ar[rd] & \resource{object file} \\ \variable{ECSIMPORT} \ar[ru] & \resource{symbol\\files} \ar@/r/[u] & \resource{debugging\\information}}
\seeoberon\seeassembly\seeavr\seeobject\seedebugging
}

\providecommand{\obavrtt}{
\toolsection{obavr32} is a compiler for the Oberon programming language targeting the AVR32 hardware architecture.
It generates machine code for AVR32 processors from modules written in Oberon and stores it in corresponding object files.
For debugging purposes, it also creates a debugging information file as well as an assembly file containing a listing of the generated machine code.
In addition, it stores the interface of each module in a symbol file which is required when other modules import the module.
Programs generated with this compiler require additional runtime support that is stored in the \file{ob\-avr32\-run} library file.
\flowgraph{\resource{Oberon\\source code} \ar[r] & \toolbox{obavr32} \ar[r] \ar@/l/[d] \ar[rd] & \resource{object file} \\ \variable{ECSIMPORT} \ar[ru] & \resource{symbol\\files} \ar@/r/[u] & \resource{debugging\\information}}
\seeoberon\seeassembly\seeavrtt\seeobject\seedebugging
}

\providecommand{\obmabk}{
\toolsection{obm68k} is a compiler for the Oberon programming language targeting the M68000 hardware architecture.
It generates machine code for M68000 processors from modules written in Oberon and stores it in corresponding object files.
For debugging purposes, it also creates a debugging information file as well as an assembly file containing a listing of the generated machine code.
In addition, it stores the interface of each module in a symbol file which is required when other modules import the module.
Programs generated with this compiler require additional runtime support that is stored in the \file{ob\-m68k\-run} library file.
\flowgraph{\resource{Oberon\\source code} \ar[r] & \toolbox{obm68k} \ar[r] \ar@/l/[d] \ar[rd] & \resource{object file} \\ \variable{ECSIMPORT} \ar[ru] & \resource{symbol\\files} \ar@/r/[u] & \resource{debugging\\information}}
\seeoberon\seeassembly\seemabk\seeobject\seedebugging
}

\providecommand{\obmibl}{
\toolsection{obmibl} is a compiler for the Oberon programming language targeting the MicroBlaze hardware architecture.
It generates machine code for MicroBlaze processors from modules written in Oberon and stores it in corresponding object files.
For debugging purposes, it also creates a debugging information file as well as an assembly file containing a listing of the generated machine code.
In addition, it stores the interface of each module in a symbol file which is required when other modules import the module.
Programs generated with this compiler require additional runtime support that is stored in the \file{ob\-mibl\-run} library file.
\flowgraph{\resource{Oberon\\source code} \ar[r] & \toolbox{obmibl} \ar[r] \ar@/l/[d] \ar[rd] & \resource{object file} \\ \variable{ECSIMPORT} \ar[ru] & \resource{symbol\\files} \ar@/r/[u] & \resource{debugging\\information}}
\seeoberon\seeassembly\seemibl\seeobject\seedebugging
}

\providecommand{\obmipsa}{
\toolsection{obmips32} is a compiler for the Oberon programming language targeting the MIPS32 hardware architecture.
It generates machine code for MIPS32 processors from modules written in Oberon and stores it in corresponding object files.
For debugging purposes, it also creates a debugging information file as well as an assembly file containing a listing of the generated machine code.
In addition, it stores the interface of each module in a symbol file which is required when other modules import the module.
Programs generated with this compiler require additional runtime support that is stored in the \file{ob\-mips32\-run} library file.
\flowgraph{\resource{Oberon\\source code} \ar[r] & \toolbox{obmips32} \ar[r] \ar@/l/[d] \ar[rd] & \resource{object file} \\ \variable{ECSIMPORT} \ar[ru] & \resource{symbol\\files} \ar@/r/[u] & \resource{debugging\\information}}
\seeoberon\seeassembly\seemips\seeobject\seedebugging
}

\providecommand{\obmipsb}{
\toolsection{obmips64} is a compiler for the Oberon programming language targeting the MIPS64 hardware architecture.
It generates machine code for MIPS64 processors from modules written in Oberon and stores it in corresponding object files.
For debugging purposes, it also creates a debugging information file as well as an assembly file containing a listing of the generated machine code.
In addition, it stores the interface of each module in a symbol file which is required when other modules import the module.
Programs generated with this compiler require additional runtime support that is stored in the \file{ob\-mips64\-run} library file.
\flowgraph{\resource{Oberon\\source code} \ar[r] & \toolbox{obmips64} \ar[r] \ar@/l/[d] \ar[rd] & \resource{object file} \\ \variable{ECSIMPORT} \ar[ru] & \resource{symbol\\files} \ar@/r/[u] & \resource{debugging\\information}}
\seeoberon\seeassembly\seemips\seeobject\seedebugging
}

\providecommand{\obmmix}{
\toolsection{obmmix} is a compiler for the Oberon programming language targeting the MMIX hardware architecture.
It generates machine code for MMIX processors from modules written in Oberon and stores it in corresponding object files.
For debugging purposes, it also creates a debugging information file as well as an assembly file containing a listing of the generated machine code.
In addition, it stores the interface of each module in a symbol file which is required when other modules import the module.
Programs generated with this compiler require additional runtime support that is stored in the \file{ob\-mmix\-run} library file.
\flowgraph{\resource{Oberon\\source code} \ar[r] & \toolbox{obmmix} \ar[r] \ar@/l/[d] \ar[rd] & \resource{object file} \\ \variable{ECSIMPORT} \ar[ru] & \resource{symbol\\files} \ar@/r/[u] & \resource{debugging\\information}}
\seeoberon\seeassembly\seemmix\seeobject\seedebugging
}

\providecommand{\oborok}{
\toolsection{obor1k} is a compiler for the Oberon programming language targeting the OpenRISC 1000 hardware architecture.
It generates machine code for OpenRISC 1000 processors from modules written in Oberon and stores it in corresponding object files.
For debugging purposes, it also creates a debugging information file as well as an assembly file containing a listing of the generated machine code.
In addition, it stores the interface of each module in a symbol file which is required when other modules import the module.
Programs generated with this compiler require additional runtime support that is stored in the \file{ob\-or1k\-run} library file.
\flowgraph{\resource{Oberon\\source code} \ar[r] & \toolbox{obor1k} \ar[r] \ar@/l/[d] \ar[rd] & \resource{object file} \\ \variable{ECSIMPORT} \ar[ru] & \resource{symbol\\files} \ar@/r/[u] & \resource{debugging\\information}}
\seeoberon\seeassembly\seeorok\seeobject\seedebugging
}

\providecommand{\obppca}{
\toolsection{obppc32} is a compiler for the Oberon programming language targeting the PowerPC hardware architecture.
It generates machine code for PowerPC processors from modules written in Oberon and stores it in corresponding object files.
The compiler generates machine code for the 32-bit operating mode defined by the PowerPC architecture.
For debugging purposes, it also creates a debugging information file as well as an assembly file containing a listing of the generated machine code.
In addition, it stores the interface of each module in a symbol file which is required when other modules import the module.
Programs generated with this compiler require additional runtime support that is stored in the \file{ob\-ppc32\-run} library file.
\flowgraph{\resource{Oberon\\source code} \ar[r] & \toolbox{obppc32} \ar[r] \ar@/l/[d] \ar[rd] & \resource{object file} \\ \variable{ECSIMPORT} \ar[ru] & \resource{symbol\\files} \ar@/r/[u] & \resource{debugging\\information}}
\seeoberon\seeassembly\seeppc\seeobject\seedebugging
}

\providecommand{\obppcb}{
\toolsection{obppc64} is a compiler for the Oberon programming language targeting the PowerPC hardware architecture.
It generates machine code for PowerPC processors from modules written in Oberon and stores it in corresponding object files.
The compiler generates machine code for the 64-bit operating mode defined by the PowerPC architecture.
For debugging purposes, it also creates a debugging information file as well as an assembly file containing a listing of the generated machine code.
In addition, it stores the interface of each module in a symbol file which is required when other modules import the module.
Programs generated with this compiler require additional runtime support that is stored in the \file{ob\-ppc64\-run} library file.
\flowgraph{\resource{Oberon\\source code} \ar[r] & \toolbox{obppc64} \ar[r] \ar@/l/[d] \ar[rd] & \resource{object file} \\ \variable{ECSIMPORT} \ar[ru] & \resource{symbol\\files} \ar@/r/[u] & \resource{debugging\\information}}
\seeoberon\seeassembly\seeppc\seeobject\seedebugging
}

\providecommand{\obrisc}{
\toolsection{obrisc} is a compiler for the Oberon programming language targeting the RISC hardware architecture.
It generates machine code for RISC processors from modules written in Oberon and stores it in corresponding object files.
For debugging purposes, it also creates a debugging information file as well as an assembly file containing a listing of the generated machine code.
In addition, it stores the interface of each module in a symbol file which is required when other modules import the module.
Programs generated with this compiler require additional runtime support that is stored in the \file{ob\-risc\-run} library file.
\flowgraph{\resource{Oberon\\source code} \ar[r] & \toolbox{obrisc} \ar[r] \ar@/l/[d] \ar[rd] & \resource{object file} \\ \variable{ECSIMPORT} \ar[ru] & \resource{symbol\\files} \ar@/r/[u] & \resource{debugging\\information}}
\seeoberon\seeassembly\seerisc\seeobject\seedebugging
}

\providecommand{\obwasm}{
\toolsection{obwasm} is a compiler for the Oberon programming language targeting the WebAssembly architecture.
It generates machine code for WebAssembly targets from modules written in Oberon and stores it in corresponding object files.
For debugging purposes, it also creates a debugging information file as well as an assembly file containing a listing of the generated machine code.
In addition, it stores the interface of each module in a symbol file which is required when other modules import the module.
Programs generated with this compiler require additional runtime support that is stored in the \file{ob\-wasm\-run} library file.
\flowgraph{\resource{Oberon\\source code} \ar[r] & \toolbox{obwasm} \ar[r] \ar@/l/[d] \ar[rd] & \resource{object file} \\ \variable{ECSIMPORT} \ar[ru] & \resource{symbol\\files} \ar@/r/[u] & \resource{debugging\\information}}
\seeoberon\seeassembly\seewasm\seeobject\seedebugging
}

% converter tools

\providecommand{\dbgdwarf}{
\toolsection{dbgdwarf} is a DWARF debugging information converter tool.
It converts debugging information into the DWARF debugging data format and stores it in corresponding object files~\cite{dwarffile}.
The resulting debugging object files can be combined with runtime support that creates Executable and Linking Format (ELF) files~\cite{elffile}.
\flowgraph{\resource{debugging\\information} \ar[r] & \toolbox{dbgdwarf} \ar[r] & \resource{debugging\\object file}}
\seeobject\seedebugging
}

% assembler tools

\providecommand{\asmprint}{
\toolsection{asmprint} is a pretty printer for generic assembly code.
It reformats generic assembly code and writes it to the standard output stream.
\flowgraph{\resource{generic assembly\\source code} \ar[r] & \toolbox{asmprint} \ar[r] & \resource{reformatted\\source code}}
\seeassembly
}

\providecommand{\amdaasm}{
\toolsection{amd16asm} is an assembler for the AMD64 hardware architecture.
It translates assembly code into machine code for AMD64 processors and stores it in corresponding object files.
By default, the assembler generates machine code for the 16-bit operating mode defined by the AMD64 architecture.
\flowgraph{\resource{AMD16 assembly\\source code} \ar[r] & \toolbox{amd16asm} \ar[r] & \resource{object file}}
\seeassembly\seeamd\seeobject
}

\providecommand{\amdadism}{
\toolsection{amd16dism} is a disassembler for the AMD64 hardware architecture.
It translates machine code from object files targeting AMD64 processors into assembly code and writes it to the standard output stream.
It assumes that the machine code was generated for the 16-bit operating mode defined by the AMD64 architecture.
\flowgraph{\resource{object file} \ar[r] & \toolbox{amd16dism} \ar[r] & \resource{disassembly\\listing}}
\seeassembly\seeamd\seeobject
}

\providecommand{\amdbasm}{
\toolsection{amd32asm} is an assembler for the AMD64 hardware architecture.
It translates assembly code into machine code for AMD64 processors and stores it in corresponding object files.
By default, the assembler generates machine code for the 32-bit operating mode defined by the AMD64 architecture.
\flowgraph{\resource{AMD32 assembly\\source code} \ar[r] & \toolbox{amd32asm} \ar[r] & \resource{object file}}
\seeassembly\seeamd\seeobject
}

\providecommand{\amdbdism}{
\toolsection{amd32dism} is a disassembler for the AMD64 hardware architecture.
It translates machine code from object files targeting AMD64 processors into assembly code and writes it to the standard output stream.
It assumes that the machine code was generated for the 32-bit operating mode defined by the AMD64 architecture.
\flowgraph{\resource{object file} \ar[r] & \toolbox{amd32dism} \ar[r] & \resource{disassembly\\listing}}
\seeassembly\seeamd\seeobject
}

\providecommand{\amdcasm}{
\toolsection{amd64asm} is an assembler for the AMD64 hardware architecture.
It translates assembly code into machine code for AMD64 processors and stores it in corresponding object files.
By default, the assembler generates machine code for the 64-bit operating mode defined by the AMD64 architecture.
\flowgraph{\resource{AMD64 assembly\\source code} \ar[r] & \toolbox{amd64asm} \ar[r] & \resource{object file}}
\seeassembly\seeamd\seeobject
}

\providecommand{\amdcdism}{
\toolsection{amd64dism} is a disassembler for the AMD64 hardware architecture.
It translates machine code from object files targeting AMD64 processors into assembly code and writes it to the standard output stream.
It assumes that the machine code was generated for the 64-bit operating mode defined by the AMD64 architecture.
\flowgraph{\resource{object file} \ar[r] & \toolbox{amd64dism} \ar[r] & \resource{disassembly\\listing}}
\seeassembly\seeamd\seeobject
}

\providecommand{\armaasm}{
\toolsection{arma32asm} is an assembler for the ARM hardware architecture.
It translates assembly code into machine code for ARM processors executing A32 instructions and stores it in corresponding object files.
\flowgraph{\resource{ARM A32 assembly\\source code} \ar[r] & \toolbox{arma32asm} \ar[r] & \resource{object file}}
\seeassembly\seearm\seeobject
}

\providecommand{\armadism}{
\toolsection{arma32dism} is a disassembler for the ARM hardware architecture.
It translates machine code from object files targeting ARM processors executing A32 instructions into assembly code and writes it to the standard output stream.
\flowgraph{\resource{object file} \ar[r] & \toolbox{arma32dism} \ar[r] & \resource{disassembly\\listing}}
\seeassembly\seearm\seeobject
}

\providecommand{\armbasm}{
\toolsection{arma64asm} is an assembler for the ARM hardware architecture.
It translates assembly code into machine code for ARM processors executing A64 instructions and stores it in corresponding object files.
\flowgraph{\resource{ARM A64 assembly\\source code} \ar[r] & \toolbox{arma64asm} \ar[r] & \resource{object file}}
\seeassembly\seearm\seeobject
}

\providecommand{\armbdism}{
\toolsection{arma64dism} is a disassembler for the ARM hardware architecture.
It translates machine code from object files targeting ARM processors executing A64 instructions into assembly code and writes it to the standard output stream.
\flowgraph{\resource{object file} \ar[r] & \toolbox{arma64dism} \ar[r] & \resource{disassembly\\listing}}
\seeassembly\seearm\seeobject
}

\providecommand{\armcasm}{
\toolsection{armt32asm} is an assembler for the ARM hardware architecture.
It translates assembly code into machine code for ARM processors executing T32 instructions and stores it in corresponding object files.
\flowgraph{\resource{ARM T32 assembly\\source code} \ar[r] & \toolbox{armt32asm} \ar[r] & \resource{object file}}
\seeassembly\seearm\seeobject
}

\providecommand{\armcdism}{
\toolsection{armt32dism} is a disassembler for the ARM hardware architecture.
It translates machine code from object files targeting ARM processors executing T32 instructions into assembly code and writes it to the standard output stream.
\flowgraph{\resource{object file} \ar[r] & \toolbox{armt32dism} \ar[r] & \resource{disassembly\\listing}}
\seeassembly\seearm\seeobject
}

\providecommand{\avrasm}{
\toolsection{avrasm} is an assembler for the AVR hardware architecture.
It translates assembly code into machine code for AVR processors and stores it in corresponding object files.
The identifiers \texttt{RXL}, \texttt{RXH}, \texttt{RYL}, \texttt{RYH}, \texttt{RZL}, and \texttt{RZH} are predefined and name the corresponding registers.
The identifiers \texttt{SPL} and \texttt{SPH} are also predefined and evaluate to the address of the corresponding registers.
\flowgraph{\resource{AVR assembly\\source code} \ar[r] & \toolbox{avrasm} \ar[r] & \resource{object file}}
\seeassembly\seeavr\seeobject
}

\providecommand{\avrdism}{
\toolsection{avrdism} is a disassembler for the AVR hardware architecture.
It translates machine code from object files targeting AVR processors into assembly code and writes it to the standard output stream.
\flowgraph{\resource{object file} \ar[r] & \toolbox{avrdism} \ar[r] & \resource{disassembly\\listing}}
\seeassembly\seeavr\seeobject
}

\providecommand{\avrttasm}{
\toolsection{avr32asm} is an assembler for the AVR32 hardware architecture.
It translates assembly code into machine code for AVR32 processors and stores it in corresponding object files.
\flowgraph{\resource{AVR32 assembly\\source code} \ar[r] & \toolbox{avr32asm} \ar[r] & \resource{object file}}
\seeassembly\seeavrtt\seeobject
}

\providecommand{\avrttdism}{
\toolsection{avr32dism} is a disassembler for the AVR32 hardware architecture.
It translates machine code from object files targeting AVR32 processors into assembly code and writes it to the standard output stream.
\flowgraph{\resource{object file} \ar[r] & \toolbox{avr32dism} \ar[r] & \resource{disassembly\\listing}}
\seeassembly\seeavrtt\seeobject
}

\providecommand{\mabkasm}{
\toolsection{m68kasm} is an assembler for the M68000 hardware architecture.
It translates assembly code into machine code for M68000 processors and stores it in corresponding object files.
\flowgraph{\resource{68000 assembly\\source code} \ar[r] & \toolbox{m68kasm} \ar[r] & \resource{object file}}
\seeassembly\seemabk\seeobject
}

\providecommand{\mabkdism}{
\toolsection{m68kdism} is a disassembler for the M68000 hardware architecture.
It translates machine code from object files targeting M68000 processors into assembly code and writes it to the standard output stream.
\flowgraph{\resource{object file} \ar[r] & \toolbox{m68kdism} \ar[r] & \resource{disassembly\\listing}}
\seeassembly\seemabk\seeobject
}

\providecommand{\miblasm}{
\toolsection{miblasm} is an assembler for the MicroBlaze hardware architecture.
It translates assembly code into machine code for MicroBlaze processors and stores it in corresponding object files.
\flowgraph{\resource{MicroBlaze assembly\\source code} \ar[r] & \toolbox{miblasm} \ar[r] & \resource{object file}}
\seeassembly\seemibl\seeobject
}

\providecommand{\mibldism}{
\toolsection{mibldism} is a disassembler for the MicroBlaze hardware architecture.
It translates machine code from object files targeting MicroBlaze processors into assembly code and writes it to the standard output stream.
\flowgraph{\resource{object file} \ar[r] & \toolbox{mibldism} \ar[r] & \resource{disassembly\\listing}}
\seeassembly\seemibl\seeobject
}

\providecommand{\mipsaasm}{
\toolsection{mips32asm} is an assembler for the MIPS32 hardware architecture.
It translates assembly code into machine code for MIPS32 processors and stores it in corresponding object files.
\flowgraph{\resource{MIPS32 assembly\\source code} \ar[r] & \toolbox{mips32asm} \ar[r] & \resource{object file}}
\seeassembly\seemips\seeobject
}

\providecommand{\mipsadism}{
\toolsection{mips32dism} is a disassembler for the MIPS32 hardware architecture.
It translates machine code from object files targeting MIPS32 processors into assembly code and writes it to the standard output stream.
\flowgraph{\resource{object file} \ar[r] & \toolbox{mips32dism} \ar[r] & \resource{disassembly\\listing}}
\seeassembly\seemips\seeobject
}

\providecommand{\mipsbasm}{
\toolsection{mips64asm} is an assembler for the MIPS64 hardware architecture.
It translates assembly code into machine code for MIPS64 processors and stores it in corresponding object files.
\flowgraph{\resource{MIPS64 assembly\\source code} \ar[r] & \toolbox{mips64asm} \ar[r] & \resource{object file}}
\seeassembly\seemips\seeobject
}

\providecommand{\mipsbdism}{
\toolsection{mips64dism} is a disassembler for the MIPS64 hardware architecture.
It translates machine code from object files targeting MIPS64 processors into assembly code and writes it to the standard output stream.
\flowgraph{\resource{object file} \ar[r] & \toolbox{mips64dism} \ar[r] & \resource{disassembly\\listing}}
\seeassembly\seemips\seeobject
}

\providecommand{\mmixasm}{
\toolsection{mmixasm} is an assembler for the MMIX hardware architecture.
It translates assembly code into machine code for MMIX processors and stores it in corresponding object files.
The names of all special registers are predefined and evaluate to the corresponding number.
\flowgraph{\resource{MMIX assembly\\source code} \ar[r] & \toolbox{mmixasm} \ar[r] & \resource{object file}}
\seeassembly\seemmix\seeobject
}

\providecommand{\mmixdism}{
\toolsection{mmixdism} is a disassembler for the MMIX hardware architecture.
It translates machine code from object files targeting MMIX processors into assembly code and writes it to the standard output stream.
\flowgraph{\resource{object file} \ar[r] & \toolbox{mmixdism} \ar[r] & \resource{disassembly\\listing}}
\seeassembly\seemmix\seeobject
}

\providecommand{\orokasm}{
\toolsection{or1kasm} is an assembler for the OpenRISC 1000 hardware architecture.
It translates assembly code into machine code for OpenRISC 1000 processors and stores it in corresponding object files.
\flowgraph{\resource{OpenRISC 1000 assembly\\source code} \ar[r] & \toolbox{or1kasm} \ar[r] & \resource{object file}}
\seeassembly\seeorok\seeobject
}

\providecommand{\orokdism}{
\toolsection{or1kdism} is a disassembler for the OpenRISC 1000 hardware architecture.
It translates machine code from object files targeting OpenRISC 1000 processors into assembly code and writes it to the standard output stream.
\flowgraph{\resource{object file} \ar[r] & \toolbox{or1kdism} \ar[r] & \resource{disassembly\\listing}}
\seeassembly\seeorok\seeobject
}

\providecommand{\ppcaasm}{
\toolsection{ppc32asm} is an assembler for the PowerPC hardware architecture.
It translates assembly code into machine code for PowerPC processors and stores it in corresponding object files.
By default, the assembler generates machine code for the 32-bit operating mode defined by the PowerPC architecture.
\flowgraph{\resource{PowerPC assembly\\source code} \ar[r] & \toolbox{ppc32asm} \ar[r] & \resource{object file}}
\seeassembly\seeppc\seeobject
}

\providecommand{\ppcadism}{
\toolsection{ppc32dism} is a disassembler for the PowerPC hardware architecture.
It translates machine code from object files targeting PowerPC processors into assembly code and writes it to the standard output stream.
It assumes that the machine code was generated for the 32-bit operating mode defined by the PowerPC architecture.
\flowgraph{\resource{object file} \ar[r] & \toolbox{ppc32dism} \ar[r] & \resource{disassembly\\listing}}
\seeassembly\seeppc\seeobject
}

\providecommand{\ppcbasm}{
\toolsection{ppc64asm} is an assembler for the PowerPC hardware architecture.
It translates assembly code into machine code for PowerPC processors and stores it in corresponding object files.
By default, the assembler generates machine code for the 64-bit operating mode defined by the PowerPC architecture.
\flowgraph{\resource{PowerPC assembly\\source code} \ar[r] & \toolbox{ppc64asm} \ar[r] & \resource{object file}}
\seeassembly\seeppc\seeobject
}

\providecommand{\ppcbdism}{
\toolsection{ppc64dism} is a disassembler for the PowerPC hardware architecture.
It translates machine code from object files targeting PowerPC processors into assembly code and writes it to the standard output stream.
It assumes that the machine code was generated for the 64-bit operating mode defined by the PowerPC architecture.
\flowgraph{\resource{object file} \ar[r] & \toolbox{ppc64dism} \ar[r] & \resource{disassembly\\listing}}
\seeassembly\seeppc\seeobject
}

\providecommand{\riscasm}{
\toolsection{riscasm} is an assembler for the RISC hardware architecture.
It translates assembly code into machine code for RISC processors and stores it in corresponding object files.
The names of all special registers are predefined and evaluate to the corresponding number.
\flowgraph{\resource{RISC assembly\\source code} \ar[r] & \toolbox{riscasm} \ar[r] & \resource{object file}}
\seeassembly\seerisc\seeobject
}

\providecommand{\riscdism}{
\toolsection{riscdism} is a disassembler for the RISC hardware architecture.
It translates machine code from object files targeting RISC processors into assembly code and writes it to the standard output stream.
\flowgraph{\resource{object file} \ar[r] & \toolbox{riscdism} \ar[r] & \resource{disassembly\\listing}}
\seeassembly\seerisc\seeobject
}

\providecommand{\wasmasm}{
\toolsection{wasmasm} is an assembler for the WebAssembly architecture.
It translates assembly code into machine code for WebAssembly targets and stores it in corresponding object files.
The names of all special registers are predefined and evaluate to the corresponding number.
\flowgraph{\resource{WebAssembly assembly\\source code} \ar[r] & \toolbox{wasmasm} \ar[r] & \resource{object file}}
\seeassembly\seewasm\seeobject
}

\providecommand{\wasmdism}{
\toolsection{wasmdism} is a disassembler for the WebAssembly architecture.
It translates machine code from object files targeting WebAssembly targets into assembly code and writes it to the standard output stream.
\flowgraph{\resource{object file} \ar[r] & \toolbox{wasmdism} \ar[r] & \resource{disassembly\\listing}}
\seeassembly\seewasm\seeobject
}

% linker tools

\providecommand{\linklib}{
\toolsection{linklib} is an object file combiner.
It creates a static library file by combining all object files given to it into a single one.
\flowgraph{\resource{object files} \ar[r] & \toolbox{linklib} \ar[r] & \resource{library file}}
\seeobject
}

\providecommand{\linkbin}{
\toolsection{linkbin} is a linker for plain binary files.
It links all object files given to it into a single image and stores it in a binary file that begins with the first linked section.
It also creates a map file that lists the address, type, name and size of all used sections.
The filename extension of the resulting binary file can be specified by putting it into a constant data section called \texttt{\_extension}.
\flowgraph{\resource{object files} \ar[r] & \toolbox{linkbin} \ar[r] \ar[d] & \resource{binary file} \\ & \resource{map file}}
\seeobject
}

\providecommand{\linkmem}{
\toolsection{linkmem} is a linker for plain binary files partitioned into random-access and read-only memory.
It links all object files given to it into two distinct images, one for data sections and one for code and constant data sections, and stores each image in a binary file that begins with the first linked section of the corresponding type.
It also creates a map file that lists the address, type, name and size of all used sections.
\flowgraph{\resource{object files} \ar[r] & \toolbox{linkmem} \ar[r] \ar[d] & \resource{RAM file/\\ROM file} \\ & \resource{map file}}
\seeobject
}

\providecommand{\linkprg}{
\toolsection{linkprg} is a linker for GEMDOS executable files.
It links all object files given to it into a single image and stores the image in an Atari GEMDOS executable file~\cite{gemdosfile}.
It also creates a map file that lists the address relative to the text segment, type, name and size of all used sections.
The filename extension of the resulting executable file can be specified by putting it into a constant data section called \texttt{\_extension}.
The GEMDOS executable file format requires all patch patterns of absolute link patches to consist of four full bitmasks with descending offsets.
\flowgraph{\resource{object files} \ar[r] & \toolbox{linkprg} \ar[r] \ar[d] & \resource{executable file} \\ & \resource{map file}}
\seeobject
}

\providecommand{\linkhex}{
\toolsection{linkhex} is a linker for Intel HEX files.
It links all code sections of the object files given to it into single image and stores the image in an Intel HEX file~\cite{hexfile} that begins with the first linked section.
It also creates a map file that lists the address, type, name and size of all used sections.
\flowgraph{\resource{object files} \ar[r] & \toolbox{linkhex} \ar[r] \ar[d] & \resource{HEX file} \\ & \resource{map file}}
\seeobject
}

\providecommand{\mapsearch}{
\toolsection{mapsearch} is a debugging tool.
It searches map files generated by linker tools for the name of a binary section that encompasses a memory address read from the standard input stream.
If additionally provided with one or more object files, it also stores an excerpt thereof in a separate object file called map search result which only contains the identified binary section for disassembling purposes.
\flowgraph{& \resource{map files/\\object files} \ar[d] \\ \resource{memory\\address} \ar[r] & \toolbox{mapsearch} \ar[r] \ar[d] & \resource{section name/\\relative offset} \\ & \resource{object file\\excerpt}}
\seeobject
}

\renewcommand{\seefalse}{}

\startchapter{FALSE}{User Manual for FALSE}{false}
{FALSE is a stack-oriented programming language that supports lambda abstractions and is quite powerful for its size.
Although FALSE is quite cryptic by design and therefore considered esoteric, it provides language features that enable interoperability with other programming languages.
This \documentation{} describes the language and its implementation by the \ecs{}.}

\epigraph{What is written without effort \\ is in general read without pleasure.}{Samuel Johnson}

\section{Introduction}

FALSE is an esoteric but still powerful programming language designed by Wouter van~Oortmerssen, named after his favorite truth value~\cite{oortmerssen1993}.

\begin{center}\fallogo{2em}\end{center}

Each FALSE program consists of comments enclosed in braces and expressions.
Expressions consist of symbols which are mostly represented as a single character.
Table~\ref{tab:falsymbols} lists all of these symbols and the way the operate on the stack and the 26 variables predefined by the FALSE programming language.
The notation $u, v \to s, t$ means that $u$ and $v$ are popped from the stack in that order and $s$ and $t$ are pushed onto it afterward.
An $\epsilon$ on either side of that notation denotes that the corresponding operation does not push or pop anything respectively.
The symbols listed under the category named special-purpose are actually implementation-defined and are explained in more detail in Section~\ref{sec:falimplementation}.

\begin{table}
\centering
\begin{tabular}{@{}llll@{}}
\toprule Category & Symbol & Description & Stack Operation \\
\midrule Literals
& \texttt{number} & Value of integer $n$ & $\epsilon \to n$ \\
& \texttt{'character} & Value of character $c$ & $\epsilon \to c$ \\
& \texttt{a\ldots z} & Address $a$ of variable & $\epsilon \to a$ \\
& \texttt{[\ldots ]} & Address $f$ of function & $\epsilon \to f$ \\
\midrule Arithmetic
& \texttt{\_} & Negation & $x \to -x$ \\
Operations
& \texttt{+} & Add & $y, x \to x + y$ \\
& \texttt{-} & Subtract & $y, x \rightarrow x - y$ \\
& \texttt{*} & Multiply & $y, x \rightarrow x \times y$ \\
& \texttt{/} & Divide & $y, x \rightarrow x \div y$ \\
\midrule Logical
& \texttt{\textasciitilde} & Complement & $x \to \neg x$ \\
Operations
& \texttt{\&} & And & $y, x \to x \land y$ \\
& \texttt{|} & Or & $y, x \to x \lor y$ \\
\midrule Comparison
& \texttt{=} & Test if equal & $y, x \to x = y$ \\
Operations
& \texttt{>} & Test if greater & $y, x \to x > y$ \\
\midrule Stack
& \texttt{\$} & Duplicate & $x \to x, x$ \\
Operations
& \texttt{\%} & Delete & $x \to \epsilon$ \\
& \texttt{\textbackslash} & Swap & $x, y \to x, y$ \\
& \texttt{@} & Rotate & $x, y, z \to y, x, z$ \\
& \texttt{\o} & Select & $n \to n^{\mathit th}$ element \\
\midrule Statements
& \texttt{:} & Write $x$ to address $a$ & $a, x \to \epsilon$ \\
& \texttt{;} & Read $x$ from address $a$ & $a \to x$ \\
& \texttt{!} & Call function $f$ & $f \to \epsilon$ \\
& \texttt{?} & Call function $f$ if $x$ is true & $f, x \to \epsilon$ \\
& \texttt{\#} & Call function $f$ while & $f, g \to \epsilon$ \\ & & function $g$ returns true \\
\midrule Input/
& \texttt{.} & Print $x$ as integer & $x \to \epsilon$ \\
Output
& \texttt{,} & Print $x$ as character & $x \to \epsilon$ \\
& \texttt{\^} & Read $x$ as character & $\epsilon \to x$ \\
& \texttt{"\ldots "} & Print string $s$ & $\epsilon \to \epsilon$ \\
& \texttt{\ss} & Flush output stream & $\epsilon \to \epsilon$ \\
\midrule Language
& \texttt{"\ldots "V} & Address $a$ of external variable $s$ & $\epsilon \to a$ \\
extensions
& \texttt{"\ldots "F} & Address $a$ of external function $s$ & $\epsilon \to a$ \\
& \texttt{"\ldots "\`} & Emit inline assembly & $\epsilon \to \epsilon$ \\
\bottomrule
\end{tabular}
\caption{The symbols of the FALSE programming language}
\label{tab:falsymbols}\index{Symbols, of FALSE}\index{FALSE!Symbols}
\end{table}

In general, there are three kinds of symbols which either push values onto the stack, change values on the stack, or just pop values from it in order to perform memory accesses or input and output.
The type of the actual value that is pushed onto or popped from the stack is defined by the operation performed by the symbol.
Arithmetic operations like \texttt{+} and \texttt{-} for example pop two integer values from the stack and push their result back onto the stack again.
A single character symbol like \texttt{a} or \texttt{b} on the other hand names a variable and pushes the address of that variable onto the stack.
The actual contents of the variable can be an integer or another address and is accessed using the dereference symbol \texttt{;}.
The third kind of symbols are statements like assignments that perform their operation without pushing any new values onto the stack.

\section{Implementation-Defined Behavior}\label{sec:falimplementation}

Some issues in the specification of the otherwise abstract programming language are either left unspecified or depend on its original implementation.
This section describes the concrete implementation specific behavior defined by the \ecs{}:

\begin{itemize}

\item
The actual size of the values stored on the stack and in variables is equal to the address size of the target hardware architecture.

\item
The logical and comparison operations generate Boolean values where false is represented using the integer zero and true is represented using the integer one.
For evaluations of Boolean values the integer zero denotes false while all other values denote true.

\item
The behavior of programs that pop more elements from the stack than were pushed onto it beforehand is undefined.
The same holds for programs that access memory using invalid addresses.

\item
After the execution of each program, the value on top of the stack is taken as the return code for the runtime environment.
In order to allow programs that do not explicitly push a return code, each program first pushes a value indicating successful execution.

\item
The original specification allowed emitting inline assembly for the Motorola M68000 family of microprocessors using integer numbers in the range 0 up to 65335 followed by an apostrophe.
The \ecs{} does not provide this form of code generation because it supports more than one hardware architecture.
All of them define their own instruction set encoding which affects the length of an instruction as well as endianness issues.

Instead, the \ecs{} allows emitting inline assembly by providing the actual inline assembly code as a string followed by an apostrophe.
This is a more generic solution which does not need the programmer to apply the instruction set encoding manually.
Nevertheless, representing binary code and data with plain integers is still possible using the double byte data directive.
\seeassembly\seemabk

\item
The \ecs{} defines two new operations that allow retrieving the address of variables and functions that are not defined by FALSE programs.
This enables interoperability with other languages as described in Section~\ref{sec:falinteroperability}.

\end{itemize}

Accessing external symbols as well as the emission of inline assembly is only available in the compilers tools provided by the \ecs{}.
The interpreter does not support these operations and issues a corresponding error message.

\section{FALSE Tools}

The \ecs{} provides several different tools that process programs written in FALSE\@.
\interface

The tools process FALSE programs in several consecutive stages.
In each stage, the internal representation of the program is changed and transformed.
Figure~\ref{fig:faldataflow} shows all stages and the different representations.

\begin{figure}
\flowgraph{
& \resource{FALSE\\source code} \ar[d] \\
& \converter{Lexer} \ar[d] \\
& \resource{tokens} \ar[d] \\
& \converter{Parser} \ar[d] \\
\converter{Serializer} \ar[d] & \resource{abstract\\syntax tree} \ar[l] \ar[d] \ar[r] & \converter{Pretty Printer} \ar[d] \\
\resource{internal\\representation} & \converter{Semantic\\Checker} \ar[d] & \resource{reformatted\\source code} \\
\converter{Interpreter} \ar@/l/[d] & \resource{attributed\\syntax tree} \ar[l] \ar[d] \ar[r] & \converter{Transpiler} \ar[d] \\
\resource{input/\\output} \ar@/r/[u] & \converter{Intermediate\\Code Emitter} \ar[d] & \resource{translated\\source code} \\
& \resource{intermediate\\code} \ar[d] \ar@/u/[r] & \converter{Optimizer} \ar@/d/[l] \\
& \converter{Machine Code\\Generator} \ar[d] \\
& \resource{object file} \\
}\caption{Data flow within the tools for FALSE}
\label{fig:faldataflow}
\end{figure}

\falprint
\falcheck
\faldump
\falrun
\falcpp
\falcode
\falamda
\falamdb
\falamdc
\falarma
\falarmb
\falarmc
\falarmcfpe
\falavr
\falavrtt
\falmabk
\falmibl
\falmipsa
\falmipsb
\falmmix
\falorok
\falppca
\falppcb
\falrisc
\falwasm
\falxtensa

\section{Interoperability}\label{sec:falinteroperability}

The compilers for FALSE enable interoperability with other programming languages implemented by the \ecs{}.
The interoperability is enabled by a common intermediate code representation and calling convention. \seecode
The compilers define several intermediate code sections for each program and maintain the following naming convention.

The main program is defined in a code section called \texttt{main}.
For each function inside the main program, the compilers define a code section called \texttt{function} followed by an integer index.
The index is incremented with each function discovered lexicographically in the source code and begins with zero.
In addition, each variable that is actually used is defined in a data section with the same name.

Accessing code and data sections that are defined elsewhere is enabled by two special-purpose symbols that push the corresponding address onto the stack.
Since all stack operations operate on the actual call stack, it is even possible to provide arguments for external functions.
However, the arguments are always passed as addresses.
The return value of a function is pushed onto the stack after the call.

Functions defined by the compilers can also be called from other programs.
They first pop the return address from the stack such that the top of the stack corresponds to the last argument passed by the caller.
In the end, right before returning to the caller, the top of the stack is taken as return value of the function.

In addition, the compilers for the FALSE programming language also allow writing inline assembly code.
The corresponding symbol is described in Section~\ref{sec:falimplementation} and enables arbitrary access to any section.
\seeassembly

\concludechapter

% User manual for Oberon
% Copyright (C) Florian Negele

% This file is part of the Eigen Compiler Suite.

% Permission is granted to copy, distribute and/or modify this document
% under the terms of the GNU Free Documentation License, Version 1.3
% or any later version published by the Free Software Foundation.

% You should have received a copy of the GNU Free Documentation License
% along with the ECS.  If not, see <https://www.gnu.org/licenses/>.

% Generic documentation utilities
% Copyright (C) Florian Negele

% This file is part of the Eigen Compiler Suite.

% Permission is granted to copy, distribute and/or modify this document
% under the terms of the GNU Free Documentation License, Version 1.3
% or any later version published by the Free Software Foundation.

% You should have received a copy of the GNU Free Documentation License
% along with the ECS.  If not, see <https://www.gnu.org/licenses/>.

\providecommand{\cpp}{C\texttt{++}}
\providecommand{\opt}{_\mathit{opt}}
\providecommand{\tool}[1]{\texttt{#1}}
\providecommand{\version}{Version 0.0.40}
\providecommand{\resource}[1]{*++\txt{#1}}
\providecommand{\ecs}{Eigen Compiler Suite}
\providecommand{\changed}[1]{\underline{#1}}
\providecommand{\toolbox}[1]{\converter{#1}}
\providecommand{\file}{}\renewcommand{\file}[1]{\texttt{#1}}
\providecommand{\alignright}{\hfill\linebreak[0]\hspace*{\fill}}
\providecommand{\converter}[1]{*++[F][F*:white][F,:gray]\txt{#1}}
\providecommand{\documentation}{\ifbook chapter\else document\fi}
\providecommand{\Documentation}{\ifbook Chapter\else Document\fi}
\providecommand{\variable}[1]{\resource{\texttt{\small#1}\\variable}}
\providecommand{\documentationref}[2]{\ifbook\ref{#1}\else``\href{#1}{#2}''~\cite{#1}\fi}
\providecommand{\objfile}[1]{\texttt{#1}\index[runtime]{#1 object file@\texttt{#1} object file}}
\providecommand{\libfile}[1]{\texttt{#1}\index[runtime]{#1 library file@\texttt{#1} library file}}
\providecommand{\epigraph}[2]{\ifbook\begin{quote}\flushright\textit{#1}\par--- #2\end{quote}\fi}
\providecommand{\environmentvariable}[1]{\texttt{#1}\index{Environment variables!#1@\texttt{#1}}}
\providecommand{\environment}[1]{\texttt{#1}\index[environment]{#1 environment@\texttt{#1} environment}}
\providecommand{\toolsection}{}\renewcommand{\toolsection}[1]{\subsection{#1}\label{\prefix:#1}\tool{#1}}
\providecommand{\instruction}{}\renewcommand{\instruction}[2]{\noindent\qquad\pdftooltip{\texttt{#1}}{#2}\refstepcounter{instruction}\par}
\providecommand{\flowgraph}{}\renewcommand{\flowgraph}[1]{\par\sffamily\begin{displaymath}\xymatrix@=4ex{#1}\end{displaymath}\normalfont\par}
\providecommand{\instructionset}{}\renewcommand{\instructionset}[4]{\setcounter{instruction}{0}\begin{multicols}{\ifbook#3\else#4\fi}[{\captionof{table}[#2]{#2 (\ref*{#1:instructions}~instructions)}\label{tab:#1set}\vspace{-2ex}}]\footnotesize\raggedcolumns\input{#1.set}\label{#1:instructions}\end{multicols}}

\providecommand{\gpl}{GNU General Public License}
\providecommand{\rse}{ECS Runtime Support Exception}
\providecommand{\fdl}{\href{https://www.gnu.org/licenses/fdl.html}{GNU Free Documentation License}}

\providecommand{\docbegin}{}
\providecommand{\docend}{}
\providecommand{\doclabel}[1]{\hypertarget{#1}}
\providecommand{\doclink}[2]{\hyperlink{#1}{#2}}
\providecommand{\docsection}[3]{\hypertarget{#1}{\subsection{#2}}\label{sec:#1}\index[library]{#2@#3}}
\providecommand{\docsectionstar}[1]{}
\providecommand{\docsubbegin}{\begin{description}}
\providecommand{\docsubend}{\end{description}}
\providecommand{\docsubsection}[3]{\item[\hypertarget{#1}{#2}]\index[library]{#2@#3}}
\providecommand{\docsubsectionstar}[1]{\smallskip}
\providecommand{\docsubsubsection}[3]{\docsubsection{#1}{#2}{#3}}
\providecommand{\docsubsubsectionstar}[1]{}
\providecommand{\docsubsubsubsection}[3]{}
\providecommand{\docsubsubsubsectionstar}[1]{}
\providecommand{\doctable}{}

\providecommand{\debuggingtool}{}\renewcommand{\debuggingtool}{This tool is provided for debugging purposes.
It allows exposing and modifying an internal data structure that is usually not accessible.
}

\providecommand{\interface}{All tools accept command-line arguments which are taken as names of plain text files containing the source code.
If no arguments are provided, the standard input stream is used instead.
Output files are generated in the current working directory and have the same name as the input file being processed whereas the filename extension gets replaced by an appropriate suffix.
\seeinterface
}

\providecommand{\license}{\noindent Copyright \copyright{} Florian Negele\par\medskip\noindent
Permission is granted to copy, distribute and/or modify this document under the terms of the
\fdl{}, Version 1.3 or any later version published by the \href{https://fsf.org/}{Free Software Foundation}.
}

\providecommand{\ecslogosurface}{
\fill[darkgray] (0,0,0) -- (0,0,3) -- (0,3,3) -- (0,3,1) -- (0,4,1) -- (0,4,3) -- (0,5,3) -- (0,5,0) -- (0,2,0) -- (0,2,2) -- (0,1,2) -- (0,1,0) -- cycle;
\fill[gray] (0,5,0) -- (0,5,3) -- (1,5,3) -- (1,5,1) -- (2,5,1) -- (2,5,3) -- (3,5,3) -- (3,5,0) -- cycle;
\fill[lightgray] (0,0,0) -- (0,1,0) -- (2,1,0) -- (2,4,0) -- (1,4,0) -- (1,3,0) -- (2,3,0) -- (2,2,0) -- (0,2,0) -- (0,5,0) -- (3,5,0) -- (3,0,0) -- cycle;
\begin{scope}[line width=0.5]
\begin{scope}[gray]
\draw (0,0,0) -- (0,1,0);
\draw (2,1,0) -- (2,2,0);
\draw (0,1,2) -- (0,2,2);
\draw (0,2,0) -- (0,5,0);
\draw (2,3,0) -- (2,4,0);
\end{scope}
\begin{scope}[lightgray]
\draw (0,1,0) -- (0,1,2);
\draw (0,3,1) -- (0,3,3);
\draw (0,5,0) -- (0,5,3);
\draw (2,5,1) -- (2,5,3);
\end{scope}
\begin{scope}[white]
\draw (0,1,0) -- (2,1,0);
\draw (1,3,0) -- (2,3,0);
\draw (0,5,0) -- (3,5,0);
\end{scope}
\end{scope}
}

\providecommand{\ecslogo}[1]{
\begin{tikzpicture}[scale={(#1)/((sin(45)+cos(45))*3cm)},x={({-cos(45)*1cm},{sin(45)*sin(30)*1cm})},y={({0cm},{(cos(30)*1cm})},z={({sin(45)*1cm},{cos(45)*sin(30)*1cm})}]
\begin{scope}[darkgray,line width=1]
\draw (0,0,0) -- (0,0,3) -- (0,3,3) -- (2,3,3) -- (2,5,3) -- (3,5,3) -- (3,5,0) -- (3,0,0) -- cycle;
\draw (0,3,1) -- (0,4,1) -- (0,4,3) -- (0,5,3) -- (1,5,3) -- (1,5,1) -- (2,5,1);
\draw (1,3,0) -- (1,4,0) -- (2,4,0);
\end{scope}
\fill[darkgray] (2,0,0) -- (2,0,3) -- (2,5,3) -- (2,5,1) -- (2,4,1) -- (2,4,0) -- cycle;
\fill[lightgray] (2,0,2) -- (0,0,2) -- (0,2,2) -- (2,2,2) -- cycle;
\fill[gray] (0,1,0) -- (2,1,0) -- (2,1,2) -- (0,1,2) -- cycle;
\fill[gray] (0,3,1) -- (0,3,3) -- (2,3,3) -- (2,3,0) -- (1,3,0) -- (1,3,1) -- cycle;
\ecslogosurface
\end{tikzpicture}
}

\providecommand{\shadowedecslogo}[3]{
\begin{tikzpicture}[scale={(#1)/((sin(#2)+cos(#2))*3cm)},x={({-cos(#2)*1cm},{sin(#2)*sin(#3)*1cm})},y={({0cm},{(cos(#3)*1cm})},z={({sin(#2)*1cm},{cos(#2)*sin(#3)*1cm})}]
\shade[top color=lightgray!50!white,bottom color=white,middle color=lightgray!50!white] (0,0,0) -- (3,0,0) -- (3,{-0.5-3*sin(#2)*sin(#3)/cos(#3)},0) -- (0,-0.5,0) -- cycle;
\shade[top color=darkgray!50!gray,bottom color=white,middle color=darkgray!50!white] (0,0,0) -- (0,0,3) -- (0,{-0.5-3*cos(#2)*sin(#3)/cos(#3)},3) -- (0,-0.5,0) -- cycle;
\begin{scope}[y={({(cos(#2)+sin(#2))*0.5cm},{(cos(#2)*sin(#3)-sin(#2)*sin(#3))*0.5cm})}]
\useasboundingbox (3,0,0) -- (0,0,0) -- (0,0,3);
\shade[left color=darkgray!80!black,right color=lightgray,middle color=gray] (0,0,0) -- (0,1,0) -- (0,1,0.5) -- (0,2,0) -- (0,5,0) -- (0,5,3) -- (1,5,3) -- (1,4,3) -- (1,4,2.5) -- (1,3,3) -- (2,5,3) -- (3,5,3) -- (3,0,3) -- cycle;
\clip (0,0,0) -- (0,0,3) -- ({-3*sin(#2)/cos(#2)},0,0) -- cycle;
\shade[left color=darkgray,right color=lightgray!50!gray] (0,0,0) -- (0,1,0) -- (0,1,0.5) -- (0,2,0) -- (0,5,0) -- (0,5,3) -- (1,5,3) -- (1,4,3) -- (1,4,2.5) -- (1,3,3) -- (2,5,3) -- (3,5,3) -- (3,0,3) -- cycle;
\end{scope}
\shade[left color=darkgray,right color=darkgray!80!black] (2,0,0) -- (2,0,3) -- (2,5,3) -- (2,5,1) -- (2,4,1) -- (2,4,0) -- cycle;
\shade[left color=darkgray!90!black,right color=gray!80!darkgray] (2,0,2) -- (0,0,2) -- (0,2,2) -- (2,2,2) -- cycle;
\shade[top color=darkgray!90!black,bottom color=gray!80!darkgray] (0,1,0) -- (2,1,0) -- (2,1,2) -- (0,1,2) -- cycle;
\shade[top color=darkgray!90!black,bottom color=gray!80!darkgray] (0,3,1) -- (0,3,3) -- (2,3,3) -- (2,3,0) -- (1,3,0) -- (1,3,1) -- cycle;
\fill[gray] (2,1,0) -- (1.5,1,0.5) -- (0,1,0.5) -- (0,1,0) -- cycle;
\fill[gray] (1,3,2) -- (0.5,3,2) -- (0.5,3,3) -- (1,3,3) -- cycle;
\fill[gray] (2,3,0) -- (1.5,3,0.5) -- (1,3,0.5) -- (1,3,0) -- cycle;
\ecslogosurface
\end{tikzpicture}
}

\providecommand{\cpplogo}[1]{
\begin{tikzpicture}[scale=(#1)/512em]
\fill[gray] (435.2794,398.7159) -- (247.1911,507.3075) .. controls (236.3563,513.5642) and (218.6240,513.5642) .. (207.7892,507.3075) -- (19.7009,398.7159) .. controls (8.8646,392.4606) and (0.0000,377.1043) .. (0.0000,364.5924) -- (0.0000,147.4076) .. controls (0.8430,132.8363) and (8.2856,120.7683) .. (19.7009,113.2842) -- (207.7892,4.6926) .. controls (218.6240,-1.5642) and (236.3564,-1.5642) .. (247.1911,4.6926) -- (435.2794,113.2842) .. controls (447.5273,121.4304) and (454.4987,133.6918) .. (454.9803,147.4076) -- (454.9803,364.5924) .. controls (454.5404,377.7571) and (446.6566,391.0351) .. (435.2794,398.7159) -- cycle(75.8301,255.9993) .. controls (74.9389,404.0881) and (273.2892,469.4783) .. (358.8263,331.8769) -- (293.1917,293.8965) .. controls (253.5702,359.4301) and (155.1909,335.9977) .. (151.6601,255.9993) .. controls (152.7204,182.2703) and (249.4137,148.0211) .. (293.1961,218.1065) -- (358.8308,180.1276) .. controls (283.4477,49.2645) and (79.6318,96.3470) .. (75.8301,255.9993) -- cycle(379.1503,247.5747) -- (362.2982,247.5747) -- (362.2982,230.7226) -- (345.4490,230.7226) -- (345.4490,247.5747) -- (328.5969,247.5747) -- (328.5969,264.4254) -- (345.4490,264.4254) -- (345.4490,281.2759) -- (362.2982,281.2759) -- (362.2982,264.4254) -- (379.1503,264.4254) -- cycle(442.3420,247.5747) -- (425.4899,247.5747) -- (425.4899,230.7226) -- (408.6408,230.7226) -- (408.6408,247.5747) -- (391.7886,247.5747) -- (391.7886,264.4254) -- (408.6408,264.4254) -- (408.6408,281.2759) -- (425.4899,281.2759) -- (425.4899,264.4254) -- (442.3420,264.4254) -- cycle;
\end{tikzpicture}
}

\providecommand{\fallogo}[1]{
\begin{tikzpicture}[scale=(#1)/512em]
\fill[gray] (185.7774,0.0000) .. controls (200.4486,15.9798) and (226.8966,8.7148) .. (235.0426,31.5836) .. controls (249.5297,58.0598) and (247.9581,97.9161) .. (280.3335,110.9762) .. controls (309.1690,120.3496) and (337.8406,104.2727) .. (366.5753,103.9379) .. controls (373.4449,111.5171) and (379.2885,128.2574) .. (383.9755,108.9744) .. controls (396.6979,102.5615) and (437.2808,107.6681) .. (426.9652,124.3252) .. controls (408.9822,121.0785) and (412.4742,146.0729) .. (426.5192,131.4996) .. controls (433.8413,120.8489) and (465.1541,126.5522) .. (441.9067,135.7950) .. controls (396.1879,157.7478) and (344.1112,161.5079) .. (298.5528,183.5702) .. controls (277.7471,193.5198) and (284.6941,218.7163) .. (285.2127,236.9640) .. controls (292.3599,316.2826) and (307.3929,394.6311) .. (317.1198,473.6154) .. controls (329.0637,505.4736) and (292.1195,528.5004) .. (265.9183,511.2761) .. controls (237.9284,499.2462) and (237.3684,465.2681) .. (230.9102,439.9421) .. controls (218.6692,374.3397) and (215.6307,306.9662) .. (198.1732,242.3977) .. controls (183.1379,232.7444) and (164.4245,256.0298) .. (149.0430,261.4799) .. controls (116.9328,279.2585) and (87.1822,308.5851) .. (48.2293,307.8914) .. controls (21.3220,306.9037) and (-15.9107,281.8761) .. (7.2921,252.7908) .. controls (29.7799,220.6177) and (67.5177,204.3028) .. (100.9287,185.9449) .. controls (130.8217,170.8906) and (161.1548,156.5903) .. (191.0278,141.5847) .. controls (196.1738,120.0520) and (186.6049,95.2409) .. (186.8382,72.4353) .. controls (185.5234,48.4204) and (183.1700,23.9341) .. (185.7774,0.0000) -- cycle;
\end{tikzpicture}
}

\providecommand{\oblogo}[1]{
\begin{tikzpicture}[scale=(#1)/512em]
\fill[gray] (160.3865,208.9117) .. controls (154.0879,214.6478) and (149.0735,221.2409) .. (145.4125,228.5384) .. controls (184.8790,248.4273) and (234.7122,269.8787) .. (297.5493,291.8782) .. controls (300.3943,281.4769) and (300.9552,268.7619) .. (300.4023,255.2389) .. controls (248.9909,244.7891) and (200.0310,225.9279) .. (160.3865,208.9117) -- cycle(225.7398,392.6996) .. controls (308.0209,392.1716) and (359.3326,345.9277) .. (368.7203,285.2098) .. controls (376.6742,197.1784) and (311.7194,141.3342) .. (205.4287,142.1456) .. controls (139.9485,141.4804) and (88.7155,166.1957) .. (73.5775,228.0086) .. controls (52.0297,320.3408) and (123.4078,391.0103) .. (225.7398,392.6996) -- cycle(216.0739,176.4733) .. controls (268.9183,179.2424) and (315.8292,206.5488) .. (312.7454,265.1139) .. controls (313.2769,315.6384) and (286.5993,353.4946) .. (216.6040,355.7934) .. controls (162.4657,355.7934) and (126.0914,317.5023) .. (126.0914,260.5103) .. controls (126.1733,214.2900) and (163.3363,176.2849) .. (216.0739,176.4733) -- cycle(76.4897,189.1754) .. controls (13.1586,147.5631) and (0.0000,119.4207) .. (0.0000,119.4207) -- (90.6499,170.1632) .. controls (85.3004,175.8497) and (80.5994,182.1633) .. (76.4897,189.1754) -- cycle(353.9486,119.3004) -- (402.9482,119.3004) .. controls (427.0025,137.0797) and (450.9893,162.7034) .. (474.9529,191.0213) .. controls (509.3540,228.5339) and (531.3391,294.2091) .. (487.8149,312.1206) .. controls (462.8165,324.7652) and (394.3874,316.8943) .. (373.8912,313.6651) .. controls (379.9291,297.7449) and (383.2899,278.4204) .. (381.4989,257.7214) .. controls (420.3069,248.0321) and (421.9610,218.3461) .. (407.7867,192.6417) .. controls (391.1113,162.4018) and (370.1114,132.9097) .. (353.9486,119.3004) -- cycle;
\end{tikzpicture}
}

\providecommand{\markuptable}{
\begin{table}
\sffamily\centering
\begin{tabular}{@{}lcl@{}}
\toprule
\texttt{//italics//} & $\rightarrow$ & \textit{italics} \\
\midrule
\texttt{**bold**} & $\rightarrow$ & \textbf{bold} \\
\midrule
\texttt{\# ordered list} & & 1 ordered list \\
\texttt{\# second item} & $\rightarrow$ & 2 second item \\
\texttt{\#\# sub item} & & \hspace{1em} 1 sub item \\
\midrule
\texttt{* unordered list} & & $\bullet$ unordered list \\
\texttt{* second item} & $\rightarrow$ & $\bullet$ second item \\
\texttt{** sub item} & & \hspace{1em} $\bullet$ sub item \\
\midrule
\texttt{link to [[label]]} & $\rightarrow$ & link to \underline{label} \\
\midrule
\texttt{<{}<label>{}> definition } & $\rightarrow$ & definition \\
\midrule
\texttt{[[url|link name]]} & $\rightarrow$ & \underline{link name} \\
\midrule\addlinespace
\texttt{= large heading} & & {\Large large heading} \smallskip \\
\texttt{== medium heading} & $\rightarrow$ & {\large medium heading} \\
\texttt{=== small heading} & & small heading \\
\midrule
\texttt{no line break} & & no line break for paragraphs \\
\texttt{for paragraphs} & $\rightarrow$ \\
& & use empty line \\
\texttt{use empty line} \\
\midrule
\texttt{force\textbackslash\textbackslash line break} & $\rightarrow$ & force \\
& & line break \\
\midrule
\texttt{horizontal line} & $\rightarrow$ & horizontal line \\
\texttt{----} & & \hrulefill \\
\midrule
\texttt{|=a|=table|=header} & & \underline{a \enspace table \enspace header} \\
\texttt{|a|table|row} & $\rightarrow$ & a \enspace table \enspace row \\
\texttt{|b|table|row} & & b \enspace table \enspace row \\
\midrule
\texttt{\{\{\{} \\
\texttt{unformatted} & $\rightarrow$ & \texttt{unformatted} \\
\texttt{code} & & \texttt{code} \\
\texttt{\}\}\}} \\
\midrule\addlinespace
\texttt{@ new article} & & {\Large 1.\ new article} \smallskip \\
\texttt{@ second article} & $\rightarrow$ & {\Large 2.\ second article} \smallskip \\
\texttt{@@ sub article} & & {\large 2.1.\ sub article} \\
\bottomrule
\end{tabular}
\normalfont\caption{Elements of the generic documentation markup language}
\label{tab:docmarkup}
\end{table}
}

\providecommand{\startchapter}[4]{
\documentclass[11pt,a4paper]{article}
\usepackage{booktabs}
\usepackage[format=hang,labelfont=bf]{caption}
\usepackage{changepage}
\usepackage[T1]{fontenc}
\usepackage[margin=2cm]{geometry}
\usepackage{hyperref}
\usepackage[american]{isodate}
\usepackage{lmodern}
\usepackage{longtable}
\usepackage{mathptmx}
\usepackage{microtype}
\usepackage[toc]{multitoc}
\usepackage{multirow}
\usepackage[all]{nowidow}
\usepackage{pdfcomment}
\usepackage{syntax}
\usepackage{tikz}
\usepackage[all]{xy}
\hypersetup{pdfborder={0 0 0},bookmarksnumbered=true,pdftitle={\ecs{}: #2},pdfauthor={Florian Negele},pdfsubject={\ecs{}},pdfkeywords={#1}}
\setlength{\grammarindent}{8em}\setlength{\grammarparsep}{0.2ex}
\setlength{\columnsep}{2em}
\newcommand{\prefix}{}
\newcounter{instruction}
\bibliographystyle{unsrt}
\renewcommand{\index}[2][]{}
\renewcommand{\arraystretch}{1.05}
\renewcommand{\floatpagefraction}{0.7}
\renewcommand{\syntleft}{\itshape}\renewcommand{\syntright}{}
\title{\vspace{-5ex}\Huge{\ecs{}}\medskip\hrule}
\author{\huge{#2}}
\date{\medskip\version}
\newif\ifbook\bookfalse
\pagestyle{headings}
\frenchspacing
\begin{document}
\maketitle\thispagestyle{empty}\noindent#4\setlength{\columnseprule}{0.4pt}\tableofcontents\setlength{\columnseprule}{0pt}\vfill\pagebreak[3]\null\vfill\bigskip\noindent
\parbox{\textwidth-4em}{\license The contents of this \documentation{} are part of the \href{manual}{\ecs{} User Manual}~\cite{manual} and correspond to Chapter ``\href{manual\##3}{#1}''.\alignright\mbox{\today}}
\parbox{4em}{\flushright\ecslogo{3em}}
\clearpage
}

\providecommand{\concludechapter}{
\vfill\pagebreak[3]\null\vfill
\thispagestyle{myheadings}\markright{REFERENCES}
\noindent\begin{minipage}{\textwidth}\begin{multicols}{2}[\section*{References}]
\renewcommand{\section}[2]{}\small\bibliography{references}
\end{multicols}\end{minipage}\end{document}
}

\providecommand{\startpresentation}[2]{
\documentclass[14pt,aspectratio=43,usepdftitle=false]{beamer}
\usepackage{booktabs}
\usepackage{etex}
\usepackage{multicol}
\usepackage{tikz}
\usepackage[all]{xy}
\bibliographystyle{unsrt}
\setlength{\columnsep}{1em}
\setlength{\leftmargini}{1em}
\setbeamercolor{title}{fg=black}
\setbeamercolor{structure}{fg=darkgray}
\setbeamercolor{bibliography item}{fg=darkgray}
\setbeamerfont{title}{series=\bfseries}
\setbeamerfont{subtitle}{series=\normalfont}
\setbeamerfont*{frametitle}{parent=title}
\setbeamerfont{block title}{series=\bfseries}
\setbeamerfont*{framesubtitle}{parent=subtitle}
\setbeamersize{text margin left=1em,text margin right=1em}
\setbeamertemplate{navigation symbols}{}
\setbeamertemplate{itemize item}[circle]{}
\setbeamertemplate{bibliography item}[triangle]{}
\setbeamertemplate{bibliography entry author}{\usebeamercolor[fg]{bibliography item}}
\setbeamertemplate{frametitle}{\medskip\usebeamerfont{frametitle}\color{gray}\raisebox{-2.5ex}[0ex][0ex]{\rule{0.1em}{4.5ex}}}
\addtobeamertemplate{frametitle}{}{\hspace{0.4em}\usebeamercolor[fg]{title}\insertframetitle\par\vspace{0.2ex}\hspace{0.5em}\usebeamerfont{framesubtitle}\insertframesubtitle}
\hypersetup{pdfborder={0 0 0},bookmarksnumbered=true,bookmarksopen=true,bookmarksopenlevel=0,pdftitle={\ecs{}: #1},pdfauthor={Florian Negele},pdfsubject={\ecs{}},pdfkeywords={#1}}
\renewcommand{\flowgraph}[1]{\resizebox{\textwidth}{!}{$$\xymatrix{##1}$$}}
\title{\ecs{}\medskip\hrule\medskip}
\institute{\shadowedecslogo{5em}{30}{15}}
\date{\version}
\subtitle{#1}
\begin{document}
\begin{frame}[plain]\titlepage\nocite{manual}\end{frame}
\begin{frame}{Contents}{#1}\begin{center}\tableofcontents\end{center}\end{frame}
}

\providecommand{\concludepresentation}{
\begin{frame}{References}\begin{footnotesize}\setlength{\columnseprule}{0.4pt}\begin{multicols}{2}\bibliography{references}\end{multicols}\end{footnotesize}\end{frame}
\end{document}
}

\providecommand{\startbook}[1]{
\documentclass[10pt,paper=17cm:24cm,DIV=13,twoside=semi,headings=normal,numbers=noendperiod,cleardoublepage=plain]{scrbook}
\usepackage{atveryend}
\usepackage{booktabs}
\usepackage{caption}
\usepackage{changepage}
\usepackage[T1]{fontenc}
\usepackage{imakeidx}
\usepackage{hyperref}
\usepackage[american]{isodate}
\usepackage{lmodern}
\usepackage{longtable}
\usepackage{mathptmx}
\usepackage[final]{microtype}
\usepackage{multicol}
\usepackage{multirow}
\usepackage[all]{nowidow}
\usepackage{pdfcomment}
\usepackage{scrlayer-scrpage}
\usepackage{setspace}
\usepackage{syntax}
\usepackage[eventxtindent=4pt,oddtxtexdent=4pt]{thumbs}
\usepackage{tikz}
\usepackage[all]{xy}
\hyphenation{Micro-Blaze Open-Cores Open-RISC Power-PC}
\hypersetup{pdfborder={0 0 0},bookmarksnumbered=true,bookmarksopen=true,bookmarksopenlevel=0,pdftitle={\ecs{}: #1},pdfauthor={Florian Negele},pdfsubject={\ecs{}},pdfkeywords={#1}}
\setlength{\grammarindent}{8em}\setlength{\grammarparsep}{0.7ex}
\setkomafont{captionlabel}{\usekomafont{descriptionlabel}}
\renewcommand{\arraystretch}{1.05}\setstretch{1.1}
\renewcommand{\chapterformat}{\thechapter\autodot\enskip\raisebox{-1ex}[0ex][0ex]{\color{gray}\rule{0.1em}{3.5ex}}\enskip}
\renewcommand{\startchapter}[4]{\hypertarget{##3}{\chapter{##1}}\label{##3}##4\addthumb{##1}{\LARGE\sffamily\bfseries\thechapter}{white}{gray}\renewcommand{\prefix}{##3}}
\renewcommand{\concludechapter}{\clearpage{\stopthumb\cleardoublepage}}
\renewcommand{\syntleft}{\itshape}\renewcommand{\syntright}{}
\renewcommand{\floatpagefraction}{0.7}
\renewcommand{\partheademptypage}{}
\DeclareMicrotypeAlias{lmss}{cmr}
\newcommand{\prefix}{}
\newcounter{instruction}
\bibliographystyle{unsrt}
\newif\ifbook\booktrue
\makeindex[intoc,title=Index]
\makeindex[intoc,name=tools,title=Index of Tools,columns=3]
\makeindex[intoc,name=library,title=Index of Library Names]
\makeindex[intoc,name=runtime,title=Index of Runtime Support]
\makeindex[intoc,name=environment,title=Index of Target Environments]
\indexsetup{toclevel=chapter,headers={\indexname}{\indexname}}
\frenchspacing
\begin{document}
\pagenumbering{alph}
\begin{titlepage}\centering
\huge\sffamily\null\vfill\textbf{\ecs{}}\bigskip\hrule\bigskip#1
\normalsize\normalfont\vfill\vfill\shadowedecslogo{10em}{30}{15}
\large\vfill\vfill\version
\end{titlepage}
\null\vfill
\thispagestyle{empty}
\noindent\today\par\medskip
\license A copy of this license is included in Appendix~\ref{fdl} on page~\pageref{fdl}.
All product names used herein are for identification purposes only and may be trademarks of their respective companies.
\concludechapter
\frontmatter
\setcounter{tocdepth}{1}
\tableofcontents
\setcounter{tocdepth}{2}
\concludechapter
\listoffigures
\concludechapter
\listoftables
\concludechapter
}

\providecommand{\concludebook}{
\backmatter
\addtocontents{toc}{\protect\setcounter{tocdepth}{-1}}
\phantomsection\addcontentsline{toc}{part}{Bibliography}
\bibliography{references}
\concludechapter
\phantomsection\addcontentsline{toc}{part}{Indexes}
\printindex
\concludechapter
\indexprologue{\label{idx:tools}}
\printindex[tools]
\concludechapter
\printindex[library]
\concludechapter
\indexprologue{\label{idx:runtime}}
\printindex[runtime]
\concludechapter
\indexprologue{\label{idx:environment}}
\printindex[environment]
\concludechapter
\pagestyle{empty}\pagenumbering{Alph}\null\clearpage
\null\vfill\centering\ecslogo{4em}\par\medskip\license
\end{document}
}

% chapter references

\providecommand{\seedocumentationref}{}\renewcommand{\seedocumentationref}[3]{#1, see \Documentation{}~\documentationref{#2}{#3}. }
\providecommand{\seeinterface}{}\renewcommand{\seeinterface}{\ifbook See \Documentation{}~\documentationref{interface}{User Interface} for more information about the common user interface of all of these tools. \fi}
\providecommand{\seeguide}{}\renewcommand{\seeguide}{\seedocumentationref{For basic examples of using some of these tools in practice}{guide}{User Guide}}
\providecommand{\seecpp}{}\renewcommand{\seecpp}{\seedocumentationref{For more information about the \cpp{} programming language and its implementation by the \ecs{}}{cpp}{User Manual for \cpp{}}}
\providecommand{\seefalse}{}\renewcommand{\seefalse}{\seedocumentationref{For more information about the FALSE programming language and its implementation by the \ecs{}}{false}{User Manual for FALSE}}
\providecommand{\seeoberon}{}\renewcommand{\seeoberon}{\seedocumentationref{For more information about the Oberon programming language and its implementation by the \ecs{}}{oberon}{User Manual for Oberon}}
\providecommand{\seeassembly}{}\renewcommand{\seeassembly}{\seedocumentationref{For more information about the generic assembly language and how to use it}{assembly}{Generic Assembly Language Specification}}
\providecommand{\seeamd}{}\renewcommand{\seeamd}{\seedocumentationref{For more information about how the \ecs{} supports the AMD64 hardware architecture}{amd64}{AMD64 Hardware Architecture Support}}
\providecommand{\seearm}{}\renewcommand{\seearm}{\seedocumentationref{For more information about how the \ecs{} supports the ARM hardware architecture}{arm}{ARM Hardware Architecture Support}}
\providecommand{\seeavr}{}\renewcommand{\seeavr}{\seedocumentationref{For more information about how the \ecs{} supports the AVR hardware architecture}{avr}{AVR Hardware Architecture Support}}
\providecommand{\seeavrtt}{}\renewcommand{\seeavrtt}{\seedocumentationref{For more information about how the \ecs{} supports the AVR32 hardware architecture}{avr32}{AVR32 Hardware Architecture Support}}
\providecommand{\seemabk}{}\renewcommand{\seemabk}{\seedocumentationref{For more information about how the \ecs{} supports the M68000 hardware architecture}{m68k}{M68000 Hardware Architecture Support}}
\providecommand{\seemibl}{}\renewcommand{\seemibl}{\seedocumentationref{For more information about how the \ecs{} supports the MicroBlaze hardware architecture}{mibl}{MicroBlaze Hardware Architecture Support}}
\providecommand{\seemips}{}\renewcommand{\seemips}{\seedocumentationref{For more information about how the \ecs{} supports the MIPS32 and MIPS64 hardware architectures}{mips}{MIPS Hardware Architecture Support}}
\providecommand{\seemmix}{}\renewcommand{\seemmix}{\seedocumentationref{For more information about how the \ecs{} supports the MMIX hardware architecture}{mmix}{MMIX Hardware Architecture Support}}
\providecommand{\seeorok}{}\renewcommand{\seeorok}{\seedocumentationref{For more information about how the \ecs{} supports the OpenRISC 1000 hardware architecture}{or1k}{OpenRISC 1000 Hardware Architecture Support}}
\providecommand{\seeppc}{}\renewcommand{\seeppc}{\seedocumentationref{For more information about how the \ecs{} supports the PowerPC hardware architecture}{ppc}{PowerPC Hardware Architecture Support}}
\providecommand{\seerisc}{}\renewcommand{\seerisc}{\seedocumentationref{For more information about how the \ecs{} supports the RISC hardware architecture}{risc}{RISC Hardware Architecture Support}}
\providecommand{\seewasm}{}\renewcommand{\seewasm}{\seedocumentationref{For more information about how the \ecs{} supports the WebAssembly architecture}{wasm}{WebAssembly Architecture Support}}
\providecommand{\seedocumentation}{}\renewcommand{\seedocumentation}{\seedocumentationref{For more information about generic documentations and their generation by the \ecs{}}{documentation}{Generic Documentation Generation}}
\providecommand{\seedebugging}{}\renewcommand{\seedebugging}{\seedocumentationref{For more information about debugging information and its representation}{debugging}{Debugging Information Representation}}
\providecommand{\seecode}{}\renewcommand{\seecode}{\seedocumentationref{For more information about intermediate code and its purpose}{code}{Intermediate Code Representation}}
\providecommand{\seeobject}{}\renewcommand{\seeobject}{\seedocumentationref{For more information about object files and their purpose}{object}{Object File Representation}}

% generic documentation tools

\providecommand{\docprint}{
\toolsection{docprint} is a pretty printer for generic documentations.
It reformats generic documentations and writes it to the standard output stream.
\debuggingtool
\flowgraph{\resource{generic\\documentation} \ar[r] & \toolbox{docprint} \ar[r] & \resource{generic\\documentation}}
\seedocumentation
}

\providecommand{\doccheck}{
\toolsection{doccheck} is a syntactic and semantic checker for generic documentations.
It just performs syntactic and semantic checks on generic documentations and writes its diagnostic messages to the standard error stream.
\debuggingtool
\flowgraph{\resource{generic\\documentation} \ar[r] & \toolbox{doccheck} \ar[r] & \resource{diagnostic\\messages}}
\seedocumentation
}

\providecommand{\dochtml}{
\toolsection{dochtml} is an HTML documentation generator for generic documentations.
It processes several generic documentations and assembles all information therein into an HTML document.
\debuggingtool
\flowgraph{\resource{generic\\documentation} \ar[r] & \toolbox{dochtml} \ar[r] & \resource{HTML\\document}}
\seedocumentation
}

\providecommand{\doclatex}{
\toolsection{doclatex} is a Latex documentation generator for generic documentations.
It processes several generic documentations and assembles all information therein into a Latex document.
\debuggingtool
\flowgraph{\resource{generic\\documentation} \ar[r] & \toolbox{doclatex} \ar[r] & \resource{Latex\\document}}
\seedocumentation
}

% intermediate code tools

\providecommand{\cdcheck}{
\toolsection{cdcheck} is a syntactic and semantic checker for intermediate code.
It just performs syntactic and semantic checks on programs written in intermediate code and writes its diagnostic messages to the standard error stream.
\debuggingtool
\flowgraph{\resource{intermediate\\code} \ar[r] & \toolbox{cdcheck} \ar[r] & \resource{diagnostic\\messages}}
\seeassembly\seecode
}

\providecommand{\cdopt}{
\toolsection{cdopt} is an optimizer for intermediate code.
It performs various optimizations on programs written in intermediate code and writes the result to the standard output stream.
\debuggingtool
\flowgraph{\resource{intermediate\\code} \ar[r] & \toolbox{cdopt} \ar[r] & \resource{optimized\\code}}
\seeassembly\seecode
}

\providecommand{\cdrun}{
\toolsection{cdrun} is an interpreter for intermediate code.
It processes and executes programs written in intermediate code.
The following code sections are predefined and have the usual semantics:
\texttt{abort}, \texttt{\_Exit}, \texttt{fflush}, \texttt{floor}, \texttt{fputc}, \texttt{free}, \texttt{getchar}, \texttt{malloc}, and \texttt{putchar}.
Diagnostic messages about invalid operations include the name of the executed code section and the index of the erroneous instruction.
\debuggingtool
\flowgraph{\resource{intermediate\\code} \ar[r] & \toolbox{cdrun} \ar@/u/[r] & \resource{input/\\output} \ar@/d/[l]}
\seeassembly\seecode
}

\providecommand{\cdamda}{
\toolsection{cdamd16} is a compiler for intermediate code targeting the AMD64 hardware architecture.
It generates machine code for AMD64 processors from programs written in intermediate code and stores it in corresponding object files.
The compiler generates machine code for the 16-bit operating mode defined by the AMD64 architecture.
It also creates a debugging information file as well as an assembly file containing a listing of the generated machine code.
\debuggingtool
\flowgraph{\resource{intermediate\\code} \ar[r] & \toolbox{cdamd16} \ar[r] \ar[d] \ar[rd] & \resource{object file} \\ & \resource{assembly\\listing} & \resource{debugging\\information}}
\seeassembly\seeamd\seeobject\seecode\seedebugging
}

\providecommand{\cdamdb}{
\toolsection{cdamd32} is a compiler for intermediate code targeting the AMD64 hardware architecture.
It generates machine code for AMD64 processors from programs written in intermediate code and stores it in corresponding object files.
The compiler generates machine code for the 32-bit operating mode defined by the AMD64 architecture.
It also creates a debugging information file as well as an assembly file containing a listing of the generated machine code.
\debuggingtool
\flowgraph{\resource{intermediate\\code} \ar[r] & \toolbox{cdamd32} \ar[r] \ar[d] \ar[rd] & \resource{object file} \\ & \resource{assembly\\listing} & \resource{debugging\\information}}
\seeassembly\seeamd\seeobject\seecode\seedebugging
}

\providecommand{\cdamdc}{
\toolsection{cdamd64} is a compiler for intermediate code targeting the AMD64 hardware architecture.
It generates machine code for AMD64 processors from programs written in intermediate code and stores it in corresponding object files.
The compiler generates machine code for the 64-bit operating mode defined by the AMD64 architecture.
It also creates a debugging information file as well as an assembly file containing a listing of the generated machine code.
\debuggingtool
\flowgraph{\resource{intermediate\\code} \ar[r] & \toolbox{cdamd64} \ar[r] \ar[d] \ar[rd] & \resource{object file} \\ & \resource{assembly\\listing} & \resource{debugging\\information}}
\seeassembly\seeamd\seeobject\seecode\seedebugging
}

\providecommand{\cdarma}{
\toolsection{cdarma32} is a compiler for intermediate code targeting the ARM hardware architecture.
It generates machine code for ARM processors executing A32 instructions from programs written in intermediate code and stores it in corresponding object files.
It also creates a debugging information file as well as an assembly file containing a listing of the generated machine code.
\debuggingtool
\flowgraph{\resource{intermediate\\code} \ar[r] & \toolbox{cdarma32} \ar[r] \ar[d] \ar[rd] & \resource{object file} \\ & \resource{assembly\\listing} & \resource{debugging\\information}}
\seeassembly\seearm\seeobject\seecode\seedebugging
}

\providecommand{\cdarmb}{
\toolsection{cdarma64} is a compiler for intermediate code targeting the ARM hardware architecture.
It generates machine code for ARM processors executing A64 instructions from programs written in intermediate code and stores it in corresponding object files.
It also creates a debugging information file as well as an assembly file containing a listing of the generated machine code.
\debuggingtool
\flowgraph{\resource{intermediate\\code} \ar[r] & \toolbox{cdarma64} \ar[r] \ar[d] \ar[rd] & \resource{object file} \\ & \resource{assembly\\listing} & \resource{debugging\\information}}
\seeassembly\seearm\seeobject\seecode\seedebugging
}

\providecommand{\cdarmc}{
\toolsection{cdarmt32} is a compiler for intermediate code targeting the ARM hardware architecture.
It generates machine code for ARM processors without floating-point extension executing T32 instructions from programs written in intermediate code and stores it in corresponding object files.
It also creates a debugging information file as well as an assembly file containing a listing of the generated machine code.
\debuggingtool
\flowgraph{\resource{intermediate\\code} \ar[r] & \toolbox{cdarmt32} \ar[r] \ar[d] \ar[rd] & \resource{object file} \\ & \resource{assembly\\listing} & \resource{debugging\\information}}
\seeassembly\seearm\seeobject\seecode\seedebugging
}

\providecommand{\cdarmcfpe}{
\toolsection{cdarmt32fpe} is a compiler for intermediate code targeting the ARM hardware architecture.
It generates machine code for ARM processors with floating-point extension executing T32 instructions from programs written in intermediate code and stores it in corresponding object files.
It also creates a debugging information file as well as an assembly file containing a listing of the generated machine code.
\debuggingtool
\flowgraph{\resource{intermediate\\code} \ar[r] & \toolbox{cdarmt32fpe} \ar[r] \ar[d] \ar[rd] & \resource{object file} \\ & \resource{assembly\\listing} & \resource{debugging\\information}}
\seeassembly\seearm\seeobject\seecode\seedebugging
}

\providecommand{\cdavr}{
\toolsection{cdavr} is a compiler for intermediate code targeting the AVR hardware architecture.
It generates machine code for AVR processors from programs written in intermediate code and stores it in corresponding object files.
It also creates a debugging information file as well as an assembly file containing a listing of the generated machine code.
\debuggingtool
\flowgraph{\resource{intermediate\\code} \ar[r] & \toolbox{cdavr} \ar[r] \ar[d] \ar[rd] & \resource{object file} \\ & \resource{assembly\\listing} & \resource{debugging\\information}}
\seeassembly\seeavr\seeobject\seecode\seedebugging
}

\providecommand{\cdavrtt}{
\toolsection{cdavr32} is a compiler for intermediate code targeting the AVR32 hardware architecture.
It generates machine code for AVR32 processors from programs written in intermediate code and stores it in corresponding object files.
It also creates a debugging information file as well as an assembly file containing a listing of the generated machine code.
\debuggingtool
\flowgraph{\resource{intermediate\\code} \ar[r] & \toolbox{cdavr32} \ar[r] \ar[d] \ar[rd] & \resource{object file} \\ & \resource{assembly\\listing} & \resource{debugging\\information}}
\seeassembly\seeavrtt\seeobject\seecode\seedebugging
}

\providecommand{\cdmabk}{
\toolsection{cdm68k} is a compiler for intermediate code targeting the M68000 hardware architecture.
It generates machine code for M68000 processors from programs written in intermediate code and stores it in corresponding object files.
It also creates a debugging information file as well as an assembly file containing a listing of the generated machine code.
\debuggingtool
\flowgraph{\resource{intermediate\\code} \ar[r] & \toolbox{cdm68k} \ar[r] \ar[d] \ar[rd] & \resource{object file} \\ & \resource{assembly\\listing} & \resource{debugging\\information}}
\seeassembly\seemabk\seeobject\seecode\seedebugging
}

\providecommand{\cdmibl}{
\toolsection{cdmibl} is a compiler for intermediate code targeting the MicroBlaze hardware architecture.
It generates machine code for MicroBlaze processors from programs written in intermediate code and stores it in corresponding object files.
It also creates a debugging information file as well as an assembly file containing a listing of the generated machine code.
\debuggingtool
\flowgraph{\resource{intermediate\\code} \ar[r] & \toolbox{cdmibl} \ar[r] \ar[d] \ar[rd] & \resource{object file} \\ & \resource{assembly\\listing} & \resource{debugging\\information}}
\seeassembly\seemibl\seeobject\seecode\seedebugging
}

\providecommand{\cdmipsa}{
\toolsection{cdmips32} is a compiler for intermediate code targeting the MIPS32 hardware architecture.
It generates machine code for MIPS32 processors from programs written in intermediate code and stores it in corresponding object files.
It also creates a debugging information file as well as an assembly file containing a listing of the generated machine code.
\debuggingtool
\flowgraph{\resource{intermediate\\code} \ar[r] & \toolbox{cdmips32} \ar[r] \ar[d] \ar[rd] & \resource{object file} \\ & \resource{assembly\\listing} & \resource{debugging\\information}}
\seeassembly\seemips\seeobject\seecode\seedebugging
}

\providecommand{\cdmipsb}{
\toolsection{cdmips64} is a compiler for intermediate code targeting the MIPS64 hardware architecture.
It generates machine code for MIPS64 processors from programs written in intermediate code and stores it in corresponding object files.
It also creates a debugging information file as well as an assembly file containing a listing of the generated machine code.
\debuggingtool
\flowgraph{\resource{intermediate\\code} \ar[r] & \toolbox{cdmips64} \ar[r] \ar[d] \ar[rd] & \resource{object file} \\ & \resource{assembly\\listing} & \resource{debugging\\information}}
\seeassembly\seemips\seeobject\seecode\seedebugging
}

\providecommand{\cdmmix}{
\toolsection{cdmmix} is a compiler for intermediate code targeting the MMIX hardware architecture.
It generates machine code for MMIX processors from programs written in intermediate code and stores it in corresponding object files.
It also creates a debugging information file as well as an assembly file containing a listing of the generated machine code.
\debuggingtool
\flowgraph{\resource{intermediate\\code} \ar[r] & \toolbox{cdmmix} \ar[r] \ar[d] \ar[rd] & \resource{object file} \\ & \resource{assembly\\listing} & \resource{debugging\\information}}
\seeassembly\seemmix\seeobject\seecode\seedebugging
}

\providecommand{\cdorok}{
\toolsection{cdor1k} is a compiler for intermediate code targeting the OpenRISC 1000 hardware architecture.
It generates machine code for OpenRISC 1000 processors from programs written in intermediate code and stores it in corresponding object files.
It also creates a debugging information file as well as an assembly file containing a listing of the generated machine code.
\debuggingtool
\flowgraph{\resource{intermediate\\code} \ar[r] & \toolbox{cdor1k} \ar[r] \ar[d] \ar[rd] & \resource{object file} \\ & \resource{assembly\\listing} & \resource{debugging\\information}}
\seeassembly\seeorok\seeobject\seecode\seedebugging
}

\providecommand{\cdppca}{
\toolsection{cdppc32} is a compiler for intermediate code targeting the PowerPC hardware architecture.
It generates machine code for PowerPC processors from programs written in intermediate code and stores it in corresponding object files.
The compiler generates machine code for the 32-bit operating mode defined by the PowerPC architecture.
It also creates a debugging information file as well as an assembly file containing a listing of the generated machine code.
\debuggingtool
\flowgraph{\resource{intermediate\\code} \ar[r] & \toolbox{cdppc32} \ar[r] \ar[d] \ar[rd] & \resource{object file} \\ & \resource{assembly\\listing} & \resource{debugging\\information}}
\seeassembly\seeppc\seeobject\seecode\seedebugging
}

\providecommand{\cdppcb}{
\toolsection{cdppc64} is a compiler for intermediate code targeting the PowerPC hardware architecture.
It generates machine code for PowerPC processors from programs written in intermediate code and stores it in corresponding object files.
The compiler generates machine code for the 64-bit operating mode defined by the PowerPC architecture.
It also creates a debugging information file as well as an assembly file containing a listing of the generated machine code.
\debuggingtool
\flowgraph{\resource{intermediate\\code} \ar[r] & \toolbox{cdppc64} \ar[r] \ar[d] \ar[rd] & \resource{object file} \\ & \resource{assembly\\listing} & \resource{debugging\\information}}
\seeassembly\seeppc\seeobject\seecode\seedebugging
}

\providecommand{\cdrisc}{
\toolsection{cdrisc} is a compiler for intermediate code targeting the RISC hardware architecture.
It generates machine code for RISC processors from programs written in intermediate code and stores it in corresponding object files.
It also creates a debugging information file as well as an assembly file containing a listing of the generated machine code.
\debuggingtool
\flowgraph{\resource{intermediate\\code} \ar[r] & \toolbox{cdrisc} \ar[r] \ar[d] \ar[rd] & \resource{object file} \\ & \resource{assembly\\listing} & \resource{debugging\\information}}
\seeassembly\seerisc\seeobject\seecode\seedebugging
}

\providecommand{\cdwasm}{
\toolsection{cdwasm} is a compiler for intermediate code targeting the WebAssembly architecture.
It generates machine code for WebAssembly targets from programs written in intermediate code and stores it in corresponding object files.
It also creates a debugging information file as well as an assembly file containing a listing of the generated machine code.
\debuggingtool
\flowgraph{\resource{intermediate\\code} \ar[r] & \toolbox{cdwasm} \ar[r] \ar[d] \ar[rd] & \resource{object file} \\ & \resource{assembly\\listing} & \resource{debugging\\information}}
\seeassembly\seewasm\seeobject\seecode\seedebugging
}

% C++ tools

\providecommand{\cppprep}{
\toolsection{cppprep} is a preprocessor for the \cpp{} programming language.
It preprocesses source code according to the rules of \cpp{} and writes it to the standard output stream.
Only the macro names \texttt{\_\_DATE\_\_}, \texttt{\_\_FILE\_\_}, \texttt{\_\_LINE\_\_}, and \texttt{\_\_TIME\_\_} are predefined.
\flowgraph{\resource{\cpp{} or other\\source code} \ar[r] & \toolbox{cppprep} \ar[r] & \resource{preprocessed\\source code} \\ & \variable{ECSINCLUDE} \ar[u]}
\seecpp
}

\providecommand{\cppprint}{
\toolsection{cppprint} is a pretty printer for the \cpp{} programming language.
It reformats the source code of \cpp{} programs and writes it to the standard output stream.
\flowgraph{\resource{\cpp{}\\source code} \ar[r] & \toolbox{cppprint} \ar[r] & \resource{reformatted\\source code} \\ & \variable{ECSINCLUDE} \ar[u]}
\seecpp
}

\providecommand{\cppcheck}{
\toolsection{cppcheck} is a syntactic and semantic checker for the \cpp{} programming language.
It just performs syntactic and semantic checks on \cpp{} programs and writes its diagnostic messages to the standard error stream.
\flowgraph{\resource{\cpp{}\\source code} \ar[r] & \toolbox{cppcheck} \ar[r] & \resource{diagnostic\\messages} \\ & \variable{ECSINCLUDE} \ar[u]}
\seecpp
}

\providecommand{\cppdump}{
\toolsection{cppdump} is a serializer for the \cpp{} programming language.
It dumps the complete internal representation of programs written in \cpp{} into an XML document.
\debuggingtool
\flowgraph{\resource{\cpp{}\\source code} \ar[r] & \toolbox{cppdump} \ar[r] & \resource{internal\\representation} \\ & \variable{ECSINCLUDE} \ar[u]}
\seecpp
}

\providecommand{\cpprun}{
\toolsection{cpprun} is an interpreter for the \cpp{} programming language.
It processes and executes programs written in \cpp{}.
The macro \texttt{\_\_run\_\_} is predefined in order to enable programmers to identify this tool while interpreting.
\flowgraph{\resource{\cpp{}\\source code} \ar[r] & \toolbox{cpprun} \ar@/u/[r] & \resource{input/\\output} \ar@/d/[l] \\ & \variable{ECSINCLUDE} \ar[u]}
\seecpp
}

\providecommand{\cppdoc}{
\toolsection{cppdoc} is a generic documentation generator for the \cpp{} programming language.
It processes several \cpp{} source files and assembles all information therein into a generic documentation.
\debuggingtool
\flowgraph{\resource{\cpp{}\\source code} \ar[r] & \toolbox{cppdoc} \ar[r] & \resource{generic\\documentation} \\ & \variable{ECSINCLUDE} \ar[u]}
\seecpp\seedocumentation
}

\providecommand{\cpphtml}{
\toolsection{cpphtml} is an HTML documentation generator for the \cpp{} programming language.
It processes several \cpp{} source files and assembles all information therein into an HTML document.
\flowgraph{\resource{\cpp{}\\source code} \ar[r] & \toolbox{cpphtml} \ar[r] & \resource{HTML\\document} \\ & \variable{ECSINCLUDE} \ar[u]}
\seecpp\seedocumentation
}

\providecommand{\cpplatex}{
\toolsection{cpplatex} is a Latex documentation generator for the \cpp{} programming language.
It processes several \cpp{} source files and assembles all information therein into a Latex document.
\flowgraph{\resource{\cpp{}\\source code} \ar[r] & \toolbox{cpplatex} \ar[r] & \resource{Latex\\document} \\ & \variable{ECSINCLUDE} \ar[u]}
\seecpp\seedocumentation
}

\providecommand{\cppcode}{
\toolsection{cppcode} is an intermediate code generator for the \cpp{} programming language.
It generates intermediate code from programs written in \cpp{} and stores it in corresponding assembly files.
The macro \texttt{\_\_code\_\_} is predefined in order to enable programmers to identify this tool while generating intermediate code.
Programs generated with this tool require additional runtime support that is stored in the \file{cpp\-code\-run} library file.
\debuggingtool
\flowgraph{\resource{\cpp{}\\source code} \ar[r] & \toolbox{cppcode} \ar[r] & \resource{intermediate\\code} \\ & \variable{ECSINCLUDE} \ar[u]}
\seecpp\seeassembly\seecode
}

\providecommand{\cppamda}{
\toolsection{cppamd16} is a compiler for the \cpp{} programming language targeting the AMD64 hardware architecture.
It generates machine code for AMD64 processors from programs written in \cpp{} and stores it in corresponding object files.
The compiler generates machine code for the 16-bit operating mode defined by the AMD64 architecture.
For debugging purposes, it also creates a debugging information file as well as an assembly file containing a listing of the generated machine code.
The macro \texttt{\_\_amd16\_\_} is predefined in order to enable programmers to identify this tool and its target architecture while compiling.
Programs generated with this compiler require additional runtime support that is stored in the \file{cpp\-amd16\-run} library file.
\flowgraph{\resource{\cpp{}\\source code} \ar[r] & \toolbox{cppamd16} \ar[r] \ar[d] \ar[rd] & \resource{object file} \\ \variable{ECSINCLUDE} \ar[ru] & \resource{debugging\\information} & \resource{assembly\\listing}}
\seecpp\seeassembly\seeamd\seeobject\seedebugging
}

\providecommand{\cppamdb}{
\toolsection{cppamd32} is a compiler for the \cpp{} programming language targeting the AMD64 hardware architecture.
It generates machine code for AMD64 processors from programs written in \cpp{} and stores it in corresponding object files.
The compiler generates machine code for the 32-bit operating mode defined by the AMD64 architecture.
For debugging purposes, it also creates a debugging information file as well as an assembly file containing a listing of the generated machine code.
The macro \texttt{\_\_amd32\_\_} is predefined in order to enable programmers to identify this tool and its target architecture while compiling.
Programs generated with this compiler require additional runtime support that is stored in the \file{cpp\-amd32\-run} library file.
\flowgraph{\resource{\cpp{}\\source code} \ar[r] & \toolbox{cppamd32} \ar[r] \ar[d] \ar[rd] & \resource{object file} \\ \variable{ECSINCLUDE} \ar[ru] & \resource{debugging\\information} & \resource{assembly\\listing}}
\seecpp\seeassembly\seeamd\seeobject\seedebugging
}

\providecommand{\cppamdc}{
\toolsection{cppamd64} is a compiler for the \cpp{} programming language targeting the AMD64 hardware architecture.
It generates machine code for AMD64 processors from programs written in \cpp{} and stores it in corresponding object files.
The compiler generates machine code for the 64-bit operating mode defined by the AMD64 architecture.
For debugging purposes, it also creates a debugging information file as well as an assembly file containing a listing of the generated machine code.
The macro \texttt{\_\_amd64\_\_} is predefined in order to enable programmers to identify this tool and its target architecture while compiling.
Programs generated with this compiler require additional runtime support that is stored in the \file{cpp\-amd64\-run} library file.
\flowgraph{\resource{\cpp{}\\source code} \ar[r] & \toolbox{cppamd64} \ar[r] \ar[d] \ar[rd] & \resource{object file} \\ \variable{ECSINCLUDE} \ar[ru] & \resource{debugging\\information} & \resource{assembly\\listing}}
\seecpp\seeassembly\seeamd\seeobject\seedebugging
}

\providecommand{\cpparma}{
\toolsection{cpparma32} is a compiler for the \cpp{} programming language targeting the ARM hardware architecture.
It generates machine code for ARM processors executing A32 instructions from programs written in \cpp{} and stores it in corresponding object files.
For debugging purposes, it also creates a debugging information file as well as an assembly file containing a listing of the generated machine code.
The macro \texttt{\_\_arma32\_\_} is predefined in order to enable programmers to identify this tool and its target architecture while compiling.
Programs generated with this compiler require additional runtime support that is stored in the \file{cpp\-arma32\-run} library file.
\flowgraph{\resource{\cpp{}\\source code} \ar[r] & \toolbox{cpparma32} \ar[r] \ar[d] \ar[rd] & \resource{object file} \\ \variable{ECSINCLUDE} \ar[ru] & \resource{debugging\\information} & \resource{assembly\\listing}}
\seecpp\seeassembly\seearm\seeobject\seedebugging
}

\providecommand{\cpparmb}{
\toolsection{cpparma64} is a compiler for the \cpp{} programming language targeting the ARM hardware architecture.
It generates machine code for ARM processors executing A64 instructions from programs written in \cpp{} and stores it in corresponding object files.
For debugging purposes, it also creates a debugging information file as well as an assembly file containing a listing of the generated machine code.
The macro \texttt{\_\_arma64\_\_} is predefined in order to enable programmers to identify this tool and its target architecture while compiling.
Programs generated with this compiler require additional runtime support that is stored in the \file{cpp\-arma64\-run} library file.
\flowgraph{\resource{\cpp{}\\source code} \ar[r] & \toolbox{cpparma64} \ar[r] \ar[d] \ar[rd] & \resource{object file} \\ \variable{ECSINCLUDE} \ar[ru] & \resource{debugging\\information} & \resource{assembly\\listing}}
\seecpp\seeassembly\seearm\seeobject\seedebugging
}

\providecommand{\cpparmc}{
\toolsection{cpparmt32} is a compiler for the \cpp{} programming language targeting the ARM hardware architecture.
It generates machine code for ARM processors without floating-point extension executing T32 instructions from programs written in \cpp{} and stores it in corresponding object files.
For debugging purposes, it also creates a debugging information file as well as an assembly file containing a listing of the generated machine code.
The macro \texttt{\_\_armt32\_\_} is predefined in order to enable programmers to identify this tool and its target architecture while compiling.
Programs generated with this compiler require additional runtime support that is stored in the \file{cpp\-armt32\-run} library file.
\flowgraph{\resource{\cpp{}\\source code} \ar[r] & \toolbox{cpparmt32} \ar[r] \ar[d] \ar[rd] & \resource{object file} \\ \variable{ECSINCLUDE} \ar[ru] & \resource{debugging\\information} & \resource{assembly\\listing}}
\seecpp\seeassembly\seearm\seeobject\seedebugging
}

\providecommand{\cpparmcfpe}{
\toolsection{cpparmt32fpe} is a compiler for the \cpp{} programming language targeting the ARM hardware architecture.
It generates machine code for ARM processors with floating-point extension executing T32 instructions from programs written in \cpp{} and stores it in corresponding object files.
For debugging purposes, it also creates a debugging information file as well as an assembly file containing a listing of the generated machine code.
The macro \texttt{\_\_armt32fpe\_\_} is predefined in order to enable programmers to identify this tool and its target architecture while compiling.
Programs generated with this compiler require additional runtime support that is stored in the \file{cpp\-armt32\-fpe\-run} library file.
\flowgraph{\resource{\cpp{}\\source code} \ar[r] & \toolbox{cpparmt32fpe} \ar[r] \ar[d] \ar[rd] & \resource{object file} \\ \variable{ECSINCLUDE} \ar[ru] & \resource{debugging\\information} & \resource{assembly\\listing}}
\seecpp\seeassembly\seearm\seeobject\seedebugging
}

\providecommand{\cppavr}{
\toolsection{cppavr} is a compiler for the \cpp{} programming language targeting the AVR hardware architecture.
It generates machine code for AVR processors from programs written in \cpp{} and stores it in corresponding object files.
For debugging purposes, it also creates a debugging information file as well as an assembly file containing a listing of the generated machine code.
The macro \texttt{\_\_avr\_\_} is predefined in order to enable programmers to identify this tool and its target architecture while compiling.
Programs generated with this compiler require additional runtime support that is stored in the \file{cpp\-avr\-run} library file.
\flowgraph{\resource{\cpp{}\\source code} \ar[r] & \toolbox{cppavr} \ar[r] \ar[d] \ar[rd] & \resource{object file} \\ \variable{ECSINCLUDE} \ar[ru] & \resource{debugging\\information} & \resource{assembly\\listing}}
\seecpp\seeassembly\seeavr\seeobject\seedebugging
}

\providecommand{\cppavrtt}{
\toolsection{cppavr32} is a compiler for the \cpp{} programming language targeting the AVR32 hardware architecture.
It generates machine code for AVR32 processors from programs written in \cpp{} and stores it in corresponding object files.
For debugging purposes, it also creates a debugging information file as well as an assembly file containing a listing of the generated machine code.
The macro \texttt{\_\_avr32\_\_} is predefined in order to enable programmers to identify this tool and its target architecture while compiling.
Programs generated with this compiler require additional runtime support that is stored in the \file{cpp\-avr32\-run} library file.
\flowgraph{\resource{\cpp{}\\source code} \ar[r] & \toolbox{cppavr32} \ar[r] \ar[d] \ar[rd] & \resource{object file} \\ \variable{ECSINCLUDE} \ar[ru] & \resource{debugging\\information} & \resource{assembly\\listing}}
\seecpp\seeassembly\seeavrtt\seeobject\seedebugging
}

\providecommand{\cppmabk}{
\toolsection{cppm68k} is a compiler for the \cpp{} programming language targeting the M68000 hardware architecture.
It generates machine code for M68000 processors from programs written in \cpp{} and stores it in corresponding object files.
For debugging purposes, it also creates a debugging information file as well as an assembly file containing a listing of the generated machine code.
The macro \texttt{\_\_m68k\_\_} is predefined in order to enable programmers to identify this tool and its target architecture while compiling.
Programs generated with this compiler require additional runtime support that is stored in the \file{cpp\-m68k\-run} library file.
\flowgraph{\resource{\cpp{}\\source code} \ar[r] & \toolbox{cppm68k} \ar[r] \ar[d] \ar[rd] & \resource{object file} \\ \variable{ECSINCLUDE} \ar[ru] & \resource{debugging\\information} & \resource{assembly\\listing}}
\seecpp\seeassembly\seemabk\seeobject\seedebugging
}

\providecommand{\cppmibl}{
\toolsection{cppmibl} is a compiler for the \cpp{} programming language targeting the MicroBlaze hardware architecture.
It generates machine code for MicroBlaze processors from programs written in \cpp{} and stores it in corresponding object files.
For debugging purposes, it also creates a debugging information file as well as an assembly file containing a listing of the generated machine code.
The macro \texttt{\_\_mibl\_\_} is predefined in order to enable programmers to identify this tool and its target architecture while compiling.
Programs generated with this compiler require additional runtime support that is stored in the \file{cpp\-mibl\-run} library file.
\flowgraph{\resource{\cpp{}\\source code} \ar[r] & \toolbox{cppmibl} \ar[r] \ar[d] \ar[rd] & \resource{object file} \\ \variable{ECSINCLUDE} \ar[ru] & \resource{debugging\\information} & \resource{assembly\\listing}}
\seecpp\seeassembly\seemibl\seeobject\seedebugging
}

\providecommand{\cppmipsa}{
\toolsection{cppmips32} is a compiler for the \cpp{} programming language targeting the MIPS32 hardware architecture.
It generates machine code for MIPS32 processors from programs written in \cpp{} and stores it in corresponding object files.
For debugging purposes, it also creates a debugging information file as well as an assembly file containing a listing of the generated machine code.
The macro \texttt{\_\_mips32\_\_} is predefined in order to enable programmers to identify this tool and its target architecture while compiling.
Programs generated with this compiler require additional runtime support that is stored in the \file{cpp\-mips32\-run} library file.
\flowgraph{\resource{\cpp{}\\source code} \ar[r] & \toolbox{cppmips32} \ar[r] \ar[d] \ar[rd] & \resource{object file} \\ \variable{ECSINCLUDE} \ar[ru] & \resource{debugging\\information} & \resource{assembly\\listing}}
\seecpp\seeassembly\seemips\seeobject\seedebugging
}

\providecommand{\cppmipsb}{
\toolsection{cppmips64} is a compiler for the \cpp{} programming language targeting the MIPS64 hardware architecture.
It generates machine code for MIPS64 processors from programs written in \cpp{} and stores it in corresponding object files.
For debugging purposes, it also creates a debugging information file as well as an assembly file containing a listing of the generated machine code.
The macro \texttt{\_\_mips64\_\_} is predefined in order to enable programmers to identify this tool and its target architecture while compiling.
Programs generated with this compiler require additional runtime support that is stored in the \file{cpp\-mips64\-run} library file.
\flowgraph{\resource{\cpp{}\\source code} \ar[r] & \toolbox{cppmips64} \ar[r] \ar[d] \ar[rd] & \resource{object file} \\ \variable{ECSINCLUDE} \ar[ru] & \resource{debugging\\information} & \resource{assembly\\listing}}
\seecpp\seeassembly\seemips\seeobject\seedebugging
}

\providecommand{\cppmmix}{
\toolsection{cppmmix} is a compiler for the \cpp{} programming language targeting the MMIX hardware architecture.
It generates machine code for MMIX processors from programs written in \cpp{} and stores it in corresponding object files.
For debugging purposes, it also creates a debugging information file as well as an assembly file containing a listing of the generated machine code.
The macro \texttt{\_\_mmix\_\_} is predefined in order to enable programmers to identify this tool and its target architecture while compiling.
Programs generated with this compiler require additional runtime support that is stored in the \file{cpp\-mmix\-run} library file.
\flowgraph{\resource{\cpp{}\\source code} \ar[r] & \toolbox{cppmmix} \ar[r] \ar[d] \ar[rd] & \resource{object file} \\ \variable{ECSINCLUDE} \ar[ru] & \resource{debugging\\information} & \resource{assembly\\listing}}
\seecpp\seeassembly\seemmix\seeobject\seedebugging
}

\providecommand{\cpporok}{
\toolsection{cppor1k} is a compiler for the \cpp{} programming language targeting the OpenRISC 1000 hardware architecture.
It generates machine code for OpenRISC 1000 processors from programs written in \cpp{} and stores it in corresponding object files.
For debugging purposes, it also creates a debugging information file as well as an assembly file containing a listing of the generated machine code.
The macro \texttt{\_\_or1k\_\_} is predefined in order to enable programmers to identify this tool and its target architecture while compiling.
Programs generated with this compiler require additional runtime support that is stored in the \file{cpp\-or1k\-run} library file.
\flowgraph{\resource{\cpp{}\\source code} \ar[r] & \toolbox{cppor1k} \ar[r] \ar[d] \ar[rd] & \resource{object file} \\ \variable{ECSINCLUDE} \ar[ru] & \resource{debugging\\information} & \resource{assembly\\listing}}
\seecpp\seeassembly\seeorok\seeobject\seedebugging
}

\providecommand{\cppppca}{
\toolsection{cppppc32} is a compiler for the \cpp{} programming language targeting the PowerPC hardware architecture.
It generates machine code for PowerPC processors from programs written in \cpp{} and stores it in corresponding object files.
The compiler generates machine code for the 32-bit operating mode defined by the PowerPC architecture.
For debugging purposes, it also creates a debugging information file as well as an assembly file containing a listing of the generated machine code.
The macro \texttt{\_\_ppc32\_\_} is predefined in order to enable programmers to identify this tool and its target architecture while compiling.
Programs generated with this compiler require additional runtime support that is stored in the \file{cpp\-ppc32\-run} library file.
\flowgraph{\resource{\cpp{}\\source code} \ar[r] & \toolbox{cppppc32} \ar[r] \ar[d] \ar[rd] & \resource{object file} \\ \variable{ECSINCLUDE} \ar[ru] & \resource{debugging\\information} & \resource{assembly\\listing}}
\seecpp\seeassembly\seeppc\seeobject\seedebugging
}

\providecommand{\cppppcb}{
\toolsection{cppppc64} is a compiler for the \cpp{} programming language targeting the PowerPC hardware architecture.
It generates machine code for PowerPC processors from programs written in \cpp{} and stores it in corresponding object files.
The compiler generates machine code for the 64-bit operating mode defined by the PowerPC architecture.
For debugging purposes, it also creates a debugging information file as well as an assembly file containing a listing of the generated machine code.
The macro \texttt{\_\_ppc64\_\_} is predefined in order to enable programmers to identify this tool and its target architecture while compiling.
Programs generated with this compiler require additional runtime support that is stored in the \file{cpp\-ppc64\-run} library file.
\flowgraph{\resource{\cpp{}\\source code} \ar[r] & \toolbox{cppppc64} \ar[r] \ar[d] \ar[rd] & \resource{object file} \\ \variable{ECSINCLUDE} \ar[ru] & \resource{debugging\\information} & \resource{assembly\\listing}}
\seecpp\seeassembly\seeppc\seeobject\seedebugging
}

\providecommand{\cpprisc}{
\toolsection{cpprisc} is a compiler for the \cpp{} programming language targeting the RISC hardware architecture.
It generates machine code for RISC processors from programs written in \cpp{} and stores it in corresponding object files.
For debugging purposes, it also creates a debugging information file as well as an assembly file containing a listing of the generated machine code.
The macro \texttt{\_\_risc\_\_} is predefined in order to enable programmers to identify this tool and its target architecture while compiling.
Programs generated with this compiler require additional runtime support that is stored in the \file{cpp\-risc\-run} library file.
\flowgraph{\resource{\cpp{}\\source code} \ar[r] & \toolbox{cpprisc} \ar[r] \ar[d] \ar[rd] & \resource{object file} \\ \variable{ECSINCLUDE} \ar[ru] & \resource{debugging\\information} & \resource{assembly\\listing}}
\seecpp\seeassembly\seerisc\seeobject\seedebugging
}

\providecommand{\cppwasm}{
\toolsection{cppwasm} is a compiler for the \cpp{} programming language targeting the WebAssembly architecture.
It generates machine code for WebAssembly targets from programs written in \cpp{} and stores it in corresponding object files.
For debugging purposes, it also creates a debugging information file as well as an assembly file containing a listing of the generated machine code.
The macro \texttt{\_\_wasm\_\_} is predefined in order to enable programmers to identify this tool and its target architecture while compiling.
Programs generated with this compiler require additional runtime support that is stored in the \file{cpp\-wasm\-run} library file.
\flowgraph{\resource{\cpp{}\\source code} \ar[r] & \toolbox{cppwasm} \ar[r] \ar[d] \ar[rd] & \resource{object file} \\ \variable{ECSINCLUDE} \ar[ru] & \resource{debugging\\information} & \resource{assembly\\listing}}
\seecpp\seeassembly\seewasm\seeobject\seedebugging
}

% FALSE tools

\providecommand{\falprint}{
\toolsection{falprint} is a pretty printer for the FALSE programming language.
It reformats the source code of FALSE programs and writes it to the standard output stream.
\flowgraph{\resource{FALSE\\source code} \ar[r] & \toolbox{falprint} \ar[r] & \resource{reformatted\\source code}}
\seefalse
}

\providecommand{\falcheck}{
\toolsection{falcheck} is a syntactic and semantic checker for the FALSE programming language.
It just performs syntactic and semantic checks on FALSE programs and writes its diagnostic messages to the standard error stream.
\flowgraph{\resource{FALSE\\source code} \ar[r] & \toolbox{falcheck} \ar[r] & \resource{diagnostic\\messages}}
\seefalse
}

\providecommand{\faldump}{
\toolsection{faldump} is a serializer for the FALSE programming language.
It dumps the complete internal representation of programs written in FALSE into an XML document.
\debuggingtool
\flowgraph{\resource{FALSE\\source code} \ar[r] & \toolbox{faldump} \ar[r] & \resource{internal\\representation}}
\seefalse
}

\providecommand{\falrun}{
\toolsection{falrun} is an interpreter for the FALSE programming language.
It processes and executes programs written in FALSE\@.
\flowgraph{\resource{FALSE\\source code} \ar[r] & \toolbox{falrun} \ar@/u/[r] & \resource{input/\\output} \ar@/d/[l]}
\seefalse
}

\providecommand{\falcpp}{
\toolsection{falcpp} is a transpiler for the FALSE programming language.
It translates programs written in FALSE into \cpp{} programs and stores them in corresponding source files.
\flowgraph{\resource{FALSE\\source code} \ar[r] & \toolbox{falcpp} \ar[r] & \resource{\cpp{}\\source file}}
\seefalse\seecpp
}

\providecommand{\falcode}{
\toolsection{falcode} is an intermediate code generator for the FALSE programming language.
It generates intermediate code from programs written in FALSE and stores it in corresponding assembly files.
\debuggingtool
\flowgraph{\resource{FALSE\\source code} \ar[r] & \toolbox{falcode} \ar[r] & \resource{intermediate\\code}}
\seefalse\seeassembly\seecode
}

\providecommand{\falamda}{
\toolsection{falamd16} is a compiler for the FALSE programming language targeting the AMD64 hardware architecture.
It generates machine code for AMD64 processors from programs written in FALSE and stores it in corresponding object files.
The compiler generates machine code for the 16-bit operating mode defined by the AMD64 architecture.
\flowgraph{\resource{FALSE\\source code} \ar[r] & \toolbox{falamd16} \ar[r] & \resource{object file}}
\seefalse\seeamd\seeobject
}

\providecommand{\falamdb}{
\toolsection{falamd32} is a compiler for the FALSE programming language targeting the AMD64 hardware architecture.
It generates machine code for AMD64 processors from programs written in FALSE and stores it in corresponding object files.
The compiler generates machine code for the 32-bit operating mode defined by the AMD64 architecture.
\flowgraph{\resource{FALSE\\source code} \ar[r] & \toolbox{falamd32} \ar[r] & \resource{object file}}
\seefalse\seeamd\seeobject
}

\providecommand{\falamdc}{
\toolsection{falamd64} is a compiler for the FALSE programming language targeting the AMD64 hardware architecture.
It generates machine code for AMD64 processors from programs written in FALSE and stores it in corresponding object files.
The compiler generates machine code for the 64-bit operating mode defined by the AMD64 architecture.
\flowgraph{\resource{FALSE\\source code} \ar[r] & \toolbox{falamd64} \ar[r] & \resource{object file}}
\seefalse\seeamd\seeobject
}

\providecommand{\falarma}{
\toolsection{falarma32} is a compiler for the FALSE programming language targeting the ARM hardware architecture.
It generates machine code for ARM processors executing A32 instructions from programs written in FALSE and stores it in corresponding object files.
\flowgraph{\resource{FALSE\\source code} \ar[r] & \toolbox{falarma32} \ar[r] & \resource{object file}}
\seefalse\seearm\seeobject
}

\providecommand{\falarmb}{
\toolsection{falarma64} is a compiler for the FALSE programming language targeting the ARM hardware architecture.
It generates machine code for ARM processors executing A64 instructions from programs written in FALSE and stores it in corresponding object files.
\flowgraph{\resource{FALSE\\source code} \ar[r] & \toolbox{falarma64} \ar[r] & \resource{object file}}
\seefalse\seearm\seeobject
}

\providecommand{\falarmc}{
\toolsection{falarmt32} is a compiler for the FALSE programming language targeting the ARM hardware architecture.
It generates machine code for ARM processors without floating-point extension executing T32 instructions from programs written in FALSE and stores it in corresponding object files.
\flowgraph{\resource{FALSE\\source code} \ar[r] & \toolbox{falarmt32} \ar[r] & \resource{object file}}
\seefalse\seearm\seeobject
}

\providecommand{\falarmcfpe}{
\toolsection{falarmt32fpe} is a compiler for the FALSE programming language targeting the ARM hardware architecture.
It generates machine code for ARM processors with floating-point extension executing T32 instructions from programs written in FALSE and stores it in corresponding object files.
\flowgraph{\resource{FALSE\\source code} \ar[r] & \toolbox{falarmt32fpe} \ar[r] & \resource{object file}}
\seefalse\seearm\seeobject
}

\providecommand{\falavr}{
\toolsection{falavr} is a compiler for the FALSE programming language targeting the AVR hardware architecture.
It generates machine code for AVR processors from programs written in FALSE and stores it in corresponding object files.
\flowgraph{\resource{FALSE\\source code} \ar[r] & \toolbox{falavr} \ar[r] & \resource{object file}}
\seefalse\seeavr\seeobject
}

\providecommand{\falavrtt}{
\toolsection{falavr32} is a compiler for the FALSE programming language targeting the AVR32 hardware architecture.
It generates machine code for AVR32 processors from programs written in FALSE and stores it in corresponding object files.
\flowgraph{\resource{FALSE\\source code} \ar[r] & \toolbox{falavr32} \ar[r] & \resource{object file}}
\seefalse\seeavrtt\seeobject
}

\providecommand{\falmabk}{
\toolsection{falm68k} is a compiler for the FALSE programming language targeting the M68000 hardware architecture.
It generates machine code for M68000 processors from programs written in FALSE and stores it in corresponding object files.
\flowgraph{\resource{FALSE\\source code} \ar[r] & \toolbox{falm68k} \ar[r] & \resource{object file}}
\seefalse\seemabk\seeobject
}

\providecommand{\falmibl}{
\toolsection{falmibl} is a compiler for the FALSE programming language targeting the MicroBlaze hardware architecture.
It generates machine code for MicroBlaze processors from programs written in FALSE and stores it in corresponding object files.
\flowgraph{\resource{FALSE\\source code} \ar[r] & \toolbox{falmibl} \ar[r] & \resource{object file}}
\seefalse\seemibl\seeobject
}

\providecommand{\falmipsa}{
\toolsection{falmips32} is a compiler for the FALSE programming language targeting the MIPS32 hardware architecture.
It generates machine code for MIPS32 processors from programs written in FALSE and stores it in corresponding object files.
\flowgraph{\resource{FALSE\\source code} \ar[r] & \toolbox{falmips32} \ar[r] & \resource{object file}}
\seefalse\seemips\seeobject
}

\providecommand{\falmipsb}{
\toolsection{falmips64} is a compiler for the FALSE programming language targeting the MIPS64 hardware architecture.
It generates machine code for MIPS64 processors from programs written in FALSE and stores it in corresponding object files.
\flowgraph{\resource{FALSE\\source code} \ar[r] & \toolbox{falmips64} \ar[r] & \resource{object file}}
\seefalse\seemips\seeobject
}

\providecommand{\falmmix}{
\toolsection{falmmix} is a compiler for the FALSE programming language targeting the MMIX hardware architecture.
It generates machine code for MMIX processors from programs written in FALSE and stores it in corresponding object files.
\flowgraph{\resource{FALSE\\source code} \ar[r] & \toolbox{falmmix} \ar[r] & \resource{object file}}
\seefalse\seemmix\seeobject
}

\providecommand{\falorok}{
\toolsection{falor1k} is a compiler for the FALSE programming language targeting the OpenRISC 1000 hardware architecture.
It generates machine code for OpenRISC 1000 processors from programs written in FALSE and stores it in corresponding object files.
\flowgraph{\resource{FALSE\\source code} \ar[r] & \toolbox{falor1k} \ar[r] & \resource{object file}}
\seefalse\seeorok\seeobject
}

\providecommand{\falppca}{
\toolsection{falppc32} is a compiler for the FALSE programming language targeting the PowerPC hardware architecture.
It generates machine code for PowerPC processors from programs written in FALSE and stores it in corresponding object files.
The compiler generates machine code for the 32-bit operating mode defined by the PowerPC architecture.
\flowgraph{\resource{FALSE\\source code} \ar[r] & \toolbox{falppc32} \ar[r] & \resource{object file}}
\seefalse\seeppc\seeobject
}

\providecommand{\falppcb}{
\toolsection{falppc64} is a compiler for the FALSE programming language targeting the PowerPC hardware architecture.
It generates machine code for PowerPC processors from programs written in FALSE and stores it in corresponding object files.
The compiler generates machine code for the 64-bit operating mode defined by the PowerPC architecture.
\flowgraph{\resource{FALSE\\source code} \ar[r] & \toolbox{falppc64} \ar[r] & \resource{object file}}
\seefalse\seeppc\seeobject
}

\providecommand{\falrisc}{
\toolsection{falrisc} is a compiler for the FALSE programming language targeting the RISC hardware architecture.
It generates machine code for RISC processors from programs written in FALSE and stores it in corresponding object files.
\flowgraph{\resource{FALSE\\source code} \ar[r] & \toolbox{falrisc} \ar[r] & \resource{object file}}
\seefalse\seerisc\seeobject
}

\providecommand{\falwasm}{
\toolsection{falwasm} is a compiler for the FALSE programming language targeting the WebAssembly architecture.
It generates machine code for WebAssembly targets from programs written in FALSE and stores it in corresponding object files.
\flowgraph{\resource{FALSE\\source code} \ar[r] & \toolbox{falwasm} \ar[r] & \resource{object file}}
\seefalse\seewasm\seeobject
}

% Oberon tools

\providecommand{\obprint}{
\toolsection{obprint} is a pretty printer for the Oberon programming language.
It reformats the source code of Oberon modules and writes it to the standard output stream.
\flowgraph{\resource{Oberon\\source code} \ar[r] & \toolbox{obprint} \ar[r] & \resource{reformatted\\source code}}
\seeoberon
}

\providecommand{\obcheck}{
\toolsection{obcheck} is a syntactic and semantic checker for the Oberon programming language.
It just performs syntactic and semantic checks on Oberon modules and writes its diagnostic messages to the standard error stream.
In addition, it stores the interface of each module in a symbol file which is required when other modules import the module.
\flowgraph{\resource{Oberon\\source code} \ar[r] & \toolbox{obcheck} \ar[r] \ar@/l/[d] & \resource{diagnostic\\messages} \\ \variable{ECSIMPORT} \ar[ru] & \resource{symbol\\files} \ar@/r/[u]}
\seeoberon
}

\providecommand{\obdump}{
\toolsection{obdump} is a serializer for the Oberon programming language.
It dumps the complete internal representation of modules written in Oberon into an XML document.
\debuggingtool
\flowgraph{\resource{Oberon\\source code} \ar[r] & \toolbox{obdump} \ar[r] \ar@/l/[d] & \resource{internal\\representation} \\ \variable{ECSIMPORT} \ar[ru] & \resource{symbol\\files} \ar@/r/[u]}
\seeoberon
}

\providecommand{\obrun}{
\toolsection{obrun} is an interpreter for the Oberon programming language.
It processes and executes modules written in Oberon.
This tool does neither generate nor process symbol files while interpreting modules.
If a module is imported by another one, its filename has to be named before the other one in the list of command-line arguments.
\flowgraph{\resource{Oberon\\source code} \ar[r] & \toolbox{obrun} \ar@/u/[r] & \resource{input/\\output} \ar@/d/[l]}
\seeoberon
}

\providecommand{\obcpp}{
\toolsection{obcpp} is a transpiler for the Oberon programming language.
It translates programs written in Oberon into \cpp{} programs and stores them in corresponding source and header files.
In addition, it stores the interface of each module in a symbol file which is required when other modules import the module.
The same interface is provided by the generated header file which can be used in other parts of the \cpp{} program.
\flowgraph{\resource{Oberon\\source code} \ar[r] & \toolbox{obcpp} \ar[r] \ar@/l/[d] \ar[rd] & \resource{\cpp{}\\source file} \\ \variable{ECSIMPORT} \ar[ru] & \resource{symbol\\files} \ar@/r/[u] & \resource{\cpp{}\\header file}}
\seeoberon\seecpp
}

\providecommand{\obdoc}{
\toolsection{obdoc} is a generic documentation generator for the Oberon programming language.
It processes several Oberon modules and assembles all information therein into a generic documentation.
In addition, it stores the interface of each module in a symbol file which is required when other modules import the module.
\debuggingtool
\flowgraph{\resource{Oberon\\source code} \ar[r] & \toolbox{obdoc} \ar[r] \ar@/l/[d] & \resource{generic\\documentation} \\ \variable{ECSIMPORT} \ar[ru] & \resource{symbol\\files} \ar@/r/[u]}
\seeoberon\seedocumentation
}

\providecommand{\obhtml}{
\toolsection{obhtml} is an HTML documentation generator for the Oberon programming language.
It processes several Oberon modules and assembles all information therein into an HTML document.
In addition, it stores the interface of each module in a symbol file which is required when other modules import the module.
\flowgraph{\resource{Oberon\\source code} \ar[r] & \toolbox{obhtml} \ar[r] \ar@/l/[d] & \resource{HTML\\document} \\ \variable{ECSIMPORT} \ar[ru] & \resource{symbol\\files} \ar@/r/[u]}
\seeoberon\seedocumentation
}

\providecommand{\oblatex}{
\toolsection{oblatex} is a Latex documentation generator for the Oberon programming language.
It processes several Oberon modules and assembles all information therein into a Latex document.
In addition, it stores the interface of each module in a symbol file which is required when other modules import the module.
\flowgraph{\resource{Oberon\\source code} \ar[r] & \toolbox{oblatex} \ar[r] \ar@/l/[d] & \resource{Latex\\document} \\ \variable{ECSIMPORT} \ar[ru] & \resource{symbol\\files} \ar@/r/[u]}
\seeoberon\seedocumentation
}

\providecommand{\obcode}{
\toolsection{obcode} is an intermediate code generator for the Oberon programming language.
It generates intermediate code from modules written in Oberon and stores it in corresponding assembly files.
In addition, it stores the interface of each module in a symbol file which is required when other modules import the module.
Programs generated with this tool require additional runtime support that is stored in the \file{ob\-code\-run} library file.
\debuggingtool
\flowgraph{\resource{Oberon\\source code} \ar[r] & \toolbox{obcode} \ar[r] \ar@/l/[d] & \resource{intermediate\\code} \\ \variable{ECSIMPORT} \ar[ru] & \resource{symbol\\files} \ar@/r/[u]}
\seeoberon\seeassembly\seecode
}

\providecommand{\obamda}{
\toolsection{obamd16} is a compiler for the Oberon programming language targeting the AMD64 hardware architecture.
It generates machine code for AMD64 processors from modules written in Oberon and stores it in corresponding object files.
The compiler generates machine code for the 16-bit operating mode defined by the AMD64 architecture.
For debugging purposes, it also creates a debugging information file as well as an assembly file containing a listing of the generated machine code.
In addition, it stores the interface of each module in a symbol file which is required when other modules import the module.
Programs generated with this compiler require additional runtime support that is stored in the \file{ob\-amd16\-run} library file.
\flowgraph{\resource{Oberon\\source code} \ar[r] & \toolbox{obamd16} \ar[r] \ar@/l/[d] \ar[rd] & \resource{object file} \\ \variable{ECSIMPORT} \ar[ru] & \resource{symbol\\files} \ar@/r/[u] & \resource{debugging\\information}}
\seeoberon\seeassembly\seeamd\seeobject\seedebugging
}

\providecommand{\obamdb}{
\toolsection{obamd32} is a compiler for the Oberon programming language targeting the AMD64 hardware architecture.
It generates machine code for AMD64 processors from modules written in Oberon and stores it in corresponding object files.
The compiler generates machine code for the 32-bit operating mode defined by the AMD64 architecture.
For debugging purposes, it also creates a debugging information file as well as an assembly file containing a listing of the generated machine code.
In addition, it stores the interface of each module in a symbol file which is required when other modules import the module.
Programs generated with this compiler require additional runtime support that is stored in the \file{ob\-amd32\-run} library file.
\flowgraph{\resource{Oberon\\source code} \ar[r] & \toolbox{obamd32} \ar[r] \ar@/l/[d] \ar[rd] & \resource{object file} \\ \variable{ECSIMPORT} \ar[ru] & \resource{symbol\\files} \ar@/r/[u] & \resource{debugging\\information}}
\seeoberon\seeassembly\seeamd\seeobject\seedebugging
}

\providecommand{\obamdc}{
\toolsection{obamd64} is a compiler for the Oberon programming language targeting the AMD64 hardware architecture.
It generates machine code for AMD64 processors from modules written in Oberon and stores it in corresponding object files.
The compiler generates machine code for the 64-bit operating mode defined by the AMD64 architecture.
For debugging purposes, it also creates a debugging information file as well as an assembly file containing a listing of the generated machine code.
In addition, it stores the interface of each module in a symbol file which is required when other modules import the module.
Programs generated with this compiler require additional runtime support that is stored in the \file{ob\-amd64\-run} library file.
\flowgraph{\resource{Oberon\\source code} \ar[r] & \toolbox{obamd64} \ar[r] \ar@/l/[d] \ar[rd] & \resource{object file} \\ \variable{ECSIMPORT} \ar[ru] & \resource{symbol\\files} \ar@/r/[u] & \resource{debugging\\information}}
\seeoberon\seeassembly\seeamd\seeobject\seedebugging
}

\providecommand{\obarma}{
\toolsection{obarma32} is a compiler for the Oberon programming language targeting the ARM hardware architecture.
It generates machine code for ARM processors executing A32 instructions from modules written in Oberon and stores it in corresponding object files.
For debugging purposes, it also creates a debugging information file as well as an assembly file containing a listing of the generated machine code.
In addition, it stores the interface of each module in a symbol file which is required when other modules import the module.
Programs generated with this compiler require additional runtime support that is stored in the \file{ob\-arma32\-run} library file.
\flowgraph{\resource{Oberon\\source code} \ar[r] & \toolbox{obarma32} \ar[r] \ar@/l/[d] \ar[rd] & \resource{object file} \\ \variable{ECSIMPORT} \ar[ru] & \resource{symbol\\files} \ar@/r/[u] & \resource{debugging\\information}}
\seeoberon\seeassembly\seearm\seeobject\seedebugging
}

\providecommand{\obarmb}{
\toolsection{obarma64} is a compiler for the Oberon programming language targeting the ARM hardware architecture.
It generates machine code for ARM processors executing A64 instructions from modules written in Oberon and stores it in corresponding object files.
For debugging purposes, it also creates a debugging information file as well as an assembly file containing a listing of the generated machine code.
In addition, it stores the interface of each module in a symbol file which is required when other modules import the module.
Programs generated with this compiler require additional runtime support that is stored in the \file{ob\-arma64\-run} library file.
\flowgraph{\resource{Oberon\\source code} \ar[r] & \toolbox{obarma64} \ar[r] \ar@/l/[d] \ar[rd] & \resource{object file} \\ \variable{ECSIMPORT} \ar[ru] & \resource{symbol\\files} \ar@/r/[u] & \resource{debugging\\information}}
\seeoberon\seeassembly\seearm\seeobject\seedebugging
}

\providecommand{\obarmc}{
\toolsection{obarmt32} is a compiler for the Oberon programming language targeting the ARM hardware architecture.
It generates machine code for ARM processors without floating-point extension executing T32 instructions from modules written in Oberon and stores it in corresponding object files.
For debugging purposes, it also creates a debugging information file as well as an assembly file containing a listing of the generated machine code.
In addition, it stores the interface of each module in a symbol file which is required when other modules import the module.
Programs generated with this compiler require additional runtime support that is stored in the \file{ob\-armt32\-run} library file.
\flowgraph{\resource{Oberon\\source code} \ar[r] & \toolbox{obarmt32} \ar[r] \ar@/l/[d] \ar[rd] & \resource{object file} \\ \variable{ECSIMPORT} \ar[ru] & \resource{symbol\\files} \ar@/r/[u] & \resource{debugging\\information}}
\seeoberon\seeassembly\seearm\seeobject\seedebugging
}

\providecommand{\obarmcfpe}{
\toolsection{obarmt32fpe} is a compiler for the Oberon programming language targeting the ARM hardware architecture.
It generates machine code for ARM processors with floating-point extension executing T32 instructions from modules written in Oberon and stores it in corresponding object files.
For debugging purposes, it also creates a debugging information file as well as an assembly file containing a listing of the generated machine code.
In addition, it stores the interface of each module in a symbol file which is required when other modules import the module.
Programs generated with this compiler require additional runtime support that is stored in the \file{ob\-armt32\-fpe\-run} library file.
\flowgraph{\resource{Oberon\\source code} \ar[r] & \toolbox{obarmt32fpe} \ar[r] \ar@/l/[d] \ar[rd] & \resource{object file} \\ \variable{ECSIMPORT} \ar[ru] & \resource{symbol\\files} \ar@/r/[u] & \resource{debugging\\information}}
\seeoberon\seeassembly\seearm\seeobject\seedebugging
}

\providecommand{\obavr}{
\toolsection{obavr} is a compiler for the Oberon programming language targeting the AVR hardware architecture.
It generates machine code for AVR processors from modules written in Oberon and stores it in corresponding object files.
For debugging purposes, it also creates a debugging information file as well as an assembly file containing a listing of the generated machine code.
In addition, it stores the interface of each module in a symbol file which is required when other modules import the module.
Programs generated with this compiler require additional runtime support that is stored in the \file{ob\-avr\-run} library file.
\flowgraph{\resource{Oberon\\source code} \ar[r] & \toolbox{obavr} \ar[r] \ar@/l/[d] \ar[rd] & \resource{object file} \\ \variable{ECSIMPORT} \ar[ru] & \resource{symbol\\files} \ar@/r/[u] & \resource{debugging\\information}}
\seeoberon\seeassembly\seeavr\seeobject\seedebugging
}

\providecommand{\obavrtt}{
\toolsection{obavr32} is a compiler for the Oberon programming language targeting the AVR32 hardware architecture.
It generates machine code for AVR32 processors from modules written in Oberon and stores it in corresponding object files.
For debugging purposes, it also creates a debugging information file as well as an assembly file containing a listing of the generated machine code.
In addition, it stores the interface of each module in a symbol file which is required when other modules import the module.
Programs generated with this compiler require additional runtime support that is stored in the \file{ob\-avr32\-run} library file.
\flowgraph{\resource{Oberon\\source code} \ar[r] & \toolbox{obavr32} \ar[r] \ar@/l/[d] \ar[rd] & \resource{object file} \\ \variable{ECSIMPORT} \ar[ru] & \resource{symbol\\files} \ar@/r/[u] & \resource{debugging\\information}}
\seeoberon\seeassembly\seeavrtt\seeobject\seedebugging
}

\providecommand{\obmabk}{
\toolsection{obm68k} is a compiler for the Oberon programming language targeting the M68000 hardware architecture.
It generates machine code for M68000 processors from modules written in Oberon and stores it in corresponding object files.
For debugging purposes, it also creates a debugging information file as well as an assembly file containing a listing of the generated machine code.
In addition, it stores the interface of each module in a symbol file which is required when other modules import the module.
Programs generated with this compiler require additional runtime support that is stored in the \file{ob\-m68k\-run} library file.
\flowgraph{\resource{Oberon\\source code} \ar[r] & \toolbox{obm68k} \ar[r] \ar@/l/[d] \ar[rd] & \resource{object file} \\ \variable{ECSIMPORT} \ar[ru] & \resource{symbol\\files} \ar@/r/[u] & \resource{debugging\\information}}
\seeoberon\seeassembly\seemabk\seeobject\seedebugging
}

\providecommand{\obmibl}{
\toolsection{obmibl} is a compiler for the Oberon programming language targeting the MicroBlaze hardware architecture.
It generates machine code for MicroBlaze processors from modules written in Oberon and stores it in corresponding object files.
For debugging purposes, it also creates a debugging information file as well as an assembly file containing a listing of the generated machine code.
In addition, it stores the interface of each module in a symbol file which is required when other modules import the module.
Programs generated with this compiler require additional runtime support that is stored in the \file{ob\-mibl\-run} library file.
\flowgraph{\resource{Oberon\\source code} \ar[r] & \toolbox{obmibl} \ar[r] \ar@/l/[d] \ar[rd] & \resource{object file} \\ \variable{ECSIMPORT} \ar[ru] & \resource{symbol\\files} \ar@/r/[u] & \resource{debugging\\information}}
\seeoberon\seeassembly\seemibl\seeobject\seedebugging
}

\providecommand{\obmipsa}{
\toolsection{obmips32} is a compiler for the Oberon programming language targeting the MIPS32 hardware architecture.
It generates machine code for MIPS32 processors from modules written in Oberon and stores it in corresponding object files.
For debugging purposes, it also creates a debugging information file as well as an assembly file containing a listing of the generated machine code.
In addition, it stores the interface of each module in a symbol file which is required when other modules import the module.
Programs generated with this compiler require additional runtime support that is stored in the \file{ob\-mips32\-run} library file.
\flowgraph{\resource{Oberon\\source code} \ar[r] & \toolbox{obmips32} \ar[r] \ar@/l/[d] \ar[rd] & \resource{object file} \\ \variable{ECSIMPORT} \ar[ru] & \resource{symbol\\files} \ar@/r/[u] & \resource{debugging\\information}}
\seeoberon\seeassembly\seemips\seeobject\seedebugging
}

\providecommand{\obmipsb}{
\toolsection{obmips64} is a compiler for the Oberon programming language targeting the MIPS64 hardware architecture.
It generates machine code for MIPS64 processors from modules written in Oberon and stores it in corresponding object files.
For debugging purposes, it also creates a debugging information file as well as an assembly file containing a listing of the generated machine code.
In addition, it stores the interface of each module in a symbol file which is required when other modules import the module.
Programs generated with this compiler require additional runtime support that is stored in the \file{ob\-mips64\-run} library file.
\flowgraph{\resource{Oberon\\source code} \ar[r] & \toolbox{obmips64} \ar[r] \ar@/l/[d] \ar[rd] & \resource{object file} \\ \variable{ECSIMPORT} \ar[ru] & \resource{symbol\\files} \ar@/r/[u] & \resource{debugging\\information}}
\seeoberon\seeassembly\seemips\seeobject\seedebugging
}

\providecommand{\obmmix}{
\toolsection{obmmix} is a compiler for the Oberon programming language targeting the MMIX hardware architecture.
It generates machine code for MMIX processors from modules written in Oberon and stores it in corresponding object files.
For debugging purposes, it also creates a debugging information file as well as an assembly file containing a listing of the generated machine code.
In addition, it stores the interface of each module in a symbol file which is required when other modules import the module.
Programs generated with this compiler require additional runtime support that is stored in the \file{ob\-mmix\-run} library file.
\flowgraph{\resource{Oberon\\source code} \ar[r] & \toolbox{obmmix} \ar[r] \ar@/l/[d] \ar[rd] & \resource{object file} \\ \variable{ECSIMPORT} \ar[ru] & \resource{symbol\\files} \ar@/r/[u] & \resource{debugging\\information}}
\seeoberon\seeassembly\seemmix\seeobject\seedebugging
}

\providecommand{\oborok}{
\toolsection{obor1k} is a compiler for the Oberon programming language targeting the OpenRISC 1000 hardware architecture.
It generates machine code for OpenRISC 1000 processors from modules written in Oberon and stores it in corresponding object files.
For debugging purposes, it also creates a debugging information file as well as an assembly file containing a listing of the generated machine code.
In addition, it stores the interface of each module in a symbol file which is required when other modules import the module.
Programs generated with this compiler require additional runtime support that is stored in the \file{ob\-or1k\-run} library file.
\flowgraph{\resource{Oberon\\source code} \ar[r] & \toolbox{obor1k} \ar[r] \ar@/l/[d] \ar[rd] & \resource{object file} \\ \variable{ECSIMPORT} \ar[ru] & \resource{symbol\\files} \ar@/r/[u] & \resource{debugging\\information}}
\seeoberon\seeassembly\seeorok\seeobject\seedebugging
}

\providecommand{\obppca}{
\toolsection{obppc32} is a compiler for the Oberon programming language targeting the PowerPC hardware architecture.
It generates machine code for PowerPC processors from modules written in Oberon and stores it in corresponding object files.
The compiler generates machine code for the 32-bit operating mode defined by the PowerPC architecture.
For debugging purposes, it also creates a debugging information file as well as an assembly file containing a listing of the generated machine code.
In addition, it stores the interface of each module in a symbol file which is required when other modules import the module.
Programs generated with this compiler require additional runtime support that is stored in the \file{ob\-ppc32\-run} library file.
\flowgraph{\resource{Oberon\\source code} \ar[r] & \toolbox{obppc32} \ar[r] \ar@/l/[d] \ar[rd] & \resource{object file} \\ \variable{ECSIMPORT} \ar[ru] & \resource{symbol\\files} \ar@/r/[u] & \resource{debugging\\information}}
\seeoberon\seeassembly\seeppc\seeobject\seedebugging
}

\providecommand{\obppcb}{
\toolsection{obppc64} is a compiler for the Oberon programming language targeting the PowerPC hardware architecture.
It generates machine code for PowerPC processors from modules written in Oberon and stores it in corresponding object files.
The compiler generates machine code for the 64-bit operating mode defined by the PowerPC architecture.
For debugging purposes, it also creates a debugging information file as well as an assembly file containing a listing of the generated machine code.
In addition, it stores the interface of each module in a symbol file which is required when other modules import the module.
Programs generated with this compiler require additional runtime support that is stored in the \file{ob\-ppc64\-run} library file.
\flowgraph{\resource{Oberon\\source code} \ar[r] & \toolbox{obppc64} \ar[r] \ar@/l/[d] \ar[rd] & \resource{object file} \\ \variable{ECSIMPORT} \ar[ru] & \resource{symbol\\files} \ar@/r/[u] & \resource{debugging\\information}}
\seeoberon\seeassembly\seeppc\seeobject\seedebugging
}

\providecommand{\obrisc}{
\toolsection{obrisc} is a compiler for the Oberon programming language targeting the RISC hardware architecture.
It generates machine code for RISC processors from modules written in Oberon and stores it in corresponding object files.
For debugging purposes, it also creates a debugging information file as well as an assembly file containing a listing of the generated machine code.
In addition, it stores the interface of each module in a symbol file which is required when other modules import the module.
Programs generated with this compiler require additional runtime support that is stored in the \file{ob\-risc\-run} library file.
\flowgraph{\resource{Oberon\\source code} \ar[r] & \toolbox{obrisc} \ar[r] \ar@/l/[d] \ar[rd] & \resource{object file} \\ \variable{ECSIMPORT} \ar[ru] & \resource{symbol\\files} \ar@/r/[u] & \resource{debugging\\information}}
\seeoberon\seeassembly\seerisc\seeobject\seedebugging
}

\providecommand{\obwasm}{
\toolsection{obwasm} is a compiler for the Oberon programming language targeting the WebAssembly architecture.
It generates machine code for WebAssembly targets from modules written in Oberon and stores it in corresponding object files.
For debugging purposes, it also creates a debugging information file as well as an assembly file containing a listing of the generated machine code.
In addition, it stores the interface of each module in a symbol file which is required when other modules import the module.
Programs generated with this compiler require additional runtime support that is stored in the \file{ob\-wasm\-run} library file.
\flowgraph{\resource{Oberon\\source code} \ar[r] & \toolbox{obwasm} \ar[r] \ar@/l/[d] \ar[rd] & \resource{object file} \\ \variable{ECSIMPORT} \ar[ru] & \resource{symbol\\files} \ar@/r/[u] & \resource{debugging\\information}}
\seeoberon\seeassembly\seewasm\seeobject\seedebugging
}

% converter tools

\providecommand{\dbgdwarf}{
\toolsection{dbgdwarf} is a DWARF debugging information converter tool.
It converts debugging information into the DWARF debugging data format and stores it in corresponding object files~\cite{dwarffile}.
The resulting debugging object files can be combined with runtime support that creates Executable and Linking Format (ELF) files~\cite{elffile}.
\flowgraph{\resource{debugging\\information} \ar[r] & \toolbox{dbgdwarf} \ar[r] & \resource{debugging\\object file}}
\seeobject\seedebugging
}

% assembler tools

\providecommand{\asmprint}{
\toolsection{asmprint} is a pretty printer for generic assembly code.
It reformats generic assembly code and writes it to the standard output stream.
\flowgraph{\resource{generic assembly\\source code} \ar[r] & \toolbox{asmprint} \ar[r] & \resource{reformatted\\source code}}
\seeassembly
}

\providecommand{\amdaasm}{
\toolsection{amd16asm} is an assembler for the AMD64 hardware architecture.
It translates assembly code into machine code for AMD64 processors and stores it in corresponding object files.
By default, the assembler generates machine code for the 16-bit operating mode defined by the AMD64 architecture.
\flowgraph{\resource{AMD16 assembly\\source code} \ar[r] & \toolbox{amd16asm} \ar[r] & \resource{object file}}
\seeassembly\seeamd\seeobject
}

\providecommand{\amdadism}{
\toolsection{amd16dism} is a disassembler for the AMD64 hardware architecture.
It translates machine code from object files targeting AMD64 processors into assembly code and writes it to the standard output stream.
It assumes that the machine code was generated for the 16-bit operating mode defined by the AMD64 architecture.
\flowgraph{\resource{object file} \ar[r] & \toolbox{amd16dism} \ar[r] & \resource{disassembly\\listing}}
\seeassembly\seeamd\seeobject
}

\providecommand{\amdbasm}{
\toolsection{amd32asm} is an assembler for the AMD64 hardware architecture.
It translates assembly code into machine code for AMD64 processors and stores it in corresponding object files.
By default, the assembler generates machine code for the 32-bit operating mode defined by the AMD64 architecture.
\flowgraph{\resource{AMD32 assembly\\source code} \ar[r] & \toolbox{amd32asm} \ar[r] & \resource{object file}}
\seeassembly\seeamd\seeobject
}

\providecommand{\amdbdism}{
\toolsection{amd32dism} is a disassembler for the AMD64 hardware architecture.
It translates machine code from object files targeting AMD64 processors into assembly code and writes it to the standard output stream.
It assumes that the machine code was generated for the 32-bit operating mode defined by the AMD64 architecture.
\flowgraph{\resource{object file} \ar[r] & \toolbox{amd32dism} \ar[r] & \resource{disassembly\\listing}}
\seeassembly\seeamd\seeobject
}

\providecommand{\amdcasm}{
\toolsection{amd64asm} is an assembler for the AMD64 hardware architecture.
It translates assembly code into machine code for AMD64 processors and stores it in corresponding object files.
By default, the assembler generates machine code for the 64-bit operating mode defined by the AMD64 architecture.
\flowgraph{\resource{AMD64 assembly\\source code} \ar[r] & \toolbox{amd64asm} \ar[r] & \resource{object file}}
\seeassembly\seeamd\seeobject
}

\providecommand{\amdcdism}{
\toolsection{amd64dism} is a disassembler for the AMD64 hardware architecture.
It translates machine code from object files targeting AMD64 processors into assembly code and writes it to the standard output stream.
It assumes that the machine code was generated for the 64-bit operating mode defined by the AMD64 architecture.
\flowgraph{\resource{object file} \ar[r] & \toolbox{amd64dism} \ar[r] & \resource{disassembly\\listing}}
\seeassembly\seeamd\seeobject
}

\providecommand{\armaasm}{
\toolsection{arma32asm} is an assembler for the ARM hardware architecture.
It translates assembly code into machine code for ARM processors executing A32 instructions and stores it in corresponding object files.
\flowgraph{\resource{ARM A32 assembly\\source code} \ar[r] & \toolbox{arma32asm} \ar[r] & \resource{object file}}
\seeassembly\seearm\seeobject
}

\providecommand{\armadism}{
\toolsection{arma32dism} is a disassembler for the ARM hardware architecture.
It translates machine code from object files targeting ARM processors executing A32 instructions into assembly code and writes it to the standard output stream.
\flowgraph{\resource{object file} \ar[r] & \toolbox{arma32dism} \ar[r] & \resource{disassembly\\listing}}
\seeassembly\seearm\seeobject
}

\providecommand{\armbasm}{
\toolsection{arma64asm} is an assembler for the ARM hardware architecture.
It translates assembly code into machine code for ARM processors executing A64 instructions and stores it in corresponding object files.
\flowgraph{\resource{ARM A64 assembly\\source code} \ar[r] & \toolbox{arma64asm} \ar[r] & \resource{object file}}
\seeassembly\seearm\seeobject
}

\providecommand{\armbdism}{
\toolsection{arma64dism} is a disassembler for the ARM hardware architecture.
It translates machine code from object files targeting ARM processors executing A64 instructions into assembly code and writes it to the standard output stream.
\flowgraph{\resource{object file} \ar[r] & \toolbox{arma64dism} \ar[r] & \resource{disassembly\\listing}}
\seeassembly\seearm\seeobject
}

\providecommand{\armcasm}{
\toolsection{armt32asm} is an assembler for the ARM hardware architecture.
It translates assembly code into machine code for ARM processors executing T32 instructions and stores it in corresponding object files.
\flowgraph{\resource{ARM T32 assembly\\source code} \ar[r] & \toolbox{armt32asm} \ar[r] & \resource{object file}}
\seeassembly\seearm\seeobject
}

\providecommand{\armcdism}{
\toolsection{armt32dism} is a disassembler for the ARM hardware architecture.
It translates machine code from object files targeting ARM processors executing T32 instructions into assembly code and writes it to the standard output stream.
\flowgraph{\resource{object file} \ar[r] & \toolbox{armt32dism} \ar[r] & \resource{disassembly\\listing}}
\seeassembly\seearm\seeobject
}

\providecommand{\avrasm}{
\toolsection{avrasm} is an assembler for the AVR hardware architecture.
It translates assembly code into machine code for AVR processors and stores it in corresponding object files.
The identifiers \texttt{RXL}, \texttt{RXH}, \texttt{RYL}, \texttt{RYH}, \texttt{RZL}, and \texttt{RZH} are predefined and name the corresponding registers.
The identifiers \texttt{SPL} and \texttt{SPH} are also predefined and evaluate to the address of the corresponding registers.
\flowgraph{\resource{AVR assembly\\source code} \ar[r] & \toolbox{avrasm} \ar[r] & \resource{object file}}
\seeassembly\seeavr\seeobject
}

\providecommand{\avrdism}{
\toolsection{avrdism} is a disassembler for the AVR hardware architecture.
It translates machine code from object files targeting AVR processors into assembly code and writes it to the standard output stream.
\flowgraph{\resource{object file} \ar[r] & \toolbox{avrdism} \ar[r] & \resource{disassembly\\listing}}
\seeassembly\seeavr\seeobject
}

\providecommand{\avrttasm}{
\toolsection{avr32asm} is an assembler for the AVR32 hardware architecture.
It translates assembly code into machine code for AVR32 processors and stores it in corresponding object files.
\flowgraph{\resource{AVR32 assembly\\source code} \ar[r] & \toolbox{avr32asm} \ar[r] & \resource{object file}}
\seeassembly\seeavrtt\seeobject
}

\providecommand{\avrttdism}{
\toolsection{avr32dism} is a disassembler for the AVR32 hardware architecture.
It translates machine code from object files targeting AVR32 processors into assembly code and writes it to the standard output stream.
\flowgraph{\resource{object file} \ar[r] & \toolbox{avr32dism} \ar[r] & \resource{disassembly\\listing}}
\seeassembly\seeavrtt\seeobject
}

\providecommand{\mabkasm}{
\toolsection{m68kasm} is an assembler for the M68000 hardware architecture.
It translates assembly code into machine code for M68000 processors and stores it in corresponding object files.
\flowgraph{\resource{68000 assembly\\source code} \ar[r] & \toolbox{m68kasm} \ar[r] & \resource{object file}}
\seeassembly\seemabk\seeobject
}

\providecommand{\mabkdism}{
\toolsection{m68kdism} is a disassembler for the M68000 hardware architecture.
It translates machine code from object files targeting M68000 processors into assembly code and writes it to the standard output stream.
\flowgraph{\resource{object file} \ar[r] & \toolbox{m68kdism} \ar[r] & \resource{disassembly\\listing}}
\seeassembly\seemabk\seeobject
}

\providecommand{\miblasm}{
\toolsection{miblasm} is an assembler for the MicroBlaze hardware architecture.
It translates assembly code into machine code for MicroBlaze processors and stores it in corresponding object files.
\flowgraph{\resource{MicroBlaze assembly\\source code} \ar[r] & \toolbox{miblasm} \ar[r] & \resource{object file}}
\seeassembly\seemibl\seeobject
}

\providecommand{\mibldism}{
\toolsection{mibldism} is a disassembler for the MicroBlaze hardware architecture.
It translates machine code from object files targeting MicroBlaze processors into assembly code and writes it to the standard output stream.
\flowgraph{\resource{object file} \ar[r] & \toolbox{mibldism} \ar[r] & \resource{disassembly\\listing}}
\seeassembly\seemibl\seeobject
}

\providecommand{\mipsaasm}{
\toolsection{mips32asm} is an assembler for the MIPS32 hardware architecture.
It translates assembly code into machine code for MIPS32 processors and stores it in corresponding object files.
\flowgraph{\resource{MIPS32 assembly\\source code} \ar[r] & \toolbox{mips32asm} \ar[r] & \resource{object file}}
\seeassembly\seemips\seeobject
}

\providecommand{\mipsadism}{
\toolsection{mips32dism} is a disassembler for the MIPS32 hardware architecture.
It translates machine code from object files targeting MIPS32 processors into assembly code and writes it to the standard output stream.
\flowgraph{\resource{object file} \ar[r] & \toolbox{mips32dism} \ar[r] & \resource{disassembly\\listing}}
\seeassembly\seemips\seeobject
}

\providecommand{\mipsbasm}{
\toolsection{mips64asm} is an assembler for the MIPS64 hardware architecture.
It translates assembly code into machine code for MIPS64 processors and stores it in corresponding object files.
\flowgraph{\resource{MIPS64 assembly\\source code} \ar[r] & \toolbox{mips64asm} \ar[r] & \resource{object file}}
\seeassembly\seemips\seeobject
}

\providecommand{\mipsbdism}{
\toolsection{mips64dism} is a disassembler for the MIPS64 hardware architecture.
It translates machine code from object files targeting MIPS64 processors into assembly code and writes it to the standard output stream.
\flowgraph{\resource{object file} \ar[r] & \toolbox{mips64dism} \ar[r] & \resource{disassembly\\listing}}
\seeassembly\seemips\seeobject
}

\providecommand{\mmixasm}{
\toolsection{mmixasm} is an assembler for the MMIX hardware architecture.
It translates assembly code into machine code for MMIX processors and stores it in corresponding object files.
The names of all special registers are predefined and evaluate to the corresponding number.
\flowgraph{\resource{MMIX assembly\\source code} \ar[r] & \toolbox{mmixasm} \ar[r] & \resource{object file}}
\seeassembly\seemmix\seeobject
}

\providecommand{\mmixdism}{
\toolsection{mmixdism} is a disassembler for the MMIX hardware architecture.
It translates machine code from object files targeting MMIX processors into assembly code and writes it to the standard output stream.
\flowgraph{\resource{object file} \ar[r] & \toolbox{mmixdism} \ar[r] & \resource{disassembly\\listing}}
\seeassembly\seemmix\seeobject
}

\providecommand{\orokasm}{
\toolsection{or1kasm} is an assembler for the OpenRISC 1000 hardware architecture.
It translates assembly code into machine code for OpenRISC 1000 processors and stores it in corresponding object files.
\flowgraph{\resource{OpenRISC 1000 assembly\\source code} \ar[r] & \toolbox{or1kasm} \ar[r] & \resource{object file}}
\seeassembly\seeorok\seeobject
}

\providecommand{\orokdism}{
\toolsection{or1kdism} is a disassembler for the OpenRISC 1000 hardware architecture.
It translates machine code from object files targeting OpenRISC 1000 processors into assembly code and writes it to the standard output stream.
\flowgraph{\resource{object file} \ar[r] & \toolbox{or1kdism} \ar[r] & \resource{disassembly\\listing}}
\seeassembly\seeorok\seeobject
}

\providecommand{\ppcaasm}{
\toolsection{ppc32asm} is an assembler for the PowerPC hardware architecture.
It translates assembly code into machine code for PowerPC processors and stores it in corresponding object files.
By default, the assembler generates machine code for the 32-bit operating mode defined by the PowerPC architecture.
\flowgraph{\resource{PowerPC assembly\\source code} \ar[r] & \toolbox{ppc32asm} \ar[r] & \resource{object file}}
\seeassembly\seeppc\seeobject
}

\providecommand{\ppcadism}{
\toolsection{ppc32dism} is a disassembler for the PowerPC hardware architecture.
It translates machine code from object files targeting PowerPC processors into assembly code and writes it to the standard output stream.
It assumes that the machine code was generated for the 32-bit operating mode defined by the PowerPC architecture.
\flowgraph{\resource{object file} \ar[r] & \toolbox{ppc32dism} \ar[r] & \resource{disassembly\\listing}}
\seeassembly\seeppc\seeobject
}

\providecommand{\ppcbasm}{
\toolsection{ppc64asm} is an assembler for the PowerPC hardware architecture.
It translates assembly code into machine code for PowerPC processors and stores it in corresponding object files.
By default, the assembler generates machine code for the 64-bit operating mode defined by the PowerPC architecture.
\flowgraph{\resource{PowerPC assembly\\source code} \ar[r] & \toolbox{ppc64asm} \ar[r] & \resource{object file}}
\seeassembly\seeppc\seeobject
}

\providecommand{\ppcbdism}{
\toolsection{ppc64dism} is a disassembler for the PowerPC hardware architecture.
It translates machine code from object files targeting PowerPC processors into assembly code and writes it to the standard output stream.
It assumes that the machine code was generated for the 64-bit operating mode defined by the PowerPC architecture.
\flowgraph{\resource{object file} \ar[r] & \toolbox{ppc64dism} \ar[r] & \resource{disassembly\\listing}}
\seeassembly\seeppc\seeobject
}

\providecommand{\riscasm}{
\toolsection{riscasm} is an assembler for the RISC hardware architecture.
It translates assembly code into machine code for RISC processors and stores it in corresponding object files.
The names of all special registers are predefined and evaluate to the corresponding number.
\flowgraph{\resource{RISC assembly\\source code} \ar[r] & \toolbox{riscasm} \ar[r] & \resource{object file}}
\seeassembly\seerisc\seeobject
}

\providecommand{\riscdism}{
\toolsection{riscdism} is a disassembler for the RISC hardware architecture.
It translates machine code from object files targeting RISC processors into assembly code and writes it to the standard output stream.
\flowgraph{\resource{object file} \ar[r] & \toolbox{riscdism} \ar[r] & \resource{disassembly\\listing}}
\seeassembly\seerisc\seeobject
}

\providecommand{\wasmasm}{
\toolsection{wasmasm} is an assembler for the WebAssembly architecture.
It translates assembly code into machine code for WebAssembly targets and stores it in corresponding object files.
The names of all special registers are predefined and evaluate to the corresponding number.
\flowgraph{\resource{WebAssembly assembly\\source code} \ar[r] & \toolbox{wasmasm} \ar[r] & \resource{object file}}
\seeassembly\seewasm\seeobject
}

\providecommand{\wasmdism}{
\toolsection{wasmdism} is a disassembler for the WebAssembly architecture.
It translates machine code from object files targeting WebAssembly targets into assembly code and writes it to the standard output stream.
\flowgraph{\resource{object file} \ar[r] & \toolbox{wasmdism} \ar[r] & \resource{disassembly\\listing}}
\seeassembly\seewasm\seeobject
}

% linker tools

\providecommand{\linklib}{
\toolsection{linklib} is an object file combiner.
It creates a static library file by combining all object files given to it into a single one.
\flowgraph{\resource{object files} \ar[r] & \toolbox{linklib} \ar[r] & \resource{library file}}
\seeobject
}

\providecommand{\linkbin}{
\toolsection{linkbin} is a linker for plain binary files.
It links all object files given to it into a single image and stores it in a binary file that begins with the first linked section.
It also creates a map file that lists the address, type, name and size of all used sections.
The filename extension of the resulting binary file can be specified by putting it into a constant data section called \texttt{\_extension}.
\flowgraph{\resource{object files} \ar[r] & \toolbox{linkbin} \ar[r] \ar[d] & \resource{binary file} \\ & \resource{map file}}
\seeobject
}

\providecommand{\linkmem}{
\toolsection{linkmem} is a linker for plain binary files partitioned into random-access and read-only memory.
It links all object files given to it into two distinct images, one for data sections and one for code and constant data sections, and stores each image in a binary file that begins with the first linked section of the corresponding type.
It also creates a map file that lists the address, type, name and size of all used sections.
\flowgraph{\resource{object files} \ar[r] & \toolbox{linkmem} \ar[r] \ar[d] & \resource{RAM file/\\ROM file} \\ & \resource{map file}}
\seeobject
}

\providecommand{\linkprg}{
\toolsection{linkprg} is a linker for GEMDOS executable files.
It links all object files given to it into a single image and stores the image in an Atari GEMDOS executable file~\cite{gemdosfile}.
It also creates a map file that lists the address relative to the text segment, type, name and size of all used sections.
The filename extension of the resulting executable file can be specified by putting it into a constant data section called \texttt{\_extension}.
The GEMDOS executable file format requires all patch patterns of absolute link patches to consist of four full bitmasks with descending offsets.
\flowgraph{\resource{object files} \ar[r] & \toolbox{linkprg} \ar[r] \ar[d] & \resource{executable file} \\ & \resource{map file}}
\seeobject
}

\providecommand{\linkhex}{
\toolsection{linkhex} is a linker for Intel HEX files.
It links all code sections of the object files given to it into single image and stores the image in an Intel HEX file~\cite{hexfile} that begins with the first linked section.
It also creates a map file that lists the address, type, name and size of all used sections.
\flowgraph{\resource{object files} \ar[r] & \toolbox{linkhex} \ar[r] \ar[d] & \resource{HEX file} \\ & \resource{map file}}
\seeobject
}

\providecommand{\mapsearch}{
\toolsection{mapsearch} is a debugging tool.
It searches map files generated by linker tools for the name of a binary section that encompasses a memory address read from the standard input stream.
If additionally provided with one or more object files, it also stores an excerpt thereof in a separate object file called map search result which only contains the identified binary section for disassembling purposes.
\flowgraph{& \resource{map files/\\object files} \ar[d] \\ \resource{memory\\address} \ar[r] & \toolbox{mapsearch} \ar[r] \ar[d] & \resource{section name/\\relative offset} \\ & \resource{object file\\excerpt}}
\seeobject
}

\renewcommand{\seeoberon}{}

\startchapter{Oberon}{User Manual for Oberon}{oberon}
{Oberon is a general-purpose programming language based on Pascal and Modula-2.
It supports type extension with type-bound procedures which makes it an object-oriented language.
This \documentation{} describes Oberon and its implementation by the \ecs{}.}

\epigraph{Die Kunst ist eine Tochter der Freiheit.}{Friedrich Schiller}

\section{Introduction}

The \ecs{} implements the Oberon programming language according to the Oberon-2 language report~\cite{moessenboeck1996}.
The most important language features are block structure, modularity, separate compilation, static typing with strong type checking, and type extension with type-bound procedures.

\begin{center}\oblogo{4em}\end{center}

The \ecs{} adds some completely backwards-compatible features for portability, genericity, and interoperability which are described in the remainder of this \documentation{}.

\section{Implementation-Defined Behavior}

Although the semantics of the Oberon programming language is well defined, there are some issues that depend on its actual implementation and need detailed description.
This section lists all implementation-defined behavior and language extensions along with the corresponding section numbers of the Oberon-2 language report where all changes to the syntax and the set of predeclared identifiers are \changed{underlined}.

\newcommand{\obref}[1]{\alignright(#1)\nopagebreak}
\newcommand{\obsection}[2]{\subsection[#1]{#1\alignright(#2)}}

\obsection{Vocabulary and Representation}{3}

\begin{itemize}

\item Underscore characters in identifiers\obref{3.1}

For interoperability with other languages and interfacing with external libraries the \ecs{} allows underscore characters in identifiers.
The first character must still be a letter.

\item Binary integer constants\obref{3.2}

If an integer constant is specified with the suffix \texttt{B}, its digits and representation are binary.

\item Digit grouping\obref{3.2}

The integer part of a number may contain separating single quotes between its digits which are ignored when determining its value.

\item Type of real numbers\obref{3.2}

The type of a real number is the minimal type to which the constant value belongs.
The letter contained in the scale factor has no meaning with respect to the type.

\item Line breaks in strings\obref{3.4}

The \ecs{} allows line breaks in strings in order to support inline assembly code with multiple lines, see Section~\ref{sec:obsystemmodule}.

\end{itemize}

\obsection{Declarations and Scope Rules}{4}\label{sec:obdeclarations}

\begin{itemize}

\item Nested qualified identifiers\obref{4}

Qualified identifiers may be nested:

\begin{quote}\begin{grammar}
<Qualident> = $[$\changed{\synt{Qualident}}"."$]$<ident> \par
\end{grammar}\end{quote}

\item External declarations\obref{4}

The identifier definition of a variable or procedure declared in a module block may be followed by a constant expression enclosed in brackets which marks the declaration as \emph{external}:

\begin{quote}\begin{grammar}
<IdentDef> = <ident>$[$"*"$\mid$"-"$]$ \changed{$[$"["\synt{ConstExpression}"]"$]$} \par
\end{grammar}\end{quote}

An external declaration either defines a global alias name using a string or specifies its address using an integer included by \texttt{SYSTEM.AD\-DRESS} and requires the module \texttt{SYSTEM} to be imported, see Section~\ref{sec:obsystemmodule}.
Forward declarations marked as external do not have to be declared later in the text and refer to arbitrary entities with the respective name or address.

\item Predeclared identifiers\obref{4}

The \ecs{} adds the identifiers listed in Table~\ref{tab:obpredeclaredidentifiers} to the set of predeclared identifiers.

\begin{table}
\centering
\begin{tabular}{@{}lp{0.6\textwidth}r@{}}
\toprule Category & Identifier & Section \\
\midrule Constants & \texttt{I}, \texttt{INF}, \texttt{NAN} & \ref{sec:obtypedeclarations} \\
\midrule Types & \texttt{CARDINAL}, \texttt{COMPLEX}, \texttt{COMPLEX32}, \texttt{COMPLEX64}, \texttt{HUGECARD}, \texttt{HUGEINT}, \texttt{LENGTH}, \texttt{LONGCARD}, \texttt{LONGCOMPLEX}, \texttt{LONGSET}, \texttt{REAL32}, \texttt{REAL64}, \texttt{SET8}, \texttt{SET16}, \texttt{SET32}, \texttt{SET64}, \texttt{SHORTCARD}, \texttt{SHORTCOMPLEX}, \texttt{SHORTREAL}, \texttt{SIGNED8}, \texttt{SIGNED16}, \texttt{SIGNED32}, \texttt{SIGNED64}, \texttt{UNSIGNED8}, \texttt{UNSIGNED16}, \texttt{UNSIGNED32}, \texttt{UNSIGNED64} & \ref{sec:obtypedeclarations} \\
\midrule Procedures & \texttt{DISPOSE}, \texttt{IGNORE}, \texttt{IM}, \texttt{PTR}, \texttt{RE}, \texttt{SEL}, \texttt{TRACE} & \ref{sec:obpredeclaredprocedures} \\
\bottomrule
\end{tabular}
\caption{Additionally predeclared identifiers in Oberon}
\label{tab:obpredeclaredidentifiers}
\end{table}

\end{itemize}

\obsection{Type Declarations}{6}\label{sec:obtypedeclarations}

\begin{itemize}

\item Additional real values\obref{6.1}

Real numbers also contain the predeclared values \texttt{\changed{INF}} and \texttt{\changed{NAN}} denoting positive infinity and a quiet not a number.

\item Additional complex values\obref{6.1}

The \ecs{} supports complex numbers which belong to the numeric types and contain two real numbers representing the real and imaginary part of the complex number.
Complex numbers also contain the predeclared value \texttt{\changed{I}} denoting the imaginary unit \emph{i}.

\item Additional numeric types\obref{6.1}

The \ecs{} provides additional predeclared identifiers for numeric types with a fixed bit length.
They consist of four signed integer types called \texttt{\changed{SIGNED8}}, \texttt{\changed{SIGNED16}}, \texttt{\changed{SIGNED32}}, and \texttt{\changed{SIGNED64}}, four unsigned integer types called \texttt{\changed{UNSIGNED8}}, \texttt{\changed{UNSIGNED16}}, \texttt{\changed{UNSIGNED32}}, and \texttt{\changed{UNSIGNED64}}, two real types called \texttt{\changed{REAL32}} and \texttt{\changed{REAL64}}, as well as two complex types called \texttt{\changed{COMPLEX32}} and \texttt{\changed{COMPLEX64}}.
They form the following hierarchy where larger types include smaller ones:

\newcommand{\includesr}{\ar@{}[r]|{\txt{$\supseteq$}}}
\newcommand{\includesu}{\ar@{}[u]|{\txt{\rotatebox{90}{$\supseteq$}}}}
\newcommand{\includesd}{\ar@{}[d]|{\txt{\rotatebox{270}{$\supseteq$}}}}

\flowgraph{\texttt{COMPLEX32} \includesr & \texttt{REAL32} \includesr & \texttt{UNSIGNED64} \includesr \includesd & \texttt{SIGNED64} \includesd \\
\texttt{COMPLEX64} \includesr \includesu & \texttt{REAL64} \includesu & \texttt{UNSIGNED32} \includesr \includesd & \texttt{SIGNED32} \includesd \\
& & \texttt{UNSIGNED16} \includesr \includesd & \texttt{SIGNED16} \includesd \\
& & \texttt{UNSIGNED8} \includesr & \texttt{SIGNED8}}

The predefined set of numeric types is additionally extended by a signed integer type called \texttt{\changed{HUGEINT}}, four unsigned integer types called \texttt{\changed{SHORTCARD}}, \texttt{\changed{CARDINAL}}, \texttt{\changed{LONGCARD}}, and \texttt{\changed{HUGECARD}}, a real type called \texttt{\changed{SHORTREAL}}, as well as three complex types called \texttt{\changed{SHORTCOMPLEX}}, \texttt{\changed{COMPLEX}}, and \texttt{\changed{LONGCOMPLEX}}.
They form the following hierarchy where the types \texttt{INTEGER}, \texttt{CARDINAL}, and \texttt{REAL} name the default integer and real types of the execution environment:

\flowgraph{\texttt{SHORTCOMPLEX} \includesr & \texttt{SHORTREAL} \includesr & \texttt{HUGECARD} \includesr \includesd & \texttt{HUGEINT} \includesd \\
\texttt{COMPLEX} \includesr \includesu & \texttt{REAL} \includesu & \texttt{LONGCARD} \includesr \includesd & \texttt{LONGINT} \includesd \\
\texttt{LONGCOMPLEX} \includesr \includesu & \texttt{LONGREAL} \includesu & \texttt{CARDINAL} \includesr \includesd & \texttt{INTEGER} \includesd \\
& & \texttt{SHORTCARD} \includesr & \texttt{SHORTINT}}

Finally, the \ecs{} also predeclares a signed integer type called \texttt{\changed{LENGTH}} which is big enough to represent any array length of the execution environment.

\item Additional set types\obref{6.1}

The \ecs{} provides additional predeclared identifiers for set types with a fixed bit length.
They consist of four set types called \texttt{\changed{SET8}}, \texttt{\changed{SET16}}, \texttt{\changed{SET32}}, and \texttt{\changed{SET64}}.
They form the following hierarchy where larger types include smaller ones:

\flowgraph{\texttt{SET64} \includesr & \texttt{SET32} \includesr & \texttt{SET16} \includesr & \texttt{SET8}}

The predefined set of set types is additionally extended by three set types called \texttt{\changed{SHORTSET}}, \texttt{\changed{LONGSET}}, and \texttt{\changed{HUGESET}}.
They form the following hierarchy where the type \texttt{SET} names the default set type of the execution environment:

\flowgraph{\texttt{HUGESET} \includesr & \texttt{LONGSET} \includesr & \texttt{SET} \includesr & \texttt{SHORTSET}}

\item Type values\obref{6.1}

The sizes and value ranges of all basic types as defined by the \ecs{} are listed in Table~\ref{tab:obbasictypes}.
The sizes of the types \texttt{CARDINAL}, \texttt{INTEGER}, \texttt{LENGTH}, \texttt{REAL}, \texttt{SET}, and pointer types depend on the execution environment and are listed in Table~\ref{tab:obdefaulttypes} for each Oberon compiler provided by the \ecs{}.
The interpreter and all other tools reuse the respective type sizes of their own execution environment instead.

\begin{table}
\centering
\begin{tabular}{@{}lllcl@{}}
\toprule Category & Type & Alias & Size & Value Range \\
\midrule Boolean
& \texttt{BOOLEAN} & & 1 & \texttt{TRUE} or \texttt{FALSE} \\
\midrule Character
& \texttt{CHAR} & & 1 & \texttt{0X} to \texttt{0FFX} \\
\midrule Signed
& \texttt{SIGNED8} & & 1 & $-2^{7}$ to $+2^{7}-1$ \\
integer
& \texttt{SIGNED16} & \texttt{SHORTINT} & 2 & $-2^{15}$ to $+2^{15}-1$ \\
& & \texttt{INTEGER} & 2/4 & \emph{See Table~\ref{tab:obdefaulttypes}} \\
& \texttt{SIGNED32} & \texttt{LONGINT} & 4 & $-2^{31}$ to $+2^{31}-1$ \\
& \texttt{SIGNED64} & \texttt{HUGEINT} & 8 & $-2^{63}$ to $+2^{63}-1$ \\
& & \texttt{LENGTH} & 2/4/8 & \emph{See Table~\ref{tab:obdefaulttypes}} \\
\midrule Unsigned
& \texttt{UNSIGNED8} & & 1 & $0$ to $2^{8}-1$ \\
integer
& \texttt{UNSIGNED16} & \texttt{SHORTCARD} & 2 & $0$ to $2^{16}-1$ \\
& & \texttt{CARDINAL} & 2/4 & \emph{See Table~\ref{tab:obdefaulttypes}} \\
& \texttt{UNSIGNED32} & \texttt{LONGCARD} & 4 & $0$ to $2^{32}-1$ \\
& \texttt{UNSIGNED64} & \texttt{HUGECARD} & 8 & $0$ to $2^{64}-1$ \\
\midrule Real
& \texttt{REAL32} & \texttt{SHORTREAL} & 4 & $\pm 3.4028234 \times 10^{38}$ \\
number
& & \texttt{REAL} & 4/8 & \emph{See Table~\ref{tab:obdefaulttypes}} \\
& \texttt{REAL64} & \texttt{LONGREAL} & 8 & $\pm 1.7976931348623157 \times 10^{308}$ \\
\midrule Complex
& \texttt{COMPLEX32} & \texttt{SHORTCOMPLEX} & 8 & $\pm 3.4028234 \times 10^{38}i$ \\
number
& & \texttt{COMPLEX} & 8/16 & \emph{Two real numbers} \\
& \texttt{COMPLEX64} & \texttt{LONGCOMPLEX} & 16 & $\pm 1.7976931348623157 \times 10^{308}i$ \\
\midrule Set
& \texttt{SET8} & & 1 & \texttt{\{\}} to \texttt{\{0..7\}} \\
& \texttt{SET16} & \texttt{SHORTSET} & 2 & \texttt{\{\}} to \texttt{\{0..15\}} \\
& & \texttt{SET} & 2/4 & \emph{See Table~\ref{tab:obdefaulttypes}} \\
& \texttt{SET32} & \texttt{LONGSET} & 4 & \texttt{\{\}} to \texttt{\{0..31\}} \\
& \texttt{SET64} & \texttt{HUGESET} & 8 & \texttt{\{\}} to \texttt{\{0..63\}} \\
\bottomrule
\end{tabular}
\caption{Sizes and value ranges of basic Oberon types}
\label{tab:obbasictypes}
\end{table}

\begin{table}
\centering
\begin{tabular}{@{}lrlccc@{}}
\toprule \multicolumn{2}{@{}l}{Hardware} & Oberon & \texttt{CARDINAL}, & & \texttt{LENGTH}, \\ \multicolumn{2}{@{}l}{Architecture} & Compiler & \texttt{INTEGER}, \texttt{SET} & \texttt{REAL} & \texttt{POINTER} \\
\midrule AMD64 & 16-bit & \tool{obamd16} & 2 & 8 & 2 \\ & 32-bit & \tool{obamd32} & 4 & 8 & 4 \\ & 64-bit & \tool{obamd64} & 4 & 8 & 8 \\
\midrule ARM & A32 & \tool{obarma32} & 4 & 8 & 4 \\ & A64 & \tool{obarma64} & 4 & 8 & 8 \\ & T32 & \tool{obarmt32} & 4 & 4 & 4 \\ & & \tool{obarmt32fpe} & 4 & 8 & 4 \\
\midrule \multicolumn{2}{@{}l}{AVR} & \tool{obavr} & 2 & 4 & 2 \\
\midrule \multicolumn{2}{@{}l}{AVR32} & \tool{obavr32} & 4 & 4 & 4 \\
\midrule \multicolumn{2}{@{}l}{M68000} & \tool{obm68k} & 2 & 4 & 4 \\
\midrule \multicolumn{2}{@{}l}{MicroBlaze} & \tool{obmibl} & 4 & 4 & 4 \\
\midrule MIPS & 32-bit & \tool{obmips32} & 4 & 8 & 4 \\ & 64-bit & \tool{obmips64} & 4 & 8 & 8 \\
\midrule \multicolumn{2}{@{}l}{MMIX} & \tool{obmmix} & 4 & 8 & 8 \\
\midrule \multicolumn{2}{@{}l}{OpenRISC 1000} & \tool{obor1k} & 4 & 4 & 4 \\
\midrule PowerPC & 32-bit & \tool{obppc32} & 4 & 4 & 4 \\ & 64-bit & \tool{obppc64} & 4 & 4 & 4 \\
\midrule \multicolumn{2}{@{}l}{RISC} & \tool{obrisc} & 4 & 4 & 4 \\
\midrule \multicolumn{2}{@{}l}{WebAssembly} & \tool{obwasm} & 4 & 8 & 4 \\
\bottomrule
\end{tabular}
\caption{Sizes of hardware-dependent Oberon types}
\label{tab:obdefaulttypes}
\end{table}

\item Abstract and final record types\obref{6.3}

The \texttt{RECORD} keyword of a record type declaration may be followed by an export mark where "\texttt{*}" indicates that the record type is \emph{abstract} and "\texttt{-}" indicates that it is \emph{final}:

\begin{quote}\begin{grammar}
<RecordType> = "RECORD" \changed{$[$"*"$\mid$"-"$]$} $[$"("<BaseType>")"$]$ <FieldList> $\{$";" <FieldList>$\}$ "END" \par
\end{grammar}\end{quote}

An abstract record type must be extended before it can be assigned, allocated, or used as the type of array elements, variables, fields, value parameters, or function results.
A final record type is not extensible.

\item Initialization of pointers\obref{6.4}

All pointer variables are initialized to \texttt{NIL}.

\item Pointer to variables\obref{6.4}

A pointer type containing the \texttt{VAR} keyword may point to any variable:

\begin{quote}\begin{grammar}
<PointerType> = "POINTER" "TO" \changed{$[$"VAR"$]$} $[$"-"$]$ <Type>.
\end{grammar}\end{quote}

A pointer to a variable can be obtained using the predeclared procedure \texttt{PTR}, see Section~\ref{sec:obpredeclaredprocedures}, and is valid as long as the variable exists.
Pointers to variables are only expression compatible with other pointers to variables of extended or array compatible types where any type extends itself.

\item Read-only pointers\obref{6.4}

Pointer type declarations may be marked as read-only:

\begin{quote}\begin{grammar}
<PointerType> = "POINTER" "TO" $[$"VAR"$]$ \changed{$[$"-"$]$} <Type>.
\end{grammar}\end{quote}

Read-only pointers are only assignment compatible with read-only pointers.
The referenced variable of a read-only pointer is also read-only.

\end{itemize}

\obsection{Variable Declarations}{7}

\begin{itemize}

\item Forward declarations\obref{7}

A forward declaration allows forward references to variables whose actual declaration appears later in the text.
The data type of the forward declaration and the actual declaration must be the same.

\begin{quote}\begin{grammar}
<VariableDeclaration> = \changed{$[$\lit*{\^}$]$} <IdentList>":" <Type> \par
\end{grammar}\end{quote}

\end{itemize}

\obsection{Expressions}{8}

\begin{itemize}

\item Module designators\obref{8.1}

The identifier of a designator may refer to an imported module.
Selectors following a module designator are treated as qualified identifiers.

\item Value conversions\obref{8.2}

Any identifier denoting a basic type can also be used as the name of a function procedure that accepts a single value parameter of basic type.
The result of calling such a procedure is the value of the parameter converted to the named type.

\item Integer quotient and modulus\obref{8.2.2}

The fractional part of the integer quotient is discarded such that the modulus has the same sign as the dividend.

\item Typed set constructors\obref{8.2.3}

A set constructor may be prefixed by an identifier which specifies the set type of the result.

\begin{quote}\begin{grammar}
<Set> = \changed{$[$\synt{Qualident}$]$} "{" $[$<Element> $\{$"," <Element>$\}]$ "}"
\end{grammar}\end{quote}

If omitted, the type of the result defaults to the minimal set type to which its constant value belongs, or \texttt{SET} if the result is not constant.

\item Same type test\obref{8.2.4}

Type tests are also applicable on two types and yield whether they are the same.

\end{itemize}

\obsection{Procedure Declarations}{10}

\begin{itemize}

\item Abstract and final procedures\obref{10}

The \texttt{PROCEDURE} keyword of a type-bound procedure declaration may be followed by an export mark where "\texttt{*}" indicates that the procedure is \emph{abstract} and "\texttt{-}" indicates that it is \emph{final}.
The corresponding export marks of a forward declaration and its actual declaration must be identical:

\begin{quote}\begin{grammar}
<ProcedureHeading> = "PROCEDURE" \changed{$[$"*"$\mid$"-"$]$} $[$<Receiver>$]$ <IdentDef> $[$<FormalParameters>$]$ \par
<ForwardDeclaration> = "PROCEDURE" \changed{$[$"*"$\mid$"-"$]$} "^" $[$<Receiver>$]$ <IdentDef> $[$<FormalParameters>$]$ \par
\end{grammar}\end{quote}

An abstract procedure has no procedure body and causes all record types to which it is bound to be abstract as well.
A final procedure is not redefinable.

\item Read-only parameters\obref{10.1}

Receivers and formal parameters may be marked as read-only just like variables and fields:

\begin{quote}\begin{grammar}
<FPSection> = $[$"VAR"$]$ \changed{\synt{IdentList}}":" <Type> \par
<Receiver> = "("$[$"VAR"$]$ \changed{\synt{IdentDef}}":" <ident>")" \par
\end{grammar}\end{quote}

A type-bound procedure can be called using a read-only variable or field of record type if its receiver is read-only as well.

\item Structured result types\obref{10.1}

The result type of a function can be structured but neither an abstract record nor an open array.

\end{itemize}

\obsection{Predeclared Procedures}{10.3}\label{sec:obpredeclaredprocedures}

The \ecs{} supports all function procedures predeclared in the Oberon-2 language report and adds the following extensions:

\begin{itemize}

\item Arithmetic shift\alignright\texttt{ASH}\nopagebreak

The result type of the predeclared function procedure \texttt{ASH} is the integer type of its arguments.

\item Imaginary part\alignright\changed{\texttt{IM}}\nopagebreak

The predeclared function procedure \texttt{IM} accepts a complex value and returns its imaginary part.

\item Array length\alignright\texttt{LEN}\nopagebreak

The result type of the predeclared function procedure \texttt{LEN} is \texttt{LENGTH}, see Section~\ref{sec:obtypedeclarations}.

\item Variable pointer\alignright\changed{\texttt{PTR}}\nopagebreak

The predeclared function procedure \texttt{PTR} accepts a variable of any type and returns a pointer to it, see Section~\ref{sec:obtypedeclarations}.

\item Real part\alignright\changed{\texttt{RE}}\nopagebreak

The predeclared function procedure \texttt{RE} accepts a complex value and returns its real part.

\item Binary selection\alignright\texttt{\changed{SEL}}\nopagebreak

The predeclared function procedure \texttt{SEL} accepts a Boolean expression and two compatible expressions of any basic, pointer or procedure type.
If the condition specified by the first argument is satisfied, it evaluates the second argument and otherwise the third.

\end{itemize}

The \ecs{} supports all proper procedures predeclared in the Oberon-2 language report and adds the following extensions:

\begin{itemize}

\item Assertion\alignright\texttt{ASSERT}\nopagebreak

The optional second argument of the predeclared proper procedure \texttt{ASSERT} may be an integer constant between 0 and 255 that represents the exit status of the program.
If the first argument has a constant value, the assertion is static and terminates the compilation rather than the program execution.

\item Variable deallocation\alignright\texttt{\changed{DISPOSE}}\nopagebreak

The predeclared proper procedure \texttt{DISPOSE} accepts a pointer variable that stores a pointer to a variable previously allocated with \texttt{NEW}.
It deallocates the referenced variable and sets the pointer variable to \texttt{NIL}.

\item Program termination\alignright\texttt{HALT}\nopagebreak

The predeclared proper procedure \texttt{HALT} accepts an integer constant between 0 and 255 that represents the exit status of the program.

\item Value ignoring\alignright\texttt{\changed{IGNORE}}\nopagebreak

The predeclared proper procedure \texttt{IGNORE} accepts an expression of any type and ignores its value.

\item Variable allocation\alignright\texttt{NEW}\nopagebreak

The type of the array lengths accepted by the predeclared proper procedure \texttt{NEW} is \texttt{LENGTH}.
If a variable allocation fails, the corresponding pointer variable assumes the value \texttt{NIL}.

\item Expression evaluation\alignright\texttt{\changed{TRACE}}\nopagebreak

The predeclared proper procedure \texttt{TRACE} allows printing the value of an arbitrary expression of any basic, pointer or procedure type for debugging purposes.

\end{itemize}

\obsection{Modules}{11}

\begin{itemize}

\item Generic modules\obref{11}

Modules with a list of identifiers following their name are called \emph{generic} and can be parameterized using the same number of constant expressions when they are imported:

\begin{quote}\begin{grammar}
<Module> = "MODULE" <ident> \changed{$[$"("\synt{IdentList}")"$]$} $[$"IN" <Qualident>$]$";" \\ $\{$<ImportList>$\}$ <DeclarationSequence> \\ $[$"BEGIN" <StatementSequence>$]$ "END" <ident>"." \par
<Import> = $[$<ident> ":="$]$ <ident> \changed{$[$"("\synt{ExpressionList}")"$]$} $[$"IN" <Qualident>$]$\par
\end{grammar}\end{quote}

During an import of a generic module, each identifier is declared as an optionally exported constant or type using the constant value or type identifier of the corresponding parameter.

\item Module packages\obref{11}

Modules can be associated with a \emph{package} which provides a separate scope for all modules contained therein.
Module packages are named using the optional \texttt{IN} keyword:

\begin{quote}\begin{grammar}
<Module> = "MODULE" <ident> $[$"("<IdentList>")"$]$ \changed{$[$"IN" \synt{Qualident}$]$}";" \\ $\{$<ImportList>$\}$ <DeclarationSequence> \\ $[$"BEGIN" <StatementSequence>$]$ "END" <ident>"." \par
\end{grammar}\end{quote}

The package of imported modules can be specified either in front of the import list or following their name:

\begin{quote}\begin{grammar}
<ImportList> = \changed{$[$"IN" \synt{Qualident}$]$} "IMPORT" <Import> $\{$"," <Import>$\}$";" \par
<Import> = $[$<ident> ":="$]$ <ident> $[$"("<ExpressionList>")"$]$ \changed{$[$"IN" \synt{Qualident}$]$} \par
\end{grammar}\end{quote}

\item Multiple import lists\obref{11}

A module may contain multiple import lists:

\begin{quote}\begin{grammar}
<Module> = "MODULE" <ident> $[$"("<IdentList>")"$]$ $[$"IN" <Qualident>$]$";" \\ \changed{$\{$}<ImportList>\changed{$\}$} <DeclarationSequence> \\ $[$"BEGIN" <StatementSequence>$]$ "END" <ident>"." \par
\end{grammar}\end{quote}

\item Text following a module\obref{11}

The remaining text of the source file after the concluding dot of a module is completely ignored.

\item Module loading\obref{11}

A program can consist of an arbitrary number of modules which are loaded automatically.
At least one of them however must have a module body which serves as the entry point of the program if not provided by other compilers or assemblers.

\end{itemize}

\obsection{The Module SYSTEM}{C}\label{sec:obsystemmodule}

The \ecs{} omits the procedure \texttt{SYSTEM.\changed{CC}} and adds the types \texttt{SYSTEM.\changed{AD\-DRESS}} and \texttt{SYSTEM.\changed{SET}}.
These are unsigned integer and set types that have the same size as \texttt{LENGTH}, see Section~\ref{sec:obtypedeclarations}.
All other function procedures predeclared in the Oberon-2 language report are supported with the following extensions:

\begin{itemize}

\item Memory address\alignright\texttt{SYSTEM.ADR}\nopagebreak

The result type of the predeclared function procedure \texttt{SYSTEM.ADR} is \texttt{SYSTEM.AD\-DRESS}.

\item Memory bit\alignright\texttt{SYSTEM.BIT}\nopagebreak

The types of the arguments of the predeclared function procedure \texttt{SYSTEM.BIT} are \texttt{SYSTEM.AD\-DRESS} and \texttt{LENGTH} respectively.

\item Type interpretation\alignright\texttt{SYSTEM.VAL}\nopagebreak

If the second argument of the predeclared function procedure \texttt{SYSTEM.VAL} refers to a variable, the result is that variable interpreted as if it was declared with the specified type.
Otherwise, the result is the value of the second argument converted to the specified basic, pointer, or procedure type.

\end{itemize}

The \ecs{} omits the procedures \texttt{SYSTEM.\changed{GETREG}} and \texttt{SYSTEM.\changed{PUTREG}}.
All other proper procedures predeclared in the Oberon-2 language report are supported with the following extensions:

\begin{itemize}

\item Inline assembly code\alignright\texttt{SYSTEM.\changed{ASM}}\nopagebreak

The predeclared proper procedure \texttt{SYSTEM.ASM} allows writing inline assembly code using one of the various compilers for the Oberon programming language.
It accepts a string storing the actual assembly code and passes it to the assembler used to generate the machine code.
The available instruction set therefore depends on the actual compiler used to compile the module.
\seeassembly

The names and values of all accessible constants with boolean, character, integer, or set type are predefined in the assembly code.
Within a procedure, the names of its local variables and parameters are also predefined.
The values of these definitions correspond to the offset of the variable or parameter relative to the frame pointer.
For debugging purposes, all names predefined in inline assembly code and their actual values are accessible using the expression evaluation directive.

\item Inline intermediate code\alignright\texttt{SYSTEM.\changed{CODE}}\nopagebreak

The predeclared proper procedure \texttt{SYSTEM.CODE} allows writing inline intermediate code using one of the various compilers for the Oberon programming language.
It accepts a string storing the actual intermediate code and predefines the same names as inline assembly code.
\seecode

\item Memory deallocation\alignright\texttt{SYSTEM.\changed{DISPOSE}}\nopagebreak

The predeclared proper procedure \texttt{SYSTEM.DISPOSE} accepts any pointer variable that stores a pointer to memory previously allocated with \texttt{SYSTEM.NEW}.
It deallocates the memory and sets the pointer variable to \texttt{NIL}.

\item Memory read\alignright\texttt{SYSTEM.GET}\nopagebreak

The type of first argument of the predeclared proper procedure \texttt{SYSTEM.GET} is \texttt{SYSTEM.AD\-DRESS}.

\item Memory copy\alignright\texttt{SYSTEM.MOVE}\nopagebreak

The type of the first two arguments of the predeclared function procedure \texttt{SYSTEM.MOVE} is \texttt{SYSTEM.AD\-DRESS} whereas the type of the third argument is \texttt{LENGTH}.

\item Memory allocation\alignright\texttt{SYSTEM.NEW}\nopagebreak

The type of the second argument of the predeclared proper procedure \texttt{SYSTEM.NEW} is \texttt{LENGTH}.

\item Memory write\alignright\texttt{SYSTEM.PUT}\nopagebreak

The type of the first argument of the predeclared proper procedure \texttt{SYSTEM.PUT} is \texttt{SYSTEM.AD\-DRESS}.

\end{itemize}

\obsection{The Oberon Environment}{D}

\begin{itemize}

\item Commands\obref{D1}

The Oberon environment provided by the \ecs{} does not support a shell that allows activating commands.
It does however make the command-line arguments passed to the program itself available, see Section~\ref{sec:OBL:Arguments}.

\item Dynamic loading of modules\obref{D2}

The \ecs{} does not support dynamic loading of modules.

\item Garbage collection\obref{D3}

The \ecs{} does not support garbage collection but provides a predeclared proper procedure called \texttt{DISPOSE} which allows deallocating variables manually, see Section~\ref{sec:obpredeclaredprocedures}.

\item Browser\obref{D4}

The Oberon environment does not provide a separate browser tool for extracting the human-readable interface of a compiled module since it is already available in its symbol file, see Section~\ref{sec:obtools}.
Instead, the \ecs{} provides tools that can generate cross-referenced documents from the interfaces of modules and their user-defined annotations, see Section~\ref{sec:obgeneration}.

\item Run Time Data Structures\obref{D5}

The are at most eight different extension levels for record types.
All other quantities like the number of modules, imports, declarations, fields, parameters and statements, or the length of identifiers, strings, and arrays have no intrinsic limit and are only restricted by available memory.
The actual limit therefore depends only on the execution environment of the tool used to process the Oberon module.

\item Trap Numbers

Program execution is aborted when assertions like type guards are not satisified.
The runtime environment may indicate the reason for program terminations using trap numbers whose meaning is listed in Table~\ref{tab:obtraps}.

\begin{table}
\centering
\begin{tabular}{@{}lll@{}}
\toprule Trap & Meaning & Context \\
\midrule 0 & Failed assertion & Assert procedures \obref{10.3} \\
1 & Unmatched case label & Case statements \obref{9.5} \\
2 & Invalid array element index & Array designators \obref{8.1} \\
3 & Failed type guard & Type guards \obref{8.1} \\
4 & Unsatisfied type test & With statements \obref{9.11} \\
\bottomrule
\end{tabular}
\caption{Oberon trap numbers}
\label{tab:obtraps}
\end{table}

\end{itemize}

\section{The Oberon Library}

The \ecs{} provides a collection of modules that comprise the \emph{Oberon Library}\index{Oberon Library}\index{Libraries!Oberon Library}.
This library provides some useful utilities for programmers and is available by importing one or more of the modules listed in Table~\ref{tab:oboberonlibrary} from package \texttt{OBL}.
All of these modules are governed by the \rse{} which is an additional permission to the \gpl{} that allows users of the \ecs{} to create proprietary programs.
\ifbook Copies of these licenses are included in Appendices~\ref{gpl} and~\ref{rse} on pages~\pageref{gpl} and~\pageref{rse} respectively. \fi
The remainder of this section lists all modules provided by the Oberon Library in alphabetical order and describes their exported interface.

\newcommand{\obmoduleref}[2]{& \texttt{#1} & \ref{sec:OBL:#1} & \ifx#2\empty\else\texttt{#2} & \ref{sec:OBL:#2}\fi \\}

\begin{table}
\centering
\begin{tabular}{@{}llrlr@{}}
\toprule Category & \multicolumn{2}{l}{Module \alignright Section} & \multicolumn{2}{l@{}}{Module \alignright Section} \\
\midrule Basic Types
\obmoduleref{Booleans}{Characters}
\midrule Containers
\obmoduleref{DynamicArrays}{HashMaps}
\obmoduleref{Iterators}{Lists}
\midrule Coroutines
\obmoduleref{Coroutines}{Functions}
\obmoduleref{Generators}{}
\midrule Input/
\obmoduleref{In}{Out}
Output
\obmoduleref{Streams}{}
\midrule Numerics
\obmoduleref{Math}{Random}
\midrule Utilities
\obmoduleref{Arguments}{Arrays}
\obmoduleref{BasicTypes}{Exceptions}
\obmoduleref{Hashes}{Pairs}
\obmoduleref{Sets}{Strings}
\bottomrule
\end{tabular}
\caption{Modules of the Oberon Library}
\label{tab:oboberonlibrary}
\end{table}

\input{oblibrary.doc}

\section{The Oakwood Library}

For compatibility with other implementations, the \ecs{} provides additional library modules recommended by the Oakwood guidelines for Oberon-2 compiler developers~\cite{oakwood1995}.
This library is called the \emph{Oakwood Library}\index{Oakwood Library}\index{Libraries!Oakwood Library} and is available by importing one or more of the modules listed in Table~\ref{tab:oboakwoodlibrary}.
All of these modules are governed by the \rse{} which is an additional permission to the \gpl{} that allows users of the \ecs{} to create proprietary programs.
\ifbook Copies of these licenses are included in Appendices~\ref{gpl} and~\ref{rse} on pages~\pageref{gpl} and~\pageref{rse} respectively. \fi
The remainder of this section lists all supported modules of the Oakwood Library in alphabetical order and describes their exported interface.

\renewcommand{\obmoduleref}[1]{& \texttt{#1} & \ref{sec:#1} \\}

\begin{table}
\centering
\begin{tabular}{@{}llr@{}}
\toprule Category & \multicolumn{2}{l@{}}{Module \alignright Section} \\
\midrule Basic Modules
\obmoduleref{Strings}
\obmoduleref{Math}
\obmoduleref{MathL}
\obmoduleref{XYplane}
\midrule Additional Modules
\obmoduleref{Coroutines}
\bottomrule
\end{tabular}
\caption{Modules of the Oakwood Library}
\label{tab:oboakwoodlibrary}
\end{table}

\input{oaklibrary.doc}

\section{Documentation Generation}\label{sec:obgeneration}

The \ecs{} provides several tools that are able to extract the structure of modules written in Oberon and generate documentations for them.
This section describes the contents of the extracted information and explains how programmers can provide a user-defined description of it.
An example of a completely annotated module is given in Figure~\ref{fig:obdocexample} at the end of this section.

\subsection{Annotations}

Programmers can annotate their program using a special notation for comments.
Comments beginning with two asterisks are still ignored during parsing but recognized and merged into a single \emph{annotation} for the immediately following syntax element.
Annotations can be formatted using a lightweight markup language which is an extension of the Creole markup language used for formatting wikis~\cite{sauer2007}.
Table~\ref{tab:docmarkup} \ifbook on page~\pageref{tab:docmarkup} \fi summarizes all elements of this markup language and shows how they are formatted.
\seedocumentation

\ifbook\else\markuptable\fi

In addition to annotations for syntax elements like modules and procedures, the very last annotation after the end of a module is also recognized.
The documentation for a module begins with the user-defined content consisting of paragraphs and articles defined in this annotation.
Afterward there is an article for the module and each of its exported declarations.
Since all these articles are predefined, the article markup has a different meaning therein and allows tagging an element of the article with further information.
These so-called \emph{tags} begin with one at sign for general-purpose tags and with two at signs for descriptions of nested syntax elements.

The following sections define how these articles look like and what tags are available therein.
The tag \texttt{@remarks} is always available and allows giving more detailed information at the end of an article.

\subsection{Modules}

A module can be annotated in front of the \texttt{MODULE} keyword.
The generated article is labeled with the name and optional package of the module and contains its description as well as a summary of all exported declarations.
The summary consists of a table that groups these declarations by topic which can be defined using the \texttt{@topic} tag.
It defaults to the type of the declaration if no tag is given.
If the module is generic, the article additionally provides a fully qualified label for each exported parameter and a nested tag for its description:

\begin{quote}\begin{verbatim}
(** description for a module *)
(** @@Value its first parameter *)
(** @remarks further information about the module *)
MODULE Module (Value*);
END Module.
(** the documentation starts with this description *)
(** @ and may contain user-defined articles *)
(** as well as links to the [[Module]] *)
\end{verbatim}\end{quote}

\subsection{Constant Declarations}

A constant can be annotated in front of its identifier or the preceding \texttt{CONST} keyword.
The generated article is labeled with the fully qualified name of the constant and contains its description as well as its syntax definition:

\begin{quote}\begin{verbatim}
(** description of a constant *)
(** @topic grouping of the constant *)
(** @remarks further information about the constant *)
CONST Pi* = 3.1415;
\end{verbatim}\end{quote}

\subsection{Type Declarations}

A type can be annotated in front of its identifier or the preceding \texttt{TYPE} keyword.
The generated article is labeled with the fully qualified name of the type and contains its description as well as its syntax definition.
For record type declarations it additionally contains a summary of the record interface similar to the summary of the module.
Fields can be annotated like variables:

\begin{quote}\begin{verbatim}
(** description of a record *)
TYPE Record* = RECORD
  (** description of the first field *)
  field-: Number;
END;
\end{verbatim}\end{quote}

\subsection{Variable Declarations}

A variable can be annotated in front of its identifier or the preceding \texttt{VAR} keyword.
The generated article is labeled with the fully qualified name of the variable and contains its description as well as its syntax definition:

\begin{quote}\begin{verbatim}
(** description of a variable *)
VAR features*: SET;
\end{verbatim}\end{quote}

\subsection{Procedure Declarations}

A procedure can be annotated in front of the \texttt{PROCEDURE} keyword.
The generated article is labeled with the fully qualified name of the procedure and contains its description as well as its syntax definition.
It additionally provides a fully qualified label and a nested tag for the description of each parameter and the optional receiver.
Function procedures have an additonal tag called \texttt{@result} that allows describing the result value:

\begin{quote}\begin{verbatim}
(** description for a procedure *)
(** @@number its first parameter *)
(** @result and its result value *)
PROCEDURE Increment* (VAR number: Number): Number;
BEGIN INC (number, Value); RETURN number;
END Increment;
\end{verbatim}\end{quote}

\begin{figure}
\ttfamily\centering
\begin{minipage}{26em}\begin{verbatim}
(** description for a module *)
(** @@Value its first parameter *)
(** @remarks further information about the module *)
MODULE Module (Value*);

(** description of a constant *)
(** @topic grouping of the constant *)
(** @remarks further information about the constant *)
CONST Pi* = 3.1415;

TYPE
  (** description of a type declaration *)
  Number* = INTEGER;

  (** description of a record *)
  Record* = RECORD
    (** description of the first field *)
    field-: Number;
  END;

(** description of a variable *)
VAR features*: SET;

(** description for a procedure *)
(** @@number its first parameter *)
(** @result and its result value *)
PROCEDURE Increment* (VAR number: Number): Number;
BEGIN INC (number, Value); RETURN number;
END Increment;

END Module.
(** the documentation starts with this description *)
(** @ and may contain user-defined articles *)
(** as well as links to the [[Module]] *)
\end{verbatim}\end{minipage}
\normalfont\caption{Example of a completely annotated Oberon module}
\label{fig:obdocexample}
\end{figure}

\section{Runtime Support}\label{sec:obruntimesupport}\index{Runtime support!for Oberon}

Some language features of Oberon such as predeclared procedures require some additional runtime support.
This runtime support is stored in library files which are collections of object files. \seeobject
The \ecs{} provides the required runtime support in one library file for each hardware architecture it supports.
The name of the corresponding library file consists of a leading \file{ob}, the name of the target hardware architecture, and a trailing \file{run} as in \file{ob\-arma64\-run}.

The procedures of the module \texttt{Math} described in Section~\ref{sec:OBL:Math} are wrappers for the corresponding functions of the Standard \cpp{} Library.
Programs using these procedures therefore require additional runtime support for \cpp{}. \seecpp

\section{Oberon Tools}\label{sec:obtools}

The \ecs{} provides several different tools that process modules written in Oberon.
\interface\seeguide

The tools process Oberon modules in several consecutive stages.
In each stage, the internal representation of the module is changed and transformed.
Figure~\ref{fig:obdataflow} shows all stages and the different representations.
Additionally, each tool except for the pretty printer and the interpreter generates a so-called \emph{symbol file} for each source file being processed.
Symbol files contain information about the interface of a module and are needed whenever a module attempts to import another one.
When importing a module, its symbol file is first searched in the current working directory and then in the relative directory given by the \environmentvariable{ECSIMPORT} environment variable which must include a trailing path separator.

\begin{figure}
\flowgraph{
& \resource{Oberon\\source code} \ar[d] \\
& \converter{Lexer} \ar[d] \\
& \resource{tokens} \ar[d] \\
& \converter{Parser} \ar[d] \\
\variable{ECSIMPORT} \ar[rd] & \resource{abstract\\syntax tree} \ar[d] \ar[r] & \converter{Pretty Printer} \ar[d] \\
\resource{symbol\\files} \ar@/u/[r] & \converter{Semantic\\Checker} \ar@/d/[l] \ar[d] & \resource{reformatted\\source code} \\
\converter{Interpreter} \ar@/l/[d] & \resource{attributed\\syntax tree} \ar[l] \ar[d] \ar[r] & \converter{Transpiler} \ar[d] \\
\resource{input/\\output} \ar@/r/[u] & \converter{Intermediate\\Code Emitter} \ar[d] & \resource{translated\\source code} \\
& \resource{intermediate\\code} \ar[d] \ar@/u/[r] & \converter{Optimizer} \ar@/d/[l] \\
\resource{assembly\\listing} & \converter{Machine Code\\Generator} \ar[l] \ar[d] \ar[r] & \resource{debugging\\information} \\
& \resource{object file} \\
}\caption{Data flow within the tools for Oberon}
\label{fig:obdataflow}
\end{figure}

\obprint
\obcheck
\obdump
\obrun
\obcpp
\obdoc
\obhtml
\oblatex
\obcode
\obamda
\obamdb
\obamdc
\obarma
\obarmb
\obarmc
\obarmcfpe
\obavr
\obavrtt
\obmabk
\obmibl
\obmipsa
\obmipsb
\obmmix
\oborok
\obppca
\obppcb
\obrisc
\obwasm

\section{Interoperability}

In accordance with the goal of the \ecs{} to enable interoperability between its implemented programming languages,
the compilers for Oberon provide different mechanisms to exchange data with programs written with other tools of the \ecs{}.
The interoperability is enabled by a common intermediate code representation and calling convention. \seecode

This section describes the naming conventions used to uniquely identify the intermediate code sections defined by the compilers for Oberon
as well as the ways of accessing sections defined by other compilers and assemblers.

\subsection{Naming Conventions}

The compilers for the Oberon programming language define a code section for each body of a module and each procedure defined therein.
Additional data sections are defined for global variables as well as record descriptors which contain the type information of records needed at runtime.
The name of each section is the identifier of the actual declaration prefixed by the name of its containing scope and an attached period such that names resemble qualified identifiers.
The name of module packages are followed by colons rather than periods.

\subsection{Accessing Sections}\label{sec:obaccessingsections}

Oberon as implemented by the \ecs{} allows two different ways of accessing sections defined by other compilers or assemblers.

\begin{itemize}

\item
The predeclared procedure \texttt{SYSTEM.ASM} allows writing inline assembly code which naturally enables arbitrary access to any section, see Section~\ref{sec:obsystemmodule}.
\seeassembly

\item
Forward declarations marked as external allow referring to data and code sections that are defined elsewhere, see Section~\ref{sec:obdeclarations}.

\end{itemize}

\concludechapter

// Generic assembly language definitions
// Copyright (C) Florian Negele

// This file is part of the Eigen Compiler Suite.

// The ECS is free software: you can redistribute it and/or modify
// it under the terms of the GNU General Public License as published by
// the Free Software Foundation, either version 3 of the License, or
// (at your option) any later version.

// The ECS is distributed in the hope that it will be useful,
// but WITHOUT ANY WARRANTY; without even the implied warranty of
// MERCHANTABILITY or FITNESS FOR A PARTICULAR PURPOSE.  See the
// GNU General Public License for more details.

// You should have received a copy of the GNU General Public License
// along with the ECS.  If not, see <https://www.gnu.org/licenses/>.

#ifndef SYMBOL
	#define SYMBOL(symbol, name)
#endif

// preprocessing directives

SYMBOL (Define,  "#define")
SYMBOL (Elif,    "#elif")
SYMBOL (Else,    "#else")
SYMBOL (End,     "#end")
SYMBOL (Enddef,  "#enddef")
SYMBOL (Endif,   "#endif")
SYMBOL (Endrep,  "#endrep")
SYMBOL (If,      "#if")
SYMBOL (Line,    "#line")
SYMBOL (Repeat,  "#repeat")
SYMBOL (Undef,   "#undef")

// directives

SYMBOL (Alias,        ".alias")
SYMBOL (Align,        ".align")
SYMBOL (Alignment,    ".alignment")
SYMBOL (Assembly,     ".assembly")
SYMBOL (Assert,       ".assert")
SYMBOL (Big,          ".big")
SYMBOL (BitMode,      ".bitmode")
SYMBOL (Byte,         ".byte")
SYMBOL (Code,         ".code")
SYMBOL (Const,        ".const")
SYMBOL (Data,         ".data")
SYMBOL (DByte,        ".dbyte")
SYMBOL (Duplicable,   ".duplicable")
SYMBOL (Embed,        ".embed")
SYMBOL (Equals,       ".equals")
SYMBOL (Group,        ".group")
SYMBOL (Header,       ".header")
SYMBOL (InitCode,     ".initcode")
SYMBOL (InitData,     ".initdata")
SYMBOL (Little,       ".little")
SYMBOL (OByte,        ".obyte")
SYMBOL (Origin,       ".origin")
SYMBOL (Pad,          ".pad")
SYMBOL (QByte,        ".qbyte")
SYMBOL (Replaceable,  ".replaceable")
SYMBOL (Require,      ".require")
SYMBOL (Required,     ".required")
SYMBOL (Reserve,      ".reserve")
SYMBOL (TByte,        ".tbyte")
SYMBOL (Trace,        ".trace")
SYMBOL (Trailer,      ".trailer")
SYMBOL (Type,         ".type")

// functions

SYMBOL (Count,     "count")
SYMBOL (Extent,    "extent")
SYMBOL (Index,     "index")
SYMBOL (Offset,    "offset")
SYMBOL (Position,  "position")
SYMBOL (Size,      "size")

// operators

SYMBOL (Plus,          "+")
SYMBOL (Minus,         "-")
SYMBOL (Times,         "*")
SYMBOL (Slash,         "/")
SYMBOL (Modulo,        "%")
SYMBOL (LeftShift,     "<<")
SYMBOL (RightShift,    ">>")
SYMBOL (Less,          "<")
SYMBOL (LessEqual,     "<=")
SYMBOL (Greater,       ">")
SYMBOL (GreaterEqual,  ">=")
SYMBOL (Equal,         "==")
SYMBOL (Unequal,       "!=")
SYMBOL (Identical,     "===")
SYMBOL (Unidentical,   "!==")
SYMBOL (BitwiseNot,    "~")
SYMBOL (BitwiseOr,     "|")
SYMBOL (BitwiseAnd,    "&")
SYMBOL (BitwiseXor,    "^")
SYMBOL (LogicalNot,    "!")
SYMBOL (LogicalOr,     "||")
SYMBOL (LogicalAnd,    "&&")

// separators

SYMBOL (Comma,    ",")
SYMBOL (Colon,    ":")
SYMBOL (Newline,  "new-line character")

// parentheses

SYMBOL (LeftParen,     "(")
SYMBOL (RightParen,    ")")
SYMBOL (LeftBracket,   "[")
SYMBOL (RightBracket,  "]")
SYMBOL (LeftBrace,     "{")
SYMBOL (RightBrace,    "}")

// literals

SYMBOL (String,      "string")
SYMBOL (Address,     "address")
SYMBOL (Character,   "character")
SYMBOL (Identifier,  "identifier")
SYMBOL (Integer,     "integer")
SYMBOL (BinInteger,  "binary integer")
SYMBOL (OctInteger,  "octal integer")
SYMBOL (DecInteger,  "decimal integer")
SYMBOL (HexInteger,  "hexadecimal integer")
SYMBOL (Real,        "real")

#undef SYMBOL


\part{Supported Hardware Architectures}
// AMD64 instruction set definitions
// Copyright (C) Florian Negele

// This file is part of the Eigen Compiler Suite.

// The ECS is free software: you can redistribute it and/or modify
// it under the terms of the GNU General Public License as published by
// the Free Software Foundation, either version 3 of the License, or
// (at your option) any later version.

// The ECS is distributed in the hope that it will be useful,
// but WITHOUT ANY WARRANTY; without even the implied warranty of
// MERCHANTABILITY or FITNESS FOR A PARTICULAR PURPOSE.  See the
// GNU General Public License for more details.

// You should have received a copy of the GNU General Public License
// along with the ECS.  If not, see <https://www.gnu.org/licenses/>.

#ifndef CODE
	#define CODE(code)
#endif

#ifndef FLAG
	#define FLAG(flag, value)
#endif

#ifndef INSTR
	#define INSTR(mnem, type1, type2, type3, type4, exprefix, opcode, code1, code2, suffix, flags)
#endif

#ifndef MNEM
	#define MNEM(name, mnem, ...)
#endif

#ifndef PREFIX
	#define PREFIX(prefix, byte)
#endif

#ifndef REG
	#define REG(reg, name)
#endif

#ifndef TYPE
	#define TYPE(type)
#endif

// mnemonics

MNEM (aaa,               AAA,               ASCII Adjust After Addition)
MNEM (aad,               AAD,               ASCII Adjust Before Division)
MNEM (aam,               AAM,               ASCII Adjust After Multiply)
MNEM (aas,               AAS,               ASCII Adjust After Subtraction)
MNEM (adc,               ADC,               Add with Carry)
MNEM (adcx,              ADCX,              Unsigned Add with Carry Flag)
MNEM (add,               ADD,               Signed or Unsigned Add)
MNEM (addpd,             ADDPD,             Add Packed Double-Precision Floating-Point)
MNEM (addps,             ADDPS,             Add Packed Single-Precision Floating-Point)
MNEM (addsd,             ADDSD,             Add Scalar Double-Precision Floating-Point)
MNEM (addss,             ADDSS,             Add Scalar Single-Precision Floating-Point)
MNEM (addsubpd,          ADDSUBPD,          Add and Subtract Packed Double-Precision)
MNEM (addsubps,          ADDSUBPS,          Add and Subtract Packed Single-Precision)
MNEM (adox,              ADOX,              Unsigned Add with Overflow Flag)
MNEM (aesdec,            AESDEC,            AES Decryption Round)
MNEM (aesdeclast,        AESDECLAST,        AES Last Decryption Round)
MNEM (aesenc,            AESENC,            AES Encryption Round)
MNEM (aesenclast,        AESENCLAST,        AES Last Encryption Round)
MNEM (aesimc,            AESIMC,            AES InvMixColumn Transformation)
MNEM (aeskeygenassist,   AESKEYGENASSIST,   AES Assist Round Key Generation)
MNEM (and,               AND,               Logical AND)
MNEM (andn,              ANDN,              Logical And-Not)
MNEM (andnpd,            ANDNPD,            Logical Bitwise AND NOT Packed Double-Precision Floating-Point)
MNEM (andnps,            ANDNPS,            Logical Bitwise AND NOT Packed Single-Precision Floating-Point)
MNEM (andpd,             ANDPD,             Logical Bitwise AND Packed Double-Precision Floating-Point)
MNEM (andps,             ANDPS,             Logical Bitwise AND Packed Single-Precision Floating-Point)
MNEM (arpl,              ARPL,              Adjust Requestor Privilege Level)
MNEM (bextr,             BEXTR,             Bit Field Extract)
MNEM (blcfill,           BLCFILL,           Fill From Lowest Clear Bit)
MNEM (blci,              BLCI,              Isolate Lowest Clear Bit)
MNEM (blcic,             BLCIC,             Isolate Lowest Clear Bit and Complement)
MNEM (blcmsk,            BLCMSK,            Mask From Lowest Clear Bit)
MNEM (blcs,              BLCS,              Set Lowest Clear Bit)
MNEM (blendpd,           BLENDPD,           Blend Packed Double-Precision Floating-Point)
MNEM (blendps,           BLENDPS,           Blend Packed Single-Precision Floating-Point)
MNEM (blendvpd,          BLENDVPD,          Variable Blend Packed Double-Precision Floating-Point)
MNEM (blendvps,          BLENDVPS,          Variable Blend Packed Single-Precision Floating-Point)
MNEM (blsfill,           BLSFILL,           Fill From Lowest Set Bit)
MNEM (blsi,              BLSI,              Isolate Lowest Set Bit)
MNEM (blsic,             BLSIC,             Isolate Lowest Set Bit and Complement)
MNEM (blsmsk,            BLSMSK,            Mask From Lowest Set Bit)
MNEM (blsr,              BLSR,              Reset Lowest Set Bit)
MNEM (bound,             BOUND,             Check Array Bound)
MNEM (bsf,               BSF,               Bit Scan Forward)
MNEM (bsr,               BSR,               Bit Scan Reverse)
MNEM (bswap,             BSWAP,             Byte Swap)
MNEM (bt,                BT,                Bit Test)
MNEM (btc,               BTC,               Bit Test and Complement)
MNEM (btr,               BTR,               Bit Test and Reset)
MNEM (bts,               BTS,               Bit Test and Set)
MNEM (bzhi,              BZHI,              Zero High Bits)
MNEM (call,              CALL,              Near Procedure Call)
MNEM (callfar,           CALLFAR,           Far Procedure Call)
MNEM (cbw,               CBW,               Convert AL to Sign-Extended AX)
MNEM (cdq,               CDQ,               Convert EAX to Sign-Extended EDX:EAX)
MNEM (cdqe,              CDQE,              Convert EAX to Sign-Extended RAX)
MNEM (clac,              CLAC,              Clear Alignment Check Flag)
MNEM (clc,               CLC,               Clear Carry Flag)
MNEM (cld,               CLD,               Clear Direction Flag)
MNEM (clflush,           CLFLUSH,           Cache Line Flush)
MNEM (clflushopt,        CLFLUSHOPT,        Optimized Cache Line Flush)
MNEM (clgi,              CLGI,              Clear Global Interrupt Flag)
MNEM (cli,               CLI,               Clear Interrupt Flag)
MNEM (clrssbsy,          CLRSSBSY,          Clear Shadow Stack Busy)
MNEM (clts,              CLTS,              Clear Task-Switched Flag in CR0)
MNEM (clwb,              CLWB,              Cache Line Write Back and Retain)
MNEM (clzero,            CLZERO,            Zero Cache Line)
MNEM (cmc,               CMC,               Complement Carry Flag)
MNEM (cmova,             CMOVA,             Move if above (CF = 0 and ZF = 0))
MNEM (cmovae,            CMOVAE,            Move if above or equal (CF = 0))
MNEM (cmovb,             CMOVB,             Move if below (CF = 1))
MNEM (cmovbe,            CMOVBE,            Move if below or equal (CF = 1 or ZF = 1))
MNEM (cmovc,             CMOVC,             Move if carry (CF = 1))
MNEM (cmove,             CMOVE,             Move if equal (ZF =1))
MNEM (cmovg,             CMOVG,             Move if greater (ZF = 0 and SF = OF))
MNEM (cmovge,            CMOVGE,            Move if greater or equal (SF = OF))
MNEM (cmovl,             CMOVL,             Move if less (SF <> OF))
MNEM (cmovle,            CMOVLE,            Move if less or equal (ZF = 1 or SF <> OF))
MNEM (cmovna,            CMOVNA,            Move if not above (CF = 1 or ZF = 1))
MNEM (cmovnae,           CMOVNAE,           Move if not above or equal (CF = 1))
MNEM (cmovnb,            CMOVNB,            Move if not below (CF = 0))
MNEM (cmovnbe,           CMOVNBE,           Move if not below or equal (CF = 0 and ZF = 0))
MNEM (cmovnc,            CMOVNC,            Move if not carry (CF = 0))
MNEM (cmovne,            CMOVNE,            Move if not equal (ZF = 0))
MNEM (cmovng,            CMOVNG,            Move if not greater (ZF = 1 or SF <> OF))
MNEM (cmovnge,           CMOVNGE,           Move if not greater or equal (SF <> OF))
MNEM (cmovnl,            CMOVNL,            Move if not less (SF = OF))
MNEM (cmovnle,           CMOVNLE,           Move if not less or equal (ZF = 0 and SF = OF))
MNEM (cmovno,            CMOVNO,            Move if not overflow (OF = 0))
MNEM (cmovnp,            CMOVNP,            Move if not parity (PF = 0))
MNEM (cmovns,            CMOVNS,            Move if not sign (SF = 0))
MNEM (cmovnz,            CMOVNZ,            Move if not zero (ZF = 0))
MNEM (cmovo,             CMOVO,             Move if overflow (OF = 1))
MNEM (cmovp,             CMOVP,             Move if parity (PF = 1))
MNEM (cmovpe,            CMOVPE,            Move if parity even (PF = 1))
MNEM (cmovpo,            CMOVPO,            Move if parity odd (PF = 0))
MNEM (cmovs,             CMOVS,             Move if sign (SF =1))
MNEM (cmovz,             CMOVZ,             Move if zero (ZF = 1))
MNEM (cmp,               CMP,               Compare)
MNEM (cmpeqpd,           CMPEQPD,           Compare Packed Double-Precision Floating-Point Equal)
MNEM (cmpeqps,           CMPEQPS,           Compare Packed Single-Precision Floating-Point Equal)
MNEM (cmpeqsd,           CMPEQSD,           Compare Scalar Double-Precision Floating-Point Equal)
MNEM (cmpeqss,           CMPEQSS,           Compare Scalar Single-Precision Floating-Point Equal)
MNEM (cmplepd,           CMPLEPD,           Compare Packed Double-Precision Floating-Point Less or Equal)
MNEM (cmpleps,           CMPLEPS,           Compare Packed Single-Precision Floating-Point Less or Equal)
MNEM (cmplesd,           CMPLESD,           Compare Scalar Double-Precision Floating-Point Less or Equal)
MNEM (cmpless,           CMPLESS,           Compare Scalar Single-Precision Floating-Point Less or Equal)
MNEM (cmpltpd,           CMPLTPD,           Compare Packed Double-Precision Floating-Point Less Than)
MNEM (cmpltps,           CMPLTPS,           Compare Packed Single-Precision Floating-Point Less Than)
MNEM (cmpltsd,           CMPLTSD,           Compare Scalar Double-Precision Floating-Point Less Than)
MNEM (cmpltss,           CMPLTSS,           Compare Scalar Single-Precision Floating-Point Less Than)
MNEM (cmpneqpd,          CMPNEQPD,          Compare Packed Double-Precision Floating-Point Not Equal)
MNEM (cmpneqps,          CMPNEQPS,          Compare Packed Single-Precision Floating-Point Not Equal)
MNEM (cmpneqsd,          CMPNEQSD,          Compare Scalar Double-Precision Floating-Point Not Equal)
MNEM (cmpneqss,          CMPNEQSS,          Compare Scalar Single-Precision Floating-Point Not Equal)
MNEM (cmpnlepd,          CMPNLEPD,          Compare Packed Double-Precision Floating-Point Not Less or Equal)
MNEM (cmpnleps,          CMPNLEPS,          Compare Packed Single-Precision Floating-Point Not Less or Equal)
MNEM (cmpnlesd,          CMPNLESD,          Compare Scalar Double-Precision Floating-Point Not Less or Equal)
MNEM (cmpnless,          CMPNLESS,          Compare Scalar Single-Precision Floating-Point Not Less or Equal)
MNEM (cmpnltpd,          CMPNLTPD,          Compare Packed Double-Precision Floating-Point Not Less Than)
MNEM (cmpnltps,          CMPNLTPS,          Compare Packed Single-Precision Floating-Point Not Less Than)
MNEM (cmpnltsd,          CMPNLTSD,          Compare Scalar Double-Precision Floating-Point Not Less Than)
MNEM (cmpnltss,          CMPNLTSS,          Compare Scalar Single-Precision Floating-Point Not Less Than)
MNEM (cmpordpd,          CMPORDPD,          Compare Packed Double-Precision Floating-Point Ordered)
MNEM (cmpordps,          CMPORDPS,          Compare Packed Single-Precision Floating-Point Ordered)
MNEM (cmpordsd,          CMPORDSD,          Compare Scalar Double-Precision Floating-Point Ordered)
MNEM (cmpordss,          CMPORDSS,          Compare Scalar Single-Precision Floating-Point Ordered)
MNEM (cmppd,             CMPPD,             Compare Packed Double-Precision Floating-Point)
MNEM (cmpps,             CMPPS,             Compare Packed Single-Precision Floating-Point)
MNEM (cmpsb,             CMPSB,             Compare Bytes)
MNEM (cmpsd,             CMPSD,             Compare Doublewords)
MNEM (cmpsq,             CMPSQ,             Compare Quadwords)
MNEM (cmpss,             CMPSS,             Compare Scalar Single-Precision Floating-Point)
MNEM (cmpsw,             CMPSW,             Compare Words)
MNEM (cmpunordpd,        CMPUNORDPD,        Compare Packed Double-Precision Floating-Point Unordered)
MNEM (cmpunordps,        CMPUNORDPS,        Compare Packed Single-Precision Floating-Point Unordered)
MNEM (cmpunordsd,        CMPUNORDSD,        Compare Scalar Double-Precision Floating-Point Unordered)
MNEM (cmpunordss,        CMPUNORDSS,        Compare Scalar Single-Precision Floating-Point Unordered)
MNEM (cmpxchg,           CMPXCHG,           Compare and Exchange)
MNEM (cmpxchg16b,        CMPXCHG16B,        Compare and Exchange Sixteen Bytes)
MNEM (cmpxchg8b,         CMPXCHG8B,         Compare and Exchange Eight Bytes)
MNEM (comisd,            COMISD,            Compare Ordered Scalar Double-Precision Floating-Point)
MNEM (comiss,            COMISS,            Compare Ordered Scalar Single-Precision Floating-Point)
MNEM (cpuid,             CPUID,             Processor Identification)
MNEM (cqo,               CQO,               Convert RAX to Sign-Extended RDX:RAX)
MNEM (crc32,             CRC32,             CRC32 Cyclical Redundancy Check)
MNEM (cvtdq2pd,          CVTDQ2PD,          Convert Packed Doubleword Integers to Packed Double-Precision Floating-Point)
MNEM (cvtdq2ps,          CVTDQ2PS,          Convert Packed Doubleword Integers to Packed Single-Precision Floating-Point)
MNEM (cvtpd2dq,          CVTPD2DQ,          Convert Packed Double-Precision Floating-Point to Packed Doubleword Integers)
MNEM (cvtpd2pi,          CVTPD2PI,          Convert Packed Double-Precision Floating-Point to Packed Doubleword Integers)
MNEM (cvtpd2ps,          CVTPD2PS,          Convert Packed Double-Precision Floating-Point to Packed Single-Precision Floating-Point)
MNEM (cvtpi2pd,          CVTPI2PD,          Convert Packed Doubleword Integers to Packed Double-Precision Floating-Point)
MNEM (cvtpi2ps,          CVTPI2PS,          Convert Packed Doubleword Integers to Packed Single-Precision Floating-Point)
MNEM (cvtps2dq,          CVTPS2DQ,          Convert Packed Single-Precision Floating-Point to Packed Doubleword Integers)
MNEM (cvtps2pd,          CVTPS2PD,          Convert Packed Single-Precision Floating-Point to Packed Double-Precision Floating-Point)
MNEM (cvtps2pi,          CVTPS2PI,          Convert Packed Single-Precision Floating-Point to Packed Doubleword Integers)
MNEM (cvtsd2si,          CVTSD2SI,          Convert Scalar Double-Precision Floating-Point to Signed Doubleword or Quadword Integer)
MNEM (cvtsd2ss,          CVTSD2SS,          Convert Scalar Double-Precision Floating-Point to Scalar Single-Precision Floating-Point)
MNEM (cvtsi2sd,          CVTSI2SD,          Convert Signed Doubleword or Quadword Integer to Scalar Double-Precision Floating-Point)
MNEM (cvtsi2ss,          CVTSI2SS,          Convert Signed Doubleword or Quadword Integer to Scalar Single-Precision Floating-Point)
MNEM (cvtss2sd,          CVTSS2SD,          Convert Scalar Single-Precision Floating-Point to Scalar Double-Precision Floating-Point)
MNEM (cvtss2si,          CVTSS2SI,          Convert Scalar Single-Precision Floating-Point to Signed Doubleword or Quadword Integer)
MNEM (cvttpd2dq,         CVTTPD2DQ,         Convert Packed Double-Precision Floating-Point to Packed Doubleword Integers Truncated)
MNEM (cvttpd2pi,         CVTTPD2PI,         Convert Packed Double-Precision Floating-Point to Packed Doubleword Integers Truncated)
MNEM (cvttps2dq,         CVTTPS2DQ,         Convert Packed Single-Precision Floating-Point to Packed Doubleword Integers Truncated)
MNEM (cvttps2pi,         CVTTPS2PI,         Convert Packed Single-Precision Floating-Point to Packed Doubleword Integers Truncated)
MNEM (cvttsd2si,         CVTTSD2SI,         Convert Scalar Double-Precision Floating-Point to Signed Doubleword of Quadword Integer Truncated)
MNEM (cvttss2si,         CVTTSS2SI,         Convert Scalar Single-Precision Floating-Point to Signed Doubleword or Quadword Integer Truncated)
MNEM (cwd,               CWD,               Convert AX to Sign-Extended DX:AX)
MNEM (cwde,              CWDE,              Convert AX to Sign-Extended EAX)
MNEM (daa,               DAA,               Decimal Adjust after Addition)
MNEM (das,               DAS,               Decimal Adjust after Subtraction)
MNEM (dec,               DEC,               Decrement by 1)
MNEM (div,               DIV,               Unsigned Divide)
MNEM (divpd,             DIVPD,             Divide Packed Double-Precision Floating-Point)
MNEM (divps,             DIVPS,             Divide Packed Single-Precision Floating-Point)
MNEM (divsd,             DIVSD,             Divide Scalar Double-Precision Floating-Point)
MNEM (divss,             DIVSS,             Divide Scalar Single-Precision Floating-Point)
MNEM (dppd,              DPPD,              Dot Product Packed Double-Precision Floating-Point)
MNEM (dpps,              DPPS,              Dot Product Packed Single-Precision Floating-Point)
MNEM (emms,              EMMS,              Exit Multimedia State)
MNEM (enter,             ENTER,             Create Procedure Stack Frame)
MNEM (extractps,         EXTRACTPS,         Extract Packed Single-Precision Floating-Point)
MNEM (extrq,             EXTRQ,             Extract Field From Register)
MNEM (f2xm1,             F2XM1,             Floating-Point Compute 2 to the Power of x-1)
MNEM (fabs,              FABS,              Floating-Point Absolute Value)
MNEM (fadd,              FADD,              Floating-Point Add)
MNEM (faddp,             FADDP,             Floating-Point Add and Pop)
MNEM (fbld,              FBLD,              Floating-Point Load Binary-Coded Decimal)
MNEM (fbstp,             FBSTP,             Floating-Point Store Binary-Coded Decimal and Pop)
MNEM (fchs,              FCHS,              Floating-Point Change Sign)
MNEM (fcmovb,            FCMOVB,            Floating-Point Move if below (CF = 1))
MNEM (fcmovbe,           FCMOVBE,           Floating-Point Move if below or equal (CF = 1 or ZF = 1))
MNEM (fcmove,            FCMOVE,            Floating-Point Move if equal (ZF = 1))
MNEM (fcmovnb,           FCMOVNB,           Floating-Point Move if not below (CF = 0))
MNEM (fcmovnbe,          FCMOVNBE,          Floating-Point Move if not below or equal (CF = 0 and ZF = 0))
MNEM (fcmovne,           FCMOVNE,           Floating-Point Move if not equal (ZF = 0))
MNEM (fcmovnu,           FCMOVNU,           Floating-Point Move if not unordered (PF = 0))
MNEM (fcmovu,            FCMOVU,            Floating-Point Move if unordered (PF = 1))
MNEM (fcom,              FCOM,              Floating-Point Compare)
MNEM (fcomi,             FCOMI,             Floating-Point Compare and Set Flags)
MNEM (fcomip,            FCOMIP,            Floating-Point Compare and Set Flags and Pop)
MNEM (fcomp,             FCOMP,             Floating-Point Compare and Pop)
MNEM (fcompp,            FCOMPP,            Floating-Point Compare and Pop twice)
MNEM (fcos,              FCOS,              Floating-Point Cosine)
MNEM (fdecstp,           FDECSTP,           Floating-Point Decrement Stack-Top Pointer)
MNEM (fdiv,              FDIV,              Floating-Point Divide)
MNEM (fdivp,             FDIVP,             Floating-Point Divide and Pop)
MNEM (fdivr,             FDIVR,             Floating-Point Divide Reverse)
MNEM (fdivrp,            FDIVRP,            Floating-Point Divide Reverse and Pop)
MNEM (femms,             FEMMS,             Fast Exit Multimedia State)
MNEM (ffree,             FFREE,             Floating-Point Free Register)
MNEM (fiadd,             FIADD,             Floating-Point Integer Add)
MNEM (ficom,             FICOM,             Floating-Point Integer Compare)
MNEM (ficomp,            FICOMP,            Floating-Point Integer Compare and Pop)
MNEM (fidiv,             FIDIV,             Floating-Point Integer Divide)
MNEM (fidivr,            FIDIVR,            Floating-Point Integer Divide Reverse)
MNEM (fild,              FILD,              Floating-Point Load Integer)
MNEM (fimul,             FIMUL,             Floating-Point Integer Multiply)
MNEM (fincstp,           FINCSTP,           Floating-Point Increment Stack-Top Pointer)
MNEM (fist,              FIST,              Floating-Point Integer Store)
MNEM (fistp,             FISTP,             Floating-Point Integer Store and Pop)
MNEM (fisttp,            FISTTP,            Floating-Point Integer Truncate and Store)
MNEM (fisub,             FISUB,             Floating-Point Integer Subtract)
MNEM (fisubr,            FISUBR,            Floating-Point Integer Subtract Reverse)
MNEM (fld,               FLD,               Floating-Point Load)
MNEM (fld1,              FLD1,              Floating-Point Load +1.0)
MNEM (fldcw,             FLDCW,             Floating-Point Load x87 Control Word)
MNEM (fldenv,            FLDENV,            Floating-Point Load x87 Environment)
MNEM (fldl2e,            FLDL2E,            Floating-Point Load Log2(e))
MNEM (fldl2t,            FLDL2T,            Floating-Point Load Log2(10))
MNEM (fldlg2,            FLDLG2,            Floating-Point Load Log10(2))
MNEM (fldln2,            FLDLN2,            Floating-Point Load Ln(2))
MNEM (fldpi,             FLDPI,             Floating-Point Load Pi)
MNEM (fldz,              FLDZ,              Floating-Point Load +0.0)
MNEM (fmul,              FMUL,              Floating-Point Multiply)
MNEM (fmulp,             FMULP,             Floating-Point Multiply and Pop)
MNEM (fnclex,            FNCLEX,            Floating-Point Clear Flags)
MNEM (fninit,            FNINIT,            Floating-Point Initialize)
MNEM (fnop,              FNOP,              Floating-Point No Operation)
MNEM (fnsave,            FNSAVE,            Floating-Point Save x87 and MMX State)
MNEM (fnstcw,            FNSTCW,            Floating-Point Store Control Word)
MNEM (fnstenv,           FNSTENV,           Floating-Point Store Environment)
MNEM (fnstsw,            FNSTSW,            Floating-Point Store Status Word)
MNEM (fpatan,            FPATAN,            Floating-Point Partial Arctangent)
MNEM (fprem,             FPREM,             Floating-Point Partial Remainder)
MNEM (fprem1,            FPREM1,            Floating-Point Partial Remainder)
MNEM (fptan,             FPTAN,             Floating-Point Partial Tangent)
MNEM (frndint,           FRNDINT,           Floating-Point Round to Integer)
MNEM (frstor,            FRSTOR,            Floating-Point Restore x87 and MMX State)
MNEM (fscale,            FSCALE,            Floating-Point Scale)
MNEM (fsin,              FSIN,              Floating-Point Sine)
MNEM (fsincos,           FSINCOS,           Floating-Point Sine and Cosine)
MNEM (fsqrt,             FSQRT,             Floating-Point Square Root)
MNEM (fst,               FST,               Floating-Point Store Stack Top)
MNEM (fstp,              FSTP,              Floating-Point Store Stack Top and Pop)
MNEM (fsub,              FSUB,              Floating-Point Subtract)
MNEM (fsubp,             FSUBP,             Floating-Point Subtract and Pop)
MNEM (fsubr,             FSUBR,             Floating-Point Subtract Reverse)
MNEM (fsubrp,            FSUBRP,            Floating-Point Subtract Reverse and Pop)
MNEM (ftst,              FTST,              Floating-Point Test with Zero)
MNEM (fucom,             FUCOM,             Floating-Point Unordered Compare)
MNEM (fucomi,            FUCOMI,            Floating-Point Unordered Compare and Set Flags)
MNEM (fucomip,           FUCOMIP,           Floating-Point Unordered Compare and Set Flags and Pop)
MNEM (fucomp,            FUCOMP,            Floating-Point Unordered Compare and Pop)
MNEM (fucompp,           FUCOMPP,           Floating-Point Unordered Compare and Pop twice)
MNEM (fwait,             FWAIT,             Wait for Unmasked x87 Floating-Point Exceptions)
MNEM (fxam,              FXAM,              Floating-Point Examine)
MNEM (fxch,              FXCH,              Floating-Point Exchange)
MNEM (fxrstor,           FXRSTOR,           Restore XMM MMX and x87 State)
MNEM (fxsave,            FXSAVE,            Save XMM MMX and x87 State)
MNEM (fxtract,           FXTRACT,           Floating-Point Extract Exponent and Significand)
MNEM (fyl2x,             FYL2X,             Floating-Point y*Log2(x))
MNEM (fyl2xp1,           FYL2XP1,           Floating-Point y*Log2(x+1))
MNEM (haddpd,            HADDPD,            Horizontal Add Packed Double)
MNEM (haddps,            HADDPS,            Horizontal Add Packed Single)
MNEM (hlt,               HLT,               Halt)
MNEM (hsubpd,            HSUBPD,            Horizontal Sub Packed Double)
MNEM (hsubps,            HSUBPS,            Horizontal Sub Packed Single)
MNEM (idiv,              IDIV,              Signed Divide)
MNEM (imul,              IMUL,              Signed Multiply)
MNEM (in,                IN,                Input from Port)
MNEM (inc,               INC,               Increment by 1)
MNEM (incssp,            INCSSP,            Increment Shadow Stack Pointer)
MNEM (insb,              INSB,              Input Bytes)
MNEM (insd,              INSD,              Input Doublewords)
MNEM (insertps,          INSERTPS,          Insert Packed Single-Precision Floating-Point)
MNEM (insertq,           INSERTQ,           Insert Field)
MNEM (insw,              INSW,              Input Words)
MNEM (int,               INT,               Interrupt to Vector)
MNEM (int3,              INT3,              Interrupt to Debug Vector)
MNEM (into,              INTO,              Interrupt to Overflow Vector)
MNEM (invd,              INVD,              Invalidate Caches)
MNEM (invlpg,            INVLPG,            Invalidate TLB Entry)
MNEM (invlpga,           INVLPGA,           Invalidate TLB Entry in a Specified ASID)
MNEM (invlpgb,           INVLPGB,           Invalidate TLB Entry with Broadcast)
MNEM (invpcid,           INVPCID,           Invalidate TLB Entry in a Specified PCID)
MNEM (iret,              IRET,              Return from Interrupt)
MNEM (iretd,             IRETD,             Return from Interrupt)
MNEM (iretq,             IRETQ,             Return from Interrupt)
MNEM (ja,                JA,                Jump if above (CF = 0 and ZF = 0))
MNEM (jae,               JAE,               Jump if above or equal (CF = 0))
MNEM (jb,                JB,                Jump if below (CF = 1))
MNEM (jbe,               JBE,               Jump if below or equal (CF = 1 or ZF = 1))
MNEM (jc,                JC,                Jump if carry (CF = 1))
MNEM (jcxz,              JCXZ,              Jump if CX Zero)
MNEM (je,                JE,                Jump if equal (ZF =1))
MNEM (jecxz,             JECXZ,             Jump if ECX Zero)
MNEM (jg,                JG,                Jump if greater (ZF = 0 and SF = OF))
MNEM (jge,               JGE,               Jump if greater or equal (SF = OF))
MNEM (jl,                JL,                Jump if less (SF <> OF))
MNEM (jle,               JLE,               Jump if less or equal (ZF = 1 or SF <> OF))
MNEM (jmp,               JMP,               Near Jump)
MNEM (jmpfar,            JMPFAR,            Far Jump)
MNEM (jna,               JNA,               Jump if not above (CF = 1 or ZF = 1))
MNEM (jnae,              JNAE,              Jump if not above or equal (CF = 1))
MNEM (jnb,               JNB,               Jump if not below (CF = 0))
MNEM (jnbe,              JNBE,              Jump if not below or equal (CF = 0 and ZF = 0))
MNEM (jnc,               JNC,               Jump if not carry (CF = 0))
MNEM (jne,               JNE,               Jump if not equal (ZF = 0))
MNEM (jng,               JNG,               Jump if not greater (ZF = 1 or SF <> OF))
MNEM (jnge,              JNGE,              Jump if not greater or equal (SF <> OF))
MNEM (jnl,               JNL,               Jump if not less (SF = OF))
MNEM (jnle,              JNLE,              Jump if not less or equal (ZF = 0 and SF = OF))
MNEM (jno,               JNO,               Jump if not overflow (OF = 0))
MNEM (jnp,               JNP,               Jump if not parity (PF = 0))
MNEM (jns,               JNS,               Jump if not sign (SF = 0))
MNEM (jnz,               JNZ,               Jump if not zero (ZF = 0))
MNEM (jo,                JO,                Jump if overflow (OF = 1))
MNEM (jp,                JP,                Jump if parity (PF = 1))
MNEM (jpe,               JPE,               Jump if parity even (PF = 1))
MNEM (jpo,               JPO,               Jump if parity odd (PF = 0))
MNEM (jrcxz,             JRCXZ,             Jump if RCX Zero)
MNEM (js,                JS,                Jump if sign (SF =1))
MNEM (jz,                JZ,                Jump if zero (ZF = 1))
MNEM (lahf,              LAHF,              Load Status Flags into AH Register)
MNEM (lar,               LAR,               Load Access Rights Byte)
MNEM (lddqu,             LDDQU,             Load Unaligned Double Quadword)
MNEM (ldmxcsr,           LDMXCSR,           Load MXCSR Control/Status Register)
MNEM (lds,               LDS,               Load Far Pointer)
MNEM (lea,               LEA,               Load Effective Address)
MNEM (leave,             LEAVE,             Delete Procedure Stack Frame)
MNEM (les,               LES,               Load Far Pointer)
MNEM (lfence,            LFENCE,            Load Fence)
MNEM (lfs,               LFS,               Load Far Pointer)
MNEM (lgdt,              LGDT,              Load Global Descriptor Table Register)
MNEM (lgs,               LGS,               Load Far Pointer)
MNEM (lidt,              LIDT,              Load Interrupt Descriptor Table Register)
MNEM (lldt,              LLDT,              Load Local Descriptor Table Register)
MNEM (llwpcb,            LLWPCB,            Load Lightweight Profiling Control Block Address)
MNEM (lmsw,              LMSW,              Load Machine Status Word)
MNEM (lodsb,             LODSB,             Load Bytes)
MNEM (lodsd,             LODSD,             Load Doublewords)
MNEM (lodsq,             LODSQ,             Load Quadwords)
MNEM (lodsw,             LODSW,             Load Words)
MNEM (loop,              LOOP,              Loop if rCX is not 0)
MNEM (loope,             LOOPE,             Loop if rCX is not 0 and ZF is 1)
MNEM (loopne,            LOOPNE,            Loop if rCX is not 0 and ZF is 0)
MNEM (loopnz,            LOOPNZ,            Loop if rCX is not 0 and ZF is 0)
MNEM (loopz,             LOOPZ,             Loop if rCX is not 0 and ZF is 1)
MNEM (lsl,               LSL,               Load Segment Limit)
MNEM (lss,               LSS,               Load Far Pointer)
MNEM (ltr,               LTR,               Load Task Register)
MNEM (lwpins,            LWPINS,            Lightweight Profiling Insert Record)
MNEM (lwpval,            LWPVAL,            Lightweight Profiling Insert Value)
MNEM (lzcnt,             LZCNT,             Count Leading Zeros)
MNEM (maskmovdqu,        MASKMOVDQU,        Masked Move Double Quadword Unaligned)
MNEM (maskmovq,          MASKMOVQ,          Masked Move Quadword)
MNEM (maxpd,             MAXPD,             Maximum Packed Double-Precision Floating-Point)
MNEM (maxps,             MAXPS,             Maximum Packed Single-Precision Floating-Point)
MNEM (maxsd,             MAXSD,             Maximum Scalar Double-Precision Floating-Point)
MNEM (maxss,             MAXSS,             Maximum Scalar Single-Precision Floating-Point)
MNEM (mcommit,           MCOMMIT,           Commit Stores to Memory)
MNEM (mfence,            MFENCE,            Memory Fence)
MNEM (minpd,             MINPD,             Minimum Packed Double-Precision Floating-Point)
MNEM (minps,             MINPS,             Minimum Packed Single-Precision Floating-Point)
MNEM (minsd,             MINSD,             Minimum Scalar Double-Precision Floating-Point)
MNEM (minss,             MINSS,             Minimum Scalar Single-Precision Floating-Point)
MNEM (monitor,           MONITOR,           Setup Monitor Address)
MNEM (monitorx,          MONITORX,          Setup Monitor Address)
MNEM (mov,               MOV,               Move)
MNEM (movapd,            MOVAPD,            Move Aligned Packed Double-Precision Floating-Point)
MNEM (movaps,            MOVAPS,            Move Aligned Packed Single-Precision Floating-Point)
MNEM (movbe,             MOVBE,             Move Big Endian)
MNEM (movd,              MOVD,              Move Doubleword or Quadword)
MNEM (movddup,           MOVDDUP,           Move Double-Precision and Duplicate)
MNEM (movdq2q,           MOVDQ2Q,           Move Quadword to Quadword)
MNEM (movdqa,            MOVDQA,            Move Aligned Double Quadword)
MNEM (movdqu,            MOVDQU,            Move Unaligned Double Quadword)
MNEM (movhlps,           MOVHLPS,           Move Packed Single-Precision Floating-Point High to Low)
MNEM (movhpd,            MOVHPD,            Move High Packed Double-Precision Floating-Point)
MNEM (movhps,            MOVHPS,            Move High Packed Single-Precision Floating-Point)
MNEM (movlhps,           MOVLHPS,           Move Packed Single-Precision Floating-Point Low to High)
MNEM (movlpd,            MOVLPD,            Move Low Packed Double-Precision Floating-Point)
MNEM (movlps,            MOVLPS,            Move Low Packed Single-Precision Floating-Point)
MNEM (movmskpd,          MOVMSKPD,          Extract Packed Double-Precision Floating-Point Sign Mask)
MNEM (movmskps,          MOVMSKPS,          Extract Packed Single-Precision Floating-Point Sign Mask)
MNEM (movntdq,           MOVNTDQ,           Move Non-Temporal Double Quadword)
MNEM (movntdqa,          MOVNTDQA,          Move Non-Temporal Double Quadword Aligned)
MNEM (movnti,            MOVNTI,            Move Non-Temporal Doubleword or Quadword)
MNEM (movntpd,           MOVNTPD,           Move Non-Temporal Packed Double-Precision Floating-Point)
MNEM (movntps,           MOVNTPS,           Move Non-Temporal Packed Single-Precision Floating-Point)
MNEM (movntq,            MOVNTQ,            Move Non-Temporal Double Quadword)
MNEM (movntsd,           MOVNTSD,           Move Non-Temporal Scalar Double-Precision Floating-Point)
MNEM (movntss,           MOVNTSS,           Move Non-Temporal Scalar Single-Precision Floating-Point)
MNEM (movq,              MOVQ,              Move Quadword)
MNEM (movq2dq,           MOVQ2DQ,           Move Quadword to Quadword)
MNEM (movsb,             MOVSB,             Move Bytes)
MNEM (movsd,             MOVSD,             Move Doublewords)
MNEM (movshdup,          MOVSHDUP,          Move Single-Precision High and Duplicate)
MNEM (movsldup,          MOVSLDUP,          Move Single-Precision Low and Duplicate)
MNEM (movsq,             MOVSQ,             Move Quadwords)
MNEM (movss,             MOVSS,             Move Scalar Single-Precision Floating-Point)
MNEM (movsw,             MOVSW,             Move Words)
MNEM (movsx,             MOVSX,             Move with Sign-Extension)
MNEM (movsxd,            MOVSXD,            Move with Sign-Extend Doubleword)
MNEM (movupd,            MOVUPD,            Move Unaligned Packed Double-Precision Floating-Point)
MNEM (movups,            MOVUPS,            Move Unaligned Packed Single-Precision Floating-Point)
MNEM (movzx,             MOVZX,             Move with Zero-Extension)
MNEM (mpsadbw,           MPSADBW,           Multiple Sum of Absolute Differences)
MNEM (mul,               MUL,               Unsigned Multiply)
MNEM (mulpd,             MULPD,             Multiply Packed Double-Precision Floating-Point)
MNEM (mulps,             MULPS,             Multiply Packed Single-Precision Floating-Point)
MNEM (mulsd,             MULSD,             Multiply Scalar Double-Precision Floating-Point)
MNEM (mulss,             MULSS,             Multiply Scalar Single-Precision Floating-Point)
MNEM (mulx,              MULX,              Multiply Unsigned)
MNEM (mwait,             MWAIT,             Monitor Wait)
MNEM (mwaitx,            MWAITX,            Monitor Wait with Timeout)
MNEM (neg,               NEG,               Two`s Complement Negation)
MNEM (nop,               NOP,               No Operation)
MNEM (not,               NOT,               One`s Complement Negation)
MNEM (or,                OR,                Logical OR)
MNEM (orpd,              ORPD,              Logical Bitwise OR Packed Double-Precision Floating-Point)
MNEM (orps,              ORPS,              Logical Bitwise OR Packed Single-Precision Floating-Point)
MNEM (out,               OUT,               Output to Port)
MNEM (outsb,             OUTSB,             Output Bytes)
MNEM (outsd,             OUTSD,             Output Doublewords)
MNEM (outsw,             OUTSW,             Output Words)
MNEM (pabsb,             PABSB,             Packed Absolute Value Signed Byte)
MNEM (pabsd,             PABSD,             Packed Absolute Value Signed Doubleword)
MNEM (pabsw,             PABSW,             Packed Absolute Value Signed Word)
MNEM (packssdw,          PACKSSDW,          Pack with Saturation Signed Doubleword to Word)
MNEM (packsswb,          PACKSSWB,          Pack with Saturation Signed Word to Byte)
MNEM (packusdw,          PACKUSDW,          Pack with Saturation Signed Word to Unsigned Byte)
MNEM (packuswb,          PACKUSWB,          Pack with Saturation Signed Word to Unsigned Byte)
MNEM (paddb,             PADDB,             Packed Add Bytes)
MNEM (paddd,             PADDD,             Packed Add Doublewords)
MNEM (paddq,             PADDQ,             Packed Add Quadwords)
MNEM (paddsb,            PADDSB,            Packed Add Signed with Saturation Bytes)
MNEM (paddsw,            PADDSW,            Packed Add Signed with Saturation Words)
MNEM (paddusb,           PADDUSB,           Packed Add Unsigned with Saturation Bytes)
MNEM (paddusw,           PADDUSW,           Packed Add Unsigned with Saturation Words)
MNEM (paddw,             PADDW,             Packed Add Words)
MNEM (palignr,           PALIGNR,           Packed Align Right)
MNEM (pand,              PAND,              Packed Logical Bitwise AND)
MNEM (pandn,             PANDN,             Packed Logical Bitwise AND NOT)
MNEM (pause,             PAUSE,             Pause)
MNEM (pavgb,             PAVGB,             Packed Average Unsigned Bytes)
MNEM (pavgusb,           PAVGUSB,           Packed Average Unsigned Bytes)
MNEM (pavgw,             PAVGW,             Packed Average Unsigned Words)
MNEM (pblendvb,          PBLENDVB,          Variable Blend Packed Bytes)
MNEM (pblendw,           PBLENDW,           Blend Packed Words)
MNEM (pclmulqdq,         PCLMULQDQ,         Carry-less Multiply Quadwords)
MNEM (pcmpeqb,           PCMPEQB,           Packed Compare Equal Bytes)
MNEM (pcmpeqd,           PCMPEQD,           Packed Compare Equal Doublewords)
MNEM (pcmpeqq,           PCMPEQQ,           Packed Compare Equal Quadwords)
MNEM (pcmpeqw,           PCMPEQW,           Packed Compare Equal Words)
MNEM (pcmpestri,         PCMPESTRI,         Packed Compare Explicit Length Strings Return Index)
MNEM (pcmpestrm,         PCMPESTRM,         Packed Compare Explicit Length Strings Return Mask)
MNEM (pcmpgtb,           PCMPGTB,           Packed Compare Greater Than Signed Bytes)
MNEM (pcmpgtd,           PCMPGTD,           Packed Compare Greater Than Signed Doublewords)
MNEM (pcmpgtq,           PCMPGTQ,           Packed Compare Greater Than Signed Quadwords)
MNEM (pcmpgtw,           PCMPGTW,           Packed Compare Greater Than Signed Words)
MNEM (pcmpistri,         PCMPISTRI,         Packed Compare Implicit Length Strings Return Index)
MNEM (pcmpistrm,         PCMPISTRM,         Packed Compare Implicit Length Strings Return Mask)
MNEM (pdep,              PDEP,              Parallel Deposit Bits)
MNEM (pext,              PEXT,              Parallel Extract Bits)
MNEM (pextrb,            PEXTRB,            Extract Packed Byte)
MNEM (pextrd,            PEXTRD,            Extract Packed Doubleword)
MNEM (pextrq,            PEXTRQ,            Extract Packed Quadword)
MNEM (pextrw,            PEXTRW,            Extract Packed Word)
MNEM (pf2id,             PF2ID,             Packed Floating-Point to Integer Doubleword Conversion)
MNEM (pf2iw,             PF2IW,             Packed Floating-Point to Integer Word Conversion)
MNEM (pfacc,             PFACC,             Packed Floating-Point Accumulate)
MNEM (pfadd,             PFADD,             Packed Floating-Point Add)
MNEM (pfcmpeq,           PFCMPEQ,           Packed Floating-Point Compare Equal)
MNEM (pfcmpge,           PFCMPGE,           Packed Floating-Point Compare Greater or Equal)
MNEM (pfcmpgt,           PFCMPGT,           Packed Floating-Point Compare Greater Than)
MNEM (pfmax,             PFMAX,             Packed Single-Precision Floating-Point Maximum)
MNEM (pfmin,             PFMIN,             Packed Single-Precision Floating-Point Minimum)
MNEM (pfmul,             PFMUL,             Packed Floating-Point Multiply)
MNEM (pfnacc,            PFNACC,            Packed Floating-Point Negative Accumulate)
MNEM (pfpnacc,           PFPNACC,           Packed Floating-Point Positive-Negative Accumulate)
MNEM (pfrcp,             PFRCP,             Packed Floating-Point Positive-Negative Accumulate)
MNEM (pfrcpit1,          PFRCPIT1,          Packed Floating-Point Reciprocal Iteration 1)
MNEM (pfrcpit2,          PFRCPIT2,          Packed Floating-Point Reciprocal Iteration 2)
MNEM (pfrsqit1,          PFRSQIT1,          Packed Floating-Point Reciprocal Square Root Iteration 1)
MNEM (pfrsqrt,           PFRSQRT,           Packed Floating-Point Reciprocal Square Root Approximation)
MNEM (pfsub,             PFSUB,             Packed Floating-Point Subtract)
MNEM (pfsubr,            PFSUBR,            Packed Floating-Point Subtract Reverse)
MNEM (phaddd,            PHADDD,            Packed Horizontal Add Doubleword)
MNEM (phaddsw,           PHADDSW,           Packed Horizontal Add with Saturation Word)
MNEM (phaddw,            PHADDW,            Packed Horizontal Add Word)
MNEM (phminposuw,        PHMINPOSUW,        Horizontal Minimum and Position)
MNEM (phsubd,            PHSUBD,            Packed Horizontal Subtract Doubleword)
MNEM (phsubsw,           PHSUBSW,           Packed Horizontal Subtract with Saturation Word)
MNEM (phsubw,            PHSUBW,            Packed Horizontal Subtract Word)
MNEM (pi2fd,             PI2FD,             Packed Integer to Floating-Point Doubleword Conversion)
MNEM (pi2fw,             PI2FW,             Packed Integer to Floating-Point Word Conversion)
MNEM (pinsrb,            PINSRB,            Packed Insert Byte)
MNEM (pinsrd,            PINSRD,            Packed Insert Doubleword)
MNEM (pinsrq,            PINSRQ,            Packed Insert Quadword)
MNEM (pinsrw,            PINSRW,            Packed Insert Word)
MNEM (pmaddubsw,         PMADDUBSW,         Packed Multiply and Add Unsigned Byte to Signed Word)
MNEM (pmaddwd,           PMADDWD,           Packed Multiply Words and Add Doublewords)
MNEM (pmaxsb,            PMAXSB,            Packed Maximum Signed Bytes)
MNEM (pmaxsd,            PMAXSD,            Packed Maximum Signed Doublewords)
MNEM (pmaxsw,            PMAXSW,            Packed Maximum Signed Words)
MNEM (pmaxub,            PMAXUB,            Packed Maximum Unsigned Bytes)
MNEM (pmaxud,            PMAXUD,            Packed Maximum Unsigned Doublewords)
MNEM (pmaxuw,            PMAXUW,            Packed Maximum Unsigned Words)
MNEM (pminsb,            PMINSB,            Packed Minimum Signed Bytes)
MNEM (pminsd,            PMINSD,            Packed Minimum Signed Doublewords)
MNEM (pminsw,            PMINSW,            Packed Minimum Signed Words)
MNEM (pminub,            PMINUB,            Packed Minimum Unsigned Bytes)
MNEM (pminud,            PMINUD,            Packed Minimum Unsigned Doublewords)
MNEM (pminuw,            PMINUW,            Packed Minimum Unsigned Words)
MNEM (pmovmskb,          PMOVMSKB,          Packed Move Mask Byte)
MNEM (pmovsxbd,          PMOVSXBD,          Packed Move with Sign-Extension Byte to Doubleword)
MNEM (pmovsxbq,          PMOVSXBQ,          Packed Move with Sign-Extension Byte to Quadword)
MNEM (pmovsxbw,          PMOVSXBW,          Packed Move with Sign-Extension Byte to Word)
MNEM (pmovsxdq,          PMOVSXDQ,          Packed Move with Sign-Extension Doubleword to Quadword)
MNEM (pmovsxwd,          PMOVSXWD,          Packed Move with Sign-Extension Word to Doubleword)
MNEM (pmovsxwq,          PMOVSXWQ,          Packed Move with Sign-Extension Word to Quadword)
MNEM (pmovzxbd,          PMOVZXBD,          Packed Move with Zero-Extension Byte to Doubleword)
MNEM (pmovzxbq,          PMOVZXBQ,          Packed Move with Zero-Extension Byte to Quadword)
MNEM (pmovzxbw,          PMOVZXBW,          Packed Move with Zero-Extension Byte to Word)
MNEM (pmovzxdq,          PMOVZXDQ,          Packed Move with Zero-Extension Doubleword to Quadword)
MNEM (pmovzxwd,          PMOVZXWD,          Packed Move with Zero-Extension Word to Doubleword)
MNEM (pmovzxwq,          PMOVZXWQ,          Packed Move with Zero-Extension Word to Quadword)
MNEM (pmuldq,            PMULDQ,            Packed Multiply Signed Doubleword to Quadword)
MNEM (pmulhrsw,          PMULHRSW,          Packed Multiply High with Round and Scale Words)
MNEM (pmulhrw,           PMULHRW,           Packed Multiply High Rounded Word)
MNEM (pmulhuw,           PMULHUW,           Packed Multiply High Unsigned Word)
MNEM (pmulhw,            PMULHW,            Packed Multiply High Signed Word)
MNEM (pmulld,            PMULLD,            Packed Multiply Low Signed Doubleword)
MNEM (pmullw,            PMULLW,            Packed Multiply Low Signed Word)
MNEM (pmuludq,           PMULUDQ,           Packed Multiply Unsigned Doubleword and Store Quadword)
MNEM (pop,               POP,               Pop Stack)
MNEM (popa,              POPA,              POP All GPRs)
MNEM (popad,             POPAD,             POP All GPRs)
MNEM (popcnt,            POPCNT,            Bit Population Count)
MNEM (popf,              POPF,              POP to FLAGS)
MNEM (popfd,             POPFD,             POP to EFLAGS)
MNEM (popfq,             POPFQ,             POP to RFLAGS)
MNEM (por,               POR,               Packed Logical Bitwise OR)
MNEM (prefetch,          PREFETCH,          Prefetch L1 Data-Cache Line)
MNEM (prefetchnta,       PREFETCHNTA,       Prefetch Data to Cache Level NTA)
MNEM (prefetcht0,        PREFETCHT0,        Prefetch Data to Cache Level T0)
MNEM (prefetcht1,        PREFETCHT1,        Prefetch Data to Cache Level T1)
MNEM (prefetcht2,        PREFETCHT2,        Prefetch Data to Cache Level T2)
MNEM (prefetchw,         PREFETCHW,         Prefetch L1 Data-Cache Line)
MNEM (psadbw,            PSADBW,            Packed Sum of Absolute Differences of Bytes Into a Word)
MNEM (pshufb,            PSHUFB,            Packed Shuffle Byte)
MNEM (pshufd,            PSHUFD,            Packed Shuffle Doublewords)
MNEM (pshufhw,           PSHUFHW,           Packed Shuffle High Words)
MNEM (pshuflw,           PSHUFLW,           Packed Shuffle Low Words)
MNEM (pshufw,            PSHUFW,            Packed Shuffle Words)
MNEM (psignb,            PSIGNB,            Packed Sign Byte)
MNEM (psignd,            PSIGND,            Packed Sign Doubleword)
MNEM (psignw,            PSIGNW,            Packed Sign Word)
MNEM (pslld,             PSLLD,             Packed Shift Left Logical Doublewords)
MNEM (pslldq,            PSLLDQ,            Packed Shift Left Logical Double Quadword)
MNEM (psllq,             PSLLQ,             Packed Shift Left Logical Quadwords)
MNEM (psllw,             PSLLW,             Packed Shift Left Logical Words)
MNEM (psmash,            PSMASH,            Page Smash)
MNEM (psrad,             PSRAD,             Packed Shift Right Arithmetic Doublewords)
MNEM (psraw,             PSRAW,             Packed Shift Right Arithmetic Words)
MNEM (psrld,             PSRLD,             Packed Shift Right Logical Doublewords)
MNEM (psrldq,            PSRLDQ,            Packed Shift Right Logical Double Quadword)
MNEM (psrlq,             PSRLQ,             Packed Shift Right Logical Quadwords)
MNEM (psrlw,             PSRLW,             Packed Shift Right Logical Words)
MNEM (psubb,             PSUBB,             Packed Subtract Bytes)
MNEM (psubd,             PSUBD,             Packed Subtract Doublewords)
MNEM (psubq,             PSUBQ,             Packed Subtract Quadwords)
MNEM (psubsb,            PSUBSB,            Packed Subtract Signed with Saturation Bytes)
MNEM (psubsw,            PSUBSW,            Packed Subtract Signed with Saturation Words)
MNEM (psubusb,           PSUBUSB,           Packed Subtract Unsigned with Saturation Bytes)
MNEM (psubusw,           PSUBUSW,           Packed Subtract Unsigned with Saturation Words)
MNEM (psubw,             PSUBW,             Packed Subtract Words)
MNEM (pswapd,            PSWAPD,            Packed Swap Doubleword)
MNEM (ptest,             PTEST,             Packed Bit Test)
MNEM (punpckhbw,         PUNPCKHBW,         Unpack and Interleave High Bytes)
MNEM (punpckhdq,         PUNPCKHDQ,         Unpack and Interleave High Doublewords)
MNEM (punpckhqdq,        PUNPCKHQDQ,        Unpack and Interleave High Quadwords)
MNEM (punpckhwd,         PUNPCKHWD,         Unpack and Interleave High Words)
MNEM (punpcklbw,         PUNPCKLBW,         Unpack and Interleave Low Bytes)
MNEM (punpckldq,         PUNPCKLDQ,         Unpack and Interleave Low Doublewords)
MNEM (punpcklqdq,        PUNPCKLQDQ,        Unpack and Interleave Low Quadwords)
MNEM (punpcklwd,         PUNPCKLWD,         Unpack and Interleave Low Words)
MNEM (push,              PUSH,              Push onto Stack)
MNEM (pusha,             PUSHA,             Push All GPRs onto Stack)
MNEM (pushad,            PUSHAD,            Push All GPRs onto Stack)
MNEM (pushf,             PUSHF,             Push FLAGS onto Stack)
MNEM (pushfd,            PUSHFD,            Push EFLAGS onto Stack)
MNEM (pushfq,            PUSHFQ,            Push RFLAGS onto Stack)
MNEM (pvalidate,         PVALIDATE,         Page Validate)
MNEM (pxor,              PXOR,              Packed Logical Bitwise Exclusive OR)
MNEM (rcl,               RCL,               Rotate Through Carry Left)
MNEM (rcpps,             RCPPS,             Reciprocal Packed Single-Precision Floating-Point)
MNEM (rcpss,             RCPSS,             Reciprocal Scalar Single-Precision Floating-Point)
MNEM (rcr,               RCR,               Rotate Through Carry Right)
MNEM (rdfsbase,          RDFSBASE,          Read FS.base)
MNEM (rdgsbase,          RDGSBASE,          Read GS.base)
MNEM (rdmsr,             RDMSR,             Read Model-Specific Register)
MNEM (rdpid,             RDPID,             Read Processor ID)
MNEM (rdpkru,            RDPKRU,            Read Protection Key Rights)
MNEM (rdpmc,             RDPMC,             Read Performance-Monitoring Counter)
MNEM (rdpru,             RDPRU,             Read Processor Register)
MNEM (rdrand,            RDRAND,            Read Random)
MNEM (rdseed,            RDSEED,            Read Random Seed)
MNEM (rdsspd,            RDSSPD,            Read Shawdow Stack Pointer)
MNEM (rdsspq,            RDSSPQ,            Read Shawdow Stack Pointer)
MNEM (rdtsc,             RDTSC,             Read Time-Stamp Counter)
MNEM (rdtscp,            RDTSCP,            Read Time-Stamp Counter and Processor ID)
MNEM (ret,               RET,               Near Return from Called Procedure)
MNEM (retf,              RETF,              Far Return from Called Procedure)
MNEM (rmpadjust,         RMPADJUST,         Adjust RMP Permissions)
MNEM (rmpquery,          RMPQUERY,          Read RMP Permissions)
MNEM (rmpread,           RMPREAD,           Read RMP Entry)
MNEM (rmpupdate,         RMPUPDATE,         Write RMP Entry)
MNEM (rol,               ROL,               Rotate Left)
MNEM (ror,               ROR,               Rotate Right)
MNEM (rorx,              RORX,              Rotate Right Extended)
MNEM (roundpd,           ROUNDPD,           Round Packed Double-Precision Floating-Point)
MNEM (roundps,           ROUNDPS,           Round Packed Single-Precision Floating-Point)
MNEM (roundsd,           ROUNDSD,           Round Scalar Double-Precision Floating-Point)
MNEM (roundss,           ROUNDSS,           Round Scalar Single-Precision Floating-Point)
MNEM (rsm,               RSM,               Resume from System Management Mode)
MNEM (rsqrtps,           RSQRTPS,           Reciprocal Square Root Packed Single-Precision Floating-Point)
MNEM (rsqrtss,           RSQRTSS,           Reciprocal Square Root Scalar Single-Precision Floating-Point)
MNEM (rstorssp,          RSTORSSP,          Restore Saved Shadow Stack Pointer)
MNEM (sahf,              SAHF,              Store AH into Flags)
MNEM (sal,               SAL,               Shift Arithmetic Left)
MNEM (sar,               SAR,               Shift Arithmetic Right)
MNEM (sarx,              SARX,              Shift Right Arithmetic Extended)
MNEM (saveprevssp,       SAVEPREVSSP,       Save Previous Shadow Stack Pointer)
MNEM (sbb,               SBB,               Subtract with Borrow)
MNEM (scasb,             SCASB,             Scan Bytes)
MNEM (scasd,             SCASD,             Scan Doublewords)
MNEM (scasq,             SCASQ,             Scan Quadwords)
MNEM (scasw,             SCASW,             Scan Words)
MNEM (seta,              SETA,              Set byte if above (CF = 0 and ZF = 0))
MNEM (setae,             SETAE,             Set byte if above or equal (CF = 0))
MNEM (setb,              SETB,              Set byte if below (CF = 1))
MNEM (setbe,             SETBE,             Set byte if below or equal (CF = 1 or ZF = 1))
MNEM (setc,              SETC,              Set byte if carry (CF = 1))
MNEM (sete,              SETE,              Set byte if equal (ZF = 1))
MNEM (setg,              SETG,              Set byte if greater (ZF = 0 and SF = OF))
MNEM (setge,             SETGE,             Set byte if greater or equal (SF = OF))
MNEM (setl,              SETL,              Set byte if less (SF <> OF))
MNEM (setle,             SETLE,             Set byte if less or equal (ZF = 1 or SF <> OF))
MNEM (setna,             SETNA,             Set byte if not above (CF = 1 or ZF = 1))
MNEM (setnae,            SETNAE,            Set byte if not above or equal (CF = 1))
MNEM (setnb,             SETNB,             Set byte if not below (CF = 0))
MNEM (setnbe,            SETNBE,            Set byte if not below or equal (CF = 0 and ZF = 0))
MNEM (setnc,             SETNC,             Set byte if not carry (CF = 0))
MNEM (setne,             SETNE,             Set byte if not equal (ZF = 0))
MNEM (setng,             SETNG,             Set byte if not greater (ZF = 1 or SF <> OF))
MNEM (setnge,            SETNGE,            Set byte if not greater or equal (SF <> OF))
MNEM (setnl,             SETNL,             Set byte if not less (SF = OF))
MNEM (setnle,            SETNLE,            Set byte if not less or equal (ZF = 0 and SF = OF))
MNEM (setno,             SETNO,             Set byte if not overflow (OF = 0))
MNEM (setnp,             SETNP,             Set byte if not parity (PF = 0))
MNEM (setns,             SETNS,             Set byte if not sign (SF = 0))
MNEM (setnz,             SETNZ,             Set byte if not zero (ZF = 0))
MNEM (seto,              SETO,              Set byte if overflow (OF = 1))
MNEM (setp,              SETP,              Set byte if parity (PF = 1))
MNEM (setpe,             SETPE,             Set byte if parity even (PF = 1))
MNEM (setpo,             SETPO,             Set byte if parity odd (PF = 0))
MNEM (sets,              SETS,              Set byte if sign (SF = 1))
MNEM (setssbsy,          SETSSBSY,          Set Shadow Stack Busy)
MNEM (setz,              SETZ,              Set byte if zero (ZF = 1))
MNEM (sfence,            SFENCE,            Store Fence)
MNEM (sgdt,              SGDT,              Store Global Descriptor Table Register)
MNEM (sha1msg1,          SHA1MSG1,          Perform an Intermediate Calculation for the Next Four SHA1 Message Doublewords)
MNEM (sha1msg2,          SHA1MSG2,          Perform a Final Calculation for the Next Four SHA1 Message Doublewords)
MNEM (sha1nexte,         SHA1NEXTE,         Calculate SHA1 State Variable E after Four Rounds)
MNEM (sha1rnds4,         SHA1RNDS4,         Perform Four Rounds of SHA1 Operation)
MNEM (sha256msg1,        SHA256MSG1,        Perform an Intermediate Calculation for the Next Four SHA256 Message Doublewords)
MNEM (sha256msg2,        SHA256MSG2,        Perform a Final Calculation for the Next Four SHA256 Message Doublewords)
MNEM (sha256rnds2,       SHA256RNDS2,       Perform Two Rounds of SHA256 Operation)
MNEM (shl,               SHL,               Shift Left)
MNEM (shld,              SHLD,              Shift Left Double)
MNEM (shlx,              SHLX,              Shift Left Logical Extended)
MNEM (shr,               SHR,               Shift Right)
MNEM (shrd,              SHRD,              Shift Right Double)
MNEM (shrx,              SHRX,              Shift Right Logical Extended)
MNEM (shufpd,            SHUFPD,            Shuffle Packed Double-Precision Floating-Point)
MNEM (shufps,            SHUFPS,            Shuffle Packed Single-Precision Floating-Point)
MNEM (sidt,              SIDT,              Store Interrupt Descriptor Table Register)
MNEM (skinit,            SKINIT,            Secure Init and Jump with Attestation)
MNEM (sldt,              SLDT,              Store Local Descriptor Table Register)
MNEM (slwpcb,            SLWPCB,            Store Lightweight Profiling Control Block Address)
MNEM (smsw,              SMSW,              Store Machine Status Word)
MNEM (sqrtpd,            SQRTPD,            Square Root Packed Double-Precision Floating-Point)
MNEM (sqrtps,            SQRTPS,            Square Root Packed Single-Precision Floating-Point)
MNEM (sqrtsd,            SQRTSD,            Square Root Scalar Double-Precision Floating-Point)
MNEM (sqrtss,            SQRTSS,            Square Root Scalar Single-Precision Floating-Point)
MNEM (stac,              STAC,              Set Alignment Check Flag)
MNEM (stc,               STC,               Set Carry Flag)
MNEM (std,               STD,               Set Direction Flag)
MNEM (stgi,              STGI,              Set Global Interrupt Flag)
MNEM (sti,               STI,               Set Interrupt Flag)
MNEM (stmxcsr,           STMXCSR,           Store MXCSR Control/Status Register)
MNEM (stosb,             STOSB,             Store Bytes)
MNEM (stosd,             STOSD,             Store Doublewords)
MNEM (stosq,             STOSQ,             Store Quadwords)
MNEM (stosw,             STOSW,             Store Words)
MNEM (str,               STR,               Store Task Register)
MNEM (sub,               SUB,               Subtract)
MNEM (subpd,             SUBPD,             Subtract Packed Double-Precision Floating-Point)
MNEM (subps,             SUBPS,             Subtract Packed Single-Precision Floating-Point)
MNEM (subsd,             SUBSD,             Subtract Scalar Double-Precision Floating-Point)
MNEM (subss,             SUBSS,             Subtract Scalar Single-Precision Floating-Point)
MNEM (swapgs,            SWAPGS,            Swap GS Register with KernelGSbase MSR)
MNEM (syscall,           SYSCALL,           Fast System Call)
MNEM (sysenter,          SYSENTER,          System Call)
MNEM (sysexit,           SYSEXIT,           System Return)
MNEM (sysret,            SYSRET,            Fast System Return)
MNEM (t1mskc,            T1MSKC,            Inverse Mask From Trailing Ones)
MNEM (test,              TEST,              Test Bits)
MNEM (tlbsync,           TLBSYNC,           Synchronize TLB Invalidations)
MNEM (tzcnt,             TZCNT,             Count Trailing Zeros)
MNEM (tzmsk,             TZMSK,             Mask From Trailing Zeros)
MNEM (ucomisd,           UCOMISD,           Unordered Compare Scalar Double-Precision Floating-Point)
MNEM (ucomiss,           UCOMISS,           Unordered Compare Scalar Single-Precision Floating-Point)
MNEM (ud0,               UD0,               Undefined Operation)
MNEM (ud1,               UD1,               Undefined Operation)
MNEM (ud2,               UD2,               Undefined Operation)
MNEM (unpckhpd,          UNPCKHPD,          Unpack High Double-Precision Floating-Point)
MNEM (unpckhps,          UNPCKHPS,          Unpack High Single-Precision Floating-Point)
MNEM (unpcklpd,          UNPCKLPD,          Unpack Low Double-Precision Floating-Point)
MNEM (unpcklps,          UNPCKLPS,          Unpack Low Single-Precision Floating-Point)
MNEM (vaddpd,            VADDPD,            Add Packed Double-Precision Floating-Point)
MNEM (vaddps,            VADDPS,            Add Packed Single-Precision Floating-Point)
MNEM (vaddsd,            VADDSD,            Add Scalar Double-Precision Floating-Point)
MNEM (vaddss,            VADDSS,            Add Scalar Single-Precision Floating-Point)
MNEM (vaddsubpd,         VADDSUBPD,         Add and Subtract Packed Double-Precision)
MNEM (vaddsubps,         VADDSUBPS,         Add and Subtract Packed Single-Precision)
MNEM (vaesdec,           VAESDEC,           AES Decryption Round)
MNEM (vaesdeclast,       VAESDECLAST,       AES Last Decryption Round)
MNEM (vaesenc,           VAESENC,           AES Encryption Round)
MNEM (vaesenclast,       VAESENCLAST,       AES Last Encryption Round)
MNEM (vaesimc,           VAESIMC,           AES InvMixColumn Transformation)
MNEM (vaeskeygenassist,  VAESKEYGENASSIST,  AES Assist Round Key Generation)
MNEM (vandnpd,           VANDNPD,           Logical Bitwise AND NOT Packed Double-Precision Floating-Point)
MNEM (vandnps,           VANDNPS,           Logical Bitwise AND NOT Packed Single-Precision Floating-Point)
MNEM (vandpd,            VANDPD,            Logical Bitwise AND Packed Double-Precision Floating-Point)
MNEM (vandps,            VANDPS,            Logical Bitwise AND Packed Single-Precision Floating-Point)
MNEM (vblendpd,          VBLENDPD,          Blend Packed Double-Precision Floating-Point)
MNEM (vblendps,          VBLENDPS,          Blend Packed Single-Precision Floating-Point)
MNEM (vblendvpd,         VBLENDVPD,         Variable Blend Packed Double-Precision Floating-Point)
MNEM (vblendvps,         VBLENDVPS,         Variable Blend Packed Single-Precision Floating-Point)
MNEM (vbroadcastf128,    VBROADCASTF128,    Load With Broadcast From 128-bit Memory Location)
MNEM (vbroadcasti128,    VBROADCASTI128,    Load With Broadcast Integer From 128-bit Memory Location)
MNEM (vbroadcastsd,      VBROADCASTSD,      Load With Broadcast From 64-bit Memory Location)
MNEM (vbroadcastss,      VBROADCASTSS,      Load With Broadcast From 32-bit Memory Location)
MNEM (vcmpeqpd,          VCMPEQPD,          Compare Packed Double-Precision Floating-Point Equal)
MNEM (vcmpeqps,          VCMPEQPS,          Compare Packed Single-Precision Floating-Point Equal)
MNEM (vcmpeqsd,          VCMPEQSD,          Compare Scalar Double-Precision Floating-Point Equal)
MNEM (vcmpeqss,          VCMPEQSS,          Compare Scalar Single-Precision Floating-Point Equal)
MNEM (vcmplepd,          VCMPLEPD,          Compare Packed Double-Precision Floating-Point Less or Equal)
MNEM (vcmpleps,          VCMPLEPS,          Compare Packed Single-Precision Floating-Point Less or Equal)
MNEM (vcmplesd,          VCMPLESD,          Compare Scalar Double-Precision Floating-Point Less or Equal)
MNEM (vcmpless,          VCMPLESS,          Compare Scalar Single-Precision Floating-Point Less or Equal)
MNEM (vcmpltpd,          VCMPLTPD,          Compare Packed Double-Precision Floating-Point Less Than)
MNEM (vcmpltps,          VCMPLTPS,          Compare Packed Single-Precision Floating-Point Less Than)
MNEM (vcmpltsd,          VCMPLTSD,          Compare Scalar Double-Precision Floating-Point Less Than)
MNEM (vcmpltss,          VCMPLTSS,          Compare Scalar Single-Precision Floating-Point Less Than)
MNEM (vcmpneqpd,         VCMPNEQPD,         Compare Packed Double-Precision Floating-Point Not Equal)
MNEM (vcmpneqps,         VCMPNEQPS,         Compare Packed Single-Precision Floating-Point Not Equal)
MNEM (vcmpneqsd,         VCMPNEQSD,         Compare Scalar Double-Precision Floating-Point Not Equal)
MNEM (vcmpneqss,         VCMPNEQSS,         Compare Scalar Single-Precision Floating-Point Not Equal)
MNEM (vcmpnlepd,         VCMPNLEPD,         Compare Packed Double-Precision Floating-Point Not Less or Equal)
MNEM (vcmpnleps,         VCMPNLEPS,         Compare Packed Single-Precision Floating-Point Not Less or Equal)
MNEM (vcmpnlesd,         VCMPNLESD,         Compare Scalar Double-Precision Floating-Point Not Less or Equal)
MNEM (vcmpnless,         VCMPNLESS,         Compare Scalar Single-Precision Floating-Point Not Less or Equal)
MNEM (vcmpnltpd,         VCMPNLTPD,         Compare Packed Double-Precision Floating-Point Not Less Than)
MNEM (vcmpnltps,         VCMPNLTPS,         Compare Packed Single-Precision Floating-Point Not Less Than)
MNEM (vcmpnltsd,         VCMPNLTSD,         Compare Scalar Double-Precision Floating-Point Not Less Than)
MNEM (vcmpnltss,         VCMPNLTSS,         Compare Scalar Single-Precision Floating-Point Not Less Than)
MNEM (vcmpordpd,         VCMPORDPD,         Compare Packed Double-Precision Floating-Point Ordered)
MNEM (vcmpordps,         VCMPORDPS,         Compare Packed Single-Precision Floating-Point Ordered)
MNEM (vcmpordsd,         VCMPORDSD,         Compare Scalar Double-Precision Floating-Point Ordered)
MNEM (vcmpordss,         VCMPORDSS,         Compare Scalar Single-Precision Floating-Point Ordered)
MNEM (vcmppd,            VCMPPD,            Compare Packed Double-Precision Floating-Point)
MNEM (vcmpps,            VCMPPS,            Compare Packed Single-Precision Floating-Point)
MNEM (vcmpsd,            VCMPSD,            Compare Scalar Double-Precision Floating-Point)
MNEM (vcmpss,            VCMPSS,            Compare Scalar Single-Precision Floating-Point)
MNEM (vcmpunordpd,       VCMPUNORDPD,       Compare Packed Double-Precision Floating-Point Unordered)
MNEM (vcmpunordps,       VCMPUNORDPS,       Compare Packed Single-Precision Floating-Point Unordered)
MNEM (vcmpunordsd,       VCMPUNORDSD,       Compare Scalar Double-Precision Floating-Point Unordered)
MNEM (vcmpunordss,       VCMPUNORDSS,       Compare Scalar Single-Precision Floating-Point Unordered)
MNEM (vcomisd,           VCOMISD,           Compare Ordered Scalar Double-Precision Floating-Point)
MNEM (vcomiss,           VCOMISS,           Compare Ordered Scalar Single-Precision Floating-Point)
MNEM (vcvtdq2pd,         VCVTDQ2PD,         Convert Packed Doubleword Integers to Packed Double-Precision Floating-Point)
MNEM (vcvtdq2ps,         VCVTDQ2PS,         Convert Packed Doubleword Integers to Packed Single-Precision Floating-Point)
MNEM (vcvtpd2dq,         VCVTPD2DQ,         Convert Packed Double-Precision Floating-Point to Packed Doubleword Integers)
MNEM (vcvtpd2ps,         VCVTPD2PS,         Convert Packed Double-Precision Floating-Point to Packed Single-Precision Floating-Point)
MNEM (vcvtph2ps,         VCVTPH2PS,         Convert Packed 16-Bit Floating-Point to Single-Precision Floating-Point)
MNEM (vcvtps2dq,         VCVTPS2DQ,         Convert Packed Single-Precision Floating-Point to Packed Doubleword Integers)
MNEM (vcvtps2pd,         VCVTPS2PD,         Convert Packed Single-Precision Floating-Point to Packed Double-Precision Floating-Point)
MNEM (vcvtps2ph,         VCVTPS2PH,         Convert Packed Single-Precision Floating-Point to 16-Bit Floating-Point)
MNEM (vcvtsd2si,         VCVTSD2SI,         Convert Scalar Double-Precision Floating-Point to Signed Doubleword or Quadword Integer)
MNEM (vcvtsd2ss,         VCVTSD2SS,         Convert Scalar Double-Precision Floating-Point to Scalar Single-Precision Floating-Point)
MNEM (vcvtsi2sd,         VCVTSI2SD,         Convert Signed Doubleword or Quadword Integer to Scalar Double-Precision Floating-Point)
MNEM (vcvtsi2ss,         VCVTSI2SS,         Convert Signed Doubleword or Quadword Integer to Scalar Single-Precision Floating-Point)
MNEM (vcvtss2sd,         VCVTSS2SD,         Convert Scalar Single-Precision Floating-Point to Scalar Double-Precision Floating-Point)
MNEM (vcvtss2si,         VCVTSS2SI,         Convert Scalar Single-Precision Floating-Point to Signed Doubleword or Quadword Integer)
MNEM (vcvttpd2dq,        VCVTTPD2DQ,        Convert Packed Double-Precision Floating-Point to Packed Doubleword Integers Truncated)
MNEM (vcvttps2dq,        VCVTTPS2DQ,        Convert Packed Single-Precision Floating-Point to Packed Doubleword Integers Truncated)
MNEM (vcvttsd2si,        VCVTTSD2SI,        Convert Scalar Double-Precision Floating-Point to Signed Doubleword of Quadword Integer Truncated)
MNEM (vcvttss2si,        VCVTTSS2SI,        Convert Scalar Single-Precision Floating-Point to Signed Doubleword or Quadword Integer Truncated)
MNEM (vdivpd,            VDIVPD,            Divide Packed Double-Precision Floating-Point)
MNEM (vdivps,            VDIVPS,            Divide Packed Single-Precision Floating-Point)
MNEM (vdivsd,            VDIVSD,            Divide Scalar Double-Precision Floating-Point)
MNEM (vdivss,            VDIVSS,            Divide Scalar Single-Precision Floating-Point)
MNEM (vdppd,             VDPPD,             Dot Product Packed Double-Precision Floating-Point)
MNEM (vdpps,             VDPPS,             Dot Product Packed Single-Precision Floating-Point)
MNEM (verr,              VERR,              Verify Segment for Reads)
MNEM (verw,              VERW,              Verify Segment for Write)
MNEM (vextractf128,      VEXTRACTF128,      Extract Packed Floating-Point Values)
MNEM (vextracti128,      VEXTRACTI128,      Extract 128-bit Integer)
MNEM (vextractps,        VEXTRACTPS,        Extract Packed Single-Precision Floating-Point)
MNEM (vfmadd132pd,       VFMADD132PD,       Multiply and Add Packed Double-Precision Floating-Point)
MNEM (vfmadd132ps,       VFMADD132PS,       Multiply and Add Packed Single-Precision Floating-Point)
MNEM (vfmadd132sd,       VFMADD132SD,       Multiply and Add Scalar Double-Precision Floating-Point)
MNEM (vfmadd132ss,       VFMADD132SS,       Multiply and Add Scalar Single-Precision Floating-Point)
MNEM (vfmadd213pd,       VFMADD213PD,       Multiply and Add Packed Double-Precision Floating-Point)
MNEM (vfmadd213ps,       VFMADD213PS,       Multiply and Add Packed Single-Precision Floating-Point)
MNEM (vfmadd213sd,       VFMADD213SD,       Multiply and Add Scalar Double-Precision Floating-Point)
MNEM (vfmadd213ss,       VFMADD213SS,       Multiply and Add Scalar Single-Precision Floating-Point)
MNEM (vfmadd231pd,       VFMADD231PD,       Multiply and Add Packed Double-Precision Floating-Point)
MNEM (vfmadd231ps,       VFMADD231PS,       Multiply and Add Packed Single-Precision Floating-Point)
MNEM (vfmadd231sd,       VFMADD231SD,       Multiply and Add Scalar Double-Precision Floating-Point)
MNEM (vfmadd231ss,       VFMADD231SS,       Multiply and Add Scalar Single-Precision Floating-Point)
MNEM (vfmaddpd,          VFMADDPD,          Multiply and Add Packed Double-Precision Floating-Point)
MNEM (vfmaddps,          VFMADDPS,          Multiply and Add Packed Single-Precision Floating-Point)
MNEM (vfmaddsd,          VFMADDSD,          Multiply and Add Scalar Double-Precision Floating-Point)
MNEM (vfmaddss,          VFMADDSS,          Multiply and Add Scalar Single-Precision Floating-Point)
MNEM (vfmaddsub132pd,    VFMADDSUB132PD,    Multiply with Alternating Add/Subtract Packed Double-Precision Floating-Point)
MNEM (vfmaddsub132ps,    VFMADDSUB132PS,    Multiply with Alternating Add/Subtract Packed Single-Precision Floating-Point)
MNEM (vfmaddsub213pd,    VFMADDSUB213PD,    Multiply with Alternating Add/Subtract Packed Double-Precision Floating-Point)
MNEM (vfmaddsub213ps,    VFMADDSUB213PS,    Multiply with Alternating Add/Subtract Packed Single-Precision Floating-Point)
MNEM (vfmaddsub231pd,    VFMADDSUB231PD,    Multiply with Alternating Add/Subtract Packed Double-Precision Floating-Point)
MNEM (vfmaddsub231ps,    VFMADDSUB231PS,    Multiply with Alternating Add/Subtract Packed Single-Precision Floating-Point)
MNEM (vfmaddsubpd,       VFMADDSUBPD,       Multiply with Alternating Add/Subtract Packed Double-Precision Floating-Point)
MNEM (vfmaddsubps,       VFMADDSUBPS,       Multiply with Alternating Add/Subtract Packed Single-Precision Floating-Point)
MNEM (vfmsub132pd,       VFMSUB132PD,       Multiply and Subtract Packed Double-Precision Floating-Point)
MNEM (vfmsub132ps,       VFMSUB132PS,       Multiply and Subtract Packed Single-Precision Floating-Point)
MNEM (vfmsub132sd,       VFMSUB132SD,       Multiply and Subtract Scalar Double-Precision Floating-Point)
MNEM (vfmsub132ss,       VFMSUB132SS,       Multiply and Subtract Scalar Single-Precision Floating-Point)
MNEM (vfmsub213pd,       VFMSUB213PD,       Multiply and Subtract Packed Double-Precision Floating-Point)
MNEM (vfmsub213ps,       VFMSUB213PS,       Multiply and Subtract Packed Single-Precision Floating-Point)
MNEM (vfmsub213sd,       VFMSUB213SD,       Multiply and Subtract Scalar Double-Precision Floating-Point)
MNEM (vfmsub213ss,       VFMSUB213SS,       Multiply and Subtract Scalar Single-Precision Floating-Point)
MNEM (vfmsub231pd,       VFMSUB231PD,       Multiply and Subtract Packed Double-Precision Floating-Point)
MNEM (vfmsub231ps,       VFMSUB231PS,       Multiply and Subtract Packed Single-Precision Floating-Point)
MNEM (vfmsub231sd,       VFMSUB231SD,       Multiply and Subtract Scalar Double-Precision Floating-Point)
MNEM (vfmsub231ss,       VFMSUB231SS,       Multiply and Subtract Scalar Single-Precision Floating-Point)
MNEM (vfmsubadd132pd,    VFMSUBADD132PD,    Multiply with Alternating Subtract/Add Packed Double-Precision Floating-Point)
MNEM (vfmsubadd132ps,    VFMSUBADD132PS,    Multiply with Alternating Subtract/Add Packed Single-Precision Floating-Point)
MNEM (vfmsubadd213pd,    VFMSUBADD213PD,    Multiply with Alternating Subtract/Add Packed Double-Precision Floating-Point)
MNEM (vfmsubadd213ps,    VFMSUBADD213PS,    Multiply with Alternating Subtract/Add Packed Single-Precision Floating-Point)
MNEM (vfmsubadd231pd,    VFMSUBADD231PD,    Multiply with Alternating Subtract/Add Packed Double-Precision Floating-Point)
MNEM (vfmsubadd231ps,    VFMSUBADD231PS,    Multiply with Alternating Subtract/Add Packed Single-Precision Floating-Point)
MNEM (vfmsubaddpd,       VFMSUBADDPD,       Multiply with Alternating Subtract/Add Packed Double-Precision Floating-Point)
MNEM (vfmsubaddps,       VFMSUBADDPS,       Multiply with Alternating Subtract/Add Packed Single-Precision Floating-Point)
MNEM (vfmsubpd,          VFMSUBPD,          Multiply and Subtract Packed Double-Precision Floating-Point)
MNEM (vfmsubps,          VFMSUBPS,          Multiply and Subtract Packed Single-Precision Floating-Point)
MNEM (vfmsubsd,          VFMSUBSD,          Multiply and Subtract Scalar Double-Precision Floating-Point)
MNEM (vfmsubss,          VFMSUBSS,          Multiply and Subtract Scalar Single-Precision Floating-Point)
MNEM (vfnmadd132pd,      VFNMADD132PD,      Negative Multiply and Add Packed Double-Precision Floating-Point)
MNEM (vfnmadd132ps,      VFNMADD132PS,      Negative Multiply and Add Packed Single-Precision Floating-Point)
MNEM (vfnmadd132sd,      VFNMADD132SD,      Negative Multiply and Add Scalar Double-Precision Floating-Point)
MNEM (vfnmadd132ss,      VFNMADD132SS,      Negative Multiply and Add Scalar Single-Precision Floating-Point)
MNEM (vfnmadd213pd,      VFNMADD213PD,      Negative Multiply and Add Packed Double-Precision Floating-Point)
MNEM (vfnmadd213ps,      VFNMADD213PS,      Negative Multiply and Add Packed Single-Precision Floating-Point)
MNEM (vfnmadd213sd,      VFNMADD213SD,      Negative Multiply and Add Scalar Double-Precision Floating-Point)
MNEM (vfnmadd213ss,      VFNMADD213SS,      Negative Multiply and Add Scalar Single-Precision Floating-Point)
MNEM (vfnmadd231pd,      VFNMADD231PD,      Negative Multiply and Add Packed Double-Precision Floating-Point)
MNEM (vfnmadd231ps,      VFNMADD231PS,      Negative Multiply and Add Packed Single-Precision Floating-Point)
MNEM (vfnmadd231sd,      VFNMADD231SD,      Negative Multiply and Add Scalar Double-Precision Floating-Point)
MNEM (vfnmadd231ss,      VFNMADD231SS,      Negative Multiply and Add Scalar Single-Precision Floating-Point)
MNEM (vfnmaddpd,         VFNMADDPD,         Negative Multiply and Add Packed Double-Precision Floating-Point)
MNEM (vfnmaddps,         VFNMADDPS,         Negative Multiply and Add Packed Single-Precision Floating-Point)
MNEM (vfnmaddsd,         VFNMADDSD,         Negative Multiply and Add Scalar Double-Precision Floating-Point)
MNEM (vfnmaddss,         VFNMADDSS,         Negative Multiply and Add Scalar Single-Precision Floating-Point)
MNEM (vfnmsub132pd,      VFNMSUB132PD,      Negative Multiply and Subtract Packed Double-Precision Floating-Point)
MNEM (vfnmsub132ps,      VFNMSUB132PS,      Negative Multiply and Subtract Scalar Double-Precision Floating-Point)
MNEM (vfnmsub132sd,      VFNMSUB132SD,      Negative Multiply and Subtract Scalar Double-Precision Floating-Point)
MNEM (vfnmsub132ss,      VFNMSUB132SS,      Negative Multiply and Subtract Scalar Single-Precision Floating-Point)
MNEM (vfnmsub213pd,      VFNMSUB213PD,      Negative Multiply and Subtract Packed Double-Precision Floating-Point)
MNEM (vfnmsub213ps,      VFNMSUB213PS,      Negative Multiply and Subtract Scalar Double-Precision Floating-Point)
MNEM (vfnmsub213sd,      VFNMSUB213SD,      Negative Multiply and Subtract Scalar Double-Precision Floating-Point)
MNEM (vfnmsub213ss,      VFNMSUB213SS,      Negative Multiply and Subtract Scalar Single-Precision Floating-Point)
MNEM (vfnmsub231pd,      VFNMSUB231PD,      Negative Multiply and Subtract Packed Double-Precision Floating-Point)
MNEM (vfnmsub231ps,      VFNMSUB231PS,      Negative Multiply and Subtract Scalar Double-Precision Floating-Point)
MNEM (vfnmsub231sd,      VFNMSUB231SD,      Negative Multiply and Subtract Scalar Double-Precision Floating-Point)
MNEM (vfnmsub231ss,      VFNMSUB231SS,      Negative Multiply and Subtract Scalar Double-Precision Floating-Point)
MNEM (vfnmsubpd,         VFNMSUBPD,         Negative Multiply and Subtract Packed Double-Precision Floating-Point)
MNEM (vfnmsubps,         VFNMSUBPS,         Negative Multiply and Subtract Packed Single-Precision Floating-Point)
MNEM (vfnmsubsd,         VFNMSUBSD,         Negative Multiply and Subtract Scalar Double-Precision Floating-Point)
MNEM (vfnmsubss,         VFNMSUBSS,         Negative Multiply and Subtract Scalar Single-Precision Floating-Point)
MNEM (vfrczpd,           VFRCZPD,           Extract Fraction Packed Double-Precision Floating-Point)
MNEM (vfrczps,           VFRCZPS,           Extract Fraction Packed Single-Precision Floating-Point)
MNEM (vfrczsd,           VFRCZSD,           Extract Fraction Scalar Double-Precision Floating-Point)
MNEM (vfrczss,           VFRCZSS,           Extract Fraction Scalar Single-Precision Floating-Point)
MNEM (vhaddpd,           VHADDPD,           Horizontal Add Packed Double)
MNEM (vhaddps,           VHADDPS,           Horizontal Add Packed Single)
MNEM (vhsubpd,           VHSUBPD,           Horizontal Sub Packed Double)
MNEM (vhsubps,           VHSUBPS,           Horizontal Sub Packed Single)
MNEM (vinsertf128,       VINSERTF128,       Insert Packed Floating-Point Values 128-bit)
MNEM (vinserti128,       VINSERTI128,       Insert Packed Integer Values 128-bit)
MNEM (vinsertps,         VINSERTPS,         Insert Packed Single-Precision Floating-Point)
MNEM (vlddqu,            VLDDQU,            Load Unaligned Double Quadword)
MNEM (vldmxcsr,          VLDMXCSR,          Load MXCSR Control/Status Register)
MNEM (vmaskmovdqu,       VMASKMOVDQU,       Masked Move Double Quadword Unaligned)
MNEM (vmaskmovpd,        VMASKMOVPD,        Masked Move Packed Double-Precision)
MNEM (vmaskmovps,        VMASKMOVPS,        Masked Move Packed Single-Precision)
MNEM (vmaxpd,            VMAXPD,            Maximum Packed Double-Precision Floating-Point)
MNEM (vmaxps,            VMAXPS,            Maximum Packed Single-Precision Floating-Point)
MNEM (vmaxsd,            VMAXSD,            Maximum Scalar Double-Precision Floating-Point)
MNEM (vmaxss,            VMAXSS,            Maximum Scalar Single-Precision Floating-Point)
MNEM (vmgexit,           VMGEXIT,           SEV-ES Exit to VMM)
MNEM (vminpd,            VMINPD,            Minimum Packed Double-Precision Floating-Point)
MNEM (vminps,            VMINPS,            Minimum Packed Single-Precision Floating-Point)
MNEM (vminsd,            VMINSD,            Minimum Scalar Double-Precision Floating-Point)
MNEM (vminss,            VMINSS,            Minimum Scalar Single-Precision Floating-Point)
MNEM (vmload,            VMLOAD,            Load State from VMCB)
MNEM (vmmcall,           VMMCALL,           Call VMM)
MNEM (vmovapd,           VMOVAPD,           Move Aligned Packed Double-Precision Floating-Point)
MNEM (vmovaps,           VMOVAPS,           Move Aligned Packed Single-Precision Floating-Point)
MNEM (vmovd,             VMOVD,             Move Doubleword or Quadword)
MNEM (vmovddup,          VMOVDDUP,          Move Double-Precision and Duplicate)
MNEM (vmovdqa,           VMOVDQA,           Move Aligned Double Quadword)
MNEM (vmovdqu,           VMOVDQU,           Move Unaligned Double Quadword)
MNEM (vmovhlps,          VMOVHLPS,          Move Packed Single-Precision Floating-Point High to Low)
MNEM (vmovhpd,           VMOVHPD,           Move High Packed Double-Precision Floating-Point)
MNEM (vmovhps,           VMOVHPS,           Move High Packed Single-Precision Floating-Point)
MNEM (vmovlhps,          VMOVLHPS,          Move Packed Single-Precision Floating-Point Low to High)
MNEM (vmovlpd,           VMOVLPD,           Move Low Packed Double-Precision Floating-Point)
MNEM (vmovlps,           VMOVLPS,           Move Low Packed Single-Precision Floating-Point)
MNEM (vmovmskpd,         VMOVMSKPD,         Extract Packed Double-Precision Floating-Point Sign Mask)
MNEM (vmovmskps,         VMOVMSKPS,         Extract Packed Single-Precision Floating-Point Sign Mask)
MNEM (vmovntdq,          VMOVNTDQ,          Move Non-Temporal Double Quadword)
MNEM (vmovntdqa,         VMOVNTDQA,         Move Non-Temporal Double Quadword Aligned)
MNEM (vmovntpd,          VMOVNTPD,          Move Non-Temporal Packed Double-Precision Floating-Point)
MNEM (vmovntps,          VMOVNTPS,          Move Non-Temporal Packed Single-Precision Floating-Point)
MNEM (vmovq,             VMOVQ,             Move Quadword)
MNEM (vmovsd,            VMOVSD,            Move Doublewords)
MNEM (vmovshdup,         VMOVSHDUP,         Move Single-Precision High and Duplicate)
MNEM (vmovsldup,         VMOVSLDUP,         Move Single-Precision Low and Duplicate)
MNEM (vmovss,            VMOVSS,            Move Scalar Single-Precision Floating-Point)
MNEM (vmovupd,           VMOVUPD,           Move Unaligned Packed Double-Precision Floating-Point)
MNEM (vmovups,           VMOVUPS,           Move Unaligned Packed Single-Precision Floating-Point)
MNEM (vmpsadbw,          VMPSADBW,          Multiple Sum of Absolute Differences)
MNEM (vmrun,             VMRUN,             Run Virtual Machine)
MNEM (vmsave,            VMSAVE,            Save State to VMCB)
MNEM (vmulpd,            VMULPD,            Multiply Packed Double-Precision Floating-Point)
MNEM (vmulps,            VMULPS,            Multiply Packed Single-Precision Floating-Point)
MNEM (vmulsd,            VMULSD,            Multiply Scalar Double-Precision Floating-Point)
MNEM (vmulss,            VMULSS,            Multiply Scalar Single-Precision Floating-Point)
MNEM (vorpd,             VORPD,             Logical Bitwise OR Packed Double-Precision Floating-Point)
MNEM (vorps,             VORPS,             Logical Bitwise OR Packed Single-Precision Floating-Point)
MNEM (vpabsb,            VPABSB,            Packed Absolute Value Signed Byte)
MNEM (vpabsd,            VPABSD,            Packed Absolute Value Signed Doubleword)
MNEM (vpabsw,            VPABSW,            Packed Absolute Value Signed Word)
MNEM (vpackssdw,         VPACKSSDW,         Pack with Saturation Signed Doubleword to Word)
MNEM (vpacksswb,         VPACKSSWB,         Pack with Saturation Signed Word to Byte)
MNEM (vpackusdw,         VPACKUSDW,         Pack with Saturation Signed Word to Unsigned Byte)
MNEM (vpackuswb,         VPACKUSWB,         Pack with Saturation Signed Word to Unsigned Byte)
MNEM (vpaddb,            VPADDB,            Packed Add Bytes)
MNEM (vpaddd,            VPADDD,            Packed Add Doublewords)
MNEM (vpaddq,            VPADDQ,            Packed Add Quadwords)
MNEM (vpaddsb,           VPADDSB,           Packed Add Signed with Saturation Bytes)
MNEM (vpaddsw,           VPADDSW,           Packed Add Signed with Saturation Words)
MNEM (vpaddusb,          VPADDUSB,          Packed Add Unsigned with Saturation Bytes)
MNEM (vpaddusw,          VPADDUSW,          Packed Add Unsigned with Saturation Words)
MNEM (vpaddw,            VPADDW,            Packed Add Words)
MNEM (vpalignr,          VPALIGNR,          Packed Align Right)
MNEM (vpand,             VPAND,             Packed Logical Bitwise AND)
MNEM (vpandn,            VPANDN,            Packed Logical Bitwise AND NOT)
MNEM (vpavgb,            VPAVGB,            Packed Average Unsigned Bytes)
MNEM (vpavgw,            VPAVGW,            Packed Average Unsigned Words)
MNEM (vpblendd,          VPBLENDD,          Blend Packed Doublewords)
MNEM (vpblendvb,         VPBLENDVB,         Variable Blend Packed Bytes)
MNEM (vpblendw,          VPBLENDW,          Blend Packed Words)
MNEM (vpbroadcastb,      VPBROADCASTB,      Broadcast Packed Byte)
MNEM (vpbroadcastd,      VPBROADCASTD,      Broadcast Packed Doubleword)
MNEM (vpbroadcastq,      VPBROADCASTQ,      Broadcast Packed Quadword)
MNEM (vpbroadcastw,      VPBROADCASTW,      Broadcast Packed Quadword)
MNEM (vpclmulqdq,        VPCLMULQDQ,        Carry-less Multiply Quadwords)
MNEM (vpcmov,            VPCMOV,            Vector Conditional Moves)
MNEM (vpcmpeqb,          VPCMPEQB,          Packed Compare Equal Bytes)
MNEM (vpcmpeqd,          VPCMPEQD,          Packed Compare Equal Doublewords)
MNEM (vpcmpeqq,          VPCMPEQQ,          Packed Compare Equal Quadwords)
MNEM (vpcmpeqw,          VPCMPEQW,          Packed Compare Equal Words)
MNEM (vpcmpestri,        VPCMPESTRI,        Packed Compare Explicit Length Strings Return Index)
MNEM (vpcmpestrm,        VPCMPESTRM,        Packed Compare Explicit Length Strings Return Mask)
MNEM (vpcmpgtb,          VPCMPGTB,          Packed Compare Greater Than Signed Bytes)
MNEM (vpcmpgtd,          VPCMPGTD,          Packed Compare Greater Than Signed Doublewords)
MNEM (vpcmpgtq,          VPCMPGTQ,          Packed Compare Greater Than Signed Quadwords)
MNEM (vpcmpgtw,          VPCMPGTW,          Packed Compare Greater Than Signed Words)
MNEM (vpcmpistri,        VPCMPISTRI,        Packed Compare Implicit Length Strings Return Index)
MNEM (vpcmpistrm,        VPCMPISTRM,        Packed Compare Implicit Length Strings Return Mask)
MNEM (vpcomb,            VPCOMB,            Compare Vector Signed Bytes)
MNEM (vpcomd,            VPCOMD,            Compare Vector Signed Doublewords)
MNEM (vpcomeqb,          VPCOMEQB,          Compare Vector Signed Bytes Equal)
MNEM (vpcomeqd,          VPCOMEQD,          Compare Vector Signed Doublewords Equal)
MNEM (vpcomeqq,          VPCOMEQQ,          Compare Vector Signed Quadwords Equal)
MNEM (vpcomequb,         VPCOMEQUB,         Compare Vector Unsigned Bytes Equal)
MNEM (vpcomequd,         VPCOMEQUD,         Compare Vector Unsigned Doublewords Equal)
MNEM (vpcomequq,         VPCOMEQUQ,         Compare Vector Unsigned Quadwords Equal)
MNEM (vpcomequw,         VPCOMEQUW,         Compare Vector Unsigned Words Equal)
MNEM (vpcomeqw,          VPCOMEQW,          Compare Vector Signed Words Equal)
MNEM (vpcomfalseb,       VPCOMFALSEB,       Compare Vector Signed Bytes False)
MNEM (vpcomfalsed,       VPCOMFALSED,       Compare Vector Signed Doublewords False)
MNEM (vpcomfalseq,       VPCOMFALSEQ,       Compare Vector Signed Quadwords False)
MNEM (vpcomfalseub,      VPCOMFALSEUB,      Compare Vector Unsigned Bytes False)
MNEM (vpcomfalseud,      VPCOMFALSEUD,      Compare Vector Unsigned Doublewords False)
MNEM (vpcomfalseuq,      VPCOMFALSEUQ,      Compare Vector Unsigned Quadwords False)
MNEM (vpcomfalseuw,      VPCOMFALSEUW,      Compare Vector Unsigned Words False)
MNEM (vpcomfalsew,       VPCOMFALSEW,       Compare Vector Signed Words False)
MNEM (vpcomgeb,          VPCOMGEB,          Compare Vector Signed Bytes Greater Than or Equal)
MNEM (vpcomged,          VPCOMGED,          Compare Vector Signed Doublewords Greater Than or Equal)
MNEM (vpcomgeq,          VPCOMGEQ,          Compare Vector Signed Quadwords Greater Than or Equal)
MNEM (vpcomgeub,         VPCOMGEUB,         Compare Vector Unsigned Bytes Greater Than or Equal)
MNEM (vpcomgeud,         VPCOMGEUD,         Compare Vector Unsigned Doublewords Greater Than or Equal)
MNEM (vpcomgeuq,         VPCOMGEUQ,         Compare Vector Unsigned Quadwords Greater Than or Equal)
MNEM (vpcomgeuw,         VPCOMGEUW,         Compare Vector Unsigned Words Greater Than or Equal)
MNEM (vpcomgew,          VPCOMGEW,          Compare Vector Signed Words Greater Than or Equal)
MNEM (vpcomgtb,          VPCOMGTB,          Compare Vector Signed Bytes Greater Than)
MNEM (vpcomgtd,          VPCOMGTD,          Compare Vector Signed Doublewords Greater Than)
MNEM (vpcomgtq,          VPCOMGTQ,          Compare Vector Signed Quadwords Greater Than)
MNEM (vpcomgtub,         VPCOMGTUB,         Compare Vector Unsigned Bytes Greater Than)
MNEM (vpcomgtud,         VPCOMGTUD,         Compare Vector Unsigned Doublewords Greater Than)
MNEM (vpcomgtuq,         VPCOMGTUQ,         Compare Vector Unsigned Quadwords Greater Than)
MNEM (vpcomgtuw,         VPCOMGTUW,         Compare Vector Unsigned Words Greater Than)
MNEM (vpcomgtw,          VPCOMGTW,          Compare Vector Signed Words Greater Than)
MNEM (vpcomleb,          VPCOMLEB,          Compare Vector Signed Bytes Less Than or Equal)
MNEM (vpcomled,          VPCOMLED,          Compare Vector Signed Doublewords Less Than or Equal)
MNEM (vpcomleq,          VPCOMLEQ,          Compare Vector Signed Quadwords Less Than or Equal)
MNEM (vpcomleub,         VPCOMLEUB,         Compare Vector Unsigned Bytes Less Than or Equal)
MNEM (vpcomleud,         VPCOMLEUD,         Compare Vector Unsigned Doublewords Less Than or Equal)
MNEM (vpcomleuq,         VPCOMLEUQ,         Compare Vector Unsigned Quadwords Less Than or Equal)
MNEM (vpcomleuw,         VPCOMLEUW,         Compare Vector Unsigned Words Less Than or Equal)
MNEM (vpcomlew,          VPCOMLEW,          Compare Vector Signed Words Less Than or Equal)
MNEM (vpcomltb,          VPCOMLTB,          Compare Vector Signed Bytes Less Than)
MNEM (vpcomltd,          VPCOMLTD,          Compare Vector Signed Doublewords Less Than)
MNEM (vpcomltq,          VPCOMLTQ,          Compare Vector Signed Quadwords Less Than)
MNEM (vpcomltub,         VPCOMLTUB,         Compare Vector Unsigned Bytes Less Than)
MNEM (vpcomltud,         VPCOMLTUD,         Compare Vector Unsigned Doublewords Less Than)
MNEM (vpcomltuq,         VPCOMLTUQ,         Compare Vector Unsigned Quadwords Less Than)
MNEM (vpcomltuw,         VPCOMLTUW,         Compare Vector Unsigned Words Less Than)
MNEM (vpcomltw,          VPCOMLTW,          Compare Vector Signed Words Less Than)
MNEM (vpcomneqb,         VPCOMNEQB,         Compare Vector Signed Bytes Not Equal)
MNEM (vpcomneqd,         VPCOMNEQD,         Compare Vector Signed Doublewords Not Equal)
MNEM (vpcomneqq,         VPCOMNEQQ,         Compare Vector Signed Quadwords Not Equal)
MNEM (vpcomnequb,        VPCOMNEQUB,        Compare Vector Unsigned Bytes Not Equal)
MNEM (vpcomnequd,        VPCOMNEQUD,        Compare Vector Unsigned Doublewords Not Equal)
MNEM (vpcomnequq,        VPCOMNEQUQ,        Compare Vector Unsigned Quadwords Not Equal)
MNEM (vpcomnequw,        VPCOMNEQUW,        Compare Vector Unsigned Words Not Equal)
MNEM (vpcomneqw,         VPCOMNEQW,         Compare Vector Signed Words Not Equal)
MNEM (vpcomq,            VPCOMQ,            Compare Vector Signed Quadwords)
MNEM (vpcomtrueb,        VPCOMTRUEB,        Compare Vector Signed Bytes true)
MNEM (vpcomtrued,        VPCOMTRUED,        Compare Vector Signed Doublewords true)
MNEM (vpcomtrueq,        VPCOMTRUEQ,        Compare Vector Signed Quadwords true)
MNEM (vpcomtrueub,       VPCOMTRUEUB,       Compare Vector Unsigned Bytes true)
MNEM (vpcomtrueud,       VPCOMTRUEUD,       Compare Vector Unsigned Doublewords true)
MNEM (vpcomtrueuq,       VPCOMTRUEUQ,       Compare Vector Unsigned Quadwords true)
MNEM (vpcomtrueuw,       VPCOMTRUEUW,       Compare Vector Unsigned Words true)
MNEM (vpcomtruew,        VPCOMTRUEW,        Compare Vector Signed Words true)
MNEM (vpcomub,           VPCOMUB,           Compare Vector Unsigned Bytes)
MNEM (vpcomud,           VPCOMUD,           Compare Vector Unsigned Doublewords)
MNEM (vpcomuq,           VPCOMUQ,           Compare Vector Unsigned Quadwords)
MNEM (vpcomuw,           VPCOMUW,           Compare Vector Unsigned Words)
MNEM (vpcomw,            VPCOMW,            Compare Vector Signed Words)
MNEM (vperm2f128,        VPERM2F128,        Permute Floating-Point 128-bit)
MNEM (vperm2i128,        VPERM2I128,        Permute Integer 128-bit)
MNEM (vpermd,            VPERMD,            Packed Permute Doubleword)
MNEM (vpermpd,           VPERMPD,           Packed Permute Double-Precision Floating-Point)
MNEM (vpermps,           VPERMPS,           Packed Permute Single-Precision Floating-Point)
MNEM (vpermq,            VPERMQ,            Packed Permute Quadword)
MNEM (vpextrb,           VPEXTRB,           Extract Packed Byte)
MNEM (vpextrd,           VPEXTRD,           Extract Packed Doubleword)
MNEM (vpextrq,           VPEXTRQ,           Extract Packed Quadword)
MNEM (vpextrw,           VPEXTRW,           Extract Packed Word)
MNEM (vphaddbd,          VPHADDBD,          Packed Horizontal Add Signed Byte to Signed Doubleword)
MNEM (vphaddbq,          VPHADDBQ,          Packed Horizontal Add Signed Byte to Signed Quadword)
MNEM (vphaddbw,          VPHADDBW,          Packed Horizontal Add Signed Byte to Signed Word)
MNEM (vphaddd,           VPHADDD,           Packed Horizontal Add Doubleword)
MNEM (vphadddq,          VPHADDDQ,          Packed Horizontal Add Signed Doubleword to Signed Quadword)
MNEM (vphaddsw,          VPHADDSW,          Packed Horizontal Add with Saturation Word)
MNEM (vphaddubd,         VPHADDUBD,         Packed Horizontal Add Unsigned Byte to Doubleword)
MNEM (vphaddubq,         VPHADDUBQ,         Packed Horizontal Add Unsigned Byte to Quadword)
MNEM (vphaddubw,         VPHADDUBW,         Packed Horizontal Add Unsigned Byte to Word)
MNEM (vphaddudq,         VPHADDUDQ,         Packed Horizontal Add Unsigned Doubleword to Quadword)
MNEM (vphadduwd,         VPHADDUWD,         Packed Horizontal Add Unsigned Word to Doubleword)
MNEM (vphadduwq,         VPHADDUWQ,         Packed Horizontal Add Unsigned Word to Quadword)
MNEM (vphaddw,           VPHADDW,           Packed Horizontal Add Word)
MNEM (vphaddwd,          VPHADDWD,          Packed Horizontal Add Signed Word to Signed Doubleword)
MNEM (vphaddwq,          VPHADDWQ,          Packed Horizontal Add Signed Word to Signed Quadword)
MNEM (vphminposuw,       VPHMINPOSUW,       Horizontal Minimum and Position)
MNEM (vphsubbw,          VPHSUBBW,          Packed Horizontal Subtract Signed Byte to Signed Word)
MNEM (vphsubd,           VPHSUBD,           Packed Horizontal Subtract Doubleword)
MNEM (vphsubdq,          VPHSUBDQ,          Packed Horizontal Subtract Signed Doubleword to Signed Quadword)
MNEM (vphsubsw,          VPHSUBSW,          Packed Horizontal Subtract with Saturation Word)
MNEM (vphsubw,           VPHSUBW,           Packed Horizontal Subtract Word)
MNEM (vphsubwd,          VPHSUBWD,          Packed Horizontal Subtract Signed Word to Signed Doubleword)
MNEM (vpinsrb,           VPINSRB,           Packed Insert Byte)
MNEM (vpinsrd,           VPINSRD,           Packed Insert Doubleword)
MNEM (vpinsrq,           VPINSRQ,           Packed Insert Quadword)
MNEM (vpinsrw,           VPINSRW,           Packed Insert Word)
MNEM (vpmacsdd,          VPMACSDD,          Packed Multiply Accumulate Signed Doubleword to Signed Doubleword)
MNEM (vpmacsdqh,         VPMACSDQH,         Packed Multiply Accumulate Signed High Doubleword to Signed Quadword)
MNEM (vpmacsdql,         VPMACSDQL,         Packed Multiply Accumulate Signed Low Doubleword to Signed Quadword)
MNEM (vpmacssdd,         VPMACSSDD,         Packed Multiply Accumulate Signed Doubleword to Signed Doubleword with Saturation)
MNEM (vpmacssdqh,        VPMACSSDQH,        Packed Multiply Accumulate Signed High Doubleword to Signed Quadword with Saturation)
MNEM (vpmacssdql,        VPMACSSDQL,        Packed Multiply Accumulate Signed Low Doubleword to Signed Quadword with Saturation)
MNEM (vpmacsswd,         VPMACSSWD,         Packed Multiply Accumulate Signed Word to Signed Doubleword with Saturation)
MNEM (vpmacssww,         VPMACSSWW,         Packed Multiply Accumulate Signed Word to Signed Word with Saturation)
MNEM (vpmacswd,          VPMACSWD,          Packed Multiply Accumulate Signed Word to Signed Doubleword)
MNEM (vpmacsww,          VPMACSWW,          Packed Multiply Accumulate Signed Word to Signed Word)
MNEM (vpmadcsswd,        VPMADCSSWD,        Packed Multiply Add and Accumulate Signed Word to Signed Doubleword with Saturation)
MNEM (vpmadcswd,         VPMADCSWD,         Packed Multiply Add and Accumulate Signed Word to Signed Doubleword)
MNEM (vpmaddubsw,        VPMADDUBSW,        Packed Multiply and Add Unsigned Byte to Signed Word)
MNEM (vpmaddwd,          VPMADDWD,          Packed Multiply Words and Add Doublewords)
MNEM (vpmaskmovd,        VPMASKMOVD,        Masked Move Packed Doubleword)
MNEM (vpmaskmovq,        VPMASKMOVQ,        Masked Move Packed Quadword)
MNEM (vpmaxsb,           VPMAXSB,           Packed Maximum Signed Bytes)
MNEM (vpmaxsd,           VPMAXSD,           Packed Maximum Signed Doublewords)
MNEM (vpmaxsw,           VPMAXSW,           Packed Maximum Signed Words)
MNEM (vpmaxub,           VPMAXUB,           Packed Maximum Unsigned Bytes)
MNEM (vpmaxud,           VPMAXUD,           Packed Maximum Unsigned Doublewords)
MNEM (vpmaxuw,           VPMAXUW,           Packed Maximum Unsigned Words)
MNEM (vpminsb,           VPMINSB,           Packed Minimum Signed Bytes)
MNEM (vpminsd,           VPMINSD,           Packed Minimum Signed Doublewords)
MNEM (vpminsw,           VPMINSW,           Packed Minimum Signed Words)
MNEM (vpminub,           VPMINUB,           Packed Minimum Unsigned Bytes)
MNEM (vpminud,           VPMINUD,           Packed Minimum Unsigned Doublewords)
MNEM (vpminuw,           VPMINUW,           Packed Minimum Unsigned Words)
MNEM (vpmovmskb,         VPMOVMSKB,         Packed Move Mask Byte)
MNEM (vpmovsxbd,         VPMOVSXBD,         Packed Move with Sign-Extension Byte to Doubleword)
MNEM (vpmovsxbq,         VPMOVSXBQ,         Packed Move with Sign-Extension Byte to Quadword)
MNEM (vpmovsxbw,         VPMOVSXBW,         Packed Move with Sign-Extension Byte to Word)
MNEM (vpmovsxdq,         VPMOVSXDQ,         Packed Move with Sign-Extension Doubleword to Quadword)
MNEM (vpmovsxwd,         VPMOVSXWD,         Packed Move with Sign-Extension Word to Doubleword)
MNEM (vpmovsxwq,         VPMOVSXWQ,         Packed Move with Sign-Extension Word to Quadword)
MNEM (vpmovzxbd,         VPMOVZXBD,         Packed Move with Zero-Extension Byte to Doubleword)
MNEM (vpmovzxbq,         VPMOVZXBQ,         Packed Move with Zero-Extension Byte to Quadword)
MNEM (vpmovzxbw,         VPMOVZXBW,         Packed Move with Zero-Extension Byte to Word)
MNEM (vpmovzxdq,         VPMOVZXDQ,         Packed Move with Zero-Extension Doubleword to Quadword)
MNEM (vpmovzxwd,         VPMOVZXWD,         Packed Move with Zero-Extension Word to Doubleword)
MNEM (vpmovzxwq,         VPMOVZXWQ,         Packed Move with Zero-Extension Word to Quadword)
MNEM (vpmuldq,           VPMULDQ,           Packed Multiply Signed Doubleword to Quadword)
MNEM (vpmulhrsw,         VPMULHRSW,         Packed Multiply High with Round and Scale Words)
MNEM (vpmulhuw,          VPMULHUW,          Packed Multiply High Unsigned Word)
MNEM (vpmulhw,           VPMULHW,           Packed Multiply High Signed Word)
MNEM (vpmulld,           VPMULLD,           Packed Multiply Low Signed Doubleword)
MNEM (vpmullw,           VPMULLW,           Packed Multiply Low Signed Word)
MNEM (vpmuludq,          VPMULUDQ,          Packed Multiply Unsigned Doubleword and Store Quadword)
MNEM (vpor,              VPOR,              Packed Logical Bitwise OR)
MNEM (vpperm,            VPPERM,            Packed Permute Bytes)
MNEM (vprotb,            VPROTB,            Packed Rotate Bytes)
MNEM (vprotd,            VPROTD,            Packed Rotate Doublewords)
MNEM (vprotq,            VPROTQ,            Packed Rotate Quadwords)
MNEM (vprotw,            VPROTW,            Packed Rotate Words)
MNEM (vpsadbw,           VPSADBW,           Packed Sum of Absolute Differences of Bytes Into a Word)
MNEM (vpshab,            VPSHAB,            Packed Shift Arithmetic Bytes)
MNEM (vpshad,            VPSHAD,            Packed Shift Arithmetic Doublewords)
MNEM (vpshaq,            VPSHAQ,            Packed Shift Arithmetic Quadwords)
MNEM (vpshaw,            VPSHAW,            Packed Shift Arithmetic Words)
MNEM (vpshlb,            VPSHLB,            Packed Shift Logical Bytes)
MNEM (vpshld,            VPSHLD,            Packed Shift Logical Doublewords)
MNEM (vpshlq,            VPSHLQ,            Packed Shift Logical Quadwords)
MNEM (vpshlw,            VPSHLW,            Packed Shift Logical Words)
MNEM (vpshufb,           VPSHUFB,           Packed Shuffle Byte)
MNEM (vpshufd,           VPSHUFD,           Packed Shuffle Doublewords)
MNEM (vpshufhw,          VPSHUFHW,          Packed Shuffle High Words)
MNEM (vpshuflw,          VPSHUFLW,          Packed Shuffle Low Words)
MNEM (vpsignb,           VPSIGNB,           Packed Sign Byte)
MNEM (vpsignd,           VPSIGND,           Packed Sign Doubleword)
MNEM (vpsignw,           VPSIGNW,           Packed Sign Word)
MNEM (vpslld,            VPSLLD,            Packed Shift Left Logical Doublewords)
MNEM (vpslldq,           VPSLLDQ,           Packed Shift Left Logical Double Quadword)
MNEM (vpsllq,            VPSLLQ,            Packed Shift Left Logical Quadwords)
MNEM (vpsllvd,           VPSLLVD,           Variable Shift Left Logical Doublewords)
MNEM (vpsllvq,           VPSLLVQ,           Variable Shift Left Logical Quadwords)
MNEM (vpsllw,            VPSLLW,            Packed Shift Left Logical Words)
MNEM (vpsrad,            VPSRAD,            Packed Shift Right Arithmetic Doublewords)
MNEM (vpsravd,           VPSRAVD,           Variable Shift Right Arithmetic Doublewords)
MNEM (vpsraw,            VPSRAW,            Packed Shift Right Arithmetic Words)
MNEM (vpsrld,            VPSRLD,            Packed Shift Right Logical Doublewords)
MNEM (vpsrldq,           VPSRLDQ,           Packed Shift Right Logical Double Quadword)
MNEM (vpsrlq,            VPSRLQ,            Packed Shift Right Logical Quadwords)
MNEM (vpsrlvd,           VPSRLVD,           Variable Shift Right Logical Doublewords)
MNEM (vpsrlvq,           VPSRLVQ,           Variable Shift Right Logical Quadwords)
MNEM (vpsrlw,            VPSRLW,            Packed Shift Right Logical Words)
MNEM (vpsubb,            VPSUBB,            Packed Subtract Bytes)
MNEM (vpsubd,            VPSUBD,            Packed Subtract Doublewords)
MNEM (vpsubq,            VPSUBQ,            Packed Subtract Quadwords)
MNEM (vpsubsb,           VPSUBSB,           Packed Subtract Signed with Saturation Bytes)
MNEM (vpsubsw,           VPSUBSW,           Packed Subtract Signed with Saturation Words)
MNEM (vpsubusb,          VPSUBUSB,          Packed Subtract Unsigned with Saturation Bytes)
MNEM (vpsubusw,          VPSUBUSW,          Packed Subtract Unsigned with Saturation Words)
MNEM (vpsubw,            VPSUBW,            Packed Subtract Words)
MNEM (vptest,            VPTEST,            Packed Bit Test)
MNEM (vpunpckhbw,        VPUNPCKHBW,        Unpack and Interleave High Bytes)
MNEM (vpunpckhdq,        VPUNPCKHDQ,        Unpack and Interleave High Doublewords)
MNEM (vpunpckhqdq,       VPUNPCKHQDQ,       Unpack and Interleave High Quadwords)
MNEM (vpunpckhwd,        VPUNPCKHWD,        Unpack and Interleave High Words)
MNEM (vpunpcklbw,        VPUNPCKLBW,        Unpack and Interleave Low Bytes)
MNEM (vpunpckldq,        VPUNPCKLDQ,        Unpack and Interleave Low Doublewords)
MNEM (vpunpcklqdq,       VPUNPCKLQDQ,       Unpack and Interleave Low Quadwords)
MNEM (vpunpcklwd,        VPUNPCKLWD,        Unpack and Interleave Low Words)
MNEM (vpxor,             VPXOR,             Packed Logical Bitwise Exclusive OR)
MNEM (vrcpps,            VRCPPS,            Reciprocal Packed Single-Precision Floating-Point)
MNEM (vrcpss,            VRCPSS,            Reciprocal Scalar Single-Precision Floating-Point)
MNEM (vroundpd,          VROUNDPD,          Round Packed Double-Precision Floating-Point)
MNEM (vroundps,          VROUNDPS,          Round Packed Single-Precision Floating-Point)
MNEM (vroundsd,          VROUNDSD,          Round Scalar Double-Precision Floating-Point)
MNEM (vroundss,          VROUNDSS,          Round Scalar Single-Precision Floating-Point)
MNEM (vrsqrtps,          VRSQRTPS,          Reciprocal Square Root Packed Single-Precision Floating-Point)
MNEM (vrsqrtss,          VRSQRTSS,          Reciprocal Square Root Scalar Single-Precision Floating-Point)
MNEM (vshufpd,           VSHUFPD,           Shuffle Packed Double-Precision Floating-Point)
MNEM (vshufps,           VSHUFPS,           Shuffle Packed Single-Precision Floating-Point)
MNEM (vsqrtpd,           VSQRTPD,           Square Root Packed Double-Precision Floating-Point)
MNEM (vsqrtps,           VSQRTPS,           Square Root Packed Single-Precision Floating-Point)
MNEM (vsqrtsd,           VSQRTSD,           Square Root Scalar Double-Precision Floating-Point)
MNEM (vsqrtss,           VSQRTSS,           Square Root Scalar Single-Precision Floating-Point)
MNEM (vstmxcsr,          VSTMXCSR,          Store MXCSR Control/Status Register)
MNEM (vsubpd,            VSUBPD,            Subtract Packed Double-Precision Floating-Point)
MNEM (vsubps,            VSUBPS,            Subtract Packed Single-Precision Floating-Point)
MNEM (vsubsd,            VSUBSD,            Subtract Scalar Double-Precision Floating-Point)
MNEM (vsubss,            VSUBSS,            Subtract Scalar Single-Precision Floating-Point)
MNEM (vtestpd,           VTESTPD,           Packed Bit Test)
MNEM (vtestps,           VTESTPS,           Packed Bit Test)
MNEM (vucomisd,          VUCOMISD,          Unordered Compare Scalar Double-Precision Floating-Point)
MNEM (vucomiss,          VUCOMISS,          Unordered Compare Scalar Single-Precision Floating-Point)
MNEM (vunpckhpd,         VUNPCKHPD,         Unpack High Double-Precision Floating-Point)
MNEM (vunpckhps,         VUNPCKHPS,         Unpack High Single-Precision Floating-Point)
MNEM (vunpcklpd,         VUNPCKLPD,         Unpack Low Double-Precision Floating-Point)
MNEM (vunpcklps,         VUNPCKLPS,         Unpack Low Single-Precision Floating-Point)
MNEM (vxorpd,            VXORPD,            Logical Bitwise Exclusive OR Packed Double-Precision Floating-Point)
MNEM (vxorps,            VXORPS,            Logical Bitwise Exclusive OR Packed Single-Precision Floating-Point)
MNEM (vzeroall,          VZEROALL,          Zero All YMM Registers)
MNEM (vzeroupper,        VZEROUPPER,        Zero All YMM Registers Upper)
MNEM (wbinvd,            WBINVD,            Writeback and Invalidate Caches)
MNEM (wbnoinvd,          WBNOINVD,          Writeback With No Invalidate)
MNEM (wrfsbase,          WRFSBASE,          Write FS.base)
MNEM (wrgsbase,          WRGSBASE,          Write GS.base)
MNEM (wrmsr,             WRMSR,             Write to Model-Specific Register)
MNEM (wrpkru,            WRPKRU,            Write Protection Key Rights)
MNEM (wrssd,             WRSSD,             Write to Shadow Stack)
MNEM (wrssq,             WRSSQ,             Write to Shadow Stack)
MNEM (wrussd,            WRUSSD,            Write to User Shadow Stack)
MNEM (wrussq,            WRUSSQ,            Write to User Shadow Stack)
MNEM (xadd,              XADD,              Exchange and Add)
MNEM (xchg,              XCHG,              Exchange)
MNEM (xgetbv,            XGETBV,            Get Extended Control Register Value)
MNEM (xlatb,             XLATB,             Translate Table Index)
MNEM (xor,               XOR,               Logical Exclusive OR)
MNEM (xorpd,             XORPD,             Logical Bitwise Exclusive OR Packed Double-Precision Floating-Point)
MNEM (xorps,             XORPS,             Logical Bitwise Exclusive OR Packed Single-Precision Floating-Point)
MNEM (xrstor,            XRSTOR,            Restore Extended States)
MNEM (xrstors,           XRSTORS,           Restore Extended States Supervisor)
MNEM (xsave,             XSAVE,             Save Extended States)
MNEM (xsavec,            XSAVEC,            Save Extended States in Compacted Form)
MNEM (xsaveopt,          XSAVEOPT,          Save Extended States Performance Optimized)
MNEM (xsaves,            XSAVES,            Save Extended States Supervisor)
MNEM (xsetbv,            XSETBV,            Set Extended Control Register Value)

// General-Purpose Instruction Reference

INSTR (AAA,               none,       none,       none,       none,       0,       0x37,      no,  no,  0,     I64)

INSTR (AAD,               imm8,       none,       none,       none,       0,       0xd5,      ib,  no,  0,     I64)
INSTR (AAD,               none,       none,       none,       none,       0,       0xd5,      no,  no,  0x0a,  I64 | SFX)

INSTR (AAM,               imm8,       none,       none,       none,       0,       0xd4,      ib,  no,  0,     I64)
INSTR (AAM,               none,       none,       none,       none,       0,       0xd4,      no,  no,  0x0a,  I64 | SFX)

INSTR (AAS,               none,       none,       none,       none,       0,       0x3f,      no,  no,  0,     I64)

INSTR (ADC,               al,         imm8,       none,       none,       0,       0x14,      ib,  no,  0,     0)
INSTR (ADC,               regmem8,    imm8,       none,       none,       0,       0x80,      r2,  ib,  0,     PLOCK)
INSTR (ADC,               regmem8,    imm8,       none,       none,       0,       0x82,      r2,  ib,  0,     I64 | PLOCK)
INSTR (ADC,               regmem16,   simm8,      none,       none,       0,       0x83,      r2,  ib,  0,     O16 | PLOCK)
INSTR (ADC,               regmem32,   simm8,      none,       none,       0,       0x83,      r2,  ib,  0,     O32 | PLOCK)
INSTR (ADC,               regmem64,   simm8,      none,       none,       0,       0x83,      r2,  ib,  0,     O64 | PLOCK)
INSTR (ADC,               ax,         imm16,      none,       none,       0,       0x15,      iw,  no,  0,     O16)
INSTR (ADC,               regmem16,   imm16,      none,       none,       0,       0x81,      r2,  iw,  0,     O16 | PLOCK)
INSTR (ADC,               eax,        imm32,      none,       none,       0,       0x15,      id,  no,  0,     O32)
INSTR (ADC,               regmem32,   imm32,      none,       none,       0,       0x81,      r2,  id,  0,     O32 | PLOCK)
INSTR (ADC,               rax,        simm32,     none,       none,       0,       0x15,      id,  no,  0,     O64)
INSTR (ADC,               regmem64,   simm32,     none,       none,       0,       0x81,      r2,  id,  0,     O64 | PLOCK)
INSTR (ADC,               regmem8,    reg8,       none,       none,       0,       0x10,      r,   no,  0,     PLOCK)
INSTR (ADC,               regmem16,   reg16,      none,       none,       0,       0x11,      r,   no,  0,     O16 | PLOCK)
INSTR (ADC,               regmem32,   reg32,      none,       none,       0,       0x11,      r,   no,  0,     O32 | PLOCK)
INSTR (ADC,               regmem64,   reg64,      none,       none,       0,       0x11,      r,   no,  0,     O64 | PLOCK)
INSTR (ADC,               reg8,       regmem8,    none,       none,       0,       0x12,      r,   no,  0,     0)
INSTR (ADC,               reg16,      regmem16,   none,       none,       0,       0x13,      r,   no,  0,     O16)
INSTR (ADC,               reg32,      regmem32,   none,       none,       0,       0x13,      r,   no,  0,     O32)
INSTR (ADC,               reg64,      regmem64,   none,       none,       0,       0x13,      r,   no,  0,     O64)

INSTR (ADCX,              reg32,      regmem32,   none,       none,       0,       0x0f38f6,  r,   no,  0,     O16 | O32 | P66)
INSTR (ADCX,              reg64,      regmem64,   none,       none,       0,       0x0f38f6,  r,   no,  0,     O64 | P66)

INSTR (ADD,               al,         imm8,       none,       none,       0,       0x04,      ib,  no,  0,     0)
INSTR (ADD,               regmem8,    imm8,       none,       none,       0,       0x80,      r0,  ib,  0,     PLOCK)
INSTR (ADD,               regmem8,    imm8,       none,       none,       0,       0x82,      r0,  ib,  0,     I64 | PLOCK)
INSTR (ADD,               regmem16,   simm8,      none,       none,       0,       0x83,      r0,  ib,  0,     O16 | PLOCK)
INSTR (ADD,               regmem32,   simm8,      none,       none,       0,       0x83,      r0,  ib,  0,     O32 | PLOCK)
INSTR (ADD,               regmem64,   simm8,      none,       none,       0,       0x83,      r0,  ib,  0,     O64 | PLOCK)
INSTR (ADD,               ax,         imm16,      none,       none,       0,       0x05,      iw,  no,  0,     O16)
INSTR (ADD,               regmem16,   imm16,      none,       none,       0,       0x81,      r0,  iw,  0,     O16 | PLOCK)
INSTR (ADD,               eax,        imm32,      none,       none,       0,       0x05,      id,  no,  0,     O32)
INSTR (ADD,               regmem32,   imm32,      none,       none,       0,       0x81,      r0,  id,  0,     O32 | PLOCK)
INSTR (ADD,               rax,        simm32,     none,       none,       0,       0x05,      id,  no,  0,     O64)
INSTR (ADD,               regmem64,   simm32,     none,       none,       0,       0x81,      r0,  id,  0,     O64 | PLOCK)
INSTR (ADD,               regmem8,    reg8,       none,       none,       0,       0x00,      r,   no,  0,     PLOCK)
INSTR (ADD,               regmem16,   reg16,      none,       none,       0,       0x01,      r,   no,  0,     O16 | PLOCK)
INSTR (ADD,               regmem32,   reg32,      none,       none,       0,       0x01,      r,   no,  0,     O32 | PLOCK)
INSTR (ADD,               regmem64,   reg64,      none,       none,       0,       0x01,      r,   no,  0,     O64 | PLOCK)
INSTR (ADD,               reg8,       regmem8,    none,       none,       0,       0x02,      r,   no,  0,     0)
INSTR (ADD,               reg16,      regmem16,   none,       none,       0,       0x03,      r,   no,  0,     O16)
INSTR (ADD,               reg32,      regmem32,   none,       none,       0,       0x03,      r,   no,  0,     O32)
INSTR (ADD,               reg64,      regmem64,   none,       none,       0,       0x03,      r,   no,  0,     O64)

INSTR (ADOX,              reg32,      regmem32,   none,       none,       0,       0x0f38f6,  r,   no,  0,     O16 | O32 | PF3)
INSTR (ADOX,              reg64,      regmem64,   none,       none,       0,       0x0f38f6,  r,   no,  0,     O64 | PF3)

INSTR (AND,               al,         imm8,       none,       none,       0,       0x24,      ib,  no,  0,     0)
INSTR (AND,               regmem8,    imm8,       none,       none,       0,       0x80,      r4,  ib,  0,     PLOCK)
INSTR (AND,               regmem8,    imm8,       none,       none,       0,       0x82,      r4,  ib,  0,     I64 | PLOCK)
INSTR (AND,               regmem16,   simm8,      none,       none,       0,       0x83,      r4,  ib,  0,     O16 | PLOCK)
INSTR (AND,               regmem32,   simm8,      none,       none,       0,       0x83,      r4,  ib,  0,     O32 | PLOCK)
INSTR (AND,               regmem64,   simm8,      none,       none,       0,       0x83,      r4,  ib,  0,     O64 | PLOCK)
INSTR (AND,               ax,         imm16,      none,       none,       0,       0x25,      iw,  no,  0,     O16)
INSTR (AND,               regmem16,   imm16,      none,       none,       0,       0x81,      r4,  iw,  0,     O16 | PLOCK)
INSTR (AND,               eax,        imm32,      none,       none,       0,       0x25,      id,  no,  0,     O32)
INSTR (AND,               regmem32,   imm32,      none,       none,       0,       0x81,      r4,  id,  0,     O32 | PLOCK)
INSTR (AND,               rax,        simm32,     none,       none,       0,       0x25,      id,  no,  0,     O64)
INSTR (AND,               regmem64,   simm32,     none,       none,       0,       0x81,      r4,  id,  0,     O64 | PLOCK)
INSTR (AND,               regmem8,    reg8,       none,       none,       0,       0x20,      r,   no,  0,     PLOCK)
INSTR (AND,               regmem16,   reg16,      none,       none,       0,       0x21,      r,   no,  0,     O16 | PLOCK)
INSTR (AND,               regmem32,   reg32,      none,       none,       0,       0x21,      r,   no,  0,     O32 | PLOCK)
INSTR (AND,               regmem64,   reg64,      none,       none,       0,       0x21,      r,   no,  0,     O64 | PLOCK)
INSTR (AND,               reg8,       regmem8,    none,       none,       0,       0x22,      r,   no,  0,     0)
INSTR (AND,               reg16,      regmem16,   none,       none,       0,       0x23,      r,   no,  0,     O16)
INSTR (AND,               reg32,      regmem32,   none,       none,       0,       0x23,      r,   no,  0,     O32)
INSTR (AND,               reg64,      regmem64,   none,       none,       0,       0x23,      r,   no,  0,     O64)

INSTR (BOUND,             reg16,      mem,        none,       none,       0,       0x62,      r,   no,  0,     O16 | I64)
INSTR (BOUND,             reg32,      mem,        none,       none,       0,       0x62,      r,   no,  0,     O32 | I64)

INSTR (BSF,               reg16,      regmem16,   none,       none,       0,       0x0fbc,    r,   no,  0,     O16)
INSTR (BSF,               reg32,      regmem32,   none,       none,       0,       0x0fbc,    r,   no,  0,     O32)
INSTR (BSF,               reg64,      regmem64,   none,       none,       0,       0x0fbc,    r,   no,  0,     O64)

INSTR (LZCNT,             reg16,      regmem16,   none,       none,       0,       0x0fbd,    r,   no,  0,     O16 | PF3)
INSTR (LZCNT,             reg32,      regmem32,   none,       none,       0,       0x0fbd,    r,   no,  0,     O32 | PF3)
INSTR (LZCNT,             reg64,      regmem64,   none,       none,       0,       0x0fbd,    r,   no,  0,     O64 | PF3)

INSTR (BSR,               reg16,      regmem16,   none,       none,       0,       0x0fbd,    r,   no,  0,     O16)
INSTR (BSR,               reg32,      regmem32,   none,       none,       0,       0x0fbd,    r,   no,  0,     O32)
INSTR (BSR,               reg64,      regmem64,   none,       none,       0,       0x0fbd,    r,   no,  0,     O64)

INSTR (BSWAP,             reg32,      none,       none,       none,       0,       0x0fc8,    rv,  no,  0,     O32)
INSTR (BSWAP,             reg64,      none,       none,       none,       0,       0x0fc8,    rv,  no,  0,     O64)

INSTR (BT,                regmem16,   reg16,      none,       none,       0,       0x0fa3,    r,   no,  0,     O16)
INSTR (BT,                regmem32,   reg32,      none,       none,       0,       0x0fa3,    r,   no,  0,     O32)
INSTR (BT,                regmem64,   reg64,      none,       none,       0,       0x0fa3,    r,   no,  0,     O64)
INSTR (BT,                regmem16,   imm8,       none,       none,       0,       0x0fba,    r4,  ib,  0,     O16)
INSTR (BT,                regmem32,   imm8,       none,       none,       0,       0x0fba,    r4,  ib,  0,     O32)
INSTR (BT,                regmem64,   imm8,       none,       none,       0,       0x0fba,    r4,  ib,  0,     O64)

INSTR (BTC,               regmem16,   reg16,      none,       none,       0,       0x0fbb,    r,   no,  0,     O16 | PLOCK)
INSTR (BTC,               regmem32,   reg32,      none,       none,       0,       0x0fbb,    r,   no,  0,     O32 | PLOCK)
INSTR (BTC,               regmem64,   reg64,      none,       none,       0,       0x0fbb,    r,   no,  0,     O64 | PLOCK)
INSTR (BTC,               regmem16,   imm8,       none,       none,       0,       0x0fba,    r7,  ib,  0,     O16 | PLOCK)
INSTR (BTC,               regmem32,   imm8,       none,       none,       0,       0x0fba,    r7,  ib,  0,     O32 | PLOCK)
INSTR (BTC,               regmem64,   imm8,       none,       none,       0,       0x0fba,    r7,  ib,  0,     O64 | PLOCK)

INSTR (BTR,               regmem16,   reg16,      none,       none,       0,       0x0fb3,    r,   no,  0,     O16 | PLOCK)
INSTR (BTR,               regmem32,   reg32,      none,       none,       0,       0x0fb3,    r,   no,  0,     O32 | PLOCK)
INSTR (BTR,               regmem64,   reg64,      none,       none,       0,       0x0fb3,    r,   no,  0,     O64 | PLOCK)
INSTR (BTR,               regmem16,   imm8,       none,       none,       0,       0x0fba,    r6,  ib,  0,     O16 | PLOCK)
INSTR (BTR,               regmem32,   imm8,       none,       none,       0,       0x0fba,    r6,  ib,  0,     O32 | PLOCK)
INSTR (BTR,               regmem64,   imm8,       none,       none,       0,       0x0fba,    r6,  ib,  0,     O64 | PLOCK)

INSTR (BTS,               regmem16,   reg16,      none,       none,       0,       0x0fab,    r,   no,  0,     O16 | PLOCK)
INSTR (BTS,               regmem32,   reg32,      none,       none,       0,       0x0fab,    r,   no,  0,     O32 | PLOCK)
INSTR (BTS,               regmem64,   reg64,      none,       none,       0,       0x0fab,    r,   no,  0,     O64 | PLOCK)
INSTR (BTS,               regmem16,   imm8,       none,       none,       0,       0x0fba,    r5,  ib,  0,     O16 | PLOCK)
INSTR (BTS,               regmem32,   imm8,       none,       none,       0,       0x0fba,    r5,  ib,  0,     O32 | PLOCK)
INSTR (BTS,               regmem64,   imm8,       none,       none,       0,       0x0fba,    r5,  ib,  0,     O64 | PLOCK)

INSTR (BZHI,              reg32,      regmem32,   reg32vvvv,  none,       0x0200,  0xf5,      r,   no,  0,     PVEX)
INSTR (BZHI,              reg64,      regmem64,   reg64vvvv,  none,       0x0280,  0xf5,      r,   no,  0,     PVEX)

INSTR (CALL,              rel16off,   none,       none,       none,       0,       0xe8,      cw,  no,  0,     O16)
INSTR (CALL,              rel32off,   none,       none,       none,       0,       0xe8,      cd,  no,  0,     O32)
INSTR (CALL,              regmem16,   none,       none,       none,       0,       0xff,      r2,  no,  0,     O16)
INSTR (CALL,              regmem32,   none,       none,       none,       0,       0xff,      r2,  no,  0,     O32 | I64)
INSTR (CALL,              regmem64,   none,       none,       none,       0,       0xff,      r2,  no,  0,     O64 | D64)

INSTR (CALLFAR,           imm16,      rel16off,   none,       none,       0,       0x9a,      cw,  iw,  0,     O16 | I64 | FAR)
INSTR (CALLFAR,           imm16,      rel32off,   none,       none,       0,       0x9a,      cd,  iw,  0,     O32 | I64 | FAR)
INSTR (CALLFAR,           mem16,      none,       none,       none,       0,       0xff,      r3,  no,  0,     O16)
INSTR (CALLFAR,           mem32,      none,       none,       none,       0,       0xff,      r3,  no,  0,     O32)

INSTR (CBW,               none,       none,       none,       none,       0,       0x98,      no,  no,  0,     O16)

INSTR (CWDE,              none,       none,       none,       none,       0,       0x98,      no,  no,  0,     O32)

INSTR (CDQE,              none,       none,       none,       none,       0,       0x98,      no,  no,  0,     O64)

INSTR (CWD,               none,       none,       none,       none,       0,       0x99,      no,  no,  0,     O16)

INSTR (CDQ,               none,       none,       none,       none,       0,       0x99,      no,  no,  0,     O32)

INSTR (CQO,               none,       none,       none,       none,       0,       0x99,      no,  no,  0,     O64)

INSTR (CLAC,              none,       none,       none,       none,       0,       0x0f01,    no,  no,  0xca,  SFX)

INSTR (CLC,               none,       none,       none,       none,       0,       0xf8,      no,  no,  0,     0)

INSTR (CLD,               none,       none,       none,       none,       0,       0xfc,      no,  no,  0,     0)

INSTR (XSAVE,             mem,        none,       none,       none,       0,       0x0fae,    r4,  no,  0,     0)

INSTR (FXSAVE,            mem,        none,       none,       none,       0,       0x0fae,    r0,  no,  0,     0)

INSTR (XSAVEC,            mem,        none,       none,       none,       0,       0x0fc7,    r4,  no,  0,     0)

INSTR (CLWB,              mem,        none,       none,       none,       0,       0x0fae,    r6,  no,  0,     P66)

INSTR (XSAVEOPT,          mem,        none,       none,       none,       0,       0x0fae,    r6,  no,  0,     0)

INSTR (XSAVES,            mem,        none,       none,       none,       0,       0x0fc7,    r5,  no,  0,     0)

INSTR (XRSTOR,            mem,        none,       none,       none,       0,       0x0fae,    r5,  no,  0,     0)

INSTR (FXRSTOR,           mem,        none,       none,       none,       0,       0x0fae,    r1,  no,  0,     0)

INSTR (XRSTORS,           mem,        none,       none,       none,       0,       0x0fc7,    r3,  no,  0,     0)

INSTR (LDMXCSR,           mem32,      none,       none,       none,       0,       0x0fae,    r2,  no,  0,     0)

INSTR (VLDMXCSR,          mem32,      none,       none,       none,       0x01e8,  0xae,      r2,  no,  0,     PVEX)

INSTR (STMXCSR,           mem32,      none,       none,       none,       0,       0x0fae,    r3,  no,  0,     0)

INSTR (VSTMXCSR,          mem32,      none,       none,       none,       0x01e8,  0xae,      r3,  no,  0,     PVEX)

INSTR (CLFLUSHOPT,        mem,        none,       none,       none,       0,       0x0fae,    r7,  no,  0,     P66)

INSTR (CLFLUSH,           mem,        none,       none,       none,       0,       0x0fae,    r7,  no,  0,     0)

INSTR (CMC,               none,       none,       none,       none,       0,       0xf5,      no,  no,  0,     0)

INSTR (CMOVO,             reg16,      regmem16,   none,       none,       0,       0x0f40,    r,   no,  0,     O16)
INSTR (CMOVO,             reg32,      regmem32,   none,       none,       0,       0x0f40,    r,   no,  0,     O32)
INSTR (CMOVO,             reg64,      regmem64,   none,       none,       0,       0x0f40,    r,   no,  0,     O64)

INSTR (CMOVNO,            reg16,      regmem16,   none,       none,       0,       0x0f41,    r,   no,  0,     O16)
INSTR (CMOVNO,            reg32,      regmem32,   none,       none,       0,       0x0f41,    r,   no,  0,     O32)
INSTR (CMOVNO,            reg64,      regmem64,   none,       none,       0,       0x0f41,    r,   no,  0,     O64)

INSTR (CMOVC,             reg16,      regmem16,   none,       none,       0,       0x0f42,    r,   no,  0,     O16)
INSTR (CMOVC,             reg32,      regmem32,   none,       none,       0,       0x0f42,    r,   no,  0,     O32)
INSTR (CMOVC,             reg64,      regmem64,   none,       none,       0,       0x0f42,    r,   no,  0,     O64)

INSTR (CMOVB,             reg16,      regmem16,   none,       none,       0,       0x0f42,    r,   no,  0,     O16)
INSTR (CMOVB,             reg32,      regmem32,   none,       none,       0,       0x0f42,    r,   no,  0,     O32)
INSTR (CMOVB,             reg64,      regmem64,   none,       none,       0,       0x0f42,    r,   no,  0,     O64)

INSTR (CMOVNAE,           reg16,      regmem16,   none,       none,       0,       0x0f42,    r,   no,  0,     O16)
INSTR (CMOVNAE,           reg32,      regmem32,   none,       none,       0,       0x0f42,    r,   no,  0,     O32)
INSTR (CMOVNAE,           reg64,      regmem64,   none,       none,       0,       0x0f42,    r,   no,  0,     O64)

INSTR (CMOVNC,            reg16,      regmem16,   none,       none,       0,       0x0f43,    r,   no,  0,     O16)
INSTR (CMOVNC,            reg32,      regmem32,   none,       none,       0,       0x0f43,    r,   no,  0,     O32)
INSTR (CMOVNC,            reg64,      regmem64,   none,       none,       0,       0x0f43,    r,   no,  0,     O64)

INSTR (CMOVNB,            reg16,      regmem16,   none,       none,       0,       0x0f43,    r,   no,  0,     O16)
INSTR (CMOVNB,            reg32,      regmem32,   none,       none,       0,       0x0f43,    r,   no,  0,     O32)
INSTR (CMOVNB,            reg64,      regmem64,   none,       none,       0,       0x0f43,    r,   no,  0,     O64)

INSTR (CMOVAE,            reg16,      regmem16,   none,       none,       0,       0x0f43,    r,   no,  0,     O16)
INSTR (CMOVAE,            reg32,      regmem32,   none,       none,       0,       0x0f43,    r,   no,  0,     O32)
INSTR (CMOVAE,            reg64,      regmem64,   none,       none,       0,       0x0f43,    r,   no,  0,     O64)

INSTR (CMOVZ,             reg16,      regmem16,   none,       none,       0,       0x0f44,    r,   no,  0,     O16)
INSTR (CMOVZ,             reg32,      regmem32,   none,       none,       0,       0x0f44,    r,   no,  0,     O32)
INSTR (CMOVZ,             reg64,      regmem64,   none,       none,       0,       0x0f44,    r,   no,  0,     O64)

INSTR (CMOVE,             reg16,      regmem16,   none,       none,       0,       0x0f44,    r,   no,  0,     O16)
INSTR (CMOVE,             reg32,      regmem32,   none,       none,       0,       0x0f44,    r,   no,  0,     O32)
INSTR (CMOVE,             reg64,      regmem64,   none,       none,       0,       0x0f44,    r,   no,  0,     O64)

INSTR (CMOVNZ,            reg16,      regmem16,   none,       none,       0,       0x0f45,    r,   no,  0,     O16)
INSTR (CMOVNZ,            reg32,      regmem32,   none,       none,       0,       0x0f45,    r,   no,  0,     O32)
INSTR (CMOVNZ,            reg64,      regmem64,   none,       none,       0,       0x0f45,    r,   no,  0,     O64)

INSTR (CMOVNE,            reg16,      regmem16,   none,       none,       0,       0x0f45,    r,   no,  0,     O16)
INSTR (CMOVNE,            reg32,      regmem32,   none,       none,       0,       0x0f45,    r,   no,  0,     O32)
INSTR (CMOVNE,            reg64,      regmem64,   none,       none,       0,       0x0f45,    r,   no,  0,     O64)

INSTR (CMOVNA,            reg16,      regmem16,   none,       none,       0,       0x0f46,    r,   no,  0,     O16)
INSTR (CMOVNA,            reg32,      regmem32,   none,       none,       0,       0x0f46,    r,   no,  0,     O32)
INSTR (CMOVNA,            reg64,      regmem64,   none,       none,       0,       0x0f46,    r,   no,  0,     O64)

INSTR (CMOVBE,            reg16,      regmem16,   none,       none,       0,       0x0f46,    r,   no,  0,     O16)
INSTR (CMOVBE,            reg32,      regmem32,   none,       none,       0,       0x0f46,    r,   no,  0,     O32)
INSTR (CMOVBE,            reg64,      regmem64,   none,       none,       0,       0x0f46,    r,   no,  0,     O64)

INSTR (CMOVA,             reg16,      regmem16,   none,       none,       0,       0x0f47,    r,   no,  0,     O16)
INSTR (CMOVA,             reg32,      regmem32,   none,       none,       0,       0x0f47,    r,   no,  0,     O32)
INSTR (CMOVA,             reg64,      regmem64,   none,       none,       0,       0x0f47,    r,   no,  0,     O64)

INSTR (CMOVNBE,           reg16,      regmem16,   none,       none,       0,       0x0f47,    r,   no,  0,     O16)
INSTR (CMOVNBE,           reg32,      regmem32,   none,       none,       0,       0x0f47,    r,   no,  0,     O32)
INSTR (CMOVNBE,           reg64,      regmem64,   none,       none,       0,       0x0f47,    r,   no,  0,     O64)

INSTR (CMOVS,             reg16,      regmem16,   none,       none,       0,       0x0f48,    r,   no,  0,     O16)
INSTR (CMOVS,             reg32,      regmem32,   none,       none,       0,       0x0f48,    r,   no,  0,     O32)
INSTR (CMOVS,             reg64,      regmem64,   none,       none,       0,       0x0f48,    r,   no,  0,     O64)

INSTR (CMOVNS,            reg16,      regmem16,   none,       none,       0,       0x0f49,    r,   no,  0,     O16)
INSTR (CMOVNS,            reg32,      regmem32,   none,       none,       0,       0x0f49,    r,   no,  0,     O32)
INSTR (CMOVNS,            reg64,      regmem64,   none,       none,       0,       0x0f49,    r,   no,  0,     O64)

INSTR (CMOVPE,            reg16,      regmem16,   none,       none,       0,       0x0f4a,    r,   no,  0,     O16)
INSTR (CMOVPE,            reg32,      regmem32,   none,       none,       0,       0x0f4a,    r,   no,  0,     O32)
INSTR (CMOVPE,            reg64,      regmem64,   none,       none,       0,       0x0f4a,    r,   no,  0,     O64)

INSTR (CMOVP,             reg16,      regmem16,   none,       none,       0,       0x0f4a,    r,   no,  0,     O16)
INSTR (CMOVP,             reg32,      regmem32,   none,       none,       0,       0x0f4a,    r,   no,  0,     O32)
INSTR (CMOVP,             reg64,      regmem64,   none,       none,       0,       0x0f4a,    r,   no,  0,     O64)

INSTR (CMOVPO,            reg16,      regmem16,   none,       none,       0,       0x0f4b,    r,   no,  0,     O16)
INSTR (CMOVPO,            reg32,      regmem32,   none,       none,       0,       0x0f4b,    r,   no,  0,     O32)
INSTR (CMOVPO,            reg64,      regmem64,   none,       none,       0,       0x0f4b,    r,   no,  0,     O64)

INSTR (CMOVNP,            reg16,      regmem16,   none,       none,       0,       0x0f4b,    r,   no,  0,     O16)
INSTR (CMOVNP,            reg32,      regmem32,   none,       none,       0,       0x0f4b,    r,   no,  0,     O32)
INSTR (CMOVNP,            reg64,      regmem64,   none,       none,       0,       0x0f4b,    r,   no,  0,     O64)

INSTR (CMOVL,             reg16,      regmem16,   none,       none,       0,       0x0f4c,    r,   no,  0,     O16)
INSTR (CMOVL,             reg32,      regmem32,   none,       none,       0,       0x0f4c,    r,   no,  0,     O32)
INSTR (CMOVL,             reg64,      regmem64,   none,       none,       0,       0x0f4c,    r,   no,  0,     O64)

INSTR (CMOVNGE,           reg16,      regmem16,   none,       none,       0,       0x0f4c,    r,   no,  0,     O16)
INSTR (CMOVNGE,           reg32,      regmem32,   none,       none,       0,       0x0f4c,    r,   no,  0,     O32)
INSTR (CMOVNGE,           reg64,      regmem64,   none,       none,       0,       0x0f4c,    r,   no,  0,     O64)

INSTR (CMOVNL,            reg16,      regmem16,   none,       none,       0,       0x0f4d,    r,   no,  0,     O16)
INSTR (CMOVNL,            reg32,      regmem32,   none,       none,       0,       0x0f4d,    r,   no,  0,     O32)
INSTR (CMOVNL,            reg64,      regmem64,   none,       none,       0,       0x0f4d,    r,   no,  0,     O64)

INSTR (CMOVGE,            reg16,      regmem16,   none,       none,       0,       0x0f4d,    r,   no,  0,     O16)
INSTR (CMOVGE,            reg32,      regmem32,   none,       none,       0,       0x0f4d,    r,   no,  0,     O32)
INSTR (CMOVGE,            reg64,      regmem64,   none,       none,       0,       0x0f4d,    r,   no,  0,     O64)

INSTR (CMOVNG,            reg16,      regmem16,   none,       none,       0,       0x0f4e,    r,   no,  0,     O16)
INSTR (CMOVNG,            reg32,      regmem32,   none,       none,       0,       0x0f4e,    r,   no,  0,     O32)
INSTR (CMOVNG,            reg64,      regmem64,   none,       none,       0,       0x0f4e,    r,   no,  0,     O64)

INSTR (CMOVLE,            reg16,      regmem16,   none,       none,       0,       0x0f4e,    r,   no,  0,     O16)
INSTR (CMOVLE,            reg32,      regmem32,   none,       none,       0,       0x0f4e,    r,   no,  0,     O32)
INSTR (CMOVLE,            reg64,      regmem64,   none,       none,       0,       0x0f4e,    r,   no,  0,     O64)

INSTR (CMOVG,             reg16,      regmem16,   none,       none,       0,       0x0f4f,    r,   no,  0,     O16)
INSTR (CMOVG,             reg32,      regmem32,   none,       none,       0,       0x0f4f,    r,   no,  0,     O32)
INSTR (CMOVG,             reg64,      regmem64,   none,       none,       0,       0x0f4f,    r,   no,  0,     O64)

INSTR (CMOVNLE,           reg16,      regmem16,   none,       none,       0,       0x0f4f,    r,   no,  0,     O16)
INSTR (CMOVNLE,           reg32,      regmem32,   none,       none,       0,       0x0f4f,    r,   no,  0,     O32)
INSTR (CMOVNLE,           reg64,      regmem64,   none,       none,       0,       0x0f4f,    r,   no,  0,     O64)

INSTR (CMP,               al,         imm8,       none,       none,       0,       0x3c,      ib,  no,  0,     0)
INSTR (CMP,               regmem8,    imm8,       none,       none,       0,       0x80,      r7,  ib,  0,     0)
INSTR (CMP,               regmem8,    imm8,       none,       none,       0,       0x82,      r7,  ib,  0,     I64)
INSTR (CMP,               regmem16,   simm8,      none,       none,       0,       0x83,      r7,  ib,  0,     O16)
INSTR (CMP,               regmem32,   simm8,      none,       none,       0,       0x83,      r7,  ib,  0,     O32)
INSTR (CMP,               regmem64,   simm8,      none,       none,       0,       0x83,      r7,  ib,  0,     O64)
INSTR (CMP,               ax,         imm16,      none,       none,       0,       0x3d,      iw,  no,  0,     O16)
INSTR (CMP,               regmem16,   imm16,      none,       none,       0,       0x81,      r7,  iw,  0,     O16)
INSTR (CMP,               eax,        imm32,      none,       none,       0,       0x3d,      id,  no,  0,     O32)
INSTR (CMP,               regmem32,   imm32,      none,       none,       0,       0x81,      r7,  id,  0,     O32)
INSTR (CMP,               rax,        imm32,      none,       none,       0,       0x3d,      id,  no,  0,     O64)
INSTR (CMP,               regmem64,   simm32,     none,       none,       0,       0x81,      r7,  id,  0,     O64)
INSTR (CMP,               regmem8,    reg8,       none,       none,       0,       0x38,      r,   no,  0,     0)
INSTR (CMP,               regmem16,   reg16,      none,       none,       0,       0x39,      r,   no,  0,     O16)
INSTR (CMP,               regmem32,   reg32,      none,       none,       0,       0x39,      r,   no,  0,     O32)
INSTR (CMP,               regmem64,   reg64,      none,       none,       0,       0x39,      r,   no,  0,     O64)
INSTR (CMP,               reg8,       regmem8,    none,       none,       0,       0x3a,      r,   no,  0,     0)
INSTR (CMP,               reg16,      regmem16,   none,       none,       0,       0x3b,      r,   no,  0,     O16)
INSTR (CMP,               reg32,      regmem32,   none,       none,       0,       0x3b,      r,   no,  0,     O32)
INSTR (CMP,               reg64,      regmem64,   none,       none,       0,       0x3b,      r,   no,  0,     O64)

INSTR (CMPSB,             none,       none,       none,       none,       0,       0xa6,      no,  no,  0,     PREPE | PREPNE)

INSTR (CMPSW,             none,       none,       none,       none,       0,       0xa7,      no,  no,  0,     O16 | PREPE | PREPNE)

INSTR (CMPSD,             none,       none,       none,       none,       0,       0xa7,      no,  no,  0,     O32 | PREPE | PREPNE)
INSTR (CMPSD,             xmm,        xmmmem64,   imm8,       none,       0,       0x0fc2,    r,   ib,  0,     PF2)

INSTR (CMPSQ,             none,       none,       none,       none,       0,       0xa7,      no,  no,  0,     O64 | PREPE | PREPNE)

INSTR (CMPXCHG,           regmem8,    reg8,       none,       none,       0,       0x0fb0,    r,   no,  0,     PLOCK)
INSTR (CMPXCHG,           regmem16,   reg16,      none,       none,       0,       0x0fb1,    r,   no,  0,     O16 | PLOCK)
INSTR (CMPXCHG,           regmem32,   reg32,      none,       none,       0,       0x0fb1,    r,   no,  0,     O32 | PLOCK)
INSTR (CMPXCHG,           regmem64,   reg64,      none,       none,       0,       0x0fb1,    r,   no,  0,     O64 | PLOCK)

INSTR (CMPXCHG8B,         mem,        none,       none,       none,       0,       0x0fc7,    r1,  no,  0,     O32 | PLOCK)

INSTR (CMPXCHG16B,        mem,        none,       none,       none,       0,       0x0fc7,    r1,  no,  0,     O64 | PLOCK)

INSTR (CPUID,             none,       none,       none,       none,       0,       0x0fa2,    no,  no,  0,     0)

INSTR (CRC32,             reg32,      regmem8,    none,       none,       0,       0x0f38f0,  r,   no,  0,     PF2)
INSTR (CRC32,             reg32,      regmem16,   none,       none,       0,       0x0f38f1,  r,   no,  0,     O16 | PF2)
INSTR (CRC32,             reg32,      regmem32,   none,       none,       0,       0x0f38f1,  r,   no,  0,     O32 | PF2)
INSTR (CRC32,             reg64,      regmem8,    none,       none,       0,       0x0f38f0,  r,   no,  0,     PF2)
INSTR (CRC32,             reg64,      regmem64,   none,       none,       0,       0x0f38f1,  r,   no,  0,     O64 | PF2)

INSTR (DAA,               none,       none,       none,       none,       0,       0x27,      no,  no,  0,     I64)

INSTR (DAS,               none,       none,       none,       none,       0,       0x2f,      no,  no,  0,     I64)

INSTR (DEC,               reg16,      none,       none,       none,       0,       0x48,      rv,  no,  0,     O16 | I64)
INSTR (DEC,               reg32,      none,       none,       none,       0,       0x48,      rv,  no,  0,     O32 | I64)
INSTR (DEC,               regmem8,    none,       none,       none,       0,       0xfe,      r1,  no,  0,     PLOCK)
INSTR (DEC,               regmem16,   none,       none,       none,       0,       0xff,      r1,  no,  0,     O16 | PLOCK)
INSTR (DEC,               regmem32,   none,       none,       none,       0,       0xff,      r1,  no,  0,     O32 | PLOCK)
INSTR (DEC,               regmem64,   none,       none,       none,       0,       0xff,      r1,  no,  0,     O64 | PLOCK)

INSTR (ENTER,             rel16off,   imm8,       none,       none,       0,       0xc8,      cw,  ib,  0,     0)

INSTR (DIV,               regmem8,    none,       none,       none,       0,       0xf6,      r6,  no,  0,     0)
INSTR (DIV,               regmem16,   none,       none,       none,       0,       0xf7,      r6,  no,  0,     O16)
INSTR (DIV,               regmem32,   none,       none,       none,       0,       0xf7,      r6,  no,  0,     O32)
INSTR (DIV,               regmem64,   none,       none,       none,       0,       0xf7,      r6,  no,  0,     O64)

INSTR (IDIV,              regmem8,    none,       none,       none,       0,       0xf6,      r7,  no,  0,     0)
INSTR (IDIV,              regmem16,   none,       none,       none,       0,       0xf7,      r7,  no,  0,     O16)
INSTR (IDIV,              regmem32,   none,       none,       none,       0,       0xf7,      r7,  no,  0,     O32)
INSTR (IDIV,              regmem64,   none,       none,       none,       0,       0xf7,      r7,  no,  0,     O64)

INSTR (IMUL,              regmem8,    none,       none,       none,       0,       0xf6,      r5,  no,  0,     0)
INSTR (IMUL,              regmem16,   none,       none,       none,       0,       0xf7,      r5,  no,  0,     O16)
INSTR (IMUL,              regmem32,   none,       none,       none,       0,       0xf7,      r5,  no,  0,     O32)
INSTR (IMUL,              regmem64,   none,       none,       none,       0,       0xf7,      r5,  no,  0,     O64)
INSTR (IMUL,              reg16,      regmem16,   none,       none,       0,       0x0faf,    r,   no,  0,     O16)
INSTR (IMUL,              reg32,      regmem32,   none,       none,       0,       0x0faf,    r,   no,  0,     O32)
INSTR (IMUL,              reg64,      regmem64,   none,       none,       0,       0x0faf,    r,   no,  0,     O64)
INSTR (IMUL,              reg16,      regmem16,   simm8,      none,       0,       0x6b,      r,   ib,  0,     O16)
INSTR (IMUL,              reg32,      regmem32,   simm8,      none,       0,       0x6b,      r,   ib,  0,     O32)
INSTR (IMUL,              reg64,      regmem64,   simm8,      none,       0,       0x6b,      r,   ib,  0,     O64)
INSTR (IMUL,              reg16,      regmem16,   simm16,     none,       0,       0x69,      r,   iw,  0,     O16)
INSTR (IMUL,              reg32,      regmem32,   simm32,     none,       0,       0x69,      r,   id,  0,     O32)
INSTR (IMUL,              reg64,      regmem64,   simm32,     none,       0,       0x69,      r,   id,  0,     O64)

INSTR (IN,                al,         imm8,       none,       none,       0,       0xe4,      ib,  no,  0,     0)
INSTR (IN,                ax,         imm8,       none,       none,       0,       0xe5,      ib,  no,  0,     O16)
INSTR (IN,                eax,        imm8,       none,       none,       0,       0xe5,      ib,  no,  0,     O32)
INSTR (IN,                al,         dx,         none,       none,       0,       0xec,      no,  no,  0,     0)
INSTR (IN,                ax,         dx,         none,       none,       0,       0xed,      no,  no,  0,     O16)
INSTR (IN,                eax,        dx,         none,       none,       0,       0xed,      no,  no,  0,     O32)

INSTR (INC,               reg16,      none,       none,       none,       0,       0x40,      rv,  no,  0,     O16 | I64)
INSTR (INC,               reg32,      none,       none,       none,       0,       0x40,      rv,  no,  0,     O32 | I64)
INSTR (INC,               regmem8,    none,       none,       none,       0,       0xfe,      r0,  no,  0,     PLOCK)
INSTR (INC,               regmem16,   none,       none,       none,       0,       0xff,      r0,  no,  0,     O16 | PLOCK)
INSTR (INC,               regmem32,   none,       none,       none,       0,       0xff,      r0,  no,  0,     O32 | PLOCK)
INSTR (INC,               regmem64,   none,       none,       none,       0,       0xff,      r0,  no,  0,     O64 | PLOCK)

INSTR (INSB,              none,       none,       none,       none,       0,       0x6c,      no,  no,  0,     PREP)

INSTR (INSW,              none,       none,       none,       none,       0,       0x6d,      no,  no,  0,     O16 | PREP)

INSTR (INSD,              none,       none,       none,       none,       0,       0x6d,      no,  no,  0,     O32 | PREP)

INSTR (INT,               imm8,       none,       none,       none,       0,       0xcd,      ib,  no,  0,     0)

INSTR (INTO,              none,       none,       none,       none,       0,       0xce,      no,  no,  0,     I64)

INSTR (JO,                rel8off,    none,       none,       none,       0,       0x70,      cb,  no,  0,     0)
INSTR (JO,                rel16off,   none,       none,       none,       0,       0x0f80,    cw,  no,  0,     O16)
INSTR (JO,                rel32off,   none,       none,       none,       0,       0x0f80,    cd,  no,  0,     O32)

INSTR (JNO,               rel8off,    none,       none,       none,       0,       0x71,      cb,  no,  0,     0)
INSTR (JNO,               rel16off,   none,       none,       none,       0,       0x0f81,    cw,  no,  0,     O16)
INSTR (JNO,               rel32off,   none,       none,       none,       0,       0x0f81,    cd,  no,  0,     O32)

INSTR (JC,                rel8off,    none,       none,       none,       0,       0x72,      cb,  no,  0,     0)
INSTR (JC,                rel16off,   none,       none,       none,       0,       0x0f82,    cw,  no,  0,     O16)
INSTR (JC,                rel32off,   none,       none,       none,       0,       0x0f82,    cd,  no,  0,     O32)

INSTR (JB,                rel8off,    none,       none,       none,       0,       0x72,      cb,  no,  0,     0)
INSTR (JB,                rel16off,   none,       none,       none,       0,       0x0f82,    cw,  no,  0,     O16)
INSTR (JB,                rel32off,   none,       none,       none,       0,       0x0f82,    cd,  no,  0,     O32)

INSTR (JNAE,              rel8off,    none,       none,       none,       0,       0x72,      cb,  no,  0,     0)
INSTR (JNAE,              rel16off,   none,       none,       none,       0,       0x0f82,    cw,  no,  0,     O16)
INSTR (JNAE,              rel32off,   none,       none,       none,       0,       0x0f82,    cd,  no,  0,     O32)

INSTR (JNC,               rel8off,    none,       none,       none,       0,       0x73,      cb,  no,  0,     0)
INSTR (JNC,               rel16off,   none,       none,       none,       0,       0x0f83,    cw,  no,  0,     O16)
INSTR (JNC,               rel32off,   none,       none,       none,       0,       0x0f83,    cd,  no,  0,     O32)

INSTR (JNB,               rel8off,    none,       none,       none,       0,       0x73,      cb,  no,  0,     0)
INSTR (JNB,               rel16off,   none,       none,       none,       0,       0x0f83,    cw,  no,  0,     O16)
INSTR (JNB,               rel32off,   none,       none,       none,       0,       0x0f83,    cd,  no,  0,     O32)

INSTR (JAE,               rel8off,    none,       none,       none,       0,       0x73,      cb,  no,  0,     0)
INSTR (JAE,               rel16off,   none,       none,       none,       0,       0x0f83,    cw,  no,  0,     O16)
INSTR (JAE,               rel32off,   none,       none,       none,       0,       0x0f83,    cd,  no,  0,     O32)

INSTR (JZ,                rel8off,    none,       none,       none,       0,       0x74,      cb,  no,  0,     0)
INSTR (JZ,                rel16off,   none,       none,       none,       0,       0x0f84,    cw,  no,  0,     O16)
INSTR (JZ,                rel32off,   none,       none,       none,       0,       0x0f84,    cd,  no,  0,     O32)

INSTR (JE,                rel8off,    none,       none,       none,       0,       0x74,      cb,  no,  0,     0)
INSTR (JE,                rel16off,   none,       none,       none,       0,       0x0f84,    cw,  no,  0,     O16)
INSTR (JE,                rel32off,   none,       none,       none,       0,       0x0f84,    cd,  no,  0,     O32)

INSTR (JNZ,               rel8off,    none,       none,       none,       0,       0x75,      cb,  no,  0,     0)
INSTR (JNZ,               rel16off,   none,       none,       none,       0,       0x0f85,    cw,  no,  0,     O16)
INSTR (JNZ,               rel32off,   none,       none,       none,       0,       0x0f85,    cd,  no,  0,     O32)

INSTR (JNE,               rel8off,    none,       none,       none,       0,       0x75,      cb,  no,  0,     0)
INSTR (JNE,               rel16off,   none,       none,       none,       0,       0x0f85,    cw,  no,  0,     O16)
INSTR (JNE,               rel32off,   none,       none,       none,       0,       0x0f85,    cd,  no,  0,     O32)

INSTR (JNA,               rel8off,    none,       none,       none,       0,       0x76,      cb,  no,  0,     0)
INSTR (JNA,               rel16off,   none,       none,       none,       0,       0x0f86,    cw,  no,  0,     O16)
INSTR (JNA,               rel32off,   none,       none,       none,       0,       0x0f86,    cd,  no,  0,     O32)

INSTR (JBE,               rel8off,    none,       none,       none,       0,       0x76,      cb,  no,  0,     0)
INSTR (JBE,               rel16off,   none,       none,       none,       0,       0x0f86,    cw,  no,  0,     O16)
INSTR (JBE,               rel32off,   none,       none,       none,       0,       0x0f86,    cd,  no,  0,     O32)

INSTR (JA,                rel8off,    none,       none,       none,       0,       0x77,      cb,  no,  0,     0)
INSTR (JA,                rel16off,   none,       none,       none,       0,       0x0f87,    cw,  no,  0,     O16)
INSTR (JA,                rel32off,   none,       none,       none,       0,       0x0f87,    cd,  no,  0,     O32)

INSTR (JNBE,              rel8off,    none,       none,       none,       0,       0x77,      cb,  no,  0,     0)
INSTR (JNBE,              rel16off,   none,       none,       none,       0,       0x0f87,    cw,  no,  0,     O16)
INSTR (JNBE,              rel32off,   none,       none,       none,       0,       0x0f87,    cd,  no,  0,     O32)

INSTR (JS,                rel8off,    none,       none,       none,       0,       0x78,      cb,  no,  0,     0)
INSTR (JS,                rel16off,   none,       none,       none,       0,       0x0f88,    cw,  no,  0,     O16)
INSTR (JS,                rel32off,   none,       none,       none,       0,       0x0f88,    cd,  no,  0,     O32)

INSTR (JNS,               rel8off,    none,       none,       none,       0,       0x79,      cb,  no,  0,     0)
INSTR (JNS,               rel16off,   none,       none,       none,       0,       0x0f89,    cw,  no,  0,     O16)
INSTR (JNS,               rel32off,   none,       none,       none,       0,       0x0f89,    cd,  no,  0,     O32)

INSTR (JPE,               rel8off,    none,       none,       none,       0,       0x7a,      cb,  no,  0,     0)
INSTR (JPE,               rel16off,   none,       none,       none,       0,       0x0f8a,    cw,  no,  0,     O16)
INSTR (JPE,               rel32off,   none,       none,       none,       0,       0x0f8a,    cd,  no,  0,     O32)

INSTR (JP,                rel8off,    none,       none,       none,       0,       0x7a,      cb,  no,  0,     0)
INSTR (JP,                rel16off,   none,       none,       none,       0,       0x0f8a,    cw,  no,  0,     O16)
INSTR (JP,                rel32off,   none,       none,       none,       0,       0x0f8a,    cd,  no,  0,     O32)

INSTR (JPO,               rel8off,    none,       none,       none,       0,       0x7b,      cb,  no,  0,     0)
INSTR (JPO,               rel16off,   none,       none,       none,       0,       0x0f8b,    cw,  no,  0,     O16)
INSTR (JPO,               rel32off,   none,       none,       none,       0,       0x0f8b,    cd,  no,  0,     O32)

INSTR (JNP,               rel8off,    none,       none,       none,       0,       0x7b,      cb,  no,  0,     0)
INSTR (JNP,               rel16off,   none,       none,       none,       0,       0x0f8b,    cw,  no,  0,     O16)
INSTR (JNP,               rel32off,   none,       none,       none,       0,       0x0f8b,    cd,  no,  0,     O32)

INSTR (JL,                rel8off,    none,       none,       none,       0,       0x7c,      cb,  no,  0,     0)
INSTR (JL,                rel16off,   none,       none,       none,       0,       0x0f8c,    cw,  no,  0,     O16)
INSTR (JL,                rel32off,   none,       none,       none,       0,       0x0f8c,    cd,  no,  0,     O32)

INSTR (JNGE,              rel8off,    none,       none,       none,       0,       0x7c,      cb,  no,  0,     0)
INSTR (JNGE,              rel16off,   none,       none,       none,       0,       0x0f8c,    cw,  no,  0,     O16)
INSTR (JNGE,              rel32off,   none,       none,       none,       0,       0x0f8c,    cd,  no,  0,     O32)

INSTR (JNL,               rel8off,    none,       none,       none,       0,       0x7d,      cb,  no,  0,     0)
INSTR (JNL,               rel16off,   none,       none,       none,       0,       0x0f8d,    cw,  no,  0,     O16)
INSTR (JNL,               rel32off,   none,       none,       none,       0,       0x0f8d,    cd,  no,  0,     O32)

INSTR (JGE,               rel8off,    none,       none,       none,       0,       0x7d,      cb,  no,  0,     0)
INSTR (JGE,               rel16off,   none,       none,       none,       0,       0x0f8d,    cw,  no,  0,     O16)
INSTR (JGE,               rel32off,   none,       none,       none,       0,       0x0f8d,    cd,  no,  0,     O32)

INSTR (JNG,               rel8off,    none,       none,       none,       0,       0x7e,      cb,  no,  0,     0)
INSTR (JNG,               rel16off,   none,       none,       none,       0,       0x0f8e,    cw,  no,  0,     O16)
INSTR (JNG,               rel32off,   none,       none,       none,       0,       0x0f8e,    cd,  no,  0,     O32)

INSTR (JLE,               rel8off,    none,       none,       none,       0,       0x7e,      cb,  no,  0,     0)
INSTR (JLE,               rel16off,   none,       none,       none,       0,       0x0f8e,    cw,  no,  0,     O16)
INSTR (JLE,               rel32off,   none,       none,       none,       0,       0x0f8e,    cd,  no,  0,     O32)

INSTR (JG,                rel8off,    none,       none,       none,       0,       0x7f,      cb,  no,  0,     0)
INSTR (JG,                rel16off,   none,       none,       none,       0,       0x0f8f,    cw,  no,  0,     O16)
INSTR (JG,                rel32off,   none,       none,       none,       0,       0x0f8f,    cd,  no,  0,     O32)

INSTR (JNLE,              rel8off,    none,       none,       none,       0,       0x7f,      cb,  no,  0,     0)
INSTR (JNLE,              rel16off,   none,       none,       none,       0,       0x0f8f,    cw,  no,  0,     O16)
INSTR (JNLE,              rel32off,   none,       none,       none,       0,       0x0f8f,    cd,  no,  0,     O32)

INSTR (JCXZ,              rel8off,    none,       none,       none,       0,       0xe3,      cb,  no,  0,     O16 | I64)

INSTR (JECXZ,             rel8off,    none,       none,       none,       0,       0xe3,      cb,  no,  0,     O32)

INSTR (JRCXZ,             rel8off,    none,       none,       none,       0,       0xe3,      cb,  no,  0,     O64 | D64)

INSTR (JMP,               rel8off,    none,       none,       none,       0,       0xeb,      cb,  no,  0,     0)
INSTR (JMP,               rel16off,   none,       none,       none,       0,       0xe9,      cw,  no,  0,     O16)
INSTR (JMP,               rel32off,   none,       none,       none,       0,       0xe9,      cd,  no,  0,     O32)
INSTR (JMP,               regmem16,   none,       none,       none,       0,       0xff,      r4,  no,  0,     O16)
INSTR (JMP,               regmem32,   none,       none,       none,       0,       0xff,      r4,  no,  0,     O32 | I64)
INSTR (JMP,               regmem64,   none,       none,       none,       0,       0xff,      r4,  no,  0,     O64 | D64)

INSTR (JMPFAR,            imm16,      rel16off,   none,       none,       0,       0xea,      cw,  iw,  0,     O16 | I64 | FAR)
INSTR (JMPFAR,            imm16,      rel32off,   none,       none,       0,       0xea,      cd,  iw,  0,     O32 | I64 | FAR)
INSTR (JMPFAR,            mem16,      none,       none,       none,       0,       0xff,      r5,  no,  0,     O16)
INSTR (JMPFAR,            mem32,      none,       none,       none,       0,       0xff,      r5,  no,  0,     O32)

INSTR (LAHF,              none,       none,       none,       none,       0,       0x9f,      no,  no,  0,     0)

INSTR (LDS,               reg16,      mem,        none,       none,       0,       0xc5,      r,   no,  0,     O16 | I64)
INSTR (LDS,               reg32,      mem,        none,       none,       0,       0xc5,      r,   no,  0,     O32 | I64)

INSTR (LES,               reg16,      mem,        none,       none,       0,       0xc4,      r,   no,  0,     O16 | I64)
INSTR (LES,               reg32,      mem,        none,       none,       0,       0xc4,      r,   no,  0,     O32 | I64)

INSTR (LFS,               reg16,      mem,        none,       none,       0,       0x0fb4,    r,   no,  0,     O16)
INSTR (LFS,               reg32,      mem,        none,       none,       0,       0x0fb4,    r,   no,  0,     O32)

INSTR (LGS,               reg16,      mem,        none,       none,       0,       0x0fb5,    r,   no,  0,     O16)
INSTR (LGS,               reg32,      mem,        none,       none,       0,       0x0fb5,    r,   no,  0,     O32)

INSTR (LSS,               reg16,      mem,        none,       none,       0,       0x0fb2,    r,   no,  0,     O16)
INSTR (LSS,               reg32,      mem,        none,       none,       0,       0x0fb2,    r,   no,  0,     O32)

INSTR (LEA,               reg16,      mem,        none,       none,       0,       0x8d,      r,   no,  0,     O16)
INSTR (LEA,               reg32,      mem,        none,       none,       0,       0x8d,      r,   no,  0,     O32)
INSTR (LEA,               reg64,      mem,        none,       none,       0,       0x8d,      r,   no,  0,     O64)

INSTR (LEAVE,             none,       none,       none,       none,       0,       0xc9,      no,  no,  0,     0)

INSTR (LFENCE,            none,       none,       none,       none,       0,       0x0fae,    no,  no,  0xe8,  SFX)

INSTR (LLWPCB,            reg32rm,    none,       none,       none,       0x0978,  0x12,      r0,  no,  0,     PXOP)
INSTR (LLWPCB,            reg64rm,    none,       none,       none,       0x09f8,  0x12,      r0,  no,  0,     PXOP)

INSTR (LODSB,             none,       none,       none,       none,       0,       0xac,      no,  no,  0,     PREP)

INSTR (LODSW,             none,       none,       none,       none,       0,       0xad,      no,  no,  0,     O16 | PREP)

INSTR (LODSD,             none,       none,       none,       none,       0,       0xad,      no,  no,  0,     O32 | PREP)

INSTR (LODSQ,             none,       none,       none,       none,       0,       0xad,      no,  no,  0,     O64 | PREP)

INSTR (LOOP,              rel8off,    none,       none,       none,       0,       0xe2,      cb,  no,  0,     0)

INSTR (LOOPE,             rel8off,    none,       none,       none,       0,       0xe1,      cb,  no,  0,     0)

INSTR (LOOPNE,            rel8off,    none,       none,       none,       0,       0xe0,      cb,  no,  0,     0)

INSTR (LOOPNZ,            rel8off,    none,       none,       none,       0,       0xe0,      cb,  no,  0,     0)

INSTR (LOOPZ,             rel8off,    none,       none,       none,       0,       0xe1,      cb,  no,  0,     0)

INSTR (MFENCE,            none,       none,       none,       none,       0,       0x0fae,    no,  no,  0xf0,  SFX)

INSTR (LWPINS,            reg32vvvv,  regmem32,   imm32,      none,       0x0a10,  0x12,      r0,  id,  0,     PXOP)
INSTR (LWPINS,            reg64vvvv,  regmem32,   imm32,      none,       0x0a90,  0x12,      r0,  id,  0,     PXOP)

INSTR (LWPVAL,            reg32vvvv,  regmem32,   imm32,      none,       0x0a10,  0x12,      r1,  id,  0,     PXOP)
INSTR (LWPVAL,            reg64vvvv,  regmem32,   imm32,      none,       0x0a90,  0x12,      r1,  id,  0,     PXOP)

INSTR (RSTORSSP,          mem64,      none,       none,       none,       0,       0x0f01,    r5,  no,  0,     PF3)

INSTR (MCOMMIT,           none,       none,       none,       none,       0,       0x0f01,    no,  no,  0xfa,  PF3 | SFX)

INSTR (MOV,               al,         moffset,    none,       none,       0,       0xa0,      no,  no,  0,     0)
INSTR (MOV,               ax,         moffset,    none,       none,       0,       0xa1,      no,  no,  0,     O16)
INSTR (MOV,               eax,        moffset,    none,       none,       0,       0xa1,      no,  no,  0,     O32)
INSTR (MOV,               rax,        moffset,    none,       none,       0,       0xa1,      no,  no,  0,     O64)
INSTR (MOV,               moffset,    al,         none,       none,       0,       0xa2,      no,  no,  0,     0)
INSTR (MOV,               moffset,    ax,         none,       none,       0,       0xa3,      no,  no,  0,     O16)
INSTR (MOV,               moffset,    eax,        none,       none,       0,       0xa3,      no,  no,  0,     O32)
INSTR (MOV,               moffset,    rax,        none,       none,       0,       0xa3,      no,  no,  0,     O64)
INSTR (MOV,               regmem8,    reg8,       none,       none,       0,       0x88,      r,   no,  0,     0)
INSTR (MOV,               regmem16,   reg16,      none,       none,       0,       0x89,      r,   no,  0,     O16)
INSTR (MOV,               regmem32,   reg32,      none,       none,       0,       0x89,      r,   no,  0,     O32)
INSTR (MOV,               regmem64,   reg64,      none,       none,       0,       0x89,      r,   no,  0,     O64)
INSTR (MOV,               reg8,       regmem8,    none,       none,       0,       0x8a,      r,   no,  0,     0)
INSTR (MOV,               reg16,      regmem16,   none,       none,       0,       0x8b,      r,   no,  0,     O16)
INSTR (MOV,               reg32,      regmem32,   none,       none,       0,       0x8b,      r,   no,  0,     O32)
INSTR (MOV,               reg64,      regmem64,   none,       none,       0,       0x8b,      r,   no,  0,     O64)
INSTR (MOV,               regmem16,   segreg,     none,       none,       0,       0x8c,      r,   no,  0,     O16)
INSTR (MOV,               regmem32,   segreg,     none,       none,       0,       0x8c,      r,   no,  0,     O32)
INSTR (MOV,               regmem64,   segreg,     none,       none,       0,       0x8c,      r,   no,  0,     O64)
INSTR (MOV,               segreg,     regmem16,   none,       none,       0,       0x8e,      r,   no,  0,     0)
INSTR (MOV,               reg8,       imm8,       none,       none,       0,       0xb0,      rv,  ib,  0,     0)
INSTR (MOV,               reg16,      imm16,      none,       none,       0,       0xb8,      rv,  iw,  0,     O16)
INSTR (MOV,               reg32,      imm32,      none,       none,       0,       0xb8,      rv,  id,  0,     O32)
INSTR (MOV,               regmem8,    imm8,       none,       none,       0,       0xc6,      r0,  ib,  0,     0)
INSTR (MOV,               regmem16,   imm16,      none,       none,       0,       0xc7,      r0,  iw,  0,     O16)
INSTR (MOV,               regmem32,   imm32,      none,       none,       0,       0xc7,      r0,  id,  0,     O32)
INSTR (MOV,               regmem64,   simm32,     none,       none,       0,       0xc7,      r0,  id,  0,     O64)
INSTR (MOV,               reg64,      imm64,      none,       none,       0,       0xb8,      rv,  iq,  0,     O64)
INSTR (MOV,               cr8,        regmem32,   none,       none,       0,       0x0f22,    r,   no,  0,     I64 | PF0)
INSTR (MOV,               cr8,        regmem64,   none,       none,       0,       0x0f22,    r,   no,  0,     D64 | PF0)
INSTR (MOV,               regmem32,   cr8,        none,       none,       0,       0x0f20,    r,   no,  0,     I64 | PF0)
INSTR (MOV,               regmem64,   cr8,        none,       none,       0,       0x0f20,    r,   no,  0,     D64 | PF0)
INSTR (MOV,               cr,         regmem32,   none,       none,       0,       0x0f22,    r,   no,  0,     I64)
INSTR (MOV,               cr,         regmem64,   none,       none,       0,       0x0f22,    r,   no,  0,     D64)
INSTR (MOV,               regmem32,   cr,         none,       none,       0,       0x0f20,    r,   no,  0,     I64)
INSTR (MOV,               regmem64,   cr,         none,       none,       0,       0x0f20,    r,   no,  0,     D64)
INSTR (MOV,               dr,         regmem32,   none,       none,       0,       0x0f21,    r,   no,  0,     I64)
INSTR (MOV,               dr,         regmem64,   none,       none,       0,       0x0f21,    r,   no,  0,     D64)
INSTR (MOV,               regmem32,   dr,         none,       none,       0,       0x0f23,    r,   no,  0,     I64)
INSTR (MOV,               regmem64,   dr,         none,       none,       0,       0x0f23,    r,   no,  0,     D64)

INSTR (MOVBE,             reg16,      mem16,      none,       none,       0,       0x0f38f0,  r,   no,  0,     O16)
INSTR (MOVBE,             reg32,      mem32,      none,       none,       0,       0x0f38f0,  r,   no,  0,     O32)
INSTR (MOVBE,             reg64,      mem64,      none,       none,       0,       0x0f38f0,  r,   no,  0,     O64)
INSTR (MOVBE,             mem16,      reg16,      none,       none,       0,       0x0f38f1,  r,   no,  0,     O16)
INSTR (MOVBE,             mem32,      reg32,      none,       none,       0,       0x0f38f1,  r,   no,  0,     O32)
INSTR (MOVBE,             mem64,      reg64,      none,       none,       0,       0x0f38f1,  r,   no,  0,     O64)

INSTR (MOVDQA,            xmm,        xmmmem128,  none,       none,       0,       0x0f6f,    r,   no,  0,     P66)
INSTR (MOVDQA,            xmmmem128,  xmm,        none,       none,       0,       0x0f7f,    r,   no,  0,     P66)

INSTR (VMOVDQA,           xmm,        xmmmem128,  none,       none,       0x0179,  0x6f,      r,   no,  0,     PVEX)
INSTR (VMOVDQA,           xmmmem128,  xmm,        none,       none,       0x0179,  0x7f,      r,   no,  0,     PVEX)
INSTR (VMOVDQA,           ymm,        ymmmem256,  none,       none,       0x017d,  0x6f,      r,   no,  0,     PVEX)
INSTR (VMOVDQA,           ymmmem256,  ymm,        none,       none,       0x017d,  0x7f,      r,   no,  0,     PVEX)

INSTR (MOVDQU,            xmm,        xmmmem128,  none,       none,       0,       0x0f6f,    r,   no,  0,     PF3)
INSTR (MOVDQU,            xmmmem128,  xmm,        none,       none,       0,       0x0f7f,    r,   no,  0,     PF3)

INSTR (VMOVDQU,           xmm,        xmmmem128,  none,       none,       0x017a,  0x6f,      r,   no,  0,     PVEX)
INSTR (VMOVDQU,           xmmmem128,  xmm,        none,       none,       0x017a,  0x7f,      r,   no,  0,     PVEX)
INSTR (VMOVDQU,           ymm,        ymmmem256,  none,       none,       0x017e,  0x6f,      r,   no,  0,     PVEX)
INSTR (VMOVDQU,           ymmmem256,  ymm,        none,       none,       0x017e,  0x7f,      r,   no,  0,     PVEX)

INSTR (MOVQ,              xmm,        xmmmem64,   none,       none,       0,       0x0f7e,    r,   no,  0,     PF3)
INSTR (MOVQ,              xmmmem64,   xmm,        none,       none,       0,       0x0fd6,    r,   no,  0,     P66)
INSTR (MOVQ,              mmx,        mmxmem64,   none,       none,       0,       0x0f6f,    r,   no,  0,     0)
INSTR (MOVQ,              mmxmem64,   mmx,        none,       none,       0,       0x0f7f,    r,   no,  0,     0)

INSTR (VMOVQ,             xmm,        regmem64,   none,       none,       0x01f9,  0x6e,      r,   no,  0,     PVEX)
INSTR (VMOVQ,             regmem64,   xmm,        none,       none,       0x01fd,  0x7e,      r,   no,  0,     PVEX)
INSTR (VMOVQ,             xmm,        xmmmem64,   none,       none,       0x017a,  0x7e,      r,   no,  0,     PVEX)
INSTR (VMOVQ,             xmmmem64,   xmm,        none,       none,       0x0179,  0xd6,      r,   no,  0,     PVEX)

INSTR (MOVD,              xmm,        regmem32,   none,       none,       0,       0x0f6e,    r,   no,  0,     O16 | O32 | P66)
INSTR (MOVD,              xmm,        regmem64,   none,       none,       0,       0x0f6e,    r,   no,  0,     O64 | P66)
INSTR (MOVD,              regmem32,   xmm,        none,       none,       0,       0x0f7e,    r,   no,  0,     O16 | O32 | P66)
INSTR (MOVD,              regmem64,   xmm,        none,       none,       0,       0x0f7e,    r,   no,  0,     O64 | P66)
INSTR (MOVD,              mmx,        regmem32,   none,       none,       0,       0x0f6e,    r,   no,  0,     O16 | O32)
INSTR (MOVD,              mmx,        regmem64,   none,       none,       0,       0x0f6e,    r,   no,  0,     O64)
INSTR (MOVD,              regmem32,   mmx,        none,       none,       0,       0x0f7e,    r,   no,  0,     O16 | O32)
INSTR (MOVD,              regmem64,   mmx,        none,       none,       0,       0x0f7e,    r,   no,  0,     O64)

INSTR (VMOVD,             xmm,        regmem32,   none,       none,       0x0179,  0x6e,      r,   no,  0,     PVEX)
INSTR (VMOVD,             regmem32,   xmm,        none,       none,       0x0179,  0x7e,      r,   no,  0,     PVEX)

INSTR (MOVMSKPD,          reg32,      xmmmem32,   none,       none,       0,       0x0f50,    r,   no,  0,     O16 | O32 | P66)
INSTR (MOVMSKPD,          reg64,      xmmmem32,   none,       none,       0,       0x0f50,    r,   no,  0,     O64 | P66)

INSTR (VMOVMSKPD,         reg32,      xmmmem32,   none,       none,       0x0179,  0x50,      r,   no,  0,     PVEX)
INSTR (VMOVMSKPD,         reg64,      xmmmem64,   none,       none,       0x017d,  0x50,      r,   no,  0,     PVEX)

INSTR (MOVMSKPS,          reg32,      xmmmem32,   none,       none,       0,       0x0f50,    r,   no,  0,     O16 | O32)
INSTR (MOVMSKPS,          reg64,      xmmmem32,   none,       none,       0,       0x0f50,    r,   no,  0,     O64)

INSTR (VMOVMSKPS,         reg32,      xmmmem32,   none,       none,       0x0178,  0x50,      r,   no,  0,     PVEX)
INSTR (VMOVMSKPS,         reg64,      xmmmem64,   none,       none,       0x017c,  0x50,      r,   no,  0,     PVEX)

INSTR (MOVNTI,            mem32,      reg32,      none,       none,       0,       0x0fc3,    r,   no,  0,     O16 | O32)
INSTR (MOVNTI,            mem64,      reg64,      none,       none,       0,       0x0fc3,    r,   no,  0,     O64)

INSTR (MOVSB,             none,       none,       none,       none,       0,       0xa4,      no,  no,  0,     PREP)

INSTR (MOVSW,             none,       none,       none,       none,       0,       0xa5,      no,  no,  0,     O16 | PREP)

INSTR (MOVSD,             none,       none,       none,       none,       0,       0xa5,      no,  no,  0,     O32 | PREP)
INSTR (MOVSD,             xmm,        mem64,      none,       none,       0,       0x0f10,    r,   no,  0,     PF2)
INSTR (MOVSD,             xmmmem64,   xmm,        none,       none,       0,       0x0f11,    r,   no,  0,     PF2)

INSTR (VMOVSD,            xmm,        mem64,      none,       none,       0x017b,  0x10,      r,   no,  0,     PVEX)
INSTR (VMOVSD,            mem64,      xmm,        none,       none,       0x017b,  0x11,      r,   no,  0,     PVEX)
INSTR (VMOVSD,            xmm,        xvvvv,      xmmmem64,   none,       0x0107,  0x10,      r,   no,  0,     PVEX)
INSTR (VMOVSD,            xmm,        xvvvv,      xmmmem64,   none,       0x0107,  0x11,      r,   no,  0,     PVEX)

INSTR (MOVSQ,             none,       none,       none,       none,       0,       0xa5,      no,  no,  0,     O64 | PREP)

INSTR (MOVSX,             reg16,      regmem8,    none,       none,       0,       0x0fbe,    r,   no,  0,     O16)
INSTR (MOVSX,             reg32,      regmem8,    none,       none,       0,       0x0fbe,    r,   no,  0,     O32)
INSTR (MOVSX,             reg64,      regmem8,    none,       none,       0,       0x0fbe,    r,   no,  0,     O64)
INSTR (MOVSX,             reg32,      regmem16,   none,       none,       0,       0x0fbf,    r,   no,  0,     O32)
INSTR (MOVSX,             reg64,      regmem16,   none,       none,       0,       0x0fbf,    r,   no,  0,     O64)

INSTR (MOVSXD,            reg64,      regmem32,   none,       none,       0,       0x63,      r,   no,  0,     O64)

INSTR (MOVZX,             reg16,      regmem8,    none,       none,       0,       0x0fb6,    r,   no,  0,     O16)
INSTR (MOVZX,             reg32,      regmem8,    none,       none,       0,       0x0fb6,    r,   no,  0,     O32)
INSTR (MOVZX,             reg64,      regmem8,    none,       none,       0,       0x0fb6,    r,   no,  0,     O64)
INSTR (MOVZX,             reg32,      regmem16,   none,       none,       0,       0x0fb7,    r,   no,  0,     O32)
INSTR (MOVZX,             reg64,      regmem16,   none,       none,       0,       0x0fb7,    r,   no,  0,     O64)

INSTR (MUL,               regmem8,    none,       none,       none,       0,       0xf6,      r4,  no,  0,     0)
INSTR (MUL,               regmem16,   none,       none,       none,       0,       0xf7,      r4,  no,  0,     O16)
INSTR (MUL,               regmem32,   none,       none,       none,       0,       0xf7,      r4,  no,  0,     O32)
INSTR (MUL,               regmem64,   none,       none,       none,       0,       0xf7,      r4,  no,  0,     O64)

INSTR (MULX,              reg32,      reg32vvvv,  regmem32,   none,       0x0203,  0xf6,      r,   no,  0,     PVEX)
INSTR (MULX,              reg64,      reg64vvvv,  regmem64,   none,       0x0283,  0xf6,      r,   no,  0,     PVEX)

INSTR (NEG,               regmem8,    none,       none,       none,       0,       0xf6,      r3,  no,  0,     PLOCK)
INSTR (NEG,               regmem16,   none,       none,       none,       0,       0xf7,      r3,  no,  0,     O16 | PLOCK)
INSTR (NEG,               regmem32,   none,       none,       none,       0,       0xf7,      r3,  no,  0,     O32 | PLOCK)
INSTR (NEG,               regmem64,   none,       none,       none,       0,       0xf7,      r3,  no,  0,     O64 | PLOCK)

INSTR (PAUSE,             none,       none,       none,       none,       0,       0x90,      no,  no,  0,     PF3)

INSTR (NOP,               none,       none,       none,       none,       0,       0x90,      no,  no,  0,     0)
INSTR (NOP,               regmem16,   none,       none,       none,       0,       0x0f1f,    r0,  no,  0,     O16)
INSTR (NOP,               regmem32,   none,       none,       none,       0,       0x0f1f,    r0,  no,  0,     O32)
INSTR (NOP,               regmem64,   none,       none,       none,       0,       0x0f1f,    r0,  no,  0,     O64)

INSTR (NOT,               regmem8,    none,       none,       none,       0,       0xf6,      r2,  no,  0,     PLOCK)
INSTR (NOT,               regmem16,   none,       none,       none,       0,       0xf7,      r2,  no,  0,     O16 | PLOCK)
INSTR (NOT,               regmem32,   none,       none,       none,       0,       0xf7,      r2,  no,  0,     O32 | PLOCK)
INSTR (NOT,               regmem64,   none,       none,       none,       0,       0xf7,      r2,  no,  0,     O64 | PLOCK)

INSTR (OR,                al,         imm8,       none,       none,       0,       0x0c,      ib,  no,  0,     0)
INSTR (OR,                regmem8,    imm8,       none,       none,       0,       0x80,      r1,  ib,  0,     PLOCK)
INSTR (OR,                regmem8,    imm8,       none,       none,       0,       0x82,      r1,  ib,  0,     I64 | PLOCK)
INSTR (OR,                regmem16,   simm8,      none,       none,       0,       0x83,      r1,  ib,  0,     O16 | PLOCK)
INSTR (OR,                regmem32,   simm8,      none,       none,       0,       0x83,      r1,  ib,  0,     O32 | PLOCK)
INSTR (OR,                regmem64,   simm8,      none,       none,       0,       0x83,      r1,  ib,  0,     O64 | PLOCK)
INSTR (OR,                ax,         imm16,      none,       none,       0,       0x0d,      iw,  no,  0,     O16)
INSTR (OR,                regmem16,   imm16,      none,       none,       0,       0x81,      r1,  iw,  0,     O16 | PLOCK)
INSTR (OR,                eax,        imm32,      none,       none,       0,       0x0d,      id,  no,  0,     O32)
INSTR (OR,                regmem32,   imm32,      none,       none,       0,       0x81,      r1,  id,  0,     O32 | PLOCK)
INSTR (OR,                rax,        simm32,     none,       none,       0,       0x0d,      id,  no,  0,     O64)
INSTR (OR,                regmem64,   simm32,     none,       none,       0,       0x81,      r1,  id,  0,     O64 | PLOCK)
INSTR (OR,                regmem8,    reg8,       none,       none,       0,       0x08,      r,   no,  0,     PLOCK)
INSTR (OR,                regmem16,   reg16,      none,       none,       0,       0x09,      r,   no,  0,     O16 | PLOCK)
INSTR (OR,                regmem32,   reg32,      none,       none,       0,       0x09,      r,   no,  0,     O32 | PLOCK)
INSTR (OR,                regmem64,   reg64,      none,       none,       0,       0x09,      r,   no,  0,     O64 | PLOCK)
INSTR (OR,                reg8,       regmem8,    none,       none,       0,       0x0a,      r,   no,  0,     0)
INSTR (OR,                reg16,      regmem16,   none,       none,       0,       0x0b,      r,   no,  0,     O16)
INSTR (OR,                reg32,      regmem32,   none,       none,       0,       0x0b,      r,   no,  0,     O32)
INSTR (OR,                reg64,      regmem64,   none,       none,       0,       0x0b,      r,   no,  0,     O64)

INSTR (OUT,               imm8,       al,         none,       none,       0,       0xe6,      ib,  no,  0,     0)
INSTR (OUT,               imm8,       ax,         none,       none,       0,       0xe7,      ib,  no,  0,     O16)
INSTR (OUT,               imm8,       eax,        none,       none,       0,       0xe7,      ib,  no,  0,     O32)
INSTR (OUT,               dx,         al,         none,       none,       0,       0xee,      no,  no,  0,     0)
INSTR (OUT,               dx,         ax,         none,       none,       0,       0xef,      no,  no,  0,     O16)
INSTR (OUT,               dx,         eax,        none,       none,       0,       0xef,      no,  no,  0,     O32)

INSTR (OUTSB,             none,       none,       none,       none,       0,       0x6e,      no,  no,  0,     PREP)

INSTR (OUTSW,             none,       none,       none,       none,       0,       0x6f,      no,  no,  0,     O16 | PREP)

INSTR (OUTSD,             none,       none,       none,       none,       0,       0x6f,      no,  no,  0,     O32 | PREP)

INSTR (PDEP,              reg32,      reg32vvvv,  regmem32,   none,       0x0203,  0xf5,      r,   no,  0,     PVEX)
INSTR (PDEP,              reg64,      reg64vvvv,  regmem64,   none,       0x0283,  0xf5,      r,   no,  0,     PVEX)

INSTR (PEXT,              reg32,      reg32vvvv,  regmem32,   none,       0x0202,  0xf5,      r,   no,  0,     PVEX)
INSTR (PEXT,              reg64,      reg64vvvv,  regmem64,   none,       0x0282,  0xf5,      r,   no,  0,     PVEX)

INSTR (POP,               reg16,      none,       none,       none,       0,       0x58,      rv,  no,  0,     O16)
INSTR (POP,               reg32,      none,       none,       none,       0,       0x58,      rv,  no,  0,     O32 | I64)
INSTR (POP,               reg64,      none,       none,       none,       0,       0x58,      rv,  no,  0,     O64 | D64)
INSTR (POP,               regmem16,   none,       none,       none,       0,       0x8f,      r0,  no,  0,     O16)
INSTR (POP,               regmem32,   none,       none,       none,       0,       0x8f,      r0,  no,  0,     O32 | I64)
INSTR (POP,               regmem64,   none,       none,       none,       0,       0x8f,      r0,  no,  0,     O64 | D64)
INSTR (POP,               ds,         none,       none,       none,       0,       0x1f,      no,  no,  0,     I64)
INSTR (POP,               es,         none,       none,       none,       0,       0x07,      no,  no,  0,     I64)
INSTR (POP,               ss,         none,       none,       none,       0,       0x17,      no,  no,  0,     I64)
INSTR (POP,               fs,         none,       none,       none,       0,       0x0fa1,    no,  no,  0,     0)
INSTR (POP,               gs,         none,       none,       none,       0,       0x0fa9,    no,  no,  0,     0)

INSTR (POPA,              none,       none,       none,       none,       0,       0x61,      no,  no,  0,     O16 | I64)

INSTR (POPAD,             none,       none,       none,       none,       0,       0x61,      no,  no,  0,     O32 | I64)

INSTR (POPCNT,            reg16,      regmem16,   none,       none,       0,       0x0fb8,    r,   no,  0,     O16 | PF3)
INSTR (POPCNT,            reg32,      regmem32,   none,       none,       0,       0x0fb8,    r,   no,  0,     O32 | PF3)
INSTR (POPCNT,            reg64,      regmem64,   none,       none,       0,       0x0fb8,    r,   no,  0,     O64 | PF3)

INSTR (POPF,              none,       none,       none,       none,       0,       0x9d,      no,  no,  0,     O16)

INSTR (POPFD,             none,       none,       none,       none,       0,       0x9d,      no,  no,  0,     O32 | I64)

INSTR (POPFQ,             none,       none,       none,       none,       0,       0x9d,      no,  no,  0,     O64 | D64)

INSTR (PREFETCH,          mem,        none,       none,       none,       0,       0x0f0d,    r0,  no,  0,     0)

INSTR (PREFETCHW,         mem,        none,       none,       none,       0,       0x0f0d,    r1,  no,  0,     0)

INSTR (PREFETCHNTA,       mem,        none,       none,       none,       0,       0x0f18,    r0,  no,  0,     0)

INSTR (PREFETCHT0,        mem,        none,       none,       none,       0,       0x0f18,    r1,  no,  0,     0)

INSTR (PREFETCHT1,        mem,        none,       none,       none,       0,       0x0f18,    r2,  no,  0,     0)

INSTR (PREFETCHT2,        mem,        none,       none,       none,       0,       0x0f18,    r3,  no,  0,     0)

INSTR (PUSH,              reg16,      none,       none,       none,       0,       0x50,      rv,  no,  0,     O16)
INSTR (PUSH,              reg32,      none,       none,       none,       0,       0x50,      rv,  no,  0,     O32 | I64)
INSTR (PUSH,              reg64,      none,       none,       none,       0,       0x50,      rv,  no,  0,     O64 | D64)
INSTR (PUSH,              regmem16,   none,       none,       none,       0,       0xff,      r6,  no,  0,     O16)
INSTR (PUSH,              regmem32,   none,       none,       none,       0,       0xff,      r6,  no,  0,     O32 | I64)
INSTR (PUSH,              regmem64,   none,       none,       none,       0,       0xff,      r6,  no,  0,     O64 | D64)
INSTR (PUSH,              simm8,      none,       none,       none,       0,       0x6a,      ib,  no,  0,     0)
INSTR (PUSH,              imm16,      none,       none,       none,       0,       0x68,      iw,  no,  0,     O16)
INSTR (PUSH,              imm32,      none,       none,       none,       0,       0x68,      id,  no,  0,     O32 | I64)
INSTR (PUSH,              simm32,     none,       none,       none,       0,       0x68,      id,  no,  0,     O64 | D64)
INSTR (PUSH,              cs,         none,       none,       none,       0,       0x0e,      no,  no,  0,     I64)
INSTR (PUSH,              ss,         none,       none,       none,       0,       0x16,      no,  no,  0,     I64)
INSTR (PUSH,              ds,         none,       none,       none,       0,       0x1e,      no,  no,  0,     I64)
INSTR (PUSH,              es,         none,       none,       none,       0,       0x06,      no,  no,  0,     I64)
INSTR (PUSH,              fs,         none,       none,       none,       0,       0x0fa0,    no,  no,  0,     0)
INSTR (PUSH,              gs,         none,       none,       none,       0,       0x0fa8,    no,  no,  0,     0)

INSTR (PUSHA,             none,       none,       none,       none,       0,       0x60,      no,  no,  0,     O16 | I64)

INSTR (PUSHAD,            none,       none,       none,       none,       0,       0x60,      no,  no,  0,     O32 | I64)

INSTR (PUSHF,             none,       none,       none,       none,       0,       0x9c,      no,  no,  0,     O16)

INSTR (PUSHFD,            none,       none,       none,       none,       0,       0x9c,      no,  no,  0,     O32 | I64)

INSTR (PUSHFQ,            none,       none,       none,       none,       0,       0x9c,      no,  no,  0,     O64 | D64)

INSTR (RCL,               regmem8,    one,        none,       none,       0,       0xd0,      r2,  no,  0,     0)
INSTR (RCL,               regmem8,    cl,         none,       none,       0,       0xd2,      r2,  no,  0,     0)
INSTR (RCL,               regmem8,    imm8,       none,       none,       0,       0xc0,      r2,  ib,  0,     0)
INSTR (RCL,               regmem16,   one,        none,       none,       0,       0xd1,      r2,  no,  0,     O16)
INSTR (RCL,               regmem16,   cl,         none,       none,       0,       0xd3,      r2,  no,  0,     O16)
INSTR (RCL,               regmem16,   imm8,       none,       none,       0,       0xc1,      r2,  ib,  0,     O16)
INSTR (RCL,               regmem32,   one,        none,       none,       0,       0xd1,      r2,  no,  0,     O32)
INSTR (RCL,               regmem32,   cl,         none,       none,       0,       0xd3,      r2,  no,  0,     O32)
INSTR (RCL,               regmem32,   imm8,       none,       none,       0,       0xc1,      r2,  ib,  0,     O32)
INSTR (RCL,               regmem64,   one,        none,       none,       0,       0xd1,      r2,  no,  0,     O64)
INSTR (RCL,               regmem64,   cl,         none,       none,       0,       0xd3,      r2,  no,  0,     O64)
INSTR (RCL,               regmem64,   imm8,       none,       none,       0,       0xc1,      r2,  ib,  0,     O64)

INSTR (RCR,               regmem8,    one,        none,       none,       0,       0xd0,      r3,  no,  0,     0)
INSTR (RCR,               regmem8,    cl,         none,       none,       0,       0xd2,      r3,  no,  0,     0)
INSTR (RCR,               regmem8,    imm8,       none,       none,       0,       0xc0,      r3,  ib,  0,     0)
INSTR (RCR,               regmem16,   one,        none,       none,       0,       0xd1,      r3,  no,  0,     O16)
INSTR (RCR,               regmem16,   cl,         none,       none,       0,       0xd3,      r3,  no,  0,     O16)
INSTR (RCR,               regmem16,   imm8,       none,       none,       0,       0xc1,      r3,  ib,  0,     O16)
INSTR (RCR,               regmem32,   one,        none,       none,       0,       0xd1,      r3,  no,  0,     O32)
INSTR (RCR,               regmem32,   cl,         none,       none,       0,       0xd3,      r3,  no,  0,     O32)
INSTR (RCR,               regmem32,   imm8,       none,       none,       0,       0xc1,      r3,  ib,  0,     O32)
INSTR (RCR,               regmem64,   one,        none,       none,       0,       0xd1,      r3,  no,  0,     O64)
INSTR (RCR,               regmem64,   cl,         none,       none,       0,       0xd3,      r3,  no,  0,     O64)
INSTR (RCR,               regmem64,   imm8,       none,       none,       0,       0xc1,      r3,  ib,  0,     O64)

INSTR (RDFSBASE,          reg32rm,    none,       none,       none,       0,       0x0fae,    r0,  no,  0,     I16 | I32 | O32 | PF3)
INSTR (RDFSBASE,          reg64rm,    none,       none,       none,       0,       0x0fae,    r0,  no,  0,     I16 | I32 | O64 | PF3)

INSTR (RDGSBASE,          reg32rm,    none,       none,       none,       0,       0x0fae,    r1,  no,  0,     I16 | I32 | O32 | PF3)
INSTR (RDGSBASE,          reg64rm,    none,       none,       none,       0,       0x0fae,    r1,  no,  0,     I16 | I32 | O64 | PF3)

INSTR (RDPID,             reg32rm,    none,       none,       none,       0,       0x0fc7,    r7,  no,  0,     O32 | I64 | PF3)
INSTR (RDPID,             reg64rm,    none,       none,       none,       0,       0x0fc7,    r7,  no,  0,     O64 | D64 | PF3)

INSTR (RDPRU,             none,       none,       none,       none,       0,       0x0f01,    no,  no,  0xfd,  SFX)

INSTR (RDRAND,            reg16rm,    none,       none,       none,       0,       0x0fc7,    r6,  no,  0,     O16)
INSTR (RDRAND,            reg32rm,    none,       none,       none,       0,       0x0fc7,    r6,  no,  0,     O32)
INSTR (RDRAND,            reg64rm,    none,       none,       none,       0,       0x0fc7,    r6,  no,  0,     O64)

INSTR (RDSEED,            reg16rm,    none,       none,       none,       0,       0x0fc7,    r7,  no,  0,     O16)
INSTR (RDSEED,            reg32rm,    none,       none,       none,       0,       0x0fc7,    r7,  no,  0,     O32)
INSTR (RDSEED,            reg64rm,    none,       none,       none,       0,       0x0fc7,    r7,  no,  0,     O64)

INSTR (RET,               none,       none,       none,       none,       0,       0xc3,      no,  no,  0,     0)
INSTR (RET,               imm16,      none,       none,       none,       0,       0xc2,      iw,  no,  0,     0)

INSTR (RETF,              none,       none,       none,       none,       0,       0xcb,      no,  no,  0,     0)
INSTR (RETF,              imm16,      none,       none,       none,       0,       0xca,      iw,  no,  0,     0)

INSTR (ROL,               regmem8,    one,        none,       none,       0,       0xd0,      r0,  no,  0,     0)
INSTR (ROL,               regmem8,    cl,         none,       none,       0,       0xd2,      r0,  no,  0,     0)
INSTR (ROL,               regmem8,    imm8,       none,       none,       0,       0xc0,      r0,  ib,  0,     0)
INSTR (ROL,               regmem16,   one,        none,       none,       0,       0xd1,      r0,  no,  0,     O16)
INSTR (ROL,               regmem16,   cl,         none,       none,       0,       0xd3,      r0,  no,  0,     O16)
INSTR (ROL,               regmem16,   imm8,       none,       none,       0,       0xc1,      r0,  ib,  0,     O16)
INSTR (ROL,               regmem32,   one,        none,       none,       0,       0xd1,      r0,  no,  0,     O32)
INSTR (ROL,               regmem32,   cl,         none,       none,       0,       0xd3,      r0,  no,  0,     O32)
INSTR (ROL,               regmem32,   imm8,       none,       none,       0,       0xc1,      r0,  ib,  0,     O32)
INSTR (ROL,               regmem64,   one,        none,       none,       0,       0xd1,      r0,  no,  0,     O64)
INSTR (ROL,               regmem64,   cl,         none,       none,       0,       0xd3,      r0,  no,  0,     O64)
INSTR (ROL,               regmem64,   imm8,       none,       none,       0,       0xc1,      r0,  ib,  0,     O64)

INSTR (ROR,               regmem8,    one,        none,       none,       0,       0xd0,      r1,  no,  0,     0)
INSTR (ROR,               regmem8,    cl,         none,       none,       0,       0xd2,      r1,  no,  0,     0)
INSTR (ROR,               regmem8,    imm8,       none,       none,       0,       0xc0,      r1,  ib,  0,     0)
INSTR (ROR,               regmem16,   one,        none,       none,       0,       0xd1,      r1,  no,  0,     O16)
INSTR (ROR,               regmem16,   cl,         none,       none,       0,       0xd3,      r1,  no,  0,     O16)
INSTR (ROR,               regmem16,   imm8,       none,       none,       0,       0xc1,      r1,  ib,  0,     O16)
INSTR (ROR,               regmem32,   one,        none,       none,       0,       0xd1,      r1,  no,  0,     O32)
INSTR (ROR,               regmem32,   cl,         none,       none,       0,       0xd3,      r1,  no,  0,     O32)
INSTR (ROR,               regmem32,   imm8,       none,       none,       0,       0xc1,      r1,  ib,  0,     O32)
INSTR (ROR,               regmem64,   one,        none,       none,       0,       0xd1,      r1,  no,  0,     O64)
INSTR (ROR,               regmem64,   cl,         none,       none,       0,       0xd3,      r1,  no,  0,     O64)
INSTR (ROR,               regmem64,   imm8,       none,       none,       0,       0xc1,      r1,  ib,  0,     O64)

INSTR (RORX,              reg32,      regmem32,   imm8,       none,       0x037b,  0xf0,      r,   ib,  0,     PVEX)
INSTR (RORX,              reg64,      regmem64,   imm8,       none,       0x03fb,  0xf0,      r,   ib,  0,     PVEX)

INSTR (SAHF,              none,       none,       none,       none,       0,       0x9e,      no,  no,  0,     0)

INSTR (SHL,               regmem8,    one,        none,       none,       0,       0xd0,      r4,  no,  0,     0)
INSTR (SHL,               regmem8,    cl,         none,       none,       0,       0xd2,      r4,  no,  0,     0)
INSTR (SHL,               regmem8,    imm8,       none,       none,       0,       0xc0,      r4,  ib,  0,     0)
INSTR (SHL,               regmem16,   one,        none,       none,       0,       0xd1,      r4,  no,  0,     O16)
INSTR (SHL,               regmem16,   cl,         none,       none,       0,       0xd3,      r4,  no,  0,     O16)
INSTR (SHL,               regmem16,   imm8,       none,       none,       0,       0xc1,      r4,  ib,  0,     O16)
INSTR (SHL,               regmem32,   one,        none,       none,       0,       0xd1,      r4,  no,  0,     O32)
INSTR (SHL,               regmem32,   cl,         none,       none,       0,       0xd3,      r4,  no,  0,     O32)
INSTR (SHL,               regmem32,   imm8,       none,       none,       0,       0xc1,      r4,  ib,  0,     O32)
INSTR (SHL,               regmem64,   one,        none,       none,       0,       0xd1,      r4,  no,  0,     O64)
INSTR (SHL,               regmem64,   cl,         none,       none,       0,       0xd3,      r4,  no,  0,     O64)
INSTR (SHL,               regmem64,   imm8,       none,       none,       0,       0xc1,      r4,  ib,  0,     O64)

INSTR (SAL,               regmem8,    one,        none,       none,       0,       0xd0,      r4,  no,  0,     0)
INSTR (SAL,               regmem8,    cl,         none,       none,       0,       0xd2,      r4,  no,  0,     0)
INSTR (SAL,               regmem8,    imm8,       none,       none,       0,       0xc0,      r4,  ib,  0,     0)
INSTR (SAL,               regmem16,   one,        none,       none,       0,       0xd1,      r4,  no,  0,     O16)
INSTR (SAL,               regmem16,   cl,         none,       none,       0,       0xd3,      r4,  no,  0,     O16)
INSTR (SAL,               regmem16,   imm8,       none,       none,       0,       0xc1,      r4,  ib,  0,     O16)
INSTR (SAL,               regmem32,   one,        none,       none,       0,       0xd1,      r4,  no,  0,     O32)
INSTR (SAL,               regmem32,   cl,         none,       none,       0,       0xd3,      r4,  no,  0,     O32)
INSTR (SAL,               regmem32,   imm8,       none,       none,       0,       0xc1,      r4,  ib,  0,     O32)
INSTR (SAL,               regmem64,   one,        none,       none,       0,       0xd1,      r4,  no,  0,     O64)
INSTR (SAL,               regmem64,   cl,         none,       none,       0,       0xd3,      r4,  no,  0,     O64)
INSTR (SAL,               regmem64,   imm8,       none,       none,       0,       0xc1,      r4,  ib,  0,     O64)

INSTR (SAR,               regmem8,    one,        none,       none,       0,       0xd0,      r7,  no,  0,     0)
INSTR (SAR,               regmem8,    cl,         none,       none,       0,       0xd2,      r7,  no,  0,     0)
INSTR (SAR,               regmem8,    imm8,       none,       none,       0,       0xc0,      r7,  ib,  0,     0)
INSTR (SAR,               regmem16,   one,        none,       none,       0,       0xd1,      r7,  no,  0,     O16)
INSTR (SAR,               regmem16,   cl,         none,       none,       0,       0xd3,      r7,  no,  0,     O16)
INSTR (SAR,               regmem16,   imm8,       none,       none,       0,       0xc1,      r7,  ib,  0,     O16)
INSTR (SAR,               regmem32,   one,        none,       none,       0,       0xd1,      r7,  no,  0,     O32)
INSTR (SAR,               regmem32,   cl,         none,       none,       0,       0xd3,      r7,  no,  0,     O32)
INSTR (SAR,               regmem32,   imm8,       none,       none,       0,       0xc1,      r7,  ib,  0,     O32)
INSTR (SAR,               regmem64,   one,        none,       none,       0,       0xd1,      r7,  no,  0,     O64)
INSTR (SAR,               regmem64,   cl,         none,       none,       0,       0xd3,      r7,  no,  0,     O64)
INSTR (SAR,               regmem64,   imm8,       none,       none,       0,       0xc1,      r7,  ib,  0,     O64)

INSTR (SARX,              reg32,      regmem32,   reg32vvvv,  none,       0x0202,  0xf7,      r,   no,  0,     PVEX)
INSTR (SARX,              reg64,      regmem64,   reg64vvvv,  none,       0x0282,  0xf7,      r,   no,  0,     PVEX)

INSTR (SBB,               al,         imm8,       none,       none,       0,       0x1c,      ib,  no,  0,     0)
INSTR (SBB,               regmem8,    imm8,       none,       none,       0,       0x80,      r3,  ib,  0,     PLOCK)
INSTR (SBB,               regmem8,    imm8,       none,       none,       0,       0x82,      r3,  ib,  0,     I64 | PLOCK)
INSTR (SBB,               regmem16,   simm8,      none,       none,       0,       0x83,      r3,  ib,  0,     O16 | PLOCK)
INSTR (SBB,               regmem32,   simm8,      none,       none,       0,       0x83,      r3,  ib,  0,     O32 | PLOCK)
INSTR (SBB,               regmem64,   simm8,      none,       none,       0,       0x83,      r3,  ib,  0,     O64 | PLOCK)
INSTR (SBB,               ax,         imm16,      none,       none,       0,       0x1d,      iw,  no,  0,     O16)
INSTR (SBB,               regmem16,   imm16,      none,       none,       0,       0x81,      r3,  iw,  0,     O16 | PLOCK)
INSTR (SBB,               eax,        imm32,      none,       none,       0,       0x1d,      id,  no,  0,     O32)
INSTR (SBB,               regmem32,   imm32,      none,       none,       0,       0x81,      r3,  id,  0,     O32 | PLOCK)
INSTR (SBB,               rax,        simm32,     none,       none,       0,       0x1d,      id,  no,  0,     O64)
INSTR (SBB,               regmem64,   simm32,     none,       none,       0,       0x81,      r3,  id,  0,     O64 | PLOCK)
INSTR (SBB,               regmem8,    reg8,       none,       none,       0,       0x18,      r,   no,  0,     PLOCK)
INSTR (SBB,               regmem16,   reg16,      none,       none,       0,       0x19,      r,   no,  0,     O16 | PLOCK)
INSTR (SBB,               regmem32,   reg32,      none,       none,       0,       0x19,      r,   no,  0,     O32 | PLOCK)
INSTR (SBB,               regmem64,   reg64,      none,       none,       0,       0x19,      r,   no,  0,     O64 | PLOCK)
INSTR (SBB,               reg8,       regmem8,    none,       none,       0,       0x1a,      r,   no,  0,     0)
INSTR (SBB,               reg16,      regmem16,   none,       none,       0,       0x1b,      r,   no,  0,     O16)
INSTR (SBB,               reg32,      regmem32,   none,       none,       0,       0x1b,      r,   no,  0,     O32)
INSTR (SBB,               reg64,      regmem64,   none,       none,       0,       0x1b,      r,   no,  0,     O64)

INSTR (SCASB,             none,       none,       none,       none,       0,       0xae,      no,  no,  0,     PREPE | PREPNE)

INSTR (SCASW,             none,       none,       none,       none,       0,       0xaf,      no,  no,  0,     O16 | PREPE | PREPNE)

INSTR (SCASD,             none,       none,       none,       none,       0,       0xaf,      no,  no,  0,     O32 | PREPE | PREPNE)

INSTR (SCASQ,             none,       none,       none,       none,       0,       0xaf,      no,  no,  0,     O64 | PREPE | PREPNE)

INSTR (SETO,              regmem8,    none,       none,       none,       0,       0x0f90,    r0,  no,  0,     0)

INSTR (SETNO,             regmem8,    none,       none,       none,       0,       0x0f91,    r0,  no,  0,     0)

INSTR (SETC,              regmem8,    none,       none,       none,       0,       0x0f92,    r0,  no,  0,     0)

INSTR (SETB,              regmem8,    none,       none,       none,       0,       0x0f92,    r0,  no,  0,     0)

INSTR (SETNAE,            regmem8,    none,       none,       none,       0,       0x0f92,    r0,  no,  0,     0)

INSTR (SETNC,             regmem8,    none,       none,       none,       0,       0x0f93,    r0,  no,  0,     0)

INSTR (SETNB,             regmem8,    none,       none,       none,       0,       0x0f93,    r0,  no,  0,     0)

INSTR (SETAE,             regmem8,    none,       none,       none,       0,       0x0f93,    r0,  no,  0,     0)

INSTR (SETZ,              regmem8,    none,       none,       none,       0,       0x0f94,    r0,  no,  0,     0)

INSTR (SETE,              regmem8,    none,       none,       none,       0,       0x0f94,    r0,  no,  0,     0)

INSTR (SETNZ,             regmem8,    none,       none,       none,       0,       0x0f95,    r0,  no,  0,     0)

INSTR (SETNE,             regmem8,    none,       none,       none,       0,       0x0f95,    r0,  no,  0,     0)

INSTR (SETNA,             regmem8,    none,       none,       none,       0,       0x0f96,    r0,  no,  0,     0)

INSTR (SETBE,             regmem8,    none,       none,       none,       0,       0x0f96,    r0,  no,  0,     0)

INSTR (SETA,              regmem8,    none,       none,       none,       0,       0x0f97,    r0,  no,  0,     0)

INSTR (SETNBE,            regmem8,    none,       none,       none,       0,       0x0f97,    r0,  no,  0,     0)

INSTR (SETS,              regmem8,    none,       none,       none,       0,       0x0f98,    r0,  no,  0,     0)

INSTR (SETNS,             regmem8,    none,       none,       none,       0,       0x0f99,    r0,  no,  0,     0)

INSTR (SETPE,             regmem8,    none,       none,       none,       0,       0x0f9a,    r0,  no,  0,     0)

INSTR (SETP,              regmem8,    none,       none,       none,       0,       0x0f9a,    r0,  no,  0,     0)

INSTR (SETPO,             regmem8,    none,       none,       none,       0,       0x0f9b,    r0,  no,  0,     0)

INSTR (SETNP,             regmem8,    none,       none,       none,       0,       0x0f9b,    r0,  no,  0,     0)

INSTR (SETL,              regmem8,    none,       none,       none,       0,       0x0f9c,    r0,  no,  0,     0)

INSTR (SETNGE,            regmem8,    none,       none,       none,       0,       0x0f9c,    r0,  no,  0,     0)

INSTR (SETNL,             regmem8,    none,       none,       none,       0,       0x0f9d,    r0,  no,  0,     0)

INSTR (SETGE,             regmem8,    none,       none,       none,       0,       0x0f9d,    r0,  no,  0,     0)

INSTR (SETNG,             regmem8,    none,       none,       none,       0,       0x0f9e,    r0,  no,  0,     0)

INSTR (SETLE,             regmem8,    none,       none,       none,       0,       0x0f9e,    r0,  no,  0,     0)

INSTR (SETG,              regmem8,    none,       none,       none,       0,       0x0f9f,    r0,  no,  0,     0)

INSTR (SETNLE,            regmem8,    none,       none,       none,       0,       0x0f9f,    r0,  no,  0,     0)

INSTR (SFENCE,            none,       none,       none,       none,       0,       0x0fae,    no,  no,  0xf8,  SFX)

INSTR (SHLD,              regmem16,   reg16,      imm8,       none,       0,       0x0fa4,    r,   ib,  0,     O16)
INSTR (SHLD,              regmem16,   reg16,      cl,         none,       0,       0x0fa5,    r,   no,  0,     O16)
INSTR (SHLD,              regmem32,   reg32,      imm8,       none,       0,       0x0fa4,    r,   ib,  0,     O32)
INSTR (SHLD,              regmem32,   reg32,      cl,         none,       0,       0x0fa5,    r,   no,  0,     O32)
INSTR (SHLD,              regmem64,   reg64,      imm8,       none,       0,       0x0fa4,    r,   ib,  0,     O64)
INSTR (SHLD,              regmem64,   reg64,      cl,         none,       0,       0x0fa5,    r,   no,  0,     O64)

INSTR (SHLX,              reg32,      regmem32,   reg32vvvv,  none,       0x0201,  0xf7,      r,   no,  0,     PVEX)
INSTR (SHLX,              reg64,      regmem64,   reg64vvvv,  none,       0x0281,  0xf7,      r,   no,  0,     PVEX)

INSTR (SHR,               regmem8,    one,        none,       none,       0,       0xd0,      r5,  no,  0,     0)
INSTR (SHR,               regmem8,    cl,         none,       none,       0,       0xd2,      r5,  no,  0,     0)
INSTR (SHR,               regmem8,    imm8,       none,       none,       0,       0xc0,      r5,  ib,  0,     0)
INSTR (SHR,               regmem16,   one,        none,       none,       0,       0xd1,      r5,  no,  0,     O16)
INSTR (SHR,               regmem16,   cl,         none,       none,       0,       0xd3,      r5,  no,  0,     O16)
INSTR (SHR,               regmem16,   imm8,       none,       none,       0,       0xc1,      r5,  ib,  0,     O16)
INSTR (SHR,               regmem32,   one,        none,       none,       0,       0xd1,      r5,  no,  0,     O32)
INSTR (SHR,               regmem32,   cl,         none,       none,       0,       0xd3,      r5,  no,  0,     O32)
INSTR (SHR,               regmem32,   imm8,       none,       none,       0,       0xc1,      r5,  ib,  0,     O32)
INSTR (SHR,               regmem64,   one,        none,       none,       0,       0xd1,      r5,  no,  0,     O64)
INSTR (SHR,               regmem64,   cl,         none,       none,       0,       0xd3,      r5,  no,  0,     O64)
INSTR (SHR,               regmem64,   imm8,       none,       none,       0,       0xc1,      r5,  ib,  0,     O64)

INSTR (SHRD,              regmem16,   reg16,      imm8,       none,       0,       0x0fac,    r,   ib,  0,     O16)
INSTR (SHRD,              regmem16,   reg16,      cl,         none,       0,       0x0fad,    r,   no,  0,     O16)
INSTR (SHRD,              regmem32,   reg32,      imm8,       none,       0,       0x0fac,    r,   ib,  0,     O32)
INSTR (SHRD,              regmem32,   reg32,      cl,         none,       0,       0x0fad,    r,   no,  0,     O32)
INSTR (SHRD,              regmem64,   reg64,      imm8,       none,       0,       0x0fac,    r,   ib,  0,     O64)
INSTR (SHRD,              regmem64,   reg64,      cl,         none,       0,       0x0fad,    r,   no,  0,     O64)

INSTR (SHRX,              reg32,      regmem32,   reg32vvvv,  none,       0x0203,  0xf7,      r,   no,  0,     PVEX)
INSTR (SHRX,              reg64,      regmem64,   reg64vvvv,  none,       0x0283,  0xf7,      r,   no,  0,     PVEX)

INSTR (SLWPCB,            reg32rm,    none,       none,       none,       0x0978,  0x12,      r1,  no,  0,     PXOP)
INSTR (SLWPCB,            reg64rm,    none,       none,       none,       0x09f8,  0x12,      r1,  no,  0,     PXOP)

INSTR (STAC,              none,       none,       none,       none,       0,       0x0f01,    no,  no,  0xcb,  SFX)

INSTR (STC,               none,       none,       none,       none,       0,       0xf9,      no,  no,  0,     0)

INSTR (STD,               none,       none,       none,       none,       0,       0xfd,      no,  no,  0,     0)

INSTR (STOSB,             none,       none,       none,       none,       0,       0xaa,      no,  no,  0,     PREP)

INSTR (STOSW,             none,       none,       none,       none,       0,       0xab,      no,  no,  0,     O16 | PREP)

INSTR (STOSD,             none,       none,       none,       none,       0,       0xab,      no,  no,  0,     O32 | PREP)

INSTR (STOSQ,             none,       none,       none,       none,       0,       0xab,      no,  no,  0,     O64 | PREP)

INSTR (SUB,               al,         imm8,       none,       none,       0,       0x2c,      ib,  no,  0,     0)
INSTR (SUB,               regmem8,    imm8,       none,       none,       0,       0x80,      r5,  ib,  0,     PLOCK)
INSTR (SUB,               regmem8,    imm8,       none,       none,       0,       0x82,      r5,  ib,  0,     I64 | PLOCK)
INSTR (SUB,               regmem16,   simm8,      none,       none,       0,       0x83,      r5,  ib,  0,     O16 | PLOCK)
INSTR (SUB,               regmem32,   simm8,      none,       none,       0,       0x83,      r5,  ib,  0,     O32 | PLOCK)
INSTR (SUB,               regmem64,   simm8,      none,       none,       0,       0x83,      r5,  ib,  0,     O64 | PLOCK)
INSTR (SUB,               ax,         imm16,      none,       none,       0,       0x2d,      iw,  no,  0,     O16)
INSTR (SUB,               regmem16,   imm16,      none,       none,       0,       0x81,      r5,  iw,  0,     O16 | PLOCK)
INSTR (SUB,               eax,        imm32,      none,       none,       0,       0x2d,      id,  no,  0,     O32)
INSTR (SUB,               regmem32,   imm32,      none,       none,       0,       0x81,      r5,  id,  0,     O32 | PLOCK)
INSTR (SUB,               rax,        simm32,     none,       none,       0,       0x2d,      id,  no,  0,     O64)
INSTR (SUB,               regmem64,   simm32,     none,       none,       0,       0x81,      r5,  id,  0,     O64 | PLOCK)
INSTR (SUB,               regmem8,    reg8,       none,       none,       0,       0x28,      r,   no,  0,     PLOCK)
INSTR (SUB,               regmem16,   reg16,      none,       none,       0,       0x29,      r,   no,  0,     O16 | PLOCK)
INSTR (SUB,               regmem32,   reg32,      none,       none,       0,       0x29,      r,   no,  0,     O32 | PLOCK)
INSTR (SUB,               regmem64,   reg64,      none,       none,       0,       0x29,      r,   no,  0,     O64 | PLOCK)
INSTR (SUB,               reg8,       regmem8,    none,       none,       0,       0x2a,      r,   no,  0,     0)
INSTR (SUB,               reg16,      regmem16,   none,       none,       0,       0x2b,      r,   no,  0,     O16)
INSTR (SUB,               reg32,      regmem32,   none,       none,       0,       0x2b,      r,   no,  0,     O32)
INSTR (SUB,               reg64,      regmem64,   none,       none,       0,       0x2b,      r,   no,  0,     O64)

INSTR (TEST,              al,         imm8,       none,       none,       0,       0xa8,      ib,  no,  0,     0)
INSTR (TEST,              regmem8,    imm8,       none,       none,       0,       0xf6,      r0,  ib,  0,     0)
INSTR (TEST,              regmem8,    imm8,       none,       none,       0,       0xf6,      r1,  ib,  0,     0)
INSTR (TEST,              ax,         imm16,      none,       none,       0,       0xa9,      iw,  no,  0,     O16)
INSTR (TEST,              regmem16,   imm16,      none,       none,       0,       0xf7,      r0,  iw,  0,     O16)
INSTR (TEST,              regmem16,   imm16,      none,       none,       0,       0xf7,      r1,  iw,  0,     O16)
INSTR (TEST,              eax,        imm32,      none,       none,       0,       0xa9,      id,  no,  0,     O32)
INSTR (TEST,              regmem32,   imm32,      none,       none,       0,       0xf7,      r0,  id,  0,     O32)
INSTR (TEST,              regmem32,   imm32,      none,       none,       0,       0xf7,      r1,  id,  0,     O32)
INSTR (TEST,              rax,        simm32,     none,       none,       0,       0xa9,      id,  no,  0,     O64)
INSTR (TEST,              regmem64,   simm32,     none,       none,       0,       0xf7,      r0,  id,  0,     O64)
INSTR (TEST,              regmem64,   simm32,     none,       none,       0,       0xf7,      r1,  id,  0,     O64)
INSTR (TEST,              regmem8,    reg8,       none,       none,       0,       0x84,      r,   no,  0,     0)
INSTR (TEST,              regmem16,   reg16,      none,       none,       0,       0x85,      r,   no,  0,     O16)
INSTR (TEST,              regmem32,   reg32,      none,       none,       0,       0x85,      r,   no,  0,     O32)
INSTR (TEST,              regmem64,   reg64,      none,       none,       0,       0x85,      r,   no,  0,     O64)

INSTR (WRFSBASE,          reg32rm,    none,       none,       none,       0,       0x0fae,    r2,  no,  0,     I16 | I32 | O32 | PF3)
INSTR (WRFSBASE,          reg64rm,    none,       none,       none,       0,       0x0fae,    r2,  no,  0,     I16 | I32 | O64 | PF3)

INSTR (WRGSBASE,          reg32rm,    none,       none,       none,       0,       0x0fae,    r3,  no,  0,     I16 | I32 | O32 | PF3)
INSTR (WRGSBASE,          reg64rm,    none,       none,       none,       0,       0x0fae,    r3,  no,  0,     I16 | I32 | O64 | PF3)

INSTR (XADD,              regmem8,    reg8,       none,       none,       0,       0x0fc0,    r,   no,  0,     PLOCK)
INSTR (XADD,              regmem16,   reg16,      none,       none,       0,       0x0fc1,    r,   no,  0,     O16 | PLOCK)
INSTR (XADD,              regmem32,   reg32,      none,       none,       0,       0x0fc1,    r,   no,  0,     O32 | PLOCK)
INSTR (XADD,              regmem64,   reg64,      none,       none,       0,       0x0fc1,    r,   no,  0,     O64 | PLOCK)

INSTR (XCHG,              ax,         reg16,      none,       none,       0,       0x90,      rv,  no,  0,     O16)
INSTR (XCHG,              reg16,      ax,         none,       none,       0,       0x90,      rv,  no,  0,     O16)
INSTR (XCHG,              eax,        reg32,      none,       none,       0,       0x90,      rv,  no,  0,     O32)
INSTR (XCHG,              reg32,      eax,        none,       none,       0,       0x90,      rv,  no,  0,     O32)
INSTR (XCHG,              rax,        reg64,      none,       none,       0,       0x90,      rv,  no,  0,     O64)
INSTR (XCHG,              reg64,      rax,        none,       none,       0,       0x90,      rv,  no,  0,     O64)
INSTR (XCHG,              reg8,       regmem8,    none,       none,       0,       0x86,      r,   no,  0,     0)
INSTR (XCHG,              regmem8,    reg8,       none,       none,       0,       0x86,      r,   no,  0,     PLOCK)
INSTR (XCHG,              reg16,      regmem16,   none,       none,       0,       0x87,      r,   no,  0,     O16)
INSTR (XCHG,              regmem16,   reg16,      none,       none,       0,       0x87,      r,   no,  0,     O16 | PLOCK)
INSTR (XCHG,              reg32,      regmem32,   none,       none,       0,       0x87,      r,   no,  0,     O32)
INSTR (XCHG,              regmem32,   reg32,      none,       none,       0,       0x87,      r,   no,  0,     O32 | PLOCK)
INSTR (XCHG,              reg64,      regmem64,   none,       none,       0,       0x87,      r,   no,  0,     O64)
INSTR (XCHG,              regmem64,   reg64,      none,       none,       0,       0x87,      r,   no,  0,     O64 | PLOCK)

INSTR (XLATB,             none,       none,       none,       none,       0,       0xd7,      no,  no,  0,     0)

INSTR (XOR,               al,         imm8,       none,       none,       0,       0x34,      ib,  no,  0,     0)
INSTR (XOR,               regmem8,    imm8,       none,       none,       0,       0x80,      r6,  ib,  0,     PLOCK)
INSTR (XOR,               regmem8,    imm8,       none,       none,       0,       0x82,      r6,  ib,  0,     I64 | PLOCK)
INSTR (XOR,               regmem16,   simm8,      none,       none,       0,       0x83,      r6,  ib,  0,     O16 | PLOCK)
INSTR (XOR,               regmem32,   simm8,      none,       none,       0,       0x83,      r6,  ib,  0,     O32 | PLOCK)
INSTR (XOR,               regmem64,   simm8,      none,       none,       0,       0x83,      r6,  ib,  0,     O64 | PLOCK)
INSTR (XOR,               ax,         imm16,      none,       none,       0,       0x35,      iw,  no,  0,     O16)
INSTR (XOR,               regmem16,   imm16,      none,       none,       0,       0x81,      r6,  iw,  0,     O16 | PLOCK)
INSTR (XOR,               eax,        imm32,      none,       none,       0,       0x35,      id,  no,  0,     O32)
INSTR (XOR,               regmem32,   imm32,      none,       none,       0,       0x81,      r6,  id,  0,     O32 | PLOCK)
INSTR (XOR,               rax,        simm32,     none,       none,       0,       0x35,      id,  no,  0,     O64)
INSTR (XOR,               regmem64,   simm32,     none,       none,       0,       0x81,      r6,  id,  0,     O64 | PLOCK)
INSTR (XOR,               regmem8,    reg8,       none,       none,       0,       0x30,      r,   no,  0,     PLOCK)
INSTR (XOR,               regmem16,   reg16,      none,       none,       0,       0x31,      r,   no,  0,     O16 | PLOCK)
INSTR (XOR,               regmem32,   reg32,      none,       none,       0,       0x31,      r,   no,  0,     O32 | PLOCK)
INSTR (XOR,               regmem64,   reg64,      none,       none,       0,       0x31,      r,   no,  0,     O64 | PLOCK)
INSTR (XOR,               reg8,       regmem8,    none,       none,       0,       0x32,      r,   no,  0,     0)
INSTR (XOR,               reg16,      regmem16,   none,       none,       0,       0x33,      r,   no,  0,     O16)
INSTR (XOR,               reg32,      regmem32,   none,       none,       0,       0x33,      r,   no,  0,     O32)
INSTR (XOR,               reg64,      regmem64,   none,       none,       0,       0x33,      r,   no,  0,     O64)

// System Instruction Reference

INSTR (ARPL,              regmem16,   reg16,      none,       none,       0,       0x63,      r,   no,  0,     I64)

INSTR (CLI,               none,       none,       none,       none,       0,       0xfa,      no,  no,  0,     0)

INSTR (CLTS,              none,       none,       none,       none,       0,       0x0f06,    no,  no,  0,     0)

INSTR (CLRSSBSY,          mem64,      none,       none,       none,       0,       0x0fae,    r6,  no,  0,     PF3)

INSTR (HLT,               none,       none,       none,       none,       0,       0xf4,      no,  no,  0,     0)

INSTR (INCSSP,            reg32rm,    none,       none,       none,       0,       0x0fae,    r5,  no,  0,     O16 | O32 | PF3)
INSTR (INCSSP,            reg64rm,    none,       none,       none,       0,       0x0fae,    r5,  no,  0,     O64 | PF3)

INSTR (INT3,              none,       none,       none,       none,       0,       0xcc,      no,  no,  0,     0)

INSTR (INVD,              none,       none,       none,       none,       0,       0x0f08,    no,  no,  0,     0)

INSTR (INVLPG,            mem,        none,       none,       none,       0,       0x0f01,    r7,  no,  0,     0)

INSTR (IRET,              none,       none,       none,       none,       0,       0xcf,      no,  no,  0,     O16)

INSTR (IRETD,             none,       none,       none,       none,       0,       0xcf,      no,  no,  0,     O32)

INSTR (IRETQ,             none,       none,       none,       none,       0,       0xcf,      no,  no,  0,     O64)

INSTR (LAR,               reg16,      regmem16,   none,       none,       0,       0x0f02,    r,   no,  0,     O16)
INSTR (LAR,               reg32,      regmem16,   none,       none,       0,       0x0f02,    r,   no,  0,     O32)
INSTR (LAR,               reg64,      regmem16,   none,       none,       0,       0x0f02,    r,   no,  0,     O64)

INSTR (LGDT,              mem,        none,       none,       none,       0,       0x0f01,    r2,  no,  0,     0)

INSTR (LIDT,              mem,        none,       none,       none,       0,       0x0f01,    r3,  no,  0,     0)

INSTR (LLDT,              regmem16,   none,       none,       none,       0,       0x0f00,    r2,  no,  0,     0)

INSTR (LMSW,              regmem16,   none,       none,       none,       0,       0x0f01,    r6,  no,  0,     0)

INSTR (LSL,               reg16,      regmem16,   none,       none,       0,       0x0f03,    r,   no,  0,     O16)
INSTR (LSL,               reg32,      regmem16,   none,       none,       0,       0x0f03,    r,   no,  0,     O32)
INSTR (LSL,               reg64,      regmem16,   none,       none,       0,       0x0f03,    r,   no,  0,     O64)

INSTR (LTR,               regmem16,   none,       none,       none,       0,       0x0f00,    r3,  no,  0,     0)

INSTR (RDMSR,             none,       none,       none,       none,       0,       0x0f32,    no,  no,  0,     0)

INSTR (RDPKRU,            none,       none,       none,       none,       0,       0x0f01,    no,  no,  0xee,  SFX)

INSTR (RDPMC,             none,       none,       none,       none,       0,       0x0f33,    no,  no,  0,     0)

INSTR (RDSSPD,            reg32rm,    none,       none,       none,       0,       0x0f1e,    r1,  no,  0,     O16 | O32 | PF3)

INSTR (RDSSPQ,            reg64rm,    none,       none,       none,       0,       0x0f1e,    r1,  no,  0,     O64 | PF3)

INSTR (RDTSC,             none,       none,       none,       none,       0,       0x0f31,    no,  no,  0,     0)

INSTR (RSM,               none,       none,       none,       none,       0,       0x0faa,    no,  no,  0,     0)

INSTR (SAVEPREVSSP,       none,       none,       none,       none,       0,       0x0f01,    no,  no,  0xea,  PF3 | SFX)

INSTR (SETSSBSY,          none,       none,       none,       none,       0,       0x0f01,    no,  no,  0xe8,  PF3 | SFX)

INSTR (SGDT,              mem,        none,       none,       none,       0,       0x0f01,    r0,  no,  0,     0)

INSTR (SIDT,              mem,        none,       none,       none,       0,       0x0f01,    r1,  no,  0,     0)

INSTR (SLDT,              regmem16,   none,       none,       none,       0,       0x0f00,    r0,  no,  0,     0)

INSTR (SMSW,              regmem16,   none,       none,       none,       0,       0x0f01,    r4,  no,  0,     0)

INSTR (STI,               none,       none,       none,       none,       0,       0xfb,      no,  no,  0,     0)

INSTR (STR,               regmem16,   none,       none,       none,       0,       0x0f00,    r1,  no,  0,     0)

INSTR (SYSCALL,           none,       none,       none,       none,       0,       0x0f05,    no,  no,  0,     0)

INSTR (SYSENTER,          none,       none,       none,       none,       0,       0x0f34,    no,  no,  0,     0)

INSTR (SYSEXIT,           none,       none,       none,       none,       0,       0x0f35,    no,  no,  0,     0)

INSTR (SYSRET,            none,       none,       none,       none,       0,       0x0f07,    no,  no,  0,     0)

INSTR (TLBSYNC,           none,       none,       none,       none,       0,       0x0f01,    no,  no,  0xff,  SFX)

INSTR (UD0,               none,       none,       none,       none,       0,       0x0fff,    no,  no,  0,     0)

INSTR (UD1,               none,       none,       none,       none,       0,       0x0fb9,    r,   no,  0,     0)

INSTR (UD2,               none,       none,       none,       none,       0,       0x0f0b,    no,  no,  0,     0)

INSTR (VERR,              regmem16,   none,       none,       none,       0,       0x0f00,    r4,  no,  0,     0)

INSTR (VERW,              regmem16,   none,       none,       none,       0,       0x0f00,    r5,  no,  0,     0)

INSTR (WBINVD,            none,       none,       none,       none,       0,       0x0f09,    no,  no,  0,     0)

INSTR (WBNOINVD,          none,       none,       none,       none,       0,       0x0f09,    no,  no,  0,     PF3)

INSTR (WRMSR,             none,       none,       none,       none,       0,       0x0f30,    no,  no,  0,     0)

INSTR (WRPKRU,            none,       none,       none,       none,       0,       0x0f01,    no,  no,  0xef,  SFX)

INSTR (CLGI,              none,       none,       none,       none,       0,       0x0f01,    no,  no,  0xdd,  SFX)

INSTR (CLZERO,            none,       none,       none,       none,       0,       0x0f01,    no,  no,  0xfc,  SFX)

INSTR (INVLPGA,           ax,         ecx,        none,       none,       0,       0x0f01,    no,  no,  0xdf,  O16 | SFX)
INSTR (INVLPGA,           eax,        ecx,        none,       none,       0,       0x0f01,    no,  no,  0xdf,  O32 | SFX)
INSTR (INVLPGA,           rax,        ecx,        none,       none,       0,       0x0f01,    no,  no,  0xdf,  O64 | SFX)

INSTR (INVLPGB,           none,       none,       none,       none,       0,       0x0f01,    no,  no,  0xfe,  SFX)

INSTR (INVPCID,           reg32,      mem128,     none,       none,       0,       0x0f3882,  r,   no,  0,     O16 | O32 | P66)
INSTR (INVPCID,           reg64,      mem128,     none,       none,       0,       0x0f3882,  r,   no,  0,     O64 | P66)

INSTR (MONITOR,           none,       none,       none,       none,       0,       0x0f01,    no,  no,  0xc8,  SFX)

INSTR (MONITORX,          none,       none,       none,       none,       0,       0x0f01,    no,  no,  0xfa,  SFX)

INSTR (MWAIT,             none,       none,       none,       none,       0,       0x0f01,    no,  no,  0xc9,  SFX)

INSTR (MWAITX,            none,       none,       none,       none,       0,       0x0f01,    no,  no,  0xfb,  SFX)

INSTR (PSMASH,            none,       none,       none,       none,       0,       0x0f01,    no,  no,  0xff,  PF3 | SFX)

INSTR (PVALIDATE,         none,       none,       none,       none,       0,       0x0f01,    no,  no,  0xff,  PF2 | SFX)

INSTR (RDTSCP,            none,       none,       none,       none,       0,       0x0f01,    no,  no,  0xf9,  SFX)

INSTR (RMPADJUST,         none,       none,       none,       none,       0,       0x0f01,    no,  no,  0xfe,  PF3 | SFX)

INSTR (RMPQUERY,          none,       none,       none,       none,       0,       0x0f01,    no,  no,  0xfd,  PF3 | SFX)

INSTR (RMPREAD,           none,       none,       none,       none,       0,       0x0f01,    no,  no,  0xfd,  PF2 | SFX)

INSTR (RMPUPDATE,         none,       none,       none,       none,       0,       0x0f01,    no,  no,  0xfe,  PF2 | SFX)

INSTR (SKINIT,            eax,        none,       none,       none,       0,       0x0f01,    no,  no,  0xde,  SFX)

INSTR (STGI,              none,       none,       none,       none,       0,       0x0f01,    no,  no,  0xdc,  SFX)

INSTR (SWAPGS,            none,       none,       none,       none,       0,       0x0f01,    no,  no,  0xf8,  I16 | I32 | SFX)

INSTR (VMLOAD,            ax,         none,       none,       none,       0,       0x0f01,    no,  no,  0xda,  O16 | SFX)
INSTR (VMLOAD,            eax,        none,       none,       none,       0,       0x0f01,    no,  no,  0xda,  O32 | SFX)
INSTR (VMLOAD,            rax,        none,       none,       none,       0,       0x0f01,    no,  no,  0xda,  O64 | SFX)

INSTR (VMMCALL,           none,       none,       none,       none,       0,       0x0f01,    no,  no,  0xd9,  SFX)

INSTR (VMGEXIT,           none,       none,       none,       none,       0,       0x0f01,    no,  no,  0xd9,  PF2 | SFX)
INSTR (VMGEXIT,           none,       none,       none,       none,       0,       0x0f01,    no,  no,  0xd9,  PF3 | SFX)

INSTR (VMRUN,             ax,         none,       none,       none,       0,       0x0f01,    no,  no,  0xd8,  O16 | SFX)
INSTR (VMRUN,             eax,        none,       none,       none,       0,       0x0f01,    no,  no,  0xd8,  O32 | SFX)
INSTR (VMRUN,             rax,        none,       none,       none,       0,       0x0f01,    no,  no,  0xd8,  O64 | SFX)

INSTR (VMSAVE,            ax,         none,       none,       none,       0,       0x0f01,    no,  no,  0xdb,  O16 | SFX)
INSTR (VMSAVE,            eax,        none,       none,       none,       0,       0x0f01,    no,  no,  0xdb,  O32 | SFX)
INSTR (VMSAVE,            rax,        none,       none,       none,       0,       0x0f01,    no,  no,  0xdb,  O64 | SFX)

INSTR (WRSSD,             mem32,      reg32,      none,       none,       0,       0x0f38f6,  r,   no,  0,     O16 | O32)
INSTR (WRSSQ,             mem64,      reg64,      none,       none,       0,       0x0f38f6,  r,   no,  0,     O64)

INSTR (WRUSSD,            mem32,      reg32,      none,       none,       0,       0x0f38f5,  r,   no,  0,     O16 | O32 | P66)
INSTR (WRUSSQ,            mem64,      reg64,      none,       none,       0,       0x0f38f5,  r,   no,  0,     O64 | P66)

// 128-Bit and 256-Bit Media Instruction Reference

INSTR (ADDPD,             xmm,        xmmmem128,  none,       none,       0,       0x0f58,    r,   no,  0,     P66)

INSTR (VADDPD,            xmm,        xvvvv,      xmmmem128,  none,       0x0101,  0x58,      r,   no,  0,     PVEX)
INSTR (VADDPD,            ymm,        yvvvv,      ymmmem256,  none,       0x0105,  0x58,      r,   no,  0,     PVEX)

INSTR (ADDSD,             xmm,        xmmmem64,   none,       none,       0,       0x0f58,    r,   no,  0,     PF2)

INSTR (VADDSD,            xmm,        xvvvv,      xmmmem64,   none,       0x0103,  0x58,      r,   no,  0,     PVEX)

INSTR (ADDSS,             xmm,        xmmmem32,   none,       none,       0,       0x0f58,    r,   no,  0,     PF3)

INSTR (VADDSS,            xmm,        xvvvv,      xmmmem32,   none,       0x0102,  0x58,      r,   no,  0,     PVEX)

INSTR (ADDPS,             xmm,        xmmmem128,  none,       none,       0,       0x0f58,    r,   no,  0,     0)

INSTR (VADDPS,            xmm,        xvvvv,      xmmmem128,  none,       0x0100,  0x58,      r,   no,  0,     PVEX)
INSTR (VADDPS,            ymm,        yvvvv,      ymmmem256,  none,       0x0104,  0x58,      r,   no,  0,     PVEX)

INSTR (ADDSUBPD,          xmm,        xmmmem128,  none,       none,       0,       0x0fd0,    r,   no,  0,     P66)

INSTR (VADDSUBPD,         xmm,        xvvvv,      xmmmem128,  none,       0x0101,  0xd0,      r,   no,  0,     PVEX)
INSTR (VADDSUBPD,         ymm,        yvvvv,      ymmmem256,  none,       0x0105,  0xd0,      r,   no,  0,     PVEX)

INSTR (ADDSUBPS,          xmm,        xmmmem128,  none,       none,       0,       0x0fd0,    r,   no,  0,     PF2)

INSTR (VADDSUBPS,         xmm,        xvvvv,      xmmmem128,  none,       0x0103,  0xd0,      r,   no,  0,     PVEX)
INSTR (VADDSUBPS,         ymm,        yvvvv,      ymmmem256,  none,       0x0107,  0xd0,      r,   no,  0,     PVEX)

INSTR (AESDEC,            xmm,        xmmmem128,  none,       none,       0,       0x0f38de,  r,   no,  0,     P66)

INSTR (VAESDEC,           xmm,        xvvvv,      xmmmem128,  none,       0x0201,  0xde,      r,   no,  0,     PVEX)

INSTR (AESDECLAST,        xmm,        xmmmem128,  none,       none,       0,       0x0f38df,  r,   no,  0,     P66)

INSTR (VAESDECLAST,       xmm,        xvvvv,      xmmmem128,  none,       0x0201,  0xdf,      r,   no,  0,     PVEX)

INSTR (AESENC,            xmm,        xmmmem128,  none,       none,       0,       0x0f38dc,  r,   no,  0,     P66)

INSTR (VAESENC,           xmm,        xvvvv,      xmmmem128,  none,       0x0201,  0xdc,      r,   no,  0,     PVEX)

INSTR (AESENCLAST,        xmm,        xmmmem128,  none,       none,       0,       0x0f38dd,  r,   no,  0,     P66)

INSTR (VAESENCLAST,       xmm,        xvvvv,      xmmmem128,  none,       0x0201,  0xdd,      r,   no,  0,     PVEX)

INSTR (AESIMC,            xmm,        xmmmem128,  none,       none,       0,       0x0f38db,  r,   no,  0,     P66)

INSTR (VAESIMC,           xmm,        xmmmem128,  none,       none,       0x0201,  0xdb,      r,   no,  0,     PVEX)

INSTR (AESKEYGENASSIST,   xmm,        xmmmem128,  imm8,       none,       0,       0x0f3adf,  r,   ib,  0,     P66)

INSTR (VAESKEYGENASSIST,  xmm,        xmmmem128,  imm8,       none,       0x0301,  0xdf,      r,   ib,  0,     PVEX)

INSTR (ANDN,              reg32,      reg32vvvv,  regmem32,   none,       0x0200,  0xf2,      r,   no,  0,     PVEX)
INSTR (ANDN,              reg64,      reg64vvvv,  regmem64,   none,       0x0280,  0xf2,      r,   no,  0,     PVEX)

INSTR (ANDNPD,            xmm,        xmmmem128,  none,       none,       0,       0x0f55,    r,   no,  0,     P66)

INSTR (VANDNPD,           xmm,        xvvvv,      xmmmem128,  none,       0x0101,  0x55,      r,   no,  0,     PVEX)
INSTR (VANDNPD,           ymm,        yvvvv,      ymmmem256,  none,       0x0105,  0x55,      r,   no,  0,     PVEX)

INSTR (ANDNPS,            xmm,        xmmmem128,  none,       none,       0,       0x0f55,    r,   no,  0,     0)

INSTR (VANDNPS,           xmm,        xvvvv,      xmmmem128,  none,       0x0100,  0x55,      r,   no,  0,     PVEX)
INSTR (VANDNPS,           ymm,        yvvvv,      ymmmem256,  none,       0x0104,  0x55,      r,   no,  0,     PVEX)

INSTR (ANDPD,             xmm,        xmmmem128,  none,       none,       0,       0x0f54,    r,   no,  0,     P66)

INSTR (VANDPD,            xmm,        xvvvv,      xmmmem128,  none,       0x0101,  0x54,      r,   no,  0,     PVEX)
INSTR (VANDPD,            ymm,        yvvvv,      ymmmem256,  none,       0x0105,  0x54,      r,   no,  0,     PVEX)

INSTR (ANDPS,             xmm,        xmmmem128,  none,       none,       0,       0x0f54,    r,   no,  0,     0)

INSTR (VANDPS,            xmm,        xvvvv,      xmmmem128,  none,       0x0100,  0x54,      r,   no,  0,     PVEX)
INSTR (VANDPS,            ymm,        yvvvv,      ymmmem256,  none,       0x0104,  0x54,      r,   no,  0,     PVEX)

INSTR (BEXTR,             reg32,      regmem32,   reg32vvvv,  none,       0x0200,  0xf7,      r,   no,  0,     PVEX)
INSTR (BEXTR,             reg64,      regmem64,   reg64vvvv,  none,       0x0280,  0xf7,      r,   no,  0,     PVEX)
INSTR (BEXTR,             reg32,      regmem32,   imm32,      none,       0x0a78,  0x10,      r,   id,  0,     PXOP)
INSTR (BEXTR,             reg64,      regmem64,   imm32,      none,       0x0af8,  0x10,      r,   id,  0,     PXOP)

INSTR (BLCFILL,           reg32vvvv,  regmem32,   none,       none,       0x0900,  0x01,      r1,  no,  0,     PXOP)
INSTR (BLCFILL,           reg64vvvv,  regmem64,   none,       none,       0x0980,  0x01,      r1,  no,  0,     PXOP)

INSTR (BLCI,              reg32vvvv,  regmem32,   none,       none,       0x0900,  0x02,      r6,  no,  0,     PXOP)
INSTR (BLCI,              reg64vvvv,  regmem64,   none,       none,       0x0980,  0x02,      r6,  no,  0,     PXOP)

INSTR (BLCIC,             reg32vvvv,  regmem32,   none,       none,       0x0900,  0x01,      r5,  no,  0,     PXOP)
INSTR (BLCIC,             reg64vvvv,  regmem64,   none,       none,       0x0980,  0x01,      r5,  no,  0,     PXOP)

INSTR (BLCMSK,            reg32vvvv,  regmem32,   none,       none,       0x0900,  0x02,      r1,  no,  0,     PXOP)
INSTR (BLCMSK,            reg64vvvv,  regmem64,   none,       none,       0x0980,  0x02,      r1,  no,  0,     PXOP)

INSTR (BLCS,              reg32vvvv,  regmem32,   none,       none,       0x0900,  0x01,      r3,  no,  0,     PXOP)
INSTR (BLCS,              reg64vvvv,  regmem64,   none,       none,       0x0980,  0x01,      r3,  no,  0,     PXOP)

INSTR (BLENDPD,           xmm,        xmmmem128,  imm8,       none,       0,       0x0f3a0d,  r,   ib,  0,     P66)

INSTR (VBLENDPD,          xmm,        xvvvv,      xmmmem128,  imm8,       0x0301,  0x0d,      r,   ib,  0,     PVEX)
INSTR (VBLENDPD,          ymm,        yvvvv,      ymmmem256,  imm8,       0x0305,  0x0d,      r,   ib,  0,     PVEX)

INSTR (BLENDPS,           xmm,        xmmmem128,  imm8,       none,       0,       0x0f3a0c,  r,   ib,  0,     P66)

INSTR (VBLENDPS,          xmm,        xvvvv,      xmmmem128,  imm8,       0x0301,  0x0c,      r,   ib,  0,     PVEX)
INSTR (VBLENDPS,          ymm,        yvvvv,      ymmmem256,  imm8,       0x0305,  0x0c,      r,   ib,  0,     PVEX)

INSTR (BLENDVPD,          xmm,        xmmmem128,  none,       none,       0,       0x0f3815,  r,   no,  0,     P66)

INSTR (VBLENDVPD,         xmm,        xvvvv,      xmmmem128,  ximm,       0x0301,  0x4b,      r,   is,  0,     PVEX)
INSTR (VBLENDVPD,         ymm,        yvvvv,      ymmmem256,  yimm,       0x0305,  0x4b,      r,   is,  0,     PVEX)

INSTR (BLENDVPS,          xmm,        xmmmem128,  none,       none,       0,       0x0f3814,  r,   no,  0,     P66)

INSTR (VBLENDVPS,         xmm,        xvvvv,      xmmmem128,  ximm,       0x0301,  0x4a,      r,   is,  0,     PVEX)
INSTR (VBLENDVPS,         ymm,        yvvvv,      ymmmem256,  yimm,       0x0305,  0x4a,      r,   is,  0,     PVEX)

INSTR (BLSFILL,           reg32vvvv,  regmem32,   none,       none,       0x0900,  0x01,      r2,  no,  0,     PXOP)
INSTR (BLSFILL,           reg64vvvv,  regmem64,   none,       none,       0x0980,  0x01,      r2,  no,  0,     PXOP)

INSTR (BLSI,              reg32vvvv,  regmem32,   none,       none,       0x0200,  0xf3,      r3,  no,  0,     PXOP)
INSTR (BLSI,              reg64vvvv,  regmem64,   none,       none,       0x0280,  0xf3,      r3,  no,  0,     PXOP)

INSTR (BLSIC,             reg32vvvv,  regmem32,   none,       none,       0x0900,  0x01,      r6,  no,  0,     PXOP)
INSTR (BLSIC,             reg64vvvv,  regmem64,   none,       none,       0x0980,  0x01,      r6,  no,  0,     PXOP)

INSTR (BLSMSK,            reg32vvvv,  regmem32,   none,       none,       0x0200,  0xf3,      r2,  no,  0,     PXOP)
INSTR (BLSMSK,            reg64vvvv,  regmem64,   none,       none,       0x0280,  0xf3,      r2,  no,  0,     PXOP)

INSTR (BLSR,              reg32vvvv,  regmem32,   none,       none,       0x0200,  0xf3,      r1,  no,  0,     PXOP)
INSTR (BLSR,              reg64vvvv,  regmem64,   none,       none,       0x0280,  0xf3,      r1,  no,  0,     PXOP)

INSTR (CMPPD,             xmm,        xmmmem128,  imm8,       none,       0,       0x0fc2,    r,   ib,  0,     P66)

INSTR (VCMPPD,            xmm,        xvvvv,      xmmmem128,  imm8,       0x0101,  0xc2,      r,   ib,  0,     PVEX)
INSTR (VCMPPD,            ymm,        yvvvv,      ymmmem256,  imm8,       0x0105,  0xc2,      r,   ib,  0,     PVEX)

INSTR (CMPEQPD,           xmm,        xmmmem128,  none,       none,       0,       0x0fc2,    r,   no,  0x00,  P66 | SFX)

INSTR (VCMPEQPD,          xmm,        xvvvv,      xmmmem128,  none,       0x0101,  0xc2,      r,   no,  0x00,  PVEX | SFX)
INSTR (VCMPEQPD,          ymm,        yvvvv,      ymmmem256,  none,       0x0105,  0xc2,      r,   no,  0x00,  PVEX | SFX)

INSTR (CMPLTPD,           xmm,        xmmmem128,  none,       none,       0,       0x0fc2,    r,   no,  0x01,  P66 | SFX)

INSTR (VCMPLTPD,          xmm,        xvvvv,      xmmmem128,  none,       0x0101,  0xc2,      r,   no,  0x01,  PVEX | SFX)
INSTR (VCMPLTPD,          ymm,        yvvvv,      ymmmem256,  none,       0x0105,  0xc2,      r,   no,  0x01,  PVEX | SFX)

INSTR (CMPLEPD,           xmm,        xmmmem128,  none,       none,       0,       0x0fc2,    r,   no,  0x02,  P66 | SFX)

INSTR (VCMPLEPD,          xmm,        xvvvv,      xmmmem128,  none,       0x0101,  0xc2,      r,   no,  0x02,  PVEX | SFX)
INSTR (VCMPLEPD,          ymm,        yvvvv,      ymmmem256,  none,       0x0105,  0xc2,      r,   no,  0x02,  PVEX | SFX)

INSTR (CMPUNORDPD,        xmm,        xmmmem128,  none,       none,       0,       0x0fc2,    r,   no,  0x03,  P66 | SFX)

INSTR (VCMPUNORDPD,       xmm,        xvvvv,      xmmmem128,  none,       0x0101,  0xc2,      r,   no,  0x03,  PVEX | SFX)
INSTR (VCMPUNORDPD,       ymm,        yvvvv,      ymmmem256,  none,       0x0105,  0xc2,      r,   no,  0x03,  PVEX | SFX)

INSTR (CMPNEQPD,          xmm,        xmmmem128,  none,       none,       0,       0x0fc2,    r,   no,  0x04,  P66 | SFX)

INSTR (VCMPNEQPD,         xmm,        xvvvv,      xmmmem128,  none,       0x0101,  0xc2,      r,   no,  0x04,  PVEX | SFX)
INSTR (VCMPNEQPD,         ymm,        yvvvv,      ymmmem256,  none,       0x0105,  0xc2,      r,   no,  0x04,  PVEX | SFX)

INSTR (CMPNLTPD,          xmm,        xmmmem128,  none,       none,       0,       0x0fc2,    r,   no,  0x05,  P66 | SFX)

INSTR (VCMPNLTPD,         xmm,        xvvvv,      xmmmem128,  none,       0x0101,  0xc2,      r,   no,  0x05,  PVEX | SFX)
INSTR (VCMPNLTPD,         ymm,        yvvvv,      ymmmem256,  none,       0x0105,  0xc2,      r,   no,  0x05,  PVEX | SFX)

INSTR (CMPNLEPD,          xmm,        xmmmem128,  none,       none,       0,       0x0fc2,    r,   no,  0x06,  P66 | SFX)

INSTR (VCMPNLEPD,         xmm,        xvvvv,      xmmmem128,  none,       0x0101,  0xc2,      r,   no,  0x06,  PVEX | SFX)
INSTR (VCMPNLEPD,         ymm,        yvvvv,      ymmmem256,  none,       0x0105,  0xc2,      r,   no,  0x06,  PVEX | SFX)

INSTR (CMPORDPD,          xmm,        xmmmem128,  none,       none,       0,       0x0fc2,    r,   no,  0x07,  P66 | SFX)

INSTR (VCMPORDPD,         xmm,        xvvvv,      xmmmem128,  none,       0x0101,  0xc2,      r,   no,  0x07,  PVEX | SFX)
INSTR (VCMPORDPD,         ymm,        yvvvv,      ymmmem256,  none,       0x0105,  0xc2,      r,   no,  0x07,  PVEX | SFX)

INSTR (VCMPSD,            xmm,        xvvvv,      xmmmem64,   imm8,       0x0103,  0xc2,      r,   ib,  0,     PVEX)

INSTR (CMPEQSD,           xmm,        xmmmem64,   none,       none,       0,       0x0fc2,    r,   no,  0x00,  PF2 | SFX)

INSTR (VCMPEQSD,          xmm,        xvvvv,      xmmmem64,   none,       0x0103,  0xc2,      r,   no,  0x00,  PVEX | SFX)

INSTR (CMPLTSD,           xmm,        xmmmem64,   none,       none,       0,       0x0fc2,    r,   no,  0x01,  PF2 | SFX)

INSTR (VCMPLTSD,          xmm,        xvvvv,      xmmmem64,   none,       0x0103,  0xc2,      r,   no,  0x01,  PVEX | SFX)

INSTR (CMPLESD,           xmm,        xmmmem64,   none,       none,       0,       0x0fc2,    r,   no,  0x02,  PF2 | SFX)

INSTR (VCMPLESD,          xmm,        xvvvv,      xmmmem64,   none,       0x0103,  0xc2,      r,   no,  0x02,  PVEX | SFX)

INSTR (CMPUNORDSD,        xmm,        xmmmem64,   none,       none,       0,       0x0fc2,    r,   no,  0x03,  PF2 | SFX)

INSTR (VCMPUNORDSD,       xmm,        xvvvv,      xmmmem64,   none,       0x0103,  0xc2,      r,   no,  0x03,  PVEX | SFX)

INSTR (CMPNEQSD,          xmm,        xmmmem64,   none,       none,       0,       0x0fc2,    r,   no,  0x04,  PF2 | SFX)

INSTR (VCMPNEQSD,         xmm,        xvvvv,      xmmmem64,   none,       0x0103,  0xc2,      r,   no,  0x04,  PVEX | SFX)

INSTR (CMPNLTSD,          xmm,        xmmmem64,   none,       none,       0,       0x0fc2,    r,   no,  0x05,  PF2 | SFX)

INSTR (VCMPNLTSD,         xmm,        xvvvv,      xmmmem64,   none,       0x0103,  0xc2,      r,   no,  0x05,  PVEX | SFX)

INSTR (CMPNLESD,          xmm,        xmmmem64,   none,       none,       0,       0x0fc2,    r,   no,  0x06,  PF2 | SFX)

INSTR (VCMPNLESD,         xmm,        xvvvv,      xmmmem64,   none,       0x0103,  0xc2,      r,   no,  0x06,  PVEX | SFX)

INSTR (CMPORDSD,          xmm,        xmmmem64,   none,       none,       0,       0x0fc2,    r,   no,  0x07,  PF2 | SFX)

INSTR (VCMPORDSD,         xmm,        xvvvv,      xmmmem64,   none,       0x0103,  0xc2,      r,   no,  0x07,  PVEX | SFX)

INSTR (CMPSS,             xmm,        xmmmem32,   imm8,       none,       0,       0x0fc2,    r,   ib,  0,     PF3)

INSTR (VCMPSS,            xmm,        xvvvv,      xmmmem32,   imm8,       0x0102,  0xc2,      r,   ib,  0,     PVEX)

INSTR (CMPEQSS,           xmm,        xmmmem32,   none,       none,       0,       0x0fc2,    r,   no,  0x00,  PF3 | SFX)

INSTR (VCMPEQSS,          xmm,        xvvvv,      xmmmem32,   none,       0x0102,  0xc2,      r,   no,  0x00,  PVEX | SFX)

INSTR (CMPLTSS,           xmm,        xmmmem32,   none,       none,       0,       0x0fc2,    r,   no,  0x01,  PF3 | SFX)

INSTR (VCMPLTSS,          xmm,        xvvvv,      xmmmem32,   none,       0x0102,  0xc2,      r,   no,  0x01,  PVEX | SFX)

INSTR (CMPLESS,           xmm,        xmmmem32,   none,       none,       0,       0x0fc2,    r,   no,  0x02,  PF3 | SFX)

INSTR (VCMPLESS,          xmm,        xvvvv,      xmmmem32,   none,       0x0102,  0xc2,      r,   no,  0x02,  PVEX | SFX)

INSTR (CMPUNORDSS,        xmm,        xmmmem32,   none,       none,       0,       0x0fc2,    r,   no,  0x03,  PF3 | SFX)

INSTR (VCMPUNORDSS,       xmm,        xvvvv,      xmmmem32,   none,       0x0102,  0xc2,      r,   no,  0x03,  PVEX | SFX)

INSTR (CMPNEQSS,          xmm,        xmmmem32,   none,       none,       0,       0x0fc2,    r,   no,  0x04,  PF3 | SFX)

INSTR (VCMPNEQSS,         xmm,        xvvvv,      xmmmem32,   none,       0x0102,  0xc2,      r,   no,  0x04,  PVEX | SFX)

INSTR (CMPNLTSS,          xmm,        xmmmem32,   none,       none,       0,       0x0fc2,    r,   no,  0x05,  PF3 | SFX)

INSTR (VCMPNLTSS,         xmm,        xvvvv,      xmmmem32,   none,       0x0102,  0xc2,      r,   no,  0x05,  PVEX | SFX)

INSTR (CMPNLESS,          xmm,        xmmmem32,   none,       none,       0,       0x0fc2,    r,   no,  0x06,  PF3 | SFX)

INSTR (VCMPNLESS,         xmm,        xvvvv,      xmmmem32,   none,       0x0102,  0xc2,      r,   no,  0x06,  PVEX | SFX)

INSTR (CMPORDSS,          xmm,        xmmmem32,   none,       none,       0,       0x0fc2,    r,   no,  0x07,  PF3 | SFX)

INSTR (VCMPORDSS,         xmm,        xvvvv,      xmmmem32,   none,       0x0102,  0xc2,      r,   no,  0x07,  PVEX | SFX)

INSTR (CMPPS,             xmm,        xmmmem128,  imm8,       none,       0,       0x0fc2,    r,   ib,  0,     0)

INSTR (VCMPPS,            xmm,        xvvvv,      xmmmem128,  imm8,       0x0100,  0xc2,      r,   ib,  0,     PVEX)

INSTR (CMPEQPS,           xmm,        xmmmem128,  none,       none,       0,       0x0fc2,    r,   no,  0x00,  SFX)

INSTR (VCMPEQPS,          xmm,        xvvvv,      xmmmem128,  none,       0x0100,  0xc2,      r,   no,  0x00,  PVEX | SFX)

INSTR (CMPLTPS,           xmm,        xmmmem128,  none,       none,       0,       0x0fc2,    r,   no,  0x01,  SFX)

INSTR (VCMPLTPS,          xmm,        xvvvv,      xmmmem128,  none,       0x0100,  0xc2,      r,   no,  0x01,  PVEX | SFX)

INSTR (CMPLEPS,           xmm,        xmmmem128,  none,       none,       0,       0x0fc2,    r,   no,  0x02,  SFX)

INSTR (VCMPLEPS,          xmm,        xvvvv,      xmmmem128,  none,       0x0100,  0xc2,      r,   no,  0x02,  PVEX | SFX)

INSTR (CMPUNORDPS,        xmm,        xmmmem128,  none,       none,       0,       0x0fc2,    r,   no,  0x03,  SFX)

INSTR (VCMPUNORDPS,       xmm,        xvvvv,      xmmmem128,  none,       0x0100,  0xc2,      r,   no,  0x03,  PVEX | SFX)

INSTR (CMPNEQPS,          xmm,        xmmmem128,  none,       none,       0,       0x0fc2,    r,   no,  0x04,  SFX)

INSTR (VCMPNEQPS,         xmm,        xvvvv,      xmmmem128,  none,       0x0100,  0xc2,      r,   no,  0x04,  PVEX | SFX)

INSTR (CMPNLTPS,          xmm,        xmmmem128,  none,       none,       0,       0x0fc2,    r,   no,  0x05,  SFX)

INSTR (VCMPNLTPS,         xmm,        xvvvv,      xmmmem128,  none,       0x0100,  0xc2,      r,   no,  0x05,  PVEX | SFX)

INSTR (CMPNLEPS,          xmm,        xmmmem128,  none,       none,       0,       0x0fc2,    r,   no,  0x06,  SFX)

INSTR (VCMPNLEPS,         xmm,        xvvvv,      xmmmem128,  none,       0x0100,  0xc2,      r,   no,  0x06,  PVEX | SFX)

INSTR (CMPORDPS,          xmm,        xmmmem128,  none,       none,       0,       0x0fc2,    r,   no,  0x07,  SFX)

INSTR (VCMPORDPS,         xmm,        xvvvv,      xmmmem128,  none,       0x0100,  0xc2,      r,   no,  0x07,  PVEX | SFX)

INSTR (COMISD,            xmm,        xmmmem64,   none,       none,       0,       0x0f2f,    r,   no,  0,     P66)

INSTR (VCOMISD,           xmm,        xmmmem64,   none,       none,       0x0101,  0x2f,      r,   no,  0,     PVEX)

INSTR (COMISS,            xmm,        xmmmem32,   none,       none,       0,       0x0f2f,    r,   no,  0,     0)

INSTR (VCOMISS,           xmm,        xmmmem32,   none,       none,       0x0100,  0x2f,      r,   no,  0,     PVEX)

INSTR (CVTTPD2DQ,         xmm,        xmmmem128,  none,       none,       0,       0x0fe6,    r,   no,  0,     P66)

INSTR (VCVTTPD2DQ,        xmm,        xmmmem128,  none,       none,       0x0179,  0xe6,      r,   no,  0,     PVEX)
INSTR (VCVTTPD2DQ,        xmm,        ymmmem256,  none,       none,       0x017d,  0xe6,      r,   no,  0,     PVEX)

INSTR (CVTPD2DQ,          xmm,        xmmmem128,  none,       none,       0,       0x0fe6,    r,   no,  0,     PF2)

INSTR (VCVTPD2DQ,         xmm,        xmmmem128,  none,       none,       0x017b,  0xe6,      r,   no,  0,     PVEX)
INSTR (VCVTPD2DQ,         xmm,        ymmmem256,  none,       none,       0x017f,  0xe6,      r,   no,  0,     PVEX)

INSTR (CVTDQ2PD,          xmm,        xmmmem64,   none,       none,       0,       0x0fe6,    r,   no,  0,     PF3)

INSTR (VCVTDQ2PD,         xmm,        xmmmem64,   none,       none,       0x017a,  0xe6,      r,   no,  0,     PVEX)
INSTR (VCVTDQ2PD,         ymm,        ymmmem128,  none,       none,       0x017e,  0xe6,      r,   no,  0,     PVEX)

INSTR (CVTPS2DQ,          xmm,        xmmmem128,  none,       none,       0,       0x0f5b,    r,   no,  0,     P66)

INSTR (VCVTPS2DQ,         xmm,        xmmmem128,  none,       none,       0x0179,  0x5b,      r,   no,  0,     PVEX)
INSTR (VCVTPS2DQ,         ymm,        ymmmem256,  none,       none,       0x017d,  0x5b,      r,   no,  0,     PVEX)

INSTR (CVTTPS2DQ,         xmm,        xmmmem128,  none,       none,       0,       0x0f5b,    r,   no,  0,     PF3)

INSTR (VCVTTPS2DQ,        xmm,        xmmmem128,  none,       none,       0x017a,  0x5b,      r,   no,  0,     PVEX)
INSTR (VCVTTPS2DQ,        ymm,        ymmmem256,  none,       none,       0x017e,  0x5b,      r,   no,  0,     PVEX)

INSTR (CVTDQ2PS,          xmm,        xmmmem128,  none,       none,       0,       0x0f5b,    r,   no,  0,     0)

INSTR (VCVTDQ2PS,         xmm,        xmmmem128,  none,       none,       0x0178,  0x5b,      r,   no,  0,     PVEX)
INSTR (VCVTDQ2PS,         ymm,        ymmmem256,  none,       none,       0x017c,  0x5b,      r,   no,  0,     PVEX)

INSTR (CVTPD2PI,          mmx,        xmmmem128,  none,       none,       0,       0x0f2d,    r,   no,  0,     P66)

INSTR (CVTSD2SI,          reg32,      xmmmem64,   none,       none,       0,       0x0f2d,    r,   no,  0,     O16 | O32 | PF2)
INSTR (CVTSD2SI,          reg64,      xmmmem64,   none,       none,       0,       0x0f2d,    r,   no,  0,     O64 | PF2)

INSTR (VCVTSD2SI,         reg32,      xmmmem64,   none,       none,       0x017b,  0x2d,      r,   no,  0,     PVEX)
INSTR (VCVTSD2SI,         reg64,      xmmmem64,   none,       none,       0x01fb,  0x2d,      r,   no,  0,     PVEX)

INSTR (CVTSS2SI,          reg32,      xmmmem32,   none,       none,       0,       0x0f2d,    r,   no,  0,     O16 | O32 | PF3)
INSTR (CVTSS2SI,          reg64,      xmmmem32,   none,       none,       0,       0x0f2d,    r,   no,  0,     O64 | PF3)

INSTR (VCVTSS2SI,         reg32,      xmmmem32,   none,       none,       0x017a,  0x2d,      r,   no,  0,     PVEX)
INSTR (VCVTSS2SI,         reg64,      xmmmem32,   none,       none,       0x01fa,  0x2d,      r,   no,  0,     PVEX)

INSTR (CVTPS2PI,          mmx,        xmmmem64,   none,       none,       0,       0x0f2d,    r,   no,  0,     0)

INSTR (CVTPI2PD,          xmm,        mmxmem64,   none,       none,       0,       0x0f2a,    r,   no,  0,     P66)

INSTR (CVTSI2SD,          xmm,        regmem32,   none,       none,       0,       0x0f2a,    r,   no,  0,     O16 | O32 | PF2)
INSTR (CVTSI2SD,          xmm,        regmem64,   none,       none,       0,       0x0f2a,    r,   no,  0,     O64 | PF2)

INSTR (VCVTSI2SD,         xmm,        xvvvv,      regmem32,   none,       0x0103,  0x2a,      r,   no,  0,     PVEX)
INSTR (VCVTSI2SD,         xmm,        xvvvv,      regmem64,   none,       0x0183,  0x2a,      r,   no,  0,     PVEX)

INSTR (CVTSI2SS,          xmm,        regmem32,   none,       none,       0,       0x0f2a,    r,   no,  0,     O16 | O32 | PF3)
INSTR (CVTSI2SS,          xmm,        regmem64,   none,       none,       0,       0x0f2a,    r,   no,  0,     O64 | PF3)

INSTR (VCVTSI2SS,         xmm,        xvvvv,      regmem32,   none,       0x0102,  0x2a,      r,   no,  0,     PVEX)
INSTR (VCVTSI2SS,         xmm,        xvvvv,      regmem64,   none,       0x0182,  0x2a,      r,   no,  0,     PVEX)

INSTR (CVTPI2PS,          xmm,        mmxmem64,   none,       none,       0,       0x0f2a,    r,   no,  0,     0)

INSTR (CVTPD2PS,          xmm,        xmmmem128,  none,       none,       0,       0x0f5a,    r,   no,  0,     P66)

INSTR (VCVTPD2PS,         xmm,        xmmmem128,  none,       none,       0x0179,  0x5a,      r,   no,  0,     PVEX)
INSTR (VCVTPD2PS,         xmm,        ymmmem256,  none,       none,       0x017d,  0x5a,      r,   no,  0,     PVEX)

INSTR (CVTSD2SS,          xmm,        xmmmem64,   none,       none,       0,       0x0f5a,    r,   no,  0,     PF2)

INSTR (VCVTSD2SS,         xmm,        xvvvv,      xmmmem64,   none,       0x0103,  0x5a,      r,   no,  0,     PVEX)

INSTR (CVTSS2SD,          xmm,        xmmmem32,   none,       none,       0,       0x0f5a,    r,   no,  0,     PF3)

INSTR (VCVTSS2SD,         xmm,        xvvvv,      xmmmem32,   none,       0x0102,  0x5a,      r,   no,  0,     PVEX)

INSTR (CVTPS2PD,          xmm,        xmmmem64,   none,       none,       0,       0x0f5a,    r,   no,  0,     0)

INSTR (VCVTPS2PD,         xmm,        xmmmem64,   none,       none,       0x0178,  0x5a,      r,   no,  0,     PVEX)
INSTR (VCVTPS2PD,         ymm,        ymmmem128,  none,       none,       0x017c,  0x5a,      r,   no,  0,     PVEX)

INSTR (CVTTPD2PI,         mmx,        xmmmem128,  none,       none,       0,       0x0f2c,    r,   no,  0,     P66)

INSTR (CVTTSD2SI,         reg32,      xmmmem64,   none,       none,       0,       0x0f2c,    r,   no,  0,     O16 | O32 | PF2)
INSTR (CVTTSD2SI,         reg64,      xmmmem64,   none,       none,       0,       0x0f2c,    r,   no,  0,     O64 | PF2)

INSTR (VCVTTSD2SI,        reg32,      xmmmem64,   none,       none,       0x017b,  0x2c,      r,   no,  0,     PVEX)
INSTR (VCVTTSD2SI,        reg64,      xmmmem64,   none,       none,       0x01fb,  0x2c,      r,   no,  0,     PVEX)

INSTR (CVTTSS2SI,         reg32,      xmmmem32,   none,       none,       0,       0x0f2c,    r,   no,  0,     O16 | O32 | PF3)
INSTR (CVTTSS2SI,         reg64,      xmmmem32,   none,       none,       0,       0x0f2c,    r,   no,  0,     O64 | PF3)

INSTR (VCVTTSS2SI,        reg32,      xmmmem32,   none,       none,       0x017a,  0x2c,      r,   no,  0,     PVEX)
INSTR (VCVTTSS2SI,        reg64,      xmmmem32,   none,       none,       0x01fa,  0x2c,      r,   no,  0,     PVEX)

INSTR (CVTTPS2PI,         mmx,        xmmmem64,   none,       none,       0,       0x0f2c,    r,   no,  0,     0)

INSTR (DIVPD,             xmm,        xmmmem128,  none,       none,       0,       0x0f5e,    r,   no,  0,     P66)

INSTR (VDIVPD,            xmm,        xvvvv,      xmmmem128,  none,       0x0101,  0x5e,      r,   no,  0,     PVEX)
INSTR (VDIVPD,            ymm,        yvvvv,      ymmmem256,  none,       0x0105,  0x5e,      r,   no,  0,     PVEX)

INSTR (DIVSD,             xmm,        xmmmem64,   none,       none,       0,       0x0f5e,    r,   no,  0,     PF2)

INSTR (VDIVSD,            xmm,        xvvvv,      xmmmem64,   none,       0x0103,  0x5e,      r,   no,  0,     PVEX)

INSTR (DIVSS,             xmm,        xmmmem32,   none,       none,       0,       0x0f5e,    r,   no,  0,     PF3)

INSTR (VDIVSS,            xmm,        xvvvv,      xmmmem32,   none,       0x0102,  0x5e,      r,   no,  0,     PVEX)

INSTR (DIVPS,             xmm,        xmmmem128,  none,       none,       0,       0x0f5e,    r,   no,  0,     0)

INSTR (VDIVPS,            xmm,        xvvvv,      xmmmem128,  none,       0x0100,  0x5e,      r,   no,  0,     PVEX)
INSTR (VDIVPS,            ymm,        yvvvv,      ymmmem256,  none,       0x0104,  0x5e,      r,   no,  0,     PVEX)

INSTR (DPPD,              xmm,        xmmmem128,  imm8,       none,       0,       0x0f3a41,  r,   ib,  0,     P66)

INSTR (VDPPD,             xmm,        xvvvv,      xmmmem128,  imm8,       0x0301,  0x41,      r,   ib,  0,     PVEX)

INSTR (DPPS,              xmm,        xmmmem128,  imm8,       none,       0,       0x0f3a40,  r,   ib,  0,     P66)

INSTR (VDPPS,             xmm,        xvvvv,      xmmmem128,  imm8,       0x0301,  0x40,      r,   ib,  0,     PVEX)
INSTR (VDPPS,             ymm,        yvvvv,      ymmmem256,  imm8,       0x0305,  0x40,      r,   ib,  0,     PVEX)

INSTR (EXTRACTPS,         regmem32,   xmm,        imm8,       none,       0,       0x0f3a17,  r,   ib,  0,     P66)

INSTR (VEXTRACTPS,        regmem32,   xmm,        imm8,       none,       0x0379,  0x17,      r,   ib,  0,     PVEX)

INSTR (EXTRQ,             xmmmem64,   imm16,      none,       none,       0,       0x0f78,    r0,  iw,  0,     P66)
INSTR (EXTRQ,             xmm,        xmmmem64,   none,       none,       0,       0x0f79,    r,   no,  0,     P66)

INSTR (HADDPD,            xmm,        xmmmem128,  none,       none,       0,       0x0f7c,    r,   no,  0,     P66)

INSTR (VHADDPD,           xmm,        xvvvv,      xmmmem128,  none,       0x0101,  0x7c,      r,   no,  0,     PVEX)
INSTR (VHADDPD,           ymm,        yvvvv,      ymmmem256,  none,       0x0105,  0x7c,      r,   no,  0,     PVEX)

INSTR (HADDPS,            xmm,        xmmmem128,  none,       none,       0,       0x0f7c,    r,   no,  0,     PF2)

INSTR (VHADDPS,           xmm,        xvvvv,      xmmmem128,  none,       0x0103,  0x7c,      r,   no,  0,     PVEX)
INSTR (VHADDPS,           ymm,        yvvvv,      ymmmem256,  none,       0x0107,  0x7c,      r,   no,  0,     PVEX)

INSTR (HSUBPD,            xmm,        xmmmem128,  none,       none,       0,       0x0f7d,    r,   no,  0,     P66)

INSTR (VHSUBPD,           xmm,        xvvvv,      xmmmem128,  none,       0x0101,  0x7d,      r,   no,  0,     PVEX)
INSTR (VHSUBPD,           ymm,        yvvvv,      ymmmem256,  none,       0x0105,  0x7d,      r,   no,  0,     PVEX)

INSTR (HSUBPS,            xmm,        xmmmem128,  none,       none,       0,       0x0f7d,    r,   no,  0,     PF2)

INSTR (VHSUBPS,           xmm,        xvvvv,      xmmmem128,  none,       0x0103,  0x7d,      r,   no,  0,     PVEX)
INSTR (VHSUBPS,           ymm,        yvvvv,      ymmmem256,  none,       0x0107,  0x7d,      r,   no,  0,     PVEX)

INSTR (INSERTPS,          xmm,        xmmmem128,  imm8,       none,       0,       0x0f3a21,  r,   ib,  0,     P66)

INSTR (VINSERTPS,         xmm,        xvvvv,      xmmmem128,  imm8,       0x0301,  0x21,      r,   ib,  0,     PVEX)

INSTR (INSERTQ,           xmm,        xmmmem64,   imm16,      none,       0,       0x0f78,    r,   iw,  0,     PF2)

INSTR (LDDQU,             xmm,        mem128,     none,       none,       0,       0x0ff0,    r,   no,  0,     PF2)

INSTR (VLDDQU,            xmm,        xmmmem128,  none,       none,       0x017b,  0xf0,      r,   no,  0,     PVEX)
INSTR (VLDDQU,            ymm,        ymmmem256,  none,       none,       0x017f,  0xf0,      r,   no,  0,     PVEX)

INSTR (MASKMOVDQU,        xmm,        xmmmem128,  none,       none,       0,       0x0ff7,    r,   no,  0,     P66)

INSTR (VMASKMOVDQU,       xmm,        xmmmem128,  none,       none,       0x0179,  0xf7,      r,   no,  0,     PVEX)

INSTR (MAXPD,             xmm,        xmmmem128,  none,       none,       0,       0x0f5f,    r,   no,  0,     P66)

INSTR (VMAXPD,            xmm,        xvvvv,      xmmmem128,  none,       0x0101,  0x5f,      r,   no,  0,     PVEX)
INSTR (VMAXPD,            ymm,        yvvvv,      ymmmem256,  none,       0x0105,  0x5f,      r,   no,  0,     PVEX)

INSTR (MAXSD,             xmm,        xmmmem64,   none,       none,       0,       0x0f5f,    r,   no,  0,     PF2)

INSTR (VMAXSD,            xmm,        xvvvv,      xmmmem64,   none,       0x0103,  0x5f,      r,   no,  0,     PVEX)

INSTR (MAXSS,             xmm,        xmmmem32,   none,       none,       0,       0x0f5f,    r,   no,  0,     PF3)

INSTR (VMAXSS,            xmm,        xvvvv,      xmmmem32,   none,       0x0102,  0x5f,      r,   no,  0,     PVEX)

INSTR (MAXPS,             xmm,        xmmmem128,  none,       none,       0,       0x0f5f,    r,   no,  0,     0)

INSTR (VMAXPS,            xmm,        xvvvv,      xmmmem128,  none,       0x0100,  0x5f,      r,   no,  0,     PVEX)
INSTR (VMAXPS,            ymm,        yvvvv,      ymmmem256,  none,       0x0104,  0x5f,      r,   no,  0,     PVEX)

INSTR (MINPD,             xmm,        xmmmem128,  none,       none,       0,       0x0f5d,    r,   no,  0,     P66)

INSTR (VMINPD,            xmm,        xvvvv,      xmmmem128,  none,       0x0101,  0x5d,      r,   no,  0,     PVEX)
INSTR (VMINPD,            ymm,        yvvvv,      ymmmem256,  none,       0x0105,  0x5d,      r,   no,  0,     PVEX)

INSTR (MINSD,             xmm,        xmmmem64,   none,       none,       0,       0x0f5d,    r,   no,  0,     PF2)

INSTR (VMINSD,            xmm,        xvvvv,      xmmmem64,   none,       0x0103,  0x5d,      r,   no,  0,     PVEX)

INSTR (MINSS,             xmm,        xmmmem32,   none,       none,       0,       0x0f5d,    r,   no,  0,     PF3)

INSTR (VMINSS,            xmm,        xvvvv,      xmmmem32,   none,       0x0102,  0x5d,      r,   no,  0,     PVEX)

INSTR (MINPS,             xmm,        xmmmem128,  none,       none,       0,       0x0f5d,    r,   no,  0,     0)

INSTR (VMINPS,            xmm,        xvvvv,      xmmmem128,  none,       0x0100,  0x5d,      r,   no,  0,     PVEX)
INSTR (VMINPS,            ymm,        yvvvv,      ymmmem256,  none,       0x0104,  0x5d,      r,   no,  0,     PVEX)

INSTR (MOVAPD,            xmm,        xmmmem128,  none,       none,       0,       0x0f28,    r,   no,  0,     P66)
INSTR (MOVAPD,            xmmmem128,  xmm,        none,       none,       0,       0x0f29,    r,   no,  0,     P66)

INSTR (VMOVAPD,           xmm,        xmmmem128,  none,       none,       0x0179,  0x28,      r,   no,  0,     PVEX)
INSTR (VMOVAPD,           xmmmem128,  xmm,        none,       none,       0x0179,  0x29,      r,   no,  0,     PVEX)
INSTR (VMOVAPD,           ymm,        ymmmem256,  none,       none,       0x017d,  0x28,      r,   no,  0,     PVEX)
INSTR (VMOVAPD,           ymmmem256,  ymm,        none,       none,       0x017d,  0x29,      r,   no,  0,     PVEX)

INSTR (MOVAPS,            xmm,        xmmmem128,  none,       none,       0,       0x0f28,    r,   no,  0,     0)
INSTR (MOVAPS,            xmmmem128,  xmm,        none,       none,       0,       0x0f29,    r,   no,  0,     0)

INSTR (VMOVAPS,           xmm,        xmmmem128,  none,       none,       0x0178,  0x28,      r,   no,  0,     PVEX)
INSTR (VMOVAPS,           xmmmem128,  xmm,        none,       none,       0x0178,  0x29,      r,   no,  0,     PVEX)
INSTR (VMOVAPS,           ymm,        ymmmem256,  none,       none,       0x017c,  0x28,      r,   no,  0,     PVEX)
INSTR (VMOVAPS,           ymmmem256,  ymm,        none,       none,       0x017c,  0x29,      r,   no,  0,     PVEX)

INSTR (MOVLPD,            xmm,        mem64,      none,       none,       0,       0x0f12,    r,   no,  0,     P66)
INSTR (MOVLPD,            mem64,      xmm,        none,       none,       0,       0x0f13,    r,   no,  0,     P66)

INSTR (VMOVLPD,           xmm,        xvvvv,      mem64,      none,       0x0101,  0x12,      r,   no,  0,     PVEX)
INSTR (VMOVLPD,           mem64,      xmm,        none,       none,       0x0179,  0x13,      r,   no,  0,     PVEX)

INSTR (MOVDDUP,           xmm,        xmmmem64,   none,       none,       0,       0x0f12,    r,   no,  0,     PF2)

INSTR (VMOVDDUP,          xmm,        xmmmem64,   none,       none,       0x017b,  0x12,      r,   no,  0,     PVEX)
INSTR (VMOVDDUP,          ymm,        ymmmem256,  none,       none,       0x017f,  0x12,      r,   no,  0,     PVEX)

INSTR (MOVSLDUP,          xmm,        xmmmem128,  none,       none,       0,       0x0f12,    r,   no,  0,     PF3)

INSTR (VMOVSLDUP,         xmm,        xmmmem128,  none,       none,       0x017a,  0x12,      r,   no,  0,     PVEX)
INSTR (VMOVSLDUP,         ymm,        ymmmem256,  none,       none,       0x017e,  0x12,      r,   no,  0,     PVEX)

INSTR (MOVLPS,            xmm,        mem64,      none,       none,       0,       0x0f12,    r,   no,  0,     0)
INSTR (MOVLPS,            mem64,      xmm,        none,       none,       0,       0x0f13,    r,   no,  0,     0)

INSTR (VMOVLPS,           xmm,        xvvvv,      mem64,      none,       0x0100,  0x12,      r,   no,  0,     PVEX)
INSTR (VMOVLPS,           mem64,      xmm,        none,       none,       0x0178,  0x13,      r,   no,  0,     PVEX)

INSTR (MOVHLPS,           xmm,        xmmmem128,  none,       none,       0,       0x0f12,    r,   no,  0,     0)

INSTR (VMOVHLPS,          xmm,        xvvvv,      xmmmem128,  none,       0x0100,  0x12,      r,   no,  0,     PVEX)

INSTR (MOVDQ2Q,           mmx,        xmmmem64,   none,       none,       0,       0x0fd6,    r,   no,  0,     PF2)

INSTR (MOVQ2DQ,           xmm,        mmxmem64,   none,       none,       0,       0x0fd6,    r,   no,  0,     PF3)

INSTR (MOVHPD,            xmm,        mem64,      none,       none,       0,       0x0f16,    r,   no,  0,     P66)
INSTR (MOVHPD,            mem64,      xmm,        none,       none,       0,       0x0f17,    r,   no,  0,     P66)

INSTR (VMOVHPD,           xmm,        xvvvv,      mem64,      none,       0x0101,  0x16,      r,   no,  0,     PVEX)
INSTR (VMOVHPD,           mem64,      xmm,        none,       none,       0x0179,  0x17,      r,   no,  0,     PVEX)

INSTR (MOVSHDUP,          xmm,        xmmmem128,  none,       none,       0,       0x0f16,    r,   no,  0,     PF3)

INSTR (VMOVSHDUP,         xmm,        xmmmem128,  none,       none,       0x017a,  0x16,      r,   no,  0,     PVEX)
INSTR (VMOVSHDUP,         ymm,        ymmmem256,  none,       none,       0x017e,  0x16,      r,   no,  0,     PVEX)

INSTR (MOVHPS,            xmm,        mem64,      none,       none,       0,       0x0f16,    r,   no,  0,     0)
INSTR (MOVHPS,            mem64,      xmm,        none,       none,       0,       0x0f17,    r,   no,  0,     0)

INSTR (VMOVHPS,           xmm,        xvvvv,      mem64,      none,       0x0100,  0x16,      r,   no,  0,     PVEX)
INSTR (VMOVHPS,           mem64,      xmm,        none,       none,       0x0178,  0x17,      r,   no,  0,     PVEX)

INSTR (MOVLHPS,           xmm,        xmmmem128,  none,       none,       0,       0x0f16,    r,   no,  0,     0)

INSTR (VMOVLHPS,          xmm,        xvvvv,      xmmmem128,  none,       0x0100,  0x16,      r,   no,  0,     PVEX)

INSTR (MOVNTDQ,           mem128,     xmm,        none,       none,       0,       0x0fe7,    r,   no,  0,     P66)

INSTR (VMOVNTDQ,          mem128,     xmm,        none,       none,       0x0179,  0xe7,      r,   no,  0,     PVEX)
INSTR (VMOVNTDQ,          mem256,     ymm,        none,       none,       0x017d,  0xe7,      r,   no,  0,     PVEX)

INSTR (MOVNTDQA,          xmm,        mem128,     none,       none,       0,       0x0f382a,  r,   no,  0,     P66)

INSTR (VMOVNTDQA,         xmm,        mem128,     none,       none,       0x0279,  0x2a,      r,   no,  0,     PVEX)
INSTR (VMOVNTDQA,         ymm,        mem256,     none,       none,       0x027d,  0x2a,      r,   no,  0,     PVEX)

INSTR (MOVNTPD,           mem128,     xmm,        none,       none,       0,       0x0f2b,    r,   no,  0,     P66)

INSTR (VMOVNTPD,          mem128,     xmm,        none,       none,       0x0179,  0x2b,      r,   no,  0,     PVEX)
INSTR (VMOVNTPD,          mem256,     ymm,        none,       none,       0x017d,  0x2b,      r,   no,  0,     PVEX)

INSTR (MOVNTPS,           mem128,     xmm,        none,       none,       0,       0x0f2b,    r,   no,  0,     0)

INSTR (VMOVNTPS,          mem128,     xmm,        none,       none,       0x0178,  0x2b,      r,   no,  0,     PVEX)
INSTR (VMOVNTPS,          mem256,     ymm,        none,       none,       0x017c,  0x2b,      r,   no,  0,     PVEX)

INSTR (MOVNTSD,           mem64,      xmm,        none,       none,       0,       0x0f2b,    r,   no,  0,     PF2)

INSTR (MOVNTSS,           mem32,      xmm,        none,       none,       0,       0x0f2b,    r,   no,  0,     PF3)

INSTR (MOVUPD,            xmm,        xmmmem128,  none,       none,       0,       0x0f10,    r,   no,  0,     P66)
INSTR (MOVUPD,            xmmmem128,  xmm,        none,       none,       0,       0x0f11,    r,   no,  0,     P66)

INSTR (VMOVUPD,           xmm,        xmmmem128,  none,       none,       0x0179,  0x10,      r,   no,  0,     PVEX)
INSTR (VMOVUPD,           xmmmem128,  xmm,        none,       none,       0x0179,  0x11,      r,   no,  0,     PVEX)
INSTR (VMOVUPD,           ymm,        ymmmem256,  none,       none,       0x017d,  0x10,      r,   no,  0,     PVEX)
INSTR (VMOVUPD,           ymmmem256,  ymm,        none,       none,       0x017d,  0x11,      r,   no,  0,     PVEX)

INSTR (MOVSS,             xmm,        xmmmem32,   none,       none,       0,       0x0f10,    r,   no,  0,     PF3)
INSTR (MOVSS,             xmmmem32,   xmm,        none,       none,       0,       0x0f11,    r,   no,  0,     PF3)

INSTR (VMOVSS,            xmm,        mem32,      none,       none,       0x017a,  0x10,      r,   no,  0,     PVEX)
INSTR (VMOVSS,            mem32,      xmm,        none,       none,       0x017a,  0x11,      r,   no,  0,     PVEX)
INSTR (VMOVSS,            xmm,        xvvvv,      xmmmem32,   none,       0x0102,  0x10,      r,   no,  0,     PVEX)
INSTR (VMOVSS,            xmm,        xvvvv,      xmmmem32,   none,       0x0102,  0x11,      r,   no,  0,     PVEX)

INSTR (MOVUPS,            xmm,        xmmmem128,  none,       none,       0,       0x0f10,    r,   no,  0,     0)
INSTR (MOVUPS,            xmmmem128,  xmm,        none,       none,       0,       0x0f11,    r,   no,  0,     0)

INSTR (VMOVUPS,           xmm,        xmmmem128,  none,       none,       0x0178,  0x10,      r,   no,  0,     PVEX)
INSTR (VMOVUPS,           xmmmem128,  xmm,        none,       none,       0x0178,  0x11,      r,   no,  0,     PVEX)
INSTR (VMOVUPS,           ymm,        ymmmem256,  none,       none,       0x017c,  0x10,      r,   no,  0,     PVEX)
INSTR (VMOVUPS,           ymmmem256,  ymm,        none,       none,       0x017c,  0x11,      r,   no,  0,     PVEX)

INSTR (MPSADBW,           xmm,        xmmmem128,  imm8,       none,       0,       0x0f3a42,  r,   ib,  0,     P66)

INSTR (VMPSADBW,          xmm,        xvvvv,      xmmmem128,  imm8,       0x0301,  0x42,      r,   ib,  0,     PVEX)
INSTR (VMPSADBW,          ymm,        yvvvv,      ymmmem256,  imm8,       0x0305,  0x42,      r,   ib,  0,     PVEX)

INSTR (MULPD,             xmm,        xmmmem128,  none,       none,       0,       0x0f59,    r,   no,  0,     P66)

INSTR (VMULPD,            xmm,        xvvvv,      xmmmem128,  none,       0x0101,  0x59,      r,   no,  0,     PVEX)
INSTR (VMULPD,            ymm,        yvvvv,      ymmmem256,  none,       0x0105,  0x59,      r,   no,  0,     PVEX)

INSTR (MULSD,             xmm,        xmmmem64,   none,       none,       0,       0x0f59,    r,   no,  0,     PF2)

INSTR (VMULSD,            xmm,        xvvvv,      xmmmem64,   none,       0x0103,  0x59,      r,   no,  0,     PVEX)

INSTR (MULSS,             xmm,        xmmmem32,   none,       none,       0,       0x0f59,    r,   no,  0,     PF3)

INSTR (VMULSS,            xmm,        xvvvv,      xmmmem32,   none,       0x0102,  0x59,      r,   no,  0,     PVEX)

INSTR (MULPS,             xmm,        xmmmem128,  none,       none,       0,       0x0f59,    r,   no,  0,     0)

INSTR (VMULPS,            xmm,        xvvvv,      xmmmem128,  none,       0x0100,  0x59,      r,   no,  0,     PVEX)
INSTR (VMULPS,            ymm,        yvvvv,      ymmmem256,  none,       0x0104,  0x59,      r,   no,  0,     PVEX)

INSTR (ORPD,              xmm,        xmmmem128,  none,       none,       0,       0x0f56,    r,   no,  0,     P66)

INSTR (VORPD,             xmm,        xvvvv,      xmmmem128,  none,       0x0101,  0x56,      r,   no,  0,     PVEX)
INSTR (VORPD,             ymm,        yvvvv,      ymmmem256,  none,       0x0105,  0x56,      r,   no,  0,     PVEX)

INSTR (ORPS,              xmm,        xmmmem128,  none,       none,       0,       0x0f56,    r,   no,  0,     0)

INSTR (VORPS,             xmm,        xvvvv,      xmmmem128,  none,       0x0100,  0x56,      r,   no,  0,     PVEX)
INSTR (VORPS,             ymm,        yvvvv,      ymmmem256,  none,       0x0104,  0x56,      r,   no,  0,     PVEX)

INSTR (PABSB,             xmm,        xmmmem128,  none,       none,       0,       0x0f381c,  r,   no,  0,     0)

INSTR (VPABSB,            xmm,        xmmmem128,  none,       none,       0x0279,  0x1c,      r,   no,  0,     PVEX)
INSTR (VPABSB,            ymm,        ymmmem256,  none,       none,       0x027d,  0x1c,      r,   no,  0,     PVEX)

INSTR (PABSD,             xmm,        xmmmem128,  none,       none,       0,       0x0f381e,  r,   no,  0,     0)

INSTR (VPABSD,            xmm,        xmmmem128,  none,       none,       0x0279,  0x1e,      r,   no,  0,     PVEX)
INSTR (VPABSD,            ymm,        ymmmem256,  none,       none,       0x027d,  0x1e,      r,   no,  0,     PVEX)

INSTR (PABSW,             xmm,        xmmmem128,  none,       none,       0,       0x0f381d,  r,   no,  0,     0)

INSTR (VPABSW,            xmm,        xmmmem128,  none,       none,       0x0279,  0x1d,      r,   no,  0,     PVEX)
INSTR (VPABSW,            ymm,        ymmmem256,  none,       none,       0x027d,  0x1d,      r,   no,  0,     PVEX)

INSTR (PACKSSDW,          xmm,        xmmmem128,  none,       none,       0,       0x0f6b,    r,   no,  0,     P66)
INSTR (PACKSSDW,          mmx,        mmxmem64,   none,       none,       0,       0x0f6b,    r,   no,  0,     0)

INSTR (VPACKSSDW,         xmm,        xvvvv,      xmmmem128,  none,       0x0101,  0x6b,      r,   no,  0,     PVEX)
INSTR (VPACKSSDW,         ymm,        yvvvv,      ymmmem256,  none,       0x0105,  0x6b,      r,   no,  0,     PVEX)

INSTR (PACKSSWB,          xmm,        xmmmem128,  none,       none,       0,       0x0f63,    r,   no,  0,     P66)
INSTR (PACKSSWB,          mmx,        mmxmem64,   none,       none,       0,       0x0f63,    r,   no,  0,     0)

INSTR (VPACKSSWB,         xmm,        xvvvv,      xmmmem128,  none,       0x0101,  0x63,      r,   no,  0,     PVEX)
INSTR (VPACKSSWB,         ymm,        yvvvv,      ymmmem256,  none,       0x0105,  0x63,      r,   no,  0,     PVEX)

INSTR (PACKUSDW,          xmm,        xmmmem128,  none,       none,       0,       0x0f382b,  r,   no,  0,     P66)

INSTR (VPACKUSDW,         xmm,        xvvvv,      xmmmem128,  none,       0x0201,  0x2b,      r,   no,  0,     PVEX)
INSTR (VPACKUSDW,         ymm,        yvvvv,      ymmmem256,  none,       0x0205,  0x2b,      r,   no,  0,     PVEX)

INSTR (PACKUSWB,          xmm,        xmmmem128,  none,       none,       0,       0x0f67,    r,   no,  0,     P66)
INSTR (PACKUSWB,          mmx,        mmxmem64,   none,       none,       0,       0x0f67,    r,   no,  0,     0)

INSTR (VPACKUSWB,         xmm,        xvvvv,      xmmmem128,  none,       0x0101,  0x67,      r,   no,  0,     PVEX)
INSTR (VPACKUSWB,         ymm,        yvvvv,      ymmmem256,  none,       0x0105,  0x67,      r,   no,  0,     PVEX)

INSTR (PADDB,             xmm,        xmmmem128,  none,       none,       0,       0x0ffc,    r,   no,  0,     P66)
INSTR (PADDB,             mmx,        mmxmem64,   none,       none,       0,       0x0ffc,    r,   no,  0,     0)

INSTR (VPADDB,            xmm,        xvvvv,      xmmmem128,  none,       0x0101,  0xfc,      r,   no,  0,     PVEX)
INSTR (VPADDB,            ymm,        yvvvv,      ymmmem256,  none,       0x0105,  0xfc,      r,   no,  0,     PVEX)

INSTR (PADDD,             xmm,        xmmmem128,  none,       none,       0,       0x0ffe,    r,   no,  0,     P66)
INSTR (PADDD,             mmx,        mmxmem64,   none,       none,       0,       0x0ffe,    r,   no,  0,     0)

INSTR (VPADDD,            xmm,        xvvvv,      xmmmem128,  none,       0x0101,  0xfe,      r,   no,  0,     PVEX)
INSTR (VPADDD,            ymm,        yvvvv,      ymmmem256,  none,       0x0105,  0xfe,      r,   no,  0,     PVEX)

INSTR (PADDQ,             xmm,        xmmmem128,  none,       none,       0,       0x0fd4,    r,   no,  0,     P66)
INSTR (PADDQ,             mmx,        mmxmem64,   none,       none,       0,       0x0fd4,    r,   no,  0,     0)

INSTR (VPADDQ,            xmm,        xvvvv,      xmmmem128,  none,       0x0101,  0xd4,      r,   no,  0,     PVEX)
INSTR (VPADDQ,            ymm,        yvvvv,      ymmmem256,  none,       0x0105,  0xd4,      r,   no,  0,     PVEX)

INSTR (PADDSB,            xmm,        xmmmem128,  none,       none,       0,       0x0fec,    r,   no,  0,     P66)
INSTR (PADDSB,            mmx,        mmxmem64,   none,       none,       0,       0x0fec,    r,   no,  0,     0)

INSTR (VPADDSB,           xmm,        xvvvv,      xmmmem128,  none,       0x0101,  0xec,      r,   no,  0,     PVEX)
INSTR (VPADDSB,           ymm,        yvvvv,      ymmmem256,  none,       0x0105,  0xec,      r,   no,  0,     PVEX)

INSTR (PADDSW,            xmm,        xmmmem128,  none,       none,       0,       0x0fed,    r,   no,  0,     P66)
INSTR (PADDSW,            mmx,        mmxmem64,   none,       none,       0,       0x0fed,    r,   no,  0,     0)

INSTR (VPADDSW,           xmm,        xvvvv,      xmmmem128,  none,       0x0101,  0xed,      r,   no,  0,     PVEX)
INSTR (VPADDSW,           ymm,        yvvvv,      ymmmem256,  none,       0x0105,  0xed,      r,   no,  0,     PVEX)

INSTR (PADDUSB,           xmm,        xmmmem128,  none,       none,       0,       0x0fdc,    r,   no,  0,     P66)
INSTR (PADDUSB,           mmx,        mmxmem64,   none,       none,       0,       0x0fdc,    r,   no,  0,     0)

INSTR (VPADDUSB,          xmm,        xvvvv,      xmmmem128,  none,       0x0101,  0xdc,      r,   no,  0,     PVEX)
INSTR (VPADDUSB,          ymm,        yvvvv,      ymmmem256,  none,       0x0105,  0xdc,      r,   no,  0,     PVEX)

INSTR (PADDUSW,           xmm,        xmmmem128,  none,       none,       0,       0x0fdd,    r,   no,  0,     P66)
INSTR (PADDUSW,           mmx,        mmxmem64,   none,       none,       0,       0x0fdd,    r,   no,  0,     0)

INSTR (VPADDUSW,          xmm,        xvvvv,      xmmmem128,  none,       0x0101,  0xdd,      r,   no,  0,     PVEX)
INSTR (VPADDUSW,          ymm,        yvvvv,      ymmmem256,  none,       0x0105,  0xdd,      r,   no,  0,     PVEX)

INSTR (PADDW,             xmm,        xmmmem128,  none,       none,       0,       0x0ffd,    r,   no,  0,     P66)
INSTR (PADDW,             mmx,        mmxmem64,   none,       none,       0,       0x0ffd,    r,   no,  0,     0)

INSTR (VPADDW,            xmm,        xvvvv,      xmmmem128,  none,       0x0101,  0xfd,      r,   no,  0,     PVEX)
INSTR (VPADDW,            ymm,        yvvvv,      ymmmem256,  none,       0x0105,  0xfd,      r,   no,  0,     PVEX)

INSTR (PALIGNR,           xmm,        xmmmem128,  imm8,       none,       0,       0x0f3a0f,  r,   ib,  0,     P66)

INSTR (VPALIGNR,          xmm,        xvvvv,      xmmmem128,  imm8,       0x0301,  0x0f,      r,   ib,  0,     PVEX)
INSTR (VPALIGNR,          ymm,        yvvvv,      ymmmem256,  imm8,       0x0305,  0x0f,      r,   ib,  0,     PVEX)

INSTR (PAND,              xmm,        xmmmem128,  none,       none,       0,       0x0fdb,    r,   no,  0,     P66)
INSTR (PAND,              mmx,        mmxmem64,   none,       none,       0,       0x0fdb,    r,   no,  0,     0)

INSTR (VPAND,             xmm,        xvvvv,      xmmmem128,  none,       0x0101,  0xdb,      r,   no,  0,     PVEX)
INSTR (VPAND,             ymm,        yvvvv,      ymmmem256,  none,       0x0105,  0xdb,      r,   no,  0,     PVEX)

INSTR (PANDN,             xmm,        xmmmem128,  none,       none,       0,       0x0fdf,    r,   no,  0,     P66)
INSTR (PANDN,             mmx,        mmxmem64,   none,       none,       0,       0x0fdf,    r,   no,  0,     0)

INSTR (VPANDN,            xmm,        xvvvv,      xmmmem128,  none,       0x0101,  0xdf,      r,   no,  0,     PVEX)
INSTR (VPANDN,            ymm,        yvvvv,      ymmmem256,  none,       0x0105,  0xdf,      r,   no,  0,     PVEX)

INSTR (PAVGB,             xmm,        xmmmem128,  none,       none,       0,       0x0fe0,    r,   no,  0,     P66)
INSTR (PAVGB,             mmx,        mmxmem64,   none,       none,       0,       0x0fe0,    r,   no,  0,     0)

INSTR (VPAVGB,            xmm,        xvvvv,      xmmmem128,  none,       0x0101,  0xe0,      r,   no,  0,     PVEX)
INSTR (VPAVGB,            ymm,        yvvvv,      ymmmem256,  none,       0x0105,  0xe0,      r,   no,  0,     PVEX)

INSTR (PAVGW,             xmm,        xmmmem128,  none,       none,       0,       0x0fe3,    r,   no,  0,     P66)
INSTR (PAVGW,             mmx,        mmxmem64,   none,       none,       0,       0x0fe3,    r,   no,  0,     0)

INSTR (VPAVGW,            xmm,        xvvvv,      xmmmem128,  none,       0x0101,  0xe3,      r,   no,  0,     PVEX)
INSTR (VPAVGW,            ymm,        yvvvv,      ymmmem256,  none,       0x0105,  0xe3,      r,   no,  0,     PVEX)

INSTR (PBLENDVB,          xmm,        xmmmem128,  none,       none,       0,       0x0f3810,  r,   no,  0,     P66)

INSTR (VPBLENDVB,         xmm,        xvvvv,      xmmmem128,  ximm,       0x0301,  0x4c,      r,   is,  0,     PVEX)
INSTR (VPBLENDVB,         ymm,        yvvvv,      ymmmem256,  yimm,       0x0305,  0x4c,      r,   is,  0,     PVEX)

INSTR (PBLENDW,           xmm,        xmmmem128,  imm8,       none,       0,       0x0f3a0e,  r,   ib,  0,     P66)

INSTR (VPBLENDW,          xmm,        xvvvv,      xmmmem128,  imm8,       0x0301,  0x0e,      r,   ib,  0,     PVEX)
INSTR (VPBLENDW,          ymm,        yvvvv,      ymmmem256,  imm8,       0x0305,  0x0e,      r,   ib,  0,     PVEX)

INSTR (PCLMULQDQ,         xmm,        xmmmem128,  imm8,       none,       0,       0x0f3a44,  r,   ib,  0,     P66)

INSTR (VPCLMULQDQ,        xmm,        xvvvv,      xmmmem128,  imm8,       0x0301,  0x44,      r,   ib,  0,     PVEX)

INSTR (PCMPEQB,           xmm,        xmmmem128,  none,       none,       0,       0x0f74,    r,   no,  0,     P66)
INSTR (PCMPEQB,           mmx,        mmxmem64,   none,       none,       0,       0x0f74,    r,   no,  0,     0)

INSTR (VPCMPEQB,          xmm,        xvvvv,      xmmmem128,  none,       0x0101,  0x74,      r,   no,  0,     PVEX)
INSTR (VPCMPEQB,          ymm,        yvvvv,      ymmmem256,  none,       0x0105,  0x74,      r,   no,  0,     PVEX)

INSTR (PCMPEQD,           xmm,        xmmmem128,  none,       none,       0,       0x0f76,    r,   no,  0,     P66)
INSTR (PCMPEQD,           mmx,        mmxmem64,   none,       none,       0,       0x0f76,    r,   no,  0,     0)

INSTR (VPCMPEQD,          xmm,        xvvvv,      xmmmem128,  none,       0x0101,  0x76,      r,   no,  0,     PVEX)
INSTR (VPCMPEQD,          ymm,        yvvvv,      ymmmem256,  none,       0x0105,  0x76,      r,   no,  0,     PVEX)

INSTR (PCMPEQQ,           xmm,        xmmmem128,  none,       none,       0,       0x0f3829,  r,   no,  0,     P66)

INSTR (VPCMPEQQ,          xmm,        xvvvv,      xmmmem128,  none,       0x0201,  0x29,      r,   no,  0,     PVEX)
INSTR (VPCMPEQQ,          ymm,        yvvvv,      ymmmem256,  none,       0x0205,  0x29,      r,   no,  0,     PVEX)

INSTR (PCMPEQW,           xmm,        xmmmem128,  none,       none,       0,       0x0f75,    r,   no,  0,     P66)
INSTR (PCMPEQW,           mmx,        mmxmem64,   none,       none,       0,       0x0f75,    r,   no,  0,     0)

INSTR (VPCMPEQW,          xmm,        xvvvv,      xmmmem128,  none,       0x0101,  0x75,      r,   no,  0,     PVEX)
INSTR (VPCMPEQW,          ymm,        yvvvv,      ymmmem256,  none,       0x0105,  0x75,      r,   no,  0,     PVEX)

INSTR (PCMPESTRI,         xmm,        xmmmem128,  imm8,       none,       0,       0x0f3a61,  r,   ib,  0,     P66)

INSTR (VPCMPESTRI,        xmm,        xmmmem128,  imm8,       none,       0x0379,  0x61,      r,   ib,  0,     PVEX)

INSTR (PCMPESTRM,         xmm,        xmmmem128,  imm8,       none,       0,       0x0f3a60,  r,   ib,  0,     P66)

INSTR (VPCMPESTRM,        xmm,        xmmmem128,  imm8,       none,       0x0379,  0x60,      r,   ib,  0,     PVEX)

INSTR (PCMPGTB,           xmm,        xmmmem128,  none,       none,       0,       0x0f64,    r,   no,  0,     P66)
INSTR (PCMPGTB,           mmx,        mmxmem64,   none,       none,       0,       0x0f64,    r,   no,  0,     0)

INSTR (VPCMPGTB,          xmm,        xvvvv,      xmmmem128,  none,       0x0101,  0x64,      r,   no,  0,     PVEX)
INSTR (VPCMPGTB,          ymm,        yvvvv,      ymmmem256,  none,       0x0105,  0x64,      r,   no,  0,     PVEX)

INSTR (PCMPGTD,           xmm,        xmmmem128,  none,       none,       0,       0x0f66,    r,   no,  0,     P66)
INSTR (PCMPGTD,           mmx,        mmxmem64,   none,       none,       0,       0x0f66,    r,   no,  0,     0)

INSTR (VPCMPGTD,          xmm,        xvvvv,      xmmmem128,  none,       0x0101,  0x66,      r,   no,  0,     PVEX)
INSTR (VPCMPGTD,          ymm,        yvvvv,      ymmmem256,  none,       0x0105,  0x66,      r,   no,  0,     PVEX)

INSTR (PCMPGTQ,           xmm,        xmmmem128,  none,       none,       0,       0x0f3837,  r,   no,  0,     P66)

INSTR (VPCMPGTQ,          xmm,        xvvvv,      xmmmem128,  none,       0x0201,  0x37,      r,   no,  0,     PVEX)
INSTR (VPCMPGTQ,          ymm,        yvvvv,      ymmmem256,  none,       0x0205,  0x37,      r,   no,  0,     PVEX)

INSTR (PCMPGTW,           xmm,        xmmmem128,  none,       none,       0,       0x0f65,    r,   no,  0,     P66)
INSTR (PCMPGTW,           mmx,        mmxmem64,   none,       none,       0,       0x0f65,    r,   no,  0,     0)

INSTR (VPCMPGTW,          xmm,        xvvvv,      xmmmem128,  none,       0x0101,  0x65,      r,   no,  0,     PVEX)
INSTR (VPCMPGTW,          ymm,        yvvvv,      ymmmem256,  none,       0x0105,  0x65,      r,   no,  0,     PVEX)

INSTR (PCMPISTRI,         xmm,        xmmmem128,  imm8,       none,       0,       0x0f3a63,  r,   ib,  0,     P66)

INSTR (VPCMPISTRI,        xmm,        xmmmem128,  imm8,       none,       0x0379,  0x63,      r,   ib,  0,     PVEX)

INSTR (PCMPISTRM,         xmm,        xmmmem128,  imm8,       none,       0,       0x0f3a62,  r,   ib,  0,     P66)

INSTR (VPCMPISTRM,        xmm,        xmmmem128,  imm8,       none,       0x0379,  0x62,      r,   ib,  0,     PVEX)

INSTR (PEXTRB,            regmem8,    xmm,        imm8,       none,       0,       0x0f3a14,  r,   ib,  0,     P66)

INSTR (VPEXTRB,           regmem8,    xmm,        imm8,       none,       0x0379,  0x14,      r,   ib,  0,     PVEX)

INSTR (PEXTRD,            regmem32,   xmm,        imm8,       none,       0,       0x0f3a16,  r,   ib,  0,     O16 | O32 | P66)

INSTR (VPEXTRD,           regmem32,   xmm,        imm8,       none,       0x0379,  0x16,      r,   ib,  0,     PVEX)

INSTR (PEXTRQ,            regmem64,   xmm,        imm8,       none,       0,       0x0f3a16,  r,   ib,  0,     O64 | P66)

INSTR (VPEXTRQ,           regmem64,   xmm,        imm8,       none,       0x03f9,  0x16,      r,   ib,  0,     PVEX)

INSTR (PEXTRW,            reg32,      xmmmem128,  imm8,       none,       0,       0x0fc5,    r,   ib,  0,     P66)
INSTR (PEXTRW,            reg32,      mmxmem64,   imm8,       none,       0,       0x0fc5,    r,   ib,  0,     0)
INSTR (PEXTRW,            regmem16,   xmm,        imm8,       none,       0,       0x0f3a15,  r,   ib,  0,     P66)

INSTR (VPEXTRW,           reg32,      xmmmem128,  imm8,       none,       0x0179,  0xc5,      r,   ib,  0,     PVEX)
INSTR (VPEXTRW,           regmem16,   xmm,        imm8,       none,       0x0379,  0x15,      r,   ib,  0,     PVEX)

INSTR (PHADDD,            xmm,        xmmmem128,  none,       none,       0,       0x0f3802,  r,   no,  0,     P66)

INSTR (VPHADDD,           xmm,        xvvvv,      xmmmem128,  none,       0x0201,  0x02,      r,   no,  0,     PVEX)
INSTR (VPHADDD,           ymm,        yvvvv,      ymmmem256,  none,       0x0205,  0x02,      r,   no,  0,     PVEX)

INSTR (PHADDSW,           xmm,        xmmmem128,  none,       none,       0,       0x0f3803,  r,   no,  0,     P66)

INSTR (VPHADDSW,          xmm,        xvvvv,      xmmmem128,  none,       0x0201,  0x03,      r,   no,  0,     PVEX)
INSTR (VPHADDSW,          ymm,        yvvvv,      ymmmem256,  none,       0x0205,  0x03,      r,   no,  0,     PVEX)

INSTR (PHADDW,            xmm,        xmmmem128,  none,       none,       0,       0x0f3801,  r,   no,  0,     P66)

INSTR (VPHADDW,           xmm,        xvvvv,      xmmmem128,  none,       0x0201,  0x01,      r,   no,  0,     PVEX)
INSTR (VPHADDW,           ymm,        yvvvv,      ymmmem256,  none,       0x0205,  0x01,      r,   no,  0,     PVEX)

INSTR (PHMINPOSUW,        xmm,        xmmmem128,  none,       none,       0,       0x0f3841,  r,   no,  0,     P66)

INSTR (VPHMINPOSUW,       xmm,        xmmmem128,  none,       none,       0x0279,  0x41,      r,   no,  0,     PVEX)

INSTR (PHSUBD,            xmm,        xmmmem128,  none,       none,       0,       0x0f3806,  r,   no,  0,     P66)

INSTR (VPHSUBD,           xmm,        xvvvv,      xmmmem128,  none,       0x0201,  0x06,      r,   no,  0,     PVEX)
INSTR (VPHSUBD,           ymm,        yvvvv,      ymmmem256,  none,       0x0205,  0x06,      r,   no,  0,     PVEX)

INSTR (PHSUBSW,           xmm,        xmmmem128,  none,       none,       0,       0x0f3807,  r,   no,  0,     P66)

INSTR (VPHSUBSW,          xmm,        xvvvv,      xmmmem128,  none,       0x0201,  0x07,      r,   no,  0,     PVEX)
INSTR (VPHSUBSW,          ymm,        yvvvv,      ymmmem256,  none,       0x0205,  0x07,      r,   no,  0,     PVEX)

INSTR (PHSUBW,            xmm,        xmmmem128,  none,       none,       0,       0x0f3805,  r,   no,  0,     P66)

INSTR (VPHSUBW,           xmm,        xvvvv,      xmmmem128,  none,       0x0201,  0x05,      r,   no,  0,     PVEX)
INSTR (VPHSUBW,           ymm,        yvvvv,      ymmmem256,  none,       0x0205,  0x05,      r,   no,  0,     PVEX)

INSTR (PINSRB,            xmm,        regmem8,    imm8,       none,       0,       0x0f3a20,  r,   ib,  0,     P66)

INSTR (VPINSRB,           xmm,        regmem8,    xvvvv,      imm8,       0x0301,  0x20,      r,   ib,  0,     PVEX)

INSTR (PINSRD,            xmm,        regmem32,   imm8,       none,       0,       0x0f3a22,  r,   ib,  0,     O16 | O32 | P66)

INSTR (VPINSRD,           xmm,        regmem32,   xvvvv,      imm8,       0x0301,  0x22,      r,   ib,  0,     PVEX)

INSTR (PINSRQ,            xmm,        regmem64,   imm8,       none,       0,       0x0f3a22,  r,   ib,  0,     O64 | P66)

INSTR (VPINSRQ,           xmm,        regmem64,   xvvvv,      imm8,       0x0301,  0x22,      r,   ib,  0,     PVEX)

INSTR (PINSRW,            xmm,        regmem32,   imm8,       none,       0,       0x0fc4,    r,   ib,  0,     P66)
INSTR (PINSRW,            mmx,        regmem32,   imm8,       none,       0,       0x0fc4,    r,   ib,  0,     0)

INSTR (VPINSRW,           xmm,        regmem32,   xvvvv,      imm8,       0x0101,  0xc4,      r,   ib,  0,     PVEX)

INSTR (PMADDUBSW,         xmm,        xmmmem128,  none,       none,       0,       0x0f3804,  r,   no,  0,     P66)

INSTR (VPMADDUBSW,        xmm,        xvvvv,      xmmmem128,  none,       0x0201,  0x04,      r,   no,  0,     PVEX)
INSTR (VPMADDUBSW,        ymm,        yvvvv,      ymmmem256,  none,       0x0205,  0x04,      r,   no,  0,     PVEX)

INSTR (PMADDWD,           xmm,        xmmmem128,  none,       none,       0,       0x0ff5,    r,   no,  0,     P66)
INSTR (PMADDWD,           mmx,        mmxmem64,   none,       none,       0,       0x0ff5,    r,   no,  0,     0)

INSTR (VPMADDWD,          xmm,        xvvvv,      xmmmem128,  none,       0x0101,  0xf5,      r,   no,  0,     PVEX)
INSTR (VPMADDWD,          ymm,        yvvvv,      ymmmem256,  none,       0x0105,  0xf5,      r,   no,  0,     PVEX)

INSTR (PMAXSB,            xmm,        xmmmem128,  none,       none,       0,       0x0f383c,  r,   no,  0,     P66)

INSTR (VPMAXSB,           xmm,        xvvvv,      xmmmem128,  none,       0x0201,  0x3c,      r,   no,  0,     PVEX)
INSTR (VPMAXSB,           ymm,        yvvvv,      ymmmem256,  none,       0x0205,  0x3c,      r,   no,  0,     PVEX)

INSTR (PMAXSD,            xmm,        xmmmem128,  none,       none,       0,       0x0f383d,  r,   no,  0,     P66)

INSTR (VPMAXSD,           xmm,        xvvvv,      xmmmem128,  none,       0x0201,  0x3d,      r,   no,  0,     PVEX)
INSTR (VPMAXSD,           ymm,        yvvvv,      ymmmem256,  none,       0x0205,  0x3d,      r,   no,  0,     PVEX)

INSTR (PMAXSW,            xmm,        xmmmem128,  none,       none,       0,       0x0fee,    r,   no,  0,     P66)
INSTR (PMAXSW,            mmx,        mmxmem64,   none,       none,       0,       0x0fee,    r,   no,  0,     0)

INSTR (VPMAXSW,           xmm,        xvvvv,      xmmmem128,  none,       0x0101,  0xee,      r,   no,  0,     PVEX)
INSTR (VPMAXSW,           ymm,        yvvvv,      ymmmem256,  none,       0x0105,  0xee,      r,   no,  0,     PVEX)

INSTR (PMAXUB,            xmm,        xmmmem128,  none,       none,       0,       0x0fde,    r,   no,  0,     P66)
INSTR (PMAXUB,            mmx,        mmxmem64,   none,       none,       0,       0x0fde,    r,   no,  0,     0)

INSTR (VPMAXUB,           xmm,        xvvvv,      xmmmem128,  none,       0x0101,  0xde,      r,   no,  0,     PVEX)
INSTR (VPMAXUB,           ymm,        yvvvv,      ymmmem256,  none,       0x0105,  0xde,      r,   no,  0,     PVEX)

INSTR (PMAXUD,            xmm,        xmmmem128,  none,       none,       0,       0x0f383f,  r,   no,  0,     P66)

INSTR (VPMAXUD,           xmm,        xvvvv,      xmmmem128,  none,       0x0201,  0x3f,      r,   no,  0,     PVEX)
INSTR (VPMAXUD,           ymm,        yvvvv,      ymmmem256,  none,       0x0205,  0x3f,      r,   no,  0,     PVEX)

INSTR (PMAXUW,            xmm,        xmmmem128,  none,       none,       0,       0x0f383e,  r,   no,  0,     P66)

INSTR (VPMAXUW,           xmm,        xvvvv,      xmmmem128,  none,       0x0201,  0x3e,      r,   no,  0,     PVEX)
INSTR (VPMAXUW,           ymm,        yvvvv,      ymmmem256,  none,       0x0205,  0x3e,      r,   no,  0,     PVEX)

INSTR (PMINSB,            xmm,        xmmmem128,  none,       none,       0,       0x0f3838,  r,   no,  0,     P66)

INSTR (VPMINSB,           xmm,        xvvvv,      xmmmem128,  none,       0x0201,  0x38,      r,   no,  0,     PVEX)
INSTR (VPMINSB,           ymm,        yvvvv,      ymmmem256,  none,       0x0205,  0x38,      r,   no,  0,     PVEX)

INSTR (PMINSD,            xmm,        xmmmem128,  none,       none,       0,       0x0f3839,  r,   no,  0,     P66)

INSTR (VPMINSD,           xmm,        xvvvv,      xmmmem128,  none,       0x0201,  0x39,      r,   no,  0,     PVEX)
INSTR (VPMINSD,           ymm,        yvvvv,      ymmmem256,  none,       0x0205,  0x39,      r,   no,  0,     PVEX)

INSTR (PMINSW,            xmm,        xmmmem128,  none,       none,       0,       0x0fea,    r,   no,  0,     P66)
INSTR (PMINSW,            mmx,        mmxmem64,   none,       none,       0,       0x0fea,    r,   no,  0,     0)

INSTR (VPMINSW,           xmm,        xvvvv,      xmmmem128,  none,       0x0101,  0xea,      r,   no,  0,     PVEX)
INSTR (VPMINSW,           ymm,        yvvvv,      ymmmem256,  none,       0x0105,  0xea,      r,   no,  0,     PVEX)

INSTR (PMINUB,            xmm,        xmmmem128,  none,       none,       0,       0x0fda,    r,   no,  0,     P66)
INSTR (PMINUB,            mmx,        mmxmem64,   none,       none,       0,       0x0fda,    r,   no,  0,     0)

INSTR (VPMINUB,           xmm,        xvvvv,      xmmmem128,  none,       0x0101,  0xda,      r,   no,  0,     PVEX)
INSTR (VPMINUB,           ymm,        yvvvv,      ymmmem256,  none,       0x0105,  0xda,      r,   no,  0,     PVEX)

INSTR (PMINUD,            xmm,        xmmmem128,  none,       none,       0,       0x0f383b,  r,   no,  0,     P66)

INSTR (VPMINUD,           xmm,        xvvvv,      xmmmem128,  none,       0x0201,  0x3b,      r,   no,  0,     PVEX)
INSTR (VPMINUD,           ymm,        yvvvv,      ymmmem256,  none,       0x0205,  0x3b,      r,   no,  0,     PVEX)

INSTR (PMINUW,            xmm,        xmmmem128,  none,       none,       0,       0x0f383a,  r,   no,  0,     P66)

INSTR (VPMINUW,           xmm,        xvvvv,      xmmmem128,  none,       0x0201,  0x3a,      r,   no,  0,     PVEX)
INSTR (VPMINUW,           ymm,        yvvvv,      ymmmem256,  none,       0x0205,  0x3a,      r,   no,  0,     PVEX)

INSTR (PMOVMSKB,          reg32,      xmmmem128,  none,       none,       0,       0x0fd7,    r,   no,  0,     P66)
INSTR (PMOVMSKB,          reg32,      mmxmem64,   none,       none,       0,       0x0fd7,    r,   no,  0,     0)

INSTR (VPMOVMSKB,         reg64,      xmmmem128,  none,       none,       0x0179,  0xd7,      r,   no,  0,     PVEX)
INSTR (VPMOVMSKB,         reg64,      ymmmem256,  none,       none,       0x017d,  0xd7,      r,   no,  0,     PVEX)

INSTR (PMOVSXBD,          xmm,        xmmmem32,   none,       none,       0,       0x0f3821,  r,   no,  0,     P66)

INSTR (VPMOVSXBD,         xmm,        xmmmem32,   none,       none,       0x0279,  0x21,      r,   no,  0,     PVEX)
INSTR (VPMOVSXBD,         ymm,        xmmmem64,   none,       none,       0x027d,  0x21,      r,   no,  0,     PVEX)

INSTR (PMOVSXBQ,          xmm,        xmmmem16,   none,       none,       0,       0x0f3822,  r,   no,  0,     P66)

INSTR (VPMOVSXBQ,         xmm,        xmmmem16,   none,       none,       0x0279,  0x22,      r,   no,  0,     PVEX)
INSTR (VPMOVSXBQ,         ymm,        xmmmem32,   none,       none,       0x027d,  0x22,      r,   no,  0,     PVEX)

INSTR (PMOVSXBW,          xmm,        xmmmem64,   none,       none,       0,       0x0f3820,  r,   no,  0,     P66)

INSTR (VPMOVSXBW,         xmm,        xmmmem64,   none,       none,       0x0279,  0x20,      r,   no,  0,     PVEX)
INSTR (VPMOVSXBW,         ymm,        xmmmem128,  none,       none,       0x027d,  0x20,      r,   no,  0,     PVEX)

INSTR (PMOVSXDQ,          xmm,        xmmmem64,   none,       none,       0,       0x0f3825,  r,   no,  0,     P66)

INSTR (VPMOVSXDQ,         xmm,        xmmmem64,   none,       none,       0x0279,  0x25,      r,   no,  0,     PVEX)
INSTR (VPMOVSXDQ,         ymm,        xmmmem128,  none,       none,       0x027d,  0x25,      r,   no,  0,     PVEX)

INSTR (PMOVSXWD,          xmm,        xmmmem64,   none,       none,       0,       0x0f3823,  r,   no,  0,     P66)

INSTR (VPMOVSXWD,         xmm,        xmmmem64,   none,       none,       0x0279,  0x23,      r,   no,  0,     PVEX)
INSTR (VPMOVSXWD,         ymm,        xmmmem128,  none,       none,       0x027d,  0x23,      r,   no,  0,     PVEX)

INSTR (PMOVSXWQ,          xmm,        xmmmem32,   none,       none,       0,       0x0f3824,  r,   no,  0,     P66)

INSTR (VPMOVSXWQ,         xmm,        xmmmem32,   none,       none,       0x0279,  0x24,      r,   no,  0,     PVEX)
INSTR (VPMOVSXWQ,         ymm,        xmmmem64,   none,       none,       0x027d,  0x24,      r,   no,  0,     PVEX)

INSTR (PMOVZXBD,          xmm,        xmmmem32,   none,       none,       0,       0x0f3831,  r,   no,  0,     P66)

INSTR (VPMOVZXBD,         xmm,        xmmmem32,   none,       none,       0x0279,  0x31,      r,   no,  0,     PVEX)
INSTR (VPMOVZXBD,         ymm,        xmmmem64,   none,       none,       0x027d,  0x31,      r,   no,  0,     PVEX)

INSTR (PMOVZXBQ,          xmm,        xmmmem16,   none,       none,       0,       0x0f3832,  r,   no,  0,     P66)

INSTR (VPMOVZXBQ,         xmm,        xmmmem16,   none,       none,       0x0279,  0x32,      r,   no,  0,     PVEX)
INSTR (VPMOVZXBQ,         ymm,        xmmmem32,   none,       none,       0x027d,  0x32,      r,   no,  0,     PVEX)

INSTR (PMOVZXBW,          xmm,        xmmmem64,   none,       none,       0,       0x0f3830,  r,   no,  0,     P66)

INSTR (VPMOVZXBW,         xmm,        xmmmem64,   none,       none,       0x0279,  0x30,      r,   no,  0,     PVEX)
INSTR (VPMOVZXBW,         ymm,        xmmmem128,  none,       none,       0x027d,  0x30,      r,   no,  0,     PVEX)

INSTR (PMOVZXDQ,          xmm,        xmmmem64,   none,       none,       0,       0x0f3835,  r,   no,  0,     P66)

INSTR (VPMOVZXDQ,         xmm,        xmmmem64,   none,       none,       0x0279,  0x35,      r,   no,  0,     PVEX)
INSTR (VPMOVZXDQ,         ymm,        xmmmem128,  none,       none,       0x027d,  0x35,      r,   no,  0,     PVEX)

INSTR (PMOVZXWD,          xmm,        xmmmem64,   none,       none,       0,       0x0f3833,  r,   no,  0,     P66)

INSTR (VPMOVZXWD,         xmm,        xmmmem64,   none,       none,       0x0279,  0x33,      r,   no,  0,     PVEX)
INSTR (VPMOVZXWD,         ymm,        xmmmem128,  none,       none,       0x027d,  0x33,      r,   no,  0,     PVEX)

INSTR (PMOVZXWQ,          xmm,        xmmmem32,   none,       none,       0,       0x0f3834,  r,   no,  0,     P66)

INSTR (VPMOVZXWQ,         xmm,        xmmmem32,   none,       none,       0x0279,  0x34,      r,   no,  0,     PVEX)
INSTR (VPMOVZXWQ,         ymm,        xmmmem64,   none,       none,       0x027d,  0x34,      r,   no,  0,     PVEX)

INSTR (PMULDQ,            xmm,        xmmmem128,  none,       none,       0,       0x0f3828,  r,   no,  0,     P66)

INSTR (VPMULDQ,           xmm,        xvvvv,      xmmmem128,  none,       0x0201,  0x28,      r,   no,  0,     PVEX)
INSTR (VPMULDQ,           ymm,        yvvvv,      ymmmem256,  none,       0x0205,  0x28,      r,   no,  0,     PVEX)

INSTR (PMULHRSW,          xmm,        xmmmem128,  none,       none,       0,       0x0f380b,  r,   no,  0,     P66)

INSTR (VPMULHRSW,         xmm,        xvvvv,      xmmmem128,  none,       0x0201,  0x0b,      r,   no,  0,     PVEX)
INSTR (VPMULHRSW,         ymm,        yvvvv,      ymmmem256,  none,       0x0205,  0x0b,      r,   no,  0,     PVEX)

INSTR (PMULHUW,           xmm,        xmmmem128,  none,       none,       0,       0x0fe4,    r,   no,  0,     P66)
INSTR (PMULHUW,           mmx,        mmxmem64,   none,       none,       0,       0x0fe4,    r,   no,  0,     0)

INSTR (VPMULHUW,          xmm,        xvvvv,      xmmmem128,  none,       0x0101,  0xe4,      r,   no,  0,     PVEX)
INSTR (VPMULHUW,          ymm,        yvvvv,      ymmmem256,  none,       0x0105,  0xe4,      r,   no,  0,     PVEX)

INSTR (PMULHW,            xmm,        xmmmem128,  none,       none,       0,       0x0fe5,    r,   no,  0,     P66)
INSTR (PMULHW,            mmx,        mmxmem64,   none,       none,       0,       0x0fe5,    r,   no,  0,     0)

INSTR (VPMULHW,           xmm,        xvvvv,      xmmmem128,  none,       0x0101,  0xe5,      r,   no,  0,     PVEX)
INSTR (VPMULHW,           ymm,        yvvvv,      ymmmem256,  none,       0x0105,  0xe5,      r,   no,  0,     PVEX)

INSTR (PMULLD,            xmm,        xmmmem128,  none,       none,       0,       0x0f3840,  r,   no,  0,     P66)

INSTR (VPMULLD,           xmm,        xvvvv,      xmmmem128,  none,       0x0201,  0x40,      r,   no,  0,     PVEX)
INSTR (VPMULLD,           ymm,        yvvvv,      ymmmem256,  none,       0x0205,  0x40,      r,   no,  0,     PVEX)

INSTR (PMULLW,            xmm,        xmmmem128,  none,       none,       0,       0x0fd5,    r,   no,  0,     P66)
INSTR (PMULLW,            mmx,        mmxmem64,   none,       none,       0,       0x0fd5,    r,   no,  0,     0)

INSTR (VPMULLW,           xmm,        xvvvv,      xmmmem128,  none,       0x0101,  0xd5,      r,   no,  0,     PVEX)
INSTR (VPMULLW,           ymm,        yvvvv,      ymmmem256,  none,       0x0105,  0xd5,      r,   no,  0,     PVEX)

INSTR (PMULUDQ,           xmm,        xmmmem128,  none,       none,       0,       0x0ff4,    r,   no,  0,     P66)
INSTR (PMULUDQ,           mmx,        mmxmem64,   none,       none,       0,       0x0ff4,    r,   no,  0,     0)

INSTR (VPMULUDQ,          xmm,        xvvvv,      xmmmem128,  none,       0x0101,  0xf4,      r,   no,  0,     PVEX)
INSTR (VPMULUDQ,          ymm,        yvvvv,      ymmmem256,  none,       0x0105,  0xf4,      r,   no,  0,     PVEX)

INSTR (POR,               xmm,        xmmmem128,  none,       none,       0,       0x0feb,    r,   no,  0,     P66)
INSTR (POR,               mmx,        mmxmem64,   none,       none,       0,       0x0feb,    r,   no,  0,     0)

INSTR (VPOR,              xmm,        xvvvv,      xmmmem128,  none,       0x0101,  0xeb,      r,   no,  0,     PVEX)
INSTR (VPOR,              ymm,        yvvvv,      ymmmem256,  none,       0x0105,  0xeb,      r,   no,  0,     PVEX)

INSTR (PSADBW,            xmm,        xmmmem128,  none,       none,       0,       0x0ff6,    r,   no,  0,     P66)
INSTR (PSADBW,            mmx,        mmxmem64,   none,       none,       0,       0x0ff6,    r,   no,  0,     0)

INSTR (VPSADBW,           xmm,        xvvvv,      xmmmem128,  none,       0x0101,  0xf6,      r,   no,  0,     PVEX)
INSTR (VPSADBW,           ymm,        yvvvv,      ymmmem256,  none,       0x0105,  0xf6,      r,   no,  0,     PVEX)

INSTR (PSHUFB,            xmm,        xmmmem128,  none,       none,       0,       0x0f3800,  r,   no,  0,     P66)

INSTR (VPSHUFB,           xmm,        xvvvv,      xmmmem128,  none,       0x0201,  0x00,      r,   no,  0,     PVEX)
INSTR (VPSHUFB,           ymm,        yvvvv,      ymmmem256,  none,       0x0205,  0x00,      r,   no,  0,     PVEX)

INSTR (PSHUFD,            xmm,        xmmmem128,  imm8,       none,       0,       0x0f70,    r,   ib,  0,     P66)

INSTR (VPSHUFD,           xmm,        xmmmem128,  imm8,       none,       0x0179,  0x70,      r,   ib,  0,     PVEX)
INSTR (VPSHUFD,           ymm,        ymmmem256,  imm8,       none,       0x017d,  0x70,      r,   ib,  0,     PVEX)

INSTR (PSHUFLW,           xmm,        xmmmem128,  imm8,       none,       0,       0x0f70,    r,   ib,  0,     PF2)

INSTR (VPSHUFLW,          xmm,        xmmmem128,  imm8,       none,       0x017b,  0x70,      r,   ib,  0,     PVEX)
INSTR (VPSHUFLW,          ymm,        ymmmem256,  imm8,       none,       0x017f,  0x70,      r,   ib,  0,     PVEX)

INSTR (PSHUFHW,           xmm,        xmmmem128,  imm8,       none,       0,       0x0f70,    r,   ib,  0,     PF3)

INSTR (VPSHUFHW,          xmm,        xmmmem128,  imm8,       none,       0x017a,  0x70,      r,   ib,  0,     PVEX)
INSTR (VPSHUFHW,          ymm,        ymmmem256,  imm8,       none,       0x017e,  0x70,      r,   ib,  0,     PVEX)

INSTR (PSIGNB,            xmm,        xmmmem128,  none,       none,       0,       0x0f3808,  r,   no,  0,     P66)

INSTR (VPSIGNB,           xmm,        xvvvv,      xmmmem128,  none,       0x0201,  0x08,      r,   no,  0,     PVEX)
INSTR (VPSIGNB,           ymm,        yvvvv,      ymmmem256,  none,       0x0205,  0x08,      r,   no,  0,     PVEX)

INSTR (PSIGND,            xmm,        xmmmem128,  none,       none,       0,       0x0f380a,  r,   no,  0,     P66)

INSTR (VPSIGND,           xmm,        xvvvv,      xmmmem128,  none,       0x0201,  0x0a,      r,   no,  0,     PVEX)
INSTR (VPSIGND,           ymm,        yvvvv,      ymmmem256,  none,       0x0205,  0x0a,      r,   no,  0,     PVEX)

INSTR (PSIGNW,            xmm,        xmmmem128,  none,       none,       0,       0x0f3809,  r,   no,  0,     P66)

INSTR (VPSIGNW,           xmm,        xvvvv,      xmmmem128,  none,       0x0201,  0x09,      r,   no,  0,     PVEX)
INSTR (VPSIGNW,           ymm,        yvvvv,      ymmmem256,  none,       0x0205,  0x09,      r,   no,  0,     PVEX)

INSTR (PSLLD,             xmm,        xmmmem128,  none,       none,       0,       0x0ff2,    r,   no,  0,     P66)
INSTR (PSLLD,             mmx,        mmxmem64,   none,       none,       0,       0x0ff2,    r,   no,  0,     0)
INSTR (PSLLD,             xmmmem128,  imm8,       none,       none,       0,       0x0f72,    r6,  ib,  0,     P66)
INSTR (PSLLD,             mmxmem64,   imm8,       none,       none,       0,       0x0f72,    r6,  ib,  0,     0)

INSTR (VPSLLD,            xmm,        xvvvv,      xmmmem128,  none,       0x0101,  0xf2,      r,   no,  0,     PVEX)
INSTR (VPSLLD,            xvvvv,      xmmmem128,  imm8,       none,       0x0101,  0x72,      r6,  ib,  0,     PVEX)
INSTR (VPSLLD,            ymm,        yvvvv,      xmmmem128,  none,       0x0105,  0xf2,      r,   no,  0,     PVEX)
INSTR (VPSLLD,            yvvvv,      ymmmem256,  imm8,       none,       0x0105,  0x72,      r6,  ib,  0,     PVEX)

INSTR (PSLLDQ,            xmmmem128,  imm8,       none,       none,       0,       0x0f73,    r7,  ib,  0,     P66)

INSTR (VPSLLDQ,           xvvvv,      xmmmem128,  imm8,       none,       0x0101,  0x73,      r7,  ib,  0,     PVEX)
INSTR (VPSLLDQ,           yvvvv,      ymmmem256,  imm8,       none,       0x0105,  0x73,      r7,  ib,  0,     PVEX)

INSTR (PSLLQ,             xmm,        xmmmem128,  none,       none,       0,       0x0ff3,    r,   no,  0,     P66)
INSTR (PSLLQ,             mmx,        mmxmem64,   none,       none,       0,       0x0ff3,    r,   no,  0,     0)
INSTR (PSLLQ,             xmmmem128,  imm8,       none,       none,       0,       0x0f73,    r6,  ib,  0,     P66)
INSTR (PSLLQ,             mmxmem64,   imm8,       none,       none,       0,       0x0f73,    r6,  ib,  0,     0)

INSTR (VPSLLQ,            xmm,        xvvvv,      xmmmem128,  none,       0x0101,  0xf3,      r,   no,  0,     PVEX)
INSTR (VPSLLQ,            xvvvv,      xmmmem128,  imm8,       none,       0x0101,  0x73,      r6,  ib,  0,     PVEX)
INSTR (VPSLLQ,            ymm,        yvvvv,      xmmmem128,  none,       0x0105,  0xf3,      r,   no,  0,     PVEX)
INSTR (VPSLLQ,            yvvvv,      ymmmem256,  imm8,       none,       0x0105,  0x73,      r6,  ib,  0,     PVEX)

INSTR (PSLLW,             xmm,        xmmmem128,  none,       none,       0,       0x0ff1,    r,   no,  0,     P66)
INSTR (PSLLW,             mmx,        mmxmem64,   none,       none,       0,       0x0ff1,    r,   no,  0,     0)
INSTR (PSLLW,             xmmmem128,  imm8,       none,       none,       0,       0x0f71,    r6,  ib,  0,     P66)
INSTR (PSLLW,             mmxmem64,   imm8,       none,       none,       0,       0x0f71,    r6,  ib,  0,     0)

INSTR (VPSLLW,            xmm,        xvvvv,      xmmmem128,  none,       0x0101,  0xf1,      r,   no,  0,     PVEX)
INSTR (VPSLLW,            xvvvv,      xmmmem128,  imm8,       none,       0x0101,  0x71,      r6,  ib,  0,     PVEX)
INSTR (VPSLLW,            ymm,        yvvvv,      xmmmem128,  none,       0x0105,  0xf1,      r,   no,  0,     PVEX)
INSTR (VPSLLW,            yvvvv,      ymmmem256,  imm8,       none,       0x0105,  0x71,      r6,  ib,  0,     PVEX)

INSTR (PSRAD,             xmm,        xmmmem128,  none,       none,       0,       0x0fe2,    r,   no,  0,     P66)
INSTR (PSRAD,             mmx,        mmxmem64,   none,       none,       0,       0x0fe2,    r,   no,  0,     0)
INSTR (PSRAD,             xmmmem128,  imm8,       none,       none,       0,       0x0f72,    r4,  ib,  0,     P66)
INSTR (PSRAD,             mmxmem64,   imm8,       none,       none,       0,       0x0f72,    r4,  ib,  0,     0)

INSTR (VPSRAD,            xmm,        xvvvv,      xmmmem128,  none,       0x0101,  0xe2,      r,   no,  0,     PVEX)
INSTR (VPSRAD,            xvvvv,      xmmmem128,  imm8,       none,       0x0101,  0x72,      r4,  ib,  0,     PVEX)
INSTR (VPSRAD,            ymm,        yvvvv,      xmmmem128,  none,       0x0105,  0xe2,      r,   no,  0,     PVEX)
INSTR (VPSRAD,            yvvvv,      ymmmem256,  imm8,       none,       0x0105,  0x72,      r4,  ib,  0,     PVEX)

INSTR (PSRAW,             xmm,        xmmmem128,  none,       none,       0,       0x0fe1,    r,   no,  0,     P66)
INSTR (PSRAW,             mmx,        mmxmem64,   none,       none,       0,       0x0fe1,    r,   no,  0,     0)
INSTR (PSRAW,             xmmmem128,  imm8,       none,       none,       0,       0x0f71,    r4,  ib,  0,     P66)
INSTR (PSRAW,             mmxmem64,   imm8,       none,       none,       0,       0x0f71,    r4,  ib,  0,     0)

INSTR (VPSRAW,            xmm,        xvvvv,      xmmmem128,  none,       0x0101,  0xe1,      r,   no,  0,     PVEX)
INSTR (VPSRAW,            xvvvv,      xmmmem128,  imm8,       none,       0x0101,  0x71,      r4,  ib,  0,     PVEX)
INSTR (VPSRAW,            ymm,        yvvvv,      xmmmem128,  none,       0x0105,  0xe1,      r,   no,  0,     PVEX)
INSTR (VPSRAW,            yvvvv,      ymmmem256,  imm8,       none,       0x0105,  0x71,      r4,  ib,  0,     PVEX)

INSTR (PSRLD,             xmm,        xmmmem128,  none,       none,       0,       0x0fd2,    r,   no,  0,     P66)
INSTR (PSRLD,             mmx,        mmxmem64,   none,       none,       0,       0x0fd2,    r,   no,  0,     0)
INSTR (PSRLD,             xmmmem128,  imm8,       none,       none,       0,       0x0f72,    r2,  ib,  0,     P66)
INSTR (PSRLD,             mmxmem64,   imm8,       none,       none,       0,       0x0f72,    r2,  ib,  0,     0)

INSTR (VPSRLD,            xmm,        xvvvv,      xmmmem128,  none,       0x0101,  0xd2,      r,   no,  0,     PVEX)
INSTR (VPSRLD,            xvvvv,      xmmmem128,  imm8,       none,       0x0101,  0x72,      r2,  ib,  0,     PVEX)
INSTR (VPSRLD,            ymm,        yvvvv,      xmmmem128,  none,       0x0105,  0xd2,      r,   no,  0,     PVEX)
INSTR (VPSRLD,            yvvvv,      ymmmem256,  imm8,       none,       0x0105,  0x72,      r2,  ib,  0,     PVEX)

INSTR (PSRLDQ,            xmmmem128,  imm8,       none,       none,       0,       0x0f73,    r3,  ib,  0,     P66)

INSTR (VPSRLDQ,           xvvvv,      xmmmem128,  imm8,       none,       0x0101,  0x73,      r3,  ib,  0,     PVEX)
INSTR (VPSRLDQ,           yvvvv,      ymmmem256,  imm8,       none,       0x0105,  0x73,      r3,  ib,  0,     PVEX)

INSTR (PSRLQ,             xmm,        xmmmem128,  none,       none,       0,       0x0fd3,    r,   no,  0,     P66)
INSTR (PSRLQ,             mmx,        mmxmem64,   none,       none,       0,       0x0fd3,    r,   no,  0,     0)
INSTR (PSRLQ,             xmmmem128,  imm8,       none,       none,       0,       0x0f73,    r2,  ib,  0,     P66)
INSTR (PSRLQ,             mmxmem64,   imm8,       none,       none,       0,       0x0f73,    r2,  ib,  0,     0)

INSTR (VPSRLQ,            xmm,        xvvvv,      xmmmem128,  none,       0x0101,  0xd3,      r,   no,  0,     PVEX)
INSTR (VPSRLQ,            xvvvv,      xmmmem128,  imm8,       none,       0x0101,  0x73,      r2,  ib,  0,     PVEX)
INSTR (VPSRLQ,            ymm,        yvvvv,      xmmmem128,  none,       0x0105,  0xd3,      r,   no,  0,     PVEX)
INSTR (VPSRLQ,            yvvvv,      ymmmem256,  imm8,       none,       0x0105,  0x73,      r2,  ib,  0,     PVEX)

INSTR (PSRLW,             xmm,        xmmmem128,  none,       none,       0,       0x0fd1,    r,   no,  0,     P66)
INSTR (PSRLW,             mmx,        mmxmem64,   none,       none,       0,       0x0fd1,    r,   no,  0,     0)
INSTR (PSRLW,             xmmmem128,  imm8,       none,       none,       0,       0x0f71,    r2,  ib,  0,     P66)
INSTR (PSRLW,             mmxmem64,   imm8,       none,       none,       0,       0x0f71,    r2,  ib,  0,     0)

INSTR (VPSRLW,            xmm,        xvvvv,      xmmmem128,  none,       0x0101,  0xd1,      r,   no,  0,     PVEX)
INSTR (VPSRLW,            xvvvv,      xmmmem128,  imm8,       none,       0x0101,  0x71,      r2,  ib,  0,     PVEX)
INSTR (VPSRLW,            ymm,        yvvvv,      xmmmem128,  none,       0x0105,  0xd1,      r,   no,  0,     PVEX)
INSTR (VPSRLW,            yvvvv,      ymmmem256,  imm8,       none,       0x0105,  0x71,      r2,  ib,  0,     PVEX)

INSTR (PSUBB,             xmm,        xmmmem128,  none,       none,       0,       0x0ff8,    r,   no,  0,     P66)
INSTR (PSUBB,             mmx,        mmxmem64,   none,       none,       0,       0x0ff8,    r,   no,  0,     0)

INSTR (VPSUBB,            xmm,        xvvvv,      xmmmem128,  none,       0x0101,  0xf8,      r,   no,  0,     PVEX)
INSTR (VPSUBB,            ymm,        yvvvv,      ymmmem256,  none,       0x0105,  0xf8,      r,   no,  0,     PVEX)

INSTR (PSUBD,             xmm,        xmmmem128,  none,       none,       0,       0x0ffa,    r,   no,  0,     P66)
INSTR (PSUBD,             mmx,        mmxmem64,   none,       none,       0,       0x0ffa,    r,   no,  0,     0)

INSTR (VPSUBD,            xmm,        xvvvv,      xmmmem128,  none,       0x0101,  0xfa,      r,   no,  0,     PVEX)
INSTR (VPSUBD,            ymm,        yvvvv,      ymmmem256,  none,       0x0105,  0xfa,      r,   no,  0,     PVEX)

INSTR (PSUBQ,             xmm,        xmmmem128,  none,       none,       0,       0x0ffb,    r,   no,  0,     P66)
INSTR (PSUBQ,             mmx,        mmxmem64,   none,       none,       0,       0x0ffb,    r,   no,  0,     0)

INSTR (VPSUBQ,            xmm,        xvvvv,      xmmmem128,  none,       0x0101,  0xfb,      r,   no,  0,     PVEX)
INSTR (VPSUBQ,            ymm,        yvvvv,      ymmmem256,  none,       0x0105,  0xfb,      r,   no,  0,     PVEX)

INSTR (PSUBSB,            xmm,        xmmmem128,  none,       none,       0,       0x0fe8,    r,   no,  0,     P66)
INSTR (PSUBSB,            mmx,        mmxmem64,   none,       none,       0,       0x0fe8,    r,   no,  0,     0)

INSTR (VPSUBSB,           xmm,        xvvvv,      xmmmem128,  none,       0x0101,  0xe8,      r,   no,  0,     PVEX)
INSTR (VPSUBSB,           ymm,        yvvvv,      ymmmem256,  none,       0x0105,  0xe8,      r,   no,  0,     PVEX)

INSTR (PSUBSW,            xmm,        xmmmem128,  none,       none,       0,       0x0fe9,    r,   no,  0,     P66)
INSTR (PSUBSW,            mmx,        mmxmem64,   none,       none,       0,       0x0fe9,    r,   no,  0,     0)

INSTR (VPSUBSW,           xmm,        xvvvv,      xmmmem128,  none,       0x0101,  0xe9,      r,   no,  0,     PVEX)
INSTR (VPSUBSW,           ymm,        yvvvv,      ymmmem256,  none,       0x0105,  0xe9,      r,   no,  0,     PVEX)

INSTR (PSUBUSB,           xmm,        xmmmem128,  none,       none,       0,       0x0fd8,    r,   no,  0,     P66)
INSTR (PSUBUSB,           mmx,        mmxmem64,   none,       none,       0,       0x0fd8,    r,   no,  0,     0)

INSTR (VPSUBUSB,          xmm,        xvvvv,      xmmmem128,  none,       0x0101,  0xd8,      r,   no,  0,     PVEX)
INSTR (VPSUBUSB,          ymm,        yvvvv,      ymmmem256,  none,       0x0105,  0xd8,      r,   no,  0,     PVEX)

INSTR (PSUBUSW,           xmm,        xmmmem128,  none,       none,       0,       0x0fd9,    r,   no,  0,     P66)
INSTR (PSUBUSW,           mmx,        mmxmem64,   none,       none,       0,       0x0fd9,    r,   no,  0,     0)

INSTR (VPSUBUSW,          xmm,        xvvvv,      xmmmem128,  none,       0x0101,  0xd9,      r,   no,  0,     PVEX)
INSTR (VPSUBUSW,          ymm,        yvvvv,      ymmmem256,  none,       0x0105,  0xd9,      r,   no,  0,     PVEX)

INSTR (PSUBW,             xmm,        xmmmem128,  none,       none,       0,       0x0ff9,    r,   no,  0,     P66)
INSTR (PSUBW,             mmx,        mmxmem64,   none,       none,       0,       0x0ff9,    r,   no,  0,     0)

INSTR (VPSUBW,            xmm,        xvvvv,      xmmmem128,  none,       0x0101,  0xf9,      r,   no,  0,     PVEX)
INSTR (VPSUBW,            ymm,        yvvvv,      ymmmem256,  none,       0x0105,  0xf9,      r,   no,  0,     PVEX)

INSTR (PTEST,             xmm,        xmmmem128,  none,       none,       0,       0x0f3817,  r,   no,  0,     P66)

INSTR (VPTEST,            xmm,        xmmmem128,  none,       none,       0x0279,  0x17,      r,   no,  0,     PVEX)
INSTR (VPTEST,            ymm,        ymmmem256,  none,       none,       0x027d,  0x17,      r,   no,  0,     PVEX)

INSTR (PUNPCKHBW,         xmm,        xmmmem128,  none,       none,       0,       0x0f68,    r,   no,  0,     P66)
INSTR (PUNPCKHBW,         mmx,        mmxmem64,   none,       none,       0,       0x0f68,    r,   no,  0,     0)

INSTR (VPUNPCKHBW,        xmm,        xvvvv,      xmmmem128,  none,       0x0101,  0x68,      r,   no,  0,     PVEX)
INSTR (VPUNPCKHBW,        ymm,        yvvvv,      ymmmem256,  none,       0x0105,  0x68,      r,   no,  0,     PVEX)

INSTR (PUNPCKHDQ,         xmm,        xmmmem128,  none,       none,       0,       0x0f6a,    r,   no,  0,     P66)
INSTR (PUNPCKHDQ,         mmx,        mmxmem64,   none,       none,       0,       0x0f6a,    r,   no,  0,     0)

INSTR (VPUNPCKHDQ,        xmm,        xvvvv,      xmmmem128,  none,       0x0101,  0x6a,      r,   no,  0,     PVEX)
INSTR (VPUNPCKHDQ,        ymm,        yvvvv,      ymmmem256,  none,       0x0105,  0x6a,      r,   no,  0,     PVEX)

INSTR (PUNPCKHQDQ,        xmm,        xmmmem128,  none,       none,       0,       0x0f6d,    r,   no,  0,     P66)

INSTR (VPUNPCKHQDQ,       xmm,        xvvvv,      xmmmem128,  none,       0x0101,  0x6d,      r,   no,  0,     PVEX)
INSTR (VPUNPCKHQDQ,       ymm,        yvvvv,      ymmmem256,  none,       0x0105,  0x6d,      r,   no,  0,     PVEX)

INSTR (PUNPCKHWD,         xmm,        xmmmem128,  none,       none,       0,       0x0f69,    r,   no,  0,     P66)
INSTR (PUNPCKHWD,         mmx,        mmxmem64,   none,       none,       0,       0x0f69,    r,   no,  0,     0)

INSTR (VPUNPCKHWD,        xmm,        xvvvv,      xmmmem128,  none,       0x0101,  0x69,      r,   no,  0,     PVEX)
INSTR (VPUNPCKHWD,        ymm,        yvvvv,      ymmmem256,  none,       0x0105,  0x69,      r,   no,  0,     PVEX)

INSTR (PUNPCKLBW,         xmm,        xmmmem128,  none,       none,       0,       0x0f60,    r,   no,  0,     P66)
INSTR (PUNPCKLBW,         mmx,        mmxmem32,   none,       none,       0,       0x0f60,    r,   no,  0,     0)

INSTR (VPUNPCKLBW,        xmm,        xvvvv,      xmmmem128,  none,       0x0101,  0x60,      r,   no,  0,     PVEX)
INSTR (VPUNPCKLBW,        ymm,        yvvvv,      ymmmem256,  none,       0x0105,  0x60,      r,   no,  0,     PVEX)

INSTR (PUNPCKLDQ,         xmm,        xmmmem128,  none,       none,       0,       0x0f62,    r,   no,  0,     P66)
INSTR (PUNPCKLDQ,         mmx,        mmxmem32,   none,       none,       0,       0x0f62,    r,   no,  0,     0)

INSTR (VPUNPCKLDQ,        xmm,        xvvvv,      xmmmem128,  none,       0x0101,  0x62,      r,   no,  0,     PVEX)
INSTR (VPUNPCKLDQ,        ymm,        yvvvv,      ymmmem256,  none,       0x0105,  0x62,      r,   no,  0,     PVEX)

INSTR (PUNPCKLQDQ,        xmm,        xmmmem128,  none,       none,       0,       0x0f6c,    r,   no,  0,     P66)

INSTR (VPUNPCKLQDQ,       xmm,        xvvvv,      xmmmem128,  none,       0x0101,  0x6c,      r,   no,  0,     PVEX)
INSTR (VPUNPCKLQDQ,       ymm,        yvvvv,      ymmmem256,  none,       0x0105,  0x6c,      r,   no,  0,     PVEX)

INSTR (PUNPCKLWD,         xmm,        xmmmem128,  none,       none,       0,       0x0f61,    r,   no,  0,     P66)
INSTR (PUNPCKLWD,         mmx,        mmxmem32,   none,       none,       0,       0x0f61,    r,   no,  0,     0)

INSTR (VPUNPCKLWD,        xmm,        xvvvv,      xmmmem128,  none,       0x0101,  0x61,      r,   no,  0,     PVEX)
INSTR (VPUNPCKLWD,        ymm,        yvvvv,      ymmmem256,  none,       0x0105,  0x61,      r,   no,  0,     PVEX)

INSTR (PXOR,              xmm,        xmmmem128,  none,       none,       0,       0x0fef,    r,   no,  0,     P66)
INSTR (PXOR,              mmx,        mmxmem64,   none,       none,       0,       0x0fef,    r,   no,  0,     0)

INSTR (VPXOR,             xmm,        xvvvv,      xmmmem128,  none,       0x0101,  0xef,      r,   no,  0,     PVEX)
INSTR (VPXOR,             ymm,        yvvvv,      ymmmem256,  none,       0x0105,  0xef,      r,   no,  0,     PVEX)

INSTR (RCPPS,             xmm,        xmmmem128,  none,       none,       0,       0x0f53,    r,   no,  0,     0)

INSTR (VRCPPS,            xmm,        xmmmem128,  none,       none,       0x0178,  0x53,      r,   no,  0,     PVEX)
INSTR (VRCPPS,            ymm,        ymmmem256,  none,       none,       0x017c,  0x53,      r,   no,  0,     PVEX)

INSTR (RCPSS,             xmm,        xmmmem32,   none,       none,       0,       0x0f53,    r,   no,  0,     PF3)

INSTR (VRCPSS,            xmm,        xvvvv,      xmmmem128,  none,       0x0102,  0x53,      r,   no,  0,     PVEX)

INSTR (ROUNDPD,           xmm,        xmmmem128,  imm8,       none,       0,       0x0f3a09,  r,   ib,  0,     P66)

INSTR (VROUNDPD,          xmm,        xmmmem128,  imm8,       none,       0x0301,  0x09,      r,   ib,  0,     PVEX)
INSTR (VROUNDPD,          ymm,        ymmmem256,  imm8,       none,       0x0305,  0x09,      r,   ib,  0,     PVEX)

INSTR (ROUNDPS,           xmm,        xmmmem128,  imm8,       none,       0,       0x0f3a08,  r,   ib,  0,     P66)

INSTR (VROUNDPS,          xmm,        xmmmem128,  imm8,       none,       0x0301,  0x08,      r,   ib,  0,     PVEX)
INSTR (VROUNDPS,          ymm,        ymmmem256,  imm8,       none,       0x0305,  0x08,      r,   ib,  0,     PVEX)

INSTR (ROUNDSD,           xmm,        xmmmem64,   imm8,       none,       0,       0x0f3a0b,  r,   ib,  0,     P66)

INSTR (VROUNDSD,          xmm,        xmmmem64,   imm8,       none,       0x0301,  0x0b,      r,   ib,  0,     PVEX)

INSTR (ROUNDSS,           xmm,        xmmmem32,   imm8,       none,       0,       0x0f3a0a,  r,   ib,  0,     P66)

INSTR (VROUNDSS,          xmm,        xmmmem32,   imm8,       none,       0x0301,  0x0a,      r,   ib,  0,     PVEX)

INSTR (SHA1MSG1,          xmm,        xmmmem128,  none,       none,       0,       0x0f38c9,  r,   no,  0,     0)

INSTR (SHA1MSG2,          xmm,        xmmmem128,  none,       none,       0,       0x0f38ca,  r,   no,  0,     0)

INSTR (SHA1NEXTE,         xmm,        xmmmem128,  none,       none,       0,       0x0f38c8,  r,   no,  0,     0)

INSTR (SHA1RNDS4,         xmm,        xmmmem128,  imm8,       none,       0,       0x0f3acc,  r,   ib,  0,     0)

INSTR (SHA256MSG1,        xmm,        xmmmem128,  none,       none,       0,       0x0f38cc,  r,   no,  0,     0)

INSTR (SHA256MSG2,        xmm,        xmmmem128,  none,       none,       0,       0x0f38cd,  r,   no,  0,     0)

INSTR (SHA256RNDS2,       xmm,        xmmmem128,  none,       none,       0,       0x0f38cb,  r,   no,  0,     0)

INSTR (RSQRTSS,           xmm,        xmmmem32,   none,       none,       0,       0x0f52,    r,   no,  0,     PF3)

INSTR (VRSQRTSS,          xmm,        xvvvv,      xmmmem32,   none,       0x0102,  0x52,      r,   no,  0,     PVEX)

INSTR (RSQRTPS,           xmm,        xmmmem128,  none,       none,       0,       0x0f52,    r,   no,  0,     0)

INSTR (VRSQRTPS,          xmm,        xmmmem128,  none,       none,       0x0178,  0x52,      r,   no,  0,     PVEX)
INSTR (VRSQRTPS,          ymm,        ymmmem256,  none,       none,       0x017c,  0x52,      r,   no,  0,     PVEX)

INSTR (SHUFPD,            xmm,        xmmmem128,  imm8,       none,       0,       0x0fc6,    r,   ib,  0,     P66)

INSTR (VSHUFPD,           xmm,        xvvvv,      xmmmem128,  imm8,       0x0101,  0xc6,      r,   ib,  0,     PVEX)
INSTR (VSHUFPD,           ymm,        yvvvv,      ymmmem256,  imm8,       0x0105,  0xc6,      r,   ib,  0,     PVEX)

INSTR (SHUFPS,            xmm,        xmmmem128,  imm8,       none,       0,       0x0fc6,    r,   ib,  0,     0)

INSTR (VSHUFPS,           xmm,        xvvvv,      xmmmem128,  imm8,       0x0100,  0xc6,      r,   ib,  0,     PVEX)
INSTR (VSHUFPS,           ymm,        yvvvv,      ymmmem256,  imm8,       0x0104,  0xc6,      r,   ib,  0,     PVEX)

INSTR (SQRTPD,            xmm,        xmmmem128,  none,       none,       0,       0x0f51,    r,   no,  0,     P66)

INSTR (VSQRTPD,           xmm,        xmmmem128,  none,       none,       0x0179,  0x51,      r,   no,  0,     PVEX)
INSTR (VSQRTPD,           ymm,        ymmmem256,  none,       none,       0x017d,  0x51,      r,   no,  0,     PVEX)

INSTR (SQRTSD,            xmm,        xmmmem64,   none,       none,       0,       0x0f51,    r,   no,  0,     PF2)

INSTR (VSQRTSD,           xmm,        xvvvv,      xmmmem64,   none,       0x0103,  0x51,      r,   no,  0,     PVEX)

INSTR (SQRTSS,            xmm,        xmmmem32,   none,       none,       0,       0x0f51,    r,   no,  0,     PF3)

INSTR (VSQRTSS,           xmm,        xvvvv,      xmmmem32,   none,       0x0102,  0x51,      r,   no,  0,     PVEX)

INSTR (SQRTPS,            xmm,        xmmmem128,  none,       none,       0,       0x0f51,    r,   no,  0,     0)

INSTR (VSQRTPS,           xmm,        xmmmem128,  none,       none,       0x0178,  0x51,      r,   no,  0,     PVEX)
INSTR (VSQRTPS,           ymm,        ymmmem256,  none,       none,       0x017c,  0x51,      r,   no,  0,     PVEX)

INSTR (SUBPD,             xmm,        xmmmem128,  none,       none,       0,       0x0f5c,    r,   no,  0,     P66)

INSTR (VSUBPD,            xmm,        xvvvv,      xmmmem128,  none,       0x0101,  0x5c,      r,   no,  0,     PVEX)
INSTR (VSUBPD,            ymm,        yvvvv,      ymmmem256,  none,       0x0105,  0x5c,      r,   no,  0,     PVEX)

INSTR (SUBSD,             xmm,        xmmmem64,   none,       none,       0,       0x0f5c,    r,   no,  0,     PF2)

INSTR (VSUBSD,            xmm,        xvvvv,      xmmmem64,   none,       0x0103,  0x5c,      r,   no,  0,     PVEX)

INSTR (SUBSS,             xmm,        xmmmem32,   none,       none,       0,       0x0f5c,    r,   no,  0,     PF3)

INSTR (VSUBSS,            xmm,        xvvvv,      xmmmem32,   none,       0x0102,  0x5c,      r,   no,  0,     PVEX)

INSTR (SUBPS,             xmm,        xmmmem128,  none,       none,       0,       0x0f5c,    r,   no,  0,     0)

INSTR (VSUBPS,            xmm,        xvvvv,      xmmmem128,  none,       0x0100,  0x5c,      r,   no,  0,     PVEX)
INSTR (VSUBPS,            ymm,        yvvvv,      ymmmem256,  none,       0x0104,  0x5c,      r,   no,  0,     PVEX)

INSTR (T1MSKC,            reg32vvvv,  regmem32,   none,       none,       0x0900,  0x01,      r7,  no,  0,     PXOP)
INSTR (T1MSKC,            reg64vvvv,  regmem64,   none,       none,       0x0980,  0x01,      r7,  no,  0,     PXOP)

INSTR (TZCNT,             reg16,      regmem16,   none,       none,       0,       0x0fbc,    r,   no,  0,     O16 | PF3)
INSTR (TZCNT,             reg32,      regmem32,   none,       none,       0,       0x0fbc,    r,   no,  0,     O32 | PF3)
INSTR (TZCNT,             reg64,      regmem64,   none,       none,       0,       0x0fbc,    r,   no,  0,     O64 | PF3)

INSTR (TZMSK,             reg32vvvv,  regmem32,   none,       none,       0x0900,  0x01,      r4,  no,  0,     PXOP)
INSTR (TZMSK,             reg64vvvv,  regmem64,   none,       none,       0x0980,  0x01,      r4,  no,  0,     PXOP)

INSTR (UCOMISD,           xmm,        xmmmem64,   none,       none,       0,       0x0f2e,    r,   no,  0,     P66)

INSTR (VUCOMISD,          xmm,        xmmmem64,   none,       none,       0x0179,  0x2e,      r,   no,  0,     PVEX)

INSTR (UCOMISS,           xmm,        xmmmem32,   none,       none,       0,       0x0f2e,    r,   no,  0,     0)

INSTR (VUCOMISS,          xmm,        xmmmem32,   none,       none,       0x0178,  0x2e,      r,   no,  0,     PVEX)

INSTR (UNPCKHPD,          xmm,        xmmmem128,  none,       none,       0,       0x0f15,    r,   no,  0,     P66)

INSTR (VUNPCKHPD,         xmm,        xvvvv,      xmmmem128,  none,       0x0101,  0x15,      r,   no,  0,     PVEX)
INSTR (VUNPCKHPD,         ymm,        yvvvv,      ymmmem256,  none,       0x0105,  0x15,      r,   no,  0,     PVEX)

INSTR (UNPCKHPS,          xmm,        xmmmem128,  none,       none,       0,       0x0f15,    r,   no,  0,     0)

INSTR (VUNPCKHPS,         xmm,        xvvvv,      xmmmem128,  none,       0x0100,  0x15,      r,   no,  0,     PVEX)
INSTR (VUNPCKHPS,         ymm,        yvvvv,      ymmmem256,  none,       0x0104,  0x15,      r,   no,  0,     PVEX)

INSTR (UNPCKLPD,          xmm,        xmmmem128,  none,       none,       0,       0x0f14,    r,   no,  0,     P66)

INSTR (VUNPCKLPD,         xmm,        xvvvv,      xmmmem128,  none,       0x0101,  0x14,      r,   no,  0,     PVEX)
INSTR (VUNPCKLPD,         ymm,        yvvvv,      ymmmem256,  none,       0x0105,  0x14,      r,   no,  0,     PVEX)

INSTR (UNPCKLPS,          xmm,        xmmmem128,  none,       none,       0,       0x0f14,    r,   no,  0,     0)

INSTR (VUNPCKLPS,         xmm,        xvvvv,      xmmmem128,  none,       0x0100,  0x14,      r,   no,  0,     PVEX)
INSTR (VUNPCKLPS,         ymm,        yvvvv,      ymmmem256,  none,       0x0104,  0x14,      r,   no,  0,     PVEX)

INSTR (VBROADCASTF128,    ymm,        mem128,     none,       none,       0x027d,  0x1a,      r,   no,  0,     PVEX)

INSTR (VBROADCASTI128,    ymm,        mem128,     none,       none,       0x027d,  0x5a,      r,   no,  0,     PVEX)

INSTR (VBROADCASTSD,      ymm,        xmmmem64,   none,       none,       0x027d,  0x19,      r,   no,  0,     PVEX)

INSTR (VBROADCASTSS,      xmm,        xmmmem32,   none,       none,       0x0279,  0x18,      r,   no,  0,     PVEX)
INSTR (VBROADCASTSS,      ymm,        xmmmem32,   none,       none,       0x027d,  0x18,      r,   no,  0,     PVEX)

INSTR (VCVTPH2PS,         xmm,        xmmmem64,   none,       none,       0x0279,  0x13,      r,   no,  0,     PVEX)
INSTR (VCVTPH2PS,         ymm,        xmmmem128,  none,       none,       0x027d,  0x13,      r,   no,  0,     PVEX)

INSTR (VCVTPS2PH,         xmmmem64,   xmm,        imm8,       none,       0x0379,  0x1d,      r,   ib,  0,     PVEX)
INSTR (VCVTPS2PH,         xmmmem128,  ymm,        imm8,       none,       0x037d,  0x1d,      r,   ib,  0,     PVEX)

INSTR (VEXTRACTF128,      xmmmem128,  ymm,        imm8,       none,       0x037d,  0x19,      r,   ib,  0,     PVEX)

INSTR (VEXTRACTI128,      xmmmem128,  ymm,        imm8,       none,       0x037d,  0x39,      r,   ib,  0,     PVEX)

INSTR (VFMADDPD,          xmm,        xvvvv,      xmmmem128,  ximm,       0x0301,  0x69,      r,   is,  0,     PVEX)
INSTR (VFMADDPD,          ymm,        yvvvv,      ymmmem256,  yimm,       0x0305,  0x69,      r,   is,  0,     PVEX)
INSTR (VFMADDPD,          xmm,        xvvvv,      ximm,       xmmmem128,  0x0381,  0x69,      r,   is,  0,     PVEX)
INSTR (VFMADDPD,          ymm,        yvvvv,      yimm,       ymmmem256,  0x0385,  0x69,      r,   is,  0,     PVEX)

INSTR (VFMADD132PD,       xmm,        xvvvv,      xmmmem128,  none,       0x0281,  0x98,      r,   no,  0,     PVEX)
INSTR (VFMADD132PD,       ymm,        yvvvv,      ymmmem256,  none,       0x0285,  0x98,      r,   no,  0,     PVEX)

INSTR (VFMADD213PD,       xmm,        xvvvv,      xmmmem128,  none,       0x0281,  0xa8,      r,   no,  0,     PVEX)
INSTR (VFMADD213PD,       ymm,        yvvvv,      ymmmem256,  none,       0x0285,  0xa8,      r,   no,  0,     PVEX)

INSTR (VFMADD231PD,       xmm,        xvvvv,      xmmmem128,  none,       0x0281,  0xb8,      r,   no,  0,     PVEX)
INSTR (VFMADD231PD,       ymm,        yvvvv,      ymmmem256,  none,       0x0285,  0xb8,      r,   no,  0,     PVEX)

INSTR (VFMADDPS,          xmm,        xvvvv,      xmmmem128,  ximm,       0x0301,  0x68,      r,   is,  0,     PVEX)
INSTR (VFMADDPS,          ymm,        yvvvv,      ymmmem256,  yimm,       0x0305,  0x68,      r,   is,  0,     PVEX)
INSTR (VFMADDPS,          xmm,        xvvvv,      ximm,       xmmmem128,  0x0381,  0x68,      r,   is,  0,     PVEX)
INSTR (VFMADDPS,          ymm,        yvvvv,      yimm,       ymmmem256,  0x0385,  0x68,      r,   is,  0,     PVEX)

INSTR (VFMADD132PS,       xmm,        xvvvv,      xmmmem128,  none,       0x0201,  0x98,      r,   no,  0,     PVEX)
INSTR (VFMADD132PS,       ymm,        yvvvv,      ymmmem256,  none,       0x0205,  0x98,      r,   no,  0,     PVEX)

INSTR (VFMADD213PS,       xmm,        xvvvv,      xmmmem128,  none,       0x0201,  0xa8,      r,   no,  0,     PVEX)
INSTR (VFMADD213PS,       ymm,        yvvvv,      ymmmem256,  none,       0x0205,  0xa8,      r,   no,  0,     PVEX)

INSTR (VFMADD231PS,       xmm,        xvvvv,      xmmmem128,  none,       0x0201,  0xb8,      r,   no,  0,     PVEX)
INSTR (VFMADD231PS,       ymm,        yvvvv,      ymmmem256,  none,       0x0205,  0xb8,      r,   no,  0,     PVEX)

INSTR (VFMADDSD,          xmm,        xvvvv,      xmmmem64,   ximm,       0x0301,  0x6b,      r,   is,  0,     PVEX)
INSTR (VFMADDSD,          xmm,        xvvvv,      ximm,       xmmmem64,   0x0381,  0x6b,      r,   is,  0,     PVEX)

INSTR (VFMADD132SD,       xmm,        xvvvv,      xmmmem64,   none,       0x0281,  0x99,      r,   no,  0,     PVEX)

INSTR (VFMADD213SD,       xmm,        xvvvv,      xmmmem64,   none,       0x0281,  0xa9,      r,   no,  0,     PVEX)

INSTR (VFMADD231SD,       xmm,        xvvvv,      xmmmem64,   none,       0x0281,  0xb9,      r,   no,  0,     PVEX)

INSTR (VFMADDSS,          xmm,        xvvvv,      xmmmem32,   ximm,       0x0301,  0x6a,      r,   is,  0,     PVEX)
INSTR (VFMADDSS,          xmm,        xvvvv,      ximm,       xmmmem32,   0x0381,  0x6a,      r,   is,  0,     PVEX)

INSTR (VFMADD132SS,       xmm,        xvvvv,      xmmmem32,   none,       0x0201,  0x99,      r,   no,  0,     PVEX)

INSTR (VFMADD213SS,       xmm,        xvvvv,      xmmmem32,   none,       0x0201,  0xa9,      r,   no,  0,     PVEX)

INSTR (VFMADD231SS,       xmm,        xvvvv,      xmmmem32,   none,       0x0201,  0xb9,      r,   no,  0,     PVEX)

INSTR (VFMADDSUBPD,       xmm,        xvvvv,      xmmmem128,  ximm,       0x0301,  0x5d,      r,   is,  0,     PVEX)
INSTR (VFMADDSUBPD,       ymm,        yvvvv,      ymmmem256,  yimm,       0x0305,  0x5d,      r,   is,  0,     PVEX)
INSTR (VFMADDSUBPD,       xmm,        xvvvv,      ximm,       xmmmem128,  0x0381,  0x5d,      r,   is,  0,     PVEX)
INSTR (VFMADDSUBPD,       ymm,        yvvvv,      yimm,       ymmmem256,  0x0385,  0x5d,      r,   is,  0,     PVEX)

INSTR (VFMADDSUB132PD,    xmm,        xvvvv,      xmmmem128,  none,       0x0281,  0x96,      r,   no,  0,     PVEX)
INSTR (VFMADDSUB132PD,    ymm,        yvvvv,      ymmmem256,  none,       0x0285,  0x96,      r,   no,  0,     PVEX)

INSTR (VFMADDSUB213PD,    xmm,        xvvvv,      xmmmem128,  none,       0x0281,  0xa6,      r,   no,  0,     PVEX)
INSTR (VFMADDSUB213PD,    ymm,        yvvvv,      ymmmem256,  none,       0x0285,  0xa6,      r,   no,  0,     PVEX)

INSTR (VFMADDSUB231PD,    xmm,        xvvvv,      xmmmem128,  none,       0x0281,  0xb6,      r,   no,  0,     PVEX)
INSTR (VFMADDSUB231PD,    ymm,        yvvvv,      ymmmem256,  none,       0x0285,  0xb6,      r,   no,  0,     PVEX)

INSTR (VFMADDSUBPS,       xmm,        xvvvv,      xmmmem128,  ximm,       0x0301,  0x5c,      r,   is,  0,     PVEX)
INSTR (VFMADDSUBPS,       ymm,        yvvvv,      ymmmem256,  yimm,       0x0305,  0x5c,      r,   is,  0,     PVEX)
INSTR (VFMADDSUBPS,       xmm,        xvvvv,      ximm,       xmmmem128,  0x0381,  0x5c,      r,   is,  0,     PVEX)
INSTR (VFMADDSUBPS,       ymm,        yvvvv,      yimm,       ymmmem256,  0x0385,  0x5c,      r,   is,  0,     PVEX)

INSTR (VFMADDSUB132PS,    xmm,        xvvvv,      xmmmem128,  none,       0x0201,  0x96,      r,   no,  0,     PVEX)
INSTR (VFMADDSUB132PS,    ymm,        yvvvv,      ymmmem256,  none,       0x0205,  0x96,      r,   no,  0,     PVEX)

INSTR (VFMADDSUB213PS,    xmm,        xvvvv,      xmmmem128,  none,       0x0201,  0xa6,      r,   no,  0,     PVEX)
INSTR (VFMADDSUB213PS,    ymm,        yvvvv,      ymmmem256,  none,       0x0205,  0xa6,      r,   no,  0,     PVEX)

INSTR (VFMADDSUB231PS,    xmm,        xvvvv,      xmmmem128,  none,       0x0201,  0xb6,      r,   no,  0,     PVEX)
INSTR (VFMADDSUB231PS,    ymm,        yvvvv,      ymmmem256,  none,       0x0205,  0xb6,      r,   no,  0,     PVEX)

INSTR (VFMSUBADDPD,       xmm,        xvvvv,      xmmmem128,  ximm,       0x0301,  0x5f,      r,   is,  0,     PVEX)
INSTR (VFMSUBADDPD,       ymm,        yvvvv,      ymmmem256,  yimm,       0x0305,  0x5f,      r,   is,  0,     PVEX)
INSTR (VFMSUBADDPD,       xmm,        xvvvv,      ximm,       xmmmem128,  0x0381,  0x5f,      r,   is,  0,     PVEX)
INSTR (VFMSUBADDPD,       ymm,        yvvvv,      yimm,       ymmmem256,  0x0385,  0x5f,      r,   is,  0,     PVEX)

INSTR (VFMSUBADD132PD,    xmm,        xvvvv,      xmmmem128,  none,       0x0281,  0x97,      r,   no,  0,     PVEX)
INSTR (VFMSUBADD132PD,    ymm,        yvvvv,      ymmmem256,  none,       0x0285,  0x97,      r,   no,  0,     PVEX)

INSTR (VFMSUBADD213PD,    xmm,        xvvvv,      xmmmem128,  none,       0x0281,  0xa7,      r,   no,  0,     PVEX)
INSTR (VFMSUBADD213PD,    ymm,        yvvvv,      ymmmem256,  none,       0x0285,  0xa7,      r,   no,  0,     PVEX)

INSTR (VFMSUBADD231PD,    xmm,        xvvvv,      xmmmem128,  none,       0x0281,  0xb7,      r,   no,  0,     PVEX)
INSTR (VFMSUBADD231PD,    ymm,        yvvvv,      ymmmem256,  none,       0x0285,  0xb7,      r,   no,  0,     PVEX)

INSTR (VFMSUBADDPS,       xmm,        xvvvv,      xmmmem128,  ximm,       0x0301,  0x5e,      r,   is,  0,     PVEX)
INSTR (VFMSUBADDPS,       ymm,        yvvvv,      ymmmem256,  yimm,       0x0305,  0x5e,      r,   is,  0,     PVEX)
INSTR (VFMSUBADDPS,       xmm,        xvvvv,      ximm,       xmmmem128,  0x0381,  0x5e,      r,   is,  0,     PVEX)
INSTR (VFMSUBADDPS,       ymm,        yvvvv,      yimm,       ymmmem256,  0x0385,  0x5e,      r,   is,  0,     PVEX)

INSTR (VFMSUBADD132PS,    xmm,        xvvvv,      xmmmem128,  none,       0x0201,  0x97,      r,   no,  0,     PVEX)
INSTR (VFMSUBADD132PS,    ymm,        yvvvv,      ymmmem256,  none,       0x0205,  0x97,      r,   no,  0,     PVEX)

INSTR (VFMSUBADD213PS,    xmm,        xvvvv,      xmmmem128,  none,       0x0201,  0xa7,      r,   no,  0,     PVEX)
INSTR (VFMSUBADD213PS,    ymm,        yvvvv,      ymmmem256,  none,       0x0205,  0xa7,      r,   no,  0,     PVEX)

INSTR (VFMSUBADD231PS,    xmm,        xvvvv,      xmmmem128,  none,       0x0201,  0xb7,      r,   no,  0,     PVEX)
INSTR (VFMSUBADD231PS,    ymm,        yvvvv,      ymmmem256,  none,       0x0205,  0xb7,      r,   no,  0,     PVEX)

INSTR (VFMSUBPD,          xmm,        xvvvv,      xmmmem128,  ximm,       0x0301,  0x6d,      r,   is,  0,     PVEX)
INSTR (VFMSUBPD,          ymm,        yvvvv,      ymmmem256,  yimm,       0x0305,  0x6d,      r,   is,  0,     PVEX)
INSTR (VFMSUBPD,          xmm,        xvvvv,      ximm,       xmmmem128,  0x0381,  0x6d,      r,   is,  0,     PVEX)
INSTR (VFMSUBPD,          ymm,        yvvvv,      yimm,       ymmmem256,  0x0385,  0x6d,      r,   is,  0,     PVEX)

INSTR (VFMSUB132PD,       xmm,        xvvvv,      xmmmem128,  none,       0x0281,  0x9a,      r,   no,  0,     PVEX)
INSTR (VFMSUB132PD,       ymm,        yvvvv,      ymmmem256,  none,       0x0285,  0x9a,      r,   no,  0,     PVEX)

INSTR (VFMSUB213PD,       xmm,        xvvvv,      xmmmem128,  none,       0x0281,  0xaa,      r,   no,  0,     PVEX)
INSTR (VFMSUB213PD,       ymm,        yvvvv,      ymmmem256,  none,       0x0285,  0xaa,      r,   no,  0,     PVEX)

INSTR (VFMSUB231PD,       xmm,        xvvvv,      xmmmem128,  none,       0x0281,  0xba,      r,   no,  0,     PVEX)
INSTR (VFMSUB231PD,       ymm,        yvvvv,      ymmmem256,  none,       0x0285,  0xba,      r,   no,  0,     PVEX)

INSTR (VFMSUBPS,          xmm,        xvvvv,      xmmmem128,  ximm,       0x0301,  0x6c,      r,   is,  0,     PVEX)
INSTR (VFMSUBPS,          ymm,        yvvvv,      ymmmem256,  yimm,       0x0305,  0x6c,      r,   is,  0,     PVEX)
INSTR (VFMSUBPS,          xmm,        xvvvv,      ximm,       xmmmem128,  0x0381,  0x6c,      r,   is,  0,     PVEX)
INSTR (VFMSUBPS,          ymm,        yvvvv,      yimm,       ymmmem256,  0x0385,  0x6c,      r,   is,  0,     PVEX)

INSTR (VFMSUB132PS,       xmm,        xvvvv,      xmmmem128,  none,       0x0201,  0x9a,      r,   no,  0,     PVEX)
INSTR (VFMSUB132PS,       ymm,        yvvvv,      ymmmem256,  none,       0x0205,  0x9a,      r,   no,  0,     PVEX)

INSTR (VFMSUB213PS,       xmm,        xvvvv,      xmmmem128,  none,       0x0201,  0xaa,      r,   no,  0,     PVEX)
INSTR (VFMSUB213PS,       ymm,        yvvvv,      ymmmem256,  none,       0x0205,  0xaa,      r,   no,  0,     PVEX)

INSTR (VFMSUB231PS,       xmm,        xvvvv,      xmmmem128,  none,       0x0201,  0xba,      r,   no,  0,     PVEX)
INSTR (VFMSUB231PS,       ymm,        yvvvv,      ymmmem256,  none,       0x0205,  0xba,      r,   no,  0,     PVEX)

INSTR (VFMSUBSD,          xmm,        xvvvv,      xmmmem64,   ximm,       0x0301,  0x6f,      r,   is,  0,     PVEX)
INSTR (VFMSUBSD,          xmm,        xvvvv,      ximm,       xmmmem64,   0x0381,  0x6f,      r,   is,  0,     PVEX)

INSTR (VFMSUB132SD,       xmm,        xvvvv,      xmmmem64,   none,       0x0281,  0x9b,      r,   no,  0,     PVEX)

INSTR (VFMSUB213SD,       xmm,        xvvvv,      xmmmem64,   none,       0x0281,  0xab,      r,   no,  0,     PVEX)

INSTR (VFMSUB231SD,       xmm,        xvvvv,      xmmmem64,   none,       0x0281,  0xbb,      r,   no,  0,     PVEX)

INSTR (VFMSUBSS,          xmm,        xvvvv,      xmmmem32,   ximm,       0x0301,  0x6e,      r,   is,  0,     PVEX)
INSTR (VFMSUBSS,          xmm,        xvvvv,      ximm,       xmmmem32,   0x0381,  0x6e,      r,   is,  0,     PVEX)

INSTR (VFMSUB132SS,       xmm,        xvvvv,      xmmmem32,   none,       0x0201,  0x9b,      r,   no,  0,     PVEX)

INSTR (VFMSUB213SS,       xmm,        xvvvv,      xmmmem32,   none,       0x0201,  0xab,      r,   no,  0,     PVEX)

INSTR (VFMSUB231SS,       xmm,        xvvvv,      xmmmem32,   none,       0x0201,  0xbb,      r,   no,  0,     PVEX)

INSTR (VFNMADDPD,         xmm,        xvvvv,      xmmmem128,  ximm,       0x0301,  0x79,      r,   is,  0,     PVEX)
INSTR (VFNMADDPD,         ymm,        yvvvv,      ymmmem256,  yimm,       0x0305,  0x79,      r,   is,  0,     PVEX)
INSTR (VFNMADDPD,         xmm,        xvvvv,      ximm,       xmmmem128,  0x0381,  0x79,      r,   is,  0,     PVEX)
INSTR (VFNMADDPD,         ymm,        yvvvv,      yimm,       ymmmem256,  0x0385,  0x79,      r,   is,  0,     PVEX)

INSTR (VFNMADD132PD,      xmm,        xvvvv,      xmmmem128,  none,       0x0281,  0x9c,      r,   no,  0,     PVEX)
INSTR (VFNMADD132PD,      ymm,        yvvvv,      ymmmem256,  none,       0x0285,  0x9c,      r,   no,  0,     PVEX)

INSTR (VFNMADD213PD,      xmm,        xvvvv,      xmmmem128,  none,       0x0281,  0xac,      r,   no,  0,     PVEX)
INSTR (VFNMADD213PD,      ymm,        yvvvv,      ymmmem256,  none,       0x0285,  0xac,      r,   no,  0,     PVEX)

INSTR (VFNMADD231PD,      xmm,        xvvvv,      xmmmem128,  none,       0x0281,  0xbc,      r,   no,  0,     PVEX)
INSTR (VFNMADD231PD,      ymm,        yvvvv,      ymmmem256,  none,       0x0285,  0xbc,      r,   no,  0,     PVEX)

INSTR (VFNMADDPS,         xmm,        xvvvv,      xmmmem128,  ximm,       0x0301,  0x78,      r,   is,  0,     PVEX)
INSTR (VFNMADDPS,         ymm,        yvvvv,      ymmmem256,  yimm,       0x0305,  0x78,      r,   is,  0,     PVEX)
INSTR (VFNMADDPS,         xmm,        xvvvv,      ximm,       xmmmem128,  0x0381,  0x78,      r,   is,  0,     PVEX)
INSTR (VFNMADDPS,         ymm,        yvvvv,      yimm,       ymmmem256,  0x0385,  0x78,      r,   is,  0,     PVEX)

INSTR (VFNMADD132PS,      xmm,        xvvvv,      xmmmem128,  none,       0x0201,  0x9c,      r,   no,  0,     PVEX)
INSTR (VFNMADD132PS,      ymm,        yvvvv,      ymmmem256,  none,       0x0205,  0x9c,      r,   no,  0,     PVEX)

INSTR (VFNMADD213PS,      xmm,        xvvvv,      xmmmem128,  none,       0x0201,  0xac,      r,   no,  0,     PVEX)
INSTR (VFNMADD213PS,      ymm,        yvvvv,      ymmmem256,  none,       0x0205,  0xac,      r,   no,  0,     PVEX)

INSTR (VFNMADD231PS,      xmm,        xvvvv,      xmmmem128,  none,       0x0201,  0xbc,      r,   no,  0,     PVEX)
INSTR (VFNMADD231PS,      ymm,        yvvvv,      ymmmem256,  none,       0x0205,  0xbc,      r,   no,  0,     PVEX)

INSTR (VFNMADDSD,         xmm,        xvvvv,      xmmmem64,   ximm,       0x0301,  0x7b,      r,   is,  0,     PVEX)
INSTR (VFNMADDSD,         xmm,        xvvvv,      ximm,       xmmmem64,   0x0381,  0x7b,      r,   is,  0,     PVEX)

INSTR (VFNMADD132SD,      xmm,        xvvvv,      xmmmem64,   none,       0x0281,  0x9d,      r,   no,  0,     PVEX)

INSTR (VFNMADD213SD,      xmm,        xvvvv,      xmmmem64,   none,       0x0281,  0xad,      r,   no,  0,     PVEX)

INSTR (VFNMADD231SD,      xmm,        xvvvv,      xmmmem64,   none,       0x0281,  0xbd,      r,   no,  0,     PVEX)

INSTR (VFNMADDSS,         xmm,        xvvvv,      xmmmem32,   ximm,       0x0301,  0x7a,      r,   is,  0,     PVEX)
INSTR (VFNMADDSS,         xmm,        xvvvv,      ximm,       xmmmem32,   0x0381,  0x7a,      r,   is,  0,     PVEX)

INSTR (VFNMADD132SS,      xmm,        xvvvv,      xmmmem32,   none,       0x0201,  0x9d,      r,   no,  0,     PVEX)

INSTR (VFNMADD213SS,      xmm,        xvvvv,      xmmmem32,   none,       0x0201,  0xad,      r,   no,  0,     PVEX)

INSTR (VFNMADD231SS,      xmm,        xvvvv,      xmmmem32,   none,       0x0201,  0xbd,      r,   no,  0,     PVEX)

INSTR (VFNMSUBPD,         xmm,        xvvvv,      xmmmem128,  ximm,       0x0301,  0x7d,      r,   is,  0,     PVEX)
INSTR (VFNMSUBPD,         ymm,        yvvvv,      ymmmem256,  yimm,       0x0305,  0x7d,      r,   is,  0,     PVEX)
INSTR (VFNMSUBPD,         xmm,        xvvvv,      ximm,       xmmmem128,  0x0381,  0x7d,      r,   is,  0,     PVEX)
INSTR (VFNMSUBPD,         ymm,        yvvvv,      yimm,       ymmmem256,  0x0385,  0x7d,      r,   is,  0,     PVEX)

INSTR (VFNMSUB132PD,      xmm,        xvvvv,      xmmmem128,  none,       0x0281,  0x9e,      r,   no,  0,     PVEX)
INSTR (VFNMSUB132PD,      ymm,        yvvvv,      ymmmem256,  none,       0x0285,  0x9e,      r,   no,  0,     PVEX)

INSTR (VFNMSUB213PD,      xmm,        xvvvv,      xmmmem128,  none,       0x0281,  0xae,      r,   no,  0,     PVEX)
INSTR (VFNMSUB213PD,      ymm,        yvvvv,      ymmmem256,  none,       0x0285,  0xae,      r,   no,  0,     PVEX)

INSTR (VFNMSUB231PD,      xmm,        xvvvv,      xmmmem128,  none,       0x0281,  0xbe,      r,   no,  0,     PVEX)
INSTR (VFNMSUB231PD,      ymm,        yvvvv,      ymmmem256,  none,       0x0285,  0xbe,      r,   no,  0,     PVEX)

INSTR (VFNMSUBPS,         xmm,        xvvvv,      xmmmem128,  ximm,       0x0301,  0x7c,      r,   is,  0,     PVEX)
INSTR (VFNMSUBPS,         ymm,        yvvvv,      ymmmem256,  yimm,       0x0305,  0x7c,      r,   is,  0,     PVEX)
INSTR (VFNMSUBPS,         xmm,        xvvvv,      ximm,       xmmmem128,  0x0381,  0x7c,      r,   is,  0,     PVEX)
INSTR (VFNMSUBPS,         ymm,        yvvvv,      yimm,       ymmmem256,  0x0385,  0x7c,      r,   is,  0,     PVEX)

INSTR (VFNMSUB132PS,      xmm,        xvvvv,      xmmmem128,  none,       0x0201,  0x9e,      r,   no,  0,     PVEX)
INSTR (VFNMSUB132PS,      ymm,        yvvvv,      ymmmem256,  none,       0x0205,  0x9e,      r,   no,  0,     PVEX)

INSTR (VFNMSUB213PS,      xmm,        xvvvv,      xmmmem128,  none,       0x0201,  0xae,      r,   no,  0,     PVEX)
INSTR (VFNMSUB213PS,      ymm,        yvvvv,      ymmmem256,  none,       0x0205,  0xae,      r,   no,  0,     PVEX)

INSTR (VFNMSUB231PS,      xmm,        xvvvv,      xmmmem128,  none,       0x0201,  0xbe,      r,   no,  0,     PVEX)
INSTR (VFNMSUB231PS,      ymm,        yvvvv,      ymmmem256,  none,       0x0205,  0xbe,      r,   no,  0,     PVEX)

INSTR (VFNMSUBSD,         xmm,        xvvvv,      xmmmem64,   ximm,       0x0301,  0x7f,      r,   is,  0,     PVEX)
INSTR (VFNMSUBSD,         xmm,        xvvvv,      ximm,       xmmmem64,   0x0381,  0x7f,      r,   is,  0,     PVEX)

INSTR (VFNMSUB132SD,      xmm,        xvvvv,      xmmmem64,   none,       0x0281,  0x9f,      r,   no,  0,     PVEX)

INSTR (VFNMSUB213SD,      xmm,        xvvvv,      xmmmem64,   none,       0x0281,  0xaf,      r,   no,  0,     PVEX)

INSTR (VFNMSUB231SD,      xmm,        xvvvv,      xmmmem64,   none,       0x0281,  0xbf,      r,   no,  0,     PVEX)

INSTR (VFNMSUBSS,         xmm,        xvvvv,      xmmmem32,   ximm,       0x0301,  0x7e,      r,   is,  0,     PVEX)
INSTR (VFNMSUBSS,         xmm,        xvvvv,      ximm,       xmmmem32,   0x0381,  0x7e,      r,   is,  0,     PVEX)

INSTR (VFNMSUB132SS,      xmm,        xvvvv,      xmmmem32,   none,       0x0201,  0x9f,      r,   no,  0,     PVEX)

INSTR (VFNMSUB213SS,      xmm,        xvvvv,      xmmmem32,   none,       0x0201,  0xaf,      r,   no,  0,     PVEX)

INSTR (VFNMSUB231SS,      xmm,        xvvvv,      xmmmem32,   none,       0x0201,  0xbf,      r,   no,  0,     PVEX)

INSTR (VFRCZPD,           xmm,        xmmmem128,  none,       none,       0x0978,  0x81,      r,   no,  0,     PXOP)
INSTR (VFRCZPD,           ymm,        ymmmem256,  none,       none,       0x097c,  0x81,      r,   no,  0,     PXOP)

INSTR (VFRCZPS,           xmm,        xmmmem128,  none,       none,       0x0978,  0x80,      r,   no,  0,     PXOP)
INSTR (VFRCZPS,           ymm,        ymmmem256,  none,       none,       0x097c,  0x80,      r,   no,  0,     PXOP)

INSTR (VFRCZSD,           xmm,        xmmmem64,   none,       none,       0x0978,  0x83,      r,   no,  0,     PXOP)

INSTR (VFRCZSS,           xmm,        xmmmem32,   none,       none,       0x0978,  0x82,      r,   no,  0,     PXOP)

INSTR (VINSERTF128,       ymm,        yvvvv,      xmmmem128,  imm8,       0x0305,  0x18,      r,   ib,  0,     PVEX)

INSTR (VINSERTI128,       ymm,        yvvvv,      xmmmem128,  imm8,       0x0305,  0x38,      r,   ib,  0,     PVEX)

INSTR (VMASKMOVPD,        xmm,        xvvvv,      mem128,     none,       0x0201,  0x2d,      r,   no,  0,     PVEX)
INSTR (VMASKMOVPD,        ymm,        yvvvv,      mem256,     none,       0x0205,  0x2d,      r,   no,  0,     PVEX)
INSTR (VMASKMOVPD,        mem128,     xvvvv,      xmm,        none,       0x0201,  0x2f,      r,   no,  0,     PVEX)
INSTR (VMASKMOVPD,        mem256,     yvvvv,      ymm,        none,       0x0205,  0x2f,      r,   no,  0,     PVEX)

INSTR (VMASKMOVPS,        xmm,        xvvvv,      mem128,     none,       0x0201,  0x2c,      r,   no,  0,     PVEX)
INSTR (VMASKMOVPS,        ymm,        yvvvv,      mem256,     none,       0x0205,  0x2c,      r,   no,  0,     PVEX)
INSTR (VMASKMOVPS,        mem128,     xvvvv,      xmm,        none,       0x0201,  0x2e,      r,   no,  0,     PVEX)
INSTR (VMASKMOVPS,        mem256,     yvvvv,      ymm,        none,       0x0205,  0x2e,      r,   no,  0,     PVEX)

INSTR (VPBLENDD,          xmm,        xvvvv,      xmmmem128,  imm8,       0x0301,  0x02,      r,   ib,  0,     PVEX)
INSTR (VPBLENDD,          ymm,        yvvvv,      ymmmem256,  imm8,       0x0305,  0x02,      r,   ib,  0,     PVEX)

INSTR (VPBROADCASTB,      xmm,        xmmmem8,    none,       none,       0x0279,  0x78,      r,   no,  0,     PVEX)
INSTR (VPBROADCASTB,      ymm,        xmmmem8,    none,       none,       0x027d,  0x78,      r,   no,  0,     PVEX)

INSTR (VPBROADCASTD,      xmm,        xmmmem32,   none,       none,       0x0279,  0x58,      r,   no,  0,     PVEX)
INSTR (VPBROADCASTD,      ymm,        xmmmem32,   none,       none,       0x027d,  0x58,      r,   no,  0,     PVEX)

INSTR (VPBROADCASTQ,      xmm,        xmmmem64,   none,       none,       0x0279,  0x59,      r,   no,  0,     PVEX)
INSTR (VPBROADCASTQ,      ymm,        xmmmem64,   none,       none,       0x027d,  0x59,      r,   no,  0,     PVEX)

INSTR (VPBROADCASTW,      xmm,        xmmmem16,   none,       none,       0x0279,  0x79,      r,   no,  0,     PVEX)
INSTR (VPBROADCASTW,      ymm,        xmmmem16,   none,       none,       0x027d,  0x79,      r,   no,  0,     PVEX)

INSTR (VPCMOV,            xmm,        xvvvv,      xmmmem128,  ximm,       0x0800,  0xa2,      r,   is,  0,     PXOP)
INSTR (VPCMOV,            ymm,        yvvvv,      ymmmem256,  yimm,       0x0804,  0xa2,      r,   is,  0,     PXOP)
INSTR (VPCMOV,            xmm,        xvvvv,      ximm,       xmmmem128,  0x0880,  0xa2,      r,   is,  0,     PXOP)
INSTR (VPCMOV,            ymm,        yvvvv,      yimm,       ymmmem256,  0x0884,  0xa2,      r,   is,  0,     PXOP)

INSTR (VPCOMB,            xmm,        xvvvv,      xmmmem128,  imm8,       0x0800,  0xcc,      r,   ib,  0,     PXOP)

INSTR (VPCOMLTB,          xmm,        xvvvv,      xmmmem128,  none,       0x0800,  0xcc,      r,   no,  0x00,  PXOP | SFX)

INSTR (VPCOMLEB,          xmm,        xvvvv,      xmmmem128,  none,       0x0800,  0xcc,      r,   no,  0x01,  PXOP | SFX)

INSTR (VPCOMGTB,          xmm,        xvvvv,      xmmmem128,  none,       0x0800,  0xcc,      r,   no,  0x02,  PXOP | SFX)

INSTR (VPCOMGEB,          xmm,        xvvvv,      xmmmem128,  none,       0x0800,  0xcc,      r,   no,  0x03,  PXOP | SFX)

INSTR (VPCOMEQB,          xmm,        xvvvv,      xmmmem128,  none,       0x0800,  0xcc,      r,   no,  0x04,  PXOP | SFX)

INSTR (VPCOMNEQB,         xmm,        xvvvv,      xmmmem128,  none,       0x0800,  0xcc,      r,   no,  0x05,  PXOP | SFX)

INSTR (VPCOMFALSEB,       xmm,        xvvvv,      xmmmem128,  none,       0x0800,  0xcc,      r,   no,  0x06,  PXOP | SFX)

INSTR (VPCOMTRUEB,        xmm,        xvvvv,      xmmmem128,  none,       0x0800,  0xcc,      r,   no,  0x07,  PXOP | SFX)

INSTR (VPCOMD,            xmm,        xvvvv,      xmmmem128,  imm8,       0x0800,  0xce,      r,   ib,  0,     PXOP)

INSTR (VPCOMLTD,          xmm,        xvvvv,      xmmmem128,  none,       0x0800,  0xce,      r,   no,  0x00,  PXOP | SFX)

INSTR (VPCOMLED,          xmm,        xvvvv,      xmmmem128,  none,       0x0800,  0xce,      r,   no,  0x01,  PXOP | SFX)

INSTR (VPCOMGTD,          xmm,        xvvvv,      xmmmem128,  none,       0x0800,  0xce,      r,   no,  0x02,  PXOP | SFX)

INSTR (VPCOMGED,          xmm,        xvvvv,      xmmmem128,  none,       0x0800,  0xce,      r,   no,  0x03,  PXOP | SFX)

INSTR (VPCOMEQD,          xmm,        xvvvv,      xmmmem128,  none,       0x0800,  0xce,      r,   no,  0x04,  PXOP | SFX)

INSTR (VPCOMNEQD,         xmm,        xvvvv,      xmmmem128,  none,       0x0800,  0xce,      r,   no,  0x05,  PXOP | SFX)

INSTR (VPCOMFALSED,       xmm,        xvvvv,      xmmmem128,  none,       0x0800,  0xce,      r,   no,  0x06,  PXOP | SFX)

INSTR (VPCOMTRUED,        xmm,        xvvvv,      xmmmem128,  none,       0x0800,  0xce,      r,   no,  0x07,  PXOP | SFX)

INSTR (VPCOMQ,            xmm,        xvvvv,      xmmmem128,  imm8,       0x0800,  0xcf,      r,   ib,  0,     PXOP)

INSTR (VPCOMLTQ,          xmm,        xvvvv,      xmmmem128,  none,       0x0800,  0xcf,      r,   no,  0x00,  PXOP | SFX)

INSTR (VPCOMLEQ,          xmm,        xvvvv,      xmmmem128,  none,       0x0800,  0xcf,      r,   no,  0x01,  PXOP | SFX)

INSTR (VPCOMGTQ,          xmm,        xvvvv,      xmmmem128,  none,       0x0800,  0xcf,      r,   no,  0x02,  PXOP | SFX)

INSTR (VPCOMGEQ,          xmm,        xvvvv,      xmmmem128,  none,       0x0800,  0xcf,      r,   no,  0x03,  PXOP | SFX)

INSTR (VPCOMEQQ,          xmm,        xvvvv,      xmmmem128,  none,       0x0800,  0xcf,      r,   no,  0x04,  PXOP | SFX)

INSTR (VPCOMNEQQ,         xmm,        xvvvv,      xmmmem128,  none,       0x0800,  0xcf,      r,   no,  0x05,  PXOP | SFX)

INSTR (VPCOMFALSEQ,       xmm,        xvvvv,      xmmmem128,  none,       0x0800,  0xcf,      r,   no,  0x06,  PXOP | SFX)

INSTR (VPCOMTRUEQ,        xmm,        xvvvv,      xmmmem128,  none,       0x0800,  0xcf,      r,   no,  0x07,  PXOP | SFX)

INSTR (VPCOMUB,           xmm,        xvvvv,      xmmmem128,  imm8,       0x0800,  0xec,      r,   ib,  0,     PXOP)

INSTR (VPCOMLTUB,         xmm,        xvvvv,      xmmmem128,  none,       0x0800,  0xec,      r,   no,  0x00,  PXOP | SFX)

INSTR (VPCOMLEUB,         xmm,        xvvvv,      xmmmem128,  none,       0x0800,  0xec,      r,   no,  0x01,  PXOP | SFX)

INSTR (VPCOMGTUB,         xmm,        xvvvv,      xmmmem128,  none,       0x0800,  0xec,      r,   no,  0x02,  PXOP | SFX)

INSTR (VPCOMGEUB,         xmm,        xvvvv,      xmmmem128,  none,       0x0800,  0xec,      r,   no,  0x03,  PXOP | SFX)

INSTR (VPCOMEQUB,         xmm,        xvvvv,      xmmmem128,  none,       0x0800,  0xec,      r,   no,  0x04,  PXOP | SFX)

INSTR (VPCOMNEQUB,        xmm,        xvvvv,      xmmmem128,  none,       0x0800,  0xec,      r,   no,  0x05,  PXOP | SFX)

INSTR (VPCOMFALSEUB,      xmm,        xvvvv,      xmmmem128,  none,       0x0800,  0xec,      r,   no,  0x06,  PXOP | SFX)

INSTR (VPCOMTRUEUB,       xmm,        xvvvv,      xmmmem128,  none,       0x0800,  0xec,      r,   no,  0x07,  PXOP | SFX)

INSTR (VPCOMUD,           xmm,        xvvvv,      xmmmem128,  imm8,       0x0800,  0xee,      r,   ib,  0,     PXOP)

INSTR (VPCOMLTUD,         xmm,        xvvvv,      xmmmem128,  none,       0x0800,  0xee,      r,   no,  0x00,  PXOP | SFX)

INSTR (VPCOMLEUD,         xmm,        xvvvv,      xmmmem128,  none,       0x0800,  0xee,      r,   no,  0x01,  PXOP | SFX)

INSTR (VPCOMGTUD,         xmm,        xvvvv,      xmmmem128,  none,       0x0800,  0xee,      r,   no,  0x02,  PXOP | SFX)

INSTR (VPCOMGEUD,         xmm,        xvvvv,      xmmmem128,  none,       0x0800,  0xee,      r,   no,  0x03,  PXOP | SFX)

INSTR (VPCOMEQUD,         xmm,        xvvvv,      xmmmem128,  none,       0x0800,  0xee,      r,   no,  0x04,  PXOP | SFX)

INSTR (VPCOMNEQUD,        xmm,        xvvvv,      xmmmem128,  none,       0x0800,  0xee,      r,   no,  0x05,  PXOP | SFX)

INSTR (VPCOMFALSEUD,      xmm,        xvvvv,      xmmmem128,  none,       0x0800,  0xee,      r,   no,  0x06,  PXOP | SFX)

INSTR (VPCOMTRUEUD,       xmm,        xvvvv,      xmmmem128,  none,       0x0800,  0xee,      r,   no,  0x07,  PXOP | SFX)

INSTR (VPCOMUQ,           xmm,        xvvvv,      xmmmem128,  imm8,       0x0800,  0xef,      r,   ib,  0,     PXOP)

INSTR (VPCOMLTUQ,         xmm,        xvvvv,      xmmmem128,  none,       0x0800,  0xef,      r,   no,  0x00,  PXOP | SFX)

INSTR (VPCOMLEUQ,         xmm,        xvvvv,      xmmmem128,  none,       0x0800,  0xef,      r,   no,  0x01,  PXOP | SFX)

INSTR (VPCOMGTUQ,         xmm,        xvvvv,      xmmmem128,  none,       0x0800,  0xef,      r,   no,  0x02,  PXOP | SFX)

INSTR (VPCOMGEUQ,         xmm,        xvvvv,      xmmmem128,  none,       0x0800,  0xef,      r,   no,  0x03,  PXOP | SFX)

INSTR (VPCOMEQUQ,         xmm,        xvvvv,      xmmmem128,  none,       0x0800,  0xef,      r,   no,  0x04,  PXOP | SFX)

INSTR (VPCOMNEQUQ,        xmm,        xvvvv,      xmmmem128,  none,       0x0800,  0xef,      r,   no,  0x05,  PXOP | SFX)

INSTR (VPCOMFALSEUQ,      xmm,        xvvvv,      xmmmem128,  none,       0x0800,  0xef,      r,   no,  0x06,  PXOP | SFX)

INSTR (VPCOMTRUEUQ,       xmm,        xvvvv,      xmmmem128,  none,       0x0800,  0xef,      r,   no,  0x07,  PXOP | SFX)

INSTR (VPCOMUW,           xmm,        xvvvv,      xmmmem128,  imm8,       0x0800,  0xed,      r,   ib,  0,     PXOP)

INSTR (VPCOMLTUW,         xmm,        xvvvv,      xmmmem128,  none,       0x0800,  0xed,      r,   no,  0x00,  PXOP | SFX)

INSTR (VPCOMLEUW,         xmm,        xvvvv,      xmmmem128,  none,       0x0800,  0xed,      r,   no,  0x01,  PXOP | SFX)

INSTR (VPCOMGTUW,         xmm,        xvvvv,      xmmmem128,  none,       0x0800,  0xed,      r,   no,  0x02,  PXOP | SFX)

INSTR (VPCOMGEUW,         xmm,        xvvvv,      xmmmem128,  none,       0x0800,  0xed,      r,   no,  0x03,  PXOP | SFX)

INSTR (VPCOMEQUW,         xmm,        xvvvv,      xmmmem128,  none,       0x0800,  0xed,      r,   no,  0x04,  PXOP | SFX)

INSTR (VPCOMNEQUW,        xmm,        xvvvv,      xmmmem128,  none,       0x0800,  0xed,      r,   no,  0x05,  PXOP | SFX)

INSTR (VPCOMFALSEUW,      xmm,        xvvvv,      xmmmem128,  none,       0x0800,  0xed,      r,   no,  0x06,  PXOP | SFX)

INSTR (VPCOMTRUEUW,       xmm,        xvvvv,      xmmmem128,  none,       0x0800,  0xed,      r,   no,  0x07,  PXOP | SFX)

INSTR (VPCOMW,            xmm,        xvvvv,      xmmmem128,  imm8,       0x0800,  0xcd,      r,   ib,  0,     PXOP)

INSTR (VPCOMLTW,          xmm,        xvvvv,      xmmmem128,  none,       0x0800,  0xcd,      r,   no,  0x00,  PXOP | SFX)

INSTR (VPCOMLEW,          xmm,        xvvvv,      xmmmem128,  none,       0x0800,  0xcd,      r,   no,  0x01,  PXOP | SFX)

INSTR (VPCOMGTW,          xmm,        xvvvv,      xmmmem128,  none,       0x0800,  0xcd,      r,   no,  0x02,  PXOP | SFX)

INSTR (VPCOMGEW,          xmm,        xvvvv,      xmmmem128,  none,       0x0800,  0xcd,      r,   no,  0x03,  PXOP | SFX)

INSTR (VPCOMEQW,          xmm,        xvvvv,      xmmmem128,  none,       0x0800,  0xcd,      r,   no,  0x04,  PXOP | SFX)

INSTR (VPCOMNEQW,         xmm,        xvvvv,      xmmmem128,  none,       0x0800,  0xcd,      r,   no,  0x05,  PXOP | SFX)

INSTR (VPCOMFALSEW,       xmm,        xvvvv,      xmmmem128,  none,       0x0800,  0xcd,      r,   no,  0x06,  PXOP | SFX)

INSTR (VPCOMTRUEW,        xmm,        xvvvv,      xmmmem128,  none,       0x0800,  0xcd,      r,   no,  0x07,  PXOP | SFX)

INSTR (VPERM2F128,        ymm,        yvvvv,      ymmmem256,  imm8,       0x0305,  0x06,      r,   ib,  0,     PVEX)

INSTR (VPERM2I128,        ymm,        yvvvv,      ymmmem256,  imm8,       0x0305,  0x46,      r,   ib,  0,     PVEX)

INSTR (VPERMD,            ymm,        yvvvv,      ymmmem256,  none,       0x0205,  0x36,      r,   no,  0,     PVEX)

INSTR (VPERMPD,           ymm,        ymmmem256,  imm8,       none,       0x03fd,  0x01,      r,   ib,  0,     PVEX)

INSTR (VPERMPS,           ymm,        yvvvv,      ymmmem256,  none,       0x0205,  0x16,      r,   no,  0,     PVEX)

INSTR (VPERMQ,            ymm,        ymmmem256,  imm8,       none,       0x03fd,  0x00,      r,   ib,  0,     PVEX)

INSTR (VPHADDBD,          xmm,        xmmmem128,  none,       none,       0x0978,  0xc2,      r,   no,  0,     PXOP)

INSTR (VPHADDBQ,          xmm,        xmmmem128,  none,       none,       0x0978,  0xc3,      r,   no,  0,     PXOP)

INSTR (VPHADDBW,          xmm,        xmmmem128,  none,       none,       0x0978,  0xc1,      r,   no,  0,     PXOP)

INSTR (VPHADDDQ,          xmm,        xmmmem128,  none,       none,       0x0978,  0xcb,      r,   no,  0,     PXOP)

INSTR (VPHADDUBD,         xmm,        xmmmem128,  none,       none,       0x0978,  0xd2,      r,   no,  0,     PXOP)

INSTR (VPHADDUBQ,         xmm,        xmmmem128,  none,       none,       0x0978,  0xd3,      r,   no,  0,     PXOP)

INSTR (VPHADDUBW,         xmm,        xmmmem128,  none,       none,       0x0978,  0xd1,      r,   no,  0,     PXOP)

INSTR (VPHADDUDQ,         xmm,        xmmmem128,  none,       none,       0x0978,  0xdb,      r,   no,  0,     PXOP)

INSTR (VPHADDUWD,         xmm,        xmmmem128,  none,       none,       0x0978,  0xd6,      r,   no,  0,     PXOP)

INSTR (VPHADDUWQ,         xmm,        xmmmem128,  none,       none,       0x0978,  0xd7,      r,   no,  0,     PXOP)

INSTR (VPHADDWD,          xmm,        xmmmem128,  none,       none,       0x0978,  0xc6,      r,   no,  0,     PXOP)

INSTR (VPHADDWQ,          xmm,        xmmmem128,  none,       none,       0x0978,  0xc7,      r,   no,  0,     PXOP)

INSTR (VPHSUBBW,          xmm,        xmmmem128,  none,       none,       0x0978,  0xe1,      r,   no,  0,     PXOP)

INSTR (VPHSUBDQ,          xmm,        xmmmem128,  none,       none,       0x0978,  0xe3,      r,   no,  0,     PXOP)

INSTR (VPHSUBWD,          xmm,        xmmmem128,  none,       none,       0x0978,  0xe2,      r,   no,  0,     PXOP)

INSTR (VPMACSDD,          xmm,        xvvvv,      xmmmem128,  ximm,       0x0800,  0x9e,      r,   is,  0,     PXOP)

INSTR (VPMACSDQH,         xmm,        xvvvv,      xmmmem128,  ximm,       0x0800,  0x9f,      r,   is,  0,     PXOP)

INSTR (VPMACSDQL,         xmm,        xvvvv,      xmmmem128,  ximm,       0x0800,  0x97,      r,   is,  0,     PXOP)

INSTR (VPMACSSDD,         xmm,        xvvvv,      xmmmem128,  ximm,       0x0800,  0x8e,      r,   is,  0,     PXOP)

INSTR (VPMACSSDQH,        xmm,        xvvvv,      xmmmem128,  ximm,       0x0800,  0x8f,      r,   is,  0,     PXOP)

INSTR (VPMACSSDQL,        xmm,        xvvvv,      xmmmem128,  ximm,       0x0800,  0x87,      r,   is,  0,     PXOP)

INSTR (VPMACSSWD,         xmm,        xvvvv,      xmmmem128,  ximm,       0x0800,  0x86,      r,   is,  0,     PXOP)

INSTR (VPMACSSWW,         xmm,        xvvvv,      xmmmem128,  ximm,       0x0800,  0x85,      r,   is,  0,     PXOP)

INSTR (VPMACSWD,          xmm,        xvvvv,      xmmmem128,  ximm,       0x0800,  0x96,      r,   is,  0,     PXOP)

INSTR (VPMACSWW,          xmm,        xvvvv,      xmmmem128,  ximm,       0x0800,  0x95,      r,   is,  0,     PXOP)

INSTR (VPMADCSSWD,        xmm,        xvvvv,      xmmmem128,  ximm,       0x0800,  0xa6,      r,   is,  0,     PXOP)

INSTR (VPMADCSWD,         xmm,        xvvvv,      xmmmem128,  ximm,       0x0800,  0xb6,      r,   is,  0,     PXOP)

INSTR (VPMASKMOVD,        xmm,        xvvvv,      mem128,     none,       0x0201,  0x8c,      r,   no,  0,     PVEX)
INSTR (VPMASKMOVD,        ymm,        yvvvv,      mem256,     none,       0x0205,  0x8c,      r,   no,  0,     PVEX)
INSTR (VPMASKMOVD,        mem128,     xvvvv,      xmm,        none,       0x0201,  0x8e,      r,   no,  0,     PVEX)
INSTR (VPMASKMOVD,        mem256,     yvvvv,      ymm,        none,       0x0205,  0x8e,      r,   no,  0,     PVEX)

INSTR (VPMASKMOVQ,        xmm,        xvvvv,      mem128,     none,       0x0281,  0x8c,      r,   no,  0,     PVEX)
INSTR (VPMASKMOVQ,        ymm,        yvvvv,      mem256,     none,       0x0285,  0x8c,      r,   no,  0,     PVEX)
INSTR (VPMASKMOVQ,        mem128,     xvvvv,      xmm,        none,       0x0281,  0x8e,      r,   no,  0,     PVEX)
INSTR (VPMASKMOVQ,        mem256,     yvvvv,      ymm,        none,       0x0285,  0x8e,      r,   no,  0,     PVEX)

INSTR (VPPERM,            xmm,        xvvvv,      ximm,       xmmmem128,  0x0880,  0xa3,      r,   is,  0,     PXOP)
INSTR (VPPERM,            xmm,        xvvvv,      xmmmem128,  ximm,       0x0800,  0xa3,      r,   is,  0,     PXOP)

INSTR (VPROTB,            xmm,        xmmmem128,  xvvvv,      none,       0x0900,  0x90,      r,   no,  0,     PXOP)
INSTR (VPROTB,            xmm,        xvvvv,      xmmmem128,  none,       0x0980,  0x90,      r,   no,  0,     PXOP)
INSTR (VPROTB,            xmm,        xmmmem128,  imm8,       none,       0x0878,  0xc0,      r,   ib,  0,     PXOP)

INSTR (VPROTD,            xmm,        xmmmem128,  xvvvv,      none,       0x0900,  0x92,      r,   no,  0,     PXOP)
INSTR (VPROTD,            xmm,        xvvvv,      xmmmem128,  none,       0x0980,  0x92,      r,   no,  0,     PXOP)
INSTR (VPROTD,            xmm,        xmmmem128,  imm8,       none,       0x0878,  0xc2,      r,   ib,  0,     PXOP)

INSTR (VPROTQ,            xmm,        xmmmem128,  xvvvv,      none,       0x0900,  0x93,      r,   no,  0,     PXOP)
INSTR (VPROTQ,            xmm,        xvvvv,      xmmmem128,  none,       0x0980,  0x93,      r,   no,  0,     PXOP)
INSTR (VPROTQ,            xmm,        xmmmem128,  imm8,       none,       0x0878,  0xc3,      r,   ib,  0,     PXOP)

INSTR (VPROTW,            xmm,        xmmmem128,  xvvvv,      none,       0x0900,  0x91,      r,   no,  0,     PXOP)
INSTR (VPROTW,            xmm,        xvvvv,      xmmmem128,  none,       0x0980,  0x91,      r,   no,  0,     PXOP)
INSTR (VPROTW,            xmm,        xmmmem128,  imm8,       none,       0x0878,  0xc1,      r,   ib,  0,     PXOP)

INSTR (VPSHAB,            xmm,        xmmmem128,  xvvvv,      none,       0x0900,  0x98,      r,   no,  0,     PXOP)
INSTR (VPSHAB,            xmm,        xvvvv,      xmmmem128,  none,       0x0980,  0x98,      r,   no,  0,     PXOP)

INSTR (VPSHAD,            xmm,        xmmmem128,  xvvvv,      none,       0x0900,  0x9a,      r,   no,  0,     PXOP)
INSTR (VPSHAD,            xmm,        xvvvv,      xmmmem128,  none,       0x0980,  0x9a,      r,   no,  0,     PXOP)

INSTR (VPSHAQ,            xmm,        xmmmem128,  xvvvv,      none,       0x0900,  0x9b,      r,   no,  0,     PXOP)
INSTR (VPSHAQ,            xmm,        xvvvv,      xmmmem128,  none,       0x0980,  0x9b,      r,   no,  0,     PXOP)

INSTR (VPSHAW,            xmm,        xmmmem128,  xvvvv,      none,       0x0900,  0x99,      r,   no,  0,     PXOP)
INSTR (VPSHAW,            xmm,        xvvvv,      xmmmem128,  none,       0x0980,  0x99,      r,   no,  0,     PXOP)

INSTR (VPSHLB,            xmm,        xmmmem128,  xvvvv,      none,       0x0900,  0x94,      r,   no,  0,     PXOP)
INSTR (VPSHLB,            xmm,        xvvvv,      xmmmem128,  none,       0x0980,  0x94,      r,   no,  0,     PXOP)

INSTR (VPSHLD,            xmm,        xmmmem128,  xvvvv,      none,       0x0900,  0x96,      r,   no,  0,     PXOP)
INSTR (VPSHLD,            xmm,        xvvvv,      xmmmem128,  none,       0x0980,  0x96,      r,   no,  0,     PXOP)

INSTR (VPSHLQ,            xmm,        xmmmem128,  xvvvv,      none,       0x0900,  0x97,      r,   no,  0,     PXOP)
INSTR (VPSHLQ,            xmm,        xvvvv,      xmmmem128,  none,       0x0980,  0x97,      r,   no,  0,     PXOP)

INSTR (VPSHLW,            xmm,        xmmmem128,  xvvvv,      none,       0x0900,  0x95,      r,   no,  0,     PXOP)
INSTR (VPSHLW,            xmm,        xvvvv,      xmmmem128,  none,       0x0980,  0x95,      r,   no,  0,     PXOP)

INSTR (VPSLLVD,           xmm,        xvvvv,      xmmmem128,  none,       0x0201,  0x47,      r,   no,  0,     PVEX)
INSTR (VPSLLVD,           ymm,        yvvvv,      ymmmem256,  none,       0x0205,  0x47,      r,   no,  0,     PVEX)

INSTR (VPSLLVQ,           xmm,        xvvvv,      xmmmem128,  none,       0x0281,  0x47,      r,   no,  0,     PVEX)
INSTR (VPSLLVQ,           ymm,        yvvvv,      ymmmem256,  none,       0x0285,  0x47,      r,   no,  0,     PVEX)

INSTR (VPSRAVD,           xmm,        xvvvv,      xmmmem128,  none,       0x0201,  0x46,      r,   no,  0,     PVEX)
INSTR (VPSRAVD,           ymm,        yvvvv,      ymmmem256,  none,       0x0205,  0x46,      r,   no,  0,     PVEX)

INSTR (VPSRLVD,           xmm,        xvvvv,      xmmmem128,  none,       0x0201,  0x45,      r,   no,  0,     PVEX)
INSTR (VPSRLVD,           ymm,        yvvvv,      ymmmem256,  none,       0x0205,  0x45,      r,   no,  0,     PVEX)

INSTR (VPSRLVQ,           xmm,        xvvvv,      xmmmem128,  none,       0x0281,  0x45,      r,   no,  0,     PVEX)
INSTR (VPSRLVQ,           ymm,        yvvvv,      ymmmem256,  none,       0x0285,  0x45,      r,   no,  0,     PVEX)

INSTR (VTESTPD,           xmm,        xmmmem128,  none,       none,       0x0279,  0x0f,      r,   no,  0,     PVEX)
INSTR (VTESTPD,           ymm,        ymmmem256,  none,       none,       0x027d,  0x0f,      r,   no,  0,     PVEX)

INSTR (VTESTPS,           xmm,        xmmmem128,  none,       none,       0x0279,  0x0e,      r,   no,  0,     PVEX)
INSTR (VTESTPS,           ymm,        ymmmem256,  none,       none,       0x027d,  0x0e,      r,   no,  0,     PVEX)

INSTR (VZEROALL,          none,       none,       none,       none,       0x017c,  0x77,      no,  no,  0,     PVEX)

INSTR (VZEROUPPER,        none,       none,       none,       none,       0x0178,  0x77,      no,  no,  0,     PVEX)

INSTR (XORPD,             xmm,        xmmmem128,  none,       none,       0,       0x0f57,    r,   no,  0,     P66)

INSTR (VXORPD,            xmm,        xvvvv,      xmmmem128,  none,       0x0101,  0x57,      r,   no,  0,     PVEX)
INSTR (VXORPD,            ymm,        yvvvv,      ymmmem256,  none,       0x0105,  0x57,      r,   no,  0,     PVEX)

INSTR (XORPS,             xmm,        xmmmem128,  none,       none,       0,       0x0f57,    r,   no,  0,     0)

INSTR (VXORPS,            xmm,        xvvvv,      xmmmem128,  none,       0x0100,  0x57,      r,   no,  0,     PVEX)
INSTR (VXORPS,            ymm,        yvvvv,      ymmmem256,  none,       0x0104,  0x57,      r,   no,  0,     PVEX)

INSTR (XGETBV,            none,       none,       none,       none,       0,       0x0f01,    no,  no,  0xd0,  SFX)

INSTR (XSETBV,            none,       none,       none,       none,       0,       0x0f01,    no,  no,  0xd1,  SFX)

// 64-Bit Media Instruction Reference

INSTR (EMMS,              none,       none,       none,       none,       0,       0x0f77,    no,  no,  0,     0)

INSTR (FEMMS,             none,       none,       none,       none,       0,       0x0f0e,    no,  no,  0,     0)

INSTR (MASKMOVQ,          mmx,        mmxmem64,   none,       none,       0,       0x0ff7,    r,   no,  0,     0)

INSTR (MOVNTQ,            mem64,      mmx,        none,       none,       0,       0x0fe7,    r,   no,  0,     0)

INSTR (PAVGUSB,           mmx,        mmxmem64,   none,       none,       0,       0x0f0f,    r,   no,  0xbf,  SFX)

INSTR (PF2ID,             mmx,        mmxmem64,   none,       none,       0,       0x0f0f,    r,   no,  0x1d,  SFX)

INSTR (PF2IW,             mmx,        mmxmem64,   none,       none,       0,       0x0f0f,    r,   no,  0x1c,  SFX)

INSTR (PFACC,             mmx,        mmxmem64,   none,       none,       0,       0x0f0f,    r,   no,  0xae,  SFX)

INSTR (PFADD,             mmx,        mmxmem64,   none,       none,       0,       0x0f0f,    r,   no,  0x9e,  SFX)

INSTR (PFCMPEQ,           mmx,        mmxmem64,   none,       none,       0,       0x0f0f,    r,   no,  0xb0,  SFX)

INSTR (PFCMPGE,           mmx,        mmxmem64,   none,       none,       0,       0x0f0f,    r,   no,  0x90,  SFX)

INSTR (PFCMPGT,           mmx,        mmxmem64,   none,       none,       0,       0x0f0f,    r,   no,  0xa0,  SFX)

INSTR (PFMAX,             mmx,        mmxmem64,   none,       none,       0,       0x0f0f,    r,   no,  0xa4,  SFX)

INSTR (PFMIN,             mmx,        mmxmem64,   none,       none,       0,       0x0f0f,    r,   no,  0x94,  SFX)

INSTR (PFMUL,             mmx,        mmxmem64,   none,       none,       0,       0x0f0f,    r,   no,  0xb4,  SFX)

INSTR (PFNACC,            mmx,        mmxmem64,   none,       none,       0,       0x0f0f,    r,   no,  0x8a,  SFX)

INSTR (PFPNACC,           mmx,        mmxmem64,   none,       none,       0,       0x0f0f,    r,   no,  0x8e,  SFX)

INSTR (PFRCP,             mmx,        mmxmem64,   none,       none,       0,       0x0f0f,    r,   no,  0x96,  SFX)

INSTR (PFRCPIT1,          mmx,        mmxmem64,   none,       none,       0,       0x0f0f,    r,   no,  0xa6,  SFX)

INSTR (PFRCPIT2,          mmx,        mmxmem64,   none,       none,       0,       0x0f0f,    r,   no,  0xb6,  SFX)

INSTR (PFRSQIT1,          mmx,        mmxmem64,   none,       none,       0,       0x0f0f,    r,   no,  0xa7,  SFX)

INSTR (PFRSQRT,           mmx,        mmxmem64,   none,       none,       0,       0x0f0f,    r,   no,  0x97,  SFX)

INSTR (PFSUB,             mmx,        mmxmem64,   none,       none,       0,       0x0f0f,    r,   no,  0x9a,  SFX)

INSTR (PFSUBR,            mmx,        mmxmem64,   none,       none,       0,       0x0f0f,    r,   no,  0xaa,  SFX)

INSTR (PI2FD,             mmx,        mmxmem64,   none,       none,       0,       0x0f0f,    r,   no,  0x0d,  SFX)

INSTR (PI2FW,             mmx,        mmxmem64,   none,       none,       0,       0x0f0f,    r,   no,  0x0c,  SFX)

INSTR (PMULHRW,           mmx,        mmxmem64,   none,       none,       0,       0x0f0f,    r,   no,  0xb7,  SFX)

INSTR (PSHUFW,            mmx,        mmxmem64,   imm8,       none,       0,       0x0f70,    r,   ib,  0,     0)

INSTR (PSWAPD,            mmx,        mmxmem64,   none,       none,       0,       0x0f0f,    r,   no,  0xbb,  SFX)

// x87 Floating-Point Instruction Reference

INSTR (F2XM1,             none,       none,       none,       none,       0,       0xd9f0,    no,  no,  0,     0)

INSTR (FABS,              none,       none,       none,       none,       0,       0xd9e1,    no,  no,  0,     0)

INSTR (FADD,              st0,        sti,        none,       none,       0,       0xd8c0,    rv,  no,  0,     0)
INSTR (FADD,              sti,        st0,        none,       none,       0,       0xdcc0,    rv,  no,  0,     0)
INSTR (FADD,              mem32,      none,       none,       none,       0,       0xd8,      r0,  no,  0,     0)
INSTR (FADD,              mem64,      none,       none,       none,       0,       0xdc,      r0,  no,  0,     0)

INSTR (FADDP,             none,       none,       none,       none,       0,       0xdec1,    no,  no,  0,     0)
INSTR (FADDP,             sti,        st0,        none,       none,       0,       0xdec0,    rv,  no,  0,     0)

INSTR (FIADD,             mem16,      none,       none,       none,       0,       0xde,      r0,  no,  0,     0)
INSTR (FIADD,             mem32,      none,       none,       none,       0,       0xda,      r0,  no,  0,     0)

INSTR (FBLD,              mem,        none,       none,       none,       0,       0xdf,      r4,  no,  0,     0)

INSTR (FBSTP,             mem,        none,       none,       none,       0,       0xdf,      r6,  no,  0,     0)

INSTR (FCHS,              none,       none,       none,       none,       0,       0xd9e0,    no,  no,  0,     0)

INSTR (FNCLEX,            none,       none,       none,       none,       0,       0xdbe2,    no,  no,  0,     0)

INSTR (FCMOVB,            st0,        sti,        none,       none,       0,       0xdac0,    rv,  no,  0,     0)

INSTR (FCMOVBE,           st0,        sti,        none,       none,       0,       0xdad0,    rv,  no,  0,     0)

INSTR (FCMOVE,            st0,        sti,        none,       none,       0,       0xdac8,    rv,  no,  0,     0)

INSTR (FCMOVNB,           st0,        sti,        none,       none,       0,       0xdbc0,    rv,  no,  0,     0)

INSTR (FCMOVNBE,          st0,        sti,        none,       none,       0,       0xdbd0,    rv,  no,  0,     0)

INSTR (FCMOVNE,           st0,        sti,        none,       none,       0,       0xdbc8,    rv,  no,  0,     0)

INSTR (FCMOVNU,           st0,        sti,        none,       none,       0,       0xdbd8,    rv,  no,  0,     0)

INSTR (FCMOVU,            st0,        sti,        none,       none,       0,       0xdad8,    rv,  no,  0,     0)

INSTR (FCOM,              none,       none,       none,       none,       0,       0xd8d1,    no,  no,  0,     0)
INSTR (FCOM,              sti,        none,       none,       none,       0,       0xd8d0,    rv,  no,  0,     0)
INSTR (FCOM,              mem32,      none,       none,       none,       0,       0xd8,      r2,  no,  0,     0)
INSTR (FCOM,              mem64,      none,       none,       none,       0,       0xdc,      r2,  no,  0,     0)

INSTR (FCOMP,             none,       none,       none,       none,       0,       0xd8d9,    no,  no,  0,     0)
INSTR (FCOMP,             sti,        none,       none,       none,       0,       0xd8d8,    rv,  no,  0,     0)
INSTR (FCOMP,             mem32,      none,       none,       none,       0,       0xd8,      r3,  no,  0,     0)
INSTR (FCOMP,             mem64,      none,       none,       none,       0,       0xdc,      r3,  no,  0,     0)

INSTR (FCOMPP,            none,       none,       none,       none,       0,       0xded9,    no,  no,  0,     0)

INSTR (FCOMI,             st0,        sti,        none,       none,       0,       0xdbf0,    rv,  no,  0,     0)

INSTR (FCOMIP,            st0,        sti,        none,       none,       0,       0xdff0,    rv,  no,  0,     0)

INSTR (FCOS,              none,       none,       none,       none,       0,       0xd9ff,    no,  no,  0,     0)

INSTR (FDECSTP,           none,       none,       none,       none,       0,       0xd9f6,    no,  no,  0,     0)

INSTR (FDIV,              st0,        sti,        none,       none,       0,       0xd8f0,    rv,  no,  0,     0)
INSTR (FDIV,              sti,        st0,        none,       none,       0,       0xdcf8,    rv,  no,  0,     0)
INSTR (FDIV,              mem32,      none,       none,       none,       0,       0xd8,      r6,  no,  0,     0)
INSTR (FDIV,              mem64,      none,       none,       none,       0,       0xdc,      r6,  no,  0,     0)

INSTR (FDIVP,             none,       none,       none,       none,       0,       0xdef9,    no,  no,  0,     0)
INSTR (FDIVP,             sti,        st0,        none,       none,       0,       0xdef8,    rv,  no,  0,     0)

INSTR (FIDIV,             mem16,      none,       none,       none,       0,       0xde,      r6,  no,  0,     0)
INSTR (FIDIV,             mem32,      none,       none,       none,       0,       0xda,      r6,  no,  0,     0)

INSTR (FDIVR,             st0,        sti,        none,       none,       0,       0xd8f8,    rv,  no,  0,     0)
INSTR (FDIVR,             sti,        st0,        none,       none,       0,       0xdcf0,    rv,  no,  0,     0)
INSTR (FDIVR,             mem32,      none,       none,       none,       0,       0xd8,      r7,  no,  0,     0)
INSTR (FDIVR,             mem64,      none,       none,       none,       0,       0xdc,      r7,  no,  0,     0)

INSTR (FDIVRP,            none,       none,       none,       none,       0,       0xdef1,    no,  no,  0,     0)
INSTR (FDIVRP,            sti,        st0,        none,       none,       0,       0xdef0,    rv,  no,  0,     0)

INSTR (FIDIVR,            mem16,      none,       none,       none,       0,       0xde,      r7,  no,  0,     0)
INSTR (FIDIVR,            mem32,      none,       none,       none,       0,       0xda,      r7,  no,  0,     0)

INSTR (FFREE,             sti,        none,       none,       none,       0,       0xddc0,    rv,  no,  0,     0)

INSTR (FICOM,             mem16,      none,       none,       none,       0,       0xde,      r2,  no,  0,     0)
INSTR (FICOM,             mem32,      none,       none,       none,       0,       0xda,      r2,  no,  0,     0)

INSTR (FICOMP,            mem16,      none,       none,       none,       0,       0xde,      r3,  no,  0,     0)
INSTR (FICOMP,            mem32,      none,       none,       none,       0,       0xda,      r3,  no,  0,     0)

INSTR (FILD,              mem16,      none,       none,       none,       0,       0xdf,      r0,  no,  0,     0)
INSTR (FILD,              mem32,      none,       none,       none,       0,       0xdb,      r0,  no,  0,     0)
INSTR (FILD,              mem64,      none,       none,       none,       0,       0xdf,      r5,  no,  0,     0)

INSTR (FINCSTP,           none,       none,       none,       none,       0,       0xd9f7,    no,  no,  0,     0)

INSTR (FNINIT,            none,       none,       none,       none,       0,       0xdbe3,    no,  no,  0,     0)

INSTR (FIST,              mem16,      none,       none,       none,       0,       0xdf,      r2,  no,  0,     0)
INSTR (FIST,              mem32,      none,       none,       none,       0,       0xdb,      r2,  no,  0,     0)

INSTR (FISTP,             mem16,      none,       none,       none,       0,       0xdf,      r3,  no,  0,     0)
INSTR (FISTP,             mem32,      none,       none,       none,       0,       0xdb,      r3,  no,  0,     0)
INSTR (FISTP,             mem64,      none,       none,       none,       0,       0xdf,      r7,  no,  0,     0)

INSTR (FISTTP,            mem16,      none,       none,       none,       0,       0xdf,      r1,  no,  0,     0)
INSTR (FISTTP,            mem32,      none,       none,       none,       0,       0xdb,      r1,  no,  0,     0)
INSTR (FISTTP,            mem64,      none,       none,       none,       0,       0xdd,      r1,  no,  0,     0)

INSTR (FLD,               sti,        none,       none,       none,       0,       0xd9c0,    rv,  no,  0,     0)
INSTR (FLD,               mem32,      none,       none,       none,       0,       0xd9,      r0,  no,  0,     0)
INSTR (FLD,               mem64,      none,       none,       none,       0,       0xdd,      r0,  no,  0,     0)

INSTR (FLD1,              none,       none,       none,       none,       0,       0xd9e8,    no,  no,  0,     0)

INSTR (FLDCW,             mem,        none,       none,       none,       0,       0xd9,      r5,  no,  0,     0)

INSTR (FLDENV,            mem,        none,       none,       none,       0,       0xd9,      r4,  no,  0,     0)

INSTR (FLDL2E,            none,       none,       none,       none,       0,       0xd9ea,    no,  no,  0,     0)

INSTR (FLDL2T,            none,       none,       none,       none,       0,       0xd9e9,    no,  no,  0,     0)

INSTR (FLDLG2,            none,       none,       none,       none,       0,       0xd9ec,    no,  no,  0,     0)

INSTR (FLDLN2,            none,       none,       none,       none,       0,       0xd9ed,    no,  no,  0,     0)

INSTR (FLDPI,             none,       none,       none,       none,       0,       0xd9eb,    no,  no,  0,     0)

INSTR (FLDZ,              none,       none,       none,       none,       0,       0xd9ee,    no,  no,  0,     0)

INSTR (FMUL,              st0,        sti,        none,       none,       0,       0xd8c8,    rv,  no,  0,     0)
INSTR (FMUL,              sti,        st0,        none,       none,       0,       0xdcc8,    rv,  no,  0,     0)
INSTR (FMUL,              mem32,      none,       none,       none,       0,       0xd8,      r1,  no,  0,     0)
INSTR (FMUL,              mem64,      none,       none,       none,       0,       0xdc,      r1,  no,  0,     0)

INSTR (FMULP,             none,       none,       none,       none,       0,       0xdec9,    no,  no,  0,     0)
INSTR (FMULP,             sti,        st0,        none,       none,       0,       0xdec8,    rv,  no,  0,     0)

INSTR (FIMUL,             mem16,      none,       none,       none,       0,       0xde,      r1,  no,  0,     0)
INSTR (FIMUL,             mem32,      none,       none,       none,       0,       0xda,      r1,  no,  0,     0)

INSTR (FNOP,              none,       none,       none,       none,       0,       0xd9d0,    no,  no,  0,     0)

INSTR (FPATAN,            none,       none,       none,       none,       0,       0xd9f3,    no,  no,  0,     0)

INSTR (FPREM,             none,       none,       none,       none,       0,       0xd9f8,    no,  no,  0,     0)

INSTR (FPREM1,            none,       none,       none,       none,       0,       0xd9f5,    no,  no,  0,     0)

INSTR (FPTAN,             none,       none,       none,       none,       0,       0xd9f2,    no,  no,  0,     0)

INSTR (FRNDINT,           none,       none,       none,       none,       0,       0xd9fc,    no,  no,  0,     0)

INSTR (FRSTOR,            mem,        none,       none,       none,       0,       0xdd,      r4,  no,  0,     0)

INSTR (FNSAVE,            mem,        none,       none,       none,       0,       0xdd,      r6,  no,  0,     0)

INSTR (FSCALE,            none,       none,       none,       none,       0,       0xd9fd,    no,  no,  0,     0)

INSTR (FSIN,              none,       none,       none,       none,       0,       0xd9fe,    no,  no,  0,     0)

INSTR (FSINCOS,           none,       none,       none,       none,       0,       0xd9fb,    no,  no,  0,     0)

INSTR (FSQRT,             none,       none,       none,       none,       0,       0xd9fa,    no,  no,  0,     0)

INSTR (FST,               sti,        none,       none,       none,       0,       0xddd0,    rv,  no,  0,     0)
INSTR (FST,               mem32,      none,       none,       none,       0,       0xd9,      r2,  no,  0,     0)
INSTR (FST,               mem64,      none,       none,       none,       0,       0xdd,      r2,  no,  0,     0)

INSTR (FSTP,              sti,        none,       none,       none,       0,       0xddd8,    rv,  no,  0,     0)
INSTR (FSTP,              mem32,      none,       none,       none,       0,       0xd9,      r3,  no,  0,     0)
INSTR (FSTP,              mem64,      none,       none,       none,       0,       0xdd,      r3,  no,  0,     0)

INSTR (FNSTCW,            mem,        none,       none,       none,       0,       0xd9,      r7,  no,  0,     0)

INSTR (FNSTENV,           mem,        none,       none,       none,       0,       0xd9,      r6,  no,  0,     0)

INSTR (FNSTSW,            ax,         none,       none,       none,       0,       0xdfe0,    no,  no,  0,     0)
INSTR (FNSTSW,            mem,        none,       none,       none,       0,       0xdd,      r7,  no,  0,     0)

INSTR (FSUB,              st0,        sti,        none,       none,       0,       0xd8e0,    rv,  no,  0,     0)
INSTR (FSUB,              sti,        st0,        none,       none,       0,       0xdce8,    rv,  no,  0,     0)
INSTR (FSUB,              mem32,      none,       none,       none,       0,       0xd8,      r4,  no,  0,     0)
INSTR (FSUB,              mem64,      none,       none,       none,       0,       0xdc,      r4,  no,  0,     0)

INSTR (FSUBP,             none,       none,       none,       none,       0,       0xdee9,    no,  no,  0,     0)
INSTR (FSUBP,             sti,        st0,        none,       none,       0,       0xdee8,    rv,  no,  0,     0)

INSTR (FISUB,             mem16,      none,       none,       none,       0,       0xde,      r4,  no,  0,     0)
INSTR (FISUB,             mem32,      none,       none,       none,       0,       0xda,      r4,  no,  0,     0)

INSTR (FSUBR,             st0,        sti,        none,       none,       0,       0xd8e8,    rv,  no,  0,     0)
INSTR (FSUBR,             sti,        st0,        none,       none,       0,       0xdce0,    rv,  no,  0,     0)
INSTR (FSUBR,             mem32,      none,       none,       none,       0,       0xd8,      r5,  no,  0,     0)
INSTR (FSUBR,             mem64,      none,       none,       none,       0,       0xdc,      r5,  no,  0,     0)

INSTR (FSUBRP,            none,       none,       none,       none,       0,       0xdee1,    no,  no,  0,     0)
INSTR (FSUBRP,            sti,        st0,        none,       none,       0,       0xdee0,    rv,  no,  0,     0)

INSTR (FISUBR,            mem16,      none,       none,       none,       0,       0xde,      r5,  no,  0,     0)
INSTR (FISUBR,            mem32,      none,       none,       none,       0,       0xda,      r5,  no,  0,     0)

INSTR (FTST,              none,       none,       none,       none,       0,       0xd9e4,    no,  no,  0,     0)

INSTR (FUCOM,             none,       none,       none,       none,       0,       0xdde1,    no,  no,  0,     0)
INSTR (FUCOM,             sti,        none,       none,       none,       0,       0xdde0,    rv,  no,  0,     0)

INSTR (FUCOMP,            none,       none,       none,       none,       0,       0xdde9,    no,  no,  0,     0)
INSTR (FUCOMP,            sti,        none,       none,       none,       0,       0xdde8,    rv,  no,  0,     0)

INSTR (FUCOMPP,           none,       none,       none,       none,       0,       0xdae9,    no,  no,  0,     0)

INSTR (FUCOMI,            st0,        sti,        none,       none,       0,       0xdbe8,    rv,  no,  0,     0)

INSTR (FUCOMIP,           st0,        sti,        none,       none,       0,       0xdfe8,    rv,  no,  0,     0)

INSTR (FWAIT,             none,       none,       none,       none,       0,       0x9b,      no,  no,  0,     0)

INSTR (FXAM,              none,       none,       none,       none,       0,       0xd9e5,    no,  no,  0,     0)

INSTR (FXCH,              none,       none,       none,       none,       0,       0xd9c9,    no,  no,  0,     0)
INSTR (FXCH,              sti,        none,       none,       none,       0,       0xd9c8,    rv,  no,  0,     0)

INSTR (FXTRACT,           none,       none,       none,       none,       0,       0xd9f4,    no,  no,  0,     0)

INSTR (FYL2X,             none,       none,       none,       none,       0,       0xd9f1,    no,  no,  0,     0)

INSTR (FYL2XP1,           none,       none,       none,       none,       0,       0xd9f9,    no,  no,  0,     0)

// operand types

TYPE (one)
TYPE (imm8)
TYPE (imm16)
TYPE (imm32)
TYPE (imm64)
TYPE (simm8)
TYPE (simm16)
TYPE (simm32)
TYPE (rel8off)
TYPE (rel16off)
TYPE (rel32off)
TYPE (moffset)
TYPE (al)
TYPE (cl)
TYPE (ax)
TYPE (dx)
TYPE (eax)
TYPE (ecx)
TYPE (rax)
TYPE (es)
TYPE (cs)
TYPE (ss)
TYPE (ds)
TYPE (fs)
TYPE (gs)
TYPE (cr8)
TYPE (reg8)
TYPE (reg16)
TYPE (reg16rm)
TYPE (reg32)
TYPE (reg32rm)
TYPE (reg32vvvv)
TYPE (reg64)
TYPE (reg64rm)
TYPE (reg64vvvv)
TYPE (mmx)
TYPE (xmm)
TYPE (ximm)
TYPE (xvvvv)
TYPE (ymm)
TYPE (yimm)
TYPE (yvvvv)
TYPE (segreg)
TYPE (cr)
TYPE (dr)
TYPE (st0)
TYPE (sti)
TYPE (mem)
TYPE (mem16)
TYPE (mem32)
TYPE (mem64)
TYPE (mem128)
TYPE (mem256)
TYPE (regmem8)
TYPE (regmem16)
TYPE (regmem32)
TYPE (regmem64)
TYPE (mmxmem32)
TYPE (mmxmem64)
TYPE (xmmmem8)
TYPE (xmmmem16)
TYPE (xmmmem32)
TYPE (xmmmem64)
TYPE (xmmmem128)
TYPE (ymmmem128)
TYPE (ymmmem256)

// operand codes

CODE (ib)
CODE (iw)
CODE (id)
CODE (iq)
CODE (is)
CODE (cb)
CODE (cw)
CODE (cd)
CODE (r)
CODE (r0)
CODE (r1)
CODE (r2)
CODE (r3)
CODE (r4)
CODE (r5)
CODE (r6)
CODE (r7)
CODE (rv)

// instruction flags

FLAG (I16,     0x1)
FLAG (I32,     0x2)
FLAG (I64,     0x4)
FLAG (D64,     0x8)
FLAG (O16,     0x10)
FLAG (O32,     0x20)
FLAG (O64,     0x40)
FLAG (PLOCK,   0x80)
FLAG (PREP,    0x100)
FLAG (PREPE,   0x100)
FLAG (PREPNE,  0x200)
FLAG (P66,     0x400)
FLAG (PF0,     0x800)
FLAG (PF2,     0x1000)
FLAG (PF3,     0x2000)
FLAG (PXOP,    0x4000)
FLAG (PVEX,    0x8000)
FLAG (SFX,     0x10000)
FLAG (FAR,     0x20000)

// registers

REG (AL,     al)
REG (CL,     cl)
REG (DL,     dl)
REG (BL,     bl)
REG (AH,     ah)
REG (CH,     ch)
REG (DH,     dh)
REG (BH,     bh)
REG (R8B,    r8b)
REG (R9B,    r9b)
REG (R10B,   r10b)
REG (R11B,   r11b)
REG (R12B,   r12b)
REG (R13B,   r13b)
REG (R14B,   r14b)
REG (R15B,   r15b)

REG (SPL,    spl)
REG (BPL,    bpl)
REG (SIL,    sil)
REG (DIL,    dil)

REG (AX,     ax)
REG (CX,     cx)
REG (DX,     dx)
REG (BX,     bx)
REG (SP,     sp)
REG (BP,     bp)
REG (SI,     si)
REG (DI,     di)
REG (R8W,    r8w)
REG (R9W,    r9w)
REG (R10W,   r10w)
REG (R11W,   r11w)
REG (R12W,   r12w)
REG (R13W,   r13w)
REG (R14W,   r14w)
REG (R15W,   r15w)

REG (EAX,    eax)
REG (ECX,    ecx)
REG (EDX,    edx)
REG (EBX,    ebx)
REG (ESP,    esp)
REG (EBP,    ebp)
REG (ESI,    esi)
REG (EDI,    edi)
REG (R8D,    r8d)
REG (R9D,    r9d)
REG (R10D,   r10d)
REG (R11D,   r11d)
REG (R12D,   r12d)
REG (R13D,   r13d)
REG (R14D,   r14d)
REG (R15D,   r15d)

REG (RAX,    rax)
REG (RCX,    rcx)
REG (RDX,    rdx)
REG (RBX,    rbx)
REG (RSP,    rsp)
REG (RBP,    rbp)
REG (RSI,    rsi)
REG (RDI,    rdi)
REG (R8,     r8)
REG (R9,     r9)
REG (R10,    r10)
REG (R11,    r11)
REG (R12,    r12)
REG (R13,    r13)
REG (R14,    r14)
REG (R15,    r15)

REG (MMX0,   mmx0)
REG (MMX1,   mmx1)
REG (MMX2,   mmx2)
REG (MMX3,   mmx3)
REG (MMX4,   mmx4)
REG (MMX5,   mmx5)
REG (MMX6,   mmx6)
REG (MMX7,   mmx7)
REG (MMX8,   mmx8)
REG (MMX9,   mmx9)
REG (MMX10,  mmx10)
REG (MMX11,  mmx11)
REG (MMX12,  mmx12)
REG (MMX13,  mmx13)
REG (MMX14,  mmx14)
REG (MMX15,  mmx15)

REG (XMM0,   xmm0)
REG (XMM1,   xmm1)
REG (XMM2,   xmm2)
REG (XMM3,   xmm3)
REG (XMM4,   xmm4)
REG (XMM5,   xmm5)
REG (XMM6,   xmm6)
REG (XMM7,   xmm7)
REG (XMM8,   xmm8)
REG (XMM9,   xmm9)
REG (XMM10,  xmm10)
REG (XMM11,  xmm11)
REG (XMM12,  xmm12)
REG (XMM13,  xmm13)
REG (XMM14,  xmm14)
REG (XMM15,  xmm15)

REG (YMM0,   ymm0)
REG (YMM1,   ymm1)
REG (YMM2,   ymm2)
REG (YMM3,   ymm3)
REG (YMM4,   ymm4)
REG (YMM5,   ymm5)
REG (YMM6,   ymm6)
REG (YMM7,   ymm7)
REG (YMM8,   ymm8)
REG (YMM9,   ymm9)
REG (YMM10,  ymm10)
REG (YMM11,  ymm11)
REG (YMM12,  ymm12)
REG (YMM13,  ymm13)
REG (YMM14,  ymm14)
REG (YMM15,  ymm15)

REG (ES,     es)
REG (CS,     cs)
REG (SS,     ss)
REG (DS,     ds)
REG (FS,     fs)
REG (GS,     gs)

REG (CR0,    cr0)
REG (CR1,    cr1)
REG (CR2,    cr2)
REG (CR3,    cr3)
REG (CR4,    cr4)
REG (CR5,    cr5)
REG (CR6,    cr6)
REG (CR7,    cr7)
REG (CR8,    cr8)
REG (CR9,    cr9)
REG (CR10,   cr10)
REG (CR11,   cr11)
REG (CR12,   cr12)
REG (CR13,   cr13)
REG (CR14,   cr14)
REG (CR15,   cr15)

REG (DR0,    dr0)
REG (DR1,    dr1)
REG (DR2,    dr2)
REG (DR3,    dr3)
REG (DR4,    dr4)
REG (DR5,    dr5)
REG (DR6,    dr6)
REG (DR7,    dr7)
REG (DR8,    dr8)
REG (DR9,    dr9)
REG (DR10,   dr10)
REG (DR11,   dr11)
REG (DR12,   dr12)
REG (DR13,   dr13)
REG (DR14,   dr14)
REG (DR15,   dr15)

REG (ST0,    st0)
REG (ST1,    st1)
REG (ST2,    st2)
REG (ST3,    st3)
REG (ST4,    st4)
REG (ST5,    st5)
REG (ST6,    st6)
REG (ST7,    st7)

REG (EIP,    eip)
REG (RIP,    rip)

// prefixes

PREFIX (OS,     0x66)
PREFIX (AS,     0x67)
PREFIX (ES,     0x26)
PREFIX (CS,     0x2e)
PREFIX (SS,     0x36)
PREFIX (DS,     0x3e)
PREFIX (FS,     0x64)
PREFIX (GS,     0x65)
PREFIX (LOCK,   0xf0)
PREFIX (REP,    0xf3)
PREFIX (REPNE,  0xf2)

PREFIX (REX,    0x40)
PREFIX (REXB,   0x41)
PREFIX (REXX,   0x42)
PREFIX (REXR,   0x44)
PREFIX (REXW,   0x48)

PREFIX (XOP,    0x8f)
PREFIX (VEX,    0xc4)
PREFIX (VEX2,   0xc5)

#undef CODE
#undef FLAG
#undef INSTR
#undef MNEM
#undef PREFIX
#undef REG
#undef TYPE

// ARM instruction set definitions
// Copyright (C) Florian Negele

// This file is part of the Eigen Compiler Suite.

// The ECS is free software: you can redistribute it and/or modify
// it under the terms of the GNU General Public License as published by
// the Free Software Foundation, either version 3 of the License, or
// (at your option) any later version.

// The ECS is distributed in the hope that it will be useful,
// but WITHOUT ANY WARRANTY; without even the implied warranty of
// MERCHANTABILITY or FITNESS FOR A PARTICULAR PURPOSE.  See the
// GNU General Public License for more details.

// You should have received a copy of the GNU General Public License
// along with the ECS.  If not, see <https://www.gnu.org/licenses/>.

#ifndef FLAG
	#define FLAG(flag, value)
#endif

#ifndef MNEM
	#define MNEM(name, mnem, ...)
#endif

#ifndef SUFFIX
	#define SUFFIX(name, suffix)
#endif

// mnemonics

MNEM (adc,        ADC,        Add with Carry)
MNEM (adcs,       ADCS,       Add with Carry, setting flags)
MNEM (add,        ADD,        Add)
MNEM (adds,       ADDS,       Add, setting flags)
MNEM (addw,       ADDW,       Add)
MNEM (adr,        ADR,        Form PC-relative address)
MNEM (aesd,       AESD,       AES single round decryption)
MNEM (aese,       AESE,       AES single round encryption)
MNEM (aesimc,     AESIMC,     AES inverse mix columns)
MNEM (aesmc,      AESMC,      AES mix columns)
MNEM (and,        AND,        Bitwise AND)
MNEM (ands,       ANDS,       Bitwise AND, setting flags)
MNEM (asr,        ASR,        Arithmetic Shift Right)
MNEM (asrs,       ASRS,       Arithmetic Shift Right, setting flags)
MNEM (b,          B,          Branch)
MNEM (bfc,        BFC,        Bit Field Clear)
MNEM (bfi,        BFI,        Bit Field Insert)
MNEM (bic,        BIC,        Bitwise Bit Clear)
MNEM (bics,       BICS,       Bitwise Bit Clear, setting flags)
MNEM (bkpt,       BKPT,       Breakpoint)
MNEM (bl,         BL,         Branch with Link)
MNEM (blx,        BLX,        Branch with Link and Exchange)
MNEM (bx,         BX,         Branch and Exchange)
MNEM (bxj,        BXJ,        Branch and Exchange)
MNEM (cbnz,       CBNZ,       Compare and Branch on Nonzero)
MNEM (cbz,        CBZ,        Compare and Branch on Zero)
MNEM (clrex,      CLREX,      Clear-Exclusive)
MNEM (clz,        CLZ,        Count Leading Zeros)
MNEM (cmn,        CMN,        Compare Negative)
MNEM (cmp,        CMP,        Compare)
MNEM (cps,        CPS,        Change PE State)
MNEM (cpsid,      CPSID,      Disable PE State)
MNEM (cpsie,      CPSIE,      Enable PE State)
MNEM (crc32b,     CRC32B,     Cyclic Redundancy Check)
MNEM (crc32cb,    CRC32CB,    Cyclic Redundancy Check)
MNEM (crc32ch,    CRC32CH,    Cyclic Redundancy Check)
MNEM (crc32cw,    CRC32CW,    Cyclic Redundancy Check)
MNEM (crc32h,     CRC32H,     Cyclic Redundancy Check)
MNEM (crc32w,     CRC32W,     Cyclic Redundancy Check)
MNEM (dbg,        DBG,        Debug Hint)
MNEM (dcps1,      DCPS1,      Debug Change PE State to EL1)
MNEM (dcps2,      DCPS2,      Debug Change PE State to EL2)
MNEM (dcps3,      DCPS3,      Debug Change PE State to EL3)
MNEM (dmb,        DMB,        Data Memory Barrier)
MNEM (dsb,        DSB,        Data Synchronization Barrier)
MNEM (eor,        EOR,        Bitwise Exclusive OR)
MNEM (eors,       EORS,       Bitwise Exclusive OR, setting flags)
MNEM (eret,       ERET,       Exception Return)
MNEM (esb,        ESB,        Error Synchronization Barrier)
MNEM (fldmdbx,    FLDMDBX,    Load multiple SIMD/FP registers Decrement Before)
MNEM (fldmiax,    FLDMIAX,    Load multiple SIMD/FP registers Increment After)
MNEM (fstmdbx,    FSTMDBX,    Store multiple SIMD/FP registers Decrement Before)
MNEM (fstmiax,    FSTMIAX,    Store multiple SIMD/FP registers Increment After)
MNEM (hlt,        HLT,        Halting breakpoint)
MNEM (hvc,        HVC,        Hypervisor Call)
MNEM (isb,        ISB,        Instruction Synchronization Barrier)
MNEM (it,         IT,         If-Then)
MNEM (lda,        LDA,        Load-Acquire Word)
MNEM (ldab,       LDAB,       Load-Acquire Byte)
MNEM (ldaex,      LDAEX,      Load-Acquire Exclusive Word)
MNEM (ldaexb,     LDAEXB,     Load-Acquire Exclusive Byte)
MNEM (ldaexd,     LDAEXD,     Load-Acquire Exclusive Doubleword)
MNEM (ldaexh,     LDAEXH,     Load-Acquire Exclusive Halfword)
MNEM (ldah,       LDAH,       Load-Acquire Halfword)
MNEM (ldc,        LDC,        Load data to System register)
MNEM (ldm,        LDM,        Load Multiple)
MNEM (ldmda,      LDMDA,      Load Multiple Decrement After)
MNEM (ldmdb,      LDMDB,      Load Multiple Decrement Before)
MNEM (ldmea,      LDMEA,      Load Multiple Empty Ascending)
MNEM (ldmed,      LDMED,      Load Multiple Empty Descending)
MNEM (ldmfa,      LDMFA,      Load Multiple Full Ascending)
MNEM (ldmfd,      LDMFD,      Load Multiple Full Descending)
MNEM (ldmia,      LDMIA,      Load Multiple Increment After)
MNEM (ldmib,      LDMIB,      Load Multiple Increment Before)
MNEM (ldr,        LDR,        Load Register)
MNEM (ldrb,       LDRB,       Load Register Byte)
MNEM (ldrbt,      LDRBT,      Load Register Byte Unprivileged)
MNEM (ldrd,       LDRD,       Load Register Dual)
MNEM (ldrex,      LDREX,      Load Register Exclusive)
MNEM (ldrexb,     LDREXB,     Load Register Exclusive Byte)
MNEM (ldrexd,     LDREXD,     Load Register Exclusive Doubleword)
MNEM (ldrexh,     LDREXH,     Load Register Exclusive Halfword)
MNEM (ldrh,       LDRH,       Load Register Halfword)
MNEM (ldrht,      LDRHT,      Load Register Halfword Unprivileged)
MNEM (ldrsb,      LDRSB,      Load Register Signed Byte)
MNEM (ldrsbt,     LDRSBT,     Load Register Signed Byte Unprivileged)
MNEM (ldrsh,      LDRSH,      Load Register Signed Halfword)
MNEM (ldrsht,     LDRSHT,     Load Register Signed Halfword Unprivileged)
MNEM (ldrt,       LDRT,       Load Register Unprivileged)
MNEM (lsl,        LSL,        Logical Shift Left)
MNEM (lsls,       LSLS,       Logical Shift Left, setting flags)
MNEM (lsr,        LSR,        Logical Shift Right)
MNEM (lsrs,       LSRS,       Logical Shift Right, setting flags)
MNEM (mcr,        MCR,        Move to System register from general-purpose register)
MNEM (mcrr,       MCRR,       Move to System register from two general-purpose registers)
MNEM (mla,        MLA,        Multiply Accumulate)
MNEM (mlas,       MLAS,       Multiply Accumulate, setting flags)
MNEM (mls,        MLS,        Multiply and Subtract)
MNEM (mov,        MOV,        Move)
MNEM (movs,       MOVS,       Move, setting flags)
MNEM (movt,       MOVT,       Move Top)
MNEM (movw,       MOVW,       Move)
MNEM (mrc,        MRC,        Move to general-purpose register from System register)
MNEM (mrrc,       MRRC,       Move to two general-purpose registers from System register)
MNEM (mrs,        MRS,        Move from Banked or Special register)
MNEM (msr,        MSR,        Move to Banked or Special register)
MNEM (mul,        MUL,        Multiply)
MNEM (muls,       MULS,       Multiply, setting flags)
MNEM (mvn,        MVN,        Bitwise NOT)
MNEM (mvns,       MVNS,       Bitwise NOT, setting flags)
MNEM (nop,        NOP,        No Operation)
MNEM (orn,        ORN,        Bitwise OR NOT)
MNEM (orns,       ORNS,       Bitwise OR NOT, setting flags)
MNEM (orr,        ORR,        Bitwise OR)
MNEM (orrs,       ORRS,       Bitwise OR, setting flags)
MNEM (pkhbt,      PKHBT,      Pack Halfword Bottom to Top)
MNEM (pkhtb,      PKHTB,      Pack Halfword Top to Bottom)
MNEM (pld,        PLD,        Preload Data)
MNEM (pldw,       PLDW,       Preload Data Write)
MNEM (pli,        PLI,        Preload Instruction)
MNEM (pop,        POP,        Pop Registers from Stack)
MNEM (push,       PUSH,       Push Registers to Stack)
MNEM (qadd,       QADD,       Saturating Add)
MNEM (qadd16,     QADD16,     Saturating Add 16)
MNEM (qadd8,      QADD8,      Saturating Add 8)
MNEM (qasx,       QASX,       Saturating Add and Subtract with Exchange)
MNEM (qdadd,      QDADD,      Saturating Double and Add)
MNEM (qdsub,      QDSUB,      Saturating Double and Subtract)
MNEM (qsax,       QSAX,       Saturating Subtract and Add with Exchange)
MNEM (qsub,       QSUB,       Saturating Subtract)
MNEM (qsub16,     QSUB16,     Saturating Subtract 16)
MNEM (qsub8,      QSUB8,      Saturating Subtract 8)
MNEM (rbit,       RBIT,       Reverse Bits)
MNEM (rev,        REV,        Byte-Reverse Word)
MNEM (rev16,      REV16,      Byte-Reverse Packed Halfword)
MNEM (revsh,      REVSH,      Byte-Reverse Signed Halfword)
MNEM (rfe,        RFE,        Return From Exception)
MNEM (rfeda,      RFEDA,      Return From Exception Decrement After)
MNEM (rfedb,      RFEDB,      Return From Exception Decrement Before)
MNEM (rfeia,      RFEIA,      Return From Exception Increment After)
MNEM (rfeib,      RFEIB,      Return From Exception Increment Before)
MNEM (ror,        ROR,        Rotate Right)
MNEM (rors,       RORS,       Rotate Right, setting flags)
MNEM (rrx,        RRX,        Rotate Right with Extend)
MNEM (rrxs,       RRXS,       Rotate Right with Extend, setting flags)
MNEM (rsb,        RSB,        Reverse Subtract)
MNEM (rsbs,       RSBS,       Reverse Subtract, setting flags)
MNEM (rsc,        RSC,        Reverse Subtract with Carry)
MNEM (rscs,       RSCS,       Reverse Subtract with Carry, setting flags)
MNEM (sadd16,     SADD16,     Signed Add 16)
MNEM (sadd8,      SADD8,      Signed Add 8)
MNEM (sasx,       SASX,       Signed Add and Subtract with Exchange)
MNEM (sbc,        SBC,        Subtract with Carry)
MNEM (sbcs,       SBCS,       Subtract with Carry, setting flags)
MNEM (sbfx,       SBFX,       Signed Bit Field Extract)
MNEM (sdiv,       SDIV,       Signed Divide)
MNEM (sel,        SEL,        Select Bytes)
MNEM (setend,     SETEND,     Set Endianness)
MNEM (setpan,     SETPAN,     Set Privileged Access Never)
MNEM (sev,        SEV,        Send Event)
MNEM (sevl,       SEVL,       Send Event Local)
MNEM (sha1c,      SHA1C,      SHA1 hash update (choose))
MNEM (sha1h,      SHA1H,      SHA1 fixed rotate)
MNEM (sha1m,      SHA1M,      SHA1 hash update (majority))
MNEM (sha1p,      SHA1P,      SHA1 hash update (parity))
MNEM (sha1su0,    SHA1SU0,    SHA1 schedule update 0)
MNEM (sha1su1,    SHA1SU1,    SHA1 schedule update 1)
MNEM (sha256h,    SHA256H,    SHA256 hash update part 1)
MNEM (sha256h2,   SHA256H2,   SHA256 hash update part 2)
MNEM (sha256su0,  SHA256SU0,  SHA256 schedule update 0)
MNEM (sha256su1,  SHA256SU1,  SHA256 schedule update 1)
MNEM (shadd16,    SHADD16,    Signed Halving Add 16)
MNEM (shadd8,     SHADD8,     Signed Halving Add 8)
MNEM (shasx,      SHASX,      Signed Halving Add and Subtract with Exchange)
MNEM (shsax,      SHSAX,      Signed Halving Subtract and Add with Exchange)
MNEM (shsub16,    SHSUB16,    Signed Halving Subtract 16)
MNEM (shsub8,     SHSUB8,     Signed Halving Subtract 8)
MNEM (smc,        SMC,        Secure Monitor Call)
MNEM (smlabb,     SMLABB,     Signed Multiply Accumulate Bottom by Bottom)
MNEM (smlabt,     SMLABT,     Signed Multiply Accumulate Bottom by Top)
MNEM (smlad,      SMLAD,      Signed Multiply Accumulate Dual)
MNEM (smladx,     SMLADX,     Signed Multiply Accumulate Dual with Exchange)
MNEM (smlal,      SMLAL,      Signed Multiply Accumulate Long)
MNEM (smlalbb,    SMLALBB,    Signed Multiply Accumulate Long Bottom by Bottom)
MNEM (smlalbt,    SMLALBT,    Signed Multiply Accumulate Long Bottom by Top)
MNEM (smlald,     SMLALD,     Signed Multiply Accumulate Long)
MNEM (smlaldx,    SMLALDX,    Signed Multiply Accumulate Long with Exchange)
MNEM (smlals,     SMLALS,     Signed Multiply Accumulate Long, setting flags)
MNEM (smlaltb,    SMLALTB,    Signed Multiply Accumulate Long Top by Bottom)
MNEM (smlaltt,    SMLALTT,    Signed Multiply Accumulate Long Top by Top)
MNEM (smlatb,     SMLATB,     Signed Multiply Accumulate Top by Bottom)
MNEM (smlatt,     SMLATT,     Signed Multiply Accumulate Top by Top)
MNEM (smlawb,     SMLAWB,     Signed Multiply Accumulate Word by Bottom)
MNEM (smlawt,     SMLAWT,     Signed Multiply Accumulate Word by Top)
MNEM (smlsd,      SMLSD,      Signed Multiply Subtract Dual)
MNEM (smlsdx,     SMLSDX,     Signed Multiply Subtract Dual with Exchange)
MNEM (smlsld,     SMLSLD,     Signed Multiply Subtract Long Dual)
MNEM (smlsldx,    SMLSLDX,    Signed Multiply Subtract Long Dual with Exchange)
MNEM (smmla,      SMMLA,      Signed Most Significant Word Multiply Accumulate)
MNEM (smmlar,     SMMLAR,     Signed Most Significant Word Multiply Accumulate Rounded)
MNEM (smmls,      SMMLS,      Signed Most Significant Word Multiply Subtract)
MNEM (smmlsr,     SMMLSR,     Signed Most Significant Word Multiply Subtract Rounded)
MNEM (smmul,      SMMUL,      Signed Most Significant Word Multiply)
MNEM (smmulr,     SMMULR,     Signed Most Significant Word Multiply Rounded)
MNEM (smuad,      SMUAD,      Signed Dual Multiply Add)
MNEM (smuadx,     SMUADX,     Signed Dual Multiply Add Exchange)
MNEM (smulbb,     SMULBB,     Signed Multiply Bottom by Bottom)
MNEM (smulbt,     SMULBT,     Signed Multiply Bottom by Top)
MNEM (smull,      SMULL,      Signed Multiply Long)
MNEM (smulls,     SMULLS,     Signed Multiply Long, setting flags)
MNEM (smultb,     SMULTB,     Signed Multiply Top by Bottom)
MNEM (smultt,     SMULTT,     Signed Multiply Top by Top)
MNEM (smulwb,     SMULWB,     Signed Multiply Word by Bottom)
MNEM (smulwt,     SMULWT,     Signed Multiply Word by Top)
MNEM (smusd,      SMUSD,      Signed Multiply Subtract Dual)
MNEM (smusdx,     SMUSDX,     Signed Multiply Subtract Dual with Exchange)
MNEM (srs,        SRS,        Store Return State)
MNEM (srsda,      SRSDA,      Store Return State Decrement After)
MNEM (srsdb,      SRSDB,      Store Return State Decrement Before)
MNEM (srsia,      SRSIA,      Store Return State Increment After)
MNEM (srsib,      SRSIB,      Store Return State Increment Before)
MNEM (ssat,       SSAT,       Signed Saturate)
MNEM (ssat16,     SSAT16,     Signed Saturate 16)
MNEM (ssax,       SSAX,       Signed Subtract and Add with Exchange)
MNEM (ssub16,     SSUB16,     Signed Subtract 16)
MNEM (ssub8,      SSUB8,      Signed Subtract 8)
MNEM (stc,        STC,        Store data to System register)
MNEM (stl,        STL,        Store-Release Word)
MNEM (stlb,       STLB,       Store-Release Byte)
MNEM (stlex,      STLEX,      Store-Release Exclusive Word)
MNEM (stlexb,     STLEXB,     Store-Release Exclusive Byte)
MNEM (stlexd,     STLEXD,     Store-Release Exclusive Doubleword)
MNEM (stlexh,     STLEXH,     Store-Release Exclusive Halfword)
MNEM (stlh,       STLH,       Store-Release Halfword)
MNEM (stm,        STM,        Store Multiple)
MNEM (stmda,      STMDA,      Store Multiple Decrement After)
MNEM (stmdb,      STMDB,      Store Multiple Decrement Before)
MNEM (stmea,      STMEA,      Store Multiple Empty Ascending)
MNEM (stmed,      STMED,      Store Multiple Empty Descending)
MNEM (stmfa,      STMFA,      Store Multiple Full Ascending)
MNEM (stmfd,      STMFD,      Store Multiple Full Descending)
MNEM (stmia,      STMIA,      Store Multiple Increment After)
MNEM (stmib,      STMIB,      Store Multiple Increment Before)
MNEM (str,        STR,        Store Register)
MNEM (strb,       STRB,       Store Register Byte)
MNEM (strbt,      STRBT,      Store Register Byte Unprivileged)
MNEM (strd,       STRD,       Store Register Dual)
MNEM (strex,      STREX,      Store Register Exclusive)
MNEM (strexb,     STREXB,     Store Register Exclusive Byte)
MNEM (strexd,     STREXD,     Store Register Exclusive Doubleword)
MNEM (strexh,     STREXH,     Store Register Exclusive Halfword)
MNEM (strh,       STRH,       Store Register Halfword)
MNEM (strht,      STRHT,      Store Register Halfword Unprivileged)
MNEM (strt,       STRT,       Store Register Unprivileged)
MNEM (sub,        SUB,        Subtract)
MNEM (subs,       SUBS,       Subtract, setting flags)
MNEM (subw,       SUBW,       Subtract)
MNEM (svc,        SVC,        Supervisor Call)
MNEM (sxtab,      SXTAB,      Signed Extend and Add Byte)
MNEM (sxtab16,    SXTAB16,    Signed Extend and Add Byte 16)
MNEM (sxtah,      SXTAH,      Signed Extend and Add Halfword)
MNEM (sxtb,       SXTB,       Signed Extend Byte)
MNEM (sxtb16,     SXTB16,     Signed Extend Byte 16)
MNEM (sxth,       SXTH,       Signed Extend Halfword)
MNEM (tbb,        TBB,        Table Branch Byte)
MNEM (tbh,        TBH,        Table Branch Halfword)
MNEM (teq,        TEQ,        Test Equivalence)
MNEM (tst,        TST,        Test)
MNEM (uadd16,     UADD16,     Unsigned Add 16)
MNEM (uadd8,      UADD8,      Unsigned Add 8)
MNEM (uasx,       UASX,       Unsigned Add)
MNEM (ubfx,       UBFX,       Unsigned Bit Field Extract)
MNEM (udf,        UDF,        Permanently Undefined)
MNEM (udiv,       UDIV,       Unsigned Divide)
MNEM (uhadd16,    UHADD16,    Unsigned Halving Add 16)
MNEM (uhadd8,     UHADD8,     Unsigned Halving Add 8)
MNEM (uhasx,      UHASX,      Unsigned Halving Add)
MNEM (uhsax,      UHSAX,      Unsigned Halving Subtract)
MNEM (uhsub16,    UHSUB16,    Unsigned Halving Subtract 16)
MNEM (uhsub8,     UHSUB8,     Unsigned Halving Subtract 8)
MNEM (umaal,      UMAAL,      Unsigned Multiply Accumulate Accumulate Long)
MNEM (umlal,      UMLAL,      Unsigned Multiply Accumulate Long)
MNEM (umlals,     UMLALS,     Unsigned Multiply Accumulate Long, setting flags)
MNEM (umull,      UMULL,      Unsigned Multiply Long)
MNEM (umulls,     UMULLS,     Unsigned Multiply Long, setting flags)
MNEM (uqadd16,    UQADD16,    Unsigned Saturating Add 16)
MNEM (uqadd8,     UQADD8,     Unsigned Saturating Add 8)
MNEM (uqasx,      UQASX,      Unsigned Saturating Add and Subtract with Exchange)
MNEM (uqsax,      UQSAX,      Unsigned Saturating Subtract and Add with Exchange)
MNEM (uqsub16,    UQSUB16,    Unsigned Saturating Subtract 16)
MNEM (uqsub8,     UQSUB8,     Unsigned Saturating Subtract 8)
MNEM (usad8,      USAD8,      Unsigned Sum of Absolute Differences)
MNEM (usada8,     USADA8,     Unsigned Sum of Absolute Differences and Accumulate)
MNEM (usat,       USAT,       Unsigned Saturate)
MNEM (usat16,     USAT16,     Unsigned Saturate 16)
MNEM (usax,       USAX,       Unsigned Subtract and Add with Exchange)
MNEM (usub16,     USUB16,     Unsigned Subtract 16)
MNEM (usub8,      USUB8,      Unsigned Subtract 8)
MNEM (uxtab,      UXTAB,      Unsigned Extend and Add Byte)
MNEM (uxtab16,    UXTAB16,    Unsigned Extend and Add Byte 16)
MNEM (uxtah,      UXTAH,      Unsigned Extend and Add Halfword)
MNEM (uxtb,       UXTB,       Unsigned Extend Byte)
MNEM (uxtb16,     UXTB16,     Unsigned Extend Byte 16)
MNEM (uxth,       UXTH,       Unsigned Extend Halfword)
MNEM (vaba,       VABA,       Vector Absolute Difference and Accumulate)
MNEM (vabal,      VABAL,      Vector Absolute Difference and Accumulate Long)
MNEM (vabd,       VABD,       Vector Absolute Difference)
MNEM (vabdl,      VABDL,      Vector Absolute Difference Long)
MNEM (vabs,       VABS,       Vector Absolute)
MNEM (vacge,      VACGE,      Vector Absolute Compare Greater Than or Equal)
MNEM (vacgt,      VACGT,      Vector Absolute Compare Greater Than)
MNEM (vacle,      VACLE,      Vector Absolute Compare Less Than or Equal)
MNEM (vaclt,      VACLT,      Vector Absolute Compare Less Than)
MNEM (vadd,       VADD,       Vector Add)
MNEM (vaddhn,     VADDHN,     Vector Add and Narrow, returning High Half)
MNEM (vaddl,      VADDL,      Vector Add Long)
MNEM (vaddw,      VADDW,      Vector Add Wide)
MNEM (vand,       VAND,       Vector Bitwise AND)
MNEM (vbic,       VBIC,       Vector Bitwise Bit Clear)
MNEM (vbif,       VBIF,       Vector Bitwise Insert if False)
MNEM (vbit,       VBIT,       Vector Bitwise Insert if True)
MNEM (vbsl,       VBSL,       Vector Bitwise Select)
MNEM (vcadd,      VCADD,      Vector Complex Add)
MNEM (vceq,       VCEQ,       Vector Compare Equal)
MNEM (vcge,       VCGE,       Vector Compare Greater Than or Equal)
MNEM (vcgt,       VCGT,       Vector Compare Greater Than)
MNEM (vcle,       VCLE,       Vector Compare Less Than or Equal)
MNEM (vcls,       VCLS,       Vector Count Leading Sign Bits)
MNEM (vclt,       VCLT,       Vector Compare Less Than)
MNEM (vclz,       VCLZ,       Vector Count Leading Zeros)
MNEM (vcmla,      VCMLA,      Vector Complex Multiply Accumulate)
MNEM (vcmp,       VCMP,       Vector Compare)
MNEM (vcmpe,      VCMPE,      Vector Compare, raising Invalid Operation on NaN)
MNEM (vcnt,       VCNT,       Vector Count Set Bits)
MNEM (vcvt,       VCVT,       Convert)
MNEM (vcvta,      VCVTA,      Vector Convert floating-point to integer with Round to Nearest with Ties to Away)
MNEM (vcvtb,      VCVTB,      Convert to or from a half-precision value in the bottom half of a single-precision register)
MNEM (vcvtm,      VCVTM,      Vector Convert floating-point to integer with Round towards -Infinity)
MNEM (vcvtn,      VCVTN,      Vector Convert floating-point to integer with Round to Nearest)
MNEM (vcvtp,      VCVTP,      Vector Convert floating-point to integer with Round towards +Infinity)
MNEM (vcvtr,      VCVTR,      Convert floating-point to integer)
MNEM (vcvtt,      VCVTT,      Convert to or from a half-precision value in the top half of a single-precision register)
MNEM (vdiv,       VDIV,       Divide)
MNEM (vdup,       VDUP,       Duplicate general-purpose register to vector)
MNEM (veor,       VEOR,       Vector Bitwise Exclusive OR)
MNEM (vext,       VEXT,       Vector Extract)
MNEM (vfma,       VFMA,       Vector Fused Multiply Accumulate)
MNEM (vfmal,      VFMAL,      Vector Floating-point Multiply-Add Long to accumulator)
MNEM (vfms,       VFMS,       Vector Fused Multiply Subtract)
MNEM (vfmsl,      VFMSL,      Vector Floating-point Multiply-Subtract Long from accumulator)
MNEM (vfnma,      VFNMA,      Vector Fused Negate Multiply Accumulate)
MNEM (vfnms,      VFNMS,      Fused Negate Multiply Subtract)
MNEM (vhadd,      VHADD,      Vector Halving Add)
MNEM (vhsub,      VHSUB,      Vector Halving Subtract)
MNEM (vins,       VINS,       Vector move Insertion)
MNEM (vjcvt,      VJCVT,      Javascript Convert to signed fixed-point, rounding toward Zero)
MNEM (vld1,       VLD1,       Load 1-element structure to lane of one register)
MNEM (vld2,       VLD2,       Load 2-element structure to lane of two registers)
MNEM (vld3,       VLD3,       Load 3-element structure to lane of three registers)
MNEM (vld4,       VLD4,       Load 4-element structure to lane of four registers)
MNEM (vldm,       VLDM,       Load Multiple SIMD/FP registers)
MNEM (vldmdb,     VLDMDB,     Load Multiple SIMD/FP registers Decrement Before)
MNEM (vldmia,     VLDMIA,     Load Multiple SIMD/FP registers Increment After)
MNEM (vldr,       VLDR,       Load SIMD/FP register)
MNEM (vmax,       VMAX,       Vector Maximum)
MNEM (vmaxnm,     VMAXNM,     Floating-point Maximum Number)
MNEM (vmin,       VMIN,       Vector Minimum)
MNEM (vminnm,     VMINNM,     Floating-point Minimum Number)
MNEM (vmla,       VMLA,       Vector Multiply Accumulate)
MNEM (vmlal,      VMLAL,      Vector Multiply Accumulate Long)
MNEM (vmls,       VMLS,       Vector Multiply Subtract)
MNEM (vmlsl,      VMLSL,      Vector Multiply Subtract Long)
MNEM (vmov,       VMOV,       Copy general-purpose register to or from a SIMD/FP register)
MNEM (vmovl,      VMOVL,      Vector Move Long)
MNEM (vmovn,      VMOVN,      Vector Move and Narrow)
MNEM (vmovx,      VMOVX,      Vector Move extraction)
MNEM (vmrs,       VMRS,       Move SIMD/FP Special register to general-purpose register)
MNEM (vmsr,       VMSR,       Move general-purpose register to SIMD/FP Special register)
MNEM (vmul,       VMUL,       Vector Multiply)
MNEM (vmull,      VMULL,      Vector Multiply Long)
MNEM (vmvn,       VMVN,       Vector Bitwise NOT)
MNEM (vneg,       VNEG,       Vector Negate)
MNEM (vnmla,      VNMLA,      Vector Negate Multiply Accumulate)
MNEM (vnmls,      VNMLS,      Vector Negate Multiply Subtract)
MNEM (vnmul,      VNMUL,      Vector Negate Multiply)
MNEM (vorn,       VORN,       Vector Bitwise OR NOT)
MNEM (vorr,       VORR,       Vector Bitwise OR)
MNEM (vpadal,     VPADAL,     Vector Pairwise Add and Accumulate Long)
MNEM (vpadd,      VPADD,      Vector Pairwise Add)
MNEM (vpaddl,     VPADDL,     Vector Pairwise Add Long)
MNEM (vpmax,      VPMAX,      Vector Pairwise Maximum)
MNEM (vpmin,      VPMIN,      Vector Pairwise Minimum)
MNEM (vpop,       VPOP,       Pop SIMD/FP registers from Stack)
MNEM (vpush,      VPUSH,      Push SIMD/FP registers to Stack)
MNEM (vqabs,      VQABS,      Vector Saturating Absolute)
MNEM (vqadd,      VQADD,      Vector Saturating Add)
MNEM (vqdmlal,    VQDMLAL,    Vector Saturating Doubling Multiply Accumulate Long)
MNEM (vqdmlsl,    VQDMLSL,    Vector Saturating Doubling Multiply Subtract Long)
MNEM (vqdmulh,    VQDMULH,    Vector Saturating Doubling Multiply Returning High Half)
MNEM (vqdmull,    VQDMULL,    Vector Saturating Doubling Multiply Long)
MNEM (vqmovn,     VQMOVN,     Vector Saturating Move and Narrow Signed)
MNEM (vqmovun,    VQMOVUN,    Vector Saturating Move and Narrow Unsigned)
MNEM (vqneg,      VQNEG,      Vector Saturating Negate)
MNEM (vqrdmlah,   VQRDMLAH,   Vector Saturating Rounding Doubling Multiply Accumulate Returning High Half)
MNEM (vqrdmlsh,   VQRDMLSH,   Vector Saturating Rounding Doubling Multiply Subtract Returning High Half)
MNEM (vqrdmulh,   VQRDMULH,   Vector Saturating Rounding Doubling Multiply Returning High Half)
MNEM (vqrshl,     VQRSHL,     Vector Saturating Rounding Shift Left)
MNEM (vqrshrn,    VQRSHRN,    Vector Saturating Rounding Shift Right, Narrow Signed)
MNEM (vqrshrun,   VQRSHRUN,   Vector Saturating Rounding Shift Right, Narrow Unsigned)
MNEM (vqshl,      VQSHL,      Vector Saturating Shift Left Signed)
MNEM (vqshlu,     VQSHLU,     Vector Saturating Shift Left Unsigned)
MNEM (vqshrn,     VQSHRN,     Vector Saturating Shift Right, Narrow Signed)
MNEM (vqshrun,    VQSHRUN,    Vector Saturating Shift Right, Narrow Unsigned)
MNEM (vqsub,      VQSUB,      Vector Saturating Subtract)
MNEM (vraddhn,    VRADDHN,    Vector Rounding Add and Narrow, returning High Half)
MNEM (vrecpe,     VRECPE,     Vector Reciprocal Estimate)
MNEM (vrecps,     VRECPS,     Vector Reciprocal Step)
MNEM (vrev16,     VREV16,     Vector Reverse in halfwords)
MNEM (vrev32,     VREV32,     Vector Reverse in words)
MNEM (vrev64,     VREV64,     Vector Reverse in doublewords)
MNEM (vrhadd,     VRHADD,     Vector Rounding Halving Add)
MNEM (vrinta,     VRINTA,     Vector Round floating-point to integer towards Nearest with Ties to Away)
MNEM (vrintm,     VRINTM,     Vector Round floating-point to integer towards -Infinity)
MNEM (vrintn,     VRINTN,     Vector Round floating-point to integer to Nearest)
MNEM (vrintp,     VRINTP,     Vector Round floating-point to integer towards +Infinity)
MNEM (vrintr,     VRINTR,     Round floating-point to integer)
MNEM (vrintx,     VRINTX,     Vector round floating-point to integer inexact)
MNEM (vrintz,     VRINTZ,     Vector round floating-point to integer towards Zero)
MNEM (vrshl,      VRSHL,      Vector Rounding Shift Left)
MNEM (vrshr,      VRSHR,      Vector Rounding Shift Right)
MNEM (vrshrn,     VRSHRN,     Vector Rounding Shift Right and Narrow)
MNEM (vrsqrte,    VRSQRTE,    Vector Reciprocal Square Root Estimate)
MNEM (vrsqrts,    VRSQRTS,    Vector Reciprocal Square Root Step)
MNEM (vrsra,      VRSRA,      Vector Rounding Shift Right and Accumulate)
MNEM (vrsubhn,    VRSUBHN,    Vector Rounding Subtract and Narrow, returning High Half)
MNEM (vsdot,      VSDOT,      Dot Product Signed)
MNEM (vseleq,     VSELEQ,     Floating-point conditional select EQ)
MNEM (vselge,     VSELGE,     Floating-point conditional select GE)
MNEM (vselgt,     VSELGT,     Floating-point conditional select GT)
MNEM (vselvs,     VSELVS,     Floating-point conditional select VS)
MNEM (vshl,       VSHL,       Vector Shift Left)
MNEM (vshll,      VSHLL,      Vector Shift Left Long)
MNEM (vshr,       VSHR,       Vector Shift Right)
MNEM (vshrn,      VSHRN,      Vector Shift Right Narrow)
MNEM (vsli,       VSLI,       Vector Shift Left and Insert)
MNEM (vsqrt,      VSQRT,      Square Root)
MNEM (vsra,       VSRA,       Vector Shift Right and Accumulate)
MNEM (vsri,       VSRI,       Vector Shift Right and Insert)
MNEM (vst1,       VST1,       Store 1-element structure from register to memory)
MNEM (vst2,       VST2,       Store 2-element structure from register to memory)
MNEM (vst3,       VST3,       Store 3-element structure from register to memory)
MNEM (vst4,       VST4,       Store 4-element structure from register to memory)
MNEM (vstm,       VSTM,       Store multiple SIMD/FP registers)
MNEM (vstmdb,     VSTMDB,     Store multiple SIMD/FP registers Decrement Before)
MNEM (vstmia,     VSTMIA,     Store multiple SIMD/FP registers Increment After)
MNEM (vstr,       VSTR,       Store SIMD/FP register)
MNEM (vsub,       VSUB,       Vector Subtract)
MNEM (vsubhn,     VSUBHN,     Vector Subtract and Narrow, returning High Half)
MNEM (vsubl,      VSUBL,      Vector Subtract Long)
MNEM (vsubw,      VSUBW,      Vector Subtract Wide)
MNEM (vswp,       VSWP,       Vector Swap)
MNEM (vtbl,       VTBL,       Vector Table Lookup)
MNEM (vtbx,       VTBX,       Vector Table Lookup)
MNEM (vtrn,       VTRN,       Vector Transpose)
MNEM (vtst,       VTST,       Vector Test Bits)
MNEM (vudot,      VUDOT,      Dot Product Unsigned)
MNEM (vuzp,       VUZP,       Vector Unzip)
MNEM (vzip,       VZIP,       Vector Zip)
MNEM (wfe,        WFE,        Wait For Event)
MNEM (wfi,        WFI,        Wait For Interrupt)
MNEM (yield,      YIELD,      Yield)

// suffixes

SUFFIX (.16,       SD16)
SUFFIX (.32,       SD32)
SUFFIX (.64,       SD64)
SUFFIX (.8,        SD8)
SUFFIX (.f16,      SF16)
SUFFIX (.f16.f32,  SF16F32)
SUFFIX (.f16.f64,  SF16F64)
SUFFIX (.f16.s16,  SF16S16)
SUFFIX (.f16.s32,  SF16S32)
SUFFIX (.f16.u16,  SF16U16)
SUFFIX (.f16.u32,  SF16U32)
SUFFIX (.f32,      SF32)
SUFFIX (.f32.f16,  SF32F16)
SUFFIX (.f32.f64,  SF32F64)
SUFFIX (.f32.s16,  SF32S16)
SUFFIX (.f32.s32,  SF32S32)
SUFFIX (.f32.u16,  SF32U16)
SUFFIX (.f32.u32,  SF32U32)
SUFFIX (.f64,      SF64)
SUFFIX (.f64.f16,  SF64F16)
SUFFIX (.f64.f32,  SF64F32)
SUFFIX (.f64.s16,  SF64S16)
SUFFIX (.f64.s32,  SF64S32)
SUFFIX (.f64.u16,  SF64U16)
SUFFIX (.f64.u32,  SF64U32)
SUFFIX (.i16,      SI16)
SUFFIX (.i32,      SI32)
SUFFIX (.i64,      SI64)
SUFFIX (.i8,       SI8)
SUFFIX (.p64,      SP64)
SUFFIX (.p8,       SP8)
SUFFIX (.s16,      SS16)
SUFFIX (.s16.f16,  SS16F16)
SUFFIX (.s32,      SS32)
SUFFIX (.s32.f16,  SS32F16)
SUFFIX (.s32.f32,  SS32F32)
SUFFIX (.s32.f64,  SS32F64)
SUFFIX (.s64,      SS64)
SUFFIX (.s8,       SS8)
SUFFIX (.u16,      SU16)
SUFFIX (.u16.f16,  SU16F16)
SUFFIX (.u16.f32,  SU16F32)
SUFFIX (.u16.f64,  SU16F64)
SUFFIX (.u32,      SU32)
SUFFIX (.u32.f16,  SU32F16)
SUFFIX (.u32.f32,  SU32F32)
SUFFIX (.u32.f64,  SU32F64)
SUFFIX (.u64,      SU64)
SUFFIX (.u8,       SU8)

// instruction flags

FLAG (S,  0x3f)
FLAG (C,  0x40)
FLAG (R,  0x80)
FLAG (I,  0x100)
FLAG (O,  0x200)
FLAG (Q,  0x400)
FLAG (W,  0x800)

#undef FLAG
#undef MNEM
#undef SUFFIX

% AVR architecture documentation
% Copyright (C) Florian Negele

% This file is part of the Eigen Compiler Suite.

% Permission is granted to copy, distribute and/or modify this document
% under the terms of the GNU Free Documentation License, Version 1.3
% or any later version published by the Free Software Foundation.

% You should have received a copy of the GNU Free Documentation License
% along with the ECS.  If not, see <https://www.gnu.org/licenses/>.

% Generic documentation utilities
% Copyright (C) Florian Negele

% This file is part of the Eigen Compiler Suite.

% Permission is granted to copy, distribute and/or modify this document
% under the terms of the GNU Free Documentation License, Version 1.3
% or any later version published by the Free Software Foundation.

% You should have received a copy of the GNU Free Documentation License
% along with the ECS.  If not, see <https://www.gnu.org/licenses/>.

\providecommand{\cpp}{C\texttt{++}}
\providecommand{\opt}{_\mathit{opt}}
\providecommand{\tool}[1]{\texttt{#1}}
\providecommand{\version}{Version 0.0.40}
\providecommand{\resource}[1]{*++\txt{#1}}
\providecommand{\ecs}{Eigen Compiler Suite}
\providecommand{\changed}[1]{\underline{#1}}
\providecommand{\toolbox}[1]{\converter{#1}}
\providecommand{\file}{}\renewcommand{\file}[1]{\texttt{#1}}
\providecommand{\alignright}{\hfill\linebreak[0]\hspace*{\fill}}
\providecommand{\converter}[1]{*++[F][F*:white][F,:gray]\txt{#1}}
\providecommand{\documentation}{\ifbook chapter\else document\fi}
\providecommand{\Documentation}{\ifbook Chapter\else Document\fi}
\providecommand{\variable}[1]{\resource{\texttt{\small#1}\\variable}}
\providecommand{\documentationref}[2]{\ifbook\ref{#1}\else``\href{#1}{#2}''~\cite{#1}\fi}
\providecommand{\objfile}[1]{\texttt{#1}\index[runtime]{#1 object file@\texttt{#1} object file}}
\providecommand{\libfile}[1]{\texttt{#1}\index[runtime]{#1 library file@\texttt{#1} library file}}
\providecommand{\epigraph}[2]{\ifbook\begin{quote}\flushright\textit{#1}\par--- #2\end{quote}\fi}
\providecommand{\environmentvariable}[1]{\texttt{#1}\index{Environment variables!#1@\texttt{#1}}}
\providecommand{\environment}[1]{\texttt{#1}\index[environment]{#1 environment@\texttt{#1} environment}}
\providecommand{\toolsection}{}\renewcommand{\toolsection}[1]{\subsection{#1}\label{\prefix:#1}\tool{#1}}
\providecommand{\instruction}{}\renewcommand{\instruction}[2]{\noindent\qquad\pdftooltip{\texttt{#1}}{#2}\refstepcounter{instruction}\par}
\providecommand{\flowgraph}{}\renewcommand{\flowgraph}[1]{\par\sffamily\begin{displaymath}\xymatrix@=4ex{#1}\end{displaymath}\normalfont\par}
\providecommand{\instructionset}{}\renewcommand{\instructionset}[4]{\setcounter{instruction}{0}\begin{multicols}{\ifbook#3\else#4\fi}[{\captionof{table}[#2]{#2 (\ref*{#1:instructions}~instructions)}\label{tab:#1set}\vspace{-2ex}}]\footnotesize\raggedcolumns\input{#1.set}\label{#1:instructions}\end{multicols}}

\providecommand{\gpl}{GNU General Public License}
\providecommand{\rse}{ECS Runtime Support Exception}
\providecommand{\fdl}{\href{https://www.gnu.org/licenses/fdl.html}{GNU Free Documentation License}}

\providecommand{\docbegin}{}
\providecommand{\docend}{}
\providecommand{\doclabel}[1]{\hypertarget{#1}}
\providecommand{\doclink}[2]{\hyperlink{#1}{#2}}
\providecommand{\docsection}[3]{\hypertarget{#1}{\subsection{#2}}\label{sec:#1}\index[library]{#2@#3}}
\providecommand{\docsectionstar}[1]{}
\providecommand{\docsubbegin}{\begin{description}}
\providecommand{\docsubend}{\end{description}}
\providecommand{\docsubsection}[3]{\item[\hypertarget{#1}{#2}]\index[library]{#2@#3}}
\providecommand{\docsubsectionstar}[1]{\smallskip}
\providecommand{\docsubsubsection}[3]{\docsubsection{#1}{#2}{#3}}
\providecommand{\docsubsubsectionstar}[1]{}
\providecommand{\docsubsubsubsection}[3]{}
\providecommand{\docsubsubsubsectionstar}[1]{}
\providecommand{\doctable}{}

\providecommand{\debuggingtool}{}\renewcommand{\debuggingtool}{This tool is provided for debugging purposes.
It allows exposing and modifying an internal data structure that is usually not accessible.
}

\providecommand{\interface}{All tools accept command-line arguments which are taken as names of plain text files containing the source code.
If no arguments are provided, the standard input stream is used instead.
Output files are generated in the current working directory and have the same name as the input file being processed whereas the filename extension gets replaced by an appropriate suffix.
\seeinterface
}

\providecommand{\license}{\noindent Copyright \copyright{} Florian Negele\par\medskip\noindent
Permission is granted to copy, distribute and/or modify this document under the terms of the
\fdl{}, Version 1.3 or any later version published by the \href{https://fsf.org/}{Free Software Foundation}.
}

\providecommand{\ecslogosurface}{
\fill[darkgray] (0,0,0) -- (0,0,3) -- (0,3,3) -- (0,3,1) -- (0,4,1) -- (0,4,3) -- (0,5,3) -- (0,5,0) -- (0,2,0) -- (0,2,2) -- (0,1,2) -- (0,1,0) -- cycle;
\fill[gray] (0,5,0) -- (0,5,3) -- (1,5,3) -- (1,5,1) -- (2,5,1) -- (2,5,3) -- (3,5,3) -- (3,5,0) -- cycle;
\fill[lightgray] (0,0,0) -- (0,1,0) -- (2,1,0) -- (2,4,0) -- (1,4,0) -- (1,3,0) -- (2,3,0) -- (2,2,0) -- (0,2,0) -- (0,5,0) -- (3,5,0) -- (3,0,0) -- cycle;
\begin{scope}[line width=0.5]
\begin{scope}[gray]
\draw (0,0,0) -- (0,1,0);
\draw (2,1,0) -- (2,2,0);
\draw (0,1,2) -- (0,2,2);
\draw (0,2,0) -- (0,5,0);
\draw (2,3,0) -- (2,4,0);
\end{scope}
\begin{scope}[lightgray]
\draw (0,1,0) -- (0,1,2);
\draw (0,3,1) -- (0,3,3);
\draw (0,5,0) -- (0,5,3);
\draw (2,5,1) -- (2,5,3);
\end{scope}
\begin{scope}[white]
\draw (0,1,0) -- (2,1,0);
\draw (1,3,0) -- (2,3,0);
\draw (0,5,0) -- (3,5,0);
\end{scope}
\end{scope}
}

\providecommand{\ecslogo}[1]{
\begin{tikzpicture}[scale={(#1)/((sin(45)+cos(45))*3cm)},x={({-cos(45)*1cm},{sin(45)*sin(30)*1cm})},y={({0cm},{(cos(30)*1cm})},z={({sin(45)*1cm},{cos(45)*sin(30)*1cm})}]
\begin{scope}[darkgray,line width=1]
\draw (0,0,0) -- (0,0,3) -- (0,3,3) -- (2,3,3) -- (2,5,3) -- (3,5,3) -- (3,5,0) -- (3,0,0) -- cycle;
\draw (0,3,1) -- (0,4,1) -- (0,4,3) -- (0,5,3) -- (1,5,3) -- (1,5,1) -- (2,5,1);
\draw (1,3,0) -- (1,4,0) -- (2,4,0);
\end{scope}
\fill[darkgray] (2,0,0) -- (2,0,3) -- (2,5,3) -- (2,5,1) -- (2,4,1) -- (2,4,0) -- cycle;
\fill[lightgray] (2,0,2) -- (0,0,2) -- (0,2,2) -- (2,2,2) -- cycle;
\fill[gray] (0,1,0) -- (2,1,0) -- (2,1,2) -- (0,1,2) -- cycle;
\fill[gray] (0,3,1) -- (0,3,3) -- (2,3,3) -- (2,3,0) -- (1,3,0) -- (1,3,1) -- cycle;
\ecslogosurface
\end{tikzpicture}
}

\providecommand{\shadowedecslogo}[3]{
\begin{tikzpicture}[scale={(#1)/((sin(#2)+cos(#2))*3cm)},x={({-cos(#2)*1cm},{sin(#2)*sin(#3)*1cm})},y={({0cm},{(cos(#3)*1cm})},z={({sin(#2)*1cm},{cos(#2)*sin(#3)*1cm})}]
\shade[top color=lightgray!50!white,bottom color=white,middle color=lightgray!50!white] (0,0,0) -- (3,0,0) -- (3,{-0.5-3*sin(#2)*sin(#3)/cos(#3)},0) -- (0,-0.5,0) -- cycle;
\shade[top color=darkgray!50!gray,bottom color=white,middle color=darkgray!50!white] (0,0,0) -- (0,0,3) -- (0,{-0.5-3*cos(#2)*sin(#3)/cos(#3)},3) -- (0,-0.5,0) -- cycle;
\begin{scope}[y={({(cos(#2)+sin(#2))*0.5cm},{(cos(#2)*sin(#3)-sin(#2)*sin(#3))*0.5cm})}]
\useasboundingbox (3,0,0) -- (0,0,0) -- (0,0,3);
\shade[left color=darkgray!80!black,right color=lightgray,middle color=gray] (0,0,0) -- (0,1,0) -- (0,1,0.5) -- (0,2,0) -- (0,5,0) -- (0,5,3) -- (1,5,3) -- (1,4,3) -- (1,4,2.5) -- (1,3,3) -- (2,5,3) -- (3,5,3) -- (3,0,3) -- cycle;
\clip (0,0,0) -- (0,0,3) -- ({-3*sin(#2)/cos(#2)},0,0) -- cycle;
\shade[left color=darkgray,right color=lightgray!50!gray] (0,0,0) -- (0,1,0) -- (0,1,0.5) -- (0,2,0) -- (0,5,0) -- (0,5,3) -- (1,5,3) -- (1,4,3) -- (1,4,2.5) -- (1,3,3) -- (2,5,3) -- (3,5,3) -- (3,0,3) -- cycle;
\end{scope}
\shade[left color=darkgray,right color=darkgray!80!black] (2,0,0) -- (2,0,3) -- (2,5,3) -- (2,5,1) -- (2,4,1) -- (2,4,0) -- cycle;
\shade[left color=darkgray!90!black,right color=gray!80!darkgray] (2,0,2) -- (0,0,2) -- (0,2,2) -- (2,2,2) -- cycle;
\shade[top color=darkgray!90!black,bottom color=gray!80!darkgray] (0,1,0) -- (2,1,0) -- (2,1,2) -- (0,1,2) -- cycle;
\shade[top color=darkgray!90!black,bottom color=gray!80!darkgray] (0,3,1) -- (0,3,3) -- (2,3,3) -- (2,3,0) -- (1,3,0) -- (1,3,1) -- cycle;
\fill[gray] (2,1,0) -- (1.5,1,0.5) -- (0,1,0.5) -- (0,1,0) -- cycle;
\fill[gray] (1,3,2) -- (0.5,3,2) -- (0.5,3,3) -- (1,3,3) -- cycle;
\fill[gray] (2,3,0) -- (1.5,3,0.5) -- (1,3,0.5) -- (1,3,0) -- cycle;
\ecslogosurface
\end{tikzpicture}
}

\providecommand{\cpplogo}[1]{
\begin{tikzpicture}[scale=(#1)/512em]
\fill[gray] (435.2794,398.7159) -- (247.1911,507.3075) .. controls (236.3563,513.5642) and (218.6240,513.5642) .. (207.7892,507.3075) -- (19.7009,398.7159) .. controls (8.8646,392.4606) and (0.0000,377.1043) .. (0.0000,364.5924) -- (0.0000,147.4076) .. controls (0.8430,132.8363) and (8.2856,120.7683) .. (19.7009,113.2842) -- (207.7892,4.6926) .. controls (218.6240,-1.5642) and (236.3564,-1.5642) .. (247.1911,4.6926) -- (435.2794,113.2842) .. controls (447.5273,121.4304) and (454.4987,133.6918) .. (454.9803,147.4076) -- (454.9803,364.5924) .. controls (454.5404,377.7571) and (446.6566,391.0351) .. (435.2794,398.7159) -- cycle(75.8301,255.9993) .. controls (74.9389,404.0881) and (273.2892,469.4783) .. (358.8263,331.8769) -- (293.1917,293.8965) .. controls (253.5702,359.4301) and (155.1909,335.9977) .. (151.6601,255.9993) .. controls (152.7204,182.2703) and (249.4137,148.0211) .. (293.1961,218.1065) -- (358.8308,180.1276) .. controls (283.4477,49.2645) and (79.6318,96.3470) .. (75.8301,255.9993) -- cycle(379.1503,247.5747) -- (362.2982,247.5747) -- (362.2982,230.7226) -- (345.4490,230.7226) -- (345.4490,247.5747) -- (328.5969,247.5747) -- (328.5969,264.4254) -- (345.4490,264.4254) -- (345.4490,281.2759) -- (362.2982,281.2759) -- (362.2982,264.4254) -- (379.1503,264.4254) -- cycle(442.3420,247.5747) -- (425.4899,247.5747) -- (425.4899,230.7226) -- (408.6408,230.7226) -- (408.6408,247.5747) -- (391.7886,247.5747) -- (391.7886,264.4254) -- (408.6408,264.4254) -- (408.6408,281.2759) -- (425.4899,281.2759) -- (425.4899,264.4254) -- (442.3420,264.4254) -- cycle;
\end{tikzpicture}
}

\providecommand{\fallogo}[1]{
\begin{tikzpicture}[scale=(#1)/512em]
\fill[gray] (185.7774,0.0000) .. controls (200.4486,15.9798) and (226.8966,8.7148) .. (235.0426,31.5836) .. controls (249.5297,58.0598) and (247.9581,97.9161) .. (280.3335,110.9762) .. controls (309.1690,120.3496) and (337.8406,104.2727) .. (366.5753,103.9379) .. controls (373.4449,111.5171) and (379.2885,128.2574) .. (383.9755,108.9744) .. controls (396.6979,102.5615) and (437.2808,107.6681) .. (426.9652,124.3252) .. controls (408.9822,121.0785) and (412.4742,146.0729) .. (426.5192,131.4996) .. controls (433.8413,120.8489) and (465.1541,126.5522) .. (441.9067,135.7950) .. controls (396.1879,157.7478) and (344.1112,161.5079) .. (298.5528,183.5702) .. controls (277.7471,193.5198) and (284.6941,218.7163) .. (285.2127,236.9640) .. controls (292.3599,316.2826) and (307.3929,394.6311) .. (317.1198,473.6154) .. controls (329.0637,505.4736) and (292.1195,528.5004) .. (265.9183,511.2761) .. controls (237.9284,499.2462) and (237.3684,465.2681) .. (230.9102,439.9421) .. controls (218.6692,374.3397) and (215.6307,306.9662) .. (198.1732,242.3977) .. controls (183.1379,232.7444) and (164.4245,256.0298) .. (149.0430,261.4799) .. controls (116.9328,279.2585) and (87.1822,308.5851) .. (48.2293,307.8914) .. controls (21.3220,306.9037) and (-15.9107,281.8761) .. (7.2921,252.7908) .. controls (29.7799,220.6177) and (67.5177,204.3028) .. (100.9287,185.9449) .. controls (130.8217,170.8906) and (161.1548,156.5903) .. (191.0278,141.5847) .. controls (196.1738,120.0520) and (186.6049,95.2409) .. (186.8382,72.4353) .. controls (185.5234,48.4204) and (183.1700,23.9341) .. (185.7774,0.0000) -- cycle;
\end{tikzpicture}
}

\providecommand{\oblogo}[1]{
\begin{tikzpicture}[scale=(#1)/512em]
\fill[gray] (160.3865,208.9117) .. controls (154.0879,214.6478) and (149.0735,221.2409) .. (145.4125,228.5384) .. controls (184.8790,248.4273) and (234.7122,269.8787) .. (297.5493,291.8782) .. controls (300.3943,281.4769) and (300.9552,268.7619) .. (300.4023,255.2389) .. controls (248.9909,244.7891) and (200.0310,225.9279) .. (160.3865,208.9117) -- cycle(225.7398,392.6996) .. controls (308.0209,392.1716) and (359.3326,345.9277) .. (368.7203,285.2098) .. controls (376.6742,197.1784) and (311.7194,141.3342) .. (205.4287,142.1456) .. controls (139.9485,141.4804) and (88.7155,166.1957) .. (73.5775,228.0086) .. controls (52.0297,320.3408) and (123.4078,391.0103) .. (225.7398,392.6996) -- cycle(216.0739,176.4733) .. controls (268.9183,179.2424) and (315.8292,206.5488) .. (312.7454,265.1139) .. controls (313.2769,315.6384) and (286.5993,353.4946) .. (216.6040,355.7934) .. controls (162.4657,355.7934) and (126.0914,317.5023) .. (126.0914,260.5103) .. controls (126.1733,214.2900) and (163.3363,176.2849) .. (216.0739,176.4733) -- cycle(76.4897,189.1754) .. controls (13.1586,147.5631) and (0.0000,119.4207) .. (0.0000,119.4207) -- (90.6499,170.1632) .. controls (85.3004,175.8497) and (80.5994,182.1633) .. (76.4897,189.1754) -- cycle(353.9486,119.3004) -- (402.9482,119.3004) .. controls (427.0025,137.0797) and (450.9893,162.7034) .. (474.9529,191.0213) .. controls (509.3540,228.5339) and (531.3391,294.2091) .. (487.8149,312.1206) .. controls (462.8165,324.7652) and (394.3874,316.8943) .. (373.8912,313.6651) .. controls (379.9291,297.7449) and (383.2899,278.4204) .. (381.4989,257.7214) .. controls (420.3069,248.0321) and (421.9610,218.3461) .. (407.7867,192.6417) .. controls (391.1113,162.4018) and (370.1114,132.9097) .. (353.9486,119.3004) -- cycle;
\end{tikzpicture}
}

\providecommand{\markuptable}{
\begin{table}
\sffamily\centering
\begin{tabular}{@{}lcl@{}}
\toprule
\texttt{//italics//} & $\rightarrow$ & \textit{italics} \\
\midrule
\texttt{**bold**} & $\rightarrow$ & \textbf{bold} \\
\midrule
\texttt{\# ordered list} & & 1 ordered list \\
\texttt{\# second item} & $\rightarrow$ & 2 second item \\
\texttt{\#\# sub item} & & \hspace{1em} 1 sub item \\
\midrule
\texttt{* unordered list} & & $\bullet$ unordered list \\
\texttt{* second item} & $\rightarrow$ & $\bullet$ second item \\
\texttt{** sub item} & & \hspace{1em} $\bullet$ sub item \\
\midrule
\texttt{link to [[label]]} & $\rightarrow$ & link to \underline{label} \\
\midrule
\texttt{<{}<label>{}> definition } & $\rightarrow$ & definition \\
\midrule
\texttt{[[url|link name]]} & $\rightarrow$ & \underline{link name} \\
\midrule\addlinespace
\texttt{= large heading} & & {\Large large heading} \smallskip \\
\texttt{== medium heading} & $\rightarrow$ & {\large medium heading} \\
\texttt{=== small heading} & & small heading \\
\midrule
\texttt{no line break} & & no line break for paragraphs \\
\texttt{for paragraphs} & $\rightarrow$ \\
& & use empty line \\
\texttt{use empty line} \\
\midrule
\texttt{force\textbackslash\textbackslash line break} & $\rightarrow$ & force \\
& & line break \\
\midrule
\texttt{horizontal line} & $\rightarrow$ & horizontal line \\
\texttt{----} & & \hrulefill \\
\midrule
\texttt{|=a|=table|=header} & & \underline{a \enspace table \enspace header} \\
\texttt{|a|table|row} & $\rightarrow$ & a \enspace table \enspace row \\
\texttt{|b|table|row} & & b \enspace table \enspace row \\
\midrule
\texttt{\{\{\{} \\
\texttt{unformatted} & $\rightarrow$ & \texttt{unformatted} \\
\texttt{code} & & \texttt{code} \\
\texttt{\}\}\}} \\
\midrule\addlinespace
\texttt{@ new article} & & {\Large 1.\ new article} \smallskip \\
\texttt{@ second article} & $\rightarrow$ & {\Large 2.\ second article} \smallskip \\
\texttt{@@ sub article} & & {\large 2.1.\ sub article} \\
\bottomrule
\end{tabular}
\normalfont\caption{Elements of the generic documentation markup language}
\label{tab:docmarkup}
\end{table}
}

\providecommand{\startchapter}[4]{
\documentclass[11pt,a4paper]{article}
\usepackage{booktabs}
\usepackage[format=hang,labelfont=bf]{caption}
\usepackage{changepage}
\usepackage[T1]{fontenc}
\usepackage[margin=2cm]{geometry}
\usepackage{hyperref}
\usepackage[american]{isodate}
\usepackage{lmodern}
\usepackage{longtable}
\usepackage{mathptmx}
\usepackage{microtype}
\usepackage[toc]{multitoc}
\usepackage{multirow}
\usepackage[all]{nowidow}
\usepackage{pdfcomment}
\usepackage{syntax}
\usepackage{tikz}
\usepackage[all]{xy}
\hypersetup{pdfborder={0 0 0},bookmarksnumbered=true,pdftitle={\ecs{}: #2},pdfauthor={Florian Negele},pdfsubject={\ecs{}},pdfkeywords={#1}}
\setlength{\grammarindent}{8em}\setlength{\grammarparsep}{0.2ex}
\setlength{\columnsep}{2em}
\newcommand{\prefix}{}
\newcounter{instruction}
\bibliographystyle{unsrt}
\renewcommand{\index}[2][]{}
\renewcommand{\arraystretch}{1.05}
\renewcommand{\floatpagefraction}{0.7}
\renewcommand{\syntleft}{\itshape}\renewcommand{\syntright}{}
\title{\vspace{-5ex}\Huge{\ecs{}}\medskip\hrule}
\author{\huge{#2}}
\date{\medskip\version}
\newif\ifbook\bookfalse
\pagestyle{headings}
\frenchspacing
\begin{document}
\maketitle\thispagestyle{empty}\noindent#4\setlength{\columnseprule}{0.4pt}\tableofcontents\setlength{\columnseprule}{0pt}\vfill\pagebreak[3]\null\vfill\bigskip\noindent
\parbox{\textwidth-4em}{\license The contents of this \documentation{} are part of the \href{manual}{\ecs{} User Manual}~\cite{manual} and correspond to Chapter ``\href{manual\##3}{#1}''.\alignright\mbox{\today}}
\parbox{4em}{\flushright\ecslogo{3em}}
\clearpage
}

\providecommand{\concludechapter}{
\vfill\pagebreak[3]\null\vfill
\thispagestyle{myheadings}\markright{REFERENCES}
\noindent\begin{minipage}{\textwidth}\begin{multicols}{2}[\section*{References}]
\renewcommand{\section}[2]{}\small\bibliography{references}
\end{multicols}\end{minipage}\end{document}
}

\providecommand{\startpresentation}[2]{
\documentclass[14pt,aspectratio=43,usepdftitle=false]{beamer}
\usepackage{booktabs}
\usepackage{etex}
\usepackage{multicol}
\usepackage{tikz}
\usepackage[all]{xy}
\bibliographystyle{unsrt}
\setlength{\columnsep}{1em}
\setlength{\leftmargini}{1em}
\setbeamercolor{title}{fg=black}
\setbeamercolor{structure}{fg=darkgray}
\setbeamercolor{bibliography item}{fg=darkgray}
\setbeamerfont{title}{series=\bfseries}
\setbeamerfont{subtitle}{series=\normalfont}
\setbeamerfont*{frametitle}{parent=title}
\setbeamerfont{block title}{series=\bfseries}
\setbeamerfont*{framesubtitle}{parent=subtitle}
\setbeamersize{text margin left=1em,text margin right=1em}
\setbeamertemplate{navigation symbols}{}
\setbeamertemplate{itemize item}[circle]{}
\setbeamertemplate{bibliography item}[triangle]{}
\setbeamertemplate{bibliography entry author}{\usebeamercolor[fg]{bibliography item}}
\setbeamertemplate{frametitle}{\medskip\usebeamerfont{frametitle}\color{gray}\raisebox{-2.5ex}[0ex][0ex]{\rule{0.1em}{4.5ex}}}
\addtobeamertemplate{frametitle}{}{\hspace{0.4em}\usebeamercolor[fg]{title}\insertframetitle\par\vspace{0.2ex}\hspace{0.5em}\usebeamerfont{framesubtitle}\insertframesubtitle}
\hypersetup{pdfborder={0 0 0},bookmarksnumbered=true,bookmarksopen=true,bookmarksopenlevel=0,pdftitle={\ecs{}: #1},pdfauthor={Florian Negele},pdfsubject={\ecs{}},pdfkeywords={#1}}
\renewcommand{\flowgraph}[1]{\resizebox{\textwidth}{!}{$$\xymatrix{##1}$$}}
\title{\ecs{}\medskip\hrule\medskip}
\institute{\shadowedecslogo{5em}{30}{15}}
\date{\version}
\subtitle{#1}
\begin{document}
\begin{frame}[plain]\titlepage\nocite{manual}\end{frame}
\begin{frame}{Contents}{#1}\begin{center}\tableofcontents\end{center}\end{frame}
}

\providecommand{\concludepresentation}{
\begin{frame}{References}\begin{footnotesize}\setlength{\columnseprule}{0.4pt}\begin{multicols}{2}\bibliography{references}\end{multicols}\end{footnotesize}\end{frame}
\end{document}
}

\providecommand{\startbook}[1]{
\documentclass[10pt,paper=17cm:24cm,DIV=13,twoside=semi,headings=normal,numbers=noendperiod,cleardoublepage=plain]{scrbook}
\usepackage{atveryend}
\usepackage{booktabs}
\usepackage{caption}
\usepackage{changepage}
\usepackage[T1]{fontenc}
\usepackage{imakeidx}
\usepackage{hyperref}
\usepackage[american]{isodate}
\usepackage{lmodern}
\usepackage{longtable}
\usepackage{mathptmx}
\usepackage[final]{microtype}
\usepackage{multicol}
\usepackage{multirow}
\usepackage[all]{nowidow}
\usepackage{pdfcomment}
\usepackage{scrlayer-scrpage}
\usepackage{setspace}
\usepackage{syntax}
\usepackage[eventxtindent=4pt,oddtxtexdent=4pt]{thumbs}
\usepackage{tikz}
\usepackage[all]{xy}
\hyphenation{Micro-Blaze Open-Cores Open-RISC Power-PC}
\hypersetup{pdfborder={0 0 0},bookmarksnumbered=true,bookmarksopen=true,bookmarksopenlevel=0,pdftitle={\ecs{}: #1},pdfauthor={Florian Negele},pdfsubject={\ecs{}},pdfkeywords={#1}}
\setlength{\grammarindent}{8em}\setlength{\grammarparsep}{0.7ex}
\setkomafont{captionlabel}{\usekomafont{descriptionlabel}}
\renewcommand{\arraystretch}{1.05}\setstretch{1.1}
\renewcommand{\chapterformat}{\thechapter\autodot\enskip\raisebox{-1ex}[0ex][0ex]{\color{gray}\rule{0.1em}{3.5ex}}\enskip}
\renewcommand{\startchapter}[4]{\hypertarget{##3}{\chapter{##1}}\label{##3}##4\addthumb{##1}{\LARGE\sffamily\bfseries\thechapter}{white}{gray}\renewcommand{\prefix}{##3}}
\renewcommand{\concludechapter}{\clearpage{\stopthumb\cleardoublepage}}
\renewcommand{\syntleft}{\itshape}\renewcommand{\syntright}{}
\renewcommand{\floatpagefraction}{0.7}
\renewcommand{\partheademptypage}{}
\DeclareMicrotypeAlias{lmss}{cmr}
\newcommand{\prefix}{}
\newcounter{instruction}
\bibliographystyle{unsrt}
\newif\ifbook\booktrue
\makeindex[intoc,title=Index]
\makeindex[intoc,name=tools,title=Index of Tools,columns=3]
\makeindex[intoc,name=library,title=Index of Library Names]
\makeindex[intoc,name=runtime,title=Index of Runtime Support]
\makeindex[intoc,name=environment,title=Index of Target Environments]
\indexsetup{toclevel=chapter,headers={\indexname}{\indexname}}
\frenchspacing
\begin{document}
\pagenumbering{alph}
\begin{titlepage}\centering
\huge\sffamily\null\vfill\textbf{\ecs{}}\bigskip\hrule\bigskip#1
\normalsize\normalfont\vfill\vfill\shadowedecslogo{10em}{30}{15}
\large\vfill\vfill\version
\end{titlepage}
\null\vfill
\thispagestyle{empty}
\noindent\today\par\medskip
\license A copy of this license is included in Appendix~\ref{fdl} on page~\pageref{fdl}.
All product names used herein are for identification purposes only and may be trademarks of their respective companies.
\concludechapter
\frontmatter
\setcounter{tocdepth}{1}
\tableofcontents
\setcounter{tocdepth}{2}
\concludechapter
\listoffigures
\concludechapter
\listoftables
\concludechapter
}

\providecommand{\concludebook}{
\backmatter
\addtocontents{toc}{\protect\setcounter{tocdepth}{-1}}
\phantomsection\addcontentsline{toc}{part}{Bibliography}
\bibliography{references}
\concludechapter
\phantomsection\addcontentsline{toc}{part}{Indexes}
\printindex
\concludechapter
\indexprologue{\label{idx:tools}}
\printindex[tools]
\concludechapter
\printindex[library]
\concludechapter
\indexprologue{\label{idx:runtime}}
\printindex[runtime]
\concludechapter
\indexprologue{\label{idx:environment}}
\printindex[environment]
\concludechapter
\pagestyle{empty}\pagenumbering{Alph}\null\clearpage
\null\vfill\centering\ecslogo{4em}\par\medskip\license
\end{document}
}

% chapter references

\providecommand{\seedocumentationref}{}\renewcommand{\seedocumentationref}[3]{#1, see \Documentation{}~\documentationref{#2}{#3}. }
\providecommand{\seeinterface}{}\renewcommand{\seeinterface}{\ifbook See \Documentation{}~\documentationref{interface}{User Interface} for more information about the common user interface of all of these tools. \fi}
\providecommand{\seeguide}{}\renewcommand{\seeguide}{\seedocumentationref{For basic examples of using some of these tools in practice}{guide}{User Guide}}
\providecommand{\seecpp}{}\renewcommand{\seecpp}{\seedocumentationref{For more information about the \cpp{} programming language and its implementation by the \ecs{}}{cpp}{User Manual for \cpp{}}}
\providecommand{\seefalse}{}\renewcommand{\seefalse}{\seedocumentationref{For more information about the FALSE programming language and its implementation by the \ecs{}}{false}{User Manual for FALSE}}
\providecommand{\seeoberon}{}\renewcommand{\seeoberon}{\seedocumentationref{For more information about the Oberon programming language and its implementation by the \ecs{}}{oberon}{User Manual for Oberon}}
\providecommand{\seeassembly}{}\renewcommand{\seeassembly}{\seedocumentationref{For more information about the generic assembly language and how to use it}{assembly}{Generic Assembly Language Specification}}
\providecommand{\seeamd}{}\renewcommand{\seeamd}{\seedocumentationref{For more information about how the \ecs{} supports the AMD64 hardware architecture}{amd64}{AMD64 Hardware Architecture Support}}
\providecommand{\seearm}{}\renewcommand{\seearm}{\seedocumentationref{For more information about how the \ecs{} supports the ARM hardware architecture}{arm}{ARM Hardware Architecture Support}}
\providecommand{\seeavr}{}\renewcommand{\seeavr}{\seedocumentationref{For more information about how the \ecs{} supports the AVR hardware architecture}{avr}{AVR Hardware Architecture Support}}
\providecommand{\seeavrtt}{}\renewcommand{\seeavrtt}{\seedocumentationref{For more information about how the \ecs{} supports the AVR32 hardware architecture}{avr32}{AVR32 Hardware Architecture Support}}
\providecommand{\seemabk}{}\renewcommand{\seemabk}{\seedocumentationref{For more information about how the \ecs{} supports the M68000 hardware architecture}{m68k}{M68000 Hardware Architecture Support}}
\providecommand{\seemibl}{}\renewcommand{\seemibl}{\seedocumentationref{For more information about how the \ecs{} supports the MicroBlaze hardware architecture}{mibl}{MicroBlaze Hardware Architecture Support}}
\providecommand{\seemips}{}\renewcommand{\seemips}{\seedocumentationref{For more information about how the \ecs{} supports the MIPS32 and MIPS64 hardware architectures}{mips}{MIPS Hardware Architecture Support}}
\providecommand{\seemmix}{}\renewcommand{\seemmix}{\seedocumentationref{For more information about how the \ecs{} supports the MMIX hardware architecture}{mmix}{MMIX Hardware Architecture Support}}
\providecommand{\seeorok}{}\renewcommand{\seeorok}{\seedocumentationref{For more information about how the \ecs{} supports the OpenRISC 1000 hardware architecture}{or1k}{OpenRISC 1000 Hardware Architecture Support}}
\providecommand{\seeppc}{}\renewcommand{\seeppc}{\seedocumentationref{For more information about how the \ecs{} supports the PowerPC hardware architecture}{ppc}{PowerPC Hardware Architecture Support}}
\providecommand{\seerisc}{}\renewcommand{\seerisc}{\seedocumentationref{For more information about how the \ecs{} supports the RISC hardware architecture}{risc}{RISC Hardware Architecture Support}}
\providecommand{\seewasm}{}\renewcommand{\seewasm}{\seedocumentationref{For more information about how the \ecs{} supports the WebAssembly architecture}{wasm}{WebAssembly Architecture Support}}
\providecommand{\seedocumentation}{}\renewcommand{\seedocumentation}{\seedocumentationref{For more information about generic documentations and their generation by the \ecs{}}{documentation}{Generic Documentation Generation}}
\providecommand{\seedebugging}{}\renewcommand{\seedebugging}{\seedocumentationref{For more information about debugging information and its representation}{debugging}{Debugging Information Representation}}
\providecommand{\seecode}{}\renewcommand{\seecode}{\seedocumentationref{For more information about intermediate code and its purpose}{code}{Intermediate Code Representation}}
\providecommand{\seeobject}{}\renewcommand{\seeobject}{\seedocumentationref{For more information about object files and their purpose}{object}{Object File Representation}}

% generic documentation tools

\providecommand{\docprint}{
\toolsection{docprint} is a pretty printer for generic documentations.
It reformats generic documentations and writes it to the standard output stream.
\debuggingtool
\flowgraph{\resource{generic\\documentation} \ar[r] & \toolbox{docprint} \ar[r] & \resource{generic\\documentation}}
\seedocumentation
}

\providecommand{\doccheck}{
\toolsection{doccheck} is a syntactic and semantic checker for generic documentations.
It just performs syntactic and semantic checks on generic documentations and writes its diagnostic messages to the standard error stream.
\debuggingtool
\flowgraph{\resource{generic\\documentation} \ar[r] & \toolbox{doccheck} \ar[r] & \resource{diagnostic\\messages}}
\seedocumentation
}

\providecommand{\dochtml}{
\toolsection{dochtml} is an HTML documentation generator for generic documentations.
It processes several generic documentations and assembles all information therein into an HTML document.
\debuggingtool
\flowgraph{\resource{generic\\documentation} \ar[r] & \toolbox{dochtml} \ar[r] & \resource{HTML\\document}}
\seedocumentation
}

\providecommand{\doclatex}{
\toolsection{doclatex} is a Latex documentation generator for generic documentations.
It processes several generic documentations and assembles all information therein into a Latex document.
\debuggingtool
\flowgraph{\resource{generic\\documentation} \ar[r] & \toolbox{doclatex} \ar[r] & \resource{Latex\\document}}
\seedocumentation
}

% intermediate code tools

\providecommand{\cdcheck}{
\toolsection{cdcheck} is a syntactic and semantic checker for intermediate code.
It just performs syntactic and semantic checks on programs written in intermediate code and writes its diagnostic messages to the standard error stream.
\debuggingtool
\flowgraph{\resource{intermediate\\code} \ar[r] & \toolbox{cdcheck} \ar[r] & \resource{diagnostic\\messages}}
\seeassembly\seecode
}

\providecommand{\cdopt}{
\toolsection{cdopt} is an optimizer for intermediate code.
It performs various optimizations on programs written in intermediate code and writes the result to the standard output stream.
\debuggingtool
\flowgraph{\resource{intermediate\\code} \ar[r] & \toolbox{cdopt} \ar[r] & \resource{optimized\\code}}
\seeassembly\seecode
}

\providecommand{\cdrun}{
\toolsection{cdrun} is an interpreter for intermediate code.
It processes and executes programs written in intermediate code.
The following code sections are predefined and have the usual semantics:
\texttt{abort}, \texttt{\_Exit}, \texttt{fflush}, \texttt{floor}, \texttt{fputc}, \texttt{free}, \texttt{getchar}, \texttt{malloc}, and \texttt{putchar}.
Diagnostic messages about invalid operations include the name of the executed code section and the index of the erroneous instruction.
\debuggingtool
\flowgraph{\resource{intermediate\\code} \ar[r] & \toolbox{cdrun} \ar@/u/[r] & \resource{input/\\output} \ar@/d/[l]}
\seeassembly\seecode
}

\providecommand{\cdamda}{
\toolsection{cdamd16} is a compiler for intermediate code targeting the AMD64 hardware architecture.
It generates machine code for AMD64 processors from programs written in intermediate code and stores it in corresponding object files.
The compiler generates machine code for the 16-bit operating mode defined by the AMD64 architecture.
It also creates a debugging information file as well as an assembly file containing a listing of the generated machine code.
\debuggingtool
\flowgraph{\resource{intermediate\\code} \ar[r] & \toolbox{cdamd16} \ar[r] \ar[d] \ar[rd] & \resource{object file} \\ & \resource{assembly\\listing} & \resource{debugging\\information}}
\seeassembly\seeamd\seeobject\seecode\seedebugging
}

\providecommand{\cdamdb}{
\toolsection{cdamd32} is a compiler for intermediate code targeting the AMD64 hardware architecture.
It generates machine code for AMD64 processors from programs written in intermediate code and stores it in corresponding object files.
The compiler generates machine code for the 32-bit operating mode defined by the AMD64 architecture.
It also creates a debugging information file as well as an assembly file containing a listing of the generated machine code.
\debuggingtool
\flowgraph{\resource{intermediate\\code} \ar[r] & \toolbox{cdamd32} \ar[r] \ar[d] \ar[rd] & \resource{object file} \\ & \resource{assembly\\listing} & \resource{debugging\\information}}
\seeassembly\seeamd\seeobject\seecode\seedebugging
}

\providecommand{\cdamdc}{
\toolsection{cdamd64} is a compiler for intermediate code targeting the AMD64 hardware architecture.
It generates machine code for AMD64 processors from programs written in intermediate code and stores it in corresponding object files.
The compiler generates machine code for the 64-bit operating mode defined by the AMD64 architecture.
It also creates a debugging information file as well as an assembly file containing a listing of the generated machine code.
\debuggingtool
\flowgraph{\resource{intermediate\\code} \ar[r] & \toolbox{cdamd64} \ar[r] \ar[d] \ar[rd] & \resource{object file} \\ & \resource{assembly\\listing} & \resource{debugging\\information}}
\seeassembly\seeamd\seeobject\seecode\seedebugging
}

\providecommand{\cdarma}{
\toolsection{cdarma32} is a compiler for intermediate code targeting the ARM hardware architecture.
It generates machine code for ARM processors executing A32 instructions from programs written in intermediate code and stores it in corresponding object files.
It also creates a debugging information file as well as an assembly file containing a listing of the generated machine code.
\debuggingtool
\flowgraph{\resource{intermediate\\code} \ar[r] & \toolbox{cdarma32} \ar[r] \ar[d] \ar[rd] & \resource{object file} \\ & \resource{assembly\\listing} & \resource{debugging\\information}}
\seeassembly\seearm\seeobject\seecode\seedebugging
}

\providecommand{\cdarmb}{
\toolsection{cdarma64} is a compiler for intermediate code targeting the ARM hardware architecture.
It generates machine code for ARM processors executing A64 instructions from programs written in intermediate code and stores it in corresponding object files.
It also creates a debugging information file as well as an assembly file containing a listing of the generated machine code.
\debuggingtool
\flowgraph{\resource{intermediate\\code} \ar[r] & \toolbox{cdarma64} \ar[r] \ar[d] \ar[rd] & \resource{object file} \\ & \resource{assembly\\listing} & \resource{debugging\\information}}
\seeassembly\seearm\seeobject\seecode\seedebugging
}

\providecommand{\cdarmc}{
\toolsection{cdarmt32} is a compiler for intermediate code targeting the ARM hardware architecture.
It generates machine code for ARM processors without floating-point extension executing T32 instructions from programs written in intermediate code and stores it in corresponding object files.
It also creates a debugging information file as well as an assembly file containing a listing of the generated machine code.
\debuggingtool
\flowgraph{\resource{intermediate\\code} \ar[r] & \toolbox{cdarmt32} \ar[r] \ar[d] \ar[rd] & \resource{object file} \\ & \resource{assembly\\listing} & \resource{debugging\\information}}
\seeassembly\seearm\seeobject\seecode\seedebugging
}

\providecommand{\cdarmcfpe}{
\toolsection{cdarmt32fpe} is a compiler for intermediate code targeting the ARM hardware architecture.
It generates machine code for ARM processors with floating-point extension executing T32 instructions from programs written in intermediate code and stores it in corresponding object files.
It also creates a debugging information file as well as an assembly file containing a listing of the generated machine code.
\debuggingtool
\flowgraph{\resource{intermediate\\code} \ar[r] & \toolbox{cdarmt32fpe} \ar[r] \ar[d] \ar[rd] & \resource{object file} \\ & \resource{assembly\\listing} & \resource{debugging\\information}}
\seeassembly\seearm\seeobject\seecode\seedebugging
}

\providecommand{\cdavr}{
\toolsection{cdavr} is a compiler for intermediate code targeting the AVR hardware architecture.
It generates machine code for AVR processors from programs written in intermediate code and stores it in corresponding object files.
It also creates a debugging information file as well as an assembly file containing a listing of the generated machine code.
\debuggingtool
\flowgraph{\resource{intermediate\\code} \ar[r] & \toolbox{cdavr} \ar[r] \ar[d] \ar[rd] & \resource{object file} \\ & \resource{assembly\\listing} & \resource{debugging\\information}}
\seeassembly\seeavr\seeobject\seecode\seedebugging
}

\providecommand{\cdavrtt}{
\toolsection{cdavr32} is a compiler for intermediate code targeting the AVR32 hardware architecture.
It generates machine code for AVR32 processors from programs written in intermediate code and stores it in corresponding object files.
It also creates a debugging information file as well as an assembly file containing a listing of the generated machine code.
\debuggingtool
\flowgraph{\resource{intermediate\\code} \ar[r] & \toolbox{cdavr32} \ar[r] \ar[d] \ar[rd] & \resource{object file} \\ & \resource{assembly\\listing} & \resource{debugging\\information}}
\seeassembly\seeavrtt\seeobject\seecode\seedebugging
}

\providecommand{\cdmabk}{
\toolsection{cdm68k} is a compiler for intermediate code targeting the M68000 hardware architecture.
It generates machine code for M68000 processors from programs written in intermediate code and stores it in corresponding object files.
It also creates a debugging information file as well as an assembly file containing a listing of the generated machine code.
\debuggingtool
\flowgraph{\resource{intermediate\\code} \ar[r] & \toolbox{cdm68k} \ar[r] \ar[d] \ar[rd] & \resource{object file} \\ & \resource{assembly\\listing} & \resource{debugging\\information}}
\seeassembly\seemabk\seeobject\seecode\seedebugging
}

\providecommand{\cdmibl}{
\toolsection{cdmibl} is a compiler for intermediate code targeting the MicroBlaze hardware architecture.
It generates machine code for MicroBlaze processors from programs written in intermediate code and stores it in corresponding object files.
It also creates a debugging information file as well as an assembly file containing a listing of the generated machine code.
\debuggingtool
\flowgraph{\resource{intermediate\\code} \ar[r] & \toolbox{cdmibl} \ar[r] \ar[d] \ar[rd] & \resource{object file} \\ & \resource{assembly\\listing} & \resource{debugging\\information}}
\seeassembly\seemibl\seeobject\seecode\seedebugging
}

\providecommand{\cdmipsa}{
\toolsection{cdmips32} is a compiler for intermediate code targeting the MIPS32 hardware architecture.
It generates machine code for MIPS32 processors from programs written in intermediate code and stores it in corresponding object files.
It also creates a debugging information file as well as an assembly file containing a listing of the generated machine code.
\debuggingtool
\flowgraph{\resource{intermediate\\code} \ar[r] & \toolbox{cdmips32} \ar[r] \ar[d] \ar[rd] & \resource{object file} \\ & \resource{assembly\\listing} & \resource{debugging\\information}}
\seeassembly\seemips\seeobject\seecode\seedebugging
}

\providecommand{\cdmipsb}{
\toolsection{cdmips64} is a compiler for intermediate code targeting the MIPS64 hardware architecture.
It generates machine code for MIPS64 processors from programs written in intermediate code and stores it in corresponding object files.
It also creates a debugging information file as well as an assembly file containing a listing of the generated machine code.
\debuggingtool
\flowgraph{\resource{intermediate\\code} \ar[r] & \toolbox{cdmips64} \ar[r] \ar[d] \ar[rd] & \resource{object file} \\ & \resource{assembly\\listing} & \resource{debugging\\information}}
\seeassembly\seemips\seeobject\seecode\seedebugging
}

\providecommand{\cdmmix}{
\toolsection{cdmmix} is a compiler for intermediate code targeting the MMIX hardware architecture.
It generates machine code for MMIX processors from programs written in intermediate code and stores it in corresponding object files.
It also creates a debugging information file as well as an assembly file containing a listing of the generated machine code.
\debuggingtool
\flowgraph{\resource{intermediate\\code} \ar[r] & \toolbox{cdmmix} \ar[r] \ar[d] \ar[rd] & \resource{object file} \\ & \resource{assembly\\listing} & \resource{debugging\\information}}
\seeassembly\seemmix\seeobject\seecode\seedebugging
}

\providecommand{\cdorok}{
\toolsection{cdor1k} is a compiler for intermediate code targeting the OpenRISC 1000 hardware architecture.
It generates machine code for OpenRISC 1000 processors from programs written in intermediate code and stores it in corresponding object files.
It also creates a debugging information file as well as an assembly file containing a listing of the generated machine code.
\debuggingtool
\flowgraph{\resource{intermediate\\code} \ar[r] & \toolbox{cdor1k} \ar[r] \ar[d] \ar[rd] & \resource{object file} \\ & \resource{assembly\\listing} & \resource{debugging\\information}}
\seeassembly\seeorok\seeobject\seecode\seedebugging
}

\providecommand{\cdppca}{
\toolsection{cdppc32} is a compiler for intermediate code targeting the PowerPC hardware architecture.
It generates machine code for PowerPC processors from programs written in intermediate code and stores it in corresponding object files.
The compiler generates machine code for the 32-bit operating mode defined by the PowerPC architecture.
It also creates a debugging information file as well as an assembly file containing a listing of the generated machine code.
\debuggingtool
\flowgraph{\resource{intermediate\\code} \ar[r] & \toolbox{cdppc32} \ar[r] \ar[d] \ar[rd] & \resource{object file} \\ & \resource{assembly\\listing} & \resource{debugging\\information}}
\seeassembly\seeppc\seeobject\seecode\seedebugging
}

\providecommand{\cdppcb}{
\toolsection{cdppc64} is a compiler for intermediate code targeting the PowerPC hardware architecture.
It generates machine code for PowerPC processors from programs written in intermediate code and stores it in corresponding object files.
The compiler generates machine code for the 64-bit operating mode defined by the PowerPC architecture.
It also creates a debugging information file as well as an assembly file containing a listing of the generated machine code.
\debuggingtool
\flowgraph{\resource{intermediate\\code} \ar[r] & \toolbox{cdppc64} \ar[r] \ar[d] \ar[rd] & \resource{object file} \\ & \resource{assembly\\listing} & \resource{debugging\\information}}
\seeassembly\seeppc\seeobject\seecode\seedebugging
}

\providecommand{\cdrisc}{
\toolsection{cdrisc} is a compiler for intermediate code targeting the RISC hardware architecture.
It generates machine code for RISC processors from programs written in intermediate code and stores it in corresponding object files.
It also creates a debugging information file as well as an assembly file containing a listing of the generated machine code.
\debuggingtool
\flowgraph{\resource{intermediate\\code} \ar[r] & \toolbox{cdrisc} \ar[r] \ar[d] \ar[rd] & \resource{object file} \\ & \resource{assembly\\listing} & \resource{debugging\\information}}
\seeassembly\seerisc\seeobject\seecode\seedebugging
}

\providecommand{\cdwasm}{
\toolsection{cdwasm} is a compiler for intermediate code targeting the WebAssembly architecture.
It generates machine code for WebAssembly targets from programs written in intermediate code and stores it in corresponding object files.
It also creates a debugging information file as well as an assembly file containing a listing of the generated machine code.
\debuggingtool
\flowgraph{\resource{intermediate\\code} \ar[r] & \toolbox{cdwasm} \ar[r] \ar[d] \ar[rd] & \resource{object file} \\ & \resource{assembly\\listing} & \resource{debugging\\information}}
\seeassembly\seewasm\seeobject\seecode\seedebugging
}

% C++ tools

\providecommand{\cppprep}{
\toolsection{cppprep} is a preprocessor for the \cpp{} programming language.
It preprocesses source code according to the rules of \cpp{} and writes it to the standard output stream.
Only the macro names \texttt{\_\_DATE\_\_}, \texttt{\_\_FILE\_\_}, \texttt{\_\_LINE\_\_}, and \texttt{\_\_TIME\_\_} are predefined.
\flowgraph{\resource{\cpp{} or other\\source code} \ar[r] & \toolbox{cppprep} \ar[r] & \resource{preprocessed\\source code} \\ & \variable{ECSINCLUDE} \ar[u]}
\seecpp
}

\providecommand{\cppprint}{
\toolsection{cppprint} is a pretty printer for the \cpp{} programming language.
It reformats the source code of \cpp{} programs and writes it to the standard output stream.
\flowgraph{\resource{\cpp{}\\source code} \ar[r] & \toolbox{cppprint} \ar[r] & \resource{reformatted\\source code} \\ & \variable{ECSINCLUDE} \ar[u]}
\seecpp
}

\providecommand{\cppcheck}{
\toolsection{cppcheck} is a syntactic and semantic checker for the \cpp{} programming language.
It just performs syntactic and semantic checks on \cpp{} programs and writes its diagnostic messages to the standard error stream.
\flowgraph{\resource{\cpp{}\\source code} \ar[r] & \toolbox{cppcheck} \ar[r] & \resource{diagnostic\\messages} \\ & \variable{ECSINCLUDE} \ar[u]}
\seecpp
}

\providecommand{\cppdump}{
\toolsection{cppdump} is a serializer for the \cpp{} programming language.
It dumps the complete internal representation of programs written in \cpp{} into an XML document.
\debuggingtool
\flowgraph{\resource{\cpp{}\\source code} \ar[r] & \toolbox{cppdump} \ar[r] & \resource{internal\\representation} \\ & \variable{ECSINCLUDE} \ar[u]}
\seecpp
}

\providecommand{\cpprun}{
\toolsection{cpprun} is an interpreter for the \cpp{} programming language.
It processes and executes programs written in \cpp{}.
The macro \texttt{\_\_run\_\_} is predefined in order to enable programmers to identify this tool while interpreting.
\flowgraph{\resource{\cpp{}\\source code} \ar[r] & \toolbox{cpprun} \ar@/u/[r] & \resource{input/\\output} \ar@/d/[l] \\ & \variable{ECSINCLUDE} \ar[u]}
\seecpp
}

\providecommand{\cppdoc}{
\toolsection{cppdoc} is a generic documentation generator for the \cpp{} programming language.
It processes several \cpp{} source files and assembles all information therein into a generic documentation.
\debuggingtool
\flowgraph{\resource{\cpp{}\\source code} \ar[r] & \toolbox{cppdoc} \ar[r] & \resource{generic\\documentation} \\ & \variable{ECSINCLUDE} \ar[u]}
\seecpp\seedocumentation
}

\providecommand{\cpphtml}{
\toolsection{cpphtml} is an HTML documentation generator for the \cpp{} programming language.
It processes several \cpp{} source files and assembles all information therein into an HTML document.
\flowgraph{\resource{\cpp{}\\source code} \ar[r] & \toolbox{cpphtml} \ar[r] & \resource{HTML\\document} \\ & \variable{ECSINCLUDE} \ar[u]}
\seecpp\seedocumentation
}

\providecommand{\cpplatex}{
\toolsection{cpplatex} is a Latex documentation generator for the \cpp{} programming language.
It processes several \cpp{} source files and assembles all information therein into a Latex document.
\flowgraph{\resource{\cpp{}\\source code} \ar[r] & \toolbox{cpplatex} \ar[r] & \resource{Latex\\document} \\ & \variable{ECSINCLUDE} \ar[u]}
\seecpp\seedocumentation
}

\providecommand{\cppcode}{
\toolsection{cppcode} is an intermediate code generator for the \cpp{} programming language.
It generates intermediate code from programs written in \cpp{} and stores it in corresponding assembly files.
The macro \texttt{\_\_code\_\_} is predefined in order to enable programmers to identify this tool while generating intermediate code.
Programs generated with this tool require additional runtime support that is stored in the \file{cpp\-code\-run} library file.
\debuggingtool
\flowgraph{\resource{\cpp{}\\source code} \ar[r] & \toolbox{cppcode} \ar[r] & \resource{intermediate\\code} \\ & \variable{ECSINCLUDE} \ar[u]}
\seecpp\seeassembly\seecode
}

\providecommand{\cppamda}{
\toolsection{cppamd16} is a compiler for the \cpp{} programming language targeting the AMD64 hardware architecture.
It generates machine code for AMD64 processors from programs written in \cpp{} and stores it in corresponding object files.
The compiler generates machine code for the 16-bit operating mode defined by the AMD64 architecture.
For debugging purposes, it also creates a debugging information file as well as an assembly file containing a listing of the generated machine code.
The macro \texttt{\_\_amd16\_\_} is predefined in order to enable programmers to identify this tool and its target architecture while compiling.
Programs generated with this compiler require additional runtime support that is stored in the \file{cpp\-amd16\-run} library file.
\flowgraph{\resource{\cpp{}\\source code} \ar[r] & \toolbox{cppamd16} \ar[r] \ar[d] \ar[rd] & \resource{object file} \\ \variable{ECSINCLUDE} \ar[ru] & \resource{debugging\\information} & \resource{assembly\\listing}}
\seecpp\seeassembly\seeamd\seeobject\seedebugging
}

\providecommand{\cppamdb}{
\toolsection{cppamd32} is a compiler for the \cpp{} programming language targeting the AMD64 hardware architecture.
It generates machine code for AMD64 processors from programs written in \cpp{} and stores it in corresponding object files.
The compiler generates machine code for the 32-bit operating mode defined by the AMD64 architecture.
For debugging purposes, it also creates a debugging information file as well as an assembly file containing a listing of the generated machine code.
The macro \texttt{\_\_amd32\_\_} is predefined in order to enable programmers to identify this tool and its target architecture while compiling.
Programs generated with this compiler require additional runtime support that is stored in the \file{cpp\-amd32\-run} library file.
\flowgraph{\resource{\cpp{}\\source code} \ar[r] & \toolbox{cppamd32} \ar[r] \ar[d] \ar[rd] & \resource{object file} \\ \variable{ECSINCLUDE} \ar[ru] & \resource{debugging\\information} & \resource{assembly\\listing}}
\seecpp\seeassembly\seeamd\seeobject\seedebugging
}

\providecommand{\cppamdc}{
\toolsection{cppamd64} is a compiler for the \cpp{} programming language targeting the AMD64 hardware architecture.
It generates machine code for AMD64 processors from programs written in \cpp{} and stores it in corresponding object files.
The compiler generates machine code for the 64-bit operating mode defined by the AMD64 architecture.
For debugging purposes, it also creates a debugging information file as well as an assembly file containing a listing of the generated machine code.
The macro \texttt{\_\_amd64\_\_} is predefined in order to enable programmers to identify this tool and its target architecture while compiling.
Programs generated with this compiler require additional runtime support that is stored in the \file{cpp\-amd64\-run} library file.
\flowgraph{\resource{\cpp{}\\source code} \ar[r] & \toolbox{cppamd64} \ar[r] \ar[d] \ar[rd] & \resource{object file} \\ \variable{ECSINCLUDE} \ar[ru] & \resource{debugging\\information} & \resource{assembly\\listing}}
\seecpp\seeassembly\seeamd\seeobject\seedebugging
}

\providecommand{\cpparma}{
\toolsection{cpparma32} is a compiler for the \cpp{} programming language targeting the ARM hardware architecture.
It generates machine code for ARM processors executing A32 instructions from programs written in \cpp{} and stores it in corresponding object files.
For debugging purposes, it also creates a debugging information file as well as an assembly file containing a listing of the generated machine code.
The macro \texttt{\_\_arma32\_\_} is predefined in order to enable programmers to identify this tool and its target architecture while compiling.
Programs generated with this compiler require additional runtime support that is stored in the \file{cpp\-arma32\-run} library file.
\flowgraph{\resource{\cpp{}\\source code} \ar[r] & \toolbox{cpparma32} \ar[r] \ar[d] \ar[rd] & \resource{object file} \\ \variable{ECSINCLUDE} \ar[ru] & \resource{debugging\\information} & \resource{assembly\\listing}}
\seecpp\seeassembly\seearm\seeobject\seedebugging
}

\providecommand{\cpparmb}{
\toolsection{cpparma64} is a compiler for the \cpp{} programming language targeting the ARM hardware architecture.
It generates machine code for ARM processors executing A64 instructions from programs written in \cpp{} and stores it in corresponding object files.
For debugging purposes, it also creates a debugging information file as well as an assembly file containing a listing of the generated machine code.
The macro \texttt{\_\_arma64\_\_} is predefined in order to enable programmers to identify this tool and its target architecture while compiling.
Programs generated with this compiler require additional runtime support that is stored in the \file{cpp\-arma64\-run} library file.
\flowgraph{\resource{\cpp{}\\source code} \ar[r] & \toolbox{cpparma64} \ar[r] \ar[d] \ar[rd] & \resource{object file} \\ \variable{ECSINCLUDE} \ar[ru] & \resource{debugging\\information} & \resource{assembly\\listing}}
\seecpp\seeassembly\seearm\seeobject\seedebugging
}

\providecommand{\cpparmc}{
\toolsection{cpparmt32} is a compiler for the \cpp{} programming language targeting the ARM hardware architecture.
It generates machine code for ARM processors without floating-point extension executing T32 instructions from programs written in \cpp{} and stores it in corresponding object files.
For debugging purposes, it also creates a debugging information file as well as an assembly file containing a listing of the generated machine code.
The macro \texttt{\_\_armt32\_\_} is predefined in order to enable programmers to identify this tool and its target architecture while compiling.
Programs generated with this compiler require additional runtime support that is stored in the \file{cpp\-armt32\-run} library file.
\flowgraph{\resource{\cpp{}\\source code} \ar[r] & \toolbox{cpparmt32} \ar[r] \ar[d] \ar[rd] & \resource{object file} \\ \variable{ECSINCLUDE} \ar[ru] & \resource{debugging\\information} & \resource{assembly\\listing}}
\seecpp\seeassembly\seearm\seeobject\seedebugging
}

\providecommand{\cpparmcfpe}{
\toolsection{cpparmt32fpe} is a compiler for the \cpp{} programming language targeting the ARM hardware architecture.
It generates machine code for ARM processors with floating-point extension executing T32 instructions from programs written in \cpp{} and stores it in corresponding object files.
For debugging purposes, it also creates a debugging information file as well as an assembly file containing a listing of the generated machine code.
The macro \texttt{\_\_armt32fpe\_\_} is predefined in order to enable programmers to identify this tool and its target architecture while compiling.
Programs generated with this compiler require additional runtime support that is stored in the \file{cpp\-armt32\-fpe\-run} library file.
\flowgraph{\resource{\cpp{}\\source code} \ar[r] & \toolbox{cpparmt32fpe} \ar[r] \ar[d] \ar[rd] & \resource{object file} \\ \variable{ECSINCLUDE} \ar[ru] & \resource{debugging\\information} & \resource{assembly\\listing}}
\seecpp\seeassembly\seearm\seeobject\seedebugging
}

\providecommand{\cppavr}{
\toolsection{cppavr} is a compiler for the \cpp{} programming language targeting the AVR hardware architecture.
It generates machine code for AVR processors from programs written in \cpp{} and stores it in corresponding object files.
For debugging purposes, it also creates a debugging information file as well as an assembly file containing a listing of the generated machine code.
The macro \texttt{\_\_avr\_\_} is predefined in order to enable programmers to identify this tool and its target architecture while compiling.
Programs generated with this compiler require additional runtime support that is stored in the \file{cpp\-avr\-run} library file.
\flowgraph{\resource{\cpp{}\\source code} \ar[r] & \toolbox{cppavr} \ar[r] \ar[d] \ar[rd] & \resource{object file} \\ \variable{ECSINCLUDE} \ar[ru] & \resource{debugging\\information} & \resource{assembly\\listing}}
\seecpp\seeassembly\seeavr\seeobject\seedebugging
}

\providecommand{\cppavrtt}{
\toolsection{cppavr32} is a compiler for the \cpp{} programming language targeting the AVR32 hardware architecture.
It generates machine code for AVR32 processors from programs written in \cpp{} and stores it in corresponding object files.
For debugging purposes, it also creates a debugging information file as well as an assembly file containing a listing of the generated machine code.
The macro \texttt{\_\_avr32\_\_} is predefined in order to enable programmers to identify this tool and its target architecture while compiling.
Programs generated with this compiler require additional runtime support that is stored in the \file{cpp\-avr32\-run} library file.
\flowgraph{\resource{\cpp{}\\source code} \ar[r] & \toolbox{cppavr32} \ar[r] \ar[d] \ar[rd] & \resource{object file} \\ \variable{ECSINCLUDE} \ar[ru] & \resource{debugging\\information} & \resource{assembly\\listing}}
\seecpp\seeassembly\seeavrtt\seeobject\seedebugging
}

\providecommand{\cppmabk}{
\toolsection{cppm68k} is a compiler for the \cpp{} programming language targeting the M68000 hardware architecture.
It generates machine code for M68000 processors from programs written in \cpp{} and stores it in corresponding object files.
For debugging purposes, it also creates a debugging information file as well as an assembly file containing a listing of the generated machine code.
The macro \texttt{\_\_m68k\_\_} is predefined in order to enable programmers to identify this tool and its target architecture while compiling.
Programs generated with this compiler require additional runtime support that is stored in the \file{cpp\-m68k\-run} library file.
\flowgraph{\resource{\cpp{}\\source code} \ar[r] & \toolbox{cppm68k} \ar[r] \ar[d] \ar[rd] & \resource{object file} \\ \variable{ECSINCLUDE} \ar[ru] & \resource{debugging\\information} & \resource{assembly\\listing}}
\seecpp\seeassembly\seemabk\seeobject\seedebugging
}

\providecommand{\cppmibl}{
\toolsection{cppmibl} is a compiler for the \cpp{} programming language targeting the MicroBlaze hardware architecture.
It generates machine code for MicroBlaze processors from programs written in \cpp{} and stores it in corresponding object files.
For debugging purposes, it also creates a debugging information file as well as an assembly file containing a listing of the generated machine code.
The macro \texttt{\_\_mibl\_\_} is predefined in order to enable programmers to identify this tool and its target architecture while compiling.
Programs generated with this compiler require additional runtime support that is stored in the \file{cpp\-mibl\-run} library file.
\flowgraph{\resource{\cpp{}\\source code} \ar[r] & \toolbox{cppmibl} \ar[r] \ar[d] \ar[rd] & \resource{object file} \\ \variable{ECSINCLUDE} \ar[ru] & \resource{debugging\\information} & \resource{assembly\\listing}}
\seecpp\seeassembly\seemibl\seeobject\seedebugging
}

\providecommand{\cppmipsa}{
\toolsection{cppmips32} is a compiler for the \cpp{} programming language targeting the MIPS32 hardware architecture.
It generates machine code for MIPS32 processors from programs written in \cpp{} and stores it in corresponding object files.
For debugging purposes, it also creates a debugging information file as well as an assembly file containing a listing of the generated machine code.
The macro \texttt{\_\_mips32\_\_} is predefined in order to enable programmers to identify this tool and its target architecture while compiling.
Programs generated with this compiler require additional runtime support that is stored in the \file{cpp\-mips32\-run} library file.
\flowgraph{\resource{\cpp{}\\source code} \ar[r] & \toolbox{cppmips32} \ar[r] \ar[d] \ar[rd] & \resource{object file} \\ \variable{ECSINCLUDE} \ar[ru] & \resource{debugging\\information} & \resource{assembly\\listing}}
\seecpp\seeassembly\seemips\seeobject\seedebugging
}

\providecommand{\cppmipsb}{
\toolsection{cppmips64} is a compiler for the \cpp{} programming language targeting the MIPS64 hardware architecture.
It generates machine code for MIPS64 processors from programs written in \cpp{} and stores it in corresponding object files.
For debugging purposes, it also creates a debugging information file as well as an assembly file containing a listing of the generated machine code.
The macro \texttt{\_\_mips64\_\_} is predefined in order to enable programmers to identify this tool and its target architecture while compiling.
Programs generated with this compiler require additional runtime support that is stored in the \file{cpp\-mips64\-run} library file.
\flowgraph{\resource{\cpp{}\\source code} \ar[r] & \toolbox{cppmips64} \ar[r] \ar[d] \ar[rd] & \resource{object file} \\ \variable{ECSINCLUDE} \ar[ru] & \resource{debugging\\information} & \resource{assembly\\listing}}
\seecpp\seeassembly\seemips\seeobject\seedebugging
}

\providecommand{\cppmmix}{
\toolsection{cppmmix} is a compiler for the \cpp{} programming language targeting the MMIX hardware architecture.
It generates machine code for MMIX processors from programs written in \cpp{} and stores it in corresponding object files.
For debugging purposes, it also creates a debugging information file as well as an assembly file containing a listing of the generated machine code.
The macro \texttt{\_\_mmix\_\_} is predefined in order to enable programmers to identify this tool and its target architecture while compiling.
Programs generated with this compiler require additional runtime support that is stored in the \file{cpp\-mmix\-run} library file.
\flowgraph{\resource{\cpp{}\\source code} \ar[r] & \toolbox{cppmmix} \ar[r] \ar[d] \ar[rd] & \resource{object file} \\ \variable{ECSINCLUDE} \ar[ru] & \resource{debugging\\information} & \resource{assembly\\listing}}
\seecpp\seeassembly\seemmix\seeobject\seedebugging
}

\providecommand{\cpporok}{
\toolsection{cppor1k} is a compiler for the \cpp{} programming language targeting the OpenRISC 1000 hardware architecture.
It generates machine code for OpenRISC 1000 processors from programs written in \cpp{} and stores it in corresponding object files.
For debugging purposes, it also creates a debugging information file as well as an assembly file containing a listing of the generated machine code.
The macro \texttt{\_\_or1k\_\_} is predefined in order to enable programmers to identify this tool and its target architecture while compiling.
Programs generated with this compiler require additional runtime support that is stored in the \file{cpp\-or1k\-run} library file.
\flowgraph{\resource{\cpp{}\\source code} \ar[r] & \toolbox{cppor1k} \ar[r] \ar[d] \ar[rd] & \resource{object file} \\ \variable{ECSINCLUDE} \ar[ru] & \resource{debugging\\information} & \resource{assembly\\listing}}
\seecpp\seeassembly\seeorok\seeobject\seedebugging
}

\providecommand{\cppppca}{
\toolsection{cppppc32} is a compiler for the \cpp{} programming language targeting the PowerPC hardware architecture.
It generates machine code for PowerPC processors from programs written in \cpp{} and stores it in corresponding object files.
The compiler generates machine code for the 32-bit operating mode defined by the PowerPC architecture.
For debugging purposes, it also creates a debugging information file as well as an assembly file containing a listing of the generated machine code.
The macro \texttt{\_\_ppc32\_\_} is predefined in order to enable programmers to identify this tool and its target architecture while compiling.
Programs generated with this compiler require additional runtime support that is stored in the \file{cpp\-ppc32\-run} library file.
\flowgraph{\resource{\cpp{}\\source code} \ar[r] & \toolbox{cppppc32} \ar[r] \ar[d] \ar[rd] & \resource{object file} \\ \variable{ECSINCLUDE} \ar[ru] & \resource{debugging\\information} & \resource{assembly\\listing}}
\seecpp\seeassembly\seeppc\seeobject\seedebugging
}

\providecommand{\cppppcb}{
\toolsection{cppppc64} is a compiler for the \cpp{} programming language targeting the PowerPC hardware architecture.
It generates machine code for PowerPC processors from programs written in \cpp{} and stores it in corresponding object files.
The compiler generates machine code for the 64-bit operating mode defined by the PowerPC architecture.
For debugging purposes, it also creates a debugging information file as well as an assembly file containing a listing of the generated machine code.
The macro \texttt{\_\_ppc64\_\_} is predefined in order to enable programmers to identify this tool and its target architecture while compiling.
Programs generated with this compiler require additional runtime support that is stored in the \file{cpp\-ppc64\-run} library file.
\flowgraph{\resource{\cpp{}\\source code} \ar[r] & \toolbox{cppppc64} \ar[r] \ar[d] \ar[rd] & \resource{object file} \\ \variable{ECSINCLUDE} \ar[ru] & \resource{debugging\\information} & \resource{assembly\\listing}}
\seecpp\seeassembly\seeppc\seeobject\seedebugging
}

\providecommand{\cpprisc}{
\toolsection{cpprisc} is a compiler for the \cpp{} programming language targeting the RISC hardware architecture.
It generates machine code for RISC processors from programs written in \cpp{} and stores it in corresponding object files.
For debugging purposes, it also creates a debugging information file as well as an assembly file containing a listing of the generated machine code.
The macro \texttt{\_\_risc\_\_} is predefined in order to enable programmers to identify this tool and its target architecture while compiling.
Programs generated with this compiler require additional runtime support that is stored in the \file{cpp\-risc\-run} library file.
\flowgraph{\resource{\cpp{}\\source code} \ar[r] & \toolbox{cpprisc} \ar[r] \ar[d] \ar[rd] & \resource{object file} \\ \variable{ECSINCLUDE} \ar[ru] & \resource{debugging\\information} & \resource{assembly\\listing}}
\seecpp\seeassembly\seerisc\seeobject\seedebugging
}

\providecommand{\cppwasm}{
\toolsection{cppwasm} is a compiler for the \cpp{} programming language targeting the WebAssembly architecture.
It generates machine code for WebAssembly targets from programs written in \cpp{} and stores it in corresponding object files.
For debugging purposes, it also creates a debugging information file as well as an assembly file containing a listing of the generated machine code.
The macro \texttt{\_\_wasm\_\_} is predefined in order to enable programmers to identify this tool and its target architecture while compiling.
Programs generated with this compiler require additional runtime support that is stored in the \file{cpp\-wasm\-run} library file.
\flowgraph{\resource{\cpp{}\\source code} \ar[r] & \toolbox{cppwasm} \ar[r] \ar[d] \ar[rd] & \resource{object file} \\ \variable{ECSINCLUDE} \ar[ru] & \resource{debugging\\information} & \resource{assembly\\listing}}
\seecpp\seeassembly\seewasm\seeobject\seedebugging
}

% FALSE tools

\providecommand{\falprint}{
\toolsection{falprint} is a pretty printer for the FALSE programming language.
It reformats the source code of FALSE programs and writes it to the standard output stream.
\flowgraph{\resource{FALSE\\source code} \ar[r] & \toolbox{falprint} \ar[r] & \resource{reformatted\\source code}}
\seefalse
}

\providecommand{\falcheck}{
\toolsection{falcheck} is a syntactic and semantic checker for the FALSE programming language.
It just performs syntactic and semantic checks on FALSE programs and writes its diagnostic messages to the standard error stream.
\flowgraph{\resource{FALSE\\source code} \ar[r] & \toolbox{falcheck} \ar[r] & \resource{diagnostic\\messages}}
\seefalse
}

\providecommand{\faldump}{
\toolsection{faldump} is a serializer for the FALSE programming language.
It dumps the complete internal representation of programs written in FALSE into an XML document.
\debuggingtool
\flowgraph{\resource{FALSE\\source code} \ar[r] & \toolbox{faldump} \ar[r] & \resource{internal\\representation}}
\seefalse
}

\providecommand{\falrun}{
\toolsection{falrun} is an interpreter for the FALSE programming language.
It processes and executes programs written in FALSE\@.
\flowgraph{\resource{FALSE\\source code} \ar[r] & \toolbox{falrun} \ar@/u/[r] & \resource{input/\\output} \ar@/d/[l]}
\seefalse
}

\providecommand{\falcpp}{
\toolsection{falcpp} is a transpiler for the FALSE programming language.
It translates programs written in FALSE into \cpp{} programs and stores them in corresponding source files.
\flowgraph{\resource{FALSE\\source code} \ar[r] & \toolbox{falcpp} \ar[r] & \resource{\cpp{}\\source file}}
\seefalse\seecpp
}

\providecommand{\falcode}{
\toolsection{falcode} is an intermediate code generator for the FALSE programming language.
It generates intermediate code from programs written in FALSE and stores it in corresponding assembly files.
\debuggingtool
\flowgraph{\resource{FALSE\\source code} \ar[r] & \toolbox{falcode} \ar[r] & \resource{intermediate\\code}}
\seefalse\seeassembly\seecode
}

\providecommand{\falamda}{
\toolsection{falamd16} is a compiler for the FALSE programming language targeting the AMD64 hardware architecture.
It generates machine code for AMD64 processors from programs written in FALSE and stores it in corresponding object files.
The compiler generates machine code for the 16-bit operating mode defined by the AMD64 architecture.
\flowgraph{\resource{FALSE\\source code} \ar[r] & \toolbox{falamd16} \ar[r] & \resource{object file}}
\seefalse\seeamd\seeobject
}

\providecommand{\falamdb}{
\toolsection{falamd32} is a compiler for the FALSE programming language targeting the AMD64 hardware architecture.
It generates machine code for AMD64 processors from programs written in FALSE and stores it in corresponding object files.
The compiler generates machine code for the 32-bit operating mode defined by the AMD64 architecture.
\flowgraph{\resource{FALSE\\source code} \ar[r] & \toolbox{falamd32} \ar[r] & \resource{object file}}
\seefalse\seeamd\seeobject
}

\providecommand{\falamdc}{
\toolsection{falamd64} is a compiler for the FALSE programming language targeting the AMD64 hardware architecture.
It generates machine code for AMD64 processors from programs written in FALSE and stores it in corresponding object files.
The compiler generates machine code for the 64-bit operating mode defined by the AMD64 architecture.
\flowgraph{\resource{FALSE\\source code} \ar[r] & \toolbox{falamd64} \ar[r] & \resource{object file}}
\seefalse\seeamd\seeobject
}

\providecommand{\falarma}{
\toolsection{falarma32} is a compiler for the FALSE programming language targeting the ARM hardware architecture.
It generates machine code for ARM processors executing A32 instructions from programs written in FALSE and stores it in corresponding object files.
\flowgraph{\resource{FALSE\\source code} \ar[r] & \toolbox{falarma32} \ar[r] & \resource{object file}}
\seefalse\seearm\seeobject
}

\providecommand{\falarmb}{
\toolsection{falarma64} is a compiler for the FALSE programming language targeting the ARM hardware architecture.
It generates machine code for ARM processors executing A64 instructions from programs written in FALSE and stores it in corresponding object files.
\flowgraph{\resource{FALSE\\source code} \ar[r] & \toolbox{falarma64} \ar[r] & \resource{object file}}
\seefalse\seearm\seeobject
}

\providecommand{\falarmc}{
\toolsection{falarmt32} is a compiler for the FALSE programming language targeting the ARM hardware architecture.
It generates machine code for ARM processors without floating-point extension executing T32 instructions from programs written in FALSE and stores it in corresponding object files.
\flowgraph{\resource{FALSE\\source code} \ar[r] & \toolbox{falarmt32} \ar[r] & \resource{object file}}
\seefalse\seearm\seeobject
}

\providecommand{\falarmcfpe}{
\toolsection{falarmt32fpe} is a compiler for the FALSE programming language targeting the ARM hardware architecture.
It generates machine code for ARM processors with floating-point extension executing T32 instructions from programs written in FALSE and stores it in corresponding object files.
\flowgraph{\resource{FALSE\\source code} \ar[r] & \toolbox{falarmt32fpe} \ar[r] & \resource{object file}}
\seefalse\seearm\seeobject
}

\providecommand{\falavr}{
\toolsection{falavr} is a compiler for the FALSE programming language targeting the AVR hardware architecture.
It generates machine code for AVR processors from programs written in FALSE and stores it in corresponding object files.
\flowgraph{\resource{FALSE\\source code} \ar[r] & \toolbox{falavr} \ar[r] & \resource{object file}}
\seefalse\seeavr\seeobject
}

\providecommand{\falavrtt}{
\toolsection{falavr32} is a compiler for the FALSE programming language targeting the AVR32 hardware architecture.
It generates machine code for AVR32 processors from programs written in FALSE and stores it in corresponding object files.
\flowgraph{\resource{FALSE\\source code} \ar[r] & \toolbox{falavr32} \ar[r] & \resource{object file}}
\seefalse\seeavrtt\seeobject
}

\providecommand{\falmabk}{
\toolsection{falm68k} is a compiler for the FALSE programming language targeting the M68000 hardware architecture.
It generates machine code for M68000 processors from programs written in FALSE and stores it in corresponding object files.
\flowgraph{\resource{FALSE\\source code} \ar[r] & \toolbox{falm68k} \ar[r] & \resource{object file}}
\seefalse\seemabk\seeobject
}

\providecommand{\falmibl}{
\toolsection{falmibl} is a compiler for the FALSE programming language targeting the MicroBlaze hardware architecture.
It generates machine code for MicroBlaze processors from programs written in FALSE and stores it in corresponding object files.
\flowgraph{\resource{FALSE\\source code} \ar[r] & \toolbox{falmibl} \ar[r] & \resource{object file}}
\seefalse\seemibl\seeobject
}

\providecommand{\falmipsa}{
\toolsection{falmips32} is a compiler for the FALSE programming language targeting the MIPS32 hardware architecture.
It generates machine code for MIPS32 processors from programs written in FALSE and stores it in corresponding object files.
\flowgraph{\resource{FALSE\\source code} \ar[r] & \toolbox{falmips32} \ar[r] & \resource{object file}}
\seefalse\seemips\seeobject
}

\providecommand{\falmipsb}{
\toolsection{falmips64} is a compiler for the FALSE programming language targeting the MIPS64 hardware architecture.
It generates machine code for MIPS64 processors from programs written in FALSE and stores it in corresponding object files.
\flowgraph{\resource{FALSE\\source code} \ar[r] & \toolbox{falmips64} \ar[r] & \resource{object file}}
\seefalse\seemips\seeobject
}

\providecommand{\falmmix}{
\toolsection{falmmix} is a compiler for the FALSE programming language targeting the MMIX hardware architecture.
It generates machine code for MMIX processors from programs written in FALSE and stores it in corresponding object files.
\flowgraph{\resource{FALSE\\source code} \ar[r] & \toolbox{falmmix} \ar[r] & \resource{object file}}
\seefalse\seemmix\seeobject
}

\providecommand{\falorok}{
\toolsection{falor1k} is a compiler for the FALSE programming language targeting the OpenRISC 1000 hardware architecture.
It generates machine code for OpenRISC 1000 processors from programs written in FALSE and stores it in corresponding object files.
\flowgraph{\resource{FALSE\\source code} \ar[r] & \toolbox{falor1k} \ar[r] & \resource{object file}}
\seefalse\seeorok\seeobject
}

\providecommand{\falppca}{
\toolsection{falppc32} is a compiler for the FALSE programming language targeting the PowerPC hardware architecture.
It generates machine code for PowerPC processors from programs written in FALSE and stores it in corresponding object files.
The compiler generates machine code for the 32-bit operating mode defined by the PowerPC architecture.
\flowgraph{\resource{FALSE\\source code} \ar[r] & \toolbox{falppc32} \ar[r] & \resource{object file}}
\seefalse\seeppc\seeobject
}

\providecommand{\falppcb}{
\toolsection{falppc64} is a compiler for the FALSE programming language targeting the PowerPC hardware architecture.
It generates machine code for PowerPC processors from programs written in FALSE and stores it in corresponding object files.
The compiler generates machine code for the 64-bit operating mode defined by the PowerPC architecture.
\flowgraph{\resource{FALSE\\source code} \ar[r] & \toolbox{falppc64} \ar[r] & \resource{object file}}
\seefalse\seeppc\seeobject
}

\providecommand{\falrisc}{
\toolsection{falrisc} is a compiler for the FALSE programming language targeting the RISC hardware architecture.
It generates machine code for RISC processors from programs written in FALSE and stores it in corresponding object files.
\flowgraph{\resource{FALSE\\source code} \ar[r] & \toolbox{falrisc} \ar[r] & \resource{object file}}
\seefalse\seerisc\seeobject
}

\providecommand{\falwasm}{
\toolsection{falwasm} is a compiler for the FALSE programming language targeting the WebAssembly architecture.
It generates machine code for WebAssembly targets from programs written in FALSE and stores it in corresponding object files.
\flowgraph{\resource{FALSE\\source code} \ar[r] & \toolbox{falwasm} \ar[r] & \resource{object file}}
\seefalse\seewasm\seeobject
}

% Oberon tools

\providecommand{\obprint}{
\toolsection{obprint} is a pretty printer for the Oberon programming language.
It reformats the source code of Oberon modules and writes it to the standard output stream.
\flowgraph{\resource{Oberon\\source code} \ar[r] & \toolbox{obprint} \ar[r] & \resource{reformatted\\source code}}
\seeoberon
}

\providecommand{\obcheck}{
\toolsection{obcheck} is a syntactic and semantic checker for the Oberon programming language.
It just performs syntactic and semantic checks on Oberon modules and writes its diagnostic messages to the standard error stream.
In addition, it stores the interface of each module in a symbol file which is required when other modules import the module.
\flowgraph{\resource{Oberon\\source code} \ar[r] & \toolbox{obcheck} \ar[r] \ar@/l/[d] & \resource{diagnostic\\messages} \\ \variable{ECSIMPORT} \ar[ru] & \resource{symbol\\files} \ar@/r/[u]}
\seeoberon
}

\providecommand{\obdump}{
\toolsection{obdump} is a serializer for the Oberon programming language.
It dumps the complete internal representation of modules written in Oberon into an XML document.
\debuggingtool
\flowgraph{\resource{Oberon\\source code} \ar[r] & \toolbox{obdump} \ar[r] \ar@/l/[d] & \resource{internal\\representation} \\ \variable{ECSIMPORT} \ar[ru] & \resource{symbol\\files} \ar@/r/[u]}
\seeoberon
}

\providecommand{\obrun}{
\toolsection{obrun} is an interpreter for the Oberon programming language.
It processes and executes modules written in Oberon.
This tool does neither generate nor process symbol files while interpreting modules.
If a module is imported by another one, its filename has to be named before the other one in the list of command-line arguments.
\flowgraph{\resource{Oberon\\source code} \ar[r] & \toolbox{obrun} \ar@/u/[r] & \resource{input/\\output} \ar@/d/[l]}
\seeoberon
}

\providecommand{\obcpp}{
\toolsection{obcpp} is a transpiler for the Oberon programming language.
It translates programs written in Oberon into \cpp{} programs and stores them in corresponding source and header files.
In addition, it stores the interface of each module in a symbol file which is required when other modules import the module.
The same interface is provided by the generated header file which can be used in other parts of the \cpp{} program.
\flowgraph{\resource{Oberon\\source code} \ar[r] & \toolbox{obcpp} \ar[r] \ar@/l/[d] \ar[rd] & \resource{\cpp{}\\source file} \\ \variable{ECSIMPORT} \ar[ru] & \resource{symbol\\files} \ar@/r/[u] & \resource{\cpp{}\\header file}}
\seeoberon\seecpp
}

\providecommand{\obdoc}{
\toolsection{obdoc} is a generic documentation generator for the Oberon programming language.
It processes several Oberon modules and assembles all information therein into a generic documentation.
In addition, it stores the interface of each module in a symbol file which is required when other modules import the module.
\debuggingtool
\flowgraph{\resource{Oberon\\source code} \ar[r] & \toolbox{obdoc} \ar[r] \ar@/l/[d] & \resource{generic\\documentation} \\ \variable{ECSIMPORT} \ar[ru] & \resource{symbol\\files} \ar@/r/[u]}
\seeoberon\seedocumentation
}

\providecommand{\obhtml}{
\toolsection{obhtml} is an HTML documentation generator for the Oberon programming language.
It processes several Oberon modules and assembles all information therein into an HTML document.
In addition, it stores the interface of each module in a symbol file which is required when other modules import the module.
\flowgraph{\resource{Oberon\\source code} \ar[r] & \toolbox{obhtml} \ar[r] \ar@/l/[d] & \resource{HTML\\document} \\ \variable{ECSIMPORT} \ar[ru] & \resource{symbol\\files} \ar@/r/[u]}
\seeoberon\seedocumentation
}

\providecommand{\oblatex}{
\toolsection{oblatex} is a Latex documentation generator for the Oberon programming language.
It processes several Oberon modules and assembles all information therein into a Latex document.
In addition, it stores the interface of each module in a symbol file which is required when other modules import the module.
\flowgraph{\resource{Oberon\\source code} \ar[r] & \toolbox{oblatex} \ar[r] \ar@/l/[d] & \resource{Latex\\document} \\ \variable{ECSIMPORT} \ar[ru] & \resource{symbol\\files} \ar@/r/[u]}
\seeoberon\seedocumentation
}

\providecommand{\obcode}{
\toolsection{obcode} is an intermediate code generator for the Oberon programming language.
It generates intermediate code from modules written in Oberon and stores it in corresponding assembly files.
In addition, it stores the interface of each module in a symbol file which is required when other modules import the module.
Programs generated with this tool require additional runtime support that is stored in the \file{ob\-code\-run} library file.
\debuggingtool
\flowgraph{\resource{Oberon\\source code} \ar[r] & \toolbox{obcode} \ar[r] \ar@/l/[d] & \resource{intermediate\\code} \\ \variable{ECSIMPORT} \ar[ru] & \resource{symbol\\files} \ar@/r/[u]}
\seeoberon\seeassembly\seecode
}

\providecommand{\obamda}{
\toolsection{obamd16} is a compiler for the Oberon programming language targeting the AMD64 hardware architecture.
It generates machine code for AMD64 processors from modules written in Oberon and stores it in corresponding object files.
The compiler generates machine code for the 16-bit operating mode defined by the AMD64 architecture.
For debugging purposes, it also creates a debugging information file as well as an assembly file containing a listing of the generated machine code.
In addition, it stores the interface of each module in a symbol file which is required when other modules import the module.
Programs generated with this compiler require additional runtime support that is stored in the \file{ob\-amd16\-run} library file.
\flowgraph{\resource{Oberon\\source code} \ar[r] & \toolbox{obamd16} \ar[r] \ar@/l/[d] \ar[rd] & \resource{object file} \\ \variable{ECSIMPORT} \ar[ru] & \resource{symbol\\files} \ar@/r/[u] & \resource{debugging\\information}}
\seeoberon\seeassembly\seeamd\seeobject\seedebugging
}

\providecommand{\obamdb}{
\toolsection{obamd32} is a compiler for the Oberon programming language targeting the AMD64 hardware architecture.
It generates machine code for AMD64 processors from modules written in Oberon and stores it in corresponding object files.
The compiler generates machine code for the 32-bit operating mode defined by the AMD64 architecture.
For debugging purposes, it also creates a debugging information file as well as an assembly file containing a listing of the generated machine code.
In addition, it stores the interface of each module in a symbol file which is required when other modules import the module.
Programs generated with this compiler require additional runtime support that is stored in the \file{ob\-amd32\-run} library file.
\flowgraph{\resource{Oberon\\source code} \ar[r] & \toolbox{obamd32} \ar[r] \ar@/l/[d] \ar[rd] & \resource{object file} \\ \variable{ECSIMPORT} \ar[ru] & \resource{symbol\\files} \ar@/r/[u] & \resource{debugging\\information}}
\seeoberon\seeassembly\seeamd\seeobject\seedebugging
}

\providecommand{\obamdc}{
\toolsection{obamd64} is a compiler for the Oberon programming language targeting the AMD64 hardware architecture.
It generates machine code for AMD64 processors from modules written in Oberon and stores it in corresponding object files.
The compiler generates machine code for the 64-bit operating mode defined by the AMD64 architecture.
For debugging purposes, it also creates a debugging information file as well as an assembly file containing a listing of the generated machine code.
In addition, it stores the interface of each module in a symbol file which is required when other modules import the module.
Programs generated with this compiler require additional runtime support that is stored in the \file{ob\-amd64\-run} library file.
\flowgraph{\resource{Oberon\\source code} \ar[r] & \toolbox{obamd64} \ar[r] \ar@/l/[d] \ar[rd] & \resource{object file} \\ \variable{ECSIMPORT} \ar[ru] & \resource{symbol\\files} \ar@/r/[u] & \resource{debugging\\information}}
\seeoberon\seeassembly\seeamd\seeobject\seedebugging
}

\providecommand{\obarma}{
\toolsection{obarma32} is a compiler for the Oberon programming language targeting the ARM hardware architecture.
It generates machine code for ARM processors executing A32 instructions from modules written in Oberon and stores it in corresponding object files.
For debugging purposes, it also creates a debugging information file as well as an assembly file containing a listing of the generated machine code.
In addition, it stores the interface of each module in a symbol file which is required when other modules import the module.
Programs generated with this compiler require additional runtime support that is stored in the \file{ob\-arma32\-run} library file.
\flowgraph{\resource{Oberon\\source code} \ar[r] & \toolbox{obarma32} \ar[r] \ar@/l/[d] \ar[rd] & \resource{object file} \\ \variable{ECSIMPORT} \ar[ru] & \resource{symbol\\files} \ar@/r/[u] & \resource{debugging\\information}}
\seeoberon\seeassembly\seearm\seeobject\seedebugging
}

\providecommand{\obarmb}{
\toolsection{obarma64} is a compiler for the Oberon programming language targeting the ARM hardware architecture.
It generates machine code for ARM processors executing A64 instructions from modules written in Oberon and stores it in corresponding object files.
For debugging purposes, it also creates a debugging information file as well as an assembly file containing a listing of the generated machine code.
In addition, it stores the interface of each module in a symbol file which is required when other modules import the module.
Programs generated with this compiler require additional runtime support that is stored in the \file{ob\-arma64\-run} library file.
\flowgraph{\resource{Oberon\\source code} \ar[r] & \toolbox{obarma64} \ar[r] \ar@/l/[d] \ar[rd] & \resource{object file} \\ \variable{ECSIMPORT} \ar[ru] & \resource{symbol\\files} \ar@/r/[u] & \resource{debugging\\information}}
\seeoberon\seeassembly\seearm\seeobject\seedebugging
}

\providecommand{\obarmc}{
\toolsection{obarmt32} is a compiler for the Oberon programming language targeting the ARM hardware architecture.
It generates machine code for ARM processors without floating-point extension executing T32 instructions from modules written in Oberon and stores it in corresponding object files.
For debugging purposes, it also creates a debugging information file as well as an assembly file containing a listing of the generated machine code.
In addition, it stores the interface of each module in a symbol file which is required when other modules import the module.
Programs generated with this compiler require additional runtime support that is stored in the \file{ob\-armt32\-run} library file.
\flowgraph{\resource{Oberon\\source code} \ar[r] & \toolbox{obarmt32} \ar[r] \ar@/l/[d] \ar[rd] & \resource{object file} \\ \variable{ECSIMPORT} \ar[ru] & \resource{symbol\\files} \ar@/r/[u] & \resource{debugging\\information}}
\seeoberon\seeassembly\seearm\seeobject\seedebugging
}

\providecommand{\obarmcfpe}{
\toolsection{obarmt32fpe} is a compiler for the Oberon programming language targeting the ARM hardware architecture.
It generates machine code for ARM processors with floating-point extension executing T32 instructions from modules written in Oberon and stores it in corresponding object files.
For debugging purposes, it also creates a debugging information file as well as an assembly file containing a listing of the generated machine code.
In addition, it stores the interface of each module in a symbol file which is required when other modules import the module.
Programs generated with this compiler require additional runtime support that is stored in the \file{ob\-armt32\-fpe\-run} library file.
\flowgraph{\resource{Oberon\\source code} \ar[r] & \toolbox{obarmt32fpe} \ar[r] \ar@/l/[d] \ar[rd] & \resource{object file} \\ \variable{ECSIMPORT} \ar[ru] & \resource{symbol\\files} \ar@/r/[u] & \resource{debugging\\information}}
\seeoberon\seeassembly\seearm\seeobject\seedebugging
}

\providecommand{\obavr}{
\toolsection{obavr} is a compiler for the Oberon programming language targeting the AVR hardware architecture.
It generates machine code for AVR processors from modules written in Oberon and stores it in corresponding object files.
For debugging purposes, it also creates a debugging information file as well as an assembly file containing a listing of the generated machine code.
In addition, it stores the interface of each module in a symbol file which is required when other modules import the module.
Programs generated with this compiler require additional runtime support that is stored in the \file{ob\-avr\-run} library file.
\flowgraph{\resource{Oberon\\source code} \ar[r] & \toolbox{obavr} \ar[r] \ar@/l/[d] \ar[rd] & \resource{object file} \\ \variable{ECSIMPORT} \ar[ru] & \resource{symbol\\files} \ar@/r/[u] & \resource{debugging\\information}}
\seeoberon\seeassembly\seeavr\seeobject\seedebugging
}

\providecommand{\obavrtt}{
\toolsection{obavr32} is a compiler for the Oberon programming language targeting the AVR32 hardware architecture.
It generates machine code for AVR32 processors from modules written in Oberon and stores it in corresponding object files.
For debugging purposes, it also creates a debugging information file as well as an assembly file containing a listing of the generated machine code.
In addition, it stores the interface of each module in a symbol file which is required when other modules import the module.
Programs generated with this compiler require additional runtime support that is stored in the \file{ob\-avr32\-run} library file.
\flowgraph{\resource{Oberon\\source code} \ar[r] & \toolbox{obavr32} \ar[r] \ar@/l/[d] \ar[rd] & \resource{object file} \\ \variable{ECSIMPORT} \ar[ru] & \resource{symbol\\files} \ar@/r/[u] & \resource{debugging\\information}}
\seeoberon\seeassembly\seeavrtt\seeobject\seedebugging
}

\providecommand{\obmabk}{
\toolsection{obm68k} is a compiler for the Oberon programming language targeting the M68000 hardware architecture.
It generates machine code for M68000 processors from modules written in Oberon and stores it in corresponding object files.
For debugging purposes, it also creates a debugging information file as well as an assembly file containing a listing of the generated machine code.
In addition, it stores the interface of each module in a symbol file which is required when other modules import the module.
Programs generated with this compiler require additional runtime support that is stored in the \file{ob\-m68k\-run} library file.
\flowgraph{\resource{Oberon\\source code} \ar[r] & \toolbox{obm68k} \ar[r] \ar@/l/[d] \ar[rd] & \resource{object file} \\ \variable{ECSIMPORT} \ar[ru] & \resource{symbol\\files} \ar@/r/[u] & \resource{debugging\\information}}
\seeoberon\seeassembly\seemabk\seeobject\seedebugging
}

\providecommand{\obmibl}{
\toolsection{obmibl} is a compiler for the Oberon programming language targeting the MicroBlaze hardware architecture.
It generates machine code for MicroBlaze processors from modules written in Oberon and stores it in corresponding object files.
For debugging purposes, it also creates a debugging information file as well as an assembly file containing a listing of the generated machine code.
In addition, it stores the interface of each module in a symbol file which is required when other modules import the module.
Programs generated with this compiler require additional runtime support that is stored in the \file{ob\-mibl\-run} library file.
\flowgraph{\resource{Oberon\\source code} \ar[r] & \toolbox{obmibl} \ar[r] \ar@/l/[d] \ar[rd] & \resource{object file} \\ \variable{ECSIMPORT} \ar[ru] & \resource{symbol\\files} \ar@/r/[u] & \resource{debugging\\information}}
\seeoberon\seeassembly\seemibl\seeobject\seedebugging
}

\providecommand{\obmipsa}{
\toolsection{obmips32} is a compiler for the Oberon programming language targeting the MIPS32 hardware architecture.
It generates machine code for MIPS32 processors from modules written in Oberon and stores it in corresponding object files.
For debugging purposes, it also creates a debugging information file as well as an assembly file containing a listing of the generated machine code.
In addition, it stores the interface of each module in a symbol file which is required when other modules import the module.
Programs generated with this compiler require additional runtime support that is stored in the \file{ob\-mips32\-run} library file.
\flowgraph{\resource{Oberon\\source code} \ar[r] & \toolbox{obmips32} \ar[r] \ar@/l/[d] \ar[rd] & \resource{object file} \\ \variable{ECSIMPORT} \ar[ru] & \resource{symbol\\files} \ar@/r/[u] & \resource{debugging\\information}}
\seeoberon\seeassembly\seemips\seeobject\seedebugging
}

\providecommand{\obmipsb}{
\toolsection{obmips64} is a compiler for the Oberon programming language targeting the MIPS64 hardware architecture.
It generates machine code for MIPS64 processors from modules written in Oberon and stores it in corresponding object files.
For debugging purposes, it also creates a debugging information file as well as an assembly file containing a listing of the generated machine code.
In addition, it stores the interface of each module in a symbol file which is required when other modules import the module.
Programs generated with this compiler require additional runtime support that is stored in the \file{ob\-mips64\-run} library file.
\flowgraph{\resource{Oberon\\source code} \ar[r] & \toolbox{obmips64} \ar[r] \ar@/l/[d] \ar[rd] & \resource{object file} \\ \variable{ECSIMPORT} \ar[ru] & \resource{symbol\\files} \ar@/r/[u] & \resource{debugging\\information}}
\seeoberon\seeassembly\seemips\seeobject\seedebugging
}

\providecommand{\obmmix}{
\toolsection{obmmix} is a compiler for the Oberon programming language targeting the MMIX hardware architecture.
It generates machine code for MMIX processors from modules written in Oberon and stores it in corresponding object files.
For debugging purposes, it also creates a debugging information file as well as an assembly file containing a listing of the generated machine code.
In addition, it stores the interface of each module in a symbol file which is required when other modules import the module.
Programs generated with this compiler require additional runtime support that is stored in the \file{ob\-mmix\-run} library file.
\flowgraph{\resource{Oberon\\source code} \ar[r] & \toolbox{obmmix} \ar[r] \ar@/l/[d] \ar[rd] & \resource{object file} \\ \variable{ECSIMPORT} \ar[ru] & \resource{symbol\\files} \ar@/r/[u] & \resource{debugging\\information}}
\seeoberon\seeassembly\seemmix\seeobject\seedebugging
}

\providecommand{\oborok}{
\toolsection{obor1k} is a compiler for the Oberon programming language targeting the OpenRISC 1000 hardware architecture.
It generates machine code for OpenRISC 1000 processors from modules written in Oberon and stores it in corresponding object files.
For debugging purposes, it also creates a debugging information file as well as an assembly file containing a listing of the generated machine code.
In addition, it stores the interface of each module in a symbol file which is required when other modules import the module.
Programs generated with this compiler require additional runtime support that is stored in the \file{ob\-or1k\-run} library file.
\flowgraph{\resource{Oberon\\source code} \ar[r] & \toolbox{obor1k} \ar[r] \ar@/l/[d] \ar[rd] & \resource{object file} \\ \variable{ECSIMPORT} \ar[ru] & \resource{symbol\\files} \ar@/r/[u] & \resource{debugging\\information}}
\seeoberon\seeassembly\seeorok\seeobject\seedebugging
}

\providecommand{\obppca}{
\toolsection{obppc32} is a compiler for the Oberon programming language targeting the PowerPC hardware architecture.
It generates machine code for PowerPC processors from modules written in Oberon and stores it in corresponding object files.
The compiler generates machine code for the 32-bit operating mode defined by the PowerPC architecture.
For debugging purposes, it also creates a debugging information file as well as an assembly file containing a listing of the generated machine code.
In addition, it stores the interface of each module in a symbol file which is required when other modules import the module.
Programs generated with this compiler require additional runtime support that is stored in the \file{ob\-ppc32\-run} library file.
\flowgraph{\resource{Oberon\\source code} \ar[r] & \toolbox{obppc32} \ar[r] \ar@/l/[d] \ar[rd] & \resource{object file} \\ \variable{ECSIMPORT} \ar[ru] & \resource{symbol\\files} \ar@/r/[u] & \resource{debugging\\information}}
\seeoberon\seeassembly\seeppc\seeobject\seedebugging
}

\providecommand{\obppcb}{
\toolsection{obppc64} is a compiler for the Oberon programming language targeting the PowerPC hardware architecture.
It generates machine code for PowerPC processors from modules written in Oberon and stores it in corresponding object files.
The compiler generates machine code for the 64-bit operating mode defined by the PowerPC architecture.
For debugging purposes, it also creates a debugging information file as well as an assembly file containing a listing of the generated machine code.
In addition, it stores the interface of each module in a symbol file which is required when other modules import the module.
Programs generated with this compiler require additional runtime support that is stored in the \file{ob\-ppc64\-run} library file.
\flowgraph{\resource{Oberon\\source code} \ar[r] & \toolbox{obppc64} \ar[r] \ar@/l/[d] \ar[rd] & \resource{object file} \\ \variable{ECSIMPORT} \ar[ru] & \resource{symbol\\files} \ar@/r/[u] & \resource{debugging\\information}}
\seeoberon\seeassembly\seeppc\seeobject\seedebugging
}

\providecommand{\obrisc}{
\toolsection{obrisc} is a compiler for the Oberon programming language targeting the RISC hardware architecture.
It generates machine code for RISC processors from modules written in Oberon and stores it in corresponding object files.
For debugging purposes, it also creates a debugging information file as well as an assembly file containing a listing of the generated machine code.
In addition, it stores the interface of each module in a symbol file which is required when other modules import the module.
Programs generated with this compiler require additional runtime support that is stored in the \file{ob\-risc\-run} library file.
\flowgraph{\resource{Oberon\\source code} \ar[r] & \toolbox{obrisc} \ar[r] \ar@/l/[d] \ar[rd] & \resource{object file} \\ \variable{ECSIMPORT} \ar[ru] & \resource{symbol\\files} \ar@/r/[u] & \resource{debugging\\information}}
\seeoberon\seeassembly\seerisc\seeobject\seedebugging
}

\providecommand{\obwasm}{
\toolsection{obwasm} is a compiler for the Oberon programming language targeting the WebAssembly architecture.
It generates machine code for WebAssembly targets from modules written in Oberon and stores it in corresponding object files.
For debugging purposes, it also creates a debugging information file as well as an assembly file containing a listing of the generated machine code.
In addition, it stores the interface of each module in a symbol file which is required when other modules import the module.
Programs generated with this compiler require additional runtime support that is stored in the \file{ob\-wasm\-run} library file.
\flowgraph{\resource{Oberon\\source code} \ar[r] & \toolbox{obwasm} \ar[r] \ar@/l/[d] \ar[rd] & \resource{object file} \\ \variable{ECSIMPORT} \ar[ru] & \resource{symbol\\files} \ar@/r/[u] & \resource{debugging\\information}}
\seeoberon\seeassembly\seewasm\seeobject\seedebugging
}

% converter tools

\providecommand{\dbgdwarf}{
\toolsection{dbgdwarf} is a DWARF debugging information converter tool.
It converts debugging information into the DWARF debugging data format and stores it in corresponding object files~\cite{dwarffile}.
The resulting debugging object files can be combined with runtime support that creates Executable and Linking Format (ELF) files~\cite{elffile}.
\flowgraph{\resource{debugging\\information} \ar[r] & \toolbox{dbgdwarf} \ar[r] & \resource{debugging\\object file}}
\seeobject\seedebugging
}

% assembler tools

\providecommand{\asmprint}{
\toolsection{asmprint} is a pretty printer for generic assembly code.
It reformats generic assembly code and writes it to the standard output stream.
\flowgraph{\resource{generic assembly\\source code} \ar[r] & \toolbox{asmprint} \ar[r] & \resource{reformatted\\source code}}
\seeassembly
}

\providecommand{\amdaasm}{
\toolsection{amd16asm} is an assembler for the AMD64 hardware architecture.
It translates assembly code into machine code for AMD64 processors and stores it in corresponding object files.
By default, the assembler generates machine code for the 16-bit operating mode defined by the AMD64 architecture.
\flowgraph{\resource{AMD16 assembly\\source code} \ar[r] & \toolbox{amd16asm} \ar[r] & \resource{object file}}
\seeassembly\seeamd\seeobject
}

\providecommand{\amdadism}{
\toolsection{amd16dism} is a disassembler for the AMD64 hardware architecture.
It translates machine code from object files targeting AMD64 processors into assembly code and writes it to the standard output stream.
It assumes that the machine code was generated for the 16-bit operating mode defined by the AMD64 architecture.
\flowgraph{\resource{object file} \ar[r] & \toolbox{amd16dism} \ar[r] & \resource{disassembly\\listing}}
\seeassembly\seeamd\seeobject
}

\providecommand{\amdbasm}{
\toolsection{amd32asm} is an assembler for the AMD64 hardware architecture.
It translates assembly code into machine code for AMD64 processors and stores it in corresponding object files.
By default, the assembler generates machine code for the 32-bit operating mode defined by the AMD64 architecture.
\flowgraph{\resource{AMD32 assembly\\source code} \ar[r] & \toolbox{amd32asm} \ar[r] & \resource{object file}}
\seeassembly\seeamd\seeobject
}

\providecommand{\amdbdism}{
\toolsection{amd32dism} is a disassembler for the AMD64 hardware architecture.
It translates machine code from object files targeting AMD64 processors into assembly code and writes it to the standard output stream.
It assumes that the machine code was generated for the 32-bit operating mode defined by the AMD64 architecture.
\flowgraph{\resource{object file} \ar[r] & \toolbox{amd32dism} \ar[r] & \resource{disassembly\\listing}}
\seeassembly\seeamd\seeobject
}

\providecommand{\amdcasm}{
\toolsection{amd64asm} is an assembler for the AMD64 hardware architecture.
It translates assembly code into machine code for AMD64 processors and stores it in corresponding object files.
By default, the assembler generates machine code for the 64-bit operating mode defined by the AMD64 architecture.
\flowgraph{\resource{AMD64 assembly\\source code} \ar[r] & \toolbox{amd64asm} \ar[r] & \resource{object file}}
\seeassembly\seeamd\seeobject
}

\providecommand{\amdcdism}{
\toolsection{amd64dism} is a disassembler for the AMD64 hardware architecture.
It translates machine code from object files targeting AMD64 processors into assembly code and writes it to the standard output stream.
It assumes that the machine code was generated for the 64-bit operating mode defined by the AMD64 architecture.
\flowgraph{\resource{object file} \ar[r] & \toolbox{amd64dism} \ar[r] & \resource{disassembly\\listing}}
\seeassembly\seeamd\seeobject
}

\providecommand{\armaasm}{
\toolsection{arma32asm} is an assembler for the ARM hardware architecture.
It translates assembly code into machine code for ARM processors executing A32 instructions and stores it in corresponding object files.
\flowgraph{\resource{ARM A32 assembly\\source code} \ar[r] & \toolbox{arma32asm} \ar[r] & \resource{object file}}
\seeassembly\seearm\seeobject
}

\providecommand{\armadism}{
\toolsection{arma32dism} is a disassembler for the ARM hardware architecture.
It translates machine code from object files targeting ARM processors executing A32 instructions into assembly code and writes it to the standard output stream.
\flowgraph{\resource{object file} \ar[r] & \toolbox{arma32dism} \ar[r] & \resource{disassembly\\listing}}
\seeassembly\seearm\seeobject
}

\providecommand{\armbasm}{
\toolsection{arma64asm} is an assembler for the ARM hardware architecture.
It translates assembly code into machine code for ARM processors executing A64 instructions and stores it in corresponding object files.
\flowgraph{\resource{ARM A64 assembly\\source code} \ar[r] & \toolbox{arma64asm} \ar[r] & \resource{object file}}
\seeassembly\seearm\seeobject
}

\providecommand{\armbdism}{
\toolsection{arma64dism} is a disassembler for the ARM hardware architecture.
It translates machine code from object files targeting ARM processors executing A64 instructions into assembly code and writes it to the standard output stream.
\flowgraph{\resource{object file} \ar[r] & \toolbox{arma64dism} \ar[r] & \resource{disassembly\\listing}}
\seeassembly\seearm\seeobject
}

\providecommand{\armcasm}{
\toolsection{armt32asm} is an assembler for the ARM hardware architecture.
It translates assembly code into machine code for ARM processors executing T32 instructions and stores it in corresponding object files.
\flowgraph{\resource{ARM T32 assembly\\source code} \ar[r] & \toolbox{armt32asm} \ar[r] & \resource{object file}}
\seeassembly\seearm\seeobject
}

\providecommand{\armcdism}{
\toolsection{armt32dism} is a disassembler for the ARM hardware architecture.
It translates machine code from object files targeting ARM processors executing T32 instructions into assembly code and writes it to the standard output stream.
\flowgraph{\resource{object file} \ar[r] & \toolbox{armt32dism} \ar[r] & \resource{disassembly\\listing}}
\seeassembly\seearm\seeobject
}

\providecommand{\avrasm}{
\toolsection{avrasm} is an assembler for the AVR hardware architecture.
It translates assembly code into machine code for AVR processors and stores it in corresponding object files.
The identifiers \texttt{RXL}, \texttt{RXH}, \texttt{RYL}, \texttt{RYH}, \texttt{RZL}, and \texttt{RZH} are predefined and name the corresponding registers.
The identifiers \texttt{SPL} and \texttt{SPH} are also predefined and evaluate to the address of the corresponding registers.
\flowgraph{\resource{AVR assembly\\source code} \ar[r] & \toolbox{avrasm} \ar[r] & \resource{object file}}
\seeassembly\seeavr\seeobject
}

\providecommand{\avrdism}{
\toolsection{avrdism} is a disassembler for the AVR hardware architecture.
It translates machine code from object files targeting AVR processors into assembly code and writes it to the standard output stream.
\flowgraph{\resource{object file} \ar[r] & \toolbox{avrdism} \ar[r] & \resource{disassembly\\listing}}
\seeassembly\seeavr\seeobject
}

\providecommand{\avrttasm}{
\toolsection{avr32asm} is an assembler for the AVR32 hardware architecture.
It translates assembly code into machine code for AVR32 processors and stores it in corresponding object files.
\flowgraph{\resource{AVR32 assembly\\source code} \ar[r] & \toolbox{avr32asm} \ar[r] & \resource{object file}}
\seeassembly\seeavrtt\seeobject
}

\providecommand{\avrttdism}{
\toolsection{avr32dism} is a disassembler for the AVR32 hardware architecture.
It translates machine code from object files targeting AVR32 processors into assembly code and writes it to the standard output stream.
\flowgraph{\resource{object file} \ar[r] & \toolbox{avr32dism} \ar[r] & \resource{disassembly\\listing}}
\seeassembly\seeavrtt\seeobject
}

\providecommand{\mabkasm}{
\toolsection{m68kasm} is an assembler for the M68000 hardware architecture.
It translates assembly code into machine code for M68000 processors and stores it in corresponding object files.
\flowgraph{\resource{68000 assembly\\source code} \ar[r] & \toolbox{m68kasm} \ar[r] & \resource{object file}}
\seeassembly\seemabk\seeobject
}

\providecommand{\mabkdism}{
\toolsection{m68kdism} is a disassembler for the M68000 hardware architecture.
It translates machine code from object files targeting M68000 processors into assembly code and writes it to the standard output stream.
\flowgraph{\resource{object file} \ar[r] & \toolbox{m68kdism} \ar[r] & \resource{disassembly\\listing}}
\seeassembly\seemabk\seeobject
}

\providecommand{\miblasm}{
\toolsection{miblasm} is an assembler for the MicroBlaze hardware architecture.
It translates assembly code into machine code for MicroBlaze processors and stores it in corresponding object files.
\flowgraph{\resource{MicroBlaze assembly\\source code} \ar[r] & \toolbox{miblasm} \ar[r] & \resource{object file}}
\seeassembly\seemibl\seeobject
}

\providecommand{\mibldism}{
\toolsection{mibldism} is a disassembler for the MicroBlaze hardware architecture.
It translates machine code from object files targeting MicroBlaze processors into assembly code and writes it to the standard output stream.
\flowgraph{\resource{object file} \ar[r] & \toolbox{mibldism} \ar[r] & \resource{disassembly\\listing}}
\seeassembly\seemibl\seeobject
}

\providecommand{\mipsaasm}{
\toolsection{mips32asm} is an assembler for the MIPS32 hardware architecture.
It translates assembly code into machine code for MIPS32 processors and stores it in corresponding object files.
\flowgraph{\resource{MIPS32 assembly\\source code} \ar[r] & \toolbox{mips32asm} \ar[r] & \resource{object file}}
\seeassembly\seemips\seeobject
}

\providecommand{\mipsadism}{
\toolsection{mips32dism} is a disassembler for the MIPS32 hardware architecture.
It translates machine code from object files targeting MIPS32 processors into assembly code and writes it to the standard output stream.
\flowgraph{\resource{object file} \ar[r] & \toolbox{mips32dism} \ar[r] & \resource{disassembly\\listing}}
\seeassembly\seemips\seeobject
}

\providecommand{\mipsbasm}{
\toolsection{mips64asm} is an assembler for the MIPS64 hardware architecture.
It translates assembly code into machine code for MIPS64 processors and stores it in corresponding object files.
\flowgraph{\resource{MIPS64 assembly\\source code} \ar[r] & \toolbox{mips64asm} \ar[r] & \resource{object file}}
\seeassembly\seemips\seeobject
}

\providecommand{\mipsbdism}{
\toolsection{mips64dism} is a disassembler for the MIPS64 hardware architecture.
It translates machine code from object files targeting MIPS64 processors into assembly code and writes it to the standard output stream.
\flowgraph{\resource{object file} \ar[r] & \toolbox{mips64dism} \ar[r] & \resource{disassembly\\listing}}
\seeassembly\seemips\seeobject
}

\providecommand{\mmixasm}{
\toolsection{mmixasm} is an assembler for the MMIX hardware architecture.
It translates assembly code into machine code for MMIX processors and stores it in corresponding object files.
The names of all special registers are predefined and evaluate to the corresponding number.
\flowgraph{\resource{MMIX assembly\\source code} \ar[r] & \toolbox{mmixasm} \ar[r] & \resource{object file}}
\seeassembly\seemmix\seeobject
}

\providecommand{\mmixdism}{
\toolsection{mmixdism} is a disassembler for the MMIX hardware architecture.
It translates machine code from object files targeting MMIX processors into assembly code and writes it to the standard output stream.
\flowgraph{\resource{object file} \ar[r] & \toolbox{mmixdism} \ar[r] & \resource{disassembly\\listing}}
\seeassembly\seemmix\seeobject
}

\providecommand{\orokasm}{
\toolsection{or1kasm} is an assembler for the OpenRISC 1000 hardware architecture.
It translates assembly code into machine code for OpenRISC 1000 processors and stores it in corresponding object files.
\flowgraph{\resource{OpenRISC 1000 assembly\\source code} \ar[r] & \toolbox{or1kasm} \ar[r] & \resource{object file}}
\seeassembly\seeorok\seeobject
}

\providecommand{\orokdism}{
\toolsection{or1kdism} is a disassembler for the OpenRISC 1000 hardware architecture.
It translates machine code from object files targeting OpenRISC 1000 processors into assembly code and writes it to the standard output stream.
\flowgraph{\resource{object file} \ar[r] & \toolbox{or1kdism} \ar[r] & \resource{disassembly\\listing}}
\seeassembly\seeorok\seeobject
}

\providecommand{\ppcaasm}{
\toolsection{ppc32asm} is an assembler for the PowerPC hardware architecture.
It translates assembly code into machine code for PowerPC processors and stores it in corresponding object files.
By default, the assembler generates machine code for the 32-bit operating mode defined by the PowerPC architecture.
\flowgraph{\resource{PowerPC assembly\\source code} \ar[r] & \toolbox{ppc32asm} \ar[r] & \resource{object file}}
\seeassembly\seeppc\seeobject
}

\providecommand{\ppcadism}{
\toolsection{ppc32dism} is a disassembler for the PowerPC hardware architecture.
It translates machine code from object files targeting PowerPC processors into assembly code and writes it to the standard output stream.
It assumes that the machine code was generated for the 32-bit operating mode defined by the PowerPC architecture.
\flowgraph{\resource{object file} \ar[r] & \toolbox{ppc32dism} \ar[r] & \resource{disassembly\\listing}}
\seeassembly\seeppc\seeobject
}

\providecommand{\ppcbasm}{
\toolsection{ppc64asm} is an assembler for the PowerPC hardware architecture.
It translates assembly code into machine code for PowerPC processors and stores it in corresponding object files.
By default, the assembler generates machine code for the 64-bit operating mode defined by the PowerPC architecture.
\flowgraph{\resource{PowerPC assembly\\source code} \ar[r] & \toolbox{ppc64asm} \ar[r] & \resource{object file}}
\seeassembly\seeppc\seeobject
}

\providecommand{\ppcbdism}{
\toolsection{ppc64dism} is a disassembler for the PowerPC hardware architecture.
It translates machine code from object files targeting PowerPC processors into assembly code and writes it to the standard output stream.
It assumes that the machine code was generated for the 64-bit operating mode defined by the PowerPC architecture.
\flowgraph{\resource{object file} \ar[r] & \toolbox{ppc64dism} \ar[r] & \resource{disassembly\\listing}}
\seeassembly\seeppc\seeobject
}

\providecommand{\riscasm}{
\toolsection{riscasm} is an assembler for the RISC hardware architecture.
It translates assembly code into machine code for RISC processors and stores it in corresponding object files.
The names of all special registers are predefined and evaluate to the corresponding number.
\flowgraph{\resource{RISC assembly\\source code} \ar[r] & \toolbox{riscasm} \ar[r] & \resource{object file}}
\seeassembly\seerisc\seeobject
}

\providecommand{\riscdism}{
\toolsection{riscdism} is a disassembler for the RISC hardware architecture.
It translates machine code from object files targeting RISC processors into assembly code and writes it to the standard output stream.
\flowgraph{\resource{object file} \ar[r] & \toolbox{riscdism} \ar[r] & \resource{disassembly\\listing}}
\seeassembly\seerisc\seeobject
}

\providecommand{\wasmasm}{
\toolsection{wasmasm} is an assembler for the WebAssembly architecture.
It translates assembly code into machine code for WebAssembly targets and stores it in corresponding object files.
The names of all special registers are predefined and evaluate to the corresponding number.
\flowgraph{\resource{WebAssembly assembly\\source code} \ar[r] & \toolbox{wasmasm} \ar[r] & \resource{object file}}
\seeassembly\seewasm\seeobject
}

\providecommand{\wasmdism}{
\toolsection{wasmdism} is a disassembler for the WebAssembly architecture.
It translates machine code from object files targeting WebAssembly targets into assembly code and writes it to the standard output stream.
\flowgraph{\resource{object file} \ar[r] & \toolbox{wasmdism} \ar[r] & \resource{disassembly\\listing}}
\seeassembly\seewasm\seeobject
}

% linker tools

\providecommand{\linklib}{
\toolsection{linklib} is an object file combiner.
It creates a static library file by combining all object files given to it into a single one.
\flowgraph{\resource{object files} \ar[r] & \toolbox{linklib} \ar[r] & \resource{library file}}
\seeobject
}

\providecommand{\linkbin}{
\toolsection{linkbin} is a linker for plain binary files.
It links all object files given to it into a single image and stores it in a binary file that begins with the first linked section.
It also creates a map file that lists the address, type, name and size of all used sections.
The filename extension of the resulting binary file can be specified by putting it into a constant data section called \texttt{\_extension}.
\flowgraph{\resource{object files} \ar[r] & \toolbox{linkbin} \ar[r] \ar[d] & \resource{binary file} \\ & \resource{map file}}
\seeobject
}

\providecommand{\linkmem}{
\toolsection{linkmem} is a linker for plain binary files partitioned into random-access and read-only memory.
It links all object files given to it into two distinct images, one for data sections and one for code and constant data sections, and stores each image in a binary file that begins with the first linked section of the corresponding type.
It also creates a map file that lists the address, type, name and size of all used sections.
\flowgraph{\resource{object files} \ar[r] & \toolbox{linkmem} \ar[r] \ar[d] & \resource{RAM file/\\ROM file} \\ & \resource{map file}}
\seeobject
}

\providecommand{\linkprg}{
\toolsection{linkprg} is a linker for GEMDOS executable files.
It links all object files given to it into a single image and stores the image in an Atari GEMDOS executable file~\cite{gemdosfile}.
It also creates a map file that lists the address relative to the text segment, type, name and size of all used sections.
The filename extension of the resulting executable file can be specified by putting it into a constant data section called \texttt{\_extension}.
The GEMDOS executable file format requires all patch patterns of absolute link patches to consist of four full bitmasks with descending offsets.
\flowgraph{\resource{object files} \ar[r] & \toolbox{linkprg} \ar[r] \ar[d] & \resource{executable file} \\ & \resource{map file}}
\seeobject
}

\providecommand{\linkhex}{
\toolsection{linkhex} is a linker for Intel HEX files.
It links all code sections of the object files given to it into single image and stores the image in an Intel HEX file~\cite{hexfile} that begins with the first linked section.
It also creates a map file that lists the address, type, name and size of all used sections.
\flowgraph{\resource{object files} \ar[r] & \toolbox{linkhex} \ar[r] \ar[d] & \resource{HEX file} \\ & \resource{map file}}
\seeobject
}

\providecommand{\mapsearch}{
\toolsection{mapsearch} is a debugging tool.
It searches map files generated by linker tools for the name of a binary section that encompasses a memory address read from the standard input stream.
If additionally provided with one or more object files, it also stores an excerpt thereof in a separate object file called map search result which only contains the identified binary section for disassembling purposes.
\flowgraph{& \resource{map files/\\object files} \ar[d] \\ \resource{memory\\address} \ar[r] & \toolbox{mapsearch} \ar[r] \ar[d] & \resource{section name/\\relative offset} \\ & \resource{object file\\excerpt}}
\seeobject
}

\renewcommand{\seeavr}{}

\startchapter{AVR}{AVR Hardware Architecture Support}{avr}
{This \documentation{} describes how the \ecs{} supports the AVR hardware architecture.
This includes information about the assembler, disassembler, and the various compilers featured by the \ecs{} as well as the interoperability between these tools.}

\section{Introduction}

The \ecs{} features various compilers, an assembler, and a disassembler that target the AVR hardware architecture by Atmel.
Figure~\ref{fig:avrdataflow} shows the data flow in-between these tools.

\begin{figure}
\flowgraph{
\resource{intermediate\\code} \ar[d] & & \resource{assembly\\source code} \ar[d] \\
\converter{AVR\\Generator} \ar[r] \ar[rd] \ar[d] & \resource{assembly\\listing} \ar[r] & \converter{AVR\\Assembler} \ar[ld] \\
\resource{debugging\\information} & \resource{object file} \ar[d] \\
& \converter{AVR\\Disassembler} \ar[d] \\
& \resource{disassembly\\listing} \\
}\caption{Data flow within the tools targeting the AVR architecture}
\label{fig:avrdataflow}
\end{figure}

All compilers targeting the AVR architecture translate their programs using an intermediate code representation.
The AVR generator is able to translate the intermediate code representation of a program into machine code for AVR processors.
It stores the resulting binary code and data in so-called object files.
Additionally, the generator is able to create an assembly code listing of the machine code for debugging purposes.
This assembly code listing can also be processed by the assembler yielding exactly the same object file.
The disassembler is able to open object files and print a human-readable disassembly listing of their contents.
\seeobject\seecode

\section{Instruction Set}

Tools targeting the AVR architecture support the instruction set listed in Table~\ref{tab:avrset} and use the same assembly syntax as predefined by Atmel~\cite{avr:instructionset}.
\seeassembly

\instructionset{avr}{Supported AVR instruction set}{5}{6}

\section{Calling Convention}\index{Calling convention!of AVR}

The machine code generator and runtime support for the AVR architecture as provided by the \ecs{} use the following calling convention in order to enable interoperability.
In general, the order of memory accesses to values that consist of several octets is most significant byte first or big-endian respectively.

\subsection{Stack Operations}

Arguments for functions are in general passed using the stack according to the intermediate code specification.
See \Documentation{}~\documentationref{code}{Intermediate Code Representation} for more information about the role of the stack.
Function arguments are pushed on the stack in reverse order and cleaned by the caller.
Because the AVR instructions \texttt{push} and \texttt{pop} modify the stack register in a different order with respect to the one defined by the intermediate code,
all data on the stack is accessed using a one octet displacement.

\subsection{Floating-Point Support}

The AVR architecture does not support any floating-point operations natively.
The AVR runtime support has currently no software emulation for floating-point operations.

\subsection{Register Mapping}

The special-purpose registers defined by the intermediate code representation are mapped to their corresponding physical registers in the following way:

\begin{itemize}

\item Result Register\alignright\texttt{\$res}\nopagebreak

The intermediate code result register \texttt{\$res} is mapped to AVR registers \texttt{r0} up to \texttt{r7} depending on the size of the actual return type.

\item Stack Pointer Register\alignright\texttt{\$sp}\nopagebreak

The intermediate code stack pointer register \texttt{\$sp} is mapped to AVR registers \texttt{SPL} and \texttt{SPH} respectively.

\item Frame Pointer Register\alignright\texttt{\$fp}\nopagebreak

The intermediate code frame pointer register \texttt{\$fp} is mapped to AVR registers \texttt{r24} and \texttt{r25} respectively.

\item Link Register\alignright\texttt{\$lnk}\nopagebreak

The intermediate code link register \texttt{\$lnk} is not supported.

\end{itemize}

All other intermediate code registers are mapped as needed to the remaining physical registers.
Their contents and mapping are therefore considered volatile across function calls.

\section{Runtime Support}\index{Runtime support!for AVR}

The \ecs{} provides runtime support for the AVR architecture and runtime environments based on this hardware architecture in object files.
Users targeting a specific runtime environment have to use an appropriate linker together with these object files in order create an executable program.
This section gives information about all supported runtime environments based on the AVR hardware architecture as well as the required combination of linker and object files.

Basic architectural runtime support is provided by the object file \objfile{avr\-run}.
Users should always include this object file during linking regardless of the actual target runtime environment.
All other object files given to the linker should target the same hardware architecture.

The \ecs{} additionally supports remote execution of programs targeting AVR processors.
The runtime support is stored in the \objfile{avr\-re\-mote} object file.
It synchronizes the start of a program and its result with a host machine using the standard output stream.

Programs written in \cpp{} need additional runtime support stored in the \libfile{cpp\-avr\-run} library file.
Programs written in Oberon need additional runtime support stored in the \libfile{ob\-avr\-run} library file.
\seecpp\seeoberon

\subsection{ATmega32 Microcontrollers}

Programs targeting the ATmega32 microcontroller by Atmel~\cite{atmega32} are created using the \tool{link\-hex} linker tool.
It creates an Intel HEX file~\cite{hexfile} for programming the microcontroller if provided with the runtime support stored in \objfile{at\-mega32\-run} object file.
Calling the \tool{ecsd} utility tool using the \environment{at\-mega32} target environment achieves the same result.
The USART communication device is configured to use a baud rate of 115200~bps with eight data bits, no parity, and one stop bit.

\subsection{ATmega328 Microcontrollers}

Programs targeting the ATmega328 microcontroller by Atmel~\cite{atmega328} are created using the \tool{link\-hex} linker tool.
It creates an Intel HEX file~\cite{hexfile} for programming the microcontroller if provided with the runtime support stored in \objfile{at\-mega328\-run} object file.
Calling the \tool{ecsd} utility tool using the \environment{at\-mega328} target environment achieves the same result.
The USART communication device is configured to use a baud rate of 115200~bps with eight data bits, no parity, and one stop bit.

\subsection{ATmega8515 Microcontrollers}

Programs targeting the ATmega8515 microcontroller by Atmel~\cite{atmega8515} are created using the \tool{link\-hex} linker tool.
It creates an Intel HEX file~\cite{hexfile} for programming the microcontroller if provided with the runtime support stored in \objfile{at\-mega8515\-run} object file.
Calling the \tool{ecsd} utility tool using the \environment{at\-mega8515} target environment achieves the same result.
The USART communication device is configured to use a baud rate of 115200~bps with eight data bits, no parity, and one stop bit.

\section{AVR Tools}

The \ecs{} provides the following tools that are able to process object files targeting the AVR hardware architecture.
\interface

\cdavr
\cppavr
\falavr
\obavr
\avrasm
\avrdism
\linkhex

\concludechapter

// AVR32 instruction set definitions
// Copyright (C) Florian Negele

// This file is part of the Eigen Compiler Suite.

// The ECS is free software: you can redistribute it and/or modify
// it under the terms of the GNU General Public License as published by
// the Free Software Foundation, either version 3 of the License, or
// (at your option) any later version.

// The ECS is distributed in the hope that it will be useful,
// but WITHOUT ANY WARRANTY; without even the implied warranty of
// MERCHANTABILITY or FITNESS FOR A PARTICULAR PURPOSE.  See the
// GNU General Public License for more details.

// You should have received a copy of the GNU General Public License
// along with the ECS.  If not, see <https://www.gnu.org/licenses/>.

#ifndef INSTR
	#define INSTR(mnem, code, mask, type0, type1, type2, type3, type4)
#endif

#ifndef MNEM
	#define MNEM(name, mnem, ...)
#endif

#ifndef TYPE
	#define TYPE(type)
#endif

// mnemonics

MNEM (abs,            ABS,           Absolute Value)
MNEM (acall,          ACALL,         Application Call)
MNEM (acr,            ACR,           Add Carry to Register)
MNEM (adc,            ADC,           Add with Carry)
MNEM (add,            ADD,           Add without Carry)
MNEM (addabs,         ADDABS,        Add Absolute Value)
MNEM (addal,          ADDAL,         Conditional Add Always)
MNEM (addcc,          ADDCC,         Conditional Add if Carry Cleared)
MNEM (addcs,          ADDCS,         Conditional Add if Carry Set)
MNEM (addeq,          ADDEQ,         Conditional Add if Equal)
MNEM (addge,          ADDGE,         Conditional Add if Greater than or Equal)
MNEM (addgt,          ADDGT,         Conditional Add if Greater Than)
MNEM (addhh.w,        ADDHHW,        Add Halfwords into Word)
MNEM (addhi,          ADDHI,         Conditional Add if Higher)
MNEM (addhs,          ADDHS,         Conditional Add if Higher or Same)
MNEM (addle,          ADDLE,         Conditional Add if Less than or Equal)
MNEM (addlo,          ADDLO,         Conditional Add if Lower)
MNEM (addls,          ADDLS,         Conditional Add if Lower or Same)
MNEM (addlt,          ADDLT,         Conditional Add if Less Than)
MNEM (addmi,          ADDMI,         Conditional Add if Negative)
MNEM (addne,          ADDNE,         Conditional Add if Not Equal)
MNEM (addpl,          ADDPL,         Conditional Add if Positive)
MNEM (addqs,          ADDQS,         Conditional Add if Saturated)
MNEM (addvc,          ADDVC,         Conditional Add if Overflow Cleared)
MNEM (addvs,          ADDVS,         Conditional Add if Overflow Set)
MNEM (and,            AND,           Logical AND)
MNEM (andal,          ANDAL,         Conditional AND Always)
MNEM (andcc,          ANDCC,         Conditional AND if Carry Cleared)
MNEM (andcs,          ANDCS,         Conditional AND if Carry Set)
MNEM (andeq,          ANDEQ,         Conditional AND if Equal)
MNEM (andge,          ANDGE,         Conditional AND if Greater than or Equal)
MNEM (andgt,          ANDGT,         Conditional AND if Greater Than)
MNEM (andh,           ANDH,          Logical AND into High Half of Register)
MNEM (andhi,          ANDHI,         Conditional AND if Higher)
MNEM (andhs,          ANDHS,         Conditional AND if Higher or Same)
MNEM (andl,           ANDL,          Logical AND into Low Half of Register)
MNEM (andle,          ANDLE,         Conditional AND if Less than or Equal)
MNEM (andlo,          ANDLO,         Conditional AND if Lower)
MNEM (andls,          ANDLS,         Conditional AND if Lower or Same)
MNEM (andlt,          ANDLT,         Conditional AND if Less Than)
MNEM (andmi,          ANDMI,         Conditional AND if Negative)
MNEM (andn,           ANDN,          Logical AND NOT)
MNEM (andne,          ANDNE,         Conditional AND if Not Equal)
MNEM (andpl,          ANDPL,         Conditional AND if Positive)
MNEM (andqs,          ANDQS,         Conditional AND if Saturated)
MNEM (andvc,          ANDVC,         Conditional AND if Overflow Cleared)
MNEM (andvs,          ANDVS,         Conditional AND if Overflow Set)
MNEM (asr,            ASR,           Arithmetic Shift Right)
MNEM (bfexts,         BFEXTS,        Bitfield Extract and Sign-extend)
MNEM (bfextu,         BFEXTU,        Bitfield Extract and Zero-extend)
MNEM (bfins,          BFINS,         Bitfield Insert)
MNEM (bld,            BLD,           Bit Load from Register to C and Z)
MNEM (bral,           BRAL,          Branch Always)
MNEM (brcc,           BRCC,          Branch if Carry Cleared)
MNEM (brcs,           BRCS,          Branch if Carry Set)
MNEM (breakpoint,     BREAKPOINT,    Software Debug Breakpoint)
MNEM (breq,           BREQ,          Branch if Equal)
MNEM (brev,           BREV,          Bit Reverse)
MNEM (brge,           BRGE,          Branch if Greater than or Equal)
MNEM (brgt,           BRGT,          Branch if Greater Than)
MNEM (brhi,           BRHI,          Branch if Higher)
MNEM (brhs,           BRHS,          Branch if Higher or Same)
MNEM (brle,           BRLE,          Branch if Less than or Equal)
MNEM (brlo,           BRLO,          Branch if Lower)
MNEM (brls,           BRLS,          Branch if Lower or Same)
MNEM (brlt,           BRLT,          Branch if Less Than)
MNEM (brmi,           BRMI,          Branch if Less Negative)
MNEM (brne,           BRNE,          Branch if Not Equal)
MNEM (brpl,           BRPL,          Branch if Less Positive)
MNEM (brqs,           BRQS,          Branch if Saturated)
MNEM (brvc,           BRVC,          Branch if Overflow Cleared)
MNEM (brvs,           BRVS,          Branch if Overflow Set)
MNEM (bst,            BST,           Copy C to Register Bit)
MNEM (cache,          CACHE,         Perform Cache Control Operation)
MNEM (casts.b,        CASTSB,        Typecast Byte to Signed Word)
MNEM (casts.h,        CASTSH,        Typecast Halfword to Signed Word)
MNEM (castu.b,        CASTUB,        Typecast Byte to Unsigned Word)
MNEM (castu.h,        CASTUH,        Typecast Halfword to Unsigned Word)
MNEM (cbr,            CBR,           Clear Bit in Register)
MNEM (clz,            CLZ,           Count Leading Zeros)
MNEM (com,            COM,           One`s Compliment)
MNEM (cop,            COP,           Coprocessor Operation)
MNEM (cp.b,           CPB,           Compare Byte)
MNEM (cp.h,           CPH,           Compare Halfword)
MNEM (cp.w,           CPW,           Compare Word)
MNEM (cpc,            CPC,           Compare with Carry)
MNEM (csrf,           CSRF,          Clear Status Register Flag)
MNEM (csrfcz,         CSRFCZ,        Copy Status Register Flag to C and Z)
MNEM (divs,           DIVS,          Signed Divide)
MNEM (divu,           DIVU,          Unsigned Divide)
MNEM (eor,            EOR,           Logical EOR)
MNEM (eoral,          EORAL,         Conditional Logical EOR Always)
MNEM (eorcc,          EORCC,         Conditional Logical EOR if Carry Cleared)
MNEM (eorcs,          EORCS,         Conditional Logical EOR if Carry Set)
MNEM (eoreq,          EOREQ,         Conditional Logical EOR if Equal)
MNEM (eorge,          EORGE,         Conditional Logical EOR if Greater than or Equal)
MNEM (eorgt,          EORGT,         Conditional Logical EOR if Greater Than)
MNEM (eorh,           EORH,          Logical EOR into High Half of Register)
MNEM (eorhi,          EORHI,         Conditional Logical EOR if Higher)
MNEM (eorhs,          EORHS,         Conditional Logical EOR if Higher or Same)
MNEM (eorl,           EORL,          Logical EOR into Low Half of Register)
MNEM (eorle,          EORLE,         Conditional Logical EOR if Less than or Equal)
MNEM (eorlo,          EORLO,         Conditional Logical EOR if Lower)
MNEM (eorls,          EORLS,         Conditional Logical EOR if Lower or Same)
MNEM (eorlt,          EORLT,         Conditional Logical EOR if Less Than)
MNEM (eormi,          EORMI,         Conditional Logical EOR if Negative)
MNEM (eorne,          EORNE,         Conditional Logical EOR if Not Equal)
MNEM (eorpl,          EORPL,         Conditional Logical EOR if Positive)
MNEM (eorqs,          EORQS,         Conditional Logical EOR if Saturated)
MNEM (eorvc,          EORVC,         Conditional Logical EOR if Overflow Cleared)
MNEM (eorvs,          EORVS,         Conditional Logical EOR if Overflow Set)
MNEM (frs,            FRS,           Flush Return Stack)
MNEM (icall,          ICALL,         Indirect Call to Subroutine)
MNEM (incjosp,        INCJOSP,       Increment Java Operand Stack Pointer)
MNEM (ld.d,           LDD,           Load Doubleword)
MNEM (ld.sb,          LDSB,          Load Sign-extended Byte)
MNEM (ld.sbal,        LDSBAL,        Conditionally Load Sign-extended Byte Always)
MNEM (ld.sbcc,        LDSBCC,        Conditionally Load Sign-extended Byte if Carry Cleared)
MNEM (ld.sbcs,        LDSBCS,        Conditionally Load Sign-extended Byte if Carry Set)
MNEM (ld.sbeq,        LDSBEQ,        Conditionally Load Sign-extended Byte if Equal)
MNEM (ld.sbge,        LDSBGE,        Conditionally Load Sign-extended Byte if Greater than or Equal)
MNEM (ld.sbgt,        LDSBGT,        Conditionally Load Sign-extended Byte if Greater Than)
MNEM (ld.sbhi,        LDSBHI,        Conditionally Load Sign-extended Byte if Higher)
MNEM (ld.sbhs,        LDSBHS,        Conditionally Load Sign-extended Byte if Higher or Same)
MNEM (ld.sble,        LDSBLE,        Conditionally Load Sign-extended Byte if Less than or Equal)
MNEM (ld.sblo,        LDSBLO,        Conditionally Load Sign-extended Byte if Lower)
MNEM (ld.sbls,        LDSBLS,        Conditionally Load Sign-extended Byte if Lower or Same)
MNEM (ld.sblt,        LDSBLT,        Conditionally Load Sign-extended Byte if Less Than)
MNEM (ld.sbmi,        LDSBMI,        Conditionally Load Sign-extended Byte if Negative)
MNEM (ld.sbne,        LDSBNE,        Conditionally Load Sign-extended Byte if Not Equal)
MNEM (ld.sbpl,        LDSBPL,        Conditionally Load Sign-extended Byte if Positive)
MNEM (ld.sbqs,        LDSBQS,        Conditionally Load Sign-extended Byte if Saturated)
MNEM (ld.sbvc,        LDSBVC,        Conditionally Load Sign-extended Byte if Overflow Cleared)
MNEM (ld.sbvs,        LDSBVS,        Conditionally Load Sign-extended Byte if Overflow Set)
MNEM (ld.sh,          LDSH,          Load Sign-extended Halfword)
MNEM (ld.shal,        LDSHAL,        Conditionally Load Sign-extended Halfword Always)
MNEM (ld.shcc,        LDSHCC,        Conditionally Load Sign-extended Halfword if Carry Cleared)
MNEM (ld.shcs,        LDSHCS,        Conditionally Load Sign-extended Halfword if Carry Set)
MNEM (ld.sheq,        LDSHEQ,        Conditionally Load Sign-extended Halfword if Equal)
MNEM (ld.shge,        LDSHGE,        Conditionally Load Sign-extended Halfword if Greater than or Equal)
MNEM (ld.shgt,        LDSHGT,        Conditionally Load Sign-extended Halfword if Greater Than)
MNEM (ld.shhi,        LDSHHI,        Conditionally Load Sign-extended Halfword if Higher)
MNEM (ld.shhs,        LDSHHS,        Conditionally Load Sign-extended Halfword if Higher or Same)
MNEM (ld.shle,        LDSHLE,        Conditionally Load Sign-extended Halfword if Less than or Equal)
MNEM (ld.shlo,        LDSHLO,        Conditionally Load Sign-extended Halfword if Lower)
MNEM (ld.shls,        LDSHLS,        Conditionally Load Sign-extended Halfword if Lower or Same)
MNEM (ld.shlt,        LDSHLT,        Conditionally Load Sign-extended Halfword if Less Than)
MNEM (ld.shmi,        LDSHMI,        Conditionally Load Sign-extended Halfword if Negative)
MNEM (ld.shne,        LDSHNE,        Conditionally Load Sign-extended Halfword if Not Equal)
MNEM (ld.shpl,        LDSHPL,        Conditionally Load Sign-extended Halfword if Positive)
MNEM (ld.shqs,        LDSHQS,        Conditionally Load Sign-extended Halfword if Saturated)
MNEM (ld.shvc,        LDSHVC,        Conditionally Load Sign-extended Halfword if Overflow Cleared)
MNEM (ld.shvs,        LDSHVS,        Conditionally Load Sign-extended Halfword if Overflow Set)
MNEM (ld.ub,          LDUB,          Load Zero-extended Byte)
MNEM (ld.ubal,        LDUBAL,        Conditionally Load Zero-extended Byte Always)
MNEM (ld.ubcc,        LDUBCC,        Conditionally Load Zero-extended Byte if Carry Cleared)
MNEM (ld.ubcs,        LDUBCS,        Conditionally Load Zero-extended Byte if Carry Set)
MNEM (ld.ubeq,        LDUBEQ,        Conditionally Load Zero-extended Byte if Equal)
MNEM (ld.ubge,        LDUBGE,        Conditionally Load Zero-extended Byte if Greater than or Equal)
MNEM (ld.ubgt,        LDUBGT,        Conditionally Load Zero-extended Byte if Greater Than)
MNEM (ld.ubhi,        LDUBHI,        Conditionally Load Zero-extended Byte if Higher)
MNEM (ld.ubhs,        LDUBHS,        Conditionally Load Zero-extended Byte if Higher or Same)
MNEM (ld.uble,        LDUBLE,        Conditionally Load Zero-extended Byte if Less than or Equal)
MNEM (ld.ublo,        LDUBLO,        Conditionally Load Zero-extended Byte if Lower)
MNEM (ld.ubls,        LDUBLS,        Conditionally Load Zero-extended Byte if Lower or Same)
MNEM (ld.ublt,        LDUBLT,        Conditionally Load Zero-extended Byte if Less Than)
MNEM (ld.ubmi,        LDUBMI,        Conditionally Load Zero-extended Byte if Negative)
MNEM (ld.ubne,        LDUBNE,        Conditionally Load Zero-extended Byte if Not Equal)
MNEM (ld.ubpl,        LDUBPL,        Conditionally Load Zero-extended Byte if Positive)
MNEM (ld.ubqs,        LDUBQS,        Conditionally Load Zero-extended Byte if Saturated)
MNEM (ld.ubvc,        LDUBVC,        Conditionally Load Zero-extended Byte if Overflow Cleared)
MNEM (ld.ubvs,        LDUBVS,        Conditionally Load Zero-extended Byte if Overflow Set)
MNEM (ld.uh,          LDUH,          Load Zero-extended Halfword)
MNEM (ld.uhal,        LDUHAL,        Conditionally Load Zero-extended Halfword Always)
MNEM (ld.uhcc,        LDUHCC,        Conditionally Load Zero-extended Halfword if Carry Cleared)
MNEM (ld.uhcs,        LDUHCS,        Conditionally Load Zero-extended Halfword if Carry Set)
MNEM (ld.uheq,        LDUHEQ,        Conditionally Load Zero-extended Halfword if Equal)
MNEM (ld.uhge,        LDUHGE,        Conditionally Load Zero-extended Halfword if Greater than or Equal)
MNEM (ld.uhgt,        LDUHGT,        Conditionally Load Zero-extended Halfword if Greater Than)
MNEM (ld.uhhi,        LDUHHI,        Conditionally Load Zero-extended Halfword if Higher)
MNEM (ld.uhhs,        LDUHHS,        Conditionally Load Zero-extended Halfword if Higher or Same)
MNEM (ld.uhle,        LDUHLE,        Conditionally Load Zero-extended Halfword if Less than or Equal)
MNEM (ld.uhlo,        LDUHLO,        Conditionally Load Zero-extended Halfword if Lower)
MNEM (ld.uhls,        LDUHLS,        Conditionally Load Zero-extended Halfword if Lower or Same)
MNEM (ld.uhlt,        LDUHLT,        Conditionally Load Zero-extended Halfword if Less Than)
MNEM (ld.uhmi,        LDUHMI,        Conditionally Load Zero-extended Halfword if Negative)
MNEM (ld.uhne,        LDUHNE,        Conditionally Load Zero-extended Halfword if Not Equal)
MNEM (ld.uhpl,        LDUHPL,        Conditionally Load Zero-extended Halfword if Positive)
MNEM (ld.uhqs,        LDUHQS,        Conditionally Load Zero-extended Halfword if Saturated)
MNEM (ld.uhvc,        LDUHVC,        Conditionally Load Zero-extended Halfword if Overflow Cleared)
MNEM (ld.uhvs,        LDUHVS,        Conditionally Load Zero-extended Halfword if Overflow Set)
MNEM (ld.w,           LDW,           Load Word)
MNEM (ld.wal,         LDWAL,         Conditionally Word Always)
MNEM (ld.wcc,         LDWCC,         Conditionally Word if Carry Cleared)
MNEM (ld.wcs,         LDWCS,         Conditionally Word if Carry Set)
MNEM (ld.weq,         LDWEQ,         Conditionally Word if Equal)
MNEM (ld.wge,         LDWGE,         Conditionally Word if Greater than or Equal)
MNEM (ld.wgt,         LDWGT,         Conditionally Word if Greater Than)
MNEM (ld.whi,         LDWHI,         Conditionally Word if Higher)
MNEM (ld.whs,         LDWHS,         Conditionally Word if Higher or Same)
MNEM (ld.wle,         LDWLE,         Conditionally Word if Less than or Equal)
MNEM (ld.wlo,         LDWLO,         Conditionally Word if Lower)
MNEM (ld.wls,         LDWLS,         Conditionally Word if Lower or Same)
MNEM (ld.wlt,         LDWLT,         Conditionally Word if Less Than)
MNEM (ld.wmi,         LDWMI,         Conditionally Word if Negative)
MNEM (ld.wne,         LDWNE,         Conditionally Word if Not Equal)
MNEM (ld.wpl,         LDWPL,         Conditionally Word if Positive)
MNEM (ld.wqs,         LDWQS,         Conditionally Word if Saturated)
MNEM (ld.wvc,         LDWVC,         Conditionally Word if Overflow Cleared)
MNEM (ld.wvs,         LDWVS,         Conditionally Word if Overflow Set)
MNEM (ldc.d,          LDCD,          Load Coprocessor Doubleword)
MNEM (ldc.w,          LDCW,          Load Coprocessor Word)
MNEM (ldc0.d,         LDC0D,         Load Doubleword into Coprocessor 0)
MNEM (ldc0.w,         LDC0W,         Load Word into Coprocessor 0)
MNEM (lddpc,          LDDPC,         Load PC-relative with Displacement)
MNEM (lddsp,          LDDSP,         Load SP-relative with Displacement)
MNEM (ldins.b,        LDINSB,        Load and Insert Byte into register)
MNEM (ldins.h,        LDINSH,        Load and Insert Halfword into register)
MNEM (ldswp.sh,       LDSWPSH,       Load and Swap Sign-extended Halfword)
MNEM (ldswp.uh,       LDSWPUH,       Load and Swap Zero-extended Halfword)
MNEM (ldswp.w,        LDSWPW,        Load and Swap Word)
MNEM (lsl,            LSL,           Logical Shift Left)
MNEM (lsr,            LSR,           Logical Shift Right)
MNEM (mac,            MAC,           Multiply Accumulate)
MNEM (machh.d,        MACHHD,        Multiply Halfwords and Accumulate in Doubleword)
MNEM (machh.w,        MACHHW,        Multiply Halfwords and Accumulate in Word)
MNEM (macs.d,         MACSD,         Multiply Accumulate Signed)
MNEM (macsathh.w,     MACSATHHW,     Multiply-Accumulate Halfwords with Saturation into Word)
MNEM (macu.d,         MACUD,         Multiply Accumulate Unsigned)
MNEM (macwh.d,        MACWHD,        Multiply Word with Halfword and Accumulate in Doubleword)
MNEM (max,            MAX,           Return Maximum Value)
MNEM (mcall,          MCALL,         Subroutine Call)
MNEM (memc,           MEMC,          Clear Bit in Memory)
MNEM (mems,           MEMS,          Set Bit in Memory)
MNEM (memt,           MEMT,          Toggle Bit in Memory)
MNEM (mfdr,           MFDR,          Move from Debug Register)
MNEM (mfsr,           MFSR,          Move from System Register)
MNEM (min,            MIN,           Return Minimum Value)
MNEM (mov,            MOV,           Move Data into Register)
MNEM (moval,          MOVAL,         Conditional Move Register Always)
MNEM (movcc,          MOVCC,         Conditional Move Register if Carry Cleared)
MNEM (movcs,          MOVCS,         Conditional Move Register if Carry Set)
MNEM (moveq,          MOVEQ,         Conditional Move Register if Equal)
MNEM (movge,          MOVGE,         Conditional Move Register if Greater than or Equal)
MNEM (movgt,          MOVGT,         Conditional Move Register if Greater Than)
MNEM (movh,           MOVH,          Move Data into High Halfword of Register)
MNEM (movhi,          MOVHI,         Conditional Move Register if Higher)
MNEM (movhs,          MOVHS,         Conditional Move Register if Higher or Same)
MNEM (movl,           MOVL,          Move Data into Low Halfword of Register)
MNEM (movle,          MOVLE,         Conditional Move Register if Less than or Equal)
MNEM (movlo,          MOVLO,         Conditional Move Register if Lower)
MNEM (movls,          MOVLS,         Conditional Move Register if Lower or Same)
MNEM (movlt,          MOVLT,         Conditional Move Register if Less Than)
MNEM (movmi,          MOVMI,         Conditional Move Register if Negative)
MNEM (movne,          MOVNE,         Conditional Move Register if Not Equal)
MNEM (movpl,          MOVPL,         Conditional Move Register if Positive)
MNEM (movqs,          MOVQS,         Conditional Move Register if Saturated)
MNEM (movvc,          MOVVC,         Conditional Move Register if Overflow Cleared)
MNEM (movvs,          MOVVS,         Conditional Move Register if Overflow Set)
MNEM (mtdr,           MTDR,          Move to Debug Register)
MNEM (mtsr,           MTSR,          Move to System Register)
MNEM (mul,            MUL,           Multiply)
MNEM (mulhh.w,        MULHHW,        Multiply Halfword with Halfword)
MNEM (mulnhh.w,       MULNHHW,       Multiply Halfword with Negated Halfword)
MNEM (mulnwh.d,       MULNWHD,       Multiply Word with Negated Halfword)
MNEM (muls.d,         MULSD,         Multiply Signed)
MNEM (mulsathh.h,     MULSATHHH,     Multiply Halfwords with Saturation into Halfword)
MNEM (mulsathh.w,     MULSATHHW,     Multiply Halfwords with Saturation into Word)
MNEM (mulsatrndhh.h,  MULSATRNDHHH,  Multiply Halfwords with Saturation and Rounding into Halfword)
MNEM (mulsatrndhh.w,  MULSATRNDHHW,  Multiply Halfwords with Saturation and Rounding into Word)
MNEM (mulsatwh.w,     MULSATWHW,     Multiply Word and Halfword with Saturation into Word)
MNEM (mulu.d,         MULUD,         Multiply Unsigned)
MNEM (mulwh.d,        MULWHD,        Multiply Word with Halfword)
MNEM (musfr,          MUSFR,         Copy Register to Status Register)
MNEM (mustr,          MUSTR,         Copy Status Register to Register)
MNEM (mvcr.d,         MVCRD,         Move Doubleword Coprocessor Register to Register file)
MNEM (mvcr.w,         MVCRW,         Move Word Coprocessor Register to Register file)
MNEM (mvrc.d,         MVRCD,         Move Doubleword Register file Register to Coprocessor Register)
MNEM (mvrc.w,         MVRCW,         Move Word Coprocessor Word Register to Register file)
MNEM (neg,            NEG,           Two`s Complement)
MNEM (nop,            NOP,           No Operation)
MNEM (or,             OR,            Logical OR)
MNEM (oral,           ORAL,          Conditional Logical OR Always)
MNEM (orcc,           ORCC,          Conditional Logical OR if Carry Cleared)
MNEM (orcs,           ORCS,          Conditional Logical OR if Carry Set)
MNEM (oreq,           OREQ,          Conditional Logical OR if Equal)
MNEM (orge,           ORGE,          Conditional Logical OR if Greater than or Equal)
MNEM (orgt,           ORGT,          Conditional Logical OR if Greater Than)
MNEM (orh,            ORH,           Logical OR into High Half of Register)
MNEM (orhi,           ORHI,          Conditional Logical OR if Higher)
MNEM (orhs,           ORHS,          Conditional Logical OR if Higher or Same)
MNEM (orl,            ORL,           Logical OR into Low Half of Register)
MNEM (orle,           ORLE,          Conditional Logical OR if Less than or Equal)
MNEM (orlo,           ORLO,          Conditional Logical OR if Lower)
MNEM (orls,           ORLS,          Conditional Logical OR if Lower or Same)
MNEM (orlt,           ORLT,          Conditional Logical OR if Less Than)
MNEM (ormi,           ORMI,          Conditional Logical OR if Negative)
MNEM (orne,           ORNE,          Conditional Logical OR if Not Equal)
MNEM (orpl,           ORPL,          Conditional Logical OR if Positive)
MNEM (orqs,           ORQS,          Conditional Logical OR if Saturated)
MNEM (orvc,           ORVC,          Conditional Logical OR if Overflow Cleared)
MNEM (orvs,           ORVS,          Conditional Logical OR if Overflow Set)
MNEM (pabs.sb,        PABSSB,        Packed Absolute Value of Signed Bytes)
MNEM (pabs.sh,        PABSSH,        Packed Absolute Value of Signed Halfwords)
MNEM (packsh.sb,      PACKSHSB,      Pack Signed Halfwords to Signed Bytes)
MNEM (packsh.ub,      PACKSHUB,      Pack Signed Halfwords to Unsigned Bytes)
MNEM (packw.sh,       PACKWSHS,      Pack Words to Signed Halfwords)
MNEM (padd.b,         PADDB,         Packed Addition on Bytes)
MNEM (padd.h,         PADDH,         Packed Addition on Halfwords)
MNEM (paddh.sh,       PADDHSH,       Packed Addition with Halving on Signed Halfwords)
MNEM (paddh.ub,       PADDHUB,       Packed Addition with Halving on Unsigned Bytes)
MNEM (padds.sb,       PADDSSB,       Packed Addition with Saturation on Signed Bytes)
MNEM (padds.sh,       PADDSSH,       Packed Addition with Saturation on Signed Halfwords)
MNEM (padds.ub,       PADDSUB,       Packed Addition with Saturation on Unsigned Bytes)
MNEM (padds.uh,       PADDSUH,       Packed Addition with Saturation on Unsigned Halfwords)
MNEM (paddsub.h,      PADDSUBH,      Packed Halfword Addition and Subtraction)
MNEM (paddsubh.sh,    PADDSUBHSH,    Packed Halfword Addition and Subtraction with Halving)
MNEM (paddsubs.sh,    PADDSUBSSH,    Packed Signed Halfword Addition and Subtraction with Saturation)
MNEM (paddsubs.uh,    PADDSUBSUH,    Packed Unsigned Halfword Addition and Subtraction with Saturation)
MNEM (paddx.h,        PADDXH,        Packed Halfword Addition with Crossed Operand)
MNEM (paddxh.sh,      PADDXHSH,      Packed Signed Halfword Addition with Crossed Operand and Halving)
MNEM (paddxs.sh,      PADDXSSH,      Packed Signed Halfword Addition with Crossed Operand and Saturation)
MNEM (paddxs.uh,      PADDXSUH,      Packed Unsigned Halfword Addition with Crossed Operand and Saturation)
MNEM (pasr.b,         PASRB,         Packed Arithmetic Shift Right on Bytes)
MNEM (pasr.h,         PASRH,         Packed Arithmetic Shift Right on Halfwords)
MNEM (pavg.sh,        PAVGSH,        Packed Average of Signed Halfwords)
MNEM (pavg.ub,        PAVGUB,        Packed Average of Unsigned Bytes)
MNEM (plsl.b,         PLSLB,         Packed Logical Shift Left on Bytes)
MNEM (plsl.h,         PLSLH,         Packed Logical Shift Left on Halfwords)
MNEM (plsr.b,         PLSRB,         Packed Logical Shift Right on Bytes)
MNEM (plsr.h,         PLSRH,         Packed Logical Shift Right on Halfwords)
MNEM (pmax.sh,        PMAXSH,        Packed Maximum Value of Signed Halfwords)
MNEM (pmax.ub,        PMAXUB,        Packed Maximum Value of Unsigned Bytes)
MNEM (pmin.sh,        PMINSH,        Packed Minimum Value of Signed Halfwords)
MNEM (pmin.ub,        PMINUB,        Packed Minimum Value of Unsigned Bytes)
MNEM (popjc,          POPJC,         Pop Java Context from Frame)
MNEM (pref,           PREF,          Cache Prefetch)
MNEM (psad,           PSAD,          Packed Sum of Absolute Differences)
MNEM (psub.b,         PSUBB,         Packed Subtraction on Bytes)
MNEM (psub.h,         PSUBH,         Packed Subtraction on Halfwords)
MNEM (psubadd.h,      PSUBADDH,      Packed Halfword Subtraction and Addition)
MNEM (psubaddh.sh,    PSUBADDHSH,    Packed Signed Halfword Subtraction and Addition with Halving)
MNEM (psubadds.sh,    PSUBADDSSH,    Packed Halfword Subtraction and Addition with Saturation on Signed Halfwords)
MNEM (psubadds.uh,    PSUBADDSUH,    Packed Halfword Subtraction and Addition with Saturation on Unsigned Halfwords)
MNEM (psubh.sh,       PSUBHSH,       Packed Subtraction with Halving on Signed Halfwords)
MNEM (psubh.ub,       PSUBHUB,       Packed Subtraction with Halving on Unsigned Bytes)
MNEM (psubs.sb,       PSUBSSB,       Packed Subtraction with Saturation on Signed Bytes)
MNEM (psubs.sh,       PSUBSSH,       Packed Subtraction with Saturation on Signed Halfwords)
MNEM (psubs.ub,       PSUBSUB,       Packed Subtraction with Saturation on Unsigned Bytes)
MNEM (psubs.uh,       PSUBSUH,       Packed Subtraction with Saturation on Unsigned Halfwords)
MNEM (psubx.h,        PSUBXH,        Packed Halfword Subtraction with Crossed Operand)
MNEM (psubxh.sh,      PSUBXHSH,      Packed Signed Halfword Subtraction with Crossed Operand and Halving)
MNEM (psubxs.sh,      PSUBXSSH,      Packed Signed Halfword Subtraction with Crossed Operand and Saturation)
MNEM (psubxs.uh,      PSUBXSUH,      Packed Unsigned Halfword Subtraction with Crossed Operand and Saturation)
MNEM (punpcksb.h,     PUNPCKSBH,     Unpack Signed Bytes to Halfwords)
MNEM (punpckub.h,     PUNPCKUBH,     Unpack Unsigned Bytes to Halfwords)
MNEM (pushjc,         PUSHJC,        Push Java Context to Frame)
MNEM (rcall,          RCALL,         Relative Subroutine Call)
MNEM (retal,          RETAL,         Conditional Return from Subroutine Always)
MNEM (retcc,          RETCC,         Conditional Return from Subroutine if Carry Cleared)
MNEM (retcs,          RETCS,         Conditional Return from Subroutine if Carry Set)
MNEM (retd,           RETD,          Return from Debug Mode)
MNEM (rete,           RETE,          Return from Event Handler)
MNEM (reteq,          RETEQ,         Conditional Return from Subroutine if Equal)
MNEM (retge,          RETGE,         Conditional Return from Subroutine if Greater than or Equal)
MNEM (retgt,          RETGT,         Conditional Return from Subroutine if Greater Than)
MNEM (rethi,          RETHI,         Conditional Return from Subroutine if Higher)
MNEM (reths,          RETHS,         Conditional Return from Subroutine if Higher or Same)
MNEM (retj,           RETJ,          Return from Java Trap)
MNEM (retle,          RETLE,         Conditional Return from Subroutine if Less than or Equal)
MNEM (retlo,          RETLO,         Conditional Return from Subroutine if Lower)
MNEM (retls,          RETLS,         Conditional Return from Subroutine if Lower or Same)
MNEM (retlt,          RETLT,         Conditional Return from Subroutine if Less Than)
MNEM (retmi,          RETMI,         Conditional Return from Subroutine if Negative)
MNEM (retne,          RETNE,         Conditional Return from Subroutine if Not Equal)
MNEM (retpl,          RETPL,         Conditional Return from Subroutine if Positive)
MNEM (retqs,          RETQS,         Conditional Return from Subroutine if Saturated)
MNEM (rets,           RETS,          Return from Supervisor Call)
MNEM (retss,          RETSS,         Return from Secure State)
MNEM (retvc,          RETVC,         Conditional Return from Subroutine if Overflow Cleared)
MNEM (retvs,          RETVS,         Conditional Return from Subroutine if Overflow Set)
MNEM (rjmp,           RJMP,          Relative Jump)
MNEM (rol,            ROL,           Rotate Left through Carry)
MNEM (ror,            ROR,           Rotate Right through Carry)
MNEM (rsub,           RSUB,          Reverse Subtract)
MNEM (rsubal,         RSUBAL,        Conditional Reverse Subtract Always)
MNEM (rsubcc,         RSUBCC,        Conditional Reverse Subtract if Carry Cleared)
MNEM (rsubcs,         RSUBCS,        Conditional Reverse Subtract if Carry Set)
MNEM (rsubeq,         RSUBEQ,        Conditional Reverse Subtract if Equal)
MNEM (rsubge,         RSUBGE,        Conditional Reverse Subtract if Greater than or Equal)
MNEM (rsubgt,         RSUBGT,        Conditional Reverse Subtract if Greater Than)
MNEM (rsubhi,         RSUBHI,        Conditional Reverse Subtract if Higher)
MNEM (rsubhs,         RSUBHS,        Conditional Reverse Subtract if Higher or Same)
MNEM (rsuble,         RSUBLE,        Conditional Reverse Subtract if Less than or Equal)
MNEM (rsublo,         RSUBLO,        Conditional Reverse Subtract if Lower)
MNEM (rsubls,         RSUBLS,        Conditional Reverse Subtract if Lower or Same)
MNEM (rsublt,         RSUBLT,        Conditional Reverse Subtract if Less Than)
MNEM (rsubmi,         RSUBMI,        Conditional Reverse Subtract if Negative)
MNEM (rsubne,         RSUBNE,        Conditional Reverse Subtract if Not Equal)
MNEM (rsubpl,         RSUBPL,        Conditional Reverse Subtract if Positive)
MNEM (rsubqs,         RSUBQS,        Conditional Reverse Subtract if Saturated)
MNEM (rsubvc,         RSUBVC,        Conditional Reverse Subtract if Overflow Cleared)
MNEM (rsubvs,         RSUBVS,        Conditional Reverse Subtract if Overflow Set)
MNEM (satadd.h,       SATADDH,       Saturated Add of Halfwords)
MNEM (satadd.w,       SATADDW,       Saturated Add of Words)
MNEM (satrnds,        SATRNDS,       Saturate with Rounding Signed)
MNEM (satrndu,        SATRNDU,       Saturate with Rounding Unsigned)
MNEM (sats,           SATS,          Saturate Signed)
MNEM (satsub.h,       SATSUBH,       Saturated Subtract of Halfwords)
MNEM (satsub.w,       SATSUBW,       Saturated Subtract of Words)
MNEM (satu,           SATU,          Saturate Unsigned)
MNEM (sbc,            SBC,           Subtract with Carry)
MNEM (sbr,            SBR,           Set Bit in Register)
MNEM (scall,          SCALL,         Supervisor Call)
MNEM (scr,            SCR,           Subtract Carry from Register)
MNEM (sleep,          SLEEP,         Set CPU Activity Mode)
MNEM (sral,           SRAL,          Set Register Conditionally Always)
MNEM (srcc,           SRCC,          Set Register Conditionally if Carry Cleared)
MNEM (srcs,           SRCS,          Set Register Conditionally if Carry Set)
MNEM (sreq,           SREQ,          Set Register Conditionally if Equal)
MNEM (srge,           SRGE,          Set Register Conditionally if Greater than or Equal)
MNEM (srgt,           SRGT,          Set Register Conditionally if Greater Than)
MNEM (srhi,           SRHI,          Set Register Conditionally if Higher)
MNEM (srhs,           SRHS,          Set Register Conditionally if Higher or Same)
MNEM (srle,           SRLE,          Set Register Conditionally if Less than or Equal)
MNEM (srlo,           SRLO,          Set Register Conditionally if Lower)
MNEM (srls,           SRLS,          Set Register Conditionally if Lower or Same)
MNEM (srlt,           SRLT,          Set Register Conditionally if Less Than)
MNEM (srmi,           SRMI,          Set Register Conditionally if Negative)
MNEM (srne,           SRNE,          Set Register Conditionally if Not Equal)
MNEM (srpl,           SRPL,          Set Register Conditionally if Positive)
MNEM (srqs,           SRQS,          Set Register Conditionally if Saturated)
MNEM (srvc,           SRVC,          Set Register Conditionally if Overflow Cleared)
MNEM (srvs,           SRVS,          Set Register Conditionally if Overflow Set)
MNEM (sscall,         SSCALL,        Secure State Call)
MNEM (ssrf,           SSRF,          Set Status Register Flag)
MNEM (st.b,           STB,           Store Byte)
MNEM (st.bal,         STBAL,         Conditionally Store Byte Always)
MNEM (st.bcc,         STBCC,         Conditionally Store Byte if Carry Cleared)
MNEM (st.bcs,         STBCS,         Conditionally Store Byte if Carry Set)
MNEM (st.beq,         STBEQ,         Conditionally Store Byte if Equal)
MNEM (st.bge,         STBGE,         Conditionally Store Byte if Greater than or Equal)
MNEM (st.bgt,         STBGT,         Conditionally Store Byte if Greater Than)
MNEM (st.bhi,         STBHI,         Conditionally Store Byte if Higher)
MNEM (st.bhs,         STBHS,         Conditionally Store Byte if Higher or Same)
MNEM (st.ble,         STBLE,         Conditionally Store Byte if Less than or Equal)
MNEM (st.blo,         STBLO,         Conditionally Store Byte if Lower)
MNEM (st.bls,         STBLS,         Conditionally Store Byte if Lower or Same)
MNEM (st.blt,         STBLT,         Conditionally Store Byte if Less Than)
MNEM (st.bmi,         STBMI,         Conditionally Store Byte if Negative)
MNEM (st.bne,         STBNE,         Conditionally Store Byte if Not Equal)
MNEM (st.bpl,         STBPL,         Conditionally Store Byte if Positive)
MNEM (st.bqs,         STBQS,         Conditionally Store Byte if Saturated)
MNEM (st.bvc,         STBVC,         Conditionally Store Byte if Overflow Cleared)
MNEM (st.bvs,         STBVS,         Conditionally Store Byte if Overflow Set)
MNEM (st.d,           STD,           Store Doubleword)
MNEM (st.h,           STH,           Store Halfword)
MNEM (st.hal,         STHAL,         Conditionally Store Halfword Always)
MNEM (st.hcc,         STHCC,         Conditionally Store Halfword if Carry Cleared)
MNEM (st.hcs,         STHCS,         Conditionally Store Halfword if Carry Set)
MNEM (st.heq,         STHEQ,         Conditionally Store Halfword if Equal)
MNEM (st.hge,         STHGE,         Conditionally Store Halfword if Greater than or Equal)
MNEM (st.hgt,         STHGT,         Conditionally Store Halfword if Greater Than)
MNEM (st.hhi,         STHHI,         Conditionally Store Halfword if Higher)
MNEM (st.hhs,         STHHS,         Conditionally Store Halfword if Higher or Same)
MNEM (st.hle,         STHLE,         Conditionally Store Halfword if Less than or Equal)
MNEM (st.hlo,         STHLO,         Conditionally Store Halfword if Lower)
MNEM (st.hls,         STHLS,         Conditionally Store Halfword if Lower or Same)
MNEM (st.hlt,         STHLT,         Conditionally Store Halfword if Less Than)
MNEM (st.hmi,         STHMI,         Conditionally Store Halfword if Negative)
MNEM (st.hne,         STHNE,         Conditionally Store Halfword if Not Equal)
MNEM (st.hpl,         STHPL,         Conditionally Store Halfword if Positive)
MNEM (st.hqs,         STHQS,         Conditionally Store Halfword if Saturated)
MNEM (st.hvc,         STHVC,         Conditionally Store Halfword if Overflow Cleared)
MNEM (st.hvs,         STHVS,         Conditionally Store Halfword if Overflow Set)
MNEM (st.w,           STW,           Store Word)
MNEM (st.wal,         STWAL,         Conditionally Store Word Always)
MNEM (st.wcc,         STWCC,         Conditionally Store Word if Carry Cleared)
MNEM (st.wcs,         STWCS,         Conditionally Store Word if Carry Set)
MNEM (st.weq,         STWEQ,         Conditionally Store Word if Equal)
MNEM (st.wge,         STWGE,         Conditionally Store Word if Greater than or Equal)
MNEM (st.wgt,         STWGT,         Conditionally Store Word if Greater Than)
MNEM (st.whi,         STWHI,         Conditionally Store Word if Higher)
MNEM (st.whs,         STWHS,         Conditionally Store Word if Higher or Same)
MNEM (st.wle,         STWLE,         Conditionally Store Word if Less than or Equal)
MNEM (st.wlo,         STWLO,         Conditionally Store Word if Lower)
MNEM (st.wls,         STWLS,         Conditionally Store Word if Lower or Same)
MNEM (st.wlt,         STWLT,         Conditionally Store Word if Less Than)
MNEM (st.wmi,         STWMI,         Conditionally Store Word if Negative)
MNEM (st.wne,         STWNE,         Conditionally Store Word if Not Equal)
MNEM (st.wpl,         STWPL,         Conditionally Store Word if Positive)
MNEM (st.wqs,         STWQS,         Conditionally Store Word if Saturated)
MNEM (st.wvc,         STWVC,         Conditionally Store Word if Overflow Cleared)
MNEM (st.wvs,         STWVS,         Conditionally Store Word if Overflow Set)
MNEM (stc.d,          STCD,          Store Coprocessor Doubleword)
MNEM (stc.w,          STCW,          Store Coprocessor Word)
MNEM (stc0.d,         STC0D,         Store Coprocessor 0 Doubleword Register)
MNEM (stc0.w,         STC0W,         Store Coprocessor 0 Word Register)
MNEM (stcond,         STCOND,        Store Word Conditionally)
MNEM (stdsp,          STDSP,         Store SP-relative with Displacement)
MNEM (sthh.w,         STHHW,         Store Halfwords into Word)
MNEM (stswp.h,        STSWPH,        Swap and Store Halfword)
MNEM (stswp.w,        STSWPW,        Swap and Store Word)
MNEM (sub,            SUB,           Subtract (without Carry))
MNEM (subal,          SUBAL,         Conditionally Subtract Always)
MNEM (subcc,          SUBCC,         Conditionally Subtract if Carry Cleared)
MNEM (subcs,          SUBCS,         Conditionally Subtract if Carry Set)
MNEM (subeq,          SUBEQ,         Conditionally Subtract if Equal)
MNEM (subfal,         SUBFAL,        Conditionally Subtract and Update Flags Always)
MNEM (subfcc,         SUBFCC,        Conditionally Subtract and Update Flags if Carry Cleared)
MNEM (subfcs,         SUBFCS,        Conditionally Subtract and Update Flags if Carry Set)
MNEM (subfeq,         SUBFEQ,        Conditionally Subtract and Update Flags if Equal)
MNEM (subfge,         SUBFGE,        Conditionally Subtract and Update Flags if Greater than or Equal)
MNEM (subfgt,         SUBFGT,        Conditionally Subtract and Update Flags if Greater Than)
MNEM (subfhi,         SUBFHI,        Conditionally Subtract and Update Flags if Higher)
MNEM (subfhs,         SUBFHS,        Conditionally Subtract and Update Flags if Higher or Same)
MNEM (subfle,         SUBFLE,        Conditionally Subtract and Update Flags if Less than or Equal)
MNEM (subflo,         SUBFLO,        Conditionally Subtract and Update Flags if Lower)
MNEM (subfls,         SUBFLS,        Conditionally Subtract and Update Flags if Lower or Same)
MNEM (subflt,         SUBFLT,        Conditionally Subtract and Update Flags if Less Than)
MNEM (subfmi,         SUBFMI,        Conditionally Subtract and Update Flags if Negative)
MNEM (subfne,         SUBFNE,        Conditionally Subtract and Update Flags if Not Equal)
MNEM (subfpl,         SUBFPL,        Conditionally Subtract and Update Flags if Positive)
MNEM (subfqs,         SUBFQS,        Conditionally Subtract and Update Flags if Saturated)
MNEM (subfvc,         SUBFVC,        Conditionally Subtract and Update Flags if Overflow Cleared)
MNEM (subfvs,         SUBFVS,        Conditionally Subtract and Update Flags if Overflow Set)
MNEM (subge,          SUBGE,         Conditionally Subtract if Greater than or Equal)
MNEM (subgt,          SUBGT,         Conditionally Subtract if Greater Than)
MNEM (subhh.w,        SUBHHW,        Subtract Halfwords into Word)
MNEM (subhi,          SUBHI,         Conditionally Subtract if Higher)
MNEM (subhs,          SUBHS,         Conditionally Subtract if Higher or Same)
MNEM (suble,          SUBLE,         Conditionally Subtract if Less than or Equal)
MNEM (sublo,          SUBLO,         Conditionally Subtract if Lower)
MNEM (subls,          SUBLS,         Conditionally Subtract if Lower or Same)
MNEM (sublt,          SUBLT,         Conditionally Subtract if Less Than)
MNEM (submi,          SUBMI,         Conditionally Subtract if Negative)
MNEM (subne,          SUBNE,         Conditionally Subtract if Not Equal)
MNEM (subpl,          SUBPL,         Conditionally Subtract if Positive)
MNEM (subqs,          SUBQS,         Conditionally Subtract if Saturated)
MNEM (subvc,          SUBVC,         Conditionally Subtract if Overflow Cleared)
MNEM (subvs,          SUBVS,         Conditionally Subtract if Overflow Set)
MNEM (swap.b,         SWAPB,         Swap Bytes)
MNEM (swap.bh,        SWAPBH,        Swap Bytes in Halfword)
MNEM (swap.h,         SWAPH,         Swap Halfwords)
MNEM (sync,           SYNC,          Synchronize memory)
MNEM (tlbr,           TLBR,          Read TLB Entry)
MNEM (tlbs,           TLBS,          Search TLB For Entry)
MNEM (tlbw,           TLBW,          Write TLB Entry)
MNEM (tnbz,           TNBZ,          Test if No Byte is Equal to Zero)
MNEM (tst,            TST,           Test Register)
MNEM (xchg,           XCHG,          Exchange Register and Memory)

// instructions

INSTR (ABS,           0x00005c40,  0x0000fff0,  R0,              Void,         Void,      Void,  Void)

INSTR (ACALL,         0x0000d000,  0x0000f00f,  U48T4,           Void,         Void,      Void,  Void)

INSTR (ACR,           0x00005c00,  0x0000fff0,  R0,              Void,         Void,      Void,  Void)

INSTR (ADC,           0xe0000040,  0xe1f0fff0,  R0,              R25,          R16,       Void,  Void)

INSTR (ADD,           0x00000000,  0x0000e1f0,  R0,              R9,           Void,      Void,  Void)
INSTR (ADD,           0xe0000000,  0xe1f0ffc0,  R0,              R25,          R16L42,    Void,  Void)

INSTR (ADDEQ,         0xe1d0e000,  0xe1f0fff0,  R0,              R25,          R16,       Void,  Void)

INSTR (ADDNE,         0xe1d0e100,  0xe1f0fff0,  R0,              R25,          R16,       Void,  Void)

INSTR (ADDCC,         0xe1d0e200,  0xe1f0fff0,  R0,              R25,          R16,       Void,  Void)

INSTR (ADDHS,         0xe1d0e200,  0xe1f0fff0,  R0,              R25,          R16,       Void,  Void)

INSTR (ADDCS,         0xe1d0e300,  0xe1f0fff0,  R0,              R25,          R16,       Void,  Void)

INSTR (ADDLO,         0xe1d0e300,  0xe1f0fff0,  R0,              R25,          R16,       Void,  Void)

INSTR (ADDGE,         0xe1d0e400,  0xe1f0fff0,  R0,              R25,          R16,       Void,  Void)

INSTR (ADDLT,         0xe1d0e500,  0xe1f0fff0,  R0,              R25,          R16,       Void,  Void)

INSTR (ADDMI,         0xe1d0e600,  0xe1f0fff0,  R0,              R25,          R16,       Void,  Void)

INSTR (ADDPL,         0xe1d0e700,  0xe1f0fff0,  R0,              R25,          R16,       Void,  Void)

INSTR (ADDLS,         0xe1d0e800,  0xe1f0fff0,  R0,              R25,          R16,       Void,  Void)

INSTR (ADDGT,         0xe1d0e900,  0xe1f0fff0,  R0,              R25,          R16,       Void,  Void)

INSTR (ADDLE,         0xe1d0ea00,  0xe1f0fff0,  R0,              R25,          R16,       Void,  Void)

INSTR (ADDHI,         0xe1d0eb00,  0xe1f0fff0,  R0,              R25,          R16,       Void,  Void)

INSTR (ADDVS,         0xe1d0ec00,  0xe1f0fff0,  R0,              R25,          R16,       Void,  Void)

INSTR (ADDVC,         0xe1d0ed00,  0xe1f0fff0,  R0,              R25,          R16,       Void,  Void)

INSTR (ADDQS,         0xe1d0ee00,  0xe1f0fff0,  R0,              R25,          R16,       Void,  Void)

INSTR (ADDAL,         0xe1d0ef00,  0xe1f0fff0,  R0,              R25,          R16,       Void,  Void)

INSTR (ADDABS,        0xe0000e40,  0xe1f0fff0,  R0,              R25,          R16,       Void,  Void)

INSTR (ADDHHW,        0xe0000e00,  0xe1f0ffc0,  R0,              R25P51,       R16P41,    Void,  Void)

INSTR (AND,           0x00000060,  0x0000e1f0,  R0,              R9,           Void,      Void,  Void)
INSTR (AND,           0xe1e00000,  0xe1f0fe00,  R0,              R25,          R16L45,    Void,  Void)
INSTR (AND,           0xe1e00200,  0xe1f0fe00,  R0,              R25,          R16R45,    Void,  Void)

INSTR (ANDEQ,         0xe1d0e020,  0xe1f0fff0,  R0,              R25,          R16,       Void,  Void)

INSTR (ANDNE,         0xe1d0e120,  0xe1f0fff0,  R0,              R25,          R16,       Void,  Void)

INSTR (ANDCC,         0xe1d0e220,  0xe1f0fff0,  R0,              R25,          R16,       Void,  Void)

INSTR (ANDHS,         0xe1d0e220,  0xe1f0fff0,  R0,              R25,          R16,       Void,  Void)

INSTR (ANDCS,         0xe1d0e320,  0xe1f0fff0,  R0,              R25,          R16,       Void,  Void)

INSTR (ANDLO,         0xe1d0e320,  0xe1f0fff0,  R0,              R25,          R16,       Void,  Void)

INSTR (ANDGE,         0xe1d0e420,  0xe1f0fff0,  R0,              R25,          R16,       Void,  Void)

INSTR (ANDLT,         0xe1d0e520,  0xe1f0fff0,  R0,              R25,          R16,       Void,  Void)

INSTR (ANDMI,         0xe1d0e620,  0xe1f0fff0,  R0,              R25,          R16,       Void,  Void)

INSTR (ANDPL,         0xe1d0e720,  0xe1f0fff0,  R0,              R25,          R16,       Void,  Void)

INSTR (ANDLS,         0xe1d0e820,  0xe1f0fff0,  R0,              R25,          R16,       Void,  Void)

INSTR (ANDGT,         0xe1d0e920,  0xe1f0fff0,  R0,              R25,          R16,       Void,  Void)

INSTR (ANDLE,         0xe1d0ea20,  0xe1f0fff0,  R0,              R25,          R16,       Void,  Void)

INSTR (ANDHI,         0xe1d0eb20,  0xe1f0fff0,  R0,              R25,          R16,       Void,  Void)

INSTR (ANDVS,         0xe1d0ec20,  0xe1f0fff0,  R0,              R25,          R16,       Void,  Void)

INSTR (ANDVC,         0xe1d0ed20,  0xe1f0fff0,  R0,              R25,          R16,       Void,  Void)

INSTR (ANDQS,         0xe1d0ee20,  0xe1f0fff0,  R0,              R25,          R16,       Void,  Void)

INSTR (ANDAL,         0xe1d0ef20,  0xe1f0fff0,  R0,              R25,          R16,       Void,  Void)

INSTR (ANDH,          0xe4100000,  0xfff00000,  R16,             U016,         Void,      Void,  Void)
INSTR (ANDH,          0xe6100000,  0xfff00000,  R16,             U016,         COH,       Void,  Void)

INSTR (ANDL,          0xe0100000,  0xfff00000,  R16,             U016,         Void,      Void,  Void)
INSTR (ANDL,          0xe2100000,  0xfff00000,  R16,             U016,         COH,       Void,  Void)

INSTR (ANDN,          0x00000080,  0x0000e1f0,  R0,              R9,           Void,      Void,  Void)

INSTR (ASR,           0xe0000840,  0xe1f0fff0,  R0,              R25,          R16,       Void,  Void)
INSTR (ASR,           0x0000a140,  0x0000e1e0,  R0,              U4194,        Void,      Void,  Void)
INSTR (ASR,           0xe0001400,  0xe1f0ffe0,  R16,             R25,          U05,       Void,  Void)

INSTR (BFEXTS,        0xe1d0b000,  0xe1f0fc00,  R25,             R16,          U55,       U05,   Void)

INSTR (BFEXTU,        0xe1d0c000,  0xe1f0fc00,  R25,             R16,          U55,       U05,   Void)

INSTR (BFINS,         0xe1d0d000,  0xe1f0fc00,  R25,             R16,          U55,       U05,   Void)

INSTR (BLD,           0xedb00000,  0xfff0ffe0,  R16,             U05,          Void,      Void,  Void)

INSTR (BREQ,          0x0000c000,  0x0000f00f,  S48T2,           Void,         Void,      Void,  Void)
INSTR (BREQ,          0xe0800000,  0xe1ef0000,  S01620254T2,     Void,         Void,      Void,  Void)

INSTR (BRNE,          0x0000c001,  0x0000f00f,  S48T2,           Void,         Void,      Void,  Void)
INSTR (BRNE,          0xe0810000,  0xe1ef0000,  S01620254T2,     Void,         Void,      Void,  Void)

INSTR (BRCC,          0x0000c002,  0x0000f00f,  S48T2,           Void,         Void,      Void,  Void)
INSTR (BRCC,          0xe0820000,  0xe1ef0000,  S01620254T2,     Void,         Void,      Void,  Void)

INSTR (BRHS,          0x0000c002,  0x0000f00f,  S48T2,           Void,         Void,      Void,  Void)
INSTR (BRHS,          0xe0820000,  0xe1ef0000,  S01620254T2,     Void,         Void,      Void,  Void)

INSTR (BRCS,          0x0000c003,  0x0000f00f,  S48T2,           Void,         Void,      Void,  Void)
INSTR (BRCS,          0xe0830000,  0xe1ef0000,  S01620254T2,     Void,         Void,      Void,  Void)

INSTR (BRLO,          0x0000c003,  0x0000f00f,  S48T2,           Void,         Void,      Void,  Void)
INSTR (BRLO,          0xe0830000,  0xe1ef0000,  S01620254T2,     Void,         Void,      Void,  Void)

INSTR (BRGE,          0x0000c004,  0x0000f00f,  S48T2,           Void,         Void,      Void,  Void)
INSTR (BRGE,          0xe0840000,  0xe1ef0000,  S01620254T2,     Void,         Void,      Void,  Void)

INSTR (BRLT,          0x0000c005,  0x0000f00f,  S48T2,           Void,         Void,      Void,  Void)
INSTR (BRLT,          0xe0850000,  0xe1ef0000,  S01620254T2,     Void,         Void,      Void,  Void)

INSTR (BRMI,          0x0000c006,  0x0000f00f,  S48T2,           Void,         Void,      Void,  Void)
INSTR (BRMI,          0xe0860000,  0xe1ef0000,  S01620254T2,     Void,         Void,      Void,  Void)

INSTR (BRPL,          0x0000c007,  0x0000f00f,  S48T2,           Void,         Void,      Void,  Void)
INSTR (BRPL,          0xe0870000,  0xe1ef0000,  S01620254T2,     Void,         Void,      Void,  Void)

INSTR (BRLS,          0xe0880000,  0xe1ef0000,  S01620254T2,     Void,         Void,      Void,  Void)

INSTR (BRGT,          0xe0890000,  0xe1ef0000,  S01620254T2,     Void,         Void,      Void,  Void)

INSTR (BRLE,          0xe08a0000,  0xe1ef0000,  S01620254T2,     Void,         Void,      Void,  Void)

INSTR (BRHI,          0xe08b0000,  0xe1ef0000,  S01620254T2,     Void,         Void,      Void,  Void)

INSTR (BRVS,          0xe08c0000,  0xe1ef0000,  S01620254T2,     Void,         Void,      Void,  Void)

INSTR (BRVC,          0xe08d0000,  0xe1ef0000,  S01620254T2,     Void,         Void,      Void,  Void)

INSTR (BRQS,          0xe08e0000,  0xe1ef0000,  S01620254T2,     Void,         Void,      Void,  Void)

INSTR (BRAL,          0xe08f0000,  0xe1ef0000,  S01620254T2,     Void,         Void,      Void,  Void)

INSTR (BREAKPOINT,    0x0000d673,  0x0000ffff,  Void,            Void,         Void,      Void,  Void)

INSTR (BREV,          0x00005c90,  0x0000fff0,  R0,              Void,         Void,      Void,  Void)

INSTR (BST,           0xefb00000,  0xfff0ffe0,  R16,             U05,          Void,      Void,  Void)

INSTR (CACHE,         0xf4100000,  0xfff00000,  P16S011,         U115,         Void,      Void,  Void)

INSTR (CASTSH,        0x00005c80,  0x0000fff0,  R0,              Void,         Void,      Void,  Void)

INSTR (CASTSB,        0x00005c60,  0x0000fff0,  R0,              Void,         Void,      Void,  Void)

INSTR (CASTUH,        0x00005c70,  0x0000fff0,  R0,              Void,         Void,      Void,  Void)

INSTR (CASTUB,        0x00005c50,  0x0000fff0,  R0,              Void,         Void,      Void,  Void)

INSTR (CBR,           0x0000a1c0,  0x0000e1e0,  R0,              U4194,        Void,      Void,  Void)

INSTR (CLZ,           0xe0001200,  0xe1f0ffff,  R16,             R25,          Void,      Void,  Void)

INSTR (COM,           0x00005cd0,  0x0000fff0,  R0,              Void,         Void,      Void,  Void)

INSTR (COP,           0xe1a00000,  0xf9f00000,  CP13,            CR8,          CR4,       CR0,   COP)

INSTR (CPB,           0xe0001800,  0xe1f0ffff,  R16,             R25,          Void,      Void,  Void)

INSTR (CPH,           0xe0001900,  0xe1f0ffff,  R16,             R25,          Void,      Void,  Void)

INSTR (CPW,           0x00000030,  0x0000e1f0,  R0,              R9,           Void,      Void,  Void)
INSTR (CPW,           0x00005800,  0x0000fc00,  R0,              S46,          Void,      Void,  Void)
INSTR (CPW,           0xe0400000,  0xe1e00000,  R0,              S01620254,    Void,      Void,  Void)

INSTR (CPC,           0xe0001300,  0xe1f0ffff,  R16,             R25,          Void,      Void,  Void)
INSTR (CPC,           0x00005c20,  0x0000fff0,  R0,              Void,         Void,      Void,  Void)

INSTR (CSRF,          0x0000d403,  0x0000fe0f,  U45,             Void,         Void,      Void,  Void)

INSTR (CSRFCZ,        0x0000d003,  0x0000fe0f,  U45,             Void,         Void,      Void,  Void)

INSTR (DIVS,          0xe0000c00,  0xe1f0fff0,  R0D,             R25,          R16,       Void,  Void)

INSTR (DIVU,          0xe0000d00,  0xe1f0fff0,  R0D,             R25,          R16,       Void,  Void)

INSTR (EOR,           0x00000050,  0x0000e1f0,  R0,              R9,           Void,      Void,  Void)
INSTR (EOR,           0xe1e02000,  0xe1f0fe00,  R0,              R25,          R16L45,    Void,  Void)
INSTR (EOR,           0xe1e02200,  0xe1f0fe00,  R0,              R25,          R16R45,    Void,  Void)

INSTR (EOREQ,         0xe1d0e040,  0xe1f0fff0,  R0,              R25,          R16,       Void,  Void)

INSTR (EORNE,         0xe1d0e140,  0xe1f0fff0,  R0,              R25,          R16,       Void,  Void)

INSTR (EORCC,         0xe1d0e240,  0xe1f0fff0,  R0,              R25,          R16,       Void,  Void)

INSTR (EORHS,         0xe1d0e240,  0xe1f0fff0,  R0,              R25,          R16,       Void,  Void)

INSTR (EORCS,         0xe1d0e340,  0xe1f0fff0,  R0,              R25,          R16,       Void,  Void)

INSTR (EORLO,         0xe1d0e340,  0xe1f0fff0,  R0,              R25,          R16,       Void,  Void)

INSTR (EORGE,         0xe1d0e440,  0xe1f0fff0,  R0,              R25,          R16,       Void,  Void)

INSTR (EORLT,         0xe1d0e540,  0xe1f0fff0,  R0,              R25,          R16,       Void,  Void)

INSTR (EORMI,         0xe1d0e640,  0xe1f0fff0,  R0,              R25,          R16,       Void,  Void)

INSTR (EORPL,         0xe1d0e740,  0xe1f0fff0,  R0,              R25,          R16,       Void,  Void)

INSTR (EORLS,         0xe1d0e840,  0xe1f0fff0,  R0,              R25,          R16,       Void,  Void)

INSTR (EORGT,         0xe1d0e940,  0xe1f0fff0,  R0,              R25,          R16,       Void,  Void)

INSTR (EORLE,         0xe1d0ea40,  0xe1f0fff0,  R0,              R25,          R16,       Void,  Void)

INSTR (EORHI,         0xe1d0eb40,  0xe1f0fff0,  R0,              R25,          R16,       Void,  Void)

INSTR (EORVS,         0xe1d0ec40,  0xe1f0fff0,  R0,              R25,          R16,       Void,  Void)

INSTR (EORVC,         0xe1d0ed40,  0xe1f0fff0,  R0,              R25,          R16,       Void,  Void)

INSTR (EORQS,         0xe1d0ee40,  0xe1f0fff0,  R0,              R25,          R16,       Void,  Void)

INSTR (EORAL,         0xe1d0ef40,  0xe1f0fff0,  R0,              R25,          R16,       Void,  Void)

INSTR (EORH,          0xee100000,  0xfff00000,  R16,             U016,         Void,      Void,  Void)

INSTR (EORL,          0xec100000,  0xfff00000,  R16,             U016,         Void,      Void,  Void)

INSTR (FRS,           0x0000d743,  0x0000ffff,  Void,            Void,         Void,      Void,  Void)

INSTR (ICALL,         0x00005d10,  0x0000fff0,  R0,              Void,         Void,      Void,  Void)

INSTR (INCJOSP,       0x0000d683,  0x0000ff8f,  S43WZ,           Void,         Void,      Void,  Void)

INSTR (LDD,           0x0000a101,  0x0000e1f1,  R1D,             P9I,          Void,      Void,  Void)
INSTR (LDD,           0x0000a110,  0x0000e1f1,  R1D,             P9D,          Void,      Void,  Void)
INSTR (LDD,           0x0000a100,  0x0000e1f1,  R1D,             R9,           Void,      Void,  Void)
INSTR (LDD,           0xe0e00000,  0xe1f10000,  R16D,            P25S016,      Void,      Void,  Void)
INSTR (LDD,           0xe0000200,  0xe1f0ffc0,  R0D,             P25L16S42,    Void,      Void,  Void)

INSTR (LDSB,          0xe1200000,  0xe1f00000,  R16,             P25S016,      Void,      Void,  Void)
INSTR (LDSB,          0xe0000600,  0xe1f0ffc0,  R0,              P25L16S42,    Void,      Void,  Void)

INSTR (LDSBEQ,        0xe1f00600,  0xe1f0fe00,  R16,             P25U09,       Void,      Void,  Void)

INSTR (LDSBNE,        0xe1f01600,  0xe1f0fe00,  R16,             P25U09,       Void,      Void,  Void)

INSTR (LDSBCC,        0xe1f02600,  0xe1f0fe00,  R16,             P25U09,       Void,      Void,  Void)

INSTR (LDSBHS,        0xe1f02600,  0xe1f0fe00,  R16,             P25U09,       Void,      Void,  Void)

INSTR (LDSBCS,        0xe1f03600,  0xe1f0fe00,  R16,             P25U09,       Void,      Void,  Void)

INSTR (LDSBLO,        0xe1f03600,  0xe1f0fe00,  R16,             P25U09,       Void,      Void,  Void)

INSTR (LDSBGE,        0xe1f04600,  0xe1f0fe00,  R16,             P25U09,       Void,      Void,  Void)

INSTR (LDSBLT,        0xe1f05600,  0xe1f0fe00,  R16,             P25U09,       Void,      Void,  Void)

INSTR (LDSBMI,        0xe1f06600,  0xe1f0fe00,  R16,             P25U09,       Void,      Void,  Void)

INSTR (LDSBPL,        0xe1f07600,  0xe1f0fe00,  R16,             P25U09,       Void,      Void,  Void)

INSTR (LDSBLS,        0xe1f08600,  0xe1f0fe00,  R16,             P25U09,       Void,      Void,  Void)

INSTR (LDSBGT,        0xe1f09600,  0xe1f0fe00,  R16,             P25U09,       Void,      Void,  Void)

INSTR (LDSBLE,        0xe1f0a600,  0xe1f0fe00,  R16,             P25U09,       Void,      Void,  Void)

INSTR (LDSBHI,        0xe1f0b600,  0xe1f0fe00,  R16,             P25U09,       Void,      Void,  Void)

INSTR (LDSBVS,        0xe1f0c600,  0xe1f0fe00,  R16,             P25U09,       Void,      Void,  Void)

INSTR (LDSBVC,        0xe1f0d600,  0xe1f0fe00,  R16,             P25U09,       Void,      Void,  Void)

INSTR (LDSBQS,        0xe1f0e600,  0xe1f0fe00,  R16,             P25U09,       Void,      Void,  Void)

INSTR (LDSBAL,        0xe1f0f600,  0xe1f0fe00,  R16,             P25U09,       Void,      Void,  Void)

INSTR (LDUB,          0x00000130,  0x0000e1f0,  R0,              P9I,          Void,      Void,  Void)
INSTR (LDUB,          0x00000170,  0x0000e1f0,  R0,              P9D,          Void,      Void,  Void)
INSTR (LDUB,          0x00000180,  0x0000e180,  R0,              P9U43,        Void,      Void,  Void)
INSTR (LDUB,          0xe1300000,  0xe1f00000,  R16,             P25S016,      Void,      Void,  Void)
INSTR (LDUB,          0xe0000700,  0xe1f0ffc0,  R0,              P25L16S42,    Void,      Void,  Void)

INSTR (LDUBEQ,        0xe1f00800,  0xe1f0fe00,  R16,             P25U09,       Void,      Void,  Void)

INSTR (LDUBNE,        0xe1f01800,  0xe1f0fe00,  R16,             P25U09,       Void,      Void,  Void)

INSTR (LDUBCC,        0xe1f02800,  0xe1f0fe00,  R16,             P25U09,       Void,      Void,  Void)

INSTR (LDUBHS,        0xe1f02800,  0xe1f0fe00,  R16,             P25U09,       Void,      Void,  Void)

INSTR (LDUBCS,        0xe1f03800,  0xe1f0fe00,  R16,             P25U09,       Void,      Void,  Void)

INSTR (LDUBLO,        0xe1f03800,  0xe1f0fe00,  R16,             P25U09,       Void,      Void,  Void)

INSTR (LDUBGE,        0xe1f04800,  0xe1f0fe00,  R16,             P25U09,       Void,      Void,  Void)

INSTR (LDUBLT,        0xe1f05800,  0xe1f0fe00,  R16,             P25U09,       Void,      Void,  Void)

INSTR (LDUBMI,        0xe1f06800,  0xe1f0fe00,  R16,             P25U09,       Void,      Void,  Void)

INSTR (LDUBPL,        0xe1f07800,  0xe1f0fe00,  R16,             P25U09,       Void,      Void,  Void)

INSTR (LDUBLS,        0xe1f08800,  0xe1f0fe00,  R16,             P25U09,       Void,      Void,  Void)

INSTR (LDUBGT,        0xe1f09800,  0xe1f0fe00,  R16,             P25U09,       Void,      Void,  Void)

INSTR (LDUBLE,        0xe1f0a800,  0xe1f0fe00,  R16,             P25U09,       Void,      Void,  Void)

INSTR (LDUBHI,        0xe1f0b800,  0xe1f0fe00,  R16,             P25U09,       Void,      Void,  Void)

INSTR (LDUBVS,        0xe1f0c800,  0xe1f0fe00,  R16,             P25U09,       Void,      Void,  Void)

INSTR (LDUBVC,        0xe1f0d800,  0xe1f0fe00,  R16,             P25U09,       Void,      Void,  Void)

INSTR (LDUBQS,        0xe1f0e800,  0xe1f0fe00,  R16,             P25U09,       Void,      Void,  Void)

INSTR (LDUBAL,        0xe1f0f800,  0xe1f0fe00,  R16,             P25U09,       Void,      Void,  Void)

INSTR (LDSH,          0x00000110,  0x0000e1f0,  R0,              P9I,          Void,      Void,  Void)
INSTR (LDSH,          0x00000150,  0x0000e1f0,  R0,              P9D,          Void,      Void,  Void)
INSTR (LDSH,          0x00008000,  0x0000e180,  R0,              P9U43T2,      Void,      Void,  Void)
INSTR (LDSH,          0xe1000000,  0xe1f00000,  R16,             P25S016,      Void,      Void,  Void)
INSTR (LDSH,          0xe0000600,  0xe1f0ffc0,  R0,              P25L16S42,    Void,      Void,  Void)

INSTR (LDSHEQ,        0xe1f00200,  0xe1f0fe00,  R16,             P25U09T2,     Void,      Void,  Void)

INSTR (LDSHNE,        0xe1f01200,  0xe1f0fe00,  R16,             P25U09T2,     Void,      Void,  Void)

INSTR (LDSHCC,        0xe1f02200,  0xe1f0fe00,  R16,             P25U09T2,     Void,      Void,  Void)

INSTR (LDSHHS,        0xe1f02200,  0xe1f0fe00,  R16,             P25U09T2,     Void,      Void,  Void)

INSTR (LDSHCS,        0xe1f03200,  0xe1f0fe00,  R16,             P25U09T2,     Void,      Void,  Void)

INSTR (LDSHLO,        0xe1f03200,  0xe1f0fe00,  R16,             P25U09T2,     Void,      Void,  Void)

INSTR (LDSHGE,        0xe1f04200,  0xe1f0fe00,  R16,             P25U09T2,     Void,      Void,  Void)

INSTR (LDSHLT,        0xe1f05200,  0xe1f0fe00,  R16,             P25U09T2,     Void,      Void,  Void)

INSTR (LDSHMI,        0xe1f06200,  0xe1f0fe00,  R16,             P25U09T2,     Void,      Void,  Void)

INSTR (LDSHPL,        0xe1f07200,  0xe1f0fe00,  R16,             P25U09T2,     Void,      Void,  Void)

INSTR (LDSHLS,        0xe1f08200,  0xe1f0fe00,  R16,             P25U09T2,     Void,      Void,  Void)

INSTR (LDSHGT,        0xe1f09200,  0xe1f0fe00,  R16,             P25U09T2,     Void,      Void,  Void)

INSTR (LDSHLE,        0xe1f0a200,  0xe1f0fe00,  R16,             P25U09T2,     Void,      Void,  Void)

INSTR (LDSHHI,        0xe1f0b200,  0xe1f0fe00,  R16,             P25U09T2,     Void,      Void,  Void)

INSTR (LDSHVS,        0xe1f0c200,  0xe1f0fe00,  R16,             P25U09T2,     Void,      Void,  Void)

INSTR (LDSHVC,        0xe1f0d200,  0xe1f0fe00,  R16,             P25U09T2,     Void,      Void,  Void)

INSTR (LDSHQS,        0xe1f0e200,  0xe1f0fe00,  R16,             P25U09T2,     Void,      Void,  Void)

INSTR (LDSHAL,        0xe1f0f200,  0xe1f0fe00,  R16,             P25U09T2,     Void,      Void,  Void)

INSTR (LDUH,          0x00000120,  0x0000e1f0,  R0,              P9I,          Void,      Void,  Void)
INSTR (LDUH,          0x00000160,  0x0000e1f0,  R0,              P9D,          Void,      Void,  Void)
INSTR (LDUH,          0x00008080,  0x0000e180,  R0,              P9U43T2,      Void,      Void,  Void)
INSTR (LDUH,          0xe1100000,  0xe1f00000,  R16,             P25S016,      Void,      Void,  Void)
INSTR (LDUH,          0xe0000500,  0xe1f0ffc0,  R0,              P25L16S42,    Void,      Void,  Void)

INSTR (LDUHEQ,        0xe1f00400,  0xe1f0fe00,  R16,             P25U09T2,     Void,      Void,  Void)

INSTR (LDUHNE,        0xe1f01400,  0xe1f0fe00,  R16,             P25U09T2,     Void,      Void,  Void)

INSTR (LDUHCC,        0xe1f02400,  0xe1f0fe00,  R16,             P25U09T2,     Void,      Void,  Void)

INSTR (LDUHHS,        0xe1f02400,  0xe1f0fe00,  R16,             P25U09T2,     Void,      Void,  Void)

INSTR (LDUHCS,        0xe1f03400,  0xe1f0fe00,  R16,             P25U09T2,     Void,      Void,  Void)

INSTR (LDUHLO,        0xe1f03400,  0xe1f0fe00,  R16,             P25U09T2,     Void,      Void,  Void)

INSTR (LDUHGE,        0xe1f04400,  0xe1f0fe00,  R16,             P25U09T2,     Void,      Void,  Void)

INSTR (LDUHLT,        0xe1f05400,  0xe1f0fe00,  R16,             P25U09T2,     Void,      Void,  Void)

INSTR (LDUHMI,        0xe1f06400,  0xe1f0fe00,  R16,             P25U09T2,     Void,      Void,  Void)

INSTR (LDUHPL,        0xe1f07400,  0xe1f0fe00,  R16,             P25U09T2,     Void,      Void,  Void)

INSTR (LDUHLS,        0xe1f08400,  0xe1f0fe00,  R16,             P25U09T2,     Void,      Void,  Void)

INSTR (LDUHGT,        0xe1f09400,  0xe1f0fe00,  R16,             P25U09T2,     Void,      Void,  Void)

INSTR (LDUHLE,        0xe1f0a400,  0xe1f0fe00,  R16,             P25U09T2,     Void,      Void,  Void)

INSTR (LDUHHI,        0xe1f0b400,  0xe1f0fe00,  R16,             P25U09T2,     Void,      Void,  Void)

INSTR (LDUHVS,        0xe1f0c400,  0xe1f0fe00,  R16,             P25U09T2,     Void,      Void,  Void)

INSTR (LDUHVC,        0xe1f0d400,  0xe1f0fe00,  R16,             P25U09T2,     Void,      Void,  Void)

INSTR (LDUHQS,        0xe1f0e400,  0xe1f0fe00,  R16,             P25U09T2,     Void,      Void,  Void)

INSTR (LDUHAL,        0xe1f0f400,  0xe1f0fe00,  R16,             P25U09T2,     Void,      Void,  Void)

INSTR (LDW,           0x00000100,  0x0000e1f0,  R0,              P9I,          Void,      Void,  Void)
INSTR (LDW,           0x00000140,  0x0000e1f0,  R0,              P9D,          Void,      Void,  Void)
INSTR (LDW,           0x00006000,  0x0000e000,  R0,              P9U45T4,      Void,      Void,  Void)
INSTR (LDW,           0xe0f00000,  0xe1f00000,  R16,             P25S016,      Void,      Void,  Void)
INSTR (LDW,           0xe0000300,  0xe1f0ffc0,  R0,              P25L16S42,    Void,      Void,  Void)

INSTR (LDWEQ,         0xe1f00000,  0xe1f0fe00,  R16,             P25U09T4,     Void,      Void,  Void)

INSTR (LDWNE,         0xe1f01000,  0xe1f0fe00,  R16,             P25U09T4,     Void,      Void,  Void)

INSTR (LDWCC,         0xe1f02000,  0xe1f0fe00,  R16,             P25U09T4,     Void,      Void,  Void)

INSTR (LDWHS,         0xe1f02000,  0xe1f0fe00,  R16,             P25U09T4,     Void,      Void,  Void)

INSTR (LDWCS,         0xe1f03000,  0xe1f0fe00,  R16,             P25U09T4,     Void,      Void,  Void)

INSTR (LDWLO,         0xe1f03000,  0xe1f0fe00,  R16,             P25U09T4,     Void,      Void,  Void)

INSTR (LDWGE,         0xe1f04000,  0xe1f0fe00,  R16,             P25U09T4,     Void,      Void,  Void)

INSTR (LDWLT,         0xe1f05000,  0xe1f0fe00,  R16,             P25U09T4,     Void,      Void,  Void)

INSTR (LDWMI,         0xe1f06000,  0xe1f0fe00,  R16,             P25U09T4,     Void,      Void,  Void)

INSTR (LDWPL,         0xe1f07000,  0xe1f0fe00,  R16,             P25U09T4,     Void,      Void,  Void)

INSTR (LDWLS,         0xe1f08000,  0xe1f0fe00,  R16,             P25U09T4,     Void,      Void,  Void)

INSTR (LDWGT,         0xe1f09000,  0xe1f0fe00,  R16,             P25U09T4,     Void,      Void,  Void)

INSTR (LDWLE,         0xe1f0a000,  0xe1f0fe00,  R16,             P25U09T4,     Void,      Void,  Void)

INSTR (LDWHI,         0xe1f0b000,  0xe1f0fe00,  R16,             P25U09T4,     Void,      Void,  Void)

INSTR (LDWVS,         0xe1f0c000,  0xe1f0fe00,  R16,             P25U09T4,     Void,      Void,  Void)

INSTR (LDWVC,         0xe1f0d000,  0xe1f0fe00,  R16,             P25U09T4,     Void,      Void,  Void)

INSTR (LDWQS,         0xe1f0e000,  0xe1f0fe00,  R16,             P25U09T4,     Void,      Void,  Void)

INSTR (LDWAL,         0xe1f0f000,  0xe1f0fe00,  R16,             P25U09T4,     Void,      Void,  Void)

INSTR (LDCD,          0xe9a01000,  0xfff01100,  CP13,            CR9D,         P16U08T4,  Void,  Void)
INSTR (LDCD,          0xefa00050,  0xfff011ff,  CP13,            CR9D,         P16D,      Void,  Void)
INSTR (LDCD,          0xefa01040,  0xfff011c0,  CP13,            CR9D,         P16L0S42,  Void,  Void)

INSTR (LDCW,          0xe9a00000,  0xfff01000,  CP13,            CR8,          P16U08T4,  Void,  Void)
INSTR (LDCW,          0xefa00040,  0xfff010ff,  CP13,            CR8,          P16D,      Void,  Void)
INSTR (LDCW,          0xefa01000,  0xfff010c0,  CP13,            CR8,          P16L0S42,  Void,  Void)

INSTR (LDC0D,         0xf3a00000,  0xfff00100,  CR9D,            P16U08124T4,  Void,      Void,  Void)

INSTR (LDC0W,         0xf1a00000,  0xfff00000,  CR8,             P16U08124T4,  Void,      Void,  Void)

INSTR (LDDPC,         0x00004800,  0x0000f800,  R0,              PPCU47T4,     Void,      Void,  Void)

INSTR (LDDSP,         0x00004000,  0x0000f800,  R0,              PSPU47T4,     Void,      Void,  Void)

INSTR (LDINSB,        0xe1d04000,  0xe1f0c000,  R16P122,         P25S012,      Void,      Void,  Void)

INSTR (LDINSH,        0xe1d00000,  0xe1f0e000,  R16P121,         P25S012T2,    Void,      Void,  Void)

INSTR (LDSWPSH,       0xe1d02000,  0xe1f0f000,  R16,             P25S012T2,    Void,      Void,  Void)

INSTR (LDSWPUH,       0xe1d03000,  0xe1f0f000,  R16,             P25S012T2,    Void,      Void,  Void)

INSTR (LDSWPW,        0xe1d08000,  0xe1f0f000,  R16,             P25S012T4,    Void,      Void,  Void)

INSTR (LSL,           0xe0000940,  0xe1f0fff0,  R0,              R25,          R16,       Void,  Void)
INSTR (LSL,           0x0000a160,  0x0000e1e0,  R0,              U4194,        Void,      Void,  Void)
INSTR (LSL,           0xe0001500,  0xe1f0ffe0,  R16,             R25,          U05,       Void,  Void)

INSTR (LSR,           0xe0000a40,  0xe1f0fff0,  R0,              R25,          R16,       Void,  Void)
INSTR (LSR,           0x0000a180,  0x0000e1e0,  R0,              U4194,        Void,      Void,  Void)
INSTR (LSR,           0xe0001600,  0xe1f0ffe0,  R16,             R25,          U05,       Void,  Void)

INSTR (MAC,           0xe0000340,  0xe1f0fff0,  R0,              R25,          R16,       Void,  Void)

INSTR (MACHHD,        0xe0000580,  0xe1f0ffc0,  R0D,             R25P51,       R16P41,    Void,  Void)

INSTR (MACHHW,        0xe0000480,  0xe1f0ffc0,  R0,              R25P51,       R16P41,    Void,  Void)

INSTR (MACSD,         0xe0000540,  0xe1f0fff0,  R0D,             R25,          R16,       Void,  Void)

INSTR (MACSATHHW,     0xe0000680,  0xe1f0ffc0,  R0,              R25P51,       R16P41,    Void,  Void)

INSTR (MACUD,         0xe0000740,  0xe1f0fff0,  R0D,             R25,          R16,       Void,  Void)

INSTR (MACWHD,        0xe0000c80,  0xe1f0ffe0,  R0D,             R25,          R16P41,    Void,  Void)

INSTR (MAX,           0xe0000c40,  0xe1f0fff0,  R0,              R25,          R16,       Void,  Void)

INSTR (MCALL,         0xf0100000,  0xfff00000,  P16S016T4,       Void,         Void,      Void,  Void)

INSTR (MEMC,          0xf6100000,  0xfff00000,  S015T4,          U155,         Void,      Void,  Void)

INSTR (MEMS,          0xf8100000,  0xfff00000,  S015T4,          U155,         Void,      Void,  Void)

INSTR (MEMT,          0xfa100000,  0xfff00000,  S015T4,          U155,         Void,      Void,  Void)

INSTR (MFDR,          0xe5b00000,  0xfff0ff00,  R16,             U08T4,        Void,      Void,  Void)

INSTR (MFSR,          0xe1b00000,  0xfff0ff00,  R16,             U08T4,        Void,      Void,  Void)

INSTR (MIN,           0xe0000d40,  0xe1f0fff0,  R0,              R25,          R16,       Void,  Void)

INSTR (MOV,           0x00003000,  0x0000f000,  R0,              S48,          Void,      Void,  Void)
INSTR (MOV,           0xe0600000,  0xe1e00000,  R16,             S01620254,    Void,      Void,  Void)
INSTR (MOV,           0x00000090,  0x0000e1f0,  R0,              R9,           Void,      Void,  Void)

INSTR (MOVEQ,         0xe0001700,  0xe1f0ffff,  R16,             R25,          Void,      Void,  Void)
INSTR (MOVEQ,         0xf9b00000,  0xfff0ff00,  R16,             S08,          Void,      Void,  Void)

INSTR (MOVNE,         0xe0001710,  0xe1f0ffff,  R16,             R25,          Void,      Void,  Void)
INSTR (MOVNE,         0xf9b00100,  0xfff0ff00,  R16,             S08,          Void,      Void,  Void)

INSTR (MOVCC,         0xe0001720,  0xe1f0ffff,  R16,             R25,          Void,      Void,  Void)
INSTR (MOVCC,         0xf9b00200,  0xfff0ff00,  R16,             S08,          Void,      Void,  Void)

INSTR (MOVHS,         0xe0001720,  0xe1f0ffff,  R16,             R25,          Void,      Void,  Void)
INSTR (MOVHS,         0xf9b00200,  0xfff0ff00,  R16,             S08,          Void,      Void,  Void)

INSTR (MOVCS,         0xe0001730,  0xe1f0ffff,  R16,             R25,          Void,      Void,  Void)
INSTR (MOVCS,         0xf9b00300,  0xfff0ff00,  R16,             S08,          Void,      Void,  Void)

INSTR (MOVLO,         0xe0001730,  0xe1f0ffff,  R16,             R25,          Void,      Void,  Void)
INSTR (MOVLO,         0xf9b00300,  0xfff0ff00,  R16,             S08,          Void,      Void,  Void)

INSTR (MOVGE,         0xe0001740,  0xe1f0ffff,  R16,             R25,          Void,      Void,  Void)
INSTR (MOVGE,         0xf9b00400,  0xfff0ff00,  R16,             S08,          Void,      Void,  Void)

INSTR (MOVLT,         0xe0001750,  0xe1f0ffff,  R16,             R25,          Void,      Void,  Void)
INSTR (MOVLT,         0xf9b00500,  0xfff0ff00,  R16,             S08,          Void,      Void,  Void)

INSTR (MOVMI,         0xe0001760,  0xe1f0ffff,  R16,             R25,          Void,      Void,  Void)
INSTR (MOVMI,         0xf9b00600,  0xfff0ff00,  R16,             S08,          Void,      Void,  Void)

INSTR (MOVPL,         0xe0001770,  0xe1f0ffff,  R16,             R25,          Void,      Void,  Void)
INSTR (MOVPL,         0xf9b00700,  0xfff0ff00,  R16,             S08,          Void,      Void,  Void)

INSTR (MOVLS,         0xe0001780,  0xe1f0ffff,  R16,             R25,          Void,      Void,  Void)
INSTR (MOVLS,         0xf9b00800,  0xfff0ff00,  R16,             S08,          Void,      Void,  Void)

INSTR (MOVGT,         0xe0001790,  0xe1f0ffff,  R16,             R25,          Void,      Void,  Void)
INSTR (MOVGT,         0xf9b00900,  0xfff0ff00,  R16,             S08,          Void,      Void,  Void)

INSTR (MOVLE,         0xe00017a0,  0xe1f0ffff,  R16,             R25,          Void,      Void,  Void)
INSTR (MOVLE,         0xf9b00a00,  0xfff0ff00,  R16,             S08,          Void,      Void,  Void)

INSTR (MOVHI,         0xe00017b0,  0xe1f0ffff,  R16,             R25,          Void,      Void,  Void)
INSTR (MOVHI,         0xf9b00b00,  0xfff0ff00,  R16,             S08,          Void,      Void,  Void)

INSTR (MOVVS,         0xe00017c0,  0xe1f0ffff,  R16,             R25,          Void,      Void,  Void)
INSTR (MOVVS,         0xf9b00c00,  0xfff0ff00,  R16,             S08,          Void,      Void,  Void)

INSTR (MOVVC,         0xe00017d0,  0xe1f0ffff,  R16,             R25,          Void,      Void,  Void)
INSTR (MOVVC,         0xf9b00d00,  0xfff0ff00,  R16,             S08,          Void,      Void,  Void)

INSTR (MOVQS,         0xe00017e0,  0xe1f0ffff,  R16,             R25,          Void,      Void,  Void)
INSTR (MOVQS,         0xf9b00e00,  0xfff0ff00,  R16,             S08,          Void,      Void,  Void)

INSTR (MOVAL,         0xe00017f0,  0xe1f0ffff,  R16,             R25,          Void,      Void,  Void)
INSTR (MOVAL,         0xf9b00f00,  0xfff0ff00,  R16,             S08,          Void,      Void,  Void)

INSTR (MOVH,          0xfc100000,  0xfff00000,  R16,             U016,         Void,      Void,  Void)

INSTR (MOVL,          0xe0600000,  0xfff00000,  R16,             U016,         Void,      Void,  Void)

INSTR (MTDR,          0xe7b00000,  0xfff0ff00,  U08T4,           R16,          Void,      Void,  Void)

INSTR (MTSR,          0xe3b00000,  0xfff0ff00,  U08T4,           R16,          Void,      Void,  Void)

INSTR (MUL,           0x0000a130,  0x0000e1f0,  R0,              R9,           Void,      Void,  Void)
INSTR (MUL,           0xe0000240,  0xe1f0fff0,  R0,              R25,          R16,       Void,  Void)
INSTR (MUL,           0xe0001000,  0xe1f0ff00,  R16,             R25,          S08,       Void,  Void)

INSTR (MULHHW,        0xe0000780,  0xe1f0ffc0,  R0,              R25P51,       R16P41,    Void,  Void)

INSTR (MULNHHW,       0xe0000180,  0xe1f0ffc0,  R0,              R25P51,       R16P41,    Void,  Void)

INSTR (MULNWHD,       0xe0000280,  0xe1f0ffe0,  R0D,             R25,          R16P41,    Void,  Void)

INSTR (MULSD,         0xe0000440,  0xe1f0fff0,  R0D,             R25,          R16,       Void,  Void)

INSTR (MULSATHHH,     0xe0000880,  0xe1f0ffc0,  R0,              R25P51,       R16P41,    Void,  Void)

INSTR (MULSATHHW,     0xe0000980,  0xe1f0ffc0,  R0,              R25P51,       R16P41,    Void,  Void)

INSTR (MULSATRNDHHH,  0xe0000a80,  0xe1f0ffc0,  R0,              R25P51,       R16P41,    Void,  Void)

INSTR (MULSATRNDHHW,  0xe0000b80,  0xe1f0ffc0,  R0,              R25P51,       R16P41,    Void,  Void)

INSTR (MULSATWHW,     0xe0000e80,  0xe1f0ffe0,  R0,              R25,          R16P41,    Void,  Void)

INSTR (MULUD,         0xe0000640,  0xe1f0fff0,  R0D,             R25,          R16,       Void,  Void)

INSTR (MULWHD,        0xe0000d80,  0xe1f0ffe0,  R0D,             R25,          R16P41,    Void,  Void)

INSTR (MUSFR,         0x00005d30,  0x0000fff0,  R0,              Void,         Void,      Void,  Void)

INSTR (MUSTR,         0x00005d20,  0x0000fff0,  R0,              Void,         Void,      Void,  Void)

INSTR (MVCRD,         0xefa00010,  0xfff111ff,  CP13,            R16D,         CR9D,      Void,  Void)

INSTR (MVCRW,         0xefa00000,  0xfff010ff,  CP13,            R16,          CR8,       Void,  Void)

INSTR (MVRCD,         0xefa00030,  0xfff111ff,  CP13,            CR9D,         R16D,      Void,  Void)

INSTR (MVRCW,         0xefa00020,  0xfff010ff,  CP13,            CR8,          R16,       Void,  Void)

INSTR (NEG,           0x00005c30,  0x0000fff0,  R0,              Void,         Void,      Void,  Void)

INSTR (NOP,           0x0000d703,  0x0000ffff,  Void,            Void,         Void,      Void,  Void)

INSTR (OR,            0x00000040,  0x0000e1f0,  R0,              R9,           Void,      Void,  Void)
INSTR (OR,            0xe1e01000,  0xe1f0fe00,  R0,              R25,          R16L45,    Void,  Void)
INSTR (OR,            0xe1e01200,  0xe1f0fe00,  R0,              R25,          R16R45,    Void,  Void)

INSTR (OREQ,          0xe1d0e030,  0xe1f0fff0,  R0,              R25,          R16,       Void,  Void)

INSTR (ORNE,          0xe1d0e130,  0xe1f0fff0,  R0,              R25,          R16,       Void,  Void)

INSTR (ORCC,          0xe1d0e230,  0xe1f0fff0,  R0,              R25,          R16,       Void,  Void)

INSTR (ORHS,          0xe1d0e230,  0xe1f0fff0,  R0,              R25,          R16,       Void,  Void)

INSTR (ORCS,          0xe1d0e330,  0xe1f0fff0,  R0,              R25,          R16,       Void,  Void)

INSTR (ORLO,          0xe1d0e330,  0xe1f0fff0,  R0,              R25,          R16,       Void,  Void)

INSTR (ORGE,          0xe1d0e430,  0xe1f0fff0,  R0,              R25,          R16,       Void,  Void)

INSTR (ORLT,          0xe1d0e530,  0xe1f0fff0,  R0,              R25,          R16,       Void,  Void)

INSTR (ORMI,          0xe1d0e630,  0xe1f0fff0,  R0,              R25,          R16,       Void,  Void)

INSTR (ORPL,          0xe1d0e730,  0xe1f0fff0,  R0,              R25,          R16,       Void,  Void)

INSTR (ORLS,          0xe1d0e830,  0xe1f0fff0,  R0,              R25,          R16,       Void,  Void)

INSTR (ORGT,          0xe1d0e930,  0xe1f0fff0,  R0,              R25,          R16,       Void,  Void)

INSTR (ORLE,          0xe1d0ea30,  0xe1f0fff0,  R0,              R25,          R16,       Void,  Void)

INSTR (ORHI,          0xe1d0eb30,  0xe1f0fff0,  R0,              R25,          R16,       Void,  Void)

INSTR (ORVS,          0xe1d0ec30,  0xe1f0fff0,  R0,              R25,          R16,       Void,  Void)

INSTR (ORVC,          0xe1d0ed30,  0xe1f0fff0,  R0,              R25,          R16,       Void,  Void)

INSTR (ORQS,          0xe1d0ee30,  0xe1f0fff0,  R0,              R25,          R16,       Void,  Void)

INSTR (ORAL,          0xe1d0ef30,  0xe1f0fff0,  R0,              R25,          R16,       Void,  Void)

INSTR (ORH,           0xea100000,  0xfff00000,  R16,             U016,         Void,      Void,  Void)

INSTR (ORL,           0xe8100000,  0xfff00000,  R16,             U016,         Void,      Void,  Void)

INSTR (PABSSB,        0xe00023e0,  0xfff0fff0,  R0,              R16,          Void,      Void,  Void)

INSTR (PABSSH,        0xe00023f0,  0xfff0fff0,  R0,              R16,          Void,      Void,  Void)

INSTR (PACKSHUB,      0xe00024c0,  0xe1f0fff0,  R0,              R25,          R16,       Void,  Void)

INSTR (PACKSHSB,      0xe00024d0,  0xe1f0fff0,  R0,              R25,          R16,       Void,  Void)

INSTR (PACKWSHS,      0xe0002470,  0xe1f0fff0,  R0,              R25,          R16,       Void,  Void)

INSTR (PADDB,         0xe0002500,  0xe1f0fff0,  R0,              R25,          R16,       Void,  Void)

INSTR (PADDH,         0xe0002000,  0xe1f0fff0,  R0,              R25,          R16,       Void,  Void)

INSTR (PADDHUB,       0xe0002360,  0xe1f0fff0,  R0,              R25,          R16,       Void,  Void)

INSTR (PADDHSH,       0xe00020c0,  0xe1f0fff0,  R0,              R25,          R16,       Void,  Void)

INSTR (PADDSUB,       0xe0002340,  0xe1f0fff0,  R0,              R25,          R16,       Void,  Void)

INSTR (PADDSSB,       0xe0002320,  0xe1f0fff0,  R0,              R25,          R16,       Void,  Void)

INSTR (PADDSUH,       0xe0002080,  0xe1f0fff0,  R0,              R25,          R16,       Void,  Void)

INSTR (PADDSSH,       0xe0002040,  0xe1f0fff0,  R0,              R25,          R16,       Void,  Void)

INSTR (PADDSUBH,      0xe0002100,  0xe1f0ffc0,  R0,              R25P51,       R16P41,    Void,  Void)

INSTR (PADDSUBHSH,    0xe0002280,  0xe1f0ffc0,  R0,              R25P51,       R16P41,    Void,  Void)

INSTR (PADDSUBSUH,    0xe0002200,  0xe1f0ffc0,  R0,              R25P51,       R16P41,    Void,  Void)

INSTR (PADDSUBSSH,    0xe0002180,  0xe1f0ffc0,  R0,              R25P51,       R16P41,    Void,  Void)

INSTR (PADDXH,        0xe0002020,  0xe1f0fff0,  R0,              R25,          R16,       Void,  Void)

INSTR (PADDXHSH,      0xe00020e0,  0xe1f0fff0,  R0,              R25,          R16,       Void,  Void)

INSTR (PADDXSUH,      0xe00020a0,  0xe1f0fff0,  R0,              R25,          R16,       Void,  Void)

INSTR (PADDXSSH,      0xe0002060,  0xe1f0fff0,  R0,              R25,          R16,       Void,  Void)

INSTR (PASRB,         0xe0002410,  0xe1f8fff0,  R0,              R25,          U163,      Void,  Void)

INSTR (PASRH,         0xe0002440,  0xe1f0fff0,  R0,              R25,          U164,      Void,  Void)

INSTR (PAVGUB,        0xe00023c0,  0xe1f0fff0,  R0,              R25,          R16,       Void,  Void)

INSTR (PAVGSH,        0xe00023d0,  0xe1f0fff0,  R0,              R25,          R16,       Void,  Void)

INSTR (PLSLB,         0xe0002420,  0xe1f8fff0,  R0,              R25,          U163,      Void,  Void)

INSTR (PLSLH,         0xe0002450,  0xe1f0fff0,  R0,              R25,          U164,      Void,  Void)

INSTR (PLSRB,         0xe0002430,  0xe1f8fff0,  R0,              R25,          U163,      Void,  Void)

INSTR (PLSRH,         0xe0002460,  0xe1f0fff0,  R0,              R25,          U164,      Void,  Void)

INSTR (PMAXUB,        0xe0002380,  0xe1f0fff0,  R0,              R25,          R16,       Void,  Void)

INSTR (PMAXSH,        0xe0002390,  0xe1f0fff0,  R0,              R25,          R16,       Void,  Void)

INSTR (PMINUB,        0xe00023a0,  0xe1f0fff0,  R0,              R25,          R16,       Void,  Void)

INSTR (PMINSH,        0xe00023b0,  0xe1f0fff0,  R0,              R25,          R16,       Void,  Void)

INSTR (POPJC,         0x0000d713,  0x0000ffff,  Void,            Void,         Void,      Void,  Void)

INSTR (PREF,          0xf2100000,  0xfff00000,  P16S016,         Void,         Void,      Void,  Void)

INSTR (PSAD,          0xe0002400,  0xe1f0fff0,  R0,              R25,          R16,       Void,  Void)

INSTR (PSUBB,         0xe0002310,  0xe1f0fff0,  R0,              R25,          R16,       Void,  Void)

INSTR (PSUBH,         0xe0002010,  0xe1f0fff0,  R0,              R25,          R16,       Void,  Void)

INSTR (PSUBADDH,      0xe0002140,  0xe1f0ffc0,  R0,              R25P51,       R16P41,    Void,  Void)

INSTR (PSUBADDHSH,    0xe00022c0,  0xe1f0ffc0,  R0,              R25P51,       R16P41,    Void,  Void)

INSTR (PSUBADDSUH,    0xe0002240,  0xe1f0ffc0,  R0,              R25P51,       R16P41,    Void,  Void)

INSTR (PSUBADDSSH,    0xe00021c0,  0xe1f0ffc0,  R0,              R25P51,       R16P41,    Void,  Void)

INSTR (PSUBHUB,       0xe0002370,  0xe1f0fff0,  R0,              R25,          R16,       Void,  Void)

INSTR (PSUBHSH,       0xe00020d0,  0xe1f0fff0,  R0,              R25,          R16,       Void,  Void)

INSTR (PSUBSUB,       0xe0002350,  0xe1f0fff0,  R0,              R25,          R16,       Void,  Void)

INSTR (PSUBSSB,       0xe0002330,  0xe1f0fff0,  R0,              R25,          R16,       Void,  Void)

INSTR (PSUBSUH,       0xe0002090,  0xe1f0fff0,  R0,              R25,          R16,       Void,  Void)

INSTR (PSUBSSH,       0xe0002050,  0xe1f0fff0,  R0,              R25,          R16,       Void,  Void)

INSTR (PSUBXH,        0xe0002030,  0xe1f0fff0,  R0,              R25,          R16,       Void,  Void)

INSTR (PSUBXHSH,      0xe00020f0,  0xe1f0fff0,  R0,              R25,          R16,       Void,  Void)

INSTR (PSUBXSUH,      0xe00020b0,  0xe1f0fff0,  R0,              R25,          R16,       Void,  Void)

INSTR (PSUBXSSH,      0xe0002070,  0xe1f0fff0,  R0,              R25,          R16,       Void,  Void)

INSTR (PUNPCKSBH,     0xe00024a0,  0xe1f0ffe0,  R0,              R25P41,       Void,      Void,  Void)

INSTR (PUNPCKUBH,     0xe0002480,  0xe1f0ffe0,  R0,              R25P41,       Void,      Void,  Void)

INSTR (PUSHJC,        0x0000d723,  0x0000ffff,  Void,            Void,         Void,      Void,  Void)

INSTR (RCALL,         0x0000c00c,  0x0000f00c,  PPCS4802T2,      Void,         Void,      Void,  Void)
INSTR (RCALL,         0xe0a00000,  0xe1ef0000,  PPCS01620254T2,  Void,         Void,      Void,  Void)

INSTR (RETEQ,         0x00005e00,  0x0000fff0,  R0,              Void,         Void,      Void,  Void)

INSTR (RETNE,         0x00005e10,  0x0000fff0,  R0,              Void,         Void,      Void,  Void)

INSTR (RETCC,         0x00005e20,  0x0000fff0,  R0,              Void,         Void,      Void,  Void)

INSTR (RETHS,         0x00005e20,  0x0000fff0,  R0,              Void,         Void,      Void,  Void)

INSTR (RETCS,         0x00005e30,  0x0000fff0,  R0,              Void,         Void,      Void,  Void)

INSTR (RETLO,         0x00005e30,  0x0000fff0,  R0,              Void,         Void,      Void,  Void)

INSTR (RETGE,         0x00005e40,  0x0000fff0,  R0,              Void,         Void,      Void,  Void)

INSTR (RETLT,         0x00005e50,  0x0000fff0,  R0,              Void,         Void,      Void,  Void)

INSTR (RETMI,         0x00005e60,  0x0000fff0,  R0,              Void,         Void,      Void,  Void)

INSTR (RETPL,         0x00005e70,  0x0000fff0,  R0,              Void,         Void,      Void,  Void)

INSTR (RETLS,         0x00005e80,  0x0000fff0,  R0,              Void,         Void,      Void,  Void)

INSTR (RETGT,         0x00005e90,  0x0000fff0,  R0,              Void,         Void,      Void,  Void)

INSTR (RETLE,         0x00005ea0,  0x0000fff0,  R0,              Void,         Void,      Void,  Void)

INSTR (RETHI,         0x00005eb0,  0x0000fff0,  R0,              Void,         Void,      Void,  Void)

INSTR (RETVS,         0x00005ec0,  0x0000fff0,  R0,              Void,         Void,      Void,  Void)

INSTR (RETVC,         0x00005ed0,  0x0000fff0,  R0,              Void,         Void,      Void,  Void)

INSTR (RETQS,         0x00005ee0,  0x0000fff0,  R0,              Void,         Void,      Void,  Void)

INSTR (RETAL,         0x00005ef0,  0x0000fff0,  R0,              Void,         Void,      Void,  Void)

INSTR (RETD,          0x0000d623,  0x0000ffff,  Void,            Void,         Void,      Void,  Void)

INSTR (RETE,          0x0000d603,  0x0000ffff,  Void,            Void,         Void,      Void,  Void)

INSTR (RETJ,          0x0000d633,  0x0000ffff,  Void,            Void,         Void,      Void,  Void)

INSTR (RETS,          0x0000d613,  0x0000ffff,  Void,            Void,         Void,      Void,  Void)

INSTR (RETSS,         0x0000d763,  0x0000ffff,  Void,            Void,         Void,      Void,  Void)

INSTR (RJMP,          0x0000c008,  0x0000f00c,  PPCS4802T2,      Void,         Void,      Void,  Void)

INSTR (ROL,           0x00005cf0,  0x0000fff0,  R0,              Void,         Void,      Void,  Void)

INSTR (ROR,           0x00005d00,  0x0000fff0,  R0,              Void,         Void,      Void,  Void)

INSTR (RSUB,          0x00000020,  0x0000e1f0,  R0,              R9,           Void,      Void,  Void)
INSTR (RSUB,          0xe0001100,  0xe1f0ff00,  R16,             R25,          S08,       Void,  Void)

INSTR (RSUBEQ,        0xfbb00000,  0xfff0ff00,  R16,             S08,          Void,      Void,  Void)

INSTR (RSUBNE,        0xfbb00100,  0xfff0ff00,  R16,             S08,          Void,      Void,  Void)

INSTR (RSUBCC,        0xfbb00200,  0xfff0ff00,  R16,             S08,          Void,      Void,  Void)

INSTR (RSUBHS,        0xfbb00200,  0xfff0ff00,  R16,             S08,          Void,      Void,  Void)

INSTR (RSUBCS,        0xfbb00300,  0xfff0ff00,  R16,             S08,          Void,      Void,  Void)

INSTR (RSUBLO,        0xfbb00300,  0xfff0ff00,  R16,             S08,          Void,      Void,  Void)

INSTR (RSUBGE,        0xfbb00400,  0xfff0ff00,  R16,             S08,          Void,      Void,  Void)

INSTR (RSUBLT,        0xfbb00500,  0xfff0ff00,  R16,             S08,          Void,      Void,  Void)

INSTR (RSUBMI,        0xfbb00600,  0xfff0ff00,  R16,             S08,          Void,      Void,  Void)

INSTR (RSUBPL,        0xfbb00700,  0xfff0ff00,  R16,             S08,          Void,      Void,  Void)

INSTR (RSUBLS,        0xfbb00800,  0xfff0ff00,  R16,             S08,          Void,      Void,  Void)

INSTR (RSUBGT,        0xfbb00900,  0xfff0ff00,  R16,             S08,          Void,      Void,  Void)

INSTR (RSUBLE,        0xfbb00a00,  0xfff0ff00,  R16,             S08,          Void,      Void,  Void)

INSTR (RSUBHI,        0xfbb00b00,  0xfff0ff00,  R16,             S08,          Void,      Void,  Void)

INSTR (RSUBVS,        0xfbb00c00,  0xfff0ff00,  R16,             S08,          Void,      Void,  Void)

INSTR (RSUBVC,        0xfbb00d00,  0xfff0ff00,  R16,             S08,          Void,      Void,  Void)

INSTR (RSUBQS,        0xfbb00e00,  0xfff0ff00,  R16,             S08,          Void,      Void,  Void)

INSTR (RSUBAL,        0xfbb00f00,  0xfff0ff00,  R16,             S08,          Void,      Void,  Void)

INSTR (SATADDH,       0xe00002c0,  0xe1f0fff0,  R0,              R25,          R16,       Void,  Void)

INSTR (SATADDW,       0xe00000c0,  0xe1f0fff0,  R0,              R25,          R16,       Void,  Void)

INSTR (SATRNDS,       0xf3b00000,  0xfff0fc00,  R16R05,          U55,          Void,      Void,  Void)

INSTR (SATRNDU,       0xf3b00400,  0xfff0fc00,  R16R05,          U55,          Void,      Void,  Void)

INSTR (SATS,          0xf1b00000,  0xfff0fc00,  R16R05,          U55,          Void,      Void,  Void)

INSTR (SATSUBH,       0xe00003c0,  0xe1f0fff0,  R0,              R25,          R16,       Void,  Void)

INSTR (SATSUBW,       0xe00001c0,  0xe1f0fff0,  R0,              R25,          R16,       Void,  Void)
INSTR (SATSUBW,       0xe0d00000,  0xe1f00000,  R16,             R25,          S016,      Void,  Void)

INSTR (SATU,          0xf1b00400,  0xfff0fc00,  R16R05,          U55,          Void,      Void,  Void)

INSTR (SBC,           0xe0000140,  0xe1f0fff0,  R0,              R25,          R16,       Void,  Void)

INSTR (SBR,           0x0000a1a0,  0x0000e1e0,  R0,              U4194,        Void,      Void,  Void)

INSTR (SCALL,         0x0000d733,  0x0000ffff,  Void,            Void,         Void,      Void,  Void)

INSTR (SCR,           0x00005c10,  0x0000fff0,  R0,              Void,         Void,      Void,  Void)

INSTR (SLEEP,         0xe9b00000,  0xffffff00,  U08,             Void,         Void,      Void,  Void)

INSTR (SREQ,          0x00005f00,  0x0000fff0,  R0,              Void,         Void,      Void,  Void)

INSTR (SRNE,          0x00005f10,  0x0000fff0,  R0,              Void,         Void,      Void,  Void)

INSTR (SRCC,          0x00005f20,  0x0000fff0,  R0,              Void,         Void,      Void,  Void)

INSTR (SRHS,          0x00005f20,  0x0000fff0,  R0,              Void,         Void,      Void,  Void)

INSTR (SRCS,          0x00005f30,  0x0000fff0,  R0,              Void,         Void,      Void,  Void)

INSTR (SRLO,          0x00005f30,  0x0000fff0,  R0,              Void,         Void,      Void,  Void)

INSTR (SRGE,          0x00005f40,  0x0000fff0,  R0,              Void,         Void,      Void,  Void)

INSTR (SRLT,          0x00005f50,  0x0000fff0,  R0,              Void,         Void,      Void,  Void)

INSTR (SRMI,          0x00005f60,  0x0000fff0,  R0,              Void,         Void,      Void,  Void)

INSTR (SRPL,          0x00005f70,  0x0000fff0,  R0,              Void,         Void,      Void,  Void)

INSTR (SRLS,          0x00005f80,  0x0000fff0,  R0,              Void,         Void,      Void,  Void)

INSTR (SRGT,          0x00005f90,  0x0000fff0,  R0,              Void,         Void,      Void,  Void)

INSTR (SRLE,          0x00005fa0,  0x0000fff0,  R0,              Void,         Void,      Void,  Void)

INSTR (SRHI,          0x00005fb0,  0x0000fff0,  R0,              Void,         Void,      Void,  Void)

INSTR (SRVS,          0x00005fc0,  0x0000fff0,  R0,              Void,         Void,      Void,  Void)

INSTR (SRVC,          0x00005fd0,  0x0000fff0,  R0,              Void,         Void,      Void,  Void)

INSTR (SRQS,          0x00005fe0,  0x0000fff0,  R0,              Void,         Void,      Void,  Void)

INSTR (SRAL,          0x00005ff0,  0x0000fff0,  R0,              Void,         Void,      Void,  Void)

INSTR (SSCALL,        0x0000d753,  0x0000ffff,  Void,            Void,         Void,      Void,  Void)

INSTR (SSRF,          0x0000d203,  0x0000fe0f,  U45,             Void,         Void,      Void,  Void)

INSTR (STB,           0x000000c0,  0x0000e1f0,  P9I,             R0,           Void,      Void,  Void)
INSTR (STB,           0x000000f0,  0x0000e1f0,  P9D,             R0,           Void,      Void,  Void)
INSTR (STB,           0x0000a080,  0x0000e180,  P9U43,           R0,           Void,      Void,  Void)
INSTR (STB,           0xe1600000,  0xe1f00000,  P25S016,         R16,          Void,      Void,  Void)
INSTR (STB,           0xe0000b00,  0xe1f0ffc0,  P25L16S42,       R0,           Void,      Void,  Void)

INSTR (STBEQ,         0xe1f00e00,  0xe1f0fe00,  P25U09,          R16,          Void,      Void,  Void)

INSTR (STBNE,         0xe1f01e00,  0xe1f0fe00,  P25U09,          R16,          Void,      Void,  Void)

INSTR (STBCC,         0xe1f02e00,  0xe1f0fe00,  P25U09,          R16,          Void,      Void,  Void)

INSTR (STBHS,         0xe1f02e00,  0xe1f0fe00,  P25U09,          R16,          Void,      Void,  Void)

INSTR (STBCS,         0xe1f03e00,  0xe1f0fe00,  P25U09,          R16,          Void,      Void,  Void)

INSTR (STBLO,         0xe1f03e00,  0xe1f0fe00,  P25U09,          R16,          Void,      Void,  Void)

INSTR (STBGE,         0xe1f04e00,  0xe1f0fe00,  P25U09,          R16,          Void,      Void,  Void)

INSTR (STBLT,         0xe1f05e00,  0xe1f0fe00,  P25U09,          R16,          Void,      Void,  Void)

INSTR (STBMI,         0xe1f06e00,  0xe1f0fe00,  P25U09,          R16,          Void,      Void,  Void)

INSTR (STBPL,         0xe1f07e00,  0xe1f0fe00,  P25U09,          R16,          Void,      Void,  Void)

INSTR (STBLS,         0xe1f08e00,  0xe1f0fe00,  P25U09,          R16,          Void,      Void,  Void)

INSTR (STBGT,         0xe1f09e00,  0xe1f0fe00,  P25U09,          R16,          Void,      Void,  Void)

INSTR (STBLE,         0xe1f0ae00,  0xe1f0fe00,  P25U09,          R16,          Void,      Void,  Void)

INSTR (STBHI,         0xe1f0be00,  0xe1f0fe00,  P25U09,          R16,          Void,      Void,  Void)

INSTR (STBVS,         0xe1f0ce00,  0xe1f0fe00,  P25U09,          R16,          Void,      Void,  Void)

INSTR (STBVC,         0xe1f0de00,  0xe1f0fe00,  P25U09,          R16,          Void,      Void,  Void)

INSTR (STBQS,         0xe1f0ee00,  0xe1f0fe00,  P25U09,          R16,          Void,      Void,  Void)

INSTR (STBAL,         0xe1f0ee00,  0xe1f0fe00,  P25U09,          R16,          Void,      Void,  Void)

INSTR (STD,           0x0000a120,  0x0000e1f1,  P9I,             R1D,          Void,      Void,  Void)
INSTR (STD,           0x0000a121,  0x0000e1f1,  P9D,             R1D,          Void,      Void,  Void)
INSTR (STD,           0x0000a111,  0x0000e1f1,  R9,              R1D,          Void,      Void,  Void)
INSTR (STD,           0xe0e10000,  0xe1f10000,  P25S016,         R16D,         Void,      Void,  Void)
INSTR (STD,           0xe0000800,  0xe1f0ffc0,  P25L16S42,       R1D,          Void,      Void,  Void)

INSTR (STH,           0x000000b0,  0x0000e1f0,  P9I,             R0,           Void,      Void,  Void)
INSTR (STH,           0x000000e0,  0x0000e1f0,  P9D,             R0,           Void,      Void,  Void)
INSTR (STH,           0x0000a000,  0x0000e180,  P9U43,           R0,           Void,      Void,  Void)
INSTR (STH,           0xe1500000,  0xe1f00000,  P25S016,         R16,          Void,      Void,  Void)
INSTR (STH,           0xe0000a00,  0xe1f0ffc0,  P25L16S42,       R0,           Void,      Void,  Void)

INSTR (STHEQ,         0xe1f00c00,  0xe1f0fe00,  P25U09T2,        R16,          Void,      Void,  Void)

INSTR (STHNE,         0xe1f01c00,  0xe1f0fe00,  P25U09T2,        R16,          Void,      Void,  Void)

INSTR (STHCC,         0xe1f02c00,  0xe1f0fe00,  P25U09T2,        R16,          Void,      Void,  Void)

INSTR (STHHS,         0xe1f02c00,  0xe1f0fe00,  P25U09T2,        R16,          Void,      Void,  Void)

INSTR (STHCS,         0xe1f03c00,  0xe1f0fe00,  P25U09T2,        R16,          Void,      Void,  Void)

INSTR (STHLO,         0xe1f03c00,  0xe1f0fe00,  P25U09T2,        R16,          Void,      Void,  Void)

INSTR (STHGE,         0xe1f04c00,  0xe1f0fe00,  P25U09T2,        R16,          Void,      Void,  Void)

INSTR (STHLT,         0xe1f05c00,  0xe1f0fe00,  P25U09T2,        R16,          Void,      Void,  Void)

INSTR (STHMI,         0xe1f06c00,  0xe1f0fe00,  P25U09T2,        R16,          Void,      Void,  Void)

INSTR (STHPL,         0xe1f07c00,  0xe1f0fe00,  P25U09T2,        R16,          Void,      Void,  Void)

INSTR (STHLS,         0xe1f08c00,  0xe1f0fe00,  P25U09T2,        R16,          Void,      Void,  Void)

INSTR (STHGT,         0xe1f09c00,  0xe1f0fe00,  P25U09T2,        R16,          Void,      Void,  Void)

INSTR (STHLE,         0xe1f0ac00,  0xe1f0fe00,  P25U09T2,        R16,          Void,      Void,  Void)

INSTR (STHHI,         0xe1f0bc00,  0xe1f0fe00,  P25U09T2,        R16,          Void,      Void,  Void)

INSTR (STHVS,         0xe1f0cc00,  0xe1f0fe00,  P25U09T2,        R16,          Void,      Void,  Void)

INSTR (STHVC,         0xe1f0dc00,  0xe1f0fe00,  P25U09T2,        R16,          Void,      Void,  Void)

INSTR (STHQS,         0xe1f0ec00,  0xe1f0fe00,  P25U09T2,        R16,          Void,      Void,  Void)

INSTR (STHAL,         0xe1f0ec00,  0xe1f0fe00,  P25U09T2,        R16,          Void,      Void,  Void)

INSTR (STW,           0x000000a0,  0x0000e1f0,  P9I,             R0,           Void,      Void,  Void)
INSTR (STW,           0x000000d0,  0x0000e1f0,  P9D,             R0,           Void,      Void,  Void)
INSTR (STW,           0x00008100,  0x0000e100,  P9U44T4,         R0,           Void,      Void,  Void)
INSTR (STW,           0xe1400000,  0xe1f00000,  P25S016,         R16,          Void,      Void,  Void)
INSTR (STW,           0xe0000900,  0xe1f0ffc0,  P25L16S42,       R0,           Void,      Void,  Void)

INSTR (STWEQ,         0xe1f00a00,  0xe1f0fe00,  P25U09T4,        R16,          Void,      Void,  Void)

INSTR (STWNE,         0xe1f01a00,  0xe1f0fe00,  P25U09T4,        R16,          Void,      Void,  Void)

INSTR (STWCC,         0xe1f02a00,  0xe1f0fe00,  P25U09T4,        R16,          Void,      Void,  Void)

INSTR (STWHS,         0xe1f02a00,  0xe1f0fe00,  P25U09T4,        R16,          Void,      Void,  Void)

INSTR (STWCS,         0xe1f03a00,  0xe1f0fe00,  P25U09T4,        R16,          Void,      Void,  Void)

INSTR (STWLO,         0xe1f03a00,  0xe1f0fe00,  P25U09T4,        R16,          Void,      Void,  Void)

INSTR (STWGE,         0xe1f04a00,  0xe1f0fe00,  P25U09T4,        R16,          Void,      Void,  Void)

INSTR (STWLT,         0xe1f05a00,  0xe1f0fe00,  P25U09T4,        R16,          Void,      Void,  Void)

INSTR (STWMI,         0xe1f06a00,  0xe1f0fe00,  P25U09T4,        R16,          Void,      Void,  Void)

INSTR (STWPL,         0xe1f07a00,  0xe1f0fe00,  P25U09T4,        R16,          Void,      Void,  Void)

INSTR (STWLS,         0xe1f08a00,  0xe1f0fe00,  P25U09T4,        R16,          Void,      Void,  Void)

INSTR (STWGT,         0xe1f09a00,  0xe1f0fe00,  P25U09T4,        R16,          Void,      Void,  Void)

INSTR (STWLE,         0xe1f0aa00,  0xe1f0fe00,  P25U09T4,        R16,          Void,      Void,  Void)

INSTR (STWHI,         0xe1f0ba00,  0xe1f0fe00,  P25U09T4,        R16,          Void,      Void,  Void)

INSTR (STWVS,         0xe1f0ca00,  0xe1f0fe00,  P25U09T4,        R16,          Void,      Void,  Void)

INSTR (STWVC,         0xe1f0da00,  0xe1f0fe00,  P25U09T4,        R16,          Void,      Void,  Void)

INSTR (STWQS,         0xe1f0ea00,  0xe1f0fe00,  P25U09T4,        R16,          Void,      Void,  Void)

INSTR (STWAL,         0xe1f0ea00,  0xe1f0fe00,  P25U09T4,        R16,          Void,      Void,  Void)

INSTR (STCD,          0xeba01000,  0xfff01100,  CP13,            P16U08T4,     CR9D,      Void,  Void)
INSTR (STCD,          0xefa00070,  0xfff011ff,  CP13,            P16D,         CR9D,      Void,  Void)
INSTR (STCD,          0xefa010c0,  0xfff011c0,  CP13,            P16L0S42,     CR9D,      Void,  Void)

INSTR (STCW,          0xeba00000,  0xfff01000,  CP13,            P16U08T4,     CR8,       Void,  Void)
INSTR (STCW,          0xefa00060,  0xfff010ff,  CP13,            P16D,         CR8,       Void,  Void)
INSTR (STCW,          0xefa11000,  0xfff010c0,  CP13,            P16L0S42,     CR8,       Void,  Void)

INSTR (STC0D,         0xf7a00000,  0xfff00100,  P16U08124T4,     CR9D,         Void,      Void,  Void)

INSTR (STC0W,         0xf5a00000,  0xfff00000,  P16U08124T4,     CR8,          Void,      Void,  Void)

INSTR (STCOND,        0xe1700000,  0xe1f00000,  P25S016,         R16,          Void,      Void,  Void)

INSTR (STDSP,         0x00005000,  0x0000f800,  PSPU47T4,        R0,           Void,      Void,  Void)

INSTR (STHHW,         0xe1e0c000,  0xe1f0c000,  P0U48T4,         R25P131,      R16P121,   Void,  Void)
INSTR (STHHW,         0xe1e08000,  0xe1f0c0c0,  P0L8S42,         R25P131,      R16P121,   Void,  Void)

INSTR (STSWPH,        0xe1d09000,  0xe1f0f000,  P25S012T2,       R16,          Void,      Void,  Void)

INSTR (STSWPW,        0xe1d0a000,  0xe1f0f000,  P25S012T4,       R16,          Void,      Void,  Void)

INSTR (SUB,           0x00000010,  0x0000e1f0,  R0,              R9,           Void,      Void,  Void)
INSTR (SUB,           0xe0000100,  0xe1f0ffc0,  R0,              R25,          R16L42,    Void,  Void)
INSTR (SUB,           0x0000200d,  0x0000f00f,  RSP,             S48T4,        Void,      Void,  Void)
INSTR (SUB,           0x00002000,  0x0000f000,  R0,              S48,          Void,      Void,  Void)
INSTR (SUB,           0xe0200000,  0xe1e00000,  R16,             S01620254,    Void,      Void,  Void)
INSTR (SUB,           0xe0c00000,  0xe1f00000,  R16,             R25,          S016,      Void,  Void)

INSTR (SUBEQ,         0xf5b00000,  0xfff0ff00,  R16,             S08,          Void,      Void,  Void)
INSTR (SUBEQ,         0xe1d0e010,  0xe1f0fff0,  R0,              R25,          R16,       Void,  Void)

INSTR (SUBNE,         0xf5b00100,  0xfff0ff00,  R16,             S08,          Void,      Void,  Void)
INSTR (SUBNE,         0xe1d0e110,  0xe1f0fff0,  R0,              R25,          R16,       Void,  Void)

INSTR (SUBCC,         0xf5b00200,  0xfff0ff00,  R16,             S08,          Void,      Void,  Void)
INSTR (SUBCC,         0xe1d0e210,  0xe1f0fff0,  R0,              R25,          R16,       Void,  Void)

INSTR (SUBHS,         0xf5b00200,  0xfff0ff00,  R16,             S08,          Void,      Void,  Void)
INSTR (SUBHS,         0xe1d0e210,  0xe1f0fff0,  R0,              R25,          R16,       Void,  Void)

INSTR (SUBCS,         0xf5b00300,  0xfff0ff00,  R16,             S08,          Void,      Void,  Void)
INSTR (SUBCS,         0xe1d0e310,  0xe1f0fff0,  R0,              R25,          R16,       Void,  Void)

INSTR (SUBLO,         0xf5b00300,  0xfff0ff00,  R16,             S08,          Void,      Void,  Void)
INSTR (SUBLO,         0xe1d0e310,  0xe1f0fff0,  R0,              R25,          R16,       Void,  Void)

INSTR (SUBGE,         0xf5b00400,  0xfff0ff00,  R16,             S08,          Void,      Void,  Void)
INSTR (SUBGE,         0xe1d0e410,  0xe1f0fff0,  R0,              R25,          R16,       Void,  Void)

INSTR (SUBLT,         0xf5b00500,  0xfff0ff00,  R16,             S08,          Void,      Void,  Void)
INSTR (SUBLT,         0xe1d0e510,  0xe1f0fff0,  R0,              R25,          R16,       Void,  Void)

INSTR (SUBMI,         0xf5b00600,  0xfff0ff00,  R16,             S08,          Void,      Void,  Void)
INSTR (SUBMI,         0xe1d0e610,  0xe1f0fff0,  R0,              R25,          R16,       Void,  Void)

INSTR (SUBPL,         0xf5b00700,  0xfff0ff00,  R16,             S08,          Void,      Void,  Void)
INSTR (SUBPL,         0xe1d0e710,  0xe1f0fff0,  R0,              R25,          R16,       Void,  Void)

INSTR (SUBLS,         0xf5b00800,  0xfff0ff00,  R16,             S08,          Void,      Void,  Void)
INSTR (SUBLS,         0xe1d0e810,  0xe1f0fff0,  R0,              R25,          R16,       Void,  Void)

INSTR (SUBGT,         0xf5b00900,  0xfff0ff00,  R16,             S08,          Void,      Void,  Void)
INSTR (SUBGT,         0xe1d0e910,  0xe1f0fff0,  R0,              R25,          R16,       Void,  Void)

INSTR (SUBLE,         0xf5b00a00,  0xfff0ff00,  R16,             S08,          Void,      Void,  Void)
INSTR (SUBLE,         0xe1d0ea10,  0xe1f0fff0,  R0,              R25,          R16,       Void,  Void)

INSTR (SUBHI,         0xf5b00b00,  0xfff0ff00,  R16,             S08,          Void,      Void,  Void)
INSTR (SUBHI,         0xe1d0eb10,  0xe1f0fff0,  R0,              R25,          R16,       Void,  Void)

INSTR (SUBVS,         0xf5b00c00,  0xfff0ff00,  R16,             S08,          Void,      Void,  Void)
INSTR (SUBVS,         0xe1d0ec10,  0xe1f0fff0,  R0,              R25,          R16,       Void,  Void)

INSTR (SUBVC,         0xf5b00d00,  0xfff0ff00,  R16,             S08,          Void,      Void,  Void)
INSTR (SUBVC,         0xe1d0ed10,  0xe1f0fff0,  R0,              R25,          R16,       Void,  Void)

INSTR (SUBQS,         0xf5b00e00,  0xfff0ff00,  R16,             S08,          Void,      Void,  Void)
INSTR (SUBQS,         0xe1d0ee10,  0xe1f0fff0,  R0,              R25,          R16,       Void,  Void)

INSTR (SUBAL,         0xf5b00e00,  0xfff0ff00,  R16,             S08,          Void,      Void,  Void)
INSTR (SUBAL,         0xe1d0ee10,  0xe1f0fff0,  R0,              R25,          R16,       Void,  Void)

INSTR (SUBFEQ,        0xf7b00000,  0xfff0ff00,  R16,             S08,          Void,      Void,  Void)

INSTR (SUBFNE,        0xf7b00100,  0xfff0ff00,  R16,             S08,          Void,      Void,  Void)

INSTR (SUBFCC,        0xf7b00200,  0xfff0ff00,  R16,             S08,          Void,      Void,  Void)

INSTR (SUBFHS,        0xf7b00200,  0xfff0ff00,  R16,             S08,          Void,      Void,  Void)

INSTR (SUBFCS,        0xf7b00300,  0xfff0ff00,  R16,             S08,          Void,      Void,  Void)

INSTR (SUBFLO,        0xf7b00300,  0xfff0ff00,  R16,             S08,          Void,      Void,  Void)

INSTR (SUBFGE,        0xf7b00400,  0xfff0ff00,  R16,             S08,          Void,      Void,  Void)

INSTR (SUBFLT,        0xf7b00500,  0xfff0ff00,  R16,             S08,          Void,      Void,  Void)

INSTR (SUBFMI,        0xf7b00600,  0xfff0ff00,  R16,             S08,          Void,      Void,  Void)

INSTR (SUBFPL,        0xf7b00700,  0xfff0ff00,  R16,             S08,          Void,      Void,  Void)

INSTR (SUBFLS,        0xf7b00800,  0xfff0ff00,  R16,             S08,          Void,      Void,  Void)

INSTR (SUBFGT,        0xf7b00900,  0xfff0ff00,  R16,             S08,          Void,      Void,  Void)

INSTR (SUBFLE,        0xf7b00a00,  0xfff0ff00,  R16,             S08,          Void,      Void,  Void)

INSTR (SUBFHI,        0xf7b00b00,  0xfff0ff00,  R16,             S08,          Void,      Void,  Void)

INSTR (SUBFVS,        0xf7b00c00,  0xfff0ff00,  R16,             S08,          Void,      Void,  Void)

INSTR (SUBFVC,        0xf7b00d00,  0xfff0ff00,  R16,             S08,          Void,      Void,  Void)

INSTR (SUBFQS,        0xf7b00e00,  0xfff0ff00,  R16,             S08,          Void,      Void,  Void)

INSTR (SUBFAL,        0xf7b00e00,  0xfff0ff00,  R16,             S08,          Void,      Void,  Void)

INSTR (SUBHHW,        0xe0000f00,  0xe1f0ffc0,  R0,              R25P51,       R16P41,    Void,  Void)

INSTR (SWAPB,         0x00005cb0,  0x0000fff0,  R0,              Void,         Void,      Void,  Void)

INSTR (SWAPBH,        0x00005cc0,  0x0000fff0,  R0,              Void,         Void,      Void,  Void)

INSTR (SWAPH,         0x00005ca0,  0x0000fff0,  R0,              Void,         Void,      Void,  Void)

INSTR (SYNC,          0xebb00000,  0xffffff00,  U08,             Void,         Void,      Void,  Void)

INSTR (TLBR,          0x0000d643,  0x0000ffff,  Void,            Void,         Void,      Void,  Void)

INSTR (TLBS,          0x0000d653,  0x0000ffff,  Void,            Void,         Void,      Void,  Void)

INSTR (TLBW,          0x0000d663,  0x0000ffff,  Void,            Void,         Void,      Void,  Void)

INSTR (TNBZ,          0x00005ce0,  0x0000fff0,  R0,              Void,         Void,      Void,  Void)

INSTR (TST,           0x00000070,  0x0000e1f0,  R0,              R9,           Void,      Void,  Void)

INSTR (XCHG,          0xe0000b40,  0xe1f0fff0,  R0,              R25,          R16,       Void,  Void)

// operand types

TYPE (RSP)
TYPE (R0)
TYPE (R0D)
TYPE (R1D)
TYPE (R9)
TYPE (R16)
TYPE (R16D)
TYPE (R16L42)
TYPE (R16L45)
TYPE (R16R05)
TYPE (R16R45)
TYPE (R16P41)
TYPE (R16P121)
TYPE (R16P122)
TYPE (R25)
TYPE (R25P41)
TYPE (R25P51)
TYPE (R25P131)
TYPE (U05)
TYPE (U08)
TYPE (U08T4)
TYPE (U016)
TYPE (U4194)
TYPE (U45)
TYPE (U48T4)
TYPE (U55)
TYPE (U115)
TYPE (U155)
TYPE (U163)
TYPE (U164)
TYPE (S08)
TYPE (S015T4)
TYPE (S016)
TYPE (S01620254)
TYPE (S01620254T2)
TYPE (S43WZ)
TYPE (S46)
TYPE (S48)
TYPE (S48T2)
TYPE (S48T4)
TYPE (P0U48T4)
TYPE (P0L8S42)
TYPE (P9I)
TYPE (P9D)
TYPE (P9U43)
TYPE (P9U43T2)
TYPE (P9U44T4)
TYPE (P9U45T4)
TYPE (P16D)
TYPE (P16U08T4)
TYPE (P16U08124T4)
TYPE (P16S011)
TYPE (P16S016)
TYPE (P16S016T4)
TYPE (P16L0S42)
TYPE (P25U09)
TYPE (P25U09T2)
TYPE (P25U09T4)
TYPE (P25S012)
TYPE (P25S012T2)
TYPE (P25S012T4)
TYPE (P25S016)
TYPE (P25L16S42)
TYPE (PPCU47T4)
TYPE (PPCS01620254T2)
TYPE (PPCS4802T2)
TYPE (PSPU47T4)
TYPE (CP13)
TYPE (CR0)
TYPE (CR4)
TYPE (CR8)
TYPE (CR9D)
TYPE (COP)
TYPE (COH)

#undef INSTR
#undef MNEM
#undef TYPE

% M68000 architecture documentation
% Copyright (C) Florian Negele

% This file is part of the Eigen Compiler Suite.

% Permission is granted to copy, distribute and/or modify this document
% under the terms of the GNU Free Documentation License, Version 1.3
% or any later version published by the Free Software Foundation.

% You should have received a copy of the GNU Free Documentation License
% along with the ECS.  If not, see <https://www.gnu.org/licenses/>.

% Generic documentation utilities
% Copyright (C) Florian Negele

% This file is part of the Eigen Compiler Suite.

% Permission is granted to copy, distribute and/or modify this document
% under the terms of the GNU Free Documentation License, Version 1.3
% or any later version published by the Free Software Foundation.

% You should have received a copy of the GNU Free Documentation License
% along with the ECS.  If not, see <https://www.gnu.org/licenses/>.

\providecommand{\cpp}{C\texttt{++}}
\providecommand{\opt}{_\mathit{opt}}
\providecommand{\tool}[1]{\texttt{#1}}
\providecommand{\version}{Version 0.0.40}
\providecommand{\resource}[1]{*++\txt{#1}}
\providecommand{\ecs}{Eigen Compiler Suite}
\providecommand{\changed}[1]{\underline{#1}}
\providecommand{\toolbox}[1]{\converter{#1}}
\providecommand{\file}{}\renewcommand{\file}[1]{\texttt{#1}}
\providecommand{\alignright}{\hfill\linebreak[0]\hspace*{\fill}}
\providecommand{\converter}[1]{*++[F][F*:white][F,:gray]\txt{#1}}
\providecommand{\documentation}{\ifbook chapter\else document\fi}
\providecommand{\Documentation}{\ifbook Chapter\else Document\fi}
\providecommand{\variable}[1]{\resource{\texttt{\small#1}\\variable}}
\providecommand{\documentationref}[2]{\ifbook\ref{#1}\else``\href{#1}{#2}''~\cite{#1}\fi}
\providecommand{\objfile}[1]{\texttt{#1}\index[runtime]{#1 object file@\texttt{#1} object file}}
\providecommand{\libfile}[1]{\texttt{#1}\index[runtime]{#1 library file@\texttt{#1} library file}}
\providecommand{\epigraph}[2]{\ifbook\begin{quote}\flushright\textit{#1}\par--- #2\end{quote}\fi}
\providecommand{\environmentvariable}[1]{\texttt{#1}\index{Environment variables!#1@\texttt{#1}}}
\providecommand{\environment}[1]{\texttt{#1}\index[environment]{#1 environment@\texttt{#1} environment}}
\providecommand{\toolsection}{}\renewcommand{\toolsection}[1]{\subsection{#1}\label{\prefix:#1}\tool{#1}}
\providecommand{\instruction}{}\renewcommand{\instruction}[2]{\noindent\qquad\pdftooltip{\texttt{#1}}{#2}\refstepcounter{instruction}\par}
\providecommand{\flowgraph}{}\renewcommand{\flowgraph}[1]{\par\sffamily\begin{displaymath}\xymatrix@=4ex{#1}\end{displaymath}\normalfont\par}
\providecommand{\instructionset}{}\renewcommand{\instructionset}[4]{\setcounter{instruction}{0}\begin{multicols}{\ifbook#3\else#4\fi}[{\captionof{table}[#2]{#2 (\ref*{#1:instructions}~instructions)}\label{tab:#1set}\vspace{-2ex}}]\footnotesize\raggedcolumns\input{#1.set}\label{#1:instructions}\end{multicols}}

\providecommand{\gpl}{GNU General Public License}
\providecommand{\rse}{ECS Runtime Support Exception}
\providecommand{\fdl}{\href{https://www.gnu.org/licenses/fdl.html}{GNU Free Documentation License}}

\providecommand{\docbegin}{}
\providecommand{\docend}{}
\providecommand{\doclabel}[1]{\hypertarget{#1}}
\providecommand{\doclink}[2]{\hyperlink{#1}{#2}}
\providecommand{\docsection}[3]{\hypertarget{#1}{\subsection{#2}}\label{sec:#1}\index[library]{#2@#3}}
\providecommand{\docsectionstar}[1]{}
\providecommand{\docsubbegin}{\begin{description}}
\providecommand{\docsubend}{\end{description}}
\providecommand{\docsubsection}[3]{\item[\hypertarget{#1}{#2}]\index[library]{#2@#3}}
\providecommand{\docsubsectionstar}[1]{\smallskip}
\providecommand{\docsubsubsection}[3]{\docsubsection{#1}{#2}{#3}}
\providecommand{\docsubsubsectionstar}[1]{}
\providecommand{\docsubsubsubsection}[3]{}
\providecommand{\docsubsubsubsectionstar}[1]{}
\providecommand{\doctable}{}

\providecommand{\debuggingtool}{}\renewcommand{\debuggingtool}{This tool is provided for debugging purposes.
It allows exposing and modifying an internal data structure that is usually not accessible.
}

\providecommand{\interface}{All tools accept command-line arguments which are taken as names of plain text files containing the source code.
If no arguments are provided, the standard input stream is used instead.
Output files are generated in the current working directory and have the same name as the input file being processed whereas the filename extension gets replaced by an appropriate suffix.
\seeinterface
}

\providecommand{\license}{\noindent Copyright \copyright{} Florian Negele\par\medskip\noindent
Permission is granted to copy, distribute and/or modify this document under the terms of the
\fdl{}, Version 1.3 or any later version published by the \href{https://fsf.org/}{Free Software Foundation}.
}

\providecommand{\ecslogosurface}{
\fill[darkgray] (0,0,0) -- (0,0,3) -- (0,3,3) -- (0,3,1) -- (0,4,1) -- (0,4,3) -- (0,5,3) -- (0,5,0) -- (0,2,0) -- (0,2,2) -- (0,1,2) -- (0,1,0) -- cycle;
\fill[gray] (0,5,0) -- (0,5,3) -- (1,5,3) -- (1,5,1) -- (2,5,1) -- (2,5,3) -- (3,5,3) -- (3,5,0) -- cycle;
\fill[lightgray] (0,0,0) -- (0,1,0) -- (2,1,0) -- (2,4,0) -- (1,4,0) -- (1,3,0) -- (2,3,0) -- (2,2,0) -- (0,2,0) -- (0,5,0) -- (3,5,0) -- (3,0,0) -- cycle;
\begin{scope}[line width=0.5]
\begin{scope}[gray]
\draw (0,0,0) -- (0,1,0);
\draw (2,1,0) -- (2,2,0);
\draw (0,1,2) -- (0,2,2);
\draw (0,2,0) -- (0,5,0);
\draw (2,3,0) -- (2,4,0);
\end{scope}
\begin{scope}[lightgray]
\draw (0,1,0) -- (0,1,2);
\draw (0,3,1) -- (0,3,3);
\draw (0,5,0) -- (0,5,3);
\draw (2,5,1) -- (2,5,3);
\end{scope}
\begin{scope}[white]
\draw (0,1,0) -- (2,1,0);
\draw (1,3,0) -- (2,3,0);
\draw (0,5,0) -- (3,5,0);
\end{scope}
\end{scope}
}

\providecommand{\ecslogo}[1]{
\begin{tikzpicture}[scale={(#1)/((sin(45)+cos(45))*3cm)},x={({-cos(45)*1cm},{sin(45)*sin(30)*1cm})},y={({0cm},{(cos(30)*1cm})},z={({sin(45)*1cm},{cos(45)*sin(30)*1cm})}]
\begin{scope}[darkgray,line width=1]
\draw (0,0,0) -- (0,0,3) -- (0,3,3) -- (2,3,3) -- (2,5,3) -- (3,5,3) -- (3,5,0) -- (3,0,0) -- cycle;
\draw (0,3,1) -- (0,4,1) -- (0,4,3) -- (0,5,3) -- (1,5,3) -- (1,5,1) -- (2,5,1);
\draw (1,3,0) -- (1,4,0) -- (2,4,0);
\end{scope}
\fill[darkgray] (2,0,0) -- (2,0,3) -- (2,5,3) -- (2,5,1) -- (2,4,1) -- (2,4,0) -- cycle;
\fill[lightgray] (2,0,2) -- (0,0,2) -- (0,2,2) -- (2,2,2) -- cycle;
\fill[gray] (0,1,0) -- (2,1,0) -- (2,1,2) -- (0,1,2) -- cycle;
\fill[gray] (0,3,1) -- (0,3,3) -- (2,3,3) -- (2,3,0) -- (1,3,0) -- (1,3,1) -- cycle;
\ecslogosurface
\end{tikzpicture}
}

\providecommand{\shadowedecslogo}[3]{
\begin{tikzpicture}[scale={(#1)/((sin(#2)+cos(#2))*3cm)},x={({-cos(#2)*1cm},{sin(#2)*sin(#3)*1cm})},y={({0cm},{(cos(#3)*1cm})},z={({sin(#2)*1cm},{cos(#2)*sin(#3)*1cm})}]
\shade[top color=lightgray!50!white,bottom color=white,middle color=lightgray!50!white] (0,0,0) -- (3,0,0) -- (3,{-0.5-3*sin(#2)*sin(#3)/cos(#3)},0) -- (0,-0.5,0) -- cycle;
\shade[top color=darkgray!50!gray,bottom color=white,middle color=darkgray!50!white] (0,0,0) -- (0,0,3) -- (0,{-0.5-3*cos(#2)*sin(#3)/cos(#3)},3) -- (0,-0.5,0) -- cycle;
\begin{scope}[y={({(cos(#2)+sin(#2))*0.5cm},{(cos(#2)*sin(#3)-sin(#2)*sin(#3))*0.5cm})}]
\useasboundingbox (3,0,0) -- (0,0,0) -- (0,0,3);
\shade[left color=darkgray!80!black,right color=lightgray,middle color=gray] (0,0,0) -- (0,1,0) -- (0,1,0.5) -- (0,2,0) -- (0,5,0) -- (0,5,3) -- (1,5,3) -- (1,4,3) -- (1,4,2.5) -- (1,3,3) -- (2,5,3) -- (3,5,3) -- (3,0,3) -- cycle;
\clip (0,0,0) -- (0,0,3) -- ({-3*sin(#2)/cos(#2)},0,0) -- cycle;
\shade[left color=darkgray,right color=lightgray!50!gray] (0,0,0) -- (0,1,0) -- (0,1,0.5) -- (0,2,0) -- (0,5,0) -- (0,5,3) -- (1,5,3) -- (1,4,3) -- (1,4,2.5) -- (1,3,3) -- (2,5,3) -- (3,5,3) -- (3,0,3) -- cycle;
\end{scope}
\shade[left color=darkgray,right color=darkgray!80!black] (2,0,0) -- (2,0,3) -- (2,5,3) -- (2,5,1) -- (2,4,1) -- (2,4,0) -- cycle;
\shade[left color=darkgray!90!black,right color=gray!80!darkgray] (2,0,2) -- (0,0,2) -- (0,2,2) -- (2,2,2) -- cycle;
\shade[top color=darkgray!90!black,bottom color=gray!80!darkgray] (0,1,0) -- (2,1,0) -- (2,1,2) -- (0,1,2) -- cycle;
\shade[top color=darkgray!90!black,bottom color=gray!80!darkgray] (0,3,1) -- (0,3,3) -- (2,3,3) -- (2,3,0) -- (1,3,0) -- (1,3,1) -- cycle;
\fill[gray] (2,1,0) -- (1.5,1,0.5) -- (0,1,0.5) -- (0,1,0) -- cycle;
\fill[gray] (1,3,2) -- (0.5,3,2) -- (0.5,3,3) -- (1,3,3) -- cycle;
\fill[gray] (2,3,0) -- (1.5,3,0.5) -- (1,3,0.5) -- (1,3,0) -- cycle;
\ecslogosurface
\end{tikzpicture}
}

\providecommand{\cpplogo}[1]{
\begin{tikzpicture}[scale=(#1)/512em]
\fill[gray] (435.2794,398.7159) -- (247.1911,507.3075) .. controls (236.3563,513.5642) and (218.6240,513.5642) .. (207.7892,507.3075) -- (19.7009,398.7159) .. controls (8.8646,392.4606) and (0.0000,377.1043) .. (0.0000,364.5924) -- (0.0000,147.4076) .. controls (0.8430,132.8363) and (8.2856,120.7683) .. (19.7009,113.2842) -- (207.7892,4.6926) .. controls (218.6240,-1.5642) and (236.3564,-1.5642) .. (247.1911,4.6926) -- (435.2794,113.2842) .. controls (447.5273,121.4304) and (454.4987,133.6918) .. (454.9803,147.4076) -- (454.9803,364.5924) .. controls (454.5404,377.7571) and (446.6566,391.0351) .. (435.2794,398.7159) -- cycle(75.8301,255.9993) .. controls (74.9389,404.0881) and (273.2892,469.4783) .. (358.8263,331.8769) -- (293.1917,293.8965) .. controls (253.5702,359.4301) and (155.1909,335.9977) .. (151.6601,255.9993) .. controls (152.7204,182.2703) and (249.4137,148.0211) .. (293.1961,218.1065) -- (358.8308,180.1276) .. controls (283.4477,49.2645) and (79.6318,96.3470) .. (75.8301,255.9993) -- cycle(379.1503,247.5747) -- (362.2982,247.5747) -- (362.2982,230.7226) -- (345.4490,230.7226) -- (345.4490,247.5747) -- (328.5969,247.5747) -- (328.5969,264.4254) -- (345.4490,264.4254) -- (345.4490,281.2759) -- (362.2982,281.2759) -- (362.2982,264.4254) -- (379.1503,264.4254) -- cycle(442.3420,247.5747) -- (425.4899,247.5747) -- (425.4899,230.7226) -- (408.6408,230.7226) -- (408.6408,247.5747) -- (391.7886,247.5747) -- (391.7886,264.4254) -- (408.6408,264.4254) -- (408.6408,281.2759) -- (425.4899,281.2759) -- (425.4899,264.4254) -- (442.3420,264.4254) -- cycle;
\end{tikzpicture}
}

\providecommand{\fallogo}[1]{
\begin{tikzpicture}[scale=(#1)/512em]
\fill[gray] (185.7774,0.0000) .. controls (200.4486,15.9798) and (226.8966,8.7148) .. (235.0426,31.5836) .. controls (249.5297,58.0598) and (247.9581,97.9161) .. (280.3335,110.9762) .. controls (309.1690,120.3496) and (337.8406,104.2727) .. (366.5753,103.9379) .. controls (373.4449,111.5171) and (379.2885,128.2574) .. (383.9755,108.9744) .. controls (396.6979,102.5615) and (437.2808,107.6681) .. (426.9652,124.3252) .. controls (408.9822,121.0785) and (412.4742,146.0729) .. (426.5192,131.4996) .. controls (433.8413,120.8489) and (465.1541,126.5522) .. (441.9067,135.7950) .. controls (396.1879,157.7478) and (344.1112,161.5079) .. (298.5528,183.5702) .. controls (277.7471,193.5198) and (284.6941,218.7163) .. (285.2127,236.9640) .. controls (292.3599,316.2826) and (307.3929,394.6311) .. (317.1198,473.6154) .. controls (329.0637,505.4736) and (292.1195,528.5004) .. (265.9183,511.2761) .. controls (237.9284,499.2462) and (237.3684,465.2681) .. (230.9102,439.9421) .. controls (218.6692,374.3397) and (215.6307,306.9662) .. (198.1732,242.3977) .. controls (183.1379,232.7444) and (164.4245,256.0298) .. (149.0430,261.4799) .. controls (116.9328,279.2585) and (87.1822,308.5851) .. (48.2293,307.8914) .. controls (21.3220,306.9037) and (-15.9107,281.8761) .. (7.2921,252.7908) .. controls (29.7799,220.6177) and (67.5177,204.3028) .. (100.9287,185.9449) .. controls (130.8217,170.8906) and (161.1548,156.5903) .. (191.0278,141.5847) .. controls (196.1738,120.0520) and (186.6049,95.2409) .. (186.8382,72.4353) .. controls (185.5234,48.4204) and (183.1700,23.9341) .. (185.7774,0.0000) -- cycle;
\end{tikzpicture}
}

\providecommand{\oblogo}[1]{
\begin{tikzpicture}[scale=(#1)/512em]
\fill[gray] (160.3865,208.9117) .. controls (154.0879,214.6478) and (149.0735,221.2409) .. (145.4125,228.5384) .. controls (184.8790,248.4273) and (234.7122,269.8787) .. (297.5493,291.8782) .. controls (300.3943,281.4769) and (300.9552,268.7619) .. (300.4023,255.2389) .. controls (248.9909,244.7891) and (200.0310,225.9279) .. (160.3865,208.9117) -- cycle(225.7398,392.6996) .. controls (308.0209,392.1716) and (359.3326,345.9277) .. (368.7203,285.2098) .. controls (376.6742,197.1784) and (311.7194,141.3342) .. (205.4287,142.1456) .. controls (139.9485,141.4804) and (88.7155,166.1957) .. (73.5775,228.0086) .. controls (52.0297,320.3408) and (123.4078,391.0103) .. (225.7398,392.6996) -- cycle(216.0739,176.4733) .. controls (268.9183,179.2424) and (315.8292,206.5488) .. (312.7454,265.1139) .. controls (313.2769,315.6384) and (286.5993,353.4946) .. (216.6040,355.7934) .. controls (162.4657,355.7934) and (126.0914,317.5023) .. (126.0914,260.5103) .. controls (126.1733,214.2900) and (163.3363,176.2849) .. (216.0739,176.4733) -- cycle(76.4897,189.1754) .. controls (13.1586,147.5631) and (0.0000,119.4207) .. (0.0000,119.4207) -- (90.6499,170.1632) .. controls (85.3004,175.8497) and (80.5994,182.1633) .. (76.4897,189.1754) -- cycle(353.9486,119.3004) -- (402.9482,119.3004) .. controls (427.0025,137.0797) and (450.9893,162.7034) .. (474.9529,191.0213) .. controls (509.3540,228.5339) and (531.3391,294.2091) .. (487.8149,312.1206) .. controls (462.8165,324.7652) and (394.3874,316.8943) .. (373.8912,313.6651) .. controls (379.9291,297.7449) and (383.2899,278.4204) .. (381.4989,257.7214) .. controls (420.3069,248.0321) and (421.9610,218.3461) .. (407.7867,192.6417) .. controls (391.1113,162.4018) and (370.1114,132.9097) .. (353.9486,119.3004) -- cycle;
\end{tikzpicture}
}

\providecommand{\markuptable}{
\begin{table}
\sffamily\centering
\begin{tabular}{@{}lcl@{}}
\toprule
\texttt{//italics//} & $\rightarrow$ & \textit{italics} \\
\midrule
\texttt{**bold**} & $\rightarrow$ & \textbf{bold} \\
\midrule
\texttt{\# ordered list} & & 1 ordered list \\
\texttt{\# second item} & $\rightarrow$ & 2 second item \\
\texttt{\#\# sub item} & & \hspace{1em} 1 sub item \\
\midrule
\texttt{* unordered list} & & $\bullet$ unordered list \\
\texttt{* second item} & $\rightarrow$ & $\bullet$ second item \\
\texttt{** sub item} & & \hspace{1em} $\bullet$ sub item \\
\midrule
\texttt{link to [[label]]} & $\rightarrow$ & link to \underline{label} \\
\midrule
\texttt{<{}<label>{}> definition } & $\rightarrow$ & definition \\
\midrule
\texttt{[[url|link name]]} & $\rightarrow$ & \underline{link name} \\
\midrule\addlinespace
\texttt{= large heading} & & {\Large large heading} \smallskip \\
\texttt{== medium heading} & $\rightarrow$ & {\large medium heading} \\
\texttt{=== small heading} & & small heading \\
\midrule
\texttt{no line break} & & no line break for paragraphs \\
\texttt{for paragraphs} & $\rightarrow$ \\
& & use empty line \\
\texttt{use empty line} \\
\midrule
\texttt{force\textbackslash\textbackslash line break} & $\rightarrow$ & force \\
& & line break \\
\midrule
\texttt{horizontal line} & $\rightarrow$ & horizontal line \\
\texttt{----} & & \hrulefill \\
\midrule
\texttt{|=a|=table|=header} & & \underline{a \enspace table \enspace header} \\
\texttt{|a|table|row} & $\rightarrow$ & a \enspace table \enspace row \\
\texttt{|b|table|row} & & b \enspace table \enspace row \\
\midrule
\texttt{\{\{\{} \\
\texttt{unformatted} & $\rightarrow$ & \texttt{unformatted} \\
\texttt{code} & & \texttt{code} \\
\texttt{\}\}\}} \\
\midrule\addlinespace
\texttt{@ new article} & & {\Large 1.\ new article} \smallskip \\
\texttt{@ second article} & $\rightarrow$ & {\Large 2.\ second article} \smallskip \\
\texttt{@@ sub article} & & {\large 2.1.\ sub article} \\
\bottomrule
\end{tabular}
\normalfont\caption{Elements of the generic documentation markup language}
\label{tab:docmarkup}
\end{table}
}

\providecommand{\startchapter}[4]{
\documentclass[11pt,a4paper]{article}
\usepackage{booktabs}
\usepackage[format=hang,labelfont=bf]{caption}
\usepackage{changepage}
\usepackage[T1]{fontenc}
\usepackage[margin=2cm]{geometry}
\usepackage{hyperref}
\usepackage[american]{isodate}
\usepackage{lmodern}
\usepackage{longtable}
\usepackage{mathptmx}
\usepackage{microtype}
\usepackage[toc]{multitoc}
\usepackage{multirow}
\usepackage[all]{nowidow}
\usepackage{pdfcomment}
\usepackage{syntax}
\usepackage{tikz}
\usepackage[all]{xy}
\hypersetup{pdfborder={0 0 0},bookmarksnumbered=true,pdftitle={\ecs{}: #2},pdfauthor={Florian Negele},pdfsubject={\ecs{}},pdfkeywords={#1}}
\setlength{\grammarindent}{8em}\setlength{\grammarparsep}{0.2ex}
\setlength{\columnsep}{2em}
\newcommand{\prefix}{}
\newcounter{instruction}
\bibliographystyle{unsrt}
\renewcommand{\index}[2][]{}
\renewcommand{\arraystretch}{1.05}
\renewcommand{\floatpagefraction}{0.7}
\renewcommand{\syntleft}{\itshape}\renewcommand{\syntright}{}
\title{\vspace{-5ex}\Huge{\ecs{}}\medskip\hrule}
\author{\huge{#2}}
\date{\medskip\version}
\newif\ifbook\bookfalse
\pagestyle{headings}
\frenchspacing
\begin{document}
\maketitle\thispagestyle{empty}\noindent#4\setlength{\columnseprule}{0.4pt}\tableofcontents\setlength{\columnseprule}{0pt}\vfill\pagebreak[3]\null\vfill\bigskip\noindent
\parbox{\textwidth-4em}{\license The contents of this \documentation{} are part of the \href{manual}{\ecs{} User Manual}~\cite{manual} and correspond to Chapter ``\href{manual\##3}{#1}''.\alignright\mbox{\today}}
\parbox{4em}{\flushright\ecslogo{3em}}
\clearpage
}

\providecommand{\concludechapter}{
\vfill\pagebreak[3]\null\vfill
\thispagestyle{myheadings}\markright{REFERENCES}
\noindent\begin{minipage}{\textwidth}\begin{multicols}{2}[\section*{References}]
\renewcommand{\section}[2]{}\small\bibliography{references}
\end{multicols}\end{minipage}\end{document}
}

\providecommand{\startpresentation}[2]{
\documentclass[14pt,aspectratio=43,usepdftitle=false]{beamer}
\usepackage{booktabs}
\usepackage{etex}
\usepackage{multicol}
\usepackage{tikz}
\usepackage[all]{xy}
\bibliographystyle{unsrt}
\setlength{\columnsep}{1em}
\setlength{\leftmargini}{1em}
\setbeamercolor{title}{fg=black}
\setbeamercolor{structure}{fg=darkgray}
\setbeamercolor{bibliography item}{fg=darkgray}
\setbeamerfont{title}{series=\bfseries}
\setbeamerfont{subtitle}{series=\normalfont}
\setbeamerfont*{frametitle}{parent=title}
\setbeamerfont{block title}{series=\bfseries}
\setbeamerfont*{framesubtitle}{parent=subtitle}
\setbeamersize{text margin left=1em,text margin right=1em}
\setbeamertemplate{navigation symbols}{}
\setbeamertemplate{itemize item}[circle]{}
\setbeamertemplate{bibliography item}[triangle]{}
\setbeamertemplate{bibliography entry author}{\usebeamercolor[fg]{bibliography item}}
\setbeamertemplate{frametitle}{\medskip\usebeamerfont{frametitle}\color{gray}\raisebox{-2.5ex}[0ex][0ex]{\rule{0.1em}{4.5ex}}}
\addtobeamertemplate{frametitle}{}{\hspace{0.4em}\usebeamercolor[fg]{title}\insertframetitle\par\vspace{0.2ex}\hspace{0.5em}\usebeamerfont{framesubtitle}\insertframesubtitle}
\hypersetup{pdfborder={0 0 0},bookmarksnumbered=true,bookmarksopen=true,bookmarksopenlevel=0,pdftitle={\ecs{}: #1},pdfauthor={Florian Negele},pdfsubject={\ecs{}},pdfkeywords={#1}}
\renewcommand{\flowgraph}[1]{\resizebox{\textwidth}{!}{$$\xymatrix{##1}$$}}
\title{\ecs{}\medskip\hrule\medskip}
\institute{\shadowedecslogo{5em}{30}{15}}
\date{\version}
\subtitle{#1}
\begin{document}
\begin{frame}[plain]\titlepage\nocite{manual}\end{frame}
\begin{frame}{Contents}{#1}\begin{center}\tableofcontents\end{center}\end{frame}
}

\providecommand{\concludepresentation}{
\begin{frame}{References}\begin{footnotesize}\setlength{\columnseprule}{0.4pt}\begin{multicols}{2}\bibliography{references}\end{multicols}\end{footnotesize}\end{frame}
\end{document}
}

\providecommand{\startbook}[1]{
\documentclass[10pt,paper=17cm:24cm,DIV=13,twoside=semi,headings=normal,numbers=noendperiod,cleardoublepage=plain]{scrbook}
\usepackage{atveryend}
\usepackage{booktabs}
\usepackage{caption}
\usepackage{changepage}
\usepackage[T1]{fontenc}
\usepackage{imakeidx}
\usepackage{hyperref}
\usepackage[american]{isodate}
\usepackage{lmodern}
\usepackage{longtable}
\usepackage{mathptmx}
\usepackage[final]{microtype}
\usepackage{multicol}
\usepackage{multirow}
\usepackage[all]{nowidow}
\usepackage{pdfcomment}
\usepackage{scrlayer-scrpage}
\usepackage{setspace}
\usepackage{syntax}
\usepackage[eventxtindent=4pt,oddtxtexdent=4pt]{thumbs}
\usepackage{tikz}
\usepackage[all]{xy}
\hyphenation{Micro-Blaze Open-Cores Open-RISC Power-PC}
\hypersetup{pdfborder={0 0 0},bookmarksnumbered=true,bookmarksopen=true,bookmarksopenlevel=0,pdftitle={\ecs{}: #1},pdfauthor={Florian Negele},pdfsubject={\ecs{}},pdfkeywords={#1}}
\setlength{\grammarindent}{8em}\setlength{\grammarparsep}{0.7ex}
\setkomafont{captionlabel}{\usekomafont{descriptionlabel}}
\renewcommand{\arraystretch}{1.05}\setstretch{1.1}
\renewcommand{\chapterformat}{\thechapter\autodot\enskip\raisebox{-1ex}[0ex][0ex]{\color{gray}\rule{0.1em}{3.5ex}}\enskip}
\renewcommand{\startchapter}[4]{\hypertarget{##3}{\chapter{##1}}\label{##3}##4\addthumb{##1}{\LARGE\sffamily\bfseries\thechapter}{white}{gray}\renewcommand{\prefix}{##3}}
\renewcommand{\concludechapter}{\clearpage{\stopthumb\cleardoublepage}}
\renewcommand{\syntleft}{\itshape}\renewcommand{\syntright}{}
\renewcommand{\floatpagefraction}{0.7}
\renewcommand{\partheademptypage}{}
\DeclareMicrotypeAlias{lmss}{cmr}
\newcommand{\prefix}{}
\newcounter{instruction}
\bibliographystyle{unsrt}
\newif\ifbook\booktrue
\makeindex[intoc,title=Index]
\makeindex[intoc,name=tools,title=Index of Tools,columns=3]
\makeindex[intoc,name=library,title=Index of Library Names]
\makeindex[intoc,name=runtime,title=Index of Runtime Support]
\makeindex[intoc,name=environment,title=Index of Target Environments]
\indexsetup{toclevel=chapter,headers={\indexname}{\indexname}}
\frenchspacing
\begin{document}
\pagenumbering{alph}
\begin{titlepage}\centering
\huge\sffamily\null\vfill\textbf{\ecs{}}\bigskip\hrule\bigskip#1
\normalsize\normalfont\vfill\vfill\shadowedecslogo{10em}{30}{15}
\large\vfill\vfill\version
\end{titlepage}
\null\vfill
\thispagestyle{empty}
\noindent\today\par\medskip
\license A copy of this license is included in Appendix~\ref{fdl} on page~\pageref{fdl}.
All product names used herein are for identification purposes only and may be trademarks of their respective companies.
\concludechapter
\frontmatter
\setcounter{tocdepth}{1}
\tableofcontents
\setcounter{tocdepth}{2}
\concludechapter
\listoffigures
\concludechapter
\listoftables
\concludechapter
}

\providecommand{\concludebook}{
\backmatter
\addtocontents{toc}{\protect\setcounter{tocdepth}{-1}}
\phantomsection\addcontentsline{toc}{part}{Bibliography}
\bibliography{references}
\concludechapter
\phantomsection\addcontentsline{toc}{part}{Indexes}
\printindex
\concludechapter
\indexprologue{\label{idx:tools}}
\printindex[tools]
\concludechapter
\printindex[library]
\concludechapter
\indexprologue{\label{idx:runtime}}
\printindex[runtime]
\concludechapter
\indexprologue{\label{idx:environment}}
\printindex[environment]
\concludechapter
\pagestyle{empty}\pagenumbering{Alph}\null\clearpage
\null\vfill\centering\ecslogo{4em}\par\medskip\license
\end{document}
}

% chapter references

\providecommand{\seedocumentationref}{}\renewcommand{\seedocumentationref}[3]{#1, see \Documentation{}~\documentationref{#2}{#3}. }
\providecommand{\seeinterface}{}\renewcommand{\seeinterface}{\ifbook See \Documentation{}~\documentationref{interface}{User Interface} for more information about the common user interface of all of these tools. \fi}
\providecommand{\seeguide}{}\renewcommand{\seeguide}{\seedocumentationref{For basic examples of using some of these tools in practice}{guide}{User Guide}}
\providecommand{\seecpp}{}\renewcommand{\seecpp}{\seedocumentationref{For more information about the \cpp{} programming language and its implementation by the \ecs{}}{cpp}{User Manual for \cpp{}}}
\providecommand{\seefalse}{}\renewcommand{\seefalse}{\seedocumentationref{For more information about the FALSE programming language and its implementation by the \ecs{}}{false}{User Manual for FALSE}}
\providecommand{\seeoberon}{}\renewcommand{\seeoberon}{\seedocumentationref{For more information about the Oberon programming language and its implementation by the \ecs{}}{oberon}{User Manual for Oberon}}
\providecommand{\seeassembly}{}\renewcommand{\seeassembly}{\seedocumentationref{For more information about the generic assembly language and how to use it}{assembly}{Generic Assembly Language Specification}}
\providecommand{\seeamd}{}\renewcommand{\seeamd}{\seedocumentationref{For more information about how the \ecs{} supports the AMD64 hardware architecture}{amd64}{AMD64 Hardware Architecture Support}}
\providecommand{\seearm}{}\renewcommand{\seearm}{\seedocumentationref{For more information about how the \ecs{} supports the ARM hardware architecture}{arm}{ARM Hardware Architecture Support}}
\providecommand{\seeavr}{}\renewcommand{\seeavr}{\seedocumentationref{For more information about how the \ecs{} supports the AVR hardware architecture}{avr}{AVR Hardware Architecture Support}}
\providecommand{\seeavrtt}{}\renewcommand{\seeavrtt}{\seedocumentationref{For more information about how the \ecs{} supports the AVR32 hardware architecture}{avr32}{AVR32 Hardware Architecture Support}}
\providecommand{\seemabk}{}\renewcommand{\seemabk}{\seedocumentationref{For more information about how the \ecs{} supports the M68000 hardware architecture}{m68k}{M68000 Hardware Architecture Support}}
\providecommand{\seemibl}{}\renewcommand{\seemibl}{\seedocumentationref{For more information about how the \ecs{} supports the MicroBlaze hardware architecture}{mibl}{MicroBlaze Hardware Architecture Support}}
\providecommand{\seemips}{}\renewcommand{\seemips}{\seedocumentationref{For more information about how the \ecs{} supports the MIPS32 and MIPS64 hardware architectures}{mips}{MIPS Hardware Architecture Support}}
\providecommand{\seemmix}{}\renewcommand{\seemmix}{\seedocumentationref{For more information about how the \ecs{} supports the MMIX hardware architecture}{mmix}{MMIX Hardware Architecture Support}}
\providecommand{\seeorok}{}\renewcommand{\seeorok}{\seedocumentationref{For more information about how the \ecs{} supports the OpenRISC 1000 hardware architecture}{or1k}{OpenRISC 1000 Hardware Architecture Support}}
\providecommand{\seeppc}{}\renewcommand{\seeppc}{\seedocumentationref{For more information about how the \ecs{} supports the PowerPC hardware architecture}{ppc}{PowerPC Hardware Architecture Support}}
\providecommand{\seerisc}{}\renewcommand{\seerisc}{\seedocumentationref{For more information about how the \ecs{} supports the RISC hardware architecture}{risc}{RISC Hardware Architecture Support}}
\providecommand{\seewasm}{}\renewcommand{\seewasm}{\seedocumentationref{For more information about how the \ecs{} supports the WebAssembly architecture}{wasm}{WebAssembly Architecture Support}}
\providecommand{\seedocumentation}{}\renewcommand{\seedocumentation}{\seedocumentationref{For more information about generic documentations and their generation by the \ecs{}}{documentation}{Generic Documentation Generation}}
\providecommand{\seedebugging}{}\renewcommand{\seedebugging}{\seedocumentationref{For more information about debugging information and its representation}{debugging}{Debugging Information Representation}}
\providecommand{\seecode}{}\renewcommand{\seecode}{\seedocumentationref{For more information about intermediate code and its purpose}{code}{Intermediate Code Representation}}
\providecommand{\seeobject}{}\renewcommand{\seeobject}{\seedocumentationref{For more information about object files and their purpose}{object}{Object File Representation}}

% generic documentation tools

\providecommand{\docprint}{
\toolsection{docprint} is a pretty printer for generic documentations.
It reformats generic documentations and writes it to the standard output stream.
\debuggingtool
\flowgraph{\resource{generic\\documentation} \ar[r] & \toolbox{docprint} \ar[r] & \resource{generic\\documentation}}
\seedocumentation
}

\providecommand{\doccheck}{
\toolsection{doccheck} is a syntactic and semantic checker for generic documentations.
It just performs syntactic and semantic checks on generic documentations and writes its diagnostic messages to the standard error stream.
\debuggingtool
\flowgraph{\resource{generic\\documentation} \ar[r] & \toolbox{doccheck} \ar[r] & \resource{diagnostic\\messages}}
\seedocumentation
}

\providecommand{\dochtml}{
\toolsection{dochtml} is an HTML documentation generator for generic documentations.
It processes several generic documentations and assembles all information therein into an HTML document.
\debuggingtool
\flowgraph{\resource{generic\\documentation} \ar[r] & \toolbox{dochtml} \ar[r] & \resource{HTML\\document}}
\seedocumentation
}

\providecommand{\doclatex}{
\toolsection{doclatex} is a Latex documentation generator for generic documentations.
It processes several generic documentations and assembles all information therein into a Latex document.
\debuggingtool
\flowgraph{\resource{generic\\documentation} \ar[r] & \toolbox{doclatex} \ar[r] & \resource{Latex\\document}}
\seedocumentation
}

% intermediate code tools

\providecommand{\cdcheck}{
\toolsection{cdcheck} is a syntactic and semantic checker for intermediate code.
It just performs syntactic and semantic checks on programs written in intermediate code and writes its diagnostic messages to the standard error stream.
\debuggingtool
\flowgraph{\resource{intermediate\\code} \ar[r] & \toolbox{cdcheck} \ar[r] & \resource{diagnostic\\messages}}
\seeassembly\seecode
}

\providecommand{\cdopt}{
\toolsection{cdopt} is an optimizer for intermediate code.
It performs various optimizations on programs written in intermediate code and writes the result to the standard output stream.
\debuggingtool
\flowgraph{\resource{intermediate\\code} \ar[r] & \toolbox{cdopt} \ar[r] & \resource{optimized\\code}}
\seeassembly\seecode
}

\providecommand{\cdrun}{
\toolsection{cdrun} is an interpreter for intermediate code.
It processes and executes programs written in intermediate code.
The following code sections are predefined and have the usual semantics:
\texttt{abort}, \texttt{\_Exit}, \texttt{fflush}, \texttt{floor}, \texttt{fputc}, \texttt{free}, \texttt{getchar}, \texttt{malloc}, and \texttt{putchar}.
Diagnostic messages about invalid operations include the name of the executed code section and the index of the erroneous instruction.
\debuggingtool
\flowgraph{\resource{intermediate\\code} \ar[r] & \toolbox{cdrun} \ar@/u/[r] & \resource{input/\\output} \ar@/d/[l]}
\seeassembly\seecode
}

\providecommand{\cdamda}{
\toolsection{cdamd16} is a compiler for intermediate code targeting the AMD64 hardware architecture.
It generates machine code for AMD64 processors from programs written in intermediate code and stores it in corresponding object files.
The compiler generates machine code for the 16-bit operating mode defined by the AMD64 architecture.
It also creates a debugging information file as well as an assembly file containing a listing of the generated machine code.
\debuggingtool
\flowgraph{\resource{intermediate\\code} \ar[r] & \toolbox{cdamd16} \ar[r] \ar[d] \ar[rd] & \resource{object file} \\ & \resource{assembly\\listing} & \resource{debugging\\information}}
\seeassembly\seeamd\seeobject\seecode\seedebugging
}

\providecommand{\cdamdb}{
\toolsection{cdamd32} is a compiler for intermediate code targeting the AMD64 hardware architecture.
It generates machine code for AMD64 processors from programs written in intermediate code and stores it in corresponding object files.
The compiler generates machine code for the 32-bit operating mode defined by the AMD64 architecture.
It also creates a debugging information file as well as an assembly file containing a listing of the generated machine code.
\debuggingtool
\flowgraph{\resource{intermediate\\code} \ar[r] & \toolbox{cdamd32} \ar[r] \ar[d] \ar[rd] & \resource{object file} \\ & \resource{assembly\\listing} & \resource{debugging\\information}}
\seeassembly\seeamd\seeobject\seecode\seedebugging
}

\providecommand{\cdamdc}{
\toolsection{cdamd64} is a compiler for intermediate code targeting the AMD64 hardware architecture.
It generates machine code for AMD64 processors from programs written in intermediate code and stores it in corresponding object files.
The compiler generates machine code for the 64-bit operating mode defined by the AMD64 architecture.
It also creates a debugging information file as well as an assembly file containing a listing of the generated machine code.
\debuggingtool
\flowgraph{\resource{intermediate\\code} \ar[r] & \toolbox{cdamd64} \ar[r] \ar[d] \ar[rd] & \resource{object file} \\ & \resource{assembly\\listing} & \resource{debugging\\information}}
\seeassembly\seeamd\seeobject\seecode\seedebugging
}

\providecommand{\cdarma}{
\toolsection{cdarma32} is a compiler for intermediate code targeting the ARM hardware architecture.
It generates machine code for ARM processors executing A32 instructions from programs written in intermediate code and stores it in corresponding object files.
It also creates a debugging information file as well as an assembly file containing a listing of the generated machine code.
\debuggingtool
\flowgraph{\resource{intermediate\\code} \ar[r] & \toolbox{cdarma32} \ar[r] \ar[d] \ar[rd] & \resource{object file} \\ & \resource{assembly\\listing} & \resource{debugging\\information}}
\seeassembly\seearm\seeobject\seecode\seedebugging
}

\providecommand{\cdarmb}{
\toolsection{cdarma64} is a compiler for intermediate code targeting the ARM hardware architecture.
It generates machine code for ARM processors executing A64 instructions from programs written in intermediate code and stores it in corresponding object files.
It also creates a debugging information file as well as an assembly file containing a listing of the generated machine code.
\debuggingtool
\flowgraph{\resource{intermediate\\code} \ar[r] & \toolbox{cdarma64} \ar[r] \ar[d] \ar[rd] & \resource{object file} \\ & \resource{assembly\\listing} & \resource{debugging\\information}}
\seeassembly\seearm\seeobject\seecode\seedebugging
}

\providecommand{\cdarmc}{
\toolsection{cdarmt32} is a compiler for intermediate code targeting the ARM hardware architecture.
It generates machine code for ARM processors without floating-point extension executing T32 instructions from programs written in intermediate code and stores it in corresponding object files.
It also creates a debugging information file as well as an assembly file containing a listing of the generated machine code.
\debuggingtool
\flowgraph{\resource{intermediate\\code} \ar[r] & \toolbox{cdarmt32} \ar[r] \ar[d] \ar[rd] & \resource{object file} \\ & \resource{assembly\\listing} & \resource{debugging\\information}}
\seeassembly\seearm\seeobject\seecode\seedebugging
}

\providecommand{\cdarmcfpe}{
\toolsection{cdarmt32fpe} is a compiler for intermediate code targeting the ARM hardware architecture.
It generates machine code for ARM processors with floating-point extension executing T32 instructions from programs written in intermediate code and stores it in corresponding object files.
It also creates a debugging information file as well as an assembly file containing a listing of the generated machine code.
\debuggingtool
\flowgraph{\resource{intermediate\\code} \ar[r] & \toolbox{cdarmt32fpe} \ar[r] \ar[d] \ar[rd] & \resource{object file} \\ & \resource{assembly\\listing} & \resource{debugging\\information}}
\seeassembly\seearm\seeobject\seecode\seedebugging
}

\providecommand{\cdavr}{
\toolsection{cdavr} is a compiler for intermediate code targeting the AVR hardware architecture.
It generates machine code for AVR processors from programs written in intermediate code and stores it in corresponding object files.
It also creates a debugging information file as well as an assembly file containing a listing of the generated machine code.
\debuggingtool
\flowgraph{\resource{intermediate\\code} \ar[r] & \toolbox{cdavr} \ar[r] \ar[d] \ar[rd] & \resource{object file} \\ & \resource{assembly\\listing} & \resource{debugging\\information}}
\seeassembly\seeavr\seeobject\seecode\seedebugging
}

\providecommand{\cdavrtt}{
\toolsection{cdavr32} is a compiler for intermediate code targeting the AVR32 hardware architecture.
It generates machine code for AVR32 processors from programs written in intermediate code and stores it in corresponding object files.
It also creates a debugging information file as well as an assembly file containing a listing of the generated machine code.
\debuggingtool
\flowgraph{\resource{intermediate\\code} \ar[r] & \toolbox{cdavr32} \ar[r] \ar[d] \ar[rd] & \resource{object file} \\ & \resource{assembly\\listing} & \resource{debugging\\information}}
\seeassembly\seeavrtt\seeobject\seecode\seedebugging
}

\providecommand{\cdmabk}{
\toolsection{cdm68k} is a compiler for intermediate code targeting the M68000 hardware architecture.
It generates machine code for M68000 processors from programs written in intermediate code and stores it in corresponding object files.
It also creates a debugging information file as well as an assembly file containing a listing of the generated machine code.
\debuggingtool
\flowgraph{\resource{intermediate\\code} \ar[r] & \toolbox{cdm68k} \ar[r] \ar[d] \ar[rd] & \resource{object file} \\ & \resource{assembly\\listing} & \resource{debugging\\information}}
\seeassembly\seemabk\seeobject\seecode\seedebugging
}

\providecommand{\cdmibl}{
\toolsection{cdmibl} is a compiler for intermediate code targeting the MicroBlaze hardware architecture.
It generates machine code for MicroBlaze processors from programs written in intermediate code and stores it in corresponding object files.
It also creates a debugging information file as well as an assembly file containing a listing of the generated machine code.
\debuggingtool
\flowgraph{\resource{intermediate\\code} \ar[r] & \toolbox{cdmibl} \ar[r] \ar[d] \ar[rd] & \resource{object file} \\ & \resource{assembly\\listing} & \resource{debugging\\information}}
\seeassembly\seemibl\seeobject\seecode\seedebugging
}

\providecommand{\cdmipsa}{
\toolsection{cdmips32} is a compiler for intermediate code targeting the MIPS32 hardware architecture.
It generates machine code for MIPS32 processors from programs written in intermediate code and stores it in corresponding object files.
It also creates a debugging information file as well as an assembly file containing a listing of the generated machine code.
\debuggingtool
\flowgraph{\resource{intermediate\\code} \ar[r] & \toolbox{cdmips32} \ar[r] \ar[d] \ar[rd] & \resource{object file} \\ & \resource{assembly\\listing} & \resource{debugging\\information}}
\seeassembly\seemips\seeobject\seecode\seedebugging
}

\providecommand{\cdmipsb}{
\toolsection{cdmips64} is a compiler for intermediate code targeting the MIPS64 hardware architecture.
It generates machine code for MIPS64 processors from programs written in intermediate code and stores it in corresponding object files.
It also creates a debugging information file as well as an assembly file containing a listing of the generated machine code.
\debuggingtool
\flowgraph{\resource{intermediate\\code} \ar[r] & \toolbox{cdmips64} \ar[r] \ar[d] \ar[rd] & \resource{object file} \\ & \resource{assembly\\listing} & \resource{debugging\\information}}
\seeassembly\seemips\seeobject\seecode\seedebugging
}

\providecommand{\cdmmix}{
\toolsection{cdmmix} is a compiler for intermediate code targeting the MMIX hardware architecture.
It generates machine code for MMIX processors from programs written in intermediate code and stores it in corresponding object files.
It also creates a debugging information file as well as an assembly file containing a listing of the generated machine code.
\debuggingtool
\flowgraph{\resource{intermediate\\code} \ar[r] & \toolbox{cdmmix} \ar[r] \ar[d] \ar[rd] & \resource{object file} \\ & \resource{assembly\\listing} & \resource{debugging\\information}}
\seeassembly\seemmix\seeobject\seecode\seedebugging
}

\providecommand{\cdorok}{
\toolsection{cdor1k} is a compiler for intermediate code targeting the OpenRISC 1000 hardware architecture.
It generates machine code for OpenRISC 1000 processors from programs written in intermediate code and stores it in corresponding object files.
It also creates a debugging information file as well as an assembly file containing a listing of the generated machine code.
\debuggingtool
\flowgraph{\resource{intermediate\\code} \ar[r] & \toolbox{cdor1k} \ar[r] \ar[d] \ar[rd] & \resource{object file} \\ & \resource{assembly\\listing} & \resource{debugging\\information}}
\seeassembly\seeorok\seeobject\seecode\seedebugging
}

\providecommand{\cdppca}{
\toolsection{cdppc32} is a compiler for intermediate code targeting the PowerPC hardware architecture.
It generates machine code for PowerPC processors from programs written in intermediate code and stores it in corresponding object files.
The compiler generates machine code for the 32-bit operating mode defined by the PowerPC architecture.
It also creates a debugging information file as well as an assembly file containing a listing of the generated machine code.
\debuggingtool
\flowgraph{\resource{intermediate\\code} \ar[r] & \toolbox{cdppc32} \ar[r] \ar[d] \ar[rd] & \resource{object file} \\ & \resource{assembly\\listing} & \resource{debugging\\information}}
\seeassembly\seeppc\seeobject\seecode\seedebugging
}

\providecommand{\cdppcb}{
\toolsection{cdppc64} is a compiler for intermediate code targeting the PowerPC hardware architecture.
It generates machine code for PowerPC processors from programs written in intermediate code and stores it in corresponding object files.
The compiler generates machine code for the 64-bit operating mode defined by the PowerPC architecture.
It also creates a debugging information file as well as an assembly file containing a listing of the generated machine code.
\debuggingtool
\flowgraph{\resource{intermediate\\code} \ar[r] & \toolbox{cdppc64} \ar[r] \ar[d] \ar[rd] & \resource{object file} \\ & \resource{assembly\\listing} & \resource{debugging\\information}}
\seeassembly\seeppc\seeobject\seecode\seedebugging
}

\providecommand{\cdrisc}{
\toolsection{cdrisc} is a compiler for intermediate code targeting the RISC hardware architecture.
It generates machine code for RISC processors from programs written in intermediate code and stores it in corresponding object files.
It also creates a debugging information file as well as an assembly file containing a listing of the generated machine code.
\debuggingtool
\flowgraph{\resource{intermediate\\code} \ar[r] & \toolbox{cdrisc} \ar[r] \ar[d] \ar[rd] & \resource{object file} \\ & \resource{assembly\\listing} & \resource{debugging\\information}}
\seeassembly\seerisc\seeobject\seecode\seedebugging
}

\providecommand{\cdwasm}{
\toolsection{cdwasm} is a compiler for intermediate code targeting the WebAssembly architecture.
It generates machine code for WebAssembly targets from programs written in intermediate code and stores it in corresponding object files.
It also creates a debugging information file as well as an assembly file containing a listing of the generated machine code.
\debuggingtool
\flowgraph{\resource{intermediate\\code} \ar[r] & \toolbox{cdwasm} \ar[r] \ar[d] \ar[rd] & \resource{object file} \\ & \resource{assembly\\listing} & \resource{debugging\\information}}
\seeassembly\seewasm\seeobject\seecode\seedebugging
}

% C++ tools

\providecommand{\cppprep}{
\toolsection{cppprep} is a preprocessor for the \cpp{} programming language.
It preprocesses source code according to the rules of \cpp{} and writes it to the standard output stream.
Only the macro names \texttt{\_\_DATE\_\_}, \texttt{\_\_FILE\_\_}, \texttt{\_\_LINE\_\_}, and \texttt{\_\_TIME\_\_} are predefined.
\flowgraph{\resource{\cpp{} or other\\source code} \ar[r] & \toolbox{cppprep} \ar[r] & \resource{preprocessed\\source code} \\ & \variable{ECSINCLUDE} \ar[u]}
\seecpp
}

\providecommand{\cppprint}{
\toolsection{cppprint} is a pretty printer for the \cpp{} programming language.
It reformats the source code of \cpp{} programs and writes it to the standard output stream.
\flowgraph{\resource{\cpp{}\\source code} \ar[r] & \toolbox{cppprint} \ar[r] & \resource{reformatted\\source code} \\ & \variable{ECSINCLUDE} \ar[u]}
\seecpp
}

\providecommand{\cppcheck}{
\toolsection{cppcheck} is a syntactic and semantic checker for the \cpp{} programming language.
It just performs syntactic and semantic checks on \cpp{} programs and writes its diagnostic messages to the standard error stream.
\flowgraph{\resource{\cpp{}\\source code} \ar[r] & \toolbox{cppcheck} \ar[r] & \resource{diagnostic\\messages} \\ & \variable{ECSINCLUDE} \ar[u]}
\seecpp
}

\providecommand{\cppdump}{
\toolsection{cppdump} is a serializer for the \cpp{} programming language.
It dumps the complete internal representation of programs written in \cpp{} into an XML document.
\debuggingtool
\flowgraph{\resource{\cpp{}\\source code} \ar[r] & \toolbox{cppdump} \ar[r] & \resource{internal\\representation} \\ & \variable{ECSINCLUDE} \ar[u]}
\seecpp
}

\providecommand{\cpprun}{
\toolsection{cpprun} is an interpreter for the \cpp{} programming language.
It processes and executes programs written in \cpp{}.
The macro \texttt{\_\_run\_\_} is predefined in order to enable programmers to identify this tool while interpreting.
\flowgraph{\resource{\cpp{}\\source code} \ar[r] & \toolbox{cpprun} \ar@/u/[r] & \resource{input/\\output} \ar@/d/[l] \\ & \variable{ECSINCLUDE} \ar[u]}
\seecpp
}

\providecommand{\cppdoc}{
\toolsection{cppdoc} is a generic documentation generator for the \cpp{} programming language.
It processes several \cpp{} source files and assembles all information therein into a generic documentation.
\debuggingtool
\flowgraph{\resource{\cpp{}\\source code} \ar[r] & \toolbox{cppdoc} \ar[r] & \resource{generic\\documentation} \\ & \variable{ECSINCLUDE} \ar[u]}
\seecpp\seedocumentation
}

\providecommand{\cpphtml}{
\toolsection{cpphtml} is an HTML documentation generator for the \cpp{} programming language.
It processes several \cpp{} source files and assembles all information therein into an HTML document.
\flowgraph{\resource{\cpp{}\\source code} \ar[r] & \toolbox{cpphtml} \ar[r] & \resource{HTML\\document} \\ & \variable{ECSINCLUDE} \ar[u]}
\seecpp\seedocumentation
}

\providecommand{\cpplatex}{
\toolsection{cpplatex} is a Latex documentation generator for the \cpp{} programming language.
It processes several \cpp{} source files and assembles all information therein into a Latex document.
\flowgraph{\resource{\cpp{}\\source code} \ar[r] & \toolbox{cpplatex} \ar[r] & \resource{Latex\\document} \\ & \variable{ECSINCLUDE} \ar[u]}
\seecpp\seedocumentation
}

\providecommand{\cppcode}{
\toolsection{cppcode} is an intermediate code generator for the \cpp{} programming language.
It generates intermediate code from programs written in \cpp{} and stores it in corresponding assembly files.
The macro \texttt{\_\_code\_\_} is predefined in order to enable programmers to identify this tool while generating intermediate code.
Programs generated with this tool require additional runtime support that is stored in the \file{cpp\-code\-run} library file.
\debuggingtool
\flowgraph{\resource{\cpp{}\\source code} \ar[r] & \toolbox{cppcode} \ar[r] & \resource{intermediate\\code} \\ & \variable{ECSINCLUDE} \ar[u]}
\seecpp\seeassembly\seecode
}

\providecommand{\cppamda}{
\toolsection{cppamd16} is a compiler for the \cpp{} programming language targeting the AMD64 hardware architecture.
It generates machine code for AMD64 processors from programs written in \cpp{} and stores it in corresponding object files.
The compiler generates machine code for the 16-bit operating mode defined by the AMD64 architecture.
For debugging purposes, it also creates a debugging information file as well as an assembly file containing a listing of the generated machine code.
The macro \texttt{\_\_amd16\_\_} is predefined in order to enable programmers to identify this tool and its target architecture while compiling.
Programs generated with this compiler require additional runtime support that is stored in the \file{cpp\-amd16\-run} library file.
\flowgraph{\resource{\cpp{}\\source code} \ar[r] & \toolbox{cppamd16} \ar[r] \ar[d] \ar[rd] & \resource{object file} \\ \variable{ECSINCLUDE} \ar[ru] & \resource{debugging\\information} & \resource{assembly\\listing}}
\seecpp\seeassembly\seeamd\seeobject\seedebugging
}

\providecommand{\cppamdb}{
\toolsection{cppamd32} is a compiler for the \cpp{} programming language targeting the AMD64 hardware architecture.
It generates machine code for AMD64 processors from programs written in \cpp{} and stores it in corresponding object files.
The compiler generates machine code for the 32-bit operating mode defined by the AMD64 architecture.
For debugging purposes, it also creates a debugging information file as well as an assembly file containing a listing of the generated machine code.
The macro \texttt{\_\_amd32\_\_} is predefined in order to enable programmers to identify this tool and its target architecture while compiling.
Programs generated with this compiler require additional runtime support that is stored in the \file{cpp\-amd32\-run} library file.
\flowgraph{\resource{\cpp{}\\source code} \ar[r] & \toolbox{cppamd32} \ar[r] \ar[d] \ar[rd] & \resource{object file} \\ \variable{ECSINCLUDE} \ar[ru] & \resource{debugging\\information} & \resource{assembly\\listing}}
\seecpp\seeassembly\seeamd\seeobject\seedebugging
}

\providecommand{\cppamdc}{
\toolsection{cppamd64} is a compiler for the \cpp{} programming language targeting the AMD64 hardware architecture.
It generates machine code for AMD64 processors from programs written in \cpp{} and stores it in corresponding object files.
The compiler generates machine code for the 64-bit operating mode defined by the AMD64 architecture.
For debugging purposes, it also creates a debugging information file as well as an assembly file containing a listing of the generated machine code.
The macro \texttt{\_\_amd64\_\_} is predefined in order to enable programmers to identify this tool and its target architecture while compiling.
Programs generated with this compiler require additional runtime support that is stored in the \file{cpp\-amd64\-run} library file.
\flowgraph{\resource{\cpp{}\\source code} \ar[r] & \toolbox{cppamd64} \ar[r] \ar[d] \ar[rd] & \resource{object file} \\ \variable{ECSINCLUDE} \ar[ru] & \resource{debugging\\information} & \resource{assembly\\listing}}
\seecpp\seeassembly\seeamd\seeobject\seedebugging
}

\providecommand{\cpparma}{
\toolsection{cpparma32} is a compiler for the \cpp{} programming language targeting the ARM hardware architecture.
It generates machine code for ARM processors executing A32 instructions from programs written in \cpp{} and stores it in corresponding object files.
For debugging purposes, it also creates a debugging information file as well as an assembly file containing a listing of the generated machine code.
The macro \texttt{\_\_arma32\_\_} is predefined in order to enable programmers to identify this tool and its target architecture while compiling.
Programs generated with this compiler require additional runtime support that is stored in the \file{cpp\-arma32\-run} library file.
\flowgraph{\resource{\cpp{}\\source code} \ar[r] & \toolbox{cpparma32} \ar[r] \ar[d] \ar[rd] & \resource{object file} \\ \variable{ECSINCLUDE} \ar[ru] & \resource{debugging\\information} & \resource{assembly\\listing}}
\seecpp\seeassembly\seearm\seeobject\seedebugging
}

\providecommand{\cpparmb}{
\toolsection{cpparma64} is a compiler for the \cpp{} programming language targeting the ARM hardware architecture.
It generates machine code for ARM processors executing A64 instructions from programs written in \cpp{} and stores it in corresponding object files.
For debugging purposes, it also creates a debugging information file as well as an assembly file containing a listing of the generated machine code.
The macro \texttt{\_\_arma64\_\_} is predefined in order to enable programmers to identify this tool and its target architecture while compiling.
Programs generated with this compiler require additional runtime support that is stored in the \file{cpp\-arma64\-run} library file.
\flowgraph{\resource{\cpp{}\\source code} \ar[r] & \toolbox{cpparma64} \ar[r] \ar[d] \ar[rd] & \resource{object file} \\ \variable{ECSINCLUDE} \ar[ru] & \resource{debugging\\information} & \resource{assembly\\listing}}
\seecpp\seeassembly\seearm\seeobject\seedebugging
}

\providecommand{\cpparmc}{
\toolsection{cpparmt32} is a compiler for the \cpp{} programming language targeting the ARM hardware architecture.
It generates machine code for ARM processors without floating-point extension executing T32 instructions from programs written in \cpp{} and stores it in corresponding object files.
For debugging purposes, it also creates a debugging information file as well as an assembly file containing a listing of the generated machine code.
The macro \texttt{\_\_armt32\_\_} is predefined in order to enable programmers to identify this tool and its target architecture while compiling.
Programs generated with this compiler require additional runtime support that is stored in the \file{cpp\-armt32\-run} library file.
\flowgraph{\resource{\cpp{}\\source code} \ar[r] & \toolbox{cpparmt32} \ar[r] \ar[d] \ar[rd] & \resource{object file} \\ \variable{ECSINCLUDE} \ar[ru] & \resource{debugging\\information} & \resource{assembly\\listing}}
\seecpp\seeassembly\seearm\seeobject\seedebugging
}

\providecommand{\cpparmcfpe}{
\toolsection{cpparmt32fpe} is a compiler for the \cpp{} programming language targeting the ARM hardware architecture.
It generates machine code for ARM processors with floating-point extension executing T32 instructions from programs written in \cpp{} and stores it in corresponding object files.
For debugging purposes, it also creates a debugging information file as well as an assembly file containing a listing of the generated machine code.
The macro \texttt{\_\_armt32fpe\_\_} is predefined in order to enable programmers to identify this tool and its target architecture while compiling.
Programs generated with this compiler require additional runtime support that is stored in the \file{cpp\-armt32\-fpe\-run} library file.
\flowgraph{\resource{\cpp{}\\source code} \ar[r] & \toolbox{cpparmt32fpe} \ar[r] \ar[d] \ar[rd] & \resource{object file} \\ \variable{ECSINCLUDE} \ar[ru] & \resource{debugging\\information} & \resource{assembly\\listing}}
\seecpp\seeassembly\seearm\seeobject\seedebugging
}

\providecommand{\cppavr}{
\toolsection{cppavr} is a compiler for the \cpp{} programming language targeting the AVR hardware architecture.
It generates machine code for AVR processors from programs written in \cpp{} and stores it in corresponding object files.
For debugging purposes, it also creates a debugging information file as well as an assembly file containing a listing of the generated machine code.
The macro \texttt{\_\_avr\_\_} is predefined in order to enable programmers to identify this tool and its target architecture while compiling.
Programs generated with this compiler require additional runtime support that is stored in the \file{cpp\-avr\-run} library file.
\flowgraph{\resource{\cpp{}\\source code} \ar[r] & \toolbox{cppavr} \ar[r] \ar[d] \ar[rd] & \resource{object file} \\ \variable{ECSINCLUDE} \ar[ru] & \resource{debugging\\information} & \resource{assembly\\listing}}
\seecpp\seeassembly\seeavr\seeobject\seedebugging
}

\providecommand{\cppavrtt}{
\toolsection{cppavr32} is a compiler for the \cpp{} programming language targeting the AVR32 hardware architecture.
It generates machine code for AVR32 processors from programs written in \cpp{} and stores it in corresponding object files.
For debugging purposes, it also creates a debugging information file as well as an assembly file containing a listing of the generated machine code.
The macro \texttt{\_\_avr32\_\_} is predefined in order to enable programmers to identify this tool and its target architecture while compiling.
Programs generated with this compiler require additional runtime support that is stored in the \file{cpp\-avr32\-run} library file.
\flowgraph{\resource{\cpp{}\\source code} \ar[r] & \toolbox{cppavr32} \ar[r] \ar[d] \ar[rd] & \resource{object file} \\ \variable{ECSINCLUDE} \ar[ru] & \resource{debugging\\information} & \resource{assembly\\listing}}
\seecpp\seeassembly\seeavrtt\seeobject\seedebugging
}

\providecommand{\cppmabk}{
\toolsection{cppm68k} is a compiler for the \cpp{} programming language targeting the M68000 hardware architecture.
It generates machine code for M68000 processors from programs written in \cpp{} and stores it in corresponding object files.
For debugging purposes, it also creates a debugging information file as well as an assembly file containing a listing of the generated machine code.
The macro \texttt{\_\_m68k\_\_} is predefined in order to enable programmers to identify this tool and its target architecture while compiling.
Programs generated with this compiler require additional runtime support that is stored in the \file{cpp\-m68k\-run} library file.
\flowgraph{\resource{\cpp{}\\source code} \ar[r] & \toolbox{cppm68k} \ar[r] \ar[d] \ar[rd] & \resource{object file} \\ \variable{ECSINCLUDE} \ar[ru] & \resource{debugging\\information} & \resource{assembly\\listing}}
\seecpp\seeassembly\seemabk\seeobject\seedebugging
}

\providecommand{\cppmibl}{
\toolsection{cppmibl} is a compiler for the \cpp{} programming language targeting the MicroBlaze hardware architecture.
It generates machine code for MicroBlaze processors from programs written in \cpp{} and stores it in corresponding object files.
For debugging purposes, it also creates a debugging information file as well as an assembly file containing a listing of the generated machine code.
The macro \texttt{\_\_mibl\_\_} is predefined in order to enable programmers to identify this tool and its target architecture while compiling.
Programs generated with this compiler require additional runtime support that is stored in the \file{cpp\-mibl\-run} library file.
\flowgraph{\resource{\cpp{}\\source code} \ar[r] & \toolbox{cppmibl} \ar[r] \ar[d] \ar[rd] & \resource{object file} \\ \variable{ECSINCLUDE} \ar[ru] & \resource{debugging\\information} & \resource{assembly\\listing}}
\seecpp\seeassembly\seemibl\seeobject\seedebugging
}

\providecommand{\cppmipsa}{
\toolsection{cppmips32} is a compiler for the \cpp{} programming language targeting the MIPS32 hardware architecture.
It generates machine code for MIPS32 processors from programs written in \cpp{} and stores it in corresponding object files.
For debugging purposes, it also creates a debugging information file as well as an assembly file containing a listing of the generated machine code.
The macro \texttt{\_\_mips32\_\_} is predefined in order to enable programmers to identify this tool and its target architecture while compiling.
Programs generated with this compiler require additional runtime support that is stored in the \file{cpp\-mips32\-run} library file.
\flowgraph{\resource{\cpp{}\\source code} \ar[r] & \toolbox{cppmips32} \ar[r] \ar[d] \ar[rd] & \resource{object file} \\ \variable{ECSINCLUDE} \ar[ru] & \resource{debugging\\information} & \resource{assembly\\listing}}
\seecpp\seeassembly\seemips\seeobject\seedebugging
}

\providecommand{\cppmipsb}{
\toolsection{cppmips64} is a compiler for the \cpp{} programming language targeting the MIPS64 hardware architecture.
It generates machine code for MIPS64 processors from programs written in \cpp{} and stores it in corresponding object files.
For debugging purposes, it also creates a debugging information file as well as an assembly file containing a listing of the generated machine code.
The macro \texttt{\_\_mips64\_\_} is predefined in order to enable programmers to identify this tool and its target architecture while compiling.
Programs generated with this compiler require additional runtime support that is stored in the \file{cpp\-mips64\-run} library file.
\flowgraph{\resource{\cpp{}\\source code} \ar[r] & \toolbox{cppmips64} \ar[r] \ar[d] \ar[rd] & \resource{object file} \\ \variable{ECSINCLUDE} \ar[ru] & \resource{debugging\\information} & \resource{assembly\\listing}}
\seecpp\seeassembly\seemips\seeobject\seedebugging
}

\providecommand{\cppmmix}{
\toolsection{cppmmix} is a compiler for the \cpp{} programming language targeting the MMIX hardware architecture.
It generates machine code for MMIX processors from programs written in \cpp{} and stores it in corresponding object files.
For debugging purposes, it also creates a debugging information file as well as an assembly file containing a listing of the generated machine code.
The macro \texttt{\_\_mmix\_\_} is predefined in order to enable programmers to identify this tool and its target architecture while compiling.
Programs generated with this compiler require additional runtime support that is stored in the \file{cpp\-mmix\-run} library file.
\flowgraph{\resource{\cpp{}\\source code} \ar[r] & \toolbox{cppmmix} \ar[r] \ar[d] \ar[rd] & \resource{object file} \\ \variable{ECSINCLUDE} \ar[ru] & \resource{debugging\\information} & \resource{assembly\\listing}}
\seecpp\seeassembly\seemmix\seeobject\seedebugging
}

\providecommand{\cpporok}{
\toolsection{cppor1k} is a compiler for the \cpp{} programming language targeting the OpenRISC 1000 hardware architecture.
It generates machine code for OpenRISC 1000 processors from programs written in \cpp{} and stores it in corresponding object files.
For debugging purposes, it also creates a debugging information file as well as an assembly file containing a listing of the generated machine code.
The macro \texttt{\_\_or1k\_\_} is predefined in order to enable programmers to identify this tool and its target architecture while compiling.
Programs generated with this compiler require additional runtime support that is stored in the \file{cpp\-or1k\-run} library file.
\flowgraph{\resource{\cpp{}\\source code} \ar[r] & \toolbox{cppor1k} \ar[r] \ar[d] \ar[rd] & \resource{object file} \\ \variable{ECSINCLUDE} \ar[ru] & \resource{debugging\\information} & \resource{assembly\\listing}}
\seecpp\seeassembly\seeorok\seeobject\seedebugging
}

\providecommand{\cppppca}{
\toolsection{cppppc32} is a compiler for the \cpp{} programming language targeting the PowerPC hardware architecture.
It generates machine code for PowerPC processors from programs written in \cpp{} and stores it in corresponding object files.
The compiler generates machine code for the 32-bit operating mode defined by the PowerPC architecture.
For debugging purposes, it also creates a debugging information file as well as an assembly file containing a listing of the generated machine code.
The macro \texttt{\_\_ppc32\_\_} is predefined in order to enable programmers to identify this tool and its target architecture while compiling.
Programs generated with this compiler require additional runtime support that is stored in the \file{cpp\-ppc32\-run} library file.
\flowgraph{\resource{\cpp{}\\source code} \ar[r] & \toolbox{cppppc32} \ar[r] \ar[d] \ar[rd] & \resource{object file} \\ \variable{ECSINCLUDE} \ar[ru] & \resource{debugging\\information} & \resource{assembly\\listing}}
\seecpp\seeassembly\seeppc\seeobject\seedebugging
}

\providecommand{\cppppcb}{
\toolsection{cppppc64} is a compiler for the \cpp{} programming language targeting the PowerPC hardware architecture.
It generates machine code for PowerPC processors from programs written in \cpp{} and stores it in corresponding object files.
The compiler generates machine code for the 64-bit operating mode defined by the PowerPC architecture.
For debugging purposes, it also creates a debugging information file as well as an assembly file containing a listing of the generated machine code.
The macro \texttt{\_\_ppc64\_\_} is predefined in order to enable programmers to identify this tool and its target architecture while compiling.
Programs generated with this compiler require additional runtime support that is stored in the \file{cpp\-ppc64\-run} library file.
\flowgraph{\resource{\cpp{}\\source code} \ar[r] & \toolbox{cppppc64} \ar[r] \ar[d] \ar[rd] & \resource{object file} \\ \variable{ECSINCLUDE} \ar[ru] & \resource{debugging\\information} & \resource{assembly\\listing}}
\seecpp\seeassembly\seeppc\seeobject\seedebugging
}

\providecommand{\cpprisc}{
\toolsection{cpprisc} is a compiler for the \cpp{} programming language targeting the RISC hardware architecture.
It generates machine code for RISC processors from programs written in \cpp{} and stores it in corresponding object files.
For debugging purposes, it also creates a debugging information file as well as an assembly file containing a listing of the generated machine code.
The macro \texttt{\_\_risc\_\_} is predefined in order to enable programmers to identify this tool and its target architecture while compiling.
Programs generated with this compiler require additional runtime support that is stored in the \file{cpp\-risc\-run} library file.
\flowgraph{\resource{\cpp{}\\source code} \ar[r] & \toolbox{cpprisc} \ar[r] \ar[d] \ar[rd] & \resource{object file} \\ \variable{ECSINCLUDE} \ar[ru] & \resource{debugging\\information} & \resource{assembly\\listing}}
\seecpp\seeassembly\seerisc\seeobject\seedebugging
}

\providecommand{\cppwasm}{
\toolsection{cppwasm} is a compiler for the \cpp{} programming language targeting the WebAssembly architecture.
It generates machine code for WebAssembly targets from programs written in \cpp{} and stores it in corresponding object files.
For debugging purposes, it also creates a debugging information file as well as an assembly file containing a listing of the generated machine code.
The macro \texttt{\_\_wasm\_\_} is predefined in order to enable programmers to identify this tool and its target architecture while compiling.
Programs generated with this compiler require additional runtime support that is stored in the \file{cpp\-wasm\-run} library file.
\flowgraph{\resource{\cpp{}\\source code} \ar[r] & \toolbox{cppwasm} \ar[r] \ar[d] \ar[rd] & \resource{object file} \\ \variable{ECSINCLUDE} \ar[ru] & \resource{debugging\\information} & \resource{assembly\\listing}}
\seecpp\seeassembly\seewasm\seeobject\seedebugging
}

% FALSE tools

\providecommand{\falprint}{
\toolsection{falprint} is a pretty printer for the FALSE programming language.
It reformats the source code of FALSE programs and writes it to the standard output stream.
\flowgraph{\resource{FALSE\\source code} \ar[r] & \toolbox{falprint} \ar[r] & \resource{reformatted\\source code}}
\seefalse
}

\providecommand{\falcheck}{
\toolsection{falcheck} is a syntactic and semantic checker for the FALSE programming language.
It just performs syntactic and semantic checks on FALSE programs and writes its diagnostic messages to the standard error stream.
\flowgraph{\resource{FALSE\\source code} \ar[r] & \toolbox{falcheck} \ar[r] & \resource{diagnostic\\messages}}
\seefalse
}

\providecommand{\faldump}{
\toolsection{faldump} is a serializer for the FALSE programming language.
It dumps the complete internal representation of programs written in FALSE into an XML document.
\debuggingtool
\flowgraph{\resource{FALSE\\source code} \ar[r] & \toolbox{faldump} \ar[r] & \resource{internal\\representation}}
\seefalse
}

\providecommand{\falrun}{
\toolsection{falrun} is an interpreter for the FALSE programming language.
It processes and executes programs written in FALSE\@.
\flowgraph{\resource{FALSE\\source code} \ar[r] & \toolbox{falrun} \ar@/u/[r] & \resource{input/\\output} \ar@/d/[l]}
\seefalse
}

\providecommand{\falcpp}{
\toolsection{falcpp} is a transpiler for the FALSE programming language.
It translates programs written in FALSE into \cpp{} programs and stores them in corresponding source files.
\flowgraph{\resource{FALSE\\source code} \ar[r] & \toolbox{falcpp} \ar[r] & \resource{\cpp{}\\source file}}
\seefalse\seecpp
}

\providecommand{\falcode}{
\toolsection{falcode} is an intermediate code generator for the FALSE programming language.
It generates intermediate code from programs written in FALSE and stores it in corresponding assembly files.
\debuggingtool
\flowgraph{\resource{FALSE\\source code} \ar[r] & \toolbox{falcode} \ar[r] & \resource{intermediate\\code}}
\seefalse\seeassembly\seecode
}

\providecommand{\falamda}{
\toolsection{falamd16} is a compiler for the FALSE programming language targeting the AMD64 hardware architecture.
It generates machine code for AMD64 processors from programs written in FALSE and stores it in corresponding object files.
The compiler generates machine code for the 16-bit operating mode defined by the AMD64 architecture.
\flowgraph{\resource{FALSE\\source code} \ar[r] & \toolbox{falamd16} \ar[r] & \resource{object file}}
\seefalse\seeamd\seeobject
}

\providecommand{\falamdb}{
\toolsection{falamd32} is a compiler for the FALSE programming language targeting the AMD64 hardware architecture.
It generates machine code for AMD64 processors from programs written in FALSE and stores it in corresponding object files.
The compiler generates machine code for the 32-bit operating mode defined by the AMD64 architecture.
\flowgraph{\resource{FALSE\\source code} \ar[r] & \toolbox{falamd32} \ar[r] & \resource{object file}}
\seefalse\seeamd\seeobject
}

\providecommand{\falamdc}{
\toolsection{falamd64} is a compiler for the FALSE programming language targeting the AMD64 hardware architecture.
It generates machine code for AMD64 processors from programs written in FALSE and stores it in corresponding object files.
The compiler generates machine code for the 64-bit operating mode defined by the AMD64 architecture.
\flowgraph{\resource{FALSE\\source code} \ar[r] & \toolbox{falamd64} \ar[r] & \resource{object file}}
\seefalse\seeamd\seeobject
}

\providecommand{\falarma}{
\toolsection{falarma32} is a compiler for the FALSE programming language targeting the ARM hardware architecture.
It generates machine code for ARM processors executing A32 instructions from programs written in FALSE and stores it in corresponding object files.
\flowgraph{\resource{FALSE\\source code} \ar[r] & \toolbox{falarma32} \ar[r] & \resource{object file}}
\seefalse\seearm\seeobject
}

\providecommand{\falarmb}{
\toolsection{falarma64} is a compiler for the FALSE programming language targeting the ARM hardware architecture.
It generates machine code for ARM processors executing A64 instructions from programs written in FALSE and stores it in corresponding object files.
\flowgraph{\resource{FALSE\\source code} \ar[r] & \toolbox{falarma64} \ar[r] & \resource{object file}}
\seefalse\seearm\seeobject
}

\providecommand{\falarmc}{
\toolsection{falarmt32} is a compiler for the FALSE programming language targeting the ARM hardware architecture.
It generates machine code for ARM processors without floating-point extension executing T32 instructions from programs written in FALSE and stores it in corresponding object files.
\flowgraph{\resource{FALSE\\source code} \ar[r] & \toolbox{falarmt32} \ar[r] & \resource{object file}}
\seefalse\seearm\seeobject
}

\providecommand{\falarmcfpe}{
\toolsection{falarmt32fpe} is a compiler for the FALSE programming language targeting the ARM hardware architecture.
It generates machine code for ARM processors with floating-point extension executing T32 instructions from programs written in FALSE and stores it in corresponding object files.
\flowgraph{\resource{FALSE\\source code} \ar[r] & \toolbox{falarmt32fpe} \ar[r] & \resource{object file}}
\seefalse\seearm\seeobject
}

\providecommand{\falavr}{
\toolsection{falavr} is a compiler for the FALSE programming language targeting the AVR hardware architecture.
It generates machine code for AVR processors from programs written in FALSE and stores it in corresponding object files.
\flowgraph{\resource{FALSE\\source code} \ar[r] & \toolbox{falavr} \ar[r] & \resource{object file}}
\seefalse\seeavr\seeobject
}

\providecommand{\falavrtt}{
\toolsection{falavr32} is a compiler for the FALSE programming language targeting the AVR32 hardware architecture.
It generates machine code for AVR32 processors from programs written in FALSE and stores it in corresponding object files.
\flowgraph{\resource{FALSE\\source code} \ar[r] & \toolbox{falavr32} \ar[r] & \resource{object file}}
\seefalse\seeavrtt\seeobject
}

\providecommand{\falmabk}{
\toolsection{falm68k} is a compiler for the FALSE programming language targeting the M68000 hardware architecture.
It generates machine code for M68000 processors from programs written in FALSE and stores it in corresponding object files.
\flowgraph{\resource{FALSE\\source code} \ar[r] & \toolbox{falm68k} \ar[r] & \resource{object file}}
\seefalse\seemabk\seeobject
}

\providecommand{\falmibl}{
\toolsection{falmibl} is a compiler for the FALSE programming language targeting the MicroBlaze hardware architecture.
It generates machine code for MicroBlaze processors from programs written in FALSE and stores it in corresponding object files.
\flowgraph{\resource{FALSE\\source code} \ar[r] & \toolbox{falmibl} \ar[r] & \resource{object file}}
\seefalse\seemibl\seeobject
}

\providecommand{\falmipsa}{
\toolsection{falmips32} is a compiler for the FALSE programming language targeting the MIPS32 hardware architecture.
It generates machine code for MIPS32 processors from programs written in FALSE and stores it in corresponding object files.
\flowgraph{\resource{FALSE\\source code} \ar[r] & \toolbox{falmips32} \ar[r] & \resource{object file}}
\seefalse\seemips\seeobject
}

\providecommand{\falmipsb}{
\toolsection{falmips64} is a compiler for the FALSE programming language targeting the MIPS64 hardware architecture.
It generates machine code for MIPS64 processors from programs written in FALSE and stores it in corresponding object files.
\flowgraph{\resource{FALSE\\source code} \ar[r] & \toolbox{falmips64} \ar[r] & \resource{object file}}
\seefalse\seemips\seeobject
}

\providecommand{\falmmix}{
\toolsection{falmmix} is a compiler for the FALSE programming language targeting the MMIX hardware architecture.
It generates machine code for MMIX processors from programs written in FALSE and stores it in corresponding object files.
\flowgraph{\resource{FALSE\\source code} \ar[r] & \toolbox{falmmix} \ar[r] & \resource{object file}}
\seefalse\seemmix\seeobject
}

\providecommand{\falorok}{
\toolsection{falor1k} is a compiler for the FALSE programming language targeting the OpenRISC 1000 hardware architecture.
It generates machine code for OpenRISC 1000 processors from programs written in FALSE and stores it in corresponding object files.
\flowgraph{\resource{FALSE\\source code} \ar[r] & \toolbox{falor1k} \ar[r] & \resource{object file}}
\seefalse\seeorok\seeobject
}

\providecommand{\falppca}{
\toolsection{falppc32} is a compiler for the FALSE programming language targeting the PowerPC hardware architecture.
It generates machine code for PowerPC processors from programs written in FALSE and stores it in corresponding object files.
The compiler generates machine code for the 32-bit operating mode defined by the PowerPC architecture.
\flowgraph{\resource{FALSE\\source code} \ar[r] & \toolbox{falppc32} \ar[r] & \resource{object file}}
\seefalse\seeppc\seeobject
}

\providecommand{\falppcb}{
\toolsection{falppc64} is a compiler for the FALSE programming language targeting the PowerPC hardware architecture.
It generates machine code for PowerPC processors from programs written in FALSE and stores it in corresponding object files.
The compiler generates machine code for the 64-bit operating mode defined by the PowerPC architecture.
\flowgraph{\resource{FALSE\\source code} \ar[r] & \toolbox{falppc64} \ar[r] & \resource{object file}}
\seefalse\seeppc\seeobject
}

\providecommand{\falrisc}{
\toolsection{falrisc} is a compiler for the FALSE programming language targeting the RISC hardware architecture.
It generates machine code for RISC processors from programs written in FALSE and stores it in corresponding object files.
\flowgraph{\resource{FALSE\\source code} \ar[r] & \toolbox{falrisc} \ar[r] & \resource{object file}}
\seefalse\seerisc\seeobject
}

\providecommand{\falwasm}{
\toolsection{falwasm} is a compiler for the FALSE programming language targeting the WebAssembly architecture.
It generates machine code for WebAssembly targets from programs written in FALSE and stores it in corresponding object files.
\flowgraph{\resource{FALSE\\source code} \ar[r] & \toolbox{falwasm} \ar[r] & \resource{object file}}
\seefalse\seewasm\seeobject
}

% Oberon tools

\providecommand{\obprint}{
\toolsection{obprint} is a pretty printer for the Oberon programming language.
It reformats the source code of Oberon modules and writes it to the standard output stream.
\flowgraph{\resource{Oberon\\source code} \ar[r] & \toolbox{obprint} \ar[r] & \resource{reformatted\\source code}}
\seeoberon
}

\providecommand{\obcheck}{
\toolsection{obcheck} is a syntactic and semantic checker for the Oberon programming language.
It just performs syntactic and semantic checks on Oberon modules and writes its diagnostic messages to the standard error stream.
In addition, it stores the interface of each module in a symbol file which is required when other modules import the module.
\flowgraph{\resource{Oberon\\source code} \ar[r] & \toolbox{obcheck} \ar[r] \ar@/l/[d] & \resource{diagnostic\\messages} \\ \variable{ECSIMPORT} \ar[ru] & \resource{symbol\\files} \ar@/r/[u]}
\seeoberon
}

\providecommand{\obdump}{
\toolsection{obdump} is a serializer for the Oberon programming language.
It dumps the complete internal representation of modules written in Oberon into an XML document.
\debuggingtool
\flowgraph{\resource{Oberon\\source code} \ar[r] & \toolbox{obdump} \ar[r] \ar@/l/[d] & \resource{internal\\representation} \\ \variable{ECSIMPORT} \ar[ru] & \resource{symbol\\files} \ar@/r/[u]}
\seeoberon
}

\providecommand{\obrun}{
\toolsection{obrun} is an interpreter for the Oberon programming language.
It processes and executes modules written in Oberon.
This tool does neither generate nor process symbol files while interpreting modules.
If a module is imported by another one, its filename has to be named before the other one in the list of command-line arguments.
\flowgraph{\resource{Oberon\\source code} \ar[r] & \toolbox{obrun} \ar@/u/[r] & \resource{input/\\output} \ar@/d/[l]}
\seeoberon
}

\providecommand{\obcpp}{
\toolsection{obcpp} is a transpiler for the Oberon programming language.
It translates programs written in Oberon into \cpp{} programs and stores them in corresponding source and header files.
In addition, it stores the interface of each module in a symbol file which is required when other modules import the module.
The same interface is provided by the generated header file which can be used in other parts of the \cpp{} program.
\flowgraph{\resource{Oberon\\source code} \ar[r] & \toolbox{obcpp} \ar[r] \ar@/l/[d] \ar[rd] & \resource{\cpp{}\\source file} \\ \variable{ECSIMPORT} \ar[ru] & \resource{symbol\\files} \ar@/r/[u] & \resource{\cpp{}\\header file}}
\seeoberon\seecpp
}

\providecommand{\obdoc}{
\toolsection{obdoc} is a generic documentation generator for the Oberon programming language.
It processes several Oberon modules and assembles all information therein into a generic documentation.
In addition, it stores the interface of each module in a symbol file which is required when other modules import the module.
\debuggingtool
\flowgraph{\resource{Oberon\\source code} \ar[r] & \toolbox{obdoc} \ar[r] \ar@/l/[d] & \resource{generic\\documentation} \\ \variable{ECSIMPORT} \ar[ru] & \resource{symbol\\files} \ar@/r/[u]}
\seeoberon\seedocumentation
}

\providecommand{\obhtml}{
\toolsection{obhtml} is an HTML documentation generator for the Oberon programming language.
It processes several Oberon modules and assembles all information therein into an HTML document.
In addition, it stores the interface of each module in a symbol file which is required when other modules import the module.
\flowgraph{\resource{Oberon\\source code} \ar[r] & \toolbox{obhtml} \ar[r] \ar@/l/[d] & \resource{HTML\\document} \\ \variable{ECSIMPORT} \ar[ru] & \resource{symbol\\files} \ar@/r/[u]}
\seeoberon\seedocumentation
}

\providecommand{\oblatex}{
\toolsection{oblatex} is a Latex documentation generator for the Oberon programming language.
It processes several Oberon modules and assembles all information therein into a Latex document.
In addition, it stores the interface of each module in a symbol file which is required when other modules import the module.
\flowgraph{\resource{Oberon\\source code} \ar[r] & \toolbox{oblatex} \ar[r] \ar@/l/[d] & \resource{Latex\\document} \\ \variable{ECSIMPORT} \ar[ru] & \resource{symbol\\files} \ar@/r/[u]}
\seeoberon\seedocumentation
}

\providecommand{\obcode}{
\toolsection{obcode} is an intermediate code generator for the Oberon programming language.
It generates intermediate code from modules written in Oberon and stores it in corresponding assembly files.
In addition, it stores the interface of each module in a symbol file which is required when other modules import the module.
Programs generated with this tool require additional runtime support that is stored in the \file{ob\-code\-run} library file.
\debuggingtool
\flowgraph{\resource{Oberon\\source code} \ar[r] & \toolbox{obcode} \ar[r] \ar@/l/[d] & \resource{intermediate\\code} \\ \variable{ECSIMPORT} \ar[ru] & \resource{symbol\\files} \ar@/r/[u]}
\seeoberon\seeassembly\seecode
}

\providecommand{\obamda}{
\toolsection{obamd16} is a compiler for the Oberon programming language targeting the AMD64 hardware architecture.
It generates machine code for AMD64 processors from modules written in Oberon and stores it in corresponding object files.
The compiler generates machine code for the 16-bit operating mode defined by the AMD64 architecture.
For debugging purposes, it also creates a debugging information file as well as an assembly file containing a listing of the generated machine code.
In addition, it stores the interface of each module in a symbol file which is required when other modules import the module.
Programs generated with this compiler require additional runtime support that is stored in the \file{ob\-amd16\-run} library file.
\flowgraph{\resource{Oberon\\source code} \ar[r] & \toolbox{obamd16} \ar[r] \ar@/l/[d] \ar[rd] & \resource{object file} \\ \variable{ECSIMPORT} \ar[ru] & \resource{symbol\\files} \ar@/r/[u] & \resource{debugging\\information}}
\seeoberon\seeassembly\seeamd\seeobject\seedebugging
}

\providecommand{\obamdb}{
\toolsection{obamd32} is a compiler for the Oberon programming language targeting the AMD64 hardware architecture.
It generates machine code for AMD64 processors from modules written in Oberon and stores it in corresponding object files.
The compiler generates machine code for the 32-bit operating mode defined by the AMD64 architecture.
For debugging purposes, it also creates a debugging information file as well as an assembly file containing a listing of the generated machine code.
In addition, it stores the interface of each module in a symbol file which is required when other modules import the module.
Programs generated with this compiler require additional runtime support that is stored in the \file{ob\-amd32\-run} library file.
\flowgraph{\resource{Oberon\\source code} \ar[r] & \toolbox{obamd32} \ar[r] \ar@/l/[d] \ar[rd] & \resource{object file} \\ \variable{ECSIMPORT} \ar[ru] & \resource{symbol\\files} \ar@/r/[u] & \resource{debugging\\information}}
\seeoberon\seeassembly\seeamd\seeobject\seedebugging
}

\providecommand{\obamdc}{
\toolsection{obamd64} is a compiler for the Oberon programming language targeting the AMD64 hardware architecture.
It generates machine code for AMD64 processors from modules written in Oberon and stores it in corresponding object files.
The compiler generates machine code for the 64-bit operating mode defined by the AMD64 architecture.
For debugging purposes, it also creates a debugging information file as well as an assembly file containing a listing of the generated machine code.
In addition, it stores the interface of each module in a symbol file which is required when other modules import the module.
Programs generated with this compiler require additional runtime support that is stored in the \file{ob\-amd64\-run} library file.
\flowgraph{\resource{Oberon\\source code} \ar[r] & \toolbox{obamd64} \ar[r] \ar@/l/[d] \ar[rd] & \resource{object file} \\ \variable{ECSIMPORT} \ar[ru] & \resource{symbol\\files} \ar@/r/[u] & \resource{debugging\\information}}
\seeoberon\seeassembly\seeamd\seeobject\seedebugging
}

\providecommand{\obarma}{
\toolsection{obarma32} is a compiler for the Oberon programming language targeting the ARM hardware architecture.
It generates machine code for ARM processors executing A32 instructions from modules written in Oberon and stores it in corresponding object files.
For debugging purposes, it also creates a debugging information file as well as an assembly file containing a listing of the generated machine code.
In addition, it stores the interface of each module in a symbol file which is required when other modules import the module.
Programs generated with this compiler require additional runtime support that is stored in the \file{ob\-arma32\-run} library file.
\flowgraph{\resource{Oberon\\source code} \ar[r] & \toolbox{obarma32} \ar[r] \ar@/l/[d] \ar[rd] & \resource{object file} \\ \variable{ECSIMPORT} \ar[ru] & \resource{symbol\\files} \ar@/r/[u] & \resource{debugging\\information}}
\seeoberon\seeassembly\seearm\seeobject\seedebugging
}

\providecommand{\obarmb}{
\toolsection{obarma64} is a compiler for the Oberon programming language targeting the ARM hardware architecture.
It generates machine code for ARM processors executing A64 instructions from modules written in Oberon and stores it in corresponding object files.
For debugging purposes, it also creates a debugging information file as well as an assembly file containing a listing of the generated machine code.
In addition, it stores the interface of each module in a symbol file which is required when other modules import the module.
Programs generated with this compiler require additional runtime support that is stored in the \file{ob\-arma64\-run} library file.
\flowgraph{\resource{Oberon\\source code} \ar[r] & \toolbox{obarma64} \ar[r] \ar@/l/[d] \ar[rd] & \resource{object file} \\ \variable{ECSIMPORT} \ar[ru] & \resource{symbol\\files} \ar@/r/[u] & \resource{debugging\\information}}
\seeoberon\seeassembly\seearm\seeobject\seedebugging
}

\providecommand{\obarmc}{
\toolsection{obarmt32} is a compiler for the Oberon programming language targeting the ARM hardware architecture.
It generates machine code for ARM processors without floating-point extension executing T32 instructions from modules written in Oberon and stores it in corresponding object files.
For debugging purposes, it also creates a debugging information file as well as an assembly file containing a listing of the generated machine code.
In addition, it stores the interface of each module in a symbol file which is required when other modules import the module.
Programs generated with this compiler require additional runtime support that is stored in the \file{ob\-armt32\-run} library file.
\flowgraph{\resource{Oberon\\source code} \ar[r] & \toolbox{obarmt32} \ar[r] \ar@/l/[d] \ar[rd] & \resource{object file} \\ \variable{ECSIMPORT} \ar[ru] & \resource{symbol\\files} \ar@/r/[u] & \resource{debugging\\information}}
\seeoberon\seeassembly\seearm\seeobject\seedebugging
}

\providecommand{\obarmcfpe}{
\toolsection{obarmt32fpe} is a compiler for the Oberon programming language targeting the ARM hardware architecture.
It generates machine code for ARM processors with floating-point extension executing T32 instructions from modules written in Oberon and stores it in corresponding object files.
For debugging purposes, it also creates a debugging information file as well as an assembly file containing a listing of the generated machine code.
In addition, it stores the interface of each module in a symbol file which is required when other modules import the module.
Programs generated with this compiler require additional runtime support that is stored in the \file{ob\-armt32\-fpe\-run} library file.
\flowgraph{\resource{Oberon\\source code} \ar[r] & \toolbox{obarmt32fpe} \ar[r] \ar@/l/[d] \ar[rd] & \resource{object file} \\ \variable{ECSIMPORT} \ar[ru] & \resource{symbol\\files} \ar@/r/[u] & \resource{debugging\\information}}
\seeoberon\seeassembly\seearm\seeobject\seedebugging
}

\providecommand{\obavr}{
\toolsection{obavr} is a compiler for the Oberon programming language targeting the AVR hardware architecture.
It generates machine code for AVR processors from modules written in Oberon and stores it in corresponding object files.
For debugging purposes, it also creates a debugging information file as well as an assembly file containing a listing of the generated machine code.
In addition, it stores the interface of each module in a symbol file which is required when other modules import the module.
Programs generated with this compiler require additional runtime support that is stored in the \file{ob\-avr\-run} library file.
\flowgraph{\resource{Oberon\\source code} \ar[r] & \toolbox{obavr} \ar[r] \ar@/l/[d] \ar[rd] & \resource{object file} \\ \variable{ECSIMPORT} \ar[ru] & \resource{symbol\\files} \ar@/r/[u] & \resource{debugging\\information}}
\seeoberon\seeassembly\seeavr\seeobject\seedebugging
}

\providecommand{\obavrtt}{
\toolsection{obavr32} is a compiler for the Oberon programming language targeting the AVR32 hardware architecture.
It generates machine code for AVR32 processors from modules written in Oberon and stores it in corresponding object files.
For debugging purposes, it also creates a debugging information file as well as an assembly file containing a listing of the generated machine code.
In addition, it stores the interface of each module in a symbol file which is required when other modules import the module.
Programs generated with this compiler require additional runtime support that is stored in the \file{ob\-avr32\-run} library file.
\flowgraph{\resource{Oberon\\source code} \ar[r] & \toolbox{obavr32} \ar[r] \ar@/l/[d] \ar[rd] & \resource{object file} \\ \variable{ECSIMPORT} \ar[ru] & \resource{symbol\\files} \ar@/r/[u] & \resource{debugging\\information}}
\seeoberon\seeassembly\seeavrtt\seeobject\seedebugging
}

\providecommand{\obmabk}{
\toolsection{obm68k} is a compiler for the Oberon programming language targeting the M68000 hardware architecture.
It generates machine code for M68000 processors from modules written in Oberon and stores it in corresponding object files.
For debugging purposes, it also creates a debugging information file as well as an assembly file containing a listing of the generated machine code.
In addition, it stores the interface of each module in a symbol file which is required when other modules import the module.
Programs generated with this compiler require additional runtime support that is stored in the \file{ob\-m68k\-run} library file.
\flowgraph{\resource{Oberon\\source code} \ar[r] & \toolbox{obm68k} \ar[r] \ar@/l/[d] \ar[rd] & \resource{object file} \\ \variable{ECSIMPORT} \ar[ru] & \resource{symbol\\files} \ar@/r/[u] & \resource{debugging\\information}}
\seeoberon\seeassembly\seemabk\seeobject\seedebugging
}

\providecommand{\obmibl}{
\toolsection{obmibl} is a compiler for the Oberon programming language targeting the MicroBlaze hardware architecture.
It generates machine code for MicroBlaze processors from modules written in Oberon and stores it in corresponding object files.
For debugging purposes, it also creates a debugging information file as well as an assembly file containing a listing of the generated machine code.
In addition, it stores the interface of each module in a symbol file which is required when other modules import the module.
Programs generated with this compiler require additional runtime support that is stored in the \file{ob\-mibl\-run} library file.
\flowgraph{\resource{Oberon\\source code} \ar[r] & \toolbox{obmibl} \ar[r] \ar@/l/[d] \ar[rd] & \resource{object file} \\ \variable{ECSIMPORT} \ar[ru] & \resource{symbol\\files} \ar@/r/[u] & \resource{debugging\\information}}
\seeoberon\seeassembly\seemibl\seeobject\seedebugging
}

\providecommand{\obmipsa}{
\toolsection{obmips32} is a compiler for the Oberon programming language targeting the MIPS32 hardware architecture.
It generates machine code for MIPS32 processors from modules written in Oberon and stores it in corresponding object files.
For debugging purposes, it also creates a debugging information file as well as an assembly file containing a listing of the generated machine code.
In addition, it stores the interface of each module in a symbol file which is required when other modules import the module.
Programs generated with this compiler require additional runtime support that is stored in the \file{ob\-mips32\-run} library file.
\flowgraph{\resource{Oberon\\source code} \ar[r] & \toolbox{obmips32} \ar[r] \ar@/l/[d] \ar[rd] & \resource{object file} \\ \variable{ECSIMPORT} \ar[ru] & \resource{symbol\\files} \ar@/r/[u] & \resource{debugging\\information}}
\seeoberon\seeassembly\seemips\seeobject\seedebugging
}

\providecommand{\obmipsb}{
\toolsection{obmips64} is a compiler for the Oberon programming language targeting the MIPS64 hardware architecture.
It generates machine code for MIPS64 processors from modules written in Oberon and stores it in corresponding object files.
For debugging purposes, it also creates a debugging information file as well as an assembly file containing a listing of the generated machine code.
In addition, it stores the interface of each module in a symbol file which is required when other modules import the module.
Programs generated with this compiler require additional runtime support that is stored in the \file{ob\-mips64\-run} library file.
\flowgraph{\resource{Oberon\\source code} \ar[r] & \toolbox{obmips64} \ar[r] \ar@/l/[d] \ar[rd] & \resource{object file} \\ \variable{ECSIMPORT} \ar[ru] & \resource{symbol\\files} \ar@/r/[u] & \resource{debugging\\information}}
\seeoberon\seeassembly\seemips\seeobject\seedebugging
}

\providecommand{\obmmix}{
\toolsection{obmmix} is a compiler for the Oberon programming language targeting the MMIX hardware architecture.
It generates machine code for MMIX processors from modules written in Oberon and stores it in corresponding object files.
For debugging purposes, it also creates a debugging information file as well as an assembly file containing a listing of the generated machine code.
In addition, it stores the interface of each module in a symbol file which is required when other modules import the module.
Programs generated with this compiler require additional runtime support that is stored in the \file{ob\-mmix\-run} library file.
\flowgraph{\resource{Oberon\\source code} \ar[r] & \toolbox{obmmix} \ar[r] \ar@/l/[d] \ar[rd] & \resource{object file} \\ \variable{ECSIMPORT} \ar[ru] & \resource{symbol\\files} \ar@/r/[u] & \resource{debugging\\information}}
\seeoberon\seeassembly\seemmix\seeobject\seedebugging
}

\providecommand{\oborok}{
\toolsection{obor1k} is a compiler for the Oberon programming language targeting the OpenRISC 1000 hardware architecture.
It generates machine code for OpenRISC 1000 processors from modules written in Oberon and stores it in corresponding object files.
For debugging purposes, it also creates a debugging information file as well as an assembly file containing a listing of the generated machine code.
In addition, it stores the interface of each module in a symbol file which is required when other modules import the module.
Programs generated with this compiler require additional runtime support that is stored in the \file{ob\-or1k\-run} library file.
\flowgraph{\resource{Oberon\\source code} \ar[r] & \toolbox{obor1k} \ar[r] \ar@/l/[d] \ar[rd] & \resource{object file} \\ \variable{ECSIMPORT} \ar[ru] & \resource{symbol\\files} \ar@/r/[u] & \resource{debugging\\information}}
\seeoberon\seeassembly\seeorok\seeobject\seedebugging
}

\providecommand{\obppca}{
\toolsection{obppc32} is a compiler for the Oberon programming language targeting the PowerPC hardware architecture.
It generates machine code for PowerPC processors from modules written in Oberon and stores it in corresponding object files.
The compiler generates machine code for the 32-bit operating mode defined by the PowerPC architecture.
For debugging purposes, it also creates a debugging information file as well as an assembly file containing a listing of the generated machine code.
In addition, it stores the interface of each module in a symbol file which is required when other modules import the module.
Programs generated with this compiler require additional runtime support that is stored in the \file{ob\-ppc32\-run} library file.
\flowgraph{\resource{Oberon\\source code} \ar[r] & \toolbox{obppc32} \ar[r] \ar@/l/[d] \ar[rd] & \resource{object file} \\ \variable{ECSIMPORT} \ar[ru] & \resource{symbol\\files} \ar@/r/[u] & \resource{debugging\\information}}
\seeoberon\seeassembly\seeppc\seeobject\seedebugging
}

\providecommand{\obppcb}{
\toolsection{obppc64} is a compiler for the Oberon programming language targeting the PowerPC hardware architecture.
It generates machine code for PowerPC processors from modules written in Oberon and stores it in corresponding object files.
The compiler generates machine code for the 64-bit operating mode defined by the PowerPC architecture.
For debugging purposes, it also creates a debugging information file as well as an assembly file containing a listing of the generated machine code.
In addition, it stores the interface of each module in a symbol file which is required when other modules import the module.
Programs generated with this compiler require additional runtime support that is stored in the \file{ob\-ppc64\-run} library file.
\flowgraph{\resource{Oberon\\source code} \ar[r] & \toolbox{obppc64} \ar[r] \ar@/l/[d] \ar[rd] & \resource{object file} \\ \variable{ECSIMPORT} \ar[ru] & \resource{symbol\\files} \ar@/r/[u] & \resource{debugging\\information}}
\seeoberon\seeassembly\seeppc\seeobject\seedebugging
}

\providecommand{\obrisc}{
\toolsection{obrisc} is a compiler for the Oberon programming language targeting the RISC hardware architecture.
It generates machine code for RISC processors from modules written in Oberon and stores it in corresponding object files.
For debugging purposes, it also creates a debugging information file as well as an assembly file containing a listing of the generated machine code.
In addition, it stores the interface of each module in a symbol file which is required when other modules import the module.
Programs generated with this compiler require additional runtime support that is stored in the \file{ob\-risc\-run} library file.
\flowgraph{\resource{Oberon\\source code} \ar[r] & \toolbox{obrisc} \ar[r] \ar@/l/[d] \ar[rd] & \resource{object file} \\ \variable{ECSIMPORT} \ar[ru] & \resource{symbol\\files} \ar@/r/[u] & \resource{debugging\\information}}
\seeoberon\seeassembly\seerisc\seeobject\seedebugging
}

\providecommand{\obwasm}{
\toolsection{obwasm} is a compiler for the Oberon programming language targeting the WebAssembly architecture.
It generates machine code for WebAssembly targets from modules written in Oberon and stores it in corresponding object files.
For debugging purposes, it also creates a debugging information file as well as an assembly file containing a listing of the generated machine code.
In addition, it stores the interface of each module in a symbol file which is required when other modules import the module.
Programs generated with this compiler require additional runtime support that is stored in the \file{ob\-wasm\-run} library file.
\flowgraph{\resource{Oberon\\source code} \ar[r] & \toolbox{obwasm} \ar[r] \ar@/l/[d] \ar[rd] & \resource{object file} \\ \variable{ECSIMPORT} \ar[ru] & \resource{symbol\\files} \ar@/r/[u] & \resource{debugging\\information}}
\seeoberon\seeassembly\seewasm\seeobject\seedebugging
}

% converter tools

\providecommand{\dbgdwarf}{
\toolsection{dbgdwarf} is a DWARF debugging information converter tool.
It converts debugging information into the DWARF debugging data format and stores it in corresponding object files~\cite{dwarffile}.
The resulting debugging object files can be combined with runtime support that creates Executable and Linking Format (ELF) files~\cite{elffile}.
\flowgraph{\resource{debugging\\information} \ar[r] & \toolbox{dbgdwarf} \ar[r] & \resource{debugging\\object file}}
\seeobject\seedebugging
}

% assembler tools

\providecommand{\asmprint}{
\toolsection{asmprint} is a pretty printer for generic assembly code.
It reformats generic assembly code and writes it to the standard output stream.
\flowgraph{\resource{generic assembly\\source code} \ar[r] & \toolbox{asmprint} \ar[r] & \resource{reformatted\\source code}}
\seeassembly
}

\providecommand{\amdaasm}{
\toolsection{amd16asm} is an assembler for the AMD64 hardware architecture.
It translates assembly code into machine code for AMD64 processors and stores it in corresponding object files.
By default, the assembler generates machine code for the 16-bit operating mode defined by the AMD64 architecture.
\flowgraph{\resource{AMD16 assembly\\source code} \ar[r] & \toolbox{amd16asm} \ar[r] & \resource{object file}}
\seeassembly\seeamd\seeobject
}

\providecommand{\amdadism}{
\toolsection{amd16dism} is a disassembler for the AMD64 hardware architecture.
It translates machine code from object files targeting AMD64 processors into assembly code and writes it to the standard output stream.
It assumes that the machine code was generated for the 16-bit operating mode defined by the AMD64 architecture.
\flowgraph{\resource{object file} \ar[r] & \toolbox{amd16dism} \ar[r] & \resource{disassembly\\listing}}
\seeassembly\seeamd\seeobject
}

\providecommand{\amdbasm}{
\toolsection{amd32asm} is an assembler for the AMD64 hardware architecture.
It translates assembly code into machine code for AMD64 processors and stores it in corresponding object files.
By default, the assembler generates machine code for the 32-bit operating mode defined by the AMD64 architecture.
\flowgraph{\resource{AMD32 assembly\\source code} \ar[r] & \toolbox{amd32asm} \ar[r] & \resource{object file}}
\seeassembly\seeamd\seeobject
}

\providecommand{\amdbdism}{
\toolsection{amd32dism} is a disassembler for the AMD64 hardware architecture.
It translates machine code from object files targeting AMD64 processors into assembly code and writes it to the standard output stream.
It assumes that the machine code was generated for the 32-bit operating mode defined by the AMD64 architecture.
\flowgraph{\resource{object file} \ar[r] & \toolbox{amd32dism} \ar[r] & \resource{disassembly\\listing}}
\seeassembly\seeamd\seeobject
}

\providecommand{\amdcasm}{
\toolsection{amd64asm} is an assembler for the AMD64 hardware architecture.
It translates assembly code into machine code for AMD64 processors and stores it in corresponding object files.
By default, the assembler generates machine code for the 64-bit operating mode defined by the AMD64 architecture.
\flowgraph{\resource{AMD64 assembly\\source code} \ar[r] & \toolbox{amd64asm} \ar[r] & \resource{object file}}
\seeassembly\seeamd\seeobject
}

\providecommand{\amdcdism}{
\toolsection{amd64dism} is a disassembler for the AMD64 hardware architecture.
It translates machine code from object files targeting AMD64 processors into assembly code and writes it to the standard output stream.
It assumes that the machine code was generated for the 64-bit operating mode defined by the AMD64 architecture.
\flowgraph{\resource{object file} \ar[r] & \toolbox{amd64dism} \ar[r] & \resource{disassembly\\listing}}
\seeassembly\seeamd\seeobject
}

\providecommand{\armaasm}{
\toolsection{arma32asm} is an assembler for the ARM hardware architecture.
It translates assembly code into machine code for ARM processors executing A32 instructions and stores it in corresponding object files.
\flowgraph{\resource{ARM A32 assembly\\source code} \ar[r] & \toolbox{arma32asm} \ar[r] & \resource{object file}}
\seeassembly\seearm\seeobject
}

\providecommand{\armadism}{
\toolsection{arma32dism} is a disassembler for the ARM hardware architecture.
It translates machine code from object files targeting ARM processors executing A32 instructions into assembly code and writes it to the standard output stream.
\flowgraph{\resource{object file} \ar[r] & \toolbox{arma32dism} \ar[r] & \resource{disassembly\\listing}}
\seeassembly\seearm\seeobject
}

\providecommand{\armbasm}{
\toolsection{arma64asm} is an assembler for the ARM hardware architecture.
It translates assembly code into machine code for ARM processors executing A64 instructions and stores it in corresponding object files.
\flowgraph{\resource{ARM A64 assembly\\source code} \ar[r] & \toolbox{arma64asm} \ar[r] & \resource{object file}}
\seeassembly\seearm\seeobject
}

\providecommand{\armbdism}{
\toolsection{arma64dism} is a disassembler for the ARM hardware architecture.
It translates machine code from object files targeting ARM processors executing A64 instructions into assembly code and writes it to the standard output stream.
\flowgraph{\resource{object file} \ar[r] & \toolbox{arma64dism} \ar[r] & \resource{disassembly\\listing}}
\seeassembly\seearm\seeobject
}

\providecommand{\armcasm}{
\toolsection{armt32asm} is an assembler for the ARM hardware architecture.
It translates assembly code into machine code for ARM processors executing T32 instructions and stores it in corresponding object files.
\flowgraph{\resource{ARM T32 assembly\\source code} \ar[r] & \toolbox{armt32asm} \ar[r] & \resource{object file}}
\seeassembly\seearm\seeobject
}

\providecommand{\armcdism}{
\toolsection{armt32dism} is a disassembler for the ARM hardware architecture.
It translates machine code from object files targeting ARM processors executing T32 instructions into assembly code and writes it to the standard output stream.
\flowgraph{\resource{object file} \ar[r] & \toolbox{armt32dism} \ar[r] & \resource{disassembly\\listing}}
\seeassembly\seearm\seeobject
}

\providecommand{\avrasm}{
\toolsection{avrasm} is an assembler for the AVR hardware architecture.
It translates assembly code into machine code for AVR processors and stores it in corresponding object files.
The identifiers \texttt{RXL}, \texttt{RXH}, \texttt{RYL}, \texttt{RYH}, \texttt{RZL}, and \texttt{RZH} are predefined and name the corresponding registers.
The identifiers \texttt{SPL} and \texttt{SPH} are also predefined and evaluate to the address of the corresponding registers.
\flowgraph{\resource{AVR assembly\\source code} \ar[r] & \toolbox{avrasm} \ar[r] & \resource{object file}}
\seeassembly\seeavr\seeobject
}

\providecommand{\avrdism}{
\toolsection{avrdism} is a disassembler for the AVR hardware architecture.
It translates machine code from object files targeting AVR processors into assembly code and writes it to the standard output stream.
\flowgraph{\resource{object file} \ar[r] & \toolbox{avrdism} \ar[r] & \resource{disassembly\\listing}}
\seeassembly\seeavr\seeobject
}

\providecommand{\avrttasm}{
\toolsection{avr32asm} is an assembler for the AVR32 hardware architecture.
It translates assembly code into machine code for AVR32 processors and stores it in corresponding object files.
\flowgraph{\resource{AVR32 assembly\\source code} \ar[r] & \toolbox{avr32asm} \ar[r] & \resource{object file}}
\seeassembly\seeavrtt\seeobject
}

\providecommand{\avrttdism}{
\toolsection{avr32dism} is a disassembler for the AVR32 hardware architecture.
It translates machine code from object files targeting AVR32 processors into assembly code and writes it to the standard output stream.
\flowgraph{\resource{object file} \ar[r] & \toolbox{avr32dism} \ar[r] & \resource{disassembly\\listing}}
\seeassembly\seeavrtt\seeobject
}

\providecommand{\mabkasm}{
\toolsection{m68kasm} is an assembler for the M68000 hardware architecture.
It translates assembly code into machine code for M68000 processors and stores it in corresponding object files.
\flowgraph{\resource{68000 assembly\\source code} \ar[r] & \toolbox{m68kasm} \ar[r] & \resource{object file}}
\seeassembly\seemabk\seeobject
}

\providecommand{\mabkdism}{
\toolsection{m68kdism} is a disassembler for the M68000 hardware architecture.
It translates machine code from object files targeting M68000 processors into assembly code and writes it to the standard output stream.
\flowgraph{\resource{object file} \ar[r] & \toolbox{m68kdism} \ar[r] & \resource{disassembly\\listing}}
\seeassembly\seemabk\seeobject
}

\providecommand{\miblasm}{
\toolsection{miblasm} is an assembler for the MicroBlaze hardware architecture.
It translates assembly code into machine code for MicroBlaze processors and stores it in corresponding object files.
\flowgraph{\resource{MicroBlaze assembly\\source code} \ar[r] & \toolbox{miblasm} \ar[r] & \resource{object file}}
\seeassembly\seemibl\seeobject
}

\providecommand{\mibldism}{
\toolsection{mibldism} is a disassembler for the MicroBlaze hardware architecture.
It translates machine code from object files targeting MicroBlaze processors into assembly code and writes it to the standard output stream.
\flowgraph{\resource{object file} \ar[r] & \toolbox{mibldism} \ar[r] & \resource{disassembly\\listing}}
\seeassembly\seemibl\seeobject
}

\providecommand{\mipsaasm}{
\toolsection{mips32asm} is an assembler for the MIPS32 hardware architecture.
It translates assembly code into machine code for MIPS32 processors and stores it in corresponding object files.
\flowgraph{\resource{MIPS32 assembly\\source code} \ar[r] & \toolbox{mips32asm} \ar[r] & \resource{object file}}
\seeassembly\seemips\seeobject
}

\providecommand{\mipsadism}{
\toolsection{mips32dism} is a disassembler for the MIPS32 hardware architecture.
It translates machine code from object files targeting MIPS32 processors into assembly code and writes it to the standard output stream.
\flowgraph{\resource{object file} \ar[r] & \toolbox{mips32dism} \ar[r] & \resource{disassembly\\listing}}
\seeassembly\seemips\seeobject
}

\providecommand{\mipsbasm}{
\toolsection{mips64asm} is an assembler for the MIPS64 hardware architecture.
It translates assembly code into machine code for MIPS64 processors and stores it in corresponding object files.
\flowgraph{\resource{MIPS64 assembly\\source code} \ar[r] & \toolbox{mips64asm} \ar[r] & \resource{object file}}
\seeassembly\seemips\seeobject
}

\providecommand{\mipsbdism}{
\toolsection{mips64dism} is a disassembler for the MIPS64 hardware architecture.
It translates machine code from object files targeting MIPS64 processors into assembly code and writes it to the standard output stream.
\flowgraph{\resource{object file} \ar[r] & \toolbox{mips64dism} \ar[r] & \resource{disassembly\\listing}}
\seeassembly\seemips\seeobject
}

\providecommand{\mmixasm}{
\toolsection{mmixasm} is an assembler for the MMIX hardware architecture.
It translates assembly code into machine code for MMIX processors and stores it in corresponding object files.
The names of all special registers are predefined and evaluate to the corresponding number.
\flowgraph{\resource{MMIX assembly\\source code} \ar[r] & \toolbox{mmixasm} \ar[r] & \resource{object file}}
\seeassembly\seemmix\seeobject
}

\providecommand{\mmixdism}{
\toolsection{mmixdism} is a disassembler for the MMIX hardware architecture.
It translates machine code from object files targeting MMIX processors into assembly code and writes it to the standard output stream.
\flowgraph{\resource{object file} \ar[r] & \toolbox{mmixdism} \ar[r] & \resource{disassembly\\listing}}
\seeassembly\seemmix\seeobject
}

\providecommand{\orokasm}{
\toolsection{or1kasm} is an assembler for the OpenRISC 1000 hardware architecture.
It translates assembly code into machine code for OpenRISC 1000 processors and stores it in corresponding object files.
\flowgraph{\resource{OpenRISC 1000 assembly\\source code} \ar[r] & \toolbox{or1kasm} \ar[r] & \resource{object file}}
\seeassembly\seeorok\seeobject
}

\providecommand{\orokdism}{
\toolsection{or1kdism} is a disassembler for the OpenRISC 1000 hardware architecture.
It translates machine code from object files targeting OpenRISC 1000 processors into assembly code and writes it to the standard output stream.
\flowgraph{\resource{object file} \ar[r] & \toolbox{or1kdism} \ar[r] & \resource{disassembly\\listing}}
\seeassembly\seeorok\seeobject
}

\providecommand{\ppcaasm}{
\toolsection{ppc32asm} is an assembler for the PowerPC hardware architecture.
It translates assembly code into machine code for PowerPC processors and stores it in corresponding object files.
By default, the assembler generates machine code for the 32-bit operating mode defined by the PowerPC architecture.
\flowgraph{\resource{PowerPC assembly\\source code} \ar[r] & \toolbox{ppc32asm} \ar[r] & \resource{object file}}
\seeassembly\seeppc\seeobject
}

\providecommand{\ppcadism}{
\toolsection{ppc32dism} is a disassembler for the PowerPC hardware architecture.
It translates machine code from object files targeting PowerPC processors into assembly code and writes it to the standard output stream.
It assumes that the machine code was generated for the 32-bit operating mode defined by the PowerPC architecture.
\flowgraph{\resource{object file} \ar[r] & \toolbox{ppc32dism} \ar[r] & \resource{disassembly\\listing}}
\seeassembly\seeppc\seeobject
}

\providecommand{\ppcbasm}{
\toolsection{ppc64asm} is an assembler for the PowerPC hardware architecture.
It translates assembly code into machine code for PowerPC processors and stores it in corresponding object files.
By default, the assembler generates machine code for the 64-bit operating mode defined by the PowerPC architecture.
\flowgraph{\resource{PowerPC assembly\\source code} \ar[r] & \toolbox{ppc64asm} \ar[r] & \resource{object file}}
\seeassembly\seeppc\seeobject
}

\providecommand{\ppcbdism}{
\toolsection{ppc64dism} is a disassembler for the PowerPC hardware architecture.
It translates machine code from object files targeting PowerPC processors into assembly code and writes it to the standard output stream.
It assumes that the machine code was generated for the 64-bit operating mode defined by the PowerPC architecture.
\flowgraph{\resource{object file} \ar[r] & \toolbox{ppc64dism} \ar[r] & \resource{disassembly\\listing}}
\seeassembly\seeppc\seeobject
}

\providecommand{\riscasm}{
\toolsection{riscasm} is an assembler for the RISC hardware architecture.
It translates assembly code into machine code for RISC processors and stores it in corresponding object files.
The names of all special registers are predefined and evaluate to the corresponding number.
\flowgraph{\resource{RISC assembly\\source code} \ar[r] & \toolbox{riscasm} \ar[r] & \resource{object file}}
\seeassembly\seerisc\seeobject
}

\providecommand{\riscdism}{
\toolsection{riscdism} is a disassembler for the RISC hardware architecture.
It translates machine code from object files targeting RISC processors into assembly code and writes it to the standard output stream.
\flowgraph{\resource{object file} \ar[r] & \toolbox{riscdism} \ar[r] & \resource{disassembly\\listing}}
\seeassembly\seerisc\seeobject
}

\providecommand{\wasmasm}{
\toolsection{wasmasm} is an assembler for the WebAssembly architecture.
It translates assembly code into machine code for WebAssembly targets and stores it in corresponding object files.
The names of all special registers are predefined and evaluate to the corresponding number.
\flowgraph{\resource{WebAssembly assembly\\source code} \ar[r] & \toolbox{wasmasm} \ar[r] & \resource{object file}}
\seeassembly\seewasm\seeobject
}

\providecommand{\wasmdism}{
\toolsection{wasmdism} is a disassembler for the WebAssembly architecture.
It translates machine code from object files targeting WebAssembly targets into assembly code and writes it to the standard output stream.
\flowgraph{\resource{object file} \ar[r] & \toolbox{wasmdism} \ar[r] & \resource{disassembly\\listing}}
\seeassembly\seewasm\seeobject
}

% linker tools

\providecommand{\linklib}{
\toolsection{linklib} is an object file combiner.
It creates a static library file by combining all object files given to it into a single one.
\flowgraph{\resource{object files} \ar[r] & \toolbox{linklib} \ar[r] & \resource{library file}}
\seeobject
}

\providecommand{\linkbin}{
\toolsection{linkbin} is a linker for plain binary files.
It links all object files given to it into a single image and stores it in a binary file that begins with the first linked section.
It also creates a map file that lists the address, type, name and size of all used sections.
The filename extension of the resulting binary file can be specified by putting it into a constant data section called \texttt{\_extension}.
\flowgraph{\resource{object files} \ar[r] & \toolbox{linkbin} \ar[r] \ar[d] & \resource{binary file} \\ & \resource{map file}}
\seeobject
}

\providecommand{\linkmem}{
\toolsection{linkmem} is a linker for plain binary files partitioned into random-access and read-only memory.
It links all object files given to it into two distinct images, one for data sections and one for code and constant data sections, and stores each image in a binary file that begins with the first linked section of the corresponding type.
It also creates a map file that lists the address, type, name and size of all used sections.
\flowgraph{\resource{object files} \ar[r] & \toolbox{linkmem} \ar[r] \ar[d] & \resource{RAM file/\\ROM file} \\ & \resource{map file}}
\seeobject
}

\providecommand{\linkprg}{
\toolsection{linkprg} is a linker for GEMDOS executable files.
It links all object files given to it into a single image and stores the image in an Atari GEMDOS executable file~\cite{gemdosfile}.
It also creates a map file that lists the address relative to the text segment, type, name and size of all used sections.
The filename extension of the resulting executable file can be specified by putting it into a constant data section called \texttt{\_extension}.
The GEMDOS executable file format requires all patch patterns of absolute link patches to consist of four full bitmasks with descending offsets.
\flowgraph{\resource{object files} \ar[r] & \toolbox{linkprg} \ar[r] \ar[d] & \resource{executable file} \\ & \resource{map file}}
\seeobject
}

\providecommand{\linkhex}{
\toolsection{linkhex} is a linker for Intel HEX files.
It links all code sections of the object files given to it into single image and stores the image in an Intel HEX file~\cite{hexfile} that begins with the first linked section.
It also creates a map file that lists the address, type, name and size of all used sections.
\flowgraph{\resource{object files} \ar[r] & \toolbox{linkhex} \ar[r] \ar[d] & \resource{HEX file} \\ & \resource{map file}}
\seeobject
}

\providecommand{\mapsearch}{
\toolsection{mapsearch} is a debugging tool.
It searches map files generated by linker tools for the name of a binary section that encompasses a memory address read from the standard input stream.
If additionally provided with one or more object files, it also stores an excerpt thereof in a separate object file called map search result which only contains the identified binary section for disassembling purposes.
\flowgraph{& \resource{map files/\\object files} \ar[d] \\ \resource{memory\\address} \ar[r] & \toolbox{mapsearch} \ar[r] \ar[d] & \resource{section name/\\relative offset} \\ & \resource{object file\\excerpt}}
\seeobject
}

\renewcommand{\seemabk}{}

\startchapter{M68000}{M68000 Hardware Architecture Support}{m68k}
{This \documentation{} describes how the \ecs{} supports the M68000 hardware architecture.
This includes information about the assembler, disassembler, and the various compilers featured by the \ecs{} as well as the interoperability between these tools.}

\section{Introduction}

The \ecs{} features various compilers, an assembler, and a disassembler that target the M68000 hardware architecture by Motorola.
Figure~\ref{fig:m68kdataflow} shows the data flow in-between these tools.

\begin{figure}
\flowgraph{
\resource{intermediate\\code} \ar[d] & & \resource{assembly\\source code} \ar[d] \\
\converter{M68000\\Generator} \ar[r] \ar[rd] \ar[d] & \resource{assembly\\listing} \ar[r] & \converter{M68000\\Assembler} \ar[ld] \\
\resource{debugging\\information} & \resource{object file} \ar[d] \\
& \converter{M68000\\Disassembler} \ar[d] \\
& \resource{disassembly\\listing} \\
}\caption{Data flow within the tools targeting the M68000 architecture}
\label{fig:m68kdataflow}
\end{figure}

All compilers targeting the M68000 architecture translate their programs using an intermediate code representation.
The M68000 generator is able to translate the intermediate code representation of a program into machine code for M68000 processors.
It stores the resulting binary code and data in so-called object files.
Additionally, the generator is able to create an assembly code listing of the machine code for debugging purposes.
This assembly code listing can also be processed by the assembler yielding exactly the same object file.
The disassembler is able to open object files and print a human-readable disassembly listing of their contents.
\seeobject\seecode

\section{Instruction Set}

Tools targeting the M68000 architecture support the instruction set listed in Table~\ref{tab:m68kset} and use the same assembly syntax as predefined by Motorola~\cite{m68k:instructionset}.
The only exception are immediate values which are not prefixed by a number sign.
\seeassembly

\instructionset{m68k}{Supported M68000 instruction set}{5}{6}

\section{Calling Convention}\index{Calling convention!of M68000}

The machine code generator and runtime support for the M68000 architecture as provided by the \ecs{} use the following calling convention in order to enable interoperability.

\subsection{Stack Operations}

Arguments for functions are in general passed using the stack according to the intermediate code specification.
See \Documentation{}~\documentationref{code}{Intermediate Code Representation} for more information about the role of the stack.
Function arguments are pushed on the stack in reverse order and cleaned by the caller.

\subsection{Floating-Point Support}

There is currently no support for floating-point operations.

\subsection{Register Mapping}

The special-purpose registers defined by the intermediate code representation are mapped to their corresponding physical registers in the following way:

\begin{itemize}

\item Result Register\alignright\texttt{\$res}\nopagebreak

The intermediate code result register \texttt{\$res} is mapped to M68000 registers \texttt{a0} or \texttt{d0} and \texttt{d1} depending on the actual return type.

\item Stack Pointer Register\alignright\texttt{\$sp}\nopagebreak

The intermediate code stack pointer register \texttt{\$sp} is mapped to the M68000 register \texttt{a7} (\texttt{sp}).

\item Frame Pointer Register\alignright\texttt{\$fp}\nopagebreak

The intermediate code stack pointer register \texttt{\$fp} is mapped to the M68000 register \texttt{a6}.

\end{itemize}

All other intermediate code registers are mapped as needed to the remaining physical registers.
Their contents and mapping are therefore considered volatile across function calls.

\section{Runtime Support}\index{Runtime support!for M68000}

The \ecs{} provides runtime support for the M68000 architecture and runtime environments based on this hardware architecture in object files.
Users targeting a specific runtime environment have to use an appropriate linker together with these object files in order create an executable program.
This section gives information about all supported runtime environments based on the M68000 hardware architecture as well as the required combination of linker and object files.

Basic architectural runtime support is provided by the object file \objfile{m86k\-run}.
Users should always include this object file during linking regardless of the actual target runtime environment.
All other object files given to the linker should target the same hardware architecture.

Programs written in \cpp{} need additional runtime support stored in the \libfile{cpp\-m86k\-run} library file.
Programs written in Oberon need additional runtime support stored in the \libfile{ob\-m86k\-run} library file.
\seecpp\seeoberon

Programs targeting the Atari TOS operating system are created using the \tool{link\-prg} linker tool.
It creates GEMDOS executable files~\cite{gemdosfile} if provided with the runtime support stored in the \objfile{tos\-run} object file.
Calling the \tool{ecsd} utility tool using the \environment{tos} target environment achieves the same result.
GEMDOS, BIOS, and XBIOS functions are available using additional runtime support stored in the \objfile{tos\-api} object file.

\section{M68000 Tools}

The \ecs{} provides the following tools that are able to process object files targeting the M68000 hardware architecture.
\interface

\cdmabk
\cppmabk
\falmabk
\obmabk
\mabkasm
\mabkdism
\linkprg

\concludechapter

// MicroBlaze instruction set definitions
// Copyright (C) Florian Negele

// This file is part of the Eigen Compiler Suite.

// The ECS is free software: you can redistribute it and/or modify
// it under the terms of the GNU General Public License as published by
// the Free Software Foundation, either version 3 of the License, or
// (at your option) any later version.

// The ECS is distributed in the hope that it will be useful,
// but WITHOUT ANY WARRANTY; without even the implied warranty of
// MERCHANTABILITY or FITNESS FOR A PARTICULAR PURPOSE.  See the
// GNU General Public License for more details.

// You should have received a copy of the GNU General Public License
// along with the ECS.  If not, see <https://www.gnu.org/licenses/>.

#ifndef INSTR
	#define INSTR(mnem, code, mask, type1, type2, type3)
#endif

#ifndef MNEM
	#define MNEM(name, mnem, ...)
#endif

#ifndef SREG
	#define SREG(reg, name, number)
#endif

#ifndef TYPE
	#define TYPE(type)
#endif

// mnemonics

MNEM (add,        ADD,        Add)
MNEM (addc,       ADDC,       Add with Carry)
MNEM (addi,       ADDI,       Add Immediate)
MNEM (addic,      ADDIC,      Add Immediate with Carry)
MNEM (addik,      ADDIK,      Add Immediate and Keep Carry)
MNEM (addikc,     ADDIKC,     Add Immediate with Carry and Keep Carry)
MNEM (addk,       ADDK,       Add and Keep Carry)
MNEM (addkc,      ADDKC,      Add with Carry and Keep Carry)
MNEM (aget,       AGET,       Get Data from Stream Interface Atomic)
MNEM (agetd,      AGETD,      Get Data from Stream Interface Dynamic Atomic)
MNEM (and,        AND,        Logical AND)
MNEM (andi,       ANDI,       Logical AND with Immediate)
MNEM (andn,       ANDN,       Logical AND NOT)
MNEM (andni,      ANDNI,      Logical AND NOT with Immediate)
MNEM (aput,       APUT,       Put Data to Stream Interface Atomic)
MNEM (aputd,      APUTD,      Put Data to Stream Interface Dynamic Atomic)
MNEM (beq,        BEQ,        Branch if Equal)
MNEM (beqd,       BEQD,       Branch if Equal with Delay)
MNEM (beqi,       BEQI,       Branch Immediate if Equal)
MNEM (beqid,      BEQID,      Branch Immediate if Equal with Delay)
MNEM (bge,        BGE,        Branch if Greater or Equal)
MNEM (bged,       BGED,       Branch if Greater or Equal with Delay)
MNEM (bgei,       BGEI,       Branch Immediate if Greater or Equal)
MNEM (bgeid,      BGEID,      Branch Immediate if Greater or Equal with Delay)
MNEM (bgt,        BGT,        Branch if Greater Than)
MNEM (bgtd,       BGTD,       Branch if Greater Than with Delay)
MNEM (bgti,       BGTI,       Branch Immediate if Greater Than)
MNEM (bgtid,      BGTID,      Branch Immediate if Greater Than with Delay)
MNEM (ble,        BLE,        Branch if Less or Equal)
MNEM (bled,       BLED,       Branch if Less or Equal with Delay)
MNEM (blei,       BLEI,       Branch Immediate if Less or Equal)
MNEM (bleid,      BLEID,      Branch Immediate if Less or Equal with Delay)
MNEM (blt,        BLT,        Branch if Less Than)
MNEM (bltd,       BLTD,       Branch if Less Than with Delay)
MNEM (blti,       BLTI,       Branch Immediate if Less Than)
MNEM (bltid,      BLTID,      Branch Immediate if Less Than with Delay)
MNEM (bne,        BNE,        Branch if Not Equal)
MNEM (bned,       BNED,       Branch if Not Equal with Delay)
MNEM (bnei,       BNEI,       Branch Immediate if Not Equal)
MNEM (bneid,      BNEID,      Branch Immediate if Not Equal with Delay)
MNEM (br,         BR,         Branch)
MNEM (bra,        BRA,        Branch Absolute)
MNEM (brad,       BRAD,       Branch Absolute with Delay)
MNEM (brai,       BRAI,       Branch Absolute Immediate)
MNEM (braid,      BRAID,      Branch Absolute Immediate with Delay)
MNEM (brald,      BRALD,      Branch Absolute and Link with Delay)
MNEM (bralid,     BRALID,     Branch Absolute and Link Immediate with Delay)
MNEM (brd,        BRD,        Branch with Delay)
MNEM (bri,        BRI,        Branch Immediate)
MNEM (brid,       BRID,       Branch Immediate with Delay)
MNEM (brk,        BRK,        Break)
MNEM (brki,       BRKI,       Break Immediate)
MNEM (brld,       BRLD,       Branch and Link with Delay)
MNEM (brlid,      BRLID,      Branch and Link Immediate with Delay)
MNEM (bsll,       BSLL,       Barrel Shift Left Logical)
MNEM (bslli,      BSLLI,      Barrel Shift Left Logical Immediate)
MNEM (bsra,       BSRA,       Barrel Shift Right Arithmetical)
MNEM (bsrai,      BSRAI,      Barrel Shift Right Arithmetical Immediate)
MNEM (bsrl,       BSRL,       Barrel Shift Right Logical)
MNEM (bsrli,      BSRLI,      Barrel Shift Right Logical Immediate)
MNEM (caget,      CAGET,      Get Control from Stream Interface Atomic)
MNEM (cagetd,     CAGETD,     Get Control from Stream Interface Dynamic Atomic)
MNEM (caput,      CAPUT,      Put Control to Stream Interface Atomic)
MNEM (caputd,     CAPUTD,     Put Control to Stream Interface Dynamic Atomic)
MNEM (cget,       CGET,       Get Control from Stream Interface)
MNEM (cgetd,      CGETD,      Get Control from Stream Interface Dynamic)
MNEM (clz,        CLZ,        Count Leading Zeros)
MNEM (cmp,        CMP,        Signed Integer Compare)
MNEM (cmpu,       CMPU,       Unsigned Integer Compare)
MNEM (cput,       CPUT,       Put Control to Stream Interface)
MNEM (cputd,      CPUTD,      Put Control to Stream Interface Dynamic)
MNEM (eaget,      EAGET,      Get Data from Stream Interface Exceptional Atomic)
MNEM (eagetd,     EAGETD,     Get Data from Stream Interface Dynamic Exceptional Atomic)
MNEM (ecaget,     ECAGET,     Get Control from Stream Interface Exceptional Atomic)
MNEM (ecagetd,    ECAGETD,    Get Control from Stream Interface Dynamic Exceptional Atomic)
MNEM (ecget,      ECGET,      Get Control from Stream Interface Exceptional)
MNEM (ecgetd,     ECGETD,     Get Control from Stream Interface Dynamic Exceptional)
MNEM (eget,       EGET,       Get Data from Stream Interface Exceptional)
MNEM (egetd,      EGETD,      Get Data from Stream Interface Dynamic Exceptional)
MNEM (fadd,       FADD,       Floating Point Arithmetic Add)
MNEM (fcmp.eq,    FCMPEQ,     Equal floating point comparison)
MNEM (fcmp.ge,    FCMPGE,     Greater-or-Equal floating point comparison)
MNEM (fcmp.gt,    FCMPGT,     Greater-than floating point comparison)
MNEM (fcmp.le,    FCMPLE,     Less-or-Equal floating point comparison)
MNEM (fcmp.lt,    FCMPLT,     Less-than floating point comparison)
MNEM (fcmp.ne,    FCMPNE,     Not-Equal floating point comparison)
MNEM (fcmp.un,    FCMPUN,     Unordered floating point comparison)
MNEM (fdiv,       FDIV,       Floating Point Arithmetic Division)
MNEM (fint,       FINT,       Floating Point Convert Float to Integer)
MNEM (flt,        FLT,        Floating Point Convert Integer to Float)
MNEM (fmul,       FMUL,       Floating Point Arithmetic Multiplication)
MNEM (frsub,      FRSUB,      Reverse Floating Point Arithmetic Subtraction)
MNEM (fsqrt,      FSQRT,      Floating Point Arithmetic Square Root)
MNEM (get,        GET,        Get Data from Stream Interface)
MNEM (getd,       GETD,       Get Data from Stream Interface Dynamic)
MNEM (idiv,       IDIV,       Signed Integer Divide)
MNEM (idivu,      IDIVU,      Unsigned Integer Divide)
MNEM (imm,        IMM,        Immediate)
MNEM (lbu,        LBU,        Load Byte Unsigned)
MNEM (lbui,       LBUI,       Load Byte Unsigned Immediate)
MNEM (lbur,       LBUR,       Load Byte Unsigned Reserved)
MNEM (lhu,        LHU,        Load Halfword Unsigned)
MNEM (lhui,       LHUI,       Load Halfword Unsigned Immediate)
MNEM (lhur,       LHUR,       Load Halfword Unsigned Reserved)
MNEM (lw,         LW,         Load Word)
MNEM (lwi,        LWI,        Load Word Immediate)
MNEM (lwr,        LWR,        Load Word Reserved)
MNEM (lwx,        LWX,        Load Word Exclusive)
MNEM (mbar,       MBAR,       Memory Barrier)
MNEM (mfs,        MFS,        Move From Special Purpose Register)
MNEM (msrclr,     MSRCLR,     Read MSR and clear bits in MSR)
MNEM (msrset,     MSRSET,     Read MSR and set bits in MSR)
MNEM (mts,        MTS,        Move To Special Purpose Register)
MNEM (mul,        MUL,        Multiply)
MNEM (mulh,       MULH,       Multiply High)
MNEM (mulhsu,     MULHSU,     Multiply High Signed Unsigned)
MNEM (mulhu,      MULHU,      Multiply High Unsigned)
MNEM (muli,       MULI,       Multiply Immediate)
MNEM (naget,      NAGET,      Get Data from Stream Interface Non-Blocking Atomic)
MNEM (nagetd,     NAGETD,     Get Data from Stream Interface Dynamic Non-Blocking Atomic)
MNEM (naput,      NAPUT,      Put Data to Stream Interface Non-Blocking Atomic)
MNEM (naputd,     NAPUTD,     Put Data to Stream Interface Dynamic Non-Blocking Atomic)
MNEM (ncaget,     NCAGET,     Get Control from Stream Interface Non-Blocking Atomic)
MNEM (ncagetd,    NCAGETD,    Get Control from Stream Interface Dynamic Non-Blocking Atomic)
MNEM (ncaput,     NCAPUT,     Put Control to Stream Interface Non-Blocking Atomic)
MNEM (ncaputd,    NCAPUTD,    Put Control to Stream Interface Dynamic Non-Blocking Atomic)
MNEM (ncget,      NCGET,      Get Control from Stream Interface Non-Blocking)
MNEM (ncgetd,     NCGETD,     Get Control from Stream Interface Dynamic Non-Blocking)
MNEM (ncput,      NCPUT,      Put Control to Stream Interface Non-Blocking)
MNEM (ncputd,     NCPUTD,     Put Control to Stream Interface Dynamic Non-Blocking)
MNEM (neaget,     NEAGET,     Get Data from Stream Interface Non-Blocking Exceptional Atomic)
MNEM (neagetd,    NEAGETD,    Get Data from Stream Interface Dynamic Non-Blocking Exceptional Atomic)
MNEM (necaget,    NECAGET,    Get Control from Stream Interface Non-Blocking Exceptional Atomic)
MNEM (necagetd,   NECAGETD,   Get Control from Stream Interface Dynamic Non-Blocking Exceptional Atomic)
MNEM (necget,     NECGET,     Get Control from Stream Interface Non-Blocking Exceptional)
MNEM (necgetd,    NECGETD,    Get Control from Stream Interface Dynamic Non-Blocking Exceptional)
MNEM (neget,      NEGET,      Get Data from Stream Interface Non-Blocking Exceptional)
MNEM (negetd,     NEGETD,     Get Data from Stream Interface Dynamic Non-Blocking Exceptional)
MNEM (nget,       NGET,       Get Data from Stream Interface Non-Blocking)
MNEM (ngetd,      NGETD,      Get Data from Stream Interface Dynamic Non-Blocking)
MNEM (nop,        NOP,        No Operation)
MNEM (nput,       NPUT,       Put Data to Stream Interface Non-Blocking)
MNEM (nputd,      NPUTD,      Put Data to Stream Interface Dynamic Non-Blocking)
MNEM (or,         OR,         Logical OR)
MNEM (ori,        ORI,        Logical OR with Immediate)
MNEM (pcmpbf,     PCMPBF,     Pattern Compare Byte Find)
MNEM (pcmpeq,     PCMPEQ,     Pattern Compare Equal)
MNEM (pcmpne,     PCMPNE,     Pattern Compare Not Equal)
MNEM (put,        PUT,        Put Data to Stream Interface)
MNEM (putd,       PUTD,       Put Data to Stream Interface Dynamic)
MNEM (rsub,       RSUB,       Arithmetic Reverse Subtract)
MNEM (rsubc,      RSUBC,      Arithmetic Reverse Subtract with Carry)
MNEM (rsubi,      RSUBI,      Arithmetic Reverse Subtract Immediate)
MNEM (rsubic,     RSUBIC,     Arithmetic Reverse Subtract Immediate with Carry)
MNEM (rsubik,     RSUBIK,     Arithmetic Reverse Subtract Immediate and Keep Carry)
MNEM (rsubikc,    RSUBIKC,    Arithmetic Reverse Subtract Immediate with Carry and Keep Carry)
MNEM (rsubk,      RSUBK,      Arithmetic Reverse Subtract and Keep Carry)
MNEM (rsubkc,     RSUBKC,     Arithmetic Reverse Subtract with Carry and Keep Carry)
MNEM (rtbd,       RTBD,       Return from Break)
MNEM (rted,       RTED,       Return from Exception)
MNEM (rtid,       RTID,       Return from Interrupt)
MNEM (rtsd,       RTSD,       Return from Subroutine)
MNEM (sb,         SB,         Store Byte)
MNEM (sbi,        SBI,        Store Byte Immediate)
MNEM (sbr,        SBR,        Store Byte Reserved)
MNEM (sext16,     SEXT16,     Sign Extend Halfword)
MNEM (sext8,      SEXT8,      Sign Extend Byte)
MNEM (sh,         SH,         Store Halfword)
MNEM (shi,        SHI,        Store Halfword Immediate)
MNEM (shr,        SHR,        Store Halfword Reserved)
MNEM (sra,        SRA,        Shift Right Arithmetic)
MNEM (src,        SRC,        Shift Right with Carry)
MNEM (srl,        SRL,        Shift Right Logical)
MNEM (sw,         SW,         Store Word)
MNEM (swapb,      SWAPB,      Swap Bytes)
MNEM (swaph,      SWAPH,      Swap Halfwords)
MNEM (swi,        SWI,        Store Word Immediate)
MNEM (swr,        SWR,        Store Word Reserved)
MNEM (swx,        SWX,        Store Word Exclusive)
MNEM (taget,      TAGET,      Get Data from Stream Interface Test-Only Atomic)
MNEM (tagetd,     TAGETD,     Get Data from Stream Interface Dynamic Test-Only Atomic)
MNEM (taput,      TAPUT,      Put Data to Stream Interface Test-Only Atomic)
MNEM (taputd,     TAPUTD,     Put Data to Stream Interface Dynamic Test-Only Atomic)
MNEM (tcaget,     TCAGET,     Get Control from Stream Interface Test-Only Atomic)
MNEM (tcagetd,    TCAGETD,    Get Control from Stream Interface Dynamic Test-Only Atomic)
MNEM (tcaput,     TCAPUT,     Put Control to Stream Interface Test-Only Atomic)
MNEM (tcaputd,    TCAPUTD,    Put Control to Stream Interface Dynamic Test-Only Atomic)
MNEM (tcget,      TCGET,      Get Control from Stream Interface Test-Only)
MNEM (tcgetd,     TCGETD,     Get Control from Stream Interface Dynamic Test-Only)
MNEM (tcput,      TCPUT,      Put Control to Stream Interface Test-Only)
MNEM (tcputd,     TCPUTD,     Put Control to Stream Interface Dynamic Test-Only)
MNEM (teaget,     TEAGET,     Get Data from Stream Interface Test-Only Exceptional Atomic)
MNEM (teagetd,    TEAGETD,    Get Data from Stream Interface Dynamic Test-Only Exceptional Atomic)
MNEM (tecaget,    TECAGET,    Get Control from Stream Interface Test-Only Exceptional Atomic)
MNEM (tecagetd,   TECAGETD,   Get Control from Stream Interface Dynamic Test-Only Exceptional Atomic)
MNEM (tecget,     TECGET,     Get Control from Stream Interface Test-Only Exceptional)
MNEM (tecgetd,    TECGETD,    Get Control from Stream Interface Dynamic Test-Only Exceptional)
MNEM (teget,      TEGET,      Get Data from Stream Interface Test-Only Exceptional)
MNEM (tegetd,     TEGETD,     Get Data from Stream Interface Dynamic Test-Only Exceptional)
MNEM (tget,       TGET,       Get Data from Stream Interface Test-Only)
MNEM (tgetd,      TGETD,      Get Data from Stream Interface Dynamic Test-Only)
MNEM (tnaget,     TNAGET,     Get Data from Stream Interface Test-Only Non-Blocking Atomic)
MNEM (tnagetd,    TNAGETD,    Get Data from Stream Interface Dynamic Test-Only Non-Blocking Atomic)
MNEM (tnaput,     TNAPUT,     Put Data to Stream Interface Test-Only Non-Blocking Atomic)
MNEM (tnaputd,    TNAPUTD,    Put Data to Stream Interface Dynamic Test-Only Non-Blocking Atomic)
MNEM (tncaget,    TNCAGET,    Get Control from Stream Interface Test-Only Non-Blocking Atomic)
MNEM (tncagetd,   TNCAGETD,   Get Control from Stream Interface Dynamic Test-Only Non-Blocking Atomic)
MNEM (tncaput,    TNCAPUT,    Put Control to Stream Interface Test-Only Non-Blocking Atomic)
MNEM (tncaputd,   TNCAPUTD,   Put Control to Stream Interface Dynamic Test-Only Non-Blocking Atomic)
MNEM (tncget,     TNCGET,     Get Control from Stream Interface Test-Only Non-Blocking)
MNEM (tncgetd,    TNCGETD,    Get Control from Stream Interface Dynamic Test-Only Non-Blocking)
MNEM (tncput,     TNCPUT,     Put Control to Stream Interface Test-Only Non-Blocking)
MNEM (tncputd,    TNCPUTD,    Put Control to Stream Interface Dynamic Test-Only Non-Blocking)
MNEM (tneaget,    TNEAGET,    Get Data from Stream Interface Test-Only Non-Blocking Exceptional Atomic)
MNEM (tneagetd,   TNEAGETD,   Get Data from Stream Interface Dynamic Test-Only Non-Blocking Exceptional Atomic)
MNEM (tnecaget,   TNECAGET,   Get Control from Stream Interface Test-Only Non-Blocking Exceptional Atomic)
MNEM (tnecagetd,  TNECAGETD,  Get Control from Stream Interface Dynamic Test-Only Non-Blocking Exceptional Atomic)
MNEM (tnecget,    TNECGET,    Get Control from Stream Interface Test-Only Non-Blocking Exceptional)
MNEM (tnecgetd,   TNECGETD,   Get Control from Stream Interface Dynamic Test-Only Non-Blocking Exceptional)
MNEM (tneget,     TNEGET,     Get Data from Stream Interface Test-Only Non-Blocking Exceptional)
MNEM (tnegetd,    TNEGETD,    Get Data from Stream Interface Dynamic Test-Only Non-Blocking Exceptional)
MNEM (tnget,      TNGET,      Get Data from Stream Interface Test-Only Non-Blocking)
MNEM (tngetd,     TNGETD,     Get Data from Stream Interface Dynamic Test-Only Non-Blocking)
MNEM (tnput,      TNPUT,      Put Data to Stream Interface Test-Only Non-Blocking)
MNEM (tnputd,     TNPUTD,     Put Data to Stream Interface Dynamic Test-Only Non-Blocking)
MNEM (tput,       TPUT,       Put Data to Stream Interface Test-Only)
MNEM (tputd,      TPUTD,      Put Data to Stream Interface Dynamic Test-Only)
MNEM (wdc,        WDC,        Write to Data Cache)
MNEM (wdc.clear,  WDCCLEAR,   Write to Data Cache Clear)
MNEM (wdc.flush,  WDCFLUSH,   Write to Data Cache Flush)
MNEM (wic,        WIC,        Write to Instruction Cache)
MNEM (xor,        XOR,        Logical Exclusive OR)
MNEM (xori,       XORI,       Logical Exclusive OR with Immediate)

// instructions

INSTR (NOP,        0x00000000,  0xffffffff,  Void,  Void,  Void)
INSTR (ADD,        0x00000000,  0xfc0007ff,  RD,    RA,    RB)
INSTR (ADDC,       0x08000000,  0xfc0007ff,  RD,    RA,    RB)
INSTR (ADDK,       0x10000000,  0xfc0007ff,  RD,    RA,    RB)
INSTR (ADDKC,      0x18000000,  0xfc0007ff,  RD,    RA,    RB)
INSTR (ADDI,       0x20000000,  0xfc000000,  RD,    RA,    S16)
INSTR (ADDIC,      0x28000000,  0xfc000000,  RD,    RA,    S16)
INSTR (ADDIK,      0x30000000,  0xfc000000,  RD,    RA,    S16)
INSTR (ADDIKC,     0x38000000,  0xfc000000,  RD,    RA,    S16)
INSTR (AND,        0x84000000,  0xfc0007ff,  RD,    RA,    RB)
INSTR (ANDI,       0xa4000000,  0xfc000000,  RD,    RA,    S16)
INSTR (ANDN,       0x8c000000,  0xfc0007ff,  RD,    RA,    RB)
INSTR (ANDNI,      0xac000000,  0xfc000000,  RD,    RA,    S16)
INSTR (BEQ,        0x9c000000,  0xffe007ff,  RA,    RB,    Void)
INSTR (BEQD,       0x9e000000,  0xffe007ff,  RA,    RB,    Void)
INSTR (BEQI,       0xbc000000,  0xffe00000,  RA,    O16,   Void)
INSTR (BEQID,      0xbe000000,  0xffe00000,  RA,    O16,   Void)
INSTR (BGE,        0x9ca00000,  0xffe007ff,  RA,    RB,    Void)
INSTR (BGED,       0x9ea00000,  0xffe007ff,  RA,    RB,    Void)
INSTR (BGEI,       0xbca00000,  0xffe00000,  RA,    O16,   Void)
INSTR (BGEID,      0xbea00000,  0xffe00000,  RA,    O16,   Void)
INSTR (BGT,        0x9c800000,  0xffe007ff,  RA,    RB,    Void)
INSTR (BGTD,       0x9e800000,  0xffe007ff,  RA,    RB,    Void)
INSTR (BGTI,       0xbc800000,  0xffe00000,  RA,    O16,   Void)
INSTR (BGTID,      0xbe800000,  0xffe00000,  RA,    O16,   Void)
INSTR (BLE,        0x9c600000,  0xffe007ff,  RA,    RB,    Void)
INSTR (BLED,       0x9e600000,  0xffe007ff,  RA,    RB,    Void)
INSTR (BLEI,       0xbc600000,  0xffe00000,  RA,    O16,   Void)
INSTR (BLEID,      0xbe600000,  0xffe00000,  RA,    O16,   Void)
INSTR (BLT,        0x9c400000,  0xffe007ff,  RA,    RB,    Void)
INSTR (BLTD,       0x9e400000,  0xffe007ff,  RA,    RB,    Void)
INSTR (BLTI,       0xbc400000,  0xffe00000,  RA,    O16,   Void)
INSTR (BLTID,      0xbe400000,  0xffe00000,  RA,    O16,   Void)
INSTR (BNE,        0x9c200000,  0xffe007ff,  RA,    RB,    Void)
INSTR (BNED,       0x9e200000,  0xffe007ff,  RA,    RB,    Void)
INSTR (BNEI,       0xbc200000,  0xffe00000,  RA,    O16,   Void)
INSTR (BNEID,      0xbe200000,  0xffe00000,  RA,    O16,   Void)
INSTR (BR,         0x98000000,  0xfc1f07ff,  RB,    Void,  Void)
INSTR (BRA,        0x98080000,  0xfc1f07ff,  RB,    Void,  Void)
INSTR (BRD,        0x98100000,  0xfc1f07ff,  RB,    Void,  Void)
INSTR (BRAD,       0x98180000,  0xfc1f07ff,  RB,    Void,  Void)
INSTR (BRLD,       0x98140000,  0xfc1f07ff,  RD,    RB,    Void)
INSTR (BRALD,      0x981c0000,  0xfc1f07ff,  RD,    RB,    Void)
INSTR (BRI,        0xb8000000,  0xfc1f0000,  O16,   Void,  Void)
INSTR (BRAI,       0xb8080000,  0xfc1f0000,  S16,   Void,  Void)
INSTR (BRID,       0xb8100000,  0xfc1f0000,  O16,   Void,  Void)
INSTR (BRAID,      0xb8180000,  0xfc1f0000,  O16,   Void,  Void)
INSTR (BRLID,      0xb8140000,  0xfc1f0000,  RD,    O16,   Void)
INSTR (BRALID,     0xb81c0000,  0xfc1f0000,  RD,    S16,   Void)
INSTR (BRK,        0x980c0000,  0xfc1f07ff,  RD,    RB,    Void)
INSTR (BRKI,       0xb80c0000,  0xfc1f0000,  RD,    S16,   Void)
INSTR (BSRL,       0x44000000,  0xfc0007ff,  RD,    RA,    RB)
INSTR (BSRA,       0x44000200,  0xfc0007ff,  RD,    RA,    RB)
INSTR (BSLL,       0x44000400,  0xfc0007ff,  RD,    RA,    RB)
INSTR (BSRLI,      0x64000000,  0xfc00ffe0,  RD,    RA,    U5)
INSTR (BSRAI,      0x64000200,  0xfc00ffe0,  RD,    RA,    U5)
INSTR (BSLLI,      0x64000400,  0xfc00ffe0,  RD,    RA,    U5)
INSTR (CLZ,        0x900000e0,  0xfc00ffff,  RD,    RA,    Void)
INSTR (CMP,        0x14000001,  0xfc0007ff,  RD,    RA,    RB)
INSTR (CMPU,       0x14000003,  0xfc0007ff,  RD,    RA,    RB)
INSTR (FADD,       0x58000000,  0xfc0007ff,  RD,    RA,    RB)
INSTR (FRSUB,      0x58000080,  0xfc0007ff,  RD,    RA,    RB)
INSTR (FMUL,       0x58000100,  0xfc0007ff,  RD,    RA,    RB)
INSTR (FDIV,       0x58000180,  0xfc0007ff,  RD,    RA,    RB)
INSTR (FCMPUN,     0x58000200,  0xfc0007ff,  RD,    RA,    RB)
INSTR (FCMPLT,     0x58000210,  0xfc0007ff,  RD,    RA,    RB)
INSTR (FCMPEQ,     0x58000220,  0xfc0007ff,  RD,    RA,    RB)
INSTR (FCMPLE,     0x58000230,  0xfc0007ff,  RD,    RA,    RB)
INSTR (FCMPGT,     0x58000240,  0xfc0007ff,  RD,    RA,    RB)
INSTR (FCMPNE,     0x58000250,  0xfc0007ff,  RD,    RA,    RB)
INSTR (FCMPGE,     0x58000260,  0xfc0007ff,  RD,    RA,    RB)
INSTR (FLT,        0x58000280,  0xfc00ffff,  RD,    RA,    Void)
INSTR (FINT,       0x58000300,  0xfc00ffff,  RD,    RA,    Void)
INSTR (FSQRT,      0x58000380,  0xfc00ffff,  RD,    RA,    Void)
INSTR (GET,        0x6c000000,  0xfc1ffff0,  RD,    FSL,   Void)
INSTR (TGET,       0x6c001000,  0xfc1ffff0,  RD,    FSL,   Void)
INSTR (NGET,       0x6c004000,  0xfc1ffff0,  RD,    FSL,   Void)
INSTR (TNGET,      0x6c005000,  0xfc1ffff0,  RD,    FSL,   Void)
INSTR (EGET,       0x6c000400,  0xfc1ffff0,  RD,    FSL,   Void)
INSTR (TEGET,      0x6c001400,  0xfc1ffff0,  RD,    FSL,   Void)
INSTR (NEGET,      0x6c004400,  0xfc1ffff0,  RD,    FSL,   Void)
INSTR (TNEGET,     0x6c005400,  0xfc1ffff0,  RD,    FSL,   Void)
INSTR (AGET,       0x6c000800,  0xfc1ffff0,  RD,    FSL,   Void)
INSTR (TAGET,      0x6c001800,  0xfc1ffff0,  RD,    FSL,   Void)
INSTR (NAGET,      0x6c004800,  0xfc1ffff0,  RD,    FSL,   Void)
INSTR (TNAGET,     0x6c005800,  0xfc1ffff0,  RD,    FSL,   Void)
INSTR (EAGET,      0x6c000c00,  0xfc1ffff0,  RD,    FSL,   Void)
INSTR (TEAGET,     0x6c001c00,  0xfc1ffff0,  RD,    FSL,   Void)
INSTR (NEAGET,     0x6c004c00,  0xfc1ffff0,  RD,    FSL,   Void)
INSTR (TNEAGET,    0x6c005c00,  0xfc1ffff0,  RD,    FSL,   Void)
INSTR (CGET,       0x6c002000,  0xfc1ffff0,  RD,    FSL,   Void)
INSTR (TCGET,      0x6c003000,  0xfc1ffff0,  RD,    FSL,   Void)
INSTR (NCGET,      0x6c006000,  0xfc1ffff0,  RD,    FSL,   Void)
INSTR (TNCGET,     0x6c007000,  0xfc1ffff0,  RD,    FSL,   Void)
INSTR (ECGET,      0x6c002400,  0xfc1ffff0,  RD,    FSL,   Void)
INSTR (TECGET,     0x6c003400,  0xfc1ffff0,  RD,    FSL,   Void)
INSTR (NECGET,     0x6c006400,  0xfc1ffff0,  RD,    FSL,   Void)
INSTR (TNECGET,    0x6c007400,  0xfc1ffff0,  RD,    FSL,   Void)
INSTR (CAGET,      0x6c002800,  0xfc1ffff0,  RD,    FSL,   Void)
INSTR (TCAGET,     0x6c003800,  0xfc1ffff0,  RD,    FSL,   Void)
INSTR (NCAGET,     0x6c006800,  0xfc1ffff0,  RD,    FSL,   Void)
INSTR (TNCAGET,    0x6c007800,  0xfc1ffff0,  RD,    FSL,   Void)
INSTR (ECAGET,     0x6c002c00,  0xfc1ffff0,  RD,    FSL,   Void)
INSTR (TECAGET,    0x6c003c00,  0xfc1ffff0,  RD,    FSL,   Void)
INSTR (NECAGET,    0x6c006c00,  0xfc1ffff0,  RD,    FSL,   Void)
INSTR (TNECAGET,   0x6c007c00,  0xfc1ffff0,  RD,    FSL,   Void)
INSTR (GETD,       0x4c000000,  0xfc1f07ff,  RD,    RB,    Void)
INSTR (TGETD,      0x4c000080,  0xfc1f07ff,  RD,    RB,    Void)
INSTR (NGETD,      0x4c000200,  0xfc1f07ff,  RD,    RB,    Void)
INSTR (TNGETD,     0x4c000280,  0xfc1f07ff,  RD,    RB,    Void)
INSTR (EGETD,      0x4c000020,  0xfc1f07ff,  RD,    RB,    Void)
INSTR (TEGETD,     0x4c0000a0,  0xfc1f07ff,  RD,    RB,    Void)
INSTR (NEGETD,     0x4c000220,  0xfc1f07ff,  RD,    RB,    Void)
INSTR (TNEGETD,    0x4c0002a0,  0xfc1f07ff,  RD,    RB,    Void)
INSTR (AGETD,      0x4c000040,  0xfc1f07ff,  RD,    RB,    Void)
INSTR (TAGETD,     0x4c0002c0,  0xfc1f07ff,  RD,    RB,    Void)
INSTR (NAGETD,     0x4c000040,  0xfc1f07ff,  RD,    RB,    Void)
INSTR (TNAGETD,    0x4c0002c0,  0xfc1f07ff,  RD,    RB,    Void)
INSTR (EAGETD,     0x4c000060,  0xfc1f07ff,  RD,    RB,    Void)
INSTR (TEAGETD,    0x4c0000e0,  0xfc1f07ff,  RD,    RB,    Void)
INSTR (NEAGETD,    0x4c000260,  0xfc1f07ff,  RD,    RB,    Void)
INSTR (TNEAGETD,   0x4c0002e0,  0xfc1f07ff,  RD,    RB,    Void)
INSTR (CGETD,      0x4c000100,  0xfc1f07ff,  RD,    RB,    Void)
INSTR (TCGETD,     0x4c000180,  0xfc1f07ff,  RD,    RB,    Void)
INSTR (NCGETD,     0x4c000300,  0xfc1f07ff,  RD,    RB,    Void)
INSTR (TNCGETD,    0x4c000380,  0xfc1f07ff,  RD,    RB,    Void)
INSTR (ECGETD,     0x4c000120,  0xfc1f07ff,  RD,    RB,    Void)
INSTR (TECGETD,    0x4c0001a0,  0xfc1f07ff,  RD,    RB,    Void)
INSTR (NECGETD,    0x4c000320,  0xfc1f07ff,  RD,    RB,    Void)
INSTR (TNECGETD,   0x4c0003a0,  0xfc1f07ff,  RD,    RB,    Void)
INSTR (CAGETD,     0x4c000140,  0xfc1f07ff,  RD,    RB,    Void)
INSTR (TCAGETD,    0x4c0001c0,  0xfc1f07ff,  RD,    RB,    Void)
INSTR (NCAGETD,    0x4c000340,  0xfc1f07ff,  RD,    RB,    Void)
INSTR (TNCAGETD,   0x4c0003c0,  0xfc1f07ff,  RD,    RB,    Void)
INSTR (ECAGETD,    0x4c000160,  0xfc1f07ff,  RD,    RB,    Void)
INSTR (TECAGETD,   0x4c0001e0,  0xfc1f07ff,  RD,    RB,    Void)
INSTR (NECAGETD,   0x4c000360,  0xfc1f07ff,  RD,    RB,    Void)
INSTR (TNECAGETD,  0x4c0003e0,  0xfc1f07ff,  RD,    RB,    Void)
INSTR (IDIV,       0x48000000,  0xfc0007ff,  RD,    RA,    RB)
INSTR (IDIVU,      0x48000010,  0xfc0007ff,  RD,    RA,    RB)
INSTR (IMM,        0xd0000000,  0xffff0000,  S16,   Void,  Void)
INSTR (LBU,        0xc0000000,  0xfc0007ff,  RD,    RA,    RB)
INSTR (LBUR,       0xc0000200,  0xfc0007ff,  RD,    RA,    RB)
INSTR (LBUI,       0xe0000000,  0xfc000000,  RD,    RA,    S16)
INSTR (LHU,        0xc4000000,  0xfc0007ff,  RD,    RA,    RB)
INSTR (LHUR,       0xc4000200,  0xfc0007ff,  RD,    RA,    RB)
INSTR (LHUI,       0xe4000000,  0xfc000000,  RD,    RA,    S16)
INSTR (LW,         0xc8000000,  0xfc0007ff,  RD,    RA,    RB)
INSTR (LWR,        0xc8000200,  0xfc0007ff,  RD,    RA,    RB)
INSTR (LWI,        0xe8000000,  0xfc000000,  RD,    RA,    S16)
INSTR (LWX,        0xc8000400,  0xfc0007ff,  RD,    RA,    RB)
INSTR (MBAR,       0xd8020004,  0xfc1fffff,  U521,  Void,  Void)
INSTR (MFS,        0x94008000,  0xfc1fc000,  RD,    RS,    Void)
INSTR (MSRCLR,     0x94110000,  0xfc1f8000,  RD,    U15,   Void)
INSTR (MSRSET,     0x94100000,  0xfc1f8000,  RD,    U15,   Void)
INSTR (MTS,        0x9400c000,  0xffe0c000,  RT,    RA,    Void)
INSTR (MUL,        0x40000000,  0xfc0007ff,  RD,    RA,    RB)
INSTR (MULH,       0x40000001,  0xfc0007ff,  RD,    RA,    RB)
INSTR (MULHU,      0x40000003,  0xfc0007ff,  RD,    RA,    RB)
INSTR (MULHSU,     0x40000002,  0xfc0007ff,  RD,    RA,    RB)
INSTR (MULI,       0x60000000,  0xfc000000,  RD,    RA,    S16)
INSTR (OR,         0x80000000,  0xfc0007ff,  RD,    RA,    RB)
INSTR (ORI,        0xa0000000,  0xfc000000,  RD,    RA,    S16)
INSTR (PCMPBF,     0x80000400,  0xfc0007ff,  RD,    RA,    RB)
INSTR (PCMPEQ,     0x88000400,  0xfc0007ff,  RD,    RA,    RB)
INSTR (PCMPNE,     0x8c000400,  0xfc0007ff,  RD,    RA,    RB)
INSTR (PUT,        0x6c008000,  0xffe0fff0,  RA,    FSL,   Void)
INSTR (NPUT,       0x6c00c000,  0xffe0fff0,  RA,    FSL,   Void)
INSTR (APUT,       0x6c008800,  0xffe0fff0,  RA,    FSL,   Void)
INSTR (NAPUT,      0x6c00c800,  0xffe0fff0,  RA,    FSL,   Void)
INSTR (TPUT,       0x6c009000,  0xffe0fff0,  FSL,   Void,  Void)
INSTR (TNPUT,      0x6c00d000,  0xffe0fff0,  FSL,   Void,  Void)
INSTR (TAPUT,      0x6c009800,  0xffe0fff0,  FSL,   Void,  Void)
INSTR (TNAPUT,     0x6c00d800,  0xffe0fff0,  FSL,   Void,  Void)
INSTR (CPUT,       0x6c00a000,  0xffe0fff0,  RA,    FSL,   Void)
INSTR (NCPUT,      0x6c00e000,  0xffe0fff0,  RA,    FSL,   Void)
INSTR (CAPUT,      0x6c00a800,  0xffe0fff0,  RA,    FSL,   Void)
INSTR (NCAPUT,     0x6c00e800,  0xffe0fff0,  RA,    FSL,   Void)
INSTR (TCPUT,      0x6c00b000,  0xffe0fff0,  FSL,   Void,  Void)
INSTR (TNCPUT,     0x6c00f000,  0xffe0fff0,  FSL,   Void,  Void)
INSTR (TCAPUT,     0x6c00b800,  0xffe0fff0,  FSL,   Void,  Void)
INSTR (TNCAPUT,    0x6c00f800,  0xffe0fff0,  FSL,   Void,  Void)
INSTR (PUTD,       0x4c000400,  0xffe007ff,  RA,    RB,    Void)
INSTR (NPUTD,      0x4c000600,  0xffe007ff,  RA,    RB,    Void)
INSTR (APUTD,      0x4c000440,  0xffe007ff,  RA,    RB,    Void)
INSTR (NAPUTD,     0x4c000640,  0xffe007ff,  RA,    RB,    Void)
INSTR (TPUTD,      0x4c000480,  0xffe007ff,  RB,    Void,  Void)
INSTR (TNPUTD,     0x4c000680,  0xffe007ff,  RB,    Void,  Void)
INSTR (TAPUTD,     0x4c0004c0,  0xffe007ff,  RB,    Void,  Void)
INSTR (TNAPUTD,    0x4c0006c0,  0xffe007ff,  RB,    Void,  Void)
INSTR (CPUTD,      0x4c000500,  0xffe007ff,  RA,    RB,    Void)
INSTR (NCPUTD,     0x4c000700,  0xffe007ff,  RA,    RB,    Void)
INSTR (CAPUTD,     0x4c000540,  0xffe007ff,  RA,    RB,    Void)
INSTR (NCAPUTD,    0x4c000740,  0xffe007ff,  RA,    RB,    Void)
INSTR (TCPUTD,     0x4c000580,  0xffe007ff,  RB,    Void,  Void)
INSTR (TNCPUTD,    0x4c000780,  0xffe007ff,  RB,    Void,  Void)
INSTR (TCAPUTD,    0x4c0005c0,  0xffe007ff,  RB,    Void,  Void)
INSTR (TNCAPUTD,   0x4c0007c0,  0xffe007ff,  RB,    Void,  Void)
INSTR (RSUB,       0x04000000,  0xfc0007ff,  RD,    RA,    RB)
INSTR (RSUBC,      0x0c000000,  0xfc0007ff,  RD,    RA,    RB)
INSTR (RSUBK,      0x14000000,  0xfc0007ff,  RD,    RA,    RB)
INSTR (RSUBKC,     0x1c000000,  0xfc0007ff,  RD,    RA,    RB)
INSTR (RSUBI,      0x24000000,  0xfc000000,  RD,    RA,    S16)
INSTR (RSUBIC,     0x2c000000,  0xfc000000,  RD,    RA,    S16)
INSTR (RSUBIK,     0x34000000,  0xfc000000,  RD,    RA,    S16)
INSTR (RSUBIKC,    0x3c000000,  0xfc000000,  RD,    RA,    S16)
INSTR (RTBD,       0xb6400000,  0xffe00000,  RA,    S16,   Void)
INSTR (RTID,       0xb6200000,  0xffe00000,  RA,    S16,   Void)
INSTR (RTED,       0xb6800000,  0xffe00000,  RA,    S16,   Void)
INSTR (RTSD,       0xb6000000,  0xffe00000,  RA,    S16,   Void)
INSTR (SB,         0xd0000000,  0xfc0007ff,  RD,    RA,    RB)
INSTR (SBR,        0xd0000200,  0xfc0007ff,  RD,    RA,    RB)
INSTR (SBI,        0xf0000000,  0xfc000000,  RD,    RA,    S16)
INSTR (SEXT8,      0x90000060,  0xfc00ffff,  RD,    RA,    Void)
INSTR (SEXT16,     0x90000061,  0xfc00ffff,  RD,    RA,    Void)
INSTR (SH,         0xd4000000,  0xfc0007ff,  RD,    RA,    RB)
INSTR (SHR,        0xd4000200,  0xfc0007ff,  RD,    RA,    RB)
INSTR (SHI,        0xf4000000,  0xfc000000,  RD,    RA,    S16)
INSTR (SRA,        0x90000001,  0xfc00ffff,  RD,    RA,    Void)
INSTR (SRC,        0x90000021,  0xfc00ffff,  RD,    RA,    Void)
INSTR (SRL,        0x90000041,  0xfc00ffff,  RD,    RA,    Void)
INSTR (SW,         0xd8000000,  0xfc0007ff,  RD,    RA,    RB)
INSTR (SWR,        0xd8000200,  0xfc0007ff,  RD,    RA,    RB)
INSTR (SWAPB,      0x900001e0,  0xfc00ffff,  RD,    RA,    Void)
INSTR (SWAPH,      0x900001e2,  0xfc00ffff,  RD,    RA,    Void)
INSTR (SWI,        0xf8000000,  0xfc000000,  RD,    RA,    S16)
INSTR (SWX,        0xd8000400,  0xfc0007ff,  RD,    RA,    RB)
INSTR (WDC,        0x90000064,  0xffe007ff,  RA,    RB,    Void)
INSTR (WDCFLUSH,   0x90000074,  0xffe007ff,  RA,    RB,    Void)
INSTR (WDCCLEAR,   0x90000066,  0xffe007ff,  RA,    RB,    Void)
INSTR (WIC,        0x90000068,  0xffe007ff,  RA,    RB,    Void)
INSTR (XOR,        0x88000000,  0xfc0007ff,  RD,    RA,    RB)
INSTR (XORI,       0xa8000000,  0xfc000000,  RD,    RA,    S16)

// operand types

TYPE (RA)
TYPE (RB)
TYPE (RD)
TYPE (RS)
TYPE (RT)
TYPE (U5)
TYPE (U521)
TYPE (U15)
TYPE (S16)
TYPE (O16)
TYPE (FSL)

// special registers

SREG (RPC,     rpc,     0x0000)
SREG (RMSR,    rmsr,    0x0001)
SREG (REAR,    rear,    0x0003)
SREG (RESR,    resr,    0x0005)
SREG (RFSR,    rfsr,    0x0007)
SREG (RBTR,    rbtr,    0x000b)
SREG (REDR,    redr,    0x000d)
SREG (RSLR,    rslr,    0x0800)
SREG (RSHR,    rshr,    0x0802)
SREG (RPID,    rpid,    0x1000)
SREG (RZPR,    rzpr,    0x1001)
SREG (RTLBX,   rtlbx,   0x1002)
SREG (RTLBLO,  rtlblo,  0x1003)
SREG (RTLBHI,  rtlbhi,  0x1004)
SREG (RPVR0,   rpvr0,   0x2000)
SREG (RPVR1,   rpvr1,   0x2001)
SREG (RPVR2,   rpvr2,   0x2002)
SREG (RPVR3,   rpvr3,   0x2003)
SREG (RPVR4,   rpvr4,   0x2004)
SREG (RPVR5,   rpvr5,   0x2005)
SREG (RPVR6,   rpvr6,   0x2006)
SREG (RPVR7,   rpvr7,   0x2007)
SREG (RPVR8,   rpvr8,   0x2008)
SREG (RPVR9,   rpvr9,   0x2009)
SREG (RPVR10,  rpvr10,  0x200a)
SREG (RPVR11,  rpvr11,  0x200b)

#undef INSTR
#undef MNEM
#undef SREG
#undef TYPE

% MIPS architecture documentation
% Copyright (C) Florian Negele

% This file is part of the Eigen Compiler Suite.

% Permission is granted to copy, distribute and/or modify this document
% under the terms of the GNU Free Documentation License, Version 1.3
% or any later version published by the Free Software Foundation.

% You should have received a copy of the GNU Free Documentation License
% along with the ECS.  If not, see <https://www.gnu.org/licenses/>.

% Generic documentation utilities
% Copyright (C) Florian Negele

% This file is part of the Eigen Compiler Suite.

% Permission is granted to copy, distribute and/or modify this document
% under the terms of the GNU Free Documentation License, Version 1.3
% or any later version published by the Free Software Foundation.

% You should have received a copy of the GNU Free Documentation License
% along with the ECS.  If not, see <https://www.gnu.org/licenses/>.

\providecommand{\cpp}{C\texttt{++}}
\providecommand{\opt}{_\mathit{opt}}
\providecommand{\tool}[1]{\texttt{#1}}
\providecommand{\version}{Version 0.0.40}
\providecommand{\resource}[1]{*++\txt{#1}}
\providecommand{\ecs}{Eigen Compiler Suite}
\providecommand{\changed}[1]{\underline{#1}}
\providecommand{\toolbox}[1]{\converter{#1}}
\providecommand{\file}{}\renewcommand{\file}[1]{\texttt{#1}}
\providecommand{\alignright}{\hfill\linebreak[0]\hspace*{\fill}}
\providecommand{\converter}[1]{*++[F][F*:white][F,:gray]\txt{#1}}
\providecommand{\documentation}{\ifbook chapter\else document\fi}
\providecommand{\Documentation}{\ifbook Chapter\else Document\fi}
\providecommand{\variable}[1]{\resource{\texttt{\small#1}\\variable}}
\providecommand{\documentationref}[2]{\ifbook\ref{#1}\else``\href{#1}{#2}''~\cite{#1}\fi}
\providecommand{\objfile}[1]{\texttt{#1}\index[runtime]{#1 object file@\texttt{#1} object file}}
\providecommand{\libfile}[1]{\texttt{#1}\index[runtime]{#1 library file@\texttt{#1} library file}}
\providecommand{\epigraph}[2]{\ifbook\begin{quote}\flushright\textit{#1}\par--- #2\end{quote}\fi}
\providecommand{\environmentvariable}[1]{\texttt{#1}\index{Environment variables!#1@\texttt{#1}}}
\providecommand{\environment}[1]{\texttt{#1}\index[environment]{#1 environment@\texttt{#1} environment}}
\providecommand{\toolsection}{}\renewcommand{\toolsection}[1]{\subsection{#1}\label{\prefix:#1}\tool{#1}}
\providecommand{\instruction}{}\renewcommand{\instruction}[2]{\noindent\qquad\pdftooltip{\texttt{#1}}{#2}\refstepcounter{instruction}\par}
\providecommand{\flowgraph}{}\renewcommand{\flowgraph}[1]{\par\sffamily\begin{displaymath}\xymatrix@=4ex{#1}\end{displaymath}\normalfont\par}
\providecommand{\instructionset}{}\renewcommand{\instructionset}[4]{\setcounter{instruction}{0}\begin{multicols}{\ifbook#3\else#4\fi}[{\captionof{table}[#2]{#2 (\ref*{#1:instructions}~instructions)}\label{tab:#1set}\vspace{-2ex}}]\footnotesize\raggedcolumns\input{#1.set}\label{#1:instructions}\end{multicols}}

\providecommand{\gpl}{GNU General Public License}
\providecommand{\rse}{ECS Runtime Support Exception}
\providecommand{\fdl}{\href{https://www.gnu.org/licenses/fdl.html}{GNU Free Documentation License}}

\providecommand{\docbegin}{}
\providecommand{\docend}{}
\providecommand{\doclabel}[1]{\hypertarget{#1}}
\providecommand{\doclink}[2]{\hyperlink{#1}{#2}}
\providecommand{\docsection}[3]{\hypertarget{#1}{\subsection{#2}}\label{sec:#1}\index[library]{#2@#3}}
\providecommand{\docsectionstar}[1]{}
\providecommand{\docsubbegin}{\begin{description}}
\providecommand{\docsubend}{\end{description}}
\providecommand{\docsubsection}[3]{\item[\hypertarget{#1}{#2}]\index[library]{#2@#3}}
\providecommand{\docsubsectionstar}[1]{\smallskip}
\providecommand{\docsubsubsection}[3]{\docsubsection{#1}{#2}{#3}}
\providecommand{\docsubsubsectionstar}[1]{}
\providecommand{\docsubsubsubsection}[3]{}
\providecommand{\docsubsubsubsectionstar}[1]{}
\providecommand{\doctable}{}

\providecommand{\debuggingtool}{}\renewcommand{\debuggingtool}{This tool is provided for debugging purposes.
It allows exposing and modifying an internal data structure that is usually not accessible.
}

\providecommand{\interface}{All tools accept command-line arguments which are taken as names of plain text files containing the source code.
If no arguments are provided, the standard input stream is used instead.
Output files are generated in the current working directory and have the same name as the input file being processed whereas the filename extension gets replaced by an appropriate suffix.
\seeinterface
}

\providecommand{\license}{\noindent Copyright \copyright{} Florian Negele\par\medskip\noindent
Permission is granted to copy, distribute and/or modify this document under the terms of the
\fdl{}, Version 1.3 or any later version published by the \href{https://fsf.org/}{Free Software Foundation}.
}

\providecommand{\ecslogosurface}{
\fill[darkgray] (0,0,0) -- (0,0,3) -- (0,3,3) -- (0,3,1) -- (0,4,1) -- (0,4,3) -- (0,5,3) -- (0,5,0) -- (0,2,0) -- (0,2,2) -- (0,1,2) -- (0,1,0) -- cycle;
\fill[gray] (0,5,0) -- (0,5,3) -- (1,5,3) -- (1,5,1) -- (2,5,1) -- (2,5,3) -- (3,5,3) -- (3,5,0) -- cycle;
\fill[lightgray] (0,0,0) -- (0,1,0) -- (2,1,0) -- (2,4,0) -- (1,4,0) -- (1,3,0) -- (2,3,0) -- (2,2,0) -- (0,2,0) -- (0,5,0) -- (3,5,0) -- (3,0,0) -- cycle;
\begin{scope}[line width=0.5]
\begin{scope}[gray]
\draw (0,0,0) -- (0,1,0);
\draw (2,1,0) -- (2,2,0);
\draw (0,1,2) -- (0,2,2);
\draw (0,2,0) -- (0,5,0);
\draw (2,3,0) -- (2,4,0);
\end{scope}
\begin{scope}[lightgray]
\draw (0,1,0) -- (0,1,2);
\draw (0,3,1) -- (0,3,3);
\draw (0,5,0) -- (0,5,3);
\draw (2,5,1) -- (2,5,3);
\end{scope}
\begin{scope}[white]
\draw (0,1,0) -- (2,1,0);
\draw (1,3,0) -- (2,3,0);
\draw (0,5,0) -- (3,5,0);
\end{scope}
\end{scope}
}

\providecommand{\ecslogo}[1]{
\begin{tikzpicture}[scale={(#1)/((sin(45)+cos(45))*3cm)},x={({-cos(45)*1cm},{sin(45)*sin(30)*1cm})},y={({0cm},{(cos(30)*1cm})},z={({sin(45)*1cm},{cos(45)*sin(30)*1cm})}]
\begin{scope}[darkgray,line width=1]
\draw (0,0,0) -- (0,0,3) -- (0,3,3) -- (2,3,3) -- (2,5,3) -- (3,5,3) -- (3,5,0) -- (3,0,0) -- cycle;
\draw (0,3,1) -- (0,4,1) -- (0,4,3) -- (0,5,3) -- (1,5,3) -- (1,5,1) -- (2,5,1);
\draw (1,3,0) -- (1,4,0) -- (2,4,0);
\end{scope}
\fill[darkgray] (2,0,0) -- (2,0,3) -- (2,5,3) -- (2,5,1) -- (2,4,1) -- (2,4,0) -- cycle;
\fill[lightgray] (2,0,2) -- (0,0,2) -- (0,2,2) -- (2,2,2) -- cycle;
\fill[gray] (0,1,0) -- (2,1,0) -- (2,1,2) -- (0,1,2) -- cycle;
\fill[gray] (0,3,1) -- (0,3,3) -- (2,3,3) -- (2,3,0) -- (1,3,0) -- (1,3,1) -- cycle;
\ecslogosurface
\end{tikzpicture}
}

\providecommand{\shadowedecslogo}[3]{
\begin{tikzpicture}[scale={(#1)/((sin(#2)+cos(#2))*3cm)},x={({-cos(#2)*1cm},{sin(#2)*sin(#3)*1cm})},y={({0cm},{(cos(#3)*1cm})},z={({sin(#2)*1cm},{cos(#2)*sin(#3)*1cm})}]
\shade[top color=lightgray!50!white,bottom color=white,middle color=lightgray!50!white] (0,0,0) -- (3,0,0) -- (3,{-0.5-3*sin(#2)*sin(#3)/cos(#3)},0) -- (0,-0.5,0) -- cycle;
\shade[top color=darkgray!50!gray,bottom color=white,middle color=darkgray!50!white] (0,0,0) -- (0,0,3) -- (0,{-0.5-3*cos(#2)*sin(#3)/cos(#3)},3) -- (0,-0.5,0) -- cycle;
\begin{scope}[y={({(cos(#2)+sin(#2))*0.5cm},{(cos(#2)*sin(#3)-sin(#2)*sin(#3))*0.5cm})}]
\useasboundingbox (3,0,0) -- (0,0,0) -- (0,0,3);
\shade[left color=darkgray!80!black,right color=lightgray,middle color=gray] (0,0,0) -- (0,1,0) -- (0,1,0.5) -- (0,2,0) -- (0,5,0) -- (0,5,3) -- (1,5,3) -- (1,4,3) -- (1,4,2.5) -- (1,3,3) -- (2,5,3) -- (3,5,3) -- (3,0,3) -- cycle;
\clip (0,0,0) -- (0,0,3) -- ({-3*sin(#2)/cos(#2)},0,0) -- cycle;
\shade[left color=darkgray,right color=lightgray!50!gray] (0,0,0) -- (0,1,0) -- (0,1,0.5) -- (0,2,0) -- (0,5,0) -- (0,5,3) -- (1,5,3) -- (1,4,3) -- (1,4,2.5) -- (1,3,3) -- (2,5,3) -- (3,5,3) -- (3,0,3) -- cycle;
\end{scope}
\shade[left color=darkgray,right color=darkgray!80!black] (2,0,0) -- (2,0,3) -- (2,5,3) -- (2,5,1) -- (2,4,1) -- (2,4,0) -- cycle;
\shade[left color=darkgray!90!black,right color=gray!80!darkgray] (2,0,2) -- (0,0,2) -- (0,2,2) -- (2,2,2) -- cycle;
\shade[top color=darkgray!90!black,bottom color=gray!80!darkgray] (0,1,0) -- (2,1,0) -- (2,1,2) -- (0,1,2) -- cycle;
\shade[top color=darkgray!90!black,bottom color=gray!80!darkgray] (0,3,1) -- (0,3,3) -- (2,3,3) -- (2,3,0) -- (1,3,0) -- (1,3,1) -- cycle;
\fill[gray] (2,1,0) -- (1.5,1,0.5) -- (0,1,0.5) -- (0,1,0) -- cycle;
\fill[gray] (1,3,2) -- (0.5,3,2) -- (0.5,3,3) -- (1,3,3) -- cycle;
\fill[gray] (2,3,0) -- (1.5,3,0.5) -- (1,3,0.5) -- (1,3,0) -- cycle;
\ecslogosurface
\end{tikzpicture}
}

\providecommand{\cpplogo}[1]{
\begin{tikzpicture}[scale=(#1)/512em]
\fill[gray] (435.2794,398.7159) -- (247.1911,507.3075) .. controls (236.3563,513.5642) and (218.6240,513.5642) .. (207.7892,507.3075) -- (19.7009,398.7159) .. controls (8.8646,392.4606) and (0.0000,377.1043) .. (0.0000,364.5924) -- (0.0000,147.4076) .. controls (0.8430,132.8363) and (8.2856,120.7683) .. (19.7009,113.2842) -- (207.7892,4.6926) .. controls (218.6240,-1.5642) and (236.3564,-1.5642) .. (247.1911,4.6926) -- (435.2794,113.2842) .. controls (447.5273,121.4304) and (454.4987,133.6918) .. (454.9803,147.4076) -- (454.9803,364.5924) .. controls (454.5404,377.7571) and (446.6566,391.0351) .. (435.2794,398.7159) -- cycle(75.8301,255.9993) .. controls (74.9389,404.0881) and (273.2892,469.4783) .. (358.8263,331.8769) -- (293.1917,293.8965) .. controls (253.5702,359.4301) and (155.1909,335.9977) .. (151.6601,255.9993) .. controls (152.7204,182.2703) and (249.4137,148.0211) .. (293.1961,218.1065) -- (358.8308,180.1276) .. controls (283.4477,49.2645) and (79.6318,96.3470) .. (75.8301,255.9993) -- cycle(379.1503,247.5747) -- (362.2982,247.5747) -- (362.2982,230.7226) -- (345.4490,230.7226) -- (345.4490,247.5747) -- (328.5969,247.5747) -- (328.5969,264.4254) -- (345.4490,264.4254) -- (345.4490,281.2759) -- (362.2982,281.2759) -- (362.2982,264.4254) -- (379.1503,264.4254) -- cycle(442.3420,247.5747) -- (425.4899,247.5747) -- (425.4899,230.7226) -- (408.6408,230.7226) -- (408.6408,247.5747) -- (391.7886,247.5747) -- (391.7886,264.4254) -- (408.6408,264.4254) -- (408.6408,281.2759) -- (425.4899,281.2759) -- (425.4899,264.4254) -- (442.3420,264.4254) -- cycle;
\end{tikzpicture}
}

\providecommand{\fallogo}[1]{
\begin{tikzpicture}[scale=(#1)/512em]
\fill[gray] (185.7774,0.0000) .. controls (200.4486,15.9798) and (226.8966,8.7148) .. (235.0426,31.5836) .. controls (249.5297,58.0598) and (247.9581,97.9161) .. (280.3335,110.9762) .. controls (309.1690,120.3496) and (337.8406,104.2727) .. (366.5753,103.9379) .. controls (373.4449,111.5171) and (379.2885,128.2574) .. (383.9755,108.9744) .. controls (396.6979,102.5615) and (437.2808,107.6681) .. (426.9652,124.3252) .. controls (408.9822,121.0785) and (412.4742,146.0729) .. (426.5192,131.4996) .. controls (433.8413,120.8489) and (465.1541,126.5522) .. (441.9067,135.7950) .. controls (396.1879,157.7478) and (344.1112,161.5079) .. (298.5528,183.5702) .. controls (277.7471,193.5198) and (284.6941,218.7163) .. (285.2127,236.9640) .. controls (292.3599,316.2826) and (307.3929,394.6311) .. (317.1198,473.6154) .. controls (329.0637,505.4736) and (292.1195,528.5004) .. (265.9183,511.2761) .. controls (237.9284,499.2462) and (237.3684,465.2681) .. (230.9102,439.9421) .. controls (218.6692,374.3397) and (215.6307,306.9662) .. (198.1732,242.3977) .. controls (183.1379,232.7444) and (164.4245,256.0298) .. (149.0430,261.4799) .. controls (116.9328,279.2585) and (87.1822,308.5851) .. (48.2293,307.8914) .. controls (21.3220,306.9037) and (-15.9107,281.8761) .. (7.2921,252.7908) .. controls (29.7799,220.6177) and (67.5177,204.3028) .. (100.9287,185.9449) .. controls (130.8217,170.8906) and (161.1548,156.5903) .. (191.0278,141.5847) .. controls (196.1738,120.0520) and (186.6049,95.2409) .. (186.8382,72.4353) .. controls (185.5234,48.4204) and (183.1700,23.9341) .. (185.7774,0.0000) -- cycle;
\end{tikzpicture}
}

\providecommand{\oblogo}[1]{
\begin{tikzpicture}[scale=(#1)/512em]
\fill[gray] (160.3865,208.9117) .. controls (154.0879,214.6478) and (149.0735,221.2409) .. (145.4125,228.5384) .. controls (184.8790,248.4273) and (234.7122,269.8787) .. (297.5493,291.8782) .. controls (300.3943,281.4769) and (300.9552,268.7619) .. (300.4023,255.2389) .. controls (248.9909,244.7891) and (200.0310,225.9279) .. (160.3865,208.9117) -- cycle(225.7398,392.6996) .. controls (308.0209,392.1716) and (359.3326,345.9277) .. (368.7203,285.2098) .. controls (376.6742,197.1784) and (311.7194,141.3342) .. (205.4287,142.1456) .. controls (139.9485,141.4804) and (88.7155,166.1957) .. (73.5775,228.0086) .. controls (52.0297,320.3408) and (123.4078,391.0103) .. (225.7398,392.6996) -- cycle(216.0739,176.4733) .. controls (268.9183,179.2424) and (315.8292,206.5488) .. (312.7454,265.1139) .. controls (313.2769,315.6384) and (286.5993,353.4946) .. (216.6040,355.7934) .. controls (162.4657,355.7934) and (126.0914,317.5023) .. (126.0914,260.5103) .. controls (126.1733,214.2900) and (163.3363,176.2849) .. (216.0739,176.4733) -- cycle(76.4897,189.1754) .. controls (13.1586,147.5631) and (0.0000,119.4207) .. (0.0000,119.4207) -- (90.6499,170.1632) .. controls (85.3004,175.8497) and (80.5994,182.1633) .. (76.4897,189.1754) -- cycle(353.9486,119.3004) -- (402.9482,119.3004) .. controls (427.0025,137.0797) and (450.9893,162.7034) .. (474.9529,191.0213) .. controls (509.3540,228.5339) and (531.3391,294.2091) .. (487.8149,312.1206) .. controls (462.8165,324.7652) and (394.3874,316.8943) .. (373.8912,313.6651) .. controls (379.9291,297.7449) and (383.2899,278.4204) .. (381.4989,257.7214) .. controls (420.3069,248.0321) and (421.9610,218.3461) .. (407.7867,192.6417) .. controls (391.1113,162.4018) and (370.1114,132.9097) .. (353.9486,119.3004) -- cycle;
\end{tikzpicture}
}

\providecommand{\markuptable}{
\begin{table}
\sffamily\centering
\begin{tabular}{@{}lcl@{}}
\toprule
\texttt{//italics//} & $\rightarrow$ & \textit{italics} \\
\midrule
\texttt{**bold**} & $\rightarrow$ & \textbf{bold} \\
\midrule
\texttt{\# ordered list} & & 1 ordered list \\
\texttt{\# second item} & $\rightarrow$ & 2 second item \\
\texttt{\#\# sub item} & & \hspace{1em} 1 sub item \\
\midrule
\texttt{* unordered list} & & $\bullet$ unordered list \\
\texttt{* second item} & $\rightarrow$ & $\bullet$ second item \\
\texttt{** sub item} & & \hspace{1em} $\bullet$ sub item \\
\midrule
\texttt{link to [[label]]} & $\rightarrow$ & link to \underline{label} \\
\midrule
\texttt{<{}<label>{}> definition } & $\rightarrow$ & definition \\
\midrule
\texttt{[[url|link name]]} & $\rightarrow$ & \underline{link name} \\
\midrule\addlinespace
\texttt{= large heading} & & {\Large large heading} \smallskip \\
\texttt{== medium heading} & $\rightarrow$ & {\large medium heading} \\
\texttt{=== small heading} & & small heading \\
\midrule
\texttt{no line break} & & no line break for paragraphs \\
\texttt{for paragraphs} & $\rightarrow$ \\
& & use empty line \\
\texttt{use empty line} \\
\midrule
\texttt{force\textbackslash\textbackslash line break} & $\rightarrow$ & force \\
& & line break \\
\midrule
\texttt{horizontal line} & $\rightarrow$ & horizontal line \\
\texttt{----} & & \hrulefill \\
\midrule
\texttt{|=a|=table|=header} & & \underline{a \enspace table \enspace header} \\
\texttt{|a|table|row} & $\rightarrow$ & a \enspace table \enspace row \\
\texttt{|b|table|row} & & b \enspace table \enspace row \\
\midrule
\texttt{\{\{\{} \\
\texttt{unformatted} & $\rightarrow$ & \texttt{unformatted} \\
\texttt{code} & & \texttt{code} \\
\texttt{\}\}\}} \\
\midrule\addlinespace
\texttt{@ new article} & & {\Large 1.\ new article} \smallskip \\
\texttt{@ second article} & $\rightarrow$ & {\Large 2.\ second article} \smallskip \\
\texttt{@@ sub article} & & {\large 2.1.\ sub article} \\
\bottomrule
\end{tabular}
\normalfont\caption{Elements of the generic documentation markup language}
\label{tab:docmarkup}
\end{table}
}

\providecommand{\startchapter}[4]{
\documentclass[11pt,a4paper]{article}
\usepackage{booktabs}
\usepackage[format=hang,labelfont=bf]{caption}
\usepackage{changepage}
\usepackage[T1]{fontenc}
\usepackage[margin=2cm]{geometry}
\usepackage{hyperref}
\usepackage[american]{isodate}
\usepackage{lmodern}
\usepackage{longtable}
\usepackage{mathptmx}
\usepackage{microtype}
\usepackage[toc]{multitoc}
\usepackage{multirow}
\usepackage[all]{nowidow}
\usepackage{pdfcomment}
\usepackage{syntax}
\usepackage{tikz}
\usepackage[all]{xy}
\hypersetup{pdfborder={0 0 0},bookmarksnumbered=true,pdftitle={\ecs{}: #2},pdfauthor={Florian Negele},pdfsubject={\ecs{}},pdfkeywords={#1}}
\setlength{\grammarindent}{8em}\setlength{\grammarparsep}{0.2ex}
\setlength{\columnsep}{2em}
\newcommand{\prefix}{}
\newcounter{instruction}
\bibliographystyle{unsrt}
\renewcommand{\index}[2][]{}
\renewcommand{\arraystretch}{1.05}
\renewcommand{\floatpagefraction}{0.7}
\renewcommand{\syntleft}{\itshape}\renewcommand{\syntright}{}
\title{\vspace{-5ex}\Huge{\ecs{}}\medskip\hrule}
\author{\huge{#2}}
\date{\medskip\version}
\newif\ifbook\bookfalse
\pagestyle{headings}
\frenchspacing
\begin{document}
\maketitle\thispagestyle{empty}\noindent#4\setlength{\columnseprule}{0.4pt}\tableofcontents\setlength{\columnseprule}{0pt}\vfill\pagebreak[3]\null\vfill\bigskip\noindent
\parbox{\textwidth-4em}{\license The contents of this \documentation{} are part of the \href{manual}{\ecs{} User Manual}~\cite{manual} and correspond to Chapter ``\href{manual\##3}{#1}''.\alignright\mbox{\today}}
\parbox{4em}{\flushright\ecslogo{3em}}
\clearpage
}

\providecommand{\concludechapter}{
\vfill\pagebreak[3]\null\vfill
\thispagestyle{myheadings}\markright{REFERENCES}
\noindent\begin{minipage}{\textwidth}\begin{multicols}{2}[\section*{References}]
\renewcommand{\section}[2]{}\small\bibliography{references}
\end{multicols}\end{minipage}\end{document}
}

\providecommand{\startpresentation}[2]{
\documentclass[14pt,aspectratio=43,usepdftitle=false]{beamer}
\usepackage{booktabs}
\usepackage{etex}
\usepackage{multicol}
\usepackage{tikz}
\usepackage[all]{xy}
\bibliographystyle{unsrt}
\setlength{\columnsep}{1em}
\setlength{\leftmargini}{1em}
\setbeamercolor{title}{fg=black}
\setbeamercolor{structure}{fg=darkgray}
\setbeamercolor{bibliography item}{fg=darkgray}
\setbeamerfont{title}{series=\bfseries}
\setbeamerfont{subtitle}{series=\normalfont}
\setbeamerfont*{frametitle}{parent=title}
\setbeamerfont{block title}{series=\bfseries}
\setbeamerfont*{framesubtitle}{parent=subtitle}
\setbeamersize{text margin left=1em,text margin right=1em}
\setbeamertemplate{navigation symbols}{}
\setbeamertemplate{itemize item}[circle]{}
\setbeamertemplate{bibliography item}[triangle]{}
\setbeamertemplate{bibliography entry author}{\usebeamercolor[fg]{bibliography item}}
\setbeamertemplate{frametitle}{\medskip\usebeamerfont{frametitle}\color{gray}\raisebox{-2.5ex}[0ex][0ex]{\rule{0.1em}{4.5ex}}}
\addtobeamertemplate{frametitle}{}{\hspace{0.4em}\usebeamercolor[fg]{title}\insertframetitle\par\vspace{0.2ex}\hspace{0.5em}\usebeamerfont{framesubtitle}\insertframesubtitle}
\hypersetup{pdfborder={0 0 0},bookmarksnumbered=true,bookmarksopen=true,bookmarksopenlevel=0,pdftitle={\ecs{}: #1},pdfauthor={Florian Negele},pdfsubject={\ecs{}},pdfkeywords={#1}}
\renewcommand{\flowgraph}[1]{\resizebox{\textwidth}{!}{$$\xymatrix{##1}$$}}
\title{\ecs{}\medskip\hrule\medskip}
\institute{\shadowedecslogo{5em}{30}{15}}
\date{\version}
\subtitle{#1}
\begin{document}
\begin{frame}[plain]\titlepage\nocite{manual}\end{frame}
\begin{frame}{Contents}{#1}\begin{center}\tableofcontents\end{center}\end{frame}
}

\providecommand{\concludepresentation}{
\begin{frame}{References}\begin{footnotesize}\setlength{\columnseprule}{0.4pt}\begin{multicols}{2}\bibliography{references}\end{multicols}\end{footnotesize}\end{frame}
\end{document}
}

\providecommand{\startbook}[1]{
\documentclass[10pt,paper=17cm:24cm,DIV=13,twoside=semi,headings=normal,numbers=noendperiod,cleardoublepage=plain]{scrbook}
\usepackage{atveryend}
\usepackage{booktabs}
\usepackage{caption}
\usepackage{changepage}
\usepackage[T1]{fontenc}
\usepackage{imakeidx}
\usepackage{hyperref}
\usepackage[american]{isodate}
\usepackage{lmodern}
\usepackage{longtable}
\usepackage{mathptmx}
\usepackage[final]{microtype}
\usepackage{multicol}
\usepackage{multirow}
\usepackage[all]{nowidow}
\usepackage{pdfcomment}
\usepackage{scrlayer-scrpage}
\usepackage{setspace}
\usepackage{syntax}
\usepackage[eventxtindent=4pt,oddtxtexdent=4pt]{thumbs}
\usepackage{tikz}
\usepackage[all]{xy}
\hyphenation{Micro-Blaze Open-Cores Open-RISC Power-PC}
\hypersetup{pdfborder={0 0 0},bookmarksnumbered=true,bookmarksopen=true,bookmarksopenlevel=0,pdftitle={\ecs{}: #1},pdfauthor={Florian Negele},pdfsubject={\ecs{}},pdfkeywords={#1}}
\setlength{\grammarindent}{8em}\setlength{\grammarparsep}{0.7ex}
\setkomafont{captionlabel}{\usekomafont{descriptionlabel}}
\renewcommand{\arraystretch}{1.05}\setstretch{1.1}
\renewcommand{\chapterformat}{\thechapter\autodot\enskip\raisebox{-1ex}[0ex][0ex]{\color{gray}\rule{0.1em}{3.5ex}}\enskip}
\renewcommand{\startchapter}[4]{\hypertarget{##3}{\chapter{##1}}\label{##3}##4\addthumb{##1}{\LARGE\sffamily\bfseries\thechapter}{white}{gray}\renewcommand{\prefix}{##3}}
\renewcommand{\concludechapter}{\clearpage{\stopthumb\cleardoublepage}}
\renewcommand{\syntleft}{\itshape}\renewcommand{\syntright}{}
\renewcommand{\floatpagefraction}{0.7}
\renewcommand{\partheademptypage}{}
\DeclareMicrotypeAlias{lmss}{cmr}
\newcommand{\prefix}{}
\newcounter{instruction}
\bibliographystyle{unsrt}
\newif\ifbook\booktrue
\makeindex[intoc,title=Index]
\makeindex[intoc,name=tools,title=Index of Tools,columns=3]
\makeindex[intoc,name=library,title=Index of Library Names]
\makeindex[intoc,name=runtime,title=Index of Runtime Support]
\makeindex[intoc,name=environment,title=Index of Target Environments]
\indexsetup{toclevel=chapter,headers={\indexname}{\indexname}}
\frenchspacing
\begin{document}
\pagenumbering{alph}
\begin{titlepage}\centering
\huge\sffamily\null\vfill\textbf{\ecs{}}\bigskip\hrule\bigskip#1
\normalsize\normalfont\vfill\vfill\shadowedecslogo{10em}{30}{15}
\large\vfill\vfill\version
\end{titlepage}
\null\vfill
\thispagestyle{empty}
\noindent\today\par\medskip
\license A copy of this license is included in Appendix~\ref{fdl} on page~\pageref{fdl}.
All product names used herein are for identification purposes only and may be trademarks of their respective companies.
\concludechapter
\frontmatter
\setcounter{tocdepth}{1}
\tableofcontents
\setcounter{tocdepth}{2}
\concludechapter
\listoffigures
\concludechapter
\listoftables
\concludechapter
}

\providecommand{\concludebook}{
\backmatter
\addtocontents{toc}{\protect\setcounter{tocdepth}{-1}}
\phantomsection\addcontentsline{toc}{part}{Bibliography}
\bibliography{references}
\concludechapter
\phantomsection\addcontentsline{toc}{part}{Indexes}
\printindex
\concludechapter
\indexprologue{\label{idx:tools}}
\printindex[tools]
\concludechapter
\printindex[library]
\concludechapter
\indexprologue{\label{idx:runtime}}
\printindex[runtime]
\concludechapter
\indexprologue{\label{idx:environment}}
\printindex[environment]
\concludechapter
\pagestyle{empty}\pagenumbering{Alph}\null\clearpage
\null\vfill\centering\ecslogo{4em}\par\medskip\license
\end{document}
}

% chapter references

\providecommand{\seedocumentationref}{}\renewcommand{\seedocumentationref}[3]{#1, see \Documentation{}~\documentationref{#2}{#3}. }
\providecommand{\seeinterface}{}\renewcommand{\seeinterface}{\ifbook See \Documentation{}~\documentationref{interface}{User Interface} for more information about the common user interface of all of these tools. \fi}
\providecommand{\seeguide}{}\renewcommand{\seeguide}{\seedocumentationref{For basic examples of using some of these tools in practice}{guide}{User Guide}}
\providecommand{\seecpp}{}\renewcommand{\seecpp}{\seedocumentationref{For more information about the \cpp{} programming language and its implementation by the \ecs{}}{cpp}{User Manual for \cpp{}}}
\providecommand{\seefalse}{}\renewcommand{\seefalse}{\seedocumentationref{For more information about the FALSE programming language and its implementation by the \ecs{}}{false}{User Manual for FALSE}}
\providecommand{\seeoberon}{}\renewcommand{\seeoberon}{\seedocumentationref{For more information about the Oberon programming language and its implementation by the \ecs{}}{oberon}{User Manual for Oberon}}
\providecommand{\seeassembly}{}\renewcommand{\seeassembly}{\seedocumentationref{For more information about the generic assembly language and how to use it}{assembly}{Generic Assembly Language Specification}}
\providecommand{\seeamd}{}\renewcommand{\seeamd}{\seedocumentationref{For more information about how the \ecs{} supports the AMD64 hardware architecture}{amd64}{AMD64 Hardware Architecture Support}}
\providecommand{\seearm}{}\renewcommand{\seearm}{\seedocumentationref{For more information about how the \ecs{} supports the ARM hardware architecture}{arm}{ARM Hardware Architecture Support}}
\providecommand{\seeavr}{}\renewcommand{\seeavr}{\seedocumentationref{For more information about how the \ecs{} supports the AVR hardware architecture}{avr}{AVR Hardware Architecture Support}}
\providecommand{\seeavrtt}{}\renewcommand{\seeavrtt}{\seedocumentationref{For more information about how the \ecs{} supports the AVR32 hardware architecture}{avr32}{AVR32 Hardware Architecture Support}}
\providecommand{\seemabk}{}\renewcommand{\seemabk}{\seedocumentationref{For more information about how the \ecs{} supports the M68000 hardware architecture}{m68k}{M68000 Hardware Architecture Support}}
\providecommand{\seemibl}{}\renewcommand{\seemibl}{\seedocumentationref{For more information about how the \ecs{} supports the MicroBlaze hardware architecture}{mibl}{MicroBlaze Hardware Architecture Support}}
\providecommand{\seemips}{}\renewcommand{\seemips}{\seedocumentationref{For more information about how the \ecs{} supports the MIPS32 and MIPS64 hardware architectures}{mips}{MIPS Hardware Architecture Support}}
\providecommand{\seemmix}{}\renewcommand{\seemmix}{\seedocumentationref{For more information about how the \ecs{} supports the MMIX hardware architecture}{mmix}{MMIX Hardware Architecture Support}}
\providecommand{\seeorok}{}\renewcommand{\seeorok}{\seedocumentationref{For more information about how the \ecs{} supports the OpenRISC 1000 hardware architecture}{or1k}{OpenRISC 1000 Hardware Architecture Support}}
\providecommand{\seeppc}{}\renewcommand{\seeppc}{\seedocumentationref{For more information about how the \ecs{} supports the PowerPC hardware architecture}{ppc}{PowerPC Hardware Architecture Support}}
\providecommand{\seerisc}{}\renewcommand{\seerisc}{\seedocumentationref{For more information about how the \ecs{} supports the RISC hardware architecture}{risc}{RISC Hardware Architecture Support}}
\providecommand{\seewasm}{}\renewcommand{\seewasm}{\seedocumentationref{For more information about how the \ecs{} supports the WebAssembly architecture}{wasm}{WebAssembly Architecture Support}}
\providecommand{\seedocumentation}{}\renewcommand{\seedocumentation}{\seedocumentationref{For more information about generic documentations and their generation by the \ecs{}}{documentation}{Generic Documentation Generation}}
\providecommand{\seedebugging}{}\renewcommand{\seedebugging}{\seedocumentationref{For more information about debugging information and its representation}{debugging}{Debugging Information Representation}}
\providecommand{\seecode}{}\renewcommand{\seecode}{\seedocumentationref{For more information about intermediate code and its purpose}{code}{Intermediate Code Representation}}
\providecommand{\seeobject}{}\renewcommand{\seeobject}{\seedocumentationref{For more information about object files and their purpose}{object}{Object File Representation}}

% generic documentation tools

\providecommand{\docprint}{
\toolsection{docprint} is a pretty printer for generic documentations.
It reformats generic documentations and writes it to the standard output stream.
\debuggingtool
\flowgraph{\resource{generic\\documentation} \ar[r] & \toolbox{docprint} \ar[r] & \resource{generic\\documentation}}
\seedocumentation
}

\providecommand{\doccheck}{
\toolsection{doccheck} is a syntactic and semantic checker for generic documentations.
It just performs syntactic and semantic checks on generic documentations and writes its diagnostic messages to the standard error stream.
\debuggingtool
\flowgraph{\resource{generic\\documentation} \ar[r] & \toolbox{doccheck} \ar[r] & \resource{diagnostic\\messages}}
\seedocumentation
}

\providecommand{\dochtml}{
\toolsection{dochtml} is an HTML documentation generator for generic documentations.
It processes several generic documentations and assembles all information therein into an HTML document.
\debuggingtool
\flowgraph{\resource{generic\\documentation} \ar[r] & \toolbox{dochtml} \ar[r] & \resource{HTML\\document}}
\seedocumentation
}

\providecommand{\doclatex}{
\toolsection{doclatex} is a Latex documentation generator for generic documentations.
It processes several generic documentations and assembles all information therein into a Latex document.
\debuggingtool
\flowgraph{\resource{generic\\documentation} \ar[r] & \toolbox{doclatex} \ar[r] & \resource{Latex\\document}}
\seedocumentation
}

% intermediate code tools

\providecommand{\cdcheck}{
\toolsection{cdcheck} is a syntactic and semantic checker for intermediate code.
It just performs syntactic and semantic checks on programs written in intermediate code and writes its diagnostic messages to the standard error stream.
\debuggingtool
\flowgraph{\resource{intermediate\\code} \ar[r] & \toolbox{cdcheck} \ar[r] & \resource{diagnostic\\messages}}
\seeassembly\seecode
}

\providecommand{\cdopt}{
\toolsection{cdopt} is an optimizer for intermediate code.
It performs various optimizations on programs written in intermediate code and writes the result to the standard output stream.
\debuggingtool
\flowgraph{\resource{intermediate\\code} \ar[r] & \toolbox{cdopt} \ar[r] & \resource{optimized\\code}}
\seeassembly\seecode
}

\providecommand{\cdrun}{
\toolsection{cdrun} is an interpreter for intermediate code.
It processes and executes programs written in intermediate code.
The following code sections are predefined and have the usual semantics:
\texttt{abort}, \texttt{\_Exit}, \texttt{fflush}, \texttt{floor}, \texttt{fputc}, \texttt{free}, \texttt{getchar}, \texttt{malloc}, and \texttt{putchar}.
Diagnostic messages about invalid operations include the name of the executed code section and the index of the erroneous instruction.
\debuggingtool
\flowgraph{\resource{intermediate\\code} \ar[r] & \toolbox{cdrun} \ar@/u/[r] & \resource{input/\\output} \ar@/d/[l]}
\seeassembly\seecode
}

\providecommand{\cdamda}{
\toolsection{cdamd16} is a compiler for intermediate code targeting the AMD64 hardware architecture.
It generates machine code for AMD64 processors from programs written in intermediate code and stores it in corresponding object files.
The compiler generates machine code for the 16-bit operating mode defined by the AMD64 architecture.
It also creates a debugging information file as well as an assembly file containing a listing of the generated machine code.
\debuggingtool
\flowgraph{\resource{intermediate\\code} \ar[r] & \toolbox{cdamd16} \ar[r] \ar[d] \ar[rd] & \resource{object file} \\ & \resource{assembly\\listing} & \resource{debugging\\information}}
\seeassembly\seeamd\seeobject\seecode\seedebugging
}

\providecommand{\cdamdb}{
\toolsection{cdamd32} is a compiler for intermediate code targeting the AMD64 hardware architecture.
It generates machine code for AMD64 processors from programs written in intermediate code and stores it in corresponding object files.
The compiler generates machine code for the 32-bit operating mode defined by the AMD64 architecture.
It also creates a debugging information file as well as an assembly file containing a listing of the generated machine code.
\debuggingtool
\flowgraph{\resource{intermediate\\code} \ar[r] & \toolbox{cdamd32} \ar[r] \ar[d] \ar[rd] & \resource{object file} \\ & \resource{assembly\\listing} & \resource{debugging\\information}}
\seeassembly\seeamd\seeobject\seecode\seedebugging
}

\providecommand{\cdamdc}{
\toolsection{cdamd64} is a compiler for intermediate code targeting the AMD64 hardware architecture.
It generates machine code for AMD64 processors from programs written in intermediate code and stores it in corresponding object files.
The compiler generates machine code for the 64-bit operating mode defined by the AMD64 architecture.
It also creates a debugging information file as well as an assembly file containing a listing of the generated machine code.
\debuggingtool
\flowgraph{\resource{intermediate\\code} \ar[r] & \toolbox{cdamd64} \ar[r] \ar[d] \ar[rd] & \resource{object file} \\ & \resource{assembly\\listing} & \resource{debugging\\information}}
\seeassembly\seeamd\seeobject\seecode\seedebugging
}

\providecommand{\cdarma}{
\toolsection{cdarma32} is a compiler for intermediate code targeting the ARM hardware architecture.
It generates machine code for ARM processors executing A32 instructions from programs written in intermediate code and stores it in corresponding object files.
It also creates a debugging information file as well as an assembly file containing a listing of the generated machine code.
\debuggingtool
\flowgraph{\resource{intermediate\\code} \ar[r] & \toolbox{cdarma32} \ar[r] \ar[d] \ar[rd] & \resource{object file} \\ & \resource{assembly\\listing} & \resource{debugging\\information}}
\seeassembly\seearm\seeobject\seecode\seedebugging
}

\providecommand{\cdarmb}{
\toolsection{cdarma64} is a compiler for intermediate code targeting the ARM hardware architecture.
It generates machine code for ARM processors executing A64 instructions from programs written in intermediate code and stores it in corresponding object files.
It also creates a debugging information file as well as an assembly file containing a listing of the generated machine code.
\debuggingtool
\flowgraph{\resource{intermediate\\code} \ar[r] & \toolbox{cdarma64} \ar[r] \ar[d] \ar[rd] & \resource{object file} \\ & \resource{assembly\\listing} & \resource{debugging\\information}}
\seeassembly\seearm\seeobject\seecode\seedebugging
}

\providecommand{\cdarmc}{
\toolsection{cdarmt32} is a compiler for intermediate code targeting the ARM hardware architecture.
It generates machine code for ARM processors without floating-point extension executing T32 instructions from programs written in intermediate code and stores it in corresponding object files.
It also creates a debugging information file as well as an assembly file containing a listing of the generated machine code.
\debuggingtool
\flowgraph{\resource{intermediate\\code} \ar[r] & \toolbox{cdarmt32} \ar[r] \ar[d] \ar[rd] & \resource{object file} \\ & \resource{assembly\\listing} & \resource{debugging\\information}}
\seeassembly\seearm\seeobject\seecode\seedebugging
}

\providecommand{\cdarmcfpe}{
\toolsection{cdarmt32fpe} is a compiler for intermediate code targeting the ARM hardware architecture.
It generates machine code for ARM processors with floating-point extension executing T32 instructions from programs written in intermediate code and stores it in corresponding object files.
It also creates a debugging information file as well as an assembly file containing a listing of the generated machine code.
\debuggingtool
\flowgraph{\resource{intermediate\\code} \ar[r] & \toolbox{cdarmt32fpe} \ar[r] \ar[d] \ar[rd] & \resource{object file} \\ & \resource{assembly\\listing} & \resource{debugging\\information}}
\seeassembly\seearm\seeobject\seecode\seedebugging
}

\providecommand{\cdavr}{
\toolsection{cdavr} is a compiler for intermediate code targeting the AVR hardware architecture.
It generates machine code for AVR processors from programs written in intermediate code and stores it in corresponding object files.
It also creates a debugging information file as well as an assembly file containing a listing of the generated machine code.
\debuggingtool
\flowgraph{\resource{intermediate\\code} \ar[r] & \toolbox{cdavr} \ar[r] \ar[d] \ar[rd] & \resource{object file} \\ & \resource{assembly\\listing} & \resource{debugging\\information}}
\seeassembly\seeavr\seeobject\seecode\seedebugging
}

\providecommand{\cdavrtt}{
\toolsection{cdavr32} is a compiler for intermediate code targeting the AVR32 hardware architecture.
It generates machine code for AVR32 processors from programs written in intermediate code and stores it in corresponding object files.
It also creates a debugging information file as well as an assembly file containing a listing of the generated machine code.
\debuggingtool
\flowgraph{\resource{intermediate\\code} \ar[r] & \toolbox{cdavr32} \ar[r] \ar[d] \ar[rd] & \resource{object file} \\ & \resource{assembly\\listing} & \resource{debugging\\information}}
\seeassembly\seeavrtt\seeobject\seecode\seedebugging
}

\providecommand{\cdmabk}{
\toolsection{cdm68k} is a compiler for intermediate code targeting the M68000 hardware architecture.
It generates machine code for M68000 processors from programs written in intermediate code and stores it in corresponding object files.
It also creates a debugging information file as well as an assembly file containing a listing of the generated machine code.
\debuggingtool
\flowgraph{\resource{intermediate\\code} \ar[r] & \toolbox{cdm68k} \ar[r] \ar[d] \ar[rd] & \resource{object file} \\ & \resource{assembly\\listing} & \resource{debugging\\information}}
\seeassembly\seemabk\seeobject\seecode\seedebugging
}

\providecommand{\cdmibl}{
\toolsection{cdmibl} is a compiler for intermediate code targeting the MicroBlaze hardware architecture.
It generates machine code for MicroBlaze processors from programs written in intermediate code and stores it in corresponding object files.
It also creates a debugging information file as well as an assembly file containing a listing of the generated machine code.
\debuggingtool
\flowgraph{\resource{intermediate\\code} \ar[r] & \toolbox{cdmibl} \ar[r] \ar[d] \ar[rd] & \resource{object file} \\ & \resource{assembly\\listing} & \resource{debugging\\information}}
\seeassembly\seemibl\seeobject\seecode\seedebugging
}

\providecommand{\cdmipsa}{
\toolsection{cdmips32} is a compiler for intermediate code targeting the MIPS32 hardware architecture.
It generates machine code for MIPS32 processors from programs written in intermediate code and stores it in corresponding object files.
It also creates a debugging information file as well as an assembly file containing a listing of the generated machine code.
\debuggingtool
\flowgraph{\resource{intermediate\\code} \ar[r] & \toolbox{cdmips32} \ar[r] \ar[d] \ar[rd] & \resource{object file} \\ & \resource{assembly\\listing} & \resource{debugging\\information}}
\seeassembly\seemips\seeobject\seecode\seedebugging
}

\providecommand{\cdmipsb}{
\toolsection{cdmips64} is a compiler for intermediate code targeting the MIPS64 hardware architecture.
It generates machine code for MIPS64 processors from programs written in intermediate code and stores it in corresponding object files.
It also creates a debugging information file as well as an assembly file containing a listing of the generated machine code.
\debuggingtool
\flowgraph{\resource{intermediate\\code} \ar[r] & \toolbox{cdmips64} \ar[r] \ar[d] \ar[rd] & \resource{object file} \\ & \resource{assembly\\listing} & \resource{debugging\\information}}
\seeassembly\seemips\seeobject\seecode\seedebugging
}

\providecommand{\cdmmix}{
\toolsection{cdmmix} is a compiler for intermediate code targeting the MMIX hardware architecture.
It generates machine code for MMIX processors from programs written in intermediate code and stores it in corresponding object files.
It also creates a debugging information file as well as an assembly file containing a listing of the generated machine code.
\debuggingtool
\flowgraph{\resource{intermediate\\code} \ar[r] & \toolbox{cdmmix} \ar[r] \ar[d] \ar[rd] & \resource{object file} \\ & \resource{assembly\\listing} & \resource{debugging\\information}}
\seeassembly\seemmix\seeobject\seecode\seedebugging
}

\providecommand{\cdorok}{
\toolsection{cdor1k} is a compiler for intermediate code targeting the OpenRISC 1000 hardware architecture.
It generates machine code for OpenRISC 1000 processors from programs written in intermediate code and stores it in corresponding object files.
It also creates a debugging information file as well as an assembly file containing a listing of the generated machine code.
\debuggingtool
\flowgraph{\resource{intermediate\\code} \ar[r] & \toolbox{cdor1k} \ar[r] \ar[d] \ar[rd] & \resource{object file} \\ & \resource{assembly\\listing} & \resource{debugging\\information}}
\seeassembly\seeorok\seeobject\seecode\seedebugging
}

\providecommand{\cdppca}{
\toolsection{cdppc32} is a compiler for intermediate code targeting the PowerPC hardware architecture.
It generates machine code for PowerPC processors from programs written in intermediate code and stores it in corresponding object files.
The compiler generates machine code for the 32-bit operating mode defined by the PowerPC architecture.
It also creates a debugging information file as well as an assembly file containing a listing of the generated machine code.
\debuggingtool
\flowgraph{\resource{intermediate\\code} \ar[r] & \toolbox{cdppc32} \ar[r] \ar[d] \ar[rd] & \resource{object file} \\ & \resource{assembly\\listing} & \resource{debugging\\information}}
\seeassembly\seeppc\seeobject\seecode\seedebugging
}

\providecommand{\cdppcb}{
\toolsection{cdppc64} is a compiler for intermediate code targeting the PowerPC hardware architecture.
It generates machine code for PowerPC processors from programs written in intermediate code and stores it in corresponding object files.
The compiler generates machine code for the 64-bit operating mode defined by the PowerPC architecture.
It also creates a debugging information file as well as an assembly file containing a listing of the generated machine code.
\debuggingtool
\flowgraph{\resource{intermediate\\code} \ar[r] & \toolbox{cdppc64} \ar[r] \ar[d] \ar[rd] & \resource{object file} \\ & \resource{assembly\\listing} & \resource{debugging\\information}}
\seeassembly\seeppc\seeobject\seecode\seedebugging
}

\providecommand{\cdrisc}{
\toolsection{cdrisc} is a compiler for intermediate code targeting the RISC hardware architecture.
It generates machine code for RISC processors from programs written in intermediate code and stores it in corresponding object files.
It also creates a debugging information file as well as an assembly file containing a listing of the generated machine code.
\debuggingtool
\flowgraph{\resource{intermediate\\code} \ar[r] & \toolbox{cdrisc} \ar[r] \ar[d] \ar[rd] & \resource{object file} \\ & \resource{assembly\\listing} & \resource{debugging\\information}}
\seeassembly\seerisc\seeobject\seecode\seedebugging
}

\providecommand{\cdwasm}{
\toolsection{cdwasm} is a compiler for intermediate code targeting the WebAssembly architecture.
It generates machine code for WebAssembly targets from programs written in intermediate code and stores it in corresponding object files.
It also creates a debugging information file as well as an assembly file containing a listing of the generated machine code.
\debuggingtool
\flowgraph{\resource{intermediate\\code} \ar[r] & \toolbox{cdwasm} \ar[r] \ar[d] \ar[rd] & \resource{object file} \\ & \resource{assembly\\listing} & \resource{debugging\\information}}
\seeassembly\seewasm\seeobject\seecode\seedebugging
}

% C++ tools

\providecommand{\cppprep}{
\toolsection{cppprep} is a preprocessor for the \cpp{} programming language.
It preprocesses source code according to the rules of \cpp{} and writes it to the standard output stream.
Only the macro names \texttt{\_\_DATE\_\_}, \texttt{\_\_FILE\_\_}, \texttt{\_\_LINE\_\_}, and \texttt{\_\_TIME\_\_} are predefined.
\flowgraph{\resource{\cpp{} or other\\source code} \ar[r] & \toolbox{cppprep} \ar[r] & \resource{preprocessed\\source code} \\ & \variable{ECSINCLUDE} \ar[u]}
\seecpp
}

\providecommand{\cppprint}{
\toolsection{cppprint} is a pretty printer for the \cpp{} programming language.
It reformats the source code of \cpp{} programs and writes it to the standard output stream.
\flowgraph{\resource{\cpp{}\\source code} \ar[r] & \toolbox{cppprint} \ar[r] & \resource{reformatted\\source code} \\ & \variable{ECSINCLUDE} \ar[u]}
\seecpp
}

\providecommand{\cppcheck}{
\toolsection{cppcheck} is a syntactic and semantic checker for the \cpp{} programming language.
It just performs syntactic and semantic checks on \cpp{} programs and writes its diagnostic messages to the standard error stream.
\flowgraph{\resource{\cpp{}\\source code} \ar[r] & \toolbox{cppcheck} \ar[r] & \resource{diagnostic\\messages} \\ & \variable{ECSINCLUDE} \ar[u]}
\seecpp
}

\providecommand{\cppdump}{
\toolsection{cppdump} is a serializer for the \cpp{} programming language.
It dumps the complete internal representation of programs written in \cpp{} into an XML document.
\debuggingtool
\flowgraph{\resource{\cpp{}\\source code} \ar[r] & \toolbox{cppdump} \ar[r] & \resource{internal\\representation} \\ & \variable{ECSINCLUDE} \ar[u]}
\seecpp
}

\providecommand{\cpprun}{
\toolsection{cpprun} is an interpreter for the \cpp{} programming language.
It processes and executes programs written in \cpp{}.
The macro \texttt{\_\_run\_\_} is predefined in order to enable programmers to identify this tool while interpreting.
\flowgraph{\resource{\cpp{}\\source code} \ar[r] & \toolbox{cpprun} \ar@/u/[r] & \resource{input/\\output} \ar@/d/[l] \\ & \variable{ECSINCLUDE} \ar[u]}
\seecpp
}

\providecommand{\cppdoc}{
\toolsection{cppdoc} is a generic documentation generator for the \cpp{} programming language.
It processes several \cpp{} source files and assembles all information therein into a generic documentation.
\debuggingtool
\flowgraph{\resource{\cpp{}\\source code} \ar[r] & \toolbox{cppdoc} \ar[r] & \resource{generic\\documentation} \\ & \variable{ECSINCLUDE} \ar[u]}
\seecpp\seedocumentation
}

\providecommand{\cpphtml}{
\toolsection{cpphtml} is an HTML documentation generator for the \cpp{} programming language.
It processes several \cpp{} source files and assembles all information therein into an HTML document.
\flowgraph{\resource{\cpp{}\\source code} \ar[r] & \toolbox{cpphtml} \ar[r] & \resource{HTML\\document} \\ & \variable{ECSINCLUDE} \ar[u]}
\seecpp\seedocumentation
}

\providecommand{\cpplatex}{
\toolsection{cpplatex} is a Latex documentation generator for the \cpp{} programming language.
It processes several \cpp{} source files and assembles all information therein into a Latex document.
\flowgraph{\resource{\cpp{}\\source code} \ar[r] & \toolbox{cpplatex} \ar[r] & \resource{Latex\\document} \\ & \variable{ECSINCLUDE} \ar[u]}
\seecpp\seedocumentation
}

\providecommand{\cppcode}{
\toolsection{cppcode} is an intermediate code generator for the \cpp{} programming language.
It generates intermediate code from programs written in \cpp{} and stores it in corresponding assembly files.
The macro \texttt{\_\_code\_\_} is predefined in order to enable programmers to identify this tool while generating intermediate code.
Programs generated with this tool require additional runtime support that is stored in the \file{cpp\-code\-run} library file.
\debuggingtool
\flowgraph{\resource{\cpp{}\\source code} \ar[r] & \toolbox{cppcode} \ar[r] & \resource{intermediate\\code} \\ & \variable{ECSINCLUDE} \ar[u]}
\seecpp\seeassembly\seecode
}

\providecommand{\cppamda}{
\toolsection{cppamd16} is a compiler for the \cpp{} programming language targeting the AMD64 hardware architecture.
It generates machine code for AMD64 processors from programs written in \cpp{} and stores it in corresponding object files.
The compiler generates machine code for the 16-bit operating mode defined by the AMD64 architecture.
For debugging purposes, it also creates a debugging information file as well as an assembly file containing a listing of the generated machine code.
The macro \texttt{\_\_amd16\_\_} is predefined in order to enable programmers to identify this tool and its target architecture while compiling.
Programs generated with this compiler require additional runtime support that is stored in the \file{cpp\-amd16\-run} library file.
\flowgraph{\resource{\cpp{}\\source code} \ar[r] & \toolbox{cppamd16} \ar[r] \ar[d] \ar[rd] & \resource{object file} \\ \variable{ECSINCLUDE} \ar[ru] & \resource{debugging\\information} & \resource{assembly\\listing}}
\seecpp\seeassembly\seeamd\seeobject\seedebugging
}

\providecommand{\cppamdb}{
\toolsection{cppamd32} is a compiler for the \cpp{} programming language targeting the AMD64 hardware architecture.
It generates machine code for AMD64 processors from programs written in \cpp{} and stores it in corresponding object files.
The compiler generates machine code for the 32-bit operating mode defined by the AMD64 architecture.
For debugging purposes, it also creates a debugging information file as well as an assembly file containing a listing of the generated machine code.
The macro \texttt{\_\_amd32\_\_} is predefined in order to enable programmers to identify this tool and its target architecture while compiling.
Programs generated with this compiler require additional runtime support that is stored in the \file{cpp\-amd32\-run} library file.
\flowgraph{\resource{\cpp{}\\source code} \ar[r] & \toolbox{cppamd32} \ar[r] \ar[d] \ar[rd] & \resource{object file} \\ \variable{ECSINCLUDE} \ar[ru] & \resource{debugging\\information} & \resource{assembly\\listing}}
\seecpp\seeassembly\seeamd\seeobject\seedebugging
}

\providecommand{\cppamdc}{
\toolsection{cppamd64} is a compiler for the \cpp{} programming language targeting the AMD64 hardware architecture.
It generates machine code for AMD64 processors from programs written in \cpp{} and stores it in corresponding object files.
The compiler generates machine code for the 64-bit operating mode defined by the AMD64 architecture.
For debugging purposes, it also creates a debugging information file as well as an assembly file containing a listing of the generated machine code.
The macro \texttt{\_\_amd64\_\_} is predefined in order to enable programmers to identify this tool and its target architecture while compiling.
Programs generated with this compiler require additional runtime support that is stored in the \file{cpp\-amd64\-run} library file.
\flowgraph{\resource{\cpp{}\\source code} \ar[r] & \toolbox{cppamd64} \ar[r] \ar[d] \ar[rd] & \resource{object file} \\ \variable{ECSINCLUDE} \ar[ru] & \resource{debugging\\information} & \resource{assembly\\listing}}
\seecpp\seeassembly\seeamd\seeobject\seedebugging
}

\providecommand{\cpparma}{
\toolsection{cpparma32} is a compiler for the \cpp{} programming language targeting the ARM hardware architecture.
It generates machine code for ARM processors executing A32 instructions from programs written in \cpp{} and stores it in corresponding object files.
For debugging purposes, it also creates a debugging information file as well as an assembly file containing a listing of the generated machine code.
The macro \texttt{\_\_arma32\_\_} is predefined in order to enable programmers to identify this tool and its target architecture while compiling.
Programs generated with this compiler require additional runtime support that is stored in the \file{cpp\-arma32\-run} library file.
\flowgraph{\resource{\cpp{}\\source code} \ar[r] & \toolbox{cpparma32} \ar[r] \ar[d] \ar[rd] & \resource{object file} \\ \variable{ECSINCLUDE} \ar[ru] & \resource{debugging\\information} & \resource{assembly\\listing}}
\seecpp\seeassembly\seearm\seeobject\seedebugging
}

\providecommand{\cpparmb}{
\toolsection{cpparma64} is a compiler for the \cpp{} programming language targeting the ARM hardware architecture.
It generates machine code for ARM processors executing A64 instructions from programs written in \cpp{} and stores it in corresponding object files.
For debugging purposes, it also creates a debugging information file as well as an assembly file containing a listing of the generated machine code.
The macro \texttt{\_\_arma64\_\_} is predefined in order to enable programmers to identify this tool and its target architecture while compiling.
Programs generated with this compiler require additional runtime support that is stored in the \file{cpp\-arma64\-run} library file.
\flowgraph{\resource{\cpp{}\\source code} \ar[r] & \toolbox{cpparma64} \ar[r] \ar[d] \ar[rd] & \resource{object file} \\ \variable{ECSINCLUDE} \ar[ru] & \resource{debugging\\information} & \resource{assembly\\listing}}
\seecpp\seeassembly\seearm\seeobject\seedebugging
}

\providecommand{\cpparmc}{
\toolsection{cpparmt32} is a compiler for the \cpp{} programming language targeting the ARM hardware architecture.
It generates machine code for ARM processors without floating-point extension executing T32 instructions from programs written in \cpp{} and stores it in corresponding object files.
For debugging purposes, it also creates a debugging information file as well as an assembly file containing a listing of the generated machine code.
The macro \texttt{\_\_armt32\_\_} is predefined in order to enable programmers to identify this tool and its target architecture while compiling.
Programs generated with this compiler require additional runtime support that is stored in the \file{cpp\-armt32\-run} library file.
\flowgraph{\resource{\cpp{}\\source code} \ar[r] & \toolbox{cpparmt32} \ar[r] \ar[d] \ar[rd] & \resource{object file} \\ \variable{ECSINCLUDE} \ar[ru] & \resource{debugging\\information} & \resource{assembly\\listing}}
\seecpp\seeassembly\seearm\seeobject\seedebugging
}

\providecommand{\cpparmcfpe}{
\toolsection{cpparmt32fpe} is a compiler for the \cpp{} programming language targeting the ARM hardware architecture.
It generates machine code for ARM processors with floating-point extension executing T32 instructions from programs written in \cpp{} and stores it in corresponding object files.
For debugging purposes, it also creates a debugging information file as well as an assembly file containing a listing of the generated machine code.
The macro \texttt{\_\_armt32fpe\_\_} is predefined in order to enable programmers to identify this tool and its target architecture while compiling.
Programs generated with this compiler require additional runtime support that is stored in the \file{cpp\-armt32\-fpe\-run} library file.
\flowgraph{\resource{\cpp{}\\source code} \ar[r] & \toolbox{cpparmt32fpe} \ar[r] \ar[d] \ar[rd] & \resource{object file} \\ \variable{ECSINCLUDE} \ar[ru] & \resource{debugging\\information} & \resource{assembly\\listing}}
\seecpp\seeassembly\seearm\seeobject\seedebugging
}

\providecommand{\cppavr}{
\toolsection{cppavr} is a compiler for the \cpp{} programming language targeting the AVR hardware architecture.
It generates machine code for AVR processors from programs written in \cpp{} and stores it in corresponding object files.
For debugging purposes, it also creates a debugging information file as well as an assembly file containing a listing of the generated machine code.
The macro \texttt{\_\_avr\_\_} is predefined in order to enable programmers to identify this tool and its target architecture while compiling.
Programs generated with this compiler require additional runtime support that is stored in the \file{cpp\-avr\-run} library file.
\flowgraph{\resource{\cpp{}\\source code} \ar[r] & \toolbox{cppavr} \ar[r] \ar[d] \ar[rd] & \resource{object file} \\ \variable{ECSINCLUDE} \ar[ru] & \resource{debugging\\information} & \resource{assembly\\listing}}
\seecpp\seeassembly\seeavr\seeobject\seedebugging
}

\providecommand{\cppavrtt}{
\toolsection{cppavr32} is a compiler for the \cpp{} programming language targeting the AVR32 hardware architecture.
It generates machine code for AVR32 processors from programs written in \cpp{} and stores it in corresponding object files.
For debugging purposes, it also creates a debugging information file as well as an assembly file containing a listing of the generated machine code.
The macro \texttt{\_\_avr32\_\_} is predefined in order to enable programmers to identify this tool and its target architecture while compiling.
Programs generated with this compiler require additional runtime support that is stored in the \file{cpp\-avr32\-run} library file.
\flowgraph{\resource{\cpp{}\\source code} \ar[r] & \toolbox{cppavr32} \ar[r] \ar[d] \ar[rd] & \resource{object file} \\ \variable{ECSINCLUDE} \ar[ru] & \resource{debugging\\information} & \resource{assembly\\listing}}
\seecpp\seeassembly\seeavrtt\seeobject\seedebugging
}

\providecommand{\cppmabk}{
\toolsection{cppm68k} is a compiler for the \cpp{} programming language targeting the M68000 hardware architecture.
It generates machine code for M68000 processors from programs written in \cpp{} and stores it in corresponding object files.
For debugging purposes, it also creates a debugging information file as well as an assembly file containing a listing of the generated machine code.
The macro \texttt{\_\_m68k\_\_} is predefined in order to enable programmers to identify this tool and its target architecture while compiling.
Programs generated with this compiler require additional runtime support that is stored in the \file{cpp\-m68k\-run} library file.
\flowgraph{\resource{\cpp{}\\source code} \ar[r] & \toolbox{cppm68k} \ar[r] \ar[d] \ar[rd] & \resource{object file} \\ \variable{ECSINCLUDE} \ar[ru] & \resource{debugging\\information} & \resource{assembly\\listing}}
\seecpp\seeassembly\seemabk\seeobject\seedebugging
}

\providecommand{\cppmibl}{
\toolsection{cppmibl} is a compiler for the \cpp{} programming language targeting the MicroBlaze hardware architecture.
It generates machine code for MicroBlaze processors from programs written in \cpp{} and stores it in corresponding object files.
For debugging purposes, it also creates a debugging information file as well as an assembly file containing a listing of the generated machine code.
The macro \texttt{\_\_mibl\_\_} is predefined in order to enable programmers to identify this tool and its target architecture while compiling.
Programs generated with this compiler require additional runtime support that is stored in the \file{cpp\-mibl\-run} library file.
\flowgraph{\resource{\cpp{}\\source code} \ar[r] & \toolbox{cppmibl} \ar[r] \ar[d] \ar[rd] & \resource{object file} \\ \variable{ECSINCLUDE} \ar[ru] & \resource{debugging\\information} & \resource{assembly\\listing}}
\seecpp\seeassembly\seemibl\seeobject\seedebugging
}

\providecommand{\cppmipsa}{
\toolsection{cppmips32} is a compiler for the \cpp{} programming language targeting the MIPS32 hardware architecture.
It generates machine code for MIPS32 processors from programs written in \cpp{} and stores it in corresponding object files.
For debugging purposes, it also creates a debugging information file as well as an assembly file containing a listing of the generated machine code.
The macro \texttt{\_\_mips32\_\_} is predefined in order to enable programmers to identify this tool and its target architecture while compiling.
Programs generated with this compiler require additional runtime support that is stored in the \file{cpp\-mips32\-run} library file.
\flowgraph{\resource{\cpp{}\\source code} \ar[r] & \toolbox{cppmips32} \ar[r] \ar[d] \ar[rd] & \resource{object file} \\ \variable{ECSINCLUDE} \ar[ru] & \resource{debugging\\information} & \resource{assembly\\listing}}
\seecpp\seeassembly\seemips\seeobject\seedebugging
}

\providecommand{\cppmipsb}{
\toolsection{cppmips64} is a compiler for the \cpp{} programming language targeting the MIPS64 hardware architecture.
It generates machine code for MIPS64 processors from programs written in \cpp{} and stores it in corresponding object files.
For debugging purposes, it also creates a debugging information file as well as an assembly file containing a listing of the generated machine code.
The macro \texttt{\_\_mips64\_\_} is predefined in order to enable programmers to identify this tool and its target architecture while compiling.
Programs generated with this compiler require additional runtime support that is stored in the \file{cpp\-mips64\-run} library file.
\flowgraph{\resource{\cpp{}\\source code} \ar[r] & \toolbox{cppmips64} \ar[r] \ar[d] \ar[rd] & \resource{object file} \\ \variable{ECSINCLUDE} \ar[ru] & \resource{debugging\\information} & \resource{assembly\\listing}}
\seecpp\seeassembly\seemips\seeobject\seedebugging
}

\providecommand{\cppmmix}{
\toolsection{cppmmix} is a compiler for the \cpp{} programming language targeting the MMIX hardware architecture.
It generates machine code for MMIX processors from programs written in \cpp{} and stores it in corresponding object files.
For debugging purposes, it also creates a debugging information file as well as an assembly file containing a listing of the generated machine code.
The macro \texttt{\_\_mmix\_\_} is predefined in order to enable programmers to identify this tool and its target architecture while compiling.
Programs generated with this compiler require additional runtime support that is stored in the \file{cpp\-mmix\-run} library file.
\flowgraph{\resource{\cpp{}\\source code} \ar[r] & \toolbox{cppmmix} \ar[r] \ar[d] \ar[rd] & \resource{object file} \\ \variable{ECSINCLUDE} \ar[ru] & \resource{debugging\\information} & \resource{assembly\\listing}}
\seecpp\seeassembly\seemmix\seeobject\seedebugging
}

\providecommand{\cpporok}{
\toolsection{cppor1k} is a compiler for the \cpp{} programming language targeting the OpenRISC 1000 hardware architecture.
It generates machine code for OpenRISC 1000 processors from programs written in \cpp{} and stores it in corresponding object files.
For debugging purposes, it also creates a debugging information file as well as an assembly file containing a listing of the generated machine code.
The macro \texttt{\_\_or1k\_\_} is predefined in order to enable programmers to identify this tool and its target architecture while compiling.
Programs generated with this compiler require additional runtime support that is stored in the \file{cpp\-or1k\-run} library file.
\flowgraph{\resource{\cpp{}\\source code} \ar[r] & \toolbox{cppor1k} \ar[r] \ar[d] \ar[rd] & \resource{object file} \\ \variable{ECSINCLUDE} \ar[ru] & \resource{debugging\\information} & \resource{assembly\\listing}}
\seecpp\seeassembly\seeorok\seeobject\seedebugging
}

\providecommand{\cppppca}{
\toolsection{cppppc32} is a compiler for the \cpp{} programming language targeting the PowerPC hardware architecture.
It generates machine code for PowerPC processors from programs written in \cpp{} and stores it in corresponding object files.
The compiler generates machine code for the 32-bit operating mode defined by the PowerPC architecture.
For debugging purposes, it also creates a debugging information file as well as an assembly file containing a listing of the generated machine code.
The macro \texttt{\_\_ppc32\_\_} is predefined in order to enable programmers to identify this tool and its target architecture while compiling.
Programs generated with this compiler require additional runtime support that is stored in the \file{cpp\-ppc32\-run} library file.
\flowgraph{\resource{\cpp{}\\source code} \ar[r] & \toolbox{cppppc32} \ar[r] \ar[d] \ar[rd] & \resource{object file} \\ \variable{ECSINCLUDE} \ar[ru] & \resource{debugging\\information} & \resource{assembly\\listing}}
\seecpp\seeassembly\seeppc\seeobject\seedebugging
}

\providecommand{\cppppcb}{
\toolsection{cppppc64} is a compiler for the \cpp{} programming language targeting the PowerPC hardware architecture.
It generates machine code for PowerPC processors from programs written in \cpp{} and stores it in corresponding object files.
The compiler generates machine code for the 64-bit operating mode defined by the PowerPC architecture.
For debugging purposes, it also creates a debugging information file as well as an assembly file containing a listing of the generated machine code.
The macro \texttt{\_\_ppc64\_\_} is predefined in order to enable programmers to identify this tool and its target architecture while compiling.
Programs generated with this compiler require additional runtime support that is stored in the \file{cpp\-ppc64\-run} library file.
\flowgraph{\resource{\cpp{}\\source code} \ar[r] & \toolbox{cppppc64} \ar[r] \ar[d] \ar[rd] & \resource{object file} \\ \variable{ECSINCLUDE} \ar[ru] & \resource{debugging\\information} & \resource{assembly\\listing}}
\seecpp\seeassembly\seeppc\seeobject\seedebugging
}

\providecommand{\cpprisc}{
\toolsection{cpprisc} is a compiler for the \cpp{} programming language targeting the RISC hardware architecture.
It generates machine code for RISC processors from programs written in \cpp{} and stores it in corresponding object files.
For debugging purposes, it also creates a debugging information file as well as an assembly file containing a listing of the generated machine code.
The macro \texttt{\_\_risc\_\_} is predefined in order to enable programmers to identify this tool and its target architecture while compiling.
Programs generated with this compiler require additional runtime support that is stored in the \file{cpp\-risc\-run} library file.
\flowgraph{\resource{\cpp{}\\source code} \ar[r] & \toolbox{cpprisc} \ar[r] \ar[d] \ar[rd] & \resource{object file} \\ \variable{ECSINCLUDE} \ar[ru] & \resource{debugging\\information} & \resource{assembly\\listing}}
\seecpp\seeassembly\seerisc\seeobject\seedebugging
}

\providecommand{\cppwasm}{
\toolsection{cppwasm} is a compiler for the \cpp{} programming language targeting the WebAssembly architecture.
It generates machine code for WebAssembly targets from programs written in \cpp{} and stores it in corresponding object files.
For debugging purposes, it also creates a debugging information file as well as an assembly file containing a listing of the generated machine code.
The macro \texttt{\_\_wasm\_\_} is predefined in order to enable programmers to identify this tool and its target architecture while compiling.
Programs generated with this compiler require additional runtime support that is stored in the \file{cpp\-wasm\-run} library file.
\flowgraph{\resource{\cpp{}\\source code} \ar[r] & \toolbox{cppwasm} \ar[r] \ar[d] \ar[rd] & \resource{object file} \\ \variable{ECSINCLUDE} \ar[ru] & \resource{debugging\\information} & \resource{assembly\\listing}}
\seecpp\seeassembly\seewasm\seeobject\seedebugging
}

% FALSE tools

\providecommand{\falprint}{
\toolsection{falprint} is a pretty printer for the FALSE programming language.
It reformats the source code of FALSE programs and writes it to the standard output stream.
\flowgraph{\resource{FALSE\\source code} \ar[r] & \toolbox{falprint} \ar[r] & \resource{reformatted\\source code}}
\seefalse
}

\providecommand{\falcheck}{
\toolsection{falcheck} is a syntactic and semantic checker for the FALSE programming language.
It just performs syntactic and semantic checks on FALSE programs and writes its diagnostic messages to the standard error stream.
\flowgraph{\resource{FALSE\\source code} \ar[r] & \toolbox{falcheck} \ar[r] & \resource{diagnostic\\messages}}
\seefalse
}

\providecommand{\faldump}{
\toolsection{faldump} is a serializer for the FALSE programming language.
It dumps the complete internal representation of programs written in FALSE into an XML document.
\debuggingtool
\flowgraph{\resource{FALSE\\source code} \ar[r] & \toolbox{faldump} \ar[r] & \resource{internal\\representation}}
\seefalse
}

\providecommand{\falrun}{
\toolsection{falrun} is an interpreter for the FALSE programming language.
It processes and executes programs written in FALSE\@.
\flowgraph{\resource{FALSE\\source code} \ar[r] & \toolbox{falrun} \ar@/u/[r] & \resource{input/\\output} \ar@/d/[l]}
\seefalse
}

\providecommand{\falcpp}{
\toolsection{falcpp} is a transpiler for the FALSE programming language.
It translates programs written in FALSE into \cpp{} programs and stores them in corresponding source files.
\flowgraph{\resource{FALSE\\source code} \ar[r] & \toolbox{falcpp} \ar[r] & \resource{\cpp{}\\source file}}
\seefalse\seecpp
}

\providecommand{\falcode}{
\toolsection{falcode} is an intermediate code generator for the FALSE programming language.
It generates intermediate code from programs written in FALSE and stores it in corresponding assembly files.
\debuggingtool
\flowgraph{\resource{FALSE\\source code} \ar[r] & \toolbox{falcode} \ar[r] & \resource{intermediate\\code}}
\seefalse\seeassembly\seecode
}

\providecommand{\falamda}{
\toolsection{falamd16} is a compiler for the FALSE programming language targeting the AMD64 hardware architecture.
It generates machine code for AMD64 processors from programs written in FALSE and stores it in corresponding object files.
The compiler generates machine code for the 16-bit operating mode defined by the AMD64 architecture.
\flowgraph{\resource{FALSE\\source code} \ar[r] & \toolbox{falamd16} \ar[r] & \resource{object file}}
\seefalse\seeamd\seeobject
}

\providecommand{\falamdb}{
\toolsection{falamd32} is a compiler for the FALSE programming language targeting the AMD64 hardware architecture.
It generates machine code for AMD64 processors from programs written in FALSE and stores it in corresponding object files.
The compiler generates machine code for the 32-bit operating mode defined by the AMD64 architecture.
\flowgraph{\resource{FALSE\\source code} \ar[r] & \toolbox{falamd32} \ar[r] & \resource{object file}}
\seefalse\seeamd\seeobject
}

\providecommand{\falamdc}{
\toolsection{falamd64} is a compiler for the FALSE programming language targeting the AMD64 hardware architecture.
It generates machine code for AMD64 processors from programs written in FALSE and stores it in corresponding object files.
The compiler generates machine code for the 64-bit operating mode defined by the AMD64 architecture.
\flowgraph{\resource{FALSE\\source code} \ar[r] & \toolbox{falamd64} \ar[r] & \resource{object file}}
\seefalse\seeamd\seeobject
}

\providecommand{\falarma}{
\toolsection{falarma32} is a compiler for the FALSE programming language targeting the ARM hardware architecture.
It generates machine code for ARM processors executing A32 instructions from programs written in FALSE and stores it in corresponding object files.
\flowgraph{\resource{FALSE\\source code} \ar[r] & \toolbox{falarma32} \ar[r] & \resource{object file}}
\seefalse\seearm\seeobject
}

\providecommand{\falarmb}{
\toolsection{falarma64} is a compiler for the FALSE programming language targeting the ARM hardware architecture.
It generates machine code for ARM processors executing A64 instructions from programs written in FALSE and stores it in corresponding object files.
\flowgraph{\resource{FALSE\\source code} \ar[r] & \toolbox{falarma64} \ar[r] & \resource{object file}}
\seefalse\seearm\seeobject
}

\providecommand{\falarmc}{
\toolsection{falarmt32} is a compiler for the FALSE programming language targeting the ARM hardware architecture.
It generates machine code for ARM processors without floating-point extension executing T32 instructions from programs written in FALSE and stores it in corresponding object files.
\flowgraph{\resource{FALSE\\source code} \ar[r] & \toolbox{falarmt32} \ar[r] & \resource{object file}}
\seefalse\seearm\seeobject
}

\providecommand{\falarmcfpe}{
\toolsection{falarmt32fpe} is a compiler for the FALSE programming language targeting the ARM hardware architecture.
It generates machine code for ARM processors with floating-point extension executing T32 instructions from programs written in FALSE and stores it in corresponding object files.
\flowgraph{\resource{FALSE\\source code} \ar[r] & \toolbox{falarmt32fpe} \ar[r] & \resource{object file}}
\seefalse\seearm\seeobject
}

\providecommand{\falavr}{
\toolsection{falavr} is a compiler for the FALSE programming language targeting the AVR hardware architecture.
It generates machine code for AVR processors from programs written in FALSE and stores it in corresponding object files.
\flowgraph{\resource{FALSE\\source code} \ar[r] & \toolbox{falavr} \ar[r] & \resource{object file}}
\seefalse\seeavr\seeobject
}

\providecommand{\falavrtt}{
\toolsection{falavr32} is a compiler for the FALSE programming language targeting the AVR32 hardware architecture.
It generates machine code for AVR32 processors from programs written in FALSE and stores it in corresponding object files.
\flowgraph{\resource{FALSE\\source code} \ar[r] & \toolbox{falavr32} \ar[r] & \resource{object file}}
\seefalse\seeavrtt\seeobject
}

\providecommand{\falmabk}{
\toolsection{falm68k} is a compiler for the FALSE programming language targeting the M68000 hardware architecture.
It generates machine code for M68000 processors from programs written in FALSE and stores it in corresponding object files.
\flowgraph{\resource{FALSE\\source code} \ar[r] & \toolbox{falm68k} \ar[r] & \resource{object file}}
\seefalse\seemabk\seeobject
}

\providecommand{\falmibl}{
\toolsection{falmibl} is a compiler for the FALSE programming language targeting the MicroBlaze hardware architecture.
It generates machine code for MicroBlaze processors from programs written in FALSE and stores it in corresponding object files.
\flowgraph{\resource{FALSE\\source code} \ar[r] & \toolbox{falmibl} \ar[r] & \resource{object file}}
\seefalse\seemibl\seeobject
}

\providecommand{\falmipsa}{
\toolsection{falmips32} is a compiler for the FALSE programming language targeting the MIPS32 hardware architecture.
It generates machine code for MIPS32 processors from programs written in FALSE and stores it in corresponding object files.
\flowgraph{\resource{FALSE\\source code} \ar[r] & \toolbox{falmips32} \ar[r] & \resource{object file}}
\seefalse\seemips\seeobject
}

\providecommand{\falmipsb}{
\toolsection{falmips64} is a compiler for the FALSE programming language targeting the MIPS64 hardware architecture.
It generates machine code for MIPS64 processors from programs written in FALSE and stores it in corresponding object files.
\flowgraph{\resource{FALSE\\source code} \ar[r] & \toolbox{falmips64} \ar[r] & \resource{object file}}
\seefalse\seemips\seeobject
}

\providecommand{\falmmix}{
\toolsection{falmmix} is a compiler for the FALSE programming language targeting the MMIX hardware architecture.
It generates machine code for MMIX processors from programs written in FALSE and stores it in corresponding object files.
\flowgraph{\resource{FALSE\\source code} \ar[r] & \toolbox{falmmix} \ar[r] & \resource{object file}}
\seefalse\seemmix\seeobject
}

\providecommand{\falorok}{
\toolsection{falor1k} is a compiler for the FALSE programming language targeting the OpenRISC 1000 hardware architecture.
It generates machine code for OpenRISC 1000 processors from programs written in FALSE and stores it in corresponding object files.
\flowgraph{\resource{FALSE\\source code} \ar[r] & \toolbox{falor1k} \ar[r] & \resource{object file}}
\seefalse\seeorok\seeobject
}

\providecommand{\falppca}{
\toolsection{falppc32} is a compiler for the FALSE programming language targeting the PowerPC hardware architecture.
It generates machine code for PowerPC processors from programs written in FALSE and stores it in corresponding object files.
The compiler generates machine code for the 32-bit operating mode defined by the PowerPC architecture.
\flowgraph{\resource{FALSE\\source code} \ar[r] & \toolbox{falppc32} \ar[r] & \resource{object file}}
\seefalse\seeppc\seeobject
}

\providecommand{\falppcb}{
\toolsection{falppc64} is a compiler for the FALSE programming language targeting the PowerPC hardware architecture.
It generates machine code for PowerPC processors from programs written in FALSE and stores it in corresponding object files.
The compiler generates machine code for the 64-bit operating mode defined by the PowerPC architecture.
\flowgraph{\resource{FALSE\\source code} \ar[r] & \toolbox{falppc64} \ar[r] & \resource{object file}}
\seefalse\seeppc\seeobject
}

\providecommand{\falrisc}{
\toolsection{falrisc} is a compiler for the FALSE programming language targeting the RISC hardware architecture.
It generates machine code for RISC processors from programs written in FALSE and stores it in corresponding object files.
\flowgraph{\resource{FALSE\\source code} \ar[r] & \toolbox{falrisc} \ar[r] & \resource{object file}}
\seefalse\seerisc\seeobject
}

\providecommand{\falwasm}{
\toolsection{falwasm} is a compiler for the FALSE programming language targeting the WebAssembly architecture.
It generates machine code for WebAssembly targets from programs written in FALSE and stores it in corresponding object files.
\flowgraph{\resource{FALSE\\source code} \ar[r] & \toolbox{falwasm} \ar[r] & \resource{object file}}
\seefalse\seewasm\seeobject
}

% Oberon tools

\providecommand{\obprint}{
\toolsection{obprint} is a pretty printer for the Oberon programming language.
It reformats the source code of Oberon modules and writes it to the standard output stream.
\flowgraph{\resource{Oberon\\source code} \ar[r] & \toolbox{obprint} \ar[r] & \resource{reformatted\\source code}}
\seeoberon
}

\providecommand{\obcheck}{
\toolsection{obcheck} is a syntactic and semantic checker for the Oberon programming language.
It just performs syntactic and semantic checks on Oberon modules and writes its diagnostic messages to the standard error stream.
In addition, it stores the interface of each module in a symbol file which is required when other modules import the module.
\flowgraph{\resource{Oberon\\source code} \ar[r] & \toolbox{obcheck} \ar[r] \ar@/l/[d] & \resource{diagnostic\\messages} \\ \variable{ECSIMPORT} \ar[ru] & \resource{symbol\\files} \ar@/r/[u]}
\seeoberon
}

\providecommand{\obdump}{
\toolsection{obdump} is a serializer for the Oberon programming language.
It dumps the complete internal representation of modules written in Oberon into an XML document.
\debuggingtool
\flowgraph{\resource{Oberon\\source code} \ar[r] & \toolbox{obdump} \ar[r] \ar@/l/[d] & \resource{internal\\representation} \\ \variable{ECSIMPORT} \ar[ru] & \resource{symbol\\files} \ar@/r/[u]}
\seeoberon
}

\providecommand{\obrun}{
\toolsection{obrun} is an interpreter for the Oberon programming language.
It processes and executes modules written in Oberon.
This tool does neither generate nor process symbol files while interpreting modules.
If a module is imported by another one, its filename has to be named before the other one in the list of command-line arguments.
\flowgraph{\resource{Oberon\\source code} \ar[r] & \toolbox{obrun} \ar@/u/[r] & \resource{input/\\output} \ar@/d/[l]}
\seeoberon
}

\providecommand{\obcpp}{
\toolsection{obcpp} is a transpiler for the Oberon programming language.
It translates programs written in Oberon into \cpp{} programs and stores them in corresponding source and header files.
In addition, it stores the interface of each module in a symbol file which is required when other modules import the module.
The same interface is provided by the generated header file which can be used in other parts of the \cpp{} program.
\flowgraph{\resource{Oberon\\source code} \ar[r] & \toolbox{obcpp} \ar[r] \ar@/l/[d] \ar[rd] & \resource{\cpp{}\\source file} \\ \variable{ECSIMPORT} \ar[ru] & \resource{symbol\\files} \ar@/r/[u] & \resource{\cpp{}\\header file}}
\seeoberon\seecpp
}

\providecommand{\obdoc}{
\toolsection{obdoc} is a generic documentation generator for the Oberon programming language.
It processes several Oberon modules and assembles all information therein into a generic documentation.
In addition, it stores the interface of each module in a symbol file which is required when other modules import the module.
\debuggingtool
\flowgraph{\resource{Oberon\\source code} \ar[r] & \toolbox{obdoc} \ar[r] \ar@/l/[d] & \resource{generic\\documentation} \\ \variable{ECSIMPORT} \ar[ru] & \resource{symbol\\files} \ar@/r/[u]}
\seeoberon\seedocumentation
}

\providecommand{\obhtml}{
\toolsection{obhtml} is an HTML documentation generator for the Oberon programming language.
It processes several Oberon modules and assembles all information therein into an HTML document.
In addition, it stores the interface of each module in a symbol file which is required when other modules import the module.
\flowgraph{\resource{Oberon\\source code} \ar[r] & \toolbox{obhtml} \ar[r] \ar@/l/[d] & \resource{HTML\\document} \\ \variable{ECSIMPORT} \ar[ru] & \resource{symbol\\files} \ar@/r/[u]}
\seeoberon\seedocumentation
}

\providecommand{\oblatex}{
\toolsection{oblatex} is a Latex documentation generator for the Oberon programming language.
It processes several Oberon modules and assembles all information therein into a Latex document.
In addition, it stores the interface of each module in a symbol file which is required when other modules import the module.
\flowgraph{\resource{Oberon\\source code} \ar[r] & \toolbox{oblatex} \ar[r] \ar@/l/[d] & \resource{Latex\\document} \\ \variable{ECSIMPORT} \ar[ru] & \resource{symbol\\files} \ar@/r/[u]}
\seeoberon\seedocumentation
}

\providecommand{\obcode}{
\toolsection{obcode} is an intermediate code generator for the Oberon programming language.
It generates intermediate code from modules written in Oberon and stores it in corresponding assembly files.
In addition, it stores the interface of each module in a symbol file which is required when other modules import the module.
Programs generated with this tool require additional runtime support that is stored in the \file{ob\-code\-run} library file.
\debuggingtool
\flowgraph{\resource{Oberon\\source code} \ar[r] & \toolbox{obcode} \ar[r] \ar@/l/[d] & \resource{intermediate\\code} \\ \variable{ECSIMPORT} \ar[ru] & \resource{symbol\\files} \ar@/r/[u]}
\seeoberon\seeassembly\seecode
}

\providecommand{\obamda}{
\toolsection{obamd16} is a compiler for the Oberon programming language targeting the AMD64 hardware architecture.
It generates machine code for AMD64 processors from modules written in Oberon and stores it in corresponding object files.
The compiler generates machine code for the 16-bit operating mode defined by the AMD64 architecture.
For debugging purposes, it also creates a debugging information file as well as an assembly file containing a listing of the generated machine code.
In addition, it stores the interface of each module in a symbol file which is required when other modules import the module.
Programs generated with this compiler require additional runtime support that is stored in the \file{ob\-amd16\-run} library file.
\flowgraph{\resource{Oberon\\source code} \ar[r] & \toolbox{obamd16} \ar[r] \ar@/l/[d] \ar[rd] & \resource{object file} \\ \variable{ECSIMPORT} \ar[ru] & \resource{symbol\\files} \ar@/r/[u] & \resource{debugging\\information}}
\seeoberon\seeassembly\seeamd\seeobject\seedebugging
}

\providecommand{\obamdb}{
\toolsection{obamd32} is a compiler for the Oberon programming language targeting the AMD64 hardware architecture.
It generates machine code for AMD64 processors from modules written in Oberon and stores it in corresponding object files.
The compiler generates machine code for the 32-bit operating mode defined by the AMD64 architecture.
For debugging purposes, it also creates a debugging information file as well as an assembly file containing a listing of the generated machine code.
In addition, it stores the interface of each module in a symbol file which is required when other modules import the module.
Programs generated with this compiler require additional runtime support that is stored in the \file{ob\-amd32\-run} library file.
\flowgraph{\resource{Oberon\\source code} \ar[r] & \toolbox{obamd32} \ar[r] \ar@/l/[d] \ar[rd] & \resource{object file} \\ \variable{ECSIMPORT} \ar[ru] & \resource{symbol\\files} \ar@/r/[u] & \resource{debugging\\information}}
\seeoberon\seeassembly\seeamd\seeobject\seedebugging
}

\providecommand{\obamdc}{
\toolsection{obamd64} is a compiler for the Oberon programming language targeting the AMD64 hardware architecture.
It generates machine code for AMD64 processors from modules written in Oberon and stores it in corresponding object files.
The compiler generates machine code for the 64-bit operating mode defined by the AMD64 architecture.
For debugging purposes, it also creates a debugging information file as well as an assembly file containing a listing of the generated machine code.
In addition, it stores the interface of each module in a symbol file which is required when other modules import the module.
Programs generated with this compiler require additional runtime support that is stored in the \file{ob\-amd64\-run} library file.
\flowgraph{\resource{Oberon\\source code} \ar[r] & \toolbox{obamd64} \ar[r] \ar@/l/[d] \ar[rd] & \resource{object file} \\ \variable{ECSIMPORT} \ar[ru] & \resource{symbol\\files} \ar@/r/[u] & \resource{debugging\\information}}
\seeoberon\seeassembly\seeamd\seeobject\seedebugging
}

\providecommand{\obarma}{
\toolsection{obarma32} is a compiler for the Oberon programming language targeting the ARM hardware architecture.
It generates machine code for ARM processors executing A32 instructions from modules written in Oberon and stores it in corresponding object files.
For debugging purposes, it also creates a debugging information file as well as an assembly file containing a listing of the generated machine code.
In addition, it stores the interface of each module in a symbol file which is required when other modules import the module.
Programs generated with this compiler require additional runtime support that is stored in the \file{ob\-arma32\-run} library file.
\flowgraph{\resource{Oberon\\source code} \ar[r] & \toolbox{obarma32} \ar[r] \ar@/l/[d] \ar[rd] & \resource{object file} \\ \variable{ECSIMPORT} \ar[ru] & \resource{symbol\\files} \ar@/r/[u] & \resource{debugging\\information}}
\seeoberon\seeassembly\seearm\seeobject\seedebugging
}

\providecommand{\obarmb}{
\toolsection{obarma64} is a compiler for the Oberon programming language targeting the ARM hardware architecture.
It generates machine code for ARM processors executing A64 instructions from modules written in Oberon and stores it in corresponding object files.
For debugging purposes, it also creates a debugging information file as well as an assembly file containing a listing of the generated machine code.
In addition, it stores the interface of each module in a symbol file which is required when other modules import the module.
Programs generated with this compiler require additional runtime support that is stored in the \file{ob\-arma64\-run} library file.
\flowgraph{\resource{Oberon\\source code} \ar[r] & \toolbox{obarma64} \ar[r] \ar@/l/[d] \ar[rd] & \resource{object file} \\ \variable{ECSIMPORT} \ar[ru] & \resource{symbol\\files} \ar@/r/[u] & \resource{debugging\\information}}
\seeoberon\seeassembly\seearm\seeobject\seedebugging
}

\providecommand{\obarmc}{
\toolsection{obarmt32} is a compiler for the Oberon programming language targeting the ARM hardware architecture.
It generates machine code for ARM processors without floating-point extension executing T32 instructions from modules written in Oberon and stores it in corresponding object files.
For debugging purposes, it also creates a debugging information file as well as an assembly file containing a listing of the generated machine code.
In addition, it stores the interface of each module in a symbol file which is required when other modules import the module.
Programs generated with this compiler require additional runtime support that is stored in the \file{ob\-armt32\-run} library file.
\flowgraph{\resource{Oberon\\source code} \ar[r] & \toolbox{obarmt32} \ar[r] \ar@/l/[d] \ar[rd] & \resource{object file} \\ \variable{ECSIMPORT} \ar[ru] & \resource{symbol\\files} \ar@/r/[u] & \resource{debugging\\information}}
\seeoberon\seeassembly\seearm\seeobject\seedebugging
}

\providecommand{\obarmcfpe}{
\toolsection{obarmt32fpe} is a compiler for the Oberon programming language targeting the ARM hardware architecture.
It generates machine code for ARM processors with floating-point extension executing T32 instructions from modules written in Oberon and stores it in corresponding object files.
For debugging purposes, it also creates a debugging information file as well as an assembly file containing a listing of the generated machine code.
In addition, it stores the interface of each module in a symbol file which is required when other modules import the module.
Programs generated with this compiler require additional runtime support that is stored in the \file{ob\-armt32\-fpe\-run} library file.
\flowgraph{\resource{Oberon\\source code} \ar[r] & \toolbox{obarmt32fpe} \ar[r] \ar@/l/[d] \ar[rd] & \resource{object file} \\ \variable{ECSIMPORT} \ar[ru] & \resource{symbol\\files} \ar@/r/[u] & \resource{debugging\\information}}
\seeoberon\seeassembly\seearm\seeobject\seedebugging
}

\providecommand{\obavr}{
\toolsection{obavr} is a compiler for the Oberon programming language targeting the AVR hardware architecture.
It generates machine code for AVR processors from modules written in Oberon and stores it in corresponding object files.
For debugging purposes, it also creates a debugging information file as well as an assembly file containing a listing of the generated machine code.
In addition, it stores the interface of each module in a symbol file which is required when other modules import the module.
Programs generated with this compiler require additional runtime support that is stored in the \file{ob\-avr\-run} library file.
\flowgraph{\resource{Oberon\\source code} \ar[r] & \toolbox{obavr} \ar[r] \ar@/l/[d] \ar[rd] & \resource{object file} \\ \variable{ECSIMPORT} \ar[ru] & \resource{symbol\\files} \ar@/r/[u] & \resource{debugging\\information}}
\seeoberon\seeassembly\seeavr\seeobject\seedebugging
}

\providecommand{\obavrtt}{
\toolsection{obavr32} is a compiler for the Oberon programming language targeting the AVR32 hardware architecture.
It generates machine code for AVR32 processors from modules written in Oberon and stores it in corresponding object files.
For debugging purposes, it also creates a debugging information file as well as an assembly file containing a listing of the generated machine code.
In addition, it stores the interface of each module in a symbol file which is required when other modules import the module.
Programs generated with this compiler require additional runtime support that is stored in the \file{ob\-avr32\-run} library file.
\flowgraph{\resource{Oberon\\source code} \ar[r] & \toolbox{obavr32} \ar[r] \ar@/l/[d] \ar[rd] & \resource{object file} \\ \variable{ECSIMPORT} \ar[ru] & \resource{symbol\\files} \ar@/r/[u] & \resource{debugging\\information}}
\seeoberon\seeassembly\seeavrtt\seeobject\seedebugging
}

\providecommand{\obmabk}{
\toolsection{obm68k} is a compiler for the Oberon programming language targeting the M68000 hardware architecture.
It generates machine code for M68000 processors from modules written in Oberon and stores it in corresponding object files.
For debugging purposes, it also creates a debugging information file as well as an assembly file containing a listing of the generated machine code.
In addition, it stores the interface of each module in a symbol file which is required when other modules import the module.
Programs generated with this compiler require additional runtime support that is stored in the \file{ob\-m68k\-run} library file.
\flowgraph{\resource{Oberon\\source code} \ar[r] & \toolbox{obm68k} \ar[r] \ar@/l/[d] \ar[rd] & \resource{object file} \\ \variable{ECSIMPORT} \ar[ru] & \resource{symbol\\files} \ar@/r/[u] & \resource{debugging\\information}}
\seeoberon\seeassembly\seemabk\seeobject\seedebugging
}

\providecommand{\obmibl}{
\toolsection{obmibl} is a compiler for the Oberon programming language targeting the MicroBlaze hardware architecture.
It generates machine code for MicroBlaze processors from modules written in Oberon and stores it in corresponding object files.
For debugging purposes, it also creates a debugging information file as well as an assembly file containing a listing of the generated machine code.
In addition, it stores the interface of each module in a symbol file which is required when other modules import the module.
Programs generated with this compiler require additional runtime support that is stored in the \file{ob\-mibl\-run} library file.
\flowgraph{\resource{Oberon\\source code} \ar[r] & \toolbox{obmibl} \ar[r] \ar@/l/[d] \ar[rd] & \resource{object file} \\ \variable{ECSIMPORT} \ar[ru] & \resource{symbol\\files} \ar@/r/[u] & \resource{debugging\\information}}
\seeoberon\seeassembly\seemibl\seeobject\seedebugging
}

\providecommand{\obmipsa}{
\toolsection{obmips32} is a compiler for the Oberon programming language targeting the MIPS32 hardware architecture.
It generates machine code for MIPS32 processors from modules written in Oberon and stores it in corresponding object files.
For debugging purposes, it also creates a debugging information file as well as an assembly file containing a listing of the generated machine code.
In addition, it stores the interface of each module in a symbol file which is required when other modules import the module.
Programs generated with this compiler require additional runtime support that is stored in the \file{ob\-mips32\-run} library file.
\flowgraph{\resource{Oberon\\source code} \ar[r] & \toolbox{obmips32} \ar[r] \ar@/l/[d] \ar[rd] & \resource{object file} \\ \variable{ECSIMPORT} \ar[ru] & \resource{symbol\\files} \ar@/r/[u] & \resource{debugging\\information}}
\seeoberon\seeassembly\seemips\seeobject\seedebugging
}

\providecommand{\obmipsb}{
\toolsection{obmips64} is a compiler for the Oberon programming language targeting the MIPS64 hardware architecture.
It generates machine code for MIPS64 processors from modules written in Oberon and stores it in corresponding object files.
For debugging purposes, it also creates a debugging information file as well as an assembly file containing a listing of the generated machine code.
In addition, it stores the interface of each module in a symbol file which is required when other modules import the module.
Programs generated with this compiler require additional runtime support that is stored in the \file{ob\-mips64\-run} library file.
\flowgraph{\resource{Oberon\\source code} \ar[r] & \toolbox{obmips64} \ar[r] \ar@/l/[d] \ar[rd] & \resource{object file} \\ \variable{ECSIMPORT} \ar[ru] & \resource{symbol\\files} \ar@/r/[u] & \resource{debugging\\information}}
\seeoberon\seeassembly\seemips\seeobject\seedebugging
}

\providecommand{\obmmix}{
\toolsection{obmmix} is a compiler for the Oberon programming language targeting the MMIX hardware architecture.
It generates machine code for MMIX processors from modules written in Oberon and stores it in corresponding object files.
For debugging purposes, it also creates a debugging information file as well as an assembly file containing a listing of the generated machine code.
In addition, it stores the interface of each module in a symbol file which is required when other modules import the module.
Programs generated with this compiler require additional runtime support that is stored in the \file{ob\-mmix\-run} library file.
\flowgraph{\resource{Oberon\\source code} \ar[r] & \toolbox{obmmix} \ar[r] \ar@/l/[d] \ar[rd] & \resource{object file} \\ \variable{ECSIMPORT} \ar[ru] & \resource{symbol\\files} \ar@/r/[u] & \resource{debugging\\information}}
\seeoberon\seeassembly\seemmix\seeobject\seedebugging
}

\providecommand{\oborok}{
\toolsection{obor1k} is a compiler for the Oberon programming language targeting the OpenRISC 1000 hardware architecture.
It generates machine code for OpenRISC 1000 processors from modules written in Oberon and stores it in corresponding object files.
For debugging purposes, it also creates a debugging information file as well as an assembly file containing a listing of the generated machine code.
In addition, it stores the interface of each module in a symbol file which is required when other modules import the module.
Programs generated with this compiler require additional runtime support that is stored in the \file{ob\-or1k\-run} library file.
\flowgraph{\resource{Oberon\\source code} \ar[r] & \toolbox{obor1k} \ar[r] \ar@/l/[d] \ar[rd] & \resource{object file} \\ \variable{ECSIMPORT} \ar[ru] & \resource{symbol\\files} \ar@/r/[u] & \resource{debugging\\information}}
\seeoberon\seeassembly\seeorok\seeobject\seedebugging
}

\providecommand{\obppca}{
\toolsection{obppc32} is a compiler for the Oberon programming language targeting the PowerPC hardware architecture.
It generates machine code for PowerPC processors from modules written in Oberon and stores it in corresponding object files.
The compiler generates machine code for the 32-bit operating mode defined by the PowerPC architecture.
For debugging purposes, it also creates a debugging information file as well as an assembly file containing a listing of the generated machine code.
In addition, it stores the interface of each module in a symbol file which is required when other modules import the module.
Programs generated with this compiler require additional runtime support that is stored in the \file{ob\-ppc32\-run} library file.
\flowgraph{\resource{Oberon\\source code} \ar[r] & \toolbox{obppc32} \ar[r] \ar@/l/[d] \ar[rd] & \resource{object file} \\ \variable{ECSIMPORT} \ar[ru] & \resource{symbol\\files} \ar@/r/[u] & \resource{debugging\\information}}
\seeoberon\seeassembly\seeppc\seeobject\seedebugging
}

\providecommand{\obppcb}{
\toolsection{obppc64} is a compiler for the Oberon programming language targeting the PowerPC hardware architecture.
It generates machine code for PowerPC processors from modules written in Oberon and stores it in corresponding object files.
The compiler generates machine code for the 64-bit operating mode defined by the PowerPC architecture.
For debugging purposes, it also creates a debugging information file as well as an assembly file containing a listing of the generated machine code.
In addition, it stores the interface of each module in a symbol file which is required when other modules import the module.
Programs generated with this compiler require additional runtime support that is stored in the \file{ob\-ppc64\-run} library file.
\flowgraph{\resource{Oberon\\source code} \ar[r] & \toolbox{obppc64} \ar[r] \ar@/l/[d] \ar[rd] & \resource{object file} \\ \variable{ECSIMPORT} \ar[ru] & \resource{symbol\\files} \ar@/r/[u] & \resource{debugging\\information}}
\seeoberon\seeassembly\seeppc\seeobject\seedebugging
}

\providecommand{\obrisc}{
\toolsection{obrisc} is a compiler for the Oberon programming language targeting the RISC hardware architecture.
It generates machine code for RISC processors from modules written in Oberon and stores it in corresponding object files.
For debugging purposes, it also creates a debugging information file as well as an assembly file containing a listing of the generated machine code.
In addition, it stores the interface of each module in a symbol file which is required when other modules import the module.
Programs generated with this compiler require additional runtime support that is stored in the \file{ob\-risc\-run} library file.
\flowgraph{\resource{Oberon\\source code} \ar[r] & \toolbox{obrisc} \ar[r] \ar@/l/[d] \ar[rd] & \resource{object file} \\ \variable{ECSIMPORT} \ar[ru] & \resource{symbol\\files} \ar@/r/[u] & \resource{debugging\\information}}
\seeoberon\seeassembly\seerisc\seeobject\seedebugging
}

\providecommand{\obwasm}{
\toolsection{obwasm} is a compiler for the Oberon programming language targeting the WebAssembly architecture.
It generates machine code for WebAssembly targets from modules written in Oberon and stores it in corresponding object files.
For debugging purposes, it also creates a debugging information file as well as an assembly file containing a listing of the generated machine code.
In addition, it stores the interface of each module in a symbol file which is required when other modules import the module.
Programs generated with this compiler require additional runtime support that is stored in the \file{ob\-wasm\-run} library file.
\flowgraph{\resource{Oberon\\source code} \ar[r] & \toolbox{obwasm} \ar[r] \ar@/l/[d] \ar[rd] & \resource{object file} \\ \variable{ECSIMPORT} \ar[ru] & \resource{symbol\\files} \ar@/r/[u] & \resource{debugging\\information}}
\seeoberon\seeassembly\seewasm\seeobject\seedebugging
}

% converter tools

\providecommand{\dbgdwarf}{
\toolsection{dbgdwarf} is a DWARF debugging information converter tool.
It converts debugging information into the DWARF debugging data format and stores it in corresponding object files~\cite{dwarffile}.
The resulting debugging object files can be combined with runtime support that creates Executable and Linking Format (ELF) files~\cite{elffile}.
\flowgraph{\resource{debugging\\information} \ar[r] & \toolbox{dbgdwarf} \ar[r] & \resource{debugging\\object file}}
\seeobject\seedebugging
}

% assembler tools

\providecommand{\asmprint}{
\toolsection{asmprint} is a pretty printer for generic assembly code.
It reformats generic assembly code and writes it to the standard output stream.
\flowgraph{\resource{generic assembly\\source code} \ar[r] & \toolbox{asmprint} \ar[r] & \resource{reformatted\\source code}}
\seeassembly
}

\providecommand{\amdaasm}{
\toolsection{amd16asm} is an assembler for the AMD64 hardware architecture.
It translates assembly code into machine code for AMD64 processors and stores it in corresponding object files.
By default, the assembler generates machine code for the 16-bit operating mode defined by the AMD64 architecture.
\flowgraph{\resource{AMD16 assembly\\source code} \ar[r] & \toolbox{amd16asm} \ar[r] & \resource{object file}}
\seeassembly\seeamd\seeobject
}

\providecommand{\amdadism}{
\toolsection{amd16dism} is a disassembler for the AMD64 hardware architecture.
It translates machine code from object files targeting AMD64 processors into assembly code and writes it to the standard output stream.
It assumes that the machine code was generated for the 16-bit operating mode defined by the AMD64 architecture.
\flowgraph{\resource{object file} \ar[r] & \toolbox{amd16dism} \ar[r] & \resource{disassembly\\listing}}
\seeassembly\seeamd\seeobject
}

\providecommand{\amdbasm}{
\toolsection{amd32asm} is an assembler for the AMD64 hardware architecture.
It translates assembly code into machine code for AMD64 processors and stores it in corresponding object files.
By default, the assembler generates machine code for the 32-bit operating mode defined by the AMD64 architecture.
\flowgraph{\resource{AMD32 assembly\\source code} \ar[r] & \toolbox{amd32asm} \ar[r] & \resource{object file}}
\seeassembly\seeamd\seeobject
}

\providecommand{\amdbdism}{
\toolsection{amd32dism} is a disassembler for the AMD64 hardware architecture.
It translates machine code from object files targeting AMD64 processors into assembly code and writes it to the standard output stream.
It assumes that the machine code was generated for the 32-bit operating mode defined by the AMD64 architecture.
\flowgraph{\resource{object file} \ar[r] & \toolbox{amd32dism} \ar[r] & \resource{disassembly\\listing}}
\seeassembly\seeamd\seeobject
}

\providecommand{\amdcasm}{
\toolsection{amd64asm} is an assembler for the AMD64 hardware architecture.
It translates assembly code into machine code for AMD64 processors and stores it in corresponding object files.
By default, the assembler generates machine code for the 64-bit operating mode defined by the AMD64 architecture.
\flowgraph{\resource{AMD64 assembly\\source code} \ar[r] & \toolbox{amd64asm} \ar[r] & \resource{object file}}
\seeassembly\seeamd\seeobject
}

\providecommand{\amdcdism}{
\toolsection{amd64dism} is a disassembler for the AMD64 hardware architecture.
It translates machine code from object files targeting AMD64 processors into assembly code and writes it to the standard output stream.
It assumes that the machine code was generated for the 64-bit operating mode defined by the AMD64 architecture.
\flowgraph{\resource{object file} \ar[r] & \toolbox{amd64dism} \ar[r] & \resource{disassembly\\listing}}
\seeassembly\seeamd\seeobject
}

\providecommand{\armaasm}{
\toolsection{arma32asm} is an assembler for the ARM hardware architecture.
It translates assembly code into machine code for ARM processors executing A32 instructions and stores it in corresponding object files.
\flowgraph{\resource{ARM A32 assembly\\source code} \ar[r] & \toolbox{arma32asm} \ar[r] & \resource{object file}}
\seeassembly\seearm\seeobject
}

\providecommand{\armadism}{
\toolsection{arma32dism} is a disassembler for the ARM hardware architecture.
It translates machine code from object files targeting ARM processors executing A32 instructions into assembly code and writes it to the standard output stream.
\flowgraph{\resource{object file} \ar[r] & \toolbox{arma32dism} \ar[r] & \resource{disassembly\\listing}}
\seeassembly\seearm\seeobject
}

\providecommand{\armbasm}{
\toolsection{arma64asm} is an assembler for the ARM hardware architecture.
It translates assembly code into machine code for ARM processors executing A64 instructions and stores it in corresponding object files.
\flowgraph{\resource{ARM A64 assembly\\source code} \ar[r] & \toolbox{arma64asm} \ar[r] & \resource{object file}}
\seeassembly\seearm\seeobject
}

\providecommand{\armbdism}{
\toolsection{arma64dism} is a disassembler for the ARM hardware architecture.
It translates machine code from object files targeting ARM processors executing A64 instructions into assembly code and writes it to the standard output stream.
\flowgraph{\resource{object file} \ar[r] & \toolbox{arma64dism} \ar[r] & \resource{disassembly\\listing}}
\seeassembly\seearm\seeobject
}

\providecommand{\armcasm}{
\toolsection{armt32asm} is an assembler for the ARM hardware architecture.
It translates assembly code into machine code for ARM processors executing T32 instructions and stores it in corresponding object files.
\flowgraph{\resource{ARM T32 assembly\\source code} \ar[r] & \toolbox{armt32asm} \ar[r] & \resource{object file}}
\seeassembly\seearm\seeobject
}

\providecommand{\armcdism}{
\toolsection{armt32dism} is a disassembler for the ARM hardware architecture.
It translates machine code from object files targeting ARM processors executing T32 instructions into assembly code and writes it to the standard output stream.
\flowgraph{\resource{object file} \ar[r] & \toolbox{armt32dism} \ar[r] & \resource{disassembly\\listing}}
\seeassembly\seearm\seeobject
}

\providecommand{\avrasm}{
\toolsection{avrasm} is an assembler for the AVR hardware architecture.
It translates assembly code into machine code for AVR processors and stores it in corresponding object files.
The identifiers \texttt{RXL}, \texttt{RXH}, \texttt{RYL}, \texttt{RYH}, \texttt{RZL}, and \texttt{RZH} are predefined and name the corresponding registers.
The identifiers \texttt{SPL} and \texttt{SPH} are also predefined and evaluate to the address of the corresponding registers.
\flowgraph{\resource{AVR assembly\\source code} \ar[r] & \toolbox{avrasm} \ar[r] & \resource{object file}}
\seeassembly\seeavr\seeobject
}

\providecommand{\avrdism}{
\toolsection{avrdism} is a disassembler for the AVR hardware architecture.
It translates machine code from object files targeting AVR processors into assembly code and writes it to the standard output stream.
\flowgraph{\resource{object file} \ar[r] & \toolbox{avrdism} \ar[r] & \resource{disassembly\\listing}}
\seeassembly\seeavr\seeobject
}

\providecommand{\avrttasm}{
\toolsection{avr32asm} is an assembler for the AVR32 hardware architecture.
It translates assembly code into machine code for AVR32 processors and stores it in corresponding object files.
\flowgraph{\resource{AVR32 assembly\\source code} \ar[r] & \toolbox{avr32asm} \ar[r] & \resource{object file}}
\seeassembly\seeavrtt\seeobject
}

\providecommand{\avrttdism}{
\toolsection{avr32dism} is a disassembler for the AVR32 hardware architecture.
It translates machine code from object files targeting AVR32 processors into assembly code and writes it to the standard output stream.
\flowgraph{\resource{object file} \ar[r] & \toolbox{avr32dism} \ar[r] & \resource{disassembly\\listing}}
\seeassembly\seeavrtt\seeobject
}

\providecommand{\mabkasm}{
\toolsection{m68kasm} is an assembler for the M68000 hardware architecture.
It translates assembly code into machine code for M68000 processors and stores it in corresponding object files.
\flowgraph{\resource{68000 assembly\\source code} \ar[r] & \toolbox{m68kasm} \ar[r] & \resource{object file}}
\seeassembly\seemabk\seeobject
}

\providecommand{\mabkdism}{
\toolsection{m68kdism} is a disassembler for the M68000 hardware architecture.
It translates machine code from object files targeting M68000 processors into assembly code and writes it to the standard output stream.
\flowgraph{\resource{object file} \ar[r] & \toolbox{m68kdism} \ar[r] & \resource{disassembly\\listing}}
\seeassembly\seemabk\seeobject
}

\providecommand{\miblasm}{
\toolsection{miblasm} is an assembler for the MicroBlaze hardware architecture.
It translates assembly code into machine code for MicroBlaze processors and stores it in corresponding object files.
\flowgraph{\resource{MicroBlaze assembly\\source code} \ar[r] & \toolbox{miblasm} \ar[r] & \resource{object file}}
\seeassembly\seemibl\seeobject
}

\providecommand{\mibldism}{
\toolsection{mibldism} is a disassembler for the MicroBlaze hardware architecture.
It translates machine code from object files targeting MicroBlaze processors into assembly code and writes it to the standard output stream.
\flowgraph{\resource{object file} \ar[r] & \toolbox{mibldism} \ar[r] & \resource{disassembly\\listing}}
\seeassembly\seemibl\seeobject
}

\providecommand{\mipsaasm}{
\toolsection{mips32asm} is an assembler for the MIPS32 hardware architecture.
It translates assembly code into machine code for MIPS32 processors and stores it in corresponding object files.
\flowgraph{\resource{MIPS32 assembly\\source code} \ar[r] & \toolbox{mips32asm} \ar[r] & \resource{object file}}
\seeassembly\seemips\seeobject
}

\providecommand{\mipsadism}{
\toolsection{mips32dism} is a disassembler for the MIPS32 hardware architecture.
It translates machine code from object files targeting MIPS32 processors into assembly code and writes it to the standard output stream.
\flowgraph{\resource{object file} \ar[r] & \toolbox{mips32dism} \ar[r] & \resource{disassembly\\listing}}
\seeassembly\seemips\seeobject
}

\providecommand{\mipsbasm}{
\toolsection{mips64asm} is an assembler for the MIPS64 hardware architecture.
It translates assembly code into machine code for MIPS64 processors and stores it in corresponding object files.
\flowgraph{\resource{MIPS64 assembly\\source code} \ar[r] & \toolbox{mips64asm} \ar[r] & \resource{object file}}
\seeassembly\seemips\seeobject
}

\providecommand{\mipsbdism}{
\toolsection{mips64dism} is a disassembler for the MIPS64 hardware architecture.
It translates machine code from object files targeting MIPS64 processors into assembly code and writes it to the standard output stream.
\flowgraph{\resource{object file} \ar[r] & \toolbox{mips64dism} \ar[r] & \resource{disassembly\\listing}}
\seeassembly\seemips\seeobject
}

\providecommand{\mmixasm}{
\toolsection{mmixasm} is an assembler for the MMIX hardware architecture.
It translates assembly code into machine code for MMIX processors and stores it in corresponding object files.
The names of all special registers are predefined and evaluate to the corresponding number.
\flowgraph{\resource{MMIX assembly\\source code} \ar[r] & \toolbox{mmixasm} \ar[r] & \resource{object file}}
\seeassembly\seemmix\seeobject
}

\providecommand{\mmixdism}{
\toolsection{mmixdism} is a disassembler for the MMIX hardware architecture.
It translates machine code from object files targeting MMIX processors into assembly code and writes it to the standard output stream.
\flowgraph{\resource{object file} \ar[r] & \toolbox{mmixdism} \ar[r] & \resource{disassembly\\listing}}
\seeassembly\seemmix\seeobject
}

\providecommand{\orokasm}{
\toolsection{or1kasm} is an assembler for the OpenRISC 1000 hardware architecture.
It translates assembly code into machine code for OpenRISC 1000 processors and stores it in corresponding object files.
\flowgraph{\resource{OpenRISC 1000 assembly\\source code} \ar[r] & \toolbox{or1kasm} \ar[r] & \resource{object file}}
\seeassembly\seeorok\seeobject
}

\providecommand{\orokdism}{
\toolsection{or1kdism} is a disassembler for the OpenRISC 1000 hardware architecture.
It translates machine code from object files targeting OpenRISC 1000 processors into assembly code and writes it to the standard output stream.
\flowgraph{\resource{object file} \ar[r] & \toolbox{or1kdism} \ar[r] & \resource{disassembly\\listing}}
\seeassembly\seeorok\seeobject
}

\providecommand{\ppcaasm}{
\toolsection{ppc32asm} is an assembler for the PowerPC hardware architecture.
It translates assembly code into machine code for PowerPC processors and stores it in corresponding object files.
By default, the assembler generates machine code for the 32-bit operating mode defined by the PowerPC architecture.
\flowgraph{\resource{PowerPC assembly\\source code} \ar[r] & \toolbox{ppc32asm} \ar[r] & \resource{object file}}
\seeassembly\seeppc\seeobject
}

\providecommand{\ppcadism}{
\toolsection{ppc32dism} is a disassembler for the PowerPC hardware architecture.
It translates machine code from object files targeting PowerPC processors into assembly code and writes it to the standard output stream.
It assumes that the machine code was generated for the 32-bit operating mode defined by the PowerPC architecture.
\flowgraph{\resource{object file} \ar[r] & \toolbox{ppc32dism} \ar[r] & \resource{disassembly\\listing}}
\seeassembly\seeppc\seeobject
}

\providecommand{\ppcbasm}{
\toolsection{ppc64asm} is an assembler for the PowerPC hardware architecture.
It translates assembly code into machine code for PowerPC processors and stores it in corresponding object files.
By default, the assembler generates machine code for the 64-bit operating mode defined by the PowerPC architecture.
\flowgraph{\resource{PowerPC assembly\\source code} \ar[r] & \toolbox{ppc64asm} \ar[r] & \resource{object file}}
\seeassembly\seeppc\seeobject
}

\providecommand{\ppcbdism}{
\toolsection{ppc64dism} is a disassembler for the PowerPC hardware architecture.
It translates machine code from object files targeting PowerPC processors into assembly code and writes it to the standard output stream.
It assumes that the machine code was generated for the 64-bit operating mode defined by the PowerPC architecture.
\flowgraph{\resource{object file} \ar[r] & \toolbox{ppc64dism} \ar[r] & \resource{disassembly\\listing}}
\seeassembly\seeppc\seeobject
}

\providecommand{\riscasm}{
\toolsection{riscasm} is an assembler for the RISC hardware architecture.
It translates assembly code into machine code for RISC processors and stores it in corresponding object files.
The names of all special registers are predefined and evaluate to the corresponding number.
\flowgraph{\resource{RISC assembly\\source code} \ar[r] & \toolbox{riscasm} \ar[r] & \resource{object file}}
\seeassembly\seerisc\seeobject
}

\providecommand{\riscdism}{
\toolsection{riscdism} is a disassembler for the RISC hardware architecture.
It translates machine code from object files targeting RISC processors into assembly code and writes it to the standard output stream.
\flowgraph{\resource{object file} \ar[r] & \toolbox{riscdism} \ar[r] & \resource{disassembly\\listing}}
\seeassembly\seerisc\seeobject
}

\providecommand{\wasmasm}{
\toolsection{wasmasm} is an assembler for the WebAssembly architecture.
It translates assembly code into machine code for WebAssembly targets and stores it in corresponding object files.
The names of all special registers are predefined and evaluate to the corresponding number.
\flowgraph{\resource{WebAssembly assembly\\source code} \ar[r] & \toolbox{wasmasm} \ar[r] & \resource{object file}}
\seeassembly\seewasm\seeobject
}

\providecommand{\wasmdism}{
\toolsection{wasmdism} is a disassembler for the WebAssembly architecture.
It translates machine code from object files targeting WebAssembly targets into assembly code and writes it to the standard output stream.
\flowgraph{\resource{object file} \ar[r] & \toolbox{wasmdism} \ar[r] & \resource{disassembly\\listing}}
\seeassembly\seewasm\seeobject
}

% linker tools

\providecommand{\linklib}{
\toolsection{linklib} is an object file combiner.
It creates a static library file by combining all object files given to it into a single one.
\flowgraph{\resource{object files} \ar[r] & \toolbox{linklib} \ar[r] & \resource{library file}}
\seeobject
}

\providecommand{\linkbin}{
\toolsection{linkbin} is a linker for plain binary files.
It links all object files given to it into a single image and stores it in a binary file that begins with the first linked section.
It also creates a map file that lists the address, type, name and size of all used sections.
The filename extension of the resulting binary file can be specified by putting it into a constant data section called \texttt{\_extension}.
\flowgraph{\resource{object files} \ar[r] & \toolbox{linkbin} \ar[r] \ar[d] & \resource{binary file} \\ & \resource{map file}}
\seeobject
}

\providecommand{\linkmem}{
\toolsection{linkmem} is a linker for plain binary files partitioned into random-access and read-only memory.
It links all object files given to it into two distinct images, one for data sections and one for code and constant data sections, and stores each image in a binary file that begins with the first linked section of the corresponding type.
It also creates a map file that lists the address, type, name and size of all used sections.
\flowgraph{\resource{object files} \ar[r] & \toolbox{linkmem} \ar[r] \ar[d] & \resource{RAM file/\\ROM file} \\ & \resource{map file}}
\seeobject
}

\providecommand{\linkprg}{
\toolsection{linkprg} is a linker for GEMDOS executable files.
It links all object files given to it into a single image and stores the image in an Atari GEMDOS executable file~\cite{gemdosfile}.
It also creates a map file that lists the address relative to the text segment, type, name and size of all used sections.
The filename extension of the resulting executable file can be specified by putting it into a constant data section called \texttt{\_extension}.
The GEMDOS executable file format requires all patch patterns of absolute link patches to consist of four full bitmasks with descending offsets.
\flowgraph{\resource{object files} \ar[r] & \toolbox{linkprg} \ar[r] \ar[d] & \resource{executable file} \\ & \resource{map file}}
\seeobject
}

\providecommand{\linkhex}{
\toolsection{linkhex} is a linker for Intel HEX files.
It links all code sections of the object files given to it into single image and stores the image in an Intel HEX file~\cite{hexfile} that begins with the first linked section.
It also creates a map file that lists the address, type, name and size of all used sections.
\flowgraph{\resource{object files} \ar[r] & \toolbox{linkhex} \ar[r] \ar[d] & \resource{HEX file} \\ & \resource{map file}}
\seeobject
}

\providecommand{\mapsearch}{
\toolsection{mapsearch} is a debugging tool.
It searches map files generated by linker tools for the name of a binary section that encompasses a memory address read from the standard input stream.
If additionally provided with one or more object files, it also stores an excerpt thereof in a separate object file called map search result which only contains the identified binary section for disassembling purposes.
\flowgraph{& \resource{map files/\\object files} \ar[d] \\ \resource{memory\\address} \ar[r] & \toolbox{mapsearch} \ar[r] \ar[d] & \resource{section name/\\relative offset} \\ & \resource{object file\\excerpt}}
\seeobject
}

\renewcommand{\seemips}{}

\startchapter{MIPS}{MIPS Hardware Architecture Support}{mips}
{This \documentation{} describes how the \ecs{} supports the MIPS32 and MIPS64 hardware architectures.
This includes information about the assemblers, disassemblers, and the various compilers featured by the \ecs{} as well as the interoperability between these tools.}

\section{Introduction}

The \ecs{} features various compilers, assemblers, and disassemblers that target the MIPS32 and MIPS64 hardware architectures by MIPS Limited.
Figure~\ref{fig:mipsdataflow} shows the data flow in-between these tools.

\begin{figure}
\flowgraph{
\resource{intermediate\\code} \ar[d] & & \resource{assembly\\source code} \ar[d] \\
\converter{MIPS\\Generator} \ar[r] \ar[rd] \ar[d] & \resource{assembly\\listing} \ar[r] & \converter{MIPS\\Assembler} \ar[ld] \\
\resource{debugging\\information} & \resource{object file} \ar[d] \\
& \converter{MIPS\\Disassembler} \ar[d] \\
& \resource{disassembly\\listing} \\
}\caption{Data flow within the tools targeting the MIPS32 and MIPS64 architectures}
\label{fig:mipsdataflow}
\end{figure}

All compilers targeting the MIPS32 and MIPS64 architectures translate their programs using an intermediate code representation.
The MIPS generator is able to translate the intermediate code representation of a program into machine code for MIPS32 and MIPS64 processors.
It stores the resulting binary code and data in so-called object files.
Additionally, the generator is able to create an assembly code listing of the machine code for debugging purposes.
This assembly code listing can also be processed by the assemblers yielding exactly the same object file.
The disassemblers are able to open object files and print a human-readable disassembly listing of their contents.
\seeobject\seecode

\section{Instruction Set}

Tools targeting the MIPS32 and MIPS64 architectures support the instruction set listed in Table~\ref{tab:mipsset} and use the same assembly syntax as predefined by MIPS~\cite{mips:volume1,mips:volume2}.
\seeassembly

\instructionset{mips}{Supported MIPS32 and MIPS64 instruction set}{5}{6}

The actual architecture the compilers and assemblers generate machine code for by default, is indicated by the number in the suffix of their names.
The assemblers allow users to temporarily switch between the supported architectures by passing 32 or 64 as operand to the bit mode directive.
The default ordering of the binary encoding of instructions is least significant octet first but can be changed using the big-endian mode directive.

\section{Calling Convention}\index{Calling convention!of MIPS}

The machine code generator and runtime support for the MIPS32 and MIPS64 architectures as provided by the \ecs{} use the following calling convention in order to enable interoperability.

\subsection{Stack Operations}

Arguments for functions are in general passed using the stack according to the intermediate code specification.
See \Documentation{}~\documentationref{code}{Intermediate Code Representation} for more information about the role of the stack.
Function arguments are pushed on the stack in reverse order and cleaned by the caller.

\subsection{Floating-Point Support}

The MIPS32 and MIPS64 architectures optionally support floating-point operations.
The MIPS generator is able to generate native floating-point operations for processors that do support them.

\subsection{Register Mapping}

The special-purpose registers defined by the intermediate code representation are mapped to their corresponding physical registers in the following way:

\begin{itemize}

\item Result Register\alignright\texttt{\$res}\nopagebreak

The intermediate code result register \texttt{\$res} is mapped to MIPS registers \texttt{r2} and \texttt{r3} depending on the size of the actual result value.
If supported natively, floating-point results are stored in the register \texttt{f0}.

\item Stack Pointer Register\alignright\texttt{\$sp}\nopagebreak

The intermediate code stack pointer register \texttt{\$sp} is mapped to the MIPS register \texttt{r29}.

\item Frame Pointer Register\alignright\texttt{\$fp}\nopagebreak

The intermediate code frame pointer register \texttt{\$fp} is mapped to the MIPS register \texttt{r30}.

\item Link Register\alignright\texttt{\$lnk}\nopagebreak

The intermediate code link register \texttt{\$lnk} is supported and mapped to the MIPS register \texttt{r31}.

\end{itemize}

All other intermediate code registers are mapped as needed to the remaining physical registers.
Their contents and mapping are therefore considered volatile across function calls.
Registers holding integer and address values are mapped to the general-purpose registers,
while registers holding floating-point values are mapped to the floating-point registers.

\section{Runtime Support}\index{Runtime support!for MIPS}

The \ecs{} provides runtime support for the MIPS architecture and runtime environments based on this hardware architecture in object files.
Users targeting a specific runtime environment have to use an appropriate linker together with these object files in order create an executable program.
This section gives information about all supported runtime environments based on the MIPS hardware architecture as well as the required combination of linker and object files.

Basic architectural runtime support is provided by the object files \objfile{mips32\-run} and \objfile{mips64\-run}.
Users should always include one of these object files during linking regardless of the actual target runtime environment.
All other object files given to the linker should target the same hardware architecture.

Programs written in \cpp{} need additional runtime support stored in the \libfile{cpp\-mips32\-run} and \libfile{cpp\-mips64\-run} library files respectively.
Programs written in Oberon need additional runtime support stored in the \libfile{ob\-mips32\-run} and \libfile{ob\-mips64\-run} library files respectively.
\seecpp\seeoberon

\section{MIPS Tools}

The \ecs{} provides the following tools that are able to process object files targeting the MIPS32 and MIPS64 hardware architectures.
\interface

\cdmipsa
\cdmipsb
\cppmipsa
\cppmipsb
\falmipsa
\falmipsb
\obmipsa
\obmipsb
\mipsaasm
\mipsbasm
\mipsadism
\mipsbdism
\linkbin

\concludechapter

% MMIX architecture documentation
% Copyright (C) Florian Negele

% This file is part of the Eigen Compiler Suite.

% Permission is granted to copy, distribute and/or modify this document
% under the terms of the GNU Free Documentation License, Version 1.3
% or any later version published by the Free Software Foundation.

% You should have received a copy of the GNU Free Documentation License
% along with the ECS.  If not, see <https://www.gnu.org/licenses/>.

% Generic documentation utilities
% Copyright (C) Florian Negele

% This file is part of the Eigen Compiler Suite.

% Permission is granted to copy, distribute and/or modify this document
% under the terms of the GNU Free Documentation License, Version 1.3
% or any later version published by the Free Software Foundation.

% You should have received a copy of the GNU Free Documentation License
% along with the ECS.  If not, see <https://www.gnu.org/licenses/>.

\providecommand{\cpp}{C\texttt{++}}
\providecommand{\opt}{_\mathit{opt}}
\providecommand{\tool}[1]{\texttt{#1}}
\providecommand{\version}{Version 0.0.40}
\providecommand{\resource}[1]{*++\txt{#1}}
\providecommand{\ecs}{Eigen Compiler Suite}
\providecommand{\changed}[1]{\underline{#1}}
\providecommand{\toolbox}[1]{\converter{#1}}
\providecommand{\file}{}\renewcommand{\file}[1]{\texttt{#1}}
\providecommand{\alignright}{\hfill\linebreak[0]\hspace*{\fill}}
\providecommand{\converter}[1]{*++[F][F*:white][F,:gray]\txt{#1}}
\providecommand{\documentation}{\ifbook chapter\else document\fi}
\providecommand{\Documentation}{\ifbook Chapter\else Document\fi}
\providecommand{\variable}[1]{\resource{\texttt{\small#1}\\variable}}
\providecommand{\documentationref}[2]{\ifbook\ref{#1}\else``\href{#1}{#2}''~\cite{#1}\fi}
\providecommand{\objfile}[1]{\texttt{#1}\index[runtime]{#1 object file@\texttt{#1} object file}}
\providecommand{\libfile}[1]{\texttt{#1}\index[runtime]{#1 library file@\texttt{#1} library file}}
\providecommand{\epigraph}[2]{\ifbook\begin{quote}\flushright\textit{#1}\par--- #2\end{quote}\fi}
\providecommand{\environmentvariable}[1]{\texttt{#1}\index{Environment variables!#1@\texttt{#1}}}
\providecommand{\environment}[1]{\texttt{#1}\index[environment]{#1 environment@\texttt{#1} environment}}
\providecommand{\toolsection}{}\renewcommand{\toolsection}[1]{\subsection{#1}\label{\prefix:#1}\tool{#1}}
\providecommand{\instruction}{}\renewcommand{\instruction}[2]{\noindent\qquad\pdftooltip{\texttt{#1}}{#2}\refstepcounter{instruction}\par}
\providecommand{\flowgraph}{}\renewcommand{\flowgraph}[1]{\par\sffamily\begin{displaymath}\xymatrix@=4ex{#1}\end{displaymath}\normalfont\par}
\providecommand{\instructionset}{}\renewcommand{\instructionset}[4]{\setcounter{instruction}{0}\begin{multicols}{\ifbook#3\else#4\fi}[{\captionof{table}[#2]{#2 (\ref*{#1:instructions}~instructions)}\label{tab:#1set}\vspace{-2ex}}]\footnotesize\raggedcolumns\input{#1.set}\label{#1:instructions}\end{multicols}}

\providecommand{\gpl}{GNU General Public License}
\providecommand{\rse}{ECS Runtime Support Exception}
\providecommand{\fdl}{\href{https://www.gnu.org/licenses/fdl.html}{GNU Free Documentation License}}

\providecommand{\docbegin}{}
\providecommand{\docend}{}
\providecommand{\doclabel}[1]{\hypertarget{#1}}
\providecommand{\doclink}[2]{\hyperlink{#1}{#2}}
\providecommand{\docsection}[3]{\hypertarget{#1}{\subsection{#2}}\label{sec:#1}\index[library]{#2@#3}}
\providecommand{\docsectionstar}[1]{}
\providecommand{\docsubbegin}{\begin{description}}
\providecommand{\docsubend}{\end{description}}
\providecommand{\docsubsection}[3]{\item[\hypertarget{#1}{#2}]\index[library]{#2@#3}}
\providecommand{\docsubsectionstar}[1]{\smallskip}
\providecommand{\docsubsubsection}[3]{\docsubsection{#1}{#2}{#3}}
\providecommand{\docsubsubsectionstar}[1]{}
\providecommand{\docsubsubsubsection}[3]{}
\providecommand{\docsubsubsubsectionstar}[1]{}
\providecommand{\doctable}{}

\providecommand{\debuggingtool}{}\renewcommand{\debuggingtool}{This tool is provided for debugging purposes.
It allows exposing and modifying an internal data structure that is usually not accessible.
}

\providecommand{\interface}{All tools accept command-line arguments which are taken as names of plain text files containing the source code.
If no arguments are provided, the standard input stream is used instead.
Output files are generated in the current working directory and have the same name as the input file being processed whereas the filename extension gets replaced by an appropriate suffix.
\seeinterface
}

\providecommand{\license}{\noindent Copyright \copyright{} Florian Negele\par\medskip\noindent
Permission is granted to copy, distribute and/or modify this document under the terms of the
\fdl{}, Version 1.3 or any later version published by the \href{https://fsf.org/}{Free Software Foundation}.
}

\providecommand{\ecslogosurface}{
\fill[darkgray] (0,0,0) -- (0,0,3) -- (0,3,3) -- (0,3,1) -- (0,4,1) -- (0,4,3) -- (0,5,3) -- (0,5,0) -- (0,2,0) -- (0,2,2) -- (0,1,2) -- (0,1,0) -- cycle;
\fill[gray] (0,5,0) -- (0,5,3) -- (1,5,3) -- (1,5,1) -- (2,5,1) -- (2,5,3) -- (3,5,3) -- (3,5,0) -- cycle;
\fill[lightgray] (0,0,0) -- (0,1,0) -- (2,1,0) -- (2,4,0) -- (1,4,0) -- (1,3,0) -- (2,3,0) -- (2,2,0) -- (0,2,0) -- (0,5,0) -- (3,5,0) -- (3,0,0) -- cycle;
\begin{scope}[line width=0.5]
\begin{scope}[gray]
\draw (0,0,0) -- (0,1,0);
\draw (2,1,0) -- (2,2,0);
\draw (0,1,2) -- (0,2,2);
\draw (0,2,0) -- (0,5,0);
\draw (2,3,0) -- (2,4,0);
\end{scope}
\begin{scope}[lightgray]
\draw (0,1,0) -- (0,1,2);
\draw (0,3,1) -- (0,3,3);
\draw (0,5,0) -- (0,5,3);
\draw (2,5,1) -- (2,5,3);
\end{scope}
\begin{scope}[white]
\draw (0,1,0) -- (2,1,0);
\draw (1,3,0) -- (2,3,0);
\draw (0,5,0) -- (3,5,0);
\end{scope}
\end{scope}
}

\providecommand{\ecslogo}[1]{
\begin{tikzpicture}[scale={(#1)/((sin(45)+cos(45))*3cm)},x={({-cos(45)*1cm},{sin(45)*sin(30)*1cm})},y={({0cm},{(cos(30)*1cm})},z={({sin(45)*1cm},{cos(45)*sin(30)*1cm})}]
\begin{scope}[darkgray,line width=1]
\draw (0,0,0) -- (0,0,3) -- (0,3,3) -- (2,3,3) -- (2,5,3) -- (3,5,3) -- (3,5,0) -- (3,0,0) -- cycle;
\draw (0,3,1) -- (0,4,1) -- (0,4,3) -- (0,5,3) -- (1,5,3) -- (1,5,1) -- (2,5,1);
\draw (1,3,0) -- (1,4,0) -- (2,4,0);
\end{scope}
\fill[darkgray] (2,0,0) -- (2,0,3) -- (2,5,3) -- (2,5,1) -- (2,4,1) -- (2,4,0) -- cycle;
\fill[lightgray] (2,0,2) -- (0,0,2) -- (0,2,2) -- (2,2,2) -- cycle;
\fill[gray] (0,1,0) -- (2,1,0) -- (2,1,2) -- (0,1,2) -- cycle;
\fill[gray] (0,3,1) -- (0,3,3) -- (2,3,3) -- (2,3,0) -- (1,3,0) -- (1,3,1) -- cycle;
\ecslogosurface
\end{tikzpicture}
}

\providecommand{\shadowedecslogo}[3]{
\begin{tikzpicture}[scale={(#1)/((sin(#2)+cos(#2))*3cm)},x={({-cos(#2)*1cm},{sin(#2)*sin(#3)*1cm})},y={({0cm},{(cos(#3)*1cm})},z={({sin(#2)*1cm},{cos(#2)*sin(#3)*1cm})}]
\shade[top color=lightgray!50!white,bottom color=white,middle color=lightgray!50!white] (0,0,0) -- (3,0,0) -- (3,{-0.5-3*sin(#2)*sin(#3)/cos(#3)},0) -- (0,-0.5,0) -- cycle;
\shade[top color=darkgray!50!gray,bottom color=white,middle color=darkgray!50!white] (0,0,0) -- (0,0,3) -- (0,{-0.5-3*cos(#2)*sin(#3)/cos(#3)},3) -- (0,-0.5,0) -- cycle;
\begin{scope}[y={({(cos(#2)+sin(#2))*0.5cm},{(cos(#2)*sin(#3)-sin(#2)*sin(#3))*0.5cm})}]
\useasboundingbox (3,0,0) -- (0,0,0) -- (0,0,3);
\shade[left color=darkgray!80!black,right color=lightgray,middle color=gray] (0,0,0) -- (0,1,0) -- (0,1,0.5) -- (0,2,0) -- (0,5,0) -- (0,5,3) -- (1,5,3) -- (1,4,3) -- (1,4,2.5) -- (1,3,3) -- (2,5,3) -- (3,5,3) -- (3,0,3) -- cycle;
\clip (0,0,0) -- (0,0,3) -- ({-3*sin(#2)/cos(#2)},0,0) -- cycle;
\shade[left color=darkgray,right color=lightgray!50!gray] (0,0,0) -- (0,1,0) -- (0,1,0.5) -- (0,2,0) -- (0,5,0) -- (0,5,3) -- (1,5,3) -- (1,4,3) -- (1,4,2.5) -- (1,3,3) -- (2,5,3) -- (3,5,3) -- (3,0,3) -- cycle;
\end{scope}
\shade[left color=darkgray,right color=darkgray!80!black] (2,0,0) -- (2,0,3) -- (2,5,3) -- (2,5,1) -- (2,4,1) -- (2,4,0) -- cycle;
\shade[left color=darkgray!90!black,right color=gray!80!darkgray] (2,0,2) -- (0,0,2) -- (0,2,2) -- (2,2,2) -- cycle;
\shade[top color=darkgray!90!black,bottom color=gray!80!darkgray] (0,1,0) -- (2,1,0) -- (2,1,2) -- (0,1,2) -- cycle;
\shade[top color=darkgray!90!black,bottom color=gray!80!darkgray] (0,3,1) -- (0,3,3) -- (2,3,3) -- (2,3,0) -- (1,3,0) -- (1,3,1) -- cycle;
\fill[gray] (2,1,0) -- (1.5,1,0.5) -- (0,1,0.5) -- (0,1,0) -- cycle;
\fill[gray] (1,3,2) -- (0.5,3,2) -- (0.5,3,3) -- (1,3,3) -- cycle;
\fill[gray] (2,3,0) -- (1.5,3,0.5) -- (1,3,0.5) -- (1,3,0) -- cycle;
\ecslogosurface
\end{tikzpicture}
}

\providecommand{\cpplogo}[1]{
\begin{tikzpicture}[scale=(#1)/512em]
\fill[gray] (435.2794,398.7159) -- (247.1911,507.3075) .. controls (236.3563,513.5642) and (218.6240,513.5642) .. (207.7892,507.3075) -- (19.7009,398.7159) .. controls (8.8646,392.4606) and (0.0000,377.1043) .. (0.0000,364.5924) -- (0.0000,147.4076) .. controls (0.8430,132.8363) and (8.2856,120.7683) .. (19.7009,113.2842) -- (207.7892,4.6926) .. controls (218.6240,-1.5642) and (236.3564,-1.5642) .. (247.1911,4.6926) -- (435.2794,113.2842) .. controls (447.5273,121.4304) and (454.4987,133.6918) .. (454.9803,147.4076) -- (454.9803,364.5924) .. controls (454.5404,377.7571) and (446.6566,391.0351) .. (435.2794,398.7159) -- cycle(75.8301,255.9993) .. controls (74.9389,404.0881) and (273.2892,469.4783) .. (358.8263,331.8769) -- (293.1917,293.8965) .. controls (253.5702,359.4301) and (155.1909,335.9977) .. (151.6601,255.9993) .. controls (152.7204,182.2703) and (249.4137,148.0211) .. (293.1961,218.1065) -- (358.8308,180.1276) .. controls (283.4477,49.2645) and (79.6318,96.3470) .. (75.8301,255.9993) -- cycle(379.1503,247.5747) -- (362.2982,247.5747) -- (362.2982,230.7226) -- (345.4490,230.7226) -- (345.4490,247.5747) -- (328.5969,247.5747) -- (328.5969,264.4254) -- (345.4490,264.4254) -- (345.4490,281.2759) -- (362.2982,281.2759) -- (362.2982,264.4254) -- (379.1503,264.4254) -- cycle(442.3420,247.5747) -- (425.4899,247.5747) -- (425.4899,230.7226) -- (408.6408,230.7226) -- (408.6408,247.5747) -- (391.7886,247.5747) -- (391.7886,264.4254) -- (408.6408,264.4254) -- (408.6408,281.2759) -- (425.4899,281.2759) -- (425.4899,264.4254) -- (442.3420,264.4254) -- cycle;
\end{tikzpicture}
}

\providecommand{\fallogo}[1]{
\begin{tikzpicture}[scale=(#1)/512em]
\fill[gray] (185.7774,0.0000) .. controls (200.4486,15.9798) and (226.8966,8.7148) .. (235.0426,31.5836) .. controls (249.5297,58.0598) and (247.9581,97.9161) .. (280.3335,110.9762) .. controls (309.1690,120.3496) and (337.8406,104.2727) .. (366.5753,103.9379) .. controls (373.4449,111.5171) and (379.2885,128.2574) .. (383.9755,108.9744) .. controls (396.6979,102.5615) and (437.2808,107.6681) .. (426.9652,124.3252) .. controls (408.9822,121.0785) and (412.4742,146.0729) .. (426.5192,131.4996) .. controls (433.8413,120.8489) and (465.1541,126.5522) .. (441.9067,135.7950) .. controls (396.1879,157.7478) and (344.1112,161.5079) .. (298.5528,183.5702) .. controls (277.7471,193.5198) and (284.6941,218.7163) .. (285.2127,236.9640) .. controls (292.3599,316.2826) and (307.3929,394.6311) .. (317.1198,473.6154) .. controls (329.0637,505.4736) and (292.1195,528.5004) .. (265.9183,511.2761) .. controls (237.9284,499.2462) and (237.3684,465.2681) .. (230.9102,439.9421) .. controls (218.6692,374.3397) and (215.6307,306.9662) .. (198.1732,242.3977) .. controls (183.1379,232.7444) and (164.4245,256.0298) .. (149.0430,261.4799) .. controls (116.9328,279.2585) and (87.1822,308.5851) .. (48.2293,307.8914) .. controls (21.3220,306.9037) and (-15.9107,281.8761) .. (7.2921,252.7908) .. controls (29.7799,220.6177) and (67.5177,204.3028) .. (100.9287,185.9449) .. controls (130.8217,170.8906) and (161.1548,156.5903) .. (191.0278,141.5847) .. controls (196.1738,120.0520) and (186.6049,95.2409) .. (186.8382,72.4353) .. controls (185.5234,48.4204) and (183.1700,23.9341) .. (185.7774,0.0000) -- cycle;
\end{tikzpicture}
}

\providecommand{\oblogo}[1]{
\begin{tikzpicture}[scale=(#1)/512em]
\fill[gray] (160.3865,208.9117) .. controls (154.0879,214.6478) and (149.0735,221.2409) .. (145.4125,228.5384) .. controls (184.8790,248.4273) and (234.7122,269.8787) .. (297.5493,291.8782) .. controls (300.3943,281.4769) and (300.9552,268.7619) .. (300.4023,255.2389) .. controls (248.9909,244.7891) and (200.0310,225.9279) .. (160.3865,208.9117) -- cycle(225.7398,392.6996) .. controls (308.0209,392.1716) and (359.3326,345.9277) .. (368.7203,285.2098) .. controls (376.6742,197.1784) and (311.7194,141.3342) .. (205.4287,142.1456) .. controls (139.9485,141.4804) and (88.7155,166.1957) .. (73.5775,228.0086) .. controls (52.0297,320.3408) and (123.4078,391.0103) .. (225.7398,392.6996) -- cycle(216.0739,176.4733) .. controls (268.9183,179.2424) and (315.8292,206.5488) .. (312.7454,265.1139) .. controls (313.2769,315.6384) and (286.5993,353.4946) .. (216.6040,355.7934) .. controls (162.4657,355.7934) and (126.0914,317.5023) .. (126.0914,260.5103) .. controls (126.1733,214.2900) and (163.3363,176.2849) .. (216.0739,176.4733) -- cycle(76.4897,189.1754) .. controls (13.1586,147.5631) and (0.0000,119.4207) .. (0.0000,119.4207) -- (90.6499,170.1632) .. controls (85.3004,175.8497) and (80.5994,182.1633) .. (76.4897,189.1754) -- cycle(353.9486,119.3004) -- (402.9482,119.3004) .. controls (427.0025,137.0797) and (450.9893,162.7034) .. (474.9529,191.0213) .. controls (509.3540,228.5339) and (531.3391,294.2091) .. (487.8149,312.1206) .. controls (462.8165,324.7652) and (394.3874,316.8943) .. (373.8912,313.6651) .. controls (379.9291,297.7449) and (383.2899,278.4204) .. (381.4989,257.7214) .. controls (420.3069,248.0321) and (421.9610,218.3461) .. (407.7867,192.6417) .. controls (391.1113,162.4018) and (370.1114,132.9097) .. (353.9486,119.3004) -- cycle;
\end{tikzpicture}
}

\providecommand{\markuptable}{
\begin{table}
\sffamily\centering
\begin{tabular}{@{}lcl@{}}
\toprule
\texttt{//italics//} & $\rightarrow$ & \textit{italics} \\
\midrule
\texttt{**bold**} & $\rightarrow$ & \textbf{bold} \\
\midrule
\texttt{\# ordered list} & & 1 ordered list \\
\texttt{\# second item} & $\rightarrow$ & 2 second item \\
\texttt{\#\# sub item} & & \hspace{1em} 1 sub item \\
\midrule
\texttt{* unordered list} & & $\bullet$ unordered list \\
\texttt{* second item} & $\rightarrow$ & $\bullet$ second item \\
\texttt{** sub item} & & \hspace{1em} $\bullet$ sub item \\
\midrule
\texttt{link to [[label]]} & $\rightarrow$ & link to \underline{label} \\
\midrule
\texttt{<{}<label>{}> definition } & $\rightarrow$ & definition \\
\midrule
\texttt{[[url|link name]]} & $\rightarrow$ & \underline{link name} \\
\midrule\addlinespace
\texttt{= large heading} & & {\Large large heading} \smallskip \\
\texttt{== medium heading} & $\rightarrow$ & {\large medium heading} \\
\texttt{=== small heading} & & small heading \\
\midrule
\texttt{no line break} & & no line break for paragraphs \\
\texttt{for paragraphs} & $\rightarrow$ \\
& & use empty line \\
\texttt{use empty line} \\
\midrule
\texttt{force\textbackslash\textbackslash line break} & $\rightarrow$ & force \\
& & line break \\
\midrule
\texttt{horizontal line} & $\rightarrow$ & horizontal line \\
\texttt{----} & & \hrulefill \\
\midrule
\texttt{|=a|=table|=header} & & \underline{a \enspace table \enspace header} \\
\texttt{|a|table|row} & $\rightarrow$ & a \enspace table \enspace row \\
\texttt{|b|table|row} & & b \enspace table \enspace row \\
\midrule
\texttt{\{\{\{} \\
\texttt{unformatted} & $\rightarrow$ & \texttt{unformatted} \\
\texttt{code} & & \texttt{code} \\
\texttt{\}\}\}} \\
\midrule\addlinespace
\texttt{@ new article} & & {\Large 1.\ new article} \smallskip \\
\texttt{@ second article} & $\rightarrow$ & {\Large 2.\ second article} \smallskip \\
\texttt{@@ sub article} & & {\large 2.1.\ sub article} \\
\bottomrule
\end{tabular}
\normalfont\caption{Elements of the generic documentation markup language}
\label{tab:docmarkup}
\end{table}
}

\providecommand{\startchapter}[4]{
\documentclass[11pt,a4paper]{article}
\usepackage{booktabs}
\usepackage[format=hang,labelfont=bf]{caption}
\usepackage{changepage}
\usepackage[T1]{fontenc}
\usepackage[margin=2cm]{geometry}
\usepackage{hyperref}
\usepackage[american]{isodate}
\usepackage{lmodern}
\usepackage{longtable}
\usepackage{mathptmx}
\usepackage{microtype}
\usepackage[toc]{multitoc}
\usepackage{multirow}
\usepackage[all]{nowidow}
\usepackage{pdfcomment}
\usepackage{syntax}
\usepackage{tikz}
\usepackage[all]{xy}
\hypersetup{pdfborder={0 0 0},bookmarksnumbered=true,pdftitle={\ecs{}: #2},pdfauthor={Florian Negele},pdfsubject={\ecs{}},pdfkeywords={#1}}
\setlength{\grammarindent}{8em}\setlength{\grammarparsep}{0.2ex}
\setlength{\columnsep}{2em}
\newcommand{\prefix}{}
\newcounter{instruction}
\bibliographystyle{unsrt}
\renewcommand{\index}[2][]{}
\renewcommand{\arraystretch}{1.05}
\renewcommand{\floatpagefraction}{0.7}
\renewcommand{\syntleft}{\itshape}\renewcommand{\syntright}{}
\title{\vspace{-5ex}\Huge{\ecs{}}\medskip\hrule}
\author{\huge{#2}}
\date{\medskip\version}
\newif\ifbook\bookfalse
\pagestyle{headings}
\frenchspacing
\begin{document}
\maketitle\thispagestyle{empty}\noindent#4\setlength{\columnseprule}{0.4pt}\tableofcontents\setlength{\columnseprule}{0pt}\vfill\pagebreak[3]\null\vfill\bigskip\noindent
\parbox{\textwidth-4em}{\license The contents of this \documentation{} are part of the \href{manual}{\ecs{} User Manual}~\cite{manual} and correspond to Chapter ``\href{manual\##3}{#1}''.\alignright\mbox{\today}}
\parbox{4em}{\flushright\ecslogo{3em}}
\clearpage
}

\providecommand{\concludechapter}{
\vfill\pagebreak[3]\null\vfill
\thispagestyle{myheadings}\markright{REFERENCES}
\noindent\begin{minipage}{\textwidth}\begin{multicols}{2}[\section*{References}]
\renewcommand{\section}[2]{}\small\bibliography{references}
\end{multicols}\end{minipage}\end{document}
}

\providecommand{\startpresentation}[2]{
\documentclass[14pt,aspectratio=43,usepdftitle=false]{beamer}
\usepackage{booktabs}
\usepackage{etex}
\usepackage{multicol}
\usepackage{tikz}
\usepackage[all]{xy}
\bibliographystyle{unsrt}
\setlength{\columnsep}{1em}
\setlength{\leftmargini}{1em}
\setbeamercolor{title}{fg=black}
\setbeamercolor{structure}{fg=darkgray}
\setbeamercolor{bibliography item}{fg=darkgray}
\setbeamerfont{title}{series=\bfseries}
\setbeamerfont{subtitle}{series=\normalfont}
\setbeamerfont*{frametitle}{parent=title}
\setbeamerfont{block title}{series=\bfseries}
\setbeamerfont*{framesubtitle}{parent=subtitle}
\setbeamersize{text margin left=1em,text margin right=1em}
\setbeamertemplate{navigation symbols}{}
\setbeamertemplate{itemize item}[circle]{}
\setbeamertemplate{bibliography item}[triangle]{}
\setbeamertemplate{bibliography entry author}{\usebeamercolor[fg]{bibliography item}}
\setbeamertemplate{frametitle}{\medskip\usebeamerfont{frametitle}\color{gray}\raisebox{-2.5ex}[0ex][0ex]{\rule{0.1em}{4.5ex}}}
\addtobeamertemplate{frametitle}{}{\hspace{0.4em}\usebeamercolor[fg]{title}\insertframetitle\par\vspace{0.2ex}\hspace{0.5em}\usebeamerfont{framesubtitle}\insertframesubtitle}
\hypersetup{pdfborder={0 0 0},bookmarksnumbered=true,bookmarksopen=true,bookmarksopenlevel=0,pdftitle={\ecs{}: #1},pdfauthor={Florian Negele},pdfsubject={\ecs{}},pdfkeywords={#1}}
\renewcommand{\flowgraph}[1]{\resizebox{\textwidth}{!}{$$\xymatrix{##1}$$}}
\title{\ecs{}\medskip\hrule\medskip}
\institute{\shadowedecslogo{5em}{30}{15}}
\date{\version}
\subtitle{#1}
\begin{document}
\begin{frame}[plain]\titlepage\nocite{manual}\end{frame}
\begin{frame}{Contents}{#1}\begin{center}\tableofcontents\end{center}\end{frame}
}

\providecommand{\concludepresentation}{
\begin{frame}{References}\begin{footnotesize}\setlength{\columnseprule}{0.4pt}\begin{multicols}{2}\bibliography{references}\end{multicols}\end{footnotesize}\end{frame}
\end{document}
}

\providecommand{\startbook}[1]{
\documentclass[10pt,paper=17cm:24cm,DIV=13,twoside=semi,headings=normal,numbers=noendperiod,cleardoublepage=plain]{scrbook}
\usepackage{atveryend}
\usepackage{booktabs}
\usepackage{caption}
\usepackage{changepage}
\usepackage[T1]{fontenc}
\usepackage{imakeidx}
\usepackage{hyperref}
\usepackage[american]{isodate}
\usepackage{lmodern}
\usepackage{longtable}
\usepackage{mathptmx}
\usepackage[final]{microtype}
\usepackage{multicol}
\usepackage{multirow}
\usepackage[all]{nowidow}
\usepackage{pdfcomment}
\usepackage{scrlayer-scrpage}
\usepackage{setspace}
\usepackage{syntax}
\usepackage[eventxtindent=4pt,oddtxtexdent=4pt]{thumbs}
\usepackage{tikz}
\usepackage[all]{xy}
\hyphenation{Micro-Blaze Open-Cores Open-RISC Power-PC}
\hypersetup{pdfborder={0 0 0},bookmarksnumbered=true,bookmarksopen=true,bookmarksopenlevel=0,pdftitle={\ecs{}: #1},pdfauthor={Florian Negele},pdfsubject={\ecs{}},pdfkeywords={#1}}
\setlength{\grammarindent}{8em}\setlength{\grammarparsep}{0.7ex}
\setkomafont{captionlabel}{\usekomafont{descriptionlabel}}
\renewcommand{\arraystretch}{1.05}\setstretch{1.1}
\renewcommand{\chapterformat}{\thechapter\autodot\enskip\raisebox{-1ex}[0ex][0ex]{\color{gray}\rule{0.1em}{3.5ex}}\enskip}
\renewcommand{\startchapter}[4]{\hypertarget{##3}{\chapter{##1}}\label{##3}##4\addthumb{##1}{\LARGE\sffamily\bfseries\thechapter}{white}{gray}\renewcommand{\prefix}{##3}}
\renewcommand{\concludechapter}{\clearpage{\stopthumb\cleardoublepage}}
\renewcommand{\syntleft}{\itshape}\renewcommand{\syntright}{}
\renewcommand{\floatpagefraction}{0.7}
\renewcommand{\partheademptypage}{}
\DeclareMicrotypeAlias{lmss}{cmr}
\newcommand{\prefix}{}
\newcounter{instruction}
\bibliographystyle{unsrt}
\newif\ifbook\booktrue
\makeindex[intoc,title=Index]
\makeindex[intoc,name=tools,title=Index of Tools,columns=3]
\makeindex[intoc,name=library,title=Index of Library Names]
\makeindex[intoc,name=runtime,title=Index of Runtime Support]
\makeindex[intoc,name=environment,title=Index of Target Environments]
\indexsetup{toclevel=chapter,headers={\indexname}{\indexname}}
\frenchspacing
\begin{document}
\pagenumbering{alph}
\begin{titlepage}\centering
\huge\sffamily\null\vfill\textbf{\ecs{}}\bigskip\hrule\bigskip#1
\normalsize\normalfont\vfill\vfill\shadowedecslogo{10em}{30}{15}
\large\vfill\vfill\version
\end{titlepage}
\null\vfill
\thispagestyle{empty}
\noindent\today\par\medskip
\license A copy of this license is included in Appendix~\ref{fdl} on page~\pageref{fdl}.
All product names used herein are for identification purposes only and may be trademarks of their respective companies.
\concludechapter
\frontmatter
\setcounter{tocdepth}{1}
\tableofcontents
\setcounter{tocdepth}{2}
\concludechapter
\listoffigures
\concludechapter
\listoftables
\concludechapter
}

\providecommand{\concludebook}{
\backmatter
\addtocontents{toc}{\protect\setcounter{tocdepth}{-1}}
\phantomsection\addcontentsline{toc}{part}{Bibliography}
\bibliography{references}
\concludechapter
\phantomsection\addcontentsline{toc}{part}{Indexes}
\printindex
\concludechapter
\indexprologue{\label{idx:tools}}
\printindex[tools]
\concludechapter
\printindex[library]
\concludechapter
\indexprologue{\label{idx:runtime}}
\printindex[runtime]
\concludechapter
\indexprologue{\label{idx:environment}}
\printindex[environment]
\concludechapter
\pagestyle{empty}\pagenumbering{Alph}\null\clearpage
\null\vfill\centering\ecslogo{4em}\par\medskip\license
\end{document}
}

% chapter references

\providecommand{\seedocumentationref}{}\renewcommand{\seedocumentationref}[3]{#1, see \Documentation{}~\documentationref{#2}{#3}. }
\providecommand{\seeinterface}{}\renewcommand{\seeinterface}{\ifbook See \Documentation{}~\documentationref{interface}{User Interface} for more information about the common user interface of all of these tools. \fi}
\providecommand{\seeguide}{}\renewcommand{\seeguide}{\seedocumentationref{For basic examples of using some of these tools in practice}{guide}{User Guide}}
\providecommand{\seecpp}{}\renewcommand{\seecpp}{\seedocumentationref{For more information about the \cpp{} programming language and its implementation by the \ecs{}}{cpp}{User Manual for \cpp{}}}
\providecommand{\seefalse}{}\renewcommand{\seefalse}{\seedocumentationref{For more information about the FALSE programming language and its implementation by the \ecs{}}{false}{User Manual for FALSE}}
\providecommand{\seeoberon}{}\renewcommand{\seeoberon}{\seedocumentationref{For more information about the Oberon programming language and its implementation by the \ecs{}}{oberon}{User Manual for Oberon}}
\providecommand{\seeassembly}{}\renewcommand{\seeassembly}{\seedocumentationref{For more information about the generic assembly language and how to use it}{assembly}{Generic Assembly Language Specification}}
\providecommand{\seeamd}{}\renewcommand{\seeamd}{\seedocumentationref{For more information about how the \ecs{} supports the AMD64 hardware architecture}{amd64}{AMD64 Hardware Architecture Support}}
\providecommand{\seearm}{}\renewcommand{\seearm}{\seedocumentationref{For more information about how the \ecs{} supports the ARM hardware architecture}{arm}{ARM Hardware Architecture Support}}
\providecommand{\seeavr}{}\renewcommand{\seeavr}{\seedocumentationref{For more information about how the \ecs{} supports the AVR hardware architecture}{avr}{AVR Hardware Architecture Support}}
\providecommand{\seeavrtt}{}\renewcommand{\seeavrtt}{\seedocumentationref{For more information about how the \ecs{} supports the AVR32 hardware architecture}{avr32}{AVR32 Hardware Architecture Support}}
\providecommand{\seemabk}{}\renewcommand{\seemabk}{\seedocumentationref{For more information about how the \ecs{} supports the M68000 hardware architecture}{m68k}{M68000 Hardware Architecture Support}}
\providecommand{\seemibl}{}\renewcommand{\seemibl}{\seedocumentationref{For more information about how the \ecs{} supports the MicroBlaze hardware architecture}{mibl}{MicroBlaze Hardware Architecture Support}}
\providecommand{\seemips}{}\renewcommand{\seemips}{\seedocumentationref{For more information about how the \ecs{} supports the MIPS32 and MIPS64 hardware architectures}{mips}{MIPS Hardware Architecture Support}}
\providecommand{\seemmix}{}\renewcommand{\seemmix}{\seedocumentationref{For more information about how the \ecs{} supports the MMIX hardware architecture}{mmix}{MMIX Hardware Architecture Support}}
\providecommand{\seeorok}{}\renewcommand{\seeorok}{\seedocumentationref{For more information about how the \ecs{} supports the OpenRISC 1000 hardware architecture}{or1k}{OpenRISC 1000 Hardware Architecture Support}}
\providecommand{\seeppc}{}\renewcommand{\seeppc}{\seedocumentationref{For more information about how the \ecs{} supports the PowerPC hardware architecture}{ppc}{PowerPC Hardware Architecture Support}}
\providecommand{\seerisc}{}\renewcommand{\seerisc}{\seedocumentationref{For more information about how the \ecs{} supports the RISC hardware architecture}{risc}{RISC Hardware Architecture Support}}
\providecommand{\seewasm}{}\renewcommand{\seewasm}{\seedocumentationref{For more information about how the \ecs{} supports the WebAssembly architecture}{wasm}{WebAssembly Architecture Support}}
\providecommand{\seedocumentation}{}\renewcommand{\seedocumentation}{\seedocumentationref{For more information about generic documentations and their generation by the \ecs{}}{documentation}{Generic Documentation Generation}}
\providecommand{\seedebugging}{}\renewcommand{\seedebugging}{\seedocumentationref{For more information about debugging information and its representation}{debugging}{Debugging Information Representation}}
\providecommand{\seecode}{}\renewcommand{\seecode}{\seedocumentationref{For more information about intermediate code and its purpose}{code}{Intermediate Code Representation}}
\providecommand{\seeobject}{}\renewcommand{\seeobject}{\seedocumentationref{For more information about object files and their purpose}{object}{Object File Representation}}

% generic documentation tools

\providecommand{\docprint}{
\toolsection{docprint} is a pretty printer for generic documentations.
It reformats generic documentations and writes it to the standard output stream.
\debuggingtool
\flowgraph{\resource{generic\\documentation} \ar[r] & \toolbox{docprint} \ar[r] & \resource{generic\\documentation}}
\seedocumentation
}

\providecommand{\doccheck}{
\toolsection{doccheck} is a syntactic and semantic checker for generic documentations.
It just performs syntactic and semantic checks on generic documentations and writes its diagnostic messages to the standard error stream.
\debuggingtool
\flowgraph{\resource{generic\\documentation} \ar[r] & \toolbox{doccheck} \ar[r] & \resource{diagnostic\\messages}}
\seedocumentation
}

\providecommand{\dochtml}{
\toolsection{dochtml} is an HTML documentation generator for generic documentations.
It processes several generic documentations and assembles all information therein into an HTML document.
\debuggingtool
\flowgraph{\resource{generic\\documentation} \ar[r] & \toolbox{dochtml} \ar[r] & \resource{HTML\\document}}
\seedocumentation
}

\providecommand{\doclatex}{
\toolsection{doclatex} is a Latex documentation generator for generic documentations.
It processes several generic documentations and assembles all information therein into a Latex document.
\debuggingtool
\flowgraph{\resource{generic\\documentation} \ar[r] & \toolbox{doclatex} \ar[r] & \resource{Latex\\document}}
\seedocumentation
}

% intermediate code tools

\providecommand{\cdcheck}{
\toolsection{cdcheck} is a syntactic and semantic checker for intermediate code.
It just performs syntactic and semantic checks on programs written in intermediate code and writes its diagnostic messages to the standard error stream.
\debuggingtool
\flowgraph{\resource{intermediate\\code} \ar[r] & \toolbox{cdcheck} \ar[r] & \resource{diagnostic\\messages}}
\seeassembly\seecode
}

\providecommand{\cdopt}{
\toolsection{cdopt} is an optimizer for intermediate code.
It performs various optimizations on programs written in intermediate code and writes the result to the standard output stream.
\debuggingtool
\flowgraph{\resource{intermediate\\code} \ar[r] & \toolbox{cdopt} \ar[r] & \resource{optimized\\code}}
\seeassembly\seecode
}

\providecommand{\cdrun}{
\toolsection{cdrun} is an interpreter for intermediate code.
It processes and executes programs written in intermediate code.
The following code sections are predefined and have the usual semantics:
\texttt{abort}, \texttt{\_Exit}, \texttt{fflush}, \texttt{floor}, \texttt{fputc}, \texttt{free}, \texttt{getchar}, \texttt{malloc}, and \texttt{putchar}.
Diagnostic messages about invalid operations include the name of the executed code section and the index of the erroneous instruction.
\debuggingtool
\flowgraph{\resource{intermediate\\code} \ar[r] & \toolbox{cdrun} \ar@/u/[r] & \resource{input/\\output} \ar@/d/[l]}
\seeassembly\seecode
}

\providecommand{\cdamda}{
\toolsection{cdamd16} is a compiler for intermediate code targeting the AMD64 hardware architecture.
It generates machine code for AMD64 processors from programs written in intermediate code and stores it in corresponding object files.
The compiler generates machine code for the 16-bit operating mode defined by the AMD64 architecture.
It also creates a debugging information file as well as an assembly file containing a listing of the generated machine code.
\debuggingtool
\flowgraph{\resource{intermediate\\code} \ar[r] & \toolbox{cdamd16} \ar[r] \ar[d] \ar[rd] & \resource{object file} \\ & \resource{assembly\\listing} & \resource{debugging\\information}}
\seeassembly\seeamd\seeobject\seecode\seedebugging
}

\providecommand{\cdamdb}{
\toolsection{cdamd32} is a compiler for intermediate code targeting the AMD64 hardware architecture.
It generates machine code for AMD64 processors from programs written in intermediate code and stores it in corresponding object files.
The compiler generates machine code for the 32-bit operating mode defined by the AMD64 architecture.
It also creates a debugging information file as well as an assembly file containing a listing of the generated machine code.
\debuggingtool
\flowgraph{\resource{intermediate\\code} \ar[r] & \toolbox{cdamd32} \ar[r] \ar[d] \ar[rd] & \resource{object file} \\ & \resource{assembly\\listing} & \resource{debugging\\information}}
\seeassembly\seeamd\seeobject\seecode\seedebugging
}

\providecommand{\cdamdc}{
\toolsection{cdamd64} is a compiler for intermediate code targeting the AMD64 hardware architecture.
It generates machine code for AMD64 processors from programs written in intermediate code and stores it in corresponding object files.
The compiler generates machine code for the 64-bit operating mode defined by the AMD64 architecture.
It also creates a debugging information file as well as an assembly file containing a listing of the generated machine code.
\debuggingtool
\flowgraph{\resource{intermediate\\code} \ar[r] & \toolbox{cdamd64} \ar[r] \ar[d] \ar[rd] & \resource{object file} \\ & \resource{assembly\\listing} & \resource{debugging\\information}}
\seeassembly\seeamd\seeobject\seecode\seedebugging
}

\providecommand{\cdarma}{
\toolsection{cdarma32} is a compiler for intermediate code targeting the ARM hardware architecture.
It generates machine code for ARM processors executing A32 instructions from programs written in intermediate code and stores it in corresponding object files.
It also creates a debugging information file as well as an assembly file containing a listing of the generated machine code.
\debuggingtool
\flowgraph{\resource{intermediate\\code} \ar[r] & \toolbox{cdarma32} \ar[r] \ar[d] \ar[rd] & \resource{object file} \\ & \resource{assembly\\listing} & \resource{debugging\\information}}
\seeassembly\seearm\seeobject\seecode\seedebugging
}

\providecommand{\cdarmb}{
\toolsection{cdarma64} is a compiler for intermediate code targeting the ARM hardware architecture.
It generates machine code for ARM processors executing A64 instructions from programs written in intermediate code and stores it in corresponding object files.
It also creates a debugging information file as well as an assembly file containing a listing of the generated machine code.
\debuggingtool
\flowgraph{\resource{intermediate\\code} \ar[r] & \toolbox{cdarma64} \ar[r] \ar[d] \ar[rd] & \resource{object file} \\ & \resource{assembly\\listing} & \resource{debugging\\information}}
\seeassembly\seearm\seeobject\seecode\seedebugging
}

\providecommand{\cdarmc}{
\toolsection{cdarmt32} is a compiler for intermediate code targeting the ARM hardware architecture.
It generates machine code for ARM processors without floating-point extension executing T32 instructions from programs written in intermediate code and stores it in corresponding object files.
It also creates a debugging information file as well as an assembly file containing a listing of the generated machine code.
\debuggingtool
\flowgraph{\resource{intermediate\\code} \ar[r] & \toolbox{cdarmt32} \ar[r] \ar[d] \ar[rd] & \resource{object file} \\ & \resource{assembly\\listing} & \resource{debugging\\information}}
\seeassembly\seearm\seeobject\seecode\seedebugging
}

\providecommand{\cdarmcfpe}{
\toolsection{cdarmt32fpe} is a compiler for intermediate code targeting the ARM hardware architecture.
It generates machine code for ARM processors with floating-point extension executing T32 instructions from programs written in intermediate code and stores it in corresponding object files.
It also creates a debugging information file as well as an assembly file containing a listing of the generated machine code.
\debuggingtool
\flowgraph{\resource{intermediate\\code} \ar[r] & \toolbox{cdarmt32fpe} \ar[r] \ar[d] \ar[rd] & \resource{object file} \\ & \resource{assembly\\listing} & \resource{debugging\\information}}
\seeassembly\seearm\seeobject\seecode\seedebugging
}

\providecommand{\cdavr}{
\toolsection{cdavr} is a compiler for intermediate code targeting the AVR hardware architecture.
It generates machine code for AVR processors from programs written in intermediate code and stores it in corresponding object files.
It also creates a debugging information file as well as an assembly file containing a listing of the generated machine code.
\debuggingtool
\flowgraph{\resource{intermediate\\code} \ar[r] & \toolbox{cdavr} \ar[r] \ar[d] \ar[rd] & \resource{object file} \\ & \resource{assembly\\listing} & \resource{debugging\\information}}
\seeassembly\seeavr\seeobject\seecode\seedebugging
}

\providecommand{\cdavrtt}{
\toolsection{cdavr32} is a compiler for intermediate code targeting the AVR32 hardware architecture.
It generates machine code for AVR32 processors from programs written in intermediate code and stores it in corresponding object files.
It also creates a debugging information file as well as an assembly file containing a listing of the generated machine code.
\debuggingtool
\flowgraph{\resource{intermediate\\code} \ar[r] & \toolbox{cdavr32} \ar[r] \ar[d] \ar[rd] & \resource{object file} \\ & \resource{assembly\\listing} & \resource{debugging\\information}}
\seeassembly\seeavrtt\seeobject\seecode\seedebugging
}

\providecommand{\cdmabk}{
\toolsection{cdm68k} is a compiler for intermediate code targeting the M68000 hardware architecture.
It generates machine code for M68000 processors from programs written in intermediate code and stores it in corresponding object files.
It also creates a debugging information file as well as an assembly file containing a listing of the generated machine code.
\debuggingtool
\flowgraph{\resource{intermediate\\code} \ar[r] & \toolbox{cdm68k} \ar[r] \ar[d] \ar[rd] & \resource{object file} \\ & \resource{assembly\\listing} & \resource{debugging\\information}}
\seeassembly\seemabk\seeobject\seecode\seedebugging
}

\providecommand{\cdmibl}{
\toolsection{cdmibl} is a compiler for intermediate code targeting the MicroBlaze hardware architecture.
It generates machine code for MicroBlaze processors from programs written in intermediate code and stores it in corresponding object files.
It also creates a debugging information file as well as an assembly file containing a listing of the generated machine code.
\debuggingtool
\flowgraph{\resource{intermediate\\code} \ar[r] & \toolbox{cdmibl} \ar[r] \ar[d] \ar[rd] & \resource{object file} \\ & \resource{assembly\\listing} & \resource{debugging\\information}}
\seeassembly\seemibl\seeobject\seecode\seedebugging
}

\providecommand{\cdmipsa}{
\toolsection{cdmips32} is a compiler for intermediate code targeting the MIPS32 hardware architecture.
It generates machine code for MIPS32 processors from programs written in intermediate code and stores it in corresponding object files.
It also creates a debugging information file as well as an assembly file containing a listing of the generated machine code.
\debuggingtool
\flowgraph{\resource{intermediate\\code} \ar[r] & \toolbox{cdmips32} \ar[r] \ar[d] \ar[rd] & \resource{object file} \\ & \resource{assembly\\listing} & \resource{debugging\\information}}
\seeassembly\seemips\seeobject\seecode\seedebugging
}

\providecommand{\cdmipsb}{
\toolsection{cdmips64} is a compiler for intermediate code targeting the MIPS64 hardware architecture.
It generates machine code for MIPS64 processors from programs written in intermediate code and stores it in corresponding object files.
It also creates a debugging information file as well as an assembly file containing a listing of the generated machine code.
\debuggingtool
\flowgraph{\resource{intermediate\\code} \ar[r] & \toolbox{cdmips64} \ar[r] \ar[d] \ar[rd] & \resource{object file} \\ & \resource{assembly\\listing} & \resource{debugging\\information}}
\seeassembly\seemips\seeobject\seecode\seedebugging
}

\providecommand{\cdmmix}{
\toolsection{cdmmix} is a compiler for intermediate code targeting the MMIX hardware architecture.
It generates machine code for MMIX processors from programs written in intermediate code and stores it in corresponding object files.
It also creates a debugging information file as well as an assembly file containing a listing of the generated machine code.
\debuggingtool
\flowgraph{\resource{intermediate\\code} \ar[r] & \toolbox{cdmmix} \ar[r] \ar[d] \ar[rd] & \resource{object file} \\ & \resource{assembly\\listing} & \resource{debugging\\information}}
\seeassembly\seemmix\seeobject\seecode\seedebugging
}

\providecommand{\cdorok}{
\toolsection{cdor1k} is a compiler for intermediate code targeting the OpenRISC 1000 hardware architecture.
It generates machine code for OpenRISC 1000 processors from programs written in intermediate code and stores it in corresponding object files.
It also creates a debugging information file as well as an assembly file containing a listing of the generated machine code.
\debuggingtool
\flowgraph{\resource{intermediate\\code} \ar[r] & \toolbox{cdor1k} \ar[r] \ar[d] \ar[rd] & \resource{object file} \\ & \resource{assembly\\listing} & \resource{debugging\\information}}
\seeassembly\seeorok\seeobject\seecode\seedebugging
}

\providecommand{\cdppca}{
\toolsection{cdppc32} is a compiler for intermediate code targeting the PowerPC hardware architecture.
It generates machine code for PowerPC processors from programs written in intermediate code and stores it in corresponding object files.
The compiler generates machine code for the 32-bit operating mode defined by the PowerPC architecture.
It also creates a debugging information file as well as an assembly file containing a listing of the generated machine code.
\debuggingtool
\flowgraph{\resource{intermediate\\code} \ar[r] & \toolbox{cdppc32} \ar[r] \ar[d] \ar[rd] & \resource{object file} \\ & \resource{assembly\\listing} & \resource{debugging\\information}}
\seeassembly\seeppc\seeobject\seecode\seedebugging
}

\providecommand{\cdppcb}{
\toolsection{cdppc64} is a compiler for intermediate code targeting the PowerPC hardware architecture.
It generates machine code for PowerPC processors from programs written in intermediate code and stores it in corresponding object files.
The compiler generates machine code for the 64-bit operating mode defined by the PowerPC architecture.
It also creates a debugging information file as well as an assembly file containing a listing of the generated machine code.
\debuggingtool
\flowgraph{\resource{intermediate\\code} \ar[r] & \toolbox{cdppc64} \ar[r] \ar[d] \ar[rd] & \resource{object file} \\ & \resource{assembly\\listing} & \resource{debugging\\information}}
\seeassembly\seeppc\seeobject\seecode\seedebugging
}

\providecommand{\cdrisc}{
\toolsection{cdrisc} is a compiler for intermediate code targeting the RISC hardware architecture.
It generates machine code for RISC processors from programs written in intermediate code and stores it in corresponding object files.
It also creates a debugging information file as well as an assembly file containing a listing of the generated machine code.
\debuggingtool
\flowgraph{\resource{intermediate\\code} \ar[r] & \toolbox{cdrisc} \ar[r] \ar[d] \ar[rd] & \resource{object file} \\ & \resource{assembly\\listing} & \resource{debugging\\information}}
\seeassembly\seerisc\seeobject\seecode\seedebugging
}

\providecommand{\cdwasm}{
\toolsection{cdwasm} is a compiler for intermediate code targeting the WebAssembly architecture.
It generates machine code for WebAssembly targets from programs written in intermediate code and stores it in corresponding object files.
It also creates a debugging information file as well as an assembly file containing a listing of the generated machine code.
\debuggingtool
\flowgraph{\resource{intermediate\\code} \ar[r] & \toolbox{cdwasm} \ar[r] \ar[d] \ar[rd] & \resource{object file} \\ & \resource{assembly\\listing} & \resource{debugging\\information}}
\seeassembly\seewasm\seeobject\seecode\seedebugging
}

% C++ tools

\providecommand{\cppprep}{
\toolsection{cppprep} is a preprocessor for the \cpp{} programming language.
It preprocesses source code according to the rules of \cpp{} and writes it to the standard output stream.
Only the macro names \texttt{\_\_DATE\_\_}, \texttt{\_\_FILE\_\_}, \texttt{\_\_LINE\_\_}, and \texttt{\_\_TIME\_\_} are predefined.
\flowgraph{\resource{\cpp{} or other\\source code} \ar[r] & \toolbox{cppprep} \ar[r] & \resource{preprocessed\\source code} \\ & \variable{ECSINCLUDE} \ar[u]}
\seecpp
}

\providecommand{\cppprint}{
\toolsection{cppprint} is a pretty printer for the \cpp{} programming language.
It reformats the source code of \cpp{} programs and writes it to the standard output stream.
\flowgraph{\resource{\cpp{}\\source code} \ar[r] & \toolbox{cppprint} \ar[r] & \resource{reformatted\\source code} \\ & \variable{ECSINCLUDE} \ar[u]}
\seecpp
}

\providecommand{\cppcheck}{
\toolsection{cppcheck} is a syntactic and semantic checker for the \cpp{} programming language.
It just performs syntactic and semantic checks on \cpp{} programs and writes its diagnostic messages to the standard error stream.
\flowgraph{\resource{\cpp{}\\source code} \ar[r] & \toolbox{cppcheck} \ar[r] & \resource{diagnostic\\messages} \\ & \variable{ECSINCLUDE} \ar[u]}
\seecpp
}

\providecommand{\cppdump}{
\toolsection{cppdump} is a serializer for the \cpp{} programming language.
It dumps the complete internal representation of programs written in \cpp{} into an XML document.
\debuggingtool
\flowgraph{\resource{\cpp{}\\source code} \ar[r] & \toolbox{cppdump} \ar[r] & \resource{internal\\representation} \\ & \variable{ECSINCLUDE} \ar[u]}
\seecpp
}

\providecommand{\cpprun}{
\toolsection{cpprun} is an interpreter for the \cpp{} programming language.
It processes and executes programs written in \cpp{}.
The macro \texttt{\_\_run\_\_} is predefined in order to enable programmers to identify this tool while interpreting.
\flowgraph{\resource{\cpp{}\\source code} \ar[r] & \toolbox{cpprun} \ar@/u/[r] & \resource{input/\\output} \ar@/d/[l] \\ & \variable{ECSINCLUDE} \ar[u]}
\seecpp
}

\providecommand{\cppdoc}{
\toolsection{cppdoc} is a generic documentation generator for the \cpp{} programming language.
It processes several \cpp{} source files and assembles all information therein into a generic documentation.
\debuggingtool
\flowgraph{\resource{\cpp{}\\source code} \ar[r] & \toolbox{cppdoc} \ar[r] & \resource{generic\\documentation} \\ & \variable{ECSINCLUDE} \ar[u]}
\seecpp\seedocumentation
}

\providecommand{\cpphtml}{
\toolsection{cpphtml} is an HTML documentation generator for the \cpp{} programming language.
It processes several \cpp{} source files and assembles all information therein into an HTML document.
\flowgraph{\resource{\cpp{}\\source code} \ar[r] & \toolbox{cpphtml} \ar[r] & \resource{HTML\\document} \\ & \variable{ECSINCLUDE} \ar[u]}
\seecpp\seedocumentation
}

\providecommand{\cpplatex}{
\toolsection{cpplatex} is a Latex documentation generator for the \cpp{} programming language.
It processes several \cpp{} source files and assembles all information therein into a Latex document.
\flowgraph{\resource{\cpp{}\\source code} \ar[r] & \toolbox{cpplatex} \ar[r] & \resource{Latex\\document} \\ & \variable{ECSINCLUDE} \ar[u]}
\seecpp\seedocumentation
}

\providecommand{\cppcode}{
\toolsection{cppcode} is an intermediate code generator for the \cpp{} programming language.
It generates intermediate code from programs written in \cpp{} and stores it in corresponding assembly files.
The macro \texttt{\_\_code\_\_} is predefined in order to enable programmers to identify this tool while generating intermediate code.
Programs generated with this tool require additional runtime support that is stored in the \file{cpp\-code\-run} library file.
\debuggingtool
\flowgraph{\resource{\cpp{}\\source code} \ar[r] & \toolbox{cppcode} \ar[r] & \resource{intermediate\\code} \\ & \variable{ECSINCLUDE} \ar[u]}
\seecpp\seeassembly\seecode
}

\providecommand{\cppamda}{
\toolsection{cppamd16} is a compiler for the \cpp{} programming language targeting the AMD64 hardware architecture.
It generates machine code for AMD64 processors from programs written in \cpp{} and stores it in corresponding object files.
The compiler generates machine code for the 16-bit operating mode defined by the AMD64 architecture.
For debugging purposes, it also creates a debugging information file as well as an assembly file containing a listing of the generated machine code.
The macro \texttt{\_\_amd16\_\_} is predefined in order to enable programmers to identify this tool and its target architecture while compiling.
Programs generated with this compiler require additional runtime support that is stored in the \file{cpp\-amd16\-run} library file.
\flowgraph{\resource{\cpp{}\\source code} \ar[r] & \toolbox{cppamd16} \ar[r] \ar[d] \ar[rd] & \resource{object file} \\ \variable{ECSINCLUDE} \ar[ru] & \resource{debugging\\information} & \resource{assembly\\listing}}
\seecpp\seeassembly\seeamd\seeobject\seedebugging
}

\providecommand{\cppamdb}{
\toolsection{cppamd32} is a compiler for the \cpp{} programming language targeting the AMD64 hardware architecture.
It generates machine code for AMD64 processors from programs written in \cpp{} and stores it in corresponding object files.
The compiler generates machine code for the 32-bit operating mode defined by the AMD64 architecture.
For debugging purposes, it also creates a debugging information file as well as an assembly file containing a listing of the generated machine code.
The macro \texttt{\_\_amd32\_\_} is predefined in order to enable programmers to identify this tool and its target architecture while compiling.
Programs generated with this compiler require additional runtime support that is stored in the \file{cpp\-amd32\-run} library file.
\flowgraph{\resource{\cpp{}\\source code} \ar[r] & \toolbox{cppamd32} \ar[r] \ar[d] \ar[rd] & \resource{object file} \\ \variable{ECSINCLUDE} \ar[ru] & \resource{debugging\\information} & \resource{assembly\\listing}}
\seecpp\seeassembly\seeamd\seeobject\seedebugging
}

\providecommand{\cppamdc}{
\toolsection{cppamd64} is a compiler for the \cpp{} programming language targeting the AMD64 hardware architecture.
It generates machine code for AMD64 processors from programs written in \cpp{} and stores it in corresponding object files.
The compiler generates machine code for the 64-bit operating mode defined by the AMD64 architecture.
For debugging purposes, it also creates a debugging information file as well as an assembly file containing a listing of the generated machine code.
The macro \texttt{\_\_amd64\_\_} is predefined in order to enable programmers to identify this tool and its target architecture while compiling.
Programs generated with this compiler require additional runtime support that is stored in the \file{cpp\-amd64\-run} library file.
\flowgraph{\resource{\cpp{}\\source code} \ar[r] & \toolbox{cppamd64} \ar[r] \ar[d] \ar[rd] & \resource{object file} \\ \variable{ECSINCLUDE} \ar[ru] & \resource{debugging\\information} & \resource{assembly\\listing}}
\seecpp\seeassembly\seeamd\seeobject\seedebugging
}

\providecommand{\cpparma}{
\toolsection{cpparma32} is a compiler for the \cpp{} programming language targeting the ARM hardware architecture.
It generates machine code for ARM processors executing A32 instructions from programs written in \cpp{} and stores it in corresponding object files.
For debugging purposes, it also creates a debugging information file as well as an assembly file containing a listing of the generated machine code.
The macro \texttt{\_\_arma32\_\_} is predefined in order to enable programmers to identify this tool and its target architecture while compiling.
Programs generated with this compiler require additional runtime support that is stored in the \file{cpp\-arma32\-run} library file.
\flowgraph{\resource{\cpp{}\\source code} \ar[r] & \toolbox{cpparma32} \ar[r] \ar[d] \ar[rd] & \resource{object file} \\ \variable{ECSINCLUDE} \ar[ru] & \resource{debugging\\information} & \resource{assembly\\listing}}
\seecpp\seeassembly\seearm\seeobject\seedebugging
}

\providecommand{\cpparmb}{
\toolsection{cpparma64} is a compiler for the \cpp{} programming language targeting the ARM hardware architecture.
It generates machine code for ARM processors executing A64 instructions from programs written in \cpp{} and stores it in corresponding object files.
For debugging purposes, it also creates a debugging information file as well as an assembly file containing a listing of the generated machine code.
The macro \texttt{\_\_arma64\_\_} is predefined in order to enable programmers to identify this tool and its target architecture while compiling.
Programs generated with this compiler require additional runtime support that is stored in the \file{cpp\-arma64\-run} library file.
\flowgraph{\resource{\cpp{}\\source code} \ar[r] & \toolbox{cpparma64} \ar[r] \ar[d] \ar[rd] & \resource{object file} \\ \variable{ECSINCLUDE} \ar[ru] & \resource{debugging\\information} & \resource{assembly\\listing}}
\seecpp\seeassembly\seearm\seeobject\seedebugging
}

\providecommand{\cpparmc}{
\toolsection{cpparmt32} is a compiler for the \cpp{} programming language targeting the ARM hardware architecture.
It generates machine code for ARM processors without floating-point extension executing T32 instructions from programs written in \cpp{} and stores it in corresponding object files.
For debugging purposes, it also creates a debugging information file as well as an assembly file containing a listing of the generated machine code.
The macro \texttt{\_\_armt32\_\_} is predefined in order to enable programmers to identify this tool and its target architecture while compiling.
Programs generated with this compiler require additional runtime support that is stored in the \file{cpp\-armt32\-run} library file.
\flowgraph{\resource{\cpp{}\\source code} \ar[r] & \toolbox{cpparmt32} \ar[r] \ar[d] \ar[rd] & \resource{object file} \\ \variable{ECSINCLUDE} \ar[ru] & \resource{debugging\\information} & \resource{assembly\\listing}}
\seecpp\seeassembly\seearm\seeobject\seedebugging
}

\providecommand{\cpparmcfpe}{
\toolsection{cpparmt32fpe} is a compiler for the \cpp{} programming language targeting the ARM hardware architecture.
It generates machine code for ARM processors with floating-point extension executing T32 instructions from programs written in \cpp{} and stores it in corresponding object files.
For debugging purposes, it also creates a debugging information file as well as an assembly file containing a listing of the generated machine code.
The macro \texttt{\_\_armt32fpe\_\_} is predefined in order to enable programmers to identify this tool and its target architecture while compiling.
Programs generated with this compiler require additional runtime support that is stored in the \file{cpp\-armt32\-fpe\-run} library file.
\flowgraph{\resource{\cpp{}\\source code} \ar[r] & \toolbox{cpparmt32fpe} \ar[r] \ar[d] \ar[rd] & \resource{object file} \\ \variable{ECSINCLUDE} \ar[ru] & \resource{debugging\\information} & \resource{assembly\\listing}}
\seecpp\seeassembly\seearm\seeobject\seedebugging
}

\providecommand{\cppavr}{
\toolsection{cppavr} is a compiler for the \cpp{} programming language targeting the AVR hardware architecture.
It generates machine code for AVR processors from programs written in \cpp{} and stores it in corresponding object files.
For debugging purposes, it also creates a debugging information file as well as an assembly file containing a listing of the generated machine code.
The macro \texttt{\_\_avr\_\_} is predefined in order to enable programmers to identify this tool and its target architecture while compiling.
Programs generated with this compiler require additional runtime support that is stored in the \file{cpp\-avr\-run} library file.
\flowgraph{\resource{\cpp{}\\source code} \ar[r] & \toolbox{cppavr} \ar[r] \ar[d] \ar[rd] & \resource{object file} \\ \variable{ECSINCLUDE} \ar[ru] & \resource{debugging\\information} & \resource{assembly\\listing}}
\seecpp\seeassembly\seeavr\seeobject\seedebugging
}

\providecommand{\cppavrtt}{
\toolsection{cppavr32} is a compiler for the \cpp{} programming language targeting the AVR32 hardware architecture.
It generates machine code for AVR32 processors from programs written in \cpp{} and stores it in corresponding object files.
For debugging purposes, it also creates a debugging information file as well as an assembly file containing a listing of the generated machine code.
The macro \texttt{\_\_avr32\_\_} is predefined in order to enable programmers to identify this tool and its target architecture while compiling.
Programs generated with this compiler require additional runtime support that is stored in the \file{cpp\-avr32\-run} library file.
\flowgraph{\resource{\cpp{}\\source code} \ar[r] & \toolbox{cppavr32} \ar[r] \ar[d] \ar[rd] & \resource{object file} \\ \variable{ECSINCLUDE} \ar[ru] & \resource{debugging\\information} & \resource{assembly\\listing}}
\seecpp\seeassembly\seeavrtt\seeobject\seedebugging
}

\providecommand{\cppmabk}{
\toolsection{cppm68k} is a compiler for the \cpp{} programming language targeting the M68000 hardware architecture.
It generates machine code for M68000 processors from programs written in \cpp{} and stores it in corresponding object files.
For debugging purposes, it also creates a debugging information file as well as an assembly file containing a listing of the generated machine code.
The macro \texttt{\_\_m68k\_\_} is predefined in order to enable programmers to identify this tool and its target architecture while compiling.
Programs generated with this compiler require additional runtime support that is stored in the \file{cpp\-m68k\-run} library file.
\flowgraph{\resource{\cpp{}\\source code} \ar[r] & \toolbox{cppm68k} \ar[r] \ar[d] \ar[rd] & \resource{object file} \\ \variable{ECSINCLUDE} \ar[ru] & \resource{debugging\\information} & \resource{assembly\\listing}}
\seecpp\seeassembly\seemabk\seeobject\seedebugging
}

\providecommand{\cppmibl}{
\toolsection{cppmibl} is a compiler for the \cpp{} programming language targeting the MicroBlaze hardware architecture.
It generates machine code for MicroBlaze processors from programs written in \cpp{} and stores it in corresponding object files.
For debugging purposes, it also creates a debugging information file as well as an assembly file containing a listing of the generated machine code.
The macro \texttt{\_\_mibl\_\_} is predefined in order to enable programmers to identify this tool and its target architecture while compiling.
Programs generated with this compiler require additional runtime support that is stored in the \file{cpp\-mibl\-run} library file.
\flowgraph{\resource{\cpp{}\\source code} \ar[r] & \toolbox{cppmibl} \ar[r] \ar[d] \ar[rd] & \resource{object file} \\ \variable{ECSINCLUDE} \ar[ru] & \resource{debugging\\information} & \resource{assembly\\listing}}
\seecpp\seeassembly\seemibl\seeobject\seedebugging
}

\providecommand{\cppmipsa}{
\toolsection{cppmips32} is a compiler for the \cpp{} programming language targeting the MIPS32 hardware architecture.
It generates machine code for MIPS32 processors from programs written in \cpp{} and stores it in corresponding object files.
For debugging purposes, it also creates a debugging information file as well as an assembly file containing a listing of the generated machine code.
The macro \texttt{\_\_mips32\_\_} is predefined in order to enable programmers to identify this tool and its target architecture while compiling.
Programs generated with this compiler require additional runtime support that is stored in the \file{cpp\-mips32\-run} library file.
\flowgraph{\resource{\cpp{}\\source code} \ar[r] & \toolbox{cppmips32} \ar[r] \ar[d] \ar[rd] & \resource{object file} \\ \variable{ECSINCLUDE} \ar[ru] & \resource{debugging\\information} & \resource{assembly\\listing}}
\seecpp\seeassembly\seemips\seeobject\seedebugging
}

\providecommand{\cppmipsb}{
\toolsection{cppmips64} is a compiler for the \cpp{} programming language targeting the MIPS64 hardware architecture.
It generates machine code for MIPS64 processors from programs written in \cpp{} and stores it in corresponding object files.
For debugging purposes, it also creates a debugging information file as well as an assembly file containing a listing of the generated machine code.
The macro \texttt{\_\_mips64\_\_} is predefined in order to enable programmers to identify this tool and its target architecture while compiling.
Programs generated with this compiler require additional runtime support that is stored in the \file{cpp\-mips64\-run} library file.
\flowgraph{\resource{\cpp{}\\source code} \ar[r] & \toolbox{cppmips64} \ar[r] \ar[d] \ar[rd] & \resource{object file} \\ \variable{ECSINCLUDE} \ar[ru] & \resource{debugging\\information} & \resource{assembly\\listing}}
\seecpp\seeassembly\seemips\seeobject\seedebugging
}

\providecommand{\cppmmix}{
\toolsection{cppmmix} is a compiler for the \cpp{} programming language targeting the MMIX hardware architecture.
It generates machine code for MMIX processors from programs written in \cpp{} and stores it in corresponding object files.
For debugging purposes, it also creates a debugging information file as well as an assembly file containing a listing of the generated machine code.
The macro \texttt{\_\_mmix\_\_} is predefined in order to enable programmers to identify this tool and its target architecture while compiling.
Programs generated with this compiler require additional runtime support that is stored in the \file{cpp\-mmix\-run} library file.
\flowgraph{\resource{\cpp{}\\source code} \ar[r] & \toolbox{cppmmix} \ar[r] \ar[d] \ar[rd] & \resource{object file} \\ \variable{ECSINCLUDE} \ar[ru] & \resource{debugging\\information} & \resource{assembly\\listing}}
\seecpp\seeassembly\seemmix\seeobject\seedebugging
}

\providecommand{\cpporok}{
\toolsection{cppor1k} is a compiler for the \cpp{} programming language targeting the OpenRISC 1000 hardware architecture.
It generates machine code for OpenRISC 1000 processors from programs written in \cpp{} and stores it in corresponding object files.
For debugging purposes, it also creates a debugging information file as well as an assembly file containing a listing of the generated machine code.
The macro \texttt{\_\_or1k\_\_} is predefined in order to enable programmers to identify this tool and its target architecture while compiling.
Programs generated with this compiler require additional runtime support that is stored in the \file{cpp\-or1k\-run} library file.
\flowgraph{\resource{\cpp{}\\source code} \ar[r] & \toolbox{cppor1k} \ar[r] \ar[d] \ar[rd] & \resource{object file} \\ \variable{ECSINCLUDE} \ar[ru] & \resource{debugging\\information} & \resource{assembly\\listing}}
\seecpp\seeassembly\seeorok\seeobject\seedebugging
}

\providecommand{\cppppca}{
\toolsection{cppppc32} is a compiler for the \cpp{} programming language targeting the PowerPC hardware architecture.
It generates machine code for PowerPC processors from programs written in \cpp{} and stores it in corresponding object files.
The compiler generates machine code for the 32-bit operating mode defined by the PowerPC architecture.
For debugging purposes, it also creates a debugging information file as well as an assembly file containing a listing of the generated machine code.
The macro \texttt{\_\_ppc32\_\_} is predefined in order to enable programmers to identify this tool and its target architecture while compiling.
Programs generated with this compiler require additional runtime support that is stored in the \file{cpp\-ppc32\-run} library file.
\flowgraph{\resource{\cpp{}\\source code} \ar[r] & \toolbox{cppppc32} \ar[r] \ar[d] \ar[rd] & \resource{object file} \\ \variable{ECSINCLUDE} \ar[ru] & \resource{debugging\\information} & \resource{assembly\\listing}}
\seecpp\seeassembly\seeppc\seeobject\seedebugging
}

\providecommand{\cppppcb}{
\toolsection{cppppc64} is a compiler for the \cpp{} programming language targeting the PowerPC hardware architecture.
It generates machine code for PowerPC processors from programs written in \cpp{} and stores it in corresponding object files.
The compiler generates machine code for the 64-bit operating mode defined by the PowerPC architecture.
For debugging purposes, it also creates a debugging information file as well as an assembly file containing a listing of the generated machine code.
The macro \texttt{\_\_ppc64\_\_} is predefined in order to enable programmers to identify this tool and its target architecture while compiling.
Programs generated with this compiler require additional runtime support that is stored in the \file{cpp\-ppc64\-run} library file.
\flowgraph{\resource{\cpp{}\\source code} \ar[r] & \toolbox{cppppc64} \ar[r] \ar[d] \ar[rd] & \resource{object file} \\ \variable{ECSINCLUDE} \ar[ru] & \resource{debugging\\information} & \resource{assembly\\listing}}
\seecpp\seeassembly\seeppc\seeobject\seedebugging
}

\providecommand{\cpprisc}{
\toolsection{cpprisc} is a compiler for the \cpp{} programming language targeting the RISC hardware architecture.
It generates machine code for RISC processors from programs written in \cpp{} and stores it in corresponding object files.
For debugging purposes, it also creates a debugging information file as well as an assembly file containing a listing of the generated machine code.
The macro \texttt{\_\_risc\_\_} is predefined in order to enable programmers to identify this tool and its target architecture while compiling.
Programs generated with this compiler require additional runtime support that is stored in the \file{cpp\-risc\-run} library file.
\flowgraph{\resource{\cpp{}\\source code} \ar[r] & \toolbox{cpprisc} \ar[r] \ar[d] \ar[rd] & \resource{object file} \\ \variable{ECSINCLUDE} \ar[ru] & \resource{debugging\\information} & \resource{assembly\\listing}}
\seecpp\seeassembly\seerisc\seeobject\seedebugging
}

\providecommand{\cppwasm}{
\toolsection{cppwasm} is a compiler for the \cpp{} programming language targeting the WebAssembly architecture.
It generates machine code for WebAssembly targets from programs written in \cpp{} and stores it in corresponding object files.
For debugging purposes, it also creates a debugging information file as well as an assembly file containing a listing of the generated machine code.
The macro \texttt{\_\_wasm\_\_} is predefined in order to enable programmers to identify this tool and its target architecture while compiling.
Programs generated with this compiler require additional runtime support that is stored in the \file{cpp\-wasm\-run} library file.
\flowgraph{\resource{\cpp{}\\source code} \ar[r] & \toolbox{cppwasm} \ar[r] \ar[d] \ar[rd] & \resource{object file} \\ \variable{ECSINCLUDE} \ar[ru] & \resource{debugging\\information} & \resource{assembly\\listing}}
\seecpp\seeassembly\seewasm\seeobject\seedebugging
}

% FALSE tools

\providecommand{\falprint}{
\toolsection{falprint} is a pretty printer for the FALSE programming language.
It reformats the source code of FALSE programs and writes it to the standard output stream.
\flowgraph{\resource{FALSE\\source code} \ar[r] & \toolbox{falprint} \ar[r] & \resource{reformatted\\source code}}
\seefalse
}

\providecommand{\falcheck}{
\toolsection{falcheck} is a syntactic and semantic checker for the FALSE programming language.
It just performs syntactic and semantic checks on FALSE programs and writes its diagnostic messages to the standard error stream.
\flowgraph{\resource{FALSE\\source code} \ar[r] & \toolbox{falcheck} \ar[r] & \resource{diagnostic\\messages}}
\seefalse
}

\providecommand{\faldump}{
\toolsection{faldump} is a serializer for the FALSE programming language.
It dumps the complete internal representation of programs written in FALSE into an XML document.
\debuggingtool
\flowgraph{\resource{FALSE\\source code} \ar[r] & \toolbox{faldump} \ar[r] & \resource{internal\\representation}}
\seefalse
}

\providecommand{\falrun}{
\toolsection{falrun} is an interpreter for the FALSE programming language.
It processes and executes programs written in FALSE\@.
\flowgraph{\resource{FALSE\\source code} \ar[r] & \toolbox{falrun} \ar@/u/[r] & \resource{input/\\output} \ar@/d/[l]}
\seefalse
}

\providecommand{\falcpp}{
\toolsection{falcpp} is a transpiler for the FALSE programming language.
It translates programs written in FALSE into \cpp{} programs and stores them in corresponding source files.
\flowgraph{\resource{FALSE\\source code} \ar[r] & \toolbox{falcpp} \ar[r] & \resource{\cpp{}\\source file}}
\seefalse\seecpp
}

\providecommand{\falcode}{
\toolsection{falcode} is an intermediate code generator for the FALSE programming language.
It generates intermediate code from programs written in FALSE and stores it in corresponding assembly files.
\debuggingtool
\flowgraph{\resource{FALSE\\source code} \ar[r] & \toolbox{falcode} \ar[r] & \resource{intermediate\\code}}
\seefalse\seeassembly\seecode
}

\providecommand{\falamda}{
\toolsection{falamd16} is a compiler for the FALSE programming language targeting the AMD64 hardware architecture.
It generates machine code for AMD64 processors from programs written in FALSE and stores it in corresponding object files.
The compiler generates machine code for the 16-bit operating mode defined by the AMD64 architecture.
\flowgraph{\resource{FALSE\\source code} \ar[r] & \toolbox{falamd16} \ar[r] & \resource{object file}}
\seefalse\seeamd\seeobject
}

\providecommand{\falamdb}{
\toolsection{falamd32} is a compiler for the FALSE programming language targeting the AMD64 hardware architecture.
It generates machine code for AMD64 processors from programs written in FALSE and stores it in corresponding object files.
The compiler generates machine code for the 32-bit operating mode defined by the AMD64 architecture.
\flowgraph{\resource{FALSE\\source code} \ar[r] & \toolbox{falamd32} \ar[r] & \resource{object file}}
\seefalse\seeamd\seeobject
}

\providecommand{\falamdc}{
\toolsection{falamd64} is a compiler for the FALSE programming language targeting the AMD64 hardware architecture.
It generates machine code for AMD64 processors from programs written in FALSE and stores it in corresponding object files.
The compiler generates machine code for the 64-bit operating mode defined by the AMD64 architecture.
\flowgraph{\resource{FALSE\\source code} \ar[r] & \toolbox{falamd64} \ar[r] & \resource{object file}}
\seefalse\seeamd\seeobject
}

\providecommand{\falarma}{
\toolsection{falarma32} is a compiler for the FALSE programming language targeting the ARM hardware architecture.
It generates machine code for ARM processors executing A32 instructions from programs written in FALSE and stores it in corresponding object files.
\flowgraph{\resource{FALSE\\source code} \ar[r] & \toolbox{falarma32} \ar[r] & \resource{object file}}
\seefalse\seearm\seeobject
}

\providecommand{\falarmb}{
\toolsection{falarma64} is a compiler for the FALSE programming language targeting the ARM hardware architecture.
It generates machine code for ARM processors executing A64 instructions from programs written in FALSE and stores it in corresponding object files.
\flowgraph{\resource{FALSE\\source code} \ar[r] & \toolbox{falarma64} \ar[r] & \resource{object file}}
\seefalse\seearm\seeobject
}

\providecommand{\falarmc}{
\toolsection{falarmt32} is a compiler for the FALSE programming language targeting the ARM hardware architecture.
It generates machine code for ARM processors without floating-point extension executing T32 instructions from programs written in FALSE and stores it in corresponding object files.
\flowgraph{\resource{FALSE\\source code} \ar[r] & \toolbox{falarmt32} \ar[r] & \resource{object file}}
\seefalse\seearm\seeobject
}

\providecommand{\falarmcfpe}{
\toolsection{falarmt32fpe} is a compiler for the FALSE programming language targeting the ARM hardware architecture.
It generates machine code for ARM processors with floating-point extension executing T32 instructions from programs written in FALSE and stores it in corresponding object files.
\flowgraph{\resource{FALSE\\source code} \ar[r] & \toolbox{falarmt32fpe} \ar[r] & \resource{object file}}
\seefalse\seearm\seeobject
}

\providecommand{\falavr}{
\toolsection{falavr} is a compiler for the FALSE programming language targeting the AVR hardware architecture.
It generates machine code for AVR processors from programs written in FALSE and stores it in corresponding object files.
\flowgraph{\resource{FALSE\\source code} \ar[r] & \toolbox{falavr} \ar[r] & \resource{object file}}
\seefalse\seeavr\seeobject
}

\providecommand{\falavrtt}{
\toolsection{falavr32} is a compiler for the FALSE programming language targeting the AVR32 hardware architecture.
It generates machine code for AVR32 processors from programs written in FALSE and stores it in corresponding object files.
\flowgraph{\resource{FALSE\\source code} \ar[r] & \toolbox{falavr32} \ar[r] & \resource{object file}}
\seefalse\seeavrtt\seeobject
}

\providecommand{\falmabk}{
\toolsection{falm68k} is a compiler for the FALSE programming language targeting the M68000 hardware architecture.
It generates machine code for M68000 processors from programs written in FALSE and stores it in corresponding object files.
\flowgraph{\resource{FALSE\\source code} \ar[r] & \toolbox{falm68k} \ar[r] & \resource{object file}}
\seefalse\seemabk\seeobject
}

\providecommand{\falmibl}{
\toolsection{falmibl} is a compiler for the FALSE programming language targeting the MicroBlaze hardware architecture.
It generates machine code for MicroBlaze processors from programs written in FALSE and stores it in corresponding object files.
\flowgraph{\resource{FALSE\\source code} \ar[r] & \toolbox{falmibl} \ar[r] & \resource{object file}}
\seefalse\seemibl\seeobject
}

\providecommand{\falmipsa}{
\toolsection{falmips32} is a compiler for the FALSE programming language targeting the MIPS32 hardware architecture.
It generates machine code for MIPS32 processors from programs written in FALSE and stores it in corresponding object files.
\flowgraph{\resource{FALSE\\source code} \ar[r] & \toolbox{falmips32} \ar[r] & \resource{object file}}
\seefalse\seemips\seeobject
}

\providecommand{\falmipsb}{
\toolsection{falmips64} is a compiler for the FALSE programming language targeting the MIPS64 hardware architecture.
It generates machine code for MIPS64 processors from programs written in FALSE and stores it in corresponding object files.
\flowgraph{\resource{FALSE\\source code} \ar[r] & \toolbox{falmips64} \ar[r] & \resource{object file}}
\seefalse\seemips\seeobject
}

\providecommand{\falmmix}{
\toolsection{falmmix} is a compiler for the FALSE programming language targeting the MMIX hardware architecture.
It generates machine code for MMIX processors from programs written in FALSE and stores it in corresponding object files.
\flowgraph{\resource{FALSE\\source code} \ar[r] & \toolbox{falmmix} \ar[r] & \resource{object file}}
\seefalse\seemmix\seeobject
}

\providecommand{\falorok}{
\toolsection{falor1k} is a compiler for the FALSE programming language targeting the OpenRISC 1000 hardware architecture.
It generates machine code for OpenRISC 1000 processors from programs written in FALSE and stores it in corresponding object files.
\flowgraph{\resource{FALSE\\source code} \ar[r] & \toolbox{falor1k} \ar[r] & \resource{object file}}
\seefalse\seeorok\seeobject
}

\providecommand{\falppca}{
\toolsection{falppc32} is a compiler for the FALSE programming language targeting the PowerPC hardware architecture.
It generates machine code for PowerPC processors from programs written in FALSE and stores it in corresponding object files.
The compiler generates machine code for the 32-bit operating mode defined by the PowerPC architecture.
\flowgraph{\resource{FALSE\\source code} \ar[r] & \toolbox{falppc32} \ar[r] & \resource{object file}}
\seefalse\seeppc\seeobject
}

\providecommand{\falppcb}{
\toolsection{falppc64} is a compiler for the FALSE programming language targeting the PowerPC hardware architecture.
It generates machine code for PowerPC processors from programs written in FALSE and stores it in corresponding object files.
The compiler generates machine code for the 64-bit operating mode defined by the PowerPC architecture.
\flowgraph{\resource{FALSE\\source code} \ar[r] & \toolbox{falppc64} \ar[r] & \resource{object file}}
\seefalse\seeppc\seeobject
}

\providecommand{\falrisc}{
\toolsection{falrisc} is a compiler for the FALSE programming language targeting the RISC hardware architecture.
It generates machine code for RISC processors from programs written in FALSE and stores it in corresponding object files.
\flowgraph{\resource{FALSE\\source code} \ar[r] & \toolbox{falrisc} \ar[r] & \resource{object file}}
\seefalse\seerisc\seeobject
}

\providecommand{\falwasm}{
\toolsection{falwasm} is a compiler for the FALSE programming language targeting the WebAssembly architecture.
It generates machine code for WebAssembly targets from programs written in FALSE and stores it in corresponding object files.
\flowgraph{\resource{FALSE\\source code} \ar[r] & \toolbox{falwasm} \ar[r] & \resource{object file}}
\seefalse\seewasm\seeobject
}

% Oberon tools

\providecommand{\obprint}{
\toolsection{obprint} is a pretty printer for the Oberon programming language.
It reformats the source code of Oberon modules and writes it to the standard output stream.
\flowgraph{\resource{Oberon\\source code} \ar[r] & \toolbox{obprint} \ar[r] & \resource{reformatted\\source code}}
\seeoberon
}

\providecommand{\obcheck}{
\toolsection{obcheck} is a syntactic and semantic checker for the Oberon programming language.
It just performs syntactic and semantic checks on Oberon modules and writes its diagnostic messages to the standard error stream.
In addition, it stores the interface of each module in a symbol file which is required when other modules import the module.
\flowgraph{\resource{Oberon\\source code} \ar[r] & \toolbox{obcheck} \ar[r] \ar@/l/[d] & \resource{diagnostic\\messages} \\ \variable{ECSIMPORT} \ar[ru] & \resource{symbol\\files} \ar@/r/[u]}
\seeoberon
}

\providecommand{\obdump}{
\toolsection{obdump} is a serializer for the Oberon programming language.
It dumps the complete internal representation of modules written in Oberon into an XML document.
\debuggingtool
\flowgraph{\resource{Oberon\\source code} \ar[r] & \toolbox{obdump} \ar[r] \ar@/l/[d] & \resource{internal\\representation} \\ \variable{ECSIMPORT} \ar[ru] & \resource{symbol\\files} \ar@/r/[u]}
\seeoberon
}

\providecommand{\obrun}{
\toolsection{obrun} is an interpreter for the Oberon programming language.
It processes and executes modules written in Oberon.
This tool does neither generate nor process symbol files while interpreting modules.
If a module is imported by another one, its filename has to be named before the other one in the list of command-line arguments.
\flowgraph{\resource{Oberon\\source code} \ar[r] & \toolbox{obrun} \ar@/u/[r] & \resource{input/\\output} \ar@/d/[l]}
\seeoberon
}

\providecommand{\obcpp}{
\toolsection{obcpp} is a transpiler for the Oberon programming language.
It translates programs written in Oberon into \cpp{} programs and stores them in corresponding source and header files.
In addition, it stores the interface of each module in a symbol file which is required when other modules import the module.
The same interface is provided by the generated header file which can be used in other parts of the \cpp{} program.
\flowgraph{\resource{Oberon\\source code} \ar[r] & \toolbox{obcpp} \ar[r] \ar@/l/[d] \ar[rd] & \resource{\cpp{}\\source file} \\ \variable{ECSIMPORT} \ar[ru] & \resource{symbol\\files} \ar@/r/[u] & \resource{\cpp{}\\header file}}
\seeoberon\seecpp
}

\providecommand{\obdoc}{
\toolsection{obdoc} is a generic documentation generator for the Oberon programming language.
It processes several Oberon modules and assembles all information therein into a generic documentation.
In addition, it stores the interface of each module in a symbol file which is required when other modules import the module.
\debuggingtool
\flowgraph{\resource{Oberon\\source code} \ar[r] & \toolbox{obdoc} \ar[r] \ar@/l/[d] & \resource{generic\\documentation} \\ \variable{ECSIMPORT} \ar[ru] & \resource{symbol\\files} \ar@/r/[u]}
\seeoberon\seedocumentation
}

\providecommand{\obhtml}{
\toolsection{obhtml} is an HTML documentation generator for the Oberon programming language.
It processes several Oberon modules and assembles all information therein into an HTML document.
In addition, it stores the interface of each module in a symbol file which is required when other modules import the module.
\flowgraph{\resource{Oberon\\source code} \ar[r] & \toolbox{obhtml} \ar[r] \ar@/l/[d] & \resource{HTML\\document} \\ \variable{ECSIMPORT} \ar[ru] & \resource{symbol\\files} \ar@/r/[u]}
\seeoberon\seedocumentation
}

\providecommand{\oblatex}{
\toolsection{oblatex} is a Latex documentation generator for the Oberon programming language.
It processes several Oberon modules and assembles all information therein into a Latex document.
In addition, it stores the interface of each module in a symbol file which is required when other modules import the module.
\flowgraph{\resource{Oberon\\source code} \ar[r] & \toolbox{oblatex} \ar[r] \ar@/l/[d] & \resource{Latex\\document} \\ \variable{ECSIMPORT} \ar[ru] & \resource{symbol\\files} \ar@/r/[u]}
\seeoberon\seedocumentation
}

\providecommand{\obcode}{
\toolsection{obcode} is an intermediate code generator for the Oberon programming language.
It generates intermediate code from modules written in Oberon and stores it in corresponding assembly files.
In addition, it stores the interface of each module in a symbol file which is required when other modules import the module.
Programs generated with this tool require additional runtime support that is stored in the \file{ob\-code\-run} library file.
\debuggingtool
\flowgraph{\resource{Oberon\\source code} \ar[r] & \toolbox{obcode} \ar[r] \ar@/l/[d] & \resource{intermediate\\code} \\ \variable{ECSIMPORT} \ar[ru] & \resource{symbol\\files} \ar@/r/[u]}
\seeoberon\seeassembly\seecode
}

\providecommand{\obamda}{
\toolsection{obamd16} is a compiler for the Oberon programming language targeting the AMD64 hardware architecture.
It generates machine code for AMD64 processors from modules written in Oberon and stores it in corresponding object files.
The compiler generates machine code for the 16-bit operating mode defined by the AMD64 architecture.
For debugging purposes, it also creates a debugging information file as well as an assembly file containing a listing of the generated machine code.
In addition, it stores the interface of each module in a symbol file which is required when other modules import the module.
Programs generated with this compiler require additional runtime support that is stored in the \file{ob\-amd16\-run} library file.
\flowgraph{\resource{Oberon\\source code} \ar[r] & \toolbox{obamd16} \ar[r] \ar@/l/[d] \ar[rd] & \resource{object file} \\ \variable{ECSIMPORT} \ar[ru] & \resource{symbol\\files} \ar@/r/[u] & \resource{debugging\\information}}
\seeoberon\seeassembly\seeamd\seeobject\seedebugging
}

\providecommand{\obamdb}{
\toolsection{obamd32} is a compiler for the Oberon programming language targeting the AMD64 hardware architecture.
It generates machine code for AMD64 processors from modules written in Oberon and stores it in corresponding object files.
The compiler generates machine code for the 32-bit operating mode defined by the AMD64 architecture.
For debugging purposes, it also creates a debugging information file as well as an assembly file containing a listing of the generated machine code.
In addition, it stores the interface of each module in a symbol file which is required when other modules import the module.
Programs generated with this compiler require additional runtime support that is stored in the \file{ob\-amd32\-run} library file.
\flowgraph{\resource{Oberon\\source code} \ar[r] & \toolbox{obamd32} \ar[r] \ar@/l/[d] \ar[rd] & \resource{object file} \\ \variable{ECSIMPORT} \ar[ru] & \resource{symbol\\files} \ar@/r/[u] & \resource{debugging\\information}}
\seeoberon\seeassembly\seeamd\seeobject\seedebugging
}

\providecommand{\obamdc}{
\toolsection{obamd64} is a compiler for the Oberon programming language targeting the AMD64 hardware architecture.
It generates machine code for AMD64 processors from modules written in Oberon and stores it in corresponding object files.
The compiler generates machine code for the 64-bit operating mode defined by the AMD64 architecture.
For debugging purposes, it also creates a debugging information file as well as an assembly file containing a listing of the generated machine code.
In addition, it stores the interface of each module in a symbol file which is required when other modules import the module.
Programs generated with this compiler require additional runtime support that is stored in the \file{ob\-amd64\-run} library file.
\flowgraph{\resource{Oberon\\source code} \ar[r] & \toolbox{obamd64} \ar[r] \ar@/l/[d] \ar[rd] & \resource{object file} \\ \variable{ECSIMPORT} \ar[ru] & \resource{symbol\\files} \ar@/r/[u] & \resource{debugging\\information}}
\seeoberon\seeassembly\seeamd\seeobject\seedebugging
}

\providecommand{\obarma}{
\toolsection{obarma32} is a compiler for the Oberon programming language targeting the ARM hardware architecture.
It generates machine code for ARM processors executing A32 instructions from modules written in Oberon and stores it in corresponding object files.
For debugging purposes, it also creates a debugging information file as well as an assembly file containing a listing of the generated machine code.
In addition, it stores the interface of each module in a symbol file which is required when other modules import the module.
Programs generated with this compiler require additional runtime support that is stored in the \file{ob\-arma32\-run} library file.
\flowgraph{\resource{Oberon\\source code} \ar[r] & \toolbox{obarma32} \ar[r] \ar@/l/[d] \ar[rd] & \resource{object file} \\ \variable{ECSIMPORT} \ar[ru] & \resource{symbol\\files} \ar@/r/[u] & \resource{debugging\\information}}
\seeoberon\seeassembly\seearm\seeobject\seedebugging
}

\providecommand{\obarmb}{
\toolsection{obarma64} is a compiler for the Oberon programming language targeting the ARM hardware architecture.
It generates machine code for ARM processors executing A64 instructions from modules written in Oberon and stores it in corresponding object files.
For debugging purposes, it also creates a debugging information file as well as an assembly file containing a listing of the generated machine code.
In addition, it stores the interface of each module in a symbol file which is required when other modules import the module.
Programs generated with this compiler require additional runtime support that is stored in the \file{ob\-arma64\-run} library file.
\flowgraph{\resource{Oberon\\source code} \ar[r] & \toolbox{obarma64} \ar[r] \ar@/l/[d] \ar[rd] & \resource{object file} \\ \variable{ECSIMPORT} \ar[ru] & \resource{symbol\\files} \ar@/r/[u] & \resource{debugging\\information}}
\seeoberon\seeassembly\seearm\seeobject\seedebugging
}

\providecommand{\obarmc}{
\toolsection{obarmt32} is a compiler for the Oberon programming language targeting the ARM hardware architecture.
It generates machine code for ARM processors without floating-point extension executing T32 instructions from modules written in Oberon and stores it in corresponding object files.
For debugging purposes, it also creates a debugging information file as well as an assembly file containing a listing of the generated machine code.
In addition, it stores the interface of each module in a symbol file which is required when other modules import the module.
Programs generated with this compiler require additional runtime support that is stored in the \file{ob\-armt32\-run} library file.
\flowgraph{\resource{Oberon\\source code} \ar[r] & \toolbox{obarmt32} \ar[r] \ar@/l/[d] \ar[rd] & \resource{object file} \\ \variable{ECSIMPORT} \ar[ru] & \resource{symbol\\files} \ar@/r/[u] & \resource{debugging\\information}}
\seeoberon\seeassembly\seearm\seeobject\seedebugging
}

\providecommand{\obarmcfpe}{
\toolsection{obarmt32fpe} is a compiler for the Oberon programming language targeting the ARM hardware architecture.
It generates machine code for ARM processors with floating-point extension executing T32 instructions from modules written in Oberon and stores it in corresponding object files.
For debugging purposes, it also creates a debugging information file as well as an assembly file containing a listing of the generated machine code.
In addition, it stores the interface of each module in a symbol file which is required when other modules import the module.
Programs generated with this compiler require additional runtime support that is stored in the \file{ob\-armt32\-fpe\-run} library file.
\flowgraph{\resource{Oberon\\source code} \ar[r] & \toolbox{obarmt32fpe} \ar[r] \ar@/l/[d] \ar[rd] & \resource{object file} \\ \variable{ECSIMPORT} \ar[ru] & \resource{symbol\\files} \ar@/r/[u] & \resource{debugging\\information}}
\seeoberon\seeassembly\seearm\seeobject\seedebugging
}

\providecommand{\obavr}{
\toolsection{obavr} is a compiler for the Oberon programming language targeting the AVR hardware architecture.
It generates machine code for AVR processors from modules written in Oberon and stores it in corresponding object files.
For debugging purposes, it also creates a debugging information file as well as an assembly file containing a listing of the generated machine code.
In addition, it stores the interface of each module in a symbol file which is required when other modules import the module.
Programs generated with this compiler require additional runtime support that is stored in the \file{ob\-avr\-run} library file.
\flowgraph{\resource{Oberon\\source code} \ar[r] & \toolbox{obavr} \ar[r] \ar@/l/[d] \ar[rd] & \resource{object file} \\ \variable{ECSIMPORT} \ar[ru] & \resource{symbol\\files} \ar@/r/[u] & \resource{debugging\\information}}
\seeoberon\seeassembly\seeavr\seeobject\seedebugging
}

\providecommand{\obavrtt}{
\toolsection{obavr32} is a compiler for the Oberon programming language targeting the AVR32 hardware architecture.
It generates machine code for AVR32 processors from modules written in Oberon and stores it in corresponding object files.
For debugging purposes, it also creates a debugging information file as well as an assembly file containing a listing of the generated machine code.
In addition, it stores the interface of each module in a symbol file which is required when other modules import the module.
Programs generated with this compiler require additional runtime support that is stored in the \file{ob\-avr32\-run} library file.
\flowgraph{\resource{Oberon\\source code} \ar[r] & \toolbox{obavr32} \ar[r] \ar@/l/[d] \ar[rd] & \resource{object file} \\ \variable{ECSIMPORT} \ar[ru] & \resource{symbol\\files} \ar@/r/[u] & \resource{debugging\\information}}
\seeoberon\seeassembly\seeavrtt\seeobject\seedebugging
}

\providecommand{\obmabk}{
\toolsection{obm68k} is a compiler for the Oberon programming language targeting the M68000 hardware architecture.
It generates machine code for M68000 processors from modules written in Oberon and stores it in corresponding object files.
For debugging purposes, it also creates a debugging information file as well as an assembly file containing a listing of the generated machine code.
In addition, it stores the interface of each module in a symbol file which is required when other modules import the module.
Programs generated with this compiler require additional runtime support that is stored in the \file{ob\-m68k\-run} library file.
\flowgraph{\resource{Oberon\\source code} \ar[r] & \toolbox{obm68k} \ar[r] \ar@/l/[d] \ar[rd] & \resource{object file} \\ \variable{ECSIMPORT} \ar[ru] & \resource{symbol\\files} \ar@/r/[u] & \resource{debugging\\information}}
\seeoberon\seeassembly\seemabk\seeobject\seedebugging
}

\providecommand{\obmibl}{
\toolsection{obmibl} is a compiler for the Oberon programming language targeting the MicroBlaze hardware architecture.
It generates machine code for MicroBlaze processors from modules written in Oberon and stores it in corresponding object files.
For debugging purposes, it also creates a debugging information file as well as an assembly file containing a listing of the generated machine code.
In addition, it stores the interface of each module in a symbol file which is required when other modules import the module.
Programs generated with this compiler require additional runtime support that is stored in the \file{ob\-mibl\-run} library file.
\flowgraph{\resource{Oberon\\source code} \ar[r] & \toolbox{obmibl} \ar[r] \ar@/l/[d] \ar[rd] & \resource{object file} \\ \variable{ECSIMPORT} \ar[ru] & \resource{symbol\\files} \ar@/r/[u] & \resource{debugging\\information}}
\seeoberon\seeassembly\seemibl\seeobject\seedebugging
}

\providecommand{\obmipsa}{
\toolsection{obmips32} is a compiler for the Oberon programming language targeting the MIPS32 hardware architecture.
It generates machine code for MIPS32 processors from modules written in Oberon and stores it in corresponding object files.
For debugging purposes, it also creates a debugging information file as well as an assembly file containing a listing of the generated machine code.
In addition, it stores the interface of each module in a symbol file which is required when other modules import the module.
Programs generated with this compiler require additional runtime support that is stored in the \file{ob\-mips32\-run} library file.
\flowgraph{\resource{Oberon\\source code} \ar[r] & \toolbox{obmips32} \ar[r] \ar@/l/[d] \ar[rd] & \resource{object file} \\ \variable{ECSIMPORT} \ar[ru] & \resource{symbol\\files} \ar@/r/[u] & \resource{debugging\\information}}
\seeoberon\seeassembly\seemips\seeobject\seedebugging
}

\providecommand{\obmipsb}{
\toolsection{obmips64} is a compiler for the Oberon programming language targeting the MIPS64 hardware architecture.
It generates machine code for MIPS64 processors from modules written in Oberon and stores it in corresponding object files.
For debugging purposes, it also creates a debugging information file as well as an assembly file containing a listing of the generated machine code.
In addition, it stores the interface of each module in a symbol file which is required when other modules import the module.
Programs generated with this compiler require additional runtime support that is stored in the \file{ob\-mips64\-run} library file.
\flowgraph{\resource{Oberon\\source code} \ar[r] & \toolbox{obmips64} \ar[r] \ar@/l/[d] \ar[rd] & \resource{object file} \\ \variable{ECSIMPORT} \ar[ru] & \resource{symbol\\files} \ar@/r/[u] & \resource{debugging\\information}}
\seeoberon\seeassembly\seemips\seeobject\seedebugging
}

\providecommand{\obmmix}{
\toolsection{obmmix} is a compiler for the Oberon programming language targeting the MMIX hardware architecture.
It generates machine code for MMIX processors from modules written in Oberon and stores it in corresponding object files.
For debugging purposes, it also creates a debugging information file as well as an assembly file containing a listing of the generated machine code.
In addition, it stores the interface of each module in a symbol file which is required when other modules import the module.
Programs generated with this compiler require additional runtime support that is stored in the \file{ob\-mmix\-run} library file.
\flowgraph{\resource{Oberon\\source code} \ar[r] & \toolbox{obmmix} \ar[r] \ar@/l/[d] \ar[rd] & \resource{object file} \\ \variable{ECSIMPORT} \ar[ru] & \resource{symbol\\files} \ar@/r/[u] & \resource{debugging\\information}}
\seeoberon\seeassembly\seemmix\seeobject\seedebugging
}

\providecommand{\oborok}{
\toolsection{obor1k} is a compiler for the Oberon programming language targeting the OpenRISC 1000 hardware architecture.
It generates machine code for OpenRISC 1000 processors from modules written in Oberon and stores it in corresponding object files.
For debugging purposes, it also creates a debugging information file as well as an assembly file containing a listing of the generated machine code.
In addition, it stores the interface of each module in a symbol file which is required when other modules import the module.
Programs generated with this compiler require additional runtime support that is stored in the \file{ob\-or1k\-run} library file.
\flowgraph{\resource{Oberon\\source code} \ar[r] & \toolbox{obor1k} \ar[r] \ar@/l/[d] \ar[rd] & \resource{object file} \\ \variable{ECSIMPORT} \ar[ru] & \resource{symbol\\files} \ar@/r/[u] & \resource{debugging\\information}}
\seeoberon\seeassembly\seeorok\seeobject\seedebugging
}

\providecommand{\obppca}{
\toolsection{obppc32} is a compiler for the Oberon programming language targeting the PowerPC hardware architecture.
It generates machine code for PowerPC processors from modules written in Oberon and stores it in corresponding object files.
The compiler generates machine code for the 32-bit operating mode defined by the PowerPC architecture.
For debugging purposes, it also creates a debugging information file as well as an assembly file containing a listing of the generated machine code.
In addition, it stores the interface of each module in a symbol file which is required when other modules import the module.
Programs generated with this compiler require additional runtime support that is stored in the \file{ob\-ppc32\-run} library file.
\flowgraph{\resource{Oberon\\source code} \ar[r] & \toolbox{obppc32} \ar[r] \ar@/l/[d] \ar[rd] & \resource{object file} \\ \variable{ECSIMPORT} \ar[ru] & \resource{symbol\\files} \ar@/r/[u] & \resource{debugging\\information}}
\seeoberon\seeassembly\seeppc\seeobject\seedebugging
}

\providecommand{\obppcb}{
\toolsection{obppc64} is a compiler for the Oberon programming language targeting the PowerPC hardware architecture.
It generates machine code for PowerPC processors from modules written in Oberon and stores it in corresponding object files.
The compiler generates machine code for the 64-bit operating mode defined by the PowerPC architecture.
For debugging purposes, it also creates a debugging information file as well as an assembly file containing a listing of the generated machine code.
In addition, it stores the interface of each module in a symbol file which is required when other modules import the module.
Programs generated with this compiler require additional runtime support that is stored in the \file{ob\-ppc64\-run} library file.
\flowgraph{\resource{Oberon\\source code} \ar[r] & \toolbox{obppc64} \ar[r] \ar@/l/[d] \ar[rd] & \resource{object file} \\ \variable{ECSIMPORT} \ar[ru] & \resource{symbol\\files} \ar@/r/[u] & \resource{debugging\\information}}
\seeoberon\seeassembly\seeppc\seeobject\seedebugging
}

\providecommand{\obrisc}{
\toolsection{obrisc} is a compiler for the Oberon programming language targeting the RISC hardware architecture.
It generates machine code for RISC processors from modules written in Oberon and stores it in corresponding object files.
For debugging purposes, it also creates a debugging information file as well as an assembly file containing a listing of the generated machine code.
In addition, it stores the interface of each module in a symbol file which is required when other modules import the module.
Programs generated with this compiler require additional runtime support that is stored in the \file{ob\-risc\-run} library file.
\flowgraph{\resource{Oberon\\source code} \ar[r] & \toolbox{obrisc} \ar[r] \ar@/l/[d] \ar[rd] & \resource{object file} \\ \variable{ECSIMPORT} \ar[ru] & \resource{symbol\\files} \ar@/r/[u] & \resource{debugging\\information}}
\seeoberon\seeassembly\seerisc\seeobject\seedebugging
}

\providecommand{\obwasm}{
\toolsection{obwasm} is a compiler for the Oberon programming language targeting the WebAssembly architecture.
It generates machine code for WebAssembly targets from modules written in Oberon and stores it in corresponding object files.
For debugging purposes, it also creates a debugging information file as well as an assembly file containing a listing of the generated machine code.
In addition, it stores the interface of each module in a symbol file which is required when other modules import the module.
Programs generated with this compiler require additional runtime support that is stored in the \file{ob\-wasm\-run} library file.
\flowgraph{\resource{Oberon\\source code} \ar[r] & \toolbox{obwasm} \ar[r] \ar@/l/[d] \ar[rd] & \resource{object file} \\ \variable{ECSIMPORT} \ar[ru] & \resource{symbol\\files} \ar@/r/[u] & \resource{debugging\\information}}
\seeoberon\seeassembly\seewasm\seeobject\seedebugging
}

% converter tools

\providecommand{\dbgdwarf}{
\toolsection{dbgdwarf} is a DWARF debugging information converter tool.
It converts debugging information into the DWARF debugging data format and stores it in corresponding object files~\cite{dwarffile}.
The resulting debugging object files can be combined with runtime support that creates Executable and Linking Format (ELF) files~\cite{elffile}.
\flowgraph{\resource{debugging\\information} \ar[r] & \toolbox{dbgdwarf} \ar[r] & \resource{debugging\\object file}}
\seeobject\seedebugging
}

% assembler tools

\providecommand{\asmprint}{
\toolsection{asmprint} is a pretty printer for generic assembly code.
It reformats generic assembly code and writes it to the standard output stream.
\flowgraph{\resource{generic assembly\\source code} \ar[r] & \toolbox{asmprint} \ar[r] & \resource{reformatted\\source code}}
\seeassembly
}

\providecommand{\amdaasm}{
\toolsection{amd16asm} is an assembler for the AMD64 hardware architecture.
It translates assembly code into machine code for AMD64 processors and stores it in corresponding object files.
By default, the assembler generates machine code for the 16-bit operating mode defined by the AMD64 architecture.
\flowgraph{\resource{AMD16 assembly\\source code} \ar[r] & \toolbox{amd16asm} \ar[r] & \resource{object file}}
\seeassembly\seeamd\seeobject
}

\providecommand{\amdadism}{
\toolsection{amd16dism} is a disassembler for the AMD64 hardware architecture.
It translates machine code from object files targeting AMD64 processors into assembly code and writes it to the standard output stream.
It assumes that the machine code was generated for the 16-bit operating mode defined by the AMD64 architecture.
\flowgraph{\resource{object file} \ar[r] & \toolbox{amd16dism} \ar[r] & \resource{disassembly\\listing}}
\seeassembly\seeamd\seeobject
}

\providecommand{\amdbasm}{
\toolsection{amd32asm} is an assembler for the AMD64 hardware architecture.
It translates assembly code into machine code for AMD64 processors and stores it in corresponding object files.
By default, the assembler generates machine code for the 32-bit operating mode defined by the AMD64 architecture.
\flowgraph{\resource{AMD32 assembly\\source code} \ar[r] & \toolbox{amd32asm} \ar[r] & \resource{object file}}
\seeassembly\seeamd\seeobject
}

\providecommand{\amdbdism}{
\toolsection{amd32dism} is a disassembler for the AMD64 hardware architecture.
It translates machine code from object files targeting AMD64 processors into assembly code and writes it to the standard output stream.
It assumes that the machine code was generated for the 32-bit operating mode defined by the AMD64 architecture.
\flowgraph{\resource{object file} \ar[r] & \toolbox{amd32dism} \ar[r] & \resource{disassembly\\listing}}
\seeassembly\seeamd\seeobject
}

\providecommand{\amdcasm}{
\toolsection{amd64asm} is an assembler for the AMD64 hardware architecture.
It translates assembly code into machine code for AMD64 processors and stores it in corresponding object files.
By default, the assembler generates machine code for the 64-bit operating mode defined by the AMD64 architecture.
\flowgraph{\resource{AMD64 assembly\\source code} \ar[r] & \toolbox{amd64asm} \ar[r] & \resource{object file}}
\seeassembly\seeamd\seeobject
}

\providecommand{\amdcdism}{
\toolsection{amd64dism} is a disassembler for the AMD64 hardware architecture.
It translates machine code from object files targeting AMD64 processors into assembly code and writes it to the standard output stream.
It assumes that the machine code was generated for the 64-bit operating mode defined by the AMD64 architecture.
\flowgraph{\resource{object file} \ar[r] & \toolbox{amd64dism} \ar[r] & \resource{disassembly\\listing}}
\seeassembly\seeamd\seeobject
}

\providecommand{\armaasm}{
\toolsection{arma32asm} is an assembler for the ARM hardware architecture.
It translates assembly code into machine code for ARM processors executing A32 instructions and stores it in corresponding object files.
\flowgraph{\resource{ARM A32 assembly\\source code} \ar[r] & \toolbox{arma32asm} \ar[r] & \resource{object file}}
\seeassembly\seearm\seeobject
}

\providecommand{\armadism}{
\toolsection{arma32dism} is a disassembler for the ARM hardware architecture.
It translates machine code from object files targeting ARM processors executing A32 instructions into assembly code and writes it to the standard output stream.
\flowgraph{\resource{object file} \ar[r] & \toolbox{arma32dism} \ar[r] & \resource{disassembly\\listing}}
\seeassembly\seearm\seeobject
}

\providecommand{\armbasm}{
\toolsection{arma64asm} is an assembler for the ARM hardware architecture.
It translates assembly code into machine code for ARM processors executing A64 instructions and stores it in corresponding object files.
\flowgraph{\resource{ARM A64 assembly\\source code} \ar[r] & \toolbox{arma64asm} \ar[r] & \resource{object file}}
\seeassembly\seearm\seeobject
}

\providecommand{\armbdism}{
\toolsection{arma64dism} is a disassembler for the ARM hardware architecture.
It translates machine code from object files targeting ARM processors executing A64 instructions into assembly code and writes it to the standard output stream.
\flowgraph{\resource{object file} \ar[r] & \toolbox{arma64dism} \ar[r] & \resource{disassembly\\listing}}
\seeassembly\seearm\seeobject
}

\providecommand{\armcasm}{
\toolsection{armt32asm} is an assembler for the ARM hardware architecture.
It translates assembly code into machine code for ARM processors executing T32 instructions and stores it in corresponding object files.
\flowgraph{\resource{ARM T32 assembly\\source code} \ar[r] & \toolbox{armt32asm} \ar[r] & \resource{object file}}
\seeassembly\seearm\seeobject
}

\providecommand{\armcdism}{
\toolsection{armt32dism} is a disassembler for the ARM hardware architecture.
It translates machine code from object files targeting ARM processors executing T32 instructions into assembly code and writes it to the standard output stream.
\flowgraph{\resource{object file} \ar[r] & \toolbox{armt32dism} \ar[r] & \resource{disassembly\\listing}}
\seeassembly\seearm\seeobject
}

\providecommand{\avrasm}{
\toolsection{avrasm} is an assembler for the AVR hardware architecture.
It translates assembly code into machine code for AVR processors and stores it in corresponding object files.
The identifiers \texttt{RXL}, \texttt{RXH}, \texttt{RYL}, \texttt{RYH}, \texttt{RZL}, and \texttt{RZH} are predefined and name the corresponding registers.
The identifiers \texttt{SPL} and \texttt{SPH} are also predefined and evaluate to the address of the corresponding registers.
\flowgraph{\resource{AVR assembly\\source code} \ar[r] & \toolbox{avrasm} \ar[r] & \resource{object file}}
\seeassembly\seeavr\seeobject
}

\providecommand{\avrdism}{
\toolsection{avrdism} is a disassembler for the AVR hardware architecture.
It translates machine code from object files targeting AVR processors into assembly code and writes it to the standard output stream.
\flowgraph{\resource{object file} \ar[r] & \toolbox{avrdism} \ar[r] & \resource{disassembly\\listing}}
\seeassembly\seeavr\seeobject
}

\providecommand{\avrttasm}{
\toolsection{avr32asm} is an assembler for the AVR32 hardware architecture.
It translates assembly code into machine code for AVR32 processors and stores it in corresponding object files.
\flowgraph{\resource{AVR32 assembly\\source code} \ar[r] & \toolbox{avr32asm} \ar[r] & \resource{object file}}
\seeassembly\seeavrtt\seeobject
}

\providecommand{\avrttdism}{
\toolsection{avr32dism} is a disassembler for the AVR32 hardware architecture.
It translates machine code from object files targeting AVR32 processors into assembly code and writes it to the standard output stream.
\flowgraph{\resource{object file} \ar[r] & \toolbox{avr32dism} \ar[r] & \resource{disassembly\\listing}}
\seeassembly\seeavrtt\seeobject
}

\providecommand{\mabkasm}{
\toolsection{m68kasm} is an assembler for the M68000 hardware architecture.
It translates assembly code into machine code for M68000 processors and stores it in corresponding object files.
\flowgraph{\resource{68000 assembly\\source code} \ar[r] & \toolbox{m68kasm} \ar[r] & \resource{object file}}
\seeassembly\seemabk\seeobject
}

\providecommand{\mabkdism}{
\toolsection{m68kdism} is a disassembler for the M68000 hardware architecture.
It translates machine code from object files targeting M68000 processors into assembly code and writes it to the standard output stream.
\flowgraph{\resource{object file} \ar[r] & \toolbox{m68kdism} \ar[r] & \resource{disassembly\\listing}}
\seeassembly\seemabk\seeobject
}

\providecommand{\miblasm}{
\toolsection{miblasm} is an assembler for the MicroBlaze hardware architecture.
It translates assembly code into machine code for MicroBlaze processors and stores it in corresponding object files.
\flowgraph{\resource{MicroBlaze assembly\\source code} \ar[r] & \toolbox{miblasm} \ar[r] & \resource{object file}}
\seeassembly\seemibl\seeobject
}

\providecommand{\mibldism}{
\toolsection{mibldism} is a disassembler for the MicroBlaze hardware architecture.
It translates machine code from object files targeting MicroBlaze processors into assembly code and writes it to the standard output stream.
\flowgraph{\resource{object file} \ar[r] & \toolbox{mibldism} \ar[r] & \resource{disassembly\\listing}}
\seeassembly\seemibl\seeobject
}

\providecommand{\mipsaasm}{
\toolsection{mips32asm} is an assembler for the MIPS32 hardware architecture.
It translates assembly code into machine code for MIPS32 processors and stores it in corresponding object files.
\flowgraph{\resource{MIPS32 assembly\\source code} \ar[r] & \toolbox{mips32asm} \ar[r] & \resource{object file}}
\seeassembly\seemips\seeobject
}

\providecommand{\mipsadism}{
\toolsection{mips32dism} is a disassembler for the MIPS32 hardware architecture.
It translates machine code from object files targeting MIPS32 processors into assembly code and writes it to the standard output stream.
\flowgraph{\resource{object file} \ar[r] & \toolbox{mips32dism} \ar[r] & \resource{disassembly\\listing}}
\seeassembly\seemips\seeobject
}

\providecommand{\mipsbasm}{
\toolsection{mips64asm} is an assembler for the MIPS64 hardware architecture.
It translates assembly code into machine code for MIPS64 processors and stores it in corresponding object files.
\flowgraph{\resource{MIPS64 assembly\\source code} \ar[r] & \toolbox{mips64asm} \ar[r] & \resource{object file}}
\seeassembly\seemips\seeobject
}

\providecommand{\mipsbdism}{
\toolsection{mips64dism} is a disassembler for the MIPS64 hardware architecture.
It translates machine code from object files targeting MIPS64 processors into assembly code and writes it to the standard output stream.
\flowgraph{\resource{object file} \ar[r] & \toolbox{mips64dism} \ar[r] & \resource{disassembly\\listing}}
\seeassembly\seemips\seeobject
}

\providecommand{\mmixasm}{
\toolsection{mmixasm} is an assembler for the MMIX hardware architecture.
It translates assembly code into machine code for MMIX processors and stores it in corresponding object files.
The names of all special registers are predefined and evaluate to the corresponding number.
\flowgraph{\resource{MMIX assembly\\source code} \ar[r] & \toolbox{mmixasm} \ar[r] & \resource{object file}}
\seeassembly\seemmix\seeobject
}

\providecommand{\mmixdism}{
\toolsection{mmixdism} is a disassembler for the MMIX hardware architecture.
It translates machine code from object files targeting MMIX processors into assembly code and writes it to the standard output stream.
\flowgraph{\resource{object file} \ar[r] & \toolbox{mmixdism} \ar[r] & \resource{disassembly\\listing}}
\seeassembly\seemmix\seeobject
}

\providecommand{\orokasm}{
\toolsection{or1kasm} is an assembler for the OpenRISC 1000 hardware architecture.
It translates assembly code into machine code for OpenRISC 1000 processors and stores it in corresponding object files.
\flowgraph{\resource{OpenRISC 1000 assembly\\source code} \ar[r] & \toolbox{or1kasm} \ar[r] & \resource{object file}}
\seeassembly\seeorok\seeobject
}

\providecommand{\orokdism}{
\toolsection{or1kdism} is a disassembler for the OpenRISC 1000 hardware architecture.
It translates machine code from object files targeting OpenRISC 1000 processors into assembly code and writes it to the standard output stream.
\flowgraph{\resource{object file} \ar[r] & \toolbox{or1kdism} \ar[r] & \resource{disassembly\\listing}}
\seeassembly\seeorok\seeobject
}

\providecommand{\ppcaasm}{
\toolsection{ppc32asm} is an assembler for the PowerPC hardware architecture.
It translates assembly code into machine code for PowerPC processors and stores it in corresponding object files.
By default, the assembler generates machine code for the 32-bit operating mode defined by the PowerPC architecture.
\flowgraph{\resource{PowerPC assembly\\source code} \ar[r] & \toolbox{ppc32asm} \ar[r] & \resource{object file}}
\seeassembly\seeppc\seeobject
}

\providecommand{\ppcadism}{
\toolsection{ppc32dism} is a disassembler for the PowerPC hardware architecture.
It translates machine code from object files targeting PowerPC processors into assembly code and writes it to the standard output stream.
It assumes that the machine code was generated for the 32-bit operating mode defined by the PowerPC architecture.
\flowgraph{\resource{object file} \ar[r] & \toolbox{ppc32dism} \ar[r] & \resource{disassembly\\listing}}
\seeassembly\seeppc\seeobject
}

\providecommand{\ppcbasm}{
\toolsection{ppc64asm} is an assembler for the PowerPC hardware architecture.
It translates assembly code into machine code for PowerPC processors and stores it in corresponding object files.
By default, the assembler generates machine code for the 64-bit operating mode defined by the PowerPC architecture.
\flowgraph{\resource{PowerPC assembly\\source code} \ar[r] & \toolbox{ppc64asm} \ar[r] & \resource{object file}}
\seeassembly\seeppc\seeobject
}

\providecommand{\ppcbdism}{
\toolsection{ppc64dism} is a disassembler for the PowerPC hardware architecture.
It translates machine code from object files targeting PowerPC processors into assembly code and writes it to the standard output stream.
It assumes that the machine code was generated for the 64-bit operating mode defined by the PowerPC architecture.
\flowgraph{\resource{object file} \ar[r] & \toolbox{ppc64dism} \ar[r] & \resource{disassembly\\listing}}
\seeassembly\seeppc\seeobject
}

\providecommand{\riscasm}{
\toolsection{riscasm} is an assembler for the RISC hardware architecture.
It translates assembly code into machine code for RISC processors and stores it in corresponding object files.
The names of all special registers are predefined and evaluate to the corresponding number.
\flowgraph{\resource{RISC assembly\\source code} \ar[r] & \toolbox{riscasm} \ar[r] & \resource{object file}}
\seeassembly\seerisc\seeobject
}

\providecommand{\riscdism}{
\toolsection{riscdism} is a disassembler for the RISC hardware architecture.
It translates machine code from object files targeting RISC processors into assembly code and writes it to the standard output stream.
\flowgraph{\resource{object file} \ar[r] & \toolbox{riscdism} \ar[r] & \resource{disassembly\\listing}}
\seeassembly\seerisc\seeobject
}

\providecommand{\wasmasm}{
\toolsection{wasmasm} is an assembler for the WebAssembly architecture.
It translates assembly code into machine code for WebAssembly targets and stores it in corresponding object files.
The names of all special registers are predefined and evaluate to the corresponding number.
\flowgraph{\resource{WebAssembly assembly\\source code} \ar[r] & \toolbox{wasmasm} \ar[r] & \resource{object file}}
\seeassembly\seewasm\seeobject
}

\providecommand{\wasmdism}{
\toolsection{wasmdism} is a disassembler for the WebAssembly architecture.
It translates machine code from object files targeting WebAssembly targets into assembly code and writes it to the standard output stream.
\flowgraph{\resource{object file} \ar[r] & \toolbox{wasmdism} \ar[r] & \resource{disassembly\\listing}}
\seeassembly\seewasm\seeobject
}

% linker tools

\providecommand{\linklib}{
\toolsection{linklib} is an object file combiner.
It creates a static library file by combining all object files given to it into a single one.
\flowgraph{\resource{object files} \ar[r] & \toolbox{linklib} \ar[r] & \resource{library file}}
\seeobject
}

\providecommand{\linkbin}{
\toolsection{linkbin} is a linker for plain binary files.
It links all object files given to it into a single image and stores it in a binary file that begins with the first linked section.
It also creates a map file that lists the address, type, name and size of all used sections.
The filename extension of the resulting binary file can be specified by putting it into a constant data section called \texttt{\_extension}.
\flowgraph{\resource{object files} \ar[r] & \toolbox{linkbin} \ar[r] \ar[d] & \resource{binary file} \\ & \resource{map file}}
\seeobject
}

\providecommand{\linkmem}{
\toolsection{linkmem} is a linker for plain binary files partitioned into random-access and read-only memory.
It links all object files given to it into two distinct images, one for data sections and one for code and constant data sections, and stores each image in a binary file that begins with the first linked section of the corresponding type.
It also creates a map file that lists the address, type, name and size of all used sections.
\flowgraph{\resource{object files} \ar[r] & \toolbox{linkmem} \ar[r] \ar[d] & \resource{RAM file/\\ROM file} \\ & \resource{map file}}
\seeobject
}

\providecommand{\linkprg}{
\toolsection{linkprg} is a linker for GEMDOS executable files.
It links all object files given to it into a single image and stores the image in an Atari GEMDOS executable file~\cite{gemdosfile}.
It also creates a map file that lists the address relative to the text segment, type, name and size of all used sections.
The filename extension of the resulting executable file can be specified by putting it into a constant data section called \texttt{\_extension}.
The GEMDOS executable file format requires all patch patterns of absolute link patches to consist of four full bitmasks with descending offsets.
\flowgraph{\resource{object files} \ar[r] & \toolbox{linkprg} \ar[r] \ar[d] & \resource{executable file} \\ & \resource{map file}}
\seeobject
}

\providecommand{\linkhex}{
\toolsection{linkhex} is a linker for Intel HEX files.
It links all code sections of the object files given to it into single image and stores the image in an Intel HEX file~\cite{hexfile} that begins with the first linked section.
It also creates a map file that lists the address, type, name and size of all used sections.
\flowgraph{\resource{object files} \ar[r] & \toolbox{linkhex} \ar[r] \ar[d] & \resource{HEX file} \\ & \resource{map file}}
\seeobject
}

\providecommand{\mapsearch}{
\toolsection{mapsearch} is a debugging tool.
It searches map files generated by linker tools for the name of a binary section that encompasses a memory address read from the standard input stream.
If additionally provided with one or more object files, it also stores an excerpt thereof in a separate object file called map search result which only contains the identified binary section for disassembling purposes.
\flowgraph{& \resource{map files/\\object files} \ar[d] \\ \resource{memory\\address} \ar[r] & \toolbox{mapsearch} \ar[r] \ar[d] & \resource{section name/\\relative offset} \\ & \resource{object file\\excerpt}}
\seeobject
}

\renewcommand{\seemmix}{}

\startchapter{MMIX}{MMIX Hardware Architecture Support}{mmix}
{This \documentation{} describes how the \ecs{} supports the MMIX hardware architecture.
This includes information about the assembler, disassembler, and the various compilers featured by the \ecs{} as well as the interoperability between these tools.}

\section{Introduction}

The \ecs{} features various compilers, an assembler, and a disassembler that target the MMIX hardware architecture.
Figure~\ref{fig:mmixdataflow} shows the data flow in-between these tools.

\begin{figure}
\flowgraph{
\resource{intermediate\\code} \ar[d] & & \resource{assembly\\source code} \ar[d] \\
\converter{MMIX\\Generator} \ar[r] \ar[rd] \ar[d] & \resource{assembly\\listing} \ar[r] & \converter{MMIX\\Assembler} \ar[ld] \\
\resource{debugging\\information} & \resource{object file} \ar[d] \\
& \converter{MMIX\\Disassembler} \ar[d] \\
& \resource{disassembly\\listing} \\
}\caption{Data flow within the tools targeting the MMIX architecture}
\label{fig:mmixdataflow}
\end{figure}

All compilers targeting the MMIX architecture translate their programs using an intermediate code representation.
The MMIX generator is able to translate the intermediate code representation of a program into machine code for MMIX processors.
It stores the resulting binary code and data in so-called object files.
Additionally, the generator is able to create an assembly code listing of the machine code for debugging purposes.
This assembly code listing can also be processed by the assembler yielding exactly the same object file.
The disassembler is able to open object files and print a human-readable disassembly listing of their contents.
\seeobject\seecode

\section{Instruction Set}

Tools targeting the MMIX architecture support the instruction set listed in Table~\ref{tab:mmixset} and use the same assembly syntax as predefined by Donald~E.\ Knuth~\cite{mmixware}.
The only exception are immediate values which are not prefixed by a number sign.
\seeassembly

\instructionset{mmix}{Supported MMIX instruction set}{5}{6}

\section{Calling Convention}\index{Calling convention!of MMIX}

The machine code generator and runtime support for the MMIX architecture as provided by the \ecs{} use the following calling convention in order to enable interoperability.

\subsection{Stack Operations}

Arguments for functions as well as the return address are in general passed using the stack according to the intermediate code specification.
See \Documentation{}~\documentationref{code}{Intermediate Code Representation} for more information about the role of the stack.
Function arguments are pushed on the stack in reverse order and cleaned by the caller.

\subsection{Register Mapping}

The special-purpose registers defined by the intermediate code representation are mapped to their corresponding physical registers in the following way:

\begin{itemize}

\item Result Register\alignright\texttt{\$res}\nopagebreak

The intermediate code result register \texttt{\$res} is mapped to MMIX register \texttt{\$0}.

\item Stack Pointer Register\alignright\texttt{\$sp}\nopagebreak

The intermediate code stack pointer register \texttt{\$sp} is mapped to MMIX register \texttt{\$1}.

\item Frame Pointer Register\alignright\texttt{\$fp}\nopagebreak

The intermediate code frame pointer register \texttt{\$fp} is mapped to MMIX register \texttt{\$2}.

\item Link Register\alignright\texttt{\$lnk}\nopagebreak

The intermediate code link register \texttt{\$lnk} is not supported.

\end{itemize}

All other intermediate code registers are mapped as needed to the remaining physical registers.
Their contents and mapping are therefore considered volatile across function calls.

\section{Runtime Support}\index{Runtime support!for MMIX}

The \ecs{} provides runtime support for the MMIX architecture and runtime environments based on this hardware architecture in object files.
Users targeting a specific runtime environment have to use an appropriate linker together with these object files in order create an executable program.
This section gives information about all supported runtime environments based on the MMIX hardware architecture as well as the required combination of linker and object files.

Basic architectural runtime support is provided by the object file \objfile{mmix\-run}.
Users should always include this object file during linking regardless of the actual target runtime environment.
All other object files given to the linker should target the same hardware architecture.

Programs written in \cpp{} need additional runtime support stored in the \libfile{cpp\-mmix\-run} library file.
Programs written in Oberon need additional runtime support stored in the \libfile{ob\-mmix\-run} library file.
\seecpp\seeoberon

Programs targeting the MMIX simulator are created using the \tool{link\-bin} linker tool.
It creates a MMIX object file~\cite{mmixware} if provided with the runtime support stored in the \objfile{mmix\-sim\-run} object file.
Calling the \tool{ecsd} utility tool using the \environment{mmix\-sim} target environment achieves the same result.

\section{MMIX Tools}

The \ecs{} provides the following tools that are able to process object files targeting the MMIX hardware architecture.
\interface

\cdmmix
\cppmmix
\falmmix
\obmmix
\mmixasm
\mmixdism
\linkbin

\concludechapter

// OpenRISC 1000 instruction set definitions
// Copyright (C) Florian Negele

// This file is part of the Eigen Compiler Suite.

// The ECS is free software: you can redistribute it and/or modify
// it under the terms of the GNU General Public License as published by
// the Free Software Foundation, either version 3 of the License, or
// (at your option) any later version.

// The ECS is distributed in the hope that it will be useful,
// but WITHOUT ANY WARRANTY; without even the implied warranty of
// MERCHANTABILITY or FITNESS FOR A PARTICULAR PURPOSE.  See the
// GNU General Public License for more details.

// You should have received a copy of the GNU General Public License
// along with the ECS.  If not, see <https://www.gnu.org/licenses/>.

#ifndef CLASS
	#define CLASS(class)
#endif

#ifndef INSTR
	#define INSTR(mnem, code, mask, type1, type2, type3, class)
#endif

#ifndef MNEM
	#define MNEM(name, mnem, ...)
#endif

#ifndef TYPE
	#define TYPE(type)
#endif

// mnemonics

MNEM (l.add,        LADD,       Add)
MNEM (l.addc,       LADDC,      Add and Carry)
MNEM (l.addi,       LADDI,      Add Immediate)
MNEM (l.addic,      LADDIC,     Add Immediate and Carry)
MNEM (l.and,        LAND,       And)
MNEM (l.andi,       LANDI,      And with Immediate Half Word)
MNEM (l.bf,         LBF,        Branch if Flag)
MNEM (l.bnf,        LBNF,       Branch if No Flag)
MNEM (l.cmov,       LCMOV,      Conditional Move)
MNEM (l.csync,      LCSYNC,     Context Synchronization)
MNEM (l.div,        LDIV,       Divide Signed)
MNEM (l.divu,       LDIVU,      Divide Unsigned)
MNEM (l.extbs,      LEXTBS,     Extend Byte with Sign)
MNEM (l.extbz,      LEXTBZ,     Extend Byte with Zero)
MNEM (l.exths,      LEXTHS,     Extend Half Word with Sign)
MNEM (l.exthz,      LEXTHZ,     Extend Half Word with Zero)
MNEM (l.extws,      LEXTWS,     Extend Word with Sign)
MNEM (l.extwz,      LEXTWZ,     Extend Word with Zero)
MNEM (l.ff1,        LFF1,       Find First 1)
MNEM (l.fl1,        LFL1,       Find Last 1)
MNEM (l.j,          LJ,         Jump)
MNEM (l.jal,        LJAL,       Jump and Link)
MNEM (l.jalr,       LJALR,      Jump and Link Register)
MNEM (l.jr,         LJR,        Jump Register)
MNEM (l.lbs,        LLBS,       Load Byte and Extend with Sign)
MNEM (l.lbz,        LLBZ,       Load Byte and Extend with Zero)
MNEM (l.ld,         LLD,        Load Double Word)
MNEM (l.lhs,        LLHS,       Load Half Word and Extend with Sign)
MNEM (l.lhz,        LLHZ,       Load Half Word and Extend with Zero)
MNEM (l.lwa,        LLWA,       Load Single Word Atomic)
MNEM (l.lws,        LLWS,       Load Single Word and Extend with Sign)
MNEM (l.lwz,        LLWZ,       Load Single Word and Extend with Zero)
MNEM (l.mac,        LMAC,       Multiply and Accumulate Signed)
MNEM (l.maci,       LMACI,      Multiply Immediate and Accumulate Signed)
MNEM (l.macrc,      LMACRC,     MAC Read and Clear)
MNEM (l.macu,       LMACU,      Multiply and Accumulate Unsigned)
MNEM (l.mfspr,      LMFSPR,     Move From Special-Purpose Register)
MNEM (l.movhi,      LMOVHI,     Move Immediate High)
MNEM (l.msb,        LMSB,       Multiply and Subtract Signed)
MNEM (l.msbu,       LMSBU,      Multiply and Subtract Unsigned)
MNEM (l.msync,      LMSYNC,     Memory Synchronization)
MNEM (l.mtspr,      LMTSPR,     Move To Special-Purpose Register)
MNEM (l.mul,        LMUL,       Multiply Signed)
MNEM (l.muld,       LMULD,      Multiply Signed to Double)
MNEM (l.muldu,      LMULDU,     Multiply Unsigned to Double)
MNEM (l.muli,       LMULI,      Multiply Immediate Signed)
MNEM (l.mulu,       LMULU,      Multiply Unsigned)
MNEM (l.nop,        LNOP,       No Operation)
MNEM (l.or,         LOR,        Or)
MNEM (l.ori,        LORI,       Or with Immediate Half Word)
MNEM (l.psync,      LPSYNC,     Pipeline Synchronization)
MNEM (l.rfe,        LRFE,       Return From Exception)
MNEM (l.ror,        LROR,       Rotate Right)
MNEM (l.rori,       LRORI,      Rotate Right with Immediate)
MNEM (l.sb,         LSB,        Store Byte)
MNEM (l.sd,         LSD,        Store Double Word)
MNEM (l.sfeq,       LSFEQ,      Set Flag if Equal)
MNEM (l.sfeqi,      LSFEQI,     Set Flag if Equal Immediate)
MNEM (l.sfges,      LSFGES,     Set Flag if Greater or Equal Than Signed)
MNEM (l.sfgesi,     LSFGESI,    Set Flag if Greater or Equal Than Signed Immediate)
MNEM (l.sfgeu,      LSFGEU,     Set Flag if Greater or Equal Than Unsigned)
MNEM (l.sfgeui,     LSFGEUI,    Set Flag if Greater or Equal Than Unsigned Immediate)
MNEM (l.sfgts,      LSFGTS,     Set Flag if Greater Than Signed)
MNEM (l.sfgtsi,     LSFGTSI,    Set Flag if Greater Than Signed Immediate)
MNEM (l.sfgtu,      LSFGTU,     Set Flag if Greater Than Unsigned)
MNEM (l.sfgtui,     LSFGTUI,    Set Flag if Greater Than Unsigned Immediate)
MNEM (l.sfles,      LSFLES,     Set Flag if Less or Equal Than Signed)
MNEM (l.sflesi,     LSFLESI,    Set Flag if Less or Equal Than Signed Immediate)
MNEM (l.sfleu,      LSFLEU,     Set Flag if Less or Equal Than Unsigned)
MNEM (l.sfleui,     LSFLEUI,    Set Flag if Less or Equal Than Unsigned Immediate)
MNEM (l.sflts,      LSFLTS,     Set Flag if Less Than Signed)
MNEM (l.sfltsi,     LSFLTSI,    Set Flag if Less Than Signed Immediate)
MNEM (l.sfltu,      LSFLTU,     Set Flag if Less Than Unsigned)
MNEM (l.sfltui,     LSFLTUI,    Set Flag if Less Than Unsigned Immediate)
MNEM (l.sfne,       LSFNE,      Set Flag if Not Equal)
MNEM (l.sfnei,      LSFNEI,     Set Flag if Not Equal Immediate)
MNEM (l.sh,         LSH,        Store Half Word)
MNEM (l.sll,        LSLL,       Shift Left Logical)
MNEM (l.slli,       LSLLI,      Shift Left Logical Immediate)
MNEM (l.sra,        LSRA,       Shift Right Arithmetic)
MNEM (l.srai,       LSRAI,      Shift Right Arithmetic Immediate)
MNEM (l.srl,        LSRL,       Shift Right Logical)
MNEM (l.srli,       LSRLI,      Shift Right Logical Immediate)
MNEM (l.sub,        LSUB,       Subtract)
MNEM (l.sw,         LSW,        Store Single Word)
MNEM (l.swa,        LSWA,       Store Single Word Atomic)
MNEM (l.sys,        LSYS,       System Call)
MNEM (l.trap,       LTRAP,      Trap)
MNEM (l.xor,        LXOR,       Exclusive Or)
MNEM (l.xori,       LXORI,      Exclusive Or with Immediate Half Word)
MNEM (lf.add.d,     LFADDD,     Add Floating-Point Double-Precision)
MNEM (lf.add.s,     LFADDS,     Add Floating-Point Single-Precision)
MNEM (lf.div.d,     LFDIVD,     Divide Floating-Point Double-Precision)
MNEM (lf.div.s,     LFDIVS,     Divide Floating-Point Single-Precision)
MNEM (lf.ftoi.d,    LFFTOID,    Floating-Point Double-Precision To Integer)
MNEM (lf.ftoi.s,    LFFTOIS,    Floating-Point Single-Precision To Integer)
MNEM (lf.itof.d,    LFITOFD,    Integer To Floating-Point Double-Precision)
MNEM (lf.itof.s,    LFITOFS,    Integer To Floating-Point Single-Precision)
MNEM (lf.madd.d,    LFMADDD,    Multiply and Add Floating-Point Double-Precision)
MNEM (lf.madd.s,    LFMADDS,    Multiply and Add Floating-Point Single-Precision)
MNEM (lf.mul.d,     LFMULD,     Multiply Floating-Point Double-Precision)
MNEM (lf.mul.s,     LFMULS,     Multiply Floating-Point Single-Precision)
MNEM (lf.rem.d,     LFREMD,     Remainder Floating-Point Double-Precision)
MNEM (lf.rem.s,     LFREMS,     Remainder Floating-Point Single-Precision)
MNEM (lf.sfeq.d,    LFSFEQD,    Set Flag if Equal Floating-Point Double-Precision)
MNEM (lf.sfeq.s,    LFSFEQS,    Set Flag if Equal Floating-Point Single-Precision)
MNEM (lf.sfge.d,    LFSFGED,    Set Flag if Greater or Equal Than Floating-Point Double-Precision)
MNEM (lf.sfge.s,    LFSFGES,    Set Flag if Greater or Equal Than Floating-Point Single-Precision)
MNEM (lf.sfgt.d,    LFSFGTD,    Set Flag if Greater Than Floating-Point Double-Precision)
MNEM (lf.sfgt.s,    LFSFGTS,    Set Flag if Greater Than Floating-Point Single-Precision)
MNEM (lf.sfle.d,    LFSFLED,    Set Flag if Less or Equal Than Floating-Point Double-Precision)
MNEM (lf.sfle.s,    LFSFLES,    Set Flag if Less or Equal Than Floating-Point Single-Precision)
MNEM (lf.sflt.d,    LFSFLTD,    Set Flag if Less Than Floating-Point Double-Precision)
MNEM (lf.sflt.s,    LFSFLTS,    Set Flag if Less Than Floating-Point Single-Precision)
MNEM (lf.sfne.d,    LFSFNED,    Set Flag if Not Equal Floating-Point Double-Precision)
MNEM (lf.sfne.s,    LFSFNES,    Set Flag if Not Equal Floating-Point Single-Precision)
MNEM (lf.sub.d,     LFSUBD,     Subtract Floating-Point Double-Precision)
MNEM (lf.sub.s,     LFSUBS,     Subtract Floating-Point Single-Precision)
MNEM (lv.add.b,     LVADDB,     Vector Byte Elements Add Signed)
MNEM (lv.add.h,     LVADDH,     Vector Half-Word Elements Add Signed)
MNEM (lv.adds.b,    LVADDSB,    Vector Byte Elements Add Signed Saturated)
MNEM (lv.adds.h,    LVADDSH,    Vector Half-Word Elements Add Signed Saturated)
MNEM (lv.addu.b,    LVADDUB,    Vector Byte Elements Add Unsigned)
MNEM (lv.addu.h,    LVADDUH,    Vector Half-Word Elements Add Unsigned)
MNEM (lv.addus.b,   LVADDUSB,   Vector Byte Elements Add Unsigned Saturated)
MNEM (lv.addus.h,   LVADDUSH,   Vector Half-Word Elements Add Unsigned Saturated)
MNEM (lv.all_eq.b,  LVALLEQB,   Vector Byte Elements All Equal)
MNEM (lv.all_eq.h,  LVALLEQH,   Vector Half-Word Elements All Equal)
MNEM (lv.all_ge.b,  LVALLGEB,   Vector Byte Elements All Greater Than or Equal To)
MNEM (lv.all_ge.h,  LVALLGEH,   Vector Half-Word Elements All Greater Than or Equal To)
MNEM (lv.all_gt.b,  LVALLGTB,   Vector Byte Elements All Greater Than)
MNEM (lv.all_gt.h,  LVALLGTH,   Vector Half-Word Elements All Greater Than)
MNEM (lv.all_le.b,  LVALLLEB,   Vector Byte Elements All Less Than or Equal To)
MNEM (lv.all_le.h,  LVALLLEH,   Vector Half-Word Elements All Less Than or Equal To)
MNEM (lv.all_lt.b,  LVALLLTB,   Vector Byte Elements All Less Than)
MNEM (lv.all_lt.h,  LVALLLTH,   Vector Half-Word Elements All Less Than)
MNEM (lv.all_ne.b,  LVALLNEB,   Vector Byte Elements All Not Equal)
MNEM (lv.all_ne.h,  LVALLNEH,   Vector Half-Word Elements All Not Equal)
MNEM (lv.and,       LVAND,      Vector And)
MNEM (lv.any_eq.b,  LVANYEQB,   Vector Byte Elements Any Equal)
MNEM (lv.any_eq.h,  LVANYEQH,   Vector Half-Word Elements Any Equal)
MNEM (lv.any_ge.b,  LVANYGEB,   Vector Byte Elements Any Greater Than or Equal To)
MNEM (lv.any_ge.h,  LVANYGEH,   Vector Half-Word Elements Any Greater Than or Equal To)
MNEM (lv.any_gt.b,  LVANYGTB,   Vector Byte Elements Any Greater Than)
MNEM (lv.any_gt.h,  LVANYGTH,   Vector Half-Word Elements Any Greater Than)
MNEM (lv.any_le.b,  LVANYLEB,   Vector Byte Elements Any Less Than or Equal To)
MNEM (lv.any_le.h,  LVANYLEH,   Vector Half-Word Elements Any Less Than or Equal To)
MNEM (lv.any_lt.b,  LVANYLTB,   Vector Byte Elements Any Less Than)
MNEM (lv.any_lt.h,  LVANYLTH,   Vector Half-Word Elements Any Less Than)
MNEM (lv.any_ne.b,  LVANYNEB,   Vector Byte Elements Any Not Equal)
MNEM (lv.any_ne.h,  LVANYNEH,   Vector Half-Word Elements Any Not Equal)
MNEM (lv.avg.b,     LVAVGB,     Vector Byte Elements Average)
MNEM (lv.avg.h,     LVAVGH,     Vector Half-Word Elements Average)
MNEM (lv.cmp_eq.b,  LVCMPEQB,   Vector Byte Elements Compare Equal)
MNEM (lv.cmp_eq.h,  LVCMPEQH,   Vector Half-Word Elements Compare Equal)
MNEM (lv.cmp_ge.b,  LVCMPGEB,   Vector Byte Elements Compare Greater Than or Equal To)
MNEM (lv.cmp_ge.h,  LVCMPGEH,   Vector Half-Word Elements Compare Greater Than or Equal To)
MNEM (lv.cmp_gt.b,  LVCMPGTB,   Vector Byte Elements Compare Greater Than)
MNEM (lv.cmp_gt.h,  LVCMPGTH,   Vector Half-Word Elements Compare Greater Than)
MNEM (lv.cmp_le.b,  LVCMPLEB,   Vector Byte Elements Compare Less Than or Equal To)
MNEM (lv.cmp_le.h,  LVCMPLEH,   Vector Half-Word Elements Compare Less Than or Equal To)
MNEM (lv.cmp_lt.b,  LVCMPLTB,   Vector Byte Elements Compare Less Than)
MNEM (lv.cmp_lt.h,  LVCMPLTH,   Vector Half-Word Elements Compare Less Than)
MNEM (lv.cmp_ne.b,  LVCMPNEB,   Vector Byte Elements Compare Not Equal)
MNEM (lv.cmp_ne.h,  LVCMPNEH,   Vector Half-Word Elements Compare Not Equal)
MNEM (lv.madds.h,   LVMADDSH,   Vector Half-Word Elements Multiply Add Signed Saturated)
MNEM (lv.max.b,     LVMAXB,     Vector Byte Elements Maximum)
MNEM (lv.max.h,     LVMAXH,     Vector Half-Word Elements Maximum)
MNEM (lv.merge.b,   LVMERGEB,   Vector Byte Elements Merge)
MNEM (lv.merge.h,   LVMERGEH,   Vector Half-Word Elements Merge)
MNEM (lv.min.b,     LVMINB,     Vector Byte Elements Minimum)
MNEM (lv.min.h,     LVMINH,     Vector Half-Word Elements Minimum)
MNEM (lv.msubs.h,   LVMSUBSH,   Vector Half-Word Elements Multiply Subtract Signed Saturated)
MNEM (lv.muls.h,    LVMULSH,    Vector Half-Word Elements Multiply Signed Saturated)
MNEM (lv.nand,      LVNAND,     Vector Not And)
MNEM (lv.nor,       LVNOR,      Vector Not Or)
MNEM (lv.or,        LVOR,       Vector Or)
MNEM (lv.pack.b,    LVPACKB,    Vector Byte Elements Pack)
MNEM (lv.pack.h,    LVPACKH,    Vector Half-Word Elements Pack)
MNEM (lv.packs.b,   LVPACKSB,   Vector Byte Elements Pack Signed Saturated)
MNEM (lv.packs.h,   LVPACKSH,   Vector Half-Word Elements Pack Signed Saturated)
MNEM (lv.packus.b,  LVPACKUSB,  Vector Byte Elements Pack Unsigned Saturated)
MNEM (lv.packus.h,  LVPACKUSH,  Vector Half-Word Elements Pack Unsigned Saturated)
MNEM (lv.perm.n,    LVPERMN,    Vector Nibble Elements Permute)
MNEM (lv.rl.b,      LVRLB,      Vector Byte Elements Rotate Left)
MNEM (lv.rl.h,      LVRLH,      Vector Half-Word Elements Rotate Left)
MNEM (lv.sll,       LVSLL,      Vector Shift Left Logical)
MNEM (lv.sll.b,     LVSLLB,     Vector Byte Elements Shift Left Logical)
MNEM (lv.sll.h,     LVSLLH,     Vector Half-Word Elements Shift Left Logical)
MNEM (lv.sra.b,     LVSRAB,     Vector Byte Elements Shift Right Arithmetic)
MNEM (lv.sra.h,     LVSRAH,     Vector Half-Word Elements Shift Right Arithmetic)
MNEM (lv.srl,       LVSRL,      Vector Shift Right Logical)
MNEM (lv.srl.b,     LVSRLB,     Vector Byte Elements Shift Right Logical)
MNEM (lv.srl.h,     LVSRLH,     Vector Half-Word Elements Shift Right Logical)
MNEM (lv.sub.b,     LVSUBB,     Vector Byte Elements Subtract Signed)
MNEM (lv.sub.h,     LVSUBH,     Vector Half-Word Elements Subtract Signed)
MNEM (lv.subs.b,    LVSUBSB,    Vector Byte Elements Subtract Signed Saturated)
MNEM (lv.subs.h,    LVSUBSH,    Vector Half-Word Elements Subtract Signed Saturated)
MNEM (lv.subu.b,    LVSUBUB,    Vector Byte Elements Subtract Unsigned)
MNEM (lv.subu.h,    LVSUBUH,    Vector Half-Word Elements Subtract Unsigned)
MNEM (lv.subus.b,   LVSUBUSB,   Vector Byte Elements Subtract Unsigned Saturated)
MNEM (lv.subus.h,   LVSUBUSH,   Vector Half-Word Elements Subtract Unsigned Saturated)
MNEM (lv.unpack.b,  LVUNPACKB,  Vector Byte Elements Unpack)
MNEM (lv.unpack.h,  LVUNPACKH,  Vector Half-Word Elements Unpack)
MNEM (lv.xor,       LVXOR,      Vector Exclusive Or)

// basic instructions

INSTR (LADD,       0xe0000000,  0xfc00030f,  RD,    RA,    RB,    ORBIS32I)
INSTR (LADDC,      0xe0000001,  0xfc00030f,  RD,    RA,    RB,    ORBIS32I)
INSTR (LADDI,      0x9c000000,  0xfc000000,  RD,    RA,    I,     ORBIS32I)
INSTR (LADDIC,     0xa0000000,  0xfc000000,  RD,    RA,    I,     ORBIS32I)
INSTR (LAND,       0xe0000003,  0xfc00030f,  RD,    RA,    RB,    ORBIS32I)
INSTR (LANDI,      0xa4000000,  0xfc000000,  RD,    RA,    K,     ORBIS32I)
INSTR (LBF,        0x10000000,  0xfc000000,  N,     Void,  Void,  ORBIS32I)
INSTR (LBNF,       0x0c000000,  0xfc000000,  N,     Void,  Void,  ORBIS32I)
INSTR (LCMOV,      0xe000000e,  0xfc00030f,  RD,    RA,    RB,    ORBIS32II)
INSTR (LCSYNC,     0x23000000,  0xffffffff,  Void,  Void,  Void,  ORBIS32II)
INSTR (LDIV,       0xe0000309,  0xfc00030f,  RD,    RA,    RB,    ORBIS32II)
INSTR (LDIVU,      0xe000030a,  0xfc00030f,  RD,    RA,    RB,    ORBIS32II)
INSTR (LEXTBS,     0xe000004c,  0xfc0003cf,  RD,    RA,    Void,  ORBIS32II)
INSTR (LEXTBZ,     0xe00000cc,  0xfc0003cf,  RD,    RA,    Void,  ORBIS32II)
INSTR (LEXTHS,     0xe000000c,  0xfc0003cf,  RD,    RA,    Void,  ORBIS32II)
INSTR (LEXTHZ,     0xe000008c,  0xfc0003cf,  RD,    RA,    Void,  ORBIS32II)
INSTR (LEXTWS,     0xe000000d,  0xfc0003cf,  RD,    RA,    Void,  ORBIS64II)
INSTR (LEXTWZ,     0xe000004d,  0xfc0003cf,  RD,    RA,    Void,  ORBIS64II)
INSTR (LFF1,       0xe000000f,  0xfc00030f,  RD,    RA,    Void,  ORBIS32II)
INSTR (LFL1,       0xe000010f,  0xfc00030f,  RD,    RA,    Void,  ORBIS32II)
INSTR (LJ,         0x00000000,  0xfc000000,  N,     Void,  Void,  ORBIS32I)
INSTR (LJAL,       0x04000000,  0xfc000000,  N,     Void,  Void,  ORBIS32I)
INSTR (LJALR,      0x48000000,  0xfc000000,  RB,    Void,  Void,  ORBIS32I)
INSTR (LJR,        0x44000000,  0xfc000000,  RB,    Void,  Void,  ORBIS32I)
INSTR (LLBS,       0x90000000,  0xfc000000,  RD,    IRA,   Void,  ORBIS32I)
INSTR (LLBZ,       0x8c000000,  0xfc000000,  RD,    IRA,   Void,  ORBIS32I)
INSTR (LLD,        0x80000000,  0xfc000000,  RD,    IRA,   Void,  ORBIS64I)
INSTR (LLHS,       0x98000000,  0xfc000000,  RD,    IRA,   Void,  ORBIS32I)
INSTR (LLHZ,       0x94000000,  0xfc000000,  RD,    IRA,   Void,  ORBIS32I)
INSTR (LLWA,       0x6c000000,  0xfc000000,  RD,    IRA,   Void,  ORBIS32II)
INSTR (LLWS,       0x88000000,  0xfc000000,  RD,    IRA,   Void,  ORBIS32I)
INSTR (LLWZ,       0x84000000,  0xfc000000,  RD,    IRA,   Void,  ORBIS32I)
INSTR (LMAC,       0xc4000001,  0xfc00000f,  RA,    RB,    Void,  ORBIS32II)
INSTR (LMACI,      0x4c000000,  0xfc000000,  RA,    I,     Void,  ORBIS32II)
INSTR (LMACRC,     0x18010000,  0xfc01ffff,  RD,    Void,  Void,  ORBIS32II)
INSTR (LMACU,      0xc4000003,  0xfc00000f,  RA,    RB,    Void,  ORBIS32II)
INSTR (LMFSPR,     0xb4000000,  0xfc000000,  RD,    RA,    K,     ORBIS32I)
INSTR (LMOVHI,     0x18000000,  0xfc010000,  RD,    K,     Void,  ORBIS32I)
INSTR (LMSB,       0xc4000002,  0xfc00000f,  RA,    RB,    Void,  ORBIS32II)
INSTR (LMSBU,      0xc4000004,  0xfc00000f,  RA,    RB,    Void,  ORBIS32II)
INSTR (LMSYNC,     0x22000000,  0xffffffff,  Void,  Void,  Void,  ORBIS32II)
INSTR (LMTSPR,     0xc0000000,  0xfc000000,  RA,    RB,    K_,    ORBIS32I)
INSTR (LMUL,       0xe0000306,  0xfc00030f,  RD,    RA,    RB,    ORBIS32II)
INSTR (LMULD,      0xe0000307,  0xfc00030f,  RA,    RB,    Void,  ORBIS32II)
INSTR (LMULDU,     0xe000030c,  0xfc00030f,  RA,    RB,    Void,  ORBIS32II)
INSTR (LMULI,      0xb0000000,  0xfc000000,  RD,    RA,    I,     ORBIS32II)
INSTR (LMULU,      0xe000030b,  0xfc00030f,  RD,    RA,    RB,    ORBIS32II)
INSTR (LNOP,       0x15000000,  0xff000000,  K,     Void,  Void,  ORBIS32I)
INSTR (LOR,        0xe0000004,  0xfc00030f,  RD,    RA,    RB,    ORBIS32I)
INSTR (LORI,       0xa8000000,  0xfc000000,  RD,    RA,    K,     ORBIS32I)
INSTR (LPSYNC,     0x22800000,  0xffffffff,  Void,  Void,  Void,  ORBIS32II)
INSTR (LRFE,       0x24000000,  0xfc000000,  Void,  Void,  Void,  ORBIS32I)
INSTR (LROR,       0xe00000c8,  0xfc0003cf,  RD,    RA,    RB,    ORBIS32II)
INSTR (LRORI,      0xb80000c0,  0xfc0000c0,  RD,    RA,    L,     ORBIS32II)
INSTR (LSB,        0xd8000000,  0xfc000000,  IRA_,  RB,    Void,  ORBIS32I)
INSTR (LSD,        0xd0000000,  0xfc000000,  IRA_,  RB,    Void,  ORBIS64I)
INSTR (LSFEQ,      0xe4000000,  0xffe00000,  RA,    RB,    Void,  ORBIS32I)
INSTR (LSFEQI,     0xbc000000,  0xffe00000,  RA,    I,     Void,  ORBIS32II)
INSTR (LSFGES,     0xe5600000,  0xffe00000,  RA,    RB,    Void,  ORBIS32I)
INSTR (LSFGESI,    0xbd600000,  0xffe00000,  RA,    I,     Void,  ORBIS32II)
INSTR (LSFGEU,     0xe4600000,  0xffe00000,  RA,    RB,    Void,  ORBIS32I)
INSTR (LSFGEUI,    0xbc600000,  0xffe00000,  RA,    I,     Void,  ORBIS32II)
INSTR (LSFGTS,     0xe5400000,  0xffe00000,  RA,    RB,    Void,  ORBIS32I)
INSTR (LSFGTSI,    0xbd400000,  0xffe00000,  RA,    I,     Void,  ORBIS32II)
INSTR (LSFGTU,     0xe4400000,  0xffe00000,  RA,    RB,    Void,  ORBIS32I)
INSTR (LSFGTUI,    0xbc400000,  0xffe00000,  RA,    I,     Void,  ORBIS32II)
INSTR (LSFLES,     0xe5a00000,  0xffe00000,  RA,    RB,    Void,  ORBIS32I)
INSTR (LSFLESI,    0xbda00000,  0xffe00000,  RA,    I,     Void,  ORBIS32II)
INSTR (LSFLEU,     0xe4a00000,  0xffe00000,  RA,    RB,    Void,  ORBIS32I)
INSTR (LSFLEUI,    0xbca00000,  0xffe00000,  RA,    I,     Void,  ORBIS32II)
INSTR (LSFLTS,     0xe5800000,  0xffe00000,  RA,    RB,    Void,  ORBIS32I)
INSTR (LSFLTSI,    0xbd800000,  0xffe00000,  RA,    I,     Void,  ORBIS32II)
INSTR (LSFLTU,     0xe4800000,  0xffe00000,  RA,    RB,    Void,  ORBIS32I)
INSTR (LSFLTUI,    0xbc800000,  0xffe00000,  RA,    I,     Void,  ORBIS32II)
INSTR (LSFNE,      0xe4200000,  0xffe00000,  RA,    RB,    Void,  ORBIS32I)
INSTR (LSFNEI,     0xbc200000,  0xffe00000,  RA,    I,     Void,  ORBIS32II)
INSTR (LSH,        0xdc000000,  0xfc000000,  IRA_,  RB,    Void,  ORBIS32I)
INSTR (LSLL,       0xe0000008,  0xfc0003cf,  RD,    RA,    RB,    ORBIS32I)
INSTR (LSLLI,      0xb8000000,  0xfc0000c0,  RD,    RA,    L,     ORBIS32I)
INSTR (LSRA,       0xe0000088,  0xfc0003cf,  RD,    RA,    RB,    ORBIS32I)
INSTR (LSRAI,      0xb8000080,  0xfc0000c0,  RD,    RA,    L,     ORBIS32I)
INSTR (LSRL,       0xe0000048,  0xfc0003cf,  RD,    RA,    RB,    ORBIS32I)
INSTR (LSRLI,      0xb8000040,  0xfc0000c0,  RD,    RA,    L,     ORBIS32I)
INSTR (LSUB,       0xe0000002,  0xfc00030f,  RD,    RA,    RB,    ORBIS32I)
INSTR (LSW,        0xd4000000,  0xfc000000,  IRA_,  RB,    Void,  ORBIS32I)
INSTR (LSWA,       0xcc000000,  0xfc000000,  IRA_,  RB,    Void,  ORBIS32II)
INSTR (LSYS,       0x20000000,  0xffff0000,  K,     Void,  Void,  ORBIS32I)
INSTR (LTRAP,      0x21000000,  0xffff0000,  K,     Void,  Void,  ORBIS32II)
INSTR (LXOR,       0xe0000005,  0xfc00030f,  RD,    RA,    RB,    ORBIS32I)
INSTR (LXORI,      0xac000000,  0xfc000000,  RD,    RA,    I,     ORBIS32I)

// floating-point instructions

INSTR (LFADDD,     0xc8000010,  0xfc0000ff,  RD,    RA,    RB,    ORFPX64I)
INSTR (LFADDS,     0xc8000000,  0xfc0000ff,  RD,    RA,    RB,    ORFPX32I)
INSTR (LFDIVD,     0xc8000013,  0xfc0000ff,  RD,    RA,    RB,    ORFPX64II)
INSTR (LFDIVS,     0xc8000003,  0xfc0000ff,  RD,    RA,    RB,    ORFPX32II)
INSTR (LFFTOID,    0xc8000015,  0xfc00f8ff,  RD,    RA,    Void,  ORFPX64I)
INSTR (LFFTOIS,    0xc8000005,  0xfc00f8ff,  RD,    RA,    Void,  ORFPX32I)
INSTR (LFITOFD,    0xc8000014,  0xfc00f8ff,  RD,    RA,    Void,  ORFPX64I)
INSTR (LFITOFS,    0xc8000004,  0xfc00f8ff,  RD,    RA,    Void,  ORFPX32I)
INSTR (LFMADDD,    0xc8000017,  0xfc0000ff,  RD,    RA,    RB,    ORFPX64II)
INSTR (LFMADDS,    0xc8000007,  0xfc0000ff,  RD,    RA,    RB,    ORFPX32II)
INSTR (LFMULD,     0xc8000012,  0xfc0000ff,  RD,    RA,    RB,    ORFPX64I)
INSTR (LFMULS,     0xc8000002,  0xfc0000ff,  RD,    RA,    RB,    ORFPX32I)
INSTR (LFREMD,     0xc8000016,  0xfc0000ff,  RD,    RA,    RB,    ORFPX64II)
INSTR (LFREMS,     0xc8000006,  0xfc0000ff,  RD,    RA,    RB,    ORFPX32II)
INSTR (LFSFEQD,    0xc8000018,  0xfc0000ff,  RA,    RB,    Void,  ORFPX64I)
INSTR (LFSFEQS,    0xc8000008,  0xfc0000ff,  RA,    RB,    Void,  ORFPX32I)
INSTR (LFSFGED,    0xc800001b,  0xfc0000ff,  RA,    RB,    Void,  ORFPX64I)
INSTR (LFSFGES,    0xc800000b,  0xfc0000ff,  RA,    RB,    Void,  ORFPX32I)
INSTR (LFSFGTD,    0xc800001a,  0xfc0000ff,  RA,    RB,    Void,  ORFPX64I)
INSTR (LFSFGTS,    0xc800000a,  0xfc0000ff,  RA,    RB,    Void,  ORFPX32I)
INSTR (LFSFLED,    0xc800001d,  0xfc0000ff,  RA,    RB,    Void,  ORFPX64I)
INSTR (LFSFLES,    0xc800000d,  0xfc0000ff,  RA,    RB,    Void,  ORFPX32I)
INSTR (LFSFLTD,    0xc800001c,  0xfc0000ff,  RA,    RB,    Void,  ORFPX64I)
INSTR (LFSFLTS,    0xc800000c,  0xfc0000ff,  RA,    RB,    Void,  ORFPX32I)
INSTR (LFSFNED,    0xc8000019,  0xfc0000ff,  RA,    RB,    Void,  ORFPX64I)
INSTR (LFSFNES,    0xc8000009,  0xfc0000ff,  RA,    RB,    Void,  ORFPX32I)
INSTR (LFSUBD,     0xc8000011,  0xfc0000ff,  RD,    RA,    RB,    ORFPX64I)
INSTR (LFSUBS,     0xc8000001,  0xfc0000ff,  RD,    RA,    RB,    ORFPX32I)

// vector instructions

INSTR (LVADDB,     0x28000030,  0xfc0000ff,  RD,    RA,    RB,    ORVDX64I)
INSTR (LVADDH,     0x28000031,  0xfc0000ff,  RD,    RA,    RB,    ORVDX64I)
INSTR (LVADDSB,    0x28000032,  0xfc0000ff,  RD,    RA,    RB,    ORVDX64I)
INSTR (LVADDSH,    0x28000033,  0xfc0000ff,  RD,    RA,    RB,    ORVDX64I)
INSTR (LVADDUB,    0x28000034,  0xfc0000ff,  RD,    RA,    RB,    ORVDX64I)
INSTR (LVADDUH,    0x28000035,  0xfc0000ff,  RD,    RA,    RB,    ORVDX64I)
INSTR (LVADDUSB,   0x28000036,  0xfc0000ff,  RD,    RA,    RB,    ORVDX64I)
INSTR (LVADDUSH,   0x28000037,  0xfc0000ff,  RD,    RA,    RB,    ORVDX64I)
INSTR (LVALLEQB,   0x28000010,  0xfc0000ff,  RD,    RA,    RB,    ORVDX64I)
INSTR (LVALLEQH,   0x28000011,  0xfc0000ff,  RD,    RA,    RB,    ORVDX64I)
INSTR (LVALLGEB,   0x28000012,  0xfc0000ff,  RD,    RA,    RB,    ORVDX64I)
INSTR (LVALLGEH,   0x28000013,  0xfc0000ff,  RD,    RA,    RB,    ORVDX64I)
INSTR (LVALLGTB,   0x28000014,  0xfc0000ff,  RD,    RA,    RB,    ORVDX64I)
INSTR (LVALLGTH,   0x28000015,  0xfc0000ff,  RD,    RA,    RB,    ORVDX64I)
INSTR (LVALLLEB,   0x28000016,  0xfc0000ff,  RD,    RA,    RB,    ORVDX64I)
INSTR (LVALLLEH,   0x28000017,  0xfc0000ff,  RD,    RA,    RB,    ORVDX64I)
INSTR (LVALLLTB,   0x28000018,  0xfc0000ff,  RD,    RA,    RB,    ORVDX64I)
INSTR (LVALLLTH,   0x28000019,  0xfc0000ff,  RD,    RA,    RB,    ORVDX64I)
INSTR (LVALLNEB,   0x2800001a,  0xfc0000ff,  RD,    RA,    RB,    ORVDX64I)
INSTR (LVALLNEH,   0x2800001b,  0xfc0000ff,  RD,    RA,    RB,    ORVDX64I)
INSTR (LVAND,      0x28000038,  0xfc0000ff,  RD,    RA,    RB,    ORVDX64I)
INSTR (LVANYEQB,   0x28000020,  0xfc0000ff,  RD,    RA,    RB,    ORVDX64I)
INSTR (LVANYEQH,   0x28000021,  0xfc0000ff,  RD,    RA,    RB,    ORVDX64I)
INSTR (LVANYGEB,   0x28000022,  0xfc0000ff,  RD,    RA,    RB,    ORVDX64I)
INSTR (LVANYGEH,   0x28000023,  0xfc0000ff,  RD,    RA,    RB,    ORVDX64I)
INSTR (LVANYGTB,   0x28000024,  0xfc0000ff,  RD,    RA,    RB,    ORVDX64I)
INSTR (LVANYGTH,   0x28000025,  0xfc0000ff,  RD,    RA,    RB,    ORVDX64I)
INSTR (LVANYLEB,   0x28000026,  0xfc0000ff,  RD,    RA,    RB,    ORVDX64I)
INSTR (LVANYLEH,   0x28000027,  0xfc0000ff,  RD,    RA,    RB,    ORVDX64I)
INSTR (LVANYLTB,   0x28000028,  0xfc0000ff,  RD,    RA,    RB,    ORVDX64I)
INSTR (LVANYLTH,   0x28000029,  0xfc0000ff,  RD,    RA,    RB,    ORVDX64I)
INSTR (LVANYNEB,   0x2800002a,  0xfc0000ff,  RD,    RA,    RB,    ORVDX64I)
INSTR (LVANYNEH,   0x2800002b,  0xfc0000ff,  RD,    RA,    RB,    ORVDX64I)
INSTR (LVAVGB,     0x28000039,  0xfc0000ff,  RD,    RA,    RB,    ORVDX64I)
INSTR (LVAVGH,     0x2800003a,  0xfc0000ff,  RD,    RA,    RB,    ORVDX64I)
INSTR (LVCMPEQB,   0x28000040,  0xfc0000ff,  RD,    RA,    RB,    ORVDX64I)
INSTR (LVCMPEQH,   0x28000041,  0xfc0000ff,  RD,    RA,    RB,    ORVDX64I)
INSTR (LVCMPGEB,   0x28000042,  0xfc0000ff,  RD,    RA,    RB,    ORVDX64I)
INSTR (LVCMPGEH,   0x28000043,  0xfc0000ff,  RD,    RA,    RB,    ORVDX64I)
INSTR (LVCMPGTB,   0x28000044,  0xfc0000ff,  RD,    RA,    RB,    ORVDX64I)
INSTR (LVCMPGTH,   0x28000045,  0xfc0000ff,  RD,    RA,    RB,    ORVDX64I)
INSTR (LVCMPLEB,   0x28000046,  0xfc0000ff,  RD,    RA,    RB,    ORVDX64I)
INSTR (LVCMPLEH,   0x28000047,  0xfc0000ff,  RD,    RA,    RB,    ORVDX64I)
INSTR (LVCMPLTB,   0x28000048,  0xfc0000ff,  RD,    RA,    RB,    ORVDX64I)
INSTR (LVCMPLTH,   0x28000049,  0xfc0000ff,  RD,    RA,    RB,    ORVDX64I)
INSTR (LVCMPNEB,   0x2800004a,  0xfc0000ff,  RD,    RA,    RB,    ORVDX64I)
INSTR (LVCMPNEH,   0x2800004b,  0xfc0000ff,  RD,    RA,    RB,    ORVDX64I)
INSTR (LVMADDSH,   0x28000054,  0xfc0000ff,  RD,    RA,    RB,    ORVDX64I)
INSTR (LVMAXB,     0x28000055,  0xfc0000ff,  RD,    RA,    RB,    ORVDX64I)
INSTR (LVMAXH,     0x28000056,  0xfc0000ff,  RD,    RA,    RB,    ORVDX64I)
INSTR (LVMERGEB,   0x28000057,  0xfc0000ff,  RD,    RA,    RB,    ORVDX64I)
INSTR (LVMERGEH,   0x28000058,  0xfc0000ff,  RD,    RA,    RB,    ORVDX64I)
INSTR (LVMINB,     0x28000059,  0xfc0000ff,  RD,    RA,    RB,    ORVDX64I)
INSTR (LVMINH,     0x2800005a,  0xfc0000ff,  RD,    RA,    RB,    ORVDX64I)
INSTR (LVMSUBSH,   0x2800005b,  0xfc0000ff,  RD,    RA,    RB,    ORVDX64I)
INSTR (LVMULSH,    0x2800005c,  0xfc0000ff,  RD,    RA,    RB,    ORVDX64II)
INSTR (LVNAND,     0x2800005d,  0xfc0000ff,  RD,    RA,    RB,    ORVDX64I)
INSTR (LVNOR,      0x2800005e,  0xfc0000ff,  RD,    RA,    RB,    ORVDX64I)
INSTR (LVOR,       0x2800005f,  0xfc0000ff,  RD,    RA,    RB,    ORVDX64I)
INSTR (LVPACKB,    0x28000060,  0xfc0000ff,  RD,    RA,    RB,    ORVDX64I)
INSTR (LVPACKH,    0x28000061,  0xfc0000ff,  RD,    RA,    RB,    ORVDX64I)
INSTR (LVPACKSB,   0x28000062,  0xfc0000ff,  RD,    RA,    RB,    ORVDX64I)
INSTR (LVPACKSH,   0x28000063,  0xfc0000ff,  RD,    RA,    RB,    ORVDX64I)
INSTR (LVPACKUSB,  0x28000064,  0xfc0000ff,  RD,    RA,    RB,    ORVDX64I)
INSTR (LVPACKUSH,  0x28000065,  0xfc0000ff,  RD,    RA,    RB,    ORVDX64I)
INSTR (LVPERMN,    0x28000066,  0xfc0000ff,  RD,    RA,    RB,    ORVDX64I)
INSTR (LVRLB,      0x28000067,  0xfc0000ff,  RD,    RA,    RB,    ORVDX64I)
INSTR (LVRLH,      0x28000068,  0xfc0000ff,  RD,    RA,    RB,    ORVDX64I)
INSTR (LVSLL,      0x2800006b,  0xfc0000ff,  RD,    RA,    RB,    ORVDX64I)
INSTR (LVSLLB,     0x28000069,  0xfc0000ff,  RD,    RA,    RB,    ORVDX64I)
INSTR (LVSLLH,     0x2800006a,  0xfc0000ff,  RD,    RA,    RB,    ORVDX64I)
INSTR (LVSRAB,     0x2800006e,  0xfc0000ff,  RD,    RA,    RB,    ORVDX64I)
INSTR (LVSRAH,     0x2800006f,  0xfc0000ff,  RD,    RA,    RB,    ORVDX64I)
INSTR (LVSRL,      0x28000070,  0xfc0000ff,  RD,    RA,    RB,    ORVDX64I)
INSTR (LVSRLB,     0x2800006c,  0xfc0000ff,  RD,    RA,    RB,    ORVDX64I)
INSTR (LVSRLH,     0x2800006d,  0xfc0000ff,  RD,    RA,    RB,    ORVDX64I)
INSTR (LVSUBB,     0x28000071,  0xfc0000ff,  RD,    RA,    RB,    ORVDX64I)
INSTR (LVSUBH,     0x28000072,  0xfc0000ff,  RD,    RA,    RB,    ORVDX64I)
INSTR (LVSUBSB,    0x28000073,  0xfc0000ff,  RD,    RA,    RB,    ORVDX64I)
INSTR (LVSUBSH,    0x28000074,  0xfc0000ff,  RD,    RA,    RB,    ORVDX64I)
INSTR (LVSUBUB,    0x28000075,  0xfc0000ff,  RD,    RA,    RB,    ORVDX64I)
INSTR (LVSUBUH,    0x28000076,  0xfc0000ff,  RD,    RA,    RB,    ORVDX64I)
INSTR (LVSUBUSB,   0x28000077,  0xfc0000ff,  RD,    RA,    RB,    ORVDX64I)
INSTR (LVSUBUSH,   0x28000078,  0xfc0000ff,  RD,    RA,    RB,    ORVDX64I)
INSTR (LVUNPACKB,  0x28000079,  0xfc0000ff,  RD,    RA,    RB,    ORVDX64I)
INSTR (LVUNPACKH,  0x2800007a,  0xfc0000ff,  RD,    RA,    RB,    ORVDX64I)
INSTR (LVXOR,      0x2800007b,  0xfc0000ff,  RD,    RA,    RB,    ORVDX64I)

// operand types

TYPE (I)
TYPE (IRA)
TYPE (IRA_)
TYPE (K)
TYPE (K_)
TYPE (L)
TYPE (N)
TYPE (RA)
TYPE (RB)
TYPE (RD)

// instruction classes

CLASS (ORBIS32I)
CLASS (ORBIS32II)
CLASS (ORBIS64I)
CLASS (ORBIS64II)
CLASS (ORFPX32I)
CLASS (ORFPX32II)
CLASS (ORFPX64I)
CLASS (ORFPX64II)
CLASS (ORVDX64I)
CLASS (ORVDX64II)

#undef CLASS
#undef INSTR
#undef MNEM
#undef TYPE

// PowerPC instruction set definitions
// Copyright (C) Florian Negele

// This file is part of the Eigen Compiler Suite.

// The ECS is free software: you can redistribute it and/or modify
// it under the terms of the GNU General Public License as published by
// the Free Software Foundation, either version 3 of the License, or
// (at your option) any later version.

// The ECS is distributed in the hope that it will be useful,
// but WITHOUT ANY WARRANTY; without even the implied warranty of
// MERCHANTABILITY or FITNESS FOR A PARTICULAR PURPOSE.  See the
// GNU General Public License for more details.

// You should have received a copy of the GNU General Public License
// along with the ECS.  If not, see <https://www.gnu.org/licenses/>.

#ifndef INSTR
	#define INSTR(mnem, code, mask, type1, type2, type3, type4, type5, architecture)
#endif

#ifndef MNEM
	#define MNEM(name, mnem, ...)
#endif

#ifndef TYPE
	#define TYPE(type)
#endif

// mnemonics

MNEM (add,       ADD,       Add)
MNEM (add.,      ADDx,      Add and Record)
MNEM (addc,      ADDC,      Add Carrying)
MNEM (addc.,     ADDCx,     Add Carrying and Record)
MNEM (addco,     ADDCO,     Add Carrying with Overflow)
MNEM (addco.,    ADDCOx,    Add Carrying with Overflow and Record)
MNEM (adde,      ADDE,      Add Extended)
MNEM (adde.,     ADDEx,     Add Extended and Record)
MNEM (addeo,     ADDEO,     Add Extended with Overflow)
MNEM (addeo.,    ADDEOx,    Add Extended with Overflow and Record)
MNEM (addi,      ADDI,      Add Immediate)
MNEM (addic,     ADDIC,     Add Immediate Carrying)
MNEM (addic.,    ADDICx,    Add Immediate Carrying and Record)
MNEM (addis,     ADDIS,     Add Immediate Shifted)
MNEM (addme,     ADDME,     Add to Minus One Extended)
MNEM (addme.,    ADDMEx,    Add to Minus One Extended and Record)
MNEM (addmeo,    ADDMEO,    Add to Minus One Extended with Overflow)
MNEM (addmeo.,   ADDMEOx,   Add to Minus One Extended with Overflow and Record)
MNEM (addo,      ADDO,      Add with Overflow)
MNEM (addo.,     ADDOx,     Add with Overflow and Record)
MNEM (addze,     ADDZE,     Add to Zero Extended)
MNEM (addze.,    ADDZEx,    Add to Zero Extended and Record)
MNEM (addzeo,    ADDZEO,    Add to Zero Extended with Overflow)
MNEM (addzeo.,   ADDZEOx,   Add to Zero Extended with Overflow and Record)
MNEM (and,       AND,       AND)
MNEM (and.,      ANDx,      AND and Record)
MNEM (andc,      ANDC,      AND with Complement)
MNEM (andc.,     ANDCx,     AND with Complement and Record)
MNEM (andi.,     ANDI,      AND Immediate)
MNEM (andis.,    ANDIx,     AND Immediate Shifted)
MNEM (b,         B,         Branch)
MNEM (ba,        BA,        Branch Absolute)
MNEM (bc,        BC,        Branch Conditional)
MNEM (bca,       BCA,       Branch Conditional Absolute)
MNEM (bcctr,     BCCTR,     Branch Conditional to Count Register)
MNEM (bcctrl,    BCCTRL,    Branch Conditional to Count Register and Link)
MNEM (bcl,       BCL,       Branch Conditional and Link)
MNEM (bcla,      BCLA,      Branch Conditional and Link Absolute)
MNEM (bclr,      BCLR,      Branch Conditional to Link Register)
MNEM (bclrl,     BCLRL,     Branch Conditional to Link Register and Link)
MNEM (beq,       BEQ,       Branch if equal)
MNEM (beqa,      BEQA,      Branch if equal Absolute)
MNEM (beql,      BEQL,      Branch if equal and Link)
MNEM (beqla,     BEQLA,     Branch if equal and Link Absolute)
MNEM (bge,       BGE,       Branch if greater than or equal)
MNEM (bgea,      BGEA,      Branch if greater than or equal Absolute)
MNEM (bgel,      BGEL,      Branch if greater than or equal and Link)
MNEM (bgela,     BGELA,     Branch if greater than or equal and Link Absolute)
MNEM (bgt,       BGT,       Branch if greater than)
MNEM (bgta,      BGTA,      Branch if greater than Absolute)
MNEM (bgtl,      BGTL,      Branch if greater than and Link)
MNEM (bgtla,     BGTLA,     Branch if greater than and Link Absolute)
MNEM (bl,        BL,        Branch and Link)
MNEM (bla,       BLA,       Branch and Link Absolute)
MNEM (ble,       BLE,       Branch if less than or equal)
MNEM (blea,      BLEA,      Branch if less than or equal Absolute)
MNEM (blel,      BLEL,      Branch if less than or equal and Link)
MNEM (blela,     BLELA,     Branch if less than or equal and Link Absolute)
MNEM (blt,       BLT,       Branch if less than)
MNEM (blta,      BLTA,      Branch if less than Absolute)
MNEM (bltl,      BLTL,      Branch if less than and Link)
MNEM (bltla,     BLTLA,     Branch if less than and Link Absolute)
MNEM (bne,       BNE,       Branch if not equal)
MNEM (bnea,      BNEA,      Branch if not equal Absolute)
MNEM (bnel,      BNEL,      Branch if not equal and Link)
MNEM (bnela,     BNELA,     Branch if not equal and Link Absolute)
MNEM (bng,       BNG,       Branch if not greater than)
MNEM (bnga,      BNGA,      Branch if not greater than Absolute)
MNEM (bngl,      BNGL,      Branch if not greater than and Link)
MNEM (bngla,     BNGLA,     Branch if not greater than and Link Absolute)
MNEM (bnl,       BNL,       Branch if not less than)
MNEM (bnla,      BNLA,      Branch if not less than Absolute)
MNEM (bnll,      BNLL,      Branch if not less than and Link)
MNEM (bnlla,     BNLLA,     Branch if not less than and Link Absolute)
MNEM (bns,       BNS,       Branch if not summary overflow)
MNEM (bnsa,      BNSA,      Branch if not summary overflow Absolute)
MNEM (bnsl,      BNSL,      Branch if not summary overflow and Link)
MNEM (bnsla,     BNSLA,     Branch if not summary overflow and Link Absolute)
MNEM (bnu,       BNU,       Branch if not unordered)
MNEM (bnua,      BNUA,      Branch if not unordered Absolute)
MNEM (bnul,      BNUL,      Branch if not unordered and Link)
MNEM (bnula,     BNULA,     Branch if not unordered and Link Absolute)
MNEM (bso,       BSO,       Branch if summary overflow)
MNEM (bsoa,      BSOA,      Branch if summary overflow Absolute)
MNEM (bsol,      BSOL,      Branch if summary overflow and Link)
MNEM (bsola,     BSOLA,     Branch if summary overflow and Link Absolute)
MNEM (bun,       BUN,       Branch if unordered)
MNEM (buna,      BUNA,      Branch if unordered Absolute)
MNEM (bunl,      BUNL,      Branch if unordered and Link)
MNEM (bunla,     BUNLA,     Branch if unordered and Link Absolute)
MNEM (cmp,       CMP,       Compare)
MNEM (cmpd,      CMPD,      Compare Double Word)
MNEM (cmpdi,     CMPDI,     Compare Double Word Immediate)
MNEM (cmpi,      CMPI,      Compare Immediate)
MNEM (cmpl,      CMPL,      Compare Logical)
MNEM (cmpld,     CMPLD,     Compare Double Word Logical)
MNEM (cmpldi,    CMPLDI,    Compare Double Word Logical Immediate)
MNEM (cmpli,     CMPLI,     Compare Logical Immediate)
MNEM (cmplw,     CMPLW,     Compare Word Logical)
MNEM (cmplwi,    CMPLWI,    Compare Word Logical Immediate)
MNEM (cmpw,      CMPW,      Compare Word)
MNEM (cmpwi,     CMPWI,     Compare Word Immediate)
MNEM (cntlzd,    CNTLZD,    Count Leading Zeros Double Word)
MNEM (cntlzd.,   CNTLZDx,   Count Leading Zeros Double Word and Record)
MNEM (cntlzw,    CNTLZW,    Count Leading Zeros Double Word)
MNEM (cntlzw.,   CNTLZWx,   Count Leading Zeros Double Word and Record)
MNEM (crand,     CRAND,     Condition Register AND)
MNEM (crandc,    CRANDC,    Condition Register AND with Complement)
MNEM (creqv,     CREQV,     Condition Register Equivalent)
MNEM (crnand,    CRNAND,    Condition Register NAND)
MNEM (crnor,     CRNOR,     Condition Register NOR)
MNEM (cror,      CROR,      Condition Register OR)
MNEM (crorc,     CRORC,     Condition Register OR with Complement)
MNEM (crxor,     CRXOR,     Condition Register XOR)
MNEM (dcbf,      DCBF,      Data Cache Block Flush)
MNEM (dcbi,      DCBI,      Data Cache Block Invalidate)
MNEM (dcbst,     DCBST,     Data Cache Block Store)
MNEM (dcbt,      DCBT,      Data Cache Block Touch)
MNEM (dcbtst,    DCBTST,    Data Cache Block Touch for Store)
MNEM (dcbz,      DCBZ,      Data Cache Block Clear to Zero)
MNEM (divd,      DIVD,      Divide Double Word)
MNEM (divd.,     DIVDx,     Divide Double Word and Record)
MNEM (divdo,     DIVDO,     Divide Double Word with Overflow)
MNEM (divdo.,    DIVDOx,    Divide Double Word with Overflow and Record)
MNEM (divdu,     DIVDU,     Divide Double Word Unsigned)
MNEM (divdu.,    DIVDUx,    Divide Double Word Unsigned and Record)
MNEM (divduo,    DIVDUO,    Divide Double Word Unsigned with Overflow)
MNEM (divduo.,   DIVDUOx,   Divide Double Word Unsigned with Overflow and Record)
MNEM (divw,      DIVW,      Divide Word)
MNEM (divw.,     DIVWx,     Divide Word and Record)
MNEM (divwo,     DIVWO,     Divide Word with Overflow)
MNEM (divwo.,    DIVWOx,    Divide Word with Overflow and Record)
MNEM (divwu,     DIVWU,     Divide Word Unsigned)
MNEM (divwu.,    DIVWUx,    Divide Word Unsigned and Record)
MNEM (divwuo,    DIVWUO,    Divide Word Unsigned with Overflow)
MNEM (divwuo.,   DIVWUOx,   Divide Word Unsigned with Overflow and Record)
MNEM (eciwx,     ECIWX,     External Control In Word Indexed)
MNEM (ecowx,     ECOWX,     External Control Out Word Indexed)
MNEM (eieio,     EIEIO,     Enforce In-Order Execution of I/O)
MNEM (eqv,       EQV,       Equivalent)
MNEM (eqv.,      EQVx,      Equivalent and Record)
MNEM (extsb,     EXTSB,     Extend Sign Byte)
MNEM (extsb.,    EXTSBx,    Extend Sign Byte and Record)
MNEM (extsh,     EXTSH,     Extend Sign Half Word)
MNEM (extsh.,    EXTSHx,    Extend Sign Half Word and Record)
MNEM (extsw,     EXTSW,     Extend Sign Word)
MNEM (extsw.,    EXTSWx,    Extend Sign Word and Record)
MNEM (fabs,      FABS,      Floating Absolute Value)
MNEM (fabs.,     FABSx,     Floating Absolute Value and Record)
MNEM (fadd,      FADD,      Floating Add)
MNEM (fadd.,     FADDx,     Floating Add and Record)
MNEM (fadds,     FADDS,     Floating Add Single)
MNEM (fadds.,    FADDSx,    Floating Add Single and Record)
MNEM (fcfid,     FCFID,     Floating Convert from Integer Double Word)
MNEM (fcfid.,    FCFIDx,    Floating Convert from Integer Double Word and Record)
MNEM (fcmpo,     FCMPO,     Floating Compare Ordered)
MNEM (fcmpu,     FCMPU,     Floating Compare Unordered)
MNEM (fctid,     FCTID,     Floating Convert to Integer Double Word)
MNEM (fctid.,    FCTIDx,    Floating Convert to Integer Double Word and Record)
MNEM (fctidz,    FCTIDZ,    Floating Convert to Integer Double Word with Round toward Zero)
MNEM (fctidz.,   FCTIDZx,   Floating Convert to Integer Double Word with Round toward Zero and Record)
MNEM (fctiw,     FCTIW,     Floating Convert to Integer Word)
MNEM (fctiw.,    FCTIWx,    Floating Convert to Integer Word and Record)
MNEM (fctiwz,    FCTIWZ,    Floating Convert to Integer Word with Round toward Zero)
MNEM (fctiwz.,   FCTIWZx,   Floating Convert to Integer Word with Round toward Zero and Record)
MNEM (fdiv,      FDIV,      Floating Divide)
MNEM (fdiv.,     FDIVx,     Floating Divide and Record)
MNEM (fdivs,     FDIVS,     Floating Divide Single)
MNEM (fdivs.,    FDIVSx,    Floating Divide Single and Record)
MNEM (fmadd,     FMADD,     Floating Multiply-Add)
MNEM (fmadd.,    FMADDx,    Floating Multiply-Add and Record)
MNEM (fmadds,    FMADDS,    Floating Multiply-Add Single)
MNEM (fmadds.,   FMADDSx,   Floating Multiply-Add Single and Record)
MNEM (fmr,       FMR,       Floating Move Register)
MNEM (fmr.,      FMRx,      Floating Move Register and Record)
MNEM (fmsub,     FMSUB,     Floating Multiply-Subtract)
MNEM (fmsub.,    FMSUBx,    Floating Multiply-Subtract and Record)
MNEM (fmsubs,    FMSUBS,    Floating Multiply-Subtract Single)
MNEM (fmsubs.,   FMSUBSx,   Floating Multiply-Subtract Single and Record)
MNEM (fmul,      FMUL,      Floating Multiply)
MNEM (fmul.,     FMULx,     Floating Multiply and Record)
MNEM (fmuls,     FMULS,     Floating Multiply Single)
MNEM (fmuls.,    FMULSx,    Floating Multiply Single and Record)
MNEM (fnabs,     FNABS,     Floating Negative Absolute Value)
MNEM (fnabs.,    FNABSx,    Floating Negative Absolute Value and Record)
MNEM (fneg,      FNEG,      Floating Negate)
MNEM (fneg.,     FNEGx,     Floating Negate and Record)
MNEM (fnmadd,    FNMADD,    Floating Negative Multiply-Add)
MNEM (fnmadd.,   FNMADDx,   Floating Negative Multiply-Add and Record)
MNEM (fnmadds,   FNMADDS,   Floating Negative Multiply-Add Single)
MNEM (fnmadds.,  FNMADDSx,  Floating Negative Multiply-Add Single and Record)
MNEM (fnmsub,    FNMSUB,    Floating Negative Multiply-Subtract)
MNEM (fnmsub.,   FNMSUBx,   Floating Negative Multiply-Subtract and Record)
MNEM (fnmsubs,   FNMSUBS,   Floating Negative Multiply-Subtract Single)
MNEM (fnmsubs.,  FNMSUBSx,  Floating Negative Multiply-Subtract Single and Record)
MNEM (fres,      FRES,      Floating Reciprocal Estimate Single)
MNEM (fres.,     FRESx,     Floating Reciprocal Estimate Single and Record)
MNEM (frsp,      FRSP,      Floating Round to Single)
MNEM (frsp.,     FRSPx,     Floating Round to Single and Record)
MNEM (frsqrte,   FRSQRTE,   Floating Reciprocal Square Root Estimate)
MNEM (frsqrte.,  FRSQRTEx,  Floating Reciprocal Square Root Estimate and Record)
MNEM (fsel,      FSEL,      Floating Select)
MNEM (fsel.,     FSELx,     Floating Select)
MNEM (fsqrt,     FSQRT,     Floating Square Root)
MNEM (fsqrt.,    FSQRTx,    Floating Square Root and Record)
MNEM (fsqrts,    FSQRTS,    Floating Square Root Single)
MNEM (fsqrts.,   FSQRTSx,   Floating Square Root Single and Record)
MNEM (fsub,      FSUB,      Floating Subtract)
MNEM (fsub.,     FSUBx,     Floating Subtract and Record)
MNEM (fsubs,     FSUBS,     Floating Subtract Single)
MNEM (fsubs.,    FSUBSx,    Floating Subtract Single and Record)
MNEM (icbi,      ICBI,      Instruction Cache Block Invalidate)
MNEM (isync,     ISYNC,     Instruction Synchronize)
MNEM (la,        LA,        Load Address)
MNEM (lbz,       LBZ,       Load Byte and Zero)
MNEM (lbzu,      LBZU,      Load Byte and Zero with Update)
MNEM (lbzux,     LBZUX,     Load Byte and Zero with Update Indexed)
MNEM (lbzx,      LBZX,      Load Byte and Zero Indexed)
MNEM (ld,        LD,        Load Double Word)
MNEM (ldarx,     LDARX,     Load Double Word and Reserve Indexed)
MNEM (ldu,       LDU,       Load Double Word with Update)
MNEM (ldux,      LDUX,      Load Double Word with Update Indexed)
MNEM (ldx,       LDX,       Load Double Word Indexed)
MNEM (lfd,       LFD,       Load Floating-Point Double)
MNEM (lfdu,      LFDU,      Load Floating-Point Double with Update)
MNEM (lfdux,     LFDUX,     Load Floating-Point Double with Update Indexed)
MNEM (lfdx,      LFDX,      Load Floating-Point Double Indexed)
MNEM (lfs,       LFS,       Load Floating-Point Single)
MNEM (lfsu,      LFSU,      Load Floating-Point Single with Update)
MNEM (lfsux,     LFSUX,     Load Floating-Point Single with Update Indexed)
MNEM (lfsx,      LFSX,      Load Floating-Point Single Indexed)
MNEM (lha,       LHA,       Load Half Word Algebraic)
MNEM (lhau,      LHAU,      Load Half Word Algebraic with Update)
MNEM (lhaux,     LHAUX,     Load Half Word Algebraic with Update Indexed)
MNEM (lhax,      LHAX,      Load Half Word Algebraic Indexed)
MNEM (lhbrx,     LHBRX,     Load Half Word Byte-Reverse Indexed)
MNEM (lhz,       LHZ,       Load Half Word and Zero)
MNEM (lhzu,      LHZU,      Load Half Word and Zero with Update)
MNEM (lhzux,     LHZUX,     Load Half Word and Zero with Update Indexed)
MNEM (lhzx,      LHZX,      Load Half Word and Zero Indexed)
MNEM (li,        LI,        Load Immediate)
MNEM (lis,       LIS,       Load Immediate Shifted)
MNEM (lmw,       LMW,       Load Multiple Word)
MNEM (lswi,      LSWI,      Load String Word Immediate)
MNEM (lswx,      LSWX,      Load String Word Indexed)
MNEM (lwa,       LWA,       Load Word Algebraic)
MNEM (lwarx,     LWARX,     Load Word and Reserve Indexed)
MNEM (lwaux,     LWAUX,     Load Word Algebraic with Update Indexed)
MNEM (lwax,      LWAX,      Load Word Algebraic Indexed)
MNEM (lwbrx,     LWBRX,     Load Word Byte-Reverse Indexed)
MNEM (lwz,       LWZ,       Load Word and Zero)
MNEM (lwzu,      LWZU,      Load Word and Zero with Update)
MNEM (lwzux,     LWZUX,     Load Word and Zero with Update Indexed)
MNEM (lwzx,      LWZX,      Load Word and Zero Indexed)
MNEM (mcrf,      MCRF,      Move Condition Register Field)
MNEM (mcrfs,     MCRFS,     Move to Condition Register from FPSCR)
MNEM (mcrxr,     MCRCR,     Move to Condition Register from XER)
MNEM (mfcr,      MFCR,      Move from Condition Register)
MNEM (mffs,      MFFS,      Move from FPSCR)
MNEM (mffs.,     MFFSx,     Move from FPSCR and Record)
MNEM (mfmsr,     MFMSR,     Move from Machine State Register)
MNEM (mfocrf,    MFOCRF,    Move from One Condition Register Field)
MNEM (mfspr,     MFSPR,     Move from Special-Purpose Register)
MNEM (mfsr,      MFSR,      Move from Segment Register)
MNEM (mfsrin,    MFSRIN,    Move from Segment Register Indirect)
MNEM (mftb,      MFTB,      Move from Time Base)
MNEM (mr,        MR,        Move Register)
MNEM (mr.,       MRx,       Move Register and Record)
MNEM (mtcrf,     MTCRF,     Move to Condition Register Fields)
MNEM (mtfsb0,    MTFSB0,    Move to FPSCR Bit 0)
MNEM (mtfsb0.,   MTFSB0x,   Move to FPSCR Bit 0 and Record)
MNEM (mtfsb1,    MTFSB1,    Move to FPSCR Bit 1)
MNEM (mtfsb1.,   MTFSB1x,   Move to FPSCR Bit 1 and Record)
MNEM (mtfsf,     MTFSF,     Move to FPSCR Fields)
MNEM (mtfsf.,    MTFSFx,    Move to FPSCR Fields and Record)
MNEM (mtfsfi,    MTFSFI,    Move to FPSCR Field Immediate)
MNEM (mtfsfi.,   MTFSFIx,   Move to FPSCR Field Immediate and Record)
MNEM (mtmsr,     MTMSR,     Move to Machine State Register)
MNEM (mtmsrd,    MTMSRD,    Move to Machine State Register Double Word)
MNEM (mtocrf,    MTOCRF,    Move to One Condition Register Field)
MNEM (mtspr,     MTSPR,     Move to Special-Purpose Register)
MNEM (mtsr,      MTSR,      Move to Segment Register)
MNEM (mtsrd,     MTSRD,     Move to Segment Register Double Word)
MNEM (mtsrdin,   MTSRDIN,   Move to Segment Register Double Word Indirect)
MNEM (mtsrin,    MTSRIN,    Move to Segment Register Indirect)
MNEM (mulhd,     MULHD,     Multiply High Double Word)
MNEM (mulhd.,    MULHDx,    Multiply High Double Word and Record)
MNEM (mulhdu,    MULHDU,    Multiply High Double Word Unsigned)
MNEM (mulhdu.,   MULHDUx,   Multiply High Double Word Unsigned and Record)
MNEM (mulhw,     MULHW,     Multiply High Word)
MNEM (mulhw.,    MULHWx,    Multiply High Word and Record)
MNEM (mulhwu,    MULHWU,    Multiply High Word Unsigned)
MNEM (mulhwu.,   MULHWUx,   Multiply High Word Unsigned and Record)
MNEM (mulld,     MULLD,     Multiply Low Double Word)
MNEM (mulld.,    MULLDx,    Multiply Low Double Word and Record)
MNEM (mulldo,    MULLDO,    Multiply Low Double Word with Overflow)
MNEM (mulldo.,   MULLDOx,   Multiply Low Double Word with Overflow and Record)
MNEM (mulli,     MULLI,     Multiply Low Immediate)
MNEM (mullw,     MULLW,     Multiply Low Word)
MNEM (mullw.,    MULLWx,    Multiply Low Word and Record)
MNEM (mullwo,    MULLWO,    Multiply Low Word with Overflow)
MNEM (mullwo.,   MULLWOx,   Multiply Low Word with Overflow and Record)
MNEM (nand,      NAND,      NAND)
MNEM (nand.,     NANDx,     NAND and Record)
MNEM (neg,       NEG,       Negate)
MNEM (neg.,      NEGx,      Negate and Record)
MNEM (nego,      NEGO,      Negate with Overflow)
MNEM (nego.,     NEGOx,     Negate with Overflow and Record)
MNEM (nop,       NOP,       No-Op)
MNEM (nor,       NOR,       NOR)
MNEM (nor.,      NORx,      NOR and Record)
MNEM (not,       NOT,       Complement Register)
MNEM (not.,      NOTx,      Complement Register and Record)
MNEM (or,        OR,        OR)
MNEM (or.,       ORx,       OR and Record)
MNEM (orc,       ORC,       OR with Complement)
MNEM (orc.,      ORCx,      OR with Complement and Record)
MNEM (ori,       ORI,       OR Immediate)
MNEM (oris,      ORIS,      OR Immediate Shifted)
MNEM (rfi,       RFI,       Return from Interrupt)
MNEM (rfid,      RFID,      Return from Interrupt Double Word)
MNEM (rldcl,     RLDCL,     Rotate Left Double Word then Clear Left)
MNEM (rldcl.,    RLDCLx,    Rotate Left Double Word then Clear Left and Record)
MNEM (rldcr,     RLDCR,     Rotate Left Double Word then Clear Right)
MNEM (rldcr.,    RLDCRx,    Rotate Left Double Word then Clear Right and Record)
MNEM (rldic,     RLDIC,     Rotate Left Double Word Immediate then Clear)
MNEM (rldic.,    RLDICx,    Rotate Left Double Word Immediate then Clear and Record)
MNEM (rldicl,    RLDICL,    Rotate Left Double Word Immediate then Clear Left)
MNEM (rldicl.,   RLDICLx,   Rotate Left Double Word Immediate then Clear Left and Record)
MNEM (rldicr,    RLDICR,    Rotate Left Double Word Immediate then Clear Right)
MNEM (rldicr.,   RLDICRx,   Rotate Left Double Word Immediate then Clear Right and Record)
MNEM (rldimi,    RLDIMI,    Rotate Left Double Word Immediate then Mask Insert)
MNEM (rldimi.,   RLDIMIx,   Rotate Left Double Word Immediate then Mask Insert and Record)
MNEM (rlwimi,    RLWIMI,    Rotate Left Word Immediate then Mask Insert)
MNEM (rlwimi.,   RLWIMIx,   Rotate Left Word Immediate then Mask Insert and Record)
MNEM (rlwinm,    RLWINM,    Rotate Left Word Immediate then AND with Mask)
MNEM (rlwinm.,   RLWINMx,   Rotate Left Word Immediate then AND with Mask and Record)
MNEM (rlwnm,     RLWNM,     Rotate Left Word then AND with Mask)
MNEM (rlwnm.,    RLWNMx,    Rotate Left Word then AND with Mask and Record)
MNEM (sc,        SC,        System Call)
MNEM (slbia,     SLBIA,     SLB Invalidate All)
MNEM (slbie,     SLBIE,     SLB Invalidate Entry)
MNEM (slbmfee,   SLBMFEE,   SLB Move From Entry ESID)
MNEM (slbmfev,   SLBMFEV,   SLB Move From Entry VSID)
MNEM (slbmte,    SLBMTE,    SLB Move To Entry)
MNEM (sld,       SLD,       Shift Left Double Word)
MNEM (sld.,      SLDx,      Shift Left Double Word and Record)
MNEM (slw,       SLW,       Shift Left Word)
MNEM (slw.,      SLWx,      Shift Left Word and Record)
MNEM (srad,      SRAD,      Shift Right Algebraic Double Word)
MNEM (srad.,     SRADx,     Shift Right Algebraic Double Word and Record)
MNEM (sradi,     SRADI,     Shift Right Algebraic Double Word Immediate)
MNEM (sradi.,    SRADIx,    Shift Right Algebraic Double Word Immediate and Record)
MNEM (sraw,      SRAW,      Shift Right Algebraic Word)
MNEM (sraw.,     SRAWx,     Shift Right Algebraic Word and Record)
MNEM (srawi,     SRAWI,     Shift Right Algebraic Word Immediate)
MNEM (srawi.,    SRAWIx,    Shift Right Algebraic Word Immediate and Record)
MNEM (srd,       SRD,       Shift Right Double Word)
MNEM (srd.,      SRDx,      Shift Right Double Word and Record)
MNEM (srw,       SRW,       Shift Right Word)
MNEM (srw.,      SRWx,      Shift Right Word and Record)
MNEM (stb,       STB,       Store Byte)
MNEM (stbu,      STBU,      Store Byte with Update)
MNEM (stbux,     STBUX,     Store Byte with Update Indexed)
MNEM (stbx,      STBX,      Store Byte Indexed)
MNEM (std,       STD,       Store Double Word)
MNEM (stdcx.,    STDCX,     Store Double Word Conditional Indexed)
MNEM (stdu,      STDU,      Store Double Word with Update)
MNEM (stdux,     STDUX,     Store Double Word with Update Indexed)
MNEM (stdx,      STDX,      Store Double Word Indexed)
MNEM (stfd,      STFD,      Store Floating-Point Double)
MNEM (stfdu,     STFDU,     Store Floating-Point Double with Update)
MNEM (stfdux,    STFDUX,    Store Floating-Point Double with Update Indexed)
MNEM (stfdx,     STFDX,     Store Floating-Point Double Indexed)
MNEM (stfiwx,    STFIWX,    Store Floating-Point as Integer Word Indexed)
MNEM (stfs,      STFS,      Store Floating-Point Single)
MNEM (stfsu,     STFSU,     Store Floating-Point Single with Update)
MNEM (stfsux,    STFSUX,    Store Floating-Point Single with Update Indexed)
MNEM (stfsx,     STFSX,     Store Floating-Point Single Indexed)
MNEM (sth,       STH,       Store Half Word)
MNEM (sthbrx,    STHBRX,    Store Half Word Byte-Reverse Indexed)
MNEM (sthu,      STHU,      Store Half Word with Update)
MNEM (sthux,     STHUX,     Store Half Word with Update Indexed)
MNEM (sthx,      STHX,      Store Half Word Indexed)
MNEM (stmw,      STMW,      Store Multiple Word)
MNEM (stswi,     STSWI,     Store String Word Immediate)
MNEM (stswx,     STSWX,     Store String Word Indexed)
MNEM (stw,       STW,       Store Word)
MNEM (stwbrx,    STWBRX,    Store Word Byte-Reverse Indexed)
MNEM (stwcx.,    STWCX,     Store Word Conditional Indexed)
MNEM (stwu,      STWU,      Store Word with Update)
MNEM (stwux,     STWUX,     Store Word with Update Indexed)
MNEM (stwx,      STWX,      Store Word Indexed)
MNEM (sub,       SUB,       Subtract)
MNEM (sub.,      SUBx,      Subtract and Record)
MNEM (subc,      SUBC,      Subtract Carrying)
MNEM (subc.,     SUBCx,     Subtract Carrying and Record)
MNEM (subco,     SUBCO,     Subtract Carrying with Overflow)
MNEM (subco.,    SUBCOx,    Subtract Carrying with Overflow and Record)
MNEM (subf,      SUBF,      Subtract From)
MNEM (subf.,     SUBFx,     Subtract From and Record)
MNEM (subfc,     SUBFC,     Subtract from Carrying)
MNEM (subfc.,    SUBFCx,    Subtract from Carrying and Record)
MNEM (subfco,    SUBFCO,    Subtract from Carrying with Overflow)
MNEM (subfco.,   SUBFCOx,   Subtract from Carrying with Overflow and Record)
MNEM (subfe,     SUBFE,     Subtract from Extended)
MNEM (subfe.,    SUBFEx,    Subtract from Extended and Record)
MNEM (subfeo,    SUBFEO,    Subtract from Extended with Overflow)
MNEM (subfeo.,   SUBFEOx,   Subtract from Extended with Overflow and Record)
MNEM (subfic,    SUBFIC,    Subtract from Immediate Carrying)
MNEM (subfme,    SUBFME,    Subtract from Minus One Extended)
MNEM (subfme.,   SUBFMEx,   Subtract from Minus One Extended and Record)
MNEM (subfmeo,   SUBFMEO,   Subtract from Minus One Extended with Overflow)
MNEM (subfmeo.,  SUBFMEOx,  Subtract from Minus One Extended with Overflow and Record)
MNEM (subfo,     SUBFO,     Subtract From with Overflow)
MNEM (subfo.,    SUBFOx,    Subtract From with Overflow and Record)
MNEM (subfze,    SUBFZE,    Subtract from Zero Extended)
MNEM (subfze.,   SUBFZEx,   Subtract from Zero Extended and Record)
MNEM (subfzeo,   SUBFZEO,   Subtract from Zero Extended with Overflow)
MNEM (subfzeo.,  SUBFZEOx,  Subtract from Zero Extended with Overflow and Record)
MNEM (subi,      SUBI,      Subtract Immediate)
MNEM (subic,     SUBIC,     Subtract Immediate Carrying)
MNEM (subic.,    SUBICx,    Subtract Immediate Carrying and Record)
MNEM (subis,     SUBIS,     Subtract Immediate Shifted)
MNEM (subo,      SUBO,      Subtract with Overflow)
MNEM (subo.,     SUBOx,     Subtract with Overflow and Record)
MNEM (sync,      SYNC,      Synchronize)
MNEM (td,        TD,        Trap Double Word)
MNEM (tdi,       TDI,       Trap Double Word Immediate)
MNEM (tlbia,     TLBIA,     Translation Lookaside Buffer Invalidate All)
MNEM (tlbie,     TLBIE,     Translation Lookaside Buffer Invalidate Entry)
MNEM (tlbiel,    TLBIEL,    Translation Lookaside Buffer Invalidate Entry Local)
MNEM (tlbsync,   TLBSYNC,   TLB Synchronize)
MNEM (tw,        TW,        Trap Word)
MNEM (twi,       TWI,       Trap Word Immediate)
MNEM (xor,       XOR,       XOR)
MNEM (xor.,      XORx,      XOR and Record)
MNEM (xori,      XORI,      XOR Immediate)
MNEM (xoris,     XORIS,     XOR Immediate Shifted)

// simplified instructions

INSTR (BLT,     0x41800000,  0xffff0003,  BD,    Void,  Void,  Void,  Void,  PPC32 | PPC64)
INSTR (BLE,     0x40810000,  0xffff0003,  BD,    Void,  Void,  Void,  Void,  PPC32 | PPC64)
INSTR (BEQ,     0x41820000,  0xffff0003,  BD,    Void,  Void,  Void,  Void,  PPC32 | PPC64)
INSTR (BGE,     0x40800000,  0xffff0003,  BD,    Void,  Void,  Void,  Void,  PPC32 | PPC64)
INSTR (BGT,     0x41810000,  0xffff0003,  BD,    Void,  Void,  Void,  Void,  PPC32 | PPC64)
INSTR (BNL,     0x40800000,  0xffff0003,  BD,    Void,  Void,  Void,  Void,  PPC32 | PPC64)
INSTR (BNE,     0x40820000,  0xffff0003,  BD,    Void,  Void,  Void,  Void,  PPC32 | PPC64)
INSTR (BNG,     0x40810000,  0xffff0003,  BD,    Void,  Void,  Void,  Void,  PPC32 | PPC64)
INSTR (BSO,     0x41830000,  0xffff0003,  BD,    Void,  Void,  Void,  Void,  PPC32 | PPC64)
INSTR (BNS,     0x40830000,  0xffff0003,  BD,    Void,  Void,  Void,  Void,  PPC32 | PPC64)
INSTR (BUN,     0x41830000,  0xffff0003,  BD,    Void,  Void,  Void,  Void,  PPC32 | PPC64)
INSTR (BNU,     0x40830000,  0xffff0003,  BD,    Void,  Void,  Void,  Void,  PPC32 | PPC64)
INSTR (BLTA,    0x41800002,  0xffff0003,  BD,    Void,  Void,  Void,  Void,  PPC32 | PPC64)
INSTR (BLEA,    0x40810002,  0xffff0003,  BD,    Void,  Void,  Void,  Void,  PPC32 | PPC64)
INSTR (BEQA,    0x41820002,  0xffff0003,  BD,    Void,  Void,  Void,  Void,  PPC32 | PPC64)
INSTR (BGEA,    0x40800002,  0xffff0003,  BD,    Void,  Void,  Void,  Void,  PPC32 | PPC64)
INSTR (BGTA,    0x41810002,  0xffff0003,  BD,    Void,  Void,  Void,  Void,  PPC32 | PPC64)
INSTR (BNLA,    0x40800002,  0xffff0003,  BD,    Void,  Void,  Void,  Void,  PPC32 | PPC64)
INSTR (BNEA,    0x40820002,  0xffff0003,  BD,    Void,  Void,  Void,  Void,  PPC32 | PPC64)
INSTR (BNGA,    0x40810002,  0xffff0003,  BD,    Void,  Void,  Void,  Void,  PPC32 | PPC64)
INSTR (BSOA,    0x41830002,  0xffff0003,  BD,    Void,  Void,  Void,  Void,  PPC32 | PPC64)
INSTR (BNSA,    0x40830002,  0xffff0003,  BD,    Void,  Void,  Void,  Void,  PPC32 | PPC64)
INSTR (BUNA,    0x41830002,  0xffff0003,  BD,    Void,  Void,  Void,  Void,  PPC32 | PPC64)
INSTR (BNUA,    0x40830002,  0xffff0003,  BD,    Void,  Void,  Void,  Void,  PPC32 | PPC64)
INSTR (BLTL,    0x41800001,  0xffff0003,  BD,    Void,  Void,  Void,  Void,  PPC32 | PPC64)
INSTR (BLEL,    0x40810001,  0xffff0003,  BD,    Void,  Void,  Void,  Void,  PPC32 | PPC64)
INSTR (BEQL,    0x41820001,  0xffff0003,  BD,    Void,  Void,  Void,  Void,  PPC32 | PPC64)
INSTR (BGEL,    0x40800001,  0xffff0003,  BD,    Void,  Void,  Void,  Void,  PPC32 | PPC64)
INSTR (BGTL,    0x41810001,  0xffff0003,  BD,    Void,  Void,  Void,  Void,  PPC32 | PPC64)
INSTR (BNLL,    0x40800001,  0xffff0003,  BD,    Void,  Void,  Void,  Void,  PPC32 | PPC64)
INSTR (BNEL,    0x40820001,  0xffff0003,  BD,    Void,  Void,  Void,  Void,  PPC32 | PPC64)
INSTR (BNGL,    0x40810001,  0xffff0003,  BD,    Void,  Void,  Void,  Void,  PPC32 | PPC64)
INSTR (BSOL,    0x41830001,  0xffff0003,  BD,    Void,  Void,  Void,  Void,  PPC32 | PPC64)
INSTR (BNSL,    0x40830001,  0xffff0003,  BD,    Void,  Void,  Void,  Void,  PPC32 | PPC64)
INSTR (BUNL,    0x41830001,  0xffff0003,  BD,    Void,  Void,  Void,  Void,  PPC32 | PPC64)
INSTR (BNUL,    0x40830001,  0xffff0003,  BD,    Void,  Void,  Void,  Void,  PPC32 | PPC64)
INSTR (BLTLA,   0x41800003,  0xffff0003,  BD,    Void,  Void,  Void,  Void,  PPC32 | PPC64)
INSTR (BLELA,   0x40810003,  0xffff0003,  BD,    Void,  Void,  Void,  Void,  PPC32 | PPC64)
INSTR (BEQLA,   0x41820003,  0xffff0003,  BD,    Void,  Void,  Void,  Void,  PPC32 | PPC64)
INSTR (BGELA,   0x40800003,  0xffff0003,  BD,    Void,  Void,  Void,  Void,  PPC32 | PPC64)
INSTR (BGTLA,   0x41810003,  0xffff0003,  BD,    Void,  Void,  Void,  Void,  PPC32 | PPC64)
INSTR (BNLLA,   0x40800003,  0xffff0003,  BD,    Void,  Void,  Void,  Void,  PPC32 | PPC64)
INSTR (BNELA,   0x40820003,  0xffff0003,  BD,    Void,  Void,  Void,  Void,  PPC32 | PPC64)
INSTR (BNGLA,   0x40810003,  0xffff0003,  BD,    Void,  Void,  Void,  Void,  PPC32 | PPC64)
INSTR (BSOLA,   0x41830003,  0xffff0003,  BD,    Void,  Void,  Void,  Void,  PPC32 | PPC64)
INSTR (BNSLA,   0x40830003,  0xffff0003,  BD,    Void,  Void,  Void,  Void,  PPC32 | PPC64)
INSTR (BUNLA,   0x41830003,  0xffff0003,  BD,    Void,  Void,  Void,  Void,  PPC32 | PPC64)
INSTR (BNULA,   0x40830003,  0xffff0003,  BD,    Void,  Void,  Void,  Void,  PPC32 | PPC64)
INSTR (CMPD,    0x7c200000,  0xfc2007fe,  CRFD,  RA,    RB,    Void,  Void,  PPC64)
INSTR (CMPDI,   0x2c200000,  0xfc200000,  CRFD,  RA,    SIMM,  Void,  Void,  PPC64)
INSTR (CMPLD,   0x7c200040,  0xfc2007fe,  CRFD,  RA,    RB,    Void,  Void,  PPC64)
INSTR (CMPLDI,  0x28200000,  0xfc200000,  CRFD,  RA,    UIMM,  Void,  Void,  PPC64)
INSTR (CMPW,    0x7c000000,  0xfc2007fe,  CRFD,  RA,    RB,    Void,  Void,  PPC32 | PPC64)
INSTR (CMPWI,   0x2c000000,  0xfc200000,  CRFD,  RA,    SIMM,  Void,  Void,  PPC32 | PPC64)
INSTR (CMPLW,   0x7c000040,  0xfc2007fe,  CRFD,  RA,    RB,    Void,  Void,  PPC32 | PPC64)
INSTR (CMPLWI,  0x28000000,  0xfc200000,  CRFD,  RA,    UIMM,  Void,  Void,  PPC32 | PPC64)
INSTR (LI,      0x38000000,  0xfc1f0000,  RD,    SIMM,  Void,  Void,  Void,  PPC32 | PPC64)
INSTR (LIS,     0x3c000000,  0xfc1f0000,  RD,    SIMM,  Void,  Void,  Void,  PPC32 | PPC64)
INSTR (NOP,     0x60000000,  0xffffffff,  Void,  Void,  Void,  Void,  Void,  PPC32 | PPC64)

// instructions

INSTR (ADD,       0x7c000214,  0xfc0007ff,  RD,    RA,    RB,     Void,  Void,  PPC32 | PPC64)
INSTR (ADDx,      0x7c000215,  0xfc0007ff,  RD,    RA,    RB,     Void,  Void,  PPC32 | PPC64)
INSTR (ADDO,      0x7c000614,  0xfc0007ff,  RD,    RA,    RB,     Void,  Void,  PPC32 | PPC64)
INSTR (ADDOx,     0x7c000615,  0xfc0007ff,  RD,    RA,    RB,     Void,  Void,  PPC32 | PPC64)
INSTR (ADDC,      0x7c000014,  0xfc0007ff,  RD,    RA,    RB,     Void,  Void,  PPC32 | PPC64)
INSTR (ADDCx,     0x7c000015,  0xfc0007ff,  RD,    RA,    RB,     Void,  Void,  PPC32 | PPC64)
INSTR (ADDCO,     0x7c000414,  0xfc0007ff,  RD,    RA,    RB,     Void,  Void,  PPC32 | PPC64)
INSTR (ADDCOx,    0x7c000415,  0xfc0007ff,  RD,    RA,    RB,     Void,  Void,  PPC32 | PPC64)
INSTR (ADDE,      0x7c000114,  0xfc0007ff,  RD,    RA,    RB,     Void,  Void,  PPC32 | PPC64)
INSTR (ADDEx,     0x7c000115,  0xfc0007ff,  RD,    RA,    RB,     Void,  Void,  PPC32 | PPC64)
INSTR (ADDEO,     0x7c000514,  0xfc0007ff,  RD,    RA,    RB,     Void,  Void,  PPC32 | PPC64)
INSTR (ADDEOx,    0x7c000515,  0xfc0007ff,  RD,    RA,    RB,     Void,  Void,  PPC32 | PPC64)
INSTR (ADDI,      0x38000000,  0xfc000000,  RD,    RA,    SIMM,   Void,  Void,  PPC32 | PPC64)
INSTR (ADDIC,     0x30000000,  0xfc000000,  RD,    RA,    SIMM,   Void,  Void,  PPC32 | PPC64)
INSTR (ADDICx,    0x34000000,  0xfc000000,  RD,    RA,    SIMM,   Void,  Void,  PPC32 | PPC64)
INSTR (ADDIS,     0x3c000000,  0xfc000000,  RD,    RA,    SIMM,   Void,  Void,  PPC32 | PPC64)
INSTR (ADDME,     0x7c0001d4,  0xfc0007ff,  RD,    RA,    Void,   Void,  Void,  PPC32 | PPC64)
INSTR (ADDMEx,    0x7c0001d5,  0xfc0007ff,  RD,    RA,    Void,   Void,  Void,  PPC32 | PPC64)
INSTR (ADDMEO,    0x7c0005d4,  0xfc0007ff,  RD,    RA,    Void,   Void,  Void,  PPC32 | PPC64)
INSTR (ADDMEOx,   0x7c0005d5,  0xfc0007ff,  RD,    RA,    Void,   Void,  Void,  PPC32 | PPC64)
INSTR (ADDZE,     0x7c000194,  0xfc0007ff,  RD,    RA,    Void,   Void,  Void,  PPC32 | PPC64)
INSTR (ADDZEx,    0x7c000195,  0xfc0007ff,  RD,    RA,    Void,   Void,  Void,  PPC32 | PPC64)
INSTR (ADDZEO,    0x7c000594,  0xfc0007ff,  RD,    RA,    Void,   Void,  Void,  PPC32 | PPC64)
INSTR (ADDZEOx,   0x7c000595,  0xfc0007ff,  RD,    RA,    Void,   Void,  Void,  PPC32 | PPC64)
INSTR (AND,       0x7c000038,  0xfc0007ff,  RA,    RS,    RB,     Void,  Void,  PPC32 | PPC64)
INSTR (ANDx,      0x7c000039,  0xfc0007ff,  RA,    RS,    RB,     Void,  Void,  PPC32 | PPC64)
INSTR (ANDC,      0x7c000078,  0xfc0007ff,  RA,    RS,    RB,     Void,  Void,  PPC32 | PPC64)
INSTR (ANDCx,     0x7c000079,  0xfc0007ff,  RA,    RS,    RB,     Void,  Void,  PPC32 | PPC64)
INSTR (ANDI,      0x70000000,  0xfc000000,  RA,    RS,    UIMM,   Void,  Void,  PPC32 | PPC64)
INSTR (ANDIx,     0x74000000,  0xfc000000,  RA,    RS,    UIMM,   Void,  Void,  PPC32 | PPC64)
INSTR (B,         0x48000000,  0xfc000003,  LI,    Void,  Void,   Void,  Void,  PPC32 | PPC64)
INSTR (BA,        0x48000002,  0xfc000003,  LA,    Void,  Void,   Void,  Void,  PPC32 | PPC64)
INSTR (BL,        0x48000001,  0xfc000003,  LI,    Void,  Void,   Void,  Void,  PPC32 | PPC64)
INSTR (BLA,       0x48000003,  0xfc000003,  LA,    Void,  Void,   Void,  Void,  PPC32 | PPC64)
INSTR (BC,        0x40000000,  0xfc000003,  BO,    BI,    BD,     Void,  Void,  PPC32 | PPC64)
INSTR (BCA,       0x40000002,  0xfc000003,  BO,    BI,    BD,     Void,  Void,  PPC32 | PPC64)
INSTR (BCL,       0x40000001,  0xfc000003,  BO,    BI,    BD,     Void,  Void,  PPC32 | PPC64)
INSTR (BCLA,      0x40000003,  0xfc000003,  BO,    BI,    BD,     Void,  Void,  PPC32 | PPC64)
INSTR (BCCTR,     0x4c000420,  0xfc0007ff,  BO,    BI,    BH,     Void,  Void,  PPC32 | PPC64)
INSTR (BCCTRL,    0x4c000421,  0xfc0007ff,  BO,    BI,    BH,     Void,  Void,  PPC32 | PPC64)
INSTR (BCLR,      0x4c000020,  0xfc0007ff,  BO,    BI,    BH,     Void,  Void,  PPC32 | PPC64)
INSTR (BCLRL,     0x4c000021,  0xfc0007ff,  BO,    BI,    BH,     Void,  Void,  PPC32 | PPC64)
INSTR (CMP,       0x7c000000,  0xfc0007fe,  CRFD,  L10,   RA,     RB,    Void,  PPC32 | PPC64)
INSTR (CMPI,      0x2c000000,  0xfc000000,  CRFD,  L10,   RA,     SIMM,  Void,  PPC32 | PPC64)
INSTR (CMPL,      0x7c000040,  0xfc0007fe,  CRFD,  L10,   RA,     RB,    Void,  PPC32 | PPC64)
INSTR (CMPLI,     0x28000000,  0xfc000000,  CRFD,  L10,   RA,     UIMM,  Void,  PPC32 | PPC64)
INSTR (CNTLZD,    0x7c000074,  0xfc0007ff,  RA,    RS,    Void,   Void,  Void,  PPC64)
INSTR (CNTLZDx,   0x7c000075,  0xfc0007ff,  RA,    RS,    Void,   Void,  Void,  PPC64)
INSTR (CNTLZW,    0x7c000034,  0xfc0007ff,  RA,    RS,    Void,   Void,  Void,  PPC32 | PPC64)
INSTR (CNTLZWx,   0x7c000035,  0xfc0007ff,  RA,    RS,    Void,   Void,  Void,  PPC32 | PPC64)
INSTR (CRAND,     0x4c000202,  0xfc0007fe,  CRBD,  CRBA,  CRBB,   Void,  Void,  PPC32 | PPC64)
INSTR (CRANDC,    0x4c000102,  0xfc0007fe,  CRBD,  CRBA,  CRBB,   Void,  Void,  PPC32 | PPC64)
INSTR (CREQV,     0x4c000242,  0xfc0007fe,  CRBD,  CRBA,  CRBB,   Void,  Void,  PPC32 | PPC64)
INSTR (CRNAND,    0x4c0001c2,  0xfc0007fe,  CRBD,  CRBA,  CRBB,   Void,  Void,  PPC32 | PPC64)
INSTR (CRNOR,     0x4c000042,  0xfc0007fe,  CRBD,  CRBA,  CRBB,   Void,  Void,  PPC32 | PPC64)
INSTR (CROR,      0x4c000382,  0xfc0007fe,  CRBD,  CRBA,  CRBB,   Void,  Void,  PPC32 | PPC64)
INSTR (CRORC,     0x4c000342,  0xfc0007fe,  CRBD,  CRBA,  CRBB,   Void,  Void,  PPC32 | PPC64)
INSTR (CRXOR,     0x4c000182,  0xfc0007fe,  CRBD,  CRBA,  CRBB,   Void,  Void,  PPC32 | PPC64)
INSTR (DCBF,      0x7c0000ac,  0xfc0007fe,  RA,    RB,    Void,   Void,  Void,  PPC32 | PPC64)
INSTR (DCBI,      0x7c0003ac,  0xfc0007fe,  RA,    RB,    Void,   Void,  Void,  PPC32)
INSTR (DCBST,     0x7c00006c,  0xfc0007fe,  RA,    RB,    Void,   Void,  Void,  PPC32 | PPC64)
INSTR (DCBT,      0x7c00022c,  0xfc0007fe,  RA,    RB,    TH,     Void,  Void,  PPC32 | PPC64)
INSTR (DCBTST,    0x7c0001ec,  0xfc0007fe,  RA,    RB,    Void,   Void,  Void,  PPC32 | PPC64)
INSTR (DCBZ,      0x7c0007ec,  0xfc0007fe,  RA,    RB,    Void,   Void,  Void,  PPC32 | PPC64)
INSTR (DIVD,      0x7c0003d2,  0xfc0007ff,  RD,    RA,    RB,     Void,  Void,  PPC64)
INSTR (DIVDx,     0x7c0003d3,  0xfc0007ff,  RD,    RA,    RB,     Void,  Void,  PPC64)
INSTR (DIVDO,     0x7c0007d2,  0xfc0007ff,  RD,    RA,    RB,     Void,  Void,  PPC64)
INSTR (DIVDOx,    0x7c0007d3,  0xfc0007ff,  RD,    RA,    RB,     Void,  Void,  PPC64)
INSTR (DIVDU,     0x7c000392,  0xfc0007ff,  RD,    RA,    RB,     Void,  Void,  PPC64)
INSTR (DIVDUx,    0x7c000393,  0xfc0007ff,  RD,    RA,    RB,     Void,  Void,  PPC64)
INSTR (DIVDUO,    0x7c000792,  0xfc0007ff,  RD,    RA,    RB,     Void,  Void,  PPC64)
INSTR (DIVDUOx,   0x7c000793,  0xfc0007ff,  RD,    RA,    RB,     Void,  Void,  PPC64)
INSTR (DIVW,      0x7c0003d6,  0xfc0007ff,  RD,    RA,    RB,     Void,  Void,  PPC32 | PPC64)
INSTR (DIVWx,     0x7c0003d7,  0xfc0007ff,  RD,    RA,    RB,     Void,  Void,  PPC32 | PPC64)
INSTR (DIVWO,     0x7c0007d6,  0xfc0007ff,  RD,    RA,    RB,     Void,  Void,  PPC32 | PPC64)
INSTR (DIVWOx,    0x7c0007d7,  0xfc0007ff,  RD,    RA,    RB,     Void,  Void,  PPC32 | PPC64)
INSTR (DIVWU,     0x7c000396,  0xfc0007ff,  RD,    RA,    RB,     Void,  Void,  PPC32 | PPC64)
INSTR (DIVWUx,    0x7c000397,  0xfc0007ff,  RD,    RA,    RB,     Void,  Void,  PPC32 | PPC64)
INSTR (DIVWUO,    0x7c000796,  0xfc0007ff,  RD,    RA,    RB,     Void,  Void,  PPC32 | PPC64)
INSTR (DIVWUOx,   0x7c000797,  0xfc0007ff,  RD,    RA,    RB,     Void,  Void,  PPC32 | PPC64)
INSTR (ECIWX,     0x7c00026c,  0xfc0007fe,  RD,    RA,    RB,     Void,  Void,  PPC32 | PPC64)
INSTR (ECOWX,     0x7c00036c,  0xfc0007fe,  RS,    RA,    RB,     Void,  Void,  PPC32 | PPC64)
INSTR (EIEIO,     0x7c0006ac,  0xfc0007fe,  Void,  Void,  Void,   Void,  Void,  PPC32 | PPC64)
INSTR (EQV,       0x7c000238,  0xfc0007ff,  RA,    RS,    RB,     Void,  Void,  PPC32 | PPC64)
INSTR (EQVx,      0x7c000239,  0xfc0007ff,  RA,    RS,    RB,     Void,  Void,  PPC32 | PPC64)
INSTR (EXTSB,     0x7c000774,  0xfc0007ff,  RA,    RS,    Void,   Void,  Void,  PPC32 | PPC64)
INSTR (EXTSBx,    0x7c000775,  0xfc0007ff,  RA,    RS,    Void,   Void,  Void,  PPC32 | PPC64)
INSTR (EXTSH,     0x7c000734,  0xfc0007ff,  RA,    RS,    Void,   Void,  Void,  PPC32 | PPC64)
INSTR (EXTSHx,    0x7c000735,  0xfc0007ff,  RA,    RS,    Void,   Void,  Void,  PPC32 | PPC64)
INSTR (EXTSW,     0x7c0007b4,  0xfc0007ff,  RA,    RS,    Void,   Void,  Void,  PPC64)
INSTR (EXTSWx,    0x7c0007b5,  0xfc0007ff,  RA,    RS,    Void,   Void,  Void,  PPC64)
INSTR (FABS,      0xfc000210,  0xfc0007ff,  FRD,   FRB,   Void,   Void,  Void,  PPC32 | PPC64)
INSTR (FABSx,     0xfc000211,  0xfc0007ff,  FRD,   FRB,   Void,   Void,  Void,  PPC32 | PPC64)
INSTR (FADD,      0xfc00002a,  0xfc00003f,  FRD,   FRA,   FRB,    Void,  Void,  PPC32 | PPC64)
INSTR (FADDx,     0xfc00002b,  0xfc00003f,  FRD,   FRA,   FRB,    Void,  Void,  PPC32 | PPC64)
INSTR (FADDS,     0xec00002a,  0xfc00003f,  FRD,   FRA,   FRB,    Void,  Void,  PPC32 | PPC64)
INSTR (FADDSx,    0xec00002b,  0xfc00003f,  FRD,   FRA,   FRB,    Void,  Void,  PPC32 | PPC64)
INSTR (FCFID,     0xfc00069c,  0xfc0007ff,  FRD,   FRB,   Void,   Void,  Void,  PPC64)
INSTR (FCFIDx,    0xfc00069d,  0xfc0007ff,  FRD,   FRB,   Void,   Void,  Void,  PPC64)
INSTR (FCMPO,     0xfc000040,  0xfc0007fe,  CRFD,  FRA,   FRB,    Void,  Void,  PPC32 | PPC64)
INSTR (FCMPU,     0xfc000000,  0xfc0007fe,  CRFD,  FRA,   FRB,    Void,  Void,  PPC32 | PPC64)
INSTR (FCTID,     0xfc00065c,  0xfc0007ff,  FRD,   FRB,   Void,   Void,  Void,  PPC64)
INSTR (FCTIDx,    0xfc00065d,  0xfc0007ff,  FRD,   FRB,   Void,   Void,  Void,  PPC64)
INSTR (FCTIDZ,    0xfc00065e,  0xfc0007ff,  FRD,   FRB,   Void,   Void,  Void,  PPC64)
INSTR (FCTIDZx,   0xfc00065f,  0xfc0007ff,  FRD,   FRB,   Void,   Void,  Void,  PPC64)
INSTR (FCTIW,     0xfc00001c,  0xfc0007ff,  FRD,   FRB,   Void,   Void,  Void,  PPC32 | PPC64)
INSTR (FCTIWx,    0xfc00001d,  0xfc0007ff,  FRD,   FRB,   Void,   Void,  Void,  PPC32 | PPC64)
INSTR (FCTIWZ,    0xfc00001e,  0xfc0007ff,  FRD,   FRB,   Void,   Void,  Void,  PPC32 | PPC64)
INSTR (FCTIWZx,   0xfc00001f,  0xfc0007ff,  FRD,   FRB,   Void,   Void,  Void,  PPC32 | PPC64)
INSTR (FDIV,      0xfc000024,  0xfc00003f,  FRD,   FRA,   FRB,    Void,  Void,  PPC32 | PPC64)
INSTR (FDIVx,     0xfc000025,  0xfc00003f,  FRD,   FRA,   FRB,    Void,  Void,  PPC32 | PPC64)
INSTR (FDIVS,     0xec000024,  0xfc00003f,  FRD,   FRA,   FRB,    Void,  Void,  PPC32 | PPC64)
INSTR (FDIVSx,    0xec000025,  0xfc00003f,  FRD,   FRA,   FRB,    Void,  Void,  PPC32 | PPC64)
INSTR (FMADD,     0xfc00003a,  0xfc00003f,  FRD,   FRA,   FRC,    FRB,   Void,  PPC32 | PPC64)
INSTR (FMADDx,    0xfc00003b,  0xfc00003f,  FRD,   FRA,   FRC,    FRB,   Void,  PPC32 | PPC64)
INSTR (FMADDS,    0xec00003a,  0xfc00003f,  FRD,   FRA,   FRC,    FRB,   Void,  PPC32 | PPC64)
INSTR (FMADDSx,   0xec00003b,  0xfc00003f,  FRD,   FRA,   FRC,    FRB,   Void,  PPC32 | PPC64)
INSTR (FMR,       0xfc000090,  0xfc0007ff,  FRD,   FRB,   Void,   Void,  Void,  PPC32 | PPC64)
INSTR (FMRx,      0xfc000091,  0xfc0007ff,  FRD,   FRB,   Void,   Void,  Void,  PPC32 | PPC64)
INSTR (FMSUB,     0xfc000038,  0xfc00003f,  FRD,   FRA,   FRC,    FRB,   Void,  PPC32 | PPC64)
INSTR (FMSUBx,    0xfc000039,  0xfc00003f,  FRD,   FRA,   FRC,    FRB,   Void,  PPC32 | PPC64)
INSTR (FMSUBS,    0xec000038,  0xfc00003f,  FRD,   FRA,   FRC,    FRB,   Void,  PPC32 | PPC64)
INSTR (FMSUBSx,   0xec000039,  0xfc00003f,  FRD,   FRA,   FRC,    FRB,   Void,  PPC32 | PPC64)
INSTR (FMUL,      0xfc000032,  0xfc00003f,  FRD,   FRA,   FRC,    Void,  Void,  PPC32 | PPC64)
INSTR (FMULx,     0xfc000033,  0xfc00003f,  FRD,   FRA,   FRC,    Void,  Void,  PPC32 | PPC64)
INSTR (FMULS,     0xec000032,  0xfc00003f,  FRD,   FRA,   FRC,    Void,  Void,  PPC32 | PPC64)
INSTR (FMULSx,    0xec000033,  0xfc00003f,  FRD,   FRA,   FRC,    Void,  Void,  PPC32 | PPC64)
INSTR (FNABS,     0xfc000110,  0xfc0007ff,  FRD,   FRB,   Void,   Void,  Void,  PPC32 | PPC64)
INSTR (FNABSx,    0xfc000111,  0xfc0007ff,  FRD,   FRB,   Void,   Void,  Void,  PPC32 | PPC64)
INSTR (FNEG,      0xfc000050,  0xfc0007ff,  FRD,   FRB,   Void,   Void,  Void,  PPC32 | PPC64)
INSTR (FNEGx,     0xfc000051,  0xfc0007ff,  FRD,   FRB,   Void,   Void,  Void,  PPC32 | PPC64)
INSTR (FNMADD,    0xfc00003e,  0xfc00003f,  FRD,   FRA,   FRC,    FRB,   Void,  PPC32 | PPC64)
INSTR (FNMADDx,   0xfc00003f,  0xfc00003f,  FRD,   FRA,   FRC,    FRB,   Void,  PPC32 | PPC64)
INSTR (FNMADDS,   0xec00003e,  0xfc00003f,  FRD,   FRA,   FRC,    FRB,   Void,  PPC32 | PPC64)
INSTR (FNMADDSx,  0xec00003f,  0xfc00003f,  FRD,   FRA,   FRC,    FRB,   Void,  PPC32 | PPC64)
INSTR (FNMSUB,    0xfc00003c,  0xfc00003f,  FRD,   FRA,   FRC,    FRB,   Void,  PPC32 | PPC64)
INSTR (FNMSUBx,   0xfc00003d,  0xfc00003f,  FRD,   FRA,   FRC,    FRB,   Void,  PPC32 | PPC64)
INSTR (FNMSUBS,   0xec00003c,  0xfc00003f,  FRD,   FRA,   FRC,    FRB,   Void,  PPC32 | PPC64)
INSTR (FNMSUBSx,  0xec00003d,  0xfc00003f,  FRD,   FRA,   FRC,    FRB,   Void,  PPC32 | PPC64)
INSTR (FRES,      0xec000030,  0xfc00003f,  FRD,   FRB,   Void,   Void,  Void,  PPC32 | PPC64)
INSTR (FRESx,     0xec000031,  0xfc00003f,  FRD,   FRB,   Void,   Void,  Void,  PPC32 | PPC64)
INSTR (FRSP,      0xfc000018,  0xfc0007ff,  FRD,   FRB,   Void,   Void,  Void,  PPC32 | PPC64)
INSTR (FRSPx,     0xfc000018,  0xfc0007ff,  FRD,   FRB,   Void,   Void,  Void,  PPC32 | PPC64)
INSTR (FRSQRTE,   0xfc000034,  0xfc00003f,  FRD,   FRB,   Void,   Void,  Void,  PPC32 | PPC64)
INSTR (FRSQRTEx,  0xfc000034,  0xfc00003f,  FRD,   FRB,   Void,   Void,  Void,  PPC32 | PPC64)
INSTR (FSEL,      0xfc00002e,  0xfc00003f,  FRD,   FRA,   FRC,    FRB,   Void,  PPC32 | PPC64)
INSTR (FSELx,     0xfc00002f,  0xfc00003f,  FRD,   FRA,   FRC,    FRB,   Void,  PPC32 | PPC64)
INSTR (FSQRT,     0xfc00002c,  0xfc00003f,  FRD,   FRB,   Void,   Void,  Void,  PPC32 | PPC64)
INSTR (FSQRTx,    0xfc00002d,  0xfc00003f,  FRD,   FRB,   Void,   Void,  Void,  PPC32 | PPC64)
INSTR (FSQRTS,    0xec00002c,  0xfc00003f,  FRD,   FRB,   Void,   Void,  Void,  PPC32 | PPC64)
INSTR (FSQRTSx,   0xec00002d,  0xfc00003f,  FRD,   FRB,   Void,   Void,  Void,  PPC32 | PPC64)
INSTR (FSUB,      0xfc000028,  0xfc00003f,  FRD,   FRA,   FRB,    Void,  Void,  PPC32 | PPC64)
INSTR (FSUBx,     0xfc000029,  0xfc00003f,  FRD,   FRA,   FRB,    Void,  Void,  PPC32 | PPC64)
INSTR (FSUBS,     0xec000028,  0xfc00003f,  FRD,   FRA,   FRB,    Void,  Void,  PPC32 | PPC64)
INSTR (FSUBSx,    0xec000029,  0xfc00003f,  FRD,   FRA,   FRB,    Void,  Void,  PPC32 | PPC64)
INSTR (ICBI,      0x7c0007ac,  0xfc0007fe,  RA,    RB,    Void,   Void,  Void,  PPC32 | PPC64)
INSTR (ISYNC,     0x4c00012c,  0xfc0007fe,  Void,  Void,  Void,   Void,  Void,  PPC32 | PPC64)
INSTR (LBZ,       0x88000000,  0xfc000000,  RD,    D,     RA,     Void,  Void,  PPC32 | PPC64)
INSTR (LBZU,      0x8c000000,  0xfc000000,  RD,    D,     RA,     Void,  Void,  PPC32 | PPC64)
INSTR (LBZUX,     0x7c0000ee,  0xfc0007fe,  RD,    RA,    RB,     Void,  Void,  PPC32 | PPC64)
INSTR (LBZX,      0x7c0000ae,  0xfc0007fe,  RD,    RA,    RB,     Void,  Void,  PPC32 | PPC64)
INSTR (LA,        0x38000000,  0xfc000000,  RD,    D,     RA,     Void,  Void,  PPC32 | PPC64)
INSTR (LD,        0xe8000000,  0xfc000003,  RD,    DS,    RA,     Void,  Void,  PPC64)
INSTR (LDARX,     0x7c0000a8,  0xfc0007fe,  RD,    RA,    RB,     Void,  Void,  PPC64)
INSTR (LDU,       0xe8000001,  0xfc000003,  RD,    DS,    RA,     Void,  Void,  PPC64)
INSTR (LDUX,      0x7c00006a,  0xfc0007fe,  RD,    RA,    RB,     Void,  Void,  PPC64)
INSTR (LDX,       0x7c00002a,  0xfc0007fe,  RD,    RA,    RB,     Void,  Void,  PPC64)
INSTR (LFD,       0xc8000000,  0xfc000000,  FRD,   D,     RA,     Void,  Void,  PPC32 | PPC64)
INSTR (LFDU,      0xcc000000,  0xfc000000,  FRD,   D,     RA,     Void,  Void,  PPC32 | PPC64)
INSTR (LFDUX,     0x7c0004ee,  0xfc0007fe,  FRD,   RA,    RB,     Void,  Void,  PPC32 | PPC64)
INSTR (LFDX,      0x7c0004ae,  0xfc0007fe,  FRD,   RA,    RB,     Void,  Void,  PPC32 | PPC64)
INSTR (LFS,       0xc0000000,  0xfc000000,  FRD,   D,     RA,     Void,  Void,  PPC32 | PPC64)
INSTR (LFSU,      0xc4000000,  0xfc000000,  FRD,   D,     RA,     Void,  Void,  PPC32 | PPC64)
INSTR (LFSUX,     0x7c00046e,  0xfc0007fe,  FRD,   RA,    RB,     Void,  Void,  PPC32 | PPC64)
INSTR (LFSX,      0x7c00042e,  0xfc0007fe,  FRD,   RA,    RB,     Void,  Void,  PPC32 | PPC64)
INSTR (LHA,       0xa8000000,  0xfc000000,  RD,    D,     RA,     Void,  Void,  PPC32 | PPC64)
INSTR (LHAU,      0xac000000,  0xfc000000,  RD,    D,     RA,     Void,  Void,  PPC32 | PPC64)
INSTR (LHAUX,     0x7c0002ee,  0xfc0007fe,  RD,    RA,    RB,     Void,  Void,  PPC32 | PPC64)
INSTR (LHAX,      0x7c0002ae,  0xfc0007fe,  RD,    RA,    RB,     Void,  Void,  PPC32 | PPC64)
INSTR (LHBRX,     0x7c00062c,  0xfc0007fe,  RD,    RA,    RB,     Void,  Void,  PPC32 | PPC64)
INSTR (LHZ,       0xa0000000,  0xfc000000,  RD,    D,     RA,     Void,  Void,  PPC32 | PPC64)
INSTR (LHZU,      0xa4000000,  0xfc000000,  RD,    D,     RA,     Void,  Void,  PPC32 | PPC64)
INSTR (LHZUX,     0x7c00026e,  0xfc0007fe,  RD,    RA,    RB,     Void,  Void,  PPC32 | PPC64)
INSTR (LHZX,      0x7c00022e,  0xfc0007fe,  RD,    RA,    RB,     Void,  Void,  PPC32 | PPC64)
INSTR (LMW,       0xb8000000,  0xfc000000,  RD,    D,     RA,     Void,  Void,  PPC32 | PPC64)
INSTR (LSWI,      0x7c0004aa,  0xfc0007fe,  RD,    RA,    NB,     Void,  Void,  PPC32 | PPC64)
INSTR (LSWX,      0x7c00042a,  0xfc0007fe,  RD,    RA,    RB,     Void,  Void,  PPC32 | PPC64)
INSTR (LWA,       0xe8000002,  0xfc000003,  RD,    DS,    RA,     Void,  Void,  PPC64)
INSTR (LWARX,     0x7c000028,  0xfc0007fe,  RD,    RA,    RB,     Void,  Void,  PPC32 | PPC64)
INSTR (LWAUX,     0x7c0002e8,  0xfc0007fe,  RD,    RA,    RB,     Void,  Void,  PPC64)
INSTR (LWAX,      0x7c0002aa,  0xfc0007fe,  RD,    RA,    RB,     Void,  Void,  PPC64)
INSTR (LWBRX,     0x7c00042c,  0xfc0007fe,  RD,    RA,    RB,     Void,  Void,  PPC32 | PPC64)
INSTR (LWZ,       0x80000000,  0xfc000000,  RD,    D,     RA,     Void,  Void,  PPC32 | PPC64)
INSTR (LWZU,      0x84000000,  0xfc000000,  RD,    D,     RA,     Void,  Void,  PPC32 | PPC64)
INSTR (LWZUX,     0x7c00006e,  0xfc0007fe,  RD,    RA,    RB,     Void,  Void,  PPC32 | PPC64)
INSTR (LWZX,      0x7c00002e,  0xfc0007fe,  RD,    RA,    RB,     Void,  Void,  PPC32 | PPC64)
INSTR (MCRF,      0x4c000000,  0xfc0007fe,  CRFD,  CRFS,  Void,   Void,  Void,  PPC32 | PPC64)
INSTR (MCRFS,     0xfc000080,  0xfc0007fe,  CRFD,  CRFS,  Void,   Void,  Void,  PPC32 | PPC64)
INSTR (MCRCR,     0x7c000400,  0xfc0007fe,  CRFD,  Void,  Void,   Void,  Void,  PPC32 | PPC64)
INSTR (MFOCRF,    0x7c100026,  0xfc100fff,  RD,    CRM,   Void,   Void,  Void,  PPC32 | PPC64)
INSTR (MFCR,      0x7c000026,  0xfc0007fe,  RD,    Void,  Void,   Void,  Void,  PPC32 | PPC64)
INSTR (MFFS,      0xfc00048e,  0xfc0007ff,  FRD,   Void,  Void,   Void,  Void,  PPC32 | PPC64)
INSTR (MFFSx,     0xfc00048f,  0xfc0007ff,  FRD,   Void,  Void,   Void,  Void,  PPC32 | PPC64)
INSTR (MFMSR,     0x7c0000a6,  0xfc0007fe,  RD,    Void,  Void,   Void,  Void,  PPC32 | PPC64)
INSTR (MFSPR,     0x7c0002a6,  0xfc0007fe,  RD,    SPR,   Void,   Void,  Void,  PPC32 | PPC64)
INSTR (MFSR,      0x7c0004a6,  0xfc0007fe,  RD,    SR,    Void,   Void,  Void,  PPC32)
INSTR (MFSRIN,    0x7c000526,  0xfc0007fe,  RD,    RB,    Void,   Void,  Void,  PPC32 | PPC64)
INSTR (MFTB,      0x7c0002e6,  0xfc0007fe,  RD,    TBR,   Void,   Void,  Void,  PPC32 | PPC64)
INSTR (MTOCRF,    0x7c100120,  0xfc100ffe,  CRM,   RS,    Void,   Void,  Void,  PPC32 | PPC64)
INSTR (MTCRF,     0x7c000120,  0xfc0007fe,  CRM,   RS,    Void,   Void,  Void,  PPC32 | PPC64)
INSTR (MTFSB0,    0xfc00008c,  0xfc0007ff,  CRBD,  Void,  Void,   Void,  Void,  PPC32 | PPC64)
INSTR (MTFSB0x,   0xfc00008d,  0xfc0007ff,  CRBD,  Void,  Void,   Void,  Void,  PPC32 | PPC64)
INSTR (MTFSB1,    0xfc00004c,  0xfc0007ff,  CRBD,  Void,  Void,   Void,  Void,  PPC32 | PPC64)
INSTR (MTFSB1x,   0xfc00004d,  0xfc0007ff,  CRBD,  Void,  Void,   Void,  Void,  PPC32 | PPC64)
INSTR (MTFSF,     0xfc00058e,  0xfc0007ff,  FM,    FRB,   Void,   Void,  Void,  PPC32 | PPC64)
INSTR (MTFSFx,    0xfc00058f,  0xfc0007ff,  FM,    FRB,   Void,   Void,  Void,  PPC32 | PPC64)
INSTR (MTFSFI,    0xfc00010c,  0xfc0007ff,  CRFD,  IMM,   Void,   Void,  Void,  PPC32 | PPC64)
INSTR (MTFSFIx,   0xfc00010d,  0xfc0007ff,  CRFD,  IMM,   Void,   Void,  Void,  PPC32 | PPC64)
INSTR (MTMSR,     0x7c000124,  0xfc0007fe,  RS,    L15,   Void,   Void,  Void,  PPC32 | PPC64)
INSTR (MTMSRD,    0x7c000164,  0xfc0007fe,  RS,    L15,   Void,   Void,  Void,  PPC64)
INSTR (MTSPR,     0x7c0003a6,  0xfc0007fe,  SPR,   RD,    Void,   Void,  Void,  PPC32 | PPC64)
INSTR (MTSR,      0x7c0001a4,  0xfc0007fe,  SR,    RD,    Void,   Void,  Void,  PPC32)
INSTR (MTSRD,     0x7c0000a4,  0xfc0007fe,  SR,    RD,    Void,   Void,  Void,  PPC64)
INSTR (MTSRDIN,   0x7c0000e4,  0xfc0007fe,  RS,    RB,    Void,   Void,  Void,  PPC64)
INSTR (MTSRIN,    0x7c0001e4,  0xfc0007fe,  RS,    RB,    Void,   Void,  Void,  PPC32)
INSTR (MULHD,     0x7c000092,  0xfc0007ff,  RD,    RA,    RB,     Void,  Void,  PPC64)
INSTR (MULHDx,    0x7c000093,  0xfc0007ff,  RD,    RA,    RB,     Void,  Void,  PPC64)
INSTR (MULHDU,    0x7c000012,  0xfc0007ff,  RD,    RA,    RB,     Void,  Void,  PPC64)
INSTR (MULHDUx,   0x7c000013,  0xfc0007ff,  RD,    RA,    RB,     Void,  Void,  PPC64)
INSTR (MULHW,     0x7c000096,  0xfc0007ff,  RD,    RA,    RB,     Void,  Void,  PPC32 | PPC64)
INSTR (MULHWx,    0x7c000097,  0xfc0007ff,  RD,    RA,    RB,     Void,  Void,  PPC32 | PPC64)
INSTR (MULHWU,    0x7c000016,  0xfc0007ff,  RD,    RA,    RB,     Void,  Void,  PPC32 | PPC64)
INSTR (MULHWUx,   0x7c000017,  0xfc0007ff,  RD,    RA,    RB,     Void,  Void,  PPC32 | PPC64)
INSTR (MULLD,     0x7c0001d2,  0xfc0007ff,  RD,    RA,    RB,     Void,  Void,  PPC64)
INSTR (MULLDx,    0x7c0001d3,  0xfc0007ff,  RD,    RA,    RB,     Void,  Void,  PPC64)
INSTR (MULLDO,    0x7c0005d2,  0xfc0007ff,  RD,    RA,    RB,     Void,  Void,  PPC64)
INSTR (MULLDOx,   0x7c0005d3,  0xfc0007ff,  RD,    RA,    RB,     Void,  Void,  PPC64)
INSTR (MULLI,     0x1c000000,  0xfc000000,  RD,    RA,    SIMM,   Void,  Void,  PPC32 | PPC64)
INSTR (MULLW,     0x7c0001d6,  0xfc0007ff,  RD,    RA,    RB,     Void,  Void,  PPC32 | PPC64)
INSTR (MULLWx,    0x7c0001d7,  0xfc0007ff,  RD,    RA,    RB,     Void,  Void,  PPC32 | PPC64)
INSTR (MULLWO,    0x7c0005d6,  0xfc0007ff,  RD,    RA,    RB,     Void,  Void,  PPC32 | PPC64)
INSTR (MULLWOx,   0x7c0005d7,  0xfc0007ff,  RD,    RA,    RB,     Void,  Void,  PPC32 | PPC64)
INSTR (NAND,      0x7c0003b8,  0xfc0007ff,  RA,    RS,    RB,     Void,  Void,  PPC32 | PPC64)
INSTR (NANDx,     0x7c0003b9,  0xfc0007ff,  RA,    RS,    RB,     Void,  Void,  PPC32 | PPC64)
INSTR (NEG,       0x7c0000d0,  0xfc0007ff,  RD,    RA,    Void,   Void,  Void,  PPC32 | PPC64)
INSTR (NEGx,      0x7c0000d1,  0xfc0007ff,  RD,    RA,    Void,   Void,  Void,  PPC32 | PPC64)
INSTR (NEGO,      0x7c0004d0,  0xfc0007ff,  RD,    RA,    Void,   Void,  Void,  PPC32 | PPC64)
INSTR (NEGOx,     0x7c0004d1,  0xfc0007ff,  RD,    RA,    Void,   Void,  Void,  PPC32 | PPC64)
INSTR (NOR,       0x7c0000f8,  0xfc0007ff,  RA,    RS,    RB,     Void,  Void,  PPC32 | PPC64)
INSTR (NORx,      0x7c0000f9,  0xfc0007ff,  RA,    RS,    RB,     Void,  Void,  PPC32 | PPC64)
INSTR (NOT,       0x7c0000f8,  0xfc0007ff,  RA,    RBS,   Void,   Void,  Void,  PPC32 | PPC64)
INSTR (NOTx,      0x7c0000f9,  0xfc0007ff,  RA,    RBS,   Void,   Void,  Void,  PPC32 | PPC64)
INSTR (OR,        0x7c000378,  0xfc0007ff,  RA,    RS,    RB,     Void,  Void,  PPC32 | PPC64)
INSTR (ORx,       0x7c000379,  0xfc0007ff,  RA,    RS,    RB,     Void,  Void,  PPC32 | PPC64)
INSTR (MR,        0x7c000378,  0xfc0007ff,  RA,    RBS,   Void,   Void,  Void,  PPC32 | PPC64)
INSTR (MRx,       0x7c000379,  0xfc0007ff,  RA,    RBS,   Void,   Void,  Void,  PPC32 | PPC64)
INSTR (ORC,       0x7c000338,  0xfc0007ff,  RA,    RS,    RB,     Void,  Void,  PPC32 | PPC64)
INSTR (ORCx,      0x7c000339,  0xfc0007ff,  RA,    RS,    RB,     Void,  Void,  PPC32 | PPC64)
INSTR (ORI,       0x60000000,  0xfc000000,  RA,    RS,    UIMM,   Void,  Void,  PPC32 | PPC64)
INSTR (ORIS,      0x64000000,  0xfc000000,  RA,    RS,    UIMM,   Void,  Void,  PPC32 | PPC64)
INSTR (RFI,       0x4c000064,  0xfc0007fe,  Void,  Void,  Void,   Void,  Void,  PPC32)
INSTR (RFID,      0x4c000024,  0xfc0007fe,  Void,  Void,  Void,   Void,  Void,  PPC64)
INSTR (RLDCL,     0x78000010,  0xfc00001f,  RA,    RS,    RB,     MBD,   Void,  PPC64)
INSTR (RLDCLx,    0x78000011,  0xfc00001f,  RA,    RS,    RB,     MBD,   Void,  PPC64)
INSTR (RLDCR,     0x78000012,  0xfc00001f,  RA,    RS,    RB,     MED,   Void,  PPC64)
INSTR (RLDCRx,    0x78000013,  0xfc00001f,  RA,    RS,    RB,     MED,   Void,  PPC64)
INSTR (RLDIC,     0x78000008,  0xfc00001d,  RA,    RS,    SHD,    MBD,   Void,  PPC64)
INSTR (RLDICx,    0x78000009,  0xfc00001d,  RA,    RS,    SHD,    MBD,   Void,  PPC64)
INSTR (RLDICL,    0x78000000,  0xfc00001d,  RA,    RS,    SHD,    MBD,   Void,  PPC64)
INSTR (RLDICLx,   0x78000001,  0xfc00001d,  RA,    RS,    SHD,    MBD,   Void,  PPC64)
INSTR (RLDICR,    0x78000004,  0xfc00001d,  RA,    RS,    SHD,    MED,   Void,  PPC64)
INSTR (RLDICRx,   0x78000005,  0xfc00001d,  RA,    RS,    SHD,    MED,   Void,  PPC64)
INSTR (RLDIMI,    0x7800000c,  0xfc00001d,  RA,    RS,    SHD,    MBD,   Void,  PPC64)
INSTR (RLDIMIx,   0x7800000d,  0xfc00001d,  RA,    RS,    SHD,    MBD,   Void,  PPC64)
INSTR (RLWIMI,    0x50000000,  0xfc000001,  RA,    RS,    SH,     MB,    ME,    PPC32 | PPC64)
INSTR (RLWIMIx,   0x50000001,  0xfc000001,  RA,    RS,    SH,     MB,    ME,    PPC32 | PPC64)
INSTR (RLWINM,    0x54000000,  0xfc000001,  RA,    RS,    SH,     MB,    ME,    PPC32 | PPC64)
INSTR (RLWINMx,   0x54000001,  0xfc000001,  RA,    RS,    SH,     MB,    ME,    PPC32 | PPC64)
INSTR (RLWNM,     0x5c000000,  0xfc000001,  RA,    RS,    RB,     MB,    ME,    PPC32 | PPC64)
INSTR (RLWNMx,    0x5c000001,  0xfc000001,  RA,    RS,    RB,     MB,    ME,    PPC32 | PPC64)
INSTR (SC,        0x44000002,  0xfc000002,  Void,  Void,  Void,   Void,  Void,  PPC32 | PPC64)
INSTR (SLBIA,     0x7c0003e4,  0xfc0007fe,  Void,  Void,  Void,   Void,  Void,  PPC64)
INSTR (SLBIE,     0x7c000364,  0xfc0007fe,  RB,    Void,  Void,   Void,  Void,  PPC64)
INSTR (SLBMFEE,   0x7c000726,  0xfc0007fe,  RD,    RB,    Void,   Void,  Void,  PPC64)
INSTR (SLBMFEV,   0x7c0006a6,  0xfc0007fe,  RD,    RB,    Void,   Void,  Void,  PPC64)
INSTR (SLBMTE,    0x7c000324,  0xfc0007fe,  RS,    RB,    Void,   Void,  Void,  PPC64)
INSTR (SLD,       0x7c000036,  0xfc0007ff,  RA,    RS,    RB,     Void,  Void,  PPC64)
INSTR (SLDx,      0x7c000037,  0xfc0007ff,  RA,    RS,    RB,     Void,  Void,  PPC64)
INSTR (SLW,       0x7c000030,  0xfc0007ff,  RA,    RS,    RB,     Void,  Void,  PPC32 | PPC64)
INSTR (SLWx,      0x7c000031,  0xfc0007ff,  RA,    RS,    RB,     Void,  Void,  PPC32 | PPC64)
INSTR (SRAD,      0x7c000634,  0xfc0007ff,  RA,    RS,    RB,     Void,  Void,  PPC64)
INSTR (SRADx,     0x7c000635,  0xfc0007ff,  RA,    RS,    RB,     Void,  Void,  PPC64)
INSTR (SRADI,     0x7c000674,  0xfc0007fd,  RA,    RS,    SHD,    Void,  Void,  PPC64)
INSTR (SRADIx,    0x7c000675,  0xfc0007fd,  RA,    RS,    SHD,    Void,  Void,  PPC64)
INSTR (SRAW,      0x7c000630,  0xfc0007ff,  RA,    RS,    RB,     Void,  Void,  PPC32 | PPC64)
INSTR (SRAWx,     0x7c000631,  0xfc0007ff,  RA,    RS,    RB,     Void,  Void,  PPC32 | PPC64)
INSTR (SRAWI,     0x7c000670,  0xfc0007ff,  RA,    RS,    SH,     Void,  Void,  PPC32 | PPC64)
INSTR (SRAWIx,    0x7c000671,  0xfc0007ff,  RA,    RS,    SH,     Void,  Void,  PPC32 | PPC64)
INSTR (SRD,       0x7c000436,  0xfc0007ff,  RA,    RS,    RB,     Void,  Void,  PPC64)
INSTR (SRDx,      0x7c000437,  0xfc0007ff,  RA,    RS,    RB,     Void,  Void,  PPC64)
INSTR (SRW,       0x7c000430,  0xfc0007ff,  RA,    RS,    RB,     Void,  Void,  PPC32 | PPC64)
INSTR (SRWx,      0x7c000431,  0xfc0007ff,  RA,    RS,    RB,     Void,  Void,  PPC32 | PPC64)
INSTR (STB,       0x98000000,  0xfc000000,  RS,    D,     RA,     Void,  Void,  PPC32 | PPC64)
INSTR (STBU,      0x9c000000,  0xfc000000,  RS,    D,     RA,     Void,  Void,  PPC32 | PPC64)
INSTR (STBUX,     0x7c0001ee,  0xfc0007fe,  RS,    RA,    RB,     Void,  Void,  PPC32 | PPC64)
INSTR (STBX,      0x7c0001ae,  0xfc0007fe,  RS,    RA,    RB,     Void,  Void,  PPC32 | PPC64)
INSTR (STD,       0xf8000000,  0xfc000003,  RS,    DS,    RA,     Void,  Void,  PPC64)
INSTR (STDCX,     0x7c0001ad,  0xfc0007ff,  RS,    RA,    RB,     Void,  Void,  PPC64)
INSTR (STDU,      0xf8000001,  0xfc000003,  RS,    DS,    RA,     Void,  Void,  PPC64)
INSTR (STDUX,     0x7c00016a,  0xfc0007fe,  RS,    RA,    RB,     Void,  Void,  PPC64)
INSTR (STDX,      0x7c00012a,  0xfc0007fe,  RS,    RA,    RB,     Void,  Void,  PPC64)
INSTR (STFD,      0xd8000000,  0xfc000000,  FRS,   D,     RA,     Void,  Void,  PPC32 | PPC64)
INSTR (STFDU,     0xdc000000,  0xfc000000,  FRS,   D,     RA,     Void,  Void,  PPC32 | PPC64)
INSTR (STFDUX,    0x7c0005ee,  0xfc0007fe,  FRS,   RA,    RB,     Void,  Void,  PPC32 | PPC64)
INSTR (STFDX,     0x7c0005ae,  0xfc0007fe,  FRS,   RA,    RB,     Void,  Void,  PPC32 | PPC64)
INSTR (STFIWX,    0x7c0007ae,  0xfc0007fe,  FRS,   RA,    RB,     Void,  Void,  PPC32 | PPC64)
INSTR (STFS,      0xd0000000,  0xfc000000,  FRS,   D,     RA,     Void,  Void,  PPC32 | PPC64)
INSTR (STFSU,     0xd4000000,  0xfc000000,  FRS,   D,     RA,     Void,  Void,  PPC32 | PPC64)
INSTR (STFSUX,    0x7c00056e,  0xfc0007fe,  FRS,   RA,    RB,     Void,  Void,  PPC32 | PPC64)
INSTR (STFSX,     0x7c00052e,  0xfc0007fe,  FRS,   RA,    RB,     Void,  Void,  PPC32 | PPC64)
INSTR (STH,       0xb0000000,  0xfc000000,  RS,    D,     RA,     Void,  Void,  PPC32 | PPC64)
INSTR (STHBRX,    0x7c00072c,  0xfc0007fe,  RS,    RA,    RB,     Void,  Void,  PPC32 | PPC64)
INSTR (STHU,      0xb4000000,  0xfc000000,  RS,    D,     RA,     Void,  Void,  PPC32 | PPC64)
INSTR (STHUX,     0x7c00036e,  0xfc0007fe,  RS,    RA,    RB,     Void,  Void,  PPC32 | PPC64)
INSTR (STHX,      0x7c00032e,  0xfc0007fe,  RS,    RA,    RB,     Void,  Void,  PPC32 | PPC64)
INSTR (STMW,      0xbc000000,  0xfc000000,  RS,    D,     RA,     Void,  Void,  PPC32 | PPC64)
INSTR (STSWI,     0x7c0005aa,  0xfc0007fe,  RS,    RA,    NB,     Void,  Void,  PPC32 | PPC64)
INSTR (STSWX,     0x7c00052a,  0xfc0007fe,  RS,    RA,    RB,     Void,  Void,  PPC32 | PPC64)
INSTR (STW,       0x90000000,  0xfc000000,  RS,    D,     RA,     Void,  Void,  PPC32 | PPC64)
INSTR (STWBRX,    0x7c00052c,  0xfc0007fe,  RS,    RA,    RB,     Void,  Void,  PPC32 | PPC64)
INSTR (STWCX,     0x7c00012d,  0xfc0007ff,  RS,    RA,    RB,     Void,  Void,  PPC32 | PPC64)
INSTR (STWU,      0x94000000,  0xfc000000,  RS,    D,     RA,     Void,  Void,  PPC32 | PPC64)
INSTR (STWUX,     0x7c00016e,  0xfc0007fe,  RS,    RA,    RB,     Void,  Void,  PPC32 | PPC64)
INSTR (STWX,      0x7c00012e,  0xfc0007fe,  RS,    RA,    RB,     Void,  Void,  PPC32 | PPC64)
INSTR (SUBF,      0x7c000050,  0xfc0007ff,  RD,    RA,    RB,     Void,  Void,  PPC32 | PPC64)
INSTR (SUBFx,     0x7c000051,  0xfc0007ff,  RD,    RA,    RB,     Void,  Void,  PPC32 | PPC64)
INSTR (SUBFO,     0x7c000450,  0xfc0007ff,  RD,    RA,    RB,     Void,  Void,  PPC32 | PPC64)
INSTR (SUBFOx,    0x7c000451,  0xfc0007ff,  RD,    RA,    RB,     Void,  Void,  PPC32 | PPC64)
INSTR (SUBFC,     0x7c000010,  0xfc0007ff,  RD,    RA,    RB,     Void,  Void,  PPC32 | PPC64)
INSTR (SUBFCx,    0x7c000011,  0xfc0007ff,  RD,    RA,    RB,     Void,  Void,  PPC32 | PPC64)
INSTR (SUBFCO,    0x7c000410,  0xfc0007ff,  RD,    RA,    RB,     Void,  Void,  PPC32 | PPC64)
INSTR (SUBFCOx,   0x7c000411,  0xfc0007ff,  RD,    RA,    RB,     Void,  Void,  PPC32 | PPC64)
INSTR (SUBFE,     0x7c000110,  0xfc0007ff,  RD,    RA,    RB,     Void,  Void,  PPC32 | PPC64)
INSTR (SUBFEx,    0x7c000111,  0xfc0007ff,  RD,    RA,    RB,     Void,  Void,  PPC32 | PPC64)
INSTR (SUBFEO,    0x7c000510,  0xfc0007ff,  RD,    RA,    RB,     Void,  Void,  PPC32 | PPC64)
INSTR (SUBFEOx,   0x7c000511,  0xfc0007ff,  RD,    RA,    RB,     Void,  Void,  PPC32 | PPC64)
INSTR (SUBFIC,    0x20000000,  0xfc000000,  RD,    RA,    SIMM,   Void,  Void,  PPC32 | PPC64)
INSTR (SUBFME,    0x7c0001d0,  0xfc0007ff,  RD,    RA,    Void,   Void,  Void,  PPC32 | PPC64)
INSTR (SUBFMEx,   0x7c0001d1,  0xfc0007ff,  RD,    RA,    Void,   Void,  Void,  PPC32 | PPC64)
INSTR (SUBFMEO,   0x7c0005d0,  0xfc0007ff,  RD,    RA,    Void,   Void,  Void,  PPC32 | PPC64)
INSTR (SUBFMEOx,  0x7c0005d1,  0xfc0007ff,  RD,    RA,    Void,   Void,  Void,  PPC32 | PPC64)
INSTR (SUBFZE,    0x7c000190,  0xfc0007ff,  RD,    RA,    Void,   Void,  Void,  PPC32 | PPC64)
INSTR (SUBFZEx,   0x7c000191,  0xfc0007ff,  RD,    RA,    Void,   Void,  Void,  PPC32 | PPC64)
INSTR (SUBFZEO,   0x7c000590,  0xfc0007ff,  RD,    RA,    Void,   Void,  Void,  PPC32 | PPC64)
INSTR (SUBFZEOx,  0x7c000591,  0xfc0007ff,  RD,    RA,    Void,   Void,  Void,  PPC32 | PPC64)
INSTR (SUB,       0x7c000050,  0xfc0007ff,  RD,    RB,    RA,     Void,  Void,  PPC32 | PPC64)
INSTR (SUBx,      0x7c000051,  0xfc0007ff,  RD,    RB,    RA,     Void,  Void,  PPC32 | PPC64)
INSTR (SUBO,      0x7c000450,  0xfc0007ff,  RD,    RB,    RA,     Void,  Void,  PPC32 | PPC64)
INSTR (SUBOx,     0x7c000451,  0xfc0007ff,  RD,    RB,    RA,     Void,  Void,  PPC32 | PPC64)
INSTR (SUBC,      0x7c000010,  0xfc0007ff,  RD,    RB,    RA,     Void,  Void,  PPC32 | PPC64)
INSTR (SUBCx,     0x7c000011,  0xfc0007ff,  RD,    RB,    RA,     Void,  Void,  PPC32 | PPC64)
INSTR (SUBCO,     0x7c000410,  0xfc0007ff,  RD,    RB,    RA,     Void,  Void,  PPC32 | PPC64)
INSTR (SUBCOx,    0x7c000411,  0xfc0007ff,  RD,    RB,    RA,     Void,  Void,  PPC32 | PPC64)
INSTR (SUBI,      0x38000000,  0xfc000000,  RD,    RA,    SIMMN,  Void,  Void,  PPC32 | PPC64)
INSTR (SUBIC,     0x30000000,  0xfc000000,  RD,    RA,    SIMMN,  Void,  Void,  PPC32 | PPC64)
INSTR (SUBICx,    0x34000000,  0xfc000000,  RD,    RA,    SIMMN,  Void,  Void,  PPC32 | PPC64)
INSTR (SUBIS,     0x3c000000,  0xfc000000,  RD,    RA,    SIMMN,  Void,  Void,  PPC32 | PPC64)
INSTR (SYNC,      0x7c0004ac,  0xfc0007fe,  L9,    Void,  Void,   Void,  Void,  PPC32 | PPC64)
INSTR (TD,        0x7c000088,  0xfc0007fe,  TO,    RA,    RB,     Void,  Void,  PPC64)
INSTR (TDI,       0x08000000,  0xfc000000,  TO,    RA,    SIMM,   Void,  Void,  PPC64)
INSTR (TLBIA,     0x7c0002e4,  0xfc0007fe,  Void,  Void,  Void,   Void,  Void,  PPC32 | PPC64)
INSTR (TLBIE,     0x7c000264,  0xfc0007fe,  RB,    L10,   Void,   Void,  Void,  PPC32 | PPC64)
INSTR (TLBIEL,    0x7c000224,  0xfc0007fe,  RB,    L10,   Void,   Void,  Void,  PPC32 | PPC64)
INSTR (TLBSYNC,   0x7c00046c,  0xfc0007fe,  Void,  Void,  Void,   Void,  Void,  PPC32 | PPC64)
INSTR (TW,        0x7c000008,  0xfc0007fe,  TO,    RA,    RB,     Void,  Void,  PPC32 | PPC64)
INSTR (TWI,       0x0c000000,  0xfc000000,  TO,    RA,    SIMM,   Void,  Void,  PPC32 | PPC64)
INSTR (XOR,       0x7c000278,  0xfc0007ff,  RA,    RS,    RB,     Void,  Void,  PPC32 | PPC64)
INSTR (XORx,      0x7c000279,  0xfc0007ff,  RA,    RS,    RB,     Void,  Void,  PPC32 | PPC64)
INSTR (XORI,      0x68000000,  0xfc000000,  RA,    RS,    UIMM,   Void,  Void,  PPC32 | PPC64)
INSTR (XORIS,     0x6c000000,  0xfc000000,  RA,    RS,    UIMM,   Void,  Void,  PPC32 | PPC64)

// operand types

TYPE (BD)
TYPE (BH)
TYPE (BI)
TYPE (BO)
TYPE (CRBA)
TYPE (CRBB)
TYPE (CRBD)
TYPE (CRFD)
TYPE (CRFS)
TYPE (CRM)
TYPE (D)
TYPE (DS)
TYPE (FM)
TYPE (FRA)
TYPE (FRB)
TYPE (FRC)
TYPE (FRD)
TYPE (FRS)
TYPE (IMM)
TYPE (L9)
TYPE (L10)
TYPE (L15)
TYPE (LI)
TYPE (LA)
TYPE (NB)
TYPE (MB)
TYPE (MBD)
TYPE (ME)
TYPE (MED)
TYPE (RA)
TYPE (RB)
TYPE (RBS)
TYPE (RD)
TYPE (RS)
TYPE (SH)
TYPE (SHD)
TYPE (SIMM)
TYPE (SIMMN)
TYPE (SPR)
TYPE (SR)
TYPE (TBR)
TYPE (TH)
TYPE (TO)
TYPE (UIMM)

#undef INSTR
#undef MNEM
#undef TYPE

// RISC instruction set definitions
// Copyright (C) Florian Negele

// This file is part of the Eigen Compiler Suite.

// The ECS is free software: you can redistribute it and/or modify
// it under the terms of the GNU General Public License as published by
// the Free Software Foundation, either version 3 of the License, or
// (at your option) any later version.

// The ECS is distributed in the hope that it will be useful,
// but WITHOUT ANY WARRANTY; without even the implied warranty of
// MERCHANTABILITY or FITNESS FOR A PARTICULAR PURPOSE.  See the
// GNU General Public License for more details.

// You should have received a copy of the GNU General Public License
// along with the ECS.  If not, see <https://www.gnu.org/licenses/>.

#ifndef INSTR
	#define INSTR(mnem, code, mask, type1, type2, type3)
#endif

#ifndef MNEM
	#define MNEM(name, mnem, ...)
#endif

#ifndef TYPE
	#define TYPE(type)
#endif

// mnemonics

MNEM (adc,   ADC,   Add with Carry)
MNEM (add,   ADD,   Add)
MNEM (and,   AND,   Logical AND)
MNEM (ann,   ANN,   Logical AND NOT)
MNEM (asr,   ASR,   Arithmetic Shift Right)
MNEM (b,     B,     Branch)
MNEM (bcc,   BCC,   Branch if Carry Clear)
MNEM (bcs,   BCS,   Branch if Carry Set (Lower))
MNEM (beq,   BEQ,   Branch if Equal (Zero))
MNEM (bge,   BGE,   Branch if Greater or Equal)
MNEM (bgt,   BGT,   Branch if Greater Than)
MNEM (bhi,   BHI,   Branch if Higher)
MNEM (bl,    BL,    Branch and Link)
MNEM (blcc,  BLCC,  Branch and Link if Carry Clear)
MNEM (blcs,  BLCS,  Branch and Link if Carry Set (Lower))
MNEM (ble,   BLE,   Branch if Less or Equal)
MNEM (bleq,  BLEQ,  Branch and Link if Equal (Zero))
MNEM (blge,  BLGE,  Branch and Link if Greater or Equal)
MNEM (blgt,  BLGT,  Branch and Link if Greater Than)
MNEM (blhi,  BLHI,  Branch and Link if Higher)
MNEM (blle,  BLLE,  Branch and Link if Less or Equal)
MNEM (blls,  BLLS,  Branch and Link if Lower or Same)
MNEM (bllt,  BLLT,  Branch and Link if Less Than)
MNEM (blmi,  BLMI,  Branch and Link if Negative (Minus))
MNEM (blne,  BLNE,  Branch and Link if Not Equal)
MNEM (blpl,  BLPL,  Branch and Link if Positive (Plus))
MNEM (bls,   BLS,   Branch if Lower or Same)
MNEM (blt,   BLT,   Branch if Less Than)
MNEM (blvc,  BLVC,  Branch and Link if Overflow Clear)
MNEM (blvs,  BLVS,  Branch and Link if Overflow Set)
MNEM (bmi,   BMI,   Branch if Negative (Minus))
MNEM (bne,   BNE,   Branch if Not Equal)
MNEM (bpl,   BPL,   Branch if Positive (Plus))
MNEM (bvc,   BVC,   Branch if Overflow Clear)
MNEM (bvs,   BVS,   Branch if Overflow Set)
MNEM (cli,   CLI,   Clear interrupt)
MNEM (div,   DIV,   Divide)
MNEM (fad,   FAD,   Floating-Point Add)
MNEM (fdv,   FDV,   Floating-Point Divide)
MNEM (fml,   FML,   Floating-Point Multiply)
MNEM (fsb,   FSB,   Floating-Point Subtract)
MNEM (ior,   IOR,   Logical Inclusive OR)
MNEM (ld,    LD,    Load Word)
MNEM (ldb,   LDB,   Load Byte)
MNEM (lsl,   LSL,   Logical Shift Left)
MNEM (mlu,   MLU,   Multiply Unsigned)
MNEM (mov,   MOV,   Move)
MNEM (mul,   MUL,   Multiply)
MNEM (mvf,   MVF,   Move Flags)
MNEM (mvh,   MVH,   Move H Register)
MNEM (mvs,   MVS,   Move Shifted)
MNEM (nop,   NOP,   No Operation)
MNEM (ror,   ROR,   Rotate Right)
MNEM (rti,   RTI,   Return from interrupt)
MNEM (sbc,   SBC,   Subtract with Carry)
MNEM (st,    ST,    Store Word)
MNEM (stb,   STB,   Store Byte)
MNEM (sti,   STI,   Set interrupt)
MNEM (sub,   SUB,   Subtract)
MNEM (xor,   XOR,   Logical Exclusive OR)

// special features

INSTR (NOP,   0xef000000,  0xffffffff,  Void,  Void,  Void)

INSTR (ADC,   0x20080000,  0xf00f0000,  Ra,    Rb,    Rc)
INSTR (ADC,   0x60080000,  0xe00f0000,  Ra,    Rb,    Imm)

INSTR (SBC,   0x20090000,  0xf00f0000,  Ra,    Rb,    Rc)
INSTR (SBC,   0x60090000,  0xe00f0000,  Ra,    Rb,    Imm)

INSTR (MLU,   0x200a0000,  0xf00f0000,  Ra,    Rb,    Rc)
INSTR (MLU,   0x600a0000,  0xe00f0000,  Ra,    Rb,    Imm)

INSTR (MVH,   0x30000000,  0xf00f000f,  Ra,    Void,  Void)

INSTR (MVF,   0x20000001,  0xf00f000f,  Ra,    Void,  Void)

INSTR (MVS,   0x60000000,  0xe00f0000,  Ra,    Imm,   Void)

// interrupts

INSTR (RTI,   0xc7000010,  0xff0000f0,  Rc,    Void,  Void)

INSTR (STI,   0xcf000021,  0xff0000ff,  Void,  Void,  Void)

INSTR (CLI,   0xcf000020,  0xff0000ff,  Void,  Void,  Void)

// register instructions

INSTR (MOV,   0x00000000,  0xf00f0000,  Ra,    Rc,    Void)
INSTR (MOV,   0x40000000,  0xe00f0000,  Ra,    Imm,   Void)

INSTR (LSL,   0x00010000,  0xf00f0000,  Ra,    Rb,    Rc)
INSTR (LSL,   0x40010000,  0xe00f0000,  Ra,    Rb,    Imm)

INSTR (ASR,   0x00020000,  0xf00f0000,  Ra,    Rb,    Rc)
INSTR (ASR,   0x40020000,  0xe00f0000,  Ra,    Rb,    Imm)

INSTR (ROR,   0x00030000,  0xf00f0000,  Ra,    Rb,    Rc)
INSTR (ROR,   0x40030000,  0xe00f0000,  Ra,    Rb,    Imm)

INSTR (AND,   0x00040000,  0xf00f0000,  Ra,    Rb,    Rc)
INSTR (AND,   0x40040000,  0xe00f0000,  Ra,    Rb,    Imm)

INSTR (ANN,   0x00050000,  0xf00f0000,  Ra,    Rb,    Rc)
INSTR (ANN,   0x40050000,  0xe00f0000,  Ra,    Rb,    Imm)

INSTR (IOR,   0x00060000,  0xf00f0000,  Ra,    Rb,    Rc)
INSTR (IOR,   0x40060000,  0xe00f0000,  Ra,    Rb,    Imm)

INSTR (XOR,   0x00070000,  0xf00f0000,  Ra,    Rb,    Rc)
INSTR (XOR,   0x40070000,  0xe00f0000,  Ra,    Rb,    Imm)

INSTR (ADD,   0x00080000,  0xf00f0000,  Ra,    Rb,    Rc)
INSTR (ADD,   0x40080000,  0xe00f0000,  Ra,    Rb,    Imm)

INSTR (SUB,   0x00090000,  0xf00f0000,  Ra,    Rb,    Rc)
INSTR (SUB,   0x40090000,  0xe00f0000,  Ra,    Rb,    Imm)

INSTR (MUL,   0x000a0000,  0xf00f0000,  Ra,    Rb,    Rc)
INSTR (MUL,   0x400a0000,  0xe00f0000,  Ra,    Rb,    Imm)

INSTR (DIV,   0x000b0000,  0xf00f0000,  Ra,    Rb,    Rc)
INSTR (DIV,   0x400b0000,  0xe00f0000,  Ra,    Rb,    Imm)

INSTR (FAD,   0x000c0000,  0xf00f0000,  Ra,    Rb,    Rc)
INSTR (FAD,   0x400c0000,  0xe00f0000,  Ra,    Rb,    Imm)

INSTR (FSB,   0x000d0000,  0xf00f0000,  Ra,    Rb,    Rc)
INSTR (FSB,   0x400d0000,  0xe00f0000,  Ra,    Rb,    Imm)

INSTR (FML,   0x000e0000,  0xf00f0000,  Ra,    Rb,    Rc)
INSTR (FML,   0x400e0000,  0xe00f0000,  Ra,    Rb,    Imm)

INSTR (FDV,   0x000f0000,  0xf00f0000,  Ra,    Rb,    Rc)
INSTR (FDV,   0x400f0000,  0xe00f0000,  Ra,    Rb,    Imm)

// memory instructions

INSTR (LD,    0x80000000,  0xf0000000,  Ra,    Rb,    Dsp)
INSTR (LDB,   0x90000000,  0xf0000000,  Ra,    Rb,    Dsp)

INSTR (ST,    0xa0000000,  0xf0000000,  Ra,    Rb,    Dsp)
INSTR (STB,   0xb0000000,  0xf0000000,  Ra,    Rb,    Dsp)

// branch instructions

INSTR (BMI,   0xc0000000,  0xff000000,  Rc,    Void,  Void)
INSTR (BMI,   0xe0000000,  0xff000000,  Off,   Void,  Void)
INSTR (BLMI,  0xd0000000,  0xff000000,  Rc,    Void,  Void)
INSTR (BLMI,  0xf0000000,  0xff000000,  Off,   Void,  Void)

INSTR (BEQ,   0xc1000000,  0xff000000,  Rc,    Void,  Void)
INSTR (BEQ,   0xe1000000,  0xff000000,  Off,   Void,  Void)
INSTR (BLEQ,  0xd1000000,  0xff000000,  Rc,    Void,  Void)
INSTR (BLEQ,  0xf1000000,  0xff000000,  Off,   Void,  Void)

INSTR (BCS,   0xc2000000,  0xff000000,  Rc,    Void,  Void)
INSTR (BCS,   0xe2000000,  0xff000000,  Off,   Void,  Void)
INSTR (BLCS,  0xd2000000,  0xff000000,  Rc,    Void,  Void)
INSTR (BLCS,  0xf2000000,  0xff000000,  Off,   Void,  Void)

INSTR (BVS,   0xc3000000,  0xff000000,  Rc,    Void,  Void)
INSTR (BVS,   0xe3000000,  0xff000000,  Off,   Void,  Void)
INSTR (BLVS,  0xd3000000,  0xff000000,  Rc,    Void,  Void)
INSTR (BLVS,  0xf3000000,  0xff000000,  Off,   Void,  Void)

INSTR (BLS,   0xc4000000,  0xff000000,  Rc,    Void,  Void)
INSTR (BLS,   0xe4000000,  0xff000000,  Off,   Void,  Void)
INSTR (BLLS,  0xd4000000,  0xff000000,  Rc,    Void,  Void)
INSTR (BLLS,  0xf4000000,  0xff000000,  Off,   Void,  Void)

INSTR (BLT,   0xc5000000,  0xff000000,  Rc,    Void,  Void)
INSTR (BLT,   0xe5000000,  0xff000000,  Off,   Void,  Void)
INSTR (BLLT,  0xd5000000,  0xff000000,  Rc,    Void,  Void)
INSTR (BLLT,  0xf5000000,  0xff000000,  Off,   Void,  Void)

INSTR (BLE,   0xc6000000,  0xff000000,  Rc,    Void,  Void)
INSTR (BLE,   0xe6000000,  0xff000000,  Off,   Void,  Void)
INSTR (BLLE,  0xd6000000,  0xff000000,  Rc,    Void,  Void)
INSTR (BLLE,  0xf6000000,  0xff000000,  Off,   Void,  Void)

INSTR (B,     0xc7000000,  0xff000000,  Rc,    Void,  Void)
INSTR (B,     0xe7000000,  0xff000000,  Off,   Void,  Void)
INSTR (BL,    0xd7000000,  0xff000000,  Rc,    Void,  Void)
INSTR (BL,    0xf7000000,  0xff000000,  Off,   Void,  Void)

INSTR (BPL,   0xc8000000,  0xff000000,  Rc,    Void,  Void)
INSTR (BPL,   0xe8000000,  0xff000000,  Off,   Void,  Void)
INSTR (BLPL,  0xd8000000,  0xff000000,  Rc,    Void,  Void)
INSTR (BLPL,  0xf8000000,  0xff000000,  Off,   Void,  Void)

INSTR (BNE,   0xc9000000,  0xff000000,  Rc,    Void,  Void)
INSTR (BNE,   0xe9000000,  0xff000000,  Off,   Void,  Void)
INSTR (BLNE,  0xd9000000,  0xff000000,  Rc,    Void,  Void)
INSTR (BLNE,  0xf9000000,  0xff000000,  Off,   Void,  Void)

INSTR (BCC,   0xca000000,  0xff000000,  Rc,    Void,  Void)
INSTR (BCC,   0xea000000,  0xff000000,  Off,   Void,  Void)
INSTR (BLCC,  0xda000000,  0xff000000,  Rc,    Void,  Void)
INSTR (BLCC,  0xfa000000,  0xff000000,  Off,   Void,  Void)

INSTR (BVC,   0xcb000000,  0xff000000,  Rc,    Void,  Void)
INSTR (BVC,   0xeb000000,  0xff000000,  Off,   Void,  Void)
INSTR (BLVC,  0xdb000000,  0xff000000,  Rc,    Void,  Void)
INSTR (BLVC,  0xfb000000,  0xff000000,  Off,   Void,  Void)

INSTR (BHI,   0xcc000000,  0xff000000,  Rc,    Void,  Void)
INSTR (BHI,   0xec000000,  0xff000000,  Off,   Void,  Void)
INSTR (BLHI,  0xdc000000,  0xff000000,  Rc,    Void,  Void)
INSTR (BLHI,  0xfc000000,  0xff000000,  Off,   Void,  Void)

INSTR (BGE,   0xcd000000,  0xff000000,  Rc,    Void,  Void)
INSTR (BGE,   0xed000000,  0xff000000,  Off,   Void,  Void)
INSTR (BLGE,  0xdd000000,  0xff000000,  Rc,    Void,  Void)
INSTR (BLGE,  0xfd000000,  0xff000000,  Off,   Void,  Void)

INSTR (BGT,   0xce000000,  0xff000000,  Rc,    Void,  Void)
INSTR (BGT,   0xee000000,  0xff000000,  Off,   Void,  Void)
INSTR (BLGT,  0xde000000,  0xff000000,  Rc,    Void,  Void)
INSTR (BLGT,  0xfe000000,  0xff000000,  Off,   Void,  Void)

// operand types

TYPE (Ra)
TYPE (Rb)
TYPE (Rc)
TYPE (Imm)
TYPE (Dsp)
TYPE (Off)

#undef INSTR
#undef MNEM
#undef TYPE

% WebAssembly architecture documentation
% Copyright (C) Florian Negele

% This file is part of the Eigen Compiler Suite.

% Permission is granted to copy, distribute and/or modify this document
% under the terms of the GNU Free Documentation License, Version 1.3
% or any later version published by the Free Software Foundation.

% You should have received a copy of the GNU Free Documentation License
% along with the ECS.  If not, see <https://www.gnu.org/licenses/>.

% Generic documentation utilities
% Copyright (C) Florian Negele

% This file is part of the Eigen Compiler Suite.

% Permission is granted to copy, distribute and/or modify this document
% under the terms of the GNU Free Documentation License, Version 1.3
% or any later version published by the Free Software Foundation.

% You should have received a copy of the GNU Free Documentation License
% along with the ECS.  If not, see <https://www.gnu.org/licenses/>.

\providecommand{\cpp}{C\texttt{++}}
\providecommand{\opt}{_\mathit{opt}}
\providecommand{\tool}[1]{\texttt{#1}}
\providecommand{\version}{Version 0.0.40}
\providecommand{\resource}[1]{*++\txt{#1}}
\providecommand{\ecs}{Eigen Compiler Suite}
\providecommand{\changed}[1]{\underline{#1}}
\providecommand{\toolbox}[1]{\converter{#1}}
\providecommand{\file}{}\renewcommand{\file}[1]{\texttt{#1}}
\providecommand{\alignright}{\hfill\linebreak[0]\hspace*{\fill}}
\providecommand{\converter}[1]{*++[F][F*:white][F,:gray]\txt{#1}}
\providecommand{\documentation}{\ifbook chapter\else document\fi}
\providecommand{\Documentation}{\ifbook Chapter\else Document\fi}
\providecommand{\variable}[1]{\resource{\texttt{\small#1}\\variable}}
\providecommand{\documentationref}[2]{\ifbook\ref{#1}\else``\href{#1}{#2}''~\cite{#1}\fi}
\providecommand{\objfile}[1]{\texttt{#1}\index[runtime]{#1 object file@\texttt{#1} object file}}
\providecommand{\libfile}[1]{\texttt{#1}\index[runtime]{#1 library file@\texttt{#1} library file}}
\providecommand{\epigraph}[2]{\ifbook\begin{quote}\flushright\textit{#1}\par--- #2\end{quote}\fi}
\providecommand{\environmentvariable}[1]{\texttt{#1}\index{Environment variables!#1@\texttt{#1}}}
\providecommand{\environment}[1]{\texttt{#1}\index[environment]{#1 environment@\texttt{#1} environment}}
\providecommand{\toolsection}{}\renewcommand{\toolsection}[1]{\subsection{#1}\label{\prefix:#1}\tool{#1}}
\providecommand{\instruction}{}\renewcommand{\instruction}[2]{\noindent\qquad\pdftooltip{\texttt{#1}}{#2}\refstepcounter{instruction}\par}
\providecommand{\flowgraph}{}\renewcommand{\flowgraph}[1]{\par\sffamily\begin{displaymath}\xymatrix@=4ex{#1}\end{displaymath}\normalfont\par}
\providecommand{\instructionset}{}\renewcommand{\instructionset}[4]{\setcounter{instruction}{0}\begin{multicols}{\ifbook#3\else#4\fi}[{\captionof{table}[#2]{#2 (\ref*{#1:instructions}~instructions)}\label{tab:#1set}\vspace{-2ex}}]\footnotesize\raggedcolumns\input{#1.set}\label{#1:instructions}\end{multicols}}

\providecommand{\gpl}{GNU General Public License}
\providecommand{\rse}{ECS Runtime Support Exception}
\providecommand{\fdl}{\href{https://www.gnu.org/licenses/fdl.html}{GNU Free Documentation License}}

\providecommand{\docbegin}{}
\providecommand{\docend}{}
\providecommand{\doclabel}[1]{\hypertarget{#1}}
\providecommand{\doclink}[2]{\hyperlink{#1}{#2}}
\providecommand{\docsection}[3]{\hypertarget{#1}{\subsection{#2}}\label{sec:#1}\index[library]{#2@#3}}
\providecommand{\docsectionstar}[1]{}
\providecommand{\docsubbegin}{\begin{description}}
\providecommand{\docsubend}{\end{description}}
\providecommand{\docsubsection}[3]{\item[\hypertarget{#1}{#2}]\index[library]{#2@#3}}
\providecommand{\docsubsectionstar}[1]{\smallskip}
\providecommand{\docsubsubsection}[3]{\docsubsection{#1}{#2}{#3}}
\providecommand{\docsubsubsectionstar}[1]{}
\providecommand{\docsubsubsubsection}[3]{}
\providecommand{\docsubsubsubsectionstar}[1]{}
\providecommand{\doctable}{}

\providecommand{\debuggingtool}{}\renewcommand{\debuggingtool}{This tool is provided for debugging purposes.
It allows exposing and modifying an internal data structure that is usually not accessible.
}

\providecommand{\interface}{All tools accept command-line arguments which are taken as names of plain text files containing the source code.
If no arguments are provided, the standard input stream is used instead.
Output files are generated in the current working directory and have the same name as the input file being processed whereas the filename extension gets replaced by an appropriate suffix.
\seeinterface
}

\providecommand{\license}{\noindent Copyright \copyright{} Florian Negele\par\medskip\noindent
Permission is granted to copy, distribute and/or modify this document under the terms of the
\fdl{}, Version 1.3 or any later version published by the \href{https://fsf.org/}{Free Software Foundation}.
}

\providecommand{\ecslogosurface}{
\fill[darkgray] (0,0,0) -- (0,0,3) -- (0,3,3) -- (0,3,1) -- (0,4,1) -- (0,4,3) -- (0,5,3) -- (0,5,0) -- (0,2,0) -- (0,2,2) -- (0,1,2) -- (0,1,0) -- cycle;
\fill[gray] (0,5,0) -- (0,5,3) -- (1,5,3) -- (1,5,1) -- (2,5,1) -- (2,5,3) -- (3,5,3) -- (3,5,0) -- cycle;
\fill[lightgray] (0,0,0) -- (0,1,0) -- (2,1,0) -- (2,4,0) -- (1,4,0) -- (1,3,0) -- (2,3,0) -- (2,2,0) -- (0,2,0) -- (0,5,0) -- (3,5,0) -- (3,0,0) -- cycle;
\begin{scope}[line width=0.5]
\begin{scope}[gray]
\draw (0,0,0) -- (0,1,0);
\draw (2,1,0) -- (2,2,0);
\draw (0,1,2) -- (0,2,2);
\draw (0,2,0) -- (0,5,0);
\draw (2,3,0) -- (2,4,0);
\end{scope}
\begin{scope}[lightgray]
\draw (0,1,0) -- (0,1,2);
\draw (0,3,1) -- (0,3,3);
\draw (0,5,0) -- (0,5,3);
\draw (2,5,1) -- (2,5,3);
\end{scope}
\begin{scope}[white]
\draw (0,1,0) -- (2,1,0);
\draw (1,3,0) -- (2,3,0);
\draw (0,5,0) -- (3,5,0);
\end{scope}
\end{scope}
}

\providecommand{\ecslogo}[1]{
\begin{tikzpicture}[scale={(#1)/((sin(45)+cos(45))*3cm)},x={({-cos(45)*1cm},{sin(45)*sin(30)*1cm})},y={({0cm},{(cos(30)*1cm})},z={({sin(45)*1cm},{cos(45)*sin(30)*1cm})}]
\begin{scope}[darkgray,line width=1]
\draw (0,0,0) -- (0,0,3) -- (0,3,3) -- (2,3,3) -- (2,5,3) -- (3,5,3) -- (3,5,0) -- (3,0,0) -- cycle;
\draw (0,3,1) -- (0,4,1) -- (0,4,3) -- (0,5,3) -- (1,5,3) -- (1,5,1) -- (2,5,1);
\draw (1,3,0) -- (1,4,0) -- (2,4,0);
\end{scope}
\fill[darkgray] (2,0,0) -- (2,0,3) -- (2,5,3) -- (2,5,1) -- (2,4,1) -- (2,4,0) -- cycle;
\fill[lightgray] (2,0,2) -- (0,0,2) -- (0,2,2) -- (2,2,2) -- cycle;
\fill[gray] (0,1,0) -- (2,1,0) -- (2,1,2) -- (0,1,2) -- cycle;
\fill[gray] (0,3,1) -- (0,3,3) -- (2,3,3) -- (2,3,0) -- (1,3,0) -- (1,3,1) -- cycle;
\ecslogosurface
\end{tikzpicture}
}

\providecommand{\shadowedecslogo}[3]{
\begin{tikzpicture}[scale={(#1)/((sin(#2)+cos(#2))*3cm)},x={({-cos(#2)*1cm},{sin(#2)*sin(#3)*1cm})},y={({0cm},{(cos(#3)*1cm})},z={({sin(#2)*1cm},{cos(#2)*sin(#3)*1cm})}]
\shade[top color=lightgray!50!white,bottom color=white,middle color=lightgray!50!white] (0,0,0) -- (3,0,0) -- (3,{-0.5-3*sin(#2)*sin(#3)/cos(#3)},0) -- (0,-0.5,0) -- cycle;
\shade[top color=darkgray!50!gray,bottom color=white,middle color=darkgray!50!white] (0,0,0) -- (0,0,3) -- (0,{-0.5-3*cos(#2)*sin(#3)/cos(#3)},3) -- (0,-0.5,0) -- cycle;
\begin{scope}[y={({(cos(#2)+sin(#2))*0.5cm},{(cos(#2)*sin(#3)-sin(#2)*sin(#3))*0.5cm})}]
\useasboundingbox (3,0,0) -- (0,0,0) -- (0,0,3);
\shade[left color=darkgray!80!black,right color=lightgray,middle color=gray] (0,0,0) -- (0,1,0) -- (0,1,0.5) -- (0,2,0) -- (0,5,0) -- (0,5,3) -- (1,5,3) -- (1,4,3) -- (1,4,2.5) -- (1,3,3) -- (2,5,3) -- (3,5,3) -- (3,0,3) -- cycle;
\clip (0,0,0) -- (0,0,3) -- ({-3*sin(#2)/cos(#2)},0,0) -- cycle;
\shade[left color=darkgray,right color=lightgray!50!gray] (0,0,0) -- (0,1,0) -- (0,1,0.5) -- (0,2,0) -- (0,5,0) -- (0,5,3) -- (1,5,3) -- (1,4,3) -- (1,4,2.5) -- (1,3,3) -- (2,5,3) -- (3,5,3) -- (3,0,3) -- cycle;
\end{scope}
\shade[left color=darkgray,right color=darkgray!80!black] (2,0,0) -- (2,0,3) -- (2,5,3) -- (2,5,1) -- (2,4,1) -- (2,4,0) -- cycle;
\shade[left color=darkgray!90!black,right color=gray!80!darkgray] (2,0,2) -- (0,0,2) -- (0,2,2) -- (2,2,2) -- cycle;
\shade[top color=darkgray!90!black,bottom color=gray!80!darkgray] (0,1,0) -- (2,1,0) -- (2,1,2) -- (0,1,2) -- cycle;
\shade[top color=darkgray!90!black,bottom color=gray!80!darkgray] (0,3,1) -- (0,3,3) -- (2,3,3) -- (2,3,0) -- (1,3,0) -- (1,3,1) -- cycle;
\fill[gray] (2,1,0) -- (1.5,1,0.5) -- (0,1,0.5) -- (0,1,0) -- cycle;
\fill[gray] (1,3,2) -- (0.5,3,2) -- (0.5,3,3) -- (1,3,3) -- cycle;
\fill[gray] (2,3,0) -- (1.5,3,0.5) -- (1,3,0.5) -- (1,3,0) -- cycle;
\ecslogosurface
\end{tikzpicture}
}

\providecommand{\cpplogo}[1]{
\begin{tikzpicture}[scale=(#1)/512em]
\fill[gray] (435.2794,398.7159) -- (247.1911,507.3075) .. controls (236.3563,513.5642) and (218.6240,513.5642) .. (207.7892,507.3075) -- (19.7009,398.7159) .. controls (8.8646,392.4606) and (0.0000,377.1043) .. (0.0000,364.5924) -- (0.0000,147.4076) .. controls (0.8430,132.8363) and (8.2856,120.7683) .. (19.7009,113.2842) -- (207.7892,4.6926) .. controls (218.6240,-1.5642) and (236.3564,-1.5642) .. (247.1911,4.6926) -- (435.2794,113.2842) .. controls (447.5273,121.4304) and (454.4987,133.6918) .. (454.9803,147.4076) -- (454.9803,364.5924) .. controls (454.5404,377.7571) and (446.6566,391.0351) .. (435.2794,398.7159) -- cycle(75.8301,255.9993) .. controls (74.9389,404.0881) and (273.2892,469.4783) .. (358.8263,331.8769) -- (293.1917,293.8965) .. controls (253.5702,359.4301) and (155.1909,335.9977) .. (151.6601,255.9993) .. controls (152.7204,182.2703) and (249.4137,148.0211) .. (293.1961,218.1065) -- (358.8308,180.1276) .. controls (283.4477,49.2645) and (79.6318,96.3470) .. (75.8301,255.9993) -- cycle(379.1503,247.5747) -- (362.2982,247.5747) -- (362.2982,230.7226) -- (345.4490,230.7226) -- (345.4490,247.5747) -- (328.5969,247.5747) -- (328.5969,264.4254) -- (345.4490,264.4254) -- (345.4490,281.2759) -- (362.2982,281.2759) -- (362.2982,264.4254) -- (379.1503,264.4254) -- cycle(442.3420,247.5747) -- (425.4899,247.5747) -- (425.4899,230.7226) -- (408.6408,230.7226) -- (408.6408,247.5747) -- (391.7886,247.5747) -- (391.7886,264.4254) -- (408.6408,264.4254) -- (408.6408,281.2759) -- (425.4899,281.2759) -- (425.4899,264.4254) -- (442.3420,264.4254) -- cycle;
\end{tikzpicture}
}

\providecommand{\fallogo}[1]{
\begin{tikzpicture}[scale=(#1)/512em]
\fill[gray] (185.7774,0.0000) .. controls (200.4486,15.9798) and (226.8966,8.7148) .. (235.0426,31.5836) .. controls (249.5297,58.0598) and (247.9581,97.9161) .. (280.3335,110.9762) .. controls (309.1690,120.3496) and (337.8406,104.2727) .. (366.5753,103.9379) .. controls (373.4449,111.5171) and (379.2885,128.2574) .. (383.9755,108.9744) .. controls (396.6979,102.5615) and (437.2808,107.6681) .. (426.9652,124.3252) .. controls (408.9822,121.0785) and (412.4742,146.0729) .. (426.5192,131.4996) .. controls (433.8413,120.8489) and (465.1541,126.5522) .. (441.9067,135.7950) .. controls (396.1879,157.7478) and (344.1112,161.5079) .. (298.5528,183.5702) .. controls (277.7471,193.5198) and (284.6941,218.7163) .. (285.2127,236.9640) .. controls (292.3599,316.2826) and (307.3929,394.6311) .. (317.1198,473.6154) .. controls (329.0637,505.4736) and (292.1195,528.5004) .. (265.9183,511.2761) .. controls (237.9284,499.2462) and (237.3684,465.2681) .. (230.9102,439.9421) .. controls (218.6692,374.3397) and (215.6307,306.9662) .. (198.1732,242.3977) .. controls (183.1379,232.7444) and (164.4245,256.0298) .. (149.0430,261.4799) .. controls (116.9328,279.2585) and (87.1822,308.5851) .. (48.2293,307.8914) .. controls (21.3220,306.9037) and (-15.9107,281.8761) .. (7.2921,252.7908) .. controls (29.7799,220.6177) and (67.5177,204.3028) .. (100.9287,185.9449) .. controls (130.8217,170.8906) and (161.1548,156.5903) .. (191.0278,141.5847) .. controls (196.1738,120.0520) and (186.6049,95.2409) .. (186.8382,72.4353) .. controls (185.5234,48.4204) and (183.1700,23.9341) .. (185.7774,0.0000) -- cycle;
\end{tikzpicture}
}

\providecommand{\oblogo}[1]{
\begin{tikzpicture}[scale=(#1)/512em]
\fill[gray] (160.3865,208.9117) .. controls (154.0879,214.6478) and (149.0735,221.2409) .. (145.4125,228.5384) .. controls (184.8790,248.4273) and (234.7122,269.8787) .. (297.5493,291.8782) .. controls (300.3943,281.4769) and (300.9552,268.7619) .. (300.4023,255.2389) .. controls (248.9909,244.7891) and (200.0310,225.9279) .. (160.3865,208.9117) -- cycle(225.7398,392.6996) .. controls (308.0209,392.1716) and (359.3326,345.9277) .. (368.7203,285.2098) .. controls (376.6742,197.1784) and (311.7194,141.3342) .. (205.4287,142.1456) .. controls (139.9485,141.4804) and (88.7155,166.1957) .. (73.5775,228.0086) .. controls (52.0297,320.3408) and (123.4078,391.0103) .. (225.7398,392.6996) -- cycle(216.0739,176.4733) .. controls (268.9183,179.2424) and (315.8292,206.5488) .. (312.7454,265.1139) .. controls (313.2769,315.6384) and (286.5993,353.4946) .. (216.6040,355.7934) .. controls (162.4657,355.7934) and (126.0914,317.5023) .. (126.0914,260.5103) .. controls (126.1733,214.2900) and (163.3363,176.2849) .. (216.0739,176.4733) -- cycle(76.4897,189.1754) .. controls (13.1586,147.5631) and (0.0000,119.4207) .. (0.0000,119.4207) -- (90.6499,170.1632) .. controls (85.3004,175.8497) and (80.5994,182.1633) .. (76.4897,189.1754) -- cycle(353.9486,119.3004) -- (402.9482,119.3004) .. controls (427.0025,137.0797) and (450.9893,162.7034) .. (474.9529,191.0213) .. controls (509.3540,228.5339) and (531.3391,294.2091) .. (487.8149,312.1206) .. controls (462.8165,324.7652) and (394.3874,316.8943) .. (373.8912,313.6651) .. controls (379.9291,297.7449) and (383.2899,278.4204) .. (381.4989,257.7214) .. controls (420.3069,248.0321) and (421.9610,218.3461) .. (407.7867,192.6417) .. controls (391.1113,162.4018) and (370.1114,132.9097) .. (353.9486,119.3004) -- cycle;
\end{tikzpicture}
}

\providecommand{\markuptable}{
\begin{table}
\sffamily\centering
\begin{tabular}{@{}lcl@{}}
\toprule
\texttt{//italics//} & $\rightarrow$ & \textit{italics} \\
\midrule
\texttt{**bold**} & $\rightarrow$ & \textbf{bold} \\
\midrule
\texttt{\# ordered list} & & 1 ordered list \\
\texttt{\# second item} & $\rightarrow$ & 2 second item \\
\texttt{\#\# sub item} & & \hspace{1em} 1 sub item \\
\midrule
\texttt{* unordered list} & & $\bullet$ unordered list \\
\texttt{* second item} & $\rightarrow$ & $\bullet$ second item \\
\texttt{** sub item} & & \hspace{1em} $\bullet$ sub item \\
\midrule
\texttt{link to [[label]]} & $\rightarrow$ & link to \underline{label} \\
\midrule
\texttt{<{}<label>{}> definition } & $\rightarrow$ & definition \\
\midrule
\texttt{[[url|link name]]} & $\rightarrow$ & \underline{link name} \\
\midrule\addlinespace
\texttt{= large heading} & & {\Large large heading} \smallskip \\
\texttt{== medium heading} & $\rightarrow$ & {\large medium heading} \\
\texttt{=== small heading} & & small heading \\
\midrule
\texttt{no line break} & & no line break for paragraphs \\
\texttt{for paragraphs} & $\rightarrow$ \\
& & use empty line \\
\texttt{use empty line} \\
\midrule
\texttt{force\textbackslash\textbackslash line break} & $\rightarrow$ & force \\
& & line break \\
\midrule
\texttt{horizontal line} & $\rightarrow$ & horizontal line \\
\texttt{----} & & \hrulefill \\
\midrule
\texttt{|=a|=table|=header} & & \underline{a \enspace table \enspace header} \\
\texttt{|a|table|row} & $\rightarrow$ & a \enspace table \enspace row \\
\texttt{|b|table|row} & & b \enspace table \enspace row \\
\midrule
\texttt{\{\{\{} \\
\texttt{unformatted} & $\rightarrow$ & \texttt{unformatted} \\
\texttt{code} & & \texttt{code} \\
\texttt{\}\}\}} \\
\midrule\addlinespace
\texttt{@ new article} & & {\Large 1.\ new article} \smallskip \\
\texttt{@ second article} & $\rightarrow$ & {\Large 2.\ second article} \smallskip \\
\texttt{@@ sub article} & & {\large 2.1.\ sub article} \\
\bottomrule
\end{tabular}
\normalfont\caption{Elements of the generic documentation markup language}
\label{tab:docmarkup}
\end{table}
}

\providecommand{\startchapter}[4]{
\documentclass[11pt,a4paper]{article}
\usepackage{booktabs}
\usepackage[format=hang,labelfont=bf]{caption}
\usepackage{changepage}
\usepackage[T1]{fontenc}
\usepackage[margin=2cm]{geometry}
\usepackage{hyperref}
\usepackage[american]{isodate}
\usepackage{lmodern}
\usepackage{longtable}
\usepackage{mathptmx}
\usepackage{microtype}
\usepackage[toc]{multitoc}
\usepackage{multirow}
\usepackage[all]{nowidow}
\usepackage{pdfcomment}
\usepackage{syntax}
\usepackage{tikz}
\usepackage[all]{xy}
\hypersetup{pdfborder={0 0 0},bookmarksnumbered=true,pdftitle={\ecs{}: #2},pdfauthor={Florian Negele},pdfsubject={\ecs{}},pdfkeywords={#1}}
\setlength{\grammarindent}{8em}\setlength{\grammarparsep}{0.2ex}
\setlength{\columnsep}{2em}
\newcommand{\prefix}{}
\newcounter{instruction}
\bibliographystyle{unsrt}
\renewcommand{\index}[2][]{}
\renewcommand{\arraystretch}{1.05}
\renewcommand{\floatpagefraction}{0.7}
\renewcommand{\syntleft}{\itshape}\renewcommand{\syntright}{}
\title{\vspace{-5ex}\Huge{\ecs{}}\medskip\hrule}
\author{\huge{#2}}
\date{\medskip\version}
\newif\ifbook\bookfalse
\pagestyle{headings}
\frenchspacing
\begin{document}
\maketitle\thispagestyle{empty}\noindent#4\setlength{\columnseprule}{0.4pt}\tableofcontents\setlength{\columnseprule}{0pt}\vfill\pagebreak[3]\null\vfill\bigskip\noindent
\parbox{\textwidth-4em}{\license The contents of this \documentation{} are part of the \href{manual}{\ecs{} User Manual}~\cite{manual} and correspond to Chapter ``\href{manual\##3}{#1}''.\alignright\mbox{\today}}
\parbox{4em}{\flushright\ecslogo{3em}}
\clearpage
}

\providecommand{\concludechapter}{
\vfill\pagebreak[3]\null\vfill
\thispagestyle{myheadings}\markright{REFERENCES}
\noindent\begin{minipage}{\textwidth}\begin{multicols}{2}[\section*{References}]
\renewcommand{\section}[2]{}\small\bibliography{references}
\end{multicols}\end{minipage}\end{document}
}

\providecommand{\startpresentation}[2]{
\documentclass[14pt,aspectratio=43,usepdftitle=false]{beamer}
\usepackage{booktabs}
\usepackage{etex}
\usepackage{multicol}
\usepackage{tikz}
\usepackage[all]{xy}
\bibliographystyle{unsrt}
\setlength{\columnsep}{1em}
\setlength{\leftmargini}{1em}
\setbeamercolor{title}{fg=black}
\setbeamercolor{structure}{fg=darkgray}
\setbeamercolor{bibliography item}{fg=darkgray}
\setbeamerfont{title}{series=\bfseries}
\setbeamerfont{subtitle}{series=\normalfont}
\setbeamerfont*{frametitle}{parent=title}
\setbeamerfont{block title}{series=\bfseries}
\setbeamerfont*{framesubtitle}{parent=subtitle}
\setbeamersize{text margin left=1em,text margin right=1em}
\setbeamertemplate{navigation symbols}{}
\setbeamertemplate{itemize item}[circle]{}
\setbeamertemplate{bibliography item}[triangle]{}
\setbeamertemplate{bibliography entry author}{\usebeamercolor[fg]{bibliography item}}
\setbeamertemplate{frametitle}{\medskip\usebeamerfont{frametitle}\color{gray}\raisebox{-2.5ex}[0ex][0ex]{\rule{0.1em}{4.5ex}}}
\addtobeamertemplate{frametitle}{}{\hspace{0.4em}\usebeamercolor[fg]{title}\insertframetitle\par\vspace{0.2ex}\hspace{0.5em}\usebeamerfont{framesubtitle}\insertframesubtitle}
\hypersetup{pdfborder={0 0 0},bookmarksnumbered=true,bookmarksopen=true,bookmarksopenlevel=0,pdftitle={\ecs{}: #1},pdfauthor={Florian Negele},pdfsubject={\ecs{}},pdfkeywords={#1}}
\renewcommand{\flowgraph}[1]{\resizebox{\textwidth}{!}{$$\xymatrix{##1}$$}}
\title{\ecs{}\medskip\hrule\medskip}
\institute{\shadowedecslogo{5em}{30}{15}}
\date{\version}
\subtitle{#1}
\begin{document}
\begin{frame}[plain]\titlepage\nocite{manual}\end{frame}
\begin{frame}{Contents}{#1}\begin{center}\tableofcontents\end{center}\end{frame}
}

\providecommand{\concludepresentation}{
\begin{frame}{References}\begin{footnotesize}\setlength{\columnseprule}{0.4pt}\begin{multicols}{2}\bibliography{references}\end{multicols}\end{footnotesize}\end{frame}
\end{document}
}

\providecommand{\startbook}[1]{
\documentclass[10pt,paper=17cm:24cm,DIV=13,twoside=semi,headings=normal,numbers=noendperiod,cleardoublepage=plain]{scrbook}
\usepackage{atveryend}
\usepackage{booktabs}
\usepackage{caption}
\usepackage{changepage}
\usepackage[T1]{fontenc}
\usepackage{imakeidx}
\usepackage{hyperref}
\usepackage[american]{isodate}
\usepackage{lmodern}
\usepackage{longtable}
\usepackage{mathptmx}
\usepackage[final]{microtype}
\usepackage{multicol}
\usepackage{multirow}
\usepackage[all]{nowidow}
\usepackage{pdfcomment}
\usepackage{scrlayer-scrpage}
\usepackage{setspace}
\usepackage{syntax}
\usepackage[eventxtindent=4pt,oddtxtexdent=4pt]{thumbs}
\usepackage{tikz}
\usepackage[all]{xy}
\hyphenation{Micro-Blaze Open-Cores Open-RISC Power-PC}
\hypersetup{pdfborder={0 0 0},bookmarksnumbered=true,bookmarksopen=true,bookmarksopenlevel=0,pdftitle={\ecs{}: #1},pdfauthor={Florian Negele},pdfsubject={\ecs{}},pdfkeywords={#1}}
\setlength{\grammarindent}{8em}\setlength{\grammarparsep}{0.7ex}
\setkomafont{captionlabel}{\usekomafont{descriptionlabel}}
\renewcommand{\arraystretch}{1.05}\setstretch{1.1}
\renewcommand{\chapterformat}{\thechapter\autodot\enskip\raisebox{-1ex}[0ex][0ex]{\color{gray}\rule{0.1em}{3.5ex}}\enskip}
\renewcommand{\startchapter}[4]{\hypertarget{##3}{\chapter{##1}}\label{##3}##4\addthumb{##1}{\LARGE\sffamily\bfseries\thechapter}{white}{gray}\renewcommand{\prefix}{##3}}
\renewcommand{\concludechapter}{\clearpage{\stopthumb\cleardoublepage}}
\renewcommand{\syntleft}{\itshape}\renewcommand{\syntright}{}
\renewcommand{\floatpagefraction}{0.7}
\renewcommand{\partheademptypage}{}
\DeclareMicrotypeAlias{lmss}{cmr}
\newcommand{\prefix}{}
\newcounter{instruction}
\bibliographystyle{unsrt}
\newif\ifbook\booktrue
\makeindex[intoc,title=Index]
\makeindex[intoc,name=tools,title=Index of Tools,columns=3]
\makeindex[intoc,name=library,title=Index of Library Names]
\makeindex[intoc,name=runtime,title=Index of Runtime Support]
\makeindex[intoc,name=environment,title=Index of Target Environments]
\indexsetup{toclevel=chapter,headers={\indexname}{\indexname}}
\frenchspacing
\begin{document}
\pagenumbering{alph}
\begin{titlepage}\centering
\huge\sffamily\null\vfill\textbf{\ecs{}}\bigskip\hrule\bigskip#1
\normalsize\normalfont\vfill\vfill\shadowedecslogo{10em}{30}{15}
\large\vfill\vfill\version
\end{titlepage}
\null\vfill
\thispagestyle{empty}
\noindent\today\par\medskip
\license A copy of this license is included in Appendix~\ref{fdl} on page~\pageref{fdl}.
All product names used herein are for identification purposes only and may be trademarks of their respective companies.
\concludechapter
\frontmatter
\setcounter{tocdepth}{1}
\tableofcontents
\setcounter{tocdepth}{2}
\concludechapter
\listoffigures
\concludechapter
\listoftables
\concludechapter
}

\providecommand{\concludebook}{
\backmatter
\addtocontents{toc}{\protect\setcounter{tocdepth}{-1}}
\phantomsection\addcontentsline{toc}{part}{Bibliography}
\bibliography{references}
\concludechapter
\phantomsection\addcontentsline{toc}{part}{Indexes}
\printindex
\concludechapter
\indexprologue{\label{idx:tools}}
\printindex[tools]
\concludechapter
\printindex[library]
\concludechapter
\indexprologue{\label{idx:runtime}}
\printindex[runtime]
\concludechapter
\indexprologue{\label{idx:environment}}
\printindex[environment]
\concludechapter
\pagestyle{empty}\pagenumbering{Alph}\null\clearpage
\null\vfill\centering\ecslogo{4em}\par\medskip\license
\end{document}
}

% chapter references

\providecommand{\seedocumentationref}{}\renewcommand{\seedocumentationref}[3]{#1, see \Documentation{}~\documentationref{#2}{#3}. }
\providecommand{\seeinterface}{}\renewcommand{\seeinterface}{\ifbook See \Documentation{}~\documentationref{interface}{User Interface} for more information about the common user interface of all of these tools. \fi}
\providecommand{\seeguide}{}\renewcommand{\seeguide}{\seedocumentationref{For basic examples of using some of these tools in practice}{guide}{User Guide}}
\providecommand{\seecpp}{}\renewcommand{\seecpp}{\seedocumentationref{For more information about the \cpp{} programming language and its implementation by the \ecs{}}{cpp}{User Manual for \cpp{}}}
\providecommand{\seefalse}{}\renewcommand{\seefalse}{\seedocumentationref{For more information about the FALSE programming language and its implementation by the \ecs{}}{false}{User Manual for FALSE}}
\providecommand{\seeoberon}{}\renewcommand{\seeoberon}{\seedocumentationref{For more information about the Oberon programming language and its implementation by the \ecs{}}{oberon}{User Manual for Oberon}}
\providecommand{\seeassembly}{}\renewcommand{\seeassembly}{\seedocumentationref{For more information about the generic assembly language and how to use it}{assembly}{Generic Assembly Language Specification}}
\providecommand{\seeamd}{}\renewcommand{\seeamd}{\seedocumentationref{For more information about how the \ecs{} supports the AMD64 hardware architecture}{amd64}{AMD64 Hardware Architecture Support}}
\providecommand{\seearm}{}\renewcommand{\seearm}{\seedocumentationref{For more information about how the \ecs{} supports the ARM hardware architecture}{arm}{ARM Hardware Architecture Support}}
\providecommand{\seeavr}{}\renewcommand{\seeavr}{\seedocumentationref{For more information about how the \ecs{} supports the AVR hardware architecture}{avr}{AVR Hardware Architecture Support}}
\providecommand{\seeavrtt}{}\renewcommand{\seeavrtt}{\seedocumentationref{For more information about how the \ecs{} supports the AVR32 hardware architecture}{avr32}{AVR32 Hardware Architecture Support}}
\providecommand{\seemabk}{}\renewcommand{\seemabk}{\seedocumentationref{For more information about how the \ecs{} supports the M68000 hardware architecture}{m68k}{M68000 Hardware Architecture Support}}
\providecommand{\seemibl}{}\renewcommand{\seemibl}{\seedocumentationref{For more information about how the \ecs{} supports the MicroBlaze hardware architecture}{mibl}{MicroBlaze Hardware Architecture Support}}
\providecommand{\seemips}{}\renewcommand{\seemips}{\seedocumentationref{For more information about how the \ecs{} supports the MIPS32 and MIPS64 hardware architectures}{mips}{MIPS Hardware Architecture Support}}
\providecommand{\seemmix}{}\renewcommand{\seemmix}{\seedocumentationref{For more information about how the \ecs{} supports the MMIX hardware architecture}{mmix}{MMIX Hardware Architecture Support}}
\providecommand{\seeorok}{}\renewcommand{\seeorok}{\seedocumentationref{For more information about how the \ecs{} supports the OpenRISC 1000 hardware architecture}{or1k}{OpenRISC 1000 Hardware Architecture Support}}
\providecommand{\seeppc}{}\renewcommand{\seeppc}{\seedocumentationref{For more information about how the \ecs{} supports the PowerPC hardware architecture}{ppc}{PowerPC Hardware Architecture Support}}
\providecommand{\seerisc}{}\renewcommand{\seerisc}{\seedocumentationref{For more information about how the \ecs{} supports the RISC hardware architecture}{risc}{RISC Hardware Architecture Support}}
\providecommand{\seewasm}{}\renewcommand{\seewasm}{\seedocumentationref{For more information about how the \ecs{} supports the WebAssembly architecture}{wasm}{WebAssembly Architecture Support}}
\providecommand{\seedocumentation}{}\renewcommand{\seedocumentation}{\seedocumentationref{For more information about generic documentations and their generation by the \ecs{}}{documentation}{Generic Documentation Generation}}
\providecommand{\seedebugging}{}\renewcommand{\seedebugging}{\seedocumentationref{For more information about debugging information and its representation}{debugging}{Debugging Information Representation}}
\providecommand{\seecode}{}\renewcommand{\seecode}{\seedocumentationref{For more information about intermediate code and its purpose}{code}{Intermediate Code Representation}}
\providecommand{\seeobject}{}\renewcommand{\seeobject}{\seedocumentationref{For more information about object files and their purpose}{object}{Object File Representation}}

% generic documentation tools

\providecommand{\docprint}{
\toolsection{docprint} is a pretty printer for generic documentations.
It reformats generic documentations and writes it to the standard output stream.
\debuggingtool
\flowgraph{\resource{generic\\documentation} \ar[r] & \toolbox{docprint} \ar[r] & \resource{generic\\documentation}}
\seedocumentation
}

\providecommand{\doccheck}{
\toolsection{doccheck} is a syntactic and semantic checker for generic documentations.
It just performs syntactic and semantic checks on generic documentations and writes its diagnostic messages to the standard error stream.
\debuggingtool
\flowgraph{\resource{generic\\documentation} \ar[r] & \toolbox{doccheck} \ar[r] & \resource{diagnostic\\messages}}
\seedocumentation
}

\providecommand{\dochtml}{
\toolsection{dochtml} is an HTML documentation generator for generic documentations.
It processes several generic documentations and assembles all information therein into an HTML document.
\debuggingtool
\flowgraph{\resource{generic\\documentation} \ar[r] & \toolbox{dochtml} \ar[r] & \resource{HTML\\document}}
\seedocumentation
}

\providecommand{\doclatex}{
\toolsection{doclatex} is a Latex documentation generator for generic documentations.
It processes several generic documentations and assembles all information therein into a Latex document.
\debuggingtool
\flowgraph{\resource{generic\\documentation} \ar[r] & \toolbox{doclatex} \ar[r] & \resource{Latex\\document}}
\seedocumentation
}

% intermediate code tools

\providecommand{\cdcheck}{
\toolsection{cdcheck} is a syntactic and semantic checker for intermediate code.
It just performs syntactic and semantic checks on programs written in intermediate code and writes its diagnostic messages to the standard error stream.
\debuggingtool
\flowgraph{\resource{intermediate\\code} \ar[r] & \toolbox{cdcheck} \ar[r] & \resource{diagnostic\\messages}}
\seeassembly\seecode
}

\providecommand{\cdopt}{
\toolsection{cdopt} is an optimizer for intermediate code.
It performs various optimizations on programs written in intermediate code and writes the result to the standard output stream.
\debuggingtool
\flowgraph{\resource{intermediate\\code} \ar[r] & \toolbox{cdopt} \ar[r] & \resource{optimized\\code}}
\seeassembly\seecode
}

\providecommand{\cdrun}{
\toolsection{cdrun} is an interpreter for intermediate code.
It processes and executes programs written in intermediate code.
The following code sections are predefined and have the usual semantics:
\texttt{abort}, \texttt{\_Exit}, \texttt{fflush}, \texttt{floor}, \texttt{fputc}, \texttt{free}, \texttt{getchar}, \texttt{malloc}, and \texttt{putchar}.
Diagnostic messages about invalid operations include the name of the executed code section and the index of the erroneous instruction.
\debuggingtool
\flowgraph{\resource{intermediate\\code} \ar[r] & \toolbox{cdrun} \ar@/u/[r] & \resource{input/\\output} \ar@/d/[l]}
\seeassembly\seecode
}

\providecommand{\cdamda}{
\toolsection{cdamd16} is a compiler for intermediate code targeting the AMD64 hardware architecture.
It generates machine code for AMD64 processors from programs written in intermediate code and stores it in corresponding object files.
The compiler generates machine code for the 16-bit operating mode defined by the AMD64 architecture.
It also creates a debugging information file as well as an assembly file containing a listing of the generated machine code.
\debuggingtool
\flowgraph{\resource{intermediate\\code} \ar[r] & \toolbox{cdamd16} \ar[r] \ar[d] \ar[rd] & \resource{object file} \\ & \resource{assembly\\listing} & \resource{debugging\\information}}
\seeassembly\seeamd\seeobject\seecode\seedebugging
}

\providecommand{\cdamdb}{
\toolsection{cdamd32} is a compiler for intermediate code targeting the AMD64 hardware architecture.
It generates machine code for AMD64 processors from programs written in intermediate code and stores it in corresponding object files.
The compiler generates machine code for the 32-bit operating mode defined by the AMD64 architecture.
It also creates a debugging information file as well as an assembly file containing a listing of the generated machine code.
\debuggingtool
\flowgraph{\resource{intermediate\\code} \ar[r] & \toolbox{cdamd32} \ar[r] \ar[d] \ar[rd] & \resource{object file} \\ & \resource{assembly\\listing} & \resource{debugging\\information}}
\seeassembly\seeamd\seeobject\seecode\seedebugging
}

\providecommand{\cdamdc}{
\toolsection{cdamd64} is a compiler for intermediate code targeting the AMD64 hardware architecture.
It generates machine code for AMD64 processors from programs written in intermediate code and stores it in corresponding object files.
The compiler generates machine code for the 64-bit operating mode defined by the AMD64 architecture.
It also creates a debugging information file as well as an assembly file containing a listing of the generated machine code.
\debuggingtool
\flowgraph{\resource{intermediate\\code} \ar[r] & \toolbox{cdamd64} \ar[r] \ar[d] \ar[rd] & \resource{object file} \\ & \resource{assembly\\listing} & \resource{debugging\\information}}
\seeassembly\seeamd\seeobject\seecode\seedebugging
}

\providecommand{\cdarma}{
\toolsection{cdarma32} is a compiler for intermediate code targeting the ARM hardware architecture.
It generates machine code for ARM processors executing A32 instructions from programs written in intermediate code and stores it in corresponding object files.
It also creates a debugging information file as well as an assembly file containing a listing of the generated machine code.
\debuggingtool
\flowgraph{\resource{intermediate\\code} \ar[r] & \toolbox{cdarma32} \ar[r] \ar[d] \ar[rd] & \resource{object file} \\ & \resource{assembly\\listing} & \resource{debugging\\information}}
\seeassembly\seearm\seeobject\seecode\seedebugging
}

\providecommand{\cdarmb}{
\toolsection{cdarma64} is a compiler for intermediate code targeting the ARM hardware architecture.
It generates machine code for ARM processors executing A64 instructions from programs written in intermediate code and stores it in corresponding object files.
It also creates a debugging information file as well as an assembly file containing a listing of the generated machine code.
\debuggingtool
\flowgraph{\resource{intermediate\\code} \ar[r] & \toolbox{cdarma64} \ar[r] \ar[d] \ar[rd] & \resource{object file} \\ & \resource{assembly\\listing} & \resource{debugging\\information}}
\seeassembly\seearm\seeobject\seecode\seedebugging
}

\providecommand{\cdarmc}{
\toolsection{cdarmt32} is a compiler for intermediate code targeting the ARM hardware architecture.
It generates machine code for ARM processors without floating-point extension executing T32 instructions from programs written in intermediate code and stores it in corresponding object files.
It also creates a debugging information file as well as an assembly file containing a listing of the generated machine code.
\debuggingtool
\flowgraph{\resource{intermediate\\code} \ar[r] & \toolbox{cdarmt32} \ar[r] \ar[d] \ar[rd] & \resource{object file} \\ & \resource{assembly\\listing} & \resource{debugging\\information}}
\seeassembly\seearm\seeobject\seecode\seedebugging
}

\providecommand{\cdarmcfpe}{
\toolsection{cdarmt32fpe} is a compiler for intermediate code targeting the ARM hardware architecture.
It generates machine code for ARM processors with floating-point extension executing T32 instructions from programs written in intermediate code and stores it in corresponding object files.
It also creates a debugging information file as well as an assembly file containing a listing of the generated machine code.
\debuggingtool
\flowgraph{\resource{intermediate\\code} \ar[r] & \toolbox{cdarmt32fpe} \ar[r] \ar[d] \ar[rd] & \resource{object file} \\ & \resource{assembly\\listing} & \resource{debugging\\information}}
\seeassembly\seearm\seeobject\seecode\seedebugging
}

\providecommand{\cdavr}{
\toolsection{cdavr} is a compiler for intermediate code targeting the AVR hardware architecture.
It generates machine code for AVR processors from programs written in intermediate code and stores it in corresponding object files.
It also creates a debugging information file as well as an assembly file containing a listing of the generated machine code.
\debuggingtool
\flowgraph{\resource{intermediate\\code} \ar[r] & \toolbox{cdavr} \ar[r] \ar[d] \ar[rd] & \resource{object file} \\ & \resource{assembly\\listing} & \resource{debugging\\information}}
\seeassembly\seeavr\seeobject\seecode\seedebugging
}

\providecommand{\cdavrtt}{
\toolsection{cdavr32} is a compiler for intermediate code targeting the AVR32 hardware architecture.
It generates machine code for AVR32 processors from programs written in intermediate code and stores it in corresponding object files.
It also creates a debugging information file as well as an assembly file containing a listing of the generated machine code.
\debuggingtool
\flowgraph{\resource{intermediate\\code} \ar[r] & \toolbox{cdavr32} \ar[r] \ar[d] \ar[rd] & \resource{object file} \\ & \resource{assembly\\listing} & \resource{debugging\\information}}
\seeassembly\seeavrtt\seeobject\seecode\seedebugging
}

\providecommand{\cdmabk}{
\toolsection{cdm68k} is a compiler for intermediate code targeting the M68000 hardware architecture.
It generates machine code for M68000 processors from programs written in intermediate code and stores it in corresponding object files.
It also creates a debugging information file as well as an assembly file containing a listing of the generated machine code.
\debuggingtool
\flowgraph{\resource{intermediate\\code} \ar[r] & \toolbox{cdm68k} \ar[r] \ar[d] \ar[rd] & \resource{object file} \\ & \resource{assembly\\listing} & \resource{debugging\\information}}
\seeassembly\seemabk\seeobject\seecode\seedebugging
}

\providecommand{\cdmibl}{
\toolsection{cdmibl} is a compiler for intermediate code targeting the MicroBlaze hardware architecture.
It generates machine code for MicroBlaze processors from programs written in intermediate code and stores it in corresponding object files.
It also creates a debugging information file as well as an assembly file containing a listing of the generated machine code.
\debuggingtool
\flowgraph{\resource{intermediate\\code} \ar[r] & \toolbox{cdmibl} \ar[r] \ar[d] \ar[rd] & \resource{object file} \\ & \resource{assembly\\listing} & \resource{debugging\\information}}
\seeassembly\seemibl\seeobject\seecode\seedebugging
}

\providecommand{\cdmipsa}{
\toolsection{cdmips32} is a compiler for intermediate code targeting the MIPS32 hardware architecture.
It generates machine code for MIPS32 processors from programs written in intermediate code and stores it in corresponding object files.
It also creates a debugging information file as well as an assembly file containing a listing of the generated machine code.
\debuggingtool
\flowgraph{\resource{intermediate\\code} \ar[r] & \toolbox{cdmips32} \ar[r] \ar[d] \ar[rd] & \resource{object file} \\ & \resource{assembly\\listing} & \resource{debugging\\information}}
\seeassembly\seemips\seeobject\seecode\seedebugging
}

\providecommand{\cdmipsb}{
\toolsection{cdmips64} is a compiler for intermediate code targeting the MIPS64 hardware architecture.
It generates machine code for MIPS64 processors from programs written in intermediate code and stores it in corresponding object files.
It also creates a debugging information file as well as an assembly file containing a listing of the generated machine code.
\debuggingtool
\flowgraph{\resource{intermediate\\code} \ar[r] & \toolbox{cdmips64} \ar[r] \ar[d] \ar[rd] & \resource{object file} \\ & \resource{assembly\\listing} & \resource{debugging\\information}}
\seeassembly\seemips\seeobject\seecode\seedebugging
}

\providecommand{\cdmmix}{
\toolsection{cdmmix} is a compiler for intermediate code targeting the MMIX hardware architecture.
It generates machine code for MMIX processors from programs written in intermediate code and stores it in corresponding object files.
It also creates a debugging information file as well as an assembly file containing a listing of the generated machine code.
\debuggingtool
\flowgraph{\resource{intermediate\\code} \ar[r] & \toolbox{cdmmix} \ar[r] \ar[d] \ar[rd] & \resource{object file} \\ & \resource{assembly\\listing} & \resource{debugging\\information}}
\seeassembly\seemmix\seeobject\seecode\seedebugging
}

\providecommand{\cdorok}{
\toolsection{cdor1k} is a compiler for intermediate code targeting the OpenRISC 1000 hardware architecture.
It generates machine code for OpenRISC 1000 processors from programs written in intermediate code and stores it in corresponding object files.
It also creates a debugging information file as well as an assembly file containing a listing of the generated machine code.
\debuggingtool
\flowgraph{\resource{intermediate\\code} \ar[r] & \toolbox{cdor1k} \ar[r] \ar[d] \ar[rd] & \resource{object file} \\ & \resource{assembly\\listing} & \resource{debugging\\information}}
\seeassembly\seeorok\seeobject\seecode\seedebugging
}

\providecommand{\cdppca}{
\toolsection{cdppc32} is a compiler for intermediate code targeting the PowerPC hardware architecture.
It generates machine code for PowerPC processors from programs written in intermediate code and stores it in corresponding object files.
The compiler generates machine code for the 32-bit operating mode defined by the PowerPC architecture.
It also creates a debugging information file as well as an assembly file containing a listing of the generated machine code.
\debuggingtool
\flowgraph{\resource{intermediate\\code} \ar[r] & \toolbox{cdppc32} \ar[r] \ar[d] \ar[rd] & \resource{object file} \\ & \resource{assembly\\listing} & \resource{debugging\\information}}
\seeassembly\seeppc\seeobject\seecode\seedebugging
}

\providecommand{\cdppcb}{
\toolsection{cdppc64} is a compiler for intermediate code targeting the PowerPC hardware architecture.
It generates machine code for PowerPC processors from programs written in intermediate code and stores it in corresponding object files.
The compiler generates machine code for the 64-bit operating mode defined by the PowerPC architecture.
It also creates a debugging information file as well as an assembly file containing a listing of the generated machine code.
\debuggingtool
\flowgraph{\resource{intermediate\\code} \ar[r] & \toolbox{cdppc64} \ar[r] \ar[d] \ar[rd] & \resource{object file} \\ & \resource{assembly\\listing} & \resource{debugging\\information}}
\seeassembly\seeppc\seeobject\seecode\seedebugging
}

\providecommand{\cdrisc}{
\toolsection{cdrisc} is a compiler for intermediate code targeting the RISC hardware architecture.
It generates machine code for RISC processors from programs written in intermediate code and stores it in corresponding object files.
It also creates a debugging information file as well as an assembly file containing a listing of the generated machine code.
\debuggingtool
\flowgraph{\resource{intermediate\\code} \ar[r] & \toolbox{cdrisc} \ar[r] \ar[d] \ar[rd] & \resource{object file} \\ & \resource{assembly\\listing} & \resource{debugging\\information}}
\seeassembly\seerisc\seeobject\seecode\seedebugging
}

\providecommand{\cdwasm}{
\toolsection{cdwasm} is a compiler for intermediate code targeting the WebAssembly architecture.
It generates machine code for WebAssembly targets from programs written in intermediate code and stores it in corresponding object files.
It also creates a debugging information file as well as an assembly file containing a listing of the generated machine code.
\debuggingtool
\flowgraph{\resource{intermediate\\code} \ar[r] & \toolbox{cdwasm} \ar[r] \ar[d] \ar[rd] & \resource{object file} \\ & \resource{assembly\\listing} & \resource{debugging\\information}}
\seeassembly\seewasm\seeobject\seecode\seedebugging
}

% C++ tools

\providecommand{\cppprep}{
\toolsection{cppprep} is a preprocessor for the \cpp{} programming language.
It preprocesses source code according to the rules of \cpp{} and writes it to the standard output stream.
Only the macro names \texttt{\_\_DATE\_\_}, \texttt{\_\_FILE\_\_}, \texttt{\_\_LINE\_\_}, and \texttt{\_\_TIME\_\_} are predefined.
\flowgraph{\resource{\cpp{} or other\\source code} \ar[r] & \toolbox{cppprep} \ar[r] & \resource{preprocessed\\source code} \\ & \variable{ECSINCLUDE} \ar[u]}
\seecpp
}

\providecommand{\cppprint}{
\toolsection{cppprint} is a pretty printer for the \cpp{} programming language.
It reformats the source code of \cpp{} programs and writes it to the standard output stream.
\flowgraph{\resource{\cpp{}\\source code} \ar[r] & \toolbox{cppprint} \ar[r] & \resource{reformatted\\source code} \\ & \variable{ECSINCLUDE} \ar[u]}
\seecpp
}

\providecommand{\cppcheck}{
\toolsection{cppcheck} is a syntactic and semantic checker for the \cpp{} programming language.
It just performs syntactic and semantic checks on \cpp{} programs and writes its diagnostic messages to the standard error stream.
\flowgraph{\resource{\cpp{}\\source code} \ar[r] & \toolbox{cppcheck} \ar[r] & \resource{diagnostic\\messages} \\ & \variable{ECSINCLUDE} \ar[u]}
\seecpp
}

\providecommand{\cppdump}{
\toolsection{cppdump} is a serializer for the \cpp{} programming language.
It dumps the complete internal representation of programs written in \cpp{} into an XML document.
\debuggingtool
\flowgraph{\resource{\cpp{}\\source code} \ar[r] & \toolbox{cppdump} \ar[r] & \resource{internal\\representation} \\ & \variable{ECSINCLUDE} \ar[u]}
\seecpp
}

\providecommand{\cpprun}{
\toolsection{cpprun} is an interpreter for the \cpp{} programming language.
It processes and executes programs written in \cpp{}.
The macro \texttt{\_\_run\_\_} is predefined in order to enable programmers to identify this tool while interpreting.
\flowgraph{\resource{\cpp{}\\source code} \ar[r] & \toolbox{cpprun} \ar@/u/[r] & \resource{input/\\output} \ar@/d/[l] \\ & \variable{ECSINCLUDE} \ar[u]}
\seecpp
}

\providecommand{\cppdoc}{
\toolsection{cppdoc} is a generic documentation generator for the \cpp{} programming language.
It processes several \cpp{} source files and assembles all information therein into a generic documentation.
\debuggingtool
\flowgraph{\resource{\cpp{}\\source code} \ar[r] & \toolbox{cppdoc} \ar[r] & \resource{generic\\documentation} \\ & \variable{ECSINCLUDE} \ar[u]}
\seecpp\seedocumentation
}

\providecommand{\cpphtml}{
\toolsection{cpphtml} is an HTML documentation generator for the \cpp{} programming language.
It processes several \cpp{} source files and assembles all information therein into an HTML document.
\flowgraph{\resource{\cpp{}\\source code} \ar[r] & \toolbox{cpphtml} \ar[r] & \resource{HTML\\document} \\ & \variable{ECSINCLUDE} \ar[u]}
\seecpp\seedocumentation
}

\providecommand{\cpplatex}{
\toolsection{cpplatex} is a Latex documentation generator for the \cpp{} programming language.
It processes several \cpp{} source files and assembles all information therein into a Latex document.
\flowgraph{\resource{\cpp{}\\source code} \ar[r] & \toolbox{cpplatex} \ar[r] & \resource{Latex\\document} \\ & \variable{ECSINCLUDE} \ar[u]}
\seecpp\seedocumentation
}

\providecommand{\cppcode}{
\toolsection{cppcode} is an intermediate code generator for the \cpp{} programming language.
It generates intermediate code from programs written in \cpp{} and stores it in corresponding assembly files.
The macro \texttt{\_\_code\_\_} is predefined in order to enable programmers to identify this tool while generating intermediate code.
Programs generated with this tool require additional runtime support that is stored in the \file{cpp\-code\-run} library file.
\debuggingtool
\flowgraph{\resource{\cpp{}\\source code} \ar[r] & \toolbox{cppcode} \ar[r] & \resource{intermediate\\code} \\ & \variable{ECSINCLUDE} \ar[u]}
\seecpp\seeassembly\seecode
}

\providecommand{\cppamda}{
\toolsection{cppamd16} is a compiler for the \cpp{} programming language targeting the AMD64 hardware architecture.
It generates machine code for AMD64 processors from programs written in \cpp{} and stores it in corresponding object files.
The compiler generates machine code for the 16-bit operating mode defined by the AMD64 architecture.
For debugging purposes, it also creates a debugging information file as well as an assembly file containing a listing of the generated machine code.
The macro \texttt{\_\_amd16\_\_} is predefined in order to enable programmers to identify this tool and its target architecture while compiling.
Programs generated with this compiler require additional runtime support that is stored in the \file{cpp\-amd16\-run} library file.
\flowgraph{\resource{\cpp{}\\source code} \ar[r] & \toolbox{cppamd16} \ar[r] \ar[d] \ar[rd] & \resource{object file} \\ \variable{ECSINCLUDE} \ar[ru] & \resource{debugging\\information} & \resource{assembly\\listing}}
\seecpp\seeassembly\seeamd\seeobject\seedebugging
}

\providecommand{\cppamdb}{
\toolsection{cppamd32} is a compiler for the \cpp{} programming language targeting the AMD64 hardware architecture.
It generates machine code for AMD64 processors from programs written in \cpp{} and stores it in corresponding object files.
The compiler generates machine code for the 32-bit operating mode defined by the AMD64 architecture.
For debugging purposes, it also creates a debugging information file as well as an assembly file containing a listing of the generated machine code.
The macro \texttt{\_\_amd32\_\_} is predefined in order to enable programmers to identify this tool and its target architecture while compiling.
Programs generated with this compiler require additional runtime support that is stored in the \file{cpp\-amd32\-run} library file.
\flowgraph{\resource{\cpp{}\\source code} \ar[r] & \toolbox{cppamd32} \ar[r] \ar[d] \ar[rd] & \resource{object file} \\ \variable{ECSINCLUDE} \ar[ru] & \resource{debugging\\information} & \resource{assembly\\listing}}
\seecpp\seeassembly\seeamd\seeobject\seedebugging
}

\providecommand{\cppamdc}{
\toolsection{cppamd64} is a compiler for the \cpp{} programming language targeting the AMD64 hardware architecture.
It generates machine code for AMD64 processors from programs written in \cpp{} and stores it in corresponding object files.
The compiler generates machine code for the 64-bit operating mode defined by the AMD64 architecture.
For debugging purposes, it also creates a debugging information file as well as an assembly file containing a listing of the generated machine code.
The macro \texttt{\_\_amd64\_\_} is predefined in order to enable programmers to identify this tool and its target architecture while compiling.
Programs generated with this compiler require additional runtime support that is stored in the \file{cpp\-amd64\-run} library file.
\flowgraph{\resource{\cpp{}\\source code} \ar[r] & \toolbox{cppamd64} \ar[r] \ar[d] \ar[rd] & \resource{object file} \\ \variable{ECSINCLUDE} \ar[ru] & \resource{debugging\\information} & \resource{assembly\\listing}}
\seecpp\seeassembly\seeamd\seeobject\seedebugging
}

\providecommand{\cpparma}{
\toolsection{cpparma32} is a compiler for the \cpp{} programming language targeting the ARM hardware architecture.
It generates machine code for ARM processors executing A32 instructions from programs written in \cpp{} and stores it in corresponding object files.
For debugging purposes, it also creates a debugging information file as well as an assembly file containing a listing of the generated machine code.
The macro \texttt{\_\_arma32\_\_} is predefined in order to enable programmers to identify this tool and its target architecture while compiling.
Programs generated with this compiler require additional runtime support that is stored in the \file{cpp\-arma32\-run} library file.
\flowgraph{\resource{\cpp{}\\source code} \ar[r] & \toolbox{cpparma32} \ar[r] \ar[d] \ar[rd] & \resource{object file} \\ \variable{ECSINCLUDE} \ar[ru] & \resource{debugging\\information} & \resource{assembly\\listing}}
\seecpp\seeassembly\seearm\seeobject\seedebugging
}

\providecommand{\cpparmb}{
\toolsection{cpparma64} is a compiler for the \cpp{} programming language targeting the ARM hardware architecture.
It generates machine code for ARM processors executing A64 instructions from programs written in \cpp{} and stores it in corresponding object files.
For debugging purposes, it also creates a debugging information file as well as an assembly file containing a listing of the generated machine code.
The macro \texttt{\_\_arma64\_\_} is predefined in order to enable programmers to identify this tool and its target architecture while compiling.
Programs generated with this compiler require additional runtime support that is stored in the \file{cpp\-arma64\-run} library file.
\flowgraph{\resource{\cpp{}\\source code} \ar[r] & \toolbox{cpparma64} \ar[r] \ar[d] \ar[rd] & \resource{object file} \\ \variable{ECSINCLUDE} \ar[ru] & \resource{debugging\\information} & \resource{assembly\\listing}}
\seecpp\seeassembly\seearm\seeobject\seedebugging
}

\providecommand{\cpparmc}{
\toolsection{cpparmt32} is a compiler for the \cpp{} programming language targeting the ARM hardware architecture.
It generates machine code for ARM processors without floating-point extension executing T32 instructions from programs written in \cpp{} and stores it in corresponding object files.
For debugging purposes, it also creates a debugging information file as well as an assembly file containing a listing of the generated machine code.
The macro \texttt{\_\_armt32\_\_} is predefined in order to enable programmers to identify this tool and its target architecture while compiling.
Programs generated with this compiler require additional runtime support that is stored in the \file{cpp\-armt32\-run} library file.
\flowgraph{\resource{\cpp{}\\source code} \ar[r] & \toolbox{cpparmt32} \ar[r] \ar[d] \ar[rd] & \resource{object file} \\ \variable{ECSINCLUDE} \ar[ru] & \resource{debugging\\information} & \resource{assembly\\listing}}
\seecpp\seeassembly\seearm\seeobject\seedebugging
}

\providecommand{\cpparmcfpe}{
\toolsection{cpparmt32fpe} is a compiler for the \cpp{} programming language targeting the ARM hardware architecture.
It generates machine code for ARM processors with floating-point extension executing T32 instructions from programs written in \cpp{} and stores it in corresponding object files.
For debugging purposes, it also creates a debugging information file as well as an assembly file containing a listing of the generated machine code.
The macro \texttt{\_\_armt32fpe\_\_} is predefined in order to enable programmers to identify this tool and its target architecture while compiling.
Programs generated with this compiler require additional runtime support that is stored in the \file{cpp\-armt32\-fpe\-run} library file.
\flowgraph{\resource{\cpp{}\\source code} \ar[r] & \toolbox{cpparmt32fpe} \ar[r] \ar[d] \ar[rd] & \resource{object file} \\ \variable{ECSINCLUDE} \ar[ru] & \resource{debugging\\information} & \resource{assembly\\listing}}
\seecpp\seeassembly\seearm\seeobject\seedebugging
}

\providecommand{\cppavr}{
\toolsection{cppavr} is a compiler for the \cpp{} programming language targeting the AVR hardware architecture.
It generates machine code for AVR processors from programs written in \cpp{} and stores it in corresponding object files.
For debugging purposes, it also creates a debugging information file as well as an assembly file containing a listing of the generated machine code.
The macro \texttt{\_\_avr\_\_} is predefined in order to enable programmers to identify this tool and its target architecture while compiling.
Programs generated with this compiler require additional runtime support that is stored in the \file{cpp\-avr\-run} library file.
\flowgraph{\resource{\cpp{}\\source code} \ar[r] & \toolbox{cppavr} \ar[r] \ar[d] \ar[rd] & \resource{object file} \\ \variable{ECSINCLUDE} \ar[ru] & \resource{debugging\\information} & \resource{assembly\\listing}}
\seecpp\seeassembly\seeavr\seeobject\seedebugging
}

\providecommand{\cppavrtt}{
\toolsection{cppavr32} is a compiler for the \cpp{} programming language targeting the AVR32 hardware architecture.
It generates machine code for AVR32 processors from programs written in \cpp{} and stores it in corresponding object files.
For debugging purposes, it also creates a debugging information file as well as an assembly file containing a listing of the generated machine code.
The macro \texttt{\_\_avr32\_\_} is predefined in order to enable programmers to identify this tool and its target architecture while compiling.
Programs generated with this compiler require additional runtime support that is stored in the \file{cpp\-avr32\-run} library file.
\flowgraph{\resource{\cpp{}\\source code} \ar[r] & \toolbox{cppavr32} \ar[r] \ar[d] \ar[rd] & \resource{object file} \\ \variable{ECSINCLUDE} \ar[ru] & \resource{debugging\\information} & \resource{assembly\\listing}}
\seecpp\seeassembly\seeavrtt\seeobject\seedebugging
}

\providecommand{\cppmabk}{
\toolsection{cppm68k} is a compiler for the \cpp{} programming language targeting the M68000 hardware architecture.
It generates machine code for M68000 processors from programs written in \cpp{} and stores it in corresponding object files.
For debugging purposes, it also creates a debugging information file as well as an assembly file containing a listing of the generated machine code.
The macro \texttt{\_\_m68k\_\_} is predefined in order to enable programmers to identify this tool and its target architecture while compiling.
Programs generated with this compiler require additional runtime support that is stored in the \file{cpp\-m68k\-run} library file.
\flowgraph{\resource{\cpp{}\\source code} \ar[r] & \toolbox{cppm68k} \ar[r] \ar[d] \ar[rd] & \resource{object file} \\ \variable{ECSINCLUDE} \ar[ru] & \resource{debugging\\information} & \resource{assembly\\listing}}
\seecpp\seeassembly\seemabk\seeobject\seedebugging
}

\providecommand{\cppmibl}{
\toolsection{cppmibl} is a compiler for the \cpp{} programming language targeting the MicroBlaze hardware architecture.
It generates machine code for MicroBlaze processors from programs written in \cpp{} and stores it in corresponding object files.
For debugging purposes, it also creates a debugging information file as well as an assembly file containing a listing of the generated machine code.
The macro \texttt{\_\_mibl\_\_} is predefined in order to enable programmers to identify this tool and its target architecture while compiling.
Programs generated with this compiler require additional runtime support that is stored in the \file{cpp\-mibl\-run} library file.
\flowgraph{\resource{\cpp{}\\source code} \ar[r] & \toolbox{cppmibl} \ar[r] \ar[d] \ar[rd] & \resource{object file} \\ \variable{ECSINCLUDE} \ar[ru] & \resource{debugging\\information} & \resource{assembly\\listing}}
\seecpp\seeassembly\seemibl\seeobject\seedebugging
}

\providecommand{\cppmipsa}{
\toolsection{cppmips32} is a compiler for the \cpp{} programming language targeting the MIPS32 hardware architecture.
It generates machine code for MIPS32 processors from programs written in \cpp{} and stores it in corresponding object files.
For debugging purposes, it also creates a debugging information file as well as an assembly file containing a listing of the generated machine code.
The macro \texttt{\_\_mips32\_\_} is predefined in order to enable programmers to identify this tool and its target architecture while compiling.
Programs generated with this compiler require additional runtime support that is stored in the \file{cpp\-mips32\-run} library file.
\flowgraph{\resource{\cpp{}\\source code} \ar[r] & \toolbox{cppmips32} \ar[r] \ar[d] \ar[rd] & \resource{object file} \\ \variable{ECSINCLUDE} \ar[ru] & \resource{debugging\\information} & \resource{assembly\\listing}}
\seecpp\seeassembly\seemips\seeobject\seedebugging
}

\providecommand{\cppmipsb}{
\toolsection{cppmips64} is a compiler for the \cpp{} programming language targeting the MIPS64 hardware architecture.
It generates machine code for MIPS64 processors from programs written in \cpp{} and stores it in corresponding object files.
For debugging purposes, it also creates a debugging information file as well as an assembly file containing a listing of the generated machine code.
The macro \texttt{\_\_mips64\_\_} is predefined in order to enable programmers to identify this tool and its target architecture while compiling.
Programs generated with this compiler require additional runtime support that is stored in the \file{cpp\-mips64\-run} library file.
\flowgraph{\resource{\cpp{}\\source code} \ar[r] & \toolbox{cppmips64} \ar[r] \ar[d] \ar[rd] & \resource{object file} \\ \variable{ECSINCLUDE} \ar[ru] & \resource{debugging\\information} & \resource{assembly\\listing}}
\seecpp\seeassembly\seemips\seeobject\seedebugging
}

\providecommand{\cppmmix}{
\toolsection{cppmmix} is a compiler for the \cpp{} programming language targeting the MMIX hardware architecture.
It generates machine code for MMIX processors from programs written in \cpp{} and stores it in corresponding object files.
For debugging purposes, it also creates a debugging information file as well as an assembly file containing a listing of the generated machine code.
The macro \texttt{\_\_mmix\_\_} is predefined in order to enable programmers to identify this tool and its target architecture while compiling.
Programs generated with this compiler require additional runtime support that is stored in the \file{cpp\-mmix\-run} library file.
\flowgraph{\resource{\cpp{}\\source code} \ar[r] & \toolbox{cppmmix} \ar[r] \ar[d] \ar[rd] & \resource{object file} \\ \variable{ECSINCLUDE} \ar[ru] & \resource{debugging\\information} & \resource{assembly\\listing}}
\seecpp\seeassembly\seemmix\seeobject\seedebugging
}

\providecommand{\cpporok}{
\toolsection{cppor1k} is a compiler for the \cpp{} programming language targeting the OpenRISC 1000 hardware architecture.
It generates machine code for OpenRISC 1000 processors from programs written in \cpp{} and stores it in corresponding object files.
For debugging purposes, it also creates a debugging information file as well as an assembly file containing a listing of the generated machine code.
The macro \texttt{\_\_or1k\_\_} is predefined in order to enable programmers to identify this tool and its target architecture while compiling.
Programs generated with this compiler require additional runtime support that is stored in the \file{cpp\-or1k\-run} library file.
\flowgraph{\resource{\cpp{}\\source code} \ar[r] & \toolbox{cppor1k} \ar[r] \ar[d] \ar[rd] & \resource{object file} \\ \variable{ECSINCLUDE} \ar[ru] & \resource{debugging\\information} & \resource{assembly\\listing}}
\seecpp\seeassembly\seeorok\seeobject\seedebugging
}

\providecommand{\cppppca}{
\toolsection{cppppc32} is a compiler for the \cpp{} programming language targeting the PowerPC hardware architecture.
It generates machine code for PowerPC processors from programs written in \cpp{} and stores it in corresponding object files.
The compiler generates machine code for the 32-bit operating mode defined by the PowerPC architecture.
For debugging purposes, it also creates a debugging information file as well as an assembly file containing a listing of the generated machine code.
The macro \texttt{\_\_ppc32\_\_} is predefined in order to enable programmers to identify this tool and its target architecture while compiling.
Programs generated with this compiler require additional runtime support that is stored in the \file{cpp\-ppc32\-run} library file.
\flowgraph{\resource{\cpp{}\\source code} \ar[r] & \toolbox{cppppc32} \ar[r] \ar[d] \ar[rd] & \resource{object file} \\ \variable{ECSINCLUDE} \ar[ru] & \resource{debugging\\information} & \resource{assembly\\listing}}
\seecpp\seeassembly\seeppc\seeobject\seedebugging
}

\providecommand{\cppppcb}{
\toolsection{cppppc64} is a compiler for the \cpp{} programming language targeting the PowerPC hardware architecture.
It generates machine code for PowerPC processors from programs written in \cpp{} and stores it in corresponding object files.
The compiler generates machine code for the 64-bit operating mode defined by the PowerPC architecture.
For debugging purposes, it also creates a debugging information file as well as an assembly file containing a listing of the generated machine code.
The macro \texttt{\_\_ppc64\_\_} is predefined in order to enable programmers to identify this tool and its target architecture while compiling.
Programs generated with this compiler require additional runtime support that is stored in the \file{cpp\-ppc64\-run} library file.
\flowgraph{\resource{\cpp{}\\source code} \ar[r] & \toolbox{cppppc64} \ar[r] \ar[d] \ar[rd] & \resource{object file} \\ \variable{ECSINCLUDE} \ar[ru] & \resource{debugging\\information} & \resource{assembly\\listing}}
\seecpp\seeassembly\seeppc\seeobject\seedebugging
}

\providecommand{\cpprisc}{
\toolsection{cpprisc} is a compiler for the \cpp{} programming language targeting the RISC hardware architecture.
It generates machine code for RISC processors from programs written in \cpp{} and stores it in corresponding object files.
For debugging purposes, it also creates a debugging information file as well as an assembly file containing a listing of the generated machine code.
The macro \texttt{\_\_risc\_\_} is predefined in order to enable programmers to identify this tool and its target architecture while compiling.
Programs generated with this compiler require additional runtime support that is stored in the \file{cpp\-risc\-run} library file.
\flowgraph{\resource{\cpp{}\\source code} \ar[r] & \toolbox{cpprisc} \ar[r] \ar[d] \ar[rd] & \resource{object file} \\ \variable{ECSINCLUDE} \ar[ru] & \resource{debugging\\information} & \resource{assembly\\listing}}
\seecpp\seeassembly\seerisc\seeobject\seedebugging
}

\providecommand{\cppwasm}{
\toolsection{cppwasm} is a compiler for the \cpp{} programming language targeting the WebAssembly architecture.
It generates machine code for WebAssembly targets from programs written in \cpp{} and stores it in corresponding object files.
For debugging purposes, it also creates a debugging information file as well as an assembly file containing a listing of the generated machine code.
The macro \texttt{\_\_wasm\_\_} is predefined in order to enable programmers to identify this tool and its target architecture while compiling.
Programs generated with this compiler require additional runtime support that is stored in the \file{cpp\-wasm\-run} library file.
\flowgraph{\resource{\cpp{}\\source code} \ar[r] & \toolbox{cppwasm} \ar[r] \ar[d] \ar[rd] & \resource{object file} \\ \variable{ECSINCLUDE} \ar[ru] & \resource{debugging\\information} & \resource{assembly\\listing}}
\seecpp\seeassembly\seewasm\seeobject\seedebugging
}

% FALSE tools

\providecommand{\falprint}{
\toolsection{falprint} is a pretty printer for the FALSE programming language.
It reformats the source code of FALSE programs and writes it to the standard output stream.
\flowgraph{\resource{FALSE\\source code} \ar[r] & \toolbox{falprint} \ar[r] & \resource{reformatted\\source code}}
\seefalse
}

\providecommand{\falcheck}{
\toolsection{falcheck} is a syntactic and semantic checker for the FALSE programming language.
It just performs syntactic and semantic checks on FALSE programs and writes its diagnostic messages to the standard error stream.
\flowgraph{\resource{FALSE\\source code} \ar[r] & \toolbox{falcheck} \ar[r] & \resource{diagnostic\\messages}}
\seefalse
}

\providecommand{\faldump}{
\toolsection{faldump} is a serializer for the FALSE programming language.
It dumps the complete internal representation of programs written in FALSE into an XML document.
\debuggingtool
\flowgraph{\resource{FALSE\\source code} \ar[r] & \toolbox{faldump} \ar[r] & \resource{internal\\representation}}
\seefalse
}

\providecommand{\falrun}{
\toolsection{falrun} is an interpreter for the FALSE programming language.
It processes and executes programs written in FALSE\@.
\flowgraph{\resource{FALSE\\source code} \ar[r] & \toolbox{falrun} \ar@/u/[r] & \resource{input/\\output} \ar@/d/[l]}
\seefalse
}

\providecommand{\falcpp}{
\toolsection{falcpp} is a transpiler for the FALSE programming language.
It translates programs written in FALSE into \cpp{} programs and stores them in corresponding source files.
\flowgraph{\resource{FALSE\\source code} \ar[r] & \toolbox{falcpp} \ar[r] & \resource{\cpp{}\\source file}}
\seefalse\seecpp
}

\providecommand{\falcode}{
\toolsection{falcode} is an intermediate code generator for the FALSE programming language.
It generates intermediate code from programs written in FALSE and stores it in corresponding assembly files.
\debuggingtool
\flowgraph{\resource{FALSE\\source code} \ar[r] & \toolbox{falcode} \ar[r] & \resource{intermediate\\code}}
\seefalse\seeassembly\seecode
}

\providecommand{\falamda}{
\toolsection{falamd16} is a compiler for the FALSE programming language targeting the AMD64 hardware architecture.
It generates machine code for AMD64 processors from programs written in FALSE and stores it in corresponding object files.
The compiler generates machine code for the 16-bit operating mode defined by the AMD64 architecture.
\flowgraph{\resource{FALSE\\source code} \ar[r] & \toolbox{falamd16} \ar[r] & \resource{object file}}
\seefalse\seeamd\seeobject
}

\providecommand{\falamdb}{
\toolsection{falamd32} is a compiler for the FALSE programming language targeting the AMD64 hardware architecture.
It generates machine code for AMD64 processors from programs written in FALSE and stores it in corresponding object files.
The compiler generates machine code for the 32-bit operating mode defined by the AMD64 architecture.
\flowgraph{\resource{FALSE\\source code} \ar[r] & \toolbox{falamd32} \ar[r] & \resource{object file}}
\seefalse\seeamd\seeobject
}

\providecommand{\falamdc}{
\toolsection{falamd64} is a compiler for the FALSE programming language targeting the AMD64 hardware architecture.
It generates machine code for AMD64 processors from programs written in FALSE and stores it in corresponding object files.
The compiler generates machine code for the 64-bit operating mode defined by the AMD64 architecture.
\flowgraph{\resource{FALSE\\source code} \ar[r] & \toolbox{falamd64} \ar[r] & \resource{object file}}
\seefalse\seeamd\seeobject
}

\providecommand{\falarma}{
\toolsection{falarma32} is a compiler for the FALSE programming language targeting the ARM hardware architecture.
It generates machine code for ARM processors executing A32 instructions from programs written in FALSE and stores it in corresponding object files.
\flowgraph{\resource{FALSE\\source code} \ar[r] & \toolbox{falarma32} \ar[r] & \resource{object file}}
\seefalse\seearm\seeobject
}

\providecommand{\falarmb}{
\toolsection{falarma64} is a compiler for the FALSE programming language targeting the ARM hardware architecture.
It generates machine code for ARM processors executing A64 instructions from programs written in FALSE and stores it in corresponding object files.
\flowgraph{\resource{FALSE\\source code} \ar[r] & \toolbox{falarma64} \ar[r] & \resource{object file}}
\seefalse\seearm\seeobject
}

\providecommand{\falarmc}{
\toolsection{falarmt32} is a compiler for the FALSE programming language targeting the ARM hardware architecture.
It generates machine code for ARM processors without floating-point extension executing T32 instructions from programs written in FALSE and stores it in corresponding object files.
\flowgraph{\resource{FALSE\\source code} \ar[r] & \toolbox{falarmt32} \ar[r] & \resource{object file}}
\seefalse\seearm\seeobject
}

\providecommand{\falarmcfpe}{
\toolsection{falarmt32fpe} is a compiler for the FALSE programming language targeting the ARM hardware architecture.
It generates machine code for ARM processors with floating-point extension executing T32 instructions from programs written in FALSE and stores it in corresponding object files.
\flowgraph{\resource{FALSE\\source code} \ar[r] & \toolbox{falarmt32fpe} \ar[r] & \resource{object file}}
\seefalse\seearm\seeobject
}

\providecommand{\falavr}{
\toolsection{falavr} is a compiler for the FALSE programming language targeting the AVR hardware architecture.
It generates machine code for AVR processors from programs written in FALSE and stores it in corresponding object files.
\flowgraph{\resource{FALSE\\source code} \ar[r] & \toolbox{falavr} \ar[r] & \resource{object file}}
\seefalse\seeavr\seeobject
}

\providecommand{\falavrtt}{
\toolsection{falavr32} is a compiler for the FALSE programming language targeting the AVR32 hardware architecture.
It generates machine code for AVR32 processors from programs written in FALSE and stores it in corresponding object files.
\flowgraph{\resource{FALSE\\source code} \ar[r] & \toolbox{falavr32} \ar[r] & \resource{object file}}
\seefalse\seeavrtt\seeobject
}

\providecommand{\falmabk}{
\toolsection{falm68k} is a compiler for the FALSE programming language targeting the M68000 hardware architecture.
It generates machine code for M68000 processors from programs written in FALSE and stores it in corresponding object files.
\flowgraph{\resource{FALSE\\source code} \ar[r] & \toolbox{falm68k} \ar[r] & \resource{object file}}
\seefalse\seemabk\seeobject
}

\providecommand{\falmibl}{
\toolsection{falmibl} is a compiler for the FALSE programming language targeting the MicroBlaze hardware architecture.
It generates machine code for MicroBlaze processors from programs written in FALSE and stores it in corresponding object files.
\flowgraph{\resource{FALSE\\source code} \ar[r] & \toolbox{falmibl} \ar[r] & \resource{object file}}
\seefalse\seemibl\seeobject
}

\providecommand{\falmipsa}{
\toolsection{falmips32} is a compiler for the FALSE programming language targeting the MIPS32 hardware architecture.
It generates machine code for MIPS32 processors from programs written in FALSE and stores it in corresponding object files.
\flowgraph{\resource{FALSE\\source code} \ar[r] & \toolbox{falmips32} \ar[r] & \resource{object file}}
\seefalse\seemips\seeobject
}

\providecommand{\falmipsb}{
\toolsection{falmips64} is a compiler for the FALSE programming language targeting the MIPS64 hardware architecture.
It generates machine code for MIPS64 processors from programs written in FALSE and stores it in corresponding object files.
\flowgraph{\resource{FALSE\\source code} \ar[r] & \toolbox{falmips64} \ar[r] & \resource{object file}}
\seefalse\seemips\seeobject
}

\providecommand{\falmmix}{
\toolsection{falmmix} is a compiler for the FALSE programming language targeting the MMIX hardware architecture.
It generates machine code for MMIX processors from programs written in FALSE and stores it in corresponding object files.
\flowgraph{\resource{FALSE\\source code} \ar[r] & \toolbox{falmmix} \ar[r] & \resource{object file}}
\seefalse\seemmix\seeobject
}

\providecommand{\falorok}{
\toolsection{falor1k} is a compiler for the FALSE programming language targeting the OpenRISC 1000 hardware architecture.
It generates machine code for OpenRISC 1000 processors from programs written in FALSE and stores it in corresponding object files.
\flowgraph{\resource{FALSE\\source code} \ar[r] & \toolbox{falor1k} \ar[r] & \resource{object file}}
\seefalse\seeorok\seeobject
}

\providecommand{\falppca}{
\toolsection{falppc32} is a compiler for the FALSE programming language targeting the PowerPC hardware architecture.
It generates machine code for PowerPC processors from programs written in FALSE and stores it in corresponding object files.
The compiler generates machine code for the 32-bit operating mode defined by the PowerPC architecture.
\flowgraph{\resource{FALSE\\source code} \ar[r] & \toolbox{falppc32} \ar[r] & \resource{object file}}
\seefalse\seeppc\seeobject
}

\providecommand{\falppcb}{
\toolsection{falppc64} is a compiler for the FALSE programming language targeting the PowerPC hardware architecture.
It generates machine code for PowerPC processors from programs written in FALSE and stores it in corresponding object files.
The compiler generates machine code for the 64-bit operating mode defined by the PowerPC architecture.
\flowgraph{\resource{FALSE\\source code} \ar[r] & \toolbox{falppc64} \ar[r] & \resource{object file}}
\seefalse\seeppc\seeobject
}

\providecommand{\falrisc}{
\toolsection{falrisc} is a compiler for the FALSE programming language targeting the RISC hardware architecture.
It generates machine code for RISC processors from programs written in FALSE and stores it in corresponding object files.
\flowgraph{\resource{FALSE\\source code} \ar[r] & \toolbox{falrisc} \ar[r] & \resource{object file}}
\seefalse\seerisc\seeobject
}

\providecommand{\falwasm}{
\toolsection{falwasm} is a compiler for the FALSE programming language targeting the WebAssembly architecture.
It generates machine code for WebAssembly targets from programs written in FALSE and stores it in corresponding object files.
\flowgraph{\resource{FALSE\\source code} \ar[r] & \toolbox{falwasm} \ar[r] & \resource{object file}}
\seefalse\seewasm\seeobject
}

% Oberon tools

\providecommand{\obprint}{
\toolsection{obprint} is a pretty printer for the Oberon programming language.
It reformats the source code of Oberon modules and writes it to the standard output stream.
\flowgraph{\resource{Oberon\\source code} \ar[r] & \toolbox{obprint} \ar[r] & \resource{reformatted\\source code}}
\seeoberon
}

\providecommand{\obcheck}{
\toolsection{obcheck} is a syntactic and semantic checker for the Oberon programming language.
It just performs syntactic and semantic checks on Oberon modules and writes its diagnostic messages to the standard error stream.
In addition, it stores the interface of each module in a symbol file which is required when other modules import the module.
\flowgraph{\resource{Oberon\\source code} \ar[r] & \toolbox{obcheck} \ar[r] \ar@/l/[d] & \resource{diagnostic\\messages} \\ \variable{ECSIMPORT} \ar[ru] & \resource{symbol\\files} \ar@/r/[u]}
\seeoberon
}

\providecommand{\obdump}{
\toolsection{obdump} is a serializer for the Oberon programming language.
It dumps the complete internal representation of modules written in Oberon into an XML document.
\debuggingtool
\flowgraph{\resource{Oberon\\source code} \ar[r] & \toolbox{obdump} \ar[r] \ar@/l/[d] & \resource{internal\\representation} \\ \variable{ECSIMPORT} \ar[ru] & \resource{symbol\\files} \ar@/r/[u]}
\seeoberon
}

\providecommand{\obrun}{
\toolsection{obrun} is an interpreter for the Oberon programming language.
It processes and executes modules written in Oberon.
This tool does neither generate nor process symbol files while interpreting modules.
If a module is imported by another one, its filename has to be named before the other one in the list of command-line arguments.
\flowgraph{\resource{Oberon\\source code} \ar[r] & \toolbox{obrun} \ar@/u/[r] & \resource{input/\\output} \ar@/d/[l]}
\seeoberon
}

\providecommand{\obcpp}{
\toolsection{obcpp} is a transpiler for the Oberon programming language.
It translates programs written in Oberon into \cpp{} programs and stores them in corresponding source and header files.
In addition, it stores the interface of each module in a symbol file which is required when other modules import the module.
The same interface is provided by the generated header file which can be used in other parts of the \cpp{} program.
\flowgraph{\resource{Oberon\\source code} \ar[r] & \toolbox{obcpp} \ar[r] \ar@/l/[d] \ar[rd] & \resource{\cpp{}\\source file} \\ \variable{ECSIMPORT} \ar[ru] & \resource{symbol\\files} \ar@/r/[u] & \resource{\cpp{}\\header file}}
\seeoberon\seecpp
}

\providecommand{\obdoc}{
\toolsection{obdoc} is a generic documentation generator for the Oberon programming language.
It processes several Oberon modules and assembles all information therein into a generic documentation.
In addition, it stores the interface of each module in a symbol file which is required when other modules import the module.
\debuggingtool
\flowgraph{\resource{Oberon\\source code} \ar[r] & \toolbox{obdoc} \ar[r] \ar@/l/[d] & \resource{generic\\documentation} \\ \variable{ECSIMPORT} \ar[ru] & \resource{symbol\\files} \ar@/r/[u]}
\seeoberon\seedocumentation
}

\providecommand{\obhtml}{
\toolsection{obhtml} is an HTML documentation generator for the Oberon programming language.
It processes several Oberon modules and assembles all information therein into an HTML document.
In addition, it stores the interface of each module in a symbol file which is required when other modules import the module.
\flowgraph{\resource{Oberon\\source code} \ar[r] & \toolbox{obhtml} \ar[r] \ar@/l/[d] & \resource{HTML\\document} \\ \variable{ECSIMPORT} \ar[ru] & \resource{symbol\\files} \ar@/r/[u]}
\seeoberon\seedocumentation
}

\providecommand{\oblatex}{
\toolsection{oblatex} is a Latex documentation generator for the Oberon programming language.
It processes several Oberon modules and assembles all information therein into a Latex document.
In addition, it stores the interface of each module in a symbol file which is required when other modules import the module.
\flowgraph{\resource{Oberon\\source code} \ar[r] & \toolbox{oblatex} \ar[r] \ar@/l/[d] & \resource{Latex\\document} \\ \variable{ECSIMPORT} \ar[ru] & \resource{symbol\\files} \ar@/r/[u]}
\seeoberon\seedocumentation
}

\providecommand{\obcode}{
\toolsection{obcode} is an intermediate code generator for the Oberon programming language.
It generates intermediate code from modules written in Oberon and stores it in corresponding assembly files.
In addition, it stores the interface of each module in a symbol file which is required when other modules import the module.
Programs generated with this tool require additional runtime support that is stored in the \file{ob\-code\-run} library file.
\debuggingtool
\flowgraph{\resource{Oberon\\source code} \ar[r] & \toolbox{obcode} \ar[r] \ar@/l/[d] & \resource{intermediate\\code} \\ \variable{ECSIMPORT} \ar[ru] & \resource{symbol\\files} \ar@/r/[u]}
\seeoberon\seeassembly\seecode
}

\providecommand{\obamda}{
\toolsection{obamd16} is a compiler for the Oberon programming language targeting the AMD64 hardware architecture.
It generates machine code for AMD64 processors from modules written in Oberon and stores it in corresponding object files.
The compiler generates machine code for the 16-bit operating mode defined by the AMD64 architecture.
For debugging purposes, it also creates a debugging information file as well as an assembly file containing a listing of the generated machine code.
In addition, it stores the interface of each module in a symbol file which is required when other modules import the module.
Programs generated with this compiler require additional runtime support that is stored in the \file{ob\-amd16\-run} library file.
\flowgraph{\resource{Oberon\\source code} \ar[r] & \toolbox{obamd16} \ar[r] \ar@/l/[d] \ar[rd] & \resource{object file} \\ \variable{ECSIMPORT} \ar[ru] & \resource{symbol\\files} \ar@/r/[u] & \resource{debugging\\information}}
\seeoberon\seeassembly\seeamd\seeobject\seedebugging
}

\providecommand{\obamdb}{
\toolsection{obamd32} is a compiler for the Oberon programming language targeting the AMD64 hardware architecture.
It generates machine code for AMD64 processors from modules written in Oberon and stores it in corresponding object files.
The compiler generates machine code for the 32-bit operating mode defined by the AMD64 architecture.
For debugging purposes, it also creates a debugging information file as well as an assembly file containing a listing of the generated machine code.
In addition, it stores the interface of each module in a symbol file which is required when other modules import the module.
Programs generated with this compiler require additional runtime support that is stored in the \file{ob\-amd32\-run} library file.
\flowgraph{\resource{Oberon\\source code} \ar[r] & \toolbox{obamd32} \ar[r] \ar@/l/[d] \ar[rd] & \resource{object file} \\ \variable{ECSIMPORT} \ar[ru] & \resource{symbol\\files} \ar@/r/[u] & \resource{debugging\\information}}
\seeoberon\seeassembly\seeamd\seeobject\seedebugging
}

\providecommand{\obamdc}{
\toolsection{obamd64} is a compiler for the Oberon programming language targeting the AMD64 hardware architecture.
It generates machine code for AMD64 processors from modules written in Oberon and stores it in corresponding object files.
The compiler generates machine code for the 64-bit operating mode defined by the AMD64 architecture.
For debugging purposes, it also creates a debugging information file as well as an assembly file containing a listing of the generated machine code.
In addition, it stores the interface of each module in a symbol file which is required when other modules import the module.
Programs generated with this compiler require additional runtime support that is stored in the \file{ob\-amd64\-run} library file.
\flowgraph{\resource{Oberon\\source code} \ar[r] & \toolbox{obamd64} \ar[r] \ar@/l/[d] \ar[rd] & \resource{object file} \\ \variable{ECSIMPORT} \ar[ru] & \resource{symbol\\files} \ar@/r/[u] & \resource{debugging\\information}}
\seeoberon\seeassembly\seeamd\seeobject\seedebugging
}

\providecommand{\obarma}{
\toolsection{obarma32} is a compiler for the Oberon programming language targeting the ARM hardware architecture.
It generates machine code for ARM processors executing A32 instructions from modules written in Oberon and stores it in corresponding object files.
For debugging purposes, it also creates a debugging information file as well as an assembly file containing a listing of the generated machine code.
In addition, it stores the interface of each module in a symbol file which is required when other modules import the module.
Programs generated with this compiler require additional runtime support that is stored in the \file{ob\-arma32\-run} library file.
\flowgraph{\resource{Oberon\\source code} \ar[r] & \toolbox{obarma32} \ar[r] \ar@/l/[d] \ar[rd] & \resource{object file} \\ \variable{ECSIMPORT} \ar[ru] & \resource{symbol\\files} \ar@/r/[u] & \resource{debugging\\information}}
\seeoberon\seeassembly\seearm\seeobject\seedebugging
}

\providecommand{\obarmb}{
\toolsection{obarma64} is a compiler for the Oberon programming language targeting the ARM hardware architecture.
It generates machine code for ARM processors executing A64 instructions from modules written in Oberon and stores it in corresponding object files.
For debugging purposes, it also creates a debugging information file as well as an assembly file containing a listing of the generated machine code.
In addition, it stores the interface of each module in a symbol file which is required when other modules import the module.
Programs generated with this compiler require additional runtime support that is stored in the \file{ob\-arma64\-run} library file.
\flowgraph{\resource{Oberon\\source code} \ar[r] & \toolbox{obarma64} \ar[r] \ar@/l/[d] \ar[rd] & \resource{object file} \\ \variable{ECSIMPORT} \ar[ru] & \resource{symbol\\files} \ar@/r/[u] & \resource{debugging\\information}}
\seeoberon\seeassembly\seearm\seeobject\seedebugging
}

\providecommand{\obarmc}{
\toolsection{obarmt32} is a compiler for the Oberon programming language targeting the ARM hardware architecture.
It generates machine code for ARM processors without floating-point extension executing T32 instructions from modules written in Oberon and stores it in corresponding object files.
For debugging purposes, it also creates a debugging information file as well as an assembly file containing a listing of the generated machine code.
In addition, it stores the interface of each module in a symbol file which is required when other modules import the module.
Programs generated with this compiler require additional runtime support that is stored in the \file{ob\-armt32\-run} library file.
\flowgraph{\resource{Oberon\\source code} \ar[r] & \toolbox{obarmt32} \ar[r] \ar@/l/[d] \ar[rd] & \resource{object file} \\ \variable{ECSIMPORT} \ar[ru] & \resource{symbol\\files} \ar@/r/[u] & \resource{debugging\\information}}
\seeoberon\seeassembly\seearm\seeobject\seedebugging
}

\providecommand{\obarmcfpe}{
\toolsection{obarmt32fpe} is a compiler for the Oberon programming language targeting the ARM hardware architecture.
It generates machine code for ARM processors with floating-point extension executing T32 instructions from modules written in Oberon and stores it in corresponding object files.
For debugging purposes, it also creates a debugging information file as well as an assembly file containing a listing of the generated machine code.
In addition, it stores the interface of each module in a symbol file which is required when other modules import the module.
Programs generated with this compiler require additional runtime support that is stored in the \file{ob\-armt32\-fpe\-run} library file.
\flowgraph{\resource{Oberon\\source code} \ar[r] & \toolbox{obarmt32fpe} \ar[r] \ar@/l/[d] \ar[rd] & \resource{object file} \\ \variable{ECSIMPORT} \ar[ru] & \resource{symbol\\files} \ar@/r/[u] & \resource{debugging\\information}}
\seeoberon\seeassembly\seearm\seeobject\seedebugging
}

\providecommand{\obavr}{
\toolsection{obavr} is a compiler for the Oberon programming language targeting the AVR hardware architecture.
It generates machine code for AVR processors from modules written in Oberon and stores it in corresponding object files.
For debugging purposes, it also creates a debugging information file as well as an assembly file containing a listing of the generated machine code.
In addition, it stores the interface of each module in a symbol file which is required when other modules import the module.
Programs generated with this compiler require additional runtime support that is stored in the \file{ob\-avr\-run} library file.
\flowgraph{\resource{Oberon\\source code} \ar[r] & \toolbox{obavr} \ar[r] \ar@/l/[d] \ar[rd] & \resource{object file} \\ \variable{ECSIMPORT} \ar[ru] & \resource{symbol\\files} \ar@/r/[u] & \resource{debugging\\information}}
\seeoberon\seeassembly\seeavr\seeobject\seedebugging
}

\providecommand{\obavrtt}{
\toolsection{obavr32} is a compiler for the Oberon programming language targeting the AVR32 hardware architecture.
It generates machine code for AVR32 processors from modules written in Oberon and stores it in corresponding object files.
For debugging purposes, it also creates a debugging information file as well as an assembly file containing a listing of the generated machine code.
In addition, it stores the interface of each module in a symbol file which is required when other modules import the module.
Programs generated with this compiler require additional runtime support that is stored in the \file{ob\-avr32\-run} library file.
\flowgraph{\resource{Oberon\\source code} \ar[r] & \toolbox{obavr32} \ar[r] \ar@/l/[d] \ar[rd] & \resource{object file} \\ \variable{ECSIMPORT} \ar[ru] & \resource{symbol\\files} \ar@/r/[u] & \resource{debugging\\information}}
\seeoberon\seeassembly\seeavrtt\seeobject\seedebugging
}

\providecommand{\obmabk}{
\toolsection{obm68k} is a compiler for the Oberon programming language targeting the M68000 hardware architecture.
It generates machine code for M68000 processors from modules written in Oberon and stores it in corresponding object files.
For debugging purposes, it also creates a debugging information file as well as an assembly file containing a listing of the generated machine code.
In addition, it stores the interface of each module in a symbol file which is required when other modules import the module.
Programs generated with this compiler require additional runtime support that is stored in the \file{ob\-m68k\-run} library file.
\flowgraph{\resource{Oberon\\source code} \ar[r] & \toolbox{obm68k} \ar[r] \ar@/l/[d] \ar[rd] & \resource{object file} \\ \variable{ECSIMPORT} \ar[ru] & \resource{symbol\\files} \ar@/r/[u] & \resource{debugging\\information}}
\seeoberon\seeassembly\seemabk\seeobject\seedebugging
}

\providecommand{\obmibl}{
\toolsection{obmibl} is a compiler for the Oberon programming language targeting the MicroBlaze hardware architecture.
It generates machine code for MicroBlaze processors from modules written in Oberon and stores it in corresponding object files.
For debugging purposes, it also creates a debugging information file as well as an assembly file containing a listing of the generated machine code.
In addition, it stores the interface of each module in a symbol file which is required when other modules import the module.
Programs generated with this compiler require additional runtime support that is stored in the \file{ob\-mibl\-run} library file.
\flowgraph{\resource{Oberon\\source code} \ar[r] & \toolbox{obmibl} \ar[r] \ar@/l/[d] \ar[rd] & \resource{object file} \\ \variable{ECSIMPORT} \ar[ru] & \resource{symbol\\files} \ar@/r/[u] & \resource{debugging\\information}}
\seeoberon\seeassembly\seemibl\seeobject\seedebugging
}

\providecommand{\obmipsa}{
\toolsection{obmips32} is a compiler for the Oberon programming language targeting the MIPS32 hardware architecture.
It generates machine code for MIPS32 processors from modules written in Oberon and stores it in corresponding object files.
For debugging purposes, it also creates a debugging information file as well as an assembly file containing a listing of the generated machine code.
In addition, it stores the interface of each module in a symbol file which is required when other modules import the module.
Programs generated with this compiler require additional runtime support that is stored in the \file{ob\-mips32\-run} library file.
\flowgraph{\resource{Oberon\\source code} \ar[r] & \toolbox{obmips32} \ar[r] \ar@/l/[d] \ar[rd] & \resource{object file} \\ \variable{ECSIMPORT} \ar[ru] & \resource{symbol\\files} \ar@/r/[u] & \resource{debugging\\information}}
\seeoberon\seeassembly\seemips\seeobject\seedebugging
}

\providecommand{\obmipsb}{
\toolsection{obmips64} is a compiler for the Oberon programming language targeting the MIPS64 hardware architecture.
It generates machine code for MIPS64 processors from modules written in Oberon and stores it in corresponding object files.
For debugging purposes, it also creates a debugging information file as well as an assembly file containing a listing of the generated machine code.
In addition, it stores the interface of each module in a symbol file which is required when other modules import the module.
Programs generated with this compiler require additional runtime support that is stored in the \file{ob\-mips64\-run} library file.
\flowgraph{\resource{Oberon\\source code} \ar[r] & \toolbox{obmips64} \ar[r] \ar@/l/[d] \ar[rd] & \resource{object file} \\ \variable{ECSIMPORT} \ar[ru] & \resource{symbol\\files} \ar@/r/[u] & \resource{debugging\\information}}
\seeoberon\seeassembly\seemips\seeobject\seedebugging
}

\providecommand{\obmmix}{
\toolsection{obmmix} is a compiler for the Oberon programming language targeting the MMIX hardware architecture.
It generates machine code for MMIX processors from modules written in Oberon and stores it in corresponding object files.
For debugging purposes, it also creates a debugging information file as well as an assembly file containing a listing of the generated machine code.
In addition, it stores the interface of each module in a symbol file which is required when other modules import the module.
Programs generated with this compiler require additional runtime support that is stored in the \file{ob\-mmix\-run} library file.
\flowgraph{\resource{Oberon\\source code} \ar[r] & \toolbox{obmmix} \ar[r] \ar@/l/[d] \ar[rd] & \resource{object file} \\ \variable{ECSIMPORT} \ar[ru] & \resource{symbol\\files} \ar@/r/[u] & \resource{debugging\\information}}
\seeoberon\seeassembly\seemmix\seeobject\seedebugging
}

\providecommand{\oborok}{
\toolsection{obor1k} is a compiler for the Oberon programming language targeting the OpenRISC 1000 hardware architecture.
It generates machine code for OpenRISC 1000 processors from modules written in Oberon and stores it in corresponding object files.
For debugging purposes, it also creates a debugging information file as well as an assembly file containing a listing of the generated machine code.
In addition, it stores the interface of each module in a symbol file which is required when other modules import the module.
Programs generated with this compiler require additional runtime support that is stored in the \file{ob\-or1k\-run} library file.
\flowgraph{\resource{Oberon\\source code} \ar[r] & \toolbox{obor1k} \ar[r] \ar@/l/[d] \ar[rd] & \resource{object file} \\ \variable{ECSIMPORT} \ar[ru] & \resource{symbol\\files} \ar@/r/[u] & \resource{debugging\\information}}
\seeoberon\seeassembly\seeorok\seeobject\seedebugging
}

\providecommand{\obppca}{
\toolsection{obppc32} is a compiler for the Oberon programming language targeting the PowerPC hardware architecture.
It generates machine code for PowerPC processors from modules written in Oberon and stores it in corresponding object files.
The compiler generates machine code for the 32-bit operating mode defined by the PowerPC architecture.
For debugging purposes, it also creates a debugging information file as well as an assembly file containing a listing of the generated machine code.
In addition, it stores the interface of each module in a symbol file which is required when other modules import the module.
Programs generated with this compiler require additional runtime support that is stored in the \file{ob\-ppc32\-run} library file.
\flowgraph{\resource{Oberon\\source code} \ar[r] & \toolbox{obppc32} \ar[r] \ar@/l/[d] \ar[rd] & \resource{object file} \\ \variable{ECSIMPORT} \ar[ru] & \resource{symbol\\files} \ar@/r/[u] & \resource{debugging\\information}}
\seeoberon\seeassembly\seeppc\seeobject\seedebugging
}

\providecommand{\obppcb}{
\toolsection{obppc64} is a compiler for the Oberon programming language targeting the PowerPC hardware architecture.
It generates machine code for PowerPC processors from modules written in Oberon and stores it in corresponding object files.
The compiler generates machine code for the 64-bit operating mode defined by the PowerPC architecture.
For debugging purposes, it also creates a debugging information file as well as an assembly file containing a listing of the generated machine code.
In addition, it stores the interface of each module in a symbol file which is required when other modules import the module.
Programs generated with this compiler require additional runtime support that is stored in the \file{ob\-ppc64\-run} library file.
\flowgraph{\resource{Oberon\\source code} \ar[r] & \toolbox{obppc64} \ar[r] \ar@/l/[d] \ar[rd] & \resource{object file} \\ \variable{ECSIMPORT} \ar[ru] & \resource{symbol\\files} \ar@/r/[u] & \resource{debugging\\information}}
\seeoberon\seeassembly\seeppc\seeobject\seedebugging
}

\providecommand{\obrisc}{
\toolsection{obrisc} is a compiler for the Oberon programming language targeting the RISC hardware architecture.
It generates machine code for RISC processors from modules written in Oberon and stores it in corresponding object files.
For debugging purposes, it also creates a debugging information file as well as an assembly file containing a listing of the generated machine code.
In addition, it stores the interface of each module in a symbol file which is required when other modules import the module.
Programs generated with this compiler require additional runtime support that is stored in the \file{ob\-risc\-run} library file.
\flowgraph{\resource{Oberon\\source code} \ar[r] & \toolbox{obrisc} \ar[r] \ar@/l/[d] \ar[rd] & \resource{object file} \\ \variable{ECSIMPORT} \ar[ru] & \resource{symbol\\files} \ar@/r/[u] & \resource{debugging\\information}}
\seeoberon\seeassembly\seerisc\seeobject\seedebugging
}

\providecommand{\obwasm}{
\toolsection{obwasm} is a compiler for the Oberon programming language targeting the WebAssembly architecture.
It generates machine code for WebAssembly targets from modules written in Oberon and stores it in corresponding object files.
For debugging purposes, it also creates a debugging information file as well as an assembly file containing a listing of the generated machine code.
In addition, it stores the interface of each module in a symbol file which is required when other modules import the module.
Programs generated with this compiler require additional runtime support that is stored in the \file{ob\-wasm\-run} library file.
\flowgraph{\resource{Oberon\\source code} \ar[r] & \toolbox{obwasm} \ar[r] \ar@/l/[d] \ar[rd] & \resource{object file} \\ \variable{ECSIMPORT} \ar[ru] & \resource{symbol\\files} \ar@/r/[u] & \resource{debugging\\information}}
\seeoberon\seeassembly\seewasm\seeobject\seedebugging
}

% converter tools

\providecommand{\dbgdwarf}{
\toolsection{dbgdwarf} is a DWARF debugging information converter tool.
It converts debugging information into the DWARF debugging data format and stores it in corresponding object files~\cite{dwarffile}.
The resulting debugging object files can be combined with runtime support that creates Executable and Linking Format (ELF) files~\cite{elffile}.
\flowgraph{\resource{debugging\\information} \ar[r] & \toolbox{dbgdwarf} \ar[r] & \resource{debugging\\object file}}
\seeobject\seedebugging
}

% assembler tools

\providecommand{\asmprint}{
\toolsection{asmprint} is a pretty printer for generic assembly code.
It reformats generic assembly code and writes it to the standard output stream.
\flowgraph{\resource{generic assembly\\source code} \ar[r] & \toolbox{asmprint} \ar[r] & \resource{reformatted\\source code}}
\seeassembly
}

\providecommand{\amdaasm}{
\toolsection{amd16asm} is an assembler for the AMD64 hardware architecture.
It translates assembly code into machine code for AMD64 processors and stores it in corresponding object files.
By default, the assembler generates machine code for the 16-bit operating mode defined by the AMD64 architecture.
\flowgraph{\resource{AMD16 assembly\\source code} \ar[r] & \toolbox{amd16asm} \ar[r] & \resource{object file}}
\seeassembly\seeamd\seeobject
}

\providecommand{\amdadism}{
\toolsection{amd16dism} is a disassembler for the AMD64 hardware architecture.
It translates machine code from object files targeting AMD64 processors into assembly code and writes it to the standard output stream.
It assumes that the machine code was generated for the 16-bit operating mode defined by the AMD64 architecture.
\flowgraph{\resource{object file} \ar[r] & \toolbox{amd16dism} \ar[r] & \resource{disassembly\\listing}}
\seeassembly\seeamd\seeobject
}

\providecommand{\amdbasm}{
\toolsection{amd32asm} is an assembler for the AMD64 hardware architecture.
It translates assembly code into machine code for AMD64 processors and stores it in corresponding object files.
By default, the assembler generates machine code for the 32-bit operating mode defined by the AMD64 architecture.
\flowgraph{\resource{AMD32 assembly\\source code} \ar[r] & \toolbox{amd32asm} \ar[r] & \resource{object file}}
\seeassembly\seeamd\seeobject
}

\providecommand{\amdbdism}{
\toolsection{amd32dism} is a disassembler for the AMD64 hardware architecture.
It translates machine code from object files targeting AMD64 processors into assembly code and writes it to the standard output stream.
It assumes that the machine code was generated for the 32-bit operating mode defined by the AMD64 architecture.
\flowgraph{\resource{object file} \ar[r] & \toolbox{amd32dism} \ar[r] & \resource{disassembly\\listing}}
\seeassembly\seeamd\seeobject
}

\providecommand{\amdcasm}{
\toolsection{amd64asm} is an assembler for the AMD64 hardware architecture.
It translates assembly code into machine code for AMD64 processors and stores it in corresponding object files.
By default, the assembler generates machine code for the 64-bit operating mode defined by the AMD64 architecture.
\flowgraph{\resource{AMD64 assembly\\source code} \ar[r] & \toolbox{amd64asm} \ar[r] & \resource{object file}}
\seeassembly\seeamd\seeobject
}

\providecommand{\amdcdism}{
\toolsection{amd64dism} is a disassembler for the AMD64 hardware architecture.
It translates machine code from object files targeting AMD64 processors into assembly code and writes it to the standard output stream.
It assumes that the machine code was generated for the 64-bit operating mode defined by the AMD64 architecture.
\flowgraph{\resource{object file} \ar[r] & \toolbox{amd64dism} \ar[r] & \resource{disassembly\\listing}}
\seeassembly\seeamd\seeobject
}

\providecommand{\armaasm}{
\toolsection{arma32asm} is an assembler for the ARM hardware architecture.
It translates assembly code into machine code for ARM processors executing A32 instructions and stores it in corresponding object files.
\flowgraph{\resource{ARM A32 assembly\\source code} \ar[r] & \toolbox{arma32asm} \ar[r] & \resource{object file}}
\seeassembly\seearm\seeobject
}

\providecommand{\armadism}{
\toolsection{arma32dism} is a disassembler for the ARM hardware architecture.
It translates machine code from object files targeting ARM processors executing A32 instructions into assembly code and writes it to the standard output stream.
\flowgraph{\resource{object file} \ar[r] & \toolbox{arma32dism} \ar[r] & \resource{disassembly\\listing}}
\seeassembly\seearm\seeobject
}

\providecommand{\armbasm}{
\toolsection{arma64asm} is an assembler for the ARM hardware architecture.
It translates assembly code into machine code for ARM processors executing A64 instructions and stores it in corresponding object files.
\flowgraph{\resource{ARM A64 assembly\\source code} \ar[r] & \toolbox{arma64asm} \ar[r] & \resource{object file}}
\seeassembly\seearm\seeobject
}

\providecommand{\armbdism}{
\toolsection{arma64dism} is a disassembler for the ARM hardware architecture.
It translates machine code from object files targeting ARM processors executing A64 instructions into assembly code and writes it to the standard output stream.
\flowgraph{\resource{object file} \ar[r] & \toolbox{arma64dism} \ar[r] & \resource{disassembly\\listing}}
\seeassembly\seearm\seeobject
}

\providecommand{\armcasm}{
\toolsection{armt32asm} is an assembler for the ARM hardware architecture.
It translates assembly code into machine code for ARM processors executing T32 instructions and stores it in corresponding object files.
\flowgraph{\resource{ARM T32 assembly\\source code} \ar[r] & \toolbox{armt32asm} \ar[r] & \resource{object file}}
\seeassembly\seearm\seeobject
}

\providecommand{\armcdism}{
\toolsection{armt32dism} is a disassembler for the ARM hardware architecture.
It translates machine code from object files targeting ARM processors executing T32 instructions into assembly code and writes it to the standard output stream.
\flowgraph{\resource{object file} \ar[r] & \toolbox{armt32dism} \ar[r] & \resource{disassembly\\listing}}
\seeassembly\seearm\seeobject
}

\providecommand{\avrasm}{
\toolsection{avrasm} is an assembler for the AVR hardware architecture.
It translates assembly code into machine code for AVR processors and stores it in corresponding object files.
The identifiers \texttt{RXL}, \texttt{RXH}, \texttt{RYL}, \texttt{RYH}, \texttt{RZL}, and \texttt{RZH} are predefined and name the corresponding registers.
The identifiers \texttt{SPL} and \texttt{SPH} are also predefined and evaluate to the address of the corresponding registers.
\flowgraph{\resource{AVR assembly\\source code} \ar[r] & \toolbox{avrasm} \ar[r] & \resource{object file}}
\seeassembly\seeavr\seeobject
}

\providecommand{\avrdism}{
\toolsection{avrdism} is a disassembler for the AVR hardware architecture.
It translates machine code from object files targeting AVR processors into assembly code and writes it to the standard output stream.
\flowgraph{\resource{object file} \ar[r] & \toolbox{avrdism} \ar[r] & \resource{disassembly\\listing}}
\seeassembly\seeavr\seeobject
}

\providecommand{\avrttasm}{
\toolsection{avr32asm} is an assembler for the AVR32 hardware architecture.
It translates assembly code into machine code for AVR32 processors and stores it in corresponding object files.
\flowgraph{\resource{AVR32 assembly\\source code} \ar[r] & \toolbox{avr32asm} \ar[r] & \resource{object file}}
\seeassembly\seeavrtt\seeobject
}

\providecommand{\avrttdism}{
\toolsection{avr32dism} is a disassembler for the AVR32 hardware architecture.
It translates machine code from object files targeting AVR32 processors into assembly code and writes it to the standard output stream.
\flowgraph{\resource{object file} \ar[r] & \toolbox{avr32dism} \ar[r] & \resource{disassembly\\listing}}
\seeassembly\seeavrtt\seeobject
}

\providecommand{\mabkasm}{
\toolsection{m68kasm} is an assembler for the M68000 hardware architecture.
It translates assembly code into machine code for M68000 processors and stores it in corresponding object files.
\flowgraph{\resource{68000 assembly\\source code} \ar[r] & \toolbox{m68kasm} \ar[r] & \resource{object file}}
\seeassembly\seemabk\seeobject
}

\providecommand{\mabkdism}{
\toolsection{m68kdism} is a disassembler for the M68000 hardware architecture.
It translates machine code from object files targeting M68000 processors into assembly code and writes it to the standard output stream.
\flowgraph{\resource{object file} \ar[r] & \toolbox{m68kdism} \ar[r] & \resource{disassembly\\listing}}
\seeassembly\seemabk\seeobject
}

\providecommand{\miblasm}{
\toolsection{miblasm} is an assembler for the MicroBlaze hardware architecture.
It translates assembly code into machine code for MicroBlaze processors and stores it in corresponding object files.
\flowgraph{\resource{MicroBlaze assembly\\source code} \ar[r] & \toolbox{miblasm} \ar[r] & \resource{object file}}
\seeassembly\seemibl\seeobject
}

\providecommand{\mibldism}{
\toolsection{mibldism} is a disassembler for the MicroBlaze hardware architecture.
It translates machine code from object files targeting MicroBlaze processors into assembly code and writes it to the standard output stream.
\flowgraph{\resource{object file} \ar[r] & \toolbox{mibldism} \ar[r] & \resource{disassembly\\listing}}
\seeassembly\seemibl\seeobject
}

\providecommand{\mipsaasm}{
\toolsection{mips32asm} is an assembler for the MIPS32 hardware architecture.
It translates assembly code into machine code for MIPS32 processors and stores it in corresponding object files.
\flowgraph{\resource{MIPS32 assembly\\source code} \ar[r] & \toolbox{mips32asm} \ar[r] & \resource{object file}}
\seeassembly\seemips\seeobject
}

\providecommand{\mipsadism}{
\toolsection{mips32dism} is a disassembler for the MIPS32 hardware architecture.
It translates machine code from object files targeting MIPS32 processors into assembly code and writes it to the standard output stream.
\flowgraph{\resource{object file} \ar[r] & \toolbox{mips32dism} \ar[r] & \resource{disassembly\\listing}}
\seeassembly\seemips\seeobject
}

\providecommand{\mipsbasm}{
\toolsection{mips64asm} is an assembler for the MIPS64 hardware architecture.
It translates assembly code into machine code for MIPS64 processors and stores it in corresponding object files.
\flowgraph{\resource{MIPS64 assembly\\source code} \ar[r] & \toolbox{mips64asm} \ar[r] & \resource{object file}}
\seeassembly\seemips\seeobject
}

\providecommand{\mipsbdism}{
\toolsection{mips64dism} is a disassembler for the MIPS64 hardware architecture.
It translates machine code from object files targeting MIPS64 processors into assembly code and writes it to the standard output stream.
\flowgraph{\resource{object file} \ar[r] & \toolbox{mips64dism} \ar[r] & \resource{disassembly\\listing}}
\seeassembly\seemips\seeobject
}

\providecommand{\mmixasm}{
\toolsection{mmixasm} is an assembler for the MMIX hardware architecture.
It translates assembly code into machine code for MMIX processors and stores it in corresponding object files.
The names of all special registers are predefined and evaluate to the corresponding number.
\flowgraph{\resource{MMIX assembly\\source code} \ar[r] & \toolbox{mmixasm} \ar[r] & \resource{object file}}
\seeassembly\seemmix\seeobject
}

\providecommand{\mmixdism}{
\toolsection{mmixdism} is a disassembler for the MMIX hardware architecture.
It translates machine code from object files targeting MMIX processors into assembly code and writes it to the standard output stream.
\flowgraph{\resource{object file} \ar[r] & \toolbox{mmixdism} \ar[r] & \resource{disassembly\\listing}}
\seeassembly\seemmix\seeobject
}

\providecommand{\orokasm}{
\toolsection{or1kasm} is an assembler for the OpenRISC 1000 hardware architecture.
It translates assembly code into machine code for OpenRISC 1000 processors and stores it in corresponding object files.
\flowgraph{\resource{OpenRISC 1000 assembly\\source code} \ar[r] & \toolbox{or1kasm} \ar[r] & \resource{object file}}
\seeassembly\seeorok\seeobject
}

\providecommand{\orokdism}{
\toolsection{or1kdism} is a disassembler for the OpenRISC 1000 hardware architecture.
It translates machine code from object files targeting OpenRISC 1000 processors into assembly code and writes it to the standard output stream.
\flowgraph{\resource{object file} \ar[r] & \toolbox{or1kdism} \ar[r] & \resource{disassembly\\listing}}
\seeassembly\seeorok\seeobject
}

\providecommand{\ppcaasm}{
\toolsection{ppc32asm} is an assembler for the PowerPC hardware architecture.
It translates assembly code into machine code for PowerPC processors and stores it in corresponding object files.
By default, the assembler generates machine code for the 32-bit operating mode defined by the PowerPC architecture.
\flowgraph{\resource{PowerPC assembly\\source code} \ar[r] & \toolbox{ppc32asm} \ar[r] & \resource{object file}}
\seeassembly\seeppc\seeobject
}

\providecommand{\ppcadism}{
\toolsection{ppc32dism} is a disassembler for the PowerPC hardware architecture.
It translates machine code from object files targeting PowerPC processors into assembly code and writes it to the standard output stream.
It assumes that the machine code was generated for the 32-bit operating mode defined by the PowerPC architecture.
\flowgraph{\resource{object file} \ar[r] & \toolbox{ppc32dism} \ar[r] & \resource{disassembly\\listing}}
\seeassembly\seeppc\seeobject
}

\providecommand{\ppcbasm}{
\toolsection{ppc64asm} is an assembler for the PowerPC hardware architecture.
It translates assembly code into machine code for PowerPC processors and stores it in corresponding object files.
By default, the assembler generates machine code for the 64-bit operating mode defined by the PowerPC architecture.
\flowgraph{\resource{PowerPC assembly\\source code} \ar[r] & \toolbox{ppc64asm} \ar[r] & \resource{object file}}
\seeassembly\seeppc\seeobject
}

\providecommand{\ppcbdism}{
\toolsection{ppc64dism} is a disassembler for the PowerPC hardware architecture.
It translates machine code from object files targeting PowerPC processors into assembly code and writes it to the standard output stream.
It assumes that the machine code was generated for the 64-bit operating mode defined by the PowerPC architecture.
\flowgraph{\resource{object file} \ar[r] & \toolbox{ppc64dism} \ar[r] & \resource{disassembly\\listing}}
\seeassembly\seeppc\seeobject
}

\providecommand{\riscasm}{
\toolsection{riscasm} is an assembler for the RISC hardware architecture.
It translates assembly code into machine code for RISC processors and stores it in corresponding object files.
The names of all special registers are predefined and evaluate to the corresponding number.
\flowgraph{\resource{RISC assembly\\source code} \ar[r] & \toolbox{riscasm} \ar[r] & \resource{object file}}
\seeassembly\seerisc\seeobject
}

\providecommand{\riscdism}{
\toolsection{riscdism} is a disassembler for the RISC hardware architecture.
It translates machine code from object files targeting RISC processors into assembly code and writes it to the standard output stream.
\flowgraph{\resource{object file} \ar[r] & \toolbox{riscdism} \ar[r] & \resource{disassembly\\listing}}
\seeassembly\seerisc\seeobject
}

\providecommand{\wasmasm}{
\toolsection{wasmasm} is an assembler for the WebAssembly architecture.
It translates assembly code into machine code for WebAssembly targets and stores it in corresponding object files.
The names of all special registers are predefined and evaluate to the corresponding number.
\flowgraph{\resource{WebAssembly assembly\\source code} \ar[r] & \toolbox{wasmasm} \ar[r] & \resource{object file}}
\seeassembly\seewasm\seeobject
}

\providecommand{\wasmdism}{
\toolsection{wasmdism} is a disassembler for the WebAssembly architecture.
It translates machine code from object files targeting WebAssembly targets into assembly code and writes it to the standard output stream.
\flowgraph{\resource{object file} \ar[r] & \toolbox{wasmdism} \ar[r] & \resource{disassembly\\listing}}
\seeassembly\seewasm\seeobject
}

% linker tools

\providecommand{\linklib}{
\toolsection{linklib} is an object file combiner.
It creates a static library file by combining all object files given to it into a single one.
\flowgraph{\resource{object files} \ar[r] & \toolbox{linklib} \ar[r] & \resource{library file}}
\seeobject
}

\providecommand{\linkbin}{
\toolsection{linkbin} is a linker for plain binary files.
It links all object files given to it into a single image and stores it in a binary file that begins with the first linked section.
It also creates a map file that lists the address, type, name and size of all used sections.
The filename extension of the resulting binary file can be specified by putting it into a constant data section called \texttt{\_extension}.
\flowgraph{\resource{object files} \ar[r] & \toolbox{linkbin} \ar[r] \ar[d] & \resource{binary file} \\ & \resource{map file}}
\seeobject
}

\providecommand{\linkmem}{
\toolsection{linkmem} is a linker for plain binary files partitioned into random-access and read-only memory.
It links all object files given to it into two distinct images, one for data sections and one for code and constant data sections, and stores each image in a binary file that begins with the first linked section of the corresponding type.
It also creates a map file that lists the address, type, name and size of all used sections.
\flowgraph{\resource{object files} \ar[r] & \toolbox{linkmem} \ar[r] \ar[d] & \resource{RAM file/\\ROM file} \\ & \resource{map file}}
\seeobject
}

\providecommand{\linkprg}{
\toolsection{linkprg} is a linker for GEMDOS executable files.
It links all object files given to it into a single image and stores the image in an Atari GEMDOS executable file~\cite{gemdosfile}.
It also creates a map file that lists the address relative to the text segment, type, name and size of all used sections.
The filename extension of the resulting executable file can be specified by putting it into a constant data section called \texttt{\_extension}.
The GEMDOS executable file format requires all patch patterns of absolute link patches to consist of four full bitmasks with descending offsets.
\flowgraph{\resource{object files} \ar[r] & \toolbox{linkprg} \ar[r] \ar[d] & \resource{executable file} \\ & \resource{map file}}
\seeobject
}

\providecommand{\linkhex}{
\toolsection{linkhex} is a linker for Intel HEX files.
It links all code sections of the object files given to it into single image and stores the image in an Intel HEX file~\cite{hexfile} that begins with the first linked section.
It also creates a map file that lists the address, type, name and size of all used sections.
\flowgraph{\resource{object files} \ar[r] & \toolbox{linkhex} \ar[r] \ar[d] & \resource{HEX file} \\ & \resource{map file}}
\seeobject
}

\providecommand{\mapsearch}{
\toolsection{mapsearch} is a debugging tool.
It searches map files generated by linker tools for the name of a binary section that encompasses a memory address read from the standard input stream.
If additionally provided with one or more object files, it also stores an excerpt thereof in a separate object file called map search result which only contains the identified binary section for disassembling purposes.
\flowgraph{& \resource{map files/\\object files} \ar[d] \\ \resource{memory\\address} \ar[r] & \toolbox{mapsearch} \ar[r] \ar[d] & \resource{section name/\\relative offset} \\ & \resource{object file\\excerpt}}
\seeobject
}

\renewcommand{\seewasm}{}

\startchapter{WebAssembly}{WebAssembly Architecture Support}{wasm}
{This \documentation{} describes how the \ecs{} supports the WebAssembly architecture.
This includes information about the assembler, disassembler, and the various compilers featured by the \ecs{} as well as the interoperability between these tools.}

\section{Introduction}

The \ecs{} features various compilers, an assembler, and a disassembler that target the WebAssembly architecture.
Figure~\ref{fig:wasmdataflow} shows the data flow in-between these tools.

\begin{figure}
\flowgraph{
\resource{intermediate\\code} \ar[d] & & \resource{assembly\\source code} \ar[d] \\
\converter{WebAssembly\\Generator} \ar[r] \ar[rd] \ar[d] & \resource{assembly\\listing} \ar[r] & \converter{WebAssembly\\Assembler} \ar[ld] \\
\resource{debugging\\information} & \resource{object file} \ar[d] \\
& \converter{WebAssembly\\Disassembler} \ar[d] \\
& \resource{disassembly\\listing} \\
}\caption{Data flow within the tools targeting the WebAssembly architecture}
\label{fig:wasmdataflow}
\end{figure}

All compilers targeting the WebAssembly architecture translate their programs using an intermediate code representation.
The WebAssembly generator is able to translate the intermediate code representation of a program into machine code for WebAssembly targets.
It stores the resulting binary code and data in so-called object files.
Additionally, the generator is able to create an assembly code listing of the machine code for debugging purposes.
This assembly code listing can also be processed by the assembler yielding exactly the same object file.
The disassembler is able to open object files and print a human-readable disassembly listing of their contents.
\seeobject\seecode

\section{Instruction Set}

Tools targeting the WebAssembly architecture support the instruction set listed in Table~\ref{tab:wasmset} and use the same assembly syntax as predefined by the World Wide Web Consortium (W3C)~\cite{wasm:instructionset}.
The only exception are the addition of the pseudo instructions \texttt{i32}, \texttt{label}, \texttt{lane}, \texttt{s32}, \texttt{u32}, and \texttt{valtype}.
They encode an immediate value of the respective type with a fixed size and can be used as a replacement for operands of instructions that require immediate values of that type.
\seeassembly

\instructionset{wasm}{Supported WebAssembly instruction set}{2}{2}

\section{Calling Convention}\index{Calling convention!of WebAssembly}

The machine code generator and runtime support for the WebAssembly architecture as provided by the \ecs{} use the following calling convention in order to enable interoperability.

\subsection{Stack Operations}

Arguments for functions are in general passed using the stack according to the intermediate code specification.
See \Documentation{}~\documentationref{code}{Intermediate Code Representation} for more information about the role of the stack.
Function arguments are pushed on the stack in reverse order and cleaned by the caller.

\subsection{Register Mapping}

The special-purpose registers defined by the intermediate code representation are mapped to mutable WebAssembly globals in the following way:

\begin{itemize}

\item Result Register\alignright\texttt{\$res}\nopagebreak

The intermediate code result register \texttt{\$res} is mapped to WebAssembly globals called \texttt{_\$res_i32}, \texttt{_\$res_i64}, \texttt{_\$res_f32}, and \texttt{_\$res_f64} with the respective types.

\item Stack Pointer Register\alignright\texttt{\$sp}\nopagebreak

The intermediate code stack pointer register \texttt{\$sp} is mapped to a WebAssembly global called \texttt{_\$sp} with type \texttt{i32}.

\item Frame Pointer Register\alignright\texttt{\$fp}\nopagebreak

The intermediate code frame pointer register \texttt{\$fp} is mapped to a WebAssembly global called \texttt{_\$fp} with type \texttt{i32}.

\item Link Register\alignright\texttt{\$lnk}\nopagebreak

The intermediate code link register \texttt{\$lnk} as well as return addresses are not supported.

\end{itemize}

All other intermediate code registers are mapped as needed to WebAssembly locals.
Their contents and mapping are therefore not considered volatile across function calls.

\section{Runtime Support}\index{Runtime support!for WebAssembly}

The \ecs{} provides runtime support for the WebAssembly architecture and runtime environments based on this architecture in object files.
Users targeting a specific runtime environment have to use an appropriate linker together with these object files in order create an executable program.
This section gives information about all supported runtime environments based on the WebAssembly architecture as well as the required combination of linker and object files.

Basic architectural runtime support is provided by the object file \objfile{wasm\-run}.
Users should always include this object file during linking regardless of the actual target runtime environment.
All other object files given to the linker should target the same architecture.

Programs written in \cpp{} need additional runtime support stored in the \libfile{cpp\-wasm\-run} library file.
Programs written in Oberon need additional runtime support stored in the \libfile{ob\-wasm\-run} library file.
\seecpp\seeoberon

Programs targeting web environments are created using the \tool{link\-bin} linker tool.
It creates a WebAssembly module~\cite{wasm:instructionset} if provided with the runtime support stored in the \objfile{web\-assembly\-run} object file.
Calling the \tool{ecsd} utility tool using the \environment{web\-assembly} target environment achieves the same result.

\section{WebAssembly Tools}

The \ecs{} provides the following tools that are able to process object files targeting the WebAssembly architecture.
\interface

\cdwasm
\cppwasm
\falwasm
\obwasm
\wasmasm
\wasmdism
\linkbin

\concludechapter


\part{\ecs{} Internals}
% Object file representation
% Copyright (C) Florian Negele

% This file is part of the Eigen Compiler Suite.

% Permission is granted to copy, distribute and/or modify this document
% under the terms of the GNU Free Documentation License, Version 1.3
% or any later version published by the Free Software Foundation.

% You should have received a copy of the GNU Free Documentation License
% along with the ECS.  If not, see <https://www.gnu.org/licenses/>.

% Generic documentation utilities
% Copyright (C) Florian Negele

% This file is part of the Eigen Compiler Suite.

% Permission is granted to copy, distribute and/or modify this document
% under the terms of the GNU Free Documentation License, Version 1.3
% or any later version published by the Free Software Foundation.

% You should have received a copy of the GNU Free Documentation License
% along with the ECS.  If not, see <https://www.gnu.org/licenses/>.

\providecommand{\cpp}{C\texttt{++}}
\providecommand{\opt}{_\mathit{opt}}
\providecommand{\tool}[1]{\texttt{#1}}
\providecommand{\version}{Version 0.0.40}
\providecommand{\resource}[1]{*++\txt{#1}}
\providecommand{\ecs}{Eigen Compiler Suite}
\providecommand{\changed}[1]{\underline{#1}}
\providecommand{\toolbox}[1]{\converter{#1}}
\providecommand{\file}{}\renewcommand{\file}[1]{\texttt{#1}}
\providecommand{\alignright}{\hfill\linebreak[0]\hspace*{\fill}}
\providecommand{\converter}[1]{*++[F][F*:white][F,:gray]\txt{#1}}
\providecommand{\documentation}{\ifbook chapter\else document\fi}
\providecommand{\Documentation}{\ifbook Chapter\else Document\fi}
\providecommand{\variable}[1]{\resource{\texttt{\small#1}\\variable}}
\providecommand{\documentationref}[2]{\ifbook\ref{#1}\else``\href{#1}{#2}''~\cite{#1}\fi}
\providecommand{\objfile}[1]{\texttt{#1}\index[runtime]{#1 object file@\texttt{#1} object file}}
\providecommand{\libfile}[1]{\texttt{#1}\index[runtime]{#1 library file@\texttt{#1} library file}}
\providecommand{\epigraph}[2]{\ifbook\begin{quote}\flushright\textit{#1}\par--- #2\end{quote}\fi}
\providecommand{\environmentvariable}[1]{\texttt{#1}\index{Environment variables!#1@\texttt{#1}}}
\providecommand{\environment}[1]{\texttt{#1}\index[environment]{#1 environment@\texttt{#1} environment}}
\providecommand{\toolsection}{}\renewcommand{\toolsection}[1]{\subsection{#1}\label{\prefix:#1}\tool{#1}}
\providecommand{\instruction}{}\renewcommand{\instruction}[2]{\noindent\qquad\pdftooltip{\texttt{#1}}{#2}\refstepcounter{instruction}\par}
\providecommand{\flowgraph}{}\renewcommand{\flowgraph}[1]{\par\sffamily\begin{displaymath}\xymatrix@=4ex{#1}\end{displaymath}\normalfont\par}
\providecommand{\instructionset}{}\renewcommand{\instructionset}[4]{\setcounter{instruction}{0}\begin{multicols}{\ifbook#3\else#4\fi}[{\captionof{table}[#2]{#2 (\ref*{#1:instructions}~instructions)}\label{tab:#1set}\vspace{-2ex}}]\footnotesize\raggedcolumns\input{#1.set}\label{#1:instructions}\end{multicols}}

\providecommand{\gpl}{GNU General Public License}
\providecommand{\rse}{ECS Runtime Support Exception}
\providecommand{\fdl}{\href{https://www.gnu.org/licenses/fdl.html}{GNU Free Documentation License}}

\providecommand{\docbegin}{}
\providecommand{\docend}{}
\providecommand{\doclabel}[1]{\hypertarget{#1}}
\providecommand{\doclink}[2]{\hyperlink{#1}{#2}}
\providecommand{\docsection}[3]{\hypertarget{#1}{\subsection{#2}}\label{sec:#1}\index[library]{#2@#3}}
\providecommand{\docsectionstar}[1]{}
\providecommand{\docsubbegin}{\begin{description}}
\providecommand{\docsubend}{\end{description}}
\providecommand{\docsubsection}[3]{\item[\hypertarget{#1}{#2}]\index[library]{#2@#3}}
\providecommand{\docsubsectionstar}[1]{\smallskip}
\providecommand{\docsubsubsection}[3]{\docsubsection{#1}{#2}{#3}}
\providecommand{\docsubsubsectionstar}[1]{}
\providecommand{\docsubsubsubsection}[3]{}
\providecommand{\docsubsubsubsectionstar}[1]{}
\providecommand{\doctable}{}

\providecommand{\debuggingtool}{}\renewcommand{\debuggingtool}{This tool is provided for debugging purposes.
It allows exposing and modifying an internal data structure that is usually not accessible.
}

\providecommand{\interface}{All tools accept command-line arguments which are taken as names of plain text files containing the source code.
If no arguments are provided, the standard input stream is used instead.
Output files are generated in the current working directory and have the same name as the input file being processed whereas the filename extension gets replaced by an appropriate suffix.
\seeinterface
}

\providecommand{\license}{\noindent Copyright \copyright{} Florian Negele\par\medskip\noindent
Permission is granted to copy, distribute and/or modify this document under the terms of the
\fdl{}, Version 1.3 or any later version published by the \href{https://fsf.org/}{Free Software Foundation}.
}

\providecommand{\ecslogosurface}{
\fill[darkgray] (0,0,0) -- (0,0,3) -- (0,3,3) -- (0,3,1) -- (0,4,1) -- (0,4,3) -- (0,5,3) -- (0,5,0) -- (0,2,0) -- (0,2,2) -- (0,1,2) -- (0,1,0) -- cycle;
\fill[gray] (0,5,0) -- (0,5,3) -- (1,5,3) -- (1,5,1) -- (2,5,1) -- (2,5,3) -- (3,5,3) -- (3,5,0) -- cycle;
\fill[lightgray] (0,0,0) -- (0,1,0) -- (2,1,0) -- (2,4,0) -- (1,4,0) -- (1,3,0) -- (2,3,0) -- (2,2,0) -- (0,2,0) -- (0,5,0) -- (3,5,0) -- (3,0,0) -- cycle;
\begin{scope}[line width=0.5]
\begin{scope}[gray]
\draw (0,0,0) -- (0,1,0);
\draw (2,1,0) -- (2,2,0);
\draw (0,1,2) -- (0,2,2);
\draw (0,2,0) -- (0,5,0);
\draw (2,3,0) -- (2,4,0);
\end{scope}
\begin{scope}[lightgray]
\draw (0,1,0) -- (0,1,2);
\draw (0,3,1) -- (0,3,3);
\draw (0,5,0) -- (0,5,3);
\draw (2,5,1) -- (2,5,3);
\end{scope}
\begin{scope}[white]
\draw (0,1,0) -- (2,1,0);
\draw (1,3,0) -- (2,3,0);
\draw (0,5,0) -- (3,5,0);
\end{scope}
\end{scope}
}

\providecommand{\ecslogo}[1]{
\begin{tikzpicture}[scale={(#1)/((sin(45)+cos(45))*3cm)},x={({-cos(45)*1cm},{sin(45)*sin(30)*1cm})},y={({0cm},{(cos(30)*1cm})},z={({sin(45)*1cm},{cos(45)*sin(30)*1cm})}]
\begin{scope}[darkgray,line width=1]
\draw (0,0,0) -- (0,0,3) -- (0,3,3) -- (2,3,3) -- (2,5,3) -- (3,5,3) -- (3,5,0) -- (3,0,0) -- cycle;
\draw (0,3,1) -- (0,4,1) -- (0,4,3) -- (0,5,3) -- (1,5,3) -- (1,5,1) -- (2,5,1);
\draw (1,3,0) -- (1,4,0) -- (2,4,0);
\end{scope}
\fill[darkgray] (2,0,0) -- (2,0,3) -- (2,5,3) -- (2,5,1) -- (2,4,1) -- (2,4,0) -- cycle;
\fill[lightgray] (2,0,2) -- (0,0,2) -- (0,2,2) -- (2,2,2) -- cycle;
\fill[gray] (0,1,0) -- (2,1,0) -- (2,1,2) -- (0,1,2) -- cycle;
\fill[gray] (0,3,1) -- (0,3,3) -- (2,3,3) -- (2,3,0) -- (1,3,0) -- (1,3,1) -- cycle;
\ecslogosurface
\end{tikzpicture}
}

\providecommand{\shadowedecslogo}[3]{
\begin{tikzpicture}[scale={(#1)/((sin(#2)+cos(#2))*3cm)},x={({-cos(#2)*1cm},{sin(#2)*sin(#3)*1cm})},y={({0cm},{(cos(#3)*1cm})},z={({sin(#2)*1cm},{cos(#2)*sin(#3)*1cm})}]
\shade[top color=lightgray!50!white,bottom color=white,middle color=lightgray!50!white] (0,0,0) -- (3,0,0) -- (3,{-0.5-3*sin(#2)*sin(#3)/cos(#3)},0) -- (0,-0.5,0) -- cycle;
\shade[top color=darkgray!50!gray,bottom color=white,middle color=darkgray!50!white] (0,0,0) -- (0,0,3) -- (0,{-0.5-3*cos(#2)*sin(#3)/cos(#3)},3) -- (0,-0.5,0) -- cycle;
\begin{scope}[y={({(cos(#2)+sin(#2))*0.5cm},{(cos(#2)*sin(#3)-sin(#2)*sin(#3))*0.5cm})}]
\useasboundingbox (3,0,0) -- (0,0,0) -- (0,0,3);
\shade[left color=darkgray!80!black,right color=lightgray,middle color=gray] (0,0,0) -- (0,1,0) -- (0,1,0.5) -- (0,2,0) -- (0,5,0) -- (0,5,3) -- (1,5,3) -- (1,4,3) -- (1,4,2.5) -- (1,3,3) -- (2,5,3) -- (3,5,3) -- (3,0,3) -- cycle;
\clip (0,0,0) -- (0,0,3) -- ({-3*sin(#2)/cos(#2)},0,0) -- cycle;
\shade[left color=darkgray,right color=lightgray!50!gray] (0,0,0) -- (0,1,0) -- (0,1,0.5) -- (0,2,0) -- (0,5,0) -- (0,5,3) -- (1,5,3) -- (1,4,3) -- (1,4,2.5) -- (1,3,3) -- (2,5,3) -- (3,5,3) -- (3,0,3) -- cycle;
\end{scope}
\shade[left color=darkgray,right color=darkgray!80!black] (2,0,0) -- (2,0,3) -- (2,5,3) -- (2,5,1) -- (2,4,1) -- (2,4,0) -- cycle;
\shade[left color=darkgray!90!black,right color=gray!80!darkgray] (2,0,2) -- (0,0,2) -- (0,2,2) -- (2,2,2) -- cycle;
\shade[top color=darkgray!90!black,bottom color=gray!80!darkgray] (0,1,0) -- (2,1,0) -- (2,1,2) -- (0,1,2) -- cycle;
\shade[top color=darkgray!90!black,bottom color=gray!80!darkgray] (0,3,1) -- (0,3,3) -- (2,3,3) -- (2,3,0) -- (1,3,0) -- (1,3,1) -- cycle;
\fill[gray] (2,1,0) -- (1.5,1,0.5) -- (0,1,0.5) -- (0,1,0) -- cycle;
\fill[gray] (1,3,2) -- (0.5,3,2) -- (0.5,3,3) -- (1,3,3) -- cycle;
\fill[gray] (2,3,0) -- (1.5,3,0.5) -- (1,3,0.5) -- (1,3,0) -- cycle;
\ecslogosurface
\end{tikzpicture}
}

\providecommand{\cpplogo}[1]{
\begin{tikzpicture}[scale=(#1)/512em]
\fill[gray] (435.2794,398.7159) -- (247.1911,507.3075) .. controls (236.3563,513.5642) and (218.6240,513.5642) .. (207.7892,507.3075) -- (19.7009,398.7159) .. controls (8.8646,392.4606) and (0.0000,377.1043) .. (0.0000,364.5924) -- (0.0000,147.4076) .. controls (0.8430,132.8363) and (8.2856,120.7683) .. (19.7009,113.2842) -- (207.7892,4.6926) .. controls (218.6240,-1.5642) and (236.3564,-1.5642) .. (247.1911,4.6926) -- (435.2794,113.2842) .. controls (447.5273,121.4304) and (454.4987,133.6918) .. (454.9803,147.4076) -- (454.9803,364.5924) .. controls (454.5404,377.7571) and (446.6566,391.0351) .. (435.2794,398.7159) -- cycle(75.8301,255.9993) .. controls (74.9389,404.0881) and (273.2892,469.4783) .. (358.8263,331.8769) -- (293.1917,293.8965) .. controls (253.5702,359.4301) and (155.1909,335.9977) .. (151.6601,255.9993) .. controls (152.7204,182.2703) and (249.4137,148.0211) .. (293.1961,218.1065) -- (358.8308,180.1276) .. controls (283.4477,49.2645) and (79.6318,96.3470) .. (75.8301,255.9993) -- cycle(379.1503,247.5747) -- (362.2982,247.5747) -- (362.2982,230.7226) -- (345.4490,230.7226) -- (345.4490,247.5747) -- (328.5969,247.5747) -- (328.5969,264.4254) -- (345.4490,264.4254) -- (345.4490,281.2759) -- (362.2982,281.2759) -- (362.2982,264.4254) -- (379.1503,264.4254) -- cycle(442.3420,247.5747) -- (425.4899,247.5747) -- (425.4899,230.7226) -- (408.6408,230.7226) -- (408.6408,247.5747) -- (391.7886,247.5747) -- (391.7886,264.4254) -- (408.6408,264.4254) -- (408.6408,281.2759) -- (425.4899,281.2759) -- (425.4899,264.4254) -- (442.3420,264.4254) -- cycle;
\end{tikzpicture}
}

\providecommand{\fallogo}[1]{
\begin{tikzpicture}[scale=(#1)/512em]
\fill[gray] (185.7774,0.0000) .. controls (200.4486,15.9798) and (226.8966,8.7148) .. (235.0426,31.5836) .. controls (249.5297,58.0598) and (247.9581,97.9161) .. (280.3335,110.9762) .. controls (309.1690,120.3496) and (337.8406,104.2727) .. (366.5753,103.9379) .. controls (373.4449,111.5171) and (379.2885,128.2574) .. (383.9755,108.9744) .. controls (396.6979,102.5615) and (437.2808,107.6681) .. (426.9652,124.3252) .. controls (408.9822,121.0785) and (412.4742,146.0729) .. (426.5192,131.4996) .. controls (433.8413,120.8489) and (465.1541,126.5522) .. (441.9067,135.7950) .. controls (396.1879,157.7478) and (344.1112,161.5079) .. (298.5528,183.5702) .. controls (277.7471,193.5198) and (284.6941,218.7163) .. (285.2127,236.9640) .. controls (292.3599,316.2826) and (307.3929,394.6311) .. (317.1198,473.6154) .. controls (329.0637,505.4736) and (292.1195,528.5004) .. (265.9183,511.2761) .. controls (237.9284,499.2462) and (237.3684,465.2681) .. (230.9102,439.9421) .. controls (218.6692,374.3397) and (215.6307,306.9662) .. (198.1732,242.3977) .. controls (183.1379,232.7444) and (164.4245,256.0298) .. (149.0430,261.4799) .. controls (116.9328,279.2585) and (87.1822,308.5851) .. (48.2293,307.8914) .. controls (21.3220,306.9037) and (-15.9107,281.8761) .. (7.2921,252.7908) .. controls (29.7799,220.6177) and (67.5177,204.3028) .. (100.9287,185.9449) .. controls (130.8217,170.8906) and (161.1548,156.5903) .. (191.0278,141.5847) .. controls (196.1738,120.0520) and (186.6049,95.2409) .. (186.8382,72.4353) .. controls (185.5234,48.4204) and (183.1700,23.9341) .. (185.7774,0.0000) -- cycle;
\end{tikzpicture}
}

\providecommand{\oblogo}[1]{
\begin{tikzpicture}[scale=(#1)/512em]
\fill[gray] (160.3865,208.9117) .. controls (154.0879,214.6478) and (149.0735,221.2409) .. (145.4125,228.5384) .. controls (184.8790,248.4273) and (234.7122,269.8787) .. (297.5493,291.8782) .. controls (300.3943,281.4769) and (300.9552,268.7619) .. (300.4023,255.2389) .. controls (248.9909,244.7891) and (200.0310,225.9279) .. (160.3865,208.9117) -- cycle(225.7398,392.6996) .. controls (308.0209,392.1716) and (359.3326,345.9277) .. (368.7203,285.2098) .. controls (376.6742,197.1784) and (311.7194,141.3342) .. (205.4287,142.1456) .. controls (139.9485,141.4804) and (88.7155,166.1957) .. (73.5775,228.0086) .. controls (52.0297,320.3408) and (123.4078,391.0103) .. (225.7398,392.6996) -- cycle(216.0739,176.4733) .. controls (268.9183,179.2424) and (315.8292,206.5488) .. (312.7454,265.1139) .. controls (313.2769,315.6384) and (286.5993,353.4946) .. (216.6040,355.7934) .. controls (162.4657,355.7934) and (126.0914,317.5023) .. (126.0914,260.5103) .. controls (126.1733,214.2900) and (163.3363,176.2849) .. (216.0739,176.4733) -- cycle(76.4897,189.1754) .. controls (13.1586,147.5631) and (0.0000,119.4207) .. (0.0000,119.4207) -- (90.6499,170.1632) .. controls (85.3004,175.8497) and (80.5994,182.1633) .. (76.4897,189.1754) -- cycle(353.9486,119.3004) -- (402.9482,119.3004) .. controls (427.0025,137.0797) and (450.9893,162.7034) .. (474.9529,191.0213) .. controls (509.3540,228.5339) and (531.3391,294.2091) .. (487.8149,312.1206) .. controls (462.8165,324.7652) and (394.3874,316.8943) .. (373.8912,313.6651) .. controls (379.9291,297.7449) and (383.2899,278.4204) .. (381.4989,257.7214) .. controls (420.3069,248.0321) and (421.9610,218.3461) .. (407.7867,192.6417) .. controls (391.1113,162.4018) and (370.1114,132.9097) .. (353.9486,119.3004) -- cycle;
\end{tikzpicture}
}

\providecommand{\markuptable}{
\begin{table}
\sffamily\centering
\begin{tabular}{@{}lcl@{}}
\toprule
\texttt{//italics//} & $\rightarrow$ & \textit{italics} \\
\midrule
\texttt{**bold**} & $\rightarrow$ & \textbf{bold} \\
\midrule
\texttt{\# ordered list} & & 1 ordered list \\
\texttt{\# second item} & $\rightarrow$ & 2 second item \\
\texttt{\#\# sub item} & & \hspace{1em} 1 sub item \\
\midrule
\texttt{* unordered list} & & $\bullet$ unordered list \\
\texttt{* second item} & $\rightarrow$ & $\bullet$ second item \\
\texttt{** sub item} & & \hspace{1em} $\bullet$ sub item \\
\midrule
\texttt{link to [[label]]} & $\rightarrow$ & link to \underline{label} \\
\midrule
\texttt{<{}<label>{}> definition } & $\rightarrow$ & definition \\
\midrule
\texttt{[[url|link name]]} & $\rightarrow$ & \underline{link name} \\
\midrule\addlinespace
\texttt{= large heading} & & {\Large large heading} \smallskip \\
\texttt{== medium heading} & $\rightarrow$ & {\large medium heading} \\
\texttt{=== small heading} & & small heading \\
\midrule
\texttt{no line break} & & no line break for paragraphs \\
\texttt{for paragraphs} & $\rightarrow$ \\
& & use empty line \\
\texttt{use empty line} \\
\midrule
\texttt{force\textbackslash\textbackslash line break} & $\rightarrow$ & force \\
& & line break \\
\midrule
\texttt{horizontal line} & $\rightarrow$ & horizontal line \\
\texttt{----} & & \hrulefill \\
\midrule
\texttt{|=a|=table|=header} & & \underline{a \enspace table \enspace header} \\
\texttt{|a|table|row} & $\rightarrow$ & a \enspace table \enspace row \\
\texttt{|b|table|row} & & b \enspace table \enspace row \\
\midrule
\texttt{\{\{\{} \\
\texttt{unformatted} & $\rightarrow$ & \texttt{unformatted} \\
\texttt{code} & & \texttt{code} \\
\texttt{\}\}\}} \\
\midrule\addlinespace
\texttt{@ new article} & & {\Large 1.\ new article} \smallskip \\
\texttt{@ second article} & $\rightarrow$ & {\Large 2.\ second article} \smallskip \\
\texttt{@@ sub article} & & {\large 2.1.\ sub article} \\
\bottomrule
\end{tabular}
\normalfont\caption{Elements of the generic documentation markup language}
\label{tab:docmarkup}
\end{table}
}

\providecommand{\startchapter}[4]{
\documentclass[11pt,a4paper]{article}
\usepackage{booktabs}
\usepackage[format=hang,labelfont=bf]{caption}
\usepackage{changepage}
\usepackage[T1]{fontenc}
\usepackage[margin=2cm]{geometry}
\usepackage{hyperref}
\usepackage[american]{isodate}
\usepackage{lmodern}
\usepackage{longtable}
\usepackage{mathptmx}
\usepackage{microtype}
\usepackage[toc]{multitoc}
\usepackage{multirow}
\usepackage[all]{nowidow}
\usepackage{pdfcomment}
\usepackage{syntax}
\usepackage{tikz}
\usepackage[all]{xy}
\hypersetup{pdfborder={0 0 0},bookmarksnumbered=true,pdftitle={\ecs{}: #2},pdfauthor={Florian Negele},pdfsubject={\ecs{}},pdfkeywords={#1}}
\setlength{\grammarindent}{8em}\setlength{\grammarparsep}{0.2ex}
\setlength{\columnsep}{2em}
\newcommand{\prefix}{}
\newcounter{instruction}
\bibliographystyle{unsrt}
\renewcommand{\index}[2][]{}
\renewcommand{\arraystretch}{1.05}
\renewcommand{\floatpagefraction}{0.7}
\renewcommand{\syntleft}{\itshape}\renewcommand{\syntright}{}
\title{\vspace{-5ex}\Huge{\ecs{}}\medskip\hrule}
\author{\huge{#2}}
\date{\medskip\version}
\newif\ifbook\bookfalse
\pagestyle{headings}
\frenchspacing
\begin{document}
\maketitle\thispagestyle{empty}\noindent#4\setlength{\columnseprule}{0.4pt}\tableofcontents\setlength{\columnseprule}{0pt}\vfill\pagebreak[3]\null\vfill\bigskip\noindent
\parbox{\textwidth-4em}{\license The contents of this \documentation{} are part of the \href{manual}{\ecs{} User Manual}~\cite{manual} and correspond to Chapter ``\href{manual\##3}{#1}''.\alignright\mbox{\today}}
\parbox{4em}{\flushright\ecslogo{3em}}
\clearpage
}

\providecommand{\concludechapter}{
\vfill\pagebreak[3]\null\vfill
\thispagestyle{myheadings}\markright{REFERENCES}
\noindent\begin{minipage}{\textwidth}\begin{multicols}{2}[\section*{References}]
\renewcommand{\section}[2]{}\small\bibliography{references}
\end{multicols}\end{minipage}\end{document}
}

\providecommand{\startpresentation}[2]{
\documentclass[14pt,aspectratio=43,usepdftitle=false]{beamer}
\usepackage{booktabs}
\usepackage{etex}
\usepackage{multicol}
\usepackage{tikz}
\usepackage[all]{xy}
\bibliographystyle{unsrt}
\setlength{\columnsep}{1em}
\setlength{\leftmargini}{1em}
\setbeamercolor{title}{fg=black}
\setbeamercolor{structure}{fg=darkgray}
\setbeamercolor{bibliography item}{fg=darkgray}
\setbeamerfont{title}{series=\bfseries}
\setbeamerfont{subtitle}{series=\normalfont}
\setbeamerfont*{frametitle}{parent=title}
\setbeamerfont{block title}{series=\bfseries}
\setbeamerfont*{framesubtitle}{parent=subtitle}
\setbeamersize{text margin left=1em,text margin right=1em}
\setbeamertemplate{navigation symbols}{}
\setbeamertemplate{itemize item}[circle]{}
\setbeamertemplate{bibliography item}[triangle]{}
\setbeamertemplate{bibliography entry author}{\usebeamercolor[fg]{bibliography item}}
\setbeamertemplate{frametitle}{\medskip\usebeamerfont{frametitle}\color{gray}\raisebox{-2.5ex}[0ex][0ex]{\rule{0.1em}{4.5ex}}}
\addtobeamertemplate{frametitle}{}{\hspace{0.4em}\usebeamercolor[fg]{title}\insertframetitle\par\vspace{0.2ex}\hspace{0.5em}\usebeamerfont{framesubtitle}\insertframesubtitle}
\hypersetup{pdfborder={0 0 0},bookmarksnumbered=true,bookmarksopen=true,bookmarksopenlevel=0,pdftitle={\ecs{}: #1},pdfauthor={Florian Negele},pdfsubject={\ecs{}},pdfkeywords={#1}}
\renewcommand{\flowgraph}[1]{\resizebox{\textwidth}{!}{$$\xymatrix{##1}$$}}
\title{\ecs{}\medskip\hrule\medskip}
\institute{\shadowedecslogo{5em}{30}{15}}
\date{\version}
\subtitle{#1}
\begin{document}
\begin{frame}[plain]\titlepage\nocite{manual}\end{frame}
\begin{frame}{Contents}{#1}\begin{center}\tableofcontents\end{center}\end{frame}
}

\providecommand{\concludepresentation}{
\begin{frame}{References}\begin{footnotesize}\setlength{\columnseprule}{0.4pt}\begin{multicols}{2}\bibliography{references}\end{multicols}\end{footnotesize}\end{frame}
\end{document}
}

\providecommand{\startbook}[1]{
\documentclass[10pt,paper=17cm:24cm,DIV=13,twoside=semi,headings=normal,numbers=noendperiod,cleardoublepage=plain]{scrbook}
\usepackage{atveryend}
\usepackage{booktabs}
\usepackage{caption}
\usepackage{changepage}
\usepackage[T1]{fontenc}
\usepackage{imakeidx}
\usepackage{hyperref}
\usepackage[american]{isodate}
\usepackage{lmodern}
\usepackage{longtable}
\usepackage{mathptmx}
\usepackage[final]{microtype}
\usepackage{multicol}
\usepackage{multirow}
\usepackage[all]{nowidow}
\usepackage{pdfcomment}
\usepackage{scrlayer-scrpage}
\usepackage{setspace}
\usepackage{syntax}
\usepackage[eventxtindent=4pt,oddtxtexdent=4pt]{thumbs}
\usepackage{tikz}
\usepackage[all]{xy}
\hyphenation{Micro-Blaze Open-Cores Open-RISC Power-PC}
\hypersetup{pdfborder={0 0 0},bookmarksnumbered=true,bookmarksopen=true,bookmarksopenlevel=0,pdftitle={\ecs{}: #1},pdfauthor={Florian Negele},pdfsubject={\ecs{}},pdfkeywords={#1}}
\setlength{\grammarindent}{8em}\setlength{\grammarparsep}{0.7ex}
\setkomafont{captionlabel}{\usekomafont{descriptionlabel}}
\renewcommand{\arraystretch}{1.05}\setstretch{1.1}
\renewcommand{\chapterformat}{\thechapter\autodot\enskip\raisebox{-1ex}[0ex][0ex]{\color{gray}\rule{0.1em}{3.5ex}}\enskip}
\renewcommand{\startchapter}[4]{\hypertarget{##3}{\chapter{##1}}\label{##3}##4\addthumb{##1}{\LARGE\sffamily\bfseries\thechapter}{white}{gray}\renewcommand{\prefix}{##3}}
\renewcommand{\concludechapter}{\clearpage{\stopthumb\cleardoublepage}}
\renewcommand{\syntleft}{\itshape}\renewcommand{\syntright}{}
\renewcommand{\floatpagefraction}{0.7}
\renewcommand{\partheademptypage}{}
\DeclareMicrotypeAlias{lmss}{cmr}
\newcommand{\prefix}{}
\newcounter{instruction}
\bibliographystyle{unsrt}
\newif\ifbook\booktrue
\makeindex[intoc,title=Index]
\makeindex[intoc,name=tools,title=Index of Tools,columns=3]
\makeindex[intoc,name=library,title=Index of Library Names]
\makeindex[intoc,name=runtime,title=Index of Runtime Support]
\makeindex[intoc,name=environment,title=Index of Target Environments]
\indexsetup{toclevel=chapter,headers={\indexname}{\indexname}}
\frenchspacing
\begin{document}
\pagenumbering{alph}
\begin{titlepage}\centering
\huge\sffamily\null\vfill\textbf{\ecs{}}\bigskip\hrule\bigskip#1
\normalsize\normalfont\vfill\vfill\shadowedecslogo{10em}{30}{15}
\large\vfill\vfill\version
\end{titlepage}
\null\vfill
\thispagestyle{empty}
\noindent\today\par\medskip
\license A copy of this license is included in Appendix~\ref{fdl} on page~\pageref{fdl}.
All product names used herein are for identification purposes only and may be trademarks of their respective companies.
\concludechapter
\frontmatter
\setcounter{tocdepth}{1}
\tableofcontents
\setcounter{tocdepth}{2}
\concludechapter
\listoffigures
\concludechapter
\listoftables
\concludechapter
}

\providecommand{\concludebook}{
\backmatter
\addtocontents{toc}{\protect\setcounter{tocdepth}{-1}}
\phantomsection\addcontentsline{toc}{part}{Bibliography}
\bibliography{references}
\concludechapter
\phantomsection\addcontentsline{toc}{part}{Indexes}
\printindex
\concludechapter
\indexprologue{\label{idx:tools}}
\printindex[tools]
\concludechapter
\printindex[library]
\concludechapter
\indexprologue{\label{idx:runtime}}
\printindex[runtime]
\concludechapter
\indexprologue{\label{idx:environment}}
\printindex[environment]
\concludechapter
\pagestyle{empty}\pagenumbering{Alph}\null\clearpage
\null\vfill\centering\ecslogo{4em}\par\medskip\license
\end{document}
}

% chapter references

\providecommand{\seedocumentationref}{}\renewcommand{\seedocumentationref}[3]{#1, see \Documentation{}~\documentationref{#2}{#3}. }
\providecommand{\seeinterface}{}\renewcommand{\seeinterface}{\ifbook See \Documentation{}~\documentationref{interface}{User Interface} for more information about the common user interface of all of these tools. \fi}
\providecommand{\seeguide}{}\renewcommand{\seeguide}{\seedocumentationref{For basic examples of using some of these tools in practice}{guide}{User Guide}}
\providecommand{\seecpp}{}\renewcommand{\seecpp}{\seedocumentationref{For more information about the \cpp{} programming language and its implementation by the \ecs{}}{cpp}{User Manual for \cpp{}}}
\providecommand{\seefalse}{}\renewcommand{\seefalse}{\seedocumentationref{For more information about the FALSE programming language and its implementation by the \ecs{}}{false}{User Manual for FALSE}}
\providecommand{\seeoberon}{}\renewcommand{\seeoberon}{\seedocumentationref{For more information about the Oberon programming language and its implementation by the \ecs{}}{oberon}{User Manual for Oberon}}
\providecommand{\seeassembly}{}\renewcommand{\seeassembly}{\seedocumentationref{For more information about the generic assembly language and how to use it}{assembly}{Generic Assembly Language Specification}}
\providecommand{\seeamd}{}\renewcommand{\seeamd}{\seedocumentationref{For more information about how the \ecs{} supports the AMD64 hardware architecture}{amd64}{AMD64 Hardware Architecture Support}}
\providecommand{\seearm}{}\renewcommand{\seearm}{\seedocumentationref{For more information about how the \ecs{} supports the ARM hardware architecture}{arm}{ARM Hardware Architecture Support}}
\providecommand{\seeavr}{}\renewcommand{\seeavr}{\seedocumentationref{For more information about how the \ecs{} supports the AVR hardware architecture}{avr}{AVR Hardware Architecture Support}}
\providecommand{\seeavrtt}{}\renewcommand{\seeavrtt}{\seedocumentationref{For more information about how the \ecs{} supports the AVR32 hardware architecture}{avr32}{AVR32 Hardware Architecture Support}}
\providecommand{\seemabk}{}\renewcommand{\seemabk}{\seedocumentationref{For more information about how the \ecs{} supports the M68000 hardware architecture}{m68k}{M68000 Hardware Architecture Support}}
\providecommand{\seemibl}{}\renewcommand{\seemibl}{\seedocumentationref{For more information about how the \ecs{} supports the MicroBlaze hardware architecture}{mibl}{MicroBlaze Hardware Architecture Support}}
\providecommand{\seemips}{}\renewcommand{\seemips}{\seedocumentationref{For more information about how the \ecs{} supports the MIPS32 and MIPS64 hardware architectures}{mips}{MIPS Hardware Architecture Support}}
\providecommand{\seemmix}{}\renewcommand{\seemmix}{\seedocumentationref{For more information about how the \ecs{} supports the MMIX hardware architecture}{mmix}{MMIX Hardware Architecture Support}}
\providecommand{\seeorok}{}\renewcommand{\seeorok}{\seedocumentationref{For more information about how the \ecs{} supports the OpenRISC 1000 hardware architecture}{or1k}{OpenRISC 1000 Hardware Architecture Support}}
\providecommand{\seeppc}{}\renewcommand{\seeppc}{\seedocumentationref{For more information about how the \ecs{} supports the PowerPC hardware architecture}{ppc}{PowerPC Hardware Architecture Support}}
\providecommand{\seerisc}{}\renewcommand{\seerisc}{\seedocumentationref{For more information about how the \ecs{} supports the RISC hardware architecture}{risc}{RISC Hardware Architecture Support}}
\providecommand{\seewasm}{}\renewcommand{\seewasm}{\seedocumentationref{For more information about how the \ecs{} supports the WebAssembly architecture}{wasm}{WebAssembly Architecture Support}}
\providecommand{\seedocumentation}{}\renewcommand{\seedocumentation}{\seedocumentationref{For more information about generic documentations and their generation by the \ecs{}}{documentation}{Generic Documentation Generation}}
\providecommand{\seedebugging}{}\renewcommand{\seedebugging}{\seedocumentationref{For more information about debugging information and its representation}{debugging}{Debugging Information Representation}}
\providecommand{\seecode}{}\renewcommand{\seecode}{\seedocumentationref{For more information about intermediate code and its purpose}{code}{Intermediate Code Representation}}
\providecommand{\seeobject}{}\renewcommand{\seeobject}{\seedocumentationref{For more information about object files and their purpose}{object}{Object File Representation}}

% generic documentation tools

\providecommand{\docprint}{
\toolsection{docprint} is a pretty printer for generic documentations.
It reformats generic documentations and writes it to the standard output stream.
\debuggingtool
\flowgraph{\resource{generic\\documentation} \ar[r] & \toolbox{docprint} \ar[r] & \resource{generic\\documentation}}
\seedocumentation
}

\providecommand{\doccheck}{
\toolsection{doccheck} is a syntactic and semantic checker for generic documentations.
It just performs syntactic and semantic checks on generic documentations and writes its diagnostic messages to the standard error stream.
\debuggingtool
\flowgraph{\resource{generic\\documentation} \ar[r] & \toolbox{doccheck} \ar[r] & \resource{diagnostic\\messages}}
\seedocumentation
}

\providecommand{\dochtml}{
\toolsection{dochtml} is an HTML documentation generator for generic documentations.
It processes several generic documentations and assembles all information therein into an HTML document.
\debuggingtool
\flowgraph{\resource{generic\\documentation} \ar[r] & \toolbox{dochtml} \ar[r] & \resource{HTML\\document}}
\seedocumentation
}

\providecommand{\doclatex}{
\toolsection{doclatex} is a Latex documentation generator for generic documentations.
It processes several generic documentations and assembles all information therein into a Latex document.
\debuggingtool
\flowgraph{\resource{generic\\documentation} \ar[r] & \toolbox{doclatex} \ar[r] & \resource{Latex\\document}}
\seedocumentation
}

% intermediate code tools

\providecommand{\cdcheck}{
\toolsection{cdcheck} is a syntactic and semantic checker for intermediate code.
It just performs syntactic and semantic checks on programs written in intermediate code and writes its diagnostic messages to the standard error stream.
\debuggingtool
\flowgraph{\resource{intermediate\\code} \ar[r] & \toolbox{cdcheck} \ar[r] & \resource{diagnostic\\messages}}
\seeassembly\seecode
}

\providecommand{\cdopt}{
\toolsection{cdopt} is an optimizer for intermediate code.
It performs various optimizations on programs written in intermediate code and writes the result to the standard output stream.
\debuggingtool
\flowgraph{\resource{intermediate\\code} \ar[r] & \toolbox{cdopt} \ar[r] & \resource{optimized\\code}}
\seeassembly\seecode
}

\providecommand{\cdrun}{
\toolsection{cdrun} is an interpreter for intermediate code.
It processes and executes programs written in intermediate code.
The following code sections are predefined and have the usual semantics:
\texttt{abort}, \texttt{\_Exit}, \texttt{fflush}, \texttt{floor}, \texttt{fputc}, \texttt{free}, \texttt{getchar}, \texttt{malloc}, and \texttt{putchar}.
Diagnostic messages about invalid operations include the name of the executed code section and the index of the erroneous instruction.
\debuggingtool
\flowgraph{\resource{intermediate\\code} \ar[r] & \toolbox{cdrun} \ar@/u/[r] & \resource{input/\\output} \ar@/d/[l]}
\seeassembly\seecode
}

\providecommand{\cdamda}{
\toolsection{cdamd16} is a compiler for intermediate code targeting the AMD64 hardware architecture.
It generates machine code for AMD64 processors from programs written in intermediate code and stores it in corresponding object files.
The compiler generates machine code for the 16-bit operating mode defined by the AMD64 architecture.
It also creates a debugging information file as well as an assembly file containing a listing of the generated machine code.
\debuggingtool
\flowgraph{\resource{intermediate\\code} \ar[r] & \toolbox{cdamd16} \ar[r] \ar[d] \ar[rd] & \resource{object file} \\ & \resource{assembly\\listing} & \resource{debugging\\information}}
\seeassembly\seeamd\seeobject\seecode\seedebugging
}

\providecommand{\cdamdb}{
\toolsection{cdamd32} is a compiler for intermediate code targeting the AMD64 hardware architecture.
It generates machine code for AMD64 processors from programs written in intermediate code and stores it in corresponding object files.
The compiler generates machine code for the 32-bit operating mode defined by the AMD64 architecture.
It also creates a debugging information file as well as an assembly file containing a listing of the generated machine code.
\debuggingtool
\flowgraph{\resource{intermediate\\code} \ar[r] & \toolbox{cdamd32} \ar[r] \ar[d] \ar[rd] & \resource{object file} \\ & \resource{assembly\\listing} & \resource{debugging\\information}}
\seeassembly\seeamd\seeobject\seecode\seedebugging
}

\providecommand{\cdamdc}{
\toolsection{cdamd64} is a compiler for intermediate code targeting the AMD64 hardware architecture.
It generates machine code for AMD64 processors from programs written in intermediate code and stores it in corresponding object files.
The compiler generates machine code for the 64-bit operating mode defined by the AMD64 architecture.
It also creates a debugging information file as well as an assembly file containing a listing of the generated machine code.
\debuggingtool
\flowgraph{\resource{intermediate\\code} \ar[r] & \toolbox{cdamd64} \ar[r] \ar[d] \ar[rd] & \resource{object file} \\ & \resource{assembly\\listing} & \resource{debugging\\information}}
\seeassembly\seeamd\seeobject\seecode\seedebugging
}

\providecommand{\cdarma}{
\toolsection{cdarma32} is a compiler for intermediate code targeting the ARM hardware architecture.
It generates machine code for ARM processors executing A32 instructions from programs written in intermediate code and stores it in corresponding object files.
It also creates a debugging information file as well as an assembly file containing a listing of the generated machine code.
\debuggingtool
\flowgraph{\resource{intermediate\\code} \ar[r] & \toolbox{cdarma32} \ar[r] \ar[d] \ar[rd] & \resource{object file} \\ & \resource{assembly\\listing} & \resource{debugging\\information}}
\seeassembly\seearm\seeobject\seecode\seedebugging
}

\providecommand{\cdarmb}{
\toolsection{cdarma64} is a compiler for intermediate code targeting the ARM hardware architecture.
It generates machine code for ARM processors executing A64 instructions from programs written in intermediate code and stores it in corresponding object files.
It also creates a debugging information file as well as an assembly file containing a listing of the generated machine code.
\debuggingtool
\flowgraph{\resource{intermediate\\code} \ar[r] & \toolbox{cdarma64} \ar[r] \ar[d] \ar[rd] & \resource{object file} \\ & \resource{assembly\\listing} & \resource{debugging\\information}}
\seeassembly\seearm\seeobject\seecode\seedebugging
}

\providecommand{\cdarmc}{
\toolsection{cdarmt32} is a compiler for intermediate code targeting the ARM hardware architecture.
It generates machine code for ARM processors without floating-point extension executing T32 instructions from programs written in intermediate code and stores it in corresponding object files.
It also creates a debugging information file as well as an assembly file containing a listing of the generated machine code.
\debuggingtool
\flowgraph{\resource{intermediate\\code} \ar[r] & \toolbox{cdarmt32} \ar[r] \ar[d] \ar[rd] & \resource{object file} \\ & \resource{assembly\\listing} & \resource{debugging\\information}}
\seeassembly\seearm\seeobject\seecode\seedebugging
}

\providecommand{\cdarmcfpe}{
\toolsection{cdarmt32fpe} is a compiler for intermediate code targeting the ARM hardware architecture.
It generates machine code for ARM processors with floating-point extension executing T32 instructions from programs written in intermediate code and stores it in corresponding object files.
It also creates a debugging information file as well as an assembly file containing a listing of the generated machine code.
\debuggingtool
\flowgraph{\resource{intermediate\\code} \ar[r] & \toolbox{cdarmt32fpe} \ar[r] \ar[d] \ar[rd] & \resource{object file} \\ & \resource{assembly\\listing} & \resource{debugging\\information}}
\seeassembly\seearm\seeobject\seecode\seedebugging
}

\providecommand{\cdavr}{
\toolsection{cdavr} is a compiler for intermediate code targeting the AVR hardware architecture.
It generates machine code for AVR processors from programs written in intermediate code and stores it in corresponding object files.
It also creates a debugging information file as well as an assembly file containing a listing of the generated machine code.
\debuggingtool
\flowgraph{\resource{intermediate\\code} \ar[r] & \toolbox{cdavr} \ar[r] \ar[d] \ar[rd] & \resource{object file} \\ & \resource{assembly\\listing} & \resource{debugging\\information}}
\seeassembly\seeavr\seeobject\seecode\seedebugging
}

\providecommand{\cdavrtt}{
\toolsection{cdavr32} is a compiler for intermediate code targeting the AVR32 hardware architecture.
It generates machine code for AVR32 processors from programs written in intermediate code and stores it in corresponding object files.
It also creates a debugging information file as well as an assembly file containing a listing of the generated machine code.
\debuggingtool
\flowgraph{\resource{intermediate\\code} \ar[r] & \toolbox{cdavr32} \ar[r] \ar[d] \ar[rd] & \resource{object file} \\ & \resource{assembly\\listing} & \resource{debugging\\information}}
\seeassembly\seeavrtt\seeobject\seecode\seedebugging
}

\providecommand{\cdmabk}{
\toolsection{cdm68k} is a compiler for intermediate code targeting the M68000 hardware architecture.
It generates machine code for M68000 processors from programs written in intermediate code and stores it in corresponding object files.
It also creates a debugging information file as well as an assembly file containing a listing of the generated machine code.
\debuggingtool
\flowgraph{\resource{intermediate\\code} \ar[r] & \toolbox{cdm68k} \ar[r] \ar[d] \ar[rd] & \resource{object file} \\ & \resource{assembly\\listing} & \resource{debugging\\information}}
\seeassembly\seemabk\seeobject\seecode\seedebugging
}

\providecommand{\cdmibl}{
\toolsection{cdmibl} is a compiler for intermediate code targeting the MicroBlaze hardware architecture.
It generates machine code for MicroBlaze processors from programs written in intermediate code and stores it in corresponding object files.
It also creates a debugging information file as well as an assembly file containing a listing of the generated machine code.
\debuggingtool
\flowgraph{\resource{intermediate\\code} \ar[r] & \toolbox{cdmibl} \ar[r] \ar[d] \ar[rd] & \resource{object file} \\ & \resource{assembly\\listing} & \resource{debugging\\information}}
\seeassembly\seemibl\seeobject\seecode\seedebugging
}

\providecommand{\cdmipsa}{
\toolsection{cdmips32} is a compiler for intermediate code targeting the MIPS32 hardware architecture.
It generates machine code for MIPS32 processors from programs written in intermediate code and stores it in corresponding object files.
It also creates a debugging information file as well as an assembly file containing a listing of the generated machine code.
\debuggingtool
\flowgraph{\resource{intermediate\\code} \ar[r] & \toolbox{cdmips32} \ar[r] \ar[d] \ar[rd] & \resource{object file} \\ & \resource{assembly\\listing} & \resource{debugging\\information}}
\seeassembly\seemips\seeobject\seecode\seedebugging
}

\providecommand{\cdmipsb}{
\toolsection{cdmips64} is a compiler for intermediate code targeting the MIPS64 hardware architecture.
It generates machine code for MIPS64 processors from programs written in intermediate code and stores it in corresponding object files.
It also creates a debugging information file as well as an assembly file containing a listing of the generated machine code.
\debuggingtool
\flowgraph{\resource{intermediate\\code} \ar[r] & \toolbox{cdmips64} \ar[r] \ar[d] \ar[rd] & \resource{object file} \\ & \resource{assembly\\listing} & \resource{debugging\\information}}
\seeassembly\seemips\seeobject\seecode\seedebugging
}

\providecommand{\cdmmix}{
\toolsection{cdmmix} is a compiler for intermediate code targeting the MMIX hardware architecture.
It generates machine code for MMIX processors from programs written in intermediate code and stores it in corresponding object files.
It also creates a debugging information file as well as an assembly file containing a listing of the generated machine code.
\debuggingtool
\flowgraph{\resource{intermediate\\code} \ar[r] & \toolbox{cdmmix} \ar[r] \ar[d] \ar[rd] & \resource{object file} \\ & \resource{assembly\\listing} & \resource{debugging\\information}}
\seeassembly\seemmix\seeobject\seecode\seedebugging
}

\providecommand{\cdorok}{
\toolsection{cdor1k} is a compiler for intermediate code targeting the OpenRISC 1000 hardware architecture.
It generates machine code for OpenRISC 1000 processors from programs written in intermediate code and stores it in corresponding object files.
It also creates a debugging information file as well as an assembly file containing a listing of the generated machine code.
\debuggingtool
\flowgraph{\resource{intermediate\\code} \ar[r] & \toolbox{cdor1k} \ar[r] \ar[d] \ar[rd] & \resource{object file} \\ & \resource{assembly\\listing} & \resource{debugging\\information}}
\seeassembly\seeorok\seeobject\seecode\seedebugging
}

\providecommand{\cdppca}{
\toolsection{cdppc32} is a compiler for intermediate code targeting the PowerPC hardware architecture.
It generates machine code for PowerPC processors from programs written in intermediate code and stores it in corresponding object files.
The compiler generates machine code for the 32-bit operating mode defined by the PowerPC architecture.
It also creates a debugging information file as well as an assembly file containing a listing of the generated machine code.
\debuggingtool
\flowgraph{\resource{intermediate\\code} \ar[r] & \toolbox{cdppc32} \ar[r] \ar[d] \ar[rd] & \resource{object file} \\ & \resource{assembly\\listing} & \resource{debugging\\information}}
\seeassembly\seeppc\seeobject\seecode\seedebugging
}

\providecommand{\cdppcb}{
\toolsection{cdppc64} is a compiler for intermediate code targeting the PowerPC hardware architecture.
It generates machine code for PowerPC processors from programs written in intermediate code and stores it in corresponding object files.
The compiler generates machine code for the 64-bit operating mode defined by the PowerPC architecture.
It also creates a debugging information file as well as an assembly file containing a listing of the generated machine code.
\debuggingtool
\flowgraph{\resource{intermediate\\code} \ar[r] & \toolbox{cdppc64} \ar[r] \ar[d] \ar[rd] & \resource{object file} \\ & \resource{assembly\\listing} & \resource{debugging\\information}}
\seeassembly\seeppc\seeobject\seecode\seedebugging
}

\providecommand{\cdrisc}{
\toolsection{cdrisc} is a compiler for intermediate code targeting the RISC hardware architecture.
It generates machine code for RISC processors from programs written in intermediate code and stores it in corresponding object files.
It also creates a debugging information file as well as an assembly file containing a listing of the generated machine code.
\debuggingtool
\flowgraph{\resource{intermediate\\code} \ar[r] & \toolbox{cdrisc} \ar[r] \ar[d] \ar[rd] & \resource{object file} \\ & \resource{assembly\\listing} & \resource{debugging\\information}}
\seeassembly\seerisc\seeobject\seecode\seedebugging
}

\providecommand{\cdwasm}{
\toolsection{cdwasm} is a compiler for intermediate code targeting the WebAssembly architecture.
It generates machine code for WebAssembly targets from programs written in intermediate code and stores it in corresponding object files.
It also creates a debugging information file as well as an assembly file containing a listing of the generated machine code.
\debuggingtool
\flowgraph{\resource{intermediate\\code} \ar[r] & \toolbox{cdwasm} \ar[r] \ar[d] \ar[rd] & \resource{object file} \\ & \resource{assembly\\listing} & \resource{debugging\\information}}
\seeassembly\seewasm\seeobject\seecode\seedebugging
}

% C++ tools

\providecommand{\cppprep}{
\toolsection{cppprep} is a preprocessor for the \cpp{} programming language.
It preprocesses source code according to the rules of \cpp{} and writes it to the standard output stream.
Only the macro names \texttt{\_\_DATE\_\_}, \texttt{\_\_FILE\_\_}, \texttt{\_\_LINE\_\_}, and \texttt{\_\_TIME\_\_} are predefined.
\flowgraph{\resource{\cpp{} or other\\source code} \ar[r] & \toolbox{cppprep} \ar[r] & \resource{preprocessed\\source code} \\ & \variable{ECSINCLUDE} \ar[u]}
\seecpp
}

\providecommand{\cppprint}{
\toolsection{cppprint} is a pretty printer for the \cpp{} programming language.
It reformats the source code of \cpp{} programs and writes it to the standard output stream.
\flowgraph{\resource{\cpp{}\\source code} \ar[r] & \toolbox{cppprint} \ar[r] & \resource{reformatted\\source code} \\ & \variable{ECSINCLUDE} \ar[u]}
\seecpp
}

\providecommand{\cppcheck}{
\toolsection{cppcheck} is a syntactic and semantic checker for the \cpp{} programming language.
It just performs syntactic and semantic checks on \cpp{} programs and writes its diagnostic messages to the standard error stream.
\flowgraph{\resource{\cpp{}\\source code} \ar[r] & \toolbox{cppcheck} \ar[r] & \resource{diagnostic\\messages} \\ & \variable{ECSINCLUDE} \ar[u]}
\seecpp
}

\providecommand{\cppdump}{
\toolsection{cppdump} is a serializer for the \cpp{} programming language.
It dumps the complete internal representation of programs written in \cpp{} into an XML document.
\debuggingtool
\flowgraph{\resource{\cpp{}\\source code} \ar[r] & \toolbox{cppdump} \ar[r] & \resource{internal\\representation} \\ & \variable{ECSINCLUDE} \ar[u]}
\seecpp
}

\providecommand{\cpprun}{
\toolsection{cpprun} is an interpreter for the \cpp{} programming language.
It processes and executes programs written in \cpp{}.
The macro \texttt{\_\_run\_\_} is predefined in order to enable programmers to identify this tool while interpreting.
\flowgraph{\resource{\cpp{}\\source code} \ar[r] & \toolbox{cpprun} \ar@/u/[r] & \resource{input/\\output} \ar@/d/[l] \\ & \variable{ECSINCLUDE} \ar[u]}
\seecpp
}

\providecommand{\cppdoc}{
\toolsection{cppdoc} is a generic documentation generator for the \cpp{} programming language.
It processes several \cpp{} source files and assembles all information therein into a generic documentation.
\debuggingtool
\flowgraph{\resource{\cpp{}\\source code} \ar[r] & \toolbox{cppdoc} \ar[r] & \resource{generic\\documentation} \\ & \variable{ECSINCLUDE} \ar[u]}
\seecpp\seedocumentation
}

\providecommand{\cpphtml}{
\toolsection{cpphtml} is an HTML documentation generator for the \cpp{} programming language.
It processes several \cpp{} source files and assembles all information therein into an HTML document.
\flowgraph{\resource{\cpp{}\\source code} \ar[r] & \toolbox{cpphtml} \ar[r] & \resource{HTML\\document} \\ & \variable{ECSINCLUDE} \ar[u]}
\seecpp\seedocumentation
}

\providecommand{\cpplatex}{
\toolsection{cpplatex} is a Latex documentation generator for the \cpp{} programming language.
It processes several \cpp{} source files and assembles all information therein into a Latex document.
\flowgraph{\resource{\cpp{}\\source code} \ar[r] & \toolbox{cpplatex} \ar[r] & \resource{Latex\\document} \\ & \variable{ECSINCLUDE} \ar[u]}
\seecpp\seedocumentation
}

\providecommand{\cppcode}{
\toolsection{cppcode} is an intermediate code generator for the \cpp{} programming language.
It generates intermediate code from programs written in \cpp{} and stores it in corresponding assembly files.
The macro \texttt{\_\_code\_\_} is predefined in order to enable programmers to identify this tool while generating intermediate code.
Programs generated with this tool require additional runtime support that is stored in the \file{cpp\-code\-run} library file.
\debuggingtool
\flowgraph{\resource{\cpp{}\\source code} \ar[r] & \toolbox{cppcode} \ar[r] & \resource{intermediate\\code} \\ & \variable{ECSINCLUDE} \ar[u]}
\seecpp\seeassembly\seecode
}

\providecommand{\cppamda}{
\toolsection{cppamd16} is a compiler for the \cpp{} programming language targeting the AMD64 hardware architecture.
It generates machine code for AMD64 processors from programs written in \cpp{} and stores it in corresponding object files.
The compiler generates machine code for the 16-bit operating mode defined by the AMD64 architecture.
For debugging purposes, it also creates a debugging information file as well as an assembly file containing a listing of the generated machine code.
The macro \texttt{\_\_amd16\_\_} is predefined in order to enable programmers to identify this tool and its target architecture while compiling.
Programs generated with this compiler require additional runtime support that is stored in the \file{cpp\-amd16\-run} library file.
\flowgraph{\resource{\cpp{}\\source code} \ar[r] & \toolbox{cppamd16} \ar[r] \ar[d] \ar[rd] & \resource{object file} \\ \variable{ECSINCLUDE} \ar[ru] & \resource{debugging\\information} & \resource{assembly\\listing}}
\seecpp\seeassembly\seeamd\seeobject\seedebugging
}

\providecommand{\cppamdb}{
\toolsection{cppamd32} is a compiler for the \cpp{} programming language targeting the AMD64 hardware architecture.
It generates machine code for AMD64 processors from programs written in \cpp{} and stores it in corresponding object files.
The compiler generates machine code for the 32-bit operating mode defined by the AMD64 architecture.
For debugging purposes, it also creates a debugging information file as well as an assembly file containing a listing of the generated machine code.
The macro \texttt{\_\_amd32\_\_} is predefined in order to enable programmers to identify this tool and its target architecture while compiling.
Programs generated with this compiler require additional runtime support that is stored in the \file{cpp\-amd32\-run} library file.
\flowgraph{\resource{\cpp{}\\source code} \ar[r] & \toolbox{cppamd32} \ar[r] \ar[d] \ar[rd] & \resource{object file} \\ \variable{ECSINCLUDE} \ar[ru] & \resource{debugging\\information} & \resource{assembly\\listing}}
\seecpp\seeassembly\seeamd\seeobject\seedebugging
}

\providecommand{\cppamdc}{
\toolsection{cppamd64} is a compiler for the \cpp{} programming language targeting the AMD64 hardware architecture.
It generates machine code for AMD64 processors from programs written in \cpp{} and stores it in corresponding object files.
The compiler generates machine code for the 64-bit operating mode defined by the AMD64 architecture.
For debugging purposes, it also creates a debugging information file as well as an assembly file containing a listing of the generated machine code.
The macro \texttt{\_\_amd64\_\_} is predefined in order to enable programmers to identify this tool and its target architecture while compiling.
Programs generated with this compiler require additional runtime support that is stored in the \file{cpp\-amd64\-run} library file.
\flowgraph{\resource{\cpp{}\\source code} \ar[r] & \toolbox{cppamd64} \ar[r] \ar[d] \ar[rd] & \resource{object file} \\ \variable{ECSINCLUDE} \ar[ru] & \resource{debugging\\information} & \resource{assembly\\listing}}
\seecpp\seeassembly\seeamd\seeobject\seedebugging
}

\providecommand{\cpparma}{
\toolsection{cpparma32} is a compiler for the \cpp{} programming language targeting the ARM hardware architecture.
It generates machine code for ARM processors executing A32 instructions from programs written in \cpp{} and stores it in corresponding object files.
For debugging purposes, it also creates a debugging information file as well as an assembly file containing a listing of the generated machine code.
The macro \texttt{\_\_arma32\_\_} is predefined in order to enable programmers to identify this tool and its target architecture while compiling.
Programs generated with this compiler require additional runtime support that is stored in the \file{cpp\-arma32\-run} library file.
\flowgraph{\resource{\cpp{}\\source code} \ar[r] & \toolbox{cpparma32} \ar[r] \ar[d] \ar[rd] & \resource{object file} \\ \variable{ECSINCLUDE} \ar[ru] & \resource{debugging\\information} & \resource{assembly\\listing}}
\seecpp\seeassembly\seearm\seeobject\seedebugging
}

\providecommand{\cpparmb}{
\toolsection{cpparma64} is a compiler for the \cpp{} programming language targeting the ARM hardware architecture.
It generates machine code for ARM processors executing A64 instructions from programs written in \cpp{} and stores it in corresponding object files.
For debugging purposes, it also creates a debugging information file as well as an assembly file containing a listing of the generated machine code.
The macro \texttt{\_\_arma64\_\_} is predefined in order to enable programmers to identify this tool and its target architecture while compiling.
Programs generated with this compiler require additional runtime support that is stored in the \file{cpp\-arma64\-run} library file.
\flowgraph{\resource{\cpp{}\\source code} \ar[r] & \toolbox{cpparma64} \ar[r] \ar[d] \ar[rd] & \resource{object file} \\ \variable{ECSINCLUDE} \ar[ru] & \resource{debugging\\information} & \resource{assembly\\listing}}
\seecpp\seeassembly\seearm\seeobject\seedebugging
}

\providecommand{\cpparmc}{
\toolsection{cpparmt32} is a compiler for the \cpp{} programming language targeting the ARM hardware architecture.
It generates machine code for ARM processors without floating-point extension executing T32 instructions from programs written in \cpp{} and stores it in corresponding object files.
For debugging purposes, it also creates a debugging information file as well as an assembly file containing a listing of the generated machine code.
The macro \texttt{\_\_armt32\_\_} is predefined in order to enable programmers to identify this tool and its target architecture while compiling.
Programs generated with this compiler require additional runtime support that is stored in the \file{cpp\-armt32\-run} library file.
\flowgraph{\resource{\cpp{}\\source code} \ar[r] & \toolbox{cpparmt32} \ar[r] \ar[d] \ar[rd] & \resource{object file} \\ \variable{ECSINCLUDE} \ar[ru] & \resource{debugging\\information} & \resource{assembly\\listing}}
\seecpp\seeassembly\seearm\seeobject\seedebugging
}

\providecommand{\cpparmcfpe}{
\toolsection{cpparmt32fpe} is a compiler for the \cpp{} programming language targeting the ARM hardware architecture.
It generates machine code for ARM processors with floating-point extension executing T32 instructions from programs written in \cpp{} and stores it in corresponding object files.
For debugging purposes, it also creates a debugging information file as well as an assembly file containing a listing of the generated machine code.
The macro \texttt{\_\_armt32fpe\_\_} is predefined in order to enable programmers to identify this tool and its target architecture while compiling.
Programs generated with this compiler require additional runtime support that is stored in the \file{cpp\-armt32\-fpe\-run} library file.
\flowgraph{\resource{\cpp{}\\source code} \ar[r] & \toolbox{cpparmt32fpe} \ar[r] \ar[d] \ar[rd] & \resource{object file} \\ \variable{ECSINCLUDE} \ar[ru] & \resource{debugging\\information} & \resource{assembly\\listing}}
\seecpp\seeassembly\seearm\seeobject\seedebugging
}

\providecommand{\cppavr}{
\toolsection{cppavr} is a compiler for the \cpp{} programming language targeting the AVR hardware architecture.
It generates machine code for AVR processors from programs written in \cpp{} and stores it in corresponding object files.
For debugging purposes, it also creates a debugging information file as well as an assembly file containing a listing of the generated machine code.
The macro \texttt{\_\_avr\_\_} is predefined in order to enable programmers to identify this tool and its target architecture while compiling.
Programs generated with this compiler require additional runtime support that is stored in the \file{cpp\-avr\-run} library file.
\flowgraph{\resource{\cpp{}\\source code} \ar[r] & \toolbox{cppavr} \ar[r] \ar[d] \ar[rd] & \resource{object file} \\ \variable{ECSINCLUDE} \ar[ru] & \resource{debugging\\information} & \resource{assembly\\listing}}
\seecpp\seeassembly\seeavr\seeobject\seedebugging
}

\providecommand{\cppavrtt}{
\toolsection{cppavr32} is a compiler for the \cpp{} programming language targeting the AVR32 hardware architecture.
It generates machine code for AVR32 processors from programs written in \cpp{} and stores it in corresponding object files.
For debugging purposes, it also creates a debugging information file as well as an assembly file containing a listing of the generated machine code.
The macro \texttt{\_\_avr32\_\_} is predefined in order to enable programmers to identify this tool and its target architecture while compiling.
Programs generated with this compiler require additional runtime support that is stored in the \file{cpp\-avr32\-run} library file.
\flowgraph{\resource{\cpp{}\\source code} \ar[r] & \toolbox{cppavr32} \ar[r] \ar[d] \ar[rd] & \resource{object file} \\ \variable{ECSINCLUDE} \ar[ru] & \resource{debugging\\information} & \resource{assembly\\listing}}
\seecpp\seeassembly\seeavrtt\seeobject\seedebugging
}

\providecommand{\cppmabk}{
\toolsection{cppm68k} is a compiler for the \cpp{} programming language targeting the M68000 hardware architecture.
It generates machine code for M68000 processors from programs written in \cpp{} and stores it in corresponding object files.
For debugging purposes, it also creates a debugging information file as well as an assembly file containing a listing of the generated machine code.
The macro \texttt{\_\_m68k\_\_} is predefined in order to enable programmers to identify this tool and its target architecture while compiling.
Programs generated with this compiler require additional runtime support that is stored in the \file{cpp\-m68k\-run} library file.
\flowgraph{\resource{\cpp{}\\source code} \ar[r] & \toolbox{cppm68k} \ar[r] \ar[d] \ar[rd] & \resource{object file} \\ \variable{ECSINCLUDE} \ar[ru] & \resource{debugging\\information} & \resource{assembly\\listing}}
\seecpp\seeassembly\seemabk\seeobject\seedebugging
}

\providecommand{\cppmibl}{
\toolsection{cppmibl} is a compiler for the \cpp{} programming language targeting the MicroBlaze hardware architecture.
It generates machine code for MicroBlaze processors from programs written in \cpp{} and stores it in corresponding object files.
For debugging purposes, it also creates a debugging information file as well as an assembly file containing a listing of the generated machine code.
The macro \texttt{\_\_mibl\_\_} is predefined in order to enable programmers to identify this tool and its target architecture while compiling.
Programs generated with this compiler require additional runtime support that is stored in the \file{cpp\-mibl\-run} library file.
\flowgraph{\resource{\cpp{}\\source code} \ar[r] & \toolbox{cppmibl} \ar[r] \ar[d] \ar[rd] & \resource{object file} \\ \variable{ECSINCLUDE} \ar[ru] & \resource{debugging\\information} & \resource{assembly\\listing}}
\seecpp\seeassembly\seemibl\seeobject\seedebugging
}

\providecommand{\cppmipsa}{
\toolsection{cppmips32} is a compiler for the \cpp{} programming language targeting the MIPS32 hardware architecture.
It generates machine code for MIPS32 processors from programs written in \cpp{} and stores it in corresponding object files.
For debugging purposes, it also creates a debugging information file as well as an assembly file containing a listing of the generated machine code.
The macro \texttt{\_\_mips32\_\_} is predefined in order to enable programmers to identify this tool and its target architecture while compiling.
Programs generated with this compiler require additional runtime support that is stored in the \file{cpp\-mips32\-run} library file.
\flowgraph{\resource{\cpp{}\\source code} \ar[r] & \toolbox{cppmips32} \ar[r] \ar[d] \ar[rd] & \resource{object file} \\ \variable{ECSINCLUDE} \ar[ru] & \resource{debugging\\information} & \resource{assembly\\listing}}
\seecpp\seeassembly\seemips\seeobject\seedebugging
}

\providecommand{\cppmipsb}{
\toolsection{cppmips64} is a compiler for the \cpp{} programming language targeting the MIPS64 hardware architecture.
It generates machine code for MIPS64 processors from programs written in \cpp{} and stores it in corresponding object files.
For debugging purposes, it also creates a debugging information file as well as an assembly file containing a listing of the generated machine code.
The macro \texttt{\_\_mips64\_\_} is predefined in order to enable programmers to identify this tool and its target architecture while compiling.
Programs generated with this compiler require additional runtime support that is stored in the \file{cpp\-mips64\-run} library file.
\flowgraph{\resource{\cpp{}\\source code} \ar[r] & \toolbox{cppmips64} \ar[r] \ar[d] \ar[rd] & \resource{object file} \\ \variable{ECSINCLUDE} \ar[ru] & \resource{debugging\\information} & \resource{assembly\\listing}}
\seecpp\seeassembly\seemips\seeobject\seedebugging
}

\providecommand{\cppmmix}{
\toolsection{cppmmix} is a compiler for the \cpp{} programming language targeting the MMIX hardware architecture.
It generates machine code for MMIX processors from programs written in \cpp{} and stores it in corresponding object files.
For debugging purposes, it also creates a debugging information file as well as an assembly file containing a listing of the generated machine code.
The macro \texttt{\_\_mmix\_\_} is predefined in order to enable programmers to identify this tool and its target architecture while compiling.
Programs generated with this compiler require additional runtime support that is stored in the \file{cpp\-mmix\-run} library file.
\flowgraph{\resource{\cpp{}\\source code} \ar[r] & \toolbox{cppmmix} \ar[r] \ar[d] \ar[rd] & \resource{object file} \\ \variable{ECSINCLUDE} \ar[ru] & \resource{debugging\\information} & \resource{assembly\\listing}}
\seecpp\seeassembly\seemmix\seeobject\seedebugging
}

\providecommand{\cpporok}{
\toolsection{cppor1k} is a compiler for the \cpp{} programming language targeting the OpenRISC 1000 hardware architecture.
It generates machine code for OpenRISC 1000 processors from programs written in \cpp{} and stores it in corresponding object files.
For debugging purposes, it also creates a debugging information file as well as an assembly file containing a listing of the generated machine code.
The macro \texttt{\_\_or1k\_\_} is predefined in order to enable programmers to identify this tool and its target architecture while compiling.
Programs generated with this compiler require additional runtime support that is stored in the \file{cpp\-or1k\-run} library file.
\flowgraph{\resource{\cpp{}\\source code} \ar[r] & \toolbox{cppor1k} \ar[r] \ar[d] \ar[rd] & \resource{object file} \\ \variable{ECSINCLUDE} \ar[ru] & \resource{debugging\\information} & \resource{assembly\\listing}}
\seecpp\seeassembly\seeorok\seeobject\seedebugging
}

\providecommand{\cppppca}{
\toolsection{cppppc32} is a compiler for the \cpp{} programming language targeting the PowerPC hardware architecture.
It generates machine code for PowerPC processors from programs written in \cpp{} and stores it in corresponding object files.
The compiler generates machine code for the 32-bit operating mode defined by the PowerPC architecture.
For debugging purposes, it also creates a debugging information file as well as an assembly file containing a listing of the generated machine code.
The macro \texttt{\_\_ppc32\_\_} is predefined in order to enable programmers to identify this tool and its target architecture while compiling.
Programs generated with this compiler require additional runtime support that is stored in the \file{cpp\-ppc32\-run} library file.
\flowgraph{\resource{\cpp{}\\source code} \ar[r] & \toolbox{cppppc32} \ar[r] \ar[d] \ar[rd] & \resource{object file} \\ \variable{ECSINCLUDE} \ar[ru] & \resource{debugging\\information} & \resource{assembly\\listing}}
\seecpp\seeassembly\seeppc\seeobject\seedebugging
}

\providecommand{\cppppcb}{
\toolsection{cppppc64} is a compiler for the \cpp{} programming language targeting the PowerPC hardware architecture.
It generates machine code for PowerPC processors from programs written in \cpp{} and stores it in corresponding object files.
The compiler generates machine code for the 64-bit operating mode defined by the PowerPC architecture.
For debugging purposes, it also creates a debugging information file as well as an assembly file containing a listing of the generated machine code.
The macro \texttt{\_\_ppc64\_\_} is predefined in order to enable programmers to identify this tool and its target architecture while compiling.
Programs generated with this compiler require additional runtime support that is stored in the \file{cpp\-ppc64\-run} library file.
\flowgraph{\resource{\cpp{}\\source code} \ar[r] & \toolbox{cppppc64} \ar[r] \ar[d] \ar[rd] & \resource{object file} \\ \variable{ECSINCLUDE} \ar[ru] & \resource{debugging\\information} & \resource{assembly\\listing}}
\seecpp\seeassembly\seeppc\seeobject\seedebugging
}

\providecommand{\cpprisc}{
\toolsection{cpprisc} is a compiler for the \cpp{} programming language targeting the RISC hardware architecture.
It generates machine code for RISC processors from programs written in \cpp{} and stores it in corresponding object files.
For debugging purposes, it also creates a debugging information file as well as an assembly file containing a listing of the generated machine code.
The macro \texttt{\_\_risc\_\_} is predefined in order to enable programmers to identify this tool and its target architecture while compiling.
Programs generated with this compiler require additional runtime support that is stored in the \file{cpp\-risc\-run} library file.
\flowgraph{\resource{\cpp{}\\source code} \ar[r] & \toolbox{cpprisc} \ar[r] \ar[d] \ar[rd] & \resource{object file} \\ \variable{ECSINCLUDE} \ar[ru] & \resource{debugging\\information} & \resource{assembly\\listing}}
\seecpp\seeassembly\seerisc\seeobject\seedebugging
}

\providecommand{\cppwasm}{
\toolsection{cppwasm} is a compiler for the \cpp{} programming language targeting the WebAssembly architecture.
It generates machine code for WebAssembly targets from programs written in \cpp{} and stores it in corresponding object files.
For debugging purposes, it also creates a debugging information file as well as an assembly file containing a listing of the generated machine code.
The macro \texttt{\_\_wasm\_\_} is predefined in order to enable programmers to identify this tool and its target architecture while compiling.
Programs generated with this compiler require additional runtime support that is stored in the \file{cpp\-wasm\-run} library file.
\flowgraph{\resource{\cpp{}\\source code} \ar[r] & \toolbox{cppwasm} \ar[r] \ar[d] \ar[rd] & \resource{object file} \\ \variable{ECSINCLUDE} \ar[ru] & \resource{debugging\\information} & \resource{assembly\\listing}}
\seecpp\seeassembly\seewasm\seeobject\seedebugging
}

% FALSE tools

\providecommand{\falprint}{
\toolsection{falprint} is a pretty printer for the FALSE programming language.
It reformats the source code of FALSE programs and writes it to the standard output stream.
\flowgraph{\resource{FALSE\\source code} \ar[r] & \toolbox{falprint} \ar[r] & \resource{reformatted\\source code}}
\seefalse
}

\providecommand{\falcheck}{
\toolsection{falcheck} is a syntactic and semantic checker for the FALSE programming language.
It just performs syntactic and semantic checks on FALSE programs and writes its diagnostic messages to the standard error stream.
\flowgraph{\resource{FALSE\\source code} \ar[r] & \toolbox{falcheck} \ar[r] & \resource{diagnostic\\messages}}
\seefalse
}

\providecommand{\faldump}{
\toolsection{faldump} is a serializer for the FALSE programming language.
It dumps the complete internal representation of programs written in FALSE into an XML document.
\debuggingtool
\flowgraph{\resource{FALSE\\source code} \ar[r] & \toolbox{faldump} \ar[r] & \resource{internal\\representation}}
\seefalse
}

\providecommand{\falrun}{
\toolsection{falrun} is an interpreter for the FALSE programming language.
It processes and executes programs written in FALSE\@.
\flowgraph{\resource{FALSE\\source code} \ar[r] & \toolbox{falrun} \ar@/u/[r] & \resource{input/\\output} \ar@/d/[l]}
\seefalse
}

\providecommand{\falcpp}{
\toolsection{falcpp} is a transpiler for the FALSE programming language.
It translates programs written in FALSE into \cpp{} programs and stores them in corresponding source files.
\flowgraph{\resource{FALSE\\source code} \ar[r] & \toolbox{falcpp} \ar[r] & \resource{\cpp{}\\source file}}
\seefalse\seecpp
}

\providecommand{\falcode}{
\toolsection{falcode} is an intermediate code generator for the FALSE programming language.
It generates intermediate code from programs written in FALSE and stores it in corresponding assembly files.
\debuggingtool
\flowgraph{\resource{FALSE\\source code} \ar[r] & \toolbox{falcode} \ar[r] & \resource{intermediate\\code}}
\seefalse\seeassembly\seecode
}

\providecommand{\falamda}{
\toolsection{falamd16} is a compiler for the FALSE programming language targeting the AMD64 hardware architecture.
It generates machine code for AMD64 processors from programs written in FALSE and stores it in corresponding object files.
The compiler generates machine code for the 16-bit operating mode defined by the AMD64 architecture.
\flowgraph{\resource{FALSE\\source code} \ar[r] & \toolbox{falamd16} \ar[r] & \resource{object file}}
\seefalse\seeamd\seeobject
}

\providecommand{\falamdb}{
\toolsection{falamd32} is a compiler for the FALSE programming language targeting the AMD64 hardware architecture.
It generates machine code for AMD64 processors from programs written in FALSE and stores it in corresponding object files.
The compiler generates machine code for the 32-bit operating mode defined by the AMD64 architecture.
\flowgraph{\resource{FALSE\\source code} \ar[r] & \toolbox{falamd32} \ar[r] & \resource{object file}}
\seefalse\seeamd\seeobject
}

\providecommand{\falamdc}{
\toolsection{falamd64} is a compiler for the FALSE programming language targeting the AMD64 hardware architecture.
It generates machine code for AMD64 processors from programs written in FALSE and stores it in corresponding object files.
The compiler generates machine code for the 64-bit operating mode defined by the AMD64 architecture.
\flowgraph{\resource{FALSE\\source code} \ar[r] & \toolbox{falamd64} \ar[r] & \resource{object file}}
\seefalse\seeamd\seeobject
}

\providecommand{\falarma}{
\toolsection{falarma32} is a compiler for the FALSE programming language targeting the ARM hardware architecture.
It generates machine code for ARM processors executing A32 instructions from programs written in FALSE and stores it in corresponding object files.
\flowgraph{\resource{FALSE\\source code} \ar[r] & \toolbox{falarma32} \ar[r] & \resource{object file}}
\seefalse\seearm\seeobject
}

\providecommand{\falarmb}{
\toolsection{falarma64} is a compiler for the FALSE programming language targeting the ARM hardware architecture.
It generates machine code for ARM processors executing A64 instructions from programs written in FALSE and stores it in corresponding object files.
\flowgraph{\resource{FALSE\\source code} \ar[r] & \toolbox{falarma64} \ar[r] & \resource{object file}}
\seefalse\seearm\seeobject
}

\providecommand{\falarmc}{
\toolsection{falarmt32} is a compiler for the FALSE programming language targeting the ARM hardware architecture.
It generates machine code for ARM processors without floating-point extension executing T32 instructions from programs written in FALSE and stores it in corresponding object files.
\flowgraph{\resource{FALSE\\source code} \ar[r] & \toolbox{falarmt32} \ar[r] & \resource{object file}}
\seefalse\seearm\seeobject
}

\providecommand{\falarmcfpe}{
\toolsection{falarmt32fpe} is a compiler for the FALSE programming language targeting the ARM hardware architecture.
It generates machine code for ARM processors with floating-point extension executing T32 instructions from programs written in FALSE and stores it in corresponding object files.
\flowgraph{\resource{FALSE\\source code} \ar[r] & \toolbox{falarmt32fpe} \ar[r] & \resource{object file}}
\seefalse\seearm\seeobject
}

\providecommand{\falavr}{
\toolsection{falavr} is a compiler for the FALSE programming language targeting the AVR hardware architecture.
It generates machine code for AVR processors from programs written in FALSE and stores it in corresponding object files.
\flowgraph{\resource{FALSE\\source code} \ar[r] & \toolbox{falavr} \ar[r] & \resource{object file}}
\seefalse\seeavr\seeobject
}

\providecommand{\falavrtt}{
\toolsection{falavr32} is a compiler for the FALSE programming language targeting the AVR32 hardware architecture.
It generates machine code for AVR32 processors from programs written in FALSE and stores it in corresponding object files.
\flowgraph{\resource{FALSE\\source code} \ar[r] & \toolbox{falavr32} \ar[r] & \resource{object file}}
\seefalse\seeavrtt\seeobject
}

\providecommand{\falmabk}{
\toolsection{falm68k} is a compiler for the FALSE programming language targeting the M68000 hardware architecture.
It generates machine code for M68000 processors from programs written in FALSE and stores it in corresponding object files.
\flowgraph{\resource{FALSE\\source code} \ar[r] & \toolbox{falm68k} \ar[r] & \resource{object file}}
\seefalse\seemabk\seeobject
}

\providecommand{\falmibl}{
\toolsection{falmibl} is a compiler for the FALSE programming language targeting the MicroBlaze hardware architecture.
It generates machine code for MicroBlaze processors from programs written in FALSE and stores it in corresponding object files.
\flowgraph{\resource{FALSE\\source code} \ar[r] & \toolbox{falmibl} \ar[r] & \resource{object file}}
\seefalse\seemibl\seeobject
}

\providecommand{\falmipsa}{
\toolsection{falmips32} is a compiler for the FALSE programming language targeting the MIPS32 hardware architecture.
It generates machine code for MIPS32 processors from programs written in FALSE and stores it in corresponding object files.
\flowgraph{\resource{FALSE\\source code} \ar[r] & \toolbox{falmips32} \ar[r] & \resource{object file}}
\seefalse\seemips\seeobject
}

\providecommand{\falmipsb}{
\toolsection{falmips64} is a compiler for the FALSE programming language targeting the MIPS64 hardware architecture.
It generates machine code for MIPS64 processors from programs written in FALSE and stores it in corresponding object files.
\flowgraph{\resource{FALSE\\source code} \ar[r] & \toolbox{falmips64} \ar[r] & \resource{object file}}
\seefalse\seemips\seeobject
}

\providecommand{\falmmix}{
\toolsection{falmmix} is a compiler for the FALSE programming language targeting the MMIX hardware architecture.
It generates machine code for MMIX processors from programs written in FALSE and stores it in corresponding object files.
\flowgraph{\resource{FALSE\\source code} \ar[r] & \toolbox{falmmix} \ar[r] & \resource{object file}}
\seefalse\seemmix\seeobject
}

\providecommand{\falorok}{
\toolsection{falor1k} is a compiler for the FALSE programming language targeting the OpenRISC 1000 hardware architecture.
It generates machine code for OpenRISC 1000 processors from programs written in FALSE and stores it in corresponding object files.
\flowgraph{\resource{FALSE\\source code} \ar[r] & \toolbox{falor1k} \ar[r] & \resource{object file}}
\seefalse\seeorok\seeobject
}

\providecommand{\falppca}{
\toolsection{falppc32} is a compiler for the FALSE programming language targeting the PowerPC hardware architecture.
It generates machine code for PowerPC processors from programs written in FALSE and stores it in corresponding object files.
The compiler generates machine code for the 32-bit operating mode defined by the PowerPC architecture.
\flowgraph{\resource{FALSE\\source code} \ar[r] & \toolbox{falppc32} \ar[r] & \resource{object file}}
\seefalse\seeppc\seeobject
}

\providecommand{\falppcb}{
\toolsection{falppc64} is a compiler for the FALSE programming language targeting the PowerPC hardware architecture.
It generates machine code for PowerPC processors from programs written in FALSE and stores it in corresponding object files.
The compiler generates machine code for the 64-bit operating mode defined by the PowerPC architecture.
\flowgraph{\resource{FALSE\\source code} \ar[r] & \toolbox{falppc64} \ar[r] & \resource{object file}}
\seefalse\seeppc\seeobject
}

\providecommand{\falrisc}{
\toolsection{falrisc} is a compiler for the FALSE programming language targeting the RISC hardware architecture.
It generates machine code for RISC processors from programs written in FALSE and stores it in corresponding object files.
\flowgraph{\resource{FALSE\\source code} \ar[r] & \toolbox{falrisc} \ar[r] & \resource{object file}}
\seefalse\seerisc\seeobject
}

\providecommand{\falwasm}{
\toolsection{falwasm} is a compiler for the FALSE programming language targeting the WebAssembly architecture.
It generates machine code for WebAssembly targets from programs written in FALSE and stores it in corresponding object files.
\flowgraph{\resource{FALSE\\source code} \ar[r] & \toolbox{falwasm} \ar[r] & \resource{object file}}
\seefalse\seewasm\seeobject
}

% Oberon tools

\providecommand{\obprint}{
\toolsection{obprint} is a pretty printer for the Oberon programming language.
It reformats the source code of Oberon modules and writes it to the standard output stream.
\flowgraph{\resource{Oberon\\source code} \ar[r] & \toolbox{obprint} \ar[r] & \resource{reformatted\\source code}}
\seeoberon
}

\providecommand{\obcheck}{
\toolsection{obcheck} is a syntactic and semantic checker for the Oberon programming language.
It just performs syntactic and semantic checks on Oberon modules and writes its diagnostic messages to the standard error stream.
In addition, it stores the interface of each module in a symbol file which is required when other modules import the module.
\flowgraph{\resource{Oberon\\source code} \ar[r] & \toolbox{obcheck} \ar[r] \ar@/l/[d] & \resource{diagnostic\\messages} \\ \variable{ECSIMPORT} \ar[ru] & \resource{symbol\\files} \ar@/r/[u]}
\seeoberon
}

\providecommand{\obdump}{
\toolsection{obdump} is a serializer for the Oberon programming language.
It dumps the complete internal representation of modules written in Oberon into an XML document.
\debuggingtool
\flowgraph{\resource{Oberon\\source code} \ar[r] & \toolbox{obdump} \ar[r] \ar@/l/[d] & \resource{internal\\representation} \\ \variable{ECSIMPORT} \ar[ru] & \resource{symbol\\files} \ar@/r/[u]}
\seeoberon
}

\providecommand{\obrun}{
\toolsection{obrun} is an interpreter for the Oberon programming language.
It processes and executes modules written in Oberon.
This tool does neither generate nor process symbol files while interpreting modules.
If a module is imported by another one, its filename has to be named before the other one in the list of command-line arguments.
\flowgraph{\resource{Oberon\\source code} \ar[r] & \toolbox{obrun} \ar@/u/[r] & \resource{input/\\output} \ar@/d/[l]}
\seeoberon
}

\providecommand{\obcpp}{
\toolsection{obcpp} is a transpiler for the Oberon programming language.
It translates programs written in Oberon into \cpp{} programs and stores them in corresponding source and header files.
In addition, it stores the interface of each module in a symbol file which is required when other modules import the module.
The same interface is provided by the generated header file which can be used in other parts of the \cpp{} program.
\flowgraph{\resource{Oberon\\source code} \ar[r] & \toolbox{obcpp} \ar[r] \ar@/l/[d] \ar[rd] & \resource{\cpp{}\\source file} \\ \variable{ECSIMPORT} \ar[ru] & \resource{symbol\\files} \ar@/r/[u] & \resource{\cpp{}\\header file}}
\seeoberon\seecpp
}

\providecommand{\obdoc}{
\toolsection{obdoc} is a generic documentation generator for the Oberon programming language.
It processes several Oberon modules and assembles all information therein into a generic documentation.
In addition, it stores the interface of each module in a symbol file which is required when other modules import the module.
\debuggingtool
\flowgraph{\resource{Oberon\\source code} \ar[r] & \toolbox{obdoc} \ar[r] \ar@/l/[d] & \resource{generic\\documentation} \\ \variable{ECSIMPORT} \ar[ru] & \resource{symbol\\files} \ar@/r/[u]}
\seeoberon\seedocumentation
}

\providecommand{\obhtml}{
\toolsection{obhtml} is an HTML documentation generator for the Oberon programming language.
It processes several Oberon modules and assembles all information therein into an HTML document.
In addition, it stores the interface of each module in a symbol file which is required when other modules import the module.
\flowgraph{\resource{Oberon\\source code} \ar[r] & \toolbox{obhtml} \ar[r] \ar@/l/[d] & \resource{HTML\\document} \\ \variable{ECSIMPORT} \ar[ru] & \resource{symbol\\files} \ar@/r/[u]}
\seeoberon\seedocumentation
}

\providecommand{\oblatex}{
\toolsection{oblatex} is a Latex documentation generator for the Oberon programming language.
It processes several Oberon modules and assembles all information therein into a Latex document.
In addition, it stores the interface of each module in a symbol file which is required when other modules import the module.
\flowgraph{\resource{Oberon\\source code} \ar[r] & \toolbox{oblatex} \ar[r] \ar@/l/[d] & \resource{Latex\\document} \\ \variable{ECSIMPORT} \ar[ru] & \resource{symbol\\files} \ar@/r/[u]}
\seeoberon\seedocumentation
}

\providecommand{\obcode}{
\toolsection{obcode} is an intermediate code generator for the Oberon programming language.
It generates intermediate code from modules written in Oberon and stores it in corresponding assembly files.
In addition, it stores the interface of each module in a symbol file which is required when other modules import the module.
Programs generated with this tool require additional runtime support that is stored in the \file{ob\-code\-run} library file.
\debuggingtool
\flowgraph{\resource{Oberon\\source code} \ar[r] & \toolbox{obcode} \ar[r] \ar@/l/[d] & \resource{intermediate\\code} \\ \variable{ECSIMPORT} \ar[ru] & \resource{symbol\\files} \ar@/r/[u]}
\seeoberon\seeassembly\seecode
}

\providecommand{\obamda}{
\toolsection{obamd16} is a compiler for the Oberon programming language targeting the AMD64 hardware architecture.
It generates machine code for AMD64 processors from modules written in Oberon and stores it in corresponding object files.
The compiler generates machine code for the 16-bit operating mode defined by the AMD64 architecture.
For debugging purposes, it also creates a debugging information file as well as an assembly file containing a listing of the generated machine code.
In addition, it stores the interface of each module in a symbol file which is required when other modules import the module.
Programs generated with this compiler require additional runtime support that is stored in the \file{ob\-amd16\-run} library file.
\flowgraph{\resource{Oberon\\source code} \ar[r] & \toolbox{obamd16} \ar[r] \ar@/l/[d] \ar[rd] & \resource{object file} \\ \variable{ECSIMPORT} \ar[ru] & \resource{symbol\\files} \ar@/r/[u] & \resource{debugging\\information}}
\seeoberon\seeassembly\seeamd\seeobject\seedebugging
}

\providecommand{\obamdb}{
\toolsection{obamd32} is a compiler for the Oberon programming language targeting the AMD64 hardware architecture.
It generates machine code for AMD64 processors from modules written in Oberon and stores it in corresponding object files.
The compiler generates machine code for the 32-bit operating mode defined by the AMD64 architecture.
For debugging purposes, it also creates a debugging information file as well as an assembly file containing a listing of the generated machine code.
In addition, it stores the interface of each module in a symbol file which is required when other modules import the module.
Programs generated with this compiler require additional runtime support that is stored in the \file{ob\-amd32\-run} library file.
\flowgraph{\resource{Oberon\\source code} \ar[r] & \toolbox{obamd32} \ar[r] \ar@/l/[d] \ar[rd] & \resource{object file} \\ \variable{ECSIMPORT} \ar[ru] & \resource{symbol\\files} \ar@/r/[u] & \resource{debugging\\information}}
\seeoberon\seeassembly\seeamd\seeobject\seedebugging
}

\providecommand{\obamdc}{
\toolsection{obamd64} is a compiler for the Oberon programming language targeting the AMD64 hardware architecture.
It generates machine code for AMD64 processors from modules written in Oberon and stores it in corresponding object files.
The compiler generates machine code for the 64-bit operating mode defined by the AMD64 architecture.
For debugging purposes, it also creates a debugging information file as well as an assembly file containing a listing of the generated machine code.
In addition, it stores the interface of each module in a symbol file which is required when other modules import the module.
Programs generated with this compiler require additional runtime support that is stored in the \file{ob\-amd64\-run} library file.
\flowgraph{\resource{Oberon\\source code} \ar[r] & \toolbox{obamd64} \ar[r] \ar@/l/[d] \ar[rd] & \resource{object file} \\ \variable{ECSIMPORT} \ar[ru] & \resource{symbol\\files} \ar@/r/[u] & \resource{debugging\\information}}
\seeoberon\seeassembly\seeamd\seeobject\seedebugging
}

\providecommand{\obarma}{
\toolsection{obarma32} is a compiler for the Oberon programming language targeting the ARM hardware architecture.
It generates machine code for ARM processors executing A32 instructions from modules written in Oberon and stores it in corresponding object files.
For debugging purposes, it also creates a debugging information file as well as an assembly file containing a listing of the generated machine code.
In addition, it stores the interface of each module in a symbol file which is required when other modules import the module.
Programs generated with this compiler require additional runtime support that is stored in the \file{ob\-arma32\-run} library file.
\flowgraph{\resource{Oberon\\source code} \ar[r] & \toolbox{obarma32} \ar[r] \ar@/l/[d] \ar[rd] & \resource{object file} \\ \variable{ECSIMPORT} \ar[ru] & \resource{symbol\\files} \ar@/r/[u] & \resource{debugging\\information}}
\seeoberon\seeassembly\seearm\seeobject\seedebugging
}

\providecommand{\obarmb}{
\toolsection{obarma64} is a compiler for the Oberon programming language targeting the ARM hardware architecture.
It generates machine code for ARM processors executing A64 instructions from modules written in Oberon and stores it in corresponding object files.
For debugging purposes, it also creates a debugging information file as well as an assembly file containing a listing of the generated machine code.
In addition, it stores the interface of each module in a symbol file which is required when other modules import the module.
Programs generated with this compiler require additional runtime support that is stored in the \file{ob\-arma64\-run} library file.
\flowgraph{\resource{Oberon\\source code} \ar[r] & \toolbox{obarma64} \ar[r] \ar@/l/[d] \ar[rd] & \resource{object file} \\ \variable{ECSIMPORT} \ar[ru] & \resource{symbol\\files} \ar@/r/[u] & \resource{debugging\\information}}
\seeoberon\seeassembly\seearm\seeobject\seedebugging
}

\providecommand{\obarmc}{
\toolsection{obarmt32} is a compiler for the Oberon programming language targeting the ARM hardware architecture.
It generates machine code for ARM processors without floating-point extension executing T32 instructions from modules written in Oberon and stores it in corresponding object files.
For debugging purposes, it also creates a debugging information file as well as an assembly file containing a listing of the generated machine code.
In addition, it stores the interface of each module in a symbol file which is required when other modules import the module.
Programs generated with this compiler require additional runtime support that is stored in the \file{ob\-armt32\-run} library file.
\flowgraph{\resource{Oberon\\source code} \ar[r] & \toolbox{obarmt32} \ar[r] \ar@/l/[d] \ar[rd] & \resource{object file} \\ \variable{ECSIMPORT} \ar[ru] & \resource{symbol\\files} \ar@/r/[u] & \resource{debugging\\information}}
\seeoberon\seeassembly\seearm\seeobject\seedebugging
}

\providecommand{\obarmcfpe}{
\toolsection{obarmt32fpe} is a compiler for the Oberon programming language targeting the ARM hardware architecture.
It generates machine code for ARM processors with floating-point extension executing T32 instructions from modules written in Oberon and stores it in corresponding object files.
For debugging purposes, it also creates a debugging information file as well as an assembly file containing a listing of the generated machine code.
In addition, it stores the interface of each module in a symbol file which is required when other modules import the module.
Programs generated with this compiler require additional runtime support that is stored in the \file{ob\-armt32\-fpe\-run} library file.
\flowgraph{\resource{Oberon\\source code} \ar[r] & \toolbox{obarmt32fpe} \ar[r] \ar@/l/[d] \ar[rd] & \resource{object file} \\ \variable{ECSIMPORT} \ar[ru] & \resource{symbol\\files} \ar@/r/[u] & \resource{debugging\\information}}
\seeoberon\seeassembly\seearm\seeobject\seedebugging
}

\providecommand{\obavr}{
\toolsection{obavr} is a compiler for the Oberon programming language targeting the AVR hardware architecture.
It generates machine code for AVR processors from modules written in Oberon and stores it in corresponding object files.
For debugging purposes, it also creates a debugging information file as well as an assembly file containing a listing of the generated machine code.
In addition, it stores the interface of each module in a symbol file which is required when other modules import the module.
Programs generated with this compiler require additional runtime support that is stored in the \file{ob\-avr\-run} library file.
\flowgraph{\resource{Oberon\\source code} \ar[r] & \toolbox{obavr} \ar[r] \ar@/l/[d] \ar[rd] & \resource{object file} \\ \variable{ECSIMPORT} \ar[ru] & \resource{symbol\\files} \ar@/r/[u] & \resource{debugging\\information}}
\seeoberon\seeassembly\seeavr\seeobject\seedebugging
}

\providecommand{\obavrtt}{
\toolsection{obavr32} is a compiler for the Oberon programming language targeting the AVR32 hardware architecture.
It generates machine code for AVR32 processors from modules written in Oberon and stores it in corresponding object files.
For debugging purposes, it also creates a debugging information file as well as an assembly file containing a listing of the generated machine code.
In addition, it stores the interface of each module in a symbol file which is required when other modules import the module.
Programs generated with this compiler require additional runtime support that is stored in the \file{ob\-avr32\-run} library file.
\flowgraph{\resource{Oberon\\source code} \ar[r] & \toolbox{obavr32} \ar[r] \ar@/l/[d] \ar[rd] & \resource{object file} \\ \variable{ECSIMPORT} \ar[ru] & \resource{symbol\\files} \ar@/r/[u] & \resource{debugging\\information}}
\seeoberon\seeassembly\seeavrtt\seeobject\seedebugging
}

\providecommand{\obmabk}{
\toolsection{obm68k} is a compiler for the Oberon programming language targeting the M68000 hardware architecture.
It generates machine code for M68000 processors from modules written in Oberon and stores it in corresponding object files.
For debugging purposes, it also creates a debugging information file as well as an assembly file containing a listing of the generated machine code.
In addition, it stores the interface of each module in a symbol file which is required when other modules import the module.
Programs generated with this compiler require additional runtime support that is stored in the \file{ob\-m68k\-run} library file.
\flowgraph{\resource{Oberon\\source code} \ar[r] & \toolbox{obm68k} \ar[r] \ar@/l/[d] \ar[rd] & \resource{object file} \\ \variable{ECSIMPORT} \ar[ru] & \resource{symbol\\files} \ar@/r/[u] & \resource{debugging\\information}}
\seeoberon\seeassembly\seemabk\seeobject\seedebugging
}

\providecommand{\obmibl}{
\toolsection{obmibl} is a compiler for the Oberon programming language targeting the MicroBlaze hardware architecture.
It generates machine code for MicroBlaze processors from modules written in Oberon and stores it in corresponding object files.
For debugging purposes, it also creates a debugging information file as well as an assembly file containing a listing of the generated machine code.
In addition, it stores the interface of each module in a symbol file which is required when other modules import the module.
Programs generated with this compiler require additional runtime support that is stored in the \file{ob\-mibl\-run} library file.
\flowgraph{\resource{Oberon\\source code} \ar[r] & \toolbox{obmibl} \ar[r] \ar@/l/[d] \ar[rd] & \resource{object file} \\ \variable{ECSIMPORT} \ar[ru] & \resource{symbol\\files} \ar@/r/[u] & \resource{debugging\\information}}
\seeoberon\seeassembly\seemibl\seeobject\seedebugging
}

\providecommand{\obmipsa}{
\toolsection{obmips32} is a compiler for the Oberon programming language targeting the MIPS32 hardware architecture.
It generates machine code for MIPS32 processors from modules written in Oberon and stores it in corresponding object files.
For debugging purposes, it also creates a debugging information file as well as an assembly file containing a listing of the generated machine code.
In addition, it stores the interface of each module in a symbol file which is required when other modules import the module.
Programs generated with this compiler require additional runtime support that is stored in the \file{ob\-mips32\-run} library file.
\flowgraph{\resource{Oberon\\source code} \ar[r] & \toolbox{obmips32} \ar[r] \ar@/l/[d] \ar[rd] & \resource{object file} \\ \variable{ECSIMPORT} \ar[ru] & \resource{symbol\\files} \ar@/r/[u] & \resource{debugging\\information}}
\seeoberon\seeassembly\seemips\seeobject\seedebugging
}

\providecommand{\obmipsb}{
\toolsection{obmips64} is a compiler for the Oberon programming language targeting the MIPS64 hardware architecture.
It generates machine code for MIPS64 processors from modules written in Oberon and stores it in corresponding object files.
For debugging purposes, it also creates a debugging information file as well as an assembly file containing a listing of the generated machine code.
In addition, it stores the interface of each module in a symbol file which is required when other modules import the module.
Programs generated with this compiler require additional runtime support that is stored in the \file{ob\-mips64\-run} library file.
\flowgraph{\resource{Oberon\\source code} \ar[r] & \toolbox{obmips64} \ar[r] \ar@/l/[d] \ar[rd] & \resource{object file} \\ \variable{ECSIMPORT} \ar[ru] & \resource{symbol\\files} \ar@/r/[u] & \resource{debugging\\information}}
\seeoberon\seeassembly\seemips\seeobject\seedebugging
}

\providecommand{\obmmix}{
\toolsection{obmmix} is a compiler for the Oberon programming language targeting the MMIX hardware architecture.
It generates machine code for MMIX processors from modules written in Oberon and stores it in corresponding object files.
For debugging purposes, it also creates a debugging information file as well as an assembly file containing a listing of the generated machine code.
In addition, it stores the interface of each module in a symbol file which is required when other modules import the module.
Programs generated with this compiler require additional runtime support that is stored in the \file{ob\-mmix\-run} library file.
\flowgraph{\resource{Oberon\\source code} \ar[r] & \toolbox{obmmix} \ar[r] \ar@/l/[d] \ar[rd] & \resource{object file} \\ \variable{ECSIMPORT} \ar[ru] & \resource{symbol\\files} \ar@/r/[u] & \resource{debugging\\information}}
\seeoberon\seeassembly\seemmix\seeobject\seedebugging
}

\providecommand{\oborok}{
\toolsection{obor1k} is a compiler for the Oberon programming language targeting the OpenRISC 1000 hardware architecture.
It generates machine code for OpenRISC 1000 processors from modules written in Oberon and stores it in corresponding object files.
For debugging purposes, it also creates a debugging information file as well as an assembly file containing a listing of the generated machine code.
In addition, it stores the interface of each module in a symbol file which is required when other modules import the module.
Programs generated with this compiler require additional runtime support that is stored in the \file{ob\-or1k\-run} library file.
\flowgraph{\resource{Oberon\\source code} \ar[r] & \toolbox{obor1k} \ar[r] \ar@/l/[d] \ar[rd] & \resource{object file} \\ \variable{ECSIMPORT} \ar[ru] & \resource{symbol\\files} \ar@/r/[u] & \resource{debugging\\information}}
\seeoberon\seeassembly\seeorok\seeobject\seedebugging
}

\providecommand{\obppca}{
\toolsection{obppc32} is a compiler for the Oberon programming language targeting the PowerPC hardware architecture.
It generates machine code for PowerPC processors from modules written in Oberon and stores it in corresponding object files.
The compiler generates machine code for the 32-bit operating mode defined by the PowerPC architecture.
For debugging purposes, it also creates a debugging information file as well as an assembly file containing a listing of the generated machine code.
In addition, it stores the interface of each module in a symbol file which is required when other modules import the module.
Programs generated with this compiler require additional runtime support that is stored in the \file{ob\-ppc32\-run} library file.
\flowgraph{\resource{Oberon\\source code} \ar[r] & \toolbox{obppc32} \ar[r] \ar@/l/[d] \ar[rd] & \resource{object file} \\ \variable{ECSIMPORT} \ar[ru] & \resource{symbol\\files} \ar@/r/[u] & \resource{debugging\\information}}
\seeoberon\seeassembly\seeppc\seeobject\seedebugging
}

\providecommand{\obppcb}{
\toolsection{obppc64} is a compiler for the Oberon programming language targeting the PowerPC hardware architecture.
It generates machine code for PowerPC processors from modules written in Oberon and stores it in corresponding object files.
The compiler generates machine code for the 64-bit operating mode defined by the PowerPC architecture.
For debugging purposes, it also creates a debugging information file as well as an assembly file containing a listing of the generated machine code.
In addition, it stores the interface of each module in a symbol file which is required when other modules import the module.
Programs generated with this compiler require additional runtime support that is stored in the \file{ob\-ppc64\-run} library file.
\flowgraph{\resource{Oberon\\source code} \ar[r] & \toolbox{obppc64} \ar[r] \ar@/l/[d] \ar[rd] & \resource{object file} \\ \variable{ECSIMPORT} \ar[ru] & \resource{symbol\\files} \ar@/r/[u] & \resource{debugging\\information}}
\seeoberon\seeassembly\seeppc\seeobject\seedebugging
}

\providecommand{\obrisc}{
\toolsection{obrisc} is a compiler for the Oberon programming language targeting the RISC hardware architecture.
It generates machine code for RISC processors from modules written in Oberon and stores it in corresponding object files.
For debugging purposes, it also creates a debugging information file as well as an assembly file containing a listing of the generated machine code.
In addition, it stores the interface of each module in a symbol file which is required when other modules import the module.
Programs generated with this compiler require additional runtime support that is stored in the \file{ob\-risc\-run} library file.
\flowgraph{\resource{Oberon\\source code} \ar[r] & \toolbox{obrisc} \ar[r] \ar@/l/[d] \ar[rd] & \resource{object file} \\ \variable{ECSIMPORT} \ar[ru] & \resource{symbol\\files} \ar@/r/[u] & \resource{debugging\\information}}
\seeoberon\seeassembly\seerisc\seeobject\seedebugging
}

\providecommand{\obwasm}{
\toolsection{obwasm} is a compiler for the Oberon programming language targeting the WebAssembly architecture.
It generates machine code for WebAssembly targets from modules written in Oberon and stores it in corresponding object files.
For debugging purposes, it also creates a debugging information file as well as an assembly file containing a listing of the generated machine code.
In addition, it stores the interface of each module in a symbol file which is required when other modules import the module.
Programs generated with this compiler require additional runtime support that is stored in the \file{ob\-wasm\-run} library file.
\flowgraph{\resource{Oberon\\source code} \ar[r] & \toolbox{obwasm} \ar[r] \ar@/l/[d] \ar[rd] & \resource{object file} \\ \variable{ECSIMPORT} \ar[ru] & \resource{symbol\\files} \ar@/r/[u] & \resource{debugging\\information}}
\seeoberon\seeassembly\seewasm\seeobject\seedebugging
}

% converter tools

\providecommand{\dbgdwarf}{
\toolsection{dbgdwarf} is a DWARF debugging information converter tool.
It converts debugging information into the DWARF debugging data format and stores it in corresponding object files~\cite{dwarffile}.
The resulting debugging object files can be combined with runtime support that creates Executable and Linking Format (ELF) files~\cite{elffile}.
\flowgraph{\resource{debugging\\information} \ar[r] & \toolbox{dbgdwarf} \ar[r] & \resource{debugging\\object file}}
\seeobject\seedebugging
}

% assembler tools

\providecommand{\asmprint}{
\toolsection{asmprint} is a pretty printer for generic assembly code.
It reformats generic assembly code and writes it to the standard output stream.
\flowgraph{\resource{generic assembly\\source code} \ar[r] & \toolbox{asmprint} \ar[r] & \resource{reformatted\\source code}}
\seeassembly
}

\providecommand{\amdaasm}{
\toolsection{amd16asm} is an assembler for the AMD64 hardware architecture.
It translates assembly code into machine code for AMD64 processors and stores it in corresponding object files.
By default, the assembler generates machine code for the 16-bit operating mode defined by the AMD64 architecture.
\flowgraph{\resource{AMD16 assembly\\source code} \ar[r] & \toolbox{amd16asm} \ar[r] & \resource{object file}}
\seeassembly\seeamd\seeobject
}

\providecommand{\amdadism}{
\toolsection{amd16dism} is a disassembler for the AMD64 hardware architecture.
It translates machine code from object files targeting AMD64 processors into assembly code and writes it to the standard output stream.
It assumes that the machine code was generated for the 16-bit operating mode defined by the AMD64 architecture.
\flowgraph{\resource{object file} \ar[r] & \toolbox{amd16dism} \ar[r] & \resource{disassembly\\listing}}
\seeassembly\seeamd\seeobject
}

\providecommand{\amdbasm}{
\toolsection{amd32asm} is an assembler for the AMD64 hardware architecture.
It translates assembly code into machine code for AMD64 processors and stores it in corresponding object files.
By default, the assembler generates machine code for the 32-bit operating mode defined by the AMD64 architecture.
\flowgraph{\resource{AMD32 assembly\\source code} \ar[r] & \toolbox{amd32asm} \ar[r] & \resource{object file}}
\seeassembly\seeamd\seeobject
}

\providecommand{\amdbdism}{
\toolsection{amd32dism} is a disassembler for the AMD64 hardware architecture.
It translates machine code from object files targeting AMD64 processors into assembly code and writes it to the standard output stream.
It assumes that the machine code was generated for the 32-bit operating mode defined by the AMD64 architecture.
\flowgraph{\resource{object file} \ar[r] & \toolbox{amd32dism} \ar[r] & \resource{disassembly\\listing}}
\seeassembly\seeamd\seeobject
}

\providecommand{\amdcasm}{
\toolsection{amd64asm} is an assembler for the AMD64 hardware architecture.
It translates assembly code into machine code for AMD64 processors and stores it in corresponding object files.
By default, the assembler generates machine code for the 64-bit operating mode defined by the AMD64 architecture.
\flowgraph{\resource{AMD64 assembly\\source code} \ar[r] & \toolbox{amd64asm} \ar[r] & \resource{object file}}
\seeassembly\seeamd\seeobject
}

\providecommand{\amdcdism}{
\toolsection{amd64dism} is a disassembler for the AMD64 hardware architecture.
It translates machine code from object files targeting AMD64 processors into assembly code and writes it to the standard output stream.
It assumes that the machine code was generated for the 64-bit operating mode defined by the AMD64 architecture.
\flowgraph{\resource{object file} \ar[r] & \toolbox{amd64dism} \ar[r] & \resource{disassembly\\listing}}
\seeassembly\seeamd\seeobject
}

\providecommand{\armaasm}{
\toolsection{arma32asm} is an assembler for the ARM hardware architecture.
It translates assembly code into machine code for ARM processors executing A32 instructions and stores it in corresponding object files.
\flowgraph{\resource{ARM A32 assembly\\source code} \ar[r] & \toolbox{arma32asm} \ar[r] & \resource{object file}}
\seeassembly\seearm\seeobject
}

\providecommand{\armadism}{
\toolsection{arma32dism} is a disassembler for the ARM hardware architecture.
It translates machine code from object files targeting ARM processors executing A32 instructions into assembly code and writes it to the standard output stream.
\flowgraph{\resource{object file} \ar[r] & \toolbox{arma32dism} \ar[r] & \resource{disassembly\\listing}}
\seeassembly\seearm\seeobject
}

\providecommand{\armbasm}{
\toolsection{arma64asm} is an assembler for the ARM hardware architecture.
It translates assembly code into machine code for ARM processors executing A64 instructions and stores it in corresponding object files.
\flowgraph{\resource{ARM A64 assembly\\source code} \ar[r] & \toolbox{arma64asm} \ar[r] & \resource{object file}}
\seeassembly\seearm\seeobject
}

\providecommand{\armbdism}{
\toolsection{arma64dism} is a disassembler for the ARM hardware architecture.
It translates machine code from object files targeting ARM processors executing A64 instructions into assembly code and writes it to the standard output stream.
\flowgraph{\resource{object file} \ar[r] & \toolbox{arma64dism} \ar[r] & \resource{disassembly\\listing}}
\seeassembly\seearm\seeobject
}

\providecommand{\armcasm}{
\toolsection{armt32asm} is an assembler for the ARM hardware architecture.
It translates assembly code into machine code for ARM processors executing T32 instructions and stores it in corresponding object files.
\flowgraph{\resource{ARM T32 assembly\\source code} \ar[r] & \toolbox{armt32asm} \ar[r] & \resource{object file}}
\seeassembly\seearm\seeobject
}

\providecommand{\armcdism}{
\toolsection{armt32dism} is a disassembler for the ARM hardware architecture.
It translates machine code from object files targeting ARM processors executing T32 instructions into assembly code and writes it to the standard output stream.
\flowgraph{\resource{object file} \ar[r] & \toolbox{armt32dism} \ar[r] & \resource{disassembly\\listing}}
\seeassembly\seearm\seeobject
}

\providecommand{\avrasm}{
\toolsection{avrasm} is an assembler for the AVR hardware architecture.
It translates assembly code into machine code for AVR processors and stores it in corresponding object files.
The identifiers \texttt{RXL}, \texttt{RXH}, \texttt{RYL}, \texttt{RYH}, \texttt{RZL}, and \texttt{RZH} are predefined and name the corresponding registers.
The identifiers \texttt{SPL} and \texttt{SPH} are also predefined and evaluate to the address of the corresponding registers.
\flowgraph{\resource{AVR assembly\\source code} \ar[r] & \toolbox{avrasm} \ar[r] & \resource{object file}}
\seeassembly\seeavr\seeobject
}

\providecommand{\avrdism}{
\toolsection{avrdism} is a disassembler for the AVR hardware architecture.
It translates machine code from object files targeting AVR processors into assembly code and writes it to the standard output stream.
\flowgraph{\resource{object file} \ar[r] & \toolbox{avrdism} \ar[r] & \resource{disassembly\\listing}}
\seeassembly\seeavr\seeobject
}

\providecommand{\avrttasm}{
\toolsection{avr32asm} is an assembler for the AVR32 hardware architecture.
It translates assembly code into machine code for AVR32 processors and stores it in corresponding object files.
\flowgraph{\resource{AVR32 assembly\\source code} \ar[r] & \toolbox{avr32asm} \ar[r] & \resource{object file}}
\seeassembly\seeavrtt\seeobject
}

\providecommand{\avrttdism}{
\toolsection{avr32dism} is a disassembler for the AVR32 hardware architecture.
It translates machine code from object files targeting AVR32 processors into assembly code and writes it to the standard output stream.
\flowgraph{\resource{object file} \ar[r] & \toolbox{avr32dism} \ar[r] & \resource{disassembly\\listing}}
\seeassembly\seeavrtt\seeobject
}

\providecommand{\mabkasm}{
\toolsection{m68kasm} is an assembler for the M68000 hardware architecture.
It translates assembly code into machine code for M68000 processors and stores it in corresponding object files.
\flowgraph{\resource{68000 assembly\\source code} \ar[r] & \toolbox{m68kasm} \ar[r] & \resource{object file}}
\seeassembly\seemabk\seeobject
}

\providecommand{\mabkdism}{
\toolsection{m68kdism} is a disassembler for the M68000 hardware architecture.
It translates machine code from object files targeting M68000 processors into assembly code and writes it to the standard output stream.
\flowgraph{\resource{object file} \ar[r] & \toolbox{m68kdism} \ar[r] & \resource{disassembly\\listing}}
\seeassembly\seemabk\seeobject
}

\providecommand{\miblasm}{
\toolsection{miblasm} is an assembler for the MicroBlaze hardware architecture.
It translates assembly code into machine code for MicroBlaze processors and stores it in corresponding object files.
\flowgraph{\resource{MicroBlaze assembly\\source code} \ar[r] & \toolbox{miblasm} \ar[r] & \resource{object file}}
\seeassembly\seemibl\seeobject
}

\providecommand{\mibldism}{
\toolsection{mibldism} is a disassembler for the MicroBlaze hardware architecture.
It translates machine code from object files targeting MicroBlaze processors into assembly code and writes it to the standard output stream.
\flowgraph{\resource{object file} \ar[r] & \toolbox{mibldism} \ar[r] & \resource{disassembly\\listing}}
\seeassembly\seemibl\seeobject
}

\providecommand{\mipsaasm}{
\toolsection{mips32asm} is an assembler for the MIPS32 hardware architecture.
It translates assembly code into machine code for MIPS32 processors and stores it in corresponding object files.
\flowgraph{\resource{MIPS32 assembly\\source code} \ar[r] & \toolbox{mips32asm} \ar[r] & \resource{object file}}
\seeassembly\seemips\seeobject
}

\providecommand{\mipsadism}{
\toolsection{mips32dism} is a disassembler for the MIPS32 hardware architecture.
It translates machine code from object files targeting MIPS32 processors into assembly code and writes it to the standard output stream.
\flowgraph{\resource{object file} \ar[r] & \toolbox{mips32dism} \ar[r] & \resource{disassembly\\listing}}
\seeassembly\seemips\seeobject
}

\providecommand{\mipsbasm}{
\toolsection{mips64asm} is an assembler for the MIPS64 hardware architecture.
It translates assembly code into machine code for MIPS64 processors and stores it in corresponding object files.
\flowgraph{\resource{MIPS64 assembly\\source code} \ar[r] & \toolbox{mips64asm} \ar[r] & \resource{object file}}
\seeassembly\seemips\seeobject
}

\providecommand{\mipsbdism}{
\toolsection{mips64dism} is a disassembler for the MIPS64 hardware architecture.
It translates machine code from object files targeting MIPS64 processors into assembly code and writes it to the standard output stream.
\flowgraph{\resource{object file} \ar[r] & \toolbox{mips64dism} \ar[r] & \resource{disassembly\\listing}}
\seeassembly\seemips\seeobject
}

\providecommand{\mmixasm}{
\toolsection{mmixasm} is an assembler for the MMIX hardware architecture.
It translates assembly code into machine code for MMIX processors and stores it in corresponding object files.
The names of all special registers are predefined and evaluate to the corresponding number.
\flowgraph{\resource{MMIX assembly\\source code} \ar[r] & \toolbox{mmixasm} \ar[r] & \resource{object file}}
\seeassembly\seemmix\seeobject
}

\providecommand{\mmixdism}{
\toolsection{mmixdism} is a disassembler for the MMIX hardware architecture.
It translates machine code from object files targeting MMIX processors into assembly code and writes it to the standard output stream.
\flowgraph{\resource{object file} \ar[r] & \toolbox{mmixdism} \ar[r] & \resource{disassembly\\listing}}
\seeassembly\seemmix\seeobject
}

\providecommand{\orokasm}{
\toolsection{or1kasm} is an assembler for the OpenRISC 1000 hardware architecture.
It translates assembly code into machine code for OpenRISC 1000 processors and stores it in corresponding object files.
\flowgraph{\resource{OpenRISC 1000 assembly\\source code} \ar[r] & \toolbox{or1kasm} \ar[r] & \resource{object file}}
\seeassembly\seeorok\seeobject
}

\providecommand{\orokdism}{
\toolsection{or1kdism} is a disassembler for the OpenRISC 1000 hardware architecture.
It translates machine code from object files targeting OpenRISC 1000 processors into assembly code and writes it to the standard output stream.
\flowgraph{\resource{object file} \ar[r] & \toolbox{or1kdism} \ar[r] & \resource{disassembly\\listing}}
\seeassembly\seeorok\seeobject
}

\providecommand{\ppcaasm}{
\toolsection{ppc32asm} is an assembler for the PowerPC hardware architecture.
It translates assembly code into machine code for PowerPC processors and stores it in corresponding object files.
By default, the assembler generates machine code for the 32-bit operating mode defined by the PowerPC architecture.
\flowgraph{\resource{PowerPC assembly\\source code} \ar[r] & \toolbox{ppc32asm} \ar[r] & \resource{object file}}
\seeassembly\seeppc\seeobject
}

\providecommand{\ppcadism}{
\toolsection{ppc32dism} is a disassembler for the PowerPC hardware architecture.
It translates machine code from object files targeting PowerPC processors into assembly code and writes it to the standard output stream.
It assumes that the machine code was generated for the 32-bit operating mode defined by the PowerPC architecture.
\flowgraph{\resource{object file} \ar[r] & \toolbox{ppc32dism} \ar[r] & \resource{disassembly\\listing}}
\seeassembly\seeppc\seeobject
}

\providecommand{\ppcbasm}{
\toolsection{ppc64asm} is an assembler for the PowerPC hardware architecture.
It translates assembly code into machine code for PowerPC processors and stores it in corresponding object files.
By default, the assembler generates machine code for the 64-bit operating mode defined by the PowerPC architecture.
\flowgraph{\resource{PowerPC assembly\\source code} \ar[r] & \toolbox{ppc64asm} \ar[r] & \resource{object file}}
\seeassembly\seeppc\seeobject
}

\providecommand{\ppcbdism}{
\toolsection{ppc64dism} is a disassembler for the PowerPC hardware architecture.
It translates machine code from object files targeting PowerPC processors into assembly code and writes it to the standard output stream.
It assumes that the machine code was generated for the 64-bit operating mode defined by the PowerPC architecture.
\flowgraph{\resource{object file} \ar[r] & \toolbox{ppc64dism} \ar[r] & \resource{disassembly\\listing}}
\seeassembly\seeppc\seeobject
}

\providecommand{\riscasm}{
\toolsection{riscasm} is an assembler for the RISC hardware architecture.
It translates assembly code into machine code for RISC processors and stores it in corresponding object files.
The names of all special registers are predefined and evaluate to the corresponding number.
\flowgraph{\resource{RISC assembly\\source code} \ar[r] & \toolbox{riscasm} \ar[r] & \resource{object file}}
\seeassembly\seerisc\seeobject
}

\providecommand{\riscdism}{
\toolsection{riscdism} is a disassembler for the RISC hardware architecture.
It translates machine code from object files targeting RISC processors into assembly code and writes it to the standard output stream.
\flowgraph{\resource{object file} \ar[r] & \toolbox{riscdism} \ar[r] & \resource{disassembly\\listing}}
\seeassembly\seerisc\seeobject
}

\providecommand{\wasmasm}{
\toolsection{wasmasm} is an assembler for the WebAssembly architecture.
It translates assembly code into machine code for WebAssembly targets and stores it in corresponding object files.
The names of all special registers are predefined and evaluate to the corresponding number.
\flowgraph{\resource{WebAssembly assembly\\source code} \ar[r] & \toolbox{wasmasm} \ar[r] & \resource{object file}}
\seeassembly\seewasm\seeobject
}

\providecommand{\wasmdism}{
\toolsection{wasmdism} is a disassembler for the WebAssembly architecture.
It translates machine code from object files targeting WebAssembly targets into assembly code and writes it to the standard output stream.
\flowgraph{\resource{object file} \ar[r] & \toolbox{wasmdism} \ar[r] & \resource{disassembly\\listing}}
\seeassembly\seewasm\seeobject
}

% linker tools

\providecommand{\linklib}{
\toolsection{linklib} is an object file combiner.
It creates a static library file by combining all object files given to it into a single one.
\flowgraph{\resource{object files} \ar[r] & \toolbox{linklib} \ar[r] & \resource{library file}}
\seeobject
}

\providecommand{\linkbin}{
\toolsection{linkbin} is a linker for plain binary files.
It links all object files given to it into a single image and stores it in a binary file that begins with the first linked section.
It also creates a map file that lists the address, type, name and size of all used sections.
The filename extension of the resulting binary file can be specified by putting it into a constant data section called \texttt{\_extension}.
\flowgraph{\resource{object files} \ar[r] & \toolbox{linkbin} \ar[r] \ar[d] & \resource{binary file} \\ & \resource{map file}}
\seeobject
}

\providecommand{\linkmem}{
\toolsection{linkmem} is a linker for plain binary files partitioned into random-access and read-only memory.
It links all object files given to it into two distinct images, one for data sections and one for code and constant data sections, and stores each image in a binary file that begins with the first linked section of the corresponding type.
It also creates a map file that lists the address, type, name and size of all used sections.
\flowgraph{\resource{object files} \ar[r] & \toolbox{linkmem} \ar[r] \ar[d] & \resource{RAM file/\\ROM file} \\ & \resource{map file}}
\seeobject
}

\providecommand{\linkprg}{
\toolsection{linkprg} is a linker for GEMDOS executable files.
It links all object files given to it into a single image and stores the image in an Atari GEMDOS executable file~\cite{gemdosfile}.
It also creates a map file that lists the address relative to the text segment, type, name and size of all used sections.
The filename extension of the resulting executable file can be specified by putting it into a constant data section called \texttt{\_extension}.
The GEMDOS executable file format requires all patch patterns of absolute link patches to consist of four full bitmasks with descending offsets.
\flowgraph{\resource{object files} \ar[r] & \toolbox{linkprg} \ar[r] \ar[d] & \resource{executable file} \\ & \resource{map file}}
\seeobject
}

\providecommand{\linkhex}{
\toolsection{linkhex} is a linker for Intel HEX files.
It links all code sections of the object files given to it into single image and stores the image in an Intel HEX file~\cite{hexfile} that begins with the first linked section.
It also creates a map file that lists the address, type, name and size of all used sections.
\flowgraph{\resource{object files} \ar[r] & \toolbox{linkhex} \ar[r] \ar[d] & \resource{HEX file} \\ & \resource{map file}}
\seeobject
}

\providecommand{\mapsearch}{
\toolsection{mapsearch} is a debugging tool.
It searches map files generated by linker tools for the name of a binary section that encompasses a memory address read from the standard input stream.
If additionally provided with one or more object files, it also stores an excerpt thereof in a separate object file called map search result which only contains the identified binary section for disassembling purposes.
\flowgraph{& \resource{map files/\\object files} \ar[d] \\ \resource{memory\\address} \ar[r] & \toolbox{mapsearch} \ar[r] \ar[d] & \resource{section name/\\relative offset} \\ & \resource{object file\\excerpt}}
\seeobject
}

\renewcommand{\seeobject}{}

\startchapter{Object Files}{Object File Representation}{object}
{This \documentation{} describes the purpose and the open format of object files which are used by the \ecs{} to represent generic binary code and data.
Additionally, it describes the functionality and interface of the linker tools provided by the \ecs{}.}

\epigraph{The finest eloquence is that which gets things done.}{David Lloyd George}

\section{Introduction}

\emph{Object files} are containers for all binary information that is required to build executable programs.
This includes information about so-called \emph{sections} that describe the contents and location of the code and the global data of a program.
Additionally, object files contain information about how these sections are interconnected.

The \ecs{} features several assemblers for different hardware architectures as well as compilers for different programming languages.
Although all of these tools process different types of input files, they all generate the same type of output file, namely one object file per translation unit.
Other tools like disassemblers and linkers on the other hand are able to continue processing the generated object files.
Linkers for example allocate space for all sections and establish the necessary interconnections in order to create an executable binary program.
Which tools actually generated the object files in the first place is irrelevant during this linking process.
Object files therefore clearly separate the source code from its binary representation by forming a generic abstraction of code and data as shown in Figure~\ref{fig:objdataflow}.

\begin{figure}
\flowgraph{
\resource{language\\source code} \ar[d] & \resource{assembly\\source code} \ar[d] & \resource{debugging\\information} \ar[d] \\
\converter{Compiler} \ar[rd] & \converter{Assembler\vphantom{Compiler}} \ar[d] & \converter{Converter\vphantom{Compiler}} \ar[ld] \\
& \resource{object files} \ar[ld] \ar[d] \ar[rd] \\
\converter{Linker} \ar[d] & \converter{Disassembler} \ar[d] & \converter{Combiner} \ar[d] \\
\resource{executable\\program} & \resource{disassembly\\listing} & \resource{static\\library} \\
}\caption[Object files as a generic abstraction of code and data]{Object files as a generic abstraction of code and data in-between the various kinds of tools of the \ecs{}}
\label{fig:objdataflow}
\end{figure}

Combining object files to build programs offers several advantages.
First, it enables \emph{interoperability}\index{Interoperability} between different compilers and assemblers which allows writing code and data in one programming language and using it from another.
Furthermore, using object files as a binary intermediate representation of a program effectively decouples the compilation from the linking stage and allows compilers and assemblers to become completely independent from the runtime environments targeted by linkers and their actual executable file formats.
Additionally, the use of object files enables \emph{separate compilation}\index{Separate compilation} which allows saving development time by compiling only those parts of a program that have actually changed instead of all of them.
This also allows \emph{statically linking}\index{Statically linking} against precompiled object and library files which typically provide the necessary runtime support for programming languages, the hardware architecture and runtime environment, as well as additional external libraries.

\section{Object File Structure}

Object files consist of a list of sections called \emph{binary sections}\index{Binary sections}.
Each binary section describes a contiguous sequence of relocatable binary data like machine code and contains information about how the section interconnects with other binary sections.
Sections have a unique name and carry additional information about their type and the way they should be allocated in memory by the linker.
This section describes all of this information and how it is used during linking.

\subsection{Section Types}\label{sec:objsectiontypes}

Each binary section is characterized by its \emph{section type}\index{Section types}.
Section types distinguish between code and data as required by Harvard architectures and specify how the corresponding memory shall be allocated by the linker.
The \ecs{} defines the following section types:

\begin{itemize}

\item Standard Code Sections\alignright\syntax{"code"}\nopagebreak

Standard code sections are typically used to model functions which are usually called by other code sections.
Standard code sections have no special requirements for their placement in memory other than an optional alignment.

\item Initializing Code Sections\alignright\syntax{"initcode"}\nopagebreak

Initializing code sections are placed by linkers in front of standard code sections.
This guarantees that initializing code sections are executed automatically before any other code section.
Unlike standard code sections, initializing code sections do not mimic functions as they are executed one after the other rather than being called.

\item Data Initializing Code Sections\alignright\syntax{"initdata"}\nopagebreak

This type of section is the same as initializing code sections except that they are executed at the very beginning of the program.
They are placed by linkers in front of other initializing code sections.
They therefore allow global data to be initialized upon the execution of other initializing code sections.
Data initializing code sections are typically used to initialize global data in environments that do not allow automatic memory initializations of data sections.
This explicitly includes writing data to constant data sections.

\item Standard Data Sections\alignright\syntax{"data"}\nopagebreak

Standard data sections provide the space and contents of global data.
This data is usually modified during the execution of a program by the code within code sections.
Data sections may contain predefined data that is initialized automatically by the linker.
If there are environments that do not allow global data to be initialized this way, data initializing code sections have to be used instead.

\item Constant Data Sections\alignright\syntax{"const"}\nopagebreak

Constant data sections are data sections that are not supposed to change their contents during the execution of the program.
Linkers may therefore place them in a special read-only memory if available.

\item Metadata Sections\alignright\syntax{"header" $\mid$ "trailer"}\nopagebreak

Metadata sections are constant data sections that contain metadata about a program.
Linkers place heading metadata sections at the beginning and trailing metadata sections at the end of the resulting binary file.
This allows mimicking the layout of some specific binary file format while linking several object files into a single binary file.

\end{itemize}

\subsection{Section Options}\label{sec:objsectionoptions}

The \ecs{} defines three freely combinable \emph{section options}\index{Section options}\index{Options, of sections}.
They describe how the linker shall treat sections that are never used or have the same name:

\begin{itemize}

\item Required Sections\alignright\syntax{"required"}\nopagebreak

By default, the linker does not allocate space for binary sections that are never used.
The required section option ensures that a section is always part of the binary program.
All other sections are only used if referenced directly or indirectly by required sections.
Standard code sections called \texttt{main} represent the \emph{entry point}\index{Entry points} of programs and are always implicitly required.
They are placed right after initializing code sections and in front of all other standard code sections such that their code gets executed automatically.

\item Duplicable Sections\alignright\syntax{"duplicable"}\nopagebreak

By default, the linker diagnoses an error whenever it encounters two sections that have the same name.
The duplicable section option specifies that two sections with the same name may be merged together if they are also equal otherwise.
Binary sections are equal if not only their name is the same but also all other information they carry including the binary data.
Whenever the linker encounters two sections that are equal and marked as duplicable, it discards one of them and redirects all references to the other one.
Duplicable sections are typically used for global constants like strings that may be defined in several object files but do not actually need to be duplicated in memory.

\item Replaceable Sections\alignright\syntax{"replaceable"}\nopagebreak

The replaceable section option specifies that a section is only used as long as there is no other section with the same name.
Whenever the linker encounters two sections that have the same name and one of them is marked as replaceable, it discards the replaceable one and redirects all references to the other one.

\end{itemize}

In addition to its name, a section may also have one or more so-called \emph{alias names} which allow referring to individual parts of its binary data by name.
The semantics of duplicable and replaceable sections as described above also applies to all of their alias names.

\subsection{Section Placement}\label{sec:objsectionplacement}

Usually, code and data sections are placed by linkers anywhere in memory\index{Section placement}\index{Placement, of sections}.
During the linking stage, linkers assign a unique memory address to every non-empty section that is actually used in the program.
References of a section to other sections are later resolved by their name and yield the actual address of the named section.
Referencing sections by name has the advantage that a programmer does not need to know or specify anything about the actual address of a section.

Sometimes however, the target runtime environment or hardware architecture requires programmers to assign a predefined address to a section.
The \ecs{} therefore allows programmers to define the address where a section has to be placed by the linker in the following two ways:

\begin{itemize}

\item Aligned Sections\alignright\syntax{"aligned" <Alignment>}\nopagebreak

Aligned sections have an arbitrary address that is a multiple of a positive power of two called their \emph{alignment}.
Usually, the underlying hardware imposes requirements on the alignment of data and code when they are referenced or called.
The \ecs{} aligns the address of aligned sections according to their alignment expressed in multiples of octets.

\item Fixed Sections\alignright\syntax{"fixed" <Origin>}\nopagebreak

Fixed sections have a predefined address called their \emph{origin}.
Before linkers layout any other section in memory, they place fixed sections at their origin expressed in octets.
All others sections are then placed according to the semantics of their section type and alignment.

\end{itemize}

Sections that have the same section type and alignment may additionally be assigned to a so-called \emph{group}.
This allows grouping otherwise unrelated sections together by forcing linkers to place them adjacent to each other.
Each group can be referred to by name which yields a virtual section that contains all sections within that group.
The groups themselves are placed consecutively in lexicographic order followed by all remaining sections of the same type.

Apart from the configurable placement of sections as described above and in Section~\ref{sec:objsectiontypes}, sections are in general placed in order of occurrence.
This allows initializing code sections to depend on the results of already executed code.
The only exception are sections without binary data which always precede all other sections of the same type and group.

\subsection{Section References}

Code in code sections usually needs to call functions defined in other code sections or to refer to data stored in data sections.
Data sections on the other hand are sometimes used to store the address of other data and code sections.
In order to establish this interconnection, the \ecs{} defines \emph{section links}\index{Section links}.
Section links allow a code or data section to refer to another section by name.
Besides placing sections in memory, the main task of the linker is to resolve these references.

Each binary section maintains a list of links where each link refers to a different section.
In addition to the name of the referenced section, each link contains a list of so-called \emph{link patches}.
A link patch defines where in the data of the binary section and how exactly linkers have to write the actual address of the referenced section.
This information includes the offset within the data of the section the address has to be patched as well as a displacement that is added to the address beforehand.
Additionally, the patch specifies whether the address is absolute or has to be relative to the actual memory address of the patch.
The address can also be scaled according to potential alignment constraints.
The \emph{patch pattern} finally specifies how the resulting address has to be written to memory in order to comply with predefined instruction encodings and endianness constraints.

\section{Object File Format}

Object files are stored as plain text files according to the complete syntax specification given in Figure~\ref{fig:objfileformat}.
They consist of the textual representation of an arbitrary number of binary sections according to the following syntax:

\begin{figure}
\centering\ifbook\small\fi\setlength{\grammarparsep}{0ex}
\begin{minipage}{32em}\begin{grammar}
<Object-File> = <Sections>$\opt$ \par
<Sections> = <Section> $\mid$ <Sections> <Section> \par
<Section> = <Type> <Size> <Options>$\opt$ <Name> <Aliases>$\opt$ <Group>$\opt$ <Placement> \\ <Segments>$\opt$ <Links>$\opt$ \par
<Type> = "code" $\mid$ "initcode" $\mid$ "initdata" $\mid$ \\ "data" $\mid$ "const" $\mid$ "header" $\mid$ "trailer" \par
<Size> = decimal-integer \par
<Options> = <Option> $\mid$ <Options> <Option> \par
<Option> = "required" $\mid$ "duplicable" $\mid$ "replaceable" \par
<Name> = double-quoted-string \par
<Aliases> = <Alias> $\mid$ <Aliases> <Alias> \par
<Alias> = <Offset> <Name> \par
<Offset> = decimal-integer \par
<Group> = <Name> \par
<Placement> = "aligned" <Alignment> $\mid$ "fixed" <Origin> \par
<Alignment> = decimal-integer \par
<Origin> = decimal-integer \par
<Segments> = <Segment> $\mid$ <Segments> <Segment> \par
<Segment> = <Offset> <Octets> \par
<Octets> = <Octet> $\mid$ <Octets><\ ><Octet> \par
<Octet> = <High-Quartet><\ ><Low-Quartet> \par
<High-Quartet> = hexadecimal-digit \par
<Low-Quartet> = hexadecimal-digit \par
<Links> = <Link> $\mid$ <Links> <Link> \par
<Link> = <Name> <Patches>$\opt$ \par
<Patches> = <Patch> $\mid$ <Patches> <Patch> \par
<Patch> = <Offset> <Mode> <Displacement> <Scale> <Pattern> \par
<Mode> = "abs" $\mid$ "rel" $\mid$ "siz" $\mid$ "ext" $\mid$ "pos" $\mid$ "idx" $\mid$ "cnt" \par
<Displacement> = signed-decimal-integer \par
<Scale> = decimal-integer \par
<Pattern> = <Masks> $\mid$ <Flag><\ ><Size> \par
<Masks> = <Mask> $\mid$ <Masks><\ ><Mask> \par
<Mask> = <Offset><\ ><Bitmask> \par
<Bitmask> = <Octet> \par
<Flag> = "+" $\mid$ "-" \par
\end{grammar}\end{minipage}
\caption{Syntax of the object file format}
\label{fig:objfileformat}
\end{figure}

\begin{quote}\begin{grammar}
<Object-File> = <Sections>$\opt$ \par
\end{grammar}\end{quote}

\subsection{Binary Sections}

Binary sections are represented in the object file as text according to the following syntax:

\begin{quote}\begin{grammar}
<Sections> = <Section> $\mid$ <Sections> <Section> \par
<Section> = <Type> <Size> <Options>$\opt$ <Name> <Aliases>$\opt$ <Group>$\opt$ <Placement> \\ <Segments>$\opt$ <Links>$\opt$ \par
<Type> = "code" $\mid$ "initcode" $\mid$ "initdata" $\mid$ \ifbook\\\fi "data" $\mid$ "const" $\mid$ "header" $\mid$ "trailer" \par
<Size> = decimal-integer \par
<Options> = <Option> $\mid$ <Options> <Option> \par
<Option> = "required" $\mid$ "duplicable" $\mid$ "replaceable" \par
<Name> = double-quoted-string \par
<Group> = <Name> \par
<Placement> = "aligned" <Alignment> $\mid$ "fixed" <Origin> \par
<Alignment> = decimal-integer \par
<Origin> = decimal-integer \par
\end{grammar}\end{quote}

The valid identifiers for the type of the binary section correspond to the types mentioned in Section~\ref{sec:objsectiontypes}.
The section size specifies the total number of octets occupied by the binary data of the section.
The valid identifiers for the options of the binary section correspond to the options mentioned in Section~\ref{sec:objsectionoptions}.
The optional group name and the alignment or origin of an aligned or fixed section influence its placement as described in Section~\ref{sec:objsectionplacement}.

\subsection{Alias Names}

The alias names of binary sections are represented in the object file as text according to the following syntax:

\begin{quote}\begin{grammar}
<Aliases> = <Alias> $\mid$ <Aliases> <Alias> \par
<Alias> = <Offset> <Name> \par
<Offset> = decimal-integer \par
<Name> = double-quoted-string \par
\end{grammar}\end{quote}

Alias names may not be duplicated and should differ from the name of the binary section.
The offset of an alias allows referring to a specific octet within the data of the binary section using a different name.
All names are represented using double-quoted strings and may contain standard escape sequences.

\subsection{Binary Data}

The data of a binary section is partitioned into segments which are represented in the object file as text according to the following syntax.
There may not be any white-space character in-between the hexadecimal digits of octets:

\begin{quote}\begin{grammar}
<Segments> = <Segment> $\mid$ <Segments> <Segment> \par
<Segment> = <Offset> <Octets> \par
<Offset> = decimal-integer \par
<Octets> = <Octet> $\mid$ <Octets><\ ><Octet> \par
<Octet> = <High-Quartet><\ ><Low-Quartet> \par
<High-Quartet> = hexadecimal-digit \par
<Low-Quartet> = hexadecimal-digit \par
\end{grammar}\end{quote}

Each segment contains one or more octets that represent the binary data of the section starting at the corresponding offset.
This offset plus the number of octets may not exceed the overall section size.
Overlapping segments define the binary data of a section in order of occurrence.
Binary data not covered by any segment is initialized to zero.

\subsection{Section Links}

The links of the binary sections are represented in the object file as text according to the following syntax:

\begin{quote}\begin{grammar}
<Links> = <Link> $\mid$ <Links> <Link> \par
<Link> = <Name> <Patches>$\opt$ \par
<Name> = double-quoted-string \par
\end{grammar}\end{quote}

The name of a link identifies the section that has to be referenced.
If it contains question marks, the actual name of the referenced section begins behind the last question mark.
If none of the sections named in front of a question mark are actually used, the referenced section evaluates either to zero or to the optional section named behind a colon following the referenced section.

\subsection{Link Patches}

The patches of section links are represented in the object file as text according to the following syntax:

\begin{quote}\begin{grammar}
<Patches> = <Patch> $\mid$ <Patches> <Patch> \par
<Patch> = <Offset> <Mode> <Displacement> <Scale> <Pattern> \par
<Offset> = decimal-integer \par
<Mode> = "abs" $\mid$ "rel" $\mid$ "siz" $\mid$ "ext" $\mid$ "pos" $\mid$ "idx" $\mid$ "cnt" \par
<Displacement> = signed-decimal-integer \par
<Scale> = decimal-integer \par
\end{grammar}\end{quote}

The offset specifies the position of the octet within the current binary section where the referenced section has to be patched by linkers.
The remainder of the patch specifies how the referenced section shall actually be evaluated.
The patch mode \texttt{abs} specifies that the referenced section evaluates to its absolute address expressed in octets.
The patch mode \texttt{rel} specifies that the absolute address of the referenced section shall be decremented by the target address of the patch in order to yield a relative address.
If the patch mode \texttt{siz} is used, the referenced section evaluates to its binary size expressed in octets.
The patch mode \texttt{ext} yields the absolute address of the referenced section plus its binary size.
Using the patch mode \texttt{pos}, the referenced section evaluates to its position relative to the beginning of its associated group expressed in octets.
The patch mode \texttt{idx} specifies that the referenced section evaluates to the index in the sequence of sections of its group.
The patch mode \texttt{cnt} can be used to reference groups and evaluates to the number of sections contained therein.
The evaluated value is then incremented or decremented by the optionally signed displacement and shifted to the right by the specified scale.
Overlapping link patches are applied in order of occurrence.

\subsection{Patch Patterns}

The patterns of link patches are represented in the object file as text according to the following syntax.
There may not be any white-space character in-between the elements of a pattern:

\begin{quote}\begin{grammar}
<Pattern> = <Masks> $\mid$ <Flag><\ ><Size> \par
<Masks> = <Mask> $\mid$ <Masks><\ ><Mask> \par
<Mask> = <Offset><\ ><Bitmask> \par
<Offset> = decimal-integer \par
<Bitmask> = <Octet> \par
<Octet> = <High-Quartet><\ ><Low-Quartet> \par
<High-Quartet> = hexadecimal-digit \par
<Low-Quartet> = hexadecimal-digit \par
<Flag> = "+" $\mid$ "-" \par
<Size> = decimal-integer \par
\end{grammar}\end{quote}

A pattern consists of a list of one to eight masks which define how the actual address has to be written to binary data.
Each mask contains an offset and the corresponding bitmask.
The single-digit offset specifies the relative displacement of the mask with respect to the enclosing link patch which may not exceed the overall section size.
The bitmask is given as the value of a single octet in hexadecimal form.
It specifies the number of bits consecutively taken from the address, as well as the mask which is used to write that part of the address value to binary data.
The pattern \texttt{5ff403} for example consists of two masks.
The first mask \texttt{5ff} tells the linker to write the value of the first eight bits of the address to all bits of the octet at the target address displaced by five.
The second mask \texttt{403} tells the linker to write the value of the ninth and tenth bit of the address to the lowest two bits of the octet at the target address displaced by four.
Patterns which consist exclusively of full bitmasks with consecutive offsets can also be represented by a flag indicating ascending or descending offsets followed by the number of masks contained therein.
The pattern \texttt{0ff1ff2ff} for example can be abbreviated by \texttt{+3}.

\section{Linker Tools}

Linkers process object files that were previously generated by the various compilers and assemblers of the \ecs{}.
The task of a linker is to merge all object files given to it into an executable program and to store the resulting binary image using a specific file format.
All of these linker tools accept command-line arguments which are taken as names of the actual object files.
If no arguments are provided, object files are read from the standard input stream.
Linkers create output files in the current working directory by reusing the name of the first object file whereas the filename extension gets replaced by an appropriate suffix.
\seeinterface\seeguide

\linklib
\linkbin
\linkmem
\linkhex
\linkprg
\mapsearch

\concludechapter

% Intermediate code representation
% Copyright (C) Florian Negele

% This file is part of the Eigen Compiler Suite.

% Permission is granted to copy, distribute and/or modify this document
% under the terms of the GNU Free Documentation License, Version 1.3
% or any later version published by the Free Software Foundation.

% You should have received a copy of the GNU Free Documentation License
% along with the ECS.  If not, see <https://www.gnu.org/licenses/>.

% Generic documentation utilities
% Copyright (C) Florian Negele

% This file is part of the Eigen Compiler Suite.

% Permission is granted to copy, distribute and/or modify this document
% under the terms of the GNU Free Documentation License, Version 1.3
% or any later version published by the Free Software Foundation.

% You should have received a copy of the GNU Free Documentation License
% along with the ECS.  If not, see <https://www.gnu.org/licenses/>.

\providecommand{\cpp}{C\texttt{++}}
\providecommand{\opt}{_\mathit{opt}}
\providecommand{\tool}[1]{\texttt{#1}}
\providecommand{\version}{Version 0.0.40}
\providecommand{\resource}[1]{*++\txt{#1}}
\providecommand{\ecs}{Eigen Compiler Suite}
\providecommand{\changed}[1]{\underline{#1}}
\providecommand{\toolbox}[1]{\converter{#1}}
\providecommand{\file}{}\renewcommand{\file}[1]{\texttt{#1}}
\providecommand{\alignright}{\hfill\linebreak[0]\hspace*{\fill}}
\providecommand{\converter}[1]{*++[F][F*:white][F,:gray]\txt{#1}}
\providecommand{\documentation}{\ifbook chapter\else document\fi}
\providecommand{\Documentation}{\ifbook Chapter\else Document\fi}
\providecommand{\variable}[1]{\resource{\texttt{\small#1}\\variable}}
\providecommand{\documentationref}[2]{\ifbook\ref{#1}\else``\href{#1}{#2}''~\cite{#1}\fi}
\providecommand{\objfile}[1]{\texttt{#1}\index[runtime]{#1 object file@\texttt{#1} object file}}
\providecommand{\libfile}[1]{\texttt{#1}\index[runtime]{#1 library file@\texttt{#1} library file}}
\providecommand{\epigraph}[2]{\ifbook\begin{quote}\flushright\textit{#1}\par--- #2\end{quote}\fi}
\providecommand{\environmentvariable}[1]{\texttt{#1}\index{Environment variables!#1@\texttt{#1}}}
\providecommand{\environment}[1]{\texttt{#1}\index[environment]{#1 environment@\texttt{#1} environment}}
\providecommand{\toolsection}{}\renewcommand{\toolsection}[1]{\subsection{#1}\label{\prefix:#1}\tool{#1}}
\providecommand{\instruction}{}\renewcommand{\instruction}[2]{\noindent\qquad\pdftooltip{\texttt{#1}}{#2}\refstepcounter{instruction}\par}
\providecommand{\flowgraph}{}\renewcommand{\flowgraph}[1]{\par\sffamily\begin{displaymath}\xymatrix@=4ex{#1}\end{displaymath}\normalfont\par}
\providecommand{\instructionset}{}\renewcommand{\instructionset}[4]{\setcounter{instruction}{0}\begin{multicols}{\ifbook#3\else#4\fi}[{\captionof{table}[#2]{#2 (\ref*{#1:instructions}~instructions)}\label{tab:#1set}\vspace{-2ex}}]\footnotesize\raggedcolumns\input{#1.set}\label{#1:instructions}\end{multicols}}

\providecommand{\gpl}{GNU General Public License}
\providecommand{\rse}{ECS Runtime Support Exception}
\providecommand{\fdl}{\href{https://www.gnu.org/licenses/fdl.html}{GNU Free Documentation License}}

\providecommand{\docbegin}{}
\providecommand{\docend}{}
\providecommand{\doclabel}[1]{\hypertarget{#1}}
\providecommand{\doclink}[2]{\hyperlink{#1}{#2}}
\providecommand{\docsection}[3]{\hypertarget{#1}{\subsection{#2}}\label{sec:#1}\index[library]{#2@#3}}
\providecommand{\docsectionstar}[1]{}
\providecommand{\docsubbegin}{\begin{description}}
\providecommand{\docsubend}{\end{description}}
\providecommand{\docsubsection}[3]{\item[\hypertarget{#1}{#2}]\index[library]{#2@#3}}
\providecommand{\docsubsectionstar}[1]{\smallskip}
\providecommand{\docsubsubsection}[3]{\docsubsection{#1}{#2}{#3}}
\providecommand{\docsubsubsectionstar}[1]{}
\providecommand{\docsubsubsubsection}[3]{}
\providecommand{\docsubsubsubsectionstar}[1]{}
\providecommand{\doctable}{}

\providecommand{\debuggingtool}{}\renewcommand{\debuggingtool}{This tool is provided for debugging purposes.
It allows exposing and modifying an internal data structure that is usually not accessible.
}

\providecommand{\interface}{All tools accept command-line arguments which are taken as names of plain text files containing the source code.
If no arguments are provided, the standard input stream is used instead.
Output files are generated in the current working directory and have the same name as the input file being processed whereas the filename extension gets replaced by an appropriate suffix.
\seeinterface
}

\providecommand{\license}{\noindent Copyright \copyright{} Florian Negele\par\medskip\noindent
Permission is granted to copy, distribute and/or modify this document under the terms of the
\fdl{}, Version 1.3 or any later version published by the \href{https://fsf.org/}{Free Software Foundation}.
}

\providecommand{\ecslogosurface}{
\fill[darkgray] (0,0,0) -- (0,0,3) -- (0,3,3) -- (0,3,1) -- (0,4,1) -- (0,4,3) -- (0,5,3) -- (0,5,0) -- (0,2,0) -- (0,2,2) -- (0,1,2) -- (0,1,0) -- cycle;
\fill[gray] (0,5,0) -- (0,5,3) -- (1,5,3) -- (1,5,1) -- (2,5,1) -- (2,5,3) -- (3,5,3) -- (3,5,0) -- cycle;
\fill[lightgray] (0,0,0) -- (0,1,0) -- (2,1,0) -- (2,4,0) -- (1,4,0) -- (1,3,0) -- (2,3,0) -- (2,2,0) -- (0,2,0) -- (0,5,0) -- (3,5,0) -- (3,0,0) -- cycle;
\begin{scope}[line width=0.5]
\begin{scope}[gray]
\draw (0,0,0) -- (0,1,0);
\draw (2,1,0) -- (2,2,0);
\draw (0,1,2) -- (0,2,2);
\draw (0,2,0) -- (0,5,0);
\draw (2,3,0) -- (2,4,0);
\end{scope}
\begin{scope}[lightgray]
\draw (0,1,0) -- (0,1,2);
\draw (0,3,1) -- (0,3,3);
\draw (0,5,0) -- (0,5,3);
\draw (2,5,1) -- (2,5,3);
\end{scope}
\begin{scope}[white]
\draw (0,1,0) -- (2,1,0);
\draw (1,3,0) -- (2,3,0);
\draw (0,5,0) -- (3,5,0);
\end{scope}
\end{scope}
}

\providecommand{\ecslogo}[1]{
\begin{tikzpicture}[scale={(#1)/((sin(45)+cos(45))*3cm)},x={({-cos(45)*1cm},{sin(45)*sin(30)*1cm})},y={({0cm},{(cos(30)*1cm})},z={({sin(45)*1cm},{cos(45)*sin(30)*1cm})}]
\begin{scope}[darkgray,line width=1]
\draw (0,0,0) -- (0,0,3) -- (0,3,3) -- (2,3,3) -- (2,5,3) -- (3,5,3) -- (3,5,0) -- (3,0,0) -- cycle;
\draw (0,3,1) -- (0,4,1) -- (0,4,3) -- (0,5,3) -- (1,5,3) -- (1,5,1) -- (2,5,1);
\draw (1,3,0) -- (1,4,0) -- (2,4,0);
\end{scope}
\fill[darkgray] (2,0,0) -- (2,0,3) -- (2,5,3) -- (2,5,1) -- (2,4,1) -- (2,4,0) -- cycle;
\fill[lightgray] (2,0,2) -- (0,0,2) -- (0,2,2) -- (2,2,2) -- cycle;
\fill[gray] (0,1,0) -- (2,1,0) -- (2,1,2) -- (0,1,2) -- cycle;
\fill[gray] (0,3,1) -- (0,3,3) -- (2,3,3) -- (2,3,0) -- (1,3,0) -- (1,3,1) -- cycle;
\ecslogosurface
\end{tikzpicture}
}

\providecommand{\shadowedecslogo}[3]{
\begin{tikzpicture}[scale={(#1)/((sin(#2)+cos(#2))*3cm)},x={({-cos(#2)*1cm},{sin(#2)*sin(#3)*1cm})},y={({0cm},{(cos(#3)*1cm})},z={({sin(#2)*1cm},{cos(#2)*sin(#3)*1cm})}]
\shade[top color=lightgray!50!white,bottom color=white,middle color=lightgray!50!white] (0,0,0) -- (3,0,0) -- (3,{-0.5-3*sin(#2)*sin(#3)/cos(#3)},0) -- (0,-0.5,0) -- cycle;
\shade[top color=darkgray!50!gray,bottom color=white,middle color=darkgray!50!white] (0,0,0) -- (0,0,3) -- (0,{-0.5-3*cos(#2)*sin(#3)/cos(#3)},3) -- (0,-0.5,0) -- cycle;
\begin{scope}[y={({(cos(#2)+sin(#2))*0.5cm},{(cos(#2)*sin(#3)-sin(#2)*sin(#3))*0.5cm})}]
\useasboundingbox (3,0,0) -- (0,0,0) -- (0,0,3);
\shade[left color=darkgray!80!black,right color=lightgray,middle color=gray] (0,0,0) -- (0,1,0) -- (0,1,0.5) -- (0,2,0) -- (0,5,0) -- (0,5,3) -- (1,5,3) -- (1,4,3) -- (1,4,2.5) -- (1,3,3) -- (2,5,3) -- (3,5,3) -- (3,0,3) -- cycle;
\clip (0,0,0) -- (0,0,3) -- ({-3*sin(#2)/cos(#2)},0,0) -- cycle;
\shade[left color=darkgray,right color=lightgray!50!gray] (0,0,0) -- (0,1,0) -- (0,1,0.5) -- (0,2,0) -- (0,5,0) -- (0,5,3) -- (1,5,3) -- (1,4,3) -- (1,4,2.5) -- (1,3,3) -- (2,5,3) -- (3,5,3) -- (3,0,3) -- cycle;
\end{scope}
\shade[left color=darkgray,right color=darkgray!80!black] (2,0,0) -- (2,0,3) -- (2,5,3) -- (2,5,1) -- (2,4,1) -- (2,4,0) -- cycle;
\shade[left color=darkgray!90!black,right color=gray!80!darkgray] (2,0,2) -- (0,0,2) -- (0,2,2) -- (2,2,2) -- cycle;
\shade[top color=darkgray!90!black,bottom color=gray!80!darkgray] (0,1,0) -- (2,1,0) -- (2,1,2) -- (0,1,2) -- cycle;
\shade[top color=darkgray!90!black,bottom color=gray!80!darkgray] (0,3,1) -- (0,3,3) -- (2,3,3) -- (2,3,0) -- (1,3,0) -- (1,3,1) -- cycle;
\fill[gray] (2,1,0) -- (1.5,1,0.5) -- (0,1,0.5) -- (0,1,0) -- cycle;
\fill[gray] (1,3,2) -- (0.5,3,2) -- (0.5,3,3) -- (1,3,3) -- cycle;
\fill[gray] (2,3,0) -- (1.5,3,0.5) -- (1,3,0.5) -- (1,3,0) -- cycle;
\ecslogosurface
\end{tikzpicture}
}

\providecommand{\cpplogo}[1]{
\begin{tikzpicture}[scale=(#1)/512em]
\fill[gray] (435.2794,398.7159) -- (247.1911,507.3075) .. controls (236.3563,513.5642) and (218.6240,513.5642) .. (207.7892,507.3075) -- (19.7009,398.7159) .. controls (8.8646,392.4606) and (0.0000,377.1043) .. (0.0000,364.5924) -- (0.0000,147.4076) .. controls (0.8430,132.8363) and (8.2856,120.7683) .. (19.7009,113.2842) -- (207.7892,4.6926) .. controls (218.6240,-1.5642) and (236.3564,-1.5642) .. (247.1911,4.6926) -- (435.2794,113.2842) .. controls (447.5273,121.4304) and (454.4987,133.6918) .. (454.9803,147.4076) -- (454.9803,364.5924) .. controls (454.5404,377.7571) and (446.6566,391.0351) .. (435.2794,398.7159) -- cycle(75.8301,255.9993) .. controls (74.9389,404.0881) and (273.2892,469.4783) .. (358.8263,331.8769) -- (293.1917,293.8965) .. controls (253.5702,359.4301) and (155.1909,335.9977) .. (151.6601,255.9993) .. controls (152.7204,182.2703) and (249.4137,148.0211) .. (293.1961,218.1065) -- (358.8308,180.1276) .. controls (283.4477,49.2645) and (79.6318,96.3470) .. (75.8301,255.9993) -- cycle(379.1503,247.5747) -- (362.2982,247.5747) -- (362.2982,230.7226) -- (345.4490,230.7226) -- (345.4490,247.5747) -- (328.5969,247.5747) -- (328.5969,264.4254) -- (345.4490,264.4254) -- (345.4490,281.2759) -- (362.2982,281.2759) -- (362.2982,264.4254) -- (379.1503,264.4254) -- cycle(442.3420,247.5747) -- (425.4899,247.5747) -- (425.4899,230.7226) -- (408.6408,230.7226) -- (408.6408,247.5747) -- (391.7886,247.5747) -- (391.7886,264.4254) -- (408.6408,264.4254) -- (408.6408,281.2759) -- (425.4899,281.2759) -- (425.4899,264.4254) -- (442.3420,264.4254) -- cycle;
\end{tikzpicture}
}

\providecommand{\fallogo}[1]{
\begin{tikzpicture}[scale=(#1)/512em]
\fill[gray] (185.7774,0.0000) .. controls (200.4486,15.9798) and (226.8966,8.7148) .. (235.0426,31.5836) .. controls (249.5297,58.0598) and (247.9581,97.9161) .. (280.3335,110.9762) .. controls (309.1690,120.3496) and (337.8406,104.2727) .. (366.5753,103.9379) .. controls (373.4449,111.5171) and (379.2885,128.2574) .. (383.9755,108.9744) .. controls (396.6979,102.5615) and (437.2808,107.6681) .. (426.9652,124.3252) .. controls (408.9822,121.0785) and (412.4742,146.0729) .. (426.5192,131.4996) .. controls (433.8413,120.8489) and (465.1541,126.5522) .. (441.9067,135.7950) .. controls (396.1879,157.7478) and (344.1112,161.5079) .. (298.5528,183.5702) .. controls (277.7471,193.5198) and (284.6941,218.7163) .. (285.2127,236.9640) .. controls (292.3599,316.2826) and (307.3929,394.6311) .. (317.1198,473.6154) .. controls (329.0637,505.4736) and (292.1195,528.5004) .. (265.9183,511.2761) .. controls (237.9284,499.2462) and (237.3684,465.2681) .. (230.9102,439.9421) .. controls (218.6692,374.3397) and (215.6307,306.9662) .. (198.1732,242.3977) .. controls (183.1379,232.7444) and (164.4245,256.0298) .. (149.0430,261.4799) .. controls (116.9328,279.2585) and (87.1822,308.5851) .. (48.2293,307.8914) .. controls (21.3220,306.9037) and (-15.9107,281.8761) .. (7.2921,252.7908) .. controls (29.7799,220.6177) and (67.5177,204.3028) .. (100.9287,185.9449) .. controls (130.8217,170.8906) and (161.1548,156.5903) .. (191.0278,141.5847) .. controls (196.1738,120.0520) and (186.6049,95.2409) .. (186.8382,72.4353) .. controls (185.5234,48.4204) and (183.1700,23.9341) .. (185.7774,0.0000) -- cycle;
\end{tikzpicture}
}

\providecommand{\oblogo}[1]{
\begin{tikzpicture}[scale=(#1)/512em]
\fill[gray] (160.3865,208.9117) .. controls (154.0879,214.6478) and (149.0735,221.2409) .. (145.4125,228.5384) .. controls (184.8790,248.4273) and (234.7122,269.8787) .. (297.5493,291.8782) .. controls (300.3943,281.4769) and (300.9552,268.7619) .. (300.4023,255.2389) .. controls (248.9909,244.7891) and (200.0310,225.9279) .. (160.3865,208.9117) -- cycle(225.7398,392.6996) .. controls (308.0209,392.1716) and (359.3326,345.9277) .. (368.7203,285.2098) .. controls (376.6742,197.1784) and (311.7194,141.3342) .. (205.4287,142.1456) .. controls (139.9485,141.4804) and (88.7155,166.1957) .. (73.5775,228.0086) .. controls (52.0297,320.3408) and (123.4078,391.0103) .. (225.7398,392.6996) -- cycle(216.0739,176.4733) .. controls (268.9183,179.2424) and (315.8292,206.5488) .. (312.7454,265.1139) .. controls (313.2769,315.6384) and (286.5993,353.4946) .. (216.6040,355.7934) .. controls (162.4657,355.7934) and (126.0914,317.5023) .. (126.0914,260.5103) .. controls (126.1733,214.2900) and (163.3363,176.2849) .. (216.0739,176.4733) -- cycle(76.4897,189.1754) .. controls (13.1586,147.5631) and (0.0000,119.4207) .. (0.0000,119.4207) -- (90.6499,170.1632) .. controls (85.3004,175.8497) and (80.5994,182.1633) .. (76.4897,189.1754) -- cycle(353.9486,119.3004) -- (402.9482,119.3004) .. controls (427.0025,137.0797) and (450.9893,162.7034) .. (474.9529,191.0213) .. controls (509.3540,228.5339) and (531.3391,294.2091) .. (487.8149,312.1206) .. controls (462.8165,324.7652) and (394.3874,316.8943) .. (373.8912,313.6651) .. controls (379.9291,297.7449) and (383.2899,278.4204) .. (381.4989,257.7214) .. controls (420.3069,248.0321) and (421.9610,218.3461) .. (407.7867,192.6417) .. controls (391.1113,162.4018) and (370.1114,132.9097) .. (353.9486,119.3004) -- cycle;
\end{tikzpicture}
}

\providecommand{\markuptable}{
\begin{table}
\sffamily\centering
\begin{tabular}{@{}lcl@{}}
\toprule
\texttt{//italics//} & $\rightarrow$ & \textit{italics} \\
\midrule
\texttt{**bold**} & $\rightarrow$ & \textbf{bold} \\
\midrule
\texttt{\# ordered list} & & 1 ordered list \\
\texttt{\# second item} & $\rightarrow$ & 2 second item \\
\texttt{\#\# sub item} & & \hspace{1em} 1 sub item \\
\midrule
\texttt{* unordered list} & & $\bullet$ unordered list \\
\texttt{* second item} & $\rightarrow$ & $\bullet$ second item \\
\texttt{** sub item} & & \hspace{1em} $\bullet$ sub item \\
\midrule
\texttt{link to [[label]]} & $\rightarrow$ & link to \underline{label} \\
\midrule
\texttt{<{}<label>{}> definition } & $\rightarrow$ & definition \\
\midrule
\texttt{[[url|link name]]} & $\rightarrow$ & \underline{link name} \\
\midrule\addlinespace
\texttt{= large heading} & & {\Large large heading} \smallskip \\
\texttt{== medium heading} & $\rightarrow$ & {\large medium heading} \\
\texttt{=== small heading} & & small heading \\
\midrule
\texttt{no line break} & & no line break for paragraphs \\
\texttt{for paragraphs} & $\rightarrow$ \\
& & use empty line \\
\texttt{use empty line} \\
\midrule
\texttt{force\textbackslash\textbackslash line break} & $\rightarrow$ & force \\
& & line break \\
\midrule
\texttt{horizontal line} & $\rightarrow$ & horizontal line \\
\texttt{----} & & \hrulefill \\
\midrule
\texttt{|=a|=table|=header} & & \underline{a \enspace table \enspace header} \\
\texttt{|a|table|row} & $\rightarrow$ & a \enspace table \enspace row \\
\texttt{|b|table|row} & & b \enspace table \enspace row \\
\midrule
\texttt{\{\{\{} \\
\texttt{unformatted} & $\rightarrow$ & \texttt{unformatted} \\
\texttt{code} & & \texttt{code} \\
\texttt{\}\}\}} \\
\midrule\addlinespace
\texttt{@ new article} & & {\Large 1.\ new article} \smallskip \\
\texttt{@ second article} & $\rightarrow$ & {\Large 2.\ second article} \smallskip \\
\texttt{@@ sub article} & & {\large 2.1.\ sub article} \\
\bottomrule
\end{tabular}
\normalfont\caption{Elements of the generic documentation markup language}
\label{tab:docmarkup}
\end{table}
}

\providecommand{\startchapter}[4]{
\documentclass[11pt,a4paper]{article}
\usepackage{booktabs}
\usepackage[format=hang,labelfont=bf]{caption}
\usepackage{changepage}
\usepackage[T1]{fontenc}
\usepackage[margin=2cm]{geometry}
\usepackage{hyperref}
\usepackage[american]{isodate}
\usepackage{lmodern}
\usepackage{longtable}
\usepackage{mathptmx}
\usepackage{microtype}
\usepackage[toc]{multitoc}
\usepackage{multirow}
\usepackage[all]{nowidow}
\usepackage{pdfcomment}
\usepackage{syntax}
\usepackage{tikz}
\usepackage[all]{xy}
\hypersetup{pdfborder={0 0 0},bookmarksnumbered=true,pdftitle={\ecs{}: #2},pdfauthor={Florian Negele},pdfsubject={\ecs{}},pdfkeywords={#1}}
\setlength{\grammarindent}{8em}\setlength{\grammarparsep}{0.2ex}
\setlength{\columnsep}{2em}
\newcommand{\prefix}{}
\newcounter{instruction}
\bibliographystyle{unsrt}
\renewcommand{\index}[2][]{}
\renewcommand{\arraystretch}{1.05}
\renewcommand{\floatpagefraction}{0.7}
\renewcommand{\syntleft}{\itshape}\renewcommand{\syntright}{}
\title{\vspace{-5ex}\Huge{\ecs{}}\medskip\hrule}
\author{\huge{#2}}
\date{\medskip\version}
\newif\ifbook\bookfalse
\pagestyle{headings}
\frenchspacing
\begin{document}
\maketitle\thispagestyle{empty}\noindent#4\setlength{\columnseprule}{0.4pt}\tableofcontents\setlength{\columnseprule}{0pt}\vfill\pagebreak[3]\null\vfill\bigskip\noindent
\parbox{\textwidth-4em}{\license The contents of this \documentation{} are part of the \href{manual}{\ecs{} User Manual}~\cite{manual} and correspond to Chapter ``\href{manual\##3}{#1}''.\alignright\mbox{\today}}
\parbox{4em}{\flushright\ecslogo{3em}}
\clearpage
}

\providecommand{\concludechapter}{
\vfill\pagebreak[3]\null\vfill
\thispagestyle{myheadings}\markright{REFERENCES}
\noindent\begin{minipage}{\textwidth}\begin{multicols}{2}[\section*{References}]
\renewcommand{\section}[2]{}\small\bibliography{references}
\end{multicols}\end{minipage}\end{document}
}

\providecommand{\startpresentation}[2]{
\documentclass[14pt,aspectratio=43,usepdftitle=false]{beamer}
\usepackage{booktabs}
\usepackage{etex}
\usepackage{multicol}
\usepackage{tikz}
\usepackage[all]{xy}
\bibliographystyle{unsrt}
\setlength{\columnsep}{1em}
\setlength{\leftmargini}{1em}
\setbeamercolor{title}{fg=black}
\setbeamercolor{structure}{fg=darkgray}
\setbeamercolor{bibliography item}{fg=darkgray}
\setbeamerfont{title}{series=\bfseries}
\setbeamerfont{subtitle}{series=\normalfont}
\setbeamerfont*{frametitle}{parent=title}
\setbeamerfont{block title}{series=\bfseries}
\setbeamerfont*{framesubtitle}{parent=subtitle}
\setbeamersize{text margin left=1em,text margin right=1em}
\setbeamertemplate{navigation symbols}{}
\setbeamertemplate{itemize item}[circle]{}
\setbeamertemplate{bibliography item}[triangle]{}
\setbeamertemplate{bibliography entry author}{\usebeamercolor[fg]{bibliography item}}
\setbeamertemplate{frametitle}{\medskip\usebeamerfont{frametitle}\color{gray}\raisebox{-2.5ex}[0ex][0ex]{\rule{0.1em}{4.5ex}}}
\addtobeamertemplate{frametitle}{}{\hspace{0.4em}\usebeamercolor[fg]{title}\insertframetitle\par\vspace{0.2ex}\hspace{0.5em}\usebeamerfont{framesubtitle}\insertframesubtitle}
\hypersetup{pdfborder={0 0 0},bookmarksnumbered=true,bookmarksopen=true,bookmarksopenlevel=0,pdftitle={\ecs{}: #1},pdfauthor={Florian Negele},pdfsubject={\ecs{}},pdfkeywords={#1}}
\renewcommand{\flowgraph}[1]{\resizebox{\textwidth}{!}{$$\xymatrix{##1}$$}}
\title{\ecs{}\medskip\hrule\medskip}
\institute{\shadowedecslogo{5em}{30}{15}}
\date{\version}
\subtitle{#1}
\begin{document}
\begin{frame}[plain]\titlepage\nocite{manual}\end{frame}
\begin{frame}{Contents}{#1}\begin{center}\tableofcontents\end{center}\end{frame}
}

\providecommand{\concludepresentation}{
\begin{frame}{References}\begin{footnotesize}\setlength{\columnseprule}{0.4pt}\begin{multicols}{2}\bibliography{references}\end{multicols}\end{footnotesize}\end{frame}
\end{document}
}

\providecommand{\startbook}[1]{
\documentclass[10pt,paper=17cm:24cm,DIV=13,twoside=semi,headings=normal,numbers=noendperiod,cleardoublepage=plain]{scrbook}
\usepackage{atveryend}
\usepackage{booktabs}
\usepackage{caption}
\usepackage{changepage}
\usepackage[T1]{fontenc}
\usepackage{imakeidx}
\usepackage{hyperref}
\usepackage[american]{isodate}
\usepackage{lmodern}
\usepackage{longtable}
\usepackage{mathptmx}
\usepackage[final]{microtype}
\usepackage{multicol}
\usepackage{multirow}
\usepackage[all]{nowidow}
\usepackage{pdfcomment}
\usepackage{scrlayer-scrpage}
\usepackage{setspace}
\usepackage{syntax}
\usepackage[eventxtindent=4pt,oddtxtexdent=4pt]{thumbs}
\usepackage{tikz}
\usepackage[all]{xy}
\hyphenation{Micro-Blaze Open-Cores Open-RISC Power-PC}
\hypersetup{pdfborder={0 0 0},bookmarksnumbered=true,bookmarksopen=true,bookmarksopenlevel=0,pdftitle={\ecs{}: #1},pdfauthor={Florian Negele},pdfsubject={\ecs{}},pdfkeywords={#1}}
\setlength{\grammarindent}{8em}\setlength{\grammarparsep}{0.7ex}
\setkomafont{captionlabel}{\usekomafont{descriptionlabel}}
\renewcommand{\arraystretch}{1.05}\setstretch{1.1}
\renewcommand{\chapterformat}{\thechapter\autodot\enskip\raisebox{-1ex}[0ex][0ex]{\color{gray}\rule{0.1em}{3.5ex}}\enskip}
\renewcommand{\startchapter}[4]{\hypertarget{##3}{\chapter{##1}}\label{##3}##4\addthumb{##1}{\LARGE\sffamily\bfseries\thechapter}{white}{gray}\renewcommand{\prefix}{##3}}
\renewcommand{\concludechapter}{\clearpage{\stopthumb\cleardoublepage}}
\renewcommand{\syntleft}{\itshape}\renewcommand{\syntright}{}
\renewcommand{\floatpagefraction}{0.7}
\renewcommand{\partheademptypage}{}
\DeclareMicrotypeAlias{lmss}{cmr}
\newcommand{\prefix}{}
\newcounter{instruction}
\bibliographystyle{unsrt}
\newif\ifbook\booktrue
\makeindex[intoc,title=Index]
\makeindex[intoc,name=tools,title=Index of Tools,columns=3]
\makeindex[intoc,name=library,title=Index of Library Names]
\makeindex[intoc,name=runtime,title=Index of Runtime Support]
\makeindex[intoc,name=environment,title=Index of Target Environments]
\indexsetup{toclevel=chapter,headers={\indexname}{\indexname}}
\frenchspacing
\begin{document}
\pagenumbering{alph}
\begin{titlepage}\centering
\huge\sffamily\null\vfill\textbf{\ecs{}}\bigskip\hrule\bigskip#1
\normalsize\normalfont\vfill\vfill\shadowedecslogo{10em}{30}{15}
\large\vfill\vfill\version
\end{titlepage}
\null\vfill
\thispagestyle{empty}
\noindent\today\par\medskip
\license A copy of this license is included in Appendix~\ref{fdl} on page~\pageref{fdl}.
All product names used herein are for identification purposes only and may be trademarks of their respective companies.
\concludechapter
\frontmatter
\setcounter{tocdepth}{1}
\tableofcontents
\setcounter{tocdepth}{2}
\concludechapter
\listoffigures
\concludechapter
\listoftables
\concludechapter
}

\providecommand{\concludebook}{
\backmatter
\addtocontents{toc}{\protect\setcounter{tocdepth}{-1}}
\phantomsection\addcontentsline{toc}{part}{Bibliography}
\bibliography{references}
\concludechapter
\phantomsection\addcontentsline{toc}{part}{Indexes}
\printindex
\concludechapter
\indexprologue{\label{idx:tools}}
\printindex[tools]
\concludechapter
\printindex[library]
\concludechapter
\indexprologue{\label{idx:runtime}}
\printindex[runtime]
\concludechapter
\indexprologue{\label{idx:environment}}
\printindex[environment]
\concludechapter
\pagestyle{empty}\pagenumbering{Alph}\null\clearpage
\null\vfill\centering\ecslogo{4em}\par\medskip\license
\end{document}
}

% chapter references

\providecommand{\seedocumentationref}{}\renewcommand{\seedocumentationref}[3]{#1, see \Documentation{}~\documentationref{#2}{#3}. }
\providecommand{\seeinterface}{}\renewcommand{\seeinterface}{\ifbook See \Documentation{}~\documentationref{interface}{User Interface} for more information about the common user interface of all of these tools. \fi}
\providecommand{\seeguide}{}\renewcommand{\seeguide}{\seedocumentationref{For basic examples of using some of these tools in practice}{guide}{User Guide}}
\providecommand{\seecpp}{}\renewcommand{\seecpp}{\seedocumentationref{For more information about the \cpp{} programming language and its implementation by the \ecs{}}{cpp}{User Manual for \cpp{}}}
\providecommand{\seefalse}{}\renewcommand{\seefalse}{\seedocumentationref{For more information about the FALSE programming language and its implementation by the \ecs{}}{false}{User Manual for FALSE}}
\providecommand{\seeoberon}{}\renewcommand{\seeoberon}{\seedocumentationref{For more information about the Oberon programming language and its implementation by the \ecs{}}{oberon}{User Manual for Oberon}}
\providecommand{\seeassembly}{}\renewcommand{\seeassembly}{\seedocumentationref{For more information about the generic assembly language and how to use it}{assembly}{Generic Assembly Language Specification}}
\providecommand{\seeamd}{}\renewcommand{\seeamd}{\seedocumentationref{For more information about how the \ecs{} supports the AMD64 hardware architecture}{amd64}{AMD64 Hardware Architecture Support}}
\providecommand{\seearm}{}\renewcommand{\seearm}{\seedocumentationref{For more information about how the \ecs{} supports the ARM hardware architecture}{arm}{ARM Hardware Architecture Support}}
\providecommand{\seeavr}{}\renewcommand{\seeavr}{\seedocumentationref{For more information about how the \ecs{} supports the AVR hardware architecture}{avr}{AVR Hardware Architecture Support}}
\providecommand{\seeavrtt}{}\renewcommand{\seeavrtt}{\seedocumentationref{For more information about how the \ecs{} supports the AVR32 hardware architecture}{avr32}{AVR32 Hardware Architecture Support}}
\providecommand{\seemabk}{}\renewcommand{\seemabk}{\seedocumentationref{For more information about how the \ecs{} supports the M68000 hardware architecture}{m68k}{M68000 Hardware Architecture Support}}
\providecommand{\seemibl}{}\renewcommand{\seemibl}{\seedocumentationref{For more information about how the \ecs{} supports the MicroBlaze hardware architecture}{mibl}{MicroBlaze Hardware Architecture Support}}
\providecommand{\seemips}{}\renewcommand{\seemips}{\seedocumentationref{For more information about how the \ecs{} supports the MIPS32 and MIPS64 hardware architectures}{mips}{MIPS Hardware Architecture Support}}
\providecommand{\seemmix}{}\renewcommand{\seemmix}{\seedocumentationref{For more information about how the \ecs{} supports the MMIX hardware architecture}{mmix}{MMIX Hardware Architecture Support}}
\providecommand{\seeorok}{}\renewcommand{\seeorok}{\seedocumentationref{For more information about how the \ecs{} supports the OpenRISC 1000 hardware architecture}{or1k}{OpenRISC 1000 Hardware Architecture Support}}
\providecommand{\seeppc}{}\renewcommand{\seeppc}{\seedocumentationref{For more information about how the \ecs{} supports the PowerPC hardware architecture}{ppc}{PowerPC Hardware Architecture Support}}
\providecommand{\seerisc}{}\renewcommand{\seerisc}{\seedocumentationref{For more information about how the \ecs{} supports the RISC hardware architecture}{risc}{RISC Hardware Architecture Support}}
\providecommand{\seewasm}{}\renewcommand{\seewasm}{\seedocumentationref{For more information about how the \ecs{} supports the WebAssembly architecture}{wasm}{WebAssembly Architecture Support}}
\providecommand{\seedocumentation}{}\renewcommand{\seedocumentation}{\seedocumentationref{For more information about generic documentations and their generation by the \ecs{}}{documentation}{Generic Documentation Generation}}
\providecommand{\seedebugging}{}\renewcommand{\seedebugging}{\seedocumentationref{For more information about debugging information and its representation}{debugging}{Debugging Information Representation}}
\providecommand{\seecode}{}\renewcommand{\seecode}{\seedocumentationref{For more information about intermediate code and its purpose}{code}{Intermediate Code Representation}}
\providecommand{\seeobject}{}\renewcommand{\seeobject}{\seedocumentationref{For more information about object files and their purpose}{object}{Object File Representation}}

% generic documentation tools

\providecommand{\docprint}{
\toolsection{docprint} is a pretty printer for generic documentations.
It reformats generic documentations and writes it to the standard output stream.
\debuggingtool
\flowgraph{\resource{generic\\documentation} \ar[r] & \toolbox{docprint} \ar[r] & \resource{generic\\documentation}}
\seedocumentation
}

\providecommand{\doccheck}{
\toolsection{doccheck} is a syntactic and semantic checker for generic documentations.
It just performs syntactic and semantic checks on generic documentations and writes its diagnostic messages to the standard error stream.
\debuggingtool
\flowgraph{\resource{generic\\documentation} \ar[r] & \toolbox{doccheck} \ar[r] & \resource{diagnostic\\messages}}
\seedocumentation
}

\providecommand{\dochtml}{
\toolsection{dochtml} is an HTML documentation generator for generic documentations.
It processes several generic documentations and assembles all information therein into an HTML document.
\debuggingtool
\flowgraph{\resource{generic\\documentation} \ar[r] & \toolbox{dochtml} \ar[r] & \resource{HTML\\document}}
\seedocumentation
}

\providecommand{\doclatex}{
\toolsection{doclatex} is a Latex documentation generator for generic documentations.
It processes several generic documentations and assembles all information therein into a Latex document.
\debuggingtool
\flowgraph{\resource{generic\\documentation} \ar[r] & \toolbox{doclatex} \ar[r] & \resource{Latex\\document}}
\seedocumentation
}

% intermediate code tools

\providecommand{\cdcheck}{
\toolsection{cdcheck} is a syntactic and semantic checker for intermediate code.
It just performs syntactic and semantic checks on programs written in intermediate code and writes its diagnostic messages to the standard error stream.
\debuggingtool
\flowgraph{\resource{intermediate\\code} \ar[r] & \toolbox{cdcheck} \ar[r] & \resource{diagnostic\\messages}}
\seeassembly\seecode
}

\providecommand{\cdopt}{
\toolsection{cdopt} is an optimizer for intermediate code.
It performs various optimizations on programs written in intermediate code and writes the result to the standard output stream.
\debuggingtool
\flowgraph{\resource{intermediate\\code} \ar[r] & \toolbox{cdopt} \ar[r] & \resource{optimized\\code}}
\seeassembly\seecode
}

\providecommand{\cdrun}{
\toolsection{cdrun} is an interpreter for intermediate code.
It processes and executes programs written in intermediate code.
The following code sections are predefined and have the usual semantics:
\texttt{abort}, \texttt{\_Exit}, \texttt{fflush}, \texttt{floor}, \texttt{fputc}, \texttt{free}, \texttt{getchar}, \texttt{malloc}, and \texttt{putchar}.
Diagnostic messages about invalid operations include the name of the executed code section and the index of the erroneous instruction.
\debuggingtool
\flowgraph{\resource{intermediate\\code} \ar[r] & \toolbox{cdrun} \ar@/u/[r] & \resource{input/\\output} \ar@/d/[l]}
\seeassembly\seecode
}

\providecommand{\cdamda}{
\toolsection{cdamd16} is a compiler for intermediate code targeting the AMD64 hardware architecture.
It generates machine code for AMD64 processors from programs written in intermediate code and stores it in corresponding object files.
The compiler generates machine code for the 16-bit operating mode defined by the AMD64 architecture.
It also creates a debugging information file as well as an assembly file containing a listing of the generated machine code.
\debuggingtool
\flowgraph{\resource{intermediate\\code} \ar[r] & \toolbox{cdamd16} \ar[r] \ar[d] \ar[rd] & \resource{object file} \\ & \resource{assembly\\listing} & \resource{debugging\\information}}
\seeassembly\seeamd\seeobject\seecode\seedebugging
}

\providecommand{\cdamdb}{
\toolsection{cdamd32} is a compiler for intermediate code targeting the AMD64 hardware architecture.
It generates machine code for AMD64 processors from programs written in intermediate code and stores it in corresponding object files.
The compiler generates machine code for the 32-bit operating mode defined by the AMD64 architecture.
It also creates a debugging information file as well as an assembly file containing a listing of the generated machine code.
\debuggingtool
\flowgraph{\resource{intermediate\\code} \ar[r] & \toolbox{cdamd32} \ar[r] \ar[d] \ar[rd] & \resource{object file} \\ & \resource{assembly\\listing} & \resource{debugging\\information}}
\seeassembly\seeamd\seeobject\seecode\seedebugging
}

\providecommand{\cdamdc}{
\toolsection{cdamd64} is a compiler for intermediate code targeting the AMD64 hardware architecture.
It generates machine code for AMD64 processors from programs written in intermediate code and stores it in corresponding object files.
The compiler generates machine code for the 64-bit operating mode defined by the AMD64 architecture.
It also creates a debugging information file as well as an assembly file containing a listing of the generated machine code.
\debuggingtool
\flowgraph{\resource{intermediate\\code} \ar[r] & \toolbox{cdamd64} \ar[r] \ar[d] \ar[rd] & \resource{object file} \\ & \resource{assembly\\listing} & \resource{debugging\\information}}
\seeassembly\seeamd\seeobject\seecode\seedebugging
}

\providecommand{\cdarma}{
\toolsection{cdarma32} is a compiler for intermediate code targeting the ARM hardware architecture.
It generates machine code for ARM processors executing A32 instructions from programs written in intermediate code and stores it in corresponding object files.
It also creates a debugging information file as well as an assembly file containing a listing of the generated machine code.
\debuggingtool
\flowgraph{\resource{intermediate\\code} \ar[r] & \toolbox{cdarma32} \ar[r] \ar[d] \ar[rd] & \resource{object file} \\ & \resource{assembly\\listing} & \resource{debugging\\information}}
\seeassembly\seearm\seeobject\seecode\seedebugging
}

\providecommand{\cdarmb}{
\toolsection{cdarma64} is a compiler for intermediate code targeting the ARM hardware architecture.
It generates machine code for ARM processors executing A64 instructions from programs written in intermediate code and stores it in corresponding object files.
It also creates a debugging information file as well as an assembly file containing a listing of the generated machine code.
\debuggingtool
\flowgraph{\resource{intermediate\\code} \ar[r] & \toolbox{cdarma64} \ar[r] \ar[d] \ar[rd] & \resource{object file} \\ & \resource{assembly\\listing} & \resource{debugging\\information}}
\seeassembly\seearm\seeobject\seecode\seedebugging
}

\providecommand{\cdarmc}{
\toolsection{cdarmt32} is a compiler for intermediate code targeting the ARM hardware architecture.
It generates machine code for ARM processors without floating-point extension executing T32 instructions from programs written in intermediate code and stores it in corresponding object files.
It also creates a debugging information file as well as an assembly file containing a listing of the generated machine code.
\debuggingtool
\flowgraph{\resource{intermediate\\code} \ar[r] & \toolbox{cdarmt32} \ar[r] \ar[d] \ar[rd] & \resource{object file} \\ & \resource{assembly\\listing} & \resource{debugging\\information}}
\seeassembly\seearm\seeobject\seecode\seedebugging
}

\providecommand{\cdarmcfpe}{
\toolsection{cdarmt32fpe} is a compiler for intermediate code targeting the ARM hardware architecture.
It generates machine code for ARM processors with floating-point extension executing T32 instructions from programs written in intermediate code and stores it in corresponding object files.
It also creates a debugging information file as well as an assembly file containing a listing of the generated machine code.
\debuggingtool
\flowgraph{\resource{intermediate\\code} \ar[r] & \toolbox{cdarmt32fpe} \ar[r] \ar[d] \ar[rd] & \resource{object file} \\ & \resource{assembly\\listing} & \resource{debugging\\information}}
\seeassembly\seearm\seeobject\seecode\seedebugging
}

\providecommand{\cdavr}{
\toolsection{cdavr} is a compiler for intermediate code targeting the AVR hardware architecture.
It generates machine code for AVR processors from programs written in intermediate code and stores it in corresponding object files.
It also creates a debugging information file as well as an assembly file containing a listing of the generated machine code.
\debuggingtool
\flowgraph{\resource{intermediate\\code} \ar[r] & \toolbox{cdavr} \ar[r] \ar[d] \ar[rd] & \resource{object file} \\ & \resource{assembly\\listing} & \resource{debugging\\information}}
\seeassembly\seeavr\seeobject\seecode\seedebugging
}

\providecommand{\cdavrtt}{
\toolsection{cdavr32} is a compiler for intermediate code targeting the AVR32 hardware architecture.
It generates machine code for AVR32 processors from programs written in intermediate code and stores it in corresponding object files.
It also creates a debugging information file as well as an assembly file containing a listing of the generated machine code.
\debuggingtool
\flowgraph{\resource{intermediate\\code} \ar[r] & \toolbox{cdavr32} \ar[r] \ar[d] \ar[rd] & \resource{object file} \\ & \resource{assembly\\listing} & \resource{debugging\\information}}
\seeassembly\seeavrtt\seeobject\seecode\seedebugging
}

\providecommand{\cdmabk}{
\toolsection{cdm68k} is a compiler for intermediate code targeting the M68000 hardware architecture.
It generates machine code for M68000 processors from programs written in intermediate code and stores it in corresponding object files.
It also creates a debugging information file as well as an assembly file containing a listing of the generated machine code.
\debuggingtool
\flowgraph{\resource{intermediate\\code} \ar[r] & \toolbox{cdm68k} \ar[r] \ar[d] \ar[rd] & \resource{object file} \\ & \resource{assembly\\listing} & \resource{debugging\\information}}
\seeassembly\seemabk\seeobject\seecode\seedebugging
}

\providecommand{\cdmibl}{
\toolsection{cdmibl} is a compiler for intermediate code targeting the MicroBlaze hardware architecture.
It generates machine code for MicroBlaze processors from programs written in intermediate code and stores it in corresponding object files.
It also creates a debugging information file as well as an assembly file containing a listing of the generated machine code.
\debuggingtool
\flowgraph{\resource{intermediate\\code} \ar[r] & \toolbox{cdmibl} \ar[r] \ar[d] \ar[rd] & \resource{object file} \\ & \resource{assembly\\listing} & \resource{debugging\\information}}
\seeassembly\seemibl\seeobject\seecode\seedebugging
}

\providecommand{\cdmipsa}{
\toolsection{cdmips32} is a compiler for intermediate code targeting the MIPS32 hardware architecture.
It generates machine code for MIPS32 processors from programs written in intermediate code and stores it in corresponding object files.
It also creates a debugging information file as well as an assembly file containing a listing of the generated machine code.
\debuggingtool
\flowgraph{\resource{intermediate\\code} \ar[r] & \toolbox{cdmips32} \ar[r] \ar[d] \ar[rd] & \resource{object file} \\ & \resource{assembly\\listing} & \resource{debugging\\information}}
\seeassembly\seemips\seeobject\seecode\seedebugging
}

\providecommand{\cdmipsb}{
\toolsection{cdmips64} is a compiler for intermediate code targeting the MIPS64 hardware architecture.
It generates machine code for MIPS64 processors from programs written in intermediate code and stores it in corresponding object files.
It also creates a debugging information file as well as an assembly file containing a listing of the generated machine code.
\debuggingtool
\flowgraph{\resource{intermediate\\code} \ar[r] & \toolbox{cdmips64} \ar[r] \ar[d] \ar[rd] & \resource{object file} \\ & \resource{assembly\\listing} & \resource{debugging\\information}}
\seeassembly\seemips\seeobject\seecode\seedebugging
}

\providecommand{\cdmmix}{
\toolsection{cdmmix} is a compiler for intermediate code targeting the MMIX hardware architecture.
It generates machine code for MMIX processors from programs written in intermediate code and stores it in corresponding object files.
It also creates a debugging information file as well as an assembly file containing a listing of the generated machine code.
\debuggingtool
\flowgraph{\resource{intermediate\\code} \ar[r] & \toolbox{cdmmix} \ar[r] \ar[d] \ar[rd] & \resource{object file} \\ & \resource{assembly\\listing} & \resource{debugging\\information}}
\seeassembly\seemmix\seeobject\seecode\seedebugging
}

\providecommand{\cdorok}{
\toolsection{cdor1k} is a compiler for intermediate code targeting the OpenRISC 1000 hardware architecture.
It generates machine code for OpenRISC 1000 processors from programs written in intermediate code and stores it in corresponding object files.
It also creates a debugging information file as well as an assembly file containing a listing of the generated machine code.
\debuggingtool
\flowgraph{\resource{intermediate\\code} \ar[r] & \toolbox{cdor1k} \ar[r] \ar[d] \ar[rd] & \resource{object file} \\ & \resource{assembly\\listing} & \resource{debugging\\information}}
\seeassembly\seeorok\seeobject\seecode\seedebugging
}

\providecommand{\cdppca}{
\toolsection{cdppc32} is a compiler for intermediate code targeting the PowerPC hardware architecture.
It generates machine code for PowerPC processors from programs written in intermediate code and stores it in corresponding object files.
The compiler generates machine code for the 32-bit operating mode defined by the PowerPC architecture.
It also creates a debugging information file as well as an assembly file containing a listing of the generated machine code.
\debuggingtool
\flowgraph{\resource{intermediate\\code} \ar[r] & \toolbox{cdppc32} \ar[r] \ar[d] \ar[rd] & \resource{object file} \\ & \resource{assembly\\listing} & \resource{debugging\\information}}
\seeassembly\seeppc\seeobject\seecode\seedebugging
}

\providecommand{\cdppcb}{
\toolsection{cdppc64} is a compiler for intermediate code targeting the PowerPC hardware architecture.
It generates machine code for PowerPC processors from programs written in intermediate code and stores it in corresponding object files.
The compiler generates machine code for the 64-bit operating mode defined by the PowerPC architecture.
It also creates a debugging information file as well as an assembly file containing a listing of the generated machine code.
\debuggingtool
\flowgraph{\resource{intermediate\\code} \ar[r] & \toolbox{cdppc64} \ar[r] \ar[d] \ar[rd] & \resource{object file} \\ & \resource{assembly\\listing} & \resource{debugging\\information}}
\seeassembly\seeppc\seeobject\seecode\seedebugging
}

\providecommand{\cdrisc}{
\toolsection{cdrisc} is a compiler for intermediate code targeting the RISC hardware architecture.
It generates machine code for RISC processors from programs written in intermediate code and stores it in corresponding object files.
It also creates a debugging information file as well as an assembly file containing a listing of the generated machine code.
\debuggingtool
\flowgraph{\resource{intermediate\\code} \ar[r] & \toolbox{cdrisc} \ar[r] \ar[d] \ar[rd] & \resource{object file} \\ & \resource{assembly\\listing} & \resource{debugging\\information}}
\seeassembly\seerisc\seeobject\seecode\seedebugging
}

\providecommand{\cdwasm}{
\toolsection{cdwasm} is a compiler for intermediate code targeting the WebAssembly architecture.
It generates machine code for WebAssembly targets from programs written in intermediate code and stores it in corresponding object files.
It also creates a debugging information file as well as an assembly file containing a listing of the generated machine code.
\debuggingtool
\flowgraph{\resource{intermediate\\code} \ar[r] & \toolbox{cdwasm} \ar[r] \ar[d] \ar[rd] & \resource{object file} \\ & \resource{assembly\\listing} & \resource{debugging\\information}}
\seeassembly\seewasm\seeobject\seecode\seedebugging
}

% C++ tools

\providecommand{\cppprep}{
\toolsection{cppprep} is a preprocessor for the \cpp{} programming language.
It preprocesses source code according to the rules of \cpp{} and writes it to the standard output stream.
Only the macro names \texttt{\_\_DATE\_\_}, \texttt{\_\_FILE\_\_}, \texttt{\_\_LINE\_\_}, and \texttt{\_\_TIME\_\_} are predefined.
\flowgraph{\resource{\cpp{} or other\\source code} \ar[r] & \toolbox{cppprep} \ar[r] & \resource{preprocessed\\source code} \\ & \variable{ECSINCLUDE} \ar[u]}
\seecpp
}

\providecommand{\cppprint}{
\toolsection{cppprint} is a pretty printer for the \cpp{} programming language.
It reformats the source code of \cpp{} programs and writes it to the standard output stream.
\flowgraph{\resource{\cpp{}\\source code} \ar[r] & \toolbox{cppprint} \ar[r] & \resource{reformatted\\source code} \\ & \variable{ECSINCLUDE} \ar[u]}
\seecpp
}

\providecommand{\cppcheck}{
\toolsection{cppcheck} is a syntactic and semantic checker for the \cpp{} programming language.
It just performs syntactic and semantic checks on \cpp{} programs and writes its diagnostic messages to the standard error stream.
\flowgraph{\resource{\cpp{}\\source code} \ar[r] & \toolbox{cppcheck} \ar[r] & \resource{diagnostic\\messages} \\ & \variable{ECSINCLUDE} \ar[u]}
\seecpp
}

\providecommand{\cppdump}{
\toolsection{cppdump} is a serializer for the \cpp{} programming language.
It dumps the complete internal representation of programs written in \cpp{} into an XML document.
\debuggingtool
\flowgraph{\resource{\cpp{}\\source code} \ar[r] & \toolbox{cppdump} \ar[r] & \resource{internal\\representation} \\ & \variable{ECSINCLUDE} \ar[u]}
\seecpp
}

\providecommand{\cpprun}{
\toolsection{cpprun} is an interpreter for the \cpp{} programming language.
It processes and executes programs written in \cpp{}.
The macro \texttt{\_\_run\_\_} is predefined in order to enable programmers to identify this tool while interpreting.
\flowgraph{\resource{\cpp{}\\source code} \ar[r] & \toolbox{cpprun} \ar@/u/[r] & \resource{input/\\output} \ar@/d/[l] \\ & \variable{ECSINCLUDE} \ar[u]}
\seecpp
}

\providecommand{\cppdoc}{
\toolsection{cppdoc} is a generic documentation generator for the \cpp{} programming language.
It processes several \cpp{} source files and assembles all information therein into a generic documentation.
\debuggingtool
\flowgraph{\resource{\cpp{}\\source code} \ar[r] & \toolbox{cppdoc} \ar[r] & \resource{generic\\documentation} \\ & \variable{ECSINCLUDE} \ar[u]}
\seecpp\seedocumentation
}

\providecommand{\cpphtml}{
\toolsection{cpphtml} is an HTML documentation generator for the \cpp{} programming language.
It processes several \cpp{} source files and assembles all information therein into an HTML document.
\flowgraph{\resource{\cpp{}\\source code} \ar[r] & \toolbox{cpphtml} \ar[r] & \resource{HTML\\document} \\ & \variable{ECSINCLUDE} \ar[u]}
\seecpp\seedocumentation
}

\providecommand{\cpplatex}{
\toolsection{cpplatex} is a Latex documentation generator for the \cpp{} programming language.
It processes several \cpp{} source files and assembles all information therein into a Latex document.
\flowgraph{\resource{\cpp{}\\source code} \ar[r] & \toolbox{cpplatex} \ar[r] & \resource{Latex\\document} \\ & \variable{ECSINCLUDE} \ar[u]}
\seecpp\seedocumentation
}

\providecommand{\cppcode}{
\toolsection{cppcode} is an intermediate code generator for the \cpp{} programming language.
It generates intermediate code from programs written in \cpp{} and stores it in corresponding assembly files.
The macro \texttt{\_\_code\_\_} is predefined in order to enable programmers to identify this tool while generating intermediate code.
Programs generated with this tool require additional runtime support that is stored in the \file{cpp\-code\-run} library file.
\debuggingtool
\flowgraph{\resource{\cpp{}\\source code} \ar[r] & \toolbox{cppcode} \ar[r] & \resource{intermediate\\code} \\ & \variable{ECSINCLUDE} \ar[u]}
\seecpp\seeassembly\seecode
}

\providecommand{\cppamda}{
\toolsection{cppamd16} is a compiler for the \cpp{} programming language targeting the AMD64 hardware architecture.
It generates machine code for AMD64 processors from programs written in \cpp{} and stores it in corresponding object files.
The compiler generates machine code for the 16-bit operating mode defined by the AMD64 architecture.
For debugging purposes, it also creates a debugging information file as well as an assembly file containing a listing of the generated machine code.
The macro \texttt{\_\_amd16\_\_} is predefined in order to enable programmers to identify this tool and its target architecture while compiling.
Programs generated with this compiler require additional runtime support that is stored in the \file{cpp\-amd16\-run} library file.
\flowgraph{\resource{\cpp{}\\source code} \ar[r] & \toolbox{cppamd16} \ar[r] \ar[d] \ar[rd] & \resource{object file} \\ \variable{ECSINCLUDE} \ar[ru] & \resource{debugging\\information} & \resource{assembly\\listing}}
\seecpp\seeassembly\seeamd\seeobject\seedebugging
}

\providecommand{\cppamdb}{
\toolsection{cppamd32} is a compiler for the \cpp{} programming language targeting the AMD64 hardware architecture.
It generates machine code for AMD64 processors from programs written in \cpp{} and stores it in corresponding object files.
The compiler generates machine code for the 32-bit operating mode defined by the AMD64 architecture.
For debugging purposes, it also creates a debugging information file as well as an assembly file containing a listing of the generated machine code.
The macro \texttt{\_\_amd32\_\_} is predefined in order to enable programmers to identify this tool and its target architecture while compiling.
Programs generated with this compiler require additional runtime support that is stored in the \file{cpp\-amd32\-run} library file.
\flowgraph{\resource{\cpp{}\\source code} \ar[r] & \toolbox{cppamd32} \ar[r] \ar[d] \ar[rd] & \resource{object file} \\ \variable{ECSINCLUDE} \ar[ru] & \resource{debugging\\information} & \resource{assembly\\listing}}
\seecpp\seeassembly\seeamd\seeobject\seedebugging
}

\providecommand{\cppamdc}{
\toolsection{cppamd64} is a compiler for the \cpp{} programming language targeting the AMD64 hardware architecture.
It generates machine code for AMD64 processors from programs written in \cpp{} and stores it in corresponding object files.
The compiler generates machine code for the 64-bit operating mode defined by the AMD64 architecture.
For debugging purposes, it also creates a debugging information file as well as an assembly file containing a listing of the generated machine code.
The macro \texttt{\_\_amd64\_\_} is predefined in order to enable programmers to identify this tool and its target architecture while compiling.
Programs generated with this compiler require additional runtime support that is stored in the \file{cpp\-amd64\-run} library file.
\flowgraph{\resource{\cpp{}\\source code} \ar[r] & \toolbox{cppamd64} \ar[r] \ar[d] \ar[rd] & \resource{object file} \\ \variable{ECSINCLUDE} \ar[ru] & \resource{debugging\\information} & \resource{assembly\\listing}}
\seecpp\seeassembly\seeamd\seeobject\seedebugging
}

\providecommand{\cpparma}{
\toolsection{cpparma32} is a compiler for the \cpp{} programming language targeting the ARM hardware architecture.
It generates machine code for ARM processors executing A32 instructions from programs written in \cpp{} and stores it in corresponding object files.
For debugging purposes, it also creates a debugging information file as well as an assembly file containing a listing of the generated machine code.
The macro \texttt{\_\_arma32\_\_} is predefined in order to enable programmers to identify this tool and its target architecture while compiling.
Programs generated with this compiler require additional runtime support that is stored in the \file{cpp\-arma32\-run} library file.
\flowgraph{\resource{\cpp{}\\source code} \ar[r] & \toolbox{cpparma32} \ar[r] \ar[d] \ar[rd] & \resource{object file} \\ \variable{ECSINCLUDE} \ar[ru] & \resource{debugging\\information} & \resource{assembly\\listing}}
\seecpp\seeassembly\seearm\seeobject\seedebugging
}

\providecommand{\cpparmb}{
\toolsection{cpparma64} is a compiler for the \cpp{} programming language targeting the ARM hardware architecture.
It generates machine code for ARM processors executing A64 instructions from programs written in \cpp{} and stores it in corresponding object files.
For debugging purposes, it also creates a debugging information file as well as an assembly file containing a listing of the generated machine code.
The macro \texttt{\_\_arma64\_\_} is predefined in order to enable programmers to identify this tool and its target architecture while compiling.
Programs generated with this compiler require additional runtime support that is stored in the \file{cpp\-arma64\-run} library file.
\flowgraph{\resource{\cpp{}\\source code} \ar[r] & \toolbox{cpparma64} \ar[r] \ar[d] \ar[rd] & \resource{object file} \\ \variable{ECSINCLUDE} \ar[ru] & \resource{debugging\\information} & \resource{assembly\\listing}}
\seecpp\seeassembly\seearm\seeobject\seedebugging
}

\providecommand{\cpparmc}{
\toolsection{cpparmt32} is a compiler for the \cpp{} programming language targeting the ARM hardware architecture.
It generates machine code for ARM processors without floating-point extension executing T32 instructions from programs written in \cpp{} and stores it in corresponding object files.
For debugging purposes, it also creates a debugging information file as well as an assembly file containing a listing of the generated machine code.
The macro \texttt{\_\_armt32\_\_} is predefined in order to enable programmers to identify this tool and its target architecture while compiling.
Programs generated with this compiler require additional runtime support that is stored in the \file{cpp\-armt32\-run} library file.
\flowgraph{\resource{\cpp{}\\source code} \ar[r] & \toolbox{cpparmt32} \ar[r] \ar[d] \ar[rd] & \resource{object file} \\ \variable{ECSINCLUDE} \ar[ru] & \resource{debugging\\information} & \resource{assembly\\listing}}
\seecpp\seeassembly\seearm\seeobject\seedebugging
}

\providecommand{\cpparmcfpe}{
\toolsection{cpparmt32fpe} is a compiler for the \cpp{} programming language targeting the ARM hardware architecture.
It generates machine code for ARM processors with floating-point extension executing T32 instructions from programs written in \cpp{} and stores it in corresponding object files.
For debugging purposes, it also creates a debugging information file as well as an assembly file containing a listing of the generated machine code.
The macro \texttt{\_\_armt32fpe\_\_} is predefined in order to enable programmers to identify this tool and its target architecture while compiling.
Programs generated with this compiler require additional runtime support that is stored in the \file{cpp\-armt32\-fpe\-run} library file.
\flowgraph{\resource{\cpp{}\\source code} \ar[r] & \toolbox{cpparmt32fpe} \ar[r] \ar[d] \ar[rd] & \resource{object file} \\ \variable{ECSINCLUDE} \ar[ru] & \resource{debugging\\information} & \resource{assembly\\listing}}
\seecpp\seeassembly\seearm\seeobject\seedebugging
}

\providecommand{\cppavr}{
\toolsection{cppavr} is a compiler for the \cpp{} programming language targeting the AVR hardware architecture.
It generates machine code for AVR processors from programs written in \cpp{} and stores it in corresponding object files.
For debugging purposes, it also creates a debugging information file as well as an assembly file containing a listing of the generated machine code.
The macro \texttt{\_\_avr\_\_} is predefined in order to enable programmers to identify this tool and its target architecture while compiling.
Programs generated with this compiler require additional runtime support that is stored in the \file{cpp\-avr\-run} library file.
\flowgraph{\resource{\cpp{}\\source code} \ar[r] & \toolbox{cppavr} \ar[r] \ar[d] \ar[rd] & \resource{object file} \\ \variable{ECSINCLUDE} \ar[ru] & \resource{debugging\\information} & \resource{assembly\\listing}}
\seecpp\seeassembly\seeavr\seeobject\seedebugging
}

\providecommand{\cppavrtt}{
\toolsection{cppavr32} is a compiler for the \cpp{} programming language targeting the AVR32 hardware architecture.
It generates machine code for AVR32 processors from programs written in \cpp{} and stores it in corresponding object files.
For debugging purposes, it also creates a debugging information file as well as an assembly file containing a listing of the generated machine code.
The macro \texttt{\_\_avr32\_\_} is predefined in order to enable programmers to identify this tool and its target architecture while compiling.
Programs generated with this compiler require additional runtime support that is stored in the \file{cpp\-avr32\-run} library file.
\flowgraph{\resource{\cpp{}\\source code} \ar[r] & \toolbox{cppavr32} \ar[r] \ar[d] \ar[rd] & \resource{object file} \\ \variable{ECSINCLUDE} \ar[ru] & \resource{debugging\\information} & \resource{assembly\\listing}}
\seecpp\seeassembly\seeavrtt\seeobject\seedebugging
}

\providecommand{\cppmabk}{
\toolsection{cppm68k} is a compiler for the \cpp{} programming language targeting the M68000 hardware architecture.
It generates machine code for M68000 processors from programs written in \cpp{} and stores it in corresponding object files.
For debugging purposes, it also creates a debugging information file as well as an assembly file containing a listing of the generated machine code.
The macro \texttt{\_\_m68k\_\_} is predefined in order to enable programmers to identify this tool and its target architecture while compiling.
Programs generated with this compiler require additional runtime support that is stored in the \file{cpp\-m68k\-run} library file.
\flowgraph{\resource{\cpp{}\\source code} \ar[r] & \toolbox{cppm68k} \ar[r] \ar[d] \ar[rd] & \resource{object file} \\ \variable{ECSINCLUDE} \ar[ru] & \resource{debugging\\information} & \resource{assembly\\listing}}
\seecpp\seeassembly\seemabk\seeobject\seedebugging
}

\providecommand{\cppmibl}{
\toolsection{cppmibl} is a compiler for the \cpp{} programming language targeting the MicroBlaze hardware architecture.
It generates machine code for MicroBlaze processors from programs written in \cpp{} and stores it in corresponding object files.
For debugging purposes, it also creates a debugging information file as well as an assembly file containing a listing of the generated machine code.
The macro \texttt{\_\_mibl\_\_} is predefined in order to enable programmers to identify this tool and its target architecture while compiling.
Programs generated with this compiler require additional runtime support that is stored in the \file{cpp\-mibl\-run} library file.
\flowgraph{\resource{\cpp{}\\source code} \ar[r] & \toolbox{cppmibl} \ar[r] \ar[d] \ar[rd] & \resource{object file} \\ \variable{ECSINCLUDE} \ar[ru] & \resource{debugging\\information} & \resource{assembly\\listing}}
\seecpp\seeassembly\seemibl\seeobject\seedebugging
}

\providecommand{\cppmipsa}{
\toolsection{cppmips32} is a compiler for the \cpp{} programming language targeting the MIPS32 hardware architecture.
It generates machine code for MIPS32 processors from programs written in \cpp{} and stores it in corresponding object files.
For debugging purposes, it also creates a debugging information file as well as an assembly file containing a listing of the generated machine code.
The macro \texttt{\_\_mips32\_\_} is predefined in order to enable programmers to identify this tool and its target architecture while compiling.
Programs generated with this compiler require additional runtime support that is stored in the \file{cpp\-mips32\-run} library file.
\flowgraph{\resource{\cpp{}\\source code} \ar[r] & \toolbox{cppmips32} \ar[r] \ar[d] \ar[rd] & \resource{object file} \\ \variable{ECSINCLUDE} \ar[ru] & \resource{debugging\\information} & \resource{assembly\\listing}}
\seecpp\seeassembly\seemips\seeobject\seedebugging
}

\providecommand{\cppmipsb}{
\toolsection{cppmips64} is a compiler for the \cpp{} programming language targeting the MIPS64 hardware architecture.
It generates machine code for MIPS64 processors from programs written in \cpp{} and stores it in corresponding object files.
For debugging purposes, it also creates a debugging information file as well as an assembly file containing a listing of the generated machine code.
The macro \texttt{\_\_mips64\_\_} is predefined in order to enable programmers to identify this tool and its target architecture while compiling.
Programs generated with this compiler require additional runtime support that is stored in the \file{cpp\-mips64\-run} library file.
\flowgraph{\resource{\cpp{}\\source code} \ar[r] & \toolbox{cppmips64} \ar[r] \ar[d] \ar[rd] & \resource{object file} \\ \variable{ECSINCLUDE} \ar[ru] & \resource{debugging\\information} & \resource{assembly\\listing}}
\seecpp\seeassembly\seemips\seeobject\seedebugging
}

\providecommand{\cppmmix}{
\toolsection{cppmmix} is a compiler for the \cpp{} programming language targeting the MMIX hardware architecture.
It generates machine code for MMIX processors from programs written in \cpp{} and stores it in corresponding object files.
For debugging purposes, it also creates a debugging information file as well as an assembly file containing a listing of the generated machine code.
The macro \texttt{\_\_mmix\_\_} is predefined in order to enable programmers to identify this tool and its target architecture while compiling.
Programs generated with this compiler require additional runtime support that is stored in the \file{cpp\-mmix\-run} library file.
\flowgraph{\resource{\cpp{}\\source code} \ar[r] & \toolbox{cppmmix} \ar[r] \ar[d] \ar[rd] & \resource{object file} \\ \variable{ECSINCLUDE} \ar[ru] & \resource{debugging\\information} & \resource{assembly\\listing}}
\seecpp\seeassembly\seemmix\seeobject\seedebugging
}

\providecommand{\cpporok}{
\toolsection{cppor1k} is a compiler for the \cpp{} programming language targeting the OpenRISC 1000 hardware architecture.
It generates machine code for OpenRISC 1000 processors from programs written in \cpp{} and stores it in corresponding object files.
For debugging purposes, it also creates a debugging information file as well as an assembly file containing a listing of the generated machine code.
The macro \texttt{\_\_or1k\_\_} is predefined in order to enable programmers to identify this tool and its target architecture while compiling.
Programs generated with this compiler require additional runtime support that is stored in the \file{cpp\-or1k\-run} library file.
\flowgraph{\resource{\cpp{}\\source code} \ar[r] & \toolbox{cppor1k} \ar[r] \ar[d] \ar[rd] & \resource{object file} \\ \variable{ECSINCLUDE} \ar[ru] & \resource{debugging\\information} & \resource{assembly\\listing}}
\seecpp\seeassembly\seeorok\seeobject\seedebugging
}

\providecommand{\cppppca}{
\toolsection{cppppc32} is a compiler for the \cpp{} programming language targeting the PowerPC hardware architecture.
It generates machine code for PowerPC processors from programs written in \cpp{} and stores it in corresponding object files.
The compiler generates machine code for the 32-bit operating mode defined by the PowerPC architecture.
For debugging purposes, it also creates a debugging information file as well as an assembly file containing a listing of the generated machine code.
The macro \texttt{\_\_ppc32\_\_} is predefined in order to enable programmers to identify this tool and its target architecture while compiling.
Programs generated with this compiler require additional runtime support that is stored in the \file{cpp\-ppc32\-run} library file.
\flowgraph{\resource{\cpp{}\\source code} \ar[r] & \toolbox{cppppc32} \ar[r] \ar[d] \ar[rd] & \resource{object file} \\ \variable{ECSINCLUDE} \ar[ru] & \resource{debugging\\information} & \resource{assembly\\listing}}
\seecpp\seeassembly\seeppc\seeobject\seedebugging
}

\providecommand{\cppppcb}{
\toolsection{cppppc64} is a compiler for the \cpp{} programming language targeting the PowerPC hardware architecture.
It generates machine code for PowerPC processors from programs written in \cpp{} and stores it in corresponding object files.
The compiler generates machine code for the 64-bit operating mode defined by the PowerPC architecture.
For debugging purposes, it also creates a debugging information file as well as an assembly file containing a listing of the generated machine code.
The macro \texttt{\_\_ppc64\_\_} is predefined in order to enable programmers to identify this tool and its target architecture while compiling.
Programs generated with this compiler require additional runtime support that is stored in the \file{cpp\-ppc64\-run} library file.
\flowgraph{\resource{\cpp{}\\source code} \ar[r] & \toolbox{cppppc64} \ar[r] \ar[d] \ar[rd] & \resource{object file} \\ \variable{ECSINCLUDE} \ar[ru] & \resource{debugging\\information} & \resource{assembly\\listing}}
\seecpp\seeassembly\seeppc\seeobject\seedebugging
}

\providecommand{\cpprisc}{
\toolsection{cpprisc} is a compiler for the \cpp{} programming language targeting the RISC hardware architecture.
It generates machine code for RISC processors from programs written in \cpp{} and stores it in corresponding object files.
For debugging purposes, it also creates a debugging information file as well as an assembly file containing a listing of the generated machine code.
The macro \texttt{\_\_risc\_\_} is predefined in order to enable programmers to identify this tool and its target architecture while compiling.
Programs generated with this compiler require additional runtime support that is stored in the \file{cpp\-risc\-run} library file.
\flowgraph{\resource{\cpp{}\\source code} \ar[r] & \toolbox{cpprisc} \ar[r] \ar[d] \ar[rd] & \resource{object file} \\ \variable{ECSINCLUDE} \ar[ru] & \resource{debugging\\information} & \resource{assembly\\listing}}
\seecpp\seeassembly\seerisc\seeobject\seedebugging
}

\providecommand{\cppwasm}{
\toolsection{cppwasm} is a compiler for the \cpp{} programming language targeting the WebAssembly architecture.
It generates machine code for WebAssembly targets from programs written in \cpp{} and stores it in corresponding object files.
For debugging purposes, it also creates a debugging information file as well as an assembly file containing a listing of the generated machine code.
The macro \texttt{\_\_wasm\_\_} is predefined in order to enable programmers to identify this tool and its target architecture while compiling.
Programs generated with this compiler require additional runtime support that is stored in the \file{cpp\-wasm\-run} library file.
\flowgraph{\resource{\cpp{}\\source code} \ar[r] & \toolbox{cppwasm} \ar[r] \ar[d] \ar[rd] & \resource{object file} \\ \variable{ECSINCLUDE} \ar[ru] & \resource{debugging\\information} & \resource{assembly\\listing}}
\seecpp\seeassembly\seewasm\seeobject\seedebugging
}

% FALSE tools

\providecommand{\falprint}{
\toolsection{falprint} is a pretty printer for the FALSE programming language.
It reformats the source code of FALSE programs and writes it to the standard output stream.
\flowgraph{\resource{FALSE\\source code} \ar[r] & \toolbox{falprint} \ar[r] & \resource{reformatted\\source code}}
\seefalse
}

\providecommand{\falcheck}{
\toolsection{falcheck} is a syntactic and semantic checker for the FALSE programming language.
It just performs syntactic and semantic checks on FALSE programs and writes its diagnostic messages to the standard error stream.
\flowgraph{\resource{FALSE\\source code} \ar[r] & \toolbox{falcheck} \ar[r] & \resource{diagnostic\\messages}}
\seefalse
}

\providecommand{\faldump}{
\toolsection{faldump} is a serializer for the FALSE programming language.
It dumps the complete internal representation of programs written in FALSE into an XML document.
\debuggingtool
\flowgraph{\resource{FALSE\\source code} \ar[r] & \toolbox{faldump} \ar[r] & \resource{internal\\representation}}
\seefalse
}

\providecommand{\falrun}{
\toolsection{falrun} is an interpreter for the FALSE programming language.
It processes and executes programs written in FALSE\@.
\flowgraph{\resource{FALSE\\source code} \ar[r] & \toolbox{falrun} \ar@/u/[r] & \resource{input/\\output} \ar@/d/[l]}
\seefalse
}

\providecommand{\falcpp}{
\toolsection{falcpp} is a transpiler for the FALSE programming language.
It translates programs written in FALSE into \cpp{} programs and stores them in corresponding source files.
\flowgraph{\resource{FALSE\\source code} \ar[r] & \toolbox{falcpp} \ar[r] & \resource{\cpp{}\\source file}}
\seefalse\seecpp
}

\providecommand{\falcode}{
\toolsection{falcode} is an intermediate code generator for the FALSE programming language.
It generates intermediate code from programs written in FALSE and stores it in corresponding assembly files.
\debuggingtool
\flowgraph{\resource{FALSE\\source code} \ar[r] & \toolbox{falcode} \ar[r] & \resource{intermediate\\code}}
\seefalse\seeassembly\seecode
}

\providecommand{\falamda}{
\toolsection{falamd16} is a compiler for the FALSE programming language targeting the AMD64 hardware architecture.
It generates machine code for AMD64 processors from programs written in FALSE and stores it in corresponding object files.
The compiler generates machine code for the 16-bit operating mode defined by the AMD64 architecture.
\flowgraph{\resource{FALSE\\source code} \ar[r] & \toolbox{falamd16} \ar[r] & \resource{object file}}
\seefalse\seeamd\seeobject
}

\providecommand{\falamdb}{
\toolsection{falamd32} is a compiler for the FALSE programming language targeting the AMD64 hardware architecture.
It generates machine code for AMD64 processors from programs written in FALSE and stores it in corresponding object files.
The compiler generates machine code for the 32-bit operating mode defined by the AMD64 architecture.
\flowgraph{\resource{FALSE\\source code} \ar[r] & \toolbox{falamd32} \ar[r] & \resource{object file}}
\seefalse\seeamd\seeobject
}

\providecommand{\falamdc}{
\toolsection{falamd64} is a compiler for the FALSE programming language targeting the AMD64 hardware architecture.
It generates machine code for AMD64 processors from programs written in FALSE and stores it in corresponding object files.
The compiler generates machine code for the 64-bit operating mode defined by the AMD64 architecture.
\flowgraph{\resource{FALSE\\source code} \ar[r] & \toolbox{falamd64} \ar[r] & \resource{object file}}
\seefalse\seeamd\seeobject
}

\providecommand{\falarma}{
\toolsection{falarma32} is a compiler for the FALSE programming language targeting the ARM hardware architecture.
It generates machine code for ARM processors executing A32 instructions from programs written in FALSE and stores it in corresponding object files.
\flowgraph{\resource{FALSE\\source code} \ar[r] & \toolbox{falarma32} \ar[r] & \resource{object file}}
\seefalse\seearm\seeobject
}

\providecommand{\falarmb}{
\toolsection{falarma64} is a compiler for the FALSE programming language targeting the ARM hardware architecture.
It generates machine code for ARM processors executing A64 instructions from programs written in FALSE and stores it in corresponding object files.
\flowgraph{\resource{FALSE\\source code} \ar[r] & \toolbox{falarma64} \ar[r] & \resource{object file}}
\seefalse\seearm\seeobject
}

\providecommand{\falarmc}{
\toolsection{falarmt32} is a compiler for the FALSE programming language targeting the ARM hardware architecture.
It generates machine code for ARM processors without floating-point extension executing T32 instructions from programs written in FALSE and stores it in corresponding object files.
\flowgraph{\resource{FALSE\\source code} \ar[r] & \toolbox{falarmt32} \ar[r] & \resource{object file}}
\seefalse\seearm\seeobject
}

\providecommand{\falarmcfpe}{
\toolsection{falarmt32fpe} is a compiler for the FALSE programming language targeting the ARM hardware architecture.
It generates machine code for ARM processors with floating-point extension executing T32 instructions from programs written in FALSE and stores it in corresponding object files.
\flowgraph{\resource{FALSE\\source code} \ar[r] & \toolbox{falarmt32fpe} \ar[r] & \resource{object file}}
\seefalse\seearm\seeobject
}

\providecommand{\falavr}{
\toolsection{falavr} is a compiler for the FALSE programming language targeting the AVR hardware architecture.
It generates machine code for AVR processors from programs written in FALSE and stores it in corresponding object files.
\flowgraph{\resource{FALSE\\source code} \ar[r] & \toolbox{falavr} \ar[r] & \resource{object file}}
\seefalse\seeavr\seeobject
}

\providecommand{\falavrtt}{
\toolsection{falavr32} is a compiler for the FALSE programming language targeting the AVR32 hardware architecture.
It generates machine code for AVR32 processors from programs written in FALSE and stores it in corresponding object files.
\flowgraph{\resource{FALSE\\source code} \ar[r] & \toolbox{falavr32} \ar[r] & \resource{object file}}
\seefalse\seeavrtt\seeobject
}

\providecommand{\falmabk}{
\toolsection{falm68k} is a compiler for the FALSE programming language targeting the M68000 hardware architecture.
It generates machine code for M68000 processors from programs written in FALSE and stores it in corresponding object files.
\flowgraph{\resource{FALSE\\source code} \ar[r] & \toolbox{falm68k} \ar[r] & \resource{object file}}
\seefalse\seemabk\seeobject
}

\providecommand{\falmibl}{
\toolsection{falmibl} is a compiler for the FALSE programming language targeting the MicroBlaze hardware architecture.
It generates machine code for MicroBlaze processors from programs written in FALSE and stores it in corresponding object files.
\flowgraph{\resource{FALSE\\source code} \ar[r] & \toolbox{falmibl} \ar[r] & \resource{object file}}
\seefalse\seemibl\seeobject
}

\providecommand{\falmipsa}{
\toolsection{falmips32} is a compiler for the FALSE programming language targeting the MIPS32 hardware architecture.
It generates machine code for MIPS32 processors from programs written in FALSE and stores it in corresponding object files.
\flowgraph{\resource{FALSE\\source code} \ar[r] & \toolbox{falmips32} \ar[r] & \resource{object file}}
\seefalse\seemips\seeobject
}

\providecommand{\falmipsb}{
\toolsection{falmips64} is a compiler for the FALSE programming language targeting the MIPS64 hardware architecture.
It generates machine code for MIPS64 processors from programs written in FALSE and stores it in corresponding object files.
\flowgraph{\resource{FALSE\\source code} \ar[r] & \toolbox{falmips64} \ar[r] & \resource{object file}}
\seefalse\seemips\seeobject
}

\providecommand{\falmmix}{
\toolsection{falmmix} is a compiler for the FALSE programming language targeting the MMIX hardware architecture.
It generates machine code for MMIX processors from programs written in FALSE and stores it in corresponding object files.
\flowgraph{\resource{FALSE\\source code} \ar[r] & \toolbox{falmmix} \ar[r] & \resource{object file}}
\seefalse\seemmix\seeobject
}

\providecommand{\falorok}{
\toolsection{falor1k} is a compiler for the FALSE programming language targeting the OpenRISC 1000 hardware architecture.
It generates machine code for OpenRISC 1000 processors from programs written in FALSE and stores it in corresponding object files.
\flowgraph{\resource{FALSE\\source code} \ar[r] & \toolbox{falor1k} \ar[r] & \resource{object file}}
\seefalse\seeorok\seeobject
}

\providecommand{\falppca}{
\toolsection{falppc32} is a compiler for the FALSE programming language targeting the PowerPC hardware architecture.
It generates machine code for PowerPC processors from programs written in FALSE and stores it in corresponding object files.
The compiler generates machine code for the 32-bit operating mode defined by the PowerPC architecture.
\flowgraph{\resource{FALSE\\source code} \ar[r] & \toolbox{falppc32} \ar[r] & \resource{object file}}
\seefalse\seeppc\seeobject
}

\providecommand{\falppcb}{
\toolsection{falppc64} is a compiler for the FALSE programming language targeting the PowerPC hardware architecture.
It generates machine code for PowerPC processors from programs written in FALSE and stores it in corresponding object files.
The compiler generates machine code for the 64-bit operating mode defined by the PowerPC architecture.
\flowgraph{\resource{FALSE\\source code} \ar[r] & \toolbox{falppc64} \ar[r] & \resource{object file}}
\seefalse\seeppc\seeobject
}

\providecommand{\falrisc}{
\toolsection{falrisc} is a compiler for the FALSE programming language targeting the RISC hardware architecture.
It generates machine code for RISC processors from programs written in FALSE and stores it in corresponding object files.
\flowgraph{\resource{FALSE\\source code} \ar[r] & \toolbox{falrisc} \ar[r] & \resource{object file}}
\seefalse\seerisc\seeobject
}

\providecommand{\falwasm}{
\toolsection{falwasm} is a compiler for the FALSE programming language targeting the WebAssembly architecture.
It generates machine code for WebAssembly targets from programs written in FALSE and stores it in corresponding object files.
\flowgraph{\resource{FALSE\\source code} \ar[r] & \toolbox{falwasm} \ar[r] & \resource{object file}}
\seefalse\seewasm\seeobject
}

% Oberon tools

\providecommand{\obprint}{
\toolsection{obprint} is a pretty printer for the Oberon programming language.
It reformats the source code of Oberon modules and writes it to the standard output stream.
\flowgraph{\resource{Oberon\\source code} \ar[r] & \toolbox{obprint} \ar[r] & \resource{reformatted\\source code}}
\seeoberon
}

\providecommand{\obcheck}{
\toolsection{obcheck} is a syntactic and semantic checker for the Oberon programming language.
It just performs syntactic and semantic checks on Oberon modules and writes its diagnostic messages to the standard error stream.
In addition, it stores the interface of each module in a symbol file which is required when other modules import the module.
\flowgraph{\resource{Oberon\\source code} \ar[r] & \toolbox{obcheck} \ar[r] \ar@/l/[d] & \resource{diagnostic\\messages} \\ \variable{ECSIMPORT} \ar[ru] & \resource{symbol\\files} \ar@/r/[u]}
\seeoberon
}

\providecommand{\obdump}{
\toolsection{obdump} is a serializer for the Oberon programming language.
It dumps the complete internal representation of modules written in Oberon into an XML document.
\debuggingtool
\flowgraph{\resource{Oberon\\source code} \ar[r] & \toolbox{obdump} \ar[r] \ar@/l/[d] & \resource{internal\\representation} \\ \variable{ECSIMPORT} \ar[ru] & \resource{symbol\\files} \ar@/r/[u]}
\seeoberon
}

\providecommand{\obrun}{
\toolsection{obrun} is an interpreter for the Oberon programming language.
It processes and executes modules written in Oberon.
This tool does neither generate nor process symbol files while interpreting modules.
If a module is imported by another one, its filename has to be named before the other one in the list of command-line arguments.
\flowgraph{\resource{Oberon\\source code} \ar[r] & \toolbox{obrun} \ar@/u/[r] & \resource{input/\\output} \ar@/d/[l]}
\seeoberon
}

\providecommand{\obcpp}{
\toolsection{obcpp} is a transpiler for the Oberon programming language.
It translates programs written in Oberon into \cpp{} programs and stores them in corresponding source and header files.
In addition, it stores the interface of each module in a symbol file which is required when other modules import the module.
The same interface is provided by the generated header file which can be used in other parts of the \cpp{} program.
\flowgraph{\resource{Oberon\\source code} \ar[r] & \toolbox{obcpp} \ar[r] \ar@/l/[d] \ar[rd] & \resource{\cpp{}\\source file} \\ \variable{ECSIMPORT} \ar[ru] & \resource{symbol\\files} \ar@/r/[u] & \resource{\cpp{}\\header file}}
\seeoberon\seecpp
}

\providecommand{\obdoc}{
\toolsection{obdoc} is a generic documentation generator for the Oberon programming language.
It processes several Oberon modules and assembles all information therein into a generic documentation.
In addition, it stores the interface of each module in a symbol file which is required when other modules import the module.
\debuggingtool
\flowgraph{\resource{Oberon\\source code} \ar[r] & \toolbox{obdoc} \ar[r] \ar@/l/[d] & \resource{generic\\documentation} \\ \variable{ECSIMPORT} \ar[ru] & \resource{symbol\\files} \ar@/r/[u]}
\seeoberon\seedocumentation
}

\providecommand{\obhtml}{
\toolsection{obhtml} is an HTML documentation generator for the Oberon programming language.
It processes several Oberon modules and assembles all information therein into an HTML document.
In addition, it stores the interface of each module in a symbol file which is required when other modules import the module.
\flowgraph{\resource{Oberon\\source code} \ar[r] & \toolbox{obhtml} \ar[r] \ar@/l/[d] & \resource{HTML\\document} \\ \variable{ECSIMPORT} \ar[ru] & \resource{symbol\\files} \ar@/r/[u]}
\seeoberon\seedocumentation
}

\providecommand{\oblatex}{
\toolsection{oblatex} is a Latex documentation generator for the Oberon programming language.
It processes several Oberon modules and assembles all information therein into a Latex document.
In addition, it stores the interface of each module in a symbol file which is required when other modules import the module.
\flowgraph{\resource{Oberon\\source code} \ar[r] & \toolbox{oblatex} \ar[r] \ar@/l/[d] & \resource{Latex\\document} \\ \variable{ECSIMPORT} \ar[ru] & \resource{symbol\\files} \ar@/r/[u]}
\seeoberon\seedocumentation
}

\providecommand{\obcode}{
\toolsection{obcode} is an intermediate code generator for the Oberon programming language.
It generates intermediate code from modules written in Oberon and stores it in corresponding assembly files.
In addition, it stores the interface of each module in a symbol file which is required when other modules import the module.
Programs generated with this tool require additional runtime support that is stored in the \file{ob\-code\-run} library file.
\debuggingtool
\flowgraph{\resource{Oberon\\source code} \ar[r] & \toolbox{obcode} \ar[r] \ar@/l/[d] & \resource{intermediate\\code} \\ \variable{ECSIMPORT} \ar[ru] & \resource{symbol\\files} \ar@/r/[u]}
\seeoberon\seeassembly\seecode
}

\providecommand{\obamda}{
\toolsection{obamd16} is a compiler for the Oberon programming language targeting the AMD64 hardware architecture.
It generates machine code for AMD64 processors from modules written in Oberon and stores it in corresponding object files.
The compiler generates machine code for the 16-bit operating mode defined by the AMD64 architecture.
For debugging purposes, it also creates a debugging information file as well as an assembly file containing a listing of the generated machine code.
In addition, it stores the interface of each module in a symbol file which is required when other modules import the module.
Programs generated with this compiler require additional runtime support that is stored in the \file{ob\-amd16\-run} library file.
\flowgraph{\resource{Oberon\\source code} \ar[r] & \toolbox{obamd16} \ar[r] \ar@/l/[d] \ar[rd] & \resource{object file} \\ \variable{ECSIMPORT} \ar[ru] & \resource{symbol\\files} \ar@/r/[u] & \resource{debugging\\information}}
\seeoberon\seeassembly\seeamd\seeobject\seedebugging
}

\providecommand{\obamdb}{
\toolsection{obamd32} is a compiler for the Oberon programming language targeting the AMD64 hardware architecture.
It generates machine code for AMD64 processors from modules written in Oberon and stores it in corresponding object files.
The compiler generates machine code for the 32-bit operating mode defined by the AMD64 architecture.
For debugging purposes, it also creates a debugging information file as well as an assembly file containing a listing of the generated machine code.
In addition, it stores the interface of each module in a symbol file which is required when other modules import the module.
Programs generated with this compiler require additional runtime support that is stored in the \file{ob\-amd32\-run} library file.
\flowgraph{\resource{Oberon\\source code} \ar[r] & \toolbox{obamd32} \ar[r] \ar@/l/[d] \ar[rd] & \resource{object file} \\ \variable{ECSIMPORT} \ar[ru] & \resource{symbol\\files} \ar@/r/[u] & \resource{debugging\\information}}
\seeoberon\seeassembly\seeamd\seeobject\seedebugging
}

\providecommand{\obamdc}{
\toolsection{obamd64} is a compiler for the Oberon programming language targeting the AMD64 hardware architecture.
It generates machine code for AMD64 processors from modules written in Oberon and stores it in corresponding object files.
The compiler generates machine code for the 64-bit operating mode defined by the AMD64 architecture.
For debugging purposes, it also creates a debugging information file as well as an assembly file containing a listing of the generated machine code.
In addition, it stores the interface of each module in a symbol file which is required when other modules import the module.
Programs generated with this compiler require additional runtime support that is stored in the \file{ob\-amd64\-run} library file.
\flowgraph{\resource{Oberon\\source code} \ar[r] & \toolbox{obamd64} \ar[r] \ar@/l/[d] \ar[rd] & \resource{object file} \\ \variable{ECSIMPORT} \ar[ru] & \resource{symbol\\files} \ar@/r/[u] & \resource{debugging\\information}}
\seeoberon\seeassembly\seeamd\seeobject\seedebugging
}

\providecommand{\obarma}{
\toolsection{obarma32} is a compiler for the Oberon programming language targeting the ARM hardware architecture.
It generates machine code for ARM processors executing A32 instructions from modules written in Oberon and stores it in corresponding object files.
For debugging purposes, it also creates a debugging information file as well as an assembly file containing a listing of the generated machine code.
In addition, it stores the interface of each module in a symbol file which is required when other modules import the module.
Programs generated with this compiler require additional runtime support that is stored in the \file{ob\-arma32\-run} library file.
\flowgraph{\resource{Oberon\\source code} \ar[r] & \toolbox{obarma32} \ar[r] \ar@/l/[d] \ar[rd] & \resource{object file} \\ \variable{ECSIMPORT} \ar[ru] & \resource{symbol\\files} \ar@/r/[u] & \resource{debugging\\information}}
\seeoberon\seeassembly\seearm\seeobject\seedebugging
}

\providecommand{\obarmb}{
\toolsection{obarma64} is a compiler for the Oberon programming language targeting the ARM hardware architecture.
It generates machine code for ARM processors executing A64 instructions from modules written in Oberon and stores it in corresponding object files.
For debugging purposes, it also creates a debugging information file as well as an assembly file containing a listing of the generated machine code.
In addition, it stores the interface of each module in a symbol file which is required when other modules import the module.
Programs generated with this compiler require additional runtime support that is stored in the \file{ob\-arma64\-run} library file.
\flowgraph{\resource{Oberon\\source code} \ar[r] & \toolbox{obarma64} \ar[r] \ar@/l/[d] \ar[rd] & \resource{object file} \\ \variable{ECSIMPORT} \ar[ru] & \resource{symbol\\files} \ar@/r/[u] & \resource{debugging\\information}}
\seeoberon\seeassembly\seearm\seeobject\seedebugging
}

\providecommand{\obarmc}{
\toolsection{obarmt32} is a compiler for the Oberon programming language targeting the ARM hardware architecture.
It generates machine code for ARM processors without floating-point extension executing T32 instructions from modules written in Oberon and stores it in corresponding object files.
For debugging purposes, it also creates a debugging information file as well as an assembly file containing a listing of the generated machine code.
In addition, it stores the interface of each module in a symbol file which is required when other modules import the module.
Programs generated with this compiler require additional runtime support that is stored in the \file{ob\-armt32\-run} library file.
\flowgraph{\resource{Oberon\\source code} \ar[r] & \toolbox{obarmt32} \ar[r] \ar@/l/[d] \ar[rd] & \resource{object file} \\ \variable{ECSIMPORT} \ar[ru] & \resource{symbol\\files} \ar@/r/[u] & \resource{debugging\\information}}
\seeoberon\seeassembly\seearm\seeobject\seedebugging
}

\providecommand{\obarmcfpe}{
\toolsection{obarmt32fpe} is a compiler for the Oberon programming language targeting the ARM hardware architecture.
It generates machine code for ARM processors with floating-point extension executing T32 instructions from modules written in Oberon and stores it in corresponding object files.
For debugging purposes, it also creates a debugging information file as well as an assembly file containing a listing of the generated machine code.
In addition, it stores the interface of each module in a symbol file which is required when other modules import the module.
Programs generated with this compiler require additional runtime support that is stored in the \file{ob\-armt32\-fpe\-run} library file.
\flowgraph{\resource{Oberon\\source code} \ar[r] & \toolbox{obarmt32fpe} \ar[r] \ar@/l/[d] \ar[rd] & \resource{object file} \\ \variable{ECSIMPORT} \ar[ru] & \resource{symbol\\files} \ar@/r/[u] & \resource{debugging\\information}}
\seeoberon\seeassembly\seearm\seeobject\seedebugging
}

\providecommand{\obavr}{
\toolsection{obavr} is a compiler for the Oberon programming language targeting the AVR hardware architecture.
It generates machine code for AVR processors from modules written in Oberon and stores it in corresponding object files.
For debugging purposes, it also creates a debugging information file as well as an assembly file containing a listing of the generated machine code.
In addition, it stores the interface of each module in a symbol file which is required when other modules import the module.
Programs generated with this compiler require additional runtime support that is stored in the \file{ob\-avr\-run} library file.
\flowgraph{\resource{Oberon\\source code} \ar[r] & \toolbox{obavr} \ar[r] \ar@/l/[d] \ar[rd] & \resource{object file} \\ \variable{ECSIMPORT} \ar[ru] & \resource{symbol\\files} \ar@/r/[u] & \resource{debugging\\information}}
\seeoberon\seeassembly\seeavr\seeobject\seedebugging
}

\providecommand{\obavrtt}{
\toolsection{obavr32} is a compiler for the Oberon programming language targeting the AVR32 hardware architecture.
It generates machine code for AVR32 processors from modules written in Oberon and stores it in corresponding object files.
For debugging purposes, it also creates a debugging information file as well as an assembly file containing a listing of the generated machine code.
In addition, it stores the interface of each module in a symbol file which is required when other modules import the module.
Programs generated with this compiler require additional runtime support that is stored in the \file{ob\-avr32\-run} library file.
\flowgraph{\resource{Oberon\\source code} \ar[r] & \toolbox{obavr32} \ar[r] \ar@/l/[d] \ar[rd] & \resource{object file} \\ \variable{ECSIMPORT} \ar[ru] & \resource{symbol\\files} \ar@/r/[u] & \resource{debugging\\information}}
\seeoberon\seeassembly\seeavrtt\seeobject\seedebugging
}

\providecommand{\obmabk}{
\toolsection{obm68k} is a compiler for the Oberon programming language targeting the M68000 hardware architecture.
It generates machine code for M68000 processors from modules written in Oberon and stores it in corresponding object files.
For debugging purposes, it also creates a debugging information file as well as an assembly file containing a listing of the generated machine code.
In addition, it stores the interface of each module in a symbol file which is required when other modules import the module.
Programs generated with this compiler require additional runtime support that is stored in the \file{ob\-m68k\-run} library file.
\flowgraph{\resource{Oberon\\source code} \ar[r] & \toolbox{obm68k} \ar[r] \ar@/l/[d] \ar[rd] & \resource{object file} \\ \variable{ECSIMPORT} \ar[ru] & \resource{symbol\\files} \ar@/r/[u] & \resource{debugging\\information}}
\seeoberon\seeassembly\seemabk\seeobject\seedebugging
}

\providecommand{\obmibl}{
\toolsection{obmibl} is a compiler for the Oberon programming language targeting the MicroBlaze hardware architecture.
It generates machine code for MicroBlaze processors from modules written in Oberon and stores it in corresponding object files.
For debugging purposes, it also creates a debugging information file as well as an assembly file containing a listing of the generated machine code.
In addition, it stores the interface of each module in a symbol file which is required when other modules import the module.
Programs generated with this compiler require additional runtime support that is stored in the \file{ob\-mibl\-run} library file.
\flowgraph{\resource{Oberon\\source code} \ar[r] & \toolbox{obmibl} \ar[r] \ar@/l/[d] \ar[rd] & \resource{object file} \\ \variable{ECSIMPORT} \ar[ru] & \resource{symbol\\files} \ar@/r/[u] & \resource{debugging\\information}}
\seeoberon\seeassembly\seemibl\seeobject\seedebugging
}

\providecommand{\obmipsa}{
\toolsection{obmips32} is a compiler for the Oberon programming language targeting the MIPS32 hardware architecture.
It generates machine code for MIPS32 processors from modules written in Oberon and stores it in corresponding object files.
For debugging purposes, it also creates a debugging information file as well as an assembly file containing a listing of the generated machine code.
In addition, it stores the interface of each module in a symbol file which is required when other modules import the module.
Programs generated with this compiler require additional runtime support that is stored in the \file{ob\-mips32\-run} library file.
\flowgraph{\resource{Oberon\\source code} \ar[r] & \toolbox{obmips32} \ar[r] \ar@/l/[d] \ar[rd] & \resource{object file} \\ \variable{ECSIMPORT} \ar[ru] & \resource{symbol\\files} \ar@/r/[u] & \resource{debugging\\information}}
\seeoberon\seeassembly\seemips\seeobject\seedebugging
}

\providecommand{\obmipsb}{
\toolsection{obmips64} is a compiler for the Oberon programming language targeting the MIPS64 hardware architecture.
It generates machine code for MIPS64 processors from modules written in Oberon and stores it in corresponding object files.
For debugging purposes, it also creates a debugging information file as well as an assembly file containing a listing of the generated machine code.
In addition, it stores the interface of each module in a symbol file which is required when other modules import the module.
Programs generated with this compiler require additional runtime support that is stored in the \file{ob\-mips64\-run} library file.
\flowgraph{\resource{Oberon\\source code} \ar[r] & \toolbox{obmips64} \ar[r] \ar@/l/[d] \ar[rd] & \resource{object file} \\ \variable{ECSIMPORT} \ar[ru] & \resource{symbol\\files} \ar@/r/[u] & \resource{debugging\\information}}
\seeoberon\seeassembly\seemips\seeobject\seedebugging
}

\providecommand{\obmmix}{
\toolsection{obmmix} is a compiler for the Oberon programming language targeting the MMIX hardware architecture.
It generates machine code for MMIX processors from modules written in Oberon and stores it in corresponding object files.
For debugging purposes, it also creates a debugging information file as well as an assembly file containing a listing of the generated machine code.
In addition, it stores the interface of each module in a symbol file which is required when other modules import the module.
Programs generated with this compiler require additional runtime support that is stored in the \file{ob\-mmix\-run} library file.
\flowgraph{\resource{Oberon\\source code} \ar[r] & \toolbox{obmmix} \ar[r] \ar@/l/[d] \ar[rd] & \resource{object file} \\ \variable{ECSIMPORT} \ar[ru] & \resource{symbol\\files} \ar@/r/[u] & \resource{debugging\\information}}
\seeoberon\seeassembly\seemmix\seeobject\seedebugging
}

\providecommand{\oborok}{
\toolsection{obor1k} is a compiler for the Oberon programming language targeting the OpenRISC 1000 hardware architecture.
It generates machine code for OpenRISC 1000 processors from modules written in Oberon and stores it in corresponding object files.
For debugging purposes, it also creates a debugging information file as well as an assembly file containing a listing of the generated machine code.
In addition, it stores the interface of each module in a symbol file which is required when other modules import the module.
Programs generated with this compiler require additional runtime support that is stored in the \file{ob\-or1k\-run} library file.
\flowgraph{\resource{Oberon\\source code} \ar[r] & \toolbox{obor1k} \ar[r] \ar@/l/[d] \ar[rd] & \resource{object file} \\ \variable{ECSIMPORT} \ar[ru] & \resource{symbol\\files} \ar@/r/[u] & \resource{debugging\\information}}
\seeoberon\seeassembly\seeorok\seeobject\seedebugging
}

\providecommand{\obppca}{
\toolsection{obppc32} is a compiler for the Oberon programming language targeting the PowerPC hardware architecture.
It generates machine code for PowerPC processors from modules written in Oberon and stores it in corresponding object files.
The compiler generates machine code for the 32-bit operating mode defined by the PowerPC architecture.
For debugging purposes, it also creates a debugging information file as well as an assembly file containing a listing of the generated machine code.
In addition, it stores the interface of each module in a symbol file which is required when other modules import the module.
Programs generated with this compiler require additional runtime support that is stored in the \file{ob\-ppc32\-run} library file.
\flowgraph{\resource{Oberon\\source code} \ar[r] & \toolbox{obppc32} \ar[r] \ar@/l/[d] \ar[rd] & \resource{object file} \\ \variable{ECSIMPORT} \ar[ru] & \resource{symbol\\files} \ar@/r/[u] & \resource{debugging\\information}}
\seeoberon\seeassembly\seeppc\seeobject\seedebugging
}

\providecommand{\obppcb}{
\toolsection{obppc64} is a compiler for the Oberon programming language targeting the PowerPC hardware architecture.
It generates machine code for PowerPC processors from modules written in Oberon and stores it in corresponding object files.
The compiler generates machine code for the 64-bit operating mode defined by the PowerPC architecture.
For debugging purposes, it also creates a debugging information file as well as an assembly file containing a listing of the generated machine code.
In addition, it stores the interface of each module in a symbol file which is required when other modules import the module.
Programs generated with this compiler require additional runtime support that is stored in the \file{ob\-ppc64\-run} library file.
\flowgraph{\resource{Oberon\\source code} \ar[r] & \toolbox{obppc64} \ar[r] \ar@/l/[d] \ar[rd] & \resource{object file} \\ \variable{ECSIMPORT} \ar[ru] & \resource{symbol\\files} \ar@/r/[u] & \resource{debugging\\information}}
\seeoberon\seeassembly\seeppc\seeobject\seedebugging
}

\providecommand{\obrisc}{
\toolsection{obrisc} is a compiler for the Oberon programming language targeting the RISC hardware architecture.
It generates machine code for RISC processors from modules written in Oberon and stores it in corresponding object files.
For debugging purposes, it also creates a debugging information file as well as an assembly file containing a listing of the generated machine code.
In addition, it stores the interface of each module in a symbol file which is required when other modules import the module.
Programs generated with this compiler require additional runtime support that is stored in the \file{ob\-risc\-run} library file.
\flowgraph{\resource{Oberon\\source code} \ar[r] & \toolbox{obrisc} \ar[r] \ar@/l/[d] \ar[rd] & \resource{object file} \\ \variable{ECSIMPORT} \ar[ru] & \resource{symbol\\files} \ar@/r/[u] & \resource{debugging\\information}}
\seeoberon\seeassembly\seerisc\seeobject\seedebugging
}

\providecommand{\obwasm}{
\toolsection{obwasm} is a compiler for the Oberon programming language targeting the WebAssembly architecture.
It generates machine code for WebAssembly targets from modules written in Oberon and stores it in corresponding object files.
For debugging purposes, it also creates a debugging information file as well as an assembly file containing a listing of the generated machine code.
In addition, it stores the interface of each module in a symbol file which is required when other modules import the module.
Programs generated with this compiler require additional runtime support that is stored in the \file{ob\-wasm\-run} library file.
\flowgraph{\resource{Oberon\\source code} \ar[r] & \toolbox{obwasm} \ar[r] \ar@/l/[d] \ar[rd] & \resource{object file} \\ \variable{ECSIMPORT} \ar[ru] & \resource{symbol\\files} \ar@/r/[u] & \resource{debugging\\information}}
\seeoberon\seeassembly\seewasm\seeobject\seedebugging
}

% converter tools

\providecommand{\dbgdwarf}{
\toolsection{dbgdwarf} is a DWARF debugging information converter tool.
It converts debugging information into the DWARF debugging data format and stores it in corresponding object files~\cite{dwarffile}.
The resulting debugging object files can be combined with runtime support that creates Executable and Linking Format (ELF) files~\cite{elffile}.
\flowgraph{\resource{debugging\\information} \ar[r] & \toolbox{dbgdwarf} \ar[r] & \resource{debugging\\object file}}
\seeobject\seedebugging
}

% assembler tools

\providecommand{\asmprint}{
\toolsection{asmprint} is a pretty printer for generic assembly code.
It reformats generic assembly code and writes it to the standard output stream.
\flowgraph{\resource{generic assembly\\source code} \ar[r] & \toolbox{asmprint} \ar[r] & \resource{reformatted\\source code}}
\seeassembly
}

\providecommand{\amdaasm}{
\toolsection{amd16asm} is an assembler for the AMD64 hardware architecture.
It translates assembly code into machine code for AMD64 processors and stores it in corresponding object files.
By default, the assembler generates machine code for the 16-bit operating mode defined by the AMD64 architecture.
\flowgraph{\resource{AMD16 assembly\\source code} \ar[r] & \toolbox{amd16asm} \ar[r] & \resource{object file}}
\seeassembly\seeamd\seeobject
}

\providecommand{\amdadism}{
\toolsection{amd16dism} is a disassembler for the AMD64 hardware architecture.
It translates machine code from object files targeting AMD64 processors into assembly code and writes it to the standard output stream.
It assumes that the machine code was generated for the 16-bit operating mode defined by the AMD64 architecture.
\flowgraph{\resource{object file} \ar[r] & \toolbox{amd16dism} \ar[r] & \resource{disassembly\\listing}}
\seeassembly\seeamd\seeobject
}

\providecommand{\amdbasm}{
\toolsection{amd32asm} is an assembler for the AMD64 hardware architecture.
It translates assembly code into machine code for AMD64 processors and stores it in corresponding object files.
By default, the assembler generates machine code for the 32-bit operating mode defined by the AMD64 architecture.
\flowgraph{\resource{AMD32 assembly\\source code} \ar[r] & \toolbox{amd32asm} \ar[r] & \resource{object file}}
\seeassembly\seeamd\seeobject
}

\providecommand{\amdbdism}{
\toolsection{amd32dism} is a disassembler for the AMD64 hardware architecture.
It translates machine code from object files targeting AMD64 processors into assembly code and writes it to the standard output stream.
It assumes that the machine code was generated for the 32-bit operating mode defined by the AMD64 architecture.
\flowgraph{\resource{object file} \ar[r] & \toolbox{amd32dism} \ar[r] & \resource{disassembly\\listing}}
\seeassembly\seeamd\seeobject
}

\providecommand{\amdcasm}{
\toolsection{amd64asm} is an assembler for the AMD64 hardware architecture.
It translates assembly code into machine code for AMD64 processors and stores it in corresponding object files.
By default, the assembler generates machine code for the 64-bit operating mode defined by the AMD64 architecture.
\flowgraph{\resource{AMD64 assembly\\source code} \ar[r] & \toolbox{amd64asm} \ar[r] & \resource{object file}}
\seeassembly\seeamd\seeobject
}

\providecommand{\amdcdism}{
\toolsection{amd64dism} is a disassembler for the AMD64 hardware architecture.
It translates machine code from object files targeting AMD64 processors into assembly code and writes it to the standard output stream.
It assumes that the machine code was generated for the 64-bit operating mode defined by the AMD64 architecture.
\flowgraph{\resource{object file} \ar[r] & \toolbox{amd64dism} \ar[r] & \resource{disassembly\\listing}}
\seeassembly\seeamd\seeobject
}

\providecommand{\armaasm}{
\toolsection{arma32asm} is an assembler for the ARM hardware architecture.
It translates assembly code into machine code for ARM processors executing A32 instructions and stores it in corresponding object files.
\flowgraph{\resource{ARM A32 assembly\\source code} \ar[r] & \toolbox{arma32asm} \ar[r] & \resource{object file}}
\seeassembly\seearm\seeobject
}

\providecommand{\armadism}{
\toolsection{arma32dism} is a disassembler for the ARM hardware architecture.
It translates machine code from object files targeting ARM processors executing A32 instructions into assembly code and writes it to the standard output stream.
\flowgraph{\resource{object file} \ar[r] & \toolbox{arma32dism} \ar[r] & \resource{disassembly\\listing}}
\seeassembly\seearm\seeobject
}

\providecommand{\armbasm}{
\toolsection{arma64asm} is an assembler for the ARM hardware architecture.
It translates assembly code into machine code for ARM processors executing A64 instructions and stores it in corresponding object files.
\flowgraph{\resource{ARM A64 assembly\\source code} \ar[r] & \toolbox{arma64asm} \ar[r] & \resource{object file}}
\seeassembly\seearm\seeobject
}

\providecommand{\armbdism}{
\toolsection{arma64dism} is a disassembler for the ARM hardware architecture.
It translates machine code from object files targeting ARM processors executing A64 instructions into assembly code and writes it to the standard output stream.
\flowgraph{\resource{object file} \ar[r] & \toolbox{arma64dism} \ar[r] & \resource{disassembly\\listing}}
\seeassembly\seearm\seeobject
}

\providecommand{\armcasm}{
\toolsection{armt32asm} is an assembler for the ARM hardware architecture.
It translates assembly code into machine code for ARM processors executing T32 instructions and stores it in corresponding object files.
\flowgraph{\resource{ARM T32 assembly\\source code} \ar[r] & \toolbox{armt32asm} \ar[r] & \resource{object file}}
\seeassembly\seearm\seeobject
}

\providecommand{\armcdism}{
\toolsection{armt32dism} is a disassembler for the ARM hardware architecture.
It translates machine code from object files targeting ARM processors executing T32 instructions into assembly code and writes it to the standard output stream.
\flowgraph{\resource{object file} \ar[r] & \toolbox{armt32dism} \ar[r] & \resource{disassembly\\listing}}
\seeassembly\seearm\seeobject
}

\providecommand{\avrasm}{
\toolsection{avrasm} is an assembler for the AVR hardware architecture.
It translates assembly code into machine code for AVR processors and stores it in corresponding object files.
The identifiers \texttt{RXL}, \texttt{RXH}, \texttt{RYL}, \texttt{RYH}, \texttt{RZL}, and \texttt{RZH} are predefined and name the corresponding registers.
The identifiers \texttt{SPL} and \texttt{SPH} are also predefined and evaluate to the address of the corresponding registers.
\flowgraph{\resource{AVR assembly\\source code} \ar[r] & \toolbox{avrasm} \ar[r] & \resource{object file}}
\seeassembly\seeavr\seeobject
}

\providecommand{\avrdism}{
\toolsection{avrdism} is a disassembler for the AVR hardware architecture.
It translates machine code from object files targeting AVR processors into assembly code and writes it to the standard output stream.
\flowgraph{\resource{object file} \ar[r] & \toolbox{avrdism} \ar[r] & \resource{disassembly\\listing}}
\seeassembly\seeavr\seeobject
}

\providecommand{\avrttasm}{
\toolsection{avr32asm} is an assembler for the AVR32 hardware architecture.
It translates assembly code into machine code for AVR32 processors and stores it in corresponding object files.
\flowgraph{\resource{AVR32 assembly\\source code} \ar[r] & \toolbox{avr32asm} \ar[r] & \resource{object file}}
\seeassembly\seeavrtt\seeobject
}

\providecommand{\avrttdism}{
\toolsection{avr32dism} is a disassembler for the AVR32 hardware architecture.
It translates machine code from object files targeting AVR32 processors into assembly code and writes it to the standard output stream.
\flowgraph{\resource{object file} \ar[r] & \toolbox{avr32dism} \ar[r] & \resource{disassembly\\listing}}
\seeassembly\seeavrtt\seeobject
}

\providecommand{\mabkasm}{
\toolsection{m68kasm} is an assembler for the M68000 hardware architecture.
It translates assembly code into machine code for M68000 processors and stores it in corresponding object files.
\flowgraph{\resource{68000 assembly\\source code} \ar[r] & \toolbox{m68kasm} \ar[r] & \resource{object file}}
\seeassembly\seemabk\seeobject
}

\providecommand{\mabkdism}{
\toolsection{m68kdism} is a disassembler for the M68000 hardware architecture.
It translates machine code from object files targeting M68000 processors into assembly code and writes it to the standard output stream.
\flowgraph{\resource{object file} \ar[r] & \toolbox{m68kdism} \ar[r] & \resource{disassembly\\listing}}
\seeassembly\seemabk\seeobject
}

\providecommand{\miblasm}{
\toolsection{miblasm} is an assembler for the MicroBlaze hardware architecture.
It translates assembly code into machine code for MicroBlaze processors and stores it in corresponding object files.
\flowgraph{\resource{MicroBlaze assembly\\source code} \ar[r] & \toolbox{miblasm} \ar[r] & \resource{object file}}
\seeassembly\seemibl\seeobject
}

\providecommand{\mibldism}{
\toolsection{mibldism} is a disassembler for the MicroBlaze hardware architecture.
It translates machine code from object files targeting MicroBlaze processors into assembly code and writes it to the standard output stream.
\flowgraph{\resource{object file} \ar[r] & \toolbox{mibldism} \ar[r] & \resource{disassembly\\listing}}
\seeassembly\seemibl\seeobject
}

\providecommand{\mipsaasm}{
\toolsection{mips32asm} is an assembler for the MIPS32 hardware architecture.
It translates assembly code into machine code for MIPS32 processors and stores it in corresponding object files.
\flowgraph{\resource{MIPS32 assembly\\source code} \ar[r] & \toolbox{mips32asm} \ar[r] & \resource{object file}}
\seeassembly\seemips\seeobject
}

\providecommand{\mipsadism}{
\toolsection{mips32dism} is a disassembler for the MIPS32 hardware architecture.
It translates machine code from object files targeting MIPS32 processors into assembly code and writes it to the standard output stream.
\flowgraph{\resource{object file} \ar[r] & \toolbox{mips32dism} \ar[r] & \resource{disassembly\\listing}}
\seeassembly\seemips\seeobject
}

\providecommand{\mipsbasm}{
\toolsection{mips64asm} is an assembler for the MIPS64 hardware architecture.
It translates assembly code into machine code for MIPS64 processors and stores it in corresponding object files.
\flowgraph{\resource{MIPS64 assembly\\source code} \ar[r] & \toolbox{mips64asm} \ar[r] & \resource{object file}}
\seeassembly\seemips\seeobject
}

\providecommand{\mipsbdism}{
\toolsection{mips64dism} is a disassembler for the MIPS64 hardware architecture.
It translates machine code from object files targeting MIPS64 processors into assembly code and writes it to the standard output stream.
\flowgraph{\resource{object file} \ar[r] & \toolbox{mips64dism} \ar[r] & \resource{disassembly\\listing}}
\seeassembly\seemips\seeobject
}

\providecommand{\mmixasm}{
\toolsection{mmixasm} is an assembler for the MMIX hardware architecture.
It translates assembly code into machine code for MMIX processors and stores it in corresponding object files.
The names of all special registers are predefined and evaluate to the corresponding number.
\flowgraph{\resource{MMIX assembly\\source code} \ar[r] & \toolbox{mmixasm} \ar[r] & \resource{object file}}
\seeassembly\seemmix\seeobject
}

\providecommand{\mmixdism}{
\toolsection{mmixdism} is a disassembler for the MMIX hardware architecture.
It translates machine code from object files targeting MMIX processors into assembly code and writes it to the standard output stream.
\flowgraph{\resource{object file} \ar[r] & \toolbox{mmixdism} \ar[r] & \resource{disassembly\\listing}}
\seeassembly\seemmix\seeobject
}

\providecommand{\orokasm}{
\toolsection{or1kasm} is an assembler for the OpenRISC 1000 hardware architecture.
It translates assembly code into machine code for OpenRISC 1000 processors and stores it in corresponding object files.
\flowgraph{\resource{OpenRISC 1000 assembly\\source code} \ar[r] & \toolbox{or1kasm} \ar[r] & \resource{object file}}
\seeassembly\seeorok\seeobject
}

\providecommand{\orokdism}{
\toolsection{or1kdism} is a disassembler for the OpenRISC 1000 hardware architecture.
It translates machine code from object files targeting OpenRISC 1000 processors into assembly code and writes it to the standard output stream.
\flowgraph{\resource{object file} \ar[r] & \toolbox{or1kdism} \ar[r] & \resource{disassembly\\listing}}
\seeassembly\seeorok\seeobject
}

\providecommand{\ppcaasm}{
\toolsection{ppc32asm} is an assembler for the PowerPC hardware architecture.
It translates assembly code into machine code for PowerPC processors and stores it in corresponding object files.
By default, the assembler generates machine code for the 32-bit operating mode defined by the PowerPC architecture.
\flowgraph{\resource{PowerPC assembly\\source code} \ar[r] & \toolbox{ppc32asm} \ar[r] & \resource{object file}}
\seeassembly\seeppc\seeobject
}

\providecommand{\ppcadism}{
\toolsection{ppc32dism} is a disassembler for the PowerPC hardware architecture.
It translates machine code from object files targeting PowerPC processors into assembly code and writes it to the standard output stream.
It assumes that the machine code was generated for the 32-bit operating mode defined by the PowerPC architecture.
\flowgraph{\resource{object file} \ar[r] & \toolbox{ppc32dism} \ar[r] & \resource{disassembly\\listing}}
\seeassembly\seeppc\seeobject
}

\providecommand{\ppcbasm}{
\toolsection{ppc64asm} is an assembler for the PowerPC hardware architecture.
It translates assembly code into machine code for PowerPC processors and stores it in corresponding object files.
By default, the assembler generates machine code for the 64-bit operating mode defined by the PowerPC architecture.
\flowgraph{\resource{PowerPC assembly\\source code} \ar[r] & \toolbox{ppc64asm} \ar[r] & \resource{object file}}
\seeassembly\seeppc\seeobject
}

\providecommand{\ppcbdism}{
\toolsection{ppc64dism} is a disassembler for the PowerPC hardware architecture.
It translates machine code from object files targeting PowerPC processors into assembly code and writes it to the standard output stream.
It assumes that the machine code was generated for the 64-bit operating mode defined by the PowerPC architecture.
\flowgraph{\resource{object file} \ar[r] & \toolbox{ppc64dism} \ar[r] & \resource{disassembly\\listing}}
\seeassembly\seeppc\seeobject
}

\providecommand{\riscasm}{
\toolsection{riscasm} is an assembler for the RISC hardware architecture.
It translates assembly code into machine code for RISC processors and stores it in corresponding object files.
The names of all special registers are predefined and evaluate to the corresponding number.
\flowgraph{\resource{RISC assembly\\source code} \ar[r] & \toolbox{riscasm} \ar[r] & \resource{object file}}
\seeassembly\seerisc\seeobject
}

\providecommand{\riscdism}{
\toolsection{riscdism} is a disassembler for the RISC hardware architecture.
It translates machine code from object files targeting RISC processors into assembly code and writes it to the standard output stream.
\flowgraph{\resource{object file} \ar[r] & \toolbox{riscdism} \ar[r] & \resource{disassembly\\listing}}
\seeassembly\seerisc\seeobject
}

\providecommand{\wasmasm}{
\toolsection{wasmasm} is an assembler for the WebAssembly architecture.
It translates assembly code into machine code for WebAssembly targets and stores it in corresponding object files.
The names of all special registers are predefined and evaluate to the corresponding number.
\flowgraph{\resource{WebAssembly assembly\\source code} \ar[r] & \toolbox{wasmasm} \ar[r] & \resource{object file}}
\seeassembly\seewasm\seeobject
}

\providecommand{\wasmdism}{
\toolsection{wasmdism} is a disassembler for the WebAssembly architecture.
It translates machine code from object files targeting WebAssembly targets into assembly code and writes it to the standard output stream.
\flowgraph{\resource{object file} \ar[r] & \toolbox{wasmdism} \ar[r] & \resource{disassembly\\listing}}
\seeassembly\seewasm\seeobject
}

% linker tools

\providecommand{\linklib}{
\toolsection{linklib} is an object file combiner.
It creates a static library file by combining all object files given to it into a single one.
\flowgraph{\resource{object files} \ar[r] & \toolbox{linklib} \ar[r] & \resource{library file}}
\seeobject
}

\providecommand{\linkbin}{
\toolsection{linkbin} is a linker for plain binary files.
It links all object files given to it into a single image and stores it in a binary file that begins with the first linked section.
It also creates a map file that lists the address, type, name and size of all used sections.
The filename extension of the resulting binary file can be specified by putting it into a constant data section called \texttt{\_extension}.
\flowgraph{\resource{object files} \ar[r] & \toolbox{linkbin} \ar[r] \ar[d] & \resource{binary file} \\ & \resource{map file}}
\seeobject
}

\providecommand{\linkmem}{
\toolsection{linkmem} is a linker for plain binary files partitioned into random-access and read-only memory.
It links all object files given to it into two distinct images, one for data sections and one for code and constant data sections, and stores each image in a binary file that begins with the first linked section of the corresponding type.
It also creates a map file that lists the address, type, name and size of all used sections.
\flowgraph{\resource{object files} \ar[r] & \toolbox{linkmem} \ar[r] \ar[d] & \resource{RAM file/\\ROM file} \\ & \resource{map file}}
\seeobject
}

\providecommand{\linkprg}{
\toolsection{linkprg} is a linker for GEMDOS executable files.
It links all object files given to it into a single image and stores the image in an Atari GEMDOS executable file~\cite{gemdosfile}.
It also creates a map file that lists the address relative to the text segment, type, name and size of all used sections.
The filename extension of the resulting executable file can be specified by putting it into a constant data section called \texttt{\_extension}.
The GEMDOS executable file format requires all patch patterns of absolute link patches to consist of four full bitmasks with descending offsets.
\flowgraph{\resource{object files} \ar[r] & \toolbox{linkprg} \ar[r] \ar[d] & \resource{executable file} \\ & \resource{map file}}
\seeobject
}

\providecommand{\linkhex}{
\toolsection{linkhex} is a linker for Intel HEX files.
It links all code sections of the object files given to it into single image and stores the image in an Intel HEX file~\cite{hexfile} that begins with the first linked section.
It also creates a map file that lists the address, type, name and size of all used sections.
\flowgraph{\resource{object files} \ar[r] & \toolbox{linkhex} \ar[r] \ar[d] & \resource{HEX file} \\ & \resource{map file}}
\seeobject
}

\providecommand{\mapsearch}{
\toolsection{mapsearch} is a debugging tool.
It searches map files generated by linker tools for the name of a binary section that encompasses a memory address read from the standard input stream.
If additionally provided with one or more object files, it also stores an excerpt thereof in a separate object file called map search result which only contains the identified binary section for disassembling purposes.
\flowgraph{& \resource{map files/\\object files} \ar[d] \\ \resource{memory\\address} \ar[r] & \toolbox{mapsearch} \ar[r] \ar[d] & \resource{section name/\\relative offset} \\ & \resource{object file\\excerpt}}
\seeobject
}

\renewcommand{\seecode}{}

\startchapter{Intermediate Code}{Intermediate Code Representation}{code}
{This \documentation{} describes the intermediate code representation used by the \ecs{} as the
common interface between the various programming languages and hardware architectures it supports.}

\epigraph{Unsere Eigenschaften m\"ussen wir kultivieren, \\ nicht unsere Eigenheiten.}{Johann Wolfgang von Goethe}

\section{Introduction}

Rather than compiling source code directly into machine code, the \ecs{} makes use of a so-called \emph{intermediate code representation}.
For each programming language supported by the \ecs{}, there is a so-called \emph{front-end}\index{Front-ends} which translates source code into intermediate code.
This abstract representation of the original program is then passed to a so-called \emph{back-end}\index{Back-ends} which in turn transforms it into machine code and debugging information.
Figure~\ref{fig:cddataflow} visualizes the role and data flow of the intermediate code representation in-between front-ends and back-ends.

\begin{figure}
\flowgraph{
\resource{source code} \ar[d] & \resource{source code} \ar[d] & \resource{source code} \ar[d] \\
\converter{Front-End\\for programming\\language \textit{A}} \ar[rd] & \converter{Front-End\\for programming\\language \textit{B}} \ar[d] & \converter{Front-End\\for programming\\language \textit{C}} \ar[ld] \\
& \resource{intermediate\\code} \ar[ld] \ar[d] \ar[rd] \\
\converter{Back-End\\for hardware\\architecture \textit{X}} \ar[d] & \converter{Back-End\\for hardware\\architecture \textit{Y}} \ar[d] & \converter{Intermediate\\Code\\Interpreter} \ar[d] \\
\resource{machine code/\\debugging information} & \resource{machine code/\\debugging information} & \resource{input/\\output} \\
}\caption[Intermediate code representation of programs]{Intermediate code representation of programs in-between front-ends and back-ends}
\label{fig:cddataflow}
\end{figure}

The advantage of this design is a clear separation between the various front-ends for the programming languages and the back-ends for the hardware architectures.
Since the intermediate code representation is the only interface between these parts of a compiler, they are completely independent from each other and may therefore be freely combined.
Implementing a new programming language for all supported hardware architectures and vice versa can thus be achieved by just adding one other front-end or back-end to the \ecs{}.
Furthermore, examining the output of front-ends and providing intermediate code as input to back-ends allows their respective implementations to be validated in isolation.
Using an interpreter for intermediate code that emulates its underlying abstract machine, front-end implementations can additionally be tested by executing programs directly without having to target and rely on actual hardware architectures or runtime environments.

The following sections describe the semantics of the intermediate code representation alongside the syntax of its textual representation using the generic assembly language.
\seeassembly\seedebugging

\section{Programming Model}\index{Programming model, of Intermediate Code}\label{sec:cdprogrammingmodel}

The underlying programming model of the intermediate code representation is based on the execution of instructions grouped into \emph{sections}\index{Sections}.
Each section has a unique identifier that allows it to be referenced by instructions from other sections.
Additionally, each section has a specific \emph{type}\index{Section types}\index{Types, of sections}, that allows distinct classes of instructions to be used.
The following sections describe these section types and the overall execution environment defined by the programming model.

\subsection{Code Sections}\index{Code sections}

A code section stores the actual instructions that have to be executed at runtime.
Each functional unit of a programming language is typically mapped to a single code section.
Instructions in code sections can reference other sections by either calling them or accessing the data stored in them.
The following kinds of code sections define when and how code is executed:

\begin{itemize}

\item Standard Code Sections\index{Standard code sections}\alignright\syntax{".code" <Identifier>}\nopagebreak

Standard code sections model standard functions and are typically called by instructions of other code sections.
Functions often create stack frames and have to return the control to their caller at the end of their execution.

\item Initializing Code Sections\index{Initializing code sections}\alignright\syntax{".initcode" <Identifier>}\nopagebreak

Initializing code sections are executed automatically at the start of the program and allow its state to be initialized.
Since these code sections are just executed in sequence as opposed to being explicitly called, there is no need to return from initializing code sections because there is no caller.

\item Data Initializing Code Sections\index{Data initializing code sections}\alignright\syntax{".initdata" <Identifier>}\nopagebreak

Data initializing code sections are executed automatically at the very beginning of a program and allow initializing data sections that are required in all other code sections.
This does explicitly include constant data sections because there are architectures for which all data sections have to be initialized by code sections at runtime.

\item Assembly Code Sections\index{Assembly code sections}\alignright\syntax{".assembly"}\nopagebreak

Assembly code sections are nameless pseudo sections for representing arbitrary assembly code that defines its own code and data sections.
The actual assembly code consists of one or more inline assembly instructions as described in Section~\ref{sec:cdasm}.
Each instruction invokes the underlying assembler once and has unrestricted access to all of its features.

\end{itemize}

\subsection{Data Sections}\index{Data sections}

A data section stores instructions that describe the contents and the layout of some memory region.
This memory can be accessed by the instructions of a code section.
There are two kinds of data sections which specify the access rights to the contents of the memory region:

\begin{itemize}

\item Standard Data Sections\index{Standard data sections}\alignright\syntax{".data" <Identifier>}\nopagebreak

Standard data sections model standard memory regions with read and write access.
They are typically used to represent global variables.

\item Constant Data Sections\index{Constant data sections}\alignright\syntax{".const" <Identifier>}\nopagebreak

Constant data sections model memory regions with read-only access.
Since these data sections are not supposed to change their contents, they are typically used to represent constants and strings.

\end{itemize}

\subsection{Type Sections}\index{Type sections}

A type section stores instructions that describe the type system of the programming language for debugging purposes.
The following kind of type section declares the layout and representation of user-defined and predefined types:

\begin{itemize}

\item Standard Type Sections\index{Standard type sections}\alignright\syntax{".type" <Identifier>}\nopagebreak

Standard type sections contain a single type declaration and define a new name for that type.
They are typically used to represent type definitions of the programming language.

\end{itemize}

Since type sections contain only metadata about the types used by a program, they have no machine code representation.
They do however contribute to the debugging information generated by the back-end, see Section~\ref{sec:cddebugging}.

\subsection{Order of Execution}\index{Order of execution}\index{Execution order}

Code sections are executed automatically by the runtime system according to their type as described above.
The concrete order of execution is as follows:

\begin{enumerate}

\item
All data initializing code sections are executed in order of occurrence.
They may only depend on the results of already executed data initializing code sections.

\item
All remaining initializing code sections are executed in order of occurrence.
They may only depend on the results of already executed initializing code sections.

\item
A standard code section called \texttt{main} is executed.
This section represents the actual entry point of a program and is therefore always implicitly required.
It may depend on the results of both previous execution steps but has to eventually return the control of execution to the host environment.

\end{enumerate}

All remaining code sections are called depending on the actual code executed in these three execution steps.
They may only depend on results that are also guaranteed for the code sections they were called from.

\subsection{Stack Operations}\index{Stack operations}\label{sec:cdstackoperations}

The programming model of the intermediate code defines local memory for every thread of execution called the \emph{stack} as exemplified in Figure~\ref{fig:cdstackframe}.
The stack is represented by two special registers called the \emph{stack pointer} and the \emph{frame pointer}, see Section~\ref{sec:cdregisters}.
Although they allow direct access to the stack memory, there are some intermediate code instructions that modify the contents of these registers as well as the stack memory itself.
These instructions are \texttt{push}, \texttt{pop}, \texttt{call}, and \texttt{ret}.
See the instruction reference in Section~\ref{sec:cdreference} for detailed information about the operation of these instructions.

\begin{figure}
\centering
\sffamily\begin{tabular}{r|c|l}
& $\downarrow$ & \multirow{2}{*}{highest address} \\
\cline{2-2} & last argument \\ & $\vdots$ \\ & first argument \\
\cline{2-2} & return address \\
\cline{2-2} \multirow{2}{*}{frame pointer $\rightarrow$} & previous frame \\
\cline{2-2} & first variable \\ & $\vdots$ \\ & last variable \\
\cline{2-2} \multirow{2}{*}{stack pointer $\rightarrow$} & local data & \multirow{2}{*}{lowest address} \\
\cline{2-2} & $\downarrow$ \\
\end{tabular}\normalfont
\caption{Typical intermediate code stack layout}
\label{fig:cdstackframe}
\end{figure}

The stack pointer always points to the topmost element on the stack.
Operations like pushing and popping data to or from the stack access the topmost element and implicitly modify the stack pointer accordingly.
The stack memory grows from top to bottom such that the topmost element on the stack has the lowermost address in memory.
By pushing data on the stack, the stack pointer is therefore first decremented and afterward used to store the data at the decremented memory address.
The actual size of this decrement depends on the target hardware architecture.
There are architectures that require the stack pointer address to conform to some alignment constraints.
The underlying machine code generator of a back-end provides platform-specific information that includes alignment requirements like this.

The stack memory is generally used to store arbitrary local data or to pass arguments to another code section involved in a function call\index{Calling convention}.
The virtual general-purpose registers of the intermediate code are in general mapped arbitrarily to the physical registers of the underlying hardware architecture.
For this reason, they cannot be used in general to pass arguments to other code sections.
Likewise, their contents may be invalidated in-between function calls.
Arguments and values of registers that are still in use afterward the function call, have therefore to be stored on the stack.

Functions may maintain their own local memory by creating so-called \emph{stack frames}\index{Stack frames}.
Stack frames allow reserving a certain amount of memory on the stack that is especially useful for local variables or recursive functions.
A stack frame is represented by the frame pointer and created by the instruction \texttt{enter}, see Section~\ref{sec:cdenter}.
By creating a new stack frame, the value of the frame pointer is saved on the stack and afterward set to the current value of the stack pointer.
The stack pointer is then decremented by the size of the stack frame.
Since the stack pointer may be modified again afterward, it is best to access the local data using the frame pointer.
A stack frame is deleted using the instruction \texttt{leave}, see Section~\ref{sec:cdleave}.
It just restores the previous values of the frame and stack pointer.

\section{Intermediate Code Representation}\index{Intermediate code representation}

The intermediate representation of all code and global data is based on instructions within code or data sections.
Instructions themselves have operands of different models which may be constants, registers, or memory operands.
This section describes how these components of the intermediate code are represented within the \ecs{}.

\subsection{Types}\label{sec:cdtypes}

The intermediate code representation defines a set of several basic \emph{types}.
These types specify the way how data that is stored in memory or registers shall be interpreted by an instruction.
They consist of one of the following \emph{type models}\index{Type models} and an additional \emph{type size}\index{Type sizes} expressed in octets:

\begin{itemize}

\item Signed Integers\alignright\syntax{"s1" $\mid$ "s2" $\mid$ "s4" $\mid$ "s8"}\nopagebreak

The type model for signed integers allows accessing data of signed integral data type.
There are four different type sizes available: 1, 2, 4, and 8.
For the type size $n$ all valid integers lie within the range $-2^{8n-1}$ to $+2^{8n-1}-1$.
The result of arithmetic operations causing integer overflows is undefined.

\item Unsigned Integers\alignright\syntax{"u1" $\mid$ "u2" $\mid$ "u4" $\mid$ "u8"}\nopagebreak

The type model for unsigned integers allows accessing data of unsigned integral data type.
There are four different type sizes available: 1, 2, 4, and 8.
For the type size $n$ all valid integers lie within the range $0$ to $2^{8n}-1$.
Arithmetic operations on unsigned integers obey the laws of arithmetic modulo $2^{8n}$ and do therefore not overflow.

\item Floating-Point Numbers\alignright\syntax{"f4" $\mid$ "f8"}\nopagebreak

The type model for floating-point numbers allows accessing data stored according to formats defined in the IEEE standard for floating-point arithmetic~\cite{ieee1985}.
There are two different type sizes available: 4 and 8.
Floating-point numbers of size four are stored using the single precision format,
and floating-point numbers of size eight are stored using the double precision format.

\item Data Pointers\alignright\syntax{"ptr"}\nopagebreak

The type model for data pointers allows accessing data that represents the address of a datum within a data section.
There is only one type size available and it is predefined by the target platform.

\item Function Pointers\alignright\syntax{"fun"}\nopagebreak

The type model for function pointers allows accessing data that represents the address of an instruction within a code section.
There is only one type size available and it is predefined by the target platform.

\end{itemize}

For some hardware architectures, the representation of data with the same value but different types may also be equal physically.
But in order to ensure a consistent virtual representation, all data shall be read using the same type it was written beforehand.
If the representation of the data type has to be explicitly changed, the \ecs{} provides a special conversion instruction.
This operation shall also be used even if the types in question differ only with respect to their size.

\subsection{Registers}\label{sec:cdregisters}

The abstract intermediate code representation defines the following set of virtual general-purpose and special-purpose registers:

\begin{itemize}

\item Eight General-Purpose Registers\index{General-purpose register}\alignright\syntax{"$0" $\mid$ "$1" $\mid$ \ldots $\mid$ "$7"}\nopagebreak

These general-purpose registers can be freely used to store and restore intermediate results evaluated in expressions.
Their actual values depend on the type used to access the registers and are invalidated across function calls.
Before calling a function, it is therefore important to save the value of a register onto the stack in order to safely restore it afterward.

\item The Result Register\index{Result register}\alignright\syntax{"$res"}\nopagebreak

The result register can be used to store and restore the result of a function call.
Its value depends on the type used to access the register and is not invalidated across function calls.

\item The Stack Pointer\index{Stack pointer register}\alignright\syntax{"$sp"}\nopagebreak

The stack pointer stores an address that points to the current top of the stack.
Several instructions implicitly change the stack pointer by pushing or popping values to or from the stack, see Section~\ref{sec:cdstackoperations}.
The address stored in the stack pointer has to be aligned according to the stack alignment of the target platform.

\item The Frame Pointer\index{Frame pointer register}\alignright\syntax{"$fp"}\nopagebreak

The frame pointer stores an address that points to the beginning of the current stack frame.
This register can therefore be used to access the local variables of a function or the arguments passed to it.

\item The Link Register\index{Link register}\alignright\syntax{"$lnk"}\nopagebreak

The link register stores the address of the instruction following a \texttt{call} instruction, see Section~\ref{sec:cdcall}.
This allows function calls on architectures that support this register to bypass the stack when storing the return address.

\end{itemize}

\subsection{Operands}\label{sec:cdoperands}

The intermediate code representation defines several different instructions that can take up to three operands.
Each operand is an instance of one of the following \emph{operand models}\index{Operand models}:

\newcommand{\cdoperand}[1]{\hypertarget{cd:#1}{#1}}
\newcommand{\cdoperandref}[1]{\hyperlink{cd:#1}{\textit{#1}}}

\begin{itemize}

\item\cdoperand{Size}\alignright\syntax{<Number>}\nopagebreak

The size operand model is used to express a non-negative integer number.
It generally denotes a number of octets but can also refer to some other quantity.

\item\cdoperand{Offset}\alignright\syntax{$\pm$<Number>}\nopagebreak

The offset operand model represents relative offsets with respect to the instruction directly following the current one.
The sequential program flow of a program is altered by branching to the specified instruction within the current code section.
Offsets are only used as first operand of an instruction.

\item\cdoperand{String}\alignright\syntax{<String>}\nopagebreak

The string operand model stores character sequences of arbitrary length.
It is used to carry metadata from the front-end to the back-end.
Strings can contain standard escape sequences to carry special characters.

\item\cdoperand{Type}\alignright\syntax{<Type>}\nopagebreak

The type operand model declares a single basic type as defined in Section~\ref{sec:cdtypes}.

\item\cdoperand{Immediate} Value\alignright\syntax{<Type> <Number> $\mid$ <Type> $\pm$<Number>}\nopagebreak

The immediate value operand model is used to store a constant numeric value.
The range and format of the valid values is predefined by the given type.

\item\cdoperand{Register}\alignright\syntax{<Type> <Register> $\mid$ <Type> <Register> $\pm$ <Number>}\nopagebreak

The register operand model can be used to access the values of one of the registers defined in Section~\ref{sec:cdregisters}.
The given type specifies the representation of the value of the register and has to be of type model data pointer for the special-purpose pointer registers.
If a register is accessed using type model data pointer and is not used as destination operand of an instruction,
its evaluated address can be incremented or decremented by an optional constant displacement.

\item\cdoperand{Address}\alignright\syntax{<Type> "@"<Identifier> $\mid$ <Type> "@"<Identifier> $\pm$ <Number>}\\
\alignright\syntax{$\mid$ <Type> "@"<Identifier> $+$ <Register> $\mid$ <Type> "@"<Identifier> $+$ <Register> $\pm$ <Number>}\nopagebreak

The address operand model represents the address of a code or data section named by the identifier.
If this identifier contains question marks, the actual name of the section begins behind the last question mark.
If none of the sections named in front of a question mark are used, the actual address evaluates either to zero or to the section named behind an optional colon.
If the address names a code section, its type has to be a function pointer.
Otherwise its type has to be a data pointer and the resulting address can be incremented by an optional constant displacement.
If the operand is used within an instruction of a code section, this address can additionally be incremented by the value of a register that stores a data pointer.

\item\cdoperand{Memory}\alignright\syntax{<Type> "[" <Number> "]"}\\
\alignright\syntax{$\mid$ <Type> "[" <Register> "]" $\mid$ <Type> "[" <Register> $\pm$ <Number> "]"}\\
\alignright\syntax{$\mid$ <Type> "[" "@"<Identifier> "]" $\mid$ <Type> "[" "@"<Identifier> $\pm$ <Number> "]"}\\
\alignright\syntax{$\mid$ <Type> "[" "@"<Identifier> $+$ <Register> "]" $\mid$ <Type> "[" "@"<Identifier> $+$ <Register> $\pm$ <Number> "]"}\nopagebreak

The memory operand model allows accessing the data value stored at a specified address using the representation of the given type.
The actual address can either be an immediate address, a register that stores a data pointer, the name of a data section, or any combination thereof.
The evaluation of the address of a section is the same as for sections named in the address operand model.

\end{itemize}

\subsection{Instructions}

Instructions are composed of one of the mnemonics listed in Table~\ref{tab:codeset} followed by up to three operands.
If the instruction generates a result, it is generally stored in the first register or memory operand.
Except where otherwise noted, the types of all typed operands have to match.
Operands and their different models are described in Section~\ref{sec:cdoperands}.

\instructionset{code}{Intermediate code instruction set}{5}{6}

The operation of these instructions as well as all valid combinations of their operands are specified in the instruction reference in Section~\ref{sec:cdreference}.
The instructions themselves are grouped together in code, data, or type sections as described in Section~\ref{sec:cdprogrammingmodel}.
The actual section type defines which instruction subset is available for use.
Table~\ref{tab:cdinstructions} categorizes all instructions pairwise according to their operation and lists the section types they are available in.
A machine code generator is only required to implement instructions for data management, arithmetics, bit manipulations, function calls, and branching.

\newcommand{\cdinstructionref}[3]{& #1 & \texttt{#2} & \ref{sec:cd#2} & \ifx#3\empty\else\texttt{#3} & \ref{sec:cd#3}\fi \\}

\begin{table}
\centering
\begin{tabular}{@{}lllrlr@{}}
\toprule Instruction & Section & \multicolumn{2}{l}{Mnemonic} & \multicolumn{2}{l@{}}{Mnemonic} \\
Category & Types & \multicolumn{2}{r}{See Section} & \multicolumn{2}{r@{}}{See Section} \\
\midrule Memory Layout
\cdinstructionref{all}{alias}{req}
\cdinstructionref{data}{def}{res}
\midrule Data Management
\cdinstructionref{code}{mov}{conv}
\cdinstructionref{code}{copy}{fill}
\midrule Arithmetic
\cdinstructionref{code}{add}{sub}
\cdinstructionref{code}{mul}{div}
\cdinstructionref{code}{mod}{neg}
\midrule Bit Manipulation
\cdinstructionref{code}{and}{or}
\cdinstructionref{code}{xor}{not}
\cdinstructionref{code}{lsh}{rsh}
\midrule Function Call
\cdinstructionref{code}{push}{pop}
\cdinstructionref{code}{call}{ret}
\cdinstructionref{code}{enter}{leave}
\midrule Branching
\cdinstructionref{code}{br}{jump}
\cdinstructionref{code}{breq}{brne}
\cdinstructionref{code}{brlt}{brge}
\midrule Special Purpose
\cdinstructionref{code}{nop}{asm}
\cdinstructionref{code}{fix}{unfix}
\midrule Debugging
\cdinstructionref{all}{loc}{}
\cdinstructionref{code}{break}{trap}
\midrule Symbol Declaration
\cdinstructionref{code}{sym}{}
\cdinstructionref{all}{field}{value}
\midrule Type Declaration
\cdinstructionref{all}{void}{type}
\cdinstructionref{all}{array}{rec}
\cdinstructionref{all}{ptr}{ref}
\cdinstructionref{all}{func}{enum}
\bottomrule
\end{tabular}
\caption{Intermediate code instruction categories}
\label{tab:cdinstructions}
\end{table}

Instructions and sections are textually represented using the generic assembly language.
Tools like \tool{cpp\-code} or \tool{ob\-code}, see Section~\ref{sec:cdtools}, translate programs into intermediate code and generate an assembly code listing thereof.
Other tools like \tool{cd\-check} or \tool{cd\-run} are able to read and process assembly files written using intermediate code.
These tools predefine some constant definitions whose actual values depend on the target hardware architecture.
The definitions \texttt{int}, \texttt{intsize}, and \texttt{intalign} evaluate to the name, size, and stack alignment of the default signed integer type,
while \texttt{flt}, \texttt{fltsize}, and \texttt{fltalign} evaluate to the name, size, and stack alignment of the default floating-point type.
The sizes and stack alignments of data pointers, function pointers, return addresses and the link register are available using the definitions \texttt{ptrsize}, \texttt{ptralign}, \texttt{funsize}, \texttt{funalign}, \texttt{retsize}, \texttt{retalign}, and \texttt{lnksize} respectively.
The definition \texttt{stackdisp} evaluates to the architecture-dependent stack pointer displacement that has to be added to every memory access on the stack.
The filename of the source code and current position therein are available using the definitions \texttt{file} and \texttt{line}.
\seeassembly
Since intermediate code is an abstract representation however, all assembly language features which define, align, or otherwise refer to binary data are not available.

\subsection{Debugging Information}\label{sec:cddebugging}

Several intermediate code instructions contain metadata about the program for debugging purposes.
They do not generate any machine code but contribute to the debugging information generated by the back-end.
This debugging information consists of the following metadata for debuggers capable of program animation and memory inspection:

\begin{itemize}

\item Breakpoints\index{Breakpoints}\nopagebreak

Breakpoints map machine code addresses to their corresponding source code locations and vice versa.
They enable user requests to suspend the program execution at predefined statement units of the programming language.

\item Symbol Declarations\index{Symbol declarations}\nopagebreak

Symbol declarations map memory addresses to their corresponding symbolic names and vice versa.
They are typically used to represent storage objects like variables and parameters of the programming language.

\item Type Declarations\index{Type declarations}\nopagebreak

Type declarations describe the layout and data representation of memory regions occupied by storage objects.
They are used to represent data types of the programming language and its user-defined data structures for memory inspection.

\end{itemize}

Type declarations are represented by an arbitrary set of consecutive type declaration instructions.
This allows describing complex compound types involving pointers and arrays in order of occurrence.
All type declarations in code sections or type sections refer to the type of the preceding symbol, field, or enumerator declaration.
Otherwise they declare the type of the result returned by the code section, or the type of the data or type section itself.
\seedebugging

\section{Instruction Set Reference}\label{sec:cdreference}

This section describes the operation of all instructions available in intermediate code.
Every instruction has one mnemonic and up to three operands.
The valid combinations of mnemonics and operand models is given in the syntax definition for every instruction.
The syntax definition uses either an operand model as described in Section~\ref{sec:cdoperands} or an operand class as shown in Table~\ref{tab:cdoperandclasses}.
An operand class consists of several different operand models.
The model of the actual operand has to be one of the models named in the operand class.

\begin{table}
\centering
\begin{tabular}{@{}ll@{}}
\toprule Name & Description \\ \midrule
\cdoperand{StrTyp} & \cdoperandref{String} or \cdoperandref{Type} \\
\cdoperand{ImmAdr} & \cdoperandref{Immediate} or \cdoperandref{Address} \\
\cdoperand{RegMem} & \cdoperandref{Register} or \cdoperandref{Memory} \\
\cdoperand{ImmRegMem} & \cdoperandref{Immediate}, \cdoperandref{Register}, or \cdoperandref{Memory} \\
\cdoperand{ImmRegAdrMem} & \cdoperandref{Immediate}, \cdoperandref{Register}, \cdoperandref{Address}, or \cdoperandref{Memory} \\
\cdoperand{Pointer} & \cdoperandref{ImmRegAdrMem} of type data pointer \\
\cdoperand{Function} & \cdoperandref{ImmRegAdrMem} of type function pointer \\
\bottomrule
\end{tabular}
\caption{Intermediate code operand classes}
\label{tab:cdoperandclasses}
\end{table}

\newcommand{\cdinstruction}[5]{\subsection[#1]{\texttt{#1}\enskip\textnormal{\ifx#2\empty\else\cdoperandref{#2}\ifx#3\empty\else\texttt{,} \cdoperandref{#3}\ifx#4\empty\else\texttt{,} \cdoperandref{#4}\fi\fi\fi}\enskip\alignright\mbox{#5}}\label{sec:cd#1}}

\cdinstruction{add}{RegMem}{ImmRegAdrMem}{ImmRegAdrMem}{Addition}

Adds the value of the third operand to the value of the second operand and stores the result in the first register or memory operand.
This operation is not available for function pointers.

\cdinstruction{alias}{String}{}{}{Alias Name Definition}

Defines an alias name for the code or data following the current instruction.
The address of the alias name equals to the name of the section plus the corresponding displacement.
The operand specifies the name of the alias which may not be duplicated and should differ from the name of the section.

\cdinstruction{and}{RegMem}{ImmRegMem}{ImmRegMem}{Logical AND}

Performs a logical AND operation on the bits of the values of the second and third operand and stores the result in the first register or memory operand.
This operation is not available for function pointers and floating-point numbers.

\cdinstruction{array}{Size}{Size}{}{Array Type Declaration}

Declares an array type which represents a collection of consecutive storage objects with the same element type that can be accessed by indexing.
The first operand specifies the index of the first element of the array and the second operand the number of elements contained therein.
This compound type declaration requires a subsequent declaration for its element type.

\cdinstruction{asm}{String}{Size}{String}{Inline Assembly}

Passes inline assembly code to the underlying assembler of the back-end.
The \texttt{asm} instruction itself is then replaced by the generated machine code.
The first operand names the source file and the second operand a line within that file.
This information is needed for diagnostic messages of the assembler.
The third operand contains the actual assembly code in one string.
\seeassembly

\cdinstruction{br}{Offset}{}{}{Unconditional Branch}

Continues execution at the instruction specified by the relative instruction offset.
Indirect branches or branches into other code sections can be performed using the \texttt{jump} instruction, see Section~\ref{sec:cdjump}.

\cdinstruction{break}{}{}{}{Breakpoint}

Identifies the beginning of a suspendable statement unit of the programming language.
Breakpoints require subsequent source code locations, see Section~\ref{sec:cdloc}.

\cdinstruction{breq}{Offset}{ImmRegAdrMem}{ImmRegAdrMem}{Branch if Equal}

Continues execution at the instruction specified by the relative instruction offset if the values of the second and third operand match.

\cdinstruction{brge}{Offset}{ImmRegAdrMem}{ImmRegAdrMem}{Branch if Greater Than or Equal}

Continues execution at the instruction specified by the relative instruction offset if the value of the second operand is greater than or equal to the value of the third operand.

\cdinstruction{brlt}{Offset}{ImmRegAdrMem}{ImmRegAdrMem}{Branch if Less Than}

Continues execution at the instruction specified by the relative instruction offset if the value of the second operand is less than the value of the third operand.

\cdinstruction{brne}{Offset}{ImmRegAdrMem}{ImmRegAdrMem}{Branch if Not Equal}

Continues execution at the instruction specified by the relative instruction offset if the values of the second and third operand do not match.

\cdinstruction{call}{Function}{Size}{}{Function Call}

Continues execution in the code section specified by the function pointer of the first operand and stores the address of the immediately following instruction in order to resume its execution after returning from the function.
This address is either stored in the link register or pushed onto the stack if the architecture does not support the link register.
The second operand specifies the amount of octets to be additionally popped from the stack upon returning from a function that needed arguments.
This number has to be aligned according to the stack alignment of the target platform.
For more information about stack operations, see Section~\ref{sec:cdstackoperations}.

\cdinstruction{conv}{RegMem}{ImmRegAdrMem}{}{Datum Conversion}

Converts the value of the second operand into the type specified by the first register or memory operand and stores the result there.
Converting two non-floating-point values yields the least integer congruent to the source operand modulo $2^{8n}$ where $n$ is the size of the destination type.
This operation is not available for conversions between pointers and floating-point numbers.

\cdinstruction{copy}{Pointer}{Pointer}{ImmRegMem}{Data Copy}

Copies data from the memory address named in the second operand to the memory address named in the first operand.
The third operand specifies the amount of octets to be copied.

\cdinstruction{def}{ImmAdr}{}{}{Datum Definition}

Enlarges the current data section by a single datum and initializes the resulting memory with the binary representation of the specified value.
The datum has to be aligned according to the data alignment of the target platform.
If the operand is an address, it cannot be incremented by a register, since registers are only available in code sections.

\cdinstruction{div}{RegMem}{ImmRegMem}{ImmRegMem}{Division}

Divides the value of the second operand by the value of the third operand and stores the result in the first register or memory operand.
For signed and unsigned integers, the fractional part of the quotient is discarded.
This operation is undefined if the third operand evaluates to zero, and is not available for pointers.

\cdinstruction{enter}{Size}{}{}{Stack Frame Creation}

Pushes the value of the frame pointer onto the stack and replaces it with the stack pointer.
The stack pointer is then decremented by the amount of octets specified by the operand.
This number has to be aligned according to the stack alignment of the target platform.
For more information about stack frames, see Section~\ref{sec:cdstackoperations}.

\cdinstruction{enum}{Offset}{}{}{Enumeration Type Declaration}

Declares an enumeration type that stores a single value from a predefined set of named constants called enumerators.
The operand defines the extent of the declaration in terms of the instruction range given by the relative instruction offset.
All enumerators declared within this extent belong to the most current enumeration declaration, see Section~\ref{sec:cdvalue}.
This compound type declaration requires a subsequent declaration for its underlying type.

\cdinstruction{field}{String}{Size}{Immediate}{Field Declaration}

Declares a field which associates the symbolic name of a storage object with its offset in the most current record declaration, see Section~\ref{sec:cdrec}.
The first operand specifies the name of the field where an empty name conventionally denotes inheritance rather than composition.
The second operand defines the offset of the field while the third stores the occupied bits in case it is a bit field.
Field declarations require subsequent source code locations and type declarations.

\cdinstruction{fill}{Pointer}{Pointer}{ImmRegAdrMem}{Data Initialization}

Initializes a memory region starting from the address named in first operand with the value of the third operand.
The second operand specifies the number of data values to be initialized.
The type of the third operand does not have to be a pointer.

\cdinstruction{fix}{Register}{}{}{Fix Register Mapping}

Fixes the current mapping of an unfixed general-purpose register given by the operand to its corresponding physical register.
This mapping stays fixed until the next occurrence of a corresponding \texttt{unfix} instruction in the instruction sequence, see Section~\ref{sec:cdunfix}.
A fixed mapping is required whenever an instruction sequence needs to refer to the same physical register before and after overwriting the value of a general-purpose register.

\cdinstruction{func}{Offset}{}{}{Function Type Declaration}

Declares a function type which represents the type of a functional unit of the programming language.
The operand defines the extent of the declaration in terms of the instruction range given by the relative instruction offset.
All freestanding types declared within this extent belong to the most current function type declaration and correspond to its parameters in order of occurrence.
This compound type declaration requires a subsequent type declaration for the type of the returned result.

\cdinstruction{jump}{Function}{}{}{Indirect Branch}

Continues execution in the code section specified by the function pointer of the operand.
This instruction is useful for branching into other code sections or indirect branches like returning from a function using the link register.
Direct branches within the same code section can be performed using the \texttt{br} instruction, see Section~\ref{sec:cdbr}.

\cdinstruction{leave}{}{}{}{Stack Frame Deletion}

Sets the stack pointer to the value of the frame pointer and pops the previous value of the frame pointer from the stack.
For more information about stack frames, see Section~\ref{sec:cdstackoperations}.

\cdinstruction{loc}{String}{Size}{Size}{Source Code Location}

Maps the current section or a preceding breakpoint, symbol declaration, or type declaration to its source code location.
The first operand names the source file and the second and third operands a line and column within that file.

\cdinstruction{lsh}{RegMem}{ImmRegMem}{ImmRegMem}{Left Shift}

Performs a left shift on the bits of the value of the second operand by the amount specified by the third operand and stores the result in the first register or memory operand.
This operation is undefined if the third operand is negative or greater than or equal to its size in bits, and is not available for pointers and floating-point numbers.

\cdinstruction{mod}{RegMem}{ImmRegMem}{ImmRegMem}{Modulo}

Divides the value of the second operand by the value of the third operand and stores the remainder in the first register or memory operand.
For signed integers, the remainder has the same sign as the dividend.
This operation is undefined if the third operand evaluates to zero, and is not available for pointers and floating-point numbers.

\cdinstruction{mov}{RegMem}{ImmRegAdrMem}{}{Datum Copy}

Copies the value of the second operand to the first register or memory operand.

\cdinstruction{mul}{RegMem}{ImmRegMem}{ImmRegMem}{Multiplication}

Multiplies the value of the second operand by the value of the third operand and stores the result in the first register or memory operand.
This operation is not available for function pointers.

\cdinstruction{neg}{RegMem}{ImmRegMem}{}{Negation}

Negates the value of the second operand and stores the result in the first register or memory operand.
This operation is not available for pointers.

\cdinstruction{nop}{}{}{}{No Operation}

Performs no operation.
This instruction allows an optimizer to elide unnecessary instructions as it is completely ignored by machine code generators.

\cdinstruction{not}{RegMem}{ImmRegMem}{}{Logical NOT}

Performs a one's complement negation on the bits of the value of the second operand and stores the result in the first register or memory operand.
This operation is not available for pointers and floating-point numbers.

\cdinstruction{or}{RegMem}{ImmRegMem}{ImmRegMem}{Logical OR}

Performs a logical OR operation on the bits of the values of the second and third operand and stores the result in the first register or memory operand.
This operation is not available for pointers and floating-point numbers.

\cdinstruction{pop}{RegMem}{}{}{Pop from Stack}

Stores the value pointed to by the stack pointer register into the first register or memory operand and increments the stack pointer register by the aligned size of that value.
For more information about stack operations, see Section~\ref{sec:cdstackoperations}.

\cdinstruction{ptr}{}{}{}{Pointer Type Declaration}

Declares a pointer type which may represent the address of a storage object that can be dereferenced.
This compound type declaration requires a subsequent declaration for the type of the referenced storage object.

\cdinstruction{push}{ImmRegAdrMem}{}{}{Push onto Stack}

Decrements the stack pointer register by the aligned size of the value of the operand and stores that value to the memory pointed to by the stack pointer.
For more information about stack operations, see Section~\ref{sec:cdstackoperations}.

\cdinstruction{rec}{Offset}{Size}{}{Record Type Declaration}

Declares a record type which represents a compound data structure with an arbitrary number of elements called fields.
The first operand defines the extent of the declaration in terms of the instruction range given by the relative instruction offset.
All fields declared within this extent belong to the most current record declaration, see Section~\ref{sec:cdfield}.
The second operand defines the overall size of the data structure.

\cdinstruction{ref}{}{}{}{Reference Type Declaration}

Declares a reference type which represents the address of a storage object that can be dereferenced.
This compound type declaration requires a subsequent declaration for the type of the referenced storage object.

\cdinstruction{req}{String}{}{}{Name Requirement}

Adds a dependency on a data or code section that is not explicitly referenced by any other instruction of the current section.
The operand specifies the name of the required section.

\cdinstruction{res}{Size}{}{}{Space Reservation}

Enlarges the current data section by the specified amount without initializing the resulting memory.

\cdinstruction{ret}{}{}{}{Return from Function}

Pops the function pointer from the stack and resumes the execution of the calling code section.
For more information about stack operations, see Section~\ref{sec:cdstackoperations}.

\cdinstruction{rsh}{RegMem}{ImmRegMem}{ImmRegMem}{Right Shift}

Performs a right shift on the bits of the value of the second operand by the amount specified by the third operand and stores the result in the first register or memory operand.
The shift operation is arithmetic for signed integers and logical otherwise.
This operation is undefined if the third operand is negative or greater than or equal to its size in bits, and is not available for pointers and floating-point numbers.

\cdinstruction{sub}{RegMem}{ImmRegAdrMem}{ImmRegAdrMem}{Subtraction}

Subtracts the value of the third operand from the value of the second operand and stores the result in the first register or memory operand.
This operation is not available for function pointers.

\cdinstruction{sym}{Offset}{String}{ImmRegMem}{Symbol Declaration}

Declares a symbol which associates the symbolic name of a storage object with its value or memory address.
The first operand defines the lifetime of the symbol in terms of the instruction range given by the relative instruction offset.
The second operand specifies the name of the symbol where an empty name conventionally denotes the result of the code section rather than a local variable or parameter.
The model of the third operand describes the kind of the symbol:

\begin{itemize}
\item An immediate value operand denotes a constant with the value of the operand.
\item A register operand declares an alias name for the currently mapped physical register of the operand and automatically fixes its mapping during the lifetime of the symbol.
\item A memory operand denotes a storage object at the address of the operand.
\end{itemize}

In all three cases the symbol declaration is also available as an additional constant definition in the assembly code of all \texttt{asm} instructions occuring within the lifetime of the symbol.
Symbol declarations require subsequent source code locations and type declarations.

\cdinstruction{trap}{Size}{}{}{Abnormal Program Termination}

Terminates the program abnormally due to an exceptional condition like an unsatisfied runtime check.
The operand provides implementation-defined information for a debugger in cases where its value can be encoded in the generated machine code.

\cdinstruction{type}{StrTyp}{}{}{Basic Type Declaration}

Declares a basic type which corresponds to the type of the named type section or the operand itself.

\cdinstruction{unfix}{Register}{}{}{Unfix Register Mapping}

Unfixes the current mapping of a general-purpose register which was previously fixed by a corresponding \texttt{fix} instruction, see Section~\ref{sec:cdfix}.

\cdinstruction{value}{String}{Imm}{}{Enumerator Declaration}

Declares an enumerator with a symbolic name and a constant value for the most current enumeration declaration, see Section~\ref{sec:cdenum}.
The first operand specifies the name of the enumerator and the second operand its value.
Enumerator declarations require subsequent source code locations.

\cdinstruction{void}{}{}{}{Void Type Declaration}

Declares a void type which represents an unspecified, ambiguous, or nonexistent type of the programming language.

\cdinstruction{xor}{RegMem}{ImmRegMem}{ImmRegMem}{Logical Exclusive OR}

Performs a logical exclusive OR operation on the bits of the values of the second and third operand and stores the result in the first register or memory operand.
This operation is not available for pointers and floating-point numbers.

\section{Code Optimizations}\index{Optimizations}\label{sec:cdoptimizations}

The \ecs{} features an optimizer that is able to improve programs represented using intermediate code.
This section describes all improvements and modifications on programs performed by the optimizer.
Since the intermediate code representation of a program is the only interface between the various front-ends and back-ends,
all compilers of the \ecs{} benefit from these optimizations by design.
Figure~\ref{fig:cdoptdataflow} shows the typical data flow within an optimizing compiler.

\begin{figure}
\flowgraph{
\resource{source code} \ar[d] \\
\converter{Front-End} \ar[d] \\
\resource{intermediate\\code} \ar[d] \ar@/u/[rr]^{\txt{code sections}} & \txt{\hphantom{code sections}} & \converter{Optimizer} \ar@/d/[ll]^{\txt{optimized sections}} \\
\converter{Back-End} \ar[d] \\
\resource{machine code} \\
}\caption{Data flow within an optimizing compiler}
\label{fig:cdoptdataflow}
\end{figure}

Memory operands followed by an exclamation point denote strict data accesses which are not subject to code optimizations.
This ensures accesses to memory that is considered volatile because it is shared with peripheral devices, signal handlers, or other threads of execution.
In general however, memory accesses are neither atomic nor ordered and should therefore never be used as a synchronization mechanism.

\section{Runtime Support}\index{Runtime support!for Intermediate Code}

The \ecs{} provides an interpreter called \tool{cd\-run} which executes programs written in intermediate code.
Basic runtime support for the emulated runtime environment is provided by the object file \objfile{code\-run}.
Programs written in \cpp{} need additional runtime support stored in the \libfile{cpp\-code\-run} library file.
Programs written in Oberon need additional runtime support stored in the \libfile{ob\-code\-run} library file.
\seecpp\seeoberon

\section{Intermediate Code Tools}\label{sec:cdtools}

The \ecs{} provides several different tools that generate or process files written in intermediate code.
While real compilers use intermediate code just as an internal and therefore inaccessible representation of the code to be compiled,
these tools allow exposing and modifying the actual intermediate code for debugging purposes.
For this reason, all optimizations mentioned in Section~\ref{sec:cdoptimizations} are disabled by default.
\interface\renewcommand{\debuggingtool}{}

\cdcheck
\cdopt
\cdrun
\cdamda
\cdamdb
\cdamdc
\cdarma
\cdarmb
\cdarmc
\cdarmcfpe
\cdavr
\cdavrtt
\cdmabk
\cdmibl
\cdmipsa
\cdmipsb
\cdmmix
\cdorok
\cdppca
\cdppcb
\cdrisc
\cdwasm
\cdxtensa
\cppcode
\falcode
\obcode

\concludechapter

% Debugging information representation
% Copyright (C) Florian Negele

% This file is part of the Eigen Compiler Suite.

% Permission is granted to copy, distribute and/or modify this document
% under the terms of the GNU Free Documentation License, Version 1.3
% or any later version published by the Free Software Foundation.

% You should have received a copy of the GNU Free Documentation License
% along with the ECS.  If not, see <https://www.gnu.org/licenses/>.

% Generic documentation utilities
% Copyright (C) Florian Negele

% This file is part of the Eigen Compiler Suite.

% Permission is granted to copy, distribute and/or modify this document
% under the terms of the GNU Free Documentation License, Version 1.3
% or any later version published by the Free Software Foundation.

% You should have received a copy of the GNU Free Documentation License
% along with the ECS.  If not, see <https://www.gnu.org/licenses/>.

\providecommand{\cpp}{C\texttt{++}}
\providecommand{\opt}{_\mathit{opt}}
\providecommand{\tool}[1]{\texttt{#1}}
\providecommand{\version}{Version 0.0.40}
\providecommand{\resource}[1]{*++\txt{#1}}
\providecommand{\ecs}{Eigen Compiler Suite}
\providecommand{\changed}[1]{\underline{#1}}
\providecommand{\toolbox}[1]{\converter{#1}}
\providecommand{\file}{}\renewcommand{\file}[1]{\texttt{#1}}
\providecommand{\alignright}{\hfill\linebreak[0]\hspace*{\fill}}
\providecommand{\converter}[1]{*++[F][F*:white][F,:gray]\txt{#1}}
\providecommand{\documentation}{\ifbook chapter\else document\fi}
\providecommand{\Documentation}{\ifbook Chapter\else Document\fi}
\providecommand{\variable}[1]{\resource{\texttt{\small#1}\\variable}}
\providecommand{\documentationref}[2]{\ifbook\ref{#1}\else``\href{#1}{#2}''~\cite{#1}\fi}
\providecommand{\objfile}[1]{\texttt{#1}\index[runtime]{#1 object file@\texttt{#1} object file}}
\providecommand{\libfile}[1]{\texttt{#1}\index[runtime]{#1 library file@\texttt{#1} library file}}
\providecommand{\epigraph}[2]{\ifbook\begin{quote}\flushright\textit{#1}\par--- #2\end{quote}\fi}
\providecommand{\environmentvariable}[1]{\texttt{#1}\index{Environment variables!#1@\texttt{#1}}}
\providecommand{\environment}[1]{\texttt{#1}\index[environment]{#1 environment@\texttt{#1} environment}}
\providecommand{\toolsection}{}\renewcommand{\toolsection}[1]{\subsection{#1}\label{\prefix:#1}\tool{#1}}
\providecommand{\instruction}{}\renewcommand{\instruction}[2]{\noindent\qquad\pdftooltip{\texttt{#1}}{#2}\refstepcounter{instruction}\par}
\providecommand{\flowgraph}{}\renewcommand{\flowgraph}[1]{\par\sffamily\begin{displaymath}\xymatrix@=4ex{#1}\end{displaymath}\normalfont\par}
\providecommand{\instructionset}{}\renewcommand{\instructionset}[4]{\setcounter{instruction}{0}\begin{multicols}{\ifbook#3\else#4\fi}[{\captionof{table}[#2]{#2 (\ref*{#1:instructions}~instructions)}\label{tab:#1set}\vspace{-2ex}}]\footnotesize\raggedcolumns\input{#1.set}\label{#1:instructions}\end{multicols}}

\providecommand{\gpl}{GNU General Public License}
\providecommand{\rse}{ECS Runtime Support Exception}
\providecommand{\fdl}{\href{https://www.gnu.org/licenses/fdl.html}{GNU Free Documentation License}}

\providecommand{\docbegin}{}
\providecommand{\docend}{}
\providecommand{\doclabel}[1]{\hypertarget{#1}}
\providecommand{\doclink}[2]{\hyperlink{#1}{#2}}
\providecommand{\docsection}[3]{\hypertarget{#1}{\subsection{#2}}\label{sec:#1}\index[library]{#2@#3}}
\providecommand{\docsectionstar}[1]{}
\providecommand{\docsubbegin}{\begin{description}}
\providecommand{\docsubend}{\end{description}}
\providecommand{\docsubsection}[3]{\item[\hypertarget{#1}{#2}]\index[library]{#2@#3}}
\providecommand{\docsubsectionstar}[1]{\smallskip}
\providecommand{\docsubsubsection}[3]{\docsubsection{#1}{#2}{#3}}
\providecommand{\docsubsubsectionstar}[1]{}
\providecommand{\docsubsubsubsection}[3]{}
\providecommand{\docsubsubsubsectionstar}[1]{}
\providecommand{\doctable}{}

\providecommand{\debuggingtool}{}\renewcommand{\debuggingtool}{This tool is provided for debugging purposes.
It allows exposing and modifying an internal data structure that is usually not accessible.
}

\providecommand{\interface}{All tools accept command-line arguments which are taken as names of plain text files containing the source code.
If no arguments are provided, the standard input stream is used instead.
Output files are generated in the current working directory and have the same name as the input file being processed whereas the filename extension gets replaced by an appropriate suffix.
\seeinterface
}

\providecommand{\license}{\noindent Copyright \copyright{} Florian Negele\par\medskip\noindent
Permission is granted to copy, distribute and/or modify this document under the terms of the
\fdl{}, Version 1.3 or any later version published by the \href{https://fsf.org/}{Free Software Foundation}.
}

\providecommand{\ecslogosurface}{
\fill[darkgray] (0,0,0) -- (0,0,3) -- (0,3,3) -- (0,3,1) -- (0,4,1) -- (0,4,3) -- (0,5,3) -- (0,5,0) -- (0,2,0) -- (0,2,2) -- (0,1,2) -- (0,1,0) -- cycle;
\fill[gray] (0,5,0) -- (0,5,3) -- (1,5,3) -- (1,5,1) -- (2,5,1) -- (2,5,3) -- (3,5,3) -- (3,5,0) -- cycle;
\fill[lightgray] (0,0,0) -- (0,1,0) -- (2,1,0) -- (2,4,0) -- (1,4,0) -- (1,3,0) -- (2,3,0) -- (2,2,0) -- (0,2,0) -- (0,5,0) -- (3,5,0) -- (3,0,0) -- cycle;
\begin{scope}[line width=0.5]
\begin{scope}[gray]
\draw (0,0,0) -- (0,1,0);
\draw (2,1,0) -- (2,2,0);
\draw (0,1,2) -- (0,2,2);
\draw (0,2,0) -- (0,5,0);
\draw (2,3,0) -- (2,4,0);
\end{scope}
\begin{scope}[lightgray]
\draw (0,1,0) -- (0,1,2);
\draw (0,3,1) -- (0,3,3);
\draw (0,5,0) -- (0,5,3);
\draw (2,5,1) -- (2,5,3);
\end{scope}
\begin{scope}[white]
\draw (0,1,0) -- (2,1,0);
\draw (1,3,0) -- (2,3,0);
\draw (0,5,0) -- (3,5,0);
\end{scope}
\end{scope}
}

\providecommand{\ecslogo}[1]{
\begin{tikzpicture}[scale={(#1)/((sin(45)+cos(45))*3cm)},x={({-cos(45)*1cm},{sin(45)*sin(30)*1cm})},y={({0cm},{(cos(30)*1cm})},z={({sin(45)*1cm},{cos(45)*sin(30)*1cm})}]
\begin{scope}[darkgray,line width=1]
\draw (0,0,0) -- (0,0,3) -- (0,3,3) -- (2,3,3) -- (2,5,3) -- (3,5,3) -- (3,5,0) -- (3,0,0) -- cycle;
\draw (0,3,1) -- (0,4,1) -- (0,4,3) -- (0,5,3) -- (1,5,3) -- (1,5,1) -- (2,5,1);
\draw (1,3,0) -- (1,4,0) -- (2,4,0);
\end{scope}
\fill[darkgray] (2,0,0) -- (2,0,3) -- (2,5,3) -- (2,5,1) -- (2,4,1) -- (2,4,0) -- cycle;
\fill[lightgray] (2,0,2) -- (0,0,2) -- (0,2,2) -- (2,2,2) -- cycle;
\fill[gray] (0,1,0) -- (2,1,0) -- (2,1,2) -- (0,1,2) -- cycle;
\fill[gray] (0,3,1) -- (0,3,3) -- (2,3,3) -- (2,3,0) -- (1,3,0) -- (1,3,1) -- cycle;
\ecslogosurface
\end{tikzpicture}
}

\providecommand{\shadowedecslogo}[3]{
\begin{tikzpicture}[scale={(#1)/((sin(#2)+cos(#2))*3cm)},x={({-cos(#2)*1cm},{sin(#2)*sin(#3)*1cm})},y={({0cm},{(cos(#3)*1cm})},z={({sin(#2)*1cm},{cos(#2)*sin(#3)*1cm})}]
\shade[top color=lightgray!50!white,bottom color=white,middle color=lightgray!50!white] (0,0,0) -- (3,0,0) -- (3,{-0.5-3*sin(#2)*sin(#3)/cos(#3)},0) -- (0,-0.5,0) -- cycle;
\shade[top color=darkgray!50!gray,bottom color=white,middle color=darkgray!50!white] (0,0,0) -- (0,0,3) -- (0,{-0.5-3*cos(#2)*sin(#3)/cos(#3)},3) -- (0,-0.5,0) -- cycle;
\begin{scope}[y={({(cos(#2)+sin(#2))*0.5cm},{(cos(#2)*sin(#3)-sin(#2)*sin(#3))*0.5cm})}]
\useasboundingbox (3,0,0) -- (0,0,0) -- (0,0,3);
\shade[left color=darkgray!80!black,right color=lightgray,middle color=gray] (0,0,0) -- (0,1,0) -- (0,1,0.5) -- (0,2,0) -- (0,5,0) -- (0,5,3) -- (1,5,3) -- (1,4,3) -- (1,4,2.5) -- (1,3,3) -- (2,5,3) -- (3,5,3) -- (3,0,3) -- cycle;
\clip (0,0,0) -- (0,0,3) -- ({-3*sin(#2)/cos(#2)},0,0) -- cycle;
\shade[left color=darkgray,right color=lightgray!50!gray] (0,0,0) -- (0,1,0) -- (0,1,0.5) -- (0,2,0) -- (0,5,0) -- (0,5,3) -- (1,5,3) -- (1,4,3) -- (1,4,2.5) -- (1,3,3) -- (2,5,3) -- (3,5,3) -- (3,0,3) -- cycle;
\end{scope}
\shade[left color=darkgray,right color=darkgray!80!black] (2,0,0) -- (2,0,3) -- (2,5,3) -- (2,5,1) -- (2,4,1) -- (2,4,0) -- cycle;
\shade[left color=darkgray!90!black,right color=gray!80!darkgray] (2,0,2) -- (0,0,2) -- (0,2,2) -- (2,2,2) -- cycle;
\shade[top color=darkgray!90!black,bottom color=gray!80!darkgray] (0,1,0) -- (2,1,0) -- (2,1,2) -- (0,1,2) -- cycle;
\shade[top color=darkgray!90!black,bottom color=gray!80!darkgray] (0,3,1) -- (0,3,3) -- (2,3,3) -- (2,3,0) -- (1,3,0) -- (1,3,1) -- cycle;
\fill[gray] (2,1,0) -- (1.5,1,0.5) -- (0,1,0.5) -- (0,1,0) -- cycle;
\fill[gray] (1,3,2) -- (0.5,3,2) -- (0.5,3,3) -- (1,3,3) -- cycle;
\fill[gray] (2,3,0) -- (1.5,3,0.5) -- (1,3,0.5) -- (1,3,0) -- cycle;
\ecslogosurface
\end{tikzpicture}
}

\providecommand{\cpplogo}[1]{
\begin{tikzpicture}[scale=(#1)/512em]
\fill[gray] (435.2794,398.7159) -- (247.1911,507.3075) .. controls (236.3563,513.5642) and (218.6240,513.5642) .. (207.7892,507.3075) -- (19.7009,398.7159) .. controls (8.8646,392.4606) and (0.0000,377.1043) .. (0.0000,364.5924) -- (0.0000,147.4076) .. controls (0.8430,132.8363) and (8.2856,120.7683) .. (19.7009,113.2842) -- (207.7892,4.6926) .. controls (218.6240,-1.5642) and (236.3564,-1.5642) .. (247.1911,4.6926) -- (435.2794,113.2842) .. controls (447.5273,121.4304) and (454.4987,133.6918) .. (454.9803,147.4076) -- (454.9803,364.5924) .. controls (454.5404,377.7571) and (446.6566,391.0351) .. (435.2794,398.7159) -- cycle(75.8301,255.9993) .. controls (74.9389,404.0881) and (273.2892,469.4783) .. (358.8263,331.8769) -- (293.1917,293.8965) .. controls (253.5702,359.4301) and (155.1909,335.9977) .. (151.6601,255.9993) .. controls (152.7204,182.2703) and (249.4137,148.0211) .. (293.1961,218.1065) -- (358.8308,180.1276) .. controls (283.4477,49.2645) and (79.6318,96.3470) .. (75.8301,255.9993) -- cycle(379.1503,247.5747) -- (362.2982,247.5747) -- (362.2982,230.7226) -- (345.4490,230.7226) -- (345.4490,247.5747) -- (328.5969,247.5747) -- (328.5969,264.4254) -- (345.4490,264.4254) -- (345.4490,281.2759) -- (362.2982,281.2759) -- (362.2982,264.4254) -- (379.1503,264.4254) -- cycle(442.3420,247.5747) -- (425.4899,247.5747) -- (425.4899,230.7226) -- (408.6408,230.7226) -- (408.6408,247.5747) -- (391.7886,247.5747) -- (391.7886,264.4254) -- (408.6408,264.4254) -- (408.6408,281.2759) -- (425.4899,281.2759) -- (425.4899,264.4254) -- (442.3420,264.4254) -- cycle;
\end{tikzpicture}
}

\providecommand{\fallogo}[1]{
\begin{tikzpicture}[scale=(#1)/512em]
\fill[gray] (185.7774,0.0000) .. controls (200.4486,15.9798) and (226.8966,8.7148) .. (235.0426,31.5836) .. controls (249.5297,58.0598) and (247.9581,97.9161) .. (280.3335,110.9762) .. controls (309.1690,120.3496) and (337.8406,104.2727) .. (366.5753,103.9379) .. controls (373.4449,111.5171) and (379.2885,128.2574) .. (383.9755,108.9744) .. controls (396.6979,102.5615) and (437.2808,107.6681) .. (426.9652,124.3252) .. controls (408.9822,121.0785) and (412.4742,146.0729) .. (426.5192,131.4996) .. controls (433.8413,120.8489) and (465.1541,126.5522) .. (441.9067,135.7950) .. controls (396.1879,157.7478) and (344.1112,161.5079) .. (298.5528,183.5702) .. controls (277.7471,193.5198) and (284.6941,218.7163) .. (285.2127,236.9640) .. controls (292.3599,316.2826) and (307.3929,394.6311) .. (317.1198,473.6154) .. controls (329.0637,505.4736) and (292.1195,528.5004) .. (265.9183,511.2761) .. controls (237.9284,499.2462) and (237.3684,465.2681) .. (230.9102,439.9421) .. controls (218.6692,374.3397) and (215.6307,306.9662) .. (198.1732,242.3977) .. controls (183.1379,232.7444) and (164.4245,256.0298) .. (149.0430,261.4799) .. controls (116.9328,279.2585) and (87.1822,308.5851) .. (48.2293,307.8914) .. controls (21.3220,306.9037) and (-15.9107,281.8761) .. (7.2921,252.7908) .. controls (29.7799,220.6177) and (67.5177,204.3028) .. (100.9287,185.9449) .. controls (130.8217,170.8906) and (161.1548,156.5903) .. (191.0278,141.5847) .. controls (196.1738,120.0520) and (186.6049,95.2409) .. (186.8382,72.4353) .. controls (185.5234,48.4204) and (183.1700,23.9341) .. (185.7774,0.0000) -- cycle;
\end{tikzpicture}
}

\providecommand{\oblogo}[1]{
\begin{tikzpicture}[scale=(#1)/512em]
\fill[gray] (160.3865,208.9117) .. controls (154.0879,214.6478) and (149.0735,221.2409) .. (145.4125,228.5384) .. controls (184.8790,248.4273) and (234.7122,269.8787) .. (297.5493,291.8782) .. controls (300.3943,281.4769) and (300.9552,268.7619) .. (300.4023,255.2389) .. controls (248.9909,244.7891) and (200.0310,225.9279) .. (160.3865,208.9117) -- cycle(225.7398,392.6996) .. controls (308.0209,392.1716) and (359.3326,345.9277) .. (368.7203,285.2098) .. controls (376.6742,197.1784) and (311.7194,141.3342) .. (205.4287,142.1456) .. controls (139.9485,141.4804) and (88.7155,166.1957) .. (73.5775,228.0086) .. controls (52.0297,320.3408) and (123.4078,391.0103) .. (225.7398,392.6996) -- cycle(216.0739,176.4733) .. controls (268.9183,179.2424) and (315.8292,206.5488) .. (312.7454,265.1139) .. controls (313.2769,315.6384) and (286.5993,353.4946) .. (216.6040,355.7934) .. controls (162.4657,355.7934) and (126.0914,317.5023) .. (126.0914,260.5103) .. controls (126.1733,214.2900) and (163.3363,176.2849) .. (216.0739,176.4733) -- cycle(76.4897,189.1754) .. controls (13.1586,147.5631) and (0.0000,119.4207) .. (0.0000,119.4207) -- (90.6499,170.1632) .. controls (85.3004,175.8497) and (80.5994,182.1633) .. (76.4897,189.1754) -- cycle(353.9486,119.3004) -- (402.9482,119.3004) .. controls (427.0025,137.0797) and (450.9893,162.7034) .. (474.9529,191.0213) .. controls (509.3540,228.5339) and (531.3391,294.2091) .. (487.8149,312.1206) .. controls (462.8165,324.7652) and (394.3874,316.8943) .. (373.8912,313.6651) .. controls (379.9291,297.7449) and (383.2899,278.4204) .. (381.4989,257.7214) .. controls (420.3069,248.0321) and (421.9610,218.3461) .. (407.7867,192.6417) .. controls (391.1113,162.4018) and (370.1114,132.9097) .. (353.9486,119.3004) -- cycle;
\end{tikzpicture}
}

\providecommand{\markuptable}{
\begin{table}
\sffamily\centering
\begin{tabular}{@{}lcl@{}}
\toprule
\texttt{//italics//} & $\rightarrow$ & \textit{italics} \\
\midrule
\texttt{**bold**} & $\rightarrow$ & \textbf{bold} \\
\midrule
\texttt{\# ordered list} & & 1 ordered list \\
\texttt{\# second item} & $\rightarrow$ & 2 second item \\
\texttt{\#\# sub item} & & \hspace{1em} 1 sub item \\
\midrule
\texttt{* unordered list} & & $\bullet$ unordered list \\
\texttt{* second item} & $\rightarrow$ & $\bullet$ second item \\
\texttt{** sub item} & & \hspace{1em} $\bullet$ sub item \\
\midrule
\texttt{link to [[label]]} & $\rightarrow$ & link to \underline{label} \\
\midrule
\texttt{<{}<label>{}> definition } & $\rightarrow$ & definition \\
\midrule
\texttt{[[url|link name]]} & $\rightarrow$ & \underline{link name} \\
\midrule\addlinespace
\texttt{= large heading} & & {\Large large heading} \smallskip \\
\texttt{== medium heading} & $\rightarrow$ & {\large medium heading} \\
\texttt{=== small heading} & & small heading \\
\midrule
\texttt{no line break} & & no line break for paragraphs \\
\texttt{for paragraphs} & $\rightarrow$ \\
& & use empty line \\
\texttt{use empty line} \\
\midrule
\texttt{force\textbackslash\textbackslash line break} & $\rightarrow$ & force \\
& & line break \\
\midrule
\texttt{horizontal line} & $\rightarrow$ & horizontal line \\
\texttt{----} & & \hrulefill \\
\midrule
\texttt{|=a|=table|=header} & & \underline{a \enspace table \enspace header} \\
\texttt{|a|table|row} & $\rightarrow$ & a \enspace table \enspace row \\
\texttt{|b|table|row} & & b \enspace table \enspace row \\
\midrule
\texttt{\{\{\{} \\
\texttt{unformatted} & $\rightarrow$ & \texttt{unformatted} \\
\texttt{code} & & \texttt{code} \\
\texttt{\}\}\}} \\
\midrule\addlinespace
\texttt{@ new article} & & {\Large 1.\ new article} \smallskip \\
\texttt{@ second article} & $\rightarrow$ & {\Large 2.\ second article} \smallskip \\
\texttt{@@ sub article} & & {\large 2.1.\ sub article} \\
\bottomrule
\end{tabular}
\normalfont\caption{Elements of the generic documentation markup language}
\label{tab:docmarkup}
\end{table}
}

\providecommand{\startchapter}[4]{
\documentclass[11pt,a4paper]{article}
\usepackage{booktabs}
\usepackage[format=hang,labelfont=bf]{caption}
\usepackage{changepage}
\usepackage[T1]{fontenc}
\usepackage[margin=2cm]{geometry}
\usepackage{hyperref}
\usepackage[american]{isodate}
\usepackage{lmodern}
\usepackage{longtable}
\usepackage{mathptmx}
\usepackage{microtype}
\usepackage[toc]{multitoc}
\usepackage{multirow}
\usepackage[all]{nowidow}
\usepackage{pdfcomment}
\usepackage{syntax}
\usepackage{tikz}
\usepackage[all]{xy}
\hypersetup{pdfborder={0 0 0},bookmarksnumbered=true,pdftitle={\ecs{}: #2},pdfauthor={Florian Negele},pdfsubject={\ecs{}},pdfkeywords={#1}}
\setlength{\grammarindent}{8em}\setlength{\grammarparsep}{0.2ex}
\setlength{\columnsep}{2em}
\newcommand{\prefix}{}
\newcounter{instruction}
\bibliographystyle{unsrt}
\renewcommand{\index}[2][]{}
\renewcommand{\arraystretch}{1.05}
\renewcommand{\floatpagefraction}{0.7}
\renewcommand{\syntleft}{\itshape}\renewcommand{\syntright}{}
\title{\vspace{-5ex}\Huge{\ecs{}}\medskip\hrule}
\author{\huge{#2}}
\date{\medskip\version}
\newif\ifbook\bookfalse
\pagestyle{headings}
\frenchspacing
\begin{document}
\maketitle\thispagestyle{empty}\noindent#4\setlength{\columnseprule}{0.4pt}\tableofcontents\setlength{\columnseprule}{0pt}\vfill\pagebreak[3]\null\vfill\bigskip\noindent
\parbox{\textwidth-4em}{\license The contents of this \documentation{} are part of the \href{manual}{\ecs{} User Manual}~\cite{manual} and correspond to Chapter ``\href{manual\##3}{#1}''.\alignright\mbox{\today}}
\parbox{4em}{\flushright\ecslogo{3em}}
\clearpage
}

\providecommand{\concludechapter}{
\vfill\pagebreak[3]\null\vfill
\thispagestyle{myheadings}\markright{REFERENCES}
\noindent\begin{minipage}{\textwidth}\begin{multicols}{2}[\section*{References}]
\renewcommand{\section}[2]{}\small\bibliography{references}
\end{multicols}\end{minipage}\end{document}
}

\providecommand{\startpresentation}[2]{
\documentclass[14pt,aspectratio=43,usepdftitle=false]{beamer}
\usepackage{booktabs}
\usepackage{etex}
\usepackage{multicol}
\usepackage{tikz}
\usepackage[all]{xy}
\bibliographystyle{unsrt}
\setlength{\columnsep}{1em}
\setlength{\leftmargini}{1em}
\setbeamercolor{title}{fg=black}
\setbeamercolor{structure}{fg=darkgray}
\setbeamercolor{bibliography item}{fg=darkgray}
\setbeamerfont{title}{series=\bfseries}
\setbeamerfont{subtitle}{series=\normalfont}
\setbeamerfont*{frametitle}{parent=title}
\setbeamerfont{block title}{series=\bfseries}
\setbeamerfont*{framesubtitle}{parent=subtitle}
\setbeamersize{text margin left=1em,text margin right=1em}
\setbeamertemplate{navigation symbols}{}
\setbeamertemplate{itemize item}[circle]{}
\setbeamertemplate{bibliography item}[triangle]{}
\setbeamertemplate{bibliography entry author}{\usebeamercolor[fg]{bibliography item}}
\setbeamertemplate{frametitle}{\medskip\usebeamerfont{frametitle}\color{gray}\raisebox{-2.5ex}[0ex][0ex]{\rule{0.1em}{4.5ex}}}
\addtobeamertemplate{frametitle}{}{\hspace{0.4em}\usebeamercolor[fg]{title}\insertframetitle\par\vspace{0.2ex}\hspace{0.5em}\usebeamerfont{framesubtitle}\insertframesubtitle}
\hypersetup{pdfborder={0 0 0},bookmarksnumbered=true,bookmarksopen=true,bookmarksopenlevel=0,pdftitle={\ecs{}: #1},pdfauthor={Florian Negele},pdfsubject={\ecs{}},pdfkeywords={#1}}
\renewcommand{\flowgraph}[1]{\resizebox{\textwidth}{!}{$$\xymatrix{##1}$$}}
\title{\ecs{}\medskip\hrule\medskip}
\institute{\shadowedecslogo{5em}{30}{15}}
\date{\version}
\subtitle{#1}
\begin{document}
\begin{frame}[plain]\titlepage\nocite{manual}\end{frame}
\begin{frame}{Contents}{#1}\begin{center}\tableofcontents\end{center}\end{frame}
}

\providecommand{\concludepresentation}{
\begin{frame}{References}\begin{footnotesize}\setlength{\columnseprule}{0.4pt}\begin{multicols}{2}\bibliography{references}\end{multicols}\end{footnotesize}\end{frame}
\end{document}
}

\providecommand{\startbook}[1]{
\documentclass[10pt,paper=17cm:24cm,DIV=13,twoside=semi,headings=normal,numbers=noendperiod,cleardoublepage=plain]{scrbook}
\usepackage{atveryend}
\usepackage{booktabs}
\usepackage{caption}
\usepackage{changepage}
\usepackage[T1]{fontenc}
\usepackage{imakeidx}
\usepackage{hyperref}
\usepackage[american]{isodate}
\usepackage{lmodern}
\usepackage{longtable}
\usepackage{mathptmx}
\usepackage[final]{microtype}
\usepackage{multicol}
\usepackage{multirow}
\usepackage[all]{nowidow}
\usepackage{pdfcomment}
\usepackage{scrlayer-scrpage}
\usepackage{setspace}
\usepackage{syntax}
\usepackage[eventxtindent=4pt,oddtxtexdent=4pt]{thumbs}
\usepackage{tikz}
\usepackage[all]{xy}
\hyphenation{Micro-Blaze Open-Cores Open-RISC Power-PC}
\hypersetup{pdfborder={0 0 0},bookmarksnumbered=true,bookmarksopen=true,bookmarksopenlevel=0,pdftitle={\ecs{}: #1},pdfauthor={Florian Negele},pdfsubject={\ecs{}},pdfkeywords={#1}}
\setlength{\grammarindent}{8em}\setlength{\grammarparsep}{0.7ex}
\setkomafont{captionlabel}{\usekomafont{descriptionlabel}}
\renewcommand{\arraystretch}{1.05}\setstretch{1.1}
\renewcommand{\chapterformat}{\thechapter\autodot\enskip\raisebox{-1ex}[0ex][0ex]{\color{gray}\rule{0.1em}{3.5ex}}\enskip}
\renewcommand{\startchapter}[4]{\hypertarget{##3}{\chapter{##1}}\label{##3}##4\addthumb{##1}{\LARGE\sffamily\bfseries\thechapter}{white}{gray}\renewcommand{\prefix}{##3}}
\renewcommand{\concludechapter}{\clearpage{\stopthumb\cleardoublepage}}
\renewcommand{\syntleft}{\itshape}\renewcommand{\syntright}{}
\renewcommand{\floatpagefraction}{0.7}
\renewcommand{\partheademptypage}{}
\DeclareMicrotypeAlias{lmss}{cmr}
\newcommand{\prefix}{}
\newcounter{instruction}
\bibliographystyle{unsrt}
\newif\ifbook\booktrue
\makeindex[intoc,title=Index]
\makeindex[intoc,name=tools,title=Index of Tools,columns=3]
\makeindex[intoc,name=library,title=Index of Library Names]
\makeindex[intoc,name=runtime,title=Index of Runtime Support]
\makeindex[intoc,name=environment,title=Index of Target Environments]
\indexsetup{toclevel=chapter,headers={\indexname}{\indexname}}
\frenchspacing
\begin{document}
\pagenumbering{alph}
\begin{titlepage}\centering
\huge\sffamily\null\vfill\textbf{\ecs{}}\bigskip\hrule\bigskip#1
\normalsize\normalfont\vfill\vfill\shadowedecslogo{10em}{30}{15}
\large\vfill\vfill\version
\end{titlepage}
\null\vfill
\thispagestyle{empty}
\noindent\today\par\medskip
\license A copy of this license is included in Appendix~\ref{fdl} on page~\pageref{fdl}.
All product names used herein are for identification purposes only and may be trademarks of their respective companies.
\concludechapter
\frontmatter
\setcounter{tocdepth}{1}
\tableofcontents
\setcounter{tocdepth}{2}
\concludechapter
\listoffigures
\concludechapter
\listoftables
\concludechapter
}

\providecommand{\concludebook}{
\backmatter
\addtocontents{toc}{\protect\setcounter{tocdepth}{-1}}
\phantomsection\addcontentsline{toc}{part}{Bibliography}
\bibliography{references}
\concludechapter
\phantomsection\addcontentsline{toc}{part}{Indexes}
\printindex
\concludechapter
\indexprologue{\label{idx:tools}}
\printindex[tools]
\concludechapter
\printindex[library]
\concludechapter
\indexprologue{\label{idx:runtime}}
\printindex[runtime]
\concludechapter
\indexprologue{\label{idx:environment}}
\printindex[environment]
\concludechapter
\pagestyle{empty}\pagenumbering{Alph}\null\clearpage
\null\vfill\centering\ecslogo{4em}\par\medskip\license
\end{document}
}

% chapter references

\providecommand{\seedocumentationref}{}\renewcommand{\seedocumentationref}[3]{#1, see \Documentation{}~\documentationref{#2}{#3}. }
\providecommand{\seeinterface}{}\renewcommand{\seeinterface}{\ifbook See \Documentation{}~\documentationref{interface}{User Interface} for more information about the common user interface of all of these tools. \fi}
\providecommand{\seeguide}{}\renewcommand{\seeguide}{\seedocumentationref{For basic examples of using some of these tools in practice}{guide}{User Guide}}
\providecommand{\seecpp}{}\renewcommand{\seecpp}{\seedocumentationref{For more information about the \cpp{} programming language and its implementation by the \ecs{}}{cpp}{User Manual for \cpp{}}}
\providecommand{\seefalse}{}\renewcommand{\seefalse}{\seedocumentationref{For more information about the FALSE programming language and its implementation by the \ecs{}}{false}{User Manual for FALSE}}
\providecommand{\seeoberon}{}\renewcommand{\seeoberon}{\seedocumentationref{For more information about the Oberon programming language and its implementation by the \ecs{}}{oberon}{User Manual for Oberon}}
\providecommand{\seeassembly}{}\renewcommand{\seeassembly}{\seedocumentationref{For more information about the generic assembly language and how to use it}{assembly}{Generic Assembly Language Specification}}
\providecommand{\seeamd}{}\renewcommand{\seeamd}{\seedocumentationref{For more information about how the \ecs{} supports the AMD64 hardware architecture}{amd64}{AMD64 Hardware Architecture Support}}
\providecommand{\seearm}{}\renewcommand{\seearm}{\seedocumentationref{For more information about how the \ecs{} supports the ARM hardware architecture}{arm}{ARM Hardware Architecture Support}}
\providecommand{\seeavr}{}\renewcommand{\seeavr}{\seedocumentationref{For more information about how the \ecs{} supports the AVR hardware architecture}{avr}{AVR Hardware Architecture Support}}
\providecommand{\seeavrtt}{}\renewcommand{\seeavrtt}{\seedocumentationref{For more information about how the \ecs{} supports the AVR32 hardware architecture}{avr32}{AVR32 Hardware Architecture Support}}
\providecommand{\seemabk}{}\renewcommand{\seemabk}{\seedocumentationref{For more information about how the \ecs{} supports the M68000 hardware architecture}{m68k}{M68000 Hardware Architecture Support}}
\providecommand{\seemibl}{}\renewcommand{\seemibl}{\seedocumentationref{For more information about how the \ecs{} supports the MicroBlaze hardware architecture}{mibl}{MicroBlaze Hardware Architecture Support}}
\providecommand{\seemips}{}\renewcommand{\seemips}{\seedocumentationref{For more information about how the \ecs{} supports the MIPS32 and MIPS64 hardware architectures}{mips}{MIPS Hardware Architecture Support}}
\providecommand{\seemmix}{}\renewcommand{\seemmix}{\seedocumentationref{For more information about how the \ecs{} supports the MMIX hardware architecture}{mmix}{MMIX Hardware Architecture Support}}
\providecommand{\seeorok}{}\renewcommand{\seeorok}{\seedocumentationref{For more information about how the \ecs{} supports the OpenRISC 1000 hardware architecture}{or1k}{OpenRISC 1000 Hardware Architecture Support}}
\providecommand{\seeppc}{}\renewcommand{\seeppc}{\seedocumentationref{For more information about how the \ecs{} supports the PowerPC hardware architecture}{ppc}{PowerPC Hardware Architecture Support}}
\providecommand{\seerisc}{}\renewcommand{\seerisc}{\seedocumentationref{For more information about how the \ecs{} supports the RISC hardware architecture}{risc}{RISC Hardware Architecture Support}}
\providecommand{\seewasm}{}\renewcommand{\seewasm}{\seedocumentationref{For more information about how the \ecs{} supports the WebAssembly architecture}{wasm}{WebAssembly Architecture Support}}
\providecommand{\seedocumentation}{}\renewcommand{\seedocumentation}{\seedocumentationref{For more information about generic documentations and their generation by the \ecs{}}{documentation}{Generic Documentation Generation}}
\providecommand{\seedebugging}{}\renewcommand{\seedebugging}{\seedocumentationref{For more information about debugging information and its representation}{debugging}{Debugging Information Representation}}
\providecommand{\seecode}{}\renewcommand{\seecode}{\seedocumentationref{For more information about intermediate code and its purpose}{code}{Intermediate Code Representation}}
\providecommand{\seeobject}{}\renewcommand{\seeobject}{\seedocumentationref{For more information about object files and their purpose}{object}{Object File Representation}}

% generic documentation tools

\providecommand{\docprint}{
\toolsection{docprint} is a pretty printer for generic documentations.
It reformats generic documentations and writes it to the standard output stream.
\debuggingtool
\flowgraph{\resource{generic\\documentation} \ar[r] & \toolbox{docprint} \ar[r] & \resource{generic\\documentation}}
\seedocumentation
}

\providecommand{\doccheck}{
\toolsection{doccheck} is a syntactic and semantic checker for generic documentations.
It just performs syntactic and semantic checks on generic documentations and writes its diagnostic messages to the standard error stream.
\debuggingtool
\flowgraph{\resource{generic\\documentation} \ar[r] & \toolbox{doccheck} \ar[r] & \resource{diagnostic\\messages}}
\seedocumentation
}

\providecommand{\dochtml}{
\toolsection{dochtml} is an HTML documentation generator for generic documentations.
It processes several generic documentations and assembles all information therein into an HTML document.
\debuggingtool
\flowgraph{\resource{generic\\documentation} \ar[r] & \toolbox{dochtml} \ar[r] & \resource{HTML\\document}}
\seedocumentation
}

\providecommand{\doclatex}{
\toolsection{doclatex} is a Latex documentation generator for generic documentations.
It processes several generic documentations and assembles all information therein into a Latex document.
\debuggingtool
\flowgraph{\resource{generic\\documentation} \ar[r] & \toolbox{doclatex} \ar[r] & \resource{Latex\\document}}
\seedocumentation
}

% intermediate code tools

\providecommand{\cdcheck}{
\toolsection{cdcheck} is a syntactic and semantic checker for intermediate code.
It just performs syntactic and semantic checks on programs written in intermediate code and writes its diagnostic messages to the standard error stream.
\debuggingtool
\flowgraph{\resource{intermediate\\code} \ar[r] & \toolbox{cdcheck} \ar[r] & \resource{diagnostic\\messages}}
\seeassembly\seecode
}

\providecommand{\cdopt}{
\toolsection{cdopt} is an optimizer for intermediate code.
It performs various optimizations on programs written in intermediate code and writes the result to the standard output stream.
\debuggingtool
\flowgraph{\resource{intermediate\\code} \ar[r] & \toolbox{cdopt} \ar[r] & \resource{optimized\\code}}
\seeassembly\seecode
}

\providecommand{\cdrun}{
\toolsection{cdrun} is an interpreter for intermediate code.
It processes and executes programs written in intermediate code.
The following code sections are predefined and have the usual semantics:
\texttt{abort}, \texttt{\_Exit}, \texttt{fflush}, \texttt{floor}, \texttt{fputc}, \texttt{free}, \texttt{getchar}, \texttt{malloc}, and \texttt{putchar}.
Diagnostic messages about invalid operations include the name of the executed code section and the index of the erroneous instruction.
\debuggingtool
\flowgraph{\resource{intermediate\\code} \ar[r] & \toolbox{cdrun} \ar@/u/[r] & \resource{input/\\output} \ar@/d/[l]}
\seeassembly\seecode
}

\providecommand{\cdamda}{
\toolsection{cdamd16} is a compiler for intermediate code targeting the AMD64 hardware architecture.
It generates machine code for AMD64 processors from programs written in intermediate code and stores it in corresponding object files.
The compiler generates machine code for the 16-bit operating mode defined by the AMD64 architecture.
It also creates a debugging information file as well as an assembly file containing a listing of the generated machine code.
\debuggingtool
\flowgraph{\resource{intermediate\\code} \ar[r] & \toolbox{cdamd16} \ar[r] \ar[d] \ar[rd] & \resource{object file} \\ & \resource{assembly\\listing} & \resource{debugging\\information}}
\seeassembly\seeamd\seeobject\seecode\seedebugging
}

\providecommand{\cdamdb}{
\toolsection{cdamd32} is a compiler for intermediate code targeting the AMD64 hardware architecture.
It generates machine code for AMD64 processors from programs written in intermediate code and stores it in corresponding object files.
The compiler generates machine code for the 32-bit operating mode defined by the AMD64 architecture.
It also creates a debugging information file as well as an assembly file containing a listing of the generated machine code.
\debuggingtool
\flowgraph{\resource{intermediate\\code} \ar[r] & \toolbox{cdamd32} \ar[r] \ar[d] \ar[rd] & \resource{object file} \\ & \resource{assembly\\listing} & \resource{debugging\\information}}
\seeassembly\seeamd\seeobject\seecode\seedebugging
}

\providecommand{\cdamdc}{
\toolsection{cdamd64} is a compiler for intermediate code targeting the AMD64 hardware architecture.
It generates machine code for AMD64 processors from programs written in intermediate code and stores it in corresponding object files.
The compiler generates machine code for the 64-bit operating mode defined by the AMD64 architecture.
It also creates a debugging information file as well as an assembly file containing a listing of the generated machine code.
\debuggingtool
\flowgraph{\resource{intermediate\\code} \ar[r] & \toolbox{cdamd64} \ar[r] \ar[d] \ar[rd] & \resource{object file} \\ & \resource{assembly\\listing} & \resource{debugging\\information}}
\seeassembly\seeamd\seeobject\seecode\seedebugging
}

\providecommand{\cdarma}{
\toolsection{cdarma32} is a compiler for intermediate code targeting the ARM hardware architecture.
It generates machine code for ARM processors executing A32 instructions from programs written in intermediate code and stores it in corresponding object files.
It also creates a debugging information file as well as an assembly file containing a listing of the generated machine code.
\debuggingtool
\flowgraph{\resource{intermediate\\code} \ar[r] & \toolbox{cdarma32} \ar[r] \ar[d] \ar[rd] & \resource{object file} \\ & \resource{assembly\\listing} & \resource{debugging\\information}}
\seeassembly\seearm\seeobject\seecode\seedebugging
}

\providecommand{\cdarmb}{
\toolsection{cdarma64} is a compiler for intermediate code targeting the ARM hardware architecture.
It generates machine code for ARM processors executing A64 instructions from programs written in intermediate code and stores it in corresponding object files.
It also creates a debugging information file as well as an assembly file containing a listing of the generated machine code.
\debuggingtool
\flowgraph{\resource{intermediate\\code} \ar[r] & \toolbox{cdarma64} \ar[r] \ar[d] \ar[rd] & \resource{object file} \\ & \resource{assembly\\listing} & \resource{debugging\\information}}
\seeassembly\seearm\seeobject\seecode\seedebugging
}

\providecommand{\cdarmc}{
\toolsection{cdarmt32} is a compiler for intermediate code targeting the ARM hardware architecture.
It generates machine code for ARM processors without floating-point extension executing T32 instructions from programs written in intermediate code and stores it in corresponding object files.
It also creates a debugging information file as well as an assembly file containing a listing of the generated machine code.
\debuggingtool
\flowgraph{\resource{intermediate\\code} \ar[r] & \toolbox{cdarmt32} \ar[r] \ar[d] \ar[rd] & \resource{object file} \\ & \resource{assembly\\listing} & \resource{debugging\\information}}
\seeassembly\seearm\seeobject\seecode\seedebugging
}

\providecommand{\cdarmcfpe}{
\toolsection{cdarmt32fpe} is a compiler for intermediate code targeting the ARM hardware architecture.
It generates machine code for ARM processors with floating-point extension executing T32 instructions from programs written in intermediate code and stores it in corresponding object files.
It also creates a debugging information file as well as an assembly file containing a listing of the generated machine code.
\debuggingtool
\flowgraph{\resource{intermediate\\code} \ar[r] & \toolbox{cdarmt32fpe} \ar[r] \ar[d] \ar[rd] & \resource{object file} \\ & \resource{assembly\\listing} & \resource{debugging\\information}}
\seeassembly\seearm\seeobject\seecode\seedebugging
}

\providecommand{\cdavr}{
\toolsection{cdavr} is a compiler for intermediate code targeting the AVR hardware architecture.
It generates machine code for AVR processors from programs written in intermediate code and stores it in corresponding object files.
It also creates a debugging information file as well as an assembly file containing a listing of the generated machine code.
\debuggingtool
\flowgraph{\resource{intermediate\\code} \ar[r] & \toolbox{cdavr} \ar[r] \ar[d] \ar[rd] & \resource{object file} \\ & \resource{assembly\\listing} & \resource{debugging\\information}}
\seeassembly\seeavr\seeobject\seecode\seedebugging
}

\providecommand{\cdavrtt}{
\toolsection{cdavr32} is a compiler for intermediate code targeting the AVR32 hardware architecture.
It generates machine code for AVR32 processors from programs written in intermediate code and stores it in corresponding object files.
It also creates a debugging information file as well as an assembly file containing a listing of the generated machine code.
\debuggingtool
\flowgraph{\resource{intermediate\\code} \ar[r] & \toolbox{cdavr32} \ar[r] \ar[d] \ar[rd] & \resource{object file} \\ & \resource{assembly\\listing} & \resource{debugging\\information}}
\seeassembly\seeavrtt\seeobject\seecode\seedebugging
}

\providecommand{\cdmabk}{
\toolsection{cdm68k} is a compiler for intermediate code targeting the M68000 hardware architecture.
It generates machine code for M68000 processors from programs written in intermediate code and stores it in corresponding object files.
It also creates a debugging information file as well as an assembly file containing a listing of the generated machine code.
\debuggingtool
\flowgraph{\resource{intermediate\\code} \ar[r] & \toolbox{cdm68k} \ar[r] \ar[d] \ar[rd] & \resource{object file} \\ & \resource{assembly\\listing} & \resource{debugging\\information}}
\seeassembly\seemabk\seeobject\seecode\seedebugging
}

\providecommand{\cdmibl}{
\toolsection{cdmibl} is a compiler for intermediate code targeting the MicroBlaze hardware architecture.
It generates machine code for MicroBlaze processors from programs written in intermediate code and stores it in corresponding object files.
It also creates a debugging information file as well as an assembly file containing a listing of the generated machine code.
\debuggingtool
\flowgraph{\resource{intermediate\\code} \ar[r] & \toolbox{cdmibl} \ar[r] \ar[d] \ar[rd] & \resource{object file} \\ & \resource{assembly\\listing} & \resource{debugging\\information}}
\seeassembly\seemibl\seeobject\seecode\seedebugging
}

\providecommand{\cdmipsa}{
\toolsection{cdmips32} is a compiler for intermediate code targeting the MIPS32 hardware architecture.
It generates machine code for MIPS32 processors from programs written in intermediate code and stores it in corresponding object files.
It also creates a debugging information file as well as an assembly file containing a listing of the generated machine code.
\debuggingtool
\flowgraph{\resource{intermediate\\code} \ar[r] & \toolbox{cdmips32} \ar[r] \ar[d] \ar[rd] & \resource{object file} \\ & \resource{assembly\\listing} & \resource{debugging\\information}}
\seeassembly\seemips\seeobject\seecode\seedebugging
}

\providecommand{\cdmipsb}{
\toolsection{cdmips64} is a compiler for intermediate code targeting the MIPS64 hardware architecture.
It generates machine code for MIPS64 processors from programs written in intermediate code and stores it in corresponding object files.
It also creates a debugging information file as well as an assembly file containing a listing of the generated machine code.
\debuggingtool
\flowgraph{\resource{intermediate\\code} \ar[r] & \toolbox{cdmips64} \ar[r] \ar[d] \ar[rd] & \resource{object file} \\ & \resource{assembly\\listing} & \resource{debugging\\information}}
\seeassembly\seemips\seeobject\seecode\seedebugging
}

\providecommand{\cdmmix}{
\toolsection{cdmmix} is a compiler for intermediate code targeting the MMIX hardware architecture.
It generates machine code for MMIX processors from programs written in intermediate code and stores it in corresponding object files.
It also creates a debugging information file as well as an assembly file containing a listing of the generated machine code.
\debuggingtool
\flowgraph{\resource{intermediate\\code} \ar[r] & \toolbox{cdmmix} \ar[r] \ar[d] \ar[rd] & \resource{object file} \\ & \resource{assembly\\listing} & \resource{debugging\\information}}
\seeassembly\seemmix\seeobject\seecode\seedebugging
}

\providecommand{\cdorok}{
\toolsection{cdor1k} is a compiler for intermediate code targeting the OpenRISC 1000 hardware architecture.
It generates machine code for OpenRISC 1000 processors from programs written in intermediate code and stores it in corresponding object files.
It also creates a debugging information file as well as an assembly file containing a listing of the generated machine code.
\debuggingtool
\flowgraph{\resource{intermediate\\code} \ar[r] & \toolbox{cdor1k} \ar[r] \ar[d] \ar[rd] & \resource{object file} \\ & \resource{assembly\\listing} & \resource{debugging\\information}}
\seeassembly\seeorok\seeobject\seecode\seedebugging
}

\providecommand{\cdppca}{
\toolsection{cdppc32} is a compiler for intermediate code targeting the PowerPC hardware architecture.
It generates machine code for PowerPC processors from programs written in intermediate code and stores it in corresponding object files.
The compiler generates machine code for the 32-bit operating mode defined by the PowerPC architecture.
It also creates a debugging information file as well as an assembly file containing a listing of the generated machine code.
\debuggingtool
\flowgraph{\resource{intermediate\\code} \ar[r] & \toolbox{cdppc32} \ar[r] \ar[d] \ar[rd] & \resource{object file} \\ & \resource{assembly\\listing} & \resource{debugging\\information}}
\seeassembly\seeppc\seeobject\seecode\seedebugging
}

\providecommand{\cdppcb}{
\toolsection{cdppc64} is a compiler for intermediate code targeting the PowerPC hardware architecture.
It generates machine code for PowerPC processors from programs written in intermediate code and stores it in corresponding object files.
The compiler generates machine code for the 64-bit operating mode defined by the PowerPC architecture.
It also creates a debugging information file as well as an assembly file containing a listing of the generated machine code.
\debuggingtool
\flowgraph{\resource{intermediate\\code} \ar[r] & \toolbox{cdppc64} \ar[r] \ar[d] \ar[rd] & \resource{object file} \\ & \resource{assembly\\listing} & \resource{debugging\\information}}
\seeassembly\seeppc\seeobject\seecode\seedebugging
}

\providecommand{\cdrisc}{
\toolsection{cdrisc} is a compiler for intermediate code targeting the RISC hardware architecture.
It generates machine code for RISC processors from programs written in intermediate code and stores it in corresponding object files.
It also creates a debugging information file as well as an assembly file containing a listing of the generated machine code.
\debuggingtool
\flowgraph{\resource{intermediate\\code} \ar[r] & \toolbox{cdrisc} \ar[r] \ar[d] \ar[rd] & \resource{object file} \\ & \resource{assembly\\listing} & \resource{debugging\\information}}
\seeassembly\seerisc\seeobject\seecode\seedebugging
}

\providecommand{\cdwasm}{
\toolsection{cdwasm} is a compiler for intermediate code targeting the WebAssembly architecture.
It generates machine code for WebAssembly targets from programs written in intermediate code and stores it in corresponding object files.
It also creates a debugging information file as well as an assembly file containing a listing of the generated machine code.
\debuggingtool
\flowgraph{\resource{intermediate\\code} \ar[r] & \toolbox{cdwasm} \ar[r] \ar[d] \ar[rd] & \resource{object file} \\ & \resource{assembly\\listing} & \resource{debugging\\information}}
\seeassembly\seewasm\seeobject\seecode\seedebugging
}

% C++ tools

\providecommand{\cppprep}{
\toolsection{cppprep} is a preprocessor for the \cpp{} programming language.
It preprocesses source code according to the rules of \cpp{} and writes it to the standard output stream.
Only the macro names \texttt{\_\_DATE\_\_}, \texttt{\_\_FILE\_\_}, \texttt{\_\_LINE\_\_}, and \texttt{\_\_TIME\_\_} are predefined.
\flowgraph{\resource{\cpp{} or other\\source code} \ar[r] & \toolbox{cppprep} \ar[r] & \resource{preprocessed\\source code} \\ & \variable{ECSINCLUDE} \ar[u]}
\seecpp
}

\providecommand{\cppprint}{
\toolsection{cppprint} is a pretty printer for the \cpp{} programming language.
It reformats the source code of \cpp{} programs and writes it to the standard output stream.
\flowgraph{\resource{\cpp{}\\source code} \ar[r] & \toolbox{cppprint} \ar[r] & \resource{reformatted\\source code} \\ & \variable{ECSINCLUDE} \ar[u]}
\seecpp
}

\providecommand{\cppcheck}{
\toolsection{cppcheck} is a syntactic and semantic checker for the \cpp{} programming language.
It just performs syntactic and semantic checks on \cpp{} programs and writes its diagnostic messages to the standard error stream.
\flowgraph{\resource{\cpp{}\\source code} \ar[r] & \toolbox{cppcheck} \ar[r] & \resource{diagnostic\\messages} \\ & \variable{ECSINCLUDE} \ar[u]}
\seecpp
}

\providecommand{\cppdump}{
\toolsection{cppdump} is a serializer for the \cpp{} programming language.
It dumps the complete internal representation of programs written in \cpp{} into an XML document.
\debuggingtool
\flowgraph{\resource{\cpp{}\\source code} \ar[r] & \toolbox{cppdump} \ar[r] & \resource{internal\\representation} \\ & \variable{ECSINCLUDE} \ar[u]}
\seecpp
}

\providecommand{\cpprun}{
\toolsection{cpprun} is an interpreter for the \cpp{} programming language.
It processes and executes programs written in \cpp{}.
The macro \texttt{\_\_run\_\_} is predefined in order to enable programmers to identify this tool while interpreting.
\flowgraph{\resource{\cpp{}\\source code} \ar[r] & \toolbox{cpprun} \ar@/u/[r] & \resource{input/\\output} \ar@/d/[l] \\ & \variable{ECSINCLUDE} \ar[u]}
\seecpp
}

\providecommand{\cppdoc}{
\toolsection{cppdoc} is a generic documentation generator for the \cpp{} programming language.
It processes several \cpp{} source files and assembles all information therein into a generic documentation.
\debuggingtool
\flowgraph{\resource{\cpp{}\\source code} \ar[r] & \toolbox{cppdoc} \ar[r] & \resource{generic\\documentation} \\ & \variable{ECSINCLUDE} \ar[u]}
\seecpp\seedocumentation
}

\providecommand{\cpphtml}{
\toolsection{cpphtml} is an HTML documentation generator for the \cpp{} programming language.
It processes several \cpp{} source files and assembles all information therein into an HTML document.
\flowgraph{\resource{\cpp{}\\source code} \ar[r] & \toolbox{cpphtml} \ar[r] & \resource{HTML\\document} \\ & \variable{ECSINCLUDE} \ar[u]}
\seecpp\seedocumentation
}

\providecommand{\cpplatex}{
\toolsection{cpplatex} is a Latex documentation generator for the \cpp{} programming language.
It processes several \cpp{} source files and assembles all information therein into a Latex document.
\flowgraph{\resource{\cpp{}\\source code} \ar[r] & \toolbox{cpplatex} \ar[r] & \resource{Latex\\document} \\ & \variable{ECSINCLUDE} \ar[u]}
\seecpp\seedocumentation
}

\providecommand{\cppcode}{
\toolsection{cppcode} is an intermediate code generator for the \cpp{} programming language.
It generates intermediate code from programs written in \cpp{} and stores it in corresponding assembly files.
The macro \texttt{\_\_code\_\_} is predefined in order to enable programmers to identify this tool while generating intermediate code.
Programs generated with this tool require additional runtime support that is stored in the \file{cpp\-code\-run} library file.
\debuggingtool
\flowgraph{\resource{\cpp{}\\source code} \ar[r] & \toolbox{cppcode} \ar[r] & \resource{intermediate\\code} \\ & \variable{ECSINCLUDE} \ar[u]}
\seecpp\seeassembly\seecode
}

\providecommand{\cppamda}{
\toolsection{cppamd16} is a compiler for the \cpp{} programming language targeting the AMD64 hardware architecture.
It generates machine code for AMD64 processors from programs written in \cpp{} and stores it in corresponding object files.
The compiler generates machine code for the 16-bit operating mode defined by the AMD64 architecture.
For debugging purposes, it also creates a debugging information file as well as an assembly file containing a listing of the generated machine code.
The macro \texttt{\_\_amd16\_\_} is predefined in order to enable programmers to identify this tool and its target architecture while compiling.
Programs generated with this compiler require additional runtime support that is stored in the \file{cpp\-amd16\-run} library file.
\flowgraph{\resource{\cpp{}\\source code} \ar[r] & \toolbox{cppamd16} \ar[r] \ar[d] \ar[rd] & \resource{object file} \\ \variable{ECSINCLUDE} \ar[ru] & \resource{debugging\\information} & \resource{assembly\\listing}}
\seecpp\seeassembly\seeamd\seeobject\seedebugging
}

\providecommand{\cppamdb}{
\toolsection{cppamd32} is a compiler for the \cpp{} programming language targeting the AMD64 hardware architecture.
It generates machine code for AMD64 processors from programs written in \cpp{} and stores it in corresponding object files.
The compiler generates machine code for the 32-bit operating mode defined by the AMD64 architecture.
For debugging purposes, it also creates a debugging information file as well as an assembly file containing a listing of the generated machine code.
The macro \texttt{\_\_amd32\_\_} is predefined in order to enable programmers to identify this tool and its target architecture while compiling.
Programs generated with this compiler require additional runtime support that is stored in the \file{cpp\-amd32\-run} library file.
\flowgraph{\resource{\cpp{}\\source code} \ar[r] & \toolbox{cppamd32} \ar[r] \ar[d] \ar[rd] & \resource{object file} \\ \variable{ECSINCLUDE} \ar[ru] & \resource{debugging\\information} & \resource{assembly\\listing}}
\seecpp\seeassembly\seeamd\seeobject\seedebugging
}

\providecommand{\cppamdc}{
\toolsection{cppamd64} is a compiler for the \cpp{} programming language targeting the AMD64 hardware architecture.
It generates machine code for AMD64 processors from programs written in \cpp{} and stores it in corresponding object files.
The compiler generates machine code for the 64-bit operating mode defined by the AMD64 architecture.
For debugging purposes, it also creates a debugging information file as well as an assembly file containing a listing of the generated machine code.
The macro \texttt{\_\_amd64\_\_} is predefined in order to enable programmers to identify this tool and its target architecture while compiling.
Programs generated with this compiler require additional runtime support that is stored in the \file{cpp\-amd64\-run} library file.
\flowgraph{\resource{\cpp{}\\source code} \ar[r] & \toolbox{cppamd64} \ar[r] \ar[d] \ar[rd] & \resource{object file} \\ \variable{ECSINCLUDE} \ar[ru] & \resource{debugging\\information} & \resource{assembly\\listing}}
\seecpp\seeassembly\seeamd\seeobject\seedebugging
}

\providecommand{\cpparma}{
\toolsection{cpparma32} is a compiler for the \cpp{} programming language targeting the ARM hardware architecture.
It generates machine code for ARM processors executing A32 instructions from programs written in \cpp{} and stores it in corresponding object files.
For debugging purposes, it also creates a debugging information file as well as an assembly file containing a listing of the generated machine code.
The macro \texttt{\_\_arma32\_\_} is predefined in order to enable programmers to identify this tool and its target architecture while compiling.
Programs generated with this compiler require additional runtime support that is stored in the \file{cpp\-arma32\-run} library file.
\flowgraph{\resource{\cpp{}\\source code} \ar[r] & \toolbox{cpparma32} \ar[r] \ar[d] \ar[rd] & \resource{object file} \\ \variable{ECSINCLUDE} \ar[ru] & \resource{debugging\\information} & \resource{assembly\\listing}}
\seecpp\seeassembly\seearm\seeobject\seedebugging
}

\providecommand{\cpparmb}{
\toolsection{cpparma64} is a compiler for the \cpp{} programming language targeting the ARM hardware architecture.
It generates machine code for ARM processors executing A64 instructions from programs written in \cpp{} and stores it in corresponding object files.
For debugging purposes, it also creates a debugging information file as well as an assembly file containing a listing of the generated machine code.
The macro \texttt{\_\_arma64\_\_} is predefined in order to enable programmers to identify this tool and its target architecture while compiling.
Programs generated with this compiler require additional runtime support that is stored in the \file{cpp\-arma64\-run} library file.
\flowgraph{\resource{\cpp{}\\source code} \ar[r] & \toolbox{cpparma64} \ar[r] \ar[d] \ar[rd] & \resource{object file} \\ \variable{ECSINCLUDE} \ar[ru] & \resource{debugging\\information} & \resource{assembly\\listing}}
\seecpp\seeassembly\seearm\seeobject\seedebugging
}

\providecommand{\cpparmc}{
\toolsection{cpparmt32} is a compiler for the \cpp{} programming language targeting the ARM hardware architecture.
It generates machine code for ARM processors without floating-point extension executing T32 instructions from programs written in \cpp{} and stores it in corresponding object files.
For debugging purposes, it also creates a debugging information file as well as an assembly file containing a listing of the generated machine code.
The macro \texttt{\_\_armt32\_\_} is predefined in order to enable programmers to identify this tool and its target architecture while compiling.
Programs generated with this compiler require additional runtime support that is stored in the \file{cpp\-armt32\-run} library file.
\flowgraph{\resource{\cpp{}\\source code} \ar[r] & \toolbox{cpparmt32} \ar[r] \ar[d] \ar[rd] & \resource{object file} \\ \variable{ECSINCLUDE} \ar[ru] & \resource{debugging\\information} & \resource{assembly\\listing}}
\seecpp\seeassembly\seearm\seeobject\seedebugging
}

\providecommand{\cpparmcfpe}{
\toolsection{cpparmt32fpe} is a compiler for the \cpp{} programming language targeting the ARM hardware architecture.
It generates machine code for ARM processors with floating-point extension executing T32 instructions from programs written in \cpp{} and stores it in corresponding object files.
For debugging purposes, it also creates a debugging information file as well as an assembly file containing a listing of the generated machine code.
The macro \texttt{\_\_armt32fpe\_\_} is predefined in order to enable programmers to identify this tool and its target architecture while compiling.
Programs generated with this compiler require additional runtime support that is stored in the \file{cpp\-armt32\-fpe\-run} library file.
\flowgraph{\resource{\cpp{}\\source code} \ar[r] & \toolbox{cpparmt32fpe} \ar[r] \ar[d] \ar[rd] & \resource{object file} \\ \variable{ECSINCLUDE} \ar[ru] & \resource{debugging\\information} & \resource{assembly\\listing}}
\seecpp\seeassembly\seearm\seeobject\seedebugging
}

\providecommand{\cppavr}{
\toolsection{cppavr} is a compiler for the \cpp{} programming language targeting the AVR hardware architecture.
It generates machine code for AVR processors from programs written in \cpp{} and stores it in corresponding object files.
For debugging purposes, it also creates a debugging information file as well as an assembly file containing a listing of the generated machine code.
The macro \texttt{\_\_avr\_\_} is predefined in order to enable programmers to identify this tool and its target architecture while compiling.
Programs generated with this compiler require additional runtime support that is stored in the \file{cpp\-avr\-run} library file.
\flowgraph{\resource{\cpp{}\\source code} \ar[r] & \toolbox{cppavr} \ar[r] \ar[d] \ar[rd] & \resource{object file} \\ \variable{ECSINCLUDE} \ar[ru] & \resource{debugging\\information} & \resource{assembly\\listing}}
\seecpp\seeassembly\seeavr\seeobject\seedebugging
}

\providecommand{\cppavrtt}{
\toolsection{cppavr32} is a compiler for the \cpp{} programming language targeting the AVR32 hardware architecture.
It generates machine code for AVR32 processors from programs written in \cpp{} and stores it in corresponding object files.
For debugging purposes, it also creates a debugging information file as well as an assembly file containing a listing of the generated machine code.
The macro \texttt{\_\_avr32\_\_} is predefined in order to enable programmers to identify this tool and its target architecture while compiling.
Programs generated with this compiler require additional runtime support that is stored in the \file{cpp\-avr32\-run} library file.
\flowgraph{\resource{\cpp{}\\source code} \ar[r] & \toolbox{cppavr32} \ar[r] \ar[d] \ar[rd] & \resource{object file} \\ \variable{ECSINCLUDE} \ar[ru] & \resource{debugging\\information} & \resource{assembly\\listing}}
\seecpp\seeassembly\seeavrtt\seeobject\seedebugging
}

\providecommand{\cppmabk}{
\toolsection{cppm68k} is a compiler for the \cpp{} programming language targeting the M68000 hardware architecture.
It generates machine code for M68000 processors from programs written in \cpp{} and stores it in corresponding object files.
For debugging purposes, it also creates a debugging information file as well as an assembly file containing a listing of the generated machine code.
The macro \texttt{\_\_m68k\_\_} is predefined in order to enable programmers to identify this tool and its target architecture while compiling.
Programs generated with this compiler require additional runtime support that is stored in the \file{cpp\-m68k\-run} library file.
\flowgraph{\resource{\cpp{}\\source code} \ar[r] & \toolbox{cppm68k} \ar[r] \ar[d] \ar[rd] & \resource{object file} \\ \variable{ECSINCLUDE} \ar[ru] & \resource{debugging\\information} & \resource{assembly\\listing}}
\seecpp\seeassembly\seemabk\seeobject\seedebugging
}

\providecommand{\cppmibl}{
\toolsection{cppmibl} is a compiler for the \cpp{} programming language targeting the MicroBlaze hardware architecture.
It generates machine code for MicroBlaze processors from programs written in \cpp{} and stores it in corresponding object files.
For debugging purposes, it also creates a debugging information file as well as an assembly file containing a listing of the generated machine code.
The macro \texttt{\_\_mibl\_\_} is predefined in order to enable programmers to identify this tool and its target architecture while compiling.
Programs generated with this compiler require additional runtime support that is stored in the \file{cpp\-mibl\-run} library file.
\flowgraph{\resource{\cpp{}\\source code} \ar[r] & \toolbox{cppmibl} \ar[r] \ar[d] \ar[rd] & \resource{object file} \\ \variable{ECSINCLUDE} \ar[ru] & \resource{debugging\\information} & \resource{assembly\\listing}}
\seecpp\seeassembly\seemibl\seeobject\seedebugging
}

\providecommand{\cppmipsa}{
\toolsection{cppmips32} is a compiler for the \cpp{} programming language targeting the MIPS32 hardware architecture.
It generates machine code for MIPS32 processors from programs written in \cpp{} and stores it in corresponding object files.
For debugging purposes, it also creates a debugging information file as well as an assembly file containing a listing of the generated machine code.
The macro \texttt{\_\_mips32\_\_} is predefined in order to enable programmers to identify this tool and its target architecture while compiling.
Programs generated with this compiler require additional runtime support that is stored in the \file{cpp\-mips32\-run} library file.
\flowgraph{\resource{\cpp{}\\source code} \ar[r] & \toolbox{cppmips32} \ar[r] \ar[d] \ar[rd] & \resource{object file} \\ \variable{ECSINCLUDE} \ar[ru] & \resource{debugging\\information} & \resource{assembly\\listing}}
\seecpp\seeassembly\seemips\seeobject\seedebugging
}

\providecommand{\cppmipsb}{
\toolsection{cppmips64} is a compiler for the \cpp{} programming language targeting the MIPS64 hardware architecture.
It generates machine code for MIPS64 processors from programs written in \cpp{} and stores it in corresponding object files.
For debugging purposes, it also creates a debugging information file as well as an assembly file containing a listing of the generated machine code.
The macro \texttt{\_\_mips64\_\_} is predefined in order to enable programmers to identify this tool and its target architecture while compiling.
Programs generated with this compiler require additional runtime support that is stored in the \file{cpp\-mips64\-run} library file.
\flowgraph{\resource{\cpp{}\\source code} \ar[r] & \toolbox{cppmips64} \ar[r] \ar[d] \ar[rd] & \resource{object file} \\ \variable{ECSINCLUDE} \ar[ru] & \resource{debugging\\information} & \resource{assembly\\listing}}
\seecpp\seeassembly\seemips\seeobject\seedebugging
}

\providecommand{\cppmmix}{
\toolsection{cppmmix} is a compiler for the \cpp{} programming language targeting the MMIX hardware architecture.
It generates machine code for MMIX processors from programs written in \cpp{} and stores it in corresponding object files.
For debugging purposes, it also creates a debugging information file as well as an assembly file containing a listing of the generated machine code.
The macro \texttt{\_\_mmix\_\_} is predefined in order to enable programmers to identify this tool and its target architecture while compiling.
Programs generated with this compiler require additional runtime support that is stored in the \file{cpp\-mmix\-run} library file.
\flowgraph{\resource{\cpp{}\\source code} \ar[r] & \toolbox{cppmmix} \ar[r] \ar[d] \ar[rd] & \resource{object file} \\ \variable{ECSINCLUDE} \ar[ru] & \resource{debugging\\information} & \resource{assembly\\listing}}
\seecpp\seeassembly\seemmix\seeobject\seedebugging
}

\providecommand{\cpporok}{
\toolsection{cppor1k} is a compiler for the \cpp{} programming language targeting the OpenRISC 1000 hardware architecture.
It generates machine code for OpenRISC 1000 processors from programs written in \cpp{} and stores it in corresponding object files.
For debugging purposes, it also creates a debugging information file as well as an assembly file containing a listing of the generated machine code.
The macro \texttt{\_\_or1k\_\_} is predefined in order to enable programmers to identify this tool and its target architecture while compiling.
Programs generated with this compiler require additional runtime support that is stored in the \file{cpp\-or1k\-run} library file.
\flowgraph{\resource{\cpp{}\\source code} \ar[r] & \toolbox{cppor1k} \ar[r] \ar[d] \ar[rd] & \resource{object file} \\ \variable{ECSINCLUDE} \ar[ru] & \resource{debugging\\information} & \resource{assembly\\listing}}
\seecpp\seeassembly\seeorok\seeobject\seedebugging
}

\providecommand{\cppppca}{
\toolsection{cppppc32} is a compiler for the \cpp{} programming language targeting the PowerPC hardware architecture.
It generates machine code for PowerPC processors from programs written in \cpp{} and stores it in corresponding object files.
The compiler generates machine code for the 32-bit operating mode defined by the PowerPC architecture.
For debugging purposes, it also creates a debugging information file as well as an assembly file containing a listing of the generated machine code.
The macro \texttt{\_\_ppc32\_\_} is predefined in order to enable programmers to identify this tool and its target architecture while compiling.
Programs generated with this compiler require additional runtime support that is stored in the \file{cpp\-ppc32\-run} library file.
\flowgraph{\resource{\cpp{}\\source code} \ar[r] & \toolbox{cppppc32} \ar[r] \ar[d] \ar[rd] & \resource{object file} \\ \variable{ECSINCLUDE} \ar[ru] & \resource{debugging\\information} & \resource{assembly\\listing}}
\seecpp\seeassembly\seeppc\seeobject\seedebugging
}

\providecommand{\cppppcb}{
\toolsection{cppppc64} is a compiler for the \cpp{} programming language targeting the PowerPC hardware architecture.
It generates machine code for PowerPC processors from programs written in \cpp{} and stores it in corresponding object files.
The compiler generates machine code for the 64-bit operating mode defined by the PowerPC architecture.
For debugging purposes, it also creates a debugging information file as well as an assembly file containing a listing of the generated machine code.
The macro \texttt{\_\_ppc64\_\_} is predefined in order to enable programmers to identify this tool and its target architecture while compiling.
Programs generated with this compiler require additional runtime support that is stored in the \file{cpp\-ppc64\-run} library file.
\flowgraph{\resource{\cpp{}\\source code} \ar[r] & \toolbox{cppppc64} \ar[r] \ar[d] \ar[rd] & \resource{object file} \\ \variable{ECSINCLUDE} \ar[ru] & \resource{debugging\\information} & \resource{assembly\\listing}}
\seecpp\seeassembly\seeppc\seeobject\seedebugging
}

\providecommand{\cpprisc}{
\toolsection{cpprisc} is a compiler for the \cpp{} programming language targeting the RISC hardware architecture.
It generates machine code for RISC processors from programs written in \cpp{} and stores it in corresponding object files.
For debugging purposes, it also creates a debugging information file as well as an assembly file containing a listing of the generated machine code.
The macro \texttt{\_\_risc\_\_} is predefined in order to enable programmers to identify this tool and its target architecture while compiling.
Programs generated with this compiler require additional runtime support that is stored in the \file{cpp\-risc\-run} library file.
\flowgraph{\resource{\cpp{}\\source code} \ar[r] & \toolbox{cpprisc} \ar[r] \ar[d] \ar[rd] & \resource{object file} \\ \variable{ECSINCLUDE} \ar[ru] & \resource{debugging\\information} & \resource{assembly\\listing}}
\seecpp\seeassembly\seerisc\seeobject\seedebugging
}

\providecommand{\cppwasm}{
\toolsection{cppwasm} is a compiler for the \cpp{} programming language targeting the WebAssembly architecture.
It generates machine code for WebAssembly targets from programs written in \cpp{} and stores it in corresponding object files.
For debugging purposes, it also creates a debugging information file as well as an assembly file containing a listing of the generated machine code.
The macro \texttt{\_\_wasm\_\_} is predefined in order to enable programmers to identify this tool and its target architecture while compiling.
Programs generated with this compiler require additional runtime support that is stored in the \file{cpp\-wasm\-run} library file.
\flowgraph{\resource{\cpp{}\\source code} \ar[r] & \toolbox{cppwasm} \ar[r] \ar[d] \ar[rd] & \resource{object file} \\ \variable{ECSINCLUDE} \ar[ru] & \resource{debugging\\information} & \resource{assembly\\listing}}
\seecpp\seeassembly\seewasm\seeobject\seedebugging
}

% FALSE tools

\providecommand{\falprint}{
\toolsection{falprint} is a pretty printer for the FALSE programming language.
It reformats the source code of FALSE programs and writes it to the standard output stream.
\flowgraph{\resource{FALSE\\source code} \ar[r] & \toolbox{falprint} \ar[r] & \resource{reformatted\\source code}}
\seefalse
}

\providecommand{\falcheck}{
\toolsection{falcheck} is a syntactic and semantic checker for the FALSE programming language.
It just performs syntactic and semantic checks on FALSE programs and writes its diagnostic messages to the standard error stream.
\flowgraph{\resource{FALSE\\source code} \ar[r] & \toolbox{falcheck} \ar[r] & \resource{diagnostic\\messages}}
\seefalse
}

\providecommand{\faldump}{
\toolsection{faldump} is a serializer for the FALSE programming language.
It dumps the complete internal representation of programs written in FALSE into an XML document.
\debuggingtool
\flowgraph{\resource{FALSE\\source code} \ar[r] & \toolbox{faldump} \ar[r] & \resource{internal\\representation}}
\seefalse
}

\providecommand{\falrun}{
\toolsection{falrun} is an interpreter for the FALSE programming language.
It processes and executes programs written in FALSE\@.
\flowgraph{\resource{FALSE\\source code} \ar[r] & \toolbox{falrun} \ar@/u/[r] & \resource{input/\\output} \ar@/d/[l]}
\seefalse
}

\providecommand{\falcpp}{
\toolsection{falcpp} is a transpiler for the FALSE programming language.
It translates programs written in FALSE into \cpp{} programs and stores them in corresponding source files.
\flowgraph{\resource{FALSE\\source code} \ar[r] & \toolbox{falcpp} \ar[r] & \resource{\cpp{}\\source file}}
\seefalse\seecpp
}

\providecommand{\falcode}{
\toolsection{falcode} is an intermediate code generator for the FALSE programming language.
It generates intermediate code from programs written in FALSE and stores it in corresponding assembly files.
\debuggingtool
\flowgraph{\resource{FALSE\\source code} \ar[r] & \toolbox{falcode} \ar[r] & \resource{intermediate\\code}}
\seefalse\seeassembly\seecode
}

\providecommand{\falamda}{
\toolsection{falamd16} is a compiler for the FALSE programming language targeting the AMD64 hardware architecture.
It generates machine code for AMD64 processors from programs written in FALSE and stores it in corresponding object files.
The compiler generates machine code for the 16-bit operating mode defined by the AMD64 architecture.
\flowgraph{\resource{FALSE\\source code} \ar[r] & \toolbox{falamd16} \ar[r] & \resource{object file}}
\seefalse\seeamd\seeobject
}

\providecommand{\falamdb}{
\toolsection{falamd32} is a compiler for the FALSE programming language targeting the AMD64 hardware architecture.
It generates machine code for AMD64 processors from programs written in FALSE and stores it in corresponding object files.
The compiler generates machine code for the 32-bit operating mode defined by the AMD64 architecture.
\flowgraph{\resource{FALSE\\source code} \ar[r] & \toolbox{falamd32} \ar[r] & \resource{object file}}
\seefalse\seeamd\seeobject
}

\providecommand{\falamdc}{
\toolsection{falamd64} is a compiler for the FALSE programming language targeting the AMD64 hardware architecture.
It generates machine code for AMD64 processors from programs written in FALSE and stores it in corresponding object files.
The compiler generates machine code for the 64-bit operating mode defined by the AMD64 architecture.
\flowgraph{\resource{FALSE\\source code} \ar[r] & \toolbox{falamd64} \ar[r] & \resource{object file}}
\seefalse\seeamd\seeobject
}

\providecommand{\falarma}{
\toolsection{falarma32} is a compiler for the FALSE programming language targeting the ARM hardware architecture.
It generates machine code for ARM processors executing A32 instructions from programs written in FALSE and stores it in corresponding object files.
\flowgraph{\resource{FALSE\\source code} \ar[r] & \toolbox{falarma32} \ar[r] & \resource{object file}}
\seefalse\seearm\seeobject
}

\providecommand{\falarmb}{
\toolsection{falarma64} is a compiler for the FALSE programming language targeting the ARM hardware architecture.
It generates machine code for ARM processors executing A64 instructions from programs written in FALSE and stores it in corresponding object files.
\flowgraph{\resource{FALSE\\source code} \ar[r] & \toolbox{falarma64} \ar[r] & \resource{object file}}
\seefalse\seearm\seeobject
}

\providecommand{\falarmc}{
\toolsection{falarmt32} is a compiler for the FALSE programming language targeting the ARM hardware architecture.
It generates machine code for ARM processors without floating-point extension executing T32 instructions from programs written in FALSE and stores it in corresponding object files.
\flowgraph{\resource{FALSE\\source code} \ar[r] & \toolbox{falarmt32} \ar[r] & \resource{object file}}
\seefalse\seearm\seeobject
}

\providecommand{\falarmcfpe}{
\toolsection{falarmt32fpe} is a compiler for the FALSE programming language targeting the ARM hardware architecture.
It generates machine code for ARM processors with floating-point extension executing T32 instructions from programs written in FALSE and stores it in corresponding object files.
\flowgraph{\resource{FALSE\\source code} \ar[r] & \toolbox{falarmt32fpe} \ar[r] & \resource{object file}}
\seefalse\seearm\seeobject
}

\providecommand{\falavr}{
\toolsection{falavr} is a compiler for the FALSE programming language targeting the AVR hardware architecture.
It generates machine code for AVR processors from programs written in FALSE and stores it in corresponding object files.
\flowgraph{\resource{FALSE\\source code} \ar[r] & \toolbox{falavr} \ar[r] & \resource{object file}}
\seefalse\seeavr\seeobject
}

\providecommand{\falavrtt}{
\toolsection{falavr32} is a compiler for the FALSE programming language targeting the AVR32 hardware architecture.
It generates machine code for AVR32 processors from programs written in FALSE and stores it in corresponding object files.
\flowgraph{\resource{FALSE\\source code} \ar[r] & \toolbox{falavr32} \ar[r] & \resource{object file}}
\seefalse\seeavrtt\seeobject
}

\providecommand{\falmabk}{
\toolsection{falm68k} is a compiler for the FALSE programming language targeting the M68000 hardware architecture.
It generates machine code for M68000 processors from programs written in FALSE and stores it in corresponding object files.
\flowgraph{\resource{FALSE\\source code} \ar[r] & \toolbox{falm68k} \ar[r] & \resource{object file}}
\seefalse\seemabk\seeobject
}

\providecommand{\falmibl}{
\toolsection{falmibl} is a compiler for the FALSE programming language targeting the MicroBlaze hardware architecture.
It generates machine code for MicroBlaze processors from programs written in FALSE and stores it in corresponding object files.
\flowgraph{\resource{FALSE\\source code} \ar[r] & \toolbox{falmibl} \ar[r] & \resource{object file}}
\seefalse\seemibl\seeobject
}

\providecommand{\falmipsa}{
\toolsection{falmips32} is a compiler for the FALSE programming language targeting the MIPS32 hardware architecture.
It generates machine code for MIPS32 processors from programs written in FALSE and stores it in corresponding object files.
\flowgraph{\resource{FALSE\\source code} \ar[r] & \toolbox{falmips32} \ar[r] & \resource{object file}}
\seefalse\seemips\seeobject
}

\providecommand{\falmipsb}{
\toolsection{falmips64} is a compiler for the FALSE programming language targeting the MIPS64 hardware architecture.
It generates machine code for MIPS64 processors from programs written in FALSE and stores it in corresponding object files.
\flowgraph{\resource{FALSE\\source code} \ar[r] & \toolbox{falmips64} \ar[r] & \resource{object file}}
\seefalse\seemips\seeobject
}

\providecommand{\falmmix}{
\toolsection{falmmix} is a compiler for the FALSE programming language targeting the MMIX hardware architecture.
It generates machine code for MMIX processors from programs written in FALSE and stores it in corresponding object files.
\flowgraph{\resource{FALSE\\source code} \ar[r] & \toolbox{falmmix} \ar[r] & \resource{object file}}
\seefalse\seemmix\seeobject
}

\providecommand{\falorok}{
\toolsection{falor1k} is a compiler for the FALSE programming language targeting the OpenRISC 1000 hardware architecture.
It generates machine code for OpenRISC 1000 processors from programs written in FALSE and stores it in corresponding object files.
\flowgraph{\resource{FALSE\\source code} \ar[r] & \toolbox{falor1k} \ar[r] & \resource{object file}}
\seefalse\seeorok\seeobject
}

\providecommand{\falppca}{
\toolsection{falppc32} is a compiler for the FALSE programming language targeting the PowerPC hardware architecture.
It generates machine code for PowerPC processors from programs written in FALSE and stores it in corresponding object files.
The compiler generates machine code for the 32-bit operating mode defined by the PowerPC architecture.
\flowgraph{\resource{FALSE\\source code} \ar[r] & \toolbox{falppc32} \ar[r] & \resource{object file}}
\seefalse\seeppc\seeobject
}

\providecommand{\falppcb}{
\toolsection{falppc64} is a compiler for the FALSE programming language targeting the PowerPC hardware architecture.
It generates machine code for PowerPC processors from programs written in FALSE and stores it in corresponding object files.
The compiler generates machine code for the 64-bit operating mode defined by the PowerPC architecture.
\flowgraph{\resource{FALSE\\source code} \ar[r] & \toolbox{falppc64} \ar[r] & \resource{object file}}
\seefalse\seeppc\seeobject
}

\providecommand{\falrisc}{
\toolsection{falrisc} is a compiler for the FALSE programming language targeting the RISC hardware architecture.
It generates machine code for RISC processors from programs written in FALSE and stores it in corresponding object files.
\flowgraph{\resource{FALSE\\source code} \ar[r] & \toolbox{falrisc} \ar[r] & \resource{object file}}
\seefalse\seerisc\seeobject
}

\providecommand{\falwasm}{
\toolsection{falwasm} is a compiler for the FALSE programming language targeting the WebAssembly architecture.
It generates machine code for WebAssembly targets from programs written in FALSE and stores it in corresponding object files.
\flowgraph{\resource{FALSE\\source code} \ar[r] & \toolbox{falwasm} \ar[r] & \resource{object file}}
\seefalse\seewasm\seeobject
}

% Oberon tools

\providecommand{\obprint}{
\toolsection{obprint} is a pretty printer for the Oberon programming language.
It reformats the source code of Oberon modules and writes it to the standard output stream.
\flowgraph{\resource{Oberon\\source code} \ar[r] & \toolbox{obprint} \ar[r] & \resource{reformatted\\source code}}
\seeoberon
}

\providecommand{\obcheck}{
\toolsection{obcheck} is a syntactic and semantic checker for the Oberon programming language.
It just performs syntactic and semantic checks on Oberon modules and writes its diagnostic messages to the standard error stream.
In addition, it stores the interface of each module in a symbol file which is required when other modules import the module.
\flowgraph{\resource{Oberon\\source code} \ar[r] & \toolbox{obcheck} \ar[r] \ar@/l/[d] & \resource{diagnostic\\messages} \\ \variable{ECSIMPORT} \ar[ru] & \resource{symbol\\files} \ar@/r/[u]}
\seeoberon
}

\providecommand{\obdump}{
\toolsection{obdump} is a serializer for the Oberon programming language.
It dumps the complete internal representation of modules written in Oberon into an XML document.
\debuggingtool
\flowgraph{\resource{Oberon\\source code} \ar[r] & \toolbox{obdump} \ar[r] \ar@/l/[d] & \resource{internal\\representation} \\ \variable{ECSIMPORT} \ar[ru] & \resource{symbol\\files} \ar@/r/[u]}
\seeoberon
}

\providecommand{\obrun}{
\toolsection{obrun} is an interpreter for the Oberon programming language.
It processes and executes modules written in Oberon.
This tool does neither generate nor process symbol files while interpreting modules.
If a module is imported by another one, its filename has to be named before the other one in the list of command-line arguments.
\flowgraph{\resource{Oberon\\source code} \ar[r] & \toolbox{obrun} \ar@/u/[r] & \resource{input/\\output} \ar@/d/[l]}
\seeoberon
}

\providecommand{\obcpp}{
\toolsection{obcpp} is a transpiler for the Oberon programming language.
It translates programs written in Oberon into \cpp{} programs and stores them in corresponding source and header files.
In addition, it stores the interface of each module in a symbol file which is required when other modules import the module.
The same interface is provided by the generated header file which can be used in other parts of the \cpp{} program.
\flowgraph{\resource{Oberon\\source code} \ar[r] & \toolbox{obcpp} \ar[r] \ar@/l/[d] \ar[rd] & \resource{\cpp{}\\source file} \\ \variable{ECSIMPORT} \ar[ru] & \resource{symbol\\files} \ar@/r/[u] & \resource{\cpp{}\\header file}}
\seeoberon\seecpp
}

\providecommand{\obdoc}{
\toolsection{obdoc} is a generic documentation generator for the Oberon programming language.
It processes several Oberon modules and assembles all information therein into a generic documentation.
In addition, it stores the interface of each module in a symbol file which is required when other modules import the module.
\debuggingtool
\flowgraph{\resource{Oberon\\source code} \ar[r] & \toolbox{obdoc} \ar[r] \ar@/l/[d] & \resource{generic\\documentation} \\ \variable{ECSIMPORT} \ar[ru] & \resource{symbol\\files} \ar@/r/[u]}
\seeoberon\seedocumentation
}

\providecommand{\obhtml}{
\toolsection{obhtml} is an HTML documentation generator for the Oberon programming language.
It processes several Oberon modules and assembles all information therein into an HTML document.
In addition, it stores the interface of each module in a symbol file which is required when other modules import the module.
\flowgraph{\resource{Oberon\\source code} \ar[r] & \toolbox{obhtml} \ar[r] \ar@/l/[d] & \resource{HTML\\document} \\ \variable{ECSIMPORT} \ar[ru] & \resource{symbol\\files} \ar@/r/[u]}
\seeoberon\seedocumentation
}

\providecommand{\oblatex}{
\toolsection{oblatex} is a Latex documentation generator for the Oberon programming language.
It processes several Oberon modules and assembles all information therein into a Latex document.
In addition, it stores the interface of each module in a symbol file which is required when other modules import the module.
\flowgraph{\resource{Oberon\\source code} \ar[r] & \toolbox{oblatex} \ar[r] \ar@/l/[d] & \resource{Latex\\document} \\ \variable{ECSIMPORT} \ar[ru] & \resource{symbol\\files} \ar@/r/[u]}
\seeoberon\seedocumentation
}

\providecommand{\obcode}{
\toolsection{obcode} is an intermediate code generator for the Oberon programming language.
It generates intermediate code from modules written in Oberon and stores it in corresponding assembly files.
In addition, it stores the interface of each module in a symbol file which is required when other modules import the module.
Programs generated with this tool require additional runtime support that is stored in the \file{ob\-code\-run} library file.
\debuggingtool
\flowgraph{\resource{Oberon\\source code} \ar[r] & \toolbox{obcode} \ar[r] \ar@/l/[d] & \resource{intermediate\\code} \\ \variable{ECSIMPORT} \ar[ru] & \resource{symbol\\files} \ar@/r/[u]}
\seeoberon\seeassembly\seecode
}

\providecommand{\obamda}{
\toolsection{obamd16} is a compiler for the Oberon programming language targeting the AMD64 hardware architecture.
It generates machine code for AMD64 processors from modules written in Oberon and stores it in corresponding object files.
The compiler generates machine code for the 16-bit operating mode defined by the AMD64 architecture.
For debugging purposes, it also creates a debugging information file as well as an assembly file containing a listing of the generated machine code.
In addition, it stores the interface of each module in a symbol file which is required when other modules import the module.
Programs generated with this compiler require additional runtime support that is stored in the \file{ob\-amd16\-run} library file.
\flowgraph{\resource{Oberon\\source code} \ar[r] & \toolbox{obamd16} \ar[r] \ar@/l/[d] \ar[rd] & \resource{object file} \\ \variable{ECSIMPORT} \ar[ru] & \resource{symbol\\files} \ar@/r/[u] & \resource{debugging\\information}}
\seeoberon\seeassembly\seeamd\seeobject\seedebugging
}

\providecommand{\obamdb}{
\toolsection{obamd32} is a compiler for the Oberon programming language targeting the AMD64 hardware architecture.
It generates machine code for AMD64 processors from modules written in Oberon and stores it in corresponding object files.
The compiler generates machine code for the 32-bit operating mode defined by the AMD64 architecture.
For debugging purposes, it also creates a debugging information file as well as an assembly file containing a listing of the generated machine code.
In addition, it stores the interface of each module in a symbol file which is required when other modules import the module.
Programs generated with this compiler require additional runtime support that is stored in the \file{ob\-amd32\-run} library file.
\flowgraph{\resource{Oberon\\source code} \ar[r] & \toolbox{obamd32} \ar[r] \ar@/l/[d] \ar[rd] & \resource{object file} \\ \variable{ECSIMPORT} \ar[ru] & \resource{symbol\\files} \ar@/r/[u] & \resource{debugging\\information}}
\seeoberon\seeassembly\seeamd\seeobject\seedebugging
}

\providecommand{\obamdc}{
\toolsection{obamd64} is a compiler for the Oberon programming language targeting the AMD64 hardware architecture.
It generates machine code for AMD64 processors from modules written in Oberon and stores it in corresponding object files.
The compiler generates machine code for the 64-bit operating mode defined by the AMD64 architecture.
For debugging purposes, it also creates a debugging information file as well as an assembly file containing a listing of the generated machine code.
In addition, it stores the interface of each module in a symbol file which is required when other modules import the module.
Programs generated with this compiler require additional runtime support that is stored in the \file{ob\-amd64\-run} library file.
\flowgraph{\resource{Oberon\\source code} \ar[r] & \toolbox{obamd64} \ar[r] \ar@/l/[d] \ar[rd] & \resource{object file} \\ \variable{ECSIMPORT} \ar[ru] & \resource{symbol\\files} \ar@/r/[u] & \resource{debugging\\information}}
\seeoberon\seeassembly\seeamd\seeobject\seedebugging
}

\providecommand{\obarma}{
\toolsection{obarma32} is a compiler for the Oberon programming language targeting the ARM hardware architecture.
It generates machine code for ARM processors executing A32 instructions from modules written in Oberon and stores it in corresponding object files.
For debugging purposes, it also creates a debugging information file as well as an assembly file containing a listing of the generated machine code.
In addition, it stores the interface of each module in a symbol file which is required when other modules import the module.
Programs generated with this compiler require additional runtime support that is stored in the \file{ob\-arma32\-run} library file.
\flowgraph{\resource{Oberon\\source code} \ar[r] & \toolbox{obarma32} \ar[r] \ar@/l/[d] \ar[rd] & \resource{object file} \\ \variable{ECSIMPORT} \ar[ru] & \resource{symbol\\files} \ar@/r/[u] & \resource{debugging\\information}}
\seeoberon\seeassembly\seearm\seeobject\seedebugging
}

\providecommand{\obarmb}{
\toolsection{obarma64} is a compiler for the Oberon programming language targeting the ARM hardware architecture.
It generates machine code for ARM processors executing A64 instructions from modules written in Oberon and stores it in corresponding object files.
For debugging purposes, it also creates a debugging information file as well as an assembly file containing a listing of the generated machine code.
In addition, it stores the interface of each module in a symbol file which is required when other modules import the module.
Programs generated with this compiler require additional runtime support that is stored in the \file{ob\-arma64\-run} library file.
\flowgraph{\resource{Oberon\\source code} \ar[r] & \toolbox{obarma64} \ar[r] \ar@/l/[d] \ar[rd] & \resource{object file} \\ \variable{ECSIMPORT} \ar[ru] & \resource{symbol\\files} \ar@/r/[u] & \resource{debugging\\information}}
\seeoberon\seeassembly\seearm\seeobject\seedebugging
}

\providecommand{\obarmc}{
\toolsection{obarmt32} is a compiler for the Oberon programming language targeting the ARM hardware architecture.
It generates machine code for ARM processors without floating-point extension executing T32 instructions from modules written in Oberon and stores it in corresponding object files.
For debugging purposes, it also creates a debugging information file as well as an assembly file containing a listing of the generated machine code.
In addition, it stores the interface of each module in a symbol file which is required when other modules import the module.
Programs generated with this compiler require additional runtime support that is stored in the \file{ob\-armt32\-run} library file.
\flowgraph{\resource{Oberon\\source code} \ar[r] & \toolbox{obarmt32} \ar[r] \ar@/l/[d] \ar[rd] & \resource{object file} \\ \variable{ECSIMPORT} \ar[ru] & \resource{symbol\\files} \ar@/r/[u] & \resource{debugging\\information}}
\seeoberon\seeassembly\seearm\seeobject\seedebugging
}

\providecommand{\obarmcfpe}{
\toolsection{obarmt32fpe} is a compiler for the Oberon programming language targeting the ARM hardware architecture.
It generates machine code for ARM processors with floating-point extension executing T32 instructions from modules written in Oberon and stores it in corresponding object files.
For debugging purposes, it also creates a debugging information file as well as an assembly file containing a listing of the generated machine code.
In addition, it stores the interface of each module in a symbol file which is required when other modules import the module.
Programs generated with this compiler require additional runtime support that is stored in the \file{ob\-armt32\-fpe\-run} library file.
\flowgraph{\resource{Oberon\\source code} \ar[r] & \toolbox{obarmt32fpe} \ar[r] \ar@/l/[d] \ar[rd] & \resource{object file} \\ \variable{ECSIMPORT} \ar[ru] & \resource{symbol\\files} \ar@/r/[u] & \resource{debugging\\information}}
\seeoberon\seeassembly\seearm\seeobject\seedebugging
}

\providecommand{\obavr}{
\toolsection{obavr} is a compiler for the Oberon programming language targeting the AVR hardware architecture.
It generates machine code for AVR processors from modules written in Oberon and stores it in corresponding object files.
For debugging purposes, it also creates a debugging information file as well as an assembly file containing a listing of the generated machine code.
In addition, it stores the interface of each module in a symbol file which is required when other modules import the module.
Programs generated with this compiler require additional runtime support that is stored in the \file{ob\-avr\-run} library file.
\flowgraph{\resource{Oberon\\source code} \ar[r] & \toolbox{obavr} \ar[r] \ar@/l/[d] \ar[rd] & \resource{object file} \\ \variable{ECSIMPORT} \ar[ru] & \resource{symbol\\files} \ar@/r/[u] & \resource{debugging\\information}}
\seeoberon\seeassembly\seeavr\seeobject\seedebugging
}

\providecommand{\obavrtt}{
\toolsection{obavr32} is a compiler for the Oberon programming language targeting the AVR32 hardware architecture.
It generates machine code for AVR32 processors from modules written in Oberon and stores it in corresponding object files.
For debugging purposes, it also creates a debugging information file as well as an assembly file containing a listing of the generated machine code.
In addition, it stores the interface of each module in a symbol file which is required when other modules import the module.
Programs generated with this compiler require additional runtime support that is stored in the \file{ob\-avr32\-run} library file.
\flowgraph{\resource{Oberon\\source code} \ar[r] & \toolbox{obavr32} \ar[r] \ar@/l/[d] \ar[rd] & \resource{object file} \\ \variable{ECSIMPORT} \ar[ru] & \resource{symbol\\files} \ar@/r/[u] & \resource{debugging\\information}}
\seeoberon\seeassembly\seeavrtt\seeobject\seedebugging
}

\providecommand{\obmabk}{
\toolsection{obm68k} is a compiler for the Oberon programming language targeting the M68000 hardware architecture.
It generates machine code for M68000 processors from modules written in Oberon and stores it in corresponding object files.
For debugging purposes, it also creates a debugging information file as well as an assembly file containing a listing of the generated machine code.
In addition, it stores the interface of each module in a symbol file which is required when other modules import the module.
Programs generated with this compiler require additional runtime support that is stored in the \file{ob\-m68k\-run} library file.
\flowgraph{\resource{Oberon\\source code} \ar[r] & \toolbox{obm68k} \ar[r] \ar@/l/[d] \ar[rd] & \resource{object file} \\ \variable{ECSIMPORT} \ar[ru] & \resource{symbol\\files} \ar@/r/[u] & \resource{debugging\\information}}
\seeoberon\seeassembly\seemabk\seeobject\seedebugging
}

\providecommand{\obmibl}{
\toolsection{obmibl} is a compiler for the Oberon programming language targeting the MicroBlaze hardware architecture.
It generates machine code for MicroBlaze processors from modules written in Oberon and stores it in corresponding object files.
For debugging purposes, it also creates a debugging information file as well as an assembly file containing a listing of the generated machine code.
In addition, it stores the interface of each module in a symbol file which is required when other modules import the module.
Programs generated with this compiler require additional runtime support that is stored in the \file{ob\-mibl\-run} library file.
\flowgraph{\resource{Oberon\\source code} \ar[r] & \toolbox{obmibl} \ar[r] \ar@/l/[d] \ar[rd] & \resource{object file} \\ \variable{ECSIMPORT} \ar[ru] & \resource{symbol\\files} \ar@/r/[u] & \resource{debugging\\information}}
\seeoberon\seeassembly\seemibl\seeobject\seedebugging
}

\providecommand{\obmipsa}{
\toolsection{obmips32} is a compiler for the Oberon programming language targeting the MIPS32 hardware architecture.
It generates machine code for MIPS32 processors from modules written in Oberon and stores it in corresponding object files.
For debugging purposes, it also creates a debugging information file as well as an assembly file containing a listing of the generated machine code.
In addition, it stores the interface of each module in a symbol file which is required when other modules import the module.
Programs generated with this compiler require additional runtime support that is stored in the \file{ob\-mips32\-run} library file.
\flowgraph{\resource{Oberon\\source code} \ar[r] & \toolbox{obmips32} \ar[r] \ar@/l/[d] \ar[rd] & \resource{object file} \\ \variable{ECSIMPORT} \ar[ru] & \resource{symbol\\files} \ar@/r/[u] & \resource{debugging\\information}}
\seeoberon\seeassembly\seemips\seeobject\seedebugging
}

\providecommand{\obmipsb}{
\toolsection{obmips64} is a compiler for the Oberon programming language targeting the MIPS64 hardware architecture.
It generates machine code for MIPS64 processors from modules written in Oberon and stores it in corresponding object files.
For debugging purposes, it also creates a debugging information file as well as an assembly file containing a listing of the generated machine code.
In addition, it stores the interface of each module in a symbol file which is required when other modules import the module.
Programs generated with this compiler require additional runtime support that is stored in the \file{ob\-mips64\-run} library file.
\flowgraph{\resource{Oberon\\source code} \ar[r] & \toolbox{obmips64} \ar[r] \ar@/l/[d] \ar[rd] & \resource{object file} \\ \variable{ECSIMPORT} \ar[ru] & \resource{symbol\\files} \ar@/r/[u] & \resource{debugging\\information}}
\seeoberon\seeassembly\seemips\seeobject\seedebugging
}

\providecommand{\obmmix}{
\toolsection{obmmix} is a compiler for the Oberon programming language targeting the MMIX hardware architecture.
It generates machine code for MMIX processors from modules written in Oberon and stores it in corresponding object files.
For debugging purposes, it also creates a debugging information file as well as an assembly file containing a listing of the generated machine code.
In addition, it stores the interface of each module in a symbol file which is required when other modules import the module.
Programs generated with this compiler require additional runtime support that is stored in the \file{ob\-mmix\-run} library file.
\flowgraph{\resource{Oberon\\source code} \ar[r] & \toolbox{obmmix} \ar[r] \ar@/l/[d] \ar[rd] & \resource{object file} \\ \variable{ECSIMPORT} \ar[ru] & \resource{symbol\\files} \ar@/r/[u] & \resource{debugging\\information}}
\seeoberon\seeassembly\seemmix\seeobject\seedebugging
}

\providecommand{\oborok}{
\toolsection{obor1k} is a compiler for the Oberon programming language targeting the OpenRISC 1000 hardware architecture.
It generates machine code for OpenRISC 1000 processors from modules written in Oberon and stores it in corresponding object files.
For debugging purposes, it also creates a debugging information file as well as an assembly file containing a listing of the generated machine code.
In addition, it stores the interface of each module in a symbol file which is required when other modules import the module.
Programs generated with this compiler require additional runtime support that is stored in the \file{ob\-or1k\-run} library file.
\flowgraph{\resource{Oberon\\source code} \ar[r] & \toolbox{obor1k} \ar[r] \ar@/l/[d] \ar[rd] & \resource{object file} \\ \variable{ECSIMPORT} \ar[ru] & \resource{symbol\\files} \ar@/r/[u] & \resource{debugging\\information}}
\seeoberon\seeassembly\seeorok\seeobject\seedebugging
}

\providecommand{\obppca}{
\toolsection{obppc32} is a compiler for the Oberon programming language targeting the PowerPC hardware architecture.
It generates machine code for PowerPC processors from modules written in Oberon and stores it in corresponding object files.
The compiler generates machine code for the 32-bit operating mode defined by the PowerPC architecture.
For debugging purposes, it also creates a debugging information file as well as an assembly file containing a listing of the generated machine code.
In addition, it stores the interface of each module in a symbol file which is required when other modules import the module.
Programs generated with this compiler require additional runtime support that is stored in the \file{ob\-ppc32\-run} library file.
\flowgraph{\resource{Oberon\\source code} \ar[r] & \toolbox{obppc32} \ar[r] \ar@/l/[d] \ar[rd] & \resource{object file} \\ \variable{ECSIMPORT} \ar[ru] & \resource{symbol\\files} \ar@/r/[u] & \resource{debugging\\information}}
\seeoberon\seeassembly\seeppc\seeobject\seedebugging
}

\providecommand{\obppcb}{
\toolsection{obppc64} is a compiler for the Oberon programming language targeting the PowerPC hardware architecture.
It generates machine code for PowerPC processors from modules written in Oberon and stores it in corresponding object files.
The compiler generates machine code for the 64-bit operating mode defined by the PowerPC architecture.
For debugging purposes, it also creates a debugging information file as well as an assembly file containing a listing of the generated machine code.
In addition, it stores the interface of each module in a symbol file which is required when other modules import the module.
Programs generated with this compiler require additional runtime support that is stored in the \file{ob\-ppc64\-run} library file.
\flowgraph{\resource{Oberon\\source code} \ar[r] & \toolbox{obppc64} \ar[r] \ar@/l/[d] \ar[rd] & \resource{object file} \\ \variable{ECSIMPORT} \ar[ru] & \resource{symbol\\files} \ar@/r/[u] & \resource{debugging\\information}}
\seeoberon\seeassembly\seeppc\seeobject\seedebugging
}

\providecommand{\obrisc}{
\toolsection{obrisc} is a compiler for the Oberon programming language targeting the RISC hardware architecture.
It generates machine code for RISC processors from modules written in Oberon and stores it in corresponding object files.
For debugging purposes, it also creates a debugging information file as well as an assembly file containing a listing of the generated machine code.
In addition, it stores the interface of each module in a symbol file which is required when other modules import the module.
Programs generated with this compiler require additional runtime support that is stored in the \file{ob\-risc\-run} library file.
\flowgraph{\resource{Oberon\\source code} \ar[r] & \toolbox{obrisc} \ar[r] \ar@/l/[d] \ar[rd] & \resource{object file} \\ \variable{ECSIMPORT} \ar[ru] & \resource{symbol\\files} \ar@/r/[u] & \resource{debugging\\information}}
\seeoberon\seeassembly\seerisc\seeobject\seedebugging
}

\providecommand{\obwasm}{
\toolsection{obwasm} is a compiler for the Oberon programming language targeting the WebAssembly architecture.
It generates machine code for WebAssembly targets from modules written in Oberon and stores it in corresponding object files.
For debugging purposes, it also creates a debugging information file as well as an assembly file containing a listing of the generated machine code.
In addition, it stores the interface of each module in a symbol file which is required when other modules import the module.
Programs generated with this compiler require additional runtime support that is stored in the \file{ob\-wasm\-run} library file.
\flowgraph{\resource{Oberon\\source code} \ar[r] & \toolbox{obwasm} \ar[r] \ar@/l/[d] \ar[rd] & \resource{object file} \\ \variable{ECSIMPORT} \ar[ru] & \resource{symbol\\files} \ar@/r/[u] & \resource{debugging\\information}}
\seeoberon\seeassembly\seewasm\seeobject\seedebugging
}

% converter tools

\providecommand{\dbgdwarf}{
\toolsection{dbgdwarf} is a DWARF debugging information converter tool.
It converts debugging information into the DWARF debugging data format and stores it in corresponding object files~\cite{dwarffile}.
The resulting debugging object files can be combined with runtime support that creates Executable and Linking Format (ELF) files~\cite{elffile}.
\flowgraph{\resource{debugging\\information} \ar[r] & \toolbox{dbgdwarf} \ar[r] & \resource{debugging\\object file}}
\seeobject\seedebugging
}

% assembler tools

\providecommand{\asmprint}{
\toolsection{asmprint} is a pretty printer for generic assembly code.
It reformats generic assembly code and writes it to the standard output stream.
\flowgraph{\resource{generic assembly\\source code} \ar[r] & \toolbox{asmprint} \ar[r] & \resource{reformatted\\source code}}
\seeassembly
}

\providecommand{\amdaasm}{
\toolsection{amd16asm} is an assembler for the AMD64 hardware architecture.
It translates assembly code into machine code for AMD64 processors and stores it in corresponding object files.
By default, the assembler generates machine code for the 16-bit operating mode defined by the AMD64 architecture.
\flowgraph{\resource{AMD16 assembly\\source code} \ar[r] & \toolbox{amd16asm} \ar[r] & \resource{object file}}
\seeassembly\seeamd\seeobject
}

\providecommand{\amdadism}{
\toolsection{amd16dism} is a disassembler for the AMD64 hardware architecture.
It translates machine code from object files targeting AMD64 processors into assembly code and writes it to the standard output stream.
It assumes that the machine code was generated for the 16-bit operating mode defined by the AMD64 architecture.
\flowgraph{\resource{object file} \ar[r] & \toolbox{amd16dism} \ar[r] & \resource{disassembly\\listing}}
\seeassembly\seeamd\seeobject
}

\providecommand{\amdbasm}{
\toolsection{amd32asm} is an assembler for the AMD64 hardware architecture.
It translates assembly code into machine code for AMD64 processors and stores it in corresponding object files.
By default, the assembler generates machine code for the 32-bit operating mode defined by the AMD64 architecture.
\flowgraph{\resource{AMD32 assembly\\source code} \ar[r] & \toolbox{amd32asm} \ar[r] & \resource{object file}}
\seeassembly\seeamd\seeobject
}

\providecommand{\amdbdism}{
\toolsection{amd32dism} is a disassembler for the AMD64 hardware architecture.
It translates machine code from object files targeting AMD64 processors into assembly code and writes it to the standard output stream.
It assumes that the machine code was generated for the 32-bit operating mode defined by the AMD64 architecture.
\flowgraph{\resource{object file} \ar[r] & \toolbox{amd32dism} \ar[r] & \resource{disassembly\\listing}}
\seeassembly\seeamd\seeobject
}

\providecommand{\amdcasm}{
\toolsection{amd64asm} is an assembler for the AMD64 hardware architecture.
It translates assembly code into machine code for AMD64 processors and stores it in corresponding object files.
By default, the assembler generates machine code for the 64-bit operating mode defined by the AMD64 architecture.
\flowgraph{\resource{AMD64 assembly\\source code} \ar[r] & \toolbox{amd64asm} \ar[r] & \resource{object file}}
\seeassembly\seeamd\seeobject
}

\providecommand{\amdcdism}{
\toolsection{amd64dism} is a disassembler for the AMD64 hardware architecture.
It translates machine code from object files targeting AMD64 processors into assembly code and writes it to the standard output stream.
It assumes that the machine code was generated for the 64-bit operating mode defined by the AMD64 architecture.
\flowgraph{\resource{object file} \ar[r] & \toolbox{amd64dism} \ar[r] & \resource{disassembly\\listing}}
\seeassembly\seeamd\seeobject
}

\providecommand{\armaasm}{
\toolsection{arma32asm} is an assembler for the ARM hardware architecture.
It translates assembly code into machine code for ARM processors executing A32 instructions and stores it in corresponding object files.
\flowgraph{\resource{ARM A32 assembly\\source code} \ar[r] & \toolbox{arma32asm} \ar[r] & \resource{object file}}
\seeassembly\seearm\seeobject
}

\providecommand{\armadism}{
\toolsection{arma32dism} is a disassembler for the ARM hardware architecture.
It translates machine code from object files targeting ARM processors executing A32 instructions into assembly code and writes it to the standard output stream.
\flowgraph{\resource{object file} \ar[r] & \toolbox{arma32dism} \ar[r] & \resource{disassembly\\listing}}
\seeassembly\seearm\seeobject
}

\providecommand{\armbasm}{
\toolsection{arma64asm} is an assembler for the ARM hardware architecture.
It translates assembly code into machine code for ARM processors executing A64 instructions and stores it in corresponding object files.
\flowgraph{\resource{ARM A64 assembly\\source code} \ar[r] & \toolbox{arma64asm} \ar[r] & \resource{object file}}
\seeassembly\seearm\seeobject
}

\providecommand{\armbdism}{
\toolsection{arma64dism} is a disassembler for the ARM hardware architecture.
It translates machine code from object files targeting ARM processors executing A64 instructions into assembly code and writes it to the standard output stream.
\flowgraph{\resource{object file} \ar[r] & \toolbox{arma64dism} \ar[r] & \resource{disassembly\\listing}}
\seeassembly\seearm\seeobject
}

\providecommand{\armcasm}{
\toolsection{armt32asm} is an assembler for the ARM hardware architecture.
It translates assembly code into machine code for ARM processors executing T32 instructions and stores it in corresponding object files.
\flowgraph{\resource{ARM T32 assembly\\source code} \ar[r] & \toolbox{armt32asm} \ar[r] & \resource{object file}}
\seeassembly\seearm\seeobject
}

\providecommand{\armcdism}{
\toolsection{armt32dism} is a disassembler for the ARM hardware architecture.
It translates machine code from object files targeting ARM processors executing T32 instructions into assembly code and writes it to the standard output stream.
\flowgraph{\resource{object file} \ar[r] & \toolbox{armt32dism} \ar[r] & \resource{disassembly\\listing}}
\seeassembly\seearm\seeobject
}

\providecommand{\avrasm}{
\toolsection{avrasm} is an assembler for the AVR hardware architecture.
It translates assembly code into machine code for AVR processors and stores it in corresponding object files.
The identifiers \texttt{RXL}, \texttt{RXH}, \texttt{RYL}, \texttt{RYH}, \texttt{RZL}, and \texttt{RZH} are predefined and name the corresponding registers.
The identifiers \texttt{SPL} and \texttt{SPH} are also predefined and evaluate to the address of the corresponding registers.
\flowgraph{\resource{AVR assembly\\source code} \ar[r] & \toolbox{avrasm} \ar[r] & \resource{object file}}
\seeassembly\seeavr\seeobject
}

\providecommand{\avrdism}{
\toolsection{avrdism} is a disassembler for the AVR hardware architecture.
It translates machine code from object files targeting AVR processors into assembly code and writes it to the standard output stream.
\flowgraph{\resource{object file} \ar[r] & \toolbox{avrdism} \ar[r] & \resource{disassembly\\listing}}
\seeassembly\seeavr\seeobject
}

\providecommand{\avrttasm}{
\toolsection{avr32asm} is an assembler for the AVR32 hardware architecture.
It translates assembly code into machine code for AVR32 processors and stores it in corresponding object files.
\flowgraph{\resource{AVR32 assembly\\source code} \ar[r] & \toolbox{avr32asm} \ar[r] & \resource{object file}}
\seeassembly\seeavrtt\seeobject
}

\providecommand{\avrttdism}{
\toolsection{avr32dism} is a disassembler for the AVR32 hardware architecture.
It translates machine code from object files targeting AVR32 processors into assembly code and writes it to the standard output stream.
\flowgraph{\resource{object file} \ar[r] & \toolbox{avr32dism} \ar[r] & \resource{disassembly\\listing}}
\seeassembly\seeavrtt\seeobject
}

\providecommand{\mabkasm}{
\toolsection{m68kasm} is an assembler for the M68000 hardware architecture.
It translates assembly code into machine code for M68000 processors and stores it in corresponding object files.
\flowgraph{\resource{68000 assembly\\source code} \ar[r] & \toolbox{m68kasm} \ar[r] & \resource{object file}}
\seeassembly\seemabk\seeobject
}

\providecommand{\mabkdism}{
\toolsection{m68kdism} is a disassembler for the M68000 hardware architecture.
It translates machine code from object files targeting M68000 processors into assembly code and writes it to the standard output stream.
\flowgraph{\resource{object file} \ar[r] & \toolbox{m68kdism} \ar[r] & \resource{disassembly\\listing}}
\seeassembly\seemabk\seeobject
}

\providecommand{\miblasm}{
\toolsection{miblasm} is an assembler for the MicroBlaze hardware architecture.
It translates assembly code into machine code for MicroBlaze processors and stores it in corresponding object files.
\flowgraph{\resource{MicroBlaze assembly\\source code} \ar[r] & \toolbox{miblasm} \ar[r] & \resource{object file}}
\seeassembly\seemibl\seeobject
}

\providecommand{\mibldism}{
\toolsection{mibldism} is a disassembler for the MicroBlaze hardware architecture.
It translates machine code from object files targeting MicroBlaze processors into assembly code and writes it to the standard output stream.
\flowgraph{\resource{object file} \ar[r] & \toolbox{mibldism} \ar[r] & \resource{disassembly\\listing}}
\seeassembly\seemibl\seeobject
}

\providecommand{\mipsaasm}{
\toolsection{mips32asm} is an assembler for the MIPS32 hardware architecture.
It translates assembly code into machine code for MIPS32 processors and stores it in corresponding object files.
\flowgraph{\resource{MIPS32 assembly\\source code} \ar[r] & \toolbox{mips32asm} \ar[r] & \resource{object file}}
\seeassembly\seemips\seeobject
}

\providecommand{\mipsadism}{
\toolsection{mips32dism} is a disassembler for the MIPS32 hardware architecture.
It translates machine code from object files targeting MIPS32 processors into assembly code and writes it to the standard output stream.
\flowgraph{\resource{object file} \ar[r] & \toolbox{mips32dism} \ar[r] & \resource{disassembly\\listing}}
\seeassembly\seemips\seeobject
}

\providecommand{\mipsbasm}{
\toolsection{mips64asm} is an assembler for the MIPS64 hardware architecture.
It translates assembly code into machine code for MIPS64 processors and stores it in corresponding object files.
\flowgraph{\resource{MIPS64 assembly\\source code} \ar[r] & \toolbox{mips64asm} \ar[r] & \resource{object file}}
\seeassembly\seemips\seeobject
}

\providecommand{\mipsbdism}{
\toolsection{mips64dism} is a disassembler for the MIPS64 hardware architecture.
It translates machine code from object files targeting MIPS64 processors into assembly code and writes it to the standard output stream.
\flowgraph{\resource{object file} \ar[r] & \toolbox{mips64dism} \ar[r] & \resource{disassembly\\listing}}
\seeassembly\seemips\seeobject
}

\providecommand{\mmixasm}{
\toolsection{mmixasm} is an assembler for the MMIX hardware architecture.
It translates assembly code into machine code for MMIX processors and stores it in corresponding object files.
The names of all special registers are predefined and evaluate to the corresponding number.
\flowgraph{\resource{MMIX assembly\\source code} \ar[r] & \toolbox{mmixasm} \ar[r] & \resource{object file}}
\seeassembly\seemmix\seeobject
}

\providecommand{\mmixdism}{
\toolsection{mmixdism} is a disassembler for the MMIX hardware architecture.
It translates machine code from object files targeting MMIX processors into assembly code and writes it to the standard output stream.
\flowgraph{\resource{object file} \ar[r] & \toolbox{mmixdism} \ar[r] & \resource{disassembly\\listing}}
\seeassembly\seemmix\seeobject
}

\providecommand{\orokasm}{
\toolsection{or1kasm} is an assembler for the OpenRISC 1000 hardware architecture.
It translates assembly code into machine code for OpenRISC 1000 processors and stores it in corresponding object files.
\flowgraph{\resource{OpenRISC 1000 assembly\\source code} \ar[r] & \toolbox{or1kasm} \ar[r] & \resource{object file}}
\seeassembly\seeorok\seeobject
}

\providecommand{\orokdism}{
\toolsection{or1kdism} is a disassembler for the OpenRISC 1000 hardware architecture.
It translates machine code from object files targeting OpenRISC 1000 processors into assembly code and writes it to the standard output stream.
\flowgraph{\resource{object file} \ar[r] & \toolbox{or1kdism} \ar[r] & \resource{disassembly\\listing}}
\seeassembly\seeorok\seeobject
}

\providecommand{\ppcaasm}{
\toolsection{ppc32asm} is an assembler for the PowerPC hardware architecture.
It translates assembly code into machine code for PowerPC processors and stores it in corresponding object files.
By default, the assembler generates machine code for the 32-bit operating mode defined by the PowerPC architecture.
\flowgraph{\resource{PowerPC assembly\\source code} \ar[r] & \toolbox{ppc32asm} \ar[r] & \resource{object file}}
\seeassembly\seeppc\seeobject
}

\providecommand{\ppcadism}{
\toolsection{ppc32dism} is a disassembler for the PowerPC hardware architecture.
It translates machine code from object files targeting PowerPC processors into assembly code and writes it to the standard output stream.
It assumes that the machine code was generated for the 32-bit operating mode defined by the PowerPC architecture.
\flowgraph{\resource{object file} \ar[r] & \toolbox{ppc32dism} \ar[r] & \resource{disassembly\\listing}}
\seeassembly\seeppc\seeobject
}

\providecommand{\ppcbasm}{
\toolsection{ppc64asm} is an assembler for the PowerPC hardware architecture.
It translates assembly code into machine code for PowerPC processors and stores it in corresponding object files.
By default, the assembler generates machine code for the 64-bit operating mode defined by the PowerPC architecture.
\flowgraph{\resource{PowerPC assembly\\source code} \ar[r] & \toolbox{ppc64asm} \ar[r] & \resource{object file}}
\seeassembly\seeppc\seeobject
}

\providecommand{\ppcbdism}{
\toolsection{ppc64dism} is a disassembler for the PowerPC hardware architecture.
It translates machine code from object files targeting PowerPC processors into assembly code and writes it to the standard output stream.
It assumes that the machine code was generated for the 64-bit operating mode defined by the PowerPC architecture.
\flowgraph{\resource{object file} \ar[r] & \toolbox{ppc64dism} \ar[r] & \resource{disassembly\\listing}}
\seeassembly\seeppc\seeobject
}

\providecommand{\riscasm}{
\toolsection{riscasm} is an assembler for the RISC hardware architecture.
It translates assembly code into machine code for RISC processors and stores it in corresponding object files.
The names of all special registers are predefined and evaluate to the corresponding number.
\flowgraph{\resource{RISC assembly\\source code} \ar[r] & \toolbox{riscasm} \ar[r] & \resource{object file}}
\seeassembly\seerisc\seeobject
}

\providecommand{\riscdism}{
\toolsection{riscdism} is a disassembler for the RISC hardware architecture.
It translates machine code from object files targeting RISC processors into assembly code and writes it to the standard output stream.
\flowgraph{\resource{object file} \ar[r] & \toolbox{riscdism} \ar[r] & \resource{disassembly\\listing}}
\seeassembly\seerisc\seeobject
}

\providecommand{\wasmasm}{
\toolsection{wasmasm} is an assembler for the WebAssembly architecture.
It translates assembly code into machine code for WebAssembly targets and stores it in corresponding object files.
The names of all special registers are predefined and evaluate to the corresponding number.
\flowgraph{\resource{WebAssembly assembly\\source code} \ar[r] & \toolbox{wasmasm} \ar[r] & \resource{object file}}
\seeassembly\seewasm\seeobject
}

\providecommand{\wasmdism}{
\toolsection{wasmdism} is a disassembler for the WebAssembly architecture.
It translates machine code from object files targeting WebAssembly targets into assembly code and writes it to the standard output stream.
\flowgraph{\resource{object file} \ar[r] & \toolbox{wasmdism} \ar[r] & \resource{disassembly\\listing}}
\seeassembly\seewasm\seeobject
}

% linker tools

\providecommand{\linklib}{
\toolsection{linklib} is an object file combiner.
It creates a static library file by combining all object files given to it into a single one.
\flowgraph{\resource{object files} \ar[r] & \toolbox{linklib} \ar[r] & \resource{library file}}
\seeobject
}

\providecommand{\linkbin}{
\toolsection{linkbin} is a linker for plain binary files.
It links all object files given to it into a single image and stores it in a binary file that begins with the first linked section.
It also creates a map file that lists the address, type, name and size of all used sections.
The filename extension of the resulting binary file can be specified by putting it into a constant data section called \texttt{\_extension}.
\flowgraph{\resource{object files} \ar[r] & \toolbox{linkbin} \ar[r] \ar[d] & \resource{binary file} \\ & \resource{map file}}
\seeobject
}

\providecommand{\linkmem}{
\toolsection{linkmem} is a linker for plain binary files partitioned into random-access and read-only memory.
It links all object files given to it into two distinct images, one for data sections and one for code and constant data sections, and stores each image in a binary file that begins with the first linked section of the corresponding type.
It also creates a map file that lists the address, type, name and size of all used sections.
\flowgraph{\resource{object files} \ar[r] & \toolbox{linkmem} \ar[r] \ar[d] & \resource{RAM file/\\ROM file} \\ & \resource{map file}}
\seeobject
}

\providecommand{\linkprg}{
\toolsection{linkprg} is a linker for GEMDOS executable files.
It links all object files given to it into a single image and stores the image in an Atari GEMDOS executable file~\cite{gemdosfile}.
It also creates a map file that lists the address relative to the text segment, type, name and size of all used sections.
The filename extension of the resulting executable file can be specified by putting it into a constant data section called \texttt{\_extension}.
The GEMDOS executable file format requires all patch patterns of absolute link patches to consist of four full bitmasks with descending offsets.
\flowgraph{\resource{object files} \ar[r] & \toolbox{linkprg} \ar[r] \ar[d] & \resource{executable file} \\ & \resource{map file}}
\seeobject
}

\providecommand{\linkhex}{
\toolsection{linkhex} is a linker for Intel HEX files.
It links all code sections of the object files given to it into single image and stores the image in an Intel HEX file~\cite{hexfile} that begins with the first linked section.
It also creates a map file that lists the address, type, name and size of all used sections.
\flowgraph{\resource{object files} \ar[r] & \toolbox{linkhex} \ar[r] \ar[d] & \resource{HEX file} \\ & \resource{map file}}
\seeobject
}

\providecommand{\mapsearch}{
\toolsection{mapsearch} is a debugging tool.
It searches map files generated by linker tools for the name of a binary section that encompasses a memory address read from the standard input stream.
If additionally provided with one or more object files, it also stores an excerpt thereof in a separate object file called map search result which only contains the identified binary section for disassembling purposes.
\flowgraph{& \resource{map files/\\object files} \ar[d] \\ \resource{memory\\address} \ar[r] & \toolbox{mapsearch} \ar[r] \ar[d] & \resource{section name/\\relative offset} \\ & \resource{object file\\excerpt}}
\seeobject
}

\renewcommand{\seedebugging}{}

\startchapter{Debugging Information}{Debugging Information Representation}{debugging}
{This \documentation{} describes the debugging information generated by the compilers of the \ecs{} and its open file format.
Additionally, it describes the functionality and interface of the debugging information converter tools provided by the \ecs{}.}

\epigraph{Every failure is a step to success.}{William Whewell}

\section{Introduction}

Although the \ecs{} does not provide its own debugger tool, its compiler tools do collect and store \emph{debugging information} for external debuggers.
The debugging information generated alongside an object file consists of an abstract representation of all programming language constructs like functions, variables, data types, and statements compiled into the object file.
It is designed to enable debuggers to support \emph{program animation} and \emph{memory inspection}.
For this purpose, the \ecs{} provides converter tools which generate a binary representation of the debugging information and store it in specific debugging data formats using additional object files as shown in Figure~\ref{fig:dbgdataflow}.
The resulting debugging object files can later be optionally linked together with the original object file.
\seeobject

\begin{figure}
\flowgraph{
\resource{source code} \ar[d] & \resource{source code} \ar[d] & \resource{source code} \ar[d] \\
\converter{Compiler \textit{A}} \ar[d] \ar[rd] & \converter{Compiler \textit{B}} \ar[d] & \converter{Compiler \textit{C}} \ar[d] \ar[ld] \\
\resource{object file} & \resource{debugging\\information} \ar[ld] \ar[d] \ar[rd] & \resource{object file} \\
\converter{Converter \textit{X}} \ar[d] & \converter{Converter \textit{Y}} \ar[d] & \converter{Converter \textit{Z}} \ar[d] \\
\resource{debugging\\object file} & \resource{debugging\\object file} & \resource{debugging\\object file} \\
}\caption[Debugging information representation of programs]{Debugging information representation of programs in-between compilers and converters}
\label{fig:dbgdataflow}
\end{figure}

Converting debugging information into separate object files effectively decouples a compiler from the debugger of the target runtime environment and its required debugging data format.
It also ensures that compilers always generate the same output regardless of whether the resulting binary executable is subject to debugging or not.
The following sections describe the semantics of the debugging information representation alongside the syntax of its textual file format.

\section{Debugging Information Structure}

The debugging information generated by the various compiler tools of the \ecs{} always consists of a short description of target hardware architecture.
It also contains a list of \emph{information entries} which are generic representations of programming language constructs like functions, variables, or data types.

\subsection{Information Entries}\label{sec:dbginformationentries}

Each information entry has a unique name and a \emph{source code location} referring to its textual declaration in the source code.
The \ecs{} supports the following kinds of information entries:

\begin{itemize}

\item Code Entry\alignright\syntax{"code"}\nopagebreak

A code entry represents a functional unit of the programming language like a function or procedure and corresponds to the code section in the object file with the same name.
It contains a \emph{type declaration} which describes the memory layout and data representation of the returned result.
It additionally holds of a list of \emph{symbol declarations} which describe the storage objects declared by this information entry.
It also contains a list of \emph{breakpoints} which associate source code locations of statements with the corresponding machine code offsets in the code section.

\item Data Entry\alignright\syntax{"data"}\nopagebreak

A data entry represents a data unit of the programming language like a global variable and corresponds to the data section in the object file with the same name.
It contains a type declaration for the storage object represented by the data section.

\item Type Entry\alignright\syntax{"type"}\nopagebreak

A type entry represents a named type definition of the programming language.
It contains a type declaration and may omit the source code location for predefined types.

\end{itemize}

\subsection{Symbol Declarations}\label{sec:dbgsymboldeclarations}

A symbol declaration refers to a single storage object managed by a code section like a local variable or parameter.
It has a unique name and a source code location referring to its corresponding declaration in the source code.
It also contains a type declaration for the data type of its storage object and a description of its \emph{lifetime} in terms of a range of offsets in the code section wherein the symbol is considered alive.
The \ecs{} supports the following kinds of symbol declarations:

\begin{itemize}

\item Constant Declaration\nopagebreak

A constant declaration refers to a constant or storage object that is not supposed to change its value.
Since a constant declaration must not necessarily refer to an actually existing storage object, it is represented using its constant value.

\item Register Declaration\nopagebreak

A register declaration refers to a storage object which is stored in a register.
The register of the target hardware architecture is represented using its name.

\item Variable Declaration\nopagebreak

A variable declaration refers to a storage object that has an actual memory address.
The address refers either to a data section or to a displaced register that typically names the frame pointer.

\end{itemize}

\subsection{Type Declarations}\label{sec:dbgtypedeclarations}

A type declaration describes the layout and data representation of memory regions occupied by storage objects.
The \ecs{} supports the following kinds of type declarations:

\begin{itemize}

\item Void Type\alignright\syntax{"void"}\nopagebreak

A void type represents an unspecified, ambiguous, or nonexistent type of the programming language.

\item Type Name\alignright\syntax{<Name>}\nopagebreak

A type name refers to the specified type entry, see Section~\ref{sec:dbginformationentries}.

\item Signed Integer Type\alignright\syntax{"signed"}\nopagebreak

A signed integer type represents a storage object with a given size that stores a single signed integer value using two's complement as signed magnitude representation.

\item Unsigned Integer Type\alignright\syntax{"unsigned"}\nopagebreak

An unsigned integer type represents a storage object with a given size that stores a single unsigned integer value.

\item Floating-point Number Type\alignright\syntax{"float"}\nopagebreak

A floating-point number type represents a storage object with a given size that stores a single floating-point number value according to formats defined in the IEEE standard for floating-point arithmetic~\cite{ieee1985}.

\item Enumeration Type\alignright\syntax{"enumeration"}\nopagebreak

An enumeration type represents a storage object that stores a single integer value from a predefined set of named constants called \emph{enumerators}.
It contains the type of the underlying integer type and declarations for each enumerator.

\item Array Type\alignright\syntax{"array"}\nopagebreak

An array type represents a collection of consecutive storage objects with the same type called its elements.
It contains the index of its first element, the number of elements in case the array has a static size, and a declaration for the element type.

\item Record Type\alignright\syntax{"record"}\nopagebreak

A record type describes a user-defined data structure that contains an arbitrary number of storage objects called fields.
It contains an overall size of the data structure and declarations for each field.

\item Pointer Type\alignright\syntax{"pointer"}\nopagebreak

A pointer type represents a dereferencable storage object that stores a potentially invalid address of another storage object.
It contains the type of the referenced storage object.

\item Reference Type\alignright\syntax{"reference"}\nopagebreak

A reference type represents a dereferencable storage object that stores a valid address of another storage object.
It contains the type of the referenced storage object.

\item Function Type\alignright\syntax{"function"}\nopagebreak

A function type represents the type of a functional unit of the programming language.
It contains type declarations for the returned result and each function parameter.

\end{itemize}

\section{Debugging Information Format}

The debugging information generated by compiler tools is stored in plain text files according to the complete syntax specification given in Figure~\ref{fig:dbgfileformat}.
A debugging information file consists of a description of its target as well as an arbitrary number of sources and information entries according to the following syntax:

\begin{figure}
\centering\ifbook\small\fi\setlength{\grammarparsep}{0ex}
\begin{minipage}{34em}\begin{grammar}
<Information> = <Target> <Sources> <Entries>$\opt$ \par
<Target> = <Name> <Endianness> <Pointer> \par
<Name> = double-quoted-string \par
<Endianness> = "little" $\mid$ "big" \par
<Pointer> = <Size> \par
<Size> = decimal-integer \par
<Sources> = <Source> $\mid$ <Sources> <Source> \par
<Source> = double-quoted-string \par
<Entries> = <Entry> $\mid$ <Entries> <Entry> \par
<Entry> = "code" <Name> <Location> <Type> <Size> <Symbols>$\opt$ <Breakpoints>$\opt$ $\mid$ \\ "data" <Name> <Location> <Type> <Size> $\mid$ \\ "type" <Name> <Location>$\opt$ <Type> \par
<Location> = <Index> <Line> <Column> \par
<Index> = decimal-integer \par
<Line> = decimal-integer \par
<Column> = decimal-integer \par
<Type> = "void" $\mid$ <Name> $\mid$ "signed" <Size> $\mid$ "unsigned" <Size> $\mid$ "float" <Size> $\mid$ \\ "array" <Index> <Size> <Type> $\mid$ "record" <Size> <Fields>$\opt$ $\mid$ \\ "pointer" <Type> $\mid$ "reference" <Type> $\mid$ \\ "function" <Size> <Type> <Parameters>$\opt$ $\mid$ \\ "enumeration" <Type> <Enumerators> \par
<Fields> = <Field> $\mid$ <Fields> <Field> \par
<Field> = <Name> <Location> <Type> <Offset> <Bitmask> \par
<Offset> = decimal-integer \par
<Bitmask> = decimal-integer \par
<Parameters> = <Parameter> $\mid$ <Parameters> <Parameter> \par
<Parameter> = <Type> \par
<Enumerators> = <Enumerator> $\mid$ <Enumerators> <Enumerator> \par
<Enumerator> = <Name> <Location> <Value> \par
<Value> = signed-decimal-integer $\mid$ decimal-integer \par
<Symbols> = <Symbol> $\mid$ <Symbols> <Symbol> \par
<Symbol> = <Name> <Location> <Kind> <Type> <Lifetime> \par
<Kind> = <Constant> $\mid$ <Register> $\mid$ <Variable> \par
<Constant> = "signed" signed-decimal-integer $\mid$ \\ "unsigned" decimal-integer $\mid$ \\ "float" decimal-floating-point \par
<Register> = <Name> \par
<Variable> = <Register> <Displacement> \par
<Displacement> = signed-decimal-integer \par
<Lifetime> = <Begin> <End> \par
<Begin> = <Offset> \par
<End> = <Offset> \par
<Breakpoints> = <Breakpoint> $\mid$ <Breakpoints> <Breakpoint> \par
<Breakpoint> = <Offset> <Location> \par
\end{grammar}\end{minipage}
\caption{Syntax of the debugging information file format}
\label{fig:dbgfileformat}
\end{figure}

\begin{quote}\begin{grammar}
<Information> = <Target> <Sources> <Entries>$\opt$ \par
<Sources> = <Source> $\mid$ <Sources> <Source> \par
<Source> = double-quoted-string \par
\end{grammar}\end{quote}

The sequence of sources lists all filenames used during compilation and is referenced by index in source code locations.

\subsection{Target Description}

The description of the target architecture is represented in the debugging information file as text according to the following syntax:

\begin{quote}\begin{grammar}
<Target> = <Name> <Endianness> <Pointer> \par
<Name> = double-quoted-string \par
<Endianness> = "little" $\mid$ "big" \par
<Pointer> = <Size> \par
<Size> = decimal-integer \par
\end{grammar}\end{quote}

The description specifies the unique name of the target architecture as well as its endianness and pointer size expressed in octets.

\subsection{Information Entries}

Information entries are represented in the debugging information file as text according to the following syntax:

\begin{quote}\begin{grammar}
<Entries> = <Entry> $\mid$ <Entries> <Entry> \par
<Entry> = "code" <Name> <Location> <Type> <Size> <Symbols>$\opt$ <Breakpoints>$\opt$ $\mid$ \\ "data" <Name> <Location> <Type> <Size> $\mid$ \\ "type" <Name> <Location>$\opt$ <Type> \par
<Name> = double-quoted-string \par
<Size> = decimal-integer \par
\end{grammar}\end{quote}

The valid identifiers for the kind of the information entry correspond to the information entries described in Section~\ref{sec:dbginformationentries}.
The size of an entry is expressed in octets.

\subsection{Source Code Locations}

Source code locations associated with information entries, symbol declarations, field declarations, enumerator declarations, and breakpoints are represented in the debugging information file as text according to the following syntax:

\begin{quote}\begin{grammar}
<Location> = <Index> <Line> <Column> \par
<Index> = decimal-integer \par
<Line> = decimal-integer \par
<Column> = decimal-integer \par
\end{grammar}\end{quote}

The zero-based index refers to one of the sources listed at the beginning of the debugging information in order of occurrence.
Line and column numbering starts with one where all white-space characters are counted as single characters.

\subsection{Symbol Declarations}

Symbol declarations are represented in the debugging information file as text according to the following syntax:

\begin{quote}\begin{grammar}
<Symbols> = <Symbol> $\mid$ <Symbols> <Symbol> \par
<Symbol> = <Name> <Location> <Kind> <Type> <Lifetime> \par
<Name> = double-quoted-string \par
<Kind> = <Constant> $\mid$ <Register> $\mid$ <Variable> \par
<Constant> = "signed" signed-decimal-integer $\mid$ \\ "unsigned" decimal-integer $\mid$ \\ "float" decimal-floating-point \par
<Register> = <Name> \par
<Variable> = <Register> <Displacement> \par
<Displacement> = signed-decimal-integer \par
<Lifetime> = <Begin> <End> \par
<Begin> = <Offset> \par
<End> = <Offset> \par
<Offset> = decimal-integer \par
\end{grammar}\end{quote}

The valid kinds of symbols correspond to the symbol declarations described in Section~\ref{sec:dbgsymboldeclarations}.
The offsets of symbol lifetimes refer to machine code instructions relative to the beginning of the code section corresponding to the enclosing code entry and are expressed in octets.
A symbol declaration with an empty name conventionally denotes the result of the code section rather than a local variable or parameter.
A variable with a displacement of zero names a data section rather than a register.

\subsection{Type Declarations}

Type declarations are represented in the debugging information file as text according to the following syntax:

\begin{quote}\begin{grammar}
<Type> = "void" $\mid$ <Name> $\mid$ "signed" <Size> $\mid$ "unsigned" <Size> $\mid$ "float" <Size> $\mid$ \\ "array" <Index> <Size> <Type> $\mid$ "record" <Size> <Fields>$\opt$ $\mid$ \\ "pointer" <Type> $\mid$ "reference" <Type> $\mid$ \\ "function" <Size> <Type> <Parameters>$\opt$ $\mid$ \\ "enumeration" <Type> <Enumerators> \par
<Name> = double-quoted-string \par
<Size> = decimal-integer \par
<Index> = decimal-integer \par
<Parameters> = <Parameter> $\mid$ <Parameters> <Parameter> \par
<Parameter> = <Type> \par
\end{grammar}\end{quote}

The valid kinds of types correspond to the type declarations described in Section~\ref{sec:dbgtypedeclarations}.
The size of a type is expressed in octets except for array types where it denotes the number of array elements, and for function types where it denotes the number of parameters.

\subsection{Field Declarations}

Field declarations are represented in the debugging information file as text according to the following syntax:

\begin{quote}\begin{grammar}
<Fields> = <Field> $\mid$ <Fields> <Field> \par
<Field> = <Name> <Location> <Type> <Offset> <Bitmask> \par
<Name> = double-quoted-string \par
<Offset> = decimal-integer \par
<Bitmask> = decimal-integer \par
\end{grammar}\end{quote}

The offset specifies the address of the storage object relative to the beginning of the enclosing data structure and is expressed in octets.
The binary value of the non-zero bitmask specifies the number and consecutive sequence of bits occupied by the storage object in case a bit field is declared.
A field declaration with an empty name conventionally denotes inheritance rather than composition.

\subsection{Enumerator Declarations}

Enumerator declarations are represented in the debugging information file as text according to the following syntax:

\begin{quote}\begin{grammar}
<Enumerators> = <Enumerator> $\mid$ <Enumerators> <Enumerator> \par
<Enumerator> = <Name> <Location> <Value> \par
<Name> = double-quoted-string \par
<Value> = signed-decimal-integer $\mid$ decimal-integer \par
\end{grammar}\end{quote}

The constant value of an enumerator is an optionally signed integer that belongs to the underlying integer type of the corresponding enumeration.

\subsection{Breakpoints}

Breakpoints are represented in the debugging information file as text according to the following syntax:

\begin{quote}\begin{grammar}
<Breakpoints> = <Breakpoint> $\mid$ <Breakpoints> <Breakpoint> \par
<Breakpoint> = <Offset> <Location> \par
<Offset> = decimal-integer \par
\end{grammar}\end{quote}

The offset refers to a machine code instruction relative to the beginning of the code section corresponding to the enclosing code entry and is expressed in octets.

\section{Converter Tools}

Converters process debugging information files that were previously generated by the various compiler tools of the \ecs{}.
They store debugging information in object files that represent the same information in a binary debugging data format suitable for the debugger of the target runtime environment.
\interface

\dbgdwarf

\concludechapter

% Generic documentation generation
% Copyright (C) Florian Negele

% This file is part of the Eigen Compiler Suite.

% Permission is granted to copy, distribute and/or modify this document
% under the terms of the GNU Free Documentation License, Version 1.3
% or any later version published by the Free Software Foundation.

% You should have received a copy of the GNU Free Documentation License
% along with the ECS.  If not, see <https://www.gnu.org/licenses/>.

% Generic documentation utilities
% Copyright (C) Florian Negele

% This file is part of the Eigen Compiler Suite.

% Permission is granted to copy, distribute and/or modify this document
% under the terms of the GNU Free Documentation License, Version 1.3
% or any later version published by the Free Software Foundation.

% You should have received a copy of the GNU Free Documentation License
% along with the ECS.  If not, see <https://www.gnu.org/licenses/>.

\providecommand{\cpp}{C\texttt{++}}
\providecommand{\opt}{_\mathit{opt}}
\providecommand{\tool}[1]{\texttt{#1}}
\providecommand{\version}{Version 0.0.40}
\providecommand{\resource}[1]{*++\txt{#1}}
\providecommand{\ecs}{Eigen Compiler Suite}
\providecommand{\changed}[1]{\underline{#1}}
\providecommand{\toolbox}[1]{\converter{#1}}
\providecommand{\file}{}\renewcommand{\file}[1]{\texttt{#1}}
\providecommand{\alignright}{\hfill\linebreak[0]\hspace*{\fill}}
\providecommand{\converter}[1]{*++[F][F*:white][F,:gray]\txt{#1}}
\providecommand{\documentation}{\ifbook chapter\else document\fi}
\providecommand{\Documentation}{\ifbook Chapter\else Document\fi}
\providecommand{\variable}[1]{\resource{\texttt{\small#1}\\variable}}
\providecommand{\documentationref}[2]{\ifbook\ref{#1}\else``\href{#1}{#2}''~\cite{#1}\fi}
\providecommand{\objfile}[1]{\texttt{#1}\index[runtime]{#1 object file@\texttt{#1} object file}}
\providecommand{\libfile}[1]{\texttt{#1}\index[runtime]{#1 library file@\texttt{#1} library file}}
\providecommand{\epigraph}[2]{\ifbook\begin{quote}\flushright\textit{#1}\par--- #2\end{quote}\fi}
\providecommand{\environmentvariable}[1]{\texttt{#1}\index{Environment variables!#1@\texttt{#1}}}
\providecommand{\environment}[1]{\texttt{#1}\index[environment]{#1 environment@\texttt{#1} environment}}
\providecommand{\toolsection}{}\renewcommand{\toolsection}[1]{\subsection{#1}\label{\prefix:#1}\tool{#1}}
\providecommand{\instruction}{}\renewcommand{\instruction}[2]{\noindent\qquad\pdftooltip{\texttt{#1}}{#2}\refstepcounter{instruction}\par}
\providecommand{\flowgraph}{}\renewcommand{\flowgraph}[1]{\par\sffamily\begin{displaymath}\xymatrix@=4ex{#1}\end{displaymath}\normalfont\par}
\providecommand{\instructionset}{}\renewcommand{\instructionset}[4]{\setcounter{instruction}{0}\begin{multicols}{\ifbook#3\else#4\fi}[{\captionof{table}[#2]{#2 (\ref*{#1:instructions}~instructions)}\label{tab:#1set}\vspace{-2ex}}]\footnotesize\raggedcolumns\input{#1.set}\label{#1:instructions}\end{multicols}}

\providecommand{\gpl}{GNU General Public License}
\providecommand{\rse}{ECS Runtime Support Exception}
\providecommand{\fdl}{\href{https://www.gnu.org/licenses/fdl.html}{GNU Free Documentation License}}

\providecommand{\docbegin}{}
\providecommand{\docend}{}
\providecommand{\doclabel}[1]{\hypertarget{#1}}
\providecommand{\doclink}[2]{\hyperlink{#1}{#2}}
\providecommand{\docsection}[3]{\hypertarget{#1}{\subsection{#2}}\label{sec:#1}\index[library]{#2@#3}}
\providecommand{\docsectionstar}[1]{}
\providecommand{\docsubbegin}{\begin{description}}
\providecommand{\docsubend}{\end{description}}
\providecommand{\docsubsection}[3]{\item[\hypertarget{#1}{#2}]\index[library]{#2@#3}}
\providecommand{\docsubsectionstar}[1]{\smallskip}
\providecommand{\docsubsubsection}[3]{\docsubsection{#1}{#2}{#3}}
\providecommand{\docsubsubsectionstar}[1]{}
\providecommand{\docsubsubsubsection}[3]{}
\providecommand{\docsubsubsubsectionstar}[1]{}
\providecommand{\doctable}{}

\providecommand{\debuggingtool}{}\renewcommand{\debuggingtool}{This tool is provided for debugging purposes.
It allows exposing and modifying an internal data structure that is usually not accessible.
}

\providecommand{\interface}{All tools accept command-line arguments which are taken as names of plain text files containing the source code.
If no arguments are provided, the standard input stream is used instead.
Output files are generated in the current working directory and have the same name as the input file being processed whereas the filename extension gets replaced by an appropriate suffix.
\seeinterface
}

\providecommand{\license}{\noindent Copyright \copyright{} Florian Negele\par\medskip\noindent
Permission is granted to copy, distribute and/or modify this document under the terms of the
\fdl{}, Version 1.3 or any later version published by the \href{https://fsf.org/}{Free Software Foundation}.
}

\providecommand{\ecslogosurface}{
\fill[darkgray] (0,0,0) -- (0,0,3) -- (0,3,3) -- (0,3,1) -- (0,4,1) -- (0,4,3) -- (0,5,3) -- (0,5,0) -- (0,2,0) -- (0,2,2) -- (0,1,2) -- (0,1,0) -- cycle;
\fill[gray] (0,5,0) -- (0,5,3) -- (1,5,3) -- (1,5,1) -- (2,5,1) -- (2,5,3) -- (3,5,3) -- (3,5,0) -- cycle;
\fill[lightgray] (0,0,0) -- (0,1,0) -- (2,1,0) -- (2,4,0) -- (1,4,0) -- (1,3,0) -- (2,3,0) -- (2,2,0) -- (0,2,0) -- (0,5,0) -- (3,5,0) -- (3,0,0) -- cycle;
\begin{scope}[line width=0.5]
\begin{scope}[gray]
\draw (0,0,0) -- (0,1,0);
\draw (2,1,0) -- (2,2,0);
\draw (0,1,2) -- (0,2,2);
\draw (0,2,0) -- (0,5,0);
\draw (2,3,0) -- (2,4,0);
\end{scope}
\begin{scope}[lightgray]
\draw (0,1,0) -- (0,1,2);
\draw (0,3,1) -- (0,3,3);
\draw (0,5,0) -- (0,5,3);
\draw (2,5,1) -- (2,5,3);
\end{scope}
\begin{scope}[white]
\draw (0,1,0) -- (2,1,0);
\draw (1,3,0) -- (2,3,0);
\draw (0,5,0) -- (3,5,0);
\end{scope}
\end{scope}
}

\providecommand{\ecslogo}[1]{
\begin{tikzpicture}[scale={(#1)/((sin(45)+cos(45))*3cm)},x={({-cos(45)*1cm},{sin(45)*sin(30)*1cm})},y={({0cm},{(cos(30)*1cm})},z={({sin(45)*1cm},{cos(45)*sin(30)*1cm})}]
\begin{scope}[darkgray,line width=1]
\draw (0,0,0) -- (0,0,3) -- (0,3,3) -- (2,3,3) -- (2,5,3) -- (3,5,3) -- (3,5,0) -- (3,0,0) -- cycle;
\draw (0,3,1) -- (0,4,1) -- (0,4,3) -- (0,5,3) -- (1,5,3) -- (1,5,1) -- (2,5,1);
\draw (1,3,0) -- (1,4,0) -- (2,4,0);
\end{scope}
\fill[darkgray] (2,0,0) -- (2,0,3) -- (2,5,3) -- (2,5,1) -- (2,4,1) -- (2,4,0) -- cycle;
\fill[lightgray] (2,0,2) -- (0,0,2) -- (0,2,2) -- (2,2,2) -- cycle;
\fill[gray] (0,1,0) -- (2,1,0) -- (2,1,2) -- (0,1,2) -- cycle;
\fill[gray] (0,3,1) -- (0,3,3) -- (2,3,3) -- (2,3,0) -- (1,3,0) -- (1,3,1) -- cycle;
\ecslogosurface
\end{tikzpicture}
}

\providecommand{\shadowedecslogo}[3]{
\begin{tikzpicture}[scale={(#1)/((sin(#2)+cos(#2))*3cm)},x={({-cos(#2)*1cm},{sin(#2)*sin(#3)*1cm})},y={({0cm},{(cos(#3)*1cm})},z={({sin(#2)*1cm},{cos(#2)*sin(#3)*1cm})}]
\shade[top color=lightgray!50!white,bottom color=white,middle color=lightgray!50!white] (0,0,0) -- (3,0,0) -- (3,{-0.5-3*sin(#2)*sin(#3)/cos(#3)},0) -- (0,-0.5,0) -- cycle;
\shade[top color=darkgray!50!gray,bottom color=white,middle color=darkgray!50!white] (0,0,0) -- (0,0,3) -- (0,{-0.5-3*cos(#2)*sin(#3)/cos(#3)},3) -- (0,-0.5,0) -- cycle;
\begin{scope}[y={({(cos(#2)+sin(#2))*0.5cm},{(cos(#2)*sin(#3)-sin(#2)*sin(#3))*0.5cm})}]
\useasboundingbox (3,0,0) -- (0,0,0) -- (0,0,3);
\shade[left color=darkgray!80!black,right color=lightgray,middle color=gray] (0,0,0) -- (0,1,0) -- (0,1,0.5) -- (0,2,0) -- (0,5,0) -- (0,5,3) -- (1,5,3) -- (1,4,3) -- (1,4,2.5) -- (1,3,3) -- (2,5,3) -- (3,5,3) -- (3,0,3) -- cycle;
\clip (0,0,0) -- (0,0,3) -- ({-3*sin(#2)/cos(#2)},0,0) -- cycle;
\shade[left color=darkgray,right color=lightgray!50!gray] (0,0,0) -- (0,1,0) -- (0,1,0.5) -- (0,2,0) -- (0,5,0) -- (0,5,3) -- (1,5,3) -- (1,4,3) -- (1,4,2.5) -- (1,3,3) -- (2,5,3) -- (3,5,3) -- (3,0,3) -- cycle;
\end{scope}
\shade[left color=darkgray,right color=darkgray!80!black] (2,0,0) -- (2,0,3) -- (2,5,3) -- (2,5,1) -- (2,4,1) -- (2,4,0) -- cycle;
\shade[left color=darkgray!90!black,right color=gray!80!darkgray] (2,0,2) -- (0,0,2) -- (0,2,2) -- (2,2,2) -- cycle;
\shade[top color=darkgray!90!black,bottom color=gray!80!darkgray] (0,1,0) -- (2,1,0) -- (2,1,2) -- (0,1,2) -- cycle;
\shade[top color=darkgray!90!black,bottom color=gray!80!darkgray] (0,3,1) -- (0,3,3) -- (2,3,3) -- (2,3,0) -- (1,3,0) -- (1,3,1) -- cycle;
\fill[gray] (2,1,0) -- (1.5,1,0.5) -- (0,1,0.5) -- (0,1,0) -- cycle;
\fill[gray] (1,3,2) -- (0.5,3,2) -- (0.5,3,3) -- (1,3,3) -- cycle;
\fill[gray] (2,3,0) -- (1.5,3,0.5) -- (1,3,0.5) -- (1,3,0) -- cycle;
\ecslogosurface
\end{tikzpicture}
}

\providecommand{\cpplogo}[1]{
\begin{tikzpicture}[scale=(#1)/512em]
\fill[gray] (435.2794,398.7159) -- (247.1911,507.3075) .. controls (236.3563,513.5642) and (218.6240,513.5642) .. (207.7892,507.3075) -- (19.7009,398.7159) .. controls (8.8646,392.4606) and (0.0000,377.1043) .. (0.0000,364.5924) -- (0.0000,147.4076) .. controls (0.8430,132.8363) and (8.2856,120.7683) .. (19.7009,113.2842) -- (207.7892,4.6926) .. controls (218.6240,-1.5642) and (236.3564,-1.5642) .. (247.1911,4.6926) -- (435.2794,113.2842) .. controls (447.5273,121.4304) and (454.4987,133.6918) .. (454.9803,147.4076) -- (454.9803,364.5924) .. controls (454.5404,377.7571) and (446.6566,391.0351) .. (435.2794,398.7159) -- cycle(75.8301,255.9993) .. controls (74.9389,404.0881) and (273.2892,469.4783) .. (358.8263,331.8769) -- (293.1917,293.8965) .. controls (253.5702,359.4301) and (155.1909,335.9977) .. (151.6601,255.9993) .. controls (152.7204,182.2703) and (249.4137,148.0211) .. (293.1961,218.1065) -- (358.8308,180.1276) .. controls (283.4477,49.2645) and (79.6318,96.3470) .. (75.8301,255.9993) -- cycle(379.1503,247.5747) -- (362.2982,247.5747) -- (362.2982,230.7226) -- (345.4490,230.7226) -- (345.4490,247.5747) -- (328.5969,247.5747) -- (328.5969,264.4254) -- (345.4490,264.4254) -- (345.4490,281.2759) -- (362.2982,281.2759) -- (362.2982,264.4254) -- (379.1503,264.4254) -- cycle(442.3420,247.5747) -- (425.4899,247.5747) -- (425.4899,230.7226) -- (408.6408,230.7226) -- (408.6408,247.5747) -- (391.7886,247.5747) -- (391.7886,264.4254) -- (408.6408,264.4254) -- (408.6408,281.2759) -- (425.4899,281.2759) -- (425.4899,264.4254) -- (442.3420,264.4254) -- cycle;
\end{tikzpicture}
}

\providecommand{\fallogo}[1]{
\begin{tikzpicture}[scale=(#1)/512em]
\fill[gray] (185.7774,0.0000) .. controls (200.4486,15.9798) and (226.8966,8.7148) .. (235.0426,31.5836) .. controls (249.5297,58.0598) and (247.9581,97.9161) .. (280.3335,110.9762) .. controls (309.1690,120.3496) and (337.8406,104.2727) .. (366.5753,103.9379) .. controls (373.4449,111.5171) and (379.2885,128.2574) .. (383.9755,108.9744) .. controls (396.6979,102.5615) and (437.2808,107.6681) .. (426.9652,124.3252) .. controls (408.9822,121.0785) and (412.4742,146.0729) .. (426.5192,131.4996) .. controls (433.8413,120.8489) and (465.1541,126.5522) .. (441.9067,135.7950) .. controls (396.1879,157.7478) and (344.1112,161.5079) .. (298.5528,183.5702) .. controls (277.7471,193.5198) and (284.6941,218.7163) .. (285.2127,236.9640) .. controls (292.3599,316.2826) and (307.3929,394.6311) .. (317.1198,473.6154) .. controls (329.0637,505.4736) and (292.1195,528.5004) .. (265.9183,511.2761) .. controls (237.9284,499.2462) and (237.3684,465.2681) .. (230.9102,439.9421) .. controls (218.6692,374.3397) and (215.6307,306.9662) .. (198.1732,242.3977) .. controls (183.1379,232.7444) and (164.4245,256.0298) .. (149.0430,261.4799) .. controls (116.9328,279.2585) and (87.1822,308.5851) .. (48.2293,307.8914) .. controls (21.3220,306.9037) and (-15.9107,281.8761) .. (7.2921,252.7908) .. controls (29.7799,220.6177) and (67.5177,204.3028) .. (100.9287,185.9449) .. controls (130.8217,170.8906) and (161.1548,156.5903) .. (191.0278,141.5847) .. controls (196.1738,120.0520) and (186.6049,95.2409) .. (186.8382,72.4353) .. controls (185.5234,48.4204) and (183.1700,23.9341) .. (185.7774,0.0000) -- cycle;
\end{tikzpicture}
}

\providecommand{\oblogo}[1]{
\begin{tikzpicture}[scale=(#1)/512em]
\fill[gray] (160.3865,208.9117) .. controls (154.0879,214.6478) and (149.0735,221.2409) .. (145.4125,228.5384) .. controls (184.8790,248.4273) and (234.7122,269.8787) .. (297.5493,291.8782) .. controls (300.3943,281.4769) and (300.9552,268.7619) .. (300.4023,255.2389) .. controls (248.9909,244.7891) and (200.0310,225.9279) .. (160.3865,208.9117) -- cycle(225.7398,392.6996) .. controls (308.0209,392.1716) and (359.3326,345.9277) .. (368.7203,285.2098) .. controls (376.6742,197.1784) and (311.7194,141.3342) .. (205.4287,142.1456) .. controls (139.9485,141.4804) and (88.7155,166.1957) .. (73.5775,228.0086) .. controls (52.0297,320.3408) and (123.4078,391.0103) .. (225.7398,392.6996) -- cycle(216.0739,176.4733) .. controls (268.9183,179.2424) and (315.8292,206.5488) .. (312.7454,265.1139) .. controls (313.2769,315.6384) and (286.5993,353.4946) .. (216.6040,355.7934) .. controls (162.4657,355.7934) and (126.0914,317.5023) .. (126.0914,260.5103) .. controls (126.1733,214.2900) and (163.3363,176.2849) .. (216.0739,176.4733) -- cycle(76.4897,189.1754) .. controls (13.1586,147.5631) and (0.0000,119.4207) .. (0.0000,119.4207) -- (90.6499,170.1632) .. controls (85.3004,175.8497) and (80.5994,182.1633) .. (76.4897,189.1754) -- cycle(353.9486,119.3004) -- (402.9482,119.3004) .. controls (427.0025,137.0797) and (450.9893,162.7034) .. (474.9529,191.0213) .. controls (509.3540,228.5339) and (531.3391,294.2091) .. (487.8149,312.1206) .. controls (462.8165,324.7652) and (394.3874,316.8943) .. (373.8912,313.6651) .. controls (379.9291,297.7449) and (383.2899,278.4204) .. (381.4989,257.7214) .. controls (420.3069,248.0321) and (421.9610,218.3461) .. (407.7867,192.6417) .. controls (391.1113,162.4018) and (370.1114,132.9097) .. (353.9486,119.3004) -- cycle;
\end{tikzpicture}
}

\providecommand{\markuptable}{
\begin{table}
\sffamily\centering
\begin{tabular}{@{}lcl@{}}
\toprule
\texttt{//italics//} & $\rightarrow$ & \textit{italics} \\
\midrule
\texttt{**bold**} & $\rightarrow$ & \textbf{bold} \\
\midrule
\texttt{\# ordered list} & & 1 ordered list \\
\texttt{\# second item} & $\rightarrow$ & 2 second item \\
\texttt{\#\# sub item} & & \hspace{1em} 1 sub item \\
\midrule
\texttt{* unordered list} & & $\bullet$ unordered list \\
\texttt{* second item} & $\rightarrow$ & $\bullet$ second item \\
\texttt{** sub item} & & \hspace{1em} $\bullet$ sub item \\
\midrule
\texttt{link to [[label]]} & $\rightarrow$ & link to \underline{label} \\
\midrule
\texttt{<{}<label>{}> definition } & $\rightarrow$ & definition \\
\midrule
\texttt{[[url|link name]]} & $\rightarrow$ & \underline{link name} \\
\midrule\addlinespace
\texttt{= large heading} & & {\Large large heading} \smallskip \\
\texttt{== medium heading} & $\rightarrow$ & {\large medium heading} \\
\texttt{=== small heading} & & small heading \\
\midrule
\texttt{no line break} & & no line break for paragraphs \\
\texttt{for paragraphs} & $\rightarrow$ \\
& & use empty line \\
\texttt{use empty line} \\
\midrule
\texttt{force\textbackslash\textbackslash line break} & $\rightarrow$ & force \\
& & line break \\
\midrule
\texttt{horizontal line} & $\rightarrow$ & horizontal line \\
\texttt{----} & & \hrulefill \\
\midrule
\texttt{|=a|=table|=header} & & \underline{a \enspace table \enspace header} \\
\texttt{|a|table|row} & $\rightarrow$ & a \enspace table \enspace row \\
\texttt{|b|table|row} & & b \enspace table \enspace row \\
\midrule
\texttt{\{\{\{} \\
\texttt{unformatted} & $\rightarrow$ & \texttt{unformatted} \\
\texttt{code} & & \texttt{code} \\
\texttt{\}\}\}} \\
\midrule\addlinespace
\texttt{@ new article} & & {\Large 1.\ new article} \smallskip \\
\texttt{@ second article} & $\rightarrow$ & {\Large 2.\ second article} \smallskip \\
\texttt{@@ sub article} & & {\large 2.1.\ sub article} \\
\bottomrule
\end{tabular}
\normalfont\caption{Elements of the generic documentation markup language}
\label{tab:docmarkup}
\end{table}
}

\providecommand{\startchapter}[4]{
\documentclass[11pt,a4paper]{article}
\usepackage{booktabs}
\usepackage[format=hang,labelfont=bf]{caption}
\usepackage{changepage}
\usepackage[T1]{fontenc}
\usepackage[margin=2cm]{geometry}
\usepackage{hyperref}
\usepackage[american]{isodate}
\usepackage{lmodern}
\usepackage{longtable}
\usepackage{mathptmx}
\usepackage{microtype}
\usepackage[toc]{multitoc}
\usepackage{multirow}
\usepackage[all]{nowidow}
\usepackage{pdfcomment}
\usepackage{syntax}
\usepackage{tikz}
\usepackage[all]{xy}
\hypersetup{pdfborder={0 0 0},bookmarksnumbered=true,pdftitle={\ecs{}: #2},pdfauthor={Florian Negele},pdfsubject={\ecs{}},pdfkeywords={#1}}
\setlength{\grammarindent}{8em}\setlength{\grammarparsep}{0.2ex}
\setlength{\columnsep}{2em}
\newcommand{\prefix}{}
\newcounter{instruction}
\bibliographystyle{unsrt}
\renewcommand{\index}[2][]{}
\renewcommand{\arraystretch}{1.05}
\renewcommand{\floatpagefraction}{0.7}
\renewcommand{\syntleft}{\itshape}\renewcommand{\syntright}{}
\title{\vspace{-5ex}\Huge{\ecs{}}\medskip\hrule}
\author{\huge{#2}}
\date{\medskip\version}
\newif\ifbook\bookfalse
\pagestyle{headings}
\frenchspacing
\begin{document}
\maketitle\thispagestyle{empty}\noindent#4\setlength{\columnseprule}{0.4pt}\tableofcontents\setlength{\columnseprule}{0pt}\vfill\pagebreak[3]\null\vfill\bigskip\noindent
\parbox{\textwidth-4em}{\license The contents of this \documentation{} are part of the \href{manual}{\ecs{} User Manual}~\cite{manual} and correspond to Chapter ``\href{manual\##3}{#1}''.\alignright\mbox{\today}}
\parbox{4em}{\flushright\ecslogo{3em}}
\clearpage
}

\providecommand{\concludechapter}{
\vfill\pagebreak[3]\null\vfill
\thispagestyle{myheadings}\markright{REFERENCES}
\noindent\begin{minipage}{\textwidth}\begin{multicols}{2}[\section*{References}]
\renewcommand{\section}[2]{}\small\bibliography{references}
\end{multicols}\end{minipage}\end{document}
}

\providecommand{\startpresentation}[2]{
\documentclass[14pt,aspectratio=43,usepdftitle=false]{beamer}
\usepackage{booktabs}
\usepackage{etex}
\usepackage{multicol}
\usepackage{tikz}
\usepackage[all]{xy}
\bibliographystyle{unsrt}
\setlength{\columnsep}{1em}
\setlength{\leftmargini}{1em}
\setbeamercolor{title}{fg=black}
\setbeamercolor{structure}{fg=darkgray}
\setbeamercolor{bibliography item}{fg=darkgray}
\setbeamerfont{title}{series=\bfseries}
\setbeamerfont{subtitle}{series=\normalfont}
\setbeamerfont*{frametitle}{parent=title}
\setbeamerfont{block title}{series=\bfseries}
\setbeamerfont*{framesubtitle}{parent=subtitle}
\setbeamersize{text margin left=1em,text margin right=1em}
\setbeamertemplate{navigation symbols}{}
\setbeamertemplate{itemize item}[circle]{}
\setbeamertemplate{bibliography item}[triangle]{}
\setbeamertemplate{bibliography entry author}{\usebeamercolor[fg]{bibliography item}}
\setbeamertemplate{frametitle}{\medskip\usebeamerfont{frametitle}\color{gray}\raisebox{-2.5ex}[0ex][0ex]{\rule{0.1em}{4.5ex}}}
\addtobeamertemplate{frametitle}{}{\hspace{0.4em}\usebeamercolor[fg]{title}\insertframetitle\par\vspace{0.2ex}\hspace{0.5em}\usebeamerfont{framesubtitle}\insertframesubtitle}
\hypersetup{pdfborder={0 0 0},bookmarksnumbered=true,bookmarksopen=true,bookmarksopenlevel=0,pdftitle={\ecs{}: #1},pdfauthor={Florian Negele},pdfsubject={\ecs{}},pdfkeywords={#1}}
\renewcommand{\flowgraph}[1]{\resizebox{\textwidth}{!}{$$\xymatrix{##1}$$}}
\title{\ecs{}\medskip\hrule\medskip}
\institute{\shadowedecslogo{5em}{30}{15}}
\date{\version}
\subtitle{#1}
\begin{document}
\begin{frame}[plain]\titlepage\nocite{manual}\end{frame}
\begin{frame}{Contents}{#1}\begin{center}\tableofcontents\end{center}\end{frame}
}

\providecommand{\concludepresentation}{
\begin{frame}{References}\begin{footnotesize}\setlength{\columnseprule}{0.4pt}\begin{multicols}{2}\bibliography{references}\end{multicols}\end{footnotesize}\end{frame}
\end{document}
}

\providecommand{\startbook}[1]{
\documentclass[10pt,paper=17cm:24cm,DIV=13,twoside=semi,headings=normal,numbers=noendperiod,cleardoublepage=plain]{scrbook}
\usepackage{atveryend}
\usepackage{booktabs}
\usepackage{caption}
\usepackage{changepage}
\usepackage[T1]{fontenc}
\usepackage{imakeidx}
\usepackage{hyperref}
\usepackage[american]{isodate}
\usepackage{lmodern}
\usepackage{longtable}
\usepackage{mathptmx}
\usepackage[final]{microtype}
\usepackage{multicol}
\usepackage{multirow}
\usepackage[all]{nowidow}
\usepackage{pdfcomment}
\usepackage{scrlayer-scrpage}
\usepackage{setspace}
\usepackage{syntax}
\usepackage[eventxtindent=4pt,oddtxtexdent=4pt]{thumbs}
\usepackage{tikz}
\usepackage[all]{xy}
\hyphenation{Micro-Blaze Open-Cores Open-RISC Power-PC}
\hypersetup{pdfborder={0 0 0},bookmarksnumbered=true,bookmarksopen=true,bookmarksopenlevel=0,pdftitle={\ecs{}: #1},pdfauthor={Florian Negele},pdfsubject={\ecs{}},pdfkeywords={#1}}
\setlength{\grammarindent}{8em}\setlength{\grammarparsep}{0.7ex}
\setkomafont{captionlabel}{\usekomafont{descriptionlabel}}
\renewcommand{\arraystretch}{1.05}\setstretch{1.1}
\renewcommand{\chapterformat}{\thechapter\autodot\enskip\raisebox{-1ex}[0ex][0ex]{\color{gray}\rule{0.1em}{3.5ex}}\enskip}
\renewcommand{\startchapter}[4]{\hypertarget{##3}{\chapter{##1}}\label{##3}##4\addthumb{##1}{\LARGE\sffamily\bfseries\thechapter}{white}{gray}\renewcommand{\prefix}{##3}}
\renewcommand{\concludechapter}{\clearpage{\stopthumb\cleardoublepage}}
\renewcommand{\syntleft}{\itshape}\renewcommand{\syntright}{}
\renewcommand{\floatpagefraction}{0.7}
\renewcommand{\partheademptypage}{}
\DeclareMicrotypeAlias{lmss}{cmr}
\newcommand{\prefix}{}
\newcounter{instruction}
\bibliographystyle{unsrt}
\newif\ifbook\booktrue
\makeindex[intoc,title=Index]
\makeindex[intoc,name=tools,title=Index of Tools,columns=3]
\makeindex[intoc,name=library,title=Index of Library Names]
\makeindex[intoc,name=runtime,title=Index of Runtime Support]
\makeindex[intoc,name=environment,title=Index of Target Environments]
\indexsetup{toclevel=chapter,headers={\indexname}{\indexname}}
\frenchspacing
\begin{document}
\pagenumbering{alph}
\begin{titlepage}\centering
\huge\sffamily\null\vfill\textbf{\ecs{}}\bigskip\hrule\bigskip#1
\normalsize\normalfont\vfill\vfill\shadowedecslogo{10em}{30}{15}
\large\vfill\vfill\version
\end{titlepage}
\null\vfill
\thispagestyle{empty}
\noindent\today\par\medskip
\license A copy of this license is included in Appendix~\ref{fdl} on page~\pageref{fdl}.
All product names used herein are for identification purposes only and may be trademarks of their respective companies.
\concludechapter
\frontmatter
\setcounter{tocdepth}{1}
\tableofcontents
\setcounter{tocdepth}{2}
\concludechapter
\listoffigures
\concludechapter
\listoftables
\concludechapter
}

\providecommand{\concludebook}{
\backmatter
\addtocontents{toc}{\protect\setcounter{tocdepth}{-1}}
\phantomsection\addcontentsline{toc}{part}{Bibliography}
\bibliography{references}
\concludechapter
\phantomsection\addcontentsline{toc}{part}{Indexes}
\printindex
\concludechapter
\indexprologue{\label{idx:tools}}
\printindex[tools]
\concludechapter
\printindex[library]
\concludechapter
\indexprologue{\label{idx:runtime}}
\printindex[runtime]
\concludechapter
\indexprologue{\label{idx:environment}}
\printindex[environment]
\concludechapter
\pagestyle{empty}\pagenumbering{Alph}\null\clearpage
\null\vfill\centering\ecslogo{4em}\par\medskip\license
\end{document}
}

% chapter references

\providecommand{\seedocumentationref}{}\renewcommand{\seedocumentationref}[3]{#1, see \Documentation{}~\documentationref{#2}{#3}. }
\providecommand{\seeinterface}{}\renewcommand{\seeinterface}{\ifbook See \Documentation{}~\documentationref{interface}{User Interface} for more information about the common user interface of all of these tools. \fi}
\providecommand{\seeguide}{}\renewcommand{\seeguide}{\seedocumentationref{For basic examples of using some of these tools in practice}{guide}{User Guide}}
\providecommand{\seecpp}{}\renewcommand{\seecpp}{\seedocumentationref{For more information about the \cpp{} programming language and its implementation by the \ecs{}}{cpp}{User Manual for \cpp{}}}
\providecommand{\seefalse}{}\renewcommand{\seefalse}{\seedocumentationref{For more information about the FALSE programming language and its implementation by the \ecs{}}{false}{User Manual for FALSE}}
\providecommand{\seeoberon}{}\renewcommand{\seeoberon}{\seedocumentationref{For more information about the Oberon programming language and its implementation by the \ecs{}}{oberon}{User Manual for Oberon}}
\providecommand{\seeassembly}{}\renewcommand{\seeassembly}{\seedocumentationref{For more information about the generic assembly language and how to use it}{assembly}{Generic Assembly Language Specification}}
\providecommand{\seeamd}{}\renewcommand{\seeamd}{\seedocumentationref{For more information about how the \ecs{} supports the AMD64 hardware architecture}{amd64}{AMD64 Hardware Architecture Support}}
\providecommand{\seearm}{}\renewcommand{\seearm}{\seedocumentationref{For more information about how the \ecs{} supports the ARM hardware architecture}{arm}{ARM Hardware Architecture Support}}
\providecommand{\seeavr}{}\renewcommand{\seeavr}{\seedocumentationref{For more information about how the \ecs{} supports the AVR hardware architecture}{avr}{AVR Hardware Architecture Support}}
\providecommand{\seeavrtt}{}\renewcommand{\seeavrtt}{\seedocumentationref{For more information about how the \ecs{} supports the AVR32 hardware architecture}{avr32}{AVR32 Hardware Architecture Support}}
\providecommand{\seemabk}{}\renewcommand{\seemabk}{\seedocumentationref{For more information about how the \ecs{} supports the M68000 hardware architecture}{m68k}{M68000 Hardware Architecture Support}}
\providecommand{\seemibl}{}\renewcommand{\seemibl}{\seedocumentationref{For more information about how the \ecs{} supports the MicroBlaze hardware architecture}{mibl}{MicroBlaze Hardware Architecture Support}}
\providecommand{\seemips}{}\renewcommand{\seemips}{\seedocumentationref{For more information about how the \ecs{} supports the MIPS32 and MIPS64 hardware architectures}{mips}{MIPS Hardware Architecture Support}}
\providecommand{\seemmix}{}\renewcommand{\seemmix}{\seedocumentationref{For more information about how the \ecs{} supports the MMIX hardware architecture}{mmix}{MMIX Hardware Architecture Support}}
\providecommand{\seeorok}{}\renewcommand{\seeorok}{\seedocumentationref{For more information about how the \ecs{} supports the OpenRISC 1000 hardware architecture}{or1k}{OpenRISC 1000 Hardware Architecture Support}}
\providecommand{\seeppc}{}\renewcommand{\seeppc}{\seedocumentationref{For more information about how the \ecs{} supports the PowerPC hardware architecture}{ppc}{PowerPC Hardware Architecture Support}}
\providecommand{\seerisc}{}\renewcommand{\seerisc}{\seedocumentationref{For more information about how the \ecs{} supports the RISC hardware architecture}{risc}{RISC Hardware Architecture Support}}
\providecommand{\seewasm}{}\renewcommand{\seewasm}{\seedocumentationref{For more information about how the \ecs{} supports the WebAssembly architecture}{wasm}{WebAssembly Architecture Support}}
\providecommand{\seedocumentation}{}\renewcommand{\seedocumentation}{\seedocumentationref{For more information about generic documentations and their generation by the \ecs{}}{documentation}{Generic Documentation Generation}}
\providecommand{\seedebugging}{}\renewcommand{\seedebugging}{\seedocumentationref{For more information about debugging information and its representation}{debugging}{Debugging Information Representation}}
\providecommand{\seecode}{}\renewcommand{\seecode}{\seedocumentationref{For more information about intermediate code and its purpose}{code}{Intermediate Code Representation}}
\providecommand{\seeobject}{}\renewcommand{\seeobject}{\seedocumentationref{For more information about object files and their purpose}{object}{Object File Representation}}

% generic documentation tools

\providecommand{\docprint}{
\toolsection{docprint} is a pretty printer for generic documentations.
It reformats generic documentations and writes it to the standard output stream.
\debuggingtool
\flowgraph{\resource{generic\\documentation} \ar[r] & \toolbox{docprint} \ar[r] & \resource{generic\\documentation}}
\seedocumentation
}

\providecommand{\doccheck}{
\toolsection{doccheck} is a syntactic and semantic checker for generic documentations.
It just performs syntactic and semantic checks on generic documentations and writes its diagnostic messages to the standard error stream.
\debuggingtool
\flowgraph{\resource{generic\\documentation} \ar[r] & \toolbox{doccheck} \ar[r] & \resource{diagnostic\\messages}}
\seedocumentation
}

\providecommand{\dochtml}{
\toolsection{dochtml} is an HTML documentation generator for generic documentations.
It processes several generic documentations and assembles all information therein into an HTML document.
\debuggingtool
\flowgraph{\resource{generic\\documentation} \ar[r] & \toolbox{dochtml} \ar[r] & \resource{HTML\\document}}
\seedocumentation
}

\providecommand{\doclatex}{
\toolsection{doclatex} is a Latex documentation generator for generic documentations.
It processes several generic documentations and assembles all information therein into a Latex document.
\debuggingtool
\flowgraph{\resource{generic\\documentation} \ar[r] & \toolbox{doclatex} \ar[r] & \resource{Latex\\document}}
\seedocumentation
}

% intermediate code tools

\providecommand{\cdcheck}{
\toolsection{cdcheck} is a syntactic and semantic checker for intermediate code.
It just performs syntactic and semantic checks on programs written in intermediate code and writes its diagnostic messages to the standard error stream.
\debuggingtool
\flowgraph{\resource{intermediate\\code} \ar[r] & \toolbox{cdcheck} \ar[r] & \resource{diagnostic\\messages}}
\seeassembly\seecode
}

\providecommand{\cdopt}{
\toolsection{cdopt} is an optimizer for intermediate code.
It performs various optimizations on programs written in intermediate code and writes the result to the standard output stream.
\debuggingtool
\flowgraph{\resource{intermediate\\code} \ar[r] & \toolbox{cdopt} \ar[r] & \resource{optimized\\code}}
\seeassembly\seecode
}

\providecommand{\cdrun}{
\toolsection{cdrun} is an interpreter for intermediate code.
It processes and executes programs written in intermediate code.
The following code sections are predefined and have the usual semantics:
\texttt{abort}, \texttt{\_Exit}, \texttt{fflush}, \texttt{floor}, \texttt{fputc}, \texttt{free}, \texttt{getchar}, \texttt{malloc}, and \texttt{putchar}.
Diagnostic messages about invalid operations include the name of the executed code section and the index of the erroneous instruction.
\debuggingtool
\flowgraph{\resource{intermediate\\code} \ar[r] & \toolbox{cdrun} \ar@/u/[r] & \resource{input/\\output} \ar@/d/[l]}
\seeassembly\seecode
}

\providecommand{\cdamda}{
\toolsection{cdamd16} is a compiler for intermediate code targeting the AMD64 hardware architecture.
It generates machine code for AMD64 processors from programs written in intermediate code and stores it in corresponding object files.
The compiler generates machine code for the 16-bit operating mode defined by the AMD64 architecture.
It also creates a debugging information file as well as an assembly file containing a listing of the generated machine code.
\debuggingtool
\flowgraph{\resource{intermediate\\code} \ar[r] & \toolbox{cdamd16} \ar[r] \ar[d] \ar[rd] & \resource{object file} \\ & \resource{assembly\\listing} & \resource{debugging\\information}}
\seeassembly\seeamd\seeobject\seecode\seedebugging
}

\providecommand{\cdamdb}{
\toolsection{cdamd32} is a compiler for intermediate code targeting the AMD64 hardware architecture.
It generates machine code for AMD64 processors from programs written in intermediate code and stores it in corresponding object files.
The compiler generates machine code for the 32-bit operating mode defined by the AMD64 architecture.
It also creates a debugging information file as well as an assembly file containing a listing of the generated machine code.
\debuggingtool
\flowgraph{\resource{intermediate\\code} \ar[r] & \toolbox{cdamd32} \ar[r] \ar[d] \ar[rd] & \resource{object file} \\ & \resource{assembly\\listing} & \resource{debugging\\information}}
\seeassembly\seeamd\seeobject\seecode\seedebugging
}

\providecommand{\cdamdc}{
\toolsection{cdamd64} is a compiler for intermediate code targeting the AMD64 hardware architecture.
It generates machine code for AMD64 processors from programs written in intermediate code and stores it in corresponding object files.
The compiler generates machine code for the 64-bit operating mode defined by the AMD64 architecture.
It also creates a debugging information file as well as an assembly file containing a listing of the generated machine code.
\debuggingtool
\flowgraph{\resource{intermediate\\code} \ar[r] & \toolbox{cdamd64} \ar[r] \ar[d] \ar[rd] & \resource{object file} \\ & \resource{assembly\\listing} & \resource{debugging\\information}}
\seeassembly\seeamd\seeobject\seecode\seedebugging
}

\providecommand{\cdarma}{
\toolsection{cdarma32} is a compiler for intermediate code targeting the ARM hardware architecture.
It generates machine code for ARM processors executing A32 instructions from programs written in intermediate code and stores it in corresponding object files.
It also creates a debugging information file as well as an assembly file containing a listing of the generated machine code.
\debuggingtool
\flowgraph{\resource{intermediate\\code} \ar[r] & \toolbox{cdarma32} \ar[r] \ar[d] \ar[rd] & \resource{object file} \\ & \resource{assembly\\listing} & \resource{debugging\\information}}
\seeassembly\seearm\seeobject\seecode\seedebugging
}

\providecommand{\cdarmb}{
\toolsection{cdarma64} is a compiler for intermediate code targeting the ARM hardware architecture.
It generates machine code for ARM processors executing A64 instructions from programs written in intermediate code and stores it in corresponding object files.
It also creates a debugging information file as well as an assembly file containing a listing of the generated machine code.
\debuggingtool
\flowgraph{\resource{intermediate\\code} \ar[r] & \toolbox{cdarma64} \ar[r] \ar[d] \ar[rd] & \resource{object file} \\ & \resource{assembly\\listing} & \resource{debugging\\information}}
\seeassembly\seearm\seeobject\seecode\seedebugging
}

\providecommand{\cdarmc}{
\toolsection{cdarmt32} is a compiler for intermediate code targeting the ARM hardware architecture.
It generates machine code for ARM processors without floating-point extension executing T32 instructions from programs written in intermediate code and stores it in corresponding object files.
It also creates a debugging information file as well as an assembly file containing a listing of the generated machine code.
\debuggingtool
\flowgraph{\resource{intermediate\\code} \ar[r] & \toolbox{cdarmt32} \ar[r] \ar[d] \ar[rd] & \resource{object file} \\ & \resource{assembly\\listing} & \resource{debugging\\information}}
\seeassembly\seearm\seeobject\seecode\seedebugging
}

\providecommand{\cdarmcfpe}{
\toolsection{cdarmt32fpe} is a compiler for intermediate code targeting the ARM hardware architecture.
It generates machine code for ARM processors with floating-point extension executing T32 instructions from programs written in intermediate code and stores it in corresponding object files.
It also creates a debugging information file as well as an assembly file containing a listing of the generated machine code.
\debuggingtool
\flowgraph{\resource{intermediate\\code} \ar[r] & \toolbox{cdarmt32fpe} \ar[r] \ar[d] \ar[rd] & \resource{object file} \\ & \resource{assembly\\listing} & \resource{debugging\\information}}
\seeassembly\seearm\seeobject\seecode\seedebugging
}

\providecommand{\cdavr}{
\toolsection{cdavr} is a compiler for intermediate code targeting the AVR hardware architecture.
It generates machine code for AVR processors from programs written in intermediate code and stores it in corresponding object files.
It also creates a debugging information file as well as an assembly file containing a listing of the generated machine code.
\debuggingtool
\flowgraph{\resource{intermediate\\code} \ar[r] & \toolbox{cdavr} \ar[r] \ar[d] \ar[rd] & \resource{object file} \\ & \resource{assembly\\listing} & \resource{debugging\\information}}
\seeassembly\seeavr\seeobject\seecode\seedebugging
}

\providecommand{\cdavrtt}{
\toolsection{cdavr32} is a compiler for intermediate code targeting the AVR32 hardware architecture.
It generates machine code for AVR32 processors from programs written in intermediate code and stores it in corresponding object files.
It also creates a debugging information file as well as an assembly file containing a listing of the generated machine code.
\debuggingtool
\flowgraph{\resource{intermediate\\code} \ar[r] & \toolbox{cdavr32} \ar[r] \ar[d] \ar[rd] & \resource{object file} \\ & \resource{assembly\\listing} & \resource{debugging\\information}}
\seeassembly\seeavrtt\seeobject\seecode\seedebugging
}

\providecommand{\cdmabk}{
\toolsection{cdm68k} is a compiler for intermediate code targeting the M68000 hardware architecture.
It generates machine code for M68000 processors from programs written in intermediate code and stores it in corresponding object files.
It also creates a debugging information file as well as an assembly file containing a listing of the generated machine code.
\debuggingtool
\flowgraph{\resource{intermediate\\code} \ar[r] & \toolbox{cdm68k} \ar[r] \ar[d] \ar[rd] & \resource{object file} \\ & \resource{assembly\\listing} & \resource{debugging\\information}}
\seeassembly\seemabk\seeobject\seecode\seedebugging
}

\providecommand{\cdmibl}{
\toolsection{cdmibl} is a compiler for intermediate code targeting the MicroBlaze hardware architecture.
It generates machine code for MicroBlaze processors from programs written in intermediate code and stores it in corresponding object files.
It also creates a debugging information file as well as an assembly file containing a listing of the generated machine code.
\debuggingtool
\flowgraph{\resource{intermediate\\code} \ar[r] & \toolbox{cdmibl} \ar[r] \ar[d] \ar[rd] & \resource{object file} \\ & \resource{assembly\\listing} & \resource{debugging\\information}}
\seeassembly\seemibl\seeobject\seecode\seedebugging
}

\providecommand{\cdmipsa}{
\toolsection{cdmips32} is a compiler for intermediate code targeting the MIPS32 hardware architecture.
It generates machine code for MIPS32 processors from programs written in intermediate code and stores it in corresponding object files.
It also creates a debugging information file as well as an assembly file containing a listing of the generated machine code.
\debuggingtool
\flowgraph{\resource{intermediate\\code} \ar[r] & \toolbox{cdmips32} \ar[r] \ar[d] \ar[rd] & \resource{object file} \\ & \resource{assembly\\listing} & \resource{debugging\\information}}
\seeassembly\seemips\seeobject\seecode\seedebugging
}

\providecommand{\cdmipsb}{
\toolsection{cdmips64} is a compiler for intermediate code targeting the MIPS64 hardware architecture.
It generates machine code for MIPS64 processors from programs written in intermediate code and stores it in corresponding object files.
It also creates a debugging information file as well as an assembly file containing a listing of the generated machine code.
\debuggingtool
\flowgraph{\resource{intermediate\\code} \ar[r] & \toolbox{cdmips64} \ar[r] \ar[d] \ar[rd] & \resource{object file} \\ & \resource{assembly\\listing} & \resource{debugging\\information}}
\seeassembly\seemips\seeobject\seecode\seedebugging
}

\providecommand{\cdmmix}{
\toolsection{cdmmix} is a compiler for intermediate code targeting the MMIX hardware architecture.
It generates machine code for MMIX processors from programs written in intermediate code and stores it in corresponding object files.
It also creates a debugging information file as well as an assembly file containing a listing of the generated machine code.
\debuggingtool
\flowgraph{\resource{intermediate\\code} \ar[r] & \toolbox{cdmmix} \ar[r] \ar[d] \ar[rd] & \resource{object file} \\ & \resource{assembly\\listing} & \resource{debugging\\information}}
\seeassembly\seemmix\seeobject\seecode\seedebugging
}

\providecommand{\cdorok}{
\toolsection{cdor1k} is a compiler for intermediate code targeting the OpenRISC 1000 hardware architecture.
It generates machine code for OpenRISC 1000 processors from programs written in intermediate code and stores it in corresponding object files.
It also creates a debugging information file as well as an assembly file containing a listing of the generated machine code.
\debuggingtool
\flowgraph{\resource{intermediate\\code} \ar[r] & \toolbox{cdor1k} \ar[r] \ar[d] \ar[rd] & \resource{object file} \\ & \resource{assembly\\listing} & \resource{debugging\\information}}
\seeassembly\seeorok\seeobject\seecode\seedebugging
}

\providecommand{\cdppca}{
\toolsection{cdppc32} is a compiler for intermediate code targeting the PowerPC hardware architecture.
It generates machine code for PowerPC processors from programs written in intermediate code and stores it in corresponding object files.
The compiler generates machine code for the 32-bit operating mode defined by the PowerPC architecture.
It also creates a debugging information file as well as an assembly file containing a listing of the generated machine code.
\debuggingtool
\flowgraph{\resource{intermediate\\code} \ar[r] & \toolbox{cdppc32} \ar[r] \ar[d] \ar[rd] & \resource{object file} \\ & \resource{assembly\\listing} & \resource{debugging\\information}}
\seeassembly\seeppc\seeobject\seecode\seedebugging
}

\providecommand{\cdppcb}{
\toolsection{cdppc64} is a compiler for intermediate code targeting the PowerPC hardware architecture.
It generates machine code for PowerPC processors from programs written in intermediate code and stores it in corresponding object files.
The compiler generates machine code for the 64-bit operating mode defined by the PowerPC architecture.
It also creates a debugging information file as well as an assembly file containing a listing of the generated machine code.
\debuggingtool
\flowgraph{\resource{intermediate\\code} \ar[r] & \toolbox{cdppc64} \ar[r] \ar[d] \ar[rd] & \resource{object file} \\ & \resource{assembly\\listing} & \resource{debugging\\information}}
\seeassembly\seeppc\seeobject\seecode\seedebugging
}

\providecommand{\cdrisc}{
\toolsection{cdrisc} is a compiler for intermediate code targeting the RISC hardware architecture.
It generates machine code for RISC processors from programs written in intermediate code and stores it in corresponding object files.
It also creates a debugging information file as well as an assembly file containing a listing of the generated machine code.
\debuggingtool
\flowgraph{\resource{intermediate\\code} \ar[r] & \toolbox{cdrisc} \ar[r] \ar[d] \ar[rd] & \resource{object file} \\ & \resource{assembly\\listing} & \resource{debugging\\information}}
\seeassembly\seerisc\seeobject\seecode\seedebugging
}

\providecommand{\cdwasm}{
\toolsection{cdwasm} is a compiler for intermediate code targeting the WebAssembly architecture.
It generates machine code for WebAssembly targets from programs written in intermediate code and stores it in corresponding object files.
It also creates a debugging information file as well as an assembly file containing a listing of the generated machine code.
\debuggingtool
\flowgraph{\resource{intermediate\\code} \ar[r] & \toolbox{cdwasm} \ar[r] \ar[d] \ar[rd] & \resource{object file} \\ & \resource{assembly\\listing} & \resource{debugging\\information}}
\seeassembly\seewasm\seeobject\seecode\seedebugging
}

% C++ tools

\providecommand{\cppprep}{
\toolsection{cppprep} is a preprocessor for the \cpp{} programming language.
It preprocesses source code according to the rules of \cpp{} and writes it to the standard output stream.
Only the macro names \texttt{\_\_DATE\_\_}, \texttt{\_\_FILE\_\_}, \texttt{\_\_LINE\_\_}, and \texttt{\_\_TIME\_\_} are predefined.
\flowgraph{\resource{\cpp{} or other\\source code} \ar[r] & \toolbox{cppprep} \ar[r] & \resource{preprocessed\\source code} \\ & \variable{ECSINCLUDE} \ar[u]}
\seecpp
}

\providecommand{\cppprint}{
\toolsection{cppprint} is a pretty printer for the \cpp{} programming language.
It reformats the source code of \cpp{} programs and writes it to the standard output stream.
\flowgraph{\resource{\cpp{}\\source code} \ar[r] & \toolbox{cppprint} \ar[r] & \resource{reformatted\\source code} \\ & \variable{ECSINCLUDE} \ar[u]}
\seecpp
}

\providecommand{\cppcheck}{
\toolsection{cppcheck} is a syntactic and semantic checker for the \cpp{} programming language.
It just performs syntactic and semantic checks on \cpp{} programs and writes its diagnostic messages to the standard error stream.
\flowgraph{\resource{\cpp{}\\source code} \ar[r] & \toolbox{cppcheck} \ar[r] & \resource{diagnostic\\messages} \\ & \variable{ECSINCLUDE} \ar[u]}
\seecpp
}

\providecommand{\cppdump}{
\toolsection{cppdump} is a serializer for the \cpp{} programming language.
It dumps the complete internal representation of programs written in \cpp{} into an XML document.
\debuggingtool
\flowgraph{\resource{\cpp{}\\source code} \ar[r] & \toolbox{cppdump} \ar[r] & \resource{internal\\representation} \\ & \variable{ECSINCLUDE} \ar[u]}
\seecpp
}

\providecommand{\cpprun}{
\toolsection{cpprun} is an interpreter for the \cpp{} programming language.
It processes and executes programs written in \cpp{}.
The macro \texttt{\_\_run\_\_} is predefined in order to enable programmers to identify this tool while interpreting.
\flowgraph{\resource{\cpp{}\\source code} \ar[r] & \toolbox{cpprun} \ar@/u/[r] & \resource{input/\\output} \ar@/d/[l] \\ & \variable{ECSINCLUDE} \ar[u]}
\seecpp
}

\providecommand{\cppdoc}{
\toolsection{cppdoc} is a generic documentation generator for the \cpp{} programming language.
It processes several \cpp{} source files and assembles all information therein into a generic documentation.
\debuggingtool
\flowgraph{\resource{\cpp{}\\source code} \ar[r] & \toolbox{cppdoc} \ar[r] & \resource{generic\\documentation} \\ & \variable{ECSINCLUDE} \ar[u]}
\seecpp\seedocumentation
}

\providecommand{\cpphtml}{
\toolsection{cpphtml} is an HTML documentation generator for the \cpp{} programming language.
It processes several \cpp{} source files and assembles all information therein into an HTML document.
\flowgraph{\resource{\cpp{}\\source code} \ar[r] & \toolbox{cpphtml} \ar[r] & \resource{HTML\\document} \\ & \variable{ECSINCLUDE} \ar[u]}
\seecpp\seedocumentation
}

\providecommand{\cpplatex}{
\toolsection{cpplatex} is a Latex documentation generator for the \cpp{} programming language.
It processes several \cpp{} source files and assembles all information therein into a Latex document.
\flowgraph{\resource{\cpp{}\\source code} \ar[r] & \toolbox{cpplatex} \ar[r] & \resource{Latex\\document} \\ & \variable{ECSINCLUDE} \ar[u]}
\seecpp\seedocumentation
}

\providecommand{\cppcode}{
\toolsection{cppcode} is an intermediate code generator for the \cpp{} programming language.
It generates intermediate code from programs written in \cpp{} and stores it in corresponding assembly files.
The macro \texttt{\_\_code\_\_} is predefined in order to enable programmers to identify this tool while generating intermediate code.
Programs generated with this tool require additional runtime support that is stored in the \file{cpp\-code\-run} library file.
\debuggingtool
\flowgraph{\resource{\cpp{}\\source code} \ar[r] & \toolbox{cppcode} \ar[r] & \resource{intermediate\\code} \\ & \variable{ECSINCLUDE} \ar[u]}
\seecpp\seeassembly\seecode
}

\providecommand{\cppamda}{
\toolsection{cppamd16} is a compiler for the \cpp{} programming language targeting the AMD64 hardware architecture.
It generates machine code for AMD64 processors from programs written in \cpp{} and stores it in corresponding object files.
The compiler generates machine code for the 16-bit operating mode defined by the AMD64 architecture.
For debugging purposes, it also creates a debugging information file as well as an assembly file containing a listing of the generated machine code.
The macro \texttt{\_\_amd16\_\_} is predefined in order to enable programmers to identify this tool and its target architecture while compiling.
Programs generated with this compiler require additional runtime support that is stored in the \file{cpp\-amd16\-run} library file.
\flowgraph{\resource{\cpp{}\\source code} \ar[r] & \toolbox{cppamd16} \ar[r] \ar[d] \ar[rd] & \resource{object file} \\ \variable{ECSINCLUDE} \ar[ru] & \resource{debugging\\information} & \resource{assembly\\listing}}
\seecpp\seeassembly\seeamd\seeobject\seedebugging
}

\providecommand{\cppamdb}{
\toolsection{cppamd32} is a compiler for the \cpp{} programming language targeting the AMD64 hardware architecture.
It generates machine code for AMD64 processors from programs written in \cpp{} and stores it in corresponding object files.
The compiler generates machine code for the 32-bit operating mode defined by the AMD64 architecture.
For debugging purposes, it also creates a debugging information file as well as an assembly file containing a listing of the generated machine code.
The macro \texttt{\_\_amd32\_\_} is predefined in order to enable programmers to identify this tool and its target architecture while compiling.
Programs generated with this compiler require additional runtime support that is stored in the \file{cpp\-amd32\-run} library file.
\flowgraph{\resource{\cpp{}\\source code} \ar[r] & \toolbox{cppamd32} \ar[r] \ar[d] \ar[rd] & \resource{object file} \\ \variable{ECSINCLUDE} \ar[ru] & \resource{debugging\\information} & \resource{assembly\\listing}}
\seecpp\seeassembly\seeamd\seeobject\seedebugging
}

\providecommand{\cppamdc}{
\toolsection{cppamd64} is a compiler for the \cpp{} programming language targeting the AMD64 hardware architecture.
It generates machine code for AMD64 processors from programs written in \cpp{} and stores it in corresponding object files.
The compiler generates machine code for the 64-bit operating mode defined by the AMD64 architecture.
For debugging purposes, it also creates a debugging information file as well as an assembly file containing a listing of the generated machine code.
The macro \texttt{\_\_amd64\_\_} is predefined in order to enable programmers to identify this tool and its target architecture while compiling.
Programs generated with this compiler require additional runtime support that is stored in the \file{cpp\-amd64\-run} library file.
\flowgraph{\resource{\cpp{}\\source code} \ar[r] & \toolbox{cppamd64} \ar[r] \ar[d] \ar[rd] & \resource{object file} \\ \variable{ECSINCLUDE} \ar[ru] & \resource{debugging\\information} & \resource{assembly\\listing}}
\seecpp\seeassembly\seeamd\seeobject\seedebugging
}

\providecommand{\cpparma}{
\toolsection{cpparma32} is a compiler for the \cpp{} programming language targeting the ARM hardware architecture.
It generates machine code for ARM processors executing A32 instructions from programs written in \cpp{} and stores it in corresponding object files.
For debugging purposes, it also creates a debugging information file as well as an assembly file containing a listing of the generated machine code.
The macro \texttt{\_\_arma32\_\_} is predefined in order to enable programmers to identify this tool and its target architecture while compiling.
Programs generated with this compiler require additional runtime support that is stored in the \file{cpp\-arma32\-run} library file.
\flowgraph{\resource{\cpp{}\\source code} \ar[r] & \toolbox{cpparma32} \ar[r] \ar[d] \ar[rd] & \resource{object file} \\ \variable{ECSINCLUDE} \ar[ru] & \resource{debugging\\information} & \resource{assembly\\listing}}
\seecpp\seeassembly\seearm\seeobject\seedebugging
}

\providecommand{\cpparmb}{
\toolsection{cpparma64} is a compiler for the \cpp{} programming language targeting the ARM hardware architecture.
It generates machine code for ARM processors executing A64 instructions from programs written in \cpp{} and stores it in corresponding object files.
For debugging purposes, it also creates a debugging information file as well as an assembly file containing a listing of the generated machine code.
The macro \texttt{\_\_arma64\_\_} is predefined in order to enable programmers to identify this tool and its target architecture while compiling.
Programs generated with this compiler require additional runtime support that is stored in the \file{cpp\-arma64\-run} library file.
\flowgraph{\resource{\cpp{}\\source code} \ar[r] & \toolbox{cpparma64} \ar[r] \ar[d] \ar[rd] & \resource{object file} \\ \variable{ECSINCLUDE} \ar[ru] & \resource{debugging\\information} & \resource{assembly\\listing}}
\seecpp\seeassembly\seearm\seeobject\seedebugging
}

\providecommand{\cpparmc}{
\toolsection{cpparmt32} is a compiler for the \cpp{} programming language targeting the ARM hardware architecture.
It generates machine code for ARM processors without floating-point extension executing T32 instructions from programs written in \cpp{} and stores it in corresponding object files.
For debugging purposes, it also creates a debugging information file as well as an assembly file containing a listing of the generated machine code.
The macro \texttt{\_\_armt32\_\_} is predefined in order to enable programmers to identify this tool and its target architecture while compiling.
Programs generated with this compiler require additional runtime support that is stored in the \file{cpp\-armt32\-run} library file.
\flowgraph{\resource{\cpp{}\\source code} \ar[r] & \toolbox{cpparmt32} \ar[r] \ar[d] \ar[rd] & \resource{object file} \\ \variable{ECSINCLUDE} \ar[ru] & \resource{debugging\\information} & \resource{assembly\\listing}}
\seecpp\seeassembly\seearm\seeobject\seedebugging
}

\providecommand{\cpparmcfpe}{
\toolsection{cpparmt32fpe} is a compiler for the \cpp{} programming language targeting the ARM hardware architecture.
It generates machine code for ARM processors with floating-point extension executing T32 instructions from programs written in \cpp{} and stores it in corresponding object files.
For debugging purposes, it also creates a debugging information file as well as an assembly file containing a listing of the generated machine code.
The macro \texttt{\_\_armt32fpe\_\_} is predefined in order to enable programmers to identify this tool and its target architecture while compiling.
Programs generated with this compiler require additional runtime support that is stored in the \file{cpp\-armt32\-fpe\-run} library file.
\flowgraph{\resource{\cpp{}\\source code} \ar[r] & \toolbox{cpparmt32fpe} \ar[r] \ar[d] \ar[rd] & \resource{object file} \\ \variable{ECSINCLUDE} \ar[ru] & \resource{debugging\\information} & \resource{assembly\\listing}}
\seecpp\seeassembly\seearm\seeobject\seedebugging
}

\providecommand{\cppavr}{
\toolsection{cppavr} is a compiler for the \cpp{} programming language targeting the AVR hardware architecture.
It generates machine code for AVR processors from programs written in \cpp{} and stores it in corresponding object files.
For debugging purposes, it also creates a debugging information file as well as an assembly file containing a listing of the generated machine code.
The macro \texttt{\_\_avr\_\_} is predefined in order to enable programmers to identify this tool and its target architecture while compiling.
Programs generated with this compiler require additional runtime support that is stored in the \file{cpp\-avr\-run} library file.
\flowgraph{\resource{\cpp{}\\source code} \ar[r] & \toolbox{cppavr} \ar[r] \ar[d] \ar[rd] & \resource{object file} \\ \variable{ECSINCLUDE} \ar[ru] & \resource{debugging\\information} & \resource{assembly\\listing}}
\seecpp\seeassembly\seeavr\seeobject\seedebugging
}

\providecommand{\cppavrtt}{
\toolsection{cppavr32} is a compiler for the \cpp{} programming language targeting the AVR32 hardware architecture.
It generates machine code for AVR32 processors from programs written in \cpp{} and stores it in corresponding object files.
For debugging purposes, it also creates a debugging information file as well as an assembly file containing a listing of the generated machine code.
The macro \texttt{\_\_avr32\_\_} is predefined in order to enable programmers to identify this tool and its target architecture while compiling.
Programs generated with this compiler require additional runtime support that is stored in the \file{cpp\-avr32\-run} library file.
\flowgraph{\resource{\cpp{}\\source code} \ar[r] & \toolbox{cppavr32} \ar[r] \ar[d] \ar[rd] & \resource{object file} \\ \variable{ECSINCLUDE} \ar[ru] & \resource{debugging\\information} & \resource{assembly\\listing}}
\seecpp\seeassembly\seeavrtt\seeobject\seedebugging
}

\providecommand{\cppmabk}{
\toolsection{cppm68k} is a compiler for the \cpp{} programming language targeting the M68000 hardware architecture.
It generates machine code for M68000 processors from programs written in \cpp{} and stores it in corresponding object files.
For debugging purposes, it also creates a debugging information file as well as an assembly file containing a listing of the generated machine code.
The macro \texttt{\_\_m68k\_\_} is predefined in order to enable programmers to identify this tool and its target architecture while compiling.
Programs generated with this compiler require additional runtime support that is stored in the \file{cpp\-m68k\-run} library file.
\flowgraph{\resource{\cpp{}\\source code} \ar[r] & \toolbox{cppm68k} \ar[r] \ar[d] \ar[rd] & \resource{object file} \\ \variable{ECSINCLUDE} \ar[ru] & \resource{debugging\\information} & \resource{assembly\\listing}}
\seecpp\seeassembly\seemabk\seeobject\seedebugging
}

\providecommand{\cppmibl}{
\toolsection{cppmibl} is a compiler for the \cpp{} programming language targeting the MicroBlaze hardware architecture.
It generates machine code for MicroBlaze processors from programs written in \cpp{} and stores it in corresponding object files.
For debugging purposes, it also creates a debugging information file as well as an assembly file containing a listing of the generated machine code.
The macro \texttt{\_\_mibl\_\_} is predefined in order to enable programmers to identify this tool and its target architecture while compiling.
Programs generated with this compiler require additional runtime support that is stored in the \file{cpp\-mibl\-run} library file.
\flowgraph{\resource{\cpp{}\\source code} \ar[r] & \toolbox{cppmibl} \ar[r] \ar[d] \ar[rd] & \resource{object file} \\ \variable{ECSINCLUDE} \ar[ru] & \resource{debugging\\information} & \resource{assembly\\listing}}
\seecpp\seeassembly\seemibl\seeobject\seedebugging
}

\providecommand{\cppmipsa}{
\toolsection{cppmips32} is a compiler for the \cpp{} programming language targeting the MIPS32 hardware architecture.
It generates machine code for MIPS32 processors from programs written in \cpp{} and stores it in corresponding object files.
For debugging purposes, it also creates a debugging information file as well as an assembly file containing a listing of the generated machine code.
The macro \texttt{\_\_mips32\_\_} is predefined in order to enable programmers to identify this tool and its target architecture while compiling.
Programs generated with this compiler require additional runtime support that is stored in the \file{cpp\-mips32\-run} library file.
\flowgraph{\resource{\cpp{}\\source code} \ar[r] & \toolbox{cppmips32} \ar[r] \ar[d] \ar[rd] & \resource{object file} \\ \variable{ECSINCLUDE} \ar[ru] & \resource{debugging\\information} & \resource{assembly\\listing}}
\seecpp\seeassembly\seemips\seeobject\seedebugging
}

\providecommand{\cppmipsb}{
\toolsection{cppmips64} is a compiler for the \cpp{} programming language targeting the MIPS64 hardware architecture.
It generates machine code for MIPS64 processors from programs written in \cpp{} and stores it in corresponding object files.
For debugging purposes, it also creates a debugging information file as well as an assembly file containing a listing of the generated machine code.
The macro \texttt{\_\_mips64\_\_} is predefined in order to enable programmers to identify this tool and its target architecture while compiling.
Programs generated with this compiler require additional runtime support that is stored in the \file{cpp\-mips64\-run} library file.
\flowgraph{\resource{\cpp{}\\source code} \ar[r] & \toolbox{cppmips64} \ar[r] \ar[d] \ar[rd] & \resource{object file} \\ \variable{ECSINCLUDE} \ar[ru] & \resource{debugging\\information} & \resource{assembly\\listing}}
\seecpp\seeassembly\seemips\seeobject\seedebugging
}

\providecommand{\cppmmix}{
\toolsection{cppmmix} is a compiler for the \cpp{} programming language targeting the MMIX hardware architecture.
It generates machine code for MMIX processors from programs written in \cpp{} and stores it in corresponding object files.
For debugging purposes, it also creates a debugging information file as well as an assembly file containing a listing of the generated machine code.
The macro \texttt{\_\_mmix\_\_} is predefined in order to enable programmers to identify this tool and its target architecture while compiling.
Programs generated with this compiler require additional runtime support that is stored in the \file{cpp\-mmix\-run} library file.
\flowgraph{\resource{\cpp{}\\source code} \ar[r] & \toolbox{cppmmix} \ar[r] \ar[d] \ar[rd] & \resource{object file} \\ \variable{ECSINCLUDE} \ar[ru] & \resource{debugging\\information} & \resource{assembly\\listing}}
\seecpp\seeassembly\seemmix\seeobject\seedebugging
}

\providecommand{\cpporok}{
\toolsection{cppor1k} is a compiler for the \cpp{} programming language targeting the OpenRISC 1000 hardware architecture.
It generates machine code for OpenRISC 1000 processors from programs written in \cpp{} and stores it in corresponding object files.
For debugging purposes, it also creates a debugging information file as well as an assembly file containing a listing of the generated machine code.
The macro \texttt{\_\_or1k\_\_} is predefined in order to enable programmers to identify this tool and its target architecture while compiling.
Programs generated with this compiler require additional runtime support that is stored in the \file{cpp\-or1k\-run} library file.
\flowgraph{\resource{\cpp{}\\source code} \ar[r] & \toolbox{cppor1k} \ar[r] \ar[d] \ar[rd] & \resource{object file} \\ \variable{ECSINCLUDE} \ar[ru] & \resource{debugging\\information} & \resource{assembly\\listing}}
\seecpp\seeassembly\seeorok\seeobject\seedebugging
}

\providecommand{\cppppca}{
\toolsection{cppppc32} is a compiler for the \cpp{} programming language targeting the PowerPC hardware architecture.
It generates machine code for PowerPC processors from programs written in \cpp{} and stores it in corresponding object files.
The compiler generates machine code for the 32-bit operating mode defined by the PowerPC architecture.
For debugging purposes, it also creates a debugging information file as well as an assembly file containing a listing of the generated machine code.
The macro \texttt{\_\_ppc32\_\_} is predefined in order to enable programmers to identify this tool and its target architecture while compiling.
Programs generated with this compiler require additional runtime support that is stored in the \file{cpp\-ppc32\-run} library file.
\flowgraph{\resource{\cpp{}\\source code} \ar[r] & \toolbox{cppppc32} \ar[r] \ar[d] \ar[rd] & \resource{object file} \\ \variable{ECSINCLUDE} \ar[ru] & \resource{debugging\\information} & \resource{assembly\\listing}}
\seecpp\seeassembly\seeppc\seeobject\seedebugging
}

\providecommand{\cppppcb}{
\toolsection{cppppc64} is a compiler for the \cpp{} programming language targeting the PowerPC hardware architecture.
It generates machine code for PowerPC processors from programs written in \cpp{} and stores it in corresponding object files.
The compiler generates machine code for the 64-bit operating mode defined by the PowerPC architecture.
For debugging purposes, it also creates a debugging information file as well as an assembly file containing a listing of the generated machine code.
The macro \texttt{\_\_ppc64\_\_} is predefined in order to enable programmers to identify this tool and its target architecture while compiling.
Programs generated with this compiler require additional runtime support that is stored in the \file{cpp\-ppc64\-run} library file.
\flowgraph{\resource{\cpp{}\\source code} \ar[r] & \toolbox{cppppc64} \ar[r] \ar[d] \ar[rd] & \resource{object file} \\ \variable{ECSINCLUDE} \ar[ru] & \resource{debugging\\information} & \resource{assembly\\listing}}
\seecpp\seeassembly\seeppc\seeobject\seedebugging
}

\providecommand{\cpprisc}{
\toolsection{cpprisc} is a compiler for the \cpp{} programming language targeting the RISC hardware architecture.
It generates machine code for RISC processors from programs written in \cpp{} and stores it in corresponding object files.
For debugging purposes, it also creates a debugging information file as well as an assembly file containing a listing of the generated machine code.
The macro \texttt{\_\_risc\_\_} is predefined in order to enable programmers to identify this tool and its target architecture while compiling.
Programs generated with this compiler require additional runtime support that is stored in the \file{cpp\-risc\-run} library file.
\flowgraph{\resource{\cpp{}\\source code} \ar[r] & \toolbox{cpprisc} \ar[r] \ar[d] \ar[rd] & \resource{object file} \\ \variable{ECSINCLUDE} \ar[ru] & \resource{debugging\\information} & \resource{assembly\\listing}}
\seecpp\seeassembly\seerisc\seeobject\seedebugging
}

\providecommand{\cppwasm}{
\toolsection{cppwasm} is a compiler for the \cpp{} programming language targeting the WebAssembly architecture.
It generates machine code for WebAssembly targets from programs written in \cpp{} and stores it in corresponding object files.
For debugging purposes, it also creates a debugging information file as well as an assembly file containing a listing of the generated machine code.
The macro \texttt{\_\_wasm\_\_} is predefined in order to enable programmers to identify this tool and its target architecture while compiling.
Programs generated with this compiler require additional runtime support that is stored in the \file{cpp\-wasm\-run} library file.
\flowgraph{\resource{\cpp{}\\source code} \ar[r] & \toolbox{cppwasm} \ar[r] \ar[d] \ar[rd] & \resource{object file} \\ \variable{ECSINCLUDE} \ar[ru] & \resource{debugging\\information} & \resource{assembly\\listing}}
\seecpp\seeassembly\seewasm\seeobject\seedebugging
}

% FALSE tools

\providecommand{\falprint}{
\toolsection{falprint} is a pretty printer for the FALSE programming language.
It reformats the source code of FALSE programs and writes it to the standard output stream.
\flowgraph{\resource{FALSE\\source code} \ar[r] & \toolbox{falprint} \ar[r] & \resource{reformatted\\source code}}
\seefalse
}

\providecommand{\falcheck}{
\toolsection{falcheck} is a syntactic and semantic checker for the FALSE programming language.
It just performs syntactic and semantic checks on FALSE programs and writes its diagnostic messages to the standard error stream.
\flowgraph{\resource{FALSE\\source code} \ar[r] & \toolbox{falcheck} \ar[r] & \resource{diagnostic\\messages}}
\seefalse
}

\providecommand{\faldump}{
\toolsection{faldump} is a serializer for the FALSE programming language.
It dumps the complete internal representation of programs written in FALSE into an XML document.
\debuggingtool
\flowgraph{\resource{FALSE\\source code} \ar[r] & \toolbox{faldump} \ar[r] & \resource{internal\\representation}}
\seefalse
}

\providecommand{\falrun}{
\toolsection{falrun} is an interpreter for the FALSE programming language.
It processes and executes programs written in FALSE\@.
\flowgraph{\resource{FALSE\\source code} \ar[r] & \toolbox{falrun} \ar@/u/[r] & \resource{input/\\output} \ar@/d/[l]}
\seefalse
}

\providecommand{\falcpp}{
\toolsection{falcpp} is a transpiler for the FALSE programming language.
It translates programs written in FALSE into \cpp{} programs and stores them in corresponding source files.
\flowgraph{\resource{FALSE\\source code} \ar[r] & \toolbox{falcpp} \ar[r] & \resource{\cpp{}\\source file}}
\seefalse\seecpp
}

\providecommand{\falcode}{
\toolsection{falcode} is an intermediate code generator for the FALSE programming language.
It generates intermediate code from programs written in FALSE and stores it in corresponding assembly files.
\debuggingtool
\flowgraph{\resource{FALSE\\source code} \ar[r] & \toolbox{falcode} \ar[r] & \resource{intermediate\\code}}
\seefalse\seeassembly\seecode
}

\providecommand{\falamda}{
\toolsection{falamd16} is a compiler for the FALSE programming language targeting the AMD64 hardware architecture.
It generates machine code for AMD64 processors from programs written in FALSE and stores it in corresponding object files.
The compiler generates machine code for the 16-bit operating mode defined by the AMD64 architecture.
\flowgraph{\resource{FALSE\\source code} \ar[r] & \toolbox{falamd16} \ar[r] & \resource{object file}}
\seefalse\seeamd\seeobject
}

\providecommand{\falamdb}{
\toolsection{falamd32} is a compiler for the FALSE programming language targeting the AMD64 hardware architecture.
It generates machine code for AMD64 processors from programs written in FALSE and stores it in corresponding object files.
The compiler generates machine code for the 32-bit operating mode defined by the AMD64 architecture.
\flowgraph{\resource{FALSE\\source code} \ar[r] & \toolbox{falamd32} \ar[r] & \resource{object file}}
\seefalse\seeamd\seeobject
}

\providecommand{\falamdc}{
\toolsection{falamd64} is a compiler for the FALSE programming language targeting the AMD64 hardware architecture.
It generates machine code for AMD64 processors from programs written in FALSE and stores it in corresponding object files.
The compiler generates machine code for the 64-bit operating mode defined by the AMD64 architecture.
\flowgraph{\resource{FALSE\\source code} \ar[r] & \toolbox{falamd64} \ar[r] & \resource{object file}}
\seefalse\seeamd\seeobject
}

\providecommand{\falarma}{
\toolsection{falarma32} is a compiler for the FALSE programming language targeting the ARM hardware architecture.
It generates machine code for ARM processors executing A32 instructions from programs written in FALSE and stores it in corresponding object files.
\flowgraph{\resource{FALSE\\source code} \ar[r] & \toolbox{falarma32} \ar[r] & \resource{object file}}
\seefalse\seearm\seeobject
}

\providecommand{\falarmb}{
\toolsection{falarma64} is a compiler for the FALSE programming language targeting the ARM hardware architecture.
It generates machine code for ARM processors executing A64 instructions from programs written in FALSE and stores it in corresponding object files.
\flowgraph{\resource{FALSE\\source code} \ar[r] & \toolbox{falarma64} \ar[r] & \resource{object file}}
\seefalse\seearm\seeobject
}

\providecommand{\falarmc}{
\toolsection{falarmt32} is a compiler for the FALSE programming language targeting the ARM hardware architecture.
It generates machine code for ARM processors without floating-point extension executing T32 instructions from programs written in FALSE and stores it in corresponding object files.
\flowgraph{\resource{FALSE\\source code} \ar[r] & \toolbox{falarmt32} \ar[r] & \resource{object file}}
\seefalse\seearm\seeobject
}

\providecommand{\falarmcfpe}{
\toolsection{falarmt32fpe} is a compiler for the FALSE programming language targeting the ARM hardware architecture.
It generates machine code for ARM processors with floating-point extension executing T32 instructions from programs written in FALSE and stores it in corresponding object files.
\flowgraph{\resource{FALSE\\source code} \ar[r] & \toolbox{falarmt32fpe} \ar[r] & \resource{object file}}
\seefalse\seearm\seeobject
}

\providecommand{\falavr}{
\toolsection{falavr} is a compiler for the FALSE programming language targeting the AVR hardware architecture.
It generates machine code for AVR processors from programs written in FALSE and stores it in corresponding object files.
\flowgraph{\resource{FALSE\\source code} \ar[r] & \toolbox{falavr} \ar[r] & \resource{object file}}
\seefalse\seeavr\seeobject
}

\providecommand{\falavrtt}{
\toolsection{falavr32} is a compiler for the FALSE programming language targeting the AVR32 hardware architecture.
It generates machine code for AVR32 processors from programs written in FALSE and stores it in corresponding object files.
\flowgraph{\resource{FALSE\\source code} \ar[r] & \toolbox{falavr32} \ar[r] & \resource{object file}}
\seefalse\seeavrtt\seeobject
}

\providecommand{\falmabk}{
\toolsection{falm68k} is a compiler for the FALSE programming language targeting the M68000 hardware architecture.
It generates machine code for M68000 processors from programs written in FALSE and stores it in corresponding object files.
\flowgraph{\resource{FALSE\\source code} \ar[r] & \toolbox{falm68k} \ar[r] & \resource{object file}}
\seefalse\seemabk\seeobject
}

\providecommand{\falmibl}{
\toolsection{falmibl} is a compiler for the FALSE programming language targeting the MicroBlaze hardware architecture.
It generates machine code for MicroBlaze processors from programs written in FALSE and stores it in corresponding object files.
\flowgraph{\resource{FALSE\\source code} \ar[r] & \toolbox{falmibl} \ar[r] & \resource{object file}}
\seefalse\seemibl\seeobject
}

\providecommand{\falmipsa}{
\toolsection{falmips32} is a compiler for the FALSE programming language targeting the MIPS32 hardware architecture.
It generates machine code for MIPS32 processors from programs written in FALSE and stores it in corresponding object files.
\flowgraph{\resource{FALSE\\source code} \ar[r] & \toolbox{falmips32} \ar[r] & \resource{object file}}
\seefalse\seemips\seeobject
}

\providecommand{\falmipsb}{
\toolsection{falmips64} is a compiler for the FALSE programming language targeting the MIPS64 hardware architecture.
It generates machine code for MIPS64 processors from programs written in FALSE and stores it in corresponding object files.
\flowgraph{\resource{FALSE\\source code} \ar[r] & \toolbox{falmips64} \ar[r] & \resource{object file}}
\seefalse\seemips\seeobject
}

\providecommand{\falmmix}{
\toolsection{falmmix} is a compiler for the FALSE programming language targeting the MMIX hardware architecture.
It generates machine code for MMIX processors from programs written in FALSE and stores it in corresponding object files.
\flowgraph{\resource{FALSE\\source code} \ar[r] & \toolbox{falmmix} \ar[r] & \resource{object file}}
\seefalse\seemmix\seeobject
}

\providecommand{\falorok}{
\toolsection{falor1k} is a compiler for the FALSE programming language targeting the OpenRISC 1000 hardware architecture.
It generates machine code for OpenRISC 1000 processors from programs written in FALSE and stores it in corresponding object files.
\flowgraph{\resource{FALSE\\source code} \ar[r] & \toolbox{falor1k} \ar[r] & \resource{object file}}
\seefalse\seeorok\seeobject
}

\providecommand{\falppca}{
\toolsection{falppc32} is a compiler for the FALSE programming language targeting the PowerPC hardware architecture.
It generates machine code for PowerPC processors from programs written in FALSE and stores it in corresponding object files.
The compiler generates machine code for the 32-bit operating mode defined by the PowerPC architecture.
\flowgraph{\resource{FALSE\\source code} \ar[r] & \toolbox{falppc32} \ar[r] & \resource{object file}}
\seefalse\seeppc\seeobject
}

\providecommand{\falppcb}{
\toolsection{falppc64} is a compiler for the FALSE programming language targeting the PowerPC hardware architecture.
It generates machine code for PowerPC processors from programs written in FALSE and stores it in corresponding object files.
The compiler generates machine code for the 64-bit operating mode defined by the PowerPC architecture.
\flowgraph{\resource{FALSE\\source code} \ar[r] & \toolbox{falppc64} \ar[r] & \resource{object file}}
\seefalse\seeppc\seeobject
}

\providecommand{\falrisc}{
\toolsection{falrisc} is a compiler for the FALSE programming language targeting the RISC hardware architecture.
It generates machine code for RISC processors from programs written in FALSE and stores it in corresponding object files.
\flowgraph{\resource{FALSE\\source code} \ar[r] & \toolbox{falrisc} \ar[r] & \resource{object file}}
\seefalse\seerisc\seeobject
}

\providecommand{\falwasm}{
\toolsection{falwasm} is a compiler for the FALSE programming language targeting the WebAssembly architecture.
It generates machine code for WebAssembly targets from programs written in FALSE and stores it in corresponding object files.
\flowgraph{\resource{FALSE\\source code} \ar[r] & \toolbox{falwasm} \ar[r] & \resource{object file}}
\seefalse\seewasm\seeobject
}

% Oberon tools

\providecommand{\obprint}{
\toolsection{obprint} is a pretty printer for the Oberon programming language.
It reformats the source code of Oberon modules and writes it to the standard output stream.
\flowgraph{\resource{Oberon\\source code} \ar[r] & \toolbox{obprint} \ar[r] & \resource{reformatted\\source code}}
\seeoberon
}

\providecommand{\obcheck}{
\toolsection{obcheck} is a syntactic and semantic checker for the Oberon programming language.
It just performs syntactic and semantic checks on Oberon modules and writes its diagnostic messages to the standard error stream.
In addition, it stores the interface of each module in a symbol file which is required when other modules import the module.
\flowgraph{\resource{Oberon\\source code} \ar[r] & \toolbox{obcheck} \ar[r] \ar@/l/[d] & \resource{diagnostic\\messages} \\ \variable{ECSIMPORT} \ar[ru] & \resource{symbol\\files} \ar@/r/[u]}
\seeoberon
}

\providecommand{\obdump}{
\toolsection{obdump} is a serializer for the Oberon programming language.
It dumps the complete internal representation of modules written in Oberon into an XML document.
\debuggingtool
\flowgraph{\resource{Oberon\\source code} \ar[r] & \toolbox{obdump} \ar[r] \ar@/l/[d] & \resource{internal\\representation} \\ \variable{ECSIMPORT} \ar[ru] & \resource{symbol\\files} \ar@/r/[u]}
\seeoberon
}

\providecommand{\obrun}{
\toolsection{obrun} is an interpreter for the Oberon programming language.
It processes and executes modules written in Oberon.
This tool does neither generate nor process symbol files while interpreting modules.
If a module is imported by another one, its filename has to be named before the other one in the list of command-line arguments.
\flowgraph{\resource{Oberon\\source code} \ar[r] & \toolbox{obrun} \ar@/u/[r] & \resource{input/\\output} \ar@/d/[l]}
\seeoberon
}

\providecommand{\obcpp}{
\toolsection{obcpp} is a transpiler for the Oberon programming language.
It translates programs written in Oberon into \cpp{} programs and stores them in corresponding source and header files.
In addition, it stores the interface of each module in a symbol file which is required when other modules import the module.
The same interface is provided by the generated header file which can be used in other parts of the \cpp{} program.
\flowgraph{\resource{Oberon\\source code} \ar[r] & \toolbox{obcpp} \ar[r] \ar@/l/[d] \ar[rd] & \resource{\cpp{}\\source file} \\ \variable{ECSIMPORT} \ar[ru] & \resource{symbol\\files} \ar@/r/[u] & \resource{\cpp{}\\header file}}
\seeoberon\seecpp
}

\providecommand{\obdoc}{
\toolsection{obdoc} is a generic documentation generator for the Oberon programming language.
It processes several Oberon modules and assembles all information therein into a generic documentation.
In addition, it stores the interface of each module in a symbol file which is required when other modules import the module.
\debuggingtool
\flowgraph{\resource{Oberon\\source code} \ar[r] & \toolbox{obdoc} \ar[r] \ar@/l/[d] & \resource{generic\\documentation} \\ \variable{ECSIMPORT} \ar[ru] & \resource{symbol\\files} \ar@/r/[u]}
\seeoberon\seedocumentation
}

\providecommand{\obhtml}{
\toolsection{obhtml} is an HTML documentation generator for the Oberon programming language.
It processes several Oberon modules and assembles all information therein into an HTML document.
In addition, it stores the interface of each module in a symbol file which is required when other modules import the module.
\flowgraph{\resource{Oberon\\source code} \ar[r] & \toolbox{obhtml} \ar[r] \ar@/l/[d] & \resource{HTML\\document} \\ \variable{ECSIMPORT} \ar[ru] & \resource{symbol\\files} \ar@/r/[u]}
\seeoberon\seedocumentation
}

\providecommand{\oblatex}{
\toolsection{oblatex} is a Latex documentation generator for the Oberon programming language.
It processes several Oberon modules and assembles all information therein into a Latex document.
In addition, it stores the interface of each module in a symbol file which is required when other modules import the module.
\flowgraph{\resource{Oberon\\source code} \ar[r] & \toolbox{oblatex} \ar[r] \ar@/l/[d] & \resource{Latex\\document} \\ \variable{ECSIMPORT} \ar[ru] & \resource{symbol\\files} \ar@/r/[u]}
\seeoberon\seedocumentation
}

\providecommand{\obcode}{
\toolsection{obcode} is an intermediate code generator for the Oberon programming language.
It generates intermediate code from modules written in Oberon and stores it in corresponding assembly files.
In addition, it stores the interface of each module in a symbol file which is required when other modules import the module.
Programs generated with this tool require additional runtime support that is stored in the \file{ob\-code\-run} library file.
\debuggingtool
\flowgraph{\resource{Oberon\\source code} \ar[r] & \toolbox{obcode} \ar[r] \ar@/l/[d] & \resource{intermediate\\code} \\ \variable{ECSIMPORT} \ar[ru] & \resource{symbol\\files} \ar@/r/[u]}
\seeoberon\seeassembly\seecode
}

\providecommand{\obamda}{
\toolsection{obamd16} is a compiler for the Oberon programming language targeting the AMD64 hardware architecture.
It generates machine code for AMD64 processors from modules written in Oberon and stores it in corresponding object files.
The compiler generates machine code for the 16-bit operating mode defined by the AMD64 architecture.
For debugging purposes, it also creates a debugging information file as well as an assembly file containing a listing of the generated machine code.
In addition, it stores the interface of each module in a symbol file which is required when other modules import the module.
Programs generated with this compiler require additional runtime support that is stored in the \file{ob\-amd16\-run} library file.
\flowgraph{\resource{Oberon\\source code} \ar[r] & \toolbox{obamd16} \ar[r] \ar@/l/[d] \ar[rd] & \resource{object file} \\ \variable{ECSIMPORT} \ar[ru] & \resource{symbol\\files} \ar@/r/[u] & \resource{debugging\\information}}
\seeoberon\seeassembly\seeamd\seeobject\seedebugging
}

\providecommand{\obamdb}{
\toolsection{obamd32} is a compiler for the Oberon programming language targeting the AMD64 hardware architecture.
It generates machine code for AMD64 processors from modules written in Oberon and stores it in corresponding object files.
The compiler generates machine code for the 32-bit operating mode defined by the AMD64 architecture.
For debugging purposes, it also creates a debugging information file as well as an assembly file containing a listing of the generated machine code.
In addition, it stores the interface of each module in a symbol file which is required when other modules import the module.
Programs generated with this compiler require additional runtime support that is stored in the \file{ob\-amd32\-run} library file.
\flowgraph{\resource{Oberon\\source code} \ar[r] & \toolbox{obamd32} \ar[r] \ar@/l/[d] \ar[rd] & \resource{object file} \\ \variable{ECSIMPORT} \ar[ru] & \resource{symbol\\files} \ar@/r/[u] & \resource{debugging\\information}}
\seeoberon\seeassembly\seeamd\seeobject\seedebugging
}

\providecommand{\obamdc}{
\toolsection{obamd64} is a compiler for the Oberon programming language targeting the AMD64 hardware architecture.
It generates machine code for AMD64 processors from modules written in Oberon and stores it in corresponding object files.
The compiler generates machine code for the 64-bit operating mode defined by the AMD64 architecture.
For debugging purposes, it also creates a debugging information file as well as an assembly file containing a listing of the generated machine code.
In addition, it stores the interface of each module in a symbol file which is required when other modules import the module.
Programs generated with this compiler require additional runtime support that is stored in the \file{ob\-amd64\-run} library file.
\flowgraph{\resource{Oberon\\source code} \ar[r] & \toolbox{obamd64} \ar[r] \ar@/l/[d] \ar[rd] & \resource{object file} \\ \variable{ECSIMPORT} \ar[ru] & \resource{symbol\\files} \ar@/r/[u] & \resource{debugging\\information}}
\seeoberon\seeassembly\seeamd\seeobject\seedebugging
}

\providecommand{\obarma}{
\toolsection{obarma32} is a compiler for the Oberon programming language targeting the ARM hardware architecture.
It generates machine code for ARM processors executing A32 instructions from modules written in Oberon and stores it in corresponding object files.
For debugging purposes, it also creates a debugging information file as well as an assembly file containing a listing of the generated machine code.
In addition, it stores the interface of each module in a symbol file which is required when other modules import the module.
Programs generated with this compiler require additional runtime support that is stored in the \file{ob\-arma32\-run} library file.
\flowgraph{\resource{Oberon\\source code} \ar[r] & \toolbox{obarma32} \ar[r] \ar@/l/[d] \ar[rd] & \resource{object file} \\ \variable{ECSIMPORT} \ar[ru] & \resource{symbol\\files} \ar@/r/[u] & \resource{debugging\\information}}
\seeoberon\seeassembly\seearm\seeobject\seedebugging
}

\providecommand{\obarmb}{
\toolsection{obarma64} is a compiler for the Oberon programming language targeting the ARM hardware architecture.
It generates machine code for ARM processors executing A64 instructions from modules written in Oberon and stores it in corresponding object files.
For debugging purposes, it also creates a debugging information file as well as an assembly file containing a listing of the generated machine code.
In addition, it stores the interface of each module in a symbol file which is required when other modules import the module.
Programs generated with this compiler require additional runtime support that is stored in the \file{ob\-arma64\-run} library file.
\flowgraph{\resource{Oberon\\source code} \ar[r] & \toolbox{obarma64} \ar[r] \ar@/l/[d] \ar[rd] & \resource{object file} \\ \variable{ECSIMPORT} \ar[ru] & \resource{symbol\\files} \ar@/r/[u] & \resource{debugging\\information}}
\seeoberon\seeassembly\seearm\seeobject\seedebugging
}

\providecommand{\obarmc}{
\toolsection{obarmt32} is a compiler for the Oberon programming language targeting the ARM hardware architecture.
It generates machine code for ARM processors without floating-point extension executing T32 instructions from modules written in Oberon and stores it in corresponding object files.
For debugging purposes, it also creates a debugging information file as well as an assembly file containing a listing of the generated machine code.
In addition, it stores the interface of each module in a symbol file which is required when other modules import the module.
Programs generated with this compiler require additional runtime support that is stored in the \file{ob\-armt32\-run} library file.
\flowgraph{\resource{Oberon\\source code} \ar[r] & \toolbox{obarmt32} \ar[r] \ar@/l/[d] \ar[rd] & \resource{object file} \\ \variable{ECSIMPORT} \ar[ru] & \resource{symbol\\files} \ar@/r/[u] & \resource{debugging\\information}}
\seeoberon\seeassembly\seearm\seeobject\seedebugging
}

\providecommand{\obarmcfpe}{
\toolsection{obarmt32fpe} is a compiler for the Oberon programming language targeting the ARM hardware architecture.
It generates machine code for ARM processors with floating-point extension executing T32 instructions from modules written in Oberon and stores it in corresponding object files.
For debugging purposes, it also creates a debugging information file as well as an assembly file containing a listing of the generated machine code.
In addition, it stores the interface of each module in a symbol file which is required when other modules import the module.
Programs generated with this compiler require additional runtime support that is stored in the \file{ob\-armt32\-fpe\-run} library file.
\flowgraph{\resource{Oberon\\source code} \ar[r] & \toolbox{obarmt32fpe} \ar[r] \ar@/l/[d] \ar[rd] & \resource{object file} \\ \variable{ECSIMPORT} \ar[ru] & \resource{symbol\\files} \ar@/r/[u] & \resource{debugging\\information}}
\seeoberon\seeassembly\seearm\seeobject\seedebugging
}

\providecommand{\obavr}{
\toolsection{obavr} is a compiler for the Oberon programming language targeting the AVR hardware architecture.
It generates machine code for AVR processors from modules written in Oberon and stores it in corresponding object files.
For debugging purposes, it also creates a debugging information file as well as an assembly file containing a listing of the generated machine code.
In addition, it stores the interface of each module in a symbol file which is required when other modules import the module.
Programs generated with this compiler require additional runtime support that is stored in the \file{ob\-avr\-run} library file.
\flowgraph{\resource{Oberon\\source code} \ar[r] & \toolbox{obavr} \ar[r] \ar@/l/[d] \ar[rd] & \resource{object file} \\ \variable{ECSIMPORT} \ar[ru] & \resource{symbol\\files} \ar@/r/[u] & \resource{debugging\\information}}
\seeoberon\seeassembly\seeavr\seeobject\seedebugging
}

\providecommand{\obavrtt}{
\toolsection{obavr32} is a compiler for the Oberon programming language targeting the AVR32 hardware architecture.
It generates machine code for AVR32 processors from modules written in Oberon and stores it in corresponding object files.
For debugging purposes, it also creates a debugging information file as well as an assembly file containing a listing of the generated machine code.
In addition, it stores the interface of each module in a symbol file which is required when other modules import the module.
Programs generated with this compiler require additional runtime support that is stored in the \file{ob\-avr32\-run} library file.
\flowgraph{\resource{Oberon\\source code} \ar[r] & \toolbox{obavr32} \ar[r] \ar@/l/[d] \ar[rd] & \resource{object file} \\ \variable{ECSIMPORT} \ar[ru] & \resource{symbol\\files} \ar@/r/[u] & \resource{debugging\\information}}
\seeoberon\seeassembly\seeavrtt\seeobject\seedebugging
}

\providecommand{\obmabk}{
\toolsection{obm68k} is a compiler for the Oberon programming language targeting the M68000 hardware architecture.
It generates machine code for M68000 processors from modules written in Oberon and stores it in corresponding object files.
For debugging purposes, it also creates a debugging information file as well as an assembly file containing a listing of the generated machine code.
In addition, it stores the interface of each module in a symbol file which is required when other modules import the module.
Programs generated with this compiler require additional runtime support that is stored in the \file{ob\-m68k\-run} library file.
\flowgraph{\resource{Oberon\\source code} \ar[r] & \toolbox{obm68k} \ar[r] \ar@/l/[d] \ar[rd] & \resource{object file} \\ \variable{ECSIMPORT} \ar[ru] & \resource{symbol\\files} \ar@/r/[u] & \resource{debugging\\information}}
\seeoberon\seeassembly\seemabk\seeobject\seedebugging
}

\providecommand{\obmibl}{
\toolsection{obmibl} is a compiler for the Oberon programming language targeting the MicroBlaze hardware architecture.
It generates machine code for MicroBlaze processors from modules written in Oberon and stores it in corresponding object files.
For debugging purposes, it also creates a debugging information file as well as an assembly file containing a listing of the generated machine code.
In addition, it stores the interface of each module in a symbol file which is required when other modules import the module.
Programs generated with this compiler require additional runtime support that is stored in the \file{ob\-mibl\-run} library file.
\flowgraph{\resource{Oberon\\source code} \ar[r] & \toolbox{obmibl} \ar[r] \ar@/l/[d] \ar[rd] & \resource{object file} \\ \variable{ECSIMPORT} \ar[ru] & \resource{symbol\\files} \ar@/r/[u] & \resource{debugging\\information}}
\seeoberon\seeassembly\seemibl\seeobject\seedebugging
}

\providecommand{\obmipsa}{
\toolsection{obmips32} is a compiler for the Oberon programming language targeting the MIPS32 hardware architecture.
It generates machine code for MIPS32 processors from modules written in Oberon and stores it in corresponding object files.
For debugging purposes, it also creates a debugging information file as well as an assembly file containing a listing of the generated machine code.
In addition, it stores the interface of each module in a symbol file which is required when other modules import the module.
Programs generated with this compiler require additional runtime support that is stored in the \file{ob\-mips32\-run} library file.
\flowgraph{\resource{Oberon\\source code} \ar[r] & \toolbox{obmips32} \ar[r] \ar@/l/[d] \ar[rd] & \resource{object file} \\ \variable{ECSIMPORT} \ar[ru] & \resource{symbol\\files} \ar@/r/[u] & \resource{debugging\\information}}
\seeoberon\seeassembly\seemips\seeobject\seedebugging
}

\providecommand{\obmipsb}{
\toolsection{obmips64} is a compiler for the Oberon programming language targeting the MIPS64 hardware architecture.
It generates machine code for MIPS64 processors from modules written in Oberon and stores it in corresponding object files.
For debugging purposes, it also creates a debugging information file as well as an assembly file containing a listing of the generated machine code.
In addition, it stores the interface of each module in a symbol file which is required when other modules import the module.
Programs generated with this compiler require additional runtime support that is stored in the \file{ob\-mips64\-run} library file.
\flowgraph{\resource{Oberon\\source code} \ar[r] & \toolbox{obmips64} \ar[r] \ar@/l/[d] \ar[rd] & \resource{object file} \\ \variable{ECSIMPORT} \ar[ru] & \resource{symbol\\files} \ar@/r/[u] & \resource{debugging\\information}}
\seeoberon\seeassembly\seemips\seeobject\seedebugging
}

\providecommand{\obmmix}{
\toolsection{obmmix} is a compiler for the Oberon programming language targeting the MMIX hardware architecture.
It generates machine code for MMIX processors from modules written in Oberon and stores it in corresponding object files.
For debugging purposes, it also creates a debugging information file as well as an assembly file containing a listing of the generated machine code.
In addition, it stores the interface of each module in a symbol file which is required when other modules import the module.
Programs generated with this compiler require additional runtime support that is stored in the \file{ob\-mmix\-run} library file.
\flowgraph{\resource{Oberon\\source code} \ar[r] & \toolbox{obmmix} \ar[r] \ar@/l/[d] \ar[rd] & \resource{object file} \\ \variable{ECSIMPORT} \ar[ru] & \resource{symbol\\files} \ar@/r/[u] & \resource{debugging\\information}}
\seeoberon\seeassembly\seemmix\seeobject\seedebugging
}

\providecommand{\oborok}{
\toolsection{obor1k} is a compiler for the Oberon programming language targeting the OpenRISC 1000 hardware architecture.
It generates machine code for OpenRISC 1000 processors from modules written in Oberon and stores it in corresponding object files.
For debugging purposes, it also creates a debugging information file as well as an assembly file containing a listing of the generated machine code.
In addition, it stores the interface of each module in a symbol file which is required when other modules import the module.
Programs generated with this compiler require additional runtime support that is stored in the \file{ob\-or1k\-run} library file.
\flowgraph{\resource{Oberon\\source code} \ar[r] & \toolbox{obor1k} \ar[r] \ar@/l/[d] \ar[rd] & \resource{object file} \\ \variable{ECSIMPORT} \ar[ru] & \resource{symbol\\files} \ar@/r/[u] & \resource{debugging\\information}}
\seeoberon\seeassembly\seeorok\seeobject\seedebugging
}

\providecommand{\obppca}{
\toolsection{obppc32} is a compiler for the Oberon programming language targeting the PowerPC hardware architecture.
It generates machine code for PowerPC processors from modules written in Oberon and stores it in corresponding object files.
The compiler generates machine code for the 32-bit operating mode defined by the PowerPC architecture.
For debugging purposes, it also creates a debugging information file as well as an assembly file containing a listing of the generated machine code.
In addition, it stores the interface of each module in a symbol file which is required when other modules import the module.
Programs generated with this compiler require additional runtime support that is stored in the \file{ob\-ppc32\-run} library file.
\flowgraph{\resource{Oberon\\source code} \ar[r] & \toolbox{obppc32} \ar[r] \ar@/l/[d] \ar[rd] & \resource{object file} \\ \variable{ECSIMPORT} \ar[ru] & \resource{symbol\\files} \ar@/r/[u] & \resource{debugging\\information}}
\seeoberon\seeassembly\seeppc\seeobject\seedebugging
}

\providecommand{\obppcb}{
\toolsection{obppc64} is a compiler for the Oberon programming language targeting the PowerPC hardware architecture.
It generates machine code for PowerPC processors from modules written in Oberon and stores it in corresponding object files.
The compiler generates machine code for the 64-bit operating mode defined by the PowerPC architecture.
For debugging purposes, it also creates a debugging information file as well as an assembly file containing a listing of the generated machine code.
In addition, it stores the interface of each module in a symbol file which is required when other modules import the module.
Programs generated with this compiler require additional runtime support that is stored in the \file{ob\-ppc64\-run} library file.
\flowgraph{\resource{Oberon\\source code} \ar[r] & \toolbox{obppc64} \ar[r] \ar@/l/[d] \ar[rd] & \resource{object file} \\ \variable{ECSIMPORT} \ar[ru] & \resource{symbol\\files} \ar@/r/[u] & \resource{debugging\\information}}
\seeoberon\seeassembly\seeppc\seeobject\seedebugging
}

\providecommand{\obrisc}{
\toolsection{obrisc} is a compiler for the Oberon programming language targeting the RISC hardware architecture.
It generates machine code for RISC processors from modules written in Oberon and stores it in corresponding object files.
For debugging purposes, it also creates a debugging information file as well as an assembly file containing a listing of the generated machine code.
In addition, it stores the interface of each module in a symbol file which is required when other modules import the module.
Programs generated with this compiler require additional runtime support that is stored in the \file{ob\-risc\-run} library file.
\flowgraph{\resource{Oberon\\source code} \ar[r] & \toolbox{obrisc} \ar[r] \ar@/l/[d] \ar[rd] & \resource{object file} \\ \variable{ECSIMPORT} \ar[ru] & \resource{symbol\\files} \ar@/r/[u] & \resource{debugging\\information}}
\seeoberon\seeassembly\seerisc\seeobject\seedebugging
}

\providecommand{\obwasm}{
\toolsection{obwasm} is a compiler for the Oberon programming language targeting the WebAssembly architecture.
It generates machine code for WebAssembly targets from modules written in Oberon and stores it in corresponding object files.
For debugging purposes, it also creates a debugging information file as well as an assembly file containing a listing of the generated machine code.
In addition, it stores the interface of each module in a symbol file which is required when other modules import the module.
Programs generated with this compiler require additional runtime support that is stored in the \file{ob\-wasm\-run} library file.
\flowgraph{\resource{Oberon\\source code} \ar[r] & \toolbox{obwasm} \ar[r] \ar@/l/[d] \ar[rd] & \resource{object file} \\ \variable{ECSIMPORT} \ar[ru] & \resource{symbol\\files} \ar@/r/[u] & \resource{debugging\\information}}
\seeoberon\seeassembly\seewasm\seeobject\seedebugging
}

% converter tools

\providecommand{\dbgdwarf}{
\toolsection{dbgdwarf} is a DWARF debugging information converter tool.
It converts debugging information into the DWARF debugging data format and stores it in corresponding object files~\cite{dwarffile}.
The resulting debugging object files can be combined with runtime support that creates Executable and Linking Format (ELF) files~\cite{elffile}.
\flowgraph{\resource{debugging\\information} \ar[r] & \toolbox{dbgdwarf} \ar[r] & \resource{debugging\\object file}}
\seeobject\seedebugging
}

% assembler tools

\providecommand{\asmprint}{
\toolsection{asmprint} is a pretty printer for generic assembly code.
It reformats generic assembly code and writes it to the standard output stream.
\flowgraph{\resource{generic assembly\\source code} \ar[r] & \toolbox{asmprint} \ar[r] & \resource{reformatted\\source code}}
\seeassembly
}

\providecommand{\amdaasm}{
\toolsection{amd16asm} is an assembler for the AMD64 hardware architecture.
It translates assembly code into machine code for AMD64 processors and stores it in corresponding object files.
By default, the assembler generates machine code for the 16-bit operating mode defined by the AMD64 architecture.
\flowgraph{\resource{AMD16 assembly\\source code} \ar[r] & \toolbox{amd16asm} \ar[r] & \resource{object file}}
\seeassembly\seeamd\seeobject
}

\providecommand{\amdadism}{
\toolsection{amd16dism} is a disassembler for the AMD64 hardware architecture.
It translates machine code from object files targeting AMD64 processors into assembly code and writes it to the standard output stream.
It assumes that the machine code was generated for the 16-bit operating mode defined by the AMD64 architecture.
\flowgraph{\resource{object file} \ar[r] & \toolbox{amd16dism} \ar[r] & \resource{disassembly\\listing}}
\seeassembly\seeamd\seeobject
}

\providecommand{\amdbasm}{
\toolsection{amd32asm} is an assembler for the AMD64 hardware architecture.
It translates assembly code into machine code for AMD64 processors and stores it in corresponding object files.
By default, the assembler generates machine code for the 32-bit operating mode defined by the AMD64 architecture.
\flowgraph{\resource{AMD32 assembly\\source code} \ar[r] & \toolbox{amd32asm} \ar[r] & \resource{object file}}
\seeassembly\seeamd\seeobject
}

\providecommand{\amdbdism}{
\toolsection{amd32dism} is a disassembler for the AMD64 hardware architecture.
It translates machine code from object files targeting AMD64 processors into assembly code and writes it to the standard output stream.
It assumes that the machine code was generated for the 32-bit operating mode defined by the AMD64 architecture.
\flowgraph{\resource{object file} \ar[r] & \toolbox{amd32dism} \ar[r] & \resource{disassembly\\listing}}
\seeassembly\seeamd\seeobject
}

\providecommand{\amdcasm}{
\toolsection{amd64asm} is an assembler for the AMD64 hardware architecture.
It translates assembly code into machine code for AMD64 processors and stores it in corresponding object files.
By default, the assembler generates machine code for the 64-bit operating mode defined by the AMD64 architecture.
\flowgraph{\resource{AMD64 assembly\\source code} \ar[r] & \toolbox{amd64asm} \ar[r] & \resource{object file}}
\seeassembly\seeamd\seeobject
}

\providecommand{\amdcdism}{
\toolsection{amd64dism} is a disassembler for the AMD64 hardware architecture.
It translates machine code from object files targeting AMD64 processors into assembly code and writes it to the standard output stream.
It assumes that the machine code was generated for the 64-bit operating mode defined by the AMD64 architecture.
\flowgraph{\resource{object file} \ar[r] & \toolbox{amd64dism} \ar[r] & \resource{disassembly\\listing}}
\seeassembly\seeamd\seeobject
}

\providecommand{\armaasm}{
\toolsection{arma32asm} is an assembler for the ARM hardware architecture.
It translates assembly code into machine code for ARM processors executing A32 instructions and stores it in corresponding object files.
\flowgraph{\resource{ARM A32 assembly\\source code} \ar[r] & \toolbox{arma32asm} \ar[r] & \resource{object file}}
\seeassembly\seearm\seeobject
}

\providecommand{\armadism}{
\toolsection{arma32dism} is a disassembler for the ARM hardware architecture.
It translates machine code from object files targeting ARM processors executing A32 instructions into assembly code and writes it to the standard output stream.
\flowgraph{\resource{object file} \ar[r] & \toolbox{arma32dism} \ar[r] & \resource{disassembly\\listing}}
\seeassembly\seearm\seeobject
}

\providecommand{\armbasm}{
\toolsection{arma64asm} is an assembler for the ARM hardware architecture.
It translates assembly code into machine code for ARM processors executing A64 instructions and stores it in corresponding object files.
\flowgraph{\resource{ARM A64 assembly\\source code} \ar[r] & \toolbox{arma64asm} \ar[r] & \resource{object file}}
\seeassembly\seearm\seeobject
}

\providecommand{\armbdism}{
\toolsection{arma64dism} is a disassembler for the ARM hardware architecture.
It translates machine code from object files targeting ARM processors executing A64 instructions into assembly code and writes it to the standard output stream.
\flowgraph{\resource{object file} \ar[r] & \toolbox{arma64dism} \ar[r] & \resource{disassembly\\listing}}
\seeassembly\seearm\seeobject
}

\providecommand{\armcasm}{
\toolsection{armt32asm} is an assembler for the ARM hardware architecture.
It translates assembly code into machine code for ARM processors executing T32 instructions and stores it in corresponding object files.
\flowgraph{\resource{ARM T32 assembly\\source code} \ar[r] & \toolbox{armt32asm} \ar[r] & \resource{object file}}
\seeassembly\seearm\seeobject
}

\providecommand{\armcdism}{
\toolsection{armt32dism} is a disassembler for the ARM hardware architecture.
It translates machine code from object files targeting ARM processors executing T32 instructions into assembly code and writes it to the standard output stream.
\flowgraph{\resource{object file} \ar[r] & \toolbox{armt32dism} \ar[r] & \resource{disassembly\\listing}}
\seeassembly\seearm\seeobject
}

\providecommand{\avrasm}{
\toolsection{avrasm} is an assembler for the AVR hardware architecture.
It translates assembly code into machine code for AVR processors and stores it in corresponding object files.
The identifiers \texttt{RXL}, \texttt{RXH}, \texttt{RYL}, \texttt{RYH}, \texttt{RZL}, and \texttt{RZH} are predefined and name the corresponding registers.
The identifiers \texttt{SPL} and \texttt{SPH} are also predefined and evaluate to the address of the corresponding registers.
\flowgraph{\resource{AVR assembly\\source code} \ar[r] & \toolbox{avrasm} \ar[r] & \resource{object file}}
\seeassembly\seeavr\seeobject
}

\providecommand{\avrdism}{
\toolsection{avrdism} is a disassembler for the AVR hardware architecture.
It translates machine code from object files targeting AVR processors into assembly code and writes it to the standard output stream.
\flowgraph{\resource{object file} \ar[r] & \toolbox{avrdism} \ar[r] & \resource{disassembly\\listing}}
\seeassembly\seeavr\seeobject
}

\providecommand{\avrttasm}{
\toolsection{avr32asm} is an assembler for the AVR32 hardware architecture.
It translates assembly code into machine code for AVR32 processors and stores it in corresponding object files.
\flowgraph{\resource{AVR32 assembly\\source code} \ar[r] & \toolbox{avr32asm} \ar[r] & \resource{object file}}
\seeassembly\seeavrtt\seeobject
}

\providecommand{\avrttdism}{
\toolsection{avr32dism} is a disassembler for the AVR32 hardware architecture.
It translates machine code from object files targeting AVR32 processors into assembly code and writes it to the standard output stream.
\flowgraph{\resource{object file} \ar[r] & \toolbox{avr32dism} \ar[r] & \resource{disassembly\\listing}}
\seeassembly\seeavrtt\seeobject
}

\providecommand{\mabkasm}{
\toolsection{m68kasm} is an assembler for the M68000 hardware architecture.
It translates assembly code into machine code for M68000 processors and stores it in corresponding object files.
\flowgraph{\resource{68000 assembly\\source code} \ar[r] & \toolbox{m68kasm} \ar[r] & \resource{object file}}
\seeassembly\seemabk\seeobject
}

\providecommand{\mabkdism}{
\toolsection{m68kdism} is a disassembler for the M68000 hardware architecture.
It translates machine code from object files targeting M68000 processors into assembly code and writes it to the standard output stream.
\flowgraph{\resource{object file} \ar[r] & \toolbox{m68kdism} \ar[r] & \resource{disassembly\\listing}}
\seeassembly\seemabk\seeobject
}

\providecommand{\miblasm}{
\toolsection{miblasm} is an assembler for the MicroBlaze hardware architecture.
It translates assembly code into machine code for MicroBlaze processors and stores it in corresponding object files.
\flowgraph{\resource{MicroBlaze assembly\\source code} \ar[r] & \toolbox{miblasm} \ar[r] & \resource{object file}}
\seeassembly\seemibl\seeobject
}

\providecommand{\mibldism}{
\toolsection{mibldism} is a disassembler for the MicroBlaze hardware architecture.
It translates machine code from object files targeting MicroBlaze processors into assembly code and writes it to the standard output stream.
\flowgraph{\resource{object file} \ar[r] & \toolbox{mibldism} \ar[r] & \resource{disassembly\\listing}}
\seeassembly\seemibl\seeobject
}

\providecommand{\mipsaasm}{
\toolsection{mips32asm} is an assembler for the MIPS32 hardware architecture.
It translates assembly code into machine code for MIPS32 processors and stores it in corresponding object files.
\flowgraph{\resource{MIPS32 assembly\\source code} \ar[r] & \toolbox{mips32asm} \ar[r] & \resource{object file}}
\seeassembly\seemips\seeobject
}

\providecommand{\mipsadism}{
\toolsection{mips32dism} is a disassembler for the MIPS32 hardware architecture.
It translates machine code from object files targeting MIPS32 processors into assembly code and writes it to the standard output stream.
\flowgraph{\resource{object file} \ar[r] & \toolbox{mips32dism} \ar[r] & \resource{disassembly\\listing}}
\seeassembly\seemips\seeobject
}

\providecommand{\mipsbasm}{
\toolsection{mips64asm} is an assembler for the MIPS64 hardware architecture.
It translates assembly code into machine code for MIPS64 processors and stores it in corresponding object files.
\flowgraph{\resource{MIPS64 assembly\\source code} \ar[r] & \toolbox{mips64asm} \ar[r] & \resource{object file}}
\seeassembly\seemips\seeobject
}

\providecommand{\mipsbdism}{
\toolsection{mips64dism} is a disassembler for the MIPS64 hardware architecture.
It translates machine code from object files targeting MIPS64 processors into assembly code and writes it to the standard output stream.
\flowgraph{\resource{object file} \ar[r] & \toolbox{mips64dism} \ar[r] & \resource{disassembly\\listing}}
\seeassembly\seemips\seeobject
}

\providecommand{\mmixasm}{
\toolsection{mmixasm} is an assembler for the MMIX hardware architecture.
It translates assembly code into machine code for MMIX processors and stores it in corresponding object files.
The names of all special registers are predefined and evaluate to the corresponding number.
\flowgraph{\resource{MMIX assembly\\source code} \ar[r] & \toolbox{mmixasm} \ar[r] & \resource{object file}}
\seeassembly\seemmix\seeobject
}

\providecommand{\mmixdism}{
\toolsection{mmixdism} is a disassembler for the MMIX hardware architecture.
It translates machine code from object files targeting MMIX processors into assembly code and writes it to the standard output stream.
\flowgraph{\resource{object file} \ar[r] & \toolbox{mmixdism} \ar[r] & \resource{disassembly\\listing}}
\seeassembly\seemmix\seeobject
}

\providecommand{\orokasm}{
\toolsection{or1kasm} is an assembler for the OpenRISC 1000 hardware architecture.
It translates assembly code into machine code for OpenRISC 1000 processors and stores it in corresponding object files.
\flowgraph{\resource{OpenRISC 1000 assembly\\source code} \ar[r] & \toolbox{or1kasm} \ar[r] & \resource{object file}}
\seeassembly\seeorok\seeobject
}

\providecommand{\orokdism}{
\toolsection{or1kdism} is a disassembler for the OpenRISC 1000 hardware architecture.
It translates machine code from object files targeting OpenRISC 1000 processors into assembly code and writes it to the standard output stream.
\flowgraph{\resource{object file} \ar[r] & \toolbox{or1kdism} \ar[r] & \resource{disassembly\\listing}}
\seeassembly\seeorok\seeobject
}

\providecommand{\ppcaasm}{
\toolsection{ppc32asm} is an assembler for the PowerPC hardware architecture.
It translates assembly code into machine code for PowerPC processors and stores it in corresponding object files.
By default, the assembler generates machine code for the 32-bit operating mode defined by the PowerPC architecture.
\flowgraph{\resource{PowerPC assembly\\source code} \ar[r] & \toolbox{ppc32asm} \ar[r] & \resource{object file}}
\seeassembly\seeppc\seeobject
}

\providecommand{\ppcadism}{
\toolsection{ppc32dism} is a disassembler for the PowerPC hardware architecture.
It translates machine code from object files targeting PowerPC processors into assembly code and writes it to the standard output stream.
It assumes that the machine code was generated for the 32-bit operating mode defined by the PowerPC architecture.
\flowgraph{\resource{object file} \ar[r] & \toolbox{ppc32dism} \ar[r] & \resource{disassembly\\listing}}
\seeassembly\seeppc\seeobject
}

\providecommand{\ppcbasm}{
\toolsection{ppc64asm} is an assembler for the PowerPC hardware architecture.
It translates assembly code into machine code for PowerPC processors and stores it in corresponding object files.
By default, the assembler generates machine code for the 64-bit operating mode defined by the PowerPC architecture.
\flowgraph{\resource{PowerPC assembly\\source code} \ar[r] & \toolbox{ppc64asm} \ar[r] & \resource{object file}}
\seeassembly\seeppc\seeobject
}

\providecommand{\ppcbdism}{
\toolsection{ppc64dism} is a disassembler for the PowerPC hardware architecture.
It translates machine code from object files targeting PowerPC processors into assembly code and writes it to the standard output stream.
It assumes that the machine code was generated for the 64-bit operating mode defined by the PowerPC architecture.
\flowgraph{\resource{object file} \ar[r] & \toolbox{ppc64dism} \ar[r] & \resource{disassembly\\listing}}
\seeassembly\seeppc\seeobject
}

\providecommand{\riscasm}{
\toolsection{riscasm} is an assembler for the RISC hardware architecture.
It translates assembly code into machine code for RISC processors and stores it in corresponding object files.
The names of all special registers are predefined and evaluate to the corresponding number.
\flowgraph{\resource{RISC assembly\\source code} \ar[r] & \toolbox{riscasm} \ar[r] & \resource{object file}}
\seeassembly\seerisc\seeobject
}

\providecommand{\riscdism}{
\toolsection{riscdism} is a disassembler for the RISC hardware architecture.
It translates machine code from object files targeting RISC processors into assembly code and writes it to the standard output stream.
\flowgraph{\resource{object file} \ar[r] & \toolbox{riscdism} \ar[r] & \resource{disassembly\\listing}}
\seeassembly\seerisc\seeobject
}

\providecommand{\wasmasm}{
\toolsection{wasmasm} is an assembler for the WebAssembly architecture.
It translates assembly code into machine code for WebAssembly targets and stores it in corresponding object files.
The names of all special registers are predefined and evaluate to the corresponding number.
\flowgraph{\resource{WebAssembly assembly\\source code} \ar[r] & \toolbox{wasmasm} \ar[r] & \resource{object file}}
\seeassembly\seewasm\seeobject
}

\providecommand{\wasmdism}{
\toolsection{wasmdism} is a disassembler for the WebAssembly architecture.
It translates machine code from object files targeting WebAssembly targets into assembly code and writes it to the standard output stream.
\flowgraph{\resource{object file} \ar[r] & \toolbox{wasmdism} \ar[r] & \resource{disassembly\\listing}}
\seeassembly\seewasm\seeobject
}

% linker tools

\providecommand{\linklib}{
\toolsection{linklib} is an object file combiner.
It creates a static library file by combining all object files given to it into a single one.
\flowgraph{\resource{object files} \ar[r] & \toolbox{linklib} \ar[r] & \resource{library file}}
\seeobject
}

\providecommand{\linkbin}{
\toolsection{linkbin} is a linker for plain binary files.
It links all object files given to it into a single image and stores it in a binary file that begins with the first linked section.
It also creates a map file that lists the address, type, name and size of all used sections.
The filename extension of the resulting binary file can be specified by putting it into a constant data section called \texttt{\_extension}.
\flowgraph{\resource{object files} \ar[r] & \toolbox{linkbin} \ar[r] \ar[d] & \resource{binary file} \\ & \resource{map file}}
\seeobject
}

\providecommand{\linkmem}{
\toolsection{linkmem} is a linker for plain binary files partitioned into random-access and read-only memory.
It links all object files given to it into two distinct images, one for data sections and one for code and constant data sections, and stores each image in a binary file that begins with the first linked section of the corresponding type.
It also creates a map file that lists the address, type, name and size of all used sections.
\flowgraph{\resource{object files} \ar[r] & \toolbox{linkmem} \ar[r] \ar[d] & \resource{RAM file/\\ROM file} \\ & \resource{map file}}
\seeobject
}

\providecommand{\linkprg}{
\toolsection{linkprg} is a linker for GEMDOS executable files.
It links all object files given to it into a single image and stores the image in an Atari GEMDOS executable file~\cite{gemdosfile}.
It also creates a map file that lists the address relative to the text segment, type, name and size of all used sections.
The filename extension of the resulting executable file can be specified by putting it into a constant data section called \texttt{\_extension}.
The GEMDOS executable file format requires all patch patterns of absolute link patches to consist of four full bitmasks with descending offsets.
\flowgraph{\resource{object files} \ar[r] & \toolbox{linkprg} \ar[r] \ar[d] & \resource{executable file} \\ & \resource{map file}}
\seeobject
}

\providecommand{\linkhex}{
\toolsection{linkhex} is a linker for Intel HEX files.
It links all code sections of the object files given to it into single image and stores the image in an Intel HEX file~\cite{hexfile} that begins with the first linked section.
It also creates a map file that lists the address, type, name and size of all used sections.
\flowgraph{\resource{object files} \ar[r] & \toolbox{linkhex} \ar[r] \ar[d] & \resource{HEX file} \\ & \resource{map file}}
\seeobject
}

\providecommand{\mapsearch}{
\toolsection{mapsearch} is a debugging tool.
It searches map files generated by linker tools for the name of a binary section that encompasses a memory address read from the standard input stream.
If additionally provided with one or more object files, it also stores an excerpt thereof in a separate object file called map search result which only contains the identified binary section for disassembling purposes.
\flowgraph{& \resource{map files/\\object files} \ar[d] \\ \resource{memory\\address} \ar[r] & \toolbox{mapsearch} \ar[r] \ar[d] & \resource{section name/\\relative offset} \\ & \resource{object file\\excerpt}}
\seeobject
}

\renewcommand{\seedocumentation}{}

\startchapter{Generic Documentation}{Generic Documentation Generation}{documentation}
{This \documentation{} describes the generic documentation facility used by the \ecs{} in order
to automatically generate consistent documentations from source code written in different programming languages.}

\epigraph{Learn as much by writing as by reading.}{John Dalberg-Acton}

\section{Introduction}

Almost all programming languages allow programmers to annotate their code with additional notes and comments.
Oftentimes these annotations are part of the interface exposed to users of libraries or applications.
The \ecs{} provides a facility for extracting this information in order to automatically generate detailed and appealing documentations about this interface.
This framework is called \emph{generic documentation generation}\index{Generic documentation generation}\index{Documentation generation}\index{Generation of documentation}.

The core of this framework consists of a generic representation of the program interface as well as the source code annotations provided by the programmer.
The \ecs{} provides so-called \emph{documentation extractors}\index{Documentation extractors}\index{Extractors} for some of the supported programming languages.
They extract the structure and comments of source code and create a consistent documentation of it using the generic documentation representation.
So called \emph{documentation formatters}\index{Documentation formatters}\index{Formatters} on the other hand are able to transform the generic representation of this documentation into documents of different formats.
Figure~\ref{fig:docdataflow} visualizes the role and data flow of the generic documentation representation in-between extractors and formatters.

\begin{figure}
\flowgraph{
\resource{source code} \ar[d] & \resource{source code} \ar[d] & \resource{source code} \ar[d] \\
\converter{Documentation\\Extractor\\for programming\\language \textit{A}} \ar[rd] & \converter{Documentation\\Extractor\\for programming\\language \textit{B}} \ar[d] & \converter{Documentation\\Extractor\\for programming\\language \textit{C}} \ar[ld] \\
& \resource{generic\\documentation\\representation} \ar[ld] \ar[d] \ar[rd] \\
\converter{Documentation\\Formatter\\for document\\format \textit{X}} \ar[d] & \converter{Documentation\\Formatter\\for document\\format \textit{Y}} \ar[d] & \converter{Documentation\\Formatter\\for document\\format \textit{Z}} \ar[d] \\
\resource{formatted\\document} & \resource{formatted\\document} & \resource{formatted\\document} \\
}\caption{Extracting and formatting generic documentation}
\label{fig:docdataflow}
\end{figure}

In addition to the automatically extracted information, the framework is also able to format the annotations given by the programmer.
For this purpose, the framework defines a lightweight markup language which is an extension of the Creole markup language used for formatting wikis~\cite{sauer2007}.
Table~\ref{tab:docmarkup} summarizes all elements of this markup language and shows how they are formatted.

\markuptable

\section{Generic Documentation Representation}

Generic documentations are represented using a hierarchical structure of \emph{documents}\index{Documents}, \emph{articles}\index{Articles}, \emph{paragraphs}\index{Paragraphs}, and \emph{texts}\index{Texts}.
Figure~\ref{fig:docstructure} visualizes this hierarchy.

\begin{figure}
\flowgraph{
& & \resource{generic\\documentation} \ar[ld] \ar[d] \ar[rd] \\
\cdots & \resource{document} & \resource{document} \ar[ld] \ar[d] \ar[rd] & \resource{document} & \cdots \\
\cdots & \resource{article} & \resource{article} \ar[ld] \ar[d] \ar[rd] & \resource{article} & \cdots \\
\cdots & \resource{paragraph} & \resource{paragraph} \ar[ld] \ar[d] \ar[rd] & \resource{paragraph} & \cdots \\
\cdots & \resource{text} & \resource{text} & \resource{text} & \cdots \\
}\caption{Structure of generic documentations}
\label{fig:docstructure}
\end{figure}

The remainder of this section describes the syntax of the markup language that is used to represent generic documentations.
The complete syntax specification is given in Figure~\ref{fig:docmarkupsyntax}.

\begin{figure}
\centering\ifbook\small\fi\setlength{\grammarparsep}{0ex}
\begin{minipage}{34em}\begin{grammar}
<Documentation> = <Documents>$\opt$ \par
<Documents> = <Document> $\mid$ <Documents> <Document> \par
<Document> = <Description> <Articles>$\opt$ \par
<Description> = <Paragraphs>$\opt$ \par
<Articles> = <Article> $\mid$ <Articles> <Article> \par
<Article> = <Article-Marker> <Title> <Paragraphs>$\opt$ \par
<Article-Marker> = "@" $\mid$ "@@" $\mid$ "@@@" $\mid$ "@@@@" $\mid$ \ldots \par
<Title> = <Text> \par
<Paragraph> = <Text-Block> $\mid$ <Heading> $\mid$ <List-Item> $\mid$ <Code-Block> $\mid$ <Table> $\mid$ <Line> \par
<Text-Block> = <Text> \par
<Heading> = <Heading-Marker> <Text> \par
<Heading-Marker> = "=" $\mid$ "==" $\mid$ "===" \par
<List-Item> = <Item-Marker> <Text> \par
<Item-Marker> = <Number-Marker> $\mid$ <Bullet-Marker> \par
<Number-Marker> = "#" $\mid$ "##" $\mid$ "###" \par
<Bullet-Marker> = "*" $\mid$ "**" $\mid$ "***" \par
<Code-Block> = "{{{" any text "}}}" \par
<Table> = <Rows> \par
<Rows> = <Row> $\mid$ <Rows> <Row> \par
<Row> = <Cells> \par
<Cells> = <Cell> $\mid$ <Cells> <Cell> \par
<Cell> = <Cell-Marker> <Text> \par
<Cell-Marker> = "|" $\mid$ "|=" \par
<Line> = "--------" \par
<Text> = <Text-Elements>$\opt$ \par
<Text-Elements> = <Text-Element> $\mid$ <Text-Elements> <Text-Element> \par
<Text-Element> = <Default> $\mid$ <Italic> $\mid$ <Bold> $\mid$ <Link> $\mid$ <URL> $\mid$ <Label> $\mid$ <Code> $\mid$ <Line-Break> \par
<Default> = word \par
<Italic> = "//" <Text> "//" \par
<Bold> = "**" <Text> "**" \par
<Link> = "[[" <Target> "]]" $\mid$ "[[" <Target> "|" <Text> "]]" \par
<Target> = word \par
<URL> = "[[" url "]]" $\mid$ "[[" url "|" <Text> "]]" \par
<Label> = "<""<" <Target> ">"">" \par
<Code> = "{{{" <Text> "}}}" \par
<Line-Break> = "\\\\" \par
\end{grammar}\end{minipage}
\caption{Syntax of the generic documentation markup language}
\label{fig:docmarkupsyntax}
\end{figure}

\subsection{Documents}

Each document consists of a general description of its contents and a list of articles.
Documents are represented according to the following syntax:

\begin{quote}\begin{grammar}
<Documentation> = <Documents>$\opt$ \par
<Documents> = <Document> $\mid$ <Documents> <Document> \par
<Document> = <Description> <Articles>$\opt$ \par
<Description> = <Paragraphs>$\opt$ \par
\end{grammar}\end{quote}

If the documentation consists of several documents, the paragraphs of their descriptions are merged and placed at the beginning of the generated documentation.

\subsection{Articles}

Articles allow giving more detailed information and correspond to structured sections in traditional documents.
They are represented according to the following syntax:

\begin{quote}\begin{grammar}
<Articles> = <Article> $\mid$ <Articles> <Article> \par
<Article> = <Article-Marker> <Title> <Paragraphs>$\opt$ \par
<Article-Marker> = "@" $\mid$ "@@" $\mid$ "@@@" $\mid$ "@@@@" $\mid$ \ldots \par
<Title> = <Text> \par
\end{grammar}\end{quote}

The title of an article corresponds to the caption of a section within a traditional document.
In generated documentations, they are additionally listed in the table of contents.
The number of at signs in the marker of an article specifies the corresponding nesting level of the article.
The difference of the nesting levels of two consecutive articles within the same document may not be higher than one.

\subsection{Paragraphs}

Paragraphs are text blocks that have a specific type which specifies how the text is formatted.
They are represented according to the following syntax:

\begin{quote}\begin{grammar}
<Paragraph> = <Text-Block> $\mid$ <Heading> $\mid$ <List-Item> $\mid$ <Code-Block> $\mid$ <Table> $\mid$ <Line> \par
\end{grammar}\end{quote}

The type of a paragraph affects the formatting of its contents and may be one of the following:

\begin{itemize}

\item Text Block\nopagebreak

Plain text blocks are represented according to the following syntax:

\begin{quote}\begin{grammar}
<Text-Block> = <Text> \par
\end{grammar}\end{quote}

\item Heading\nopagebreak

Headings allow authors of an article to logically structure its contents.
They are represented according to the following syntax:

\begin{quote}\begin{grammar}
<Heading> = <Heading-Marker> <Text> \par
<Heading-Marker> = "=" $\mid$ "==" $\mid$ "===" \par
\end{grammar}\end{quote}

The number of equal signs in the marker of a heading specifies the corresponding nesting level of the heading.
The difference of the nesting levels of two consecutive headings within the same article may not be higher than one.

\item List Item\nopagebreak

List items may be ordered or unordered and are represented according to the following syntax:

\begin{quote}\begin{grammar}
<List-Item> = <Item-Marker> <Text> \par
<Item-Marker> = <Number-Marker> $\mid$ <Bullet-Marker> \par
<Number-Marker> = "#" $\mid$ "##" $\mid$ "###" \par
<Bullet-Marker> = "*" $\mid$ "**" $\mid$ "***" \par
\end{grammar}\end{quote}

The number of symbols in the marker of a list item specifies the corresponding nesting level of the item within a list.
The difference of the nesting levels of two consecutive items within the same list may not be higher than one.

\item Code Blocks\nopagebreak

Code text blocks allow marking passages of source code within the documentation.
They are represented according to the following syntax:

\begin{quote}\begin{grammar}
<Code-Block> = "{{{" any text "}}}" \par
\end{grammar}\end{quote}

The code in-between the two markers is not subject to any further processing and is always formatted as is.

\item Table\nopagebreak

Tables consist of cell rows and are represented according to the following syntax:

\begin{quote}\begin{grammar}
<Table> = <Rows> \par
<Rows> = <Row> $\mid$ <Rows> <Row> \par
<Row> = <Cells> \par
<Cells> = <Cell> $\mid$ <Cells> <Cell> \par
<Cell> = <Cell-Marker> <Text> \par
<Cell-Marker> = "|" $\mid$ "|=" \par
\end{grammar}\end{quote}

An equal sign immediately following a pipe symbol in the marker of a cell turns the cell into a header.

\item Line\nopagebreak

Lines are formatted as horizontal rules and are represented according to the following syntax:

\begin{quote}\begin{grammar}
<Line> = "--------" \par
\end{grammar}\end{quote}

\end{itemize}

Paragraphs and articles may be separated by any number of new-line characters.
A single new-line character between two text lines of the same paragraph is considered as a white-space character.

\subsection{Texts}

The contents of most paragraphs as well as the title of an article consist of text.
A text is a list of so-called \emph{text elements}\index{Text elements} which specify the formatting of the text.
Text elements may also contain text themselves according to the following syntax:

\begin{quote}\begin{grammar}
<Text> = <Text-Elements>$\opt$ \par
<Text-Elements> = <Text-Element> $\mid$ <Text-Elements> <Text-Element> \par
<Text-Element> = <Default> $\mid$ <Italic> $\mid$ <Bold> $\mid$ <Link> $\mid$ <URL> $\mid$ <Label> $\mid$ <Code> $\mid$ <Line-Break> \par
\end{grammar}\end{quote}

The formatting style of a text element may be one of the following:

\begin{itemize}

\item Default\nopagebreak

The default style does not change the current formatting.
Default text elements are just strings according to the following syntax:

\begin{quote}\begin{grammar}
<Default> = word \par
\end{grammar}\end{quote}

\item Italic\nopagebreak

Italic text passages are represented according to the following syntax:

\begin{quote}\begin{grammar}
<Italic> = "//" <Text> "//" \par
\end{grammar}\end{quote}

\item Bold\nopagebreak

Bold text passages are represented according to the following syntax:

\begin{quote}\begin{grammar}
<Bold> = "**" <Text> "**" \par
\end{grammar}\end{quote}

\item Link\nopagebreak

Internal links point to a label defined elsewhere in the generic documentation and are represented according to the following syntax:

\begin{quote}\begin{grammar}
<Link> = "[[" <Target> "]]" $\mid$ "[[" <Target> "|" <Text> "]]" \par
<Target> = word \par
\end{grammar}\end{quote}

The formatted text of an internal link equals its target if the alternative text is not given.
A link is only valid, if the target names a label that was actually defined in one of the documents of a generic documentation.

\item URL\nopagebreak

External links are URLs that point to external resources.
They are represented according to the following syntax:

\begin{quote}\begin{grammar}
<URL> = "[[" url "]]" $\mid$ "[[" url "|" <Text> "]]" \par
\end{grammar}\end{quote}

The formatted text of an external link equals its target if the alternative text is not given.
Currently, only the URL schemes "http:" and "https:" are recognized.
Freestanding URLs are automatically turned into external links.

\item Label\nopagebreak

Labels name targets for internal links and are represented according to the following syntax:

\begin{quote}\begin{grammar}
<Label> = "<""<" <Target> ">"">" \par
<Target> = word \par
\end{grammar}\end{quote}

\item Code\nopagebreak

Code text passages are represented according to the following syntax:

\begin{quote}\begin{grammar}
<Code> = "{{{" <Text> "}}}" \par
\end{grammar}\end{quote}

\item Line Breaks\nopagebreak

Line breaks request the beginning of a new line.
They are represented according to the following syntax:

\begin{quote}\begin{grammar}
<Line-Break> = "\\\\" \par
\end{grammar}\end{quote}

\end{itemize}

\section{Documentation Formatters}

This section lists all document formats supported by the \ecs{} and describes how the corresponding documentation formatters process generic documentations.

\subsection{HTML Documentations}

The HTML documentation formatter merges several generic documentations into a single HTML document.
This file consists of a merged description of all documentations and all of their articles.
In addition, it inserts a linked table of contents using the titles and hierarchical structure of all articles.
All articles are formatted the same way and begin with a horizontal rule.
At the end of each article, a link to the top of the HTML page is inserted.

\subsection{Latex Documentations}

The Latex documentation formatter merges several generic documentations into a single Latex document.
It uses the article document style and uses all flavors of the section and paragraph commands to map the structure of all articles.
Headings are formatted using the starred versions of these commands.
After the merged description of all generic documentations, the formatter issues a table of contents command.
Internal and external links are formatted using the \texttt{hyperref} package.

The generated Latex code does not use built-in commands directly but defines and calls wrappers for most of them instead.
This allows including the file in larger documents which may provide user-defined versions of these commands beforehand.

\section{Generic Documentation Tools}

The \ecs{} provides several different tools that generate or process generic documentations.
While real documentation generators use generic documentations just as an internal and therefore inaccessible representation,
these tools allow inspecting and modifying the actual documentation that is processed for debugging purposes.
\interface\renewcommand{\debuggingtool}{}

\docprint
\doccheck
\dochtml
\doclatex

\section{Documentation Generation Tools}

The \ecs{} provides several different documentation generators that process commented source code files
written in miscellaneous programming languages in order to automatically extract documentations like user guides or reference manuals out of them.
Figure~\ref{fig:docgendataflow} shows the typical data flow within these documentation generators.
For concrete information about how these tools extract the structure of the source code, users have to refer to the documentation of the corresponding programming language.
\interface

\begin{figure}
\flowgraph{
\resource{source code} \ar[d] \\
\converter{Lexer} \ar[d] \\
\resource{tokens} \ar[d] \\
\converter{Parser} \ar[d] \\
\resource{abstract\\syntax tree} \ar[d] \\
\converter{Semantic\\Checker} \ar[d] \\
\resource{attributed\\syntax tree} \ar[d] \\
\converter{Documentation\\Extractor} \ar[d] \\
\resource{generic\\documentation} \ar[d] \\
\converter{Documentation\\Formatter} \ar[d] \\
\resource{formatted\\document} \\
}\caption{Data flow within a typical documentation generator}
\label{fig:docgendataflow}
\end{figure}

\cppdoc
\cpphtml
\cpplatex
\obdoc
\obhtml
\oblatex

\concludechapter

% Extensions to the Eigen Compiler Suite
% Copyright (C) Florian Negele

% This file is part of the Eigen Compiler Suite.

% Permission is granted to copy, distribute and/or modify this document
% under the terms of the GNU Free Documentation License, Version 1.3
% or any later version published by the Free Software Foundation.

% You should have received a copy of the GNU Free Documentation License
% along with the ECS.  If not, see <https://www.gnu.org/licenses/>.

% Generic documentation utilities
% Copyright (C) Florian Negele

% This file is part of the Eigen Compiler Suite.

% Permission is granted to copy, distribute and/or modify this document
% under the terms of the GNU Free Documentation License, Version 1.3
% or any later version published by the Free Software Foundation.

% You should have received a copy of the GNU Free Documentation License
% along with the ECS.  If not, see <https://www.gnu.org/licenses/>.

\providecommand{\cpp}{C\texttt{++}}
\providecommand{\opt}{_\mathit{opt}}
\providecommand{\tool}[1]{\texttt{#1}}
\providecommand{\version}{Version 0.0.40}
\providecommand{\resource}[1]{*++\txt{#1}}
\providecommand{\ecs}{Eigen Compiler Suite}
\providecommand{\changed}[1]{\underline{#1}}
\providecommand{\toolbox}[1]{\converter{#1}}
\providecommand{\file}{}\renewcommand{\file}[1]{\texttt{#1}}
\providecommand{\alignright}{\hfill\linebreak[0]\hspace*{\fill}}
\providecommand{\converter}[1]{*++[F][F*:white][F,:gray]\txt{#1}}
\providecommand{\documentation}{\ifbook chapter\else document\fi}
\providecommand{\Documentation}{\ifbook Chapter\else Document\fi}
\providecommand{\variable}[1]{\resource{\texttt{\small#1}\\variable}}
\providecommand{\documentationref}[2]{\ifbook\ref{#1}\else``\href{#1}{#2}''~\cite{#1}\fi}
\providecommand{\objfile}[1]{\texttt{#1}\index[runtime]{#1 object file@\texttt{#1} object file}}
\providecommand{\libfile}[1]{\texttt{#1}\index[runtime]{#1 library file@\texttt{#1} library file}}
\providecommand{\epigraph}[2]{\ifbook\begin{quote}\flushright\textit{#1}\par--- #2\end{quote}\fi}
\providecommand{\environmentvariable}[1]{\texttt{#1}\index{Environment variables!#1@\texttt{#1}}}
\providecommand{\environment}[1]{\texttt{#1}\index[environment]{#1 environment@\texttt{#1} environment}}
\providecommand{\toolsection}{}\renewcommand{\toolsection}[1]{\subsection{#1}\label{\prefix:#1}\tool{#1}}
\providecommand{\instruction}{}\renewcommand{\instruction}[2]{\noindent\qquad\pdftooltip{\texttt{#1}}{#2}\refstepcounter{instruction}\par}
\providecommand{\flowgraph}{}\renewcommand{\flowgraph}[1]{\par\sffamily\begin{displaymath}\xymatrix@=4ex{#1}\end{displaymath}\normalfont\par}
\providecommand{\instructionset}{}\renewcommand{\instructionset}[4]{\setcounter{instruction}{0}\begin{multicols}{\ifbook#3\else#4\fi}[{\captionof{table}[#2]{#2 (\ref*{#1:instructions}~instructions)}\label{tab:#1set}\vspace{-2ex}}]\footnotesize\raggedcolumns\input{#1.set}\label{#1:instructions}\end{multicols}}

\providecommand{\gpl}{GNU General Public License}
\providecommand{\rse}{ECS Runtime Support Exception}
\providecommand{\fdl}{\href{https://www.gnu.org/licenses/fdl.html}{GNU Free Documentation License}}

\providecommand{\docbegin}{}
\providecommand{\docend}{}
\providecommand{\doclabel}[1]{\hypertarget{#1}}
\providecommand{\doclink}[2]{\hyperlink{#1}{#2}}
\providecommand{\docsection}[3]{\hypertarget{#1}{\subsection{#2}}\label{sec:#1}\index[library]{#2@#3}}
\providecommand{\docsectionstar}[1]{}
\providecommand{\docsubbegin}{\begin{description}}
\providecommand{\docsubend}{\end{description}}
\providecommand{\docsubsection}[3]{\item[\hypertarget{#1}{#2}]\index[library]{#2@#3}}
\providecommand{\docsubsectionstar}[1]{\smallskip}
\providecommand{\docsubsubsection}[3]{\docsubsection{#1}{#2}{#3}}
\providecommand{\docsubsubsectionstar}[1]{}
\providecommand{\docsubsubsubsection}[3]{}
\providecommand{\docsubsubsubsectionstar}[1]{}
\providecommand{\doctable}{}

\providecommand{\debuggingtool}{}\renewcommand{\debuggingtool}{This tool is provided for debugging purposes.
It allows exposing and modifying an internal data structure that is usually not accessible.
}

\providecommand{\interface}{All tools accept command-line arguments which are taken as names of plain text files containing the source code.
If no arguments are provided, the standard input stream is used instead.
Output files are generated in the current working directory and have the same name as the input file being processed whereas the filename extension gets replaced by an appropriate suffix.
\seeinterface
}

\providecommand{\license}{\noindent Copyright \copyright{} Florian Negele\par\medskip\noindent
Permission is granted to copy, distribute and/or modify this document under the terms of the
\fdl{}, Version 1.3 or any later version published by the \href{https://fsf.org/}{Free Software Foundation}.
}

\providecommand{\ecslogosurface}{
\fill[darkgray] (0,0,0) -- (0,0,3) -- (0,3,3) -- (0,3,1) -- (0,4,1) -- (0,4,3) -- (0,5,3) -- (0,5,0) -- (0,2,0) -- (0,2,2) -- (0,1,2) -- (0,1,0) -- cycle;
\fill[gray] (0,5,0) -- (0,5,3) -- (1,5,3) -- (1,5,1) -- (2,5,1) -- (2,5,3) -- (3,5,3) -- (3,5,0) -- cycle;
\fill[lightgray] (0,0,0) -- (0,1,0) -- (2,1,0) -- (2,4,0) -- (1,4,0) -- (1,3,0) -- (2,3,0) -- (2,2,0) -- (0,2,0) -- (0,5,0) -- (3,5,0) -- (3,0,0) -- cycle;
\begin{scope}[line width=0.5]
\begin{scope}[gray]
\draw (0,0,0) -- (0,1,0);
\draw (2,1,0) -- (2,2,0);
\draw (0,1,2) -- (0,2,2);
\draw (0,2,0) -- (0,5,0);
\draw (2,3,0) -- (2,4,0);
\end{scope}
\begin{scope}[lightgray]
\draw (0,1,0) -- (0,1,2);
\draw (0,3,1) -- (0,3,3);
\draw (0,5,0) -- (0,5,3);
\draw (2,5,1) -- (2,5,3);
\end{scope}
\begin{scope}[white]
\draw (0,1,0) -- (2,1,0);
\draw (1,3,0) -- (2,3,0);
\draw (0,5,0) -- (3,5,0);
\end{scope}
\end{scope}
}

\providecommand{\ecslogo}[1]{
\begin{tikzpicture}[scale={(#1)/((sin(45)+cos(45))*3cm)},x={({-cos(45)*1cm},{sin(45)*sin(30)*1cm})},y={({0cm},{(cos(30)*1cm})},z={({sin(45)*1cm},{cos(45)*sin(30)*1cm})}]
\begin{scope}[darkgray,line width=1]
\draw (0,0,0) -- (0,0,3) -- (0,3,3) -- (2,3,3) -- (2,5,3) -- (3,5,3) -- (3,5,0) -- (3,0,0) -- cycle;
\draw (0,3,1) -- (0,4,1) -- (0,4,3) -- (0,5,3) -- (1,5,3) -- (1,5,1) -- (2,5,1);
\draw (1,3,0) -- (1,4,0) -- (2,4,0);
\end{scope}
\fill[darkgray] (2,0,0) -- (2,0,3) -- (2,5,3) -- (2,5,1) -- (2,4,1) -- (2,4,0) -- cycle;
\fill[lightgray] (2,0,2) -- (0,0,2) -- (0,2,2) -- (2,2,2) -- cycle;
\fill[gray] (0,1,0) -- (2,1,0) -- (2,1,2) -- (0,1,2) -- cycle;
\fill[gray] (0,3,1) -- (0,3,3) -- (2,3,3) -- (2,3,0) -- (1,3,0) -- (1,3,1) -- cycle;
\ecslogosurface
\end{tikzpicture}
}

\providecommand{\shadowedecslogo}[3]{
\begin{tikzpicture}[scale={(#1)/((sin(#2)+cos(#2))*3cm)},x={({-cos(#2)*1cm},{sin(#2)*sin(#3)*1cm})},y={({0cm},{(cos(#3)*1cm})},z={({sin(#2)*1cm},{cos(#2)*sin(#3)*1cm})}]
\shade[top color=lightgray!50!white,bottom color=white,middle color=lightgray!50!white] (0,0,0) -- (3,0,0) -- (3,{-0.5-3*sin(#2)*sin(#3)/cos(#3)},0) -- (0,-0.5,0) -- cycle;
\shade[top color=darkgray!50!gray,bottom color=white,middle color=darkgray!50!white] (0,0,0) -- (0,0,3) -- (0,{-0.5-3*cos(#2)*sin(#3)/cos(#3)},3) -- (0,-0.5,0) -- cycle;
\begin{scope}[y={({(cos(#2)+sin(#2))*0.5cm},{(cos(#2)*sin(#3)-sin(#2)*sin(#3))*0.5cm})}]
\useasboundingbox (3,0,0) -- (0,0,0) -- (0,0,3);
\shade[left color=darkgray!80!black,right color=lightgray,middle color=gray] (0,0,0) -- (0,1,0) -- (0,1,0.5) -- (0,2,0) -- (0,5,0) -- (0,5,3) -- (1,5,3) -- (1,4,3) -- (1,4,2.5) -- (1,3,3) -- (2,5,3) -- (3,5,3) -- (3,0,3) -- cycle;
\clip (0,0,0) -- (0,0,3) -- ({-3*sin(#2)/cos(#2)},0,0) -- cycle;
\shade[left color=darkgray,right color=lightgray!50!gray] (0,0,0) -- (0,1,0) -- (0,1,0.5) -- (0,2,0) -- (0,5,0) -- (0,5,3) -- (1,5,3) -- (1,4,3) -- (1,4,2.5) -- (1,3,3) -- (2,5,3) -- (3,5,3) -- (3,0,3) -- cycle;
\end{scope}
\shade[left color=darkgray,right color=darkgray!80!black] (2,0,0) -- (2,0,3) -- (2,5,3) -- (2,5,1) -- (2,4,1) -- (2,4,0) -- cycle;
\shade[left color=darkgray!90!black,right color=gray!80!darkgray] (2,0,2) -- (0,0,2) -- (0,2,2) -- (2,2,2) -- cycle;
\shade[top color=darkgray!90!black,bottom color=gray!80!darkgray] (0,1,0) -- (2,1,0) -- (2,1,2) -- (0,1,2) -- cycle;
\shade[top color=darkgray!90!black,bottom color=gray!80!darkgray] (0,3,1) -- (0,3,3) -- (2,3,3) -- (2,3,0) -- (1,3,0) -- (1,3,1) -- cycle;
\fill[gray] (2,1,0) -- (1.5,1,0.5) -- (0,1,0.5) -- (0,1,0) -- cycle;
\fill[gray] (1,3,2) -- (0.5,3,2) -- (0.5,3,3) -- (1,3,3) -- cycle;
\fill[gray] (2,3,0) -- (1.5,3,0.5) -- (1,3,0.5) -- (1,3,0) -- cycle;
\ecslogosurface
\end{tikzpicture}
}

\providecommand{\cpplogo}[1]{
\begin{tikzpicture}[scale=(#1)/512em]
\fill[gray] (435.2794,398.7159) -- (247.1911,507.3075) .. controls (236.3563,513.5642) and (218.6240,513.5642) .. (207.7892,507.3075) -- (19.7009,398.7159) .. controls (8.8646,392.4606) and (0.0000,377.1043) .. (0.0000,364.5924) -- (0.0000,147.4076) .. controls (0.8430,132.8363) and (8.2856,120.7683) .. (19.7009,113.2842) -- (207.7892,4.6926) .. controls (218.6240,-1.5642) and (236.3564,-1.5642) .. (247.1911,4.6926) -- (435.2794,113.2842) .. controls (447.5273,121.4304) and (454.4987,133.6918) .. (454.9803,147.4076) -- (454.9803,364.5924) .. controls (454.5404,377.7571) and (446.6566,391.0351) .. (435.2794,398.7159) -- cycle(75.8301,255.9993) .. controls (74.9389,404.0881) and (273.2892,469.4783) .. (358.8263,331.8769) -- (293.1917,293.8965) .. controls (253.5702,359.4301) and (155.1909,335.9977) .. (151.6601,255.9993) .. controls (152.7204,182.2703) and (249.4137,148.0211) .. (293.1961,218.1065) -- (358.8308,180.1276) .. controls (283.4477,49.2645) and (79.6318,96.3470) .. (75.8301,255.9993) -- cycle(379.1503,247.5747) -- (362.2982,247.5747) -- (362.2982,230.7226) -- (345.4490,230.7226) -- (345.4490,247.5747) -- (328.5969,247.5747) -- (328.5969,264.4254) -- (345.4490,264.4254) -- (345.4490,281.2759) -- (362.2982,281.2759) -- (362.2982,264.4254) -- (379.1503,264.4254) -- cycle(442.3420,247.5747) -- (425.4899,247.5747) -- (425.4899,230.7226) -- (408.6408,230.7226) -- (408.6408,247.5747) -- (391.7886,247.5747) -- (391.7886,264.4254) -- (408.6408,264.4254) -- (408.6408,281.2759) -- (425.4899,281.2759) -- (425.4899,264.4254) -- (442.3420,264.4254) -- cycle;
\end{tikzpicture}
}

\providecommand{\fallogo}[1]{
\begin{tikzpicture}[scale=(#1)/512em]
\fill[gray] (185.7774,0.0000) .. controls (200.4486,15.9798) and (226.8966,8.7148) .. (235.0426,31.5836) .. controls (249.5297,58.0598) and (247.9581,97.9161) .. (280.3335,110.9762) .. controls (309.1690,120.3496) and (337.8406,104.2727) .. (366.5753,103.9379) .. controls (373.4449,111.5171) and (379.2885,128.2574) .. (383.9755,108.9744) .. controls (396.6979,102.5615) and (437.2808,107.6681) .. (426.9652,124.3252) .. controls (408.9822,121.0785) and (412.4742,146.0729) .. (426.5192,131.4996) .. controls (433.8413,120.8489) and (465.1541,126.5522) .. (441.9067,135.7950) .. controls (396.1879,157.7478) and (344.1112,161.5079) .. (298.5528,183.5702) .. controls (277.7471,193.5198) and (284.6941,218.7163) .. (285.2127,236.9640) .. controls (292.3599,316.2826) and (307.3929,394.6311) .. (317.1198,473.6154) .. controls (329.0637,505.4736) and (292.1195,528.5004) .. (265.9183,511.2761) .. controls (237.9284,499.2462) and (237.3684,465.2681) .. (230.9102,439.9421) .. controls (218.6692,374.3397) and (215.6307,306.9662) .. (198.1732,242.3977) .. controls (183.1379,232.7444) and (164.4245,256.0298) .. (149.0430,261.4799) .. controls (116.9328,279.2585) and (87.1822,308.5851) .. (48.2293,307.8914) .. controls (21.3220,306.9037) and (-15.9107,281.8761) .. (7.2921,252.7908) .. controls (29.7799,220.6177) and (67.5177,204.3028) .. (100.9287,185.9449) .. controls (130.8217,170.8906) and (161.1548,156.5903) .. (191.0278,141.5847) .. controls (196.1738,120.0520) and (186.6049,95.2409) .. (186.8382,72.4353) .. controls (185.5234,48.4204) and (183.1700,23.9341) .. (185.7774,0.0000) -- cycle;
\end{tikzpicture}
}

\providecommand{\oblogo}[1]{
\begin{tikzpicture}[scale=(#1)/512em]
\fill[gray] (160.3865,208.9117) .. controls (154.0879,214.6478) and (149.0735,221.2409) .. (145.4125,228.5384) .. controls (184.8790,248.4273) and (234.7122,269.8787) .. (297.5493,291.8782) .. controls (300.3943,281.4769) and (300.9552,268.7619) .. (300.4023,255.2389) .. controls (248.9909,244.7891) and (200.0310,225.9279) .. (160.3865,208.9117) -- cycle(225.7398,392.6996) .. controls (308.0209,392.1716) and (359.3326,345.9277) .. (368.7203,285.2098) .. controls (376.6742,197.1784) and (311.7194,141.3342) .. (205.4287,142.1456) .. controls (139.9485,141.4804) and (88.7155,166.1957) .. (73.5775,228.0086) .. controls (52.0297,320.3408) and (123.4078,391.0103) .. (225.7398,392.6996) -- cycle(216.0739,176.4733) .. controls (268.9183,179.2424) and (315.8292,206.5488) .. (312.7454,265.1139) .. controls (313.2769,315.6384) and (286.5993,353.4946) .. (216.6040,355.7934) .. controls (162.4657,355.7934) and (126.0914,317.5023) .. (126.0914,260.5103) .. controls (126.1733,214.2900) and (163.3363,176.2849) .. (216.0739,176.4733) -- cycle(76.4897,189.1754) .. controls (13.1586,147.5631) and (0.0000,119.4207) .. (0.0000,119.4207) -- (90.6499,170.1632) .. controls (85.3004,175.8497) and (80.5994,182.1633) .. (76.4897,189.1754) -- cycle(353.9486,119.3004) -- (402.9482,119.3004) .. controls (427.0025,137.0797) and (450.9893,162.7034) .. (474.9529,191.0213) .. controls (509.3540,228.5339) and (531.3391,294.2091) .. (487.8149,312.1206) .. controls (462.8165,324.7652) and (394.3874,316.8943) .. (373.8912,313.6651) .. controls (379.9291,297.7449) and (383.2899,278.4204) .. (381.4989,257.7214) .. controls (420.3069,248.0321) and (421.9610,218.3461) .. (407.7867,192.6417) .. controls (391.1113,162.4018) and (370.1114,132.9097) .. (353.9486,119.3004) -- cycle;
\end{tikzpicture}
}

\providecommand{\markuptable}{
\begin{table}
\sffamily\centering
\begin{tabular}{@{}lcl@{}}
\toprule
\texttt{//italics//} & $\rightarrow$ & \textit{italics} \\
\midrule
\texttt{**bold**} & $\rightarrow$ & \textbf{bold} \\
\midrule
\texttt{\# ordered list} & & 1 ordered list \\
\texttt{\# second item} & $\rightarrow$ & 2 second item \\
\texttt{\#\# sub item} & & \hspace{1em} 1 sub item \\
\midrule
\texttt{* unordered list} & & $\bullet$ unordered list \\
\texttt{* second item} & $\rightarrow$ & $\bullet$ second item \\
\texttt{** sub item} & & \hspace{1em} $\bullet$ sub item \\
\midrule
\texttt{link to [[label]]} & $\rightarrow$ & link to \underline{label} \\
\midrule
\texttt{<{}<label>{}> definition } & $\rightarrow$ & definition \\
\midrule
\texttt{[[url|link name]]} & $\rightarrow$ & \underline{link name} \\
\midrule\addlinespace
\texttt{= large heading} & & {\Large large heading} \smallskip \\
\texttt{== medium heading} & $\rightarrow$ & {\large medium heading} \\
\texttt{=== small heading} & & small heading \\
\midrule
\texttt{no line break} & & no line break for paragraphs \\
\texttt{for paragraphs} & $\rightarrow$ \\
& & use empty line \\
\texttt{use empty line} \\
\midrule
\texttt{force\textbackslash\textbackslash line break} & $\rightarrow$ & force \\
& & line break \\
\midrule
\texttt{horizontal line} & $\rightarrow$ & horizontal line \\
\texttt{----} & & \hrulefill \\
\midrule
\texttt{|=a|=table|=header} & & \underline{a \enspace table \enspace header} \\
\texttt{|a|table|row} & $\rightarrow$ & a \enspace table \enspace row \\
\texttt{|b|table|row} & & b \enspace table \enspace row \\
\midrule
\texttt{\{\{\{} \\
\texttt{unformatted} & $\rightarrow$ & \texttt{unformatted} \\
\texttt{code} & & \texttt{code} \\
\texttt{\}\}\}} \\
\midrule\addlinespace
\texttt{@ new article} & & {\Large 1.\ new article} \smallskip \\
\texttt{@ second article} & $\rightarrow$ & {\Large 2.\ second article} \smallskip \\
\texttt{@@ sub article} & & {\large 2.1.\ sub article} \\
\bottomrule
\end{tabular}
\normalfont\caption{Elements of the generic documentation markup language}
\label{tab:docmarkup}
\end{table}
}

\providecommand{\startchapter}[4]{
\documentclass[11pt,a4paper]{article}
\usepackage{booktabs}
\usepackage[format=hang,labelfont=bf]{caption}
\usepackage{changepage}
\usepackage[T1]{fontenc}
\usepackage[margin=2cm]{geometry}
\usepackage{hyperref}
\usepackage[american]{isodate}
\usepackage{lmodern}
\usepackage{longtable}
\usepackage{mathptmx}
\usepackage{microtype}
\usepackage[toc]{multitoc}
\usepackage{multirow}
\usepackage[all]{nowidow}
\usepackage{pdfcomment}
\usepackage{syntax}
\usepackage{tikz}
\usepackage[all]{xy}
\hypersetup{pdfborder={0 0 0},bookmarksnumbered=true,pdftitle={\ecs{}: #2},pdfauthor={Florian Negele},pdfsubject={\ecs{}},pdfkeywords={#1}}
\setlength{\grammarindent}{8em}\setlength{\grammarparsep}{0.2ex}
\setlength{\columnsep}{2em}
\newcommand{\prefix}{}
\newcounter{instruction}
\bibliographystyle{unsrt}
\renewcommand{\index}[2][]{}
\renewcommand{\arraystretch}{1.05}
\renewcommand{\floatpagefraction}{0.7}
\renewcommand{\syntleft}{\itshape}\renewcommand{\syntright}{}
\title{\vspace{-5ex}\Huge{\ecs{}}\medskip\hrule}
\author{\huge{#2}}
\date{\medskip\version}
\newif\ifbook\bookfalse
\pagestyle{headings}
\frenchspacing
\begin{document}
\maketitle\thispagestyle{empty}\noindent#4\setlength{\columnseprule}{0.4pt}\tableofcontents\setlength{\columnseprule}{0pt}\vfill\pagebreak[3]\null\vfill\bigskip\noindent
\parbox{\textwidth-4em}{\license The contents of this \documentation{} are part of the \href{manual}{\ecs{} User Manual}~\cite{manual} and correspond to Chapter ``\href{manual\##3}{#1}''.\alignright\mbox{\today}}
\parbox{4em}{\flushright\ecslogo{3em}}
\clearpage
}

\providecommand{\concludechapter}{
\vfill\pagebreak[3]\null\vfill
\thispagestyle{myheadings}\markright{REFERENCES}
\noindent\begin{minipage}{\textwidth}\begin{multicols}{2}[\section*{References}]
\renewcommand{\section}[2]{}\small\bibliography{references}
\end{multicols}\end{minipage}\end{document}
}

\providecommand{\startpresentation}[2]{
\documentclass[14pt,aspectratio=43,usepdftitle=false]{beamer}
\usepackage{booktabs}
\usepackage{etex}
\usepackage{multicol}
\usepackage{tikz}
\usepackage[all]{xy}
\bibliographystyle{unsrt}
\setlength{\columnsep}{1em}
\setlength{\leftmargini}{1em}
\setbeamercolor{title}{fg=black}
\setbeamercolor{structure}{fg=darkgray}
\setbeamercolor{bibliography item}{fg=darkgray}
\setbeamerfont{title}{series=\bfseries}
\setbeamerfont{subtitle}{series=\normalfont}
\setbeamerfont*{frametitle}{parent=title}
\setbeamerfont{block title}{series=\bfseries}
\setbeamerfont*{framesubtitle}{parent=subtitle}
\setbeamersize{text margin left=1em,text margin right=1em}
\setbeamertemplate{navigation symbols}{}
\setbeamertemplate{itemize item}[circle]{}
\setbeamertemplate{bibliography item}[triangle]{}
\setbeamertemplate{bibliography entry author}{\usebeamercolor[fg]{bibliography item}}
\setbeamertemplate{frametitle}{\medskip\usebeamerfont{frametitle}\color{gray}\raisebox{-2.5ex}[0ex][0ex]{\rule{0.1em}{4.5ex}}}
\addtobeamertemplate{frametitle}{}{\hspace{0.4em}\usebeamercolor[fg]{title}\insertframetitle\par\vspace{0.2ex}\hspace{0.5em}\usebeamerfont{framesubtitle}\insertframesubtitle}
\hypersetup{pdfborder={0 0 0},bookmarksnumbered=true,bookmarksopen=true,bookmarksopenlevel=0,pdftitle={\ecs{}: #1},pdfauthor={Florian Negele},pdfsubject={\ecs{}},pdfkeywords={#1}}
\renewcommand{\flowgraph}[1]{\resizebox{\textwidth}{!}{$$\xymatrix{##1}$$}}
\title{\ecs{}\medskip\hrule\medskip}
\institute{\shadowedecslogo{5em}{30}{15}}
\date{\version}
\subtitle{#1}
\begin{document}
\begin{frame}[plain]\titlepage\nocite{manual}\end{frame}
\begin{frame}{Contents}{#1}\begin{center}\tableofcontents\end{center}\end{frame}
}

\providecommand{\concludepresentation}{
\begin{frame}{References}\begin{footnotesize}\setlength{\columnseprule}{0.4pt}\begin{multicols}{2}\bibliography{references}\end{multicols}\end{footnotesize}\end{frame}
\end{document}
}

\providecommand{\startbook}[1]{
\documentclass[10pt,paper=17cm:24cm,DIV=13,twoside=semi,headings=normal,numbers=noendperiod,cleardoublepage=plain]{scrbook}
\usepackage{atveryend}
\usepackage{booktabs}
\usepackage{caption}
\usepackage{changepage}
\usepackage[T1]{fontenc}
\usepackage{imakeidx}
\usepackage{hyperref}
\usepackage[american]{isodate}
\usepackage{lmodern}
\usepackage{longtable}
\usepackage{mathptmx}
\usepackage[final]{microtype}
\usepackage{multicol}
\usepackage{multirow}
\usepackage[all]{nowidow}
\usepackage{pdfcomment}
\usepackage{scrlayer-scrpage}
\usepackage{setspace}
\usepackage{syntax}
\usepackage[eventxtindent=4pt,oddtxtexdent=4pt]{thumbs}
\usepackage{tikz}
\usepackage[all]{xy}
\hyphenation{Micro-Blaze Open-Cores Open-RISC Power-PC}
\hypersetup{pdfborder={0 0 0},bookmarksnumbered=true,bookmarksopen=true,bookmarksopenlevel=0,pdftitle={\ecs{}: #1},pdfauthor={Florian Negele},pdfsubject={\ecs{}},pdfkeywords={#1}}
\setlength{\grammarindent}{8em}\setlength{\grammarparsep}{0.7ex}
\setkomafont{captionlabel}{\usekomafont{descriptionlabel}}
\renewcommand{\arraystretch}{1.05}\setstretch{1.1}
\renewcommand{\chapterformat}{\thechapter\autodot\enskip\raisebox{-1ex}[0ex][0ex]{\color{gray}\rule{0.1em}{3.5ex}}\enskip}
\renewcommand{\startchapter}[4]{\hypertarget{##3}{\chapter{##1}}\label{##3}##4\addthumb{##1}{\LARGE\sffamily\bfseries\thechapter}{white}{gray}\renewcommand{\prefix}{##3}}
\renewcommand{\concludechapter}{\clearpage{\stopthumb\cleardoublepage}}
\renewcommand{\syntleft}{\itshape}\renewcommand{\syntright}{}
\renewcommand{\floatpagefraction}{0.7}
\renewcommand{\partheademptypage}{}
\DeclareMicrotypeAlias{lmss}{cmr}
\newcommand{\prefix}{}
\newcounter{instruction}
\bibliographystyle{unsrt}
\newif\ifbook\booktrue
\makeindex[intoc,title=Index]
\makeindex[intoc,name=tools,title=Index of Tools,columns=3]
\makeindex[intoc,name=library,title=Index of Library Names]
\makeindex[intoc,name=runtime,title=Index of Runtime Support]
\makeindex[intoc,name=environment,title=Index of Target Environments]
\indexsetup{toclevel=chapter,headers={\indexname}{\indexname}}
\frenchspacing
\begin{document}
\pagenumbering{alph}
\begin{titlepage}\centering
\huge\sffamily\null\vfill\textbf{\ecs{}}\bigskip\hrule\bigskip#1
\normalsize\normalfont\vfill\vfill\shadowedecslogo{10em}{30}{15}
\large\vfill\vfill\version
\end{titlepage}
\null\vfill
\thispagestyle{empty}
\noindent\today\par\medskip
\license A copy of this license is included in Appendix~\ref{fdl} on page~\pageref{fdl}.
All product names used herein are for identification purposes only and may be trademarks of their respective companies.
\concludechapter
\frontmatter
\setcounter{tocdepth}{1}
\tableofcontents
\setcounter{tocdepth}{2}
\concludechapter
\listoffigures
\concludechapter
\listoftables
\concludechapter
}

\providecommand{\concludebook}{
\backmatter
\addtocontents{toc}{\protect\setcounter{tocdepth}{-1}}
\phantomsection\addcontentsline{toc}{part}{Bibliography}
\bibliography{references}
\concludechapter
\phantomsection\addcontentsline{toc}{part}{Indexes}
\printindex
\concludechapter
\indexprologue{\label{idx:tools}}
\printindex[tools]
\concludechapter
\printindex[library]
\concludechapter
\indexprologue{\label{idx:runtime}}
\printindex[runtime]
\concludechapter
\indexprologue{\label{idx:environment}}
\printindex[environment]
\concludechapter
\pagestyle{empty}\pagenumbering{Alph}\null\clearpage
\null\vfill\centering\ecslogo{4em}\par\medskip\license
\end{document}
}

% chapter references

\providecommand{\seedocumentationref}{}\renewcommand{\seedocumentationref}[3]{#1, see \Documentation{}~\documentationref{#2}{#3}. }
\providecommand{\seeinterface}{}\renewcommand{\seeinterface}{\ifbook See \Documentation{}~\documentationref{interface}{User Interface} for more information about the common user interface of all of these tools. \fi}
\providecommand{\seeguide}{}\renewcommand{\seeguide}{\seedocumentationref{For basic examples of using some of these tools in practice}{guide}{User Guide}}
\providecommand{\seecpp}{}\renewcommand{\seecpp}{\seedocumentationref{For more information about the \cpp{} programming language and its implementation by the \ecs{}}{cpp}{User Manual for \cpp{}}}
\providecommand{\seefalse}{}\renewcommand{\seefalse}{\seedocumentationref{For more information about the FALSE programming language and its implementation by the \ecs{}}{false}{User Manual for FALSE}}
\providecommand{\seeoberon}{}\renewcommand{\seeoberon}{\seedocumentationref{For more information about the Oberon programming language and its implementation by the \ecs{}}{oberon}{User Manual for Oberon}}
\providecommand{\seeassembly}{}\renewcommand{\seeassembly}{\seedocumentationref{For more information about the generic assembly language and how to use it}{assembly}{Generic Assembly Language Specification}}
\providecommand{\seeamd}{}\renewcommand{\seeamd}{\seedocumentationref{For more information about how the \ecs{} supports the AMD64 hardware architecture}{amd64}{AMD64 Hardware Architecture Support}}
\providecommand{\seearm}{}\renewcommand{\seearm}{\seedocumentationref{For more information about how the \ecs{} supports the ARM hardware architecture}{arm}{ARM Hardware Architecture Support}}
\providecommand{\seeavr}{}\renewcommand{\seeavr}{\seedocumentationref{For more information about how the \ecs{} supports the AVR hardware architecture}{avr}{AVR Hardware Architecture Support}}
\providecommand{\seeavrtt}{}\renewcommand{\seeavrtt}{\seedocumentationref{For more information about how the \ecs{} supports the AVR32 hardware architecture}{avr32}{AVR32 Hardware Architecture Support}}
\providecommand{\seemabk}{}\renewcommand{\seemabk}{\seedocumentationref{For more information about how the \ecs{} supports the M68000 hardware architecture}{m68k}{M68000 Hardware Architecture Support}}
\providecommand{\seemibl}{}\renewcommand{\seemibl}{\seedocumentationref{For more information about how the \ecs{} supports the MicroBlaze hardware architecture}{mibl}{MicroBlaze Hardware Architecture Support}}
\providecommand{\seemips}{}\renewcommand{\seemips}{\seedocumentationref{For more information about how the \ecs{} supports the MIPS32 and MIPS64 hardware architectures}{mips}{MIPS Hardware Architecture Support}}
\providecommand{\seemmix}{}\renewcommand{\seemmix}{\seedocumentationref{For more information about how the \ecs{} supports the MMIX hardware architecture}{mmix}{MMIX Hardware Architecture Support}}
\providecommand{\seeorok}{}\renewcommand{\seeorok}{\seedocumentationref{For more information about how the \ecs{} supports the OpenRISC 1000 hardware architecture}{or1k}{OpenRISC 1000 Hardware Architecture Support}}
\providecommand{\seeppc}{}\renewcommand{\seeppc}{\seedocumentationref{For more information about how the \ecs{} supports the PowerPC hardware architecture}{ppc}{PowerPC Hardware Architecture Support}}
\providecommand{\seerisc}{}\renewcommand{\seerisc}{\seedocumentationref{For more information about how the \ecs{} supports the RISC hardware architecture}{risc}{RISC Hardware Architecture Support}}
\providecommand{\seewasm}{}\renewcommand{\seewasm}{\seedocumentationref{For more information about how the \ecs{} supports the WebAssembly architecture}{wasm}{WebAssembly Architecture Support}}
\providecommand{\seedocumentation}{}\renewcommand{\seedocumentation}{\seedocumentationref{For more information about generic documentations and their generation by the \ecs{}}{documentation}{Generic Documentation Generation}}
\providecommand{\seedebugging}{}\renewcommand{\seedebugging}{\seedocumentationref{For more information about debugging information and its representation}{debugging}{Debugging Information Representation}}
\providecommand{\seecode}{}\renewcommand{\seecode}{\seedocumentationref{For more information about intermediate code and its purpose}{code}{Intermediate Code Representation}}
\providecommand{\seeobject}{}\renewcommand{\seeobject}{\seedocumentationref{For more information about object files and their purpose}{object}{Object File Representation}}

% generic documentation tools

\providecommand{\docprint}{
\toolsection{docprint} is a pretty printer for generic documentations.
It reformats generic documentations and writes it to the standard output stream.
\debuggingtool
\flowgraph{\resource{generic\\documentation} \ar[r] & \toolbox{docprint} \ar[r] & \resource{generic\\documentation}}
\seedocumentation
}

\providecommand{\doccheck}{
\toolsection{doccheck} is a syntactic and semantic checker for generic documentations.
It just performs syntactic and semantic checks on generic documentations and writes its diagnostic messages to the standard error stream.
\debuggingtool
\flowgraph{\resource{generic\\documentation} \ar[r] & \toolbox{doccheck} \ar[r] & \resource{diagnostic\\messages}}
\seedocumentation
}

\providecommand{\dochtml}{
\toolsection{dochtml} is an HTML documentation generator for generic documentations.
It processes several generic documentations and assembles all information therein into an HTML document.
\debuggingtool
\flowgraph{\resource{generic\\documentation} \ar[r] & \toolbox{dochtml} \ar[r] & \resource{HTML\\document}}
\seedocumentation
}

\providecommand{\doclatex}{
\toolsection{doclatex} is a Latex documentation generator for generic documentations.
It processes several generic documentations and assembles all information therein into a Latex document.
\debuggingtool
\flowgraph{\resource{generic\\documentation} \ar[r] & \toolbox{doclatex} \ar[r] & \resource{Latex\\document}}
\seedocumentation
}

% intermediate code tools

\providecommand{\cdcheck}{
\toolsection{cdcheck} is a syntactic and semantic checker for intermediate code.
It just performs syntactic and semantic checks on programs written in intermediate code and writes its diagnostic messages to the standard error stream.
\debuggingtool
\flowgraph{\resource{intermediate\\code} \ar[r] & \toolbox{cdcheck} \ar[r] & \resource{diagnostic\\messages}}
\seeassembly\seecode
}

\providecommand{\cdopt}{
\toolsection{cdopt} is an optimizer for intermediate code.
It performs various optimizations on programs written in intermediate code and writes the result to the standard output stream.
\debuggingtool
\flowgraph{\resource{intermediate\\code} \ar[r] & \toolbox{cdopt} \ar[r] & \resource{optimized\\code}}
\seeassembly\seecode
}

\providecommand{\cdrun}{
\toolsection{cdrun} is an interpreter for intermediate code.
It processes and executes programs written in intermediate code.
The following code sections are predefined and have the usual semantics:
\texttt{abort}, \texttt{\_Exit}, \texttt{fflush}, \texttt{floor}, \texttt{fputc}, \texttt{free}, \texttt{getchar}, \texttt{malloc}, and \texttt{putchar}.
Diagnostic messages about invalid operations include the name of the executed code section and the index of the erroneous instruction.
\debuggingtool
\flowgraph{\resource{intermediate\\code} \ar[r] & \toolbox{cdrun} \ar@/u/[r] & \resource{input/\\output} \ar@/d/[l]}
\seeassembly\seecode
}

\providecommand{\cdamda}{
\toolsection{cdamd16} is a compiler for intermediate code targeting the AMD64 hardware architecture.
It generates machine code for AMD64 processors from programs written in intermediate code and stores it in corresponding object files.
The compiler generates machine code for the 16-bit operating mode defined by the AMD64 architecture.
It also creates a debugging information file as well as an assembly file containing a listing of the generated machine code.
\debuggingtool
\flowgraph{\resource{intermediate\\code} \ar[r] & \toolbox{cdamd16} \ar[r] \ar[d] \ar[rd] & \resource{object file} \\ & \resource{assembly\\listing} & \resource{debugging\\information}}
\seeassembly\seeamd\seeobject\seecode\seedebugging
}

\providecommand{\cdamdb}{
\toolsection{cdamd32} is a compiler for intermediate code targeting the AMD64 hardware architecture.
It generates machine code for AMD64 processors from programs written in intermediate code and stores it in corresponding object files.
The compiler generates machine code for the 32-bit operating mode defined by the AMD64 architecture.
It also creates a debugging information file as well as an assembly file containing a listing of the generated machine code.
\debuggingtool
\flowgraph{\resource{intermediate\\code} \ar[r] & \toolbox{cdamd32} \ar[r] \ar[d] \ar[rd] & \resource{object file} \\ & \resource{assembly\\listing} & \resource{debugging\\information}}
\seeassembly\seeamd\seeobject\seecode\seedebugging
}

\providecommand{\cdamdc}{
\toolsection{cdamd64} is a compiler for intermediate code targeting the AMD64 hardware architecture.
It generates machine code for AMD64 processors from programs written in intermediate code and stores it in corresponding object files.
The compiler generates machine code for the 64-bit operating mode defined by the AMD64 architecture.
It also creates a debugging information file as well as an assembly file containing a listing of the generated machine code.
\debuggingtool
\flowgraph{\resource{intermediate\\code} \ar[r] & \toolbox{cdamd64} \ar[r] \ar[d] \ar[rd] & \resource{object file} \\ & \resource{assembly\\listing} & \resource{debugging\\information}}
\seeassembly\seeamd\seeobject\seecode\seedebugging
}

\providecommand{\cdarma}{
\toolsection{cdarma32} is a compiler for intermediate code targeting the ARM hardware architecture.
It generates machine code for ARM processors executing A32 instructions from programs written in intermediate code and stores it in corresponding object files.
It also creates a debugging information file as well as an assembly file containing a listing of the generated machine code.
\debuggingtool
\flowgraph{\resource{intermediate\\code} \ar[r] & \toolbox{cdarma32} \ar[r] \ar[d] \ar[rd] & \resource{object file} \\ & \resource{assembly\\listing} & \resource{debugging\\information}}
\seeassembly\seearm\seeobject\seecode\seedebugging
}

\providecommand{\cdarmb}{
\toolsection{cdarma64} is a compiler for intermediate code targeting the ARM hardware architecture.
It generates machine code for ARM processors executing A64 instructions from programs written in intermediate code and stores it in corresponding object files.
It also creates a debugging information file as well as an assembly file containing a listing of the generated machine code.
\debuggingtool
\flowgraph{\resource{intermediate\\code} \ar[r] & \toolbox{cdarma64} \ar[r] \ar[d] \ar[rd] & \resource{object file} \\ & \resource{assembly\\listing} & \resource{debugging\\information}}
\seeassembly\seearm\seeobject\seecode\seedebugging
}

\providecommand{\cdarmc}{
\toolsection{cdarmt32} is a compiler for intermediate code targeting the ARM hardware architecture.
It generates machine code for ARM processors without floating-point extension executing T32 instructions from programs written in intermediate code and stores it in corresponding object files.
It also creates a debugging information file as well as an assembly file containing a listing of the generated machine code.
\debuggingtool
\flowgraph{\resource{intermediate\\code} \ar[r] & \toolbox{cdarmt32} \ar[r] \ar[d] \ar[rd] & \resource{object file} \\ & \resource{assembly\\listing} & \resource{debugging\\information}}
\seeassembly\seearm\seeobject\seecode\seedebugging
}

\providecommand{\cdarmcfpe}{
\toolsection{cdarmt32fpe} is a compiler for intermediate code targeting the ARM hardware architecture.
It generates machine code for ARM processors with floating-point extension executing T32 instructions from programs written in intermediate code and stores it in corresponding object files.
It also creates a debugging information file as well as an assembly file containing a listing of the generated machine code.
\debuggingtool
\flowgraph{\resource{intermediate\\code} \ar[r] & \toolbox{cdarmt32fpe} \ar[r] \ar[d] \ar[rd] & \resource{object file} \\ & \resource{assembly\\listing} & \resource{debugging\\information}}
\seeassembly\seearm\seeobject\seecode\seedebugging
}

\providecommand{\cdavr}{
\toolsection{cdavr} is a compiler for intermediate code targeting the AVR hardware architecture.
It generates machine code for AVR processors from programs written in intermediate code and stores it in corresponding object files.
It also creates a debugging information file as well as an assembly file containing a listing of the generated machine code.
\debuggingtool
\flowgraph{\resource{intermediate\\code} \ar[r] & \toolbox{cdavr} \ar[r] \ar[d] \ar[rd] & \resource{object file} \\ & \resource{assembly\\listing} & \resource{debugging\\information}}
\seeassembly\seeavr\seeobject\seecode\seedebugging
}

\providecommand{\cdavrtt}{
\toolsection{cdavr32} is a compiler for intermediate code targeting the AVR32 hardware architecture.
It generates machine code for AVR32 processors from programs written in intermediate code and stores it in corresponding object files.
It also creates a debugging information file as well as an assembly file containing a listing of the generated machine code.
\debuggingtool
\flowgraph{\resource{intermediate\\code} \ar[r] & \toolbox{cdavr32} \ar[r] \ar[d] \ar[rd] & \resource{object file} \\ & \resource{assembly\\listing} & \resource{debugging\\information}}
\seeassembly\seeavrtt\seeobject\seecode\seedebugging
}

\providecommand{\cdmabk}{
\toolsection{cdm68k} is a compiler for intermediate code targeting the M68000 hardware architecture.
It generates machine code for M68000 processors from programs written in intermediate code and stores it in corresponding object files.
It also creates a debugging information file as well as an assembly file containing a listing of the generated machine code.
\debuggingtool
\flowgraph{\resource{intermediate\\code} \ar[r] & \toolbox{cdm68k} \ar[r] \ar[d] \ar[rd] & \resource{object file} \\ & \resource{assembly\\listing} & \resource{debugging\\information}}
\seeassembly\seemabk\seeobject\seecode\seedebugging
}

\providecommand{\cdmibl}{
\toolsection{cdmibl} is a compiler for intermediate code targeting the MicroBlaze hardware architecture.
It generates machine code for MicroBlaze processors from programs written in intermediate code and stores it in corresponding object files.
It also creates a debugging information file as well as an assembly file containing a listing of the generated machine code.
\debuggingtool
\flowgraph{\resource{intermediate\\code} \ar[r] & \toolbox{cdmibl} \ar[r] \ar[d] \ar[rd] & \resource{object file} \\ & \resource{assembly\\listing} & \resource{debugging\\information}}
\seeassembly\seemibl\seeobject\seecode\seedebugging
}

\providecommand{\cdmipsa}{
\toolsection{cdmips32} is a compiler for intermediate code targeting the MIPS32 hardware architecture.
It generates machine code for MIPS32 processors from programs written in intermediate code and stores it in corresponding object files.
It also creates a debugging information file as well as an assembly file containing a listing of the generated machine code.
\debuggingtool
\flowgraph{\resource{intermediate\\code} \ar[r] & \toolbox{cdmips32} \ar[r] \ar[d] \ar[rd] & \resource{object file} \\ & \resource{assembly\\listing} & \resource{debugging\\information}}
\seeassembly\seemips\seeobject\seecode\seedebugging
}

\providecommand{\cdmipsb}{
\toolsection{cdmips64} is a compiler for intermediate code targeting the MIPS64 hardware architecture.
It generates machine code for MIPS64 processors from programs written in intermediate code and stores it in corresponding object files.
It also creates a debugging information file as well as an assembly file containing a listing of the generated machine code.
\debuggingtool
\flowgraph{\resource{intermediate\\code} \ar[r] & \toolbox{cdmips64} \ar[r] \ar[d] \ar[rd] & \resource{object file} \\ & \resource{assembly\\listing} & \resource{debugging\\information}}
\seeassembly\seemips\seeobject\seecode\seedebugging
}

\providecommand{\cdmmix}{
\toolsection{cdmmix} is a compiler for intermediate code targeting the MMIX hardware architecture.
It generates machine code for MMIX processors from programs written in intermediate code and stores it in corresponding object files.
It also creates a debugging information file as well as an assembly file containing a listing of the generated machine code.
\debuggingtool
\flowgraph{\resource{intermediate\\code} \ar[r] & \toolbox{cdmmix} \ar[r] \ar[d] \ar[rd] & \resource{object file} \\ & \resource{assembly\\listing} & \resource{debugging\\information}}
\seeassembly\seemmix\seeobject\seecode\seedebugging
}

\providecommand{\cdorok}{
\toolsection{cdor1k} is a compiler for intermediate code targeting the OpenRISC 1000 hardware architecture.
It generates machine code for OpenRISC 1000 processors from programs written in intermediate code and stores it in corresponding object files.
It also creates a debugging information file as well as an assembly file containing a listing of the generated machine code.
\debuggingtool
\flowgraph{\resource{intermediate\\code} \ar[r] & \toolbox{cdor1k} \ar[r] \ar[d] \ar[rd] & \resource{object file} \\ & \resource{assembly\\listing} & \resource{debugging\\information}}
\seeassembly\seeorok\seeobject\seecode\seedebugging
}

\providecommand{\cdppca}{
\toolsection{cdppc32} is a compiler for intermediate code targeting the PowerPC hardware architecture.
It generates machine code for PowerPC processors from programs written in intermediate code and stores it in corresponding object files.
The compiler generates machine code for the 32-bit operating mode defined by the PowerPC architecture.
It also creates a debugging information file as well as an assembly file containing a listing of the generated machine code.
\debuggingtool
\flowgraph{\resource{intermediate\\code} \ar[r] & \toolbox{cdppc32} \ar[r] \ar[d] \ar[rd] & \resource{object file} \\ & \resource{assembly\\listing} & \resource{debugging\\information}}
\seeassembly\seeppc\seeobject\seecode\seedebugging
}

\providecommand{\cdppcb}{
\toolsection{cdppc64} is a compiler for intermediate code targeting the PowerPC hardware architecture.
It generates machine code for PowerPC processors from programs written in intermediate code and stores it in corresponding object files.
The compiler generates machine code for the 64-bit operating mode defined by the PowerPC architecture.
It also creates a debugging information file as well as an assembly file containing a listing of the generated machine code.
\debuggingtool
\flowgraph{\resource{intermediate\\code} \ar[r] & \toolbox{cdppc64} \ar[r] \ar[d] \ar[rd] & \resource{object file} \\ & \resource{assembly\\listing} & \resource{debugging\\information}}
\seeassembly\seeppc\seeobject\seecode\seedebugging
}

\providecommand{\cdrisc}{
\toolsection{cdrisc} is a compiler for intermediate code targeting the RISC hardware architecture.
It generates machine code for RISC processors from programs written in intermediate code and stores it in corresponding object files.
It also creates a debugging information file as well as an assembly file containing a listing of the generated machine code.
\debuggingtool
\flowgraph{\resource{intermediate\\code} \ar[r] & \toolbox{cdrisc} \ar[r] \ar[d] \ar[rd] & \resource{object file} \\ & \resource{assembly\\listing} & \resource{debugging\\information}}
\seeassembly\seerisc\seeobject\seecode\seedebugging
}

\providecommand{\cdwasm}{
\toolsection{cdwasm} is a compiler for intermediate code targeting the WebAssembly architecture.
It generates machine code for WebAssembly targets from programs written in intermediate code and stores it in corresponding object files.
It also creates a debugging information file as well as an assembly file containing a listing of the generated machine code.
\debuggingtool
\flowgraph{\resource{intermediate\\code} \ar[r] & \toolbox{cdwasm} \ar[r] \ar[d] \ar[rd] & \resource{object file} \\ & \resource{assembly\\listing} & \resource{debugging\\information}}
\seeassembly\seewasm\seeobject\seecode\seedebugging
}

% C++ tools

\providecommand{\cppprep}{
\toolsection{cppprep} is a preprocessor for the \cpp{} programming language.
It preprocesses source code according to the rules of \cpp{} and writes it to the standard output stream.
Only the macro names \texttt{\_\_DATE\_\_}, \texttt{\_\_FILE\_\_}, \texttt{\_\_LINE\_\_}, and \texttt{\_\_TIME\_\_} are predefined.
\flowgraph{\resource{\cpp{} or other\\source code} \ar[r] & \toolbox{cppprep} \ar[r] & \resource{preprocessed\\source code} \\ & \variable{ECSINCLUDE} \ar[u]}
\seecpp
}

\providecommand{\cppprint}{
\toolsection{cppprint} is a pretty printer for the \cpp{} programming language.
It reformats the source code of \cpp{} programs and writes it to the standard output stream.
\flowgraph{\resource{\cpp{}\\source code} \ar[r] & \toolbox{cppprint} \ar[r] & \resource{reformatted\\source code} \\ & \variable{ECSINCLUDE} \ar[u]}
\seecpp
}

\providecommand{\cppcheck}{
\toolsection{cppcheck} is a syntactic and semantic checker for the \cpp{} programming language.
It just performs syntactic and semantic checks on \cpp{} programs and writes its diagnostic messages to the standard error stream.
\flowgraph{\resource{\cpp{}\\source code} \ar[r] & \toolbox{cppcheck} \ar[r] & \resource{diagnostic\\messages} \\ & \variable{ECSINCLUDE} \ar[u]}
\seecpp
}

\providecommand{\cppdump}{
\toolsection{cppdump} is a serializer for the \cpp{} programming language.
It dumps the complete internal representation of programs written in \cpp{} into an XML document.
\debuggingtool
\flowgraph{\resource{\cpp{}\\source code} \ar[r] & \toolbox{cppdump} \ar[r] & \resource{internal\\representation} \\ & \variable{ECSINCLUDE} \ar[u]}
\seecpp
}

\providecommand{\cpprun}{
\toolsection{cpprun} is an interpreter for the \cpp{} programming language.
It processes and executes programs written in \cpp{}.
The macro \texttt{\_\_run\_\_} is predefined in order to enable programmers to identify this tool while interpreting.
\flowgraph{\resource{\cpp{}\\source code} \ar[r] & \toolbox{cpprun} \ar@/u/[r] & \resource{input/\\output} \ar@/d/[l] \\ & \variable{ECSINCLUDE} \ar[u]}
\seecpp
}

\providecommand{\cppdoc}{
\toolsection{cppdoc} is a generic documentation generator for the \cpp{} programming language.
It processes several \cpp{} source files and assembles all information therein into a generic documentation.
\debuggingtool
\flowgraph{\resource{\cpp{}\\source code} \ar[r] & \toolbox{cppdoc} \ar[r] & \resource{generic\\documentation} \\ & \variable{ECSINCLUDE} \ar[u]}
\seecpp\seedocumentation
}

\providecommand{\cpphtml}{
\toolsection{cpphtml} is an HTML documentation generator for the \cpp{} programming language.
It processes several \cpp{} source files and assembles all information therein into an HTML document.
\flowgraph{\resource{\cpp{}\\source code} \ar[r] & \toolbox{cpphtml} \ar[r] & \resource{HTML\\document} \\ & \variable{ECSINCLUDE} \ar[u]}
\seecpp\seedocumentation
}

\providecommand{\cpplatex}{
\toolsection{cpplatex} is a Latex documentation generator for the \cpp{} programming language.
It processes several \cpp{} source files and assembles all information therein into a Latex document.
\flowgraph{\resource{\cpp{}\\source code} \ar[r] & \toolbox{cpplatex} \ar[r] & \resource{Latex\\document} \\ & \variable{ECSINCLUDE} \ar[u]}
\seecpp\seedocumentation
}

\providecommand{\cppcode}{
\toolsection{cppcode} is an intermediate code generator for the \cpp{} programming language.
It generates intermediate code from programs written in \cpp{} and stores it in corresponding assembly files.
The macro \texttt{\_\_code\_\_} is predefined in order to enable programmers to identify this tool while generating intermediate code.
Programs generated with this tool require additional runtime support that is stored in the \file{cpp\-code\-run} library file.
\debuggingtool
\flowgraph{\resource{\cpp{}\\source code} \ar[r] & \toolbox{cppcode} \ar[r] & \resource{intermediate\\code} \\ & \variable{ECSINCLUDE} \ar[u]}
\seecpp\seeassembly\seecode
}

\providecommand{\cppamda}{
\toolsection{cppamd16} is a compiler for the \cpp{} programming language targeting the AMD64 hardware architecture.
It generates machine code for AMD64 processors from programs written in \cpp{} and stores it in corresponding object files.
The compiler generates machine code for the 16-bit operating mode defined by the AMD64 architecture.
For debugging purposes, it also creates a debugging information file as well as an assembly file containing a listing of the generated machine code.
The macro \texttt{\_\_amd16\_\_} is predefined in order to enable programmers to identify this tool and its target architecture while compiling.
Programs generated with this compiler require additional runtime support that is stored in the \file{cpp\-amd16\-run} library file.
\flowgraph{\resource{\cpp{}\\source code} \ar[r] & \toolbox{cppamd16} \ar[r] \ar[d] \ar[rd] & \resource{object file} \\ \variable{ECSINCLUDE} \ar[ru] & \resource{debugging\\information} & \resource{assembly\\listing}}
\seecpp\seeassembly\seeamd\seeobject\seedebugging
}

\providecommand{\cppamdb}{
\toolsection{cppamd32} is a compiler for the \cpp{} programming language targeting the AMD64 hardware architecture.
It generates machine code for AMD64 processors from programs written in \cpp{} and stores it in corresponding object files.
The compiler generates machine code for the 32-bit operating mode defined by the AMD64 architecture.
For debugging purposes, it also creates a debugging information file as well as an assembly file containing a listing of the generated machine code.
The macro \texttt{\_\_amd32\_\_} is predefined in order to enable programmers to identify this tool and its target architecture while compiling.
Programs generated with this compiler require additional runtime support that is stored in the \file{cpp\-amd32\-run} library file.
\flowgraph{\resource{\cpp{}\\source code} \ar[r] & \toolbox{cppamd32} \ar[r] \ar[d] \ar[rd] & \resource{object file} \\ \variable{ECSINCLUDE} \ar[ru] & \resource{debugging\\information} & \resource{assembly\\listing}}
\seecpp\seeassembly\seeamd\seeobject\seedebugging
}

\providecommand{\cppamdc}{
\toolsection{cppamd64} is a compiler for the \cpp{} programming language targeting the AMD64 hardware architecture.
It generates machine code for AMD64 processors from programs written in \cpp{} and stores it in corresponding object files.
The compiler generates machine code for the 64-bit operating mode defined by the AMD64 architecture.
For debugging purposes, it also creates a debugging information file as well as an assembly file containing a listing of the generated machine code.
The macro \texttt{\_\_amd64\_\_} is predefined in order to enable programmers to identify this tool and its target architecture while compiling.
Programs generated with this compiler require additional runtime support that is stored in the \file{cpp\-amd64\-run} library file.
\flowgraph{\resource{\cpp{}\\source code} \ar[r] & \toolbox{cppamd64} \ar[r] \ar[d] \ar[rd] & \resource{object file} \\ \variable{ECSINCLUDE} \ar[ru] & \resource{debugging\\information} & \resource{assembly\\listing}}
\seecpp\seeassembly\seeamd\seeobject\seedebugging
}

\providecommand{\cpparma}{
\toolsection{cpparma32} is a compiler for the \cpp{} programming language targeting the ARM hardware architecture.
It generates machine code for ARM processors executing A32 instructions from programs written in \cpp{} and stores it in corresponding object files.
For debugging purposes, it also creates a debugging information file as well as an assembly file containing a listing of the generated machine code.
The macro \texttt{\_\_arma32\_\_} is predefined in order to enable programmers to identify this tool and its target architecture while compiling.
Programs generated with this compiler require additional runtime support that is stored in the \file{cpp\-arma32\-run} library file.
\flowgraph{\resource{\cpp{}\\source code} \ar[r] & \toolbox{cpparma32} \ar[r] \ar[d] \ar[rd] & \resource{object file} \\ \variable{ECSINCLUDE} \ar[ru] & \resource{debugging\\information} & \resource{assembly\\listing}}
\seecpp\seeassembly\seearm\seeobject\seedebugging
}

\providecommand{\cpparmb}{
\toolsection{cpparma64} is a compiler for the \cpp{} programming language targeting the ARM hardware architecture.
It generates machine code for ARM processors executing A64 instructions from programs written in \cpp{} and stores it in corresponding object files.
For debugging purposes, it also creates a debugging information file as well as an assembly file containing a listing of the generated machine code.
The macro \texttt{\_\_arma64\_\_} is predefined in order to enable programmers to identify this tool and its target architecture while compiling.
Programs generated with this compiler require additional runtime support that is stored in the \file{cpp\-arma64\-run} library file.
\flowgraph{\resource{\cpp{}\\source code} \ar[r] & \toolbox{cpparma64} \ar[r] \ar[d] \ar[rd] & \resource{object file} \\ \variable{ECSINCLUDE} \ar[ru] & \resource{debugging\\information} & \resource{assembly\\listing}}
\seecpp\seeassembly\seearm\seeobject\seedebugging
}

\providecommand{\cpparmc}{
\toolsection{cpparmt32} is a compiler for the \cpp{} programming language targeting the ARM hardware architecture.
It generates machine code for ARM processors without floating-point extension executing T32 instructions from programs written in \cpp{} and stores it in corresponding object files.
For debugging purposes, it also creates a debugging information file as well as an assembly file containing a listing of the generated machine code.
The macro \texttt{\_\_armt32\_\_} is predefined in order to enable programmers to identify this tool and its target architecture while compiling.
Programs generated with this compiler require additional runtime support that is stored in the \file{cpp\-armt32\-run} library file.
\flowgraph{\resource{\cpp{}\\source code} \ar[r] & \toolbox{cpparmt32} \ar[r] \ar[d] \ar[rd] & \resource{object file} \\ \variable{ECSINCLUDE} \ar[ru] & \resource{debugging\\information} & \resource{assembly\\listing}}
\seecpp\seeassembly\seearm\seeobject\seedebugging
}

\providecommand{\cpparmcfpe}{
\toolsection{cpparmt32fpe} is a compiler for the \cpp{} programming language targeting the ARM hardware architecture.
It generates machine code for ARM processors with floating-point extension executing T32 instructions from programs written in \cpp{} and stores it in corresponding object files.
For debugging purposes, it also creates a debugging information file as well as an assembly file containing a listing of the generated machine code.
The macro \texttt{\_\_armt32fpe\_\_} is predefined in order to enable programmers to identify this tool and its target architecture while compiling.
Programs generated with this compiler require additional runtime support that is stored in the \file{cpp\-armt32\-fpe\-run} library file.
\flowgraph{\resource{\cpp{}\\source code} \ar[r] & \toolbox{cpparmt32fpe} \ar[r] \ar[d] \ar[rd] & \resource{object file} \\ \variable{ECSINCLUDE} \ar[ru] & \resource{debugging\\information} & \resource{assembly\\listing}}
\seecpp\seeassembly\seearm\seeobject\seedebugging
}

\providecommand{\cppavr}{
\toolsection{cppavr} is a compiler for the \cpp{} programming language targeting the AVR hardware architecture.
It generates machine code for AVR processors from programs written in \cpp{} and stores it in corresponding object files.
For debugging purposes, it also creates a debugging information file as well as an assembly file containing a listing of the generated machine code.
The macro \texttt{\_\_avr\_\_} is predefined in order to enable programmers to identify this tool and its target architecture while compiling.
Programs generated with this compiler require additional runtime support that is stored in the \file{cpp\-avr\-run} library file.
\flowgraph{\resource{\cpp{}\\source code} \ar[r] & \toolbox{cppavr} \ar[r] \ar[d] \ar[rd] & \resource{object file} \\ \variable{ECSINCLUDE} \ar[ru] & \resource{debugging\\information} & \resource{assembly\\listing}}
\seecpp\seeassembly\seeavr\seeobject\seedebugging
}

\providecommand{\cppavrtt}{
\toolsection{cppavr32} is a compiler for the \cpp{} programming language targeting the AVR32 hardware architecture.
It generates machine code for AVR32 processors from programs written in \cpp{} and stores it in corresponding object files.
For debugging purposes, it also creates a debugging information file as well as an assembly file containing a listing of the generated machine code.
The macro \texttt{\_\_avr32\_\_} is predefined in order to enable programmers to identify this tool and its target architecture while compiling.
Programs generated with this compiler require additional runtime support that is stored in the \file{cpp\-avr32\-run} library file.
\flowgraph{\resource{\cpp{}\\source code} \ar[r] & \toolbox{cppavr32} \ar[r] \ar[d] \ar[rd] & \resource{object file} \\ \variable{ECSINCLUDE} \ar[ru] & \resource{debugging\\information} & \resource{assembly\\listing}}
\seecpp\seeassembly\seeavrtt\seeobject\seedebugging
}

\providecommand{\cppmabk}{
\toolsection{cppm68k} is a compiler for the \cpp{} programming language targeting the M68000 hardware architecture.
It generates machine code for M68000 processors from programs written in \cpp{} and stores it in corresponding object files.
For debugging purposes, it also creates a debugging information file as well as an assembly file containing a listing of the generated machine code.
The macro \texttt{\_\_m68k\_\_} is predefined in order to enable programmers to identify this tool and its target architecture while compiling.
Programs generated with this compiler require additional runtime support that is stored in the \file{cpp\-m68k\-run} library file.
\flowgraph{\resource{\cpp{}\\source code} \ar[r] & \toolbox{cppm68k} \ar[r] \ar[d] \ar[rd] & \resource{object file} \\ \variable{ECSINCLUDE} \ar[ru] & \resource{debugging\\information} & \resource{assembly\\listing}}
\seecpp\seeassembly\seemabk\seeobject\seedebugging
}

\providecommand{\cppmibl}{
\toolsection{cppmibl} is a compiler for the \cpp{} programming language targeting the MicroBlaze hardware architecture.
It generates machine code for MicroBlaze processors from programs written in \cpp{} and stores it in corresponding object files.
For debugging purposes, it also creates a debugging information file as well as an assembly file containing a listing of the generated machine code.
The macro \texttt{\_\_mibl\_\_} is predefined in order to enable programmers to identify this tool and its target architecture while compiling.
Programs generated with this compiler require additional runtime support that is stored in the \file{cpp\-mibl\-run} library file.
\flowgraph{\resource{\cpp{}\\source code} \ar[r] & \toolbox{cppmibl} \ar[r] \ar[d] \ar[rd] & \resource{object file} \\ \variable{ECSINCLUDE} \ar[ru] & \resource{debugging\\information} & \resource{assembly\\listing}}
\seecpp\seeassembly\seemibl\seeobject\seedebugging
}

\providecommand{\cppmipsa}{
\toolsection{cppmips32} is a compiler for the \cpp{} programming language targeting the MIPS32 hardware architecture.
It generates machine code for MIPS32 processors from programs written in \cpp{} and stores it in corresponding object files.
For debugging purposes, it also creates a debugging information file as well as an assembly file containing a listing of the generated machine code.
The macro \texttt{\_\_mips32\_\_} is predefined in order to enable programmers to identify this tool and its target architecture while compiling.
Programs generated with this compiler require additional runtime support that is stored in the \file{cpp\-mips32\-run} library file.
\flowgraph{\resource{\cpp{}\\source code} \ar[r] & \toolbox{cppmips32} \ar[r] \ar[d] \ar[rd] & \resource{object file} \\ \variable{ECSINCLUDE} \ar[ru] & \resource{debugging\\information} & \resource{assembly\\listing}}
\seecpp\seeassembly\seemips\seeobject\seedebugging
}

\providecommand{\cppmipsb}{
\toolsection{cppmips64} is a compiler for the \cpp{} programming language targeting the MIPS64 hardware architecture.
It generates machine code for MIPS64 processors from programs written in \cpp{} and stores it in corresponding object files.
For debugging purposes, it also creates a debugging information file as well as an assembly file containing a listing of the generated machine code.
The macro \texttt{\_\_mips64\_\_} is predefined in order to enable programmers to identify this tool and its target architecture while compiling.
Programs generated with this compiler require additional runtime support that is stored in the \file{cpp\-mips64\-run} library file.
\flowgraph{\resource{\cpp{}\\source code} \ar[r] & \toolbox{cppmips64} \ar[r] \ar[d] \ar[rd] & \resource{object file} \\ \variable{ECSINCLUDE} \ar[ru] & \resource{debugging\\information} & \resource{assembly\\listing}}
\seecpp\seeassembly\seemips\seeobject\seedebugging
}

\providecommand{\cppmmix}{
\toolsection{cppmmix} is a compiler for the \cpp{} programming language targeting the MMIX hardware architecture.
It generates machine code for MMIX processors from programs written in \cpp{} and stores it in corresponding object files.
For debugging purposes, it also creates a debugging information file as well as an assembly file containing a listing of the generated machine code.
The macro \texttt{\_\_mmix\_\_} is predefined in order to enable programmers to identify this tool and its target architecture while compiling.
Programs generated with this compiler require additional runtime support that is stored in the \file{cpp\-mmix\-run} library file.
\flowgraph{\resource{\cpp{}\\source code} \ar[r] & \toolbox{cppmmix} \ar[r] \ar[d] \ar[rd] & \resource{object file} \\ \variable{ECSINCLUDE} \ar[ru] & \resource{debugging\\information} & \resource{assembly\\listing}}
\seecpp\seeassembly\seemmix\seeobject\seedebugging
}

\providecommand{\cpporok}{
\toolsection{cppor1k} is a compiler for the \cpp{} programming language targeting the OpenRISC 1000 hardware architecture.
It generates machine code for OpenRISC 1000 processors from programs written in \cpp{} and stores it in corresponding object files.
For debugging purposes, it also creates a debugging information file as well as an assembly file containing a listing of the generated machine code.
The macro \texttt{\_\_or1k\_\_} is predefined in order to enable programmers to identify this tool and its target architecture while compiling.
Programs generated with this compiler require additional runtime support that is stored in the \file{cpp\-or1k\-run} library file.
\flowgraph{\resource{\cpp{}\\source code} \ar[r] & \toolbox{cppor1k} \ar[r] \ar[d] \ar[rd] & \resource{object file} \\ \variable{ECSINCLUDE} \ar[ru] & \resource{debugging\\information} & \resource{assembly\\listing}}
\seecpp\seeassembly\seeorok\seeobject\seedebugging
}

\providecommand{\cppppca}{
\toolsection{cppppc32} is a compiler for the \cpp{} programming language targeting the PowerPC hardware architecture.
It generates machine code for PowerPC processors from programs written in \cpp{} and stores it in corresponding object files.
The compiler generates machine code for the 32-bit operating mode defined by the PowerPC architecture.
For debugging purposes, it also creates a debugging information file as well as an assembly file containing a listing of the generated machine code.
The macro \texttt{\_\_ppc32\_\_} is predefined in order to enable programmers to identify this tool and its target architecture while compiling.
Programs generated with this compiler require additional runtime support that is stored in the \file{cpp\-ppc32\-run} library file.
\flowgraph{\resource{\cpp{}\\source code} \ar[r] & \toolbox{cppppc32} \ar[r] \ar[d] \ar[rd] & \resource{object file} \\ \variable{ECSINCLUDE} \ar[ru] & \resource{debugging\\information} & \resource{assembly\\listing}}
\seecpp\seeassembly\seeppc\seeobject\seedebugging
}

\providecommand{\cppppcb}{
\toolsection{cppppc64} is a compiler for the \cpp{} programming language targeting the PowerPC hardware architecture.
It generates machine code for PowerPC processors from programs written in \cpp{} and stores it in corresponding object files.
The compiler generates machine code for the 64-bit operating mode defined by the PowerPC architecture.
For debugging purposes, it also creates a debugging information file as well as an assembly file containing a listing of the generated machine code.
The macro \texttt{\_\_ppc64\_\_} is predefined in order to enable programmers to identify this tool and its target architecture while compiling.
Programs generated with this compiler require additional runtime support that is stored in the \file{cpp\-ppc64\-run} library file.
\flowgraph{\resource{\cpp{}\\source code} \ar[r] & \toolbox{cppppc64} \ar[r] \ar[d] \ar[rd] & \resource{object file} \\ \variable{ECSINCLUDE} \ar[ru] & \resource{debugging\\information} & \resource{assembly\\listing}}
\seecpp\seeassembly\seeppc\seeobject\seedebugging
}

\providecommand{\cpprisc}{
\toolsection{cpprisc} is a compiler for the \cpp{} programming language targeting the RISC hardware architecture.
It generates machine code for RISC processors from programs written in \cpp{} and stores it in corresponding object files.
For debugging purposes, it also creates a debugging information file as well as an assembly file containing a listing of the generated machine code.
The macro \texttt{\_\_risc\_\_} is predefined in order to enable programmers to identify this tool and its target architecture while compiling.
Programs generated with this compiler require additional runtime support that is stored in the \file{cpp\-risc\-run} library file.
\flowgraph{\resource{\cpp{}\\source code} \ar[r] & \toolbox{cpprisc} \ar[r] \ar[d] \ar[rd] & \resource{object file} \\ \variable{ECSINCLUDE} \ar[ru] & \resource{debugging\\information} & \resource{assembly\\listing}}
\seecpp\seeassembly\seerisc\seeobject\seedebugging
}

\providecommand{\cppwasm}{
\toolsection{cppwasm} is a compiler for the \cpp{} programming language targeting the WebAssembly architecture.
It generates machine code for WebAssembly targets from programs written in \cpp{} and stores it in corresponding object files.
For debugging purposes, it also creates a debugging information file as well as an assembly file containing a listing of the generated machine code.
The macro \texttt{\_\_wasm\_\_} is predefined in order to enable programmers to identify this tool and its target architecture while compiling.
Programs generated with this compiler require additional runtime support that is stored in the \file{cpp\-wasm\-run} library file.
\flowgraph{\resource{\cpp{}\\source code} \ar[r] & \toolbox{cppwasm} \ar[r] \ar[d] \ar[rd] & \resource{object file} \\ \variable{ECSINCLUDE} \ar[ru] & \resource{debugging\\information} & \resource{assembly\\listing}}
\seecpp\seeassembly\seewasm\seeobject\seedebugging
}

% FALSE tools

\providecommand{\falprint}{
\toolsection{falprint} is a pretty printer for the FALSE programming language.
It reformats the source code of FALSE programs and writes it to the standard output stream.
\flowgraph{\resource{FALSE\\source code} \ar[r] & \toolbox{falprint} \ar[r] & \resource{reformatted\\source code}}
\seefalse
}

\providecommand{\falcheck}{
\toolsection{falcheck} is a syntactic and semantic checker for the FALSE programming language.
It just performs syntactic and semantic checks on FALSE programs and writes its diagnostic messages to the standard error stream.
\flowgraph{\resource{FALSE\\source code} \ar[r] & \toolbox{falcheck} \ar[r] & \resource{diagnostic\\messages}}
\seefalse
}

\providecommand{\faldump}{
\toolsection{faldump} is a serializer for the FALSE programming language.
It dumps the complete internal representation of programs written in FALSE into an XML document.
\debuggingtool
\flowgraph{\resource{FALSE\\source code} \ar[r] & \toolbox{faldump} \ar[r] & \resource{internal\\representation}}
\seefalse
}

\providecommand{\falrun}{
\toolsection{falrun} is an interpreter for the FALSE programming language.
It processes and executes programs written in FALSE\@.
\flowgraph{\resource{FALSE\\source code} \ar[r] & \toolbox{falrun} \ar@/u/[r] & \resource{input/\\output} \ar@/d/[l]}
\seefalse
}

\providecommand{\falcpp}{
\toolsection{falcpp} is a transpiler for the FALSE programming language.
It translates programs written in FALSE into \cpp{} programs and stores them in corresponding source files.
\flowgraph{\resource{FALSE\\source code} \ar[r] & \toolbox{falcpp} \ar[r] & \resource{\cpp{}\\source file}}
\seefalse\seecpp
}

\providecommand{\falcode}{
\toolsection{falcode} is an intermediate code generator for the FALSE programming language.
It generates intermediate code from programs written in FALSE and stores it in corresponding assembly files.
\debuggingtool
\flowgraph{\resource{FALSE\\source code} \ar[r] & \toolbox{falcode} \ar[r] & \resource{intermediate\\code}}
\seefalse\seeassembly\seecode
}

\providecommand{\falamda}{
\toolsection{falamd16} is a compiler for the FALSE programming language targeting the AMD64 hardware architecture.
It generates machine code for AMD64 processors from programs written in FALSE and stores it in corresponding object files.
The compiler generates machine code for the 16-bit operating mode defined by the AMD64 architecture.
\flowgraph{\resource{FALSE\\source code} \ar[r] & \toolbox{falamd16} \ar[r] & \resource{object file}}
\seefalse\seeamd\seeobject
}

\providecommand{\falamdb}{
\toolsection{falamd32} is a compiler for the FALSE programming language targeting the AMD64 hardware architecture.
It generates machine code for AMD64 processors from programs written in FALSE and stores it in corresponding object files.
The compiler generates machine code for the 32-bit operating mode defined by the AMD64 architecture.
\flowgraph{\resource{FALSE\\source code} \ar[r] & \toolbox{falamd32} \ar[r] & \resource{object file}}
\seefalse\seeamd\seeobject
}

\providecommand{\falamdc}{
\toolsection{falamd64} is a compiler for the FALSE programming language targeting the AMD64 hardware architecture.
It generates machine code for AMD64 processors from programs written in FALSE and stores it in corresponding object files.
The compiler generates machine code for the 64-bit operating mode defined by the AMD64 architecture.
\flowgraph{\resource{FALSE\\source code} \ar[r] & \toolbox{falamd64} \ar[r] & \resource{object file}}
\seefalse\seeamd\seeobject
}

\providecommand{\falarma}{
\toolsection{falarma32} is a compiler for the FALSE programming language targeting the ARM hardware architecture.
It generates machine code for ARM processors executing A32 instructions from programs written in FALSE and stores it in corresponding object files.
\flowgraph{\resource{FALSE\\source code} \ar[r] & \toolbox{falarma32} \ar[r] & \resource{object file}}
\seefalse\seearm\seeobject
}

\providecommand{\falarmb}{
\toolsection{falarma64} is a compiler for the FALSE programming language targeting the ARM hardware architecture.
It generates machine code for ARM processors executing A64 instructions from programs written in FALSE and stores it in corresponding object files.
\flowgraph{\resource{FALSE\\source code} \ar[r] & \toolbox{falarma64} \ar[r] & \resource{object file}}
\seefalse\seearm\seeobject
}

\providecommand{\falarmc}{
\toolsection{falarmt32} is a compiler for the FALSE programming language targeting the ARM hardware architecture.
It generates machine code for ARM processors without floating-point extension executing T32 instructions from programs written in FALSE and stores it in corresponding object files.
\flowgraph{\resource{FALSE\\source code} \ar[r] & \toolbox{falarmt32} \ar[r] & \resource{object file}}
\seefalse\seearm\seeobject
}

\providecommand{\falarmcfpe}{
\toolsection{falarmt32fpe} is a compiler for the FALSE programming language targeting the ARM hardware architecture.
It generates machine code for ARM processors with floating-point extension executing T32 instructions from programs written in FALSE and stores it in corresponding object files.
\flowgraph{\resource{FALSE\\source code} \ar[r] & \toolbox{falarmt32fpe} \ar[r] & \resource{object file}}
\seefalse\seearm\seeobject
}

\providecommand{\falavr}{
\toolsection{falavr} is a compiler for the FALSE programming language targeting the AVR hardware architecture.
It generates machine code for AVR processors from programs written in FALSE and stores it in corresponding object files.
\flowgraph{\resource{FALSE\\source code} \ar[r] & \toolbox{falavr} \ar[r] & \resource{object file}}
\seefalse\seeavr\seeobject
}

\providecommand{\falavrtt}{
\toolsection{falavr32} is a compiler for the FALSE programming language targeting the AVR32 hardware architecture.
It generates machine code for AVR32 processors from programs written in FALSE and stores it in corresponding object files.
\flowgraph{\resource{FALSE\\source code} \ar[r] & \toolbox{falavr32} \ar[r] & \resource{object file}}
\seefalse\seeavrtt\seeobject
}

\providecommand{\falmabk}{
\toolsection{falm68k} is a compiler for the FALSE programming language targeting the M68000 hardware architecture.
It generates machine code for M68000 processors from programs written in FALSE and stores it in corresponding object files.
\flowgraph{\resource{FALSE\\source code} \ar[r] & \toolbox{falm68k} \ar[r] & \resource{object file}}
\seefalse\seemabk\seeobject
}

\providecommand{\falmibl}{
\toolsection{falmibl} is a compiler for the FALSE programming language targeting the MicroBlaze hardware architecture.
It generates machine code for MicroBlaze processors from programs written in FALSE and stores it in corresponding object files.
\flowgraph{\resource{FALSE\\source code} \ar[r] & \toolbox{falmibl} \ar[r] & \resource{object file}}
\seefalse\seemibl\seeobject
}

\providecommand{\falmipsa}{
\toolsection{falmips32} is a compiler for the FALSE programming language targeting the MIPS32 hardware architecture.
It generates machine code for MIPS32 processors from programs written in FALSE and stores it in corresponding object files.
\flowgraph{\resource{FALSE\\source code} \ar[r] & \toolbox{falmips32} \ar[r] & \resource{object file}}
\seefalse\seemips\seeobject
}

\providecommand{\falmipsb}{
\toolsection{falmips64} is a compiler for the FALSE programming language targeting the MIPS64 hardware architecture.
It generates machine code for MIPS64 processors from programs written in FALSE and stores it in corresponding object files.
\flowgraph{\resource{FALSE\\source code} \ar[r] & \toolbox{falmips64} \ar[r] & \resource{object file}}
\seefalse\seemips\seeobject
}

\providecommand{\falmmix}{
\toolsection{falmmix} is a compiler for the FALSE programming language targeting the MMIX hardware architecture.
It generates machine code for MMIX processors from programs written in FALSE and stores it in corresponding object files.
\flowgraph{\resource{FALSE\\source code} \ar[r] & \toolbox{falmmix} \ar[r] & \resource{object file}}
\seefalse\seemmix\seeobject
}

\providecommand{\falorok}{
\toolsection{falor1k} is a compiler for the FALSE programming language targeting the OpenRISC 1000 hardware architecture.
It generates machine code for OpenRISC 1000 processors from programs written in FALSE and stores it in corresponding object files.
\flowgraph{\resource{FALSE\\source code} \ar[r] & \toolbox{falor1k} \ar[r] & \resource{object file}}
\seefalse\seeorok\seeobject
}

\providecommand{\falppca}{
\toolsection{falppc32} is a compiler for the FALSE programming language targeting the PowerPC hardware architecture.
It generates machine code for PowerPC processors from programs written in FALSE and stores it in corresponding object files.
The compiler generates machine code for the 32-bit operating mode defined by the PowerPC architecture.
\flowgraph{\resource{FALSE\\source code} \ar[r] & \toolbox{falppc32} \ar[r] & \resource{object file}}
\seefalse\seeppc\seeobject
}

\providecommand{\falppcb}{
\toolsection{falppc64} is a compiler for the FALSE programming language targeting the PowerPC hardware architecture.
It generates machine code for PowerPC processors from programs written in FALSE and stores it in corresponding object files.
The compiler generates machine code for the 64-bit operating mode defined by the PowerPC architecture.
\flowgraph{\resource{FALSE\\source code} \ar[r] & \toolbox{falppc64} \ar[r] & \resource{object file}}
\seefalse\seeppc\seeobject
}

\providecommand{\falrisc}{
\toolsection{falrisc} is a compiler for the FALSE programming language targeting the RISC hardware architecture.
It generates machine code for RISC processors from programs written in FALSE and stores it in corresponding object files.
\flowgraph{\resource{FALSE\\source code} \ar[r] & \toolbox{falrisc} \ar[r] & \resource{object file}}
\seefalse\seerisc\seeobject
}

\providecommand{\falwasm}{
\toolsection{falwasm} is a compiler for the FALSE programming language targeting the WebAssembly architecture.
It generates machine code for WebAssembly targets from programs written in FALSE and stores it in corresponding object files.
\flowgraph{\resource{FALSE\\source code} \ar[r] & \toolbox{falwasm} \ar[r] & \resource{object file}}
\seefalse\seewasm\seeobject
}

% Oberon tools

\providecommand{\obprint}{
\toolsection{obprint} is a pretty printer for the Oberon programming language.
It reformats the source code of Oberon modules and writes it to the standard output stream.
\flowgraph{\resource{Oberon\\source code} \ar[r] & \toolbox{obprint} \ar[r] & \resource{reformatted\\source code}}
\seeoberon
}

\providecommand{\obcheck}{
\toolsection{obcheck} is a syntactic and semantic checker for the Oberon programming language.
It just performs syntactic and semantic checks on Oberon modules and writes its diagnostic messages to the standard error stream.
In addition, it stores the interface of each module in a symbol file which is required when other modules import the module.
\flowgraph{\resource{Oberon\\source code} \ar[r] & \toolbox{obcheck} \ar[r] \ar@/l/[d] & \resource{diagnostic\\messages} \\ \variable{ECSIMPORT} \ar[ru] & \resource{symbol\\files} \ar@/r/[u]}
\seeoberon
}

\providecommand{\obdump}{
\toolsection{obdump} is a serializer for the Oberon programming language.
It dumps the complete internal representation of modules written in Oberon into an XML document.
\debuggingtool
\flowgraph{\resource{Oberon\\source code} \ar[r] & \toolbox{obdump} \ar[r] \ar@/l/[d] & \resource{internal\\representation} \\ \variable{ECSIMPORT} \ar[ru] & \resource{symbol\\files} \ar@/r/[u]}
\seeoberon
}

\providecommand{\obrun}{
\toolsection{obrun} is an interpreter for the Oberon programming language.
It processes and executes modules written in Oberon.
This tool does neither generate nor process symbol files while interpreting modules.
If a module is imported by another one, its filename has to be named before the other one in the list of command-line arguments.
\flowgraph{\resource{Oberon\\source code} \ar[r] & \toolbox{obrun} \ar@/u/[r] & \resource{input/\\output} \ar@/d/[l]}
\seeoberon
}

\providecommand{\obcpp}{
\toolsection{obcpp} is a transpiler for the Oberon programming language.
It translates programs written in Oberon into \cpp{} programs and stores them in corresponding source and header files.
In addition, it stores the interface of each module in a symbol file which is required when other modules import the module.
The same interface is provided by the generated header file which can be used in other parts of the \cpp{} program.
\flowgraph{\resource{Oberon\\source code} \ar[r] & \toolbox{obcpp} \ar[r] \ar@/l/[d] \ar[rd] & \resource{\cpp{}\\source file} \\ \variable{ECSIMPORT} \ar[ru] & \resource{symbol\\files} \ar@/r/[u] & \resource{\cpp{}\\header file}}
\seeoberon\seecpp
}

\providecommand{\obdoc}{
\toolsection{obdoc} is a generic documentation generator for the Oberon programming language.
It processes several Oberon modules and assembles all information therein into a generic documentation.
In addition, it stores the interface of each module in a symbol file which is required when other modules import the module.
\debuggingtool
\flowgraph{\resource{Oberon\\source code} \ar[r] & \toolbox{obdoc} \ar[r] \ar@/l/[d] & \resource{generic\\documentation} \\ \variable{ECSIMPORT} \ar[ru] & \resource{symbol\\files} \ar@/r/[u]}
\seeoberon\seedocumentation
}

\providecommand{\obhtml}{
\toolsection{obhtml} is an HTML documentation generator for the Oberon programming language.
It processes several Oberon modules and assembles all information therein into an HTML document.
In addition, it stores the interface of each module in a symbol file which is required when other modules import the module.
\flowgraph{\resource{Oberon\\source code} \ar[r] & \toolbox{obhtml} \ar[r] \ar@/l/[d] & \resource{HTML\\document} \\ \variable{ECSIMPORT} \ar[ru] & \resource{symbol\\files} \ar@/r/[u]}
\seeoberon\seedocumentation
}

\providecommand{\oblatex}{
\toolsection{oblatex} is a Latex documentation generator for the Oberon programming language.
It processes several Oberon modules and assembles all information therein into a Latex document.
In addition, it stores the interface of each module in a symbol file which is required when other modules import the module.
\flowgraph{\resource{Oberon\\source code} \ar[r] & \toolbox{oblatex} \ar[r] \ar@/l/[d] & \resource{Latex\\document} \\ \variable{ECSIMPORT} \ar[ru] & \resource{symbol\\files} \ar@/r/[u]}
\seeoberon\seedocumentation
}

\providecommand{\obcode}{
\toolsection{obcode} is an intermediate code generator for the Oberon programming language.
It generates intermediate code from modules written in Oberon and stores it in corresponding assembly files.
In addition, it stores the interface of each module in a symbol file which is required when other modules import the module.
Programs generated with this tool require additional runtime support that is stored in the \file{ob\-code\-run} library file.
\debuggingtool
\flowgraph{\resource{Oberon\\source code} \ar[r] & \toolbox{obcode} \ar[r] \ar@/l/[d] & \resource{intermediate\\code} \\ \variable{ECSIMPORT} \ar[ru] & \resource{symbol\\files} \ar@/r/[u]}
\seeoberon\seeassembly\seecode
}

\providecommand{\obamda}{
\toolsection{obamd16} is a compiler for the Oberon programming language targeting the AMD64 hardware architecture.
It generates machine code for AMD64 processors from modules written in Oberon and stores it in corresponding object files.
The compiler generates machine code for the 16-bit operating mode defined by the AMD64 architecture.
For debugging purposes, it also creates a debugging information file as well as an assembly file containing a listing of the generated machine code.
In addition, it stores the interface of each module in a symbol file which is required when other modules import the module.
Programs generated with this compiler require additional runtime support that is stored in the \file{ob\-amd16\-run} library file.
\flowgraph{\resource{Oberon\\source code} \ar[r] & \toolbox{obamd16} \ar[r] \ar@/l/[d] \ar[rd] & \resource{object file} \\ \variable{ECSIMPORT} \ar[ru] & \resource{symbol\\files} \ar@/r/[u] & \resource{debugging\\information}}
\seeoberon\seeassembly\seeamd\seeobject\seedebugging
}

\providecommand{\obamdb}{
\toolsection{obamd32} is a compiler for the Oberon programming language targeting the AMD64 hardware architecture.
It generates machine code for AMD64 processors from modules written in Oberon and stores it in corresponding object files.
The compiler generates machine code for the 32-bit operating mode defined by the AMD64 architecture.
For debugging purposes, it also creates a debugging information file as well as an assembly file containing a listing of the generated machine code.
In addition, it stores the interface of each module in a symbol file which is required when other modules import the module.
Programs generated with this compiler require additional runtime support that is stored in the \file{ob\-amd32\-run} library file.
\flowgraph{\resource{Oberon\\source code} \ar[r] & \toolbox{obamd32} \ar[r] \ar@/l/[d] \ar[rd] & \resource{object file} \\ \variable{ECSIMPORT} \ar[ru] & \resource{symbol\\files} \ar@/r/[u] & \resource{debugging\\information}}
\seeoberon\seeassembly\seeamd\seeobject\seedebugging
}

\providecommand{\obamdc}{
\toolsection{obamd64} is a compiler for the Oberon programming language targeting the AMD64 hardware architecture.
It generates machine code for AMD64 processors from modules written in Oberon and stores it in corresponding object files.
The compiler generates machine code for the 64-bit operating mode defined by the AMD64 architecture.
For debugging purposes, it also creates a debugging information file as well as an assembly file containing a listing of the generated machine code.
In addition, it stores the interface of each module in a symbol file which is required when other modules import the module.
Programs generated with this compiler require additional runtime support that is stored in the \file{ob\-amd64\-run} library file.
\flowgraph{\resource{Oberon\\source code} \ar[r] & \toolbox{obamd64} \ar[r] \ar@/l/[d] \ar[rd] & \resource{object file} \\ \variable{ECSIMPORT} \ar[ru] & \resource{symbol\\files} \ar@/r/[u] & \resource{debugging\\information}}
\seeoberon\seeassembly\seeamd\seeobject\seedebugging
}

\providecommand{\obarma}{
\toolsection{obarma32} is a compiler for the Oberon programming language targeting the ARM hardware architecture.
It generates machine code for ARM processors executing A32 instructions from modules written in Oberon and stores it in corresponding object files.
For debugging purposes, it also creates a debugging information file as well as an assembly file containing a listing of the generated machine code.
In addition, it stores the interface of each module in a symbol file which is required when other modules import the module.
Programs generated with this compiler require additional runtime support that is stored in the \file{ob\-arma32\-run} library file.
\flowgraph{\resource{Oberon\\source code} \ar[r] & \toolbox{obarma32} \ar[r] \ar@/l/[d] \ar[rd] & \resource{object file} \\ \variable{ECSIMPORT} \ar[ru] & \resource{symbol\\files} \ar@/r/[u] & \resource{debugging\\information}}
\seeoberon\seeassembly\seearm\seeobject\seedebugging
}

\providecommand{\obarmb}{
\toolsection{obarma64} is a compiler for the Oberon programming language targeting the ARM hardware architecture.
It generates machine code for ARM processors executing A64 instructions from modules written in Oberon and stores it in corresponding object files.
For debugging purposes, it also creates a debugging information file as well as an assembly file containing a listing of the generated machine code.
In addition, it stores the interface of each module in a symbol file which is required when other modules import the module.
Programs generated with this compiler require additional runtime support that is stored in the \file{ob\-arma64\-run} library file.
\flowgraph{\resource{Oberon\\source code} \ar[r] & \toolbox{obarma64} \ar[r] \ar@/l/[d] \ar[rd] & \resource{object file} \\ \variable{ECSIMPORT} \ar[ru] & \resource{symbol\\files} \ar@/r/[u] & \resource{debugging\\information}}
\seeoberon\seeassembly\seearm\seeobject\seedebugging
}

\providecommand{\obarmc}{
\toolsection{obarmt32} is a compiler for the Oberon programming language targeting the ARM hardware architecture.
It generates machine code for ARM processors without floating-point extension executing T32 instructions from modules written in Oberon and stores it in corresponding object files.
For debugging purposes, it also creates a debugging information file as well as an assembly file containing a listing of the generated machine code.
In addition, it stores the interface of each module in a symbol file which is required when other modules import the module.
Programs generated with this compiler require additional runtime support that is stored in the \file{ob\-armt32\-run} library file.
\flowgraph{\resource{Oberon\\source code} \ar[r] & \toolbox{obarmt32} \ar[r] \ar@/l/[d] \ar[rd] & \resource{object file} \\ \variable{ECSIMPORT} \ar[ru] & \resource{symbol\\files} \ar@/r/[u] & \resource{debugging\\information}}
\seeoberon\seeassembly\seearm\seeobject\seedebugging
}

\providecommand{\obarmcfpe}{
\toolsection{obarmt32fpe} is a compiler for the Oberon programming language targeting the ARM hardware architecture.
It generates machine code for ARM processors with floating-point extension executing T32 instructions from modules written in Oberon and stores it in corresponding object files.
For debugging purposes, it also creates a debugging information file as well as an assembly file containing a listing of the generated machine code.
In addition, it stores the interface of each module in a symbol file which is required when other modules import the module.
Programs generated with this compiler require additional runtime support that is stored in the \file{ob\-armt32\-fpe\-run} library file.
\flowgraph{\resource{Oberon\\source code} \ar[r] & \toolbox{obarmt32fpe} \ar[r] \ar@/l/[d] \ar[rd] & \resource{object file} \\ \variable{ECSIMPORT} \ar[ru] & \resource{symbol\\files} \ar@/r/[u] & \resource{debugging\\information}}
\seeoberon\seeassembly\seearm\seeobject\seedebugging
}

\providecommand{\obavr}{
\toolsection{obavr} is a compiler for the Oberon programming language targeting the AVR hardware architecture.
It generates machine code for AVR processors from modules written in Oberon and stores it in corresponding object files.
For debugging purposes, it also creates a debugging information file as well as an assembly file containing a listing of the generated machine code.
In addition, it stores the interface of each module in a symbol file which is required when other modules import the module.
Programs generated with this compiler require additional runtime support that is stored in the \file{ob\-avr\-run} library file.
\flowgraph{\resource{Oberon\\source code} \ar[r] & \toolbox{obavr} \ar[r] \ar@/l/[d] \ar[rd] & \resource{object file} \\ \variable{ECSIMPORT} \ar[ru] & \resource{symbol\\files} \ar@/r/[u] & \resource{debugging\\information}}
\seeoberon\seeassembly\seeavr\seeobject\seedebugging
}

\providecommand{\obavrtt}{
\toolsection{obavr32} is a compiler for the Oberon programming language targeting the AVR32 hardware architecture.
It generates machine code for AVR32 processors from modules written in Oberon and stores it in corresponding object files.
For debugging purposes, it also creates a debugging information file as well as an assembly file containing a listing of the generated machine code.
In addition, it stores the interface of each module in a symbol file which is required when other modules import the module.
Programs generated with this compiler require additional runtime support that is stored in the \file{ob\-avr32\-run} library file.
\flowgraph{\resource{Oberon\\source code} \ar[r] & \toolbox{obavr32} \ar[r] \ar@/l/[d] \ar[rd] & \resource{object file} \\ \variable{ECSIMPORT} \ar[ru] & \resource{symbol\\files} \ar@/r/[u] & \resource{debugging\\information}}
\seeoberon\seeassembly\seeavrtt\seeobject\seedebugging
}

\providecommand{\obmabk}{
\toolsection{obm68k} is a compiler for the Oberon programming language targeting the M68000 hardware architecture.
It generates machine code for M68000 processors from modules written in Oberon and stores it in corresponding object files.
For debugging purposes, it also creates a debugging information file as well as an assembly file containing a listing of the generated machine code.
In addition, it stores the interface of each module in a symbol file which is required when other modules import the module.
Programs generated with this compiler require additional runtime support that is stored in the \file{ob\-m68k\-run} library file.
\flowgraph{\resource{Oberon\\source code} \ar[r] & \toolbox{obm68k} \ar[r] \ar@/l/[d] \ar[rd] & \resource{object file} \\ \variable{ECSIMPORT} \ar[ru] & \resource{symbol\\files} \ar@/r/[u] & \resource{debugging\\information}}
\seeoberon\seeassembly\seemabk\seeobject\seedebugging
}

\providecommand{\obmibl}{
\toolsection{obmibl} is a compiler for the Oberon programming language targeting the MicroBlaze hardware architecture.
It generates machine code for MicroBlaze processors from modules written in Oberon and stores it in corresponding object files.
For debugging purposes, it also creates a debugging information file as well as an assembly file containing a listing of the generated machine code.
In addition, it stores the interface of each module in a symbol file which is required when other modules import the module.
Programs generated with this compiler require additional runtime support that is stored in the \file{ob\-mibl\-run} library file.
\flowgraph{\resource{Oberon\\source code} \ar[r] & \toolbox{obmibl} \ar[r] \ar@/l/[d] \ar[rd] & \resource{object file} \\ \variable{ECSIMPORT} \ar[ru] & \resource{symbol\\files} \ar@/r/[u] & \resource{debugging\\information}}
\seeoberon\seeassembly\seemibl\seeobject\seedebugging
}

\providecommand{\obmipsa}{
\toolsection{obmips32} is a compiler for the Oberon programming language targeting the MIPS32 hardware architecture.
It generates machine code for MIPS32 processors from modules written in Oberon and stores it in corresponding object files.
For debugging purposes, it also creates a debugging information file as well as an assembly file containing a listing of the generated machine code.
In addition, it stores the interface of each module in a symbol file which is required when other modules import the module.
Programs generated with this compiler require additional runtime support that is stored in the \file{ob\-mips32\-run} library file.
\flowgraph{\resource{Oberon\\source code} \ar[r] & \toolbox{obmips32} \ar[r] \ar@/l/[d] \ar[rd] & \resource{object file} \\ \variable{ECSIMPORT} \ar[ru] & \resource{symbol\\files} \ar@/r/[u] & \resource{debugging\\information}}
\seeoberon\seeassembly\seemips\seeobject\seedebugging
}

\providecommand{\obmipsb}{
\toolsection{obmips64} is a compiler for the Oberon programming language targeting the MIPS64 hardware architecture.
It generates machine code for MIPS64 processors from modules written in Oberon and stores it in corresponding object files.
For debugging purposes, it also creates a debugging information file as well as an assembly file containing a listing of the generated machine code.
In addition, it stores the interface of each module in a symbol file which is required when other modules import the module.
Programs generated with this compiler require additional runtime support that is stored in the \file{ob\-mips64\-run} library file.
\flowgraph{\resource{Oberon\\source code} \ar[r] & \toolbox{obmips64} \ar[r] \ar@/l/[d] \ar[rd] & \resource{object file} \\ \variable{ECSIMPORT} \ar[ru] & \resource{symbol\\files} \ar@/r/[u] & \resource{debugging\\information}}
\seeoberon\seeassembly\seemips\seeobject\seedebugging
}

\providecommand{\obmmix}{
\toolsection{obmmix} is a compiler for the Oberon programming language targeting the MMIX hardware architecture.
It generates machine code for MMIX processors from modules written in Oberon and stores it in corresponding object files.
For debugging purposes, it also creates a debugging information file as well as an assembly file containing a listing of the generated machine code.
In addition, it stores the interface of each module in a symbol file which is required when other modules import the module.
Programs generated with this compiler require additional runtime support that is stored in the \file{ob\-mmix\-run} library file.
\flowgraph{\resource{Oberon\\source code} \ar[r] & \toolbox{obmmix} \ar[r] \ar@/l/[d] \ar[rd] & \resource{object file} \\ \variable{ECSIMPORT} \ar[ru] & \resource{symbol\\files} \ar@/r/[u] & \resource{debugging\\information}}
\seeoberon\seeassembly\seemmix\seeobject\seedebugging
}

\providecommand{\oborok}{
\toolsection{obor1k} is a compiler for the Oberon programming language targeting the OpenRISC 1000 hardware architecture.
It generates machine code for OpenRISC 1000 processors from modules written in Oberon and stores it in corresponding object files.
For debugging purposes, it also creates a debugging information file as well as an assembly file containing a listing of the generated machine code.
In addition, it stores the interface of each module in a symbol file which is required when other modules import the module.
Programs generated with this compiler require additional runtime support that is stored in the \file{ob\-or1k\-run} library file.
\flowgraph{\resource{Oberon\\source code} \ar[r] & \toolbox{obor1k} \ar[r] \ar@/l/[d] \ar[rd] & \resource{object file} \\ \variable{ECSIMPORT} \ar[ru] & \resource{symbol\\files} \ar@/r/[u] & \resource{debugging\\information}}
\seeoberon\seeassembly\seeorok\seeobject\seedebugging
}

\providecommand{\obppca}{
\toolsection{obppc32} is a compiler for the Oberon programming language targeting the PowerPC hardware architecture.
It generates machine code for PowerPC processors from modules written in Oberon and stores it in corresponding object files.
The compiler generates machine code for the 32-bit operating mode defined by the PowerPC architecture.
For debugging purposes, it also creates a debugging information file as well as an assembly file containing a listing of the generated machine code.
In addition, it stores the interface of each module in a symbol file which is required when other modules import the module.
Programs generated with this compiler require additional runtime support that is stored in the \file{ob\-ppc32\-run} library file.
\flowgraph{\resource{Oberon\\source code} \ar[r] & \toolbox{obppc32} \ar[r] \ar@/l/[d] \ar[rd] & \resource{object file} \\ \variable{ECSIMPORT} \ar[ru] & \resource{symbol\\files} \ar@/r/[u] & \resource{debugging\\information}}
\seeoberon\seeassembly\seeppc\seeobject\seedebugging
}

\providecommand{\obppcb}{
\toolsection{obppc64} is a compiler for the Oberon programming language targeting the PowerPC hardware architecture.
It generates machine code for PowerPC processors from modules written in Oberon and stores it in corresponding object files.
The compiler generates machine code for the 64-bit operating mode defined by the PowerPC architecture.
For debugging purposes, it also creates a debugging information file as well as an assembly file containing a listing of the generated machine code.
In addition, it stores the interface of each module in a symbol file which is required when other modules import the module.
Programs generated with this compiler require additional runtime support that is stored in the \file{ob\-ppc64\-run} library file.
\flowgraph{\resource{Oberon\\source code} \ar[r] & \toolbox{obppc64} \ar[r] \ar@/l/[d] \ar[rd] & \resource{object file} \\ \variable{ECSIMPORT} \ar[ru] & \resource{symbol\\files} \ar@/r/[u] & \resource{debugging\\information}}
\seeoberon\seeassembly\seeppc\seeobject\seedebugging
}

\providecommand{\obrisc}{
\toolsection{obrisc} is a compiler for the Oberon programming language targeting the RISC hardware architecture.
It generates machine code for RISC processors from modules written in Oberon and stores it in corresponding object files.
For debugging purposes, it also creates a debugging information file as well as an assembly file containing a listing of the generated machine code.
In addition, it stores the interface of each module in a symbol file which is required when other modules import the module.
Programs generated with this compiler require additional runtime support that is stored in the \file{ob\-risc\-run} library file.
\flowgraph{\resource{Oberon\\source code} \ar[r] & \toolbox{obrisc} \ar[r] \ar@/l/[d] \ar[rd] & \resource{object file} \\ \variable{ECSIMPORT} \ar[ru] & \resource{symbol\\files} \ar@/r/[u] & \resource{debugging\\information}}
\seeoberon\seeassembly\seerisc\seeobject\seedebugging
}

\providecommand{\obwasm}{
\toolsection{obwasm} is a compiler for the Oberon programming language targeting the WebAssembly architecture.
It generates machine code for WebAssembly targets from modules written in Oberon and stores it in corresponding object files.
For debugging purposes, it also creates a debugging information file as well as an assembly file containing a listing of the generated machine code.
In addition, it stores the interface of each module in a symbol file which is required when other modules import the module.
Programs generated with this compiler require additional runtime support that is stored in the \file{ob\-wasm\-run} library file.
\flowgraph{\resource{Oberon\\source code} \ar[r] & \toolbox{obwasm} \ar[r] \ar@/l/[d] \ar[rd] & \resource{object file} \\ \variable{ECSIMPORT} \ar[ru] & \resource{symbol\\files} \ar@/r/[u] & \resource{debugging\\information}}
\seeoberon\seeassembly\seewasm\seeobject\seedebugging
}

% converter tools

\providecommand{\dbgdwarf}{
\toolsection{dbgdwarf} is a DWARF debugging information converter tool.
It converts debugging information into the DWARF debugging data format and stores it in corresponding object files~\cite{dwarffile}.
The resulting debugging object files can be combined with runtime support that creates Executable and Linking Format (ELF) files~\cite{elffile}.
\flowgraph{\resource{debugging\\information} \ar[r] & \toolbox{dbgdwarf} \ar[r] & \resource{debugging\\object file}}
\seeobject\seedebugging
}

% assembler tools

\providecommand{\asmprint}{
\toolsection{asmprint} is a pretty printer for generic assembly code.
It reformats generic assembly code and writes it to the standard output stream.
\flowgraph{\resource{generic assembly\\source code} \ar[r] & \toolbox{asmprint} \ar[r] & \resource{reformatted\\source code}}
\seeassembly
}

\providecommand{\amdaasm}{
\toolsection{amd16asm} is an assembler for the AMD64 hardware architecture.
It translates assembly code into machine code for AMD64 processors and stores it in corresponding object files.
By default, the assembler generates machine code for the 16-bit operating mode defined by the AMD64 architecture.
\flowgraph{\resource{AMD16 assembly\\source code} \ar[r] & \toolbox{amd16asm} \ar[r] & \resource{object file}}
\seeassembly\seeamd\seeobject
}

\providecommand{\amdadism}{
\toolsection{amd16dism} is a disassembler for the AMD64 hardware architecture.
It translates machine code from object files targeting AMD64 processors into assembly code and writes it to the standard output stream.
It assumes that the machine code was generated for the 16-bit operating mode defined by the AMD64 architecture.
\flowgraph{\resource{object file} \ar[r] & \toolbox{amd16dism} \ar[r] & \resource{disassembly\\listing}}
\seeassembly\seeamd\seeobject
}

\providecommand{\amdbasm}{
\toolsection{amd32asm} is an assembler for the AMD64 hardware architecture.
It translates assembly code into machine code for AMD64 processors and stores it in corresponding object files.
By default, the assembler generates machine code for the 32-bit operating mode defined by the AMD64 architecture.
\flowgraph{\resource{AMD32 assembly\\source code} \ar[r] & \toolbox{amd32asm} \ar[r] & \resource{object file}}
\seeassembly\seeamd\seeobject
}

\providecommand{\amdbdism}{
\toolsection{amd32dism} is a disassembler for the AMD64 hardware architecture.
It translates machine code from object files targeting AMD64 processors into assembly code and writes it to the standard output stream.
It assumes that the machine code was generated for the 32-bit operating mode defined by the AMD64 architecture.
\flowgraph{\resource{object file} \ar[r] & \toolbox{amd32dism} \ar[r] & \resource{disassembly\\listing}}
\seeassembly\seeamd\seeobject
}

\providecommand{\amdcasm}{
\toolsection{amd64asm} is an assembler for the AMD64 hardware architecture.
It translates assembly code into machine code for AMD64 processors and stores it in corresponding object files.
By default, the assembler generates machine code for the 64-bit operating mode defined by the AMD64 architecture.
\flowgraph{\resource{AMD64 assembly\\source code} \ar[r] & \toolbox{amd64asm} \ar[r] & \resource{object file}}
\seeassembly\seeamd\seeobject
}

\providecommand{\amdcdism}{
\toolsection{amd64dism} is a disassembler for the AMD64 hardware architecture.
It translates machine code from object files targeting AMD64 processors into assembly code and writes it to the standard output stream.
It assumes that the machine code was generated for the 64-bit operating mode defined by the AMD64 architecture.
\flowgraph{\resource{object file} \ar[r] & \toolbox{amd64dism} \ar[r] & \resource{disassembly\\listing}}
\seeassembly\seeamd\seeobject
}

\providecommand{\armaasm}{
\toolsection{arma32asm} is an assembler for the ARM hardware architecture.
It translates assembly code into machine code for ARM processors executing A32 instructions and stores it in corresponding object files.
\flowgraph{\resource{ARM A32 assembly\\source code} \ar[r] & \toolbox{arma32asm} \ar[r] & \resource{object file}}
\seeassembly\seearm\seeobject
}

\providecommand{\armadism}{
\toolsection{arma32dism} is a disassembler for the ARM hardware architecture.
It translates machine code from object files targeting ARM processors executing A32 instructions into assembly code and writes it to the standard output stream.
\flowgraph{\resource{object file} \ar[r] & \toolbox{arma32dism} \ar[r] & \resource{disassembly\\listing}}
\seeassembly\seearm\seeobject
}

\providecommand{\armbasm}{
\toolsection{arma64asm} is an assembler for the ARM hardware architecture.
It translates assembly code into machine code for ARM processors executing A64 instructions and stores it in corresponding object files.
\flowgraph{\resource{ARM A64 assembly\\source code} \ar[r] & \toolbox{arma64asm} \ar[r] & \resource{object file}}
\seeassembly\seearm\seeobject
}

\providecommand{\armbdism}{
\toolsection{arma64dism} is a disassembler for the ARM hardware architecture.
It translates machine code from object files targeting ARM processors executing A64 instructions into assembly code and writes it to the standard output stream.
\flowgraph{\resource{object file} \ar[r] & \toolbox{arma64dism} \ar[r] & \resource{disassembly\\listing}}
\seeassembly\seearm\seeobject
}

\providecommand{\armcasm}{
\toolsection{armt32asm} is an assembler for the ARM hardware architecture.
It translates assembly code into machine code for ARM processors executing T32 instructions and stores it in corresponding object files.
\flowgraph{\resource{ARM T32 assembly\\source code} \ar[r] & \toolbox{armt32asm} \ar[r] & \resource{object file}}
\seeassembly\seearm\seeobject
}

\providecommand{\armcdism}{
\toolsection{armt32dism} is a disassembler for the ARM hardware architecture.
It translates machine code from object files targeting ARM processors executing T32 instructions into assembly code and writes it to the standard output stream.
\flowgraph{\resource{object file} \ar[r] & \toolbox{armt32dism} \ar[r] & \resource{disassembly\\listing}}
\seeassembly\seearm\seeobject
}

\providecommand{\avrasm}{
\toolsection{avrasm} is an assembler for the AVR hardware architecture.
It translates assembly code into machine code for AVR processors and stores it in corresponding object files.
The identifiers \texttt{RXL}, \texttt{RXH}, \texttt{RYL}, \texttt{RYH}, \texttt{RZL}, and \texttt{RZH} are predefined and name the corresponding registers.
The identifiers \texttt{SPL} and \texttt{SPH} are also predefined and evaluate to the address of the corresponding registers.
\flowgraph{\resource{AVR assembly\\source code} \ar[r] & \toolbox{avrasm} \ar[r] & \resource{object file}}
\seeassembly\seeavr\seeobject
}

\providecommand{\avrdism}{
\toolsection{avrdism} is a disassembler for the AVR hardware architecture.
It translates machine code from object files targeting AVR processors into assembly code and writes it to the standard output stream.
\flowgraph{\resource{object file} \ar[r] & \toolbox{avrdism} \ar[r] & \resource{disassembly\\listing}}
\seeassembly\seeavr\seeobject
}

\providecommand{\avrttasm}{
\toolsection{avr32asm} is an assembler for the AVR32 hardware architecture.
It translates assembly code into machine code for AVR32 processors and stores it in corresponding object files.
\flowgraph{\resource{AVR32 assembly\\source code} \ar[r] & \toolbox{avr32asm} \ar[r] & \resource{object file}}
\seeassembly\seeavrtt\seeobject
}

\providecommand{\avrttdism}{
\toolsection{avr32dism} is a disassembler for the AVR32 hardware architecture.
It translates machine code from object files targeting AVR32 processors into assembly code and writes it to the standard output stream.
\flowgraph{\resource{object file} \ar[r] & \toolbox{avr32dism} \ar[r] & \resource{disassembly\\listing}}
\seeassembly\seeavrtt\seeobject
}

\providecommand{\mabkasm}{
\toolsection{m68kasm} is an assembler for the M68000 hardware architecture.
It translates assembly code into machine code for M68000 processors and stores it in corresponding object files.
\flowgraph{\resource{68000 assembly\\source code} \ar[r] & \toolbox{m68kasm} \ar[r] & \resource{object file}}
\seeassembly\seemabk\seeobject
}

\providecommand{\mabkdism}{
\toolsection{m68kdism} is a disassembler for the M68000 hardware architecture.
It translates machine code from object files targeting M68000 processors into assembly code and writes it to the standard output stream.
\flowgraph{\resource{object file} \ar[r] & \toolbox{m68kdism} \ar[r] & \resource{disassembly\\listing}}
\seeassembly\seemabk\seeobject
}

\providecommand{\miblasm}{
\toolsection{miblasm} is an assembler for the MicroBlaze hardware architecture.
It translates assembly code into machine code for MicroBlaze processors and stores it in corresponding object files.
\flowgraph{\resource{MicroBlaze assembly\\source code} \ar[r] & \toolbox{miblasm} \ar[r] & \resource{object file}}
\seeassembly\seemibl\seeobject
}

\providecommand{\mibldism}{
\toolsection{mibldism} is a disassembler for the MicroBlaze hardware architecture.
It translates machine code from object files targeting MicroBlaze processors into assembly code and writes it to the standard output stream.
\flowgraph{\resource{object file} \ar[r] & \toolbox{mibldism} \ar[r] & \resource{disassembly\\listing}}
\seeassembly\seemibl\seeobject
}

\providecommand{\mipsaasm}{
\toolsection{mips32asm} is an assembler for the MIPS32 hardware architecture.
It translates assembly code into machine code for MIPS32 processors and stores it in corresponding object files.
\flowgraph{\resource{MIPS32 assembly\\source code} \ar[r] & \toolbox{mips32asm} \ar[r] & \resource{object file}}
\seeassembly\seemips\seeobject
}

\providecommand{\mipsadism}{
\toolsection{mips32dism} is a disassembler for the MIPS32 hardware architecture.
It translates machine code from object files targeting MIPS32 processors into assembly code and writes it to the standard output stream.
\flowgraph{\resource{object file} \ar[r] & \toolbox{mips32dism} \ar[r] & \resource{disassembly\\listing}}
\seeassembly\seemips\seeobject
}

\providecommand{\mipsbasm}{
\toolsection{mips64asm} is an assembler for the MIPS64 hardware architecture.
It translates assembly code into machine code for MIPS64 processors and stores it in corresponding object files.
\flowgraph{\resource{MIPS64 assembly\\source code} \ar[r] & \toolbox{mips64asm} \ar[r] & \resource{object file}}
\seeassembly\seemips\seeobject
}

\providecommand{\mipsbdism}{
\toolsection{mips64dism} is a disassembler for the MIPS64 hardware architecture.
It translates machine code from object files targeting MIPS64 processors into assembly code and writes it to the standard output stream.
\flowgraph{\resource{object file} \ar[r] & \toolbox{mips64dism} \ar[r] & \resource{disassembly\\listing}}
\seeassembly\seemips\seeobject
}

\providecommand{\mmixasm}{
\toolsection{mmixasm} is an assembler for the MMIX hardware architecture.
It translates assembly code into machine code for MMIX processors and stores it in corresponding object files.
The names of all special registers are predefined and evaluate to the corresponding number.
\flowgraph{\resource{MMIX assembly\\source code} \ar[r] & \toolbox{mmixasm} \ar[r] & \resource{object file}}
\seeassembly\seemmix\seeobject
}

\providecommand{\mmixdism}{
\toolsection{mmixdism} is a disassembler for the MMIX hardware architecture.
It translates machine code from object files targeting MMIX processors into assembly code and writes it to the standard output stream.
\flowgraph{\resource{object file} \ar[r] & \toolbox{mmixdism} \ar[r] & \resource{disassembly\\listing}}
\seeassembly\seemmix\seeobject
}

\providecommand{\orokasm}{
\toolsection{or1kasm} is an assembler for the OpenRISC 1000 hardware architecture.
It translates assembly code into machine code for OpenRISC 1000 processors and stores it in corresponding object files.
\flowgraph{\resource{OpenRISC 1000 assembly\\source code} \ar[r] & \toolbox{or1kasm} \ar[r] & \resource{object file}}
\seeassembly\seeorok\seeobject
}

\providecommand{\orokdism}{
\toolsection{or1kdism} is a disassembler for the OpenRISC 1000 hardware architecture.
It translates machine code from object files targeting OpenRISC 1000 processors into assembly code and writes it to the standard output stream.
\flowgraph{\resource{object file} \ar[r] & \toolbox{or1kdism} \ar[r] & \resource{disassembly\\listing}}
\seeassembly\seeorok\seeobject
}

\providecommand{\ppcaasm}{
\toolsection{ppc32asm} is an assembler for the PowerPC hardware architecture.
It translates assembly code into machine code for PowerPC processors and stores it in corresponding object files.
By default, the assembler generates machine code for the 32-bit operating mode defined by the PowerPC architecture.
\flowgraph{\resource{PowerPC assembly\\source code} \ar[r] & \toolbox{ppc32asm} \ar[r] & \resource{object file}}
\seeassembly\seeppc\seeobject
}

\providecommand{\ppcadism}{
\toolsection{ppc32dism} is a disassembler for the PowerPC hardware architecture.
It translates machine code from object files targeting PowerPC processors into assembly code and writes it to the standard output stream.
It assumes that the machine code was generated for the 32-bit operating mode defined by the PowerPC architecture.
\flowgraph{\resource{object file} \ar[r] & \toolbox{ppc32dism} \ar[r] & \resource{disassembly\\listing}}
\seeassembly\seeppc\seeobject
}

\providecommand{\ppcbasm}{
\toolsection{ppc64asm} is an assembler for the PowerPC hardware architecture.
It translates assembly code into machine code for PowerPC processors and stores it in corresponding object files.
By default, the assembler generates machine code for the 64-bit operating mode defined by the PowerPC architecture.
\flowgraph{\resource{PowerPC assembly\\source code} \ar[r] & \toolbox{ppc64asm} \ar[r] & \resource{object file}}
\seeassembly\seeppc\seeobject
}

\providecommand{\ppcbdism}{
\toolsection{ppc64dism} is a disassembler for the PowerPC hardware architecture.
It translates machine code from object files targeting PowerPC processors into assembly code and writes it to the standard output stream.
It assumes that the machine code was generated for the 64-bit operating mode defined by the PowerPC architecture.
\flowgraph{\resource{object file} \ar[r] & \toolbox{ppc64dism} \ar[r] & \resource{disassembly\\listing}}
\seeassembly\seeppc\seeobject
}

\providecommand{\riscasm}{
\toolsection{riscasm} is an assembler for the RISC hardware architecture.
It translates assembly code into machine code for RISC processors and stores it in corresponding object files.
The names of all special registers are predefined and evaluate to the corresponding number.
\flowgraph{\resource{RISC assembly\\source code} \ar[r] & \toolbox{riscasm} \ar[r] & \resource{object file}}
\seeassembly\seerisc\seeobject
}

\providecommand{\riscdism}{
\toolsection{riscdism} is a disassembler for the RISC hardware architecture.
It translates machine code from object files targeting RISC processors into assembly code and writes it to the standard output stream.
\flowgraph{\resource{object file} \ar[r] & \toolbox{riscdism} \ar[r] & \resource{disassembly\\listing}}
\seeassembly\seerisc\seeobject
}

\providecommand{\wasmasm}{
\toolsection{wasmasm} is an assembler for the WebAssembly architecture.
It translates assembly code into machine code for WebAssembly targets and stores it in corresponding object files.
The names of all special registers are predefined and evaluate to the corresponding number.
\flowgraph{\resource{WebAssembly assembly\\source code} \ar[r] & \toolbox{wasmasm} \ar[r] & \resource{object file}}
\seeassembly\seewasm\seeobject
}

\providecommand{\wasmdism}{
\toolsection{wasmdism} is a disassembler for the WebAssembly architecture.
It translates machine code from object files targeting WebAssembly targets into assembly code and writes it to the standard output stream.
\flowgraph{\resource{object file} \ar[r] & \toolbox{wasmdism} \ar[r] & \resource{disassembly\\listing}}
\seeassembly\seewasm\seeobject
}

% linker tools

\providecommand{\linklib}{
\toolsection{linklib} is an object file combiner.
It creates a static library file by combining all object files given to it into a single one.
\flowgraph{\resource{object files} \ar[r] & \toolbox{linklib} \ar[r] & \resource{library file}}
\seeobject
}

\providecommand{\linkbin}{
\toolsection{linkbin} is a linker for plain binary files.
It links all object files given to it into a single image and stores it in a binary file that begins with the first linked section.
It also creates a map file that lists the address, type, name and size of all used sections.
The filename extension of the resulting binary file can be specified by putting it into a constant data section called \texttt{\_extension}.
\flowgraph{\resource{object files} \ar[r] & \toolbox{linkbin} \ar[r] \ar[d] & \resource{binary file} \\ & \resource{map file}}
\seeobject
}

\providecommand{\linkmem}{
\toolsection{linkmem} is a linker for plain binary files partitioned into random-access and read-only memory.
It links all object files given to it into two distinct images, one for data sections and one for code and constant data sections, and stores each image in a binary file that begins with the first linked section of the corresponding type.
It also creates a map file that lists the address, type, name and size of all used sections.
\flowgraph{\resource{object files} \ar[r] & \toolbox{linkmem} \ar[r] \ar[d] & \resource{RAM file/\\ROM file} \\ & \resource{map file}}
\seeobject
}

\providecommand{\linkprg}{
\toolsection{linkprg} is a linker for GEMDOS executable files.
It links all object files given to it into a single image and stores the image in an Atari GEMDOS executable file~\cite{gemdosfile}.
It also creates a map file that lists the address relative to the text segment, type, name and size of all used sections.
The filename extension of the resulting executable file can be specified by putting it into a constant data section called \texttt{\_extension}.
The GEMDOS executable file format requires all patch patterns of absolute link patches to consist of four full bitmasks with descending offsets.
\flowgraph{\resource{object files} \ar[r] & \toolbox{linkprg} \ar[r] \ar[d] & \resource{executable file} \\ & \resource{map file}}
\seeobject
}

\providecommand{\linkhex}{
\toolsection{linkhex} is a linker for Intel HEX files.
It links all code sections of the object files given to it into single image and stores the image in an Intel HEX file~\cite{hexfile} that begins with the first linked section.
It also creates a map file that lists the address, type, name and size of all used sections.
\flowgraph{\resource{object files} \ar[r] & \toolbox{linkhex} \ar[r] \ar[d] & \resource{HEX file} \\ & \resource{map file}}
\seeobject
}

\providecommand{\mapsearch}{
\toolsection{mapsearch} is a debugging tool.
It searches map files generated by linker tools for the name of a binary section that encompasses a memory address read from the standard input stream.
If additionally provided with one or more object files, it also stores an excerpt thereof in a separate object file called map search result which only contains the identified binary section for disassembling purposes.
\flowgraph{& \resource{map files/\\object files} \ar[d] \\ \resource{memory\\address} \ar[r] & \toolbox{mapsearch} \ar[r] \ar[d] & \resource{section name/\\relative offset} \\ & \resource{object file\\excerpt}}
\seeobject
}


\startchapter{Extensions}{Extensions to the \ecs{}}{extensions}
{The \ecs{} features a variety of tools like compilers, assemblers, and linkers.
All of these tools implement different programming languages, target different hardware architectures, or support different runtime environments respectively.
This \documentation{} describes the abstractions and utilities provided by the \ecs{} that facilitate the development of additional tools and runtime environments.}

\epigraph{Nought may endure but mutability.}{Percy Bysshe Shelley}

\section{Introduction}

The \ecs{} is a complete tool chain targeting a variety of programming languages, hardware architectures, and runtime environments.
Internally, it features several abstractions that are helpful for programmers implementing these tools.
Figure~\ref{fig:extabstractions} shows all abstractions and visualizes possible combinations of the different tools.

\begin{figure}
\flowgraph{
\resource{source code} \ar[d] & \resource{source code} \ar[d] & \resource{source code} \ar[d] \\
\converter{Front-End\\for programming\\language \textit{A}} \ar[rd] & \converter{Front-End\\for programming\\language \textit{B}} \ar[d] & \converter{Front-End\\for programming\\language \textit{C}} \ar[ld] \\
& \resource{intermediate\\code} \ar[ld] \ar[d] & \resource{assembly\\source code} \ar[d] \\
\converter{Back-End\\for hardware\\architecture \textit{X}} \ar[d] \ar[rd] & \converter{Back-End\\for hardware\\architecture \textit{Y}} \ar[d] & \converter{Assembler\\for hardware\\architecture \textit{Z}} \ar[ld] \\
\resource {debugging\\information} & \resource{object\\file} \ar[ld] \ar[d] \ar[rd] \\
\converter{Linker\\for runtime\\environment \textit{1}} \ar[d] & \converter{Linker\\for runtime\\environment \textit{2}} \ar[d] & \converter{Disassembler\\for hardware\\architecture \textit{3}} \ar[d] \\
\resource{executable\\binary image} & \resource{executable\\binary image} & \resource{disassembly\\listing} \\
}\caption{The main abstractions of the \ecs{}}
\label{fig:extabstractions}
\end{figure}

The goal of the abstractions is to enable and simplify all possible combinations of the different aspects of the tool chain.
This is achieved by reducing the required programming interfaces to a minimum.
As a result, adding support for another programming language, hardware architecture, or runtime environment is also simplified.
The following sections describe the steps that are required to extend the \ecs{} into this direction.

\section{Diagnostics}

Compilers and assemblers as provided by the \ecs{} translate source code written by programmers.
Therefore, source code may contain syntax or semantic errors which have to be diagnosed by these tools.
The \ecs{} provides a generic diagnostics facility for programmers that enables a consistent reporting of errors, warnings, and additional information.
Besides the actual text, diagnostic messages also include the name of the diagnosed source code and the position therein.
Unless otherwise specified, the source code position is based on text lines.

\section{Application Drivers}

The \ecs{} provides a generic driver facility for programmers that allows all applications written with it to share the same user interface.
This framework covers the overall error handling as well as the processing scheme for all kinds of input given to the applications.
\interface

\section{Programming Languages}

This section describes the tools and representations typically provided by the \ecs{} for a specific programming language and how they are implemented.
Figure~\ref{fig:extdataflow} shows the tools and representations of typical implementations of programming languages.

\begin{figure}
\flowgraph{
& & \resource{source code} \ar[d] & & \\
& & \converter{Lexer} \ar[d] \\
& & \resource{tokens} \ar[d] \\
& & \converter{Parser} \ar[d] \\
*=<2em,0em>\txt{\rotatebox{90}{Front-End}} \ar@{-}`r[uuuu]`[rruuuu]+D \ar@{-}`r[dddd]`[rrdddd]+U & \converter{Serializer} \ar[d] & \resource{abstract\\syntax tree} \ar[l] \ar[d] \ar[r] & \converter{Pretty Printer} \ar[d] \\
& \resource{internal\\representation} & \converter{Semantic\\Checker} \ar[d] & \resource{reformatted\\source code} & *=<2em,0em>\txt{\rotatebox{270}{Compiler}} \ar@{-}`l[uuuuu]`[lluuuuu]+D \ar@{-}`l[ddddd]`[llddddd]+U \\
& \converter{Interpreter} \ar@/l/[d] & \resource{attributed\\syntax tree} \ar[l] \ar[d] \ar[r] & \converter{Transpiler} \ar[d] \\
& \resource{input/\\output} \ar@/r/[u] & \converter{Intermediate\\Code Emitter} \ar[d] & \resource{translated\\source code} \\
& & \resource{intermediate\\code} \ar[d] \ar@/u/[r] & \converter{Optimizer} \ar@/d/[l] \\
*=<2em,0em>\txt{\rotatebox{90}{Back-End}} \ar@{-}`r[u]`[rru]+D \ar@{-}`r[d]`[rrd]+U & \resource{assembly\\listing} & \converter{Machine Code\\Generator} \ar[l] \ar[d] \ar[r] & \resource{debugging\\information} \\
& & \resource{object file} & & \\
}\caption{Data flow within a typical implementation of a programming language}
\label{fig:extdataflow}
\end{figure}

The source code of a program written in a programming language is most often represented in an abstract syntax tree.
This tree is created by a \emph{parser}\index{Parsers} that recognizes the syntax of the programming language.
Parsers typically use a so-called \emph{lexer}\index{Lexers} that is able to tokenize the source code and extract symbols like keywords and operators.
The abstract syntax tree is the base for all tools described in the following sections.
Therefore, all of these tools contain a parser and a lexer in order to create the syntax tree.
Although these tools use the information represented in the syntax tree for different purposes, they all traverse it in one or several consecutive stages.

\subsection{Pretty Printers}

Pretty printers just traverse the complete abstract syntax tree by reconverting its nodes into tokens again.
These tokens are then printed using a consistent and well-arranged layout.
Pretty printers usually do not alter the abstract syntax tree and are often able to reformat source code that contains semantic errors.

\subsection{Semantic Checkers}

Semantic checkers traverse the abstract syntax tree and attribute its nodes with semantic information.
They are able to diagnose violations of the semantic rules of the programming language.
For that purpose, semantic checkers sometimes use additional semantic information that is stored separately.
The functionality provided by semantic checkers is in many cases reused by the remaining tools of this section.
Standalone semantic checkers are useful for automated testing and verification of the implementation of a programming language.

\subsection{Serializers}

Serializers dump all information stored in the internal representation of a program in a human readable format for debugging purposes.
The \ecs{} provides a generic serialization facility for representing attributed syntax trees as XML documents.
This serialization format has the advantage of being standardized and easy to parse for other development tools potentially making use of the same internal program representation.

\subsection{Interpreters}

Interpreters are able to execute the program given in the syntax tree and to emulate a complete runtime environment for it.
They do not translate the syntax tree into another form, but traverse its nodes by simulating their runtime behavior as defined by the programming language.
Interpreters are useful for automated testing and verification of the implementation of a programming language or executing scripts inside a program.
For this particular case the \ecs{} provides a generic framework that allows embedded interpreters to interface with their environment.

\subsection{Transpilers}

Transpilers translate programs written in one programming language into programs written in other programming languages.
They therefore behave like pretty printers, except that the target programming language differs from the source programming language.
Additionally, transpilers usually need the semantic information provided by semantic checkers in order to translate the source code correctly.

\subsection{Documentation Generators}

Documentation generators extract the structure of the source code and combine it with annotations provided by the programmer into generic documentations.
This generic representation of the extracted information is used afterward to generate documents of different formats.
\seedocumentation

\subsection{Intermediate Code Emitters}

The \ecs{} defines an intermediate code representation that is able to represent arbitrary programs using instructions for an abstract machine.
This intermediate representation can be translated into actual machine code by machine code generators, see Section~\ref{sec:extgenerators}
An intermediate code emitter traverses syntax trees and translates theirs nodes into intermediate code sections and instructions that correspond to the runtime behavior of the original programs.
The \ecs{} provides a generic intermediate code emitter that is a base for all concrete emitters and simplifies the generation of intermediate code.
\seecode

\subsection{Front-Ends}\label{sec:extfrontends}

A front-end combines all of the previous stages required to transform the original source code into an intermediate code representation thereof.
This most often includes the parser, the semantic checker, and the intermediate code emitter.
Since this intermediate representation can be translated later into machine code for a variety of hardware architectures,
programmers only have to provide a single front-end instead of one for each possible combination thereof.
In combination with the intermediate code interpreter, they are useful for automated testing and verification of the generated intermediate code.

\subsection{Compilers}

Compilers translate source code written in a specific programming language into machine code targeting a specific hardware architecture.
They therefore just combine the output of one specific front-end with the input for a specific back-end, see Sections~\ref{sec:extfrontends} and~\ref{sec:extbackends}.
In the end, compilers write the resulting machine code into object files.
\seeobject
If programmers add support for an additional programming language by implementing a front-end accordingly,
or if they support a new hardware architecture by implementing an additional back-end, compilers for all new combinations just get available without further ado.

\section{Hardware Architectures}

This section describes the tools and components typically provided by the \ecs{} for a specific hardware architecture and how they are implemented.
The support for a hardware architecture is generally based on a powerful representation of its processor instructions as shown in Figure~\ref{fig:extinstructions}.
This abstraction must be able to represent any valid combination of mnemonic and operands of the instruction set of the target hardware architecture.
In order to be most useful, this abstract representation must be able to be instantiated in the following three ways:

\begin{figure}
\flowgraph{
\resource{Assemblers} \ar@{-}`d[]+L`[d]+L \ar@{-}`d[]+R`[d]+R & \resource{Generators} \ar@{-}`d[]+L`[d]+L \ar@{-}`d[]+R`[d]+R & \resource{Disassemblers} \ar@{-}`d[]+L`[ddd]+L \ar@{-}`d[]+R`[ddd]+R \\
\resource{textual\\representation} \ar@{-->}[dd] \ar[rd] & \resource{program-driven\\instantiation} \ar[d] & \resource{binary\\encoding} \ar[ld] \ar@{-->}[dd] \\
& \converter{Abstract\\Instruction} \ar[ld] \ar@{-->}`d[ld][ld] \ar[rd] \ar@{-->}`d[rd][rd] \ar[rd] \\
\resource{binary\\encoding} & & \resource{textual\\representation} \\
}\caption{Abstract representation of instructions}
\label{fig:extinstructions}
\end{figure}

\begin{itemize}

\item Translation from Text\nopagebreak

A concrete instruction and its operands have to be recognized by their textual representation.
Usually, the documentation of the target hardware architecture defines how its instructions are textually represented.
This functionality is needed by assemblers, see Section~\ref{sec:extassemblers}.

\item Instantiation by Code\nopagebreak

A concrete instruction shall be instantiatable by providing its mnemonic and an abstract representation of its operands.
The operands themselves are composed of immediate values, registers, or special-purpose data addressing.
This functionality is needed for the translation from text as well as by machine code generators, see Section~\ref{sec:extgenerators}.

\item Decoding from Machine Code\nopagebreak

A representation of a concrete instruction with its operands must also be creatable by decoding one or more binary octets.
The actual encoding and decoding of instructions is defined by the instruction set of the target hardware architecture.
This functionality is needed by disassemblers, see Section~\ref{sec:extdisassemblers}.

\end{itemize}

Additionally, instances of the abstract instruction representation must also be able to emit themselves in the following two ways:

\begin{itemize}

\item Encoding into Machine Code\nopagebreak

Any representation of a valid instruction shall be encoded into one or more binary octets.
This functionality is needed by assemblers and machine code generators, see Sections~\ref{sec:extassemblers} and~\ref{sec:extgenerators}.

\item Translation into Text\nopagebreak

Any representation of a valid instruction and its operands shall be translated into its textual representation.
This functionality is needed by disassemblers and machine code generators, see Sections~\ref{sec:extdisassemblers} and~\ref{sec:extgenerators}.

\end{itemize}

Usually, all combinations of instruction mnemonics and operand types that are valid according to the instruction set of the target hardware architecture are stored in a table.
This table often also contains information about how a concrete instruction with its mnemonic and operands is represented using machine code.
This information is needed for encoding and decoding instructions as discussed above.

\subsection{Assemblers}\label{sec:extassemblers}

All assemblers featured by the \ecs{} implement the generic assembly language.
It provides a generic abstraction for any textual representation of processor instructions.
It also supports common functionality like directives or the evaluation of arithmetic, bitwise, and logical operations.
\seeassembly
The \ecs{} provides a generic assembler that is a base for all concrete assemblers and implements the generic assembly language.
It performs tasks like managing sections, evaluating expressions, and writing the resulting object files.
Concrete assemblers only have to apply the translation of a simple textual representation of a single instruction into its machine code encoding.

\subsection{Disassemblers}\label{sec:extdisassemblers}

The \ecs{} provides a generic disassembler that is a base for all concrete disassemblers.
It is able to process the sections stored in an object file and to print their textual representation.
Data sections are represented using the generic data directives supported by the generic assembly language,
while the machine code stored in code sections are passed to the concrete disassemblers.
They only have to provide the encoding of some binary octets representing a single instruction and to print its textual representation.

\subsection{Machine Code Generators}\label{sec:extgenerators}

Machine code generators translate intermediate code into equivalent machine code for the target hardware architecture.
The \ecs{} provides a generic machine code generator that is a base for all concrete generators.
It is able to manage intermediate code sections and process the instructions that have the same binary encoding across all concrete generators.
This includes all instructions contained in data sections or instructions that are not natively supported by the concrete machine code generator.
The generators translate one intermediate section at a time and create a corresponding object file section and a debugging information entry.
Additionally, generators are able to create an assembly code listing of the generated machine code for verification and debugging purposes.
Since the resulting assembly code listing is by design based on the generic assembly language, it can also be processed by the corresponding assembler yielding an exact copy of the original machine code.

\subsection{Back-Ends}\label{sec:extbackends}

Back-ends combine the assembler and the machine code generator for a specific hardware architecture and provide an abstract description of some of the architectural characteristics that are necessary for the generation of intermediate code.
This includes platform-specific information about address and default integer sizes as well as data alignment constraints for example.
The abstract description is designed to be passed to front-ends for programming languages that need this information, see Section~\ref{sec:extfrontends}.

\section{Runtime Environments}

The \ecs{} supports several different runtime environments.
A runtime environment is characterized by the actual hardware architecture it is running on, as well as the underlying operating system.

\subsection{Linkers}

In order to run programs created by compilers and assemblers of the \ecs{} on a specific operating system or machine,
there must be a linker that generates the output files that are executable on the target platform.
The \ecs{} provides a generic linker that is able to map all sections of several object files into one or two binary arrangements and do the necessary linking therein.
Two binary arrangements are needed, when data and code have to be arranged in separate address spaces.
A concrete linker has just to write the binary data into a file using the target file format.
Oftentimes however, binary file formats can also be represented using only the features already provided by the generic assembly language.
In these cases, the executable output files can also be created using the plain binary file linker provided by the \ecs{} and no specific linker is necessary.

\subsection{Runtime Support}

Runtime support is needed for the initialization at the beginning of the program execution as well as the implementation of some standard functions that interface the environment.
This support depends on the actual hardware architecture, the file format of the executable file, as well as the underlying operating system.
Additional runtime support is needed for intermediate code instructions that are not natively supported by the machine code generator for a specific hardware architecture.
This support depends only on the actual hardware architecture and is the same across different operating systems.
Some programming languages may also require runtime support for their language features.
The runtime support for a language is bundled into so-called \emph{library files}\index{Library files}.
A library file is a combination of object files targeting a single hardware architecture.

\section{Development Tools}

The source code of the \ecs{} is accompanied by a makefile and some utility tools for developers.
These facilities are not required to build, execute, test, or modify the \ecs{} but are intended to simplify these activities.
The remainder of this section describes all utilities provided by the \ecs{} and their individual command-line interfaces.
The capabilities of the makefile on the other hand are described in a readme file also supplied with the source code.

\toolsection{depwalk} is a utility tool for dependency walking.
It accepts a list of \cpp{} source files and writes the set of all directly and indirectly included header files of each source file to the standard output stream.
The output is designed to be included in the makefile of the \ecs{} in order to consistently track all dependencies when changing the include directives of one or more of its source files.

\flowgraph{\resource{\cpp{}\\source file} \ar[r] & \converter{depwalk} \ar[r] & \resource{dependency\\list}}

\toolsection{ecsd} is a driver utility tool which conveniently invokes the tools of the \ecs{} with appropriate command-line arguments and environment variables.
Its functionality and command-line interface are described in \Documentation{}~\documentationref{interface}{User Interface}.

\flowgraph{\resource{input\\files} \ar[r] & \converter{ecsd} \ar[r] & \resource{executable\\file}}

For developers of the \ecs{} it additionally provides the \texttt{-m}~command-line flag which automatically builds all dependencies of a tool before invoking it.
A special target environment called \environment{code} allows exposing and executing the intermediate code representation of a program passed between front-ends and back-ends.
For the purpose of creating and debugging runtime environments on the other hand, the driver allows targeting freestanding environments for all supported hardware architectures using the \texttt{-t} option.
\ifbook The available names correspond to the common architecture prefixes and suffixes of the \ref*{tools:compilers}~compiler and assembler tools listed in Table~\ref{tab:tools} on page~\pageref{tab:tools}. \fi

\toolsection{hexdump} is a utility tool for viewing binary files.
It accepts the name of a binary file and writes its contents in hexadecimal form to the standard output stream.
If additionally provided with a corresponding map file as generated by linkers, it lists all sections contained therein and colors their contents.

\flowgraph{\resource{binary file/\\map file} \ar[r] & \converter{hexdump} \ar[r] & \resource{binary contents/\\colored sections}}

\toolsection{linecheck} is a utility tool for checking text lines.
It accepts a list of plain text files and checks their contents for consistent use of white space.
The complete source code of the \ecs{} is stored in plain text files and validated using this tool.

\flowgraph{\resource{text\\files} \ar[r] & \converter{linecheck} \ar[r] & \resource{diagnostic\\messages}}

\toolsection{regtest} is a utility tool for regression testing.
It accepts a quoted command line and the name of a plain text file called a test suite that contains an arbitrary number of tests.
It notes the result of executing the command once for each test and summarizes the differences between consecutive runs if provided with an optional result file for regression testing.
Each test consists of a short description followed by some contrived input that gets stored in a specified temporary file and is intended to cause the command execution to either succeed or fail.

\flowgraph{\resource{command/\\test suite} \ar[r] & \converter{regtest} \ar@{-->}[ld] \ar@{~}[d] \ar@/u/[r] & \resource{test\\results} \ar@/d/[l] \\ \resource{temporary\\input file} \ar[r] & \converter{command} \ar[r] & \resource{success/\\failure} \ar@{-->}[lu]}

Each test is described by a single line of text beginning with the identifier \texttt{positive} or \texttt{negative} indicating the expected result of the command execution, followed by a colon and a unique name.
The actual input of a test consists of all indented text lines following its description ignoring the first horizontal tab character of each line.
Lines beginning with a number sign character denote comments for annotating and structuring test suites and are completely ignored.

\concludechapter


\appendix
\phantomsection\addcontentsline{toc}{part}{Addendum}
\addtocontents{toc}{\protect\setcounter{tocdepth}{0}}
\part*{Addendum}
% Presentation material for the Eigen Compiler Suite
% Copyright (C) Florian Negele

% This file is part of the Eigen Compiler Suite.

% Permission is granted to copy, distribute and/or modify this document
% under the terms of the GNU Free Documentation License, Version 1.3
% or any later version published by the Free Software Foundation.

% You should have received a copy of the GNU Free Documentation License
% along with the ECS.  If not, see <https://www.gnu.org/licenses/>.

% Generic documentation utilities
% Copyright (C) Florian Negele

% This file is part of the Eigen Compiler Suite.

% Permission is granted to copy, distribute and/or modify this document
% under the terms of the GNU Free Documentation License, Version 1.3
% or any later version published by the Free Software Foundation.

% You should have received a copy of the GNU Free Documentation License
% along with the ECS.  If not, see <https://www.gnu.org/licenses/>.

\providecommand{\cpp}{C\texttt{++}}
\providecommand{\opt}{_\mathit{opt}}
\providecommand{\tool}[1]{\texttt{#1}}
\providecommand{\version}{Version 0.0.40}
\providecommand{\resource}[1]{*++\txt{#1}}
\providecommand{\ecs}{Eigen Compiler Suite}
\providecommand{\changed}[1]{\underline{#1}}
\providecommand{\toolbox}[1]{\converter{#1}}
\providecommand{\file}{}\renewcommand{\file}[1]{\texttt{#1}}
\providecommand{\alignright}{\hfill\linebreak[0]\hspace*{\fill}}
\providecommand{\converter}[1]{*++[F][F*:white][F,:gray]\txt{#1}}
\providecommand{\documentation}{\ifbook chapter\else document\fi}
\providecommand{\Documentation}{\ifbook Chapter\else Document\fi}
\providecommand{\variable}[1]{\resource{\texttt{\small#1}\\variable}}
\providecommand{\documentationref}[2]{\ifbook\ref{#1}\else``\href{#1}{#2}''~\cite{#1}\fi}
\providecommand{\objfile}[1]{\texttt{#1}\index[runtime]{#1 object file@\texttt{#1} object file}}
\providecommand{\libfile}[1]{\texttt{#1}\index[runtime]{#1 library file@\texttt{#1} library file}}
\providecommand{\epigraph}[2]{\ifbook\begin{quote}\flushright\textit{#1}\par--- #2\end{quote}\fi}
\providecommand{\environmentvariable}[1]{\texttt{#1}\index{Environment variables!#1@\texttt{#1}}}
\providecommand{\environment}[1]{\texttt{#1}\index[environment]{#1 environment@\texttt{#1} environment}}
\providecommand{\toolsection}{}\renewcommand{\toolsection}[1]{\subsection{#1}\label{\prefix:#1}\tool{#1}}
\providecommand{\instruction}{}\renewcommand{\instruction}[2]{\noindent\qquad\pdftooltip{\texttt{#1}}{#2}\refstepcounter{instruction}\par}
\providecommand{\flowgraph}{}\renewcommand{\flowgraph}[1]{\par\sffamily\begin{displaymath}\xymatrix@=4ex{#1}\end{displaymath}\normalfont\par}
\providecommand{\instructionset}{}\renewcommand{\instructionset}[4]{\setcounter{instruction}{0}\begin{multicols}{\ifbook#3\else#4\fi}[{\captionof{table}[#2]{#2 (\ref*{#1:instructions}~instructions)}\label{tab:#1set}\vspace{-2ex}}]\footnotesize\raggedcolumns\input{#1.set}\label{#1:instructions}\end{multicols}}

\providecommand{\gpl}{GNU General Public License}
\providecommand{\rse}{ECS Runtime Support Exception}
\providecommand{\fdl}{\href{https://www.gnu.org/licenses/fdl.html}{GNU Free Documentation License}}

\providecommand{\docbegin}{}
\providecommand{\docend}{}
\providecommand{\doclabel}[1]{\hypertarget{#1}}
\providecommand{\doclink}[2]{\hyperlink{#1}{#2}}
\providecommand{\docsection}[3]{\hypertarget{#1}{\subsection{#2}}\label{sec:#1}\index[library]{#2@#3}}
\providecommand{\docsectionstar}[1]{}
\providecommand{\docsubbegin}{\begin{description}}
\providecommand{\docsubend}{\end{description}}
\providecommand{\docsubsection}[3]{\item[\hypertarget{#1}{#2}]\index[library]{#2@#3}}
\providecommand{\docsubsectionstar}[1]{\smallskip}
\providecommand{\docsubsubsection}[3]{\docsubsection{#1}{#2}{#3}}
\providecommand{\docsubsubsectionstar}[1]{}
\providecommand{\docsubsubsubsection}[3]{}
\providecommand{\docsubsubsubsectionstar}[1]{}
\providecommand{\doctable}{}

\providecommand{\debuggingtool}{}\renewcommand{\debuggingtool}{This tool is provided for debugging purposes.
It allows exposing and modifying an internal data structure that is usually not accessible.
}

\providecommand{\interface}{All tools accept command-line arguments which are taken as names of plain text files containing the source code.
If no arguments are provided, the standard input stream is used instead.
Output files are generated in the current working directory and have the same name as the input file being processed whereas the filename extension gets replaced by an appropriate suffix.
\seeinterface
}

\providecommand{\license}{\noindent Copyright \copyright{} Florian Negele\par\medskip\noindent
Permission is granted to copy, distribute and/or modify this document under the terms of the
\fdl{}, Version 1.3 or any later version published by the \href{https://fsf.org/}{Free Software Foundation}.
}

\providecommand{\ecslogosurface}{
\fill[darkgray] (0,0,0) -- (0,0,3) -- (0,3,3) -- (0,3,1) -- (0,4,1) -- (0,4,3) -- (0,5,3) -- (0,5,0) -- (0,2,0) -- (0,2,2) -- (0,1,2) -- (0,1,0) -- cycle;
\fill[gray] (0,5,0) -- (0,5,3) -- (1,5,3) -- (1,5,1) -- (2,5,1) -- (2,5,3) -- (3,5,3) -- (3,5,0) -- cycle;
\fill[lightgray] (0,0,0) -- (0,1,0) -- (2,1,0) -- (2,4,0) -- (1,4,0) -- (1,3,0) -- (2,3,0) -- (2,2,0) -- (0,2,0) -- (0,5,0) -- (3,5,0) -- (3,0,0) -- cycle;
\begin{scope}[line width=0.5]
\begin{scope}[gray]
\draw (0,0,0) -- (0,1,0);
\draw (2,1,0) -- (2,2,0);
\draw (0,1,2) -- (0,2,2);
\draw (0,2,0) -- (0,5,0);
\draw (2,3,0) -- (2,4,0);
\end{scope}
\begin{scope}[lightgray]
\draw (0,1,0) -- (0,1,2);
\draw (0,3,1) -- (0,3,3);
\draw (0,5,0) -- (0,5,3);
\draw (2,5,1) -- (2,5,3);
\end{scope}
\begin{scope}[white]
\draw (0,1,0) -- (2,1,0);
\draw (1,3,0) -- (2,3,0);
\draw (0,5,0) -- (3,5,0);
\end{scope}
\end{scope}
}

\providecommand{\ecslogo}[1]{
\begin{tikzpicture}[scale={(#1)/((sin(45)+cos(45))*3cm)},x={({-cos(45)*1cm},{sin(45)*sin(30)*1cm})},y={({0cm},{(cos(30)*1cm})},z={({sin(45)*1cm},{cos(45)*sin(30)*1cm})}]
\begin{scope}[darkgray,line width=1]
\draw (0,0,0) -- (0,0,3) -- (0,3,3) -- (2,3,3) -- (2,5,3) -- (3,5,3) -- (3,5,0) -- (3,0,0) -- cycle;
\draw (0,3,1) -- (0,4,1) -- (0,4,3) -- (0,5,3) -- (1,5,3) -- (1,5,1) -- (2,5,1);
\draw (1,3,0) -- (1,4,0) -- (2,4,0);
\end{scope}
\fill[darkgray] (2,0,0) -- (2,0,3) -- (2,5,3) -- (2,5,1) -- (2,4,1) -- (2,4,0) -- cycle;
\fill[lightgray] (2,0,2) -- (0,0,2) -- (0,2,2) -- (2,2,2) -- cycle;
\fill[gray] (0,1,0) -- (2,1,0) -- (2,1,2) -- (0,1,2) -- cycle;
\fill[gray] (0,3,1) -- (0,3,3) -- (2,3,3) -- (2,3,0) -- (1,3,0) -- (1,3,1) -- cycle;
\ecslogosurface
\end{tikzpicture}
}

\providecommand{\shadowedecslogo}[3]{
\begin{tikzpicture}[scale={(#1)/((sin(#2)+cos(#2))*3cm)},x={({-cos(#2)*1cm},{sin(#2)*sin(#3)*1cm})},y={({0cm},{(cos(#3)*1cm})},z={({sin(#2)*1cm},{cos(#2)*sin(#3)*1cm})}]
\shade[top color=lightgray!50!white,bottom color=white,middle color=lightgray!50!white] (0,0,0) -- (3,0,0) -- (3,{-0.5-3*sin(#2)*sin(#3)/cos(#3)},0) -- (0,-0.5,0) -- cycle;
\shade[top color=darkgray!50!gray,bottom color=white,middle color=darkgray!50!white] (0,0,0) -- (0,0,3) -- (0,{-0.5-3*cos(#2)*sin(#3)/cos(#3)},3) -- (0,-0.5,0) -- cycle;
\begin{scope}[y={({(cos(#2)+sin(#2))*0.5cm},{(cos(#2)*sin(#3)-sin(#2)*sin(#3))*0.5cm})}]
\useasboundingbox (3,0,0) -- (0,0,0) -- (0,0,3);
\shade[left color=darkgray!80!black,right color=lightgray,middle color=gray] (0,0,0) -- (0,1,0) -- (0,1,0.5) -- (0,2,0) -- (0,5,0) -- (0,5,3) -- (1,5,3) -- (1,4,3) -- (1,4,2.5) -- (1,3,3) -- (2,5,3) -- (3,5,3) -- (3,0,3) -- cycle;
\clip (0,0,0) -- (0,0,3) -- ({-3*sin(#2)/cos(#2)},0,0) -- cycle;
\shade[left color=darkgray,right color=lightgray!50!gray] (0,0,0) -- (0,1,0) -- (0,1,0.5) -- (0,2,0) -- (0,5,0) -- (0,5,3) -- (1,5,3) -- (1,4,3) -- (1,4,2.5) -- (1,3,3) -- (2,5,3) -- (3,5,3) -- (3,0,3) -- cycle;
\end{scope}
\shade[left color=darkgray,right color=darkgray!80!black] (2,0,0) -- (2,0,3) -- (2,5,3) -- (2,5,1) -- (2,4,1) -- (2,4,0) -- cycle;
\shade[left color=darkgray!90!black,right color=gray!80!darkgray] (2,0,2) -- (0,0,2) -- (0,2,2) -- (2,2,2) -- cycle;
\shade[top color=darkgray!90!black,bottom color=gray!80!darkgray] (0,1,0) -- (2,1,0) -- (2,1,2) -- (0,1,2) -- cycle;
\shade[top color=darkgray!90!black,bottom color=gray!80!darkgray] (0,3,1) -- (0,3,3) -- (2,3,3) -- (2,3,0) -- (1,3,0) -- (1,3,1) -- cycle;
\fill[gray] (2,1,0) -- (1.5,1,0.5) -- (0,1,0.5) -- (0,1,0) -- cycle;
\fill[gray] (1,3,2) -- (0.5,3,2) -- (0.5,3,3) -- (1,3,3) -- cycle;
\fill[gray] (2,3,0) -- (1.5,3,0.5) -- (1,3,0.5) -- (1,3,0) -- cycle;
\ecslogosurface
\end{tikzpicture}
}

\providecommand{\cpplogo}[1]{
\begin{tikzpicture}[scale=(#1)/512em]
\fill[gray] (435.2794,398.7159) -- (247.1911,507.3075) .. controls (236.3563,513.5642) and (218.6240,513.5642) .. (207.7892,507.3075) -- (19.7009,398.7159) .. controls (8.8646,392.4606) and (0.0000,377.1043) .. (0.0000,364.5924) -- (0.0000,147.4076) .. controls (0.8430,132.8363) and (8.2856,120.7683) .. (19.7009,113.2842) -- (207.7892,4.6926) .. controls (218.6240,-1.5642) and (236.3564,-1.5642) .. (247.1911,4.6926) -- (435.2794,113.2842) .. controls (447.5273,121.4304) and (454.4987,133.6918) .. (454.9803,147.4076) -- (454.9803,364.5924) .. controls (454.5404,377.7571) and (446.6566,391.0351) .. (435.2794,398.7159) -- cycle(75.8301,255.9993) .. controls (74.9389,404.0881) and (273.2892,469.4783) .. (358.8263,331.8769) -- (293.1917,293.8965) .. controls (253.5702,359.4301) and (155.1909,335.9977) .. (151.6601,255.9993) .. controls (152.7204,182.2703) and (249.4137,148.0211) .. (293.1961,218.1065) -- (358.8308,180.1276) .. controls (283.4477,49.2645) and (79.6318,96.3470) .. (75.8301,255.9993) -- cycle(379.1503,247.5747) -- (362.2982,247.5747) -- (362.2982,230.7226) -- (345.4490,230.7226) -- (345.4490,247.5747) -- (328.5969,247.5747) -- (328.5969,264.4254) -- (345.4490,264.4254) -- (345.4490,281.2759) -- (362.2982,281.2759) -- (362.2982,264.4254) -- (379.1503,264.4254) -- cycle(442.3420,247.5747) -- (425.4899,247.5747) -- (425.4899,230.7226) -- (408.6408,230.7226) -- (408.6408,247.5747) -- (391.7886,247.5747) -- (391.7886,264.4254) -- (408.6408,264.4254) -- (408.6408,281.2759) -- (425.4899,281.2759) -- (425.4899,264.4254) -- (442.3420,264.4254) -- cycle;
\end{tikzpicture}
}

\providecommand{\fallogo}[1]{
\begin{tikzpicture}[scale=(#1)/512em]
\fill[gray] (185.7774,0.0000) .. controls (200.4486,15.9798) and (226.8966,8.7148) .. (235.0426,31.5836) .. controls (249.5297,58.0598) and (247.9581,97.9161) .. (280.3335,110.9762) .. controls (309.1690,120.3496) and (337.8406,104.2727) .. (366.5753,103.9379) .. controls (373.4449,111.5171) and (379.2885,128.2574) .. (383.9755,108.9744) .. controls (396.6979,102.5615) and (437.2808,107.6681) .. (426.9652,124.3252) .. controls (408.9822,121.0785) and (412.4742,146.0729) .. (426.5192,131.4996) .. controls (433.8413,120.8489) and (465.1541,126.5522) .. (441.9067,135.7950) .. controls (396.1879,157.7478) and (344.1112,161.5079) .. (298.5528,183.5702) .. controls (277.7471,193.5198) and (284.6941,218.7163) .. (285.2127,236.9640) .. controls (292.3599,316.2826) and (307.3929,394.6311) .. (317.1198,473.6154) .. controls (329.0637,505.4736) and (292.1195,528.5004) .. (265.9183,511.2761) .. controls (237.9284,499.2462) and (237.3684,465.2681) .. (230.9102,439.9421) .. controls (218.6692,374.3397) and (215.6307,306.9662) .. (198.1732,242.3977) .. controls (183.1379,232.7444) and (164.4245,256.0298) .. (149.0430,261.4799) .. controls (116.9328,279.2585) and (87.1822,308.5851) .. (48.2293,307.8914) .. controls (21.3220,306.9037) and (-15.9107,281.8761) .. (7.2921,252.7908) .. controls (29.7799,220.6177) and (67.5177,204.3028) .. (100.9287,185.9449) .. controls (130.8217,170.8906) and (161.1548,156.5903) .. (191.0278,141.5847) .. controls (196.1738,120.0520) and (186.6049,95.2409) .. (186.8382,72.4353) .. controls (185.5234,48.4204) and (183.1700,23.9341) .. (185.7774,0.0000) -- cycle;
\end{tikzpicture}
}

\providecommand{\oblogo}[1]{
\begin{tikzpicture}[scale=(#1)/512em]
\fill[gray] (160.3865,208.9117) .. controls (154.0879,214.6478) and (149.0735,221.2409) .. (145.4125,228.5384) .. controls (184.8790,248.4273) and (234.7122,269.8787) .. (297.5493,291.8782) .. controls (300.3943,281.4769) and (300.9552,268.7619) .. (300.4023,255.2389) .. controls (248.9909,244.7891) and (200.0310,225.9279) .. (160.3865,208.9117) -- cycle(225.7398,392.6996) .. controls (308.0209,392.1716) and (359.3326,345.9277) .. (368.7203,285.2098) .. controls (376.6742,197.1784) and (311.7194,141.3342) .. (205.4287,142.1456) .. controls (139.9485,141.4804) and (88.7155,166.1957) .. (73.5775,228.0086) .. controls (52.0297,320.3408) and (123.4078,391.0103) .. (225.7398,392.6996) -- cycle(216.0739,176.4733) .. controls (268.9183,179.2424) and (315.8292,206.5488) .. (312.7454,265.1139) .. controls (313.2769,315.6384) and (286.5993,353.4946) .. (216.6040,355.7934) .. controls (162.4657,355.7934) and (126.0914,317.5023) .. (126.0914,260.5103) .. controls (126.1733,214.2900) and (163.3363,176.2849) .. (216.0739,176.4733) -- cycle(76.4897,189.1754) .. controls (13.1586,147.5631) and (0.0000,119.4207) .. (0.0000,119.4207) -- (90.6499,170.1632) .. controls (85.3004,175.8497) and (80.5994,182.1633) .. (76.4897,189.1754) -- cycle(353.9486,119.3004) -- (402.9482,119.3004) .. controls (427.0025,137.0797) and (450.9893,162.7034) .. (474.9529,191.0213) .. controls (509.3540,228.5339) and (531.3391,294.2091) .. (487.8149,312.1206) .. controls (462.8165,324.7652) and (394.3874,316.8943) .. (373.8912,313.6651) .. controls (379.9291,297.7449) and (383.2899,278.4204) .. (381.4989,257.7214) .. controls (420.3069,248.0321) and (421.9610,218.3461) .. (407.7867,192.6417) .. controls (391.1113,162.4018) and (370.1114,132.9097) .. (353.9486,119.3004) -- cycle;
\end{tikzpicture}
}

\providecommand{\markuptable}{
\begin{table}
\sffamily\centering
\begin{tabular}{@{}lcl@{}}
\toprule
\texttt{//italics//} & $\rightarrow$ & \textit{italics} \\
\midrule
\texttt{**bold**} & $\rightarrow$ & \textbf{bold} \\
\midrule
\texttt{\# ordered list} & & 1 ordered list \\
\texttt{\# second item} & $\rightarrow$ & 2 second item \\
\texttt{\#\# sub item} & & \hspace{1em} 1 sub item \\
\midrule
\texttt{* unordered list} & & $\bullet$ unordered list \\
\texttt{* second item} & $\rightarrow$ & $\bullet$ second item \\
\texttt{** sub item} & & \hspace{1em} $\bullet$ sub item \\
\midrule
\texttt{link to [[label]]} & $\rightarrow$ & link to \underline{label} \\
\midrule
\texttt{<{}<label>{}> definition } & $\rightarrow$ & definition \\
\midrule
\texttt{[[url|link name]]} & $\rightarrow$ & \underline{link name} \\
\midrule\addlinespace
\texttt{= large heading} & & {\Large large heading} \smallskip \\
\texttt{== medium heading} & $\rightarrow$ & {\large medium heading} \\
\texttt{=== small heading} & & small heading \\
\midrule
\texttt{no line break} & & no line break for paragraphs \\
\texttt{for paragraphs} & $\rightarrow$ \\
& & use empty line \\
\texttt{use empty line} \\
\midrule
\texttt{force\textbackslash\textbackslash line break} & $\rightarrow$ & force \\
& & line break \\
\midrule
\texttt{horizontal line} & $\rightarrow$ & horizontal line \\
\texttt{----} & & \hrulefill \\
\midrule
\texttt{|=a|=table|=header} & & \underline{a \enspace table \enspace header} \\
\texttt{|a|table|row} & $\rightarrow$ & a \enspace table \enspace row \\
\texttt{|b|table|row} & & b \enspace table \enspace row \\
\midrule
\texttt{\{\{\{} \\
\texttt{unformatted} & $\rightarrow$ & \texttt{unformatted} \\
\texttt{code} & & \texttt{code} \\
\texttt{\}\}\}} \\
\midrule\addlinespace
\texttt{@ new article} & & {\Large 1.\ new article} \smallskip \\
\texttt{@ second article} & $\rightarrow$ & {\Large 2.\ second article} \smallskip \\
\texttt{@@ sub article} & & {\large 2.1.\ sub article} \\
\bottomrule
\end{tabular}
\normalfont\caption{Elements of the generic documentation markup language}
\label{tab:docmarkup}
\end{table}
}

\providecommand{\startchapter}[4]{
\documentclass[11pt,a4paper]{article}
\usepackage{booktabs}
\usepackage[format=hang,labelfont=bf]{caption}
\usepackage{changepage}
\usepackage[T1]{fontenc}
\usepackage[margin=2cm]{geometry}
\usepackage{hyperref}
\usepackage[american]{isodate}
\usepackage{lmodern}
\usepackage{longtable}
\usepackage{mathptmx}
\usepackage{microtype}
\usepackage[toc]{multitoc}
\usepackage{multirow}
\usepackage[all]{nowidow}
\usepackage{pdfcomment}
\usepackage{syntax}
\usepackage{tikz}
\usepackage[all]{xy}
\hypersetup{pdfborder={0 0 0},bookmarksnumbered=true,pdftitle={\ecs{}: #2},pdfauthor={Florian Negele},pdfsubject={\ecs{}},pdfkeywords={#1}}
\setlength{\grammarindent}{8em}\setlength{\grammarparsep}{0.2ex}
\setlength{\columnsep}{2em}
\newcommand{\prefix}{}
\newcounter{instruction}
\bibliographystyle{unsrt}
\renewcommand{\index}[2][]{}
\renewcommand{\arraystretch}{1.05}
\renewcommand{\floatpagefraction}{0.7}
\renewcommand{\syntleft}{\itshape}\renewcommand{\syntright}{}
\title{\vspace{-5ex}\Huge{\ecs{}}\medskip\hrule}
\author{\huge{#2}}
\date{\medskip\version}
\newif\ifbook\bookfalse
\pagestyle{headings}
\frenchspacing
\begin{document}
\maketitle\thispagestyle{empty}\noindent#4\setlength{\columnseprule}{0.4pt}\tableofcontents\setlength{\columnseprule}{0pt}\vfill\pagebreak[3]\null\vfill\bigskip\noindent
\parbox{\textwidth-4em}{\license The contents of this \documentation{} are part of the \href{manual}{\ecs{} User Manual}~\cite{manual} and correspond to Chapter ``\href{manual\##3}{#1}''.\alignright\mbox{\today}}
\parbox{4em}{\flushright\ecslogo{3em}}
\clearpage
}

\providecommand{\concludechapter}{
\vfill\pagebreak[3]\null\vfill
\thispagestyle{myheadings}\markright{REFERENCES}
\noindent\begin{minipage}{\textwidth}\begin{multicols}{2}[\section*{References}]
\renewcommand{\section}[2]{}\small\bibliography{references}
\end{multicols}\end{minipage}\end{document}
}

\providecommand{\startpresentation}[2]{
\documentclass[14pt,aspectratio=43,usepdftitle=false]{beamer}
\usepackage{booktabs}
\usepackage{etex}
\usepackage{multicol}
\usepackage{tikz}
\usepackage[all]{xy}
\bibliographystyle{unsrt}
\setlength{\columnsep}{1em}
\setlength{\leftmargini}{1em}
\setbeamercolor{title}{fg=black}
\setbeamercolor{structure}{fg=darkgray}
\setbeamercolor{bibliography item}{fg=darkgray}
\setbeamerfont{title}{series=\bfseries}
\setbeamerfont{subtitle}{series=\normalfont}
\setbeamerfont*{frametitle}{parent=title}
\setbeamerfont{block title}{series=\bfseries}
\setbeamerfont*{framesubtitle}{parent=subtitle}
\setbeamersize{text margin left=1em,text margin right=1em}
\setbeamertemplate{navigation symbols}{}
\setbeamertemplate{itemize item}[circle]{}
\setbeamertemplate{bibliography item}[triangle]{}
\setbeamertemplate{bibliography entry author}{\usebeamercolor[fg]{bibliography item}}
\setbeamertemplate{frametitle}{\medskip\usebeamerfont{frametitle}\color{gray}\raisebox{-2.5ex}[0ex][0ex]{\rule{0.1em}{4.5ex}}}
\addtobeamertemplate{frametitle}{}{\hspace{0.4em}\usebeamercolor[fg]{title}\insertframetitle\par\vspace{0.2ex}\hspace{0.5em}\usebeamerfont{framesubtitle}\insertframesubtitle}
\hypersetup{pdfborder={0 0 0},bookmarksnumbered=true,bookmarksopen=true,bookmarksopenlevel=0,pdftitle={\ecs{}: #1},pdfauthor={Florian Negele},pdfsubject={\ecs{}},pdfkeywords={#1}}
\renewcommand{\flowgraph}[1]{\resizebox{\textwidth}{!}{$$\xymatrix{##1}$$}}
\title{\ecs{}\medskip\hrule\medskip}
\institute{\shadowedecslogo{5em}{30}{15}}
\date{\version}
\subtitle{#1}
\begin{document}
\begin{frame}[plain]\titlepage\nocite{manual}\end{frame}
\begin{frame}{Contents}{#1}\begin{center}\tableofcontents\end{center}\end{frame}
}

\providecommand{\concludepresentation}{
\begin{frame}{References}\begin{footnotesize}\setlength{\columnseprule}{0.4pt}\begin{multicols}{2}\bibliography{references}\end{multicols}\end{footnotesize}\end{frame}
\end{document}
}

\providecommand{\startbook}[1]{
\documentclass[10pt,paper=17cm:24cm,DIV=13,twoside=semi,headings=normal,numbers=noendperiod,cleardoublepage=plain]{scrbook}
\usepackage{atveryend}
\usepackage{booktabs}
\usepackage{caption}
\usepackage{changepage}
\usepackage[T1]{fontenc}
\usepackage{imakeidx}
\usepackage{hyperref}
\usepackage[american]{isodate}
\usepackage{lmodern}
\usepackage{longtable}
\usepackage{mathptmx}
\usepackage[final]{microtype}
\usepackage{multicol}
\usepackage{multirow}
\usepackage[all]{nowidow}
\usepackage{pdfcomment}
\usepackage{scrlayer-scrpage}
\usepackage{setspace}
\usepackage{syntax}
\usepackage[eventxtindent=4pt,oddtxtexdent=4pt]{thumbs}
\usepackage{tikz}
\usepackage[all]{xy}
\hyphenation{Micro-Blaze Open-Cores Open-RISC Power-PC}
\hypersetup{pdfborder={0 0 0},bookmarksnumbered=true,bookmarksopen=true,bookmarksopenlevel=0,pdftitle={\ecs{}: #1},pdfauthor={Florian Negele},pdfsubject={\ecs{}},pdfkeywords={#1}}
\setlength{\grammarindent}{8em}\setlength{\grammarparsep}{0.7ex}
\setkomafont{captionlabel}{\usekomafont{descriptionlabel}}
\renewcommand{\arraystretch}{1.05}\setstretch{1.1}
\renewcommand{\chapterformat}{\thechapter\autodot\enskip\raisebox{-1ex}[0ex][0ex]{\color{gray}\rule{0.1em}{3.5ex}}\enskip}
\renewcommand{\startchapter}[4]{\hypertarget{##3}{\chapter{##1}}\label{##3}##4\addthumb{##1}{\LARGE\sffamily\bfseries\thechapter}{white}{gray}\renewcommand{\prefix}{##3}}
\renewcommand{\concludechapter}{\clearpage{\stopthumb\cleardoublepage}}
\renewcommand{\syntleft}{\itshape}\renewcommand{\syntright}{}
\renewcommand{\floatpagefraction}{0.7}
\renewcommand{\partheademptypage}{}
\DeclareMicrotypeAlias{lmss}{cmr}
\newcommand{\prefix}{}
\newcounter{instruction}
\bibliographystyle{unsrt}
\newif\ifbook\booktrue
\makeindex[intoc,title=Index]
\makeindex[intoc,name=tools,title=Index of Tools,columns=3]
\makeindex[intoc,name=library,title=Index of Library Names]
\makeindex[intoc,name=runtime,title=Index of Runtime Support]
\makeindex[intoc,name=environment,title=Index of Target Environments]
\indexsetup{toclevel=chapter,headers={\indexname}{\indexname}}
\frenchspacing
\begin{document}
\pagenumbering{alph}
\begin{titlepage}\centering
\huge\sffamily\null\vfill\textbf{\ecs{}}\bigskip\hrule\bigskip#1
\normalsize\normalfont\vfill\vfill\shadowedecslogo{10em}{30}{15}
\large\vfill\vfill\version
\end{titlepage}
\null\vfill
\thispagestyle{empty}
\noindent\today\par\medskip
\license A copy of this license is included in Appendix~\ref{fdl} on page~\pageref{fdl}.
All product names used herein are for identification purposes only and may be trademarks of their respective companies.
\concludechapter
\frontmatter
\setcounter{tocdepth}{1}
\tableofcontents
\setcounter{tocdepth}{2}
\concludechapter
\listoffigures
\concludechapter
\listoftables
\concludechapter
}

\providecommand{\concludebook}{
\backmatter
\addtocontents{toc}{\protect\setcounter{tocdepth}{-1}}
\phantomsection\addcontentsline{toc}{part}{Bibliography}
\bibliography{references}
\concludechapter
\phantomsection\addcontentsline{toc}{part}{Indexes}
\printindex
\concludechapter
\indexprologue{\label{idx:tools}}
\printindex[tools]
\concludechapter
\printindex[library]
\concludechapter
\indexprologue{\label{idx:runtime}}
\printindex[runtime]
\concludechapter
\indexprologue{\label{idx:environment}}
\printindex[environment]
\concludechapter
\pagestyle{empty}\pagenumbering{Alph}\null\clearpage
\null\vfill\centering\ecslogo{4em}\par\medskip\license
\end{document}
}

% chapter references

\providecommand{\seedocumentationref}{}\renewcommand{\seedocumentationref}[3]{#1, see \Documentation{}~\documentationref{#2}{#3}. }
\providecommand{\seeinterface}{}\renewcommand{\seeinterface}{\ifbook See \Documentation{}~\documentationref{interface}{User Interface} for more information about the common user interface of all of these tools. \fi}
\providecommand{\seeguide}{}\renewcommand{\seeguide}{\seedocumentationref{For basic examples of using some of these tools in practice}{guide}{User Guide}}
\providecommand{\seecpp}{}\renewcommand{\seecpp}{\seedocumentationref{For more information about the \cpp{} programming language and its implementation by the \ecs{}}{cpp}{User Manual for \cpp{}}}
\providecommand{\seefalse}{}\renewcommand{\seefalse}{\seedocumentationref{For more information about the FALSE programming language and its implementation by the \ecs{}}{false}{User Manual for FALSE}}
\providecommand{\seeoberon}{}\renewcommand{\seeoberon}{\seedocumentationref{For more information about the Oberon programming language and its implementation by the \ecs{}}{oberon}{User Manual for Oberon}}
\providecommand{\seeassembly}{}\renewcommand{\seeassembly}{\seedocumentationref{For more information about the generic assembly language and how to use it}{assembly}{Generic Assembly Language Specification}}
\providecommand{\seeamd}{}\renewcommand{\seeamd}{\seedocumentationref{For more information about how the \ecs{} supports the AMD64 hardware architecture}{amd64}{AMD64 Hardware Architecture Support}}
\providecommand{\seearm}{}\renewcommand{\seearm}{\seedocumentationref{For more information about how the \ecs{} supports the ARM hardware architecture}{arm}{ARM Hardware Architecture Support}}
\providecommand{\seeavr}{}\renewcommand{\seeavr}{\seedocumentationref{For more information about how the \ecs{} supports the AVR hardware architecture}{avr}{AVR Hardware Architecture Support}}
\providecommand{\seeavrtt}{}\renewcommand{\seeavrtt}{\seedocumentationref{For more information about how the \ecs{} supports the AVR32 hardware architecture}{avr32}{AVR32 Hardware Architecture Support}}
\providecommand{\seemabk}{}\renewcommand{\seemabk}{\seedocumentationref{For more information about how the \ecs{} supports the M68000 hardware architecture}{m68k}{M68000 Hardware Architecture Support}}
\providecommand{\seemibl}{}\renewcommand{\seemibl}{\seedocumentationref{For more information about how the \ecs{} supports the MicroBlaze hardware architecture}{mibl}{MicroBlaze Hardware Architecture Support}}
\providecommand{\seemips}{}\renewcommand{\seemips}{\seedocumentationref{For more information about how the \ecs{} supports the MIPS32 and MIPS64 hardware architectures}{mips}{MIPS Hardware Architecture Support}}
\providecommand{\seemmix}{}\renewcommand{\seemmix}{\seedocumentationref{For more information about how the \ecs{} supports the MMIX hardware architecture}{mmix}{MMIX Hardware Architecture Support}}
\providecommand{\seeorok}{}\renewcommand{\seeorok}{\seedocumentationref{For more information about how the \ecs{} supports the OpenRISC 1000 hardware architecture}{or1k}{OpenRISC 1000 Hardware Architecture Support}}
\providecommand{\seeppc}{}\renewcommand{\seeppc}{\seedocumentationref{For more information about how the \ecs{} supports the PowerPC hardware architecture}{ppc}{PowerPC Hardware Architecture Support}}
\providecommand{\seerisc}{}\renewcommand{\seerisc}{\seedocumentationref{For more information about how the \ecs{} supports the RISC hardware architecture}{risc}{RISC Hardware Architecture Support}}
\providecommand{\seewasm}{}\renewcommand{\seewasm}{\seedocumentationref{For more information about how the \ecs{} supports the WebAssembly architecture}{wasm}{WebAssembly Architecture Support}}
\providecommand{\seedocumentation}{}\renewcommand{\seedocumentation}{\seedocumentationref{For more information about generic documentations and their generation by the \ecs{}}{documentation}{Generic Documentation Generation}}
\providecommand{\seedebugging}{}\renewcommand{\seedebugging}{\seedocumentationref{For more information about debugging information and its representation}{debugging}{Debugging Information Representation}}
\providecommand{\seecode}{}\renewcommand{\seecode}{\seedocumentationref{For more information about intermediate code and its purpose}{code}{Intermediate Code Representation}}
\providecommand{\seeobject}{}\renewcommand{\seeobject}{\seedocumentationref{For more information about object files and their purpose}{object}{Object File Representation}}

% generic documentation tools

\providecommand{\docprint}{
\toolsection{docprint} is a pretty printer for generic documentations.
It reformats generic documentations and writes it to the standard output stream.
\debuggingtool
\flowgraph{\resource{generic\\documentation} \ar[r] & \toolbox{docprint} \ar[r] & \resource{generic\\documentation}}
\seedocumentation
}

\providecommand{\doccheck}{
\toolsection{doccheck} is a syntactic and semantic checker for generic documentations.
It just performs syntactic and semantic checks on generic documentations and writes its diagnostic messages to the standard error stream.
\debuggingtool
\flowgraph{\resource{generic\\documentation} \ar[r] & \toolbox{doccheck} \ar[r] & \resource{diagnostic\\messages}}
\seedocumentation
}

\providecommand{\dochtml}{
\toolsection{dochtml} is an HTML documentation generator for generic documentations.
It processes several generic documentations and assembles all information therein into an HTML document.
\debuggingtool
\flowgraph{\resource{generic\\documentation} \ar[r] & \toolbox{dochtml} \ar[r] & \resource{HTML\\document}}
\seedocumentation
}

\providecommand{\doclatex}{
\toolsection{doclatex} is a Latex documentation generator for generic documentations.
It processes several generic documentations and assembles all information therein into a Latex document.
\debuggingtool
\flowgraph{\resource{generic\\documentation} \ar[r] & \toolbox{doclatex} \ar[r] & \resource{Latex\\document}}
\seedocumentation
}

% intermediate code tools

\providecommand{\cdcheck}{
\toolsection{cdcheck} is a syntactic and semantic checker for intermediate code.
It just performs syntactic and semantic checks on programs written in intermediate code and writes its diagnostic messages to the standard error stream.
\debuggingtool
\flowgraph{\resource{intermediate\\code} \ar[r] & \toolbox{cdcheck} \ar[r] & \resource{diagnostic\\messages}}
\seeassembly\seecode
}

\providecommand{\cdopt}{
\toolsection{cdopt} is an optimizer for intermediate code.
It performs various optimizations on programs written in intermediate code and writes the result to the standard output stream.
\debuggingtool
\flowgraph{\resource{intermediate\\code} \ar[r] & \toolbox{cdopt} \ar[r] & \resource{optimized\\code}}
\seeassembly\seecode
}

\providecommand{\cdrun}{
\toolsection{cdrun} is an interpreter for intermediate code.
It processes and executes programs written in intermediate code.
The following code sections are predefined and have the usual semantics:
\texttt{abort}, \texttt{\_Exit}, \texttt{fflush}, \texttt{floor}, \texttt{fputc}, \texttt{free}, \texttt{getchar}, \texttt{malloc}, and \texttt{putchar}.
Diagnostic messages about invalid operations include the name of the executed code section and the index of the erroneous instruction.
\debuggingtool
\flowgraph{\resource{intermediate\\code} \ar[r] & \toolbox{cdrun} \ar@/u/[r] & \resource{input/\\output} \ar@/d/[l]}
\seeassembly\seecode
}

\providecommand{\cdamda}{
\toolsection{cdamd16} is a compiler for intermediate code targeting the AMD64 hardware architecture.
It generates machine code for AMD64 processors from programs written in intermediate code and stores it in corresponding object files.
The compiler generates machine code for the 16-bit operating mode defined by the AMD64 architecture.
It also creates a debugging information file as well as an assembly file containing a listing of the generated machine code.
\debuggingtool
\flowgraph{\resource{intermediate\\code} \ar[r] & \toolbox{cdamd16} \ar[r] \ar[d] \ar[rd] & \resource{object file} \\ & \resource{assembly\\listing} & \resource{debugging\\information}}
\seeassembly\seeamd\seeobject\seecode\seedebugging
}

\providecommand{\cdamdb}{
\toolsection{cdamd32} is a compiler for intermediate code targeting the AMD64 hardware architecture.
It generates machine code for AMD64 processors from programs written in intermediate code and stores it in corresponding object files.
The compiler generates machine code for the 32-bit operating mode defined by the AMD64 architecture.
It also creates a debugging information file as well as an assembly file containing a listing of the generated machine code.
\debuggingtool
\flowgraph{\resource{intermediate\\code} \ar[r] & \toolbox{cdamd32} \ar[r] \ar[d] \ar[rd] & \resource{object file} \\ & \resource{assembly\\listing} & \resource{debugging\\information}}
\seeassembly\seeamd\seeobject\seecode\seedebugging
}

\providecommand{\cdamdc}{
\toolsection{cdamd64} is a compiler for intermediate code targeting the AMD64 hardware architecture.
It generates machine code for AMD64 processors from programs written in intermediate code and stores it in corresponding object files.
The compiler generates machine code for the 64-bit operating mode defined by the AMD64 architecture.
It also creates a debugging information file as well as an assembly file containing a listing of the generated machine code.
\debuggingtool
\flowgraph{\resource{intermediate\\code} \ar[r] & \toolbox{cdamd64} \ar[r] \ar[d] \ar[rd] & \resource{object file} \\ & \resource{assembly\\listing} & \resource{debugging\\information}}
\seeassembly\seeamd\seeobject\seecode\seedebugging
}

\providecommand{\cdarma}{
\toolsection{cdarma32} is a compiler for intermediate code targeting the ARM hardware architecture.
It generates machine code for ARM processors executing A32 instructions from programs written in intermediate code and stores it in corresponding object files.
It also creates a debugging information file as well as an assembly file containing a listing of the generated machine code.
\debuggingtool
\flowgraph{\resource{intermediate\\code} \ar[r] & \toolbox{cdarma32} \ar[r] \ar[d] \ar[rd] & \resource{object file} \\ & \resource{assembly\\listing} & \resource{debugging\\information}}
\seeassembly\seearm\seeobject\seecode\seedebugging
}

\providecommand{\cdarmb}{
\toolsection{cdarma64} is a compiler for intermediate code targeting the ARM hardware architecture.
It generates machine code for ARM processors executing A64 instructions from programs written in intermediate code and stores it in corresponding object files.
It also creates a debugging information file as well as an assembly file containing a listing of the generated machine code.
\debuggingtool
\flowgraph{\resource{intermediate\\code} \ar[r] & \toolbox{cdarma64} \ar[r] \ar[d] \ar[rd] & \resource{object file} \\ & \resource{assembly\\listing} & \resource{debugging\\information}}
\seeassembly\seearm\seeobject\seecode\seedebugging
}

\providecommand{\cdarmc}{
\toolsection{cdarmt32} is a compiler for intermediate code targeting the ARM hardware architecture.
It generates machine code for ARM processors without floating-point extension executing T32 instructions from programs written in intermediate code and stores it in corresponding object files.
It also creates a debugging information file as well as an assembly file containing a listing of the generated machine code.
\debuggingtool
\flowgraph{\resource{intermediate\\code} \ar[r] & \toolbox{cdarmt32} \ar[r] \ar[d] \ar[rd] & \resource{object file} \\ & \resource{assembly\\listing} & \resource{debugging\\information}}
\seeassembly\seearm\seeobject\seecode\seedebugging
}

\providecommand{\cdarmcfpe}{
\toolsection{cdarmt32fpe} is a compiler for intermediate code targeting the ARM hardware architecture.
It generates machine code for ARM processors with floating-point extension executing T32 instructions from programs written in intermediate code and stores it in corresponding object files.
It also creates a debugging information file as well as an assembly file containing a listing of the generated machine code.
\debuggingtool
\flowgraph{\resource{intermediate\\code} \ar[r] & \toolbox{cdarmt32fpe} \ar[r] \ar[d] \ar[rd] & \resource{object file} \\ & \resource{assembly\\listing} & \resource{debugging\\information}}
\seeassembly\seearm\seeobject\seecode\seedebugging
}

\providecommand{\cdavr}{
\toolsection{cdavr} is a compiler for intermediate code targeting the AVR hardware architecture.
It generates machine code for AVR processors from programs written in intermediate code and stores it in corresponding object files.
It also creates a debugging information file as well as an assembly file containing a listing of the generated machine code.
\debuggingtool
\flowgraph{\resource{intermediate\\code} \ar[r] & \toolbox{cdavr} \ar[r] \ar[d] \ar[rd] & \resource{object file} \\ & \resource{assembly\\listing} & \resource{debugging\\information}}
\seeassembly\seeavr\seeobject\seecode\seedebugging
}

\providecommand{\cdavrtt}{
\toolsection{cdavr32} is a compiler for intermediate code targeting the AVR32 hardware architecture.
It generates machine code for AVR32 processors from programs written in intermediate code and stores it in corresponding object files.
It also creates a debugging information file as well as an assembly file containing a listing of the generated machine code.
\debuggingtool
\flowgraph{\resource{intermediate\\code} \ar[r] & \toolbox{cdavr32} \ar[r] \ar[d] \ar[rd] & \resource{object file} \\ & \resource{assembly\\listing} & \resource{debugging\\information}}
\seeassembly\seeavrtt\seeobject\seecode\seedebugging
}

\providecommand{\cdmabk}{
\toolsection{cdm68k} is a compiler for intermediate code targeting the M68000 hardware architecture.
It generates machine code for M68000 processors from programs written in intermediate code and stores it in corresponding object files.
It also creates a debugging information file as well as an assembly file containing a listing of the generated machine code.
\debuggingtool
\flowgraph{\resource{intermediate\\code} \ar[r] & \toolbox{cdm68k} \ar[r] \ar[d] \ar[rd] & \resource{object file} \\ & \resource{assembly\\listing} & \resource{debugging\\information}}
\seeassembly\seemabk\seeobject\seecode\seedebugging
}

\providecommand{\cdmibl}{
\toolsection{cdmibl} is a compiler for intermediate code targeting the MicroBlaze hardware architecture.
It generates machine code for MicroBlaze processors from programs written in intermediate code and stores it in corresponding object files.
It also creates a debugging information file as well as an assembly file containing a listing of the generated machine code.
\debuggingtool
\flowgraph{\resource{intermediate\\code} \ar[r] & \toolbox{cdmibl} \ar[r] \ar[d] \ar[rd] & \resource{object file} \\ & \resource{assembly\\listing} & \resource{debugging\\information}}
\seeassembly\seemibl\seeobject\seecode\seedebugging
}

\providecommand{\cdmipsa}{
\toolsection{cdmips32} is a compiler for intermediate code targeting the MIPS32 hardware architecture.
It generates machine code for MIPS32 processors from programs written in intermediate code and stores it in corresponding object files.
It also creates a debugging information file as well as an assembly file containing a listing of the generated machine code.
\debuggingtool
\flowgraph{\resource{intermediate\\code} \ar[r] & \toolbox{cdmips32} \ar[r] \ar[d] \ar[rd] & \resource{object file} \\ & \resource{assembly\\listing} & \resource{debugging\\information}}
\seeassembly\seemips\seeobject\seecode\seedebugging
}

\providecommand{\cdmipsb}{
\toolsection{cdmips64} is a compiler for intermediate code targeting the MIPS64 hardware architecture.
It generates machine code for MIPS64 processors from programs written in intermediate code and stores it in corresponding object files.
It also creates a debugging information file as well as an assembly file containing a listing of the generated machine code.
\debuggingtool
\flowgraph{\resource{intermediate\\code} \ar[r] & \toolbox{cdmips64} \ar[r] \ar[d] \ar[rd] & \resource{object file} \\ & \resource{assembly\\listing} & \resource{debugging\\information}}
\seeassembly\seemips\seeobject\seecode\seedebugging
}

\providecommand{\cdmmix}{
\toolsection{cdmmix} is a compiler for intermediate code targeting the MMIX hardware architecture.
It generates machine code for MMIX processors from programs written in intermediate code and stores it in corresponding object files.
It also creates a debugging information file as well as an assembly file containing a listing of the generated machine code.
\debuggingtool
\flowgraph{\resource{intermediate\\code} \ar[r] & \toolbox{cdmmix} \ar[r] \ar[d] \ar[rd] & \resource{object file} \\ & \resource{assembly\\listing} & \resource{debugging\\information}}
\seeassembly\seemmix\seeobject\seecode\seedebugging
}

\providecommand{\cdorok}{
\toolsection{cdor1k} is a compiler for intermediate code targeting the OpenRISC 1000 hardware architecture.
It generates machine code for OpenRISC 1000 processors from programs written in intermediate code and stores it in corresponding object files.
It also creates a debugging information file as well as an assembly file containing a listing of the generated machine code.
\debuggingtool
\flowgraph{\resource{intermediate\\code} \ar[r] & \toolbox{cdor1k} \ar[r] \ar[d] \ar[rd] & \resource{object file} \\ & \resource{assembly\\listing} & \resource{debugging\\information}}
\seeassembly\seeorok\seeobject\seecode\seedebugging
}

\providecommand{\cdppca}{
\toolsection{cdppc32} is a compiler for intermediate code targeting the PowerPC hardware architecture.
It generates machine code for PowerPC processors from programs written in intermediate code and stores it in corresponding object files.
The compiler generates machine code for the 32-bit operating mode defined by the PowerPC architecture.
It also creates a debugging information file as well as an assembly file containing a listing of the generated machine code.
\debuggingtool
\flowgraph{\resource{intermediate\\code} \ar[r] & \toolbox{cdppc32} \ar[r] \ar[d] \ar[rd] & \resource{object file} \\ & \resource{assembly\\listing} & \resource{debugging\\information}}
\seeassembly\seeppc\seeobject\seecode\seedebugging
}

\providecommand{\cdppcb}{
\toolsection{cdppc64} is a compiler for intermediate code targeting the PowerPC hardware architecture.
It generates machine code for PowerPC processors from programs written in intermediate code and stores it in corresponding object files.
The compiler generates machine code for the 64-bit operating mode defined by the PowerPC architecture.
It also creates a debugging information file as well as an assembly file containing a listing of the generated machine code.
\debuggingtool
\flowgraph{\resource{intermediate\\code} \ar[r] & \toolbox{cdppc64} \ar[r] \ar[d] \ar[rd] & \resource{object file} \\ & \resource{assembly\\listing} & \resource{debugging\\information}}
\seeassembly\seeppc\seeobject\seecode\seedebugging
}

\providecommand{\cdrisc}{
\toolsection{cdrisc} is a compiler for intermediate code targeting the RISC hardware architecture.
It generates machine code for RISC processors from programs written in intermediate code and stores it in corresponding object files.
It also creates a debugging information file as well as an assembly file containing a listing of the generated machine code.
\debuggingtool
\flowgraph{\resource{intermediate\\code} \ar[r] & \toolbox{cdrisc} \ar[r] \ar[d] \ar[rd] & \resource{object file} \\ & \resource{assembly\\listing} & \resource{debugging\\information}}
\seeassembly\seerisc\seeobject\seecode\seedebugging
}

\providecommand{\cdwasm}{
\toolsection{cdwasm} is a compiler for intermediate code targeting the WebAssembly architecture.
It generates machine code for WebAssembly targets from programs written in intermediate code and stores it in corresponding object files.
It also creates a debugging information file as well as an assembly file containing a listing of the generated machine code.
\debuggingtool
\flowgraph{\resource{intermediate\\code} \ar[r] & \toolbox{cdwasm} \ar[r] \ar[d] \ar[rd] & \resource{object file} \\ & \resource{assembly\\listing} & \resource{debugging\\information}}
\seeassembly\seewasm\seeobject\seecode\seedebugging
}

% C++ tools

\providecommand{\cppprep}{
\toolsection{cppprep} is a preprocessor for the \cpp{} programming language.
It preprocesses source code according to the rules of \cpp{} and writes it to the standard output stream.
Only the macro names \texttt{\_\_DATE\_\_}, \texttt{\_\_FILE\_\_}, \texttt{\_\_LINE\_\_}, and \texttt{\_\_TIME\_\_} are predefined.
\flowgraph{\resource{\cpp{} or other\\source code} \ar[r] & \toolbox{cppprep} \ar[r] & \resource{preprocessed\\source code} \\ & \variable{ECSINCLUDE} \ar[u]}
\seecpp
}

\providecommand{\cppprint}{
\toolsection{cppprint} is a pretty printer for the \cpp{} programming language.
It reformats the source code of \cpp{} programs and writes it to the standard output stream.
\flowgraph{\resource{\cpp{}\\source code} \ar[r] & \toolbox{cppprint} \ar[r] & \resource{reformatted\\source code} \\ & \variable{ECSINCLUDE} \ar[u]}
\seecpp
}

\providecommand{\cppcheck}{
\toolsection{cppcheck} is a syntactic and semantic checker for the \cpp{} programming language.
It just performs syntactic and semantic checks on \cpp{} programs and writes its diagnostic messages to the standard error stream.
\flowgraph{\resource{\cpp{}\\source code} \ar[r] & \toolbox{cppcheck} \ar[r] & \resource{diagnostic\\messages} \\ & \variable{ECSINCLUDE} \ar[u]}
\seecpp
}

\providecommand{\cppdump}{
\toolsection{cppdump} is a serializer for the \cpp{} programming language.
It dumps the complete internal representation of programs written in \cpp{} into an XML document.
\debuggingtool
\flowgraph{\resource{\cpp{}\\source code} \ar[r] & \toolbox{cppdump} \ar[r] & \resource{internal\\representation} \\ & \variable{ECSINCLUDE} \ar[u]}
\seecpp
}

\providecommand{\cpprun}{
\toolsection{cpprun} is an interpreter for the \cpp{} programming language.
It processes and executes programs written in \cpp{}.
The macro \texttt{\_\_run\_\_} is predefined in order to enable programmers to identify this tool while interpreting.
\flowgraph{\resource{\cpp{}\\source code} \ar[r] & \toolbox{cpprun} \ar@/u/[r] & \resource{input/\\output} \ar@/d/[l] \\ & \variable{ECSINCLUDE} \ar[u]}
\seecpp
}

\providecommand{\cppdoc}{
\toolsection{cppdoc} is a generic documentation generator for the \cpp{} programming language.
It processes several \cpp{} source files and assembles all information therein into a generic documentation.
\debuggingtool
\flowgraph{\resource{\cpp{}\\source code} \ar[r] & \toolbox{cppdoc} \ar[r] & \resource{generic\\documentation} \\ & \variable{ECSINCLUDE} \ar[u]}
\seecpp\seedocumentation
}

\providecommand{\cpphtml}{
\toolsection{cpphtml} is an HTML documentation generator for the \cpp{} programming language.
It processes several \cpp{} source files and assembles all information therein into an HTML document.
\flowgraph{\resource{\cpp{}\\source code} \ar[r] & \toolbox{cpphtml} \ar[r] & \resource{HTML\\document} \\ & \variable{ECSINCLUDE} \ar[u]}
\seecpp\seedocumentation
}

\providecommand{\cpplatex}{
\toolsection{cpplatex} is a Latex documentation generator for the \cpp{} programming language.
It processes several \cpp{} source files and assembles all information therein into a Latex document.
\flowgraph{\resource{\cpp{}\\source code} \ar[r] & \toolbox{cpplatex} \ar[r] & \resource{Latex\\document} \\ & \variable{ECSINCLUDE} \ar[u]}
\seecpp\seedocumentation
}

\providecommand{\cppcode}{
\toolsection{cppcode} is an intermediate code generator for the \cpp{} programming language.
It generates intermediate code from programs written in \cpp{} and stores it in corresponding assembly files.
The macro \texttt{\_\_code\_\_} is predefined in order to enable programmers to identify this tool while generating intermediate code.
Programs generated with this tool require additional runtime support that is stored in the \file{cpp\-code\-run} library file.
\debuggingtool
\flowgraph{\resource{\cpp{}\\source code} \ar[r] & \toolbox{cppcode} \ar[r] & \resource{intermediate\\code} \\ & \variable{ECSINCLUDE} \ar[u]}
\seecpp\seeassembly\seecode
}

\providecommand{\cppamda}{
\toolsection{cppamd16} is a compiler for the \cpp{} programming language targeting the AMD64 hardware architecture.
It generates machine code for AMD64 processors from programs written in \cpp{} and stores it in corresponding object files.
The compiler generates machine code for the 16-bit operating mode defined by the AMD64 architecture.
For debugging purposes, it also creates a debugging information file as well as an assembly file containing a listing of the generated machine code.
The macro \texttt{\_\_amd16\_\_} is predefined in order to enable programmers to identify this tool and its target architecture while compiling.
Programs generated with this compiler require additional runtime support that is stored in the \file{cpp\-amd16\-run} library file.
\flowgraph{\resource{\cpp{}\\source code} \ar[r] & \toolbox{cppamd16} \ar[r] \ar[d] \ar[rd] & \resource{object file} \\ \variable{ECSINCLUDE} \ar[ru] & \resource{debugging\\information} & \resource{assembly\\listing}}
\seecpp\seeassembly\seeamd\seeobject\seedebugging
}

\providecommand{\cppamdb}{
\toolsection{cppamd32} is a compiler for the \cpp{} programming language targeting the AMD64 hardware architecture.
It generates machine code for AMD64 processors from programs written in \cpp{} and stores it in corresponding object files.
The compiler generates machine code for the 32-bit operating mode defined by the AMD64 architecture.
For debugging purposes, it also creates a debugging information file as well as an assembly file containing a listing of the generated machine code.
The macro \texttt{\_\_amd32\_\_} is predefined in order to enable programmers to identify this tool and its target architecture while compiling.
Programs generated with this compiler require additional runtime support that is stored in the \file{cpp\-amd32\-run} library file.
\flowgraph{\resource{\cpp{}\\source code} \ar[r] & \toolbox{cppamd32} \ar[r] \ar[d] \ar[rd] & \resource{object file} \\ \variable{ECSINCLUDE} \ar[ru] & \resource{debugging\\information} & \resource{assembly\\listing}}
\seecpp\seeassembly\seeamd\seeobject\seedebugging
}

\providecommand{\cppamdc}{
\toolsection{cppamd64} is a compiler for the \cpp{} programming language targeting the AMD64 hardware architecture.
It generates machine code for AMD64 processors from programs written in \cpp{} and stores it in corresponding object files.
The compiler generates machine code for the 64-bit operating mode defined by the AMD64 architecture.
For debugging purposes, it also creates a debugging information file as well as an assembly file containing a listing of the generated machine code.
The macro \texttt{\_\_amd64\_\_} is predefined in order to enable programmers to identify this tool and its target architecture while compiling.
Programs generated with this compiler require additional runtime support that is stored in the \file{cpp\-amd64\-run} library file.
\flowgraph{\resource{\cpp{}\\source code} \ar[r] & \toolbox{cppamd64} \ar[r] \ar[d] \ar[rd] & \resource{object file} \\ \variable{ECSINCLUDE} \ar[ru] & \resource{debugging\\information} & \resource{assembly\\listing}}
\seecpp\seeassembly\seeamd\seeobject\seedebugging
}

\providecommand{\cpparma}{
\toolsection{cpparma32} is a compiler for the \cpp{} programming language targeting the ARM hardware architecture.
It generates machine code for ARM processors executing A32 instructions from programs written in \cpp{} and stores it in corresponding object files.
For debugging purposes, it also creates a debugging information file as well as an assembly file containing a listing of the generated machine code.
The macro \texttt{\_\_arma32\_\_} is predefined in order to enable programmers to identify this tool and its target architecture while compiling.
Programs generated with this compiler require additional runtime support that is stored in the \file{cpp\-arma32\-run} library file.
\flowgraph{\resource{\cpp{}\\source code} \ar[r] & \toolbox{cpparma32} \ar[r] \ar[d] \ar[rd] & \resource{object file} \\ \variable{ECSINCLUDE} \ar[ru] & \resource{debugging\\information} & \resource{assembly\\listing}}
\seecpp\seeassembly\seearm\seeobject\seedebugging
}

\providecommand{\cpparmb}{
\toolsection{cpparma64} is a compiler for the \cpp{} programming language targeting the ARM hardware architecture.
It generates machine code for ARM processors executing A64 instructions from programs written in \cpp{} and stores it in corresponding object files.
For debugging purposes, it also creates a debugging information file as well as an assembly file containing a listing of the generated machine code.
The macro \texttt{\_\_arma64\_\_} is predefined in order to enable programmers to identify this tool and its target architecture while compiling.
Programs generated with this compiler require additional runtime support that is stored in the \file{cpp\-arma64\-run} library file.
\flowgraph{\resource{\cpp{}\\source code} \ar[r] & \toolbox{cpparma64} \ar[r] \ar[d] \ar[rd] & \resource{object file} \\ \variable{ECSINCLUDE} \ar[ru] & \resource{debugging\\information} & \resource{assembly\\listing}}
\seecpp\seeassembly\seearm\seeobject\seedebugging
}

\providecommand{\cpparmc}{
\toolsection{cpparmt32} is a compiler for the \cpp{} programming language targeting the ARM hardware architecture.
It generates machine code for ARM processors without floating-point extension executing T32 instructions from programs written in \cpp{} and stores it in corresponding object files.
For debugging purposes, it also creates a debugging information file as well as an assembly file containing a listing of the generated machine code.
The macro \texttt{\_\_armt32\_\_} is predefined in order to enable programmers to identify this tool and its target architecture while compiling.
Programs generated with this compiler require additional runtime support that is stored in the \file{cpp\-armt32\-run} library file.
\flowgraph{\resource{\cpp{}\\source code} \ar[r] & \toolbox{cpparmt32} \ar[r] \ar[d] \ar[rd] & \resource{object file} \\ \variable{ECSINCLUDE} \ar[ru] & \resource{debugging\\information} & \resource{assembly\\listing}}
\seecpp\seeassembly\seearm\seeobject\seedebugging
}

\providecommand{\cpparmcfpe}{
\toolsection{cpparmt32fpe} is a compiler for the \cpp{} programming language targeting the ARM hardware architecture.
It generates machine code for ARM processors with floating-point extension executing T32 instructions from programs written in \cpp{} and stores it in corresponding object files.
For debugging purposes, it also creates a debugging information file as well as an assembly file containing a listing of the generated machine code.
The macro \texttt{\_\_armt32fpe\_\_} is predefined in order to enable programmers to identify this tool and its target architecture while compiling.
Programs generated with this compiler require additional runtime support that is stored in the \file{cpp\-armt32\-fpe\-run} library file.
\flowgraph{\resource{\cpp{}\\source code} \ar[r] & \toolbox{cpparmt32fpe} \ar[r] \ar[d] \ar[rd] & \resource{object file} \\ \variable{ECSINCLUDE} \ar[ru] & \resource{debugging\\information} & \resource{assembly\\listing}}
\seecpp\seeassembly\seearm\seeobject\seedebugging
}

\providecommand{\cppavr}{
\toolsection{cppavr} is a compiler for the \cpp{} programming language targeting the AVR hardware architecture.
It generates machine code for AVR processors from programs written in \cpp{} and stores it in corresponding object files.
For debugging purposes, it also creates a debugging information file as well as an assembly file containing a listing of the generated machine code.
The macro \texttt{\_\_avr\_\_} is predefined in order to enable programmers to identify this tool and its target architecture while compiling.
Programs generated with this compiler require additional runtime support that is stored in the \file{cpp\-avr\-run} library file.
\flowgraph{\resource{\cpp{}\\source code} \ar[r] & \toolbox{cppavr} \ar[r] \ar[d] \ar[rd] & \resource{object file} \\ \variable{ECSINCLUDE} \ar[ru] & \resource{debugging\\information} & \resource{assembly\\listing}}
\seecpp\seeassembly\seeavr\seeobject\seedebugging
}

\providecommand{\cppavrtt}{
\toolsection{cppavr32} is a compiler for the \cpp{} programming language targeting the AVR32 hardware architecture.
It generates machine code for AVR32 processors from programs written in \cpp{} and stores it in corresponding object files.
For debugging purposes, it also creates a debugging information file as well as an assembly file containing a listing of the generated machine code.
The macro \texttt{\_\_avr32\_\_} is predefined in order to enable programmers to identify this tool and its target architecture while compiling.
Programs generated with this compiler require additional runtime support that is stored in the \file{cpp\-avr32\-run} library file.
\flowgraph{\resource{\cpp{}\\source code} \ar[r] & \toolbox{cppavr32} \ar[r] \ar[d] \ar[rd] & \resource{object file} \\ \variable{ECSINCLUDE} \ar[ru] & \resource{debugging\\information} & \resource{assembly\\listing}}
\seecpp\seeassembly\seeavrtt\seeobject\seedebugging
}

\providecommand{\cppmabk}{
\toolsection{cppm68k} is a compiler for the \cpp{} programming language targeting the M68000 hardware architecture.
It generates machine code for M68000 processors from programs written in \cpp{} and stores it in corresponding object files.
For debugging purposes, it also creates a debugging information file as well as an assembly file containing a listing of the generated machine code.
The macro \texttt{\_\_m68k\_\_} is predefined in order to enable programmers to identify this tool and its target architecture while compiling.
Programs generated with this compiler require additional runtime support that is stored in the \file{cpp\-m68k\-run} library file.
\flowgraph{\resource{\cpp{}\\source code} \ar[r] & \toolbox{cppm68k} \ar[r] \ar[d] \ar[rd] & \resource{object file} \\ \variable{ECSINCLUDE} \ar[ru] & \resource{debugging\\information} & \resource{assembly\\listing}}
\seecpp\seeassembly\seemabk\seeobject\seedebugging
}

\providecommand{\cppmibl}{
\toolsection{cppmibl} is a compiler for the \cpp{} programming language targeting the MicroBlaze hardware architecture.
It generates machine code for MicroBlaze processors from programs written in \cpp{} and stores it in corresponding object files.
For debugging purposes, it also creates a debugging information file as well as an assembly file containing a listing of the generated machine code.
The macro \texttt{\_\_mibl\_\_} is predefined in order to enable programmers to identify this tool and its target architecture while compiling.
Programs generated with this compiler require additional runtime support that is stored in the \file{cpp\-mibl\-run} library file.
\flowgraph{\resource{\cpp{}\\source code} \ar[r] & \toolbox{cppmibl} \ar[r] \ar[d] \ar[rd] & \resource{object file} \\ \variable{ECSINCLUDE} \ar[ru] & \resource{debugging\\information} & \resource{assembly\\listing}}
\seecpp\seeassembly\seemibl\seeobject\seedebugging
}

\providecommand{\cppmipsa}{
\toolsection{cppmips32} is a compiler for the \cpp{} programming language targeting the MIPS32 hardware architecture.
It generates machine code for MIPS32 processors from programs written in \cpp{} and stores it in corresponding object files.
For debugging purposes, it also creates a debugging information file as well as an assembly file containing a listing of the generated machine code.
The macro \texttt{\_\_mips32\_\_} is predefined in order to enable programmers to identify this tool and its target architecture while compiling.
Programs generated with this compiler require additional runtime support that is stored in the \file{cpp\-mips32\-run} library file.
\flowgraph{\resource{\cpp{}\\source code} \ar[r] & \toolbox{cppmips32} \ar[r] \ar[d] \ar[rd] & \resource{object file} \\ \variable{ECSINCLUDE} \ar[ru] & \resource{debugging\\information} & \resource{assembly\\listing}}
\seecpp\seeassembly\seemips\seeobject\seedebugging
}

\providecommand{\cppmipsb}{
\toolsection{cppmips64} is a compiler for the \cpp{} programming language targeting the MIPS64 hardware architecture.
It generates machine code for MIPS64 processors from programs written in \cpp{} and stores it in corresponding object files.
For debugging purposes, it also creates a debugging information file as well as an assembly file containing a listing of the generated machine code.
The macro \texttt{\_\_mips64\_\_} is predefined in order to enable programmers to identify this tool and its target architecture while compiling.
Programs generated with this compiler require additional runtime support that is stored in the \file{cpp\-mips64\-run} library file.
\flowgraph{\resource{\cpp{}\\source code} \ar[r] & \toolbox{cppmips64} \ar[r] \ar[d] \ar[rd] & \resource{object file} \\ \variable{ECSINCLUDE} \ar[ru] & \resource{debugging\\information} & \resource{assembly\\listing}}
\seecpp\seeassembly\seemips\seeobject\seedebugging
}

\providecommand{\cppmmix}{
\toolsection{cppmmix} is a compiler for the \cpp{} programming language targeting the MMIX hardware architecture.
It generates machine code for MMIX processors from programs written in \cpp{} and stores it in corresponding object files.
For debugging purposes, it also creates a debugging information file as well as an assembly file containing a listing of the generated machine code.
The macro \texttt{\_\_mmix\_\_} is predefined in order to enable programmers to identify this tool and its target architecture while compiling.
Programs generated with this compiler require additional runtime support that is stored in the \file{cpp\-mmix\-run} library file.
\flowgraph{\resource{\cpp{}\\source code} \ar[r] & \toolbox{cppmmix} \ar[r] \ar[d] \ar[rd] & \resource{object file} \\ \variable{ECSINCLUDE} \ar[ru] & \resource{debugging\\information} & \resource{assembly\\listing}}
\seecpp\seeassembly\seemmix\seeobject\seedebugging
}

\providecommand{\cpporok}{
\toolsection{cppor1k} is a compiler for the \cpp{} programming language targeting the OpenRISC 1000 hardware architecture.
It generates machine code for OpenRISC 1000 processors from programs written in \cpp{} and stores it in corresponding object files.
For debugging purposes, it also creates a debugging information file as well as an assembly file containing a listing of the generated machine code.
The macro \texttt{\_\_or1k\_\_} is predefined in order to enable programmers to identify this tool and its target architecture while compiling.
Programs generated with this compiler require additional runtime support that is stored in the \file{cpp\-or1k\-run} library file.
\flowgraph{\resource{\cpp{}\\source code} \ar[r] & \toolbox{cppor1k} \ar[r] \ar[d] \ar[rd] & \resource{object file} \\ \variable{ECSINCLUDE} \ar[ru] & \resource{debugging\\information} & \resource{assembly\\listing}}
\seecpp\seeassembly\seeorok\seeobject\seedebugging
}

\providecommand{\cppppca}{
\toolsection{cppppc32} is a compiler for the \cpp{} programming language targeting the PowerPC hardware architecture.
It generates machine code for PowerPC processors from programs written in \cpp{} and stores it in corresponding object files.
The compiler generates machine code for the 32-bit operating mode defined by the PowerPC architecture.
For debugging purposes, it also creates a debugging information file as well as an assembly file containing a listing of the generated machine code.
The macro \texttt{\_\_ppc32\_\_} is predefined in order to enable programmers to identify this tool and its target architecture while compiling.
Programs generated with this compiler require additional runtime support that is stored in the \file{cpp\-ppc32\-run} library file.
\flowgraph{\resource{\cpp{}\\source code} \ar[r] & \toolbox{cppppc32} \ar[r] \ar[d] \ar[rd] & \resource{object file} \\ \variable{ECSINCLUDE} \ar[ru] & \resource{debugging\\information} & \resource{assembly\\listing}}
\seecpp\seeassembly\seeppc\seeobject\seedebugging
}

\providecommand{\cppppcb}{
\toolsection{cppppc64} is a compiler for the \cpp{} programming language targeting the PowerPC hardware architecture.
It generates machine code for PowerPC processors from programs written in \cpp{} and stores it in corresponding object files.
The compiler generates machine code for the 64-bit operating mode defined by the PowerPC architecture.
For debugging purposes, it also creates a debugging information file as well as an assembly file containing a listing of the generated machine code.
The macro \texttt{\_\_ppc64\_\_} is predefined in order to enable programmers to identify this tool and its target architecture while compiling.
Programs generated with this compiler require additional runtime support that is stored in the \file{cpp\-ppc64\-run} library file.
\flowgraph{\resource{\cpp{}\\source code} \ar[r] & \toolbox{cppppc64} \ar[r] \ar[d] \ar[rd] & \resource{object file} \\ \variable{ECSINCLUDE} \ar[ru] & \resource{debugging\\information} & \resource{assembly\\listing}}
\seecpp\seeassembly\seeppc\seeobject\seedebugging
}

\providecommand{\cpprisc}{
\toolsection{cpprisc} is a compiler for the \cpp{} programming language targeting the RISC hardware architecture.
It generates machine code for RISC processors from programs written in \cpp{} and stores it in corresponding object files.
For debugging purposes, it also creates a debugging information file as well as an assembly file containing a listing of the generated machine code.
The macro \texttt{\_\_risc\_\_} is predefined in order to enable programmers to identify this tool and its target architecture while compiling.
Programs generated with this compiler require additional runtime support that is stored in the \file{cpp\-risc\-run} library file.
\flowgraph{\resource{\cpp{}\\source code} \ar[r] & \toolbox{cpprisc} \ar[r] \ar[d] \ar[rd] & \resource{object file} \\ \variable{ECSINCLUDE} \ar[ru] & \resource{debugging\\information} & \resource{assembly\\listing}}
\seecpp\seeassembly\seerisc\seeobject\seedebugging
}

\providecommand{\cppwasm}{
\toolsection{cppwasm} is a compiler for the \cpp{} programming language targeting the WebAssembly architecture.
It generates machine code for WebAssembly targets from programs written in \cpp{} and stores it in corresponding object files.
For debugging purposes, it also creates a debugging information file as well as an assembly file containing a listing of the generated machine code.
The macro \texttt{\_\_wasm\_\_} is predefined in order to enable programmers to identify this tool and its target architecture while compiling.
Programs generated with this compiler require additional runtime support that is stored in the \file{cpp\-wasm\-run} library file.
\flowgraph{\resource{\cpp{}\\source code} \ar[r] & \toolbox{cppwasm} \ar[r] \ar[d] \ar[rd] & \resource{object file} \\ \variable{ECSINCLUDE} \ar[ru] & \resource{debugging\\information} & \resource{assembly\\listing}}
\seecpp\seeassembly\seewasm\seeobject\seedebugging
}

% FALSE tools

\providecommand{\falprint}{
\toolsection{falprint} is a pretty printer for the FALSE programming language.
It reformats the source code of FALSE programs and writes it to the standard output stream.
\flowgraph{\resource{FALSE\\source code} \ar[r] & \toolbox{falprint} \ar[r] & \resource{reformatted\\source code}}
\seefalse
}

\providecommand{\falcheck}{
\toolsection{falcheck} is a syntactic and semantic checker for the FALSE programming language.
It just performs syntactic and semantic checks on FALSE programs and writes its diagnostic messages to the standard error stream.
\flowgraph{\resource{FALSE\\source code} \ar[r] & \toolbox{falcheck} \ar[r] & \resource{diagnostic\\messages}}
\seefalse
}

\providecommand{\faldump}{
\toolsection{faldump} is a serializer for the FALSE programming language.
It dumps the complete internal representation of programs written in FALSE into an XML document.
\debuggingtool
\flowgraph{\resource{FALSE\\source code} \ar[r] & \toolbox{faldump} \ar[r] & \resource{internal\\representation}}
\seefalse
}

\providecommand{\falrun}{
\toolsection{falrun} is an interpreter for the FALSE programming language.
It processes and executes programs written in FALSE\@.
\flowgraph{\resource{FALSE\\source code} \ar[r] & \toolbox{falrun} \ar@/u/[r] & \resource{input/\\output} \ar@/d/[l]}
\seefalse
}

\providecommand{\falcpp}{
\toolsection{falcpp} is a transpiler for the FALSE programming language.
It translates programs written in FALSE into \cpp{} programs and stores them in corresponding source files.
\flowgraph{\resource{FALSE\\source code} \ar[r] & \toolbox{falcpp} \ar[r] & \resource{\cpp{}\\source file}}
\seefalse\seecpp
}

\providecommand{\falcode}{
\toolsection{falcode} is an intermediate code generator for the FALSE programming language.
It generates intermediate code from programs written in FALSE and stores it in corresponding assembly files.
\debuggingtool
\flowgraph{\resource{FALSE\\source code} \ar[r] & \toolbox{falcode} \ar[r] & \resource{intermediate\\code}}
\seefalse\seeassembly\seecode
}

\providecommand{\falamda}{
\toolsection{falamd16} is a compiler for the FALSE programming language targeting the AMD64 hardware architecture.
It generates machine code for AMD64 processors from programs written in FALSE and stores it in corresponding object files.
The compiler generates machine code for the 16-bit operating mode defined by the AMD64 architecture.
\flowgraph{\resource{FALSE\\source code} \ar[r] & \toolbox{falamd16} \ar[r] & \resource{object file}}
\seefalse\seeamd\seeobject
}

\providecommand{\falamdb}{
\toolsection{falamd32} is a compiler for the FALSE programming language targeting the AMD64 hardware architecture.
It generates machine code for AMD64 processors from programs written in FALSE and stores it in corresponding object files.
The compiler generates machine code for the 32-bit operating mode defined by the AMD64 architecture.
\flowgraph{\resource{FALSE\\source code} \ar[r] & \toolbox{falamd32} \ar[r] & \resource{object file}}
\seefalse\seeamd\seeobject
}

\providecommand{\falamdc}{
\toolsection{falamd64} is a compiler for the FALSE programming language targeting the AMD64 hardware architecture.
It generates machine code for AMD64 processors from programs written in FALSE and stores it in corresponding object files.
The compiler generates machine code for the 64-bit operating mode defined by the AMD64 architecture.
\flowgraph{\resource{FALSE\\source code} \ar[r] & \toolbox{falamd64} \ar[r] & \resource{object file}}
\seefalse\seeamd\seeobject
}

\providecommand{\falarma}{
\toolsection{falarma32} is a compiler for the FALSE programming language targeting the ARM hardware architecture.
It generates machine code for ARM processors executing A32 instructions from programs written in FALSE and stores it in corresponding object files.
\flowgraph{\resource{FALSE\\source code} \ar[r] & \toolbox{falarma32} \ar[r] & \resource{object file}}
\seefalse\seearm\seeobject
}

\providecommand{\falarmb}{
\toolsection{falarma64} is a compiler for the FALSE programming language targeting the ARM hardware architecture.
It generates machine code for ARM processors executing A64 instructions from programs written in FALSE and stores it in corresponding object files.
\flowgraph{\resource{FALSE\\source code} \ar[r] & \toolbox{falarma64} \ar[r] & \resource{object file}}
\seefalse\seearm\seeobject
}

\providecommand{\falarmc}{
\toolsection{falarmt32} is a compiler for the FALSE programming language targeting the ARM hardware architecture.
It generates machine code for ARM processors without floating-point extension executing T32 instructions from programs written in FALSE and stores it in corresponding object files.
\flowgraph{\resource{FALSE\\source code} \ar[r] & \toolbox{falarmt32} \ar[r] & \resource{object file}}
\seefalse\seearm\seeobject
}

\providecommand{\falarmcfpe}{
\toolsection{falarmt32fpe} is a compiler for the FALSE programming language targeting the ARM hardware architecture.
It generates machine code for ARM processors with floating-point extension executing T32 instructions from programs written in FALSE and stores it in corresponding object files.
\flowgraph{\resource{FALSE\\source code} \ar[r] & \toolbox{falarmt32fpe} \ar[r] & \resource{object file}}
\seefalse\seearm\seeobject
}

\providecommand{\falavr}{
\toolsection{falavr} is a compiler for the FALSE programming language targeting the AVR hardware architecture.
It generates machine code for AVR processors from programs written in FALSE and stores it in corresponding object files.
\flowgraph{\resource{FALSE\\source code} \ar[r] & \toolbox{falavr} \ar[r] & \resource{object file}}
\seefalse\seeavr\seeobject
}

\providecommand{\falavrtt}{
\toolsection{falavr32} is a compiler for the FALSE programming language targeting the AVR32 hardware architecture.
It generates machine code for AVR32 processors from programs written in FALSE and stores it in corresponding object files.
\flowgraph{\resource{FALSE\\source code} \ar[r] & \toolbox{falavr32} \ar[r] & \resource{object file}}
\seefalse\seeavrtt\seeobject
}

\providecommand{\falmabk}{
\toolsection{falm68k} is a compiler for the FALSE programming language targeting the M68000 hardware architecture.
It generates machine code for M68000 processors from programs written in FALSE and stores it in corresponding object files.
\flowgraph{\resource{FALSE\\source code} \ar[r] & \toolbox{falm68k} \ar[r] & \resource{object file}}
\seefalse\seemabk\seeobject
}

\providecommand{\falmibl}{
\toolsection{falmibl} is a compiler for the FALSE programming language targeting the MicroBlaze hardware architecture.
It generates machine code for MicroBlaze processors from programs written in FALSE and stores it in corresponding object files.
\flowgraph{\resource{FALSE\\source code} \ar[r] & \toolbox{falmibl} \ar[r] & \resource{object file}}
\seefalse\seemibl\seeobject
}

\providecommand{\falmipsa}{
\toolsection{falmips32} is a compiler for the FALSE programming language targeting the MIPS32 hardware architecture.
It generates machine code for MIPS32 processors from programs written in FALSE and stores it in corresponding object files.
\flowgraph{\resource{FALSE\\source code} \ar[r] & \toolbox{falmips32} \ar[r] & \resource{object file}}
\seefalse\seemips\seeobject
}

\providecommand{\falmipsb}{
\toolsection{falmips64} is a compiler for the FALSE programming language targeting the MIPS64 hardware architecture.
It generates machine code for MIPS64 processors from programs written in FALSE and stores it in corresponding object files.
\flowgraph{\resource{FALSE\\source code} \ar[r] & \toolbox{falmips64} \ar[r] & \resource{object file}}
\seefalse\seemips\seeobject
}

\providecommand{\falmmix}{
\toolsection{falmmix} is a compiler for the FALSE programming language targeting the MMIX hardware architecture.
It generates machine code for MMIX processors from programs written in FALSE and stores it in corresponding object files.
\flowgraph{\resource{FALSE\\source code} \ar[r] & \toolbox{falmmix} \ar[r] & \resource{object file}}
\seefalse\seemmix\seeobject
}

\providecommand{\falorok}{
\toolsection{falor1k} is a compiler for the FALSE programming language targeting the OpenRISC 1000 hardware architecture.
It generates machine code for OpenRISC 1000 processors from programs written in FALSE and stores it in corresponding object files.
\flowgraph{\resource{FALSE\\source code} \ar[r] & \toolbox{falor1k} \ar[r] & \resource{object file}}
\seefalse\seeorok\seeobject
}

\providecommand{\falppca}{
\toolsection{falppc32} is a compiler for the FALSE programming language targeting the PowerPC hardware architecture.
It generates machine code for PowerPC processors from programs written in FALSE and stores it in corresponding object files.
The compiler generates machine code for the 32-bit operating mode defined by the PowerPC architecture.
\flowgraph{\resource{FALSE\\source code} \ar[r] & \toolbox{falppc32} \ar[r] & \resource{object file}}
\seefalse\seeppc\seeobject
}

\providecommand{\falppcb}{
\toolsection{falppc64} is a compiler for the FALSE programming language targeting the PowerPC hardware architecture.
It generates machine code for PowerPC processors from programs written in FALSE and stores it in corresponding object files.
The compiler generates machine code for the 64-bit operating mode defined by the PowerPC architecture.
\flowgraph{\resource{FALSE\\source code} \ar[r] & \toolbox{falppc64} \ar[r] & \resource{object file}}
\seefalse\seeppc\seeobject
}

\providecommand{\falrisc}{
\toolsection{falrisc} is a compiler for the FALSE programming language targeting the RISC hardware architecture.
It generates machine code for RISC processors from programs written in FALSE and stores it in corresponding object files.
\flowgraph{\resource{FALSE\\source code} \ar[r] & \toolbox{falrisc} \ar[r] & \resource{object file}}
\seefalse\seerisc\seeobject
}

\providecommand{\falwasm}{
\toolsection{falwasm} is a compiler for the FALSE programming language targeting the WebAssembly architecture.
It generates machine code for WebAssembly targets from programs written in FALSE and stores it in corresponding object files.
\flowgraph{\resource{FALSE\\source code} \ar[r] & \toolbox{falwasm} \ar[r] & \resource{object file}}
\seefalse\seewasm\seeobject
}

% Oberon tools

\providecommand{\obprint}{
\toolsection{obprint} is a pretty printer for the Oberon programming language.
It reformats the source code of Oberon modules and writes it to the standard output stream.
\flowgraph{\resource{Oberon\\source code} \ar[r] & \toolbox{obprint} \ar[r] & \resource{reformatted\\source code}}
\seeoberon
}

\providecommand{\obcheck}{
\toolsection{obcheck} is a syntactic and semantic checker for the Oberon programming language.
It just performs syntactic and semantic checks on Oberon modules and writes its diagnostic messages to the standard error stream.
In addition, it stores the interface of each module in a symbol file which is required when other modules import the module.
\flowgraph{\resource{Oberon\\source code} \ar[r] & \toolbox{obcheck} \ar[r] \ar@/l/[d] & \resource{diagnostic\\messages} \\ \variable{ECSIMPORT} \ar[ru] & \resource{symbol\\files} \ar@/r/[u]}
\seeoberon
}

\providecommand{\obdump}{
\toolsection{obdump} is a serializer for the Oberon programming language.
It dumps the complete internal representation of modules written in Oberon into an XML document.
\debuggingtool
\flowgraph{\resource{Oberon\\source code} \ar[r] & \toolbox{obdump} \ar[r] \ar@/l/[d] & \resource{internal\\representation} \\ \variable{ECSIMPORT} \ar[ru] & \resource{symbol\\files} \ar@/r/[u]}
\seeoberon
}

\providecommand{\obrun}{
\toolsection{obrun} is an interpreter for the Oberon programming language.
It processes and executes modules written in Oberon.
This tool does neither generate nor process symbol files while interpreting modules.
If a module is imported by another one, its filename has to be named before the other one in the list of command-line arguments.
\flowgraph{\resource{Oberon\\source code} \ar[r] & \toolbox{obrun} \ar@/u/[r] & \resource{input/\\output} \ar@/d/[l]}
\seeoberon
}

\providecommand{\obcpp}{
\toolsection{obcpp} is a transpiler for the Oberon programming language.
It translates programs written in Oberon into \cpp{} programs and stores them in corresponding source and header files.
In addition, it stores the interface of each module in a symbol file which is required when other modules import the module.
The same interface is provided by the generated header file which can be used in other parts of the \cpp{} program.
\flowgraph{\resource{Oberon\\source code} \ar[r] & \toolbox{obcpp} \ar[r] \ar@/l/[d] \ar[rd] & \resource{\cpp{}\\source file} \\ \variable{ECSIMPORT} \ar[ru] & \resource{symbol\\files} \ar@/r/[u] & \resource{\cpp{}\\header file}}
\seeoberon\seecpp
}

\providecommand{\obdoc}{
\toolsection{obdoc} is a generic documentation generator for the Oberon programming language.
It processes several Oberon modules and assembles all information therein into a generic documentation.
In addition, it stores the interface of each module in a symbol file which is required when other modules import the module.
\debuggingtool
\flowgraph{\resource{Oberon\\source code} \ar[r] & \toolbox{obdoc} \ar[r] \ar@/l/[d] & \resource{generic\\documentation} \\ \variable{ECSIMPORT} \ar[ru] & \resource{symbol\\files} \ar@/r/[u]}
\seeoberon\seedocumentation
}

\providecommand{\obhtml}{
\toolsection{obhtml} is an HTML documentation generator for the Oberon programming language.
It processes several Oberon modules and assembles all information therein into an HTML document.
In addition, it stores the interface of each module in a symbol file which is required when other modules import the module.
\flowgraph{\resource{Oberon\\source code} \ar[r] & \toolbox{obhtml} \ar[r] \ar@/l/[d] & \resource{HTML\\document} \\ \variable{ECSIMPORT} \ar[ru] & \resource{symbol\\files} \ar@/r/[u]}
\seeoberon\seedocumentation
}

\providecommand{\oblatex}{
\toolsection{oblatex} is a Latex documentation generator for the Oberon programming language.
It processes several Oberon modules and assembles all information therein into a Latex document.
In addition, it stores the interface of each module in a symbol file which is required when other modules import the module.
\flowgraph{\resource{Oberon\\source code} \ar[r] & \toolbox{oblatex} \ar[r] \ar@/l/[d] & \resource{Latex\\document} \\ \variable{ECSIMPORT} \ar[ru] & \resource{symbol\\files} \ar@/r/[u]}
\seeoberon\seedocumentation
}

\providecommand{\obcode}{
\toolsection{obcode} is an intermediate code generator for the Oberon programming language.
It generates intermediate code from modules written in Oberon and stores it in corresponding assembly files.
In addition, it stores the interface of each module in a symbol file which is required when other modules import the module.
Programs generated with this tool require additional runtime support that is stored in the \file{ob\-code\-run} library file.
\debuggingtool
\flowgraph{\resource{Oberon\\source code} \ar[r] & \toolbox{obcode} \ar[r] \ar@/l/[d] & \resource{intermediate\\code} \\ \variable{ECSIMPORT} \ar[ru] & \resource{symbol\\files} \ar@/r/[u]}
\seeoberon\seeassembly\seecode
}

\providecommand{\obamda}{
\toolsection{obamd16} is a compiler for the Oberon programming language targeting the AMD64 hardware architecture.
It generates machine code for AMD64 processors from modules written in Oberon and stores it in corresponding object files.
The compiler generates machine code for the 16-bit operating mode defined by the AMD64 architecture.
For debugging purposes, it also creates a debugging information file as well as an assembly file containing a listing of the generated machine code.
In addition, it stores the interface of each module in a symbol file which is required when other modules import the module.
Programs generated with this compiler require additional runtime support that is stored in the \file{ob\-amd16\-run} library file.
\flowgraph{\resource{Oberon\\source code} \ar[r] & \toolbox{obamd16} \ar[r] \ar@/l/[d] \ar[rd] & \resource{object file} \\ \variable{ECSIMPORT} \ar[ru] & \resource{symbol\\files} \ar@/r/[u] & \resource{debugging\\information}}
\seeoberon\seeassembly\seeamd\seeobject\seedebugging
}

\providecommand{\obamdb}{
\toolsection{obamd32} is a compiler for the Oberon programming language targeting the AMD64 hardware architecture.
It generates machine code for AMD64 processors from modules written in Oberon and stores it in corresponding object files.
The compiler generates machine code for the 32-bit operating mode defined by the AMD64 architecture.
For debugging purposes, it also creates a debugging information file as well as an assembly file containing a listing of the generated machine code.
In addition, it stores the interface of each module in a symbol file which is required when other modules import the module.
Programs generated with this compiler require additional runtime support that is stored in the \file{ob\-amd32\-run} library file.
\flowgraph{\resource{Oberon\\source code} \ar[r] & \toolbox{obamd32} \ar[r] \ar@/l/[d] \ar[rd] & \resource{object file} \\ \variable{ECSIMPORT} \ar[ru] & \resource{symbol\\files} \ar@/r/[u] & \resource{debugging\\information}}
\seeoberon\seeassembly\seeamd\seeobject\seedebugging
}

\providecommand{\obamdc}{
\toolsection{obamd64} is a compiler for the Oberon programming language targeting the AMD64 hardware architecture.
It generates machine code for AMD64 processors from modules written in Oberon and stores it in corresponding object files.
The compiler generates machine code for the 64-bit operating mode defined by the AMD64 architecture.
For debugging purposes, it also creates a debugging information file as well as an assembly file containing a listing of the generated machine code.
In addition, it stores the interface of each module in a symbol file which is required when other modules import the module.
Programs generated with this compiler require additional runtime support that is stored in the \file{ob\-amd64\-run} library file.
\flowgraph{\resource{Oberon\\source code} \ar[r] & \toolbox{obamd64} \ar[r] \ar@/l/[d] \ar[rd] & \resource{object file} \\ \variable{ECSIMPORT} \ar[ru] & \resource{symbol\\files} \ar@/r/[u] & \resource{debugging\\information}}
\seeoberon\seeassembly\seeamd\seeobject\seedebugging
}

\providecommand{\obarma}{
\toolsection{obarma32} is a compiler for the Oberon programming language targeting the ARM hardware architecture.
It generates machine code for ARM processors executing A32 instructions from modules written in Oberon and stores it in corresponding object files.
For debugging purposes, it also creates a debugging information file as well as an assembly file containing a listing of the generated machine code.
In addition, it stores the interface of each module in a symbol file which is required when other modules import the module.
Programs generated with this compiler require additional runtime support that is stored in the \file{ob\-arma32\-run} library file.
\flowgraph{\resource{Oberon\\source code} \ar[r] & \toolbox{obarma32} \ar[r] \ar@/l/[d] \ar[rd] & \resource{object file} \\ \variable{ECSIMPORT} \ar[ru] & \resource{symbol\\files} \ar@/r/[u] & \resource{debugging\\information}}
\seeoberon\seeassembly\seearm\seeobject\seedebugging
}

\providecommand{\obarmb}{
\toolsection{obarma64} is a compiler for the Oberon programming language targeting the ARM hardware architecture.
It generates machine code for ARM processors executing A64 instructions from modules written in Oberon and stores it in corresponding object files.
For debugging purposes, it also creates a debugging information file as well as an assembly file containing a listing of the generated machine code.
In addition, it stores the interface of each module in a symbol file which is required when other modules import the module.
Programs generated with this compiler require additional runtime support that is stored in the \file{ob\-arma64\-run} library file.
\flowgraph{\resource{Oberon\\source code} \ar[r] & \toolbox{obarma64} \ar[r] \ar@/l/[d] \ar[rd] & \resource{object file} \\ \variable{ECSIMPORT} \ar[ru] & \resource{symbol\\files} \ar@/r/[u] & \resource{debugging\\information}}
\seeoberon\seeassembly\seearm\seeobject\seedebugging
}

\providecommand{\obarmc}{
\toolsection{obarmt32} is a compiler for the Oberon programming language targeting the ARM hardware architecture.
It generates machine code for ARM processors without floating-point extension executing T32 instructions from modules written in Oberon and stores it in corresponding object files.
For debugging purposes, it also creates a debugging information file as well as an assembly file containing a listing of the generated machine code.
In addition, it stores the interface of each module in a symbol file which is required when other modules import the module.
Programs generated with this compiler require additional runtime support that is stored in the \file{ob\-armt32\-run} library file.
\flowgraph{\resource{Oberon\\source code} \ar[r] & \toolbox{obarmt32} \ar[r] \ar@/l/[d] \ar[rd] & \resource{object file} \\ \variable{ECSIMPORT} \ar[ru] & \resource{symbol\\files} \ar@/r/[u] & \resource{debugging\\information}}
\seeoberon\seeassembly\seearm\seeobject\seedebugging
}

\providecommand{\obarmcfpe}{
\toolsection{obarmt32fpe} is a compiler for the Oberon programming language targeting the ARM hardware architecture.
It generates machine code for ARM processors with floating-point extension executing T32 instructions from modules written in Oberon and stores it in corresponding object files.
For debugging purposes, it also creates a debugging information file as well as an assembly file containing a listing of the generated machine code.
In addition, it stores the interface of each module in a symbol file which is required when other modules import the module.
Programs generated with this compiler require additional runtime support that is stored in the \file{ob\-armt32\-fpe\-run} library file.
\flowgraph{\resource{Oberon\\source code} \ar[r] & \toolbox{obarmt32fpe} \ar[r] \ar@/l/[d] \ar[rd] & \resource{object file} \\ \variable{ECSIMPORT} \ar[ru] & \resource{symbol\\files} \ar@/r/[u] & \resource{debugging\\information}}
\seeoberon\seeassembly\seearm\seeobject\seedebugging
}

\providecommand{\obavr}{
\toolsection{obavr} is a compiler for the Oberon programming language targeting the AVR hardware architecture.
It generates machine code for AVR processors from modules written in Oberon and stores it in corresponding object files.
For debugging purposes, it also creates a debugging information file as well as an assembly file containing a listing of the generated machine code.
In addition, it stores the interface of each module in a symbol file which is required when other modules import the module.
Programs generated with this compiler require additional runtime support that is stored in the \file{ob\-avr\-run} library file.
\flowgraph{\resource{Oberon\\source code} \ar[r] & \toolbox{obavr} \ar[r] \ar@/l/[d] \ar[rd] & \resource{object file} \\ \variable{ECSIMPORT} \ar[ru] & \resource{symbol\\files} \ar@/r/[u] & \resource{debugging\\information}}
\seeoberon\seeassembly\seeavr\seeobject\seedebugging
}

\providecommand{\obavrtt}{
\toolsection{obavr32} is a compiler for the Oberon programming language targeting the AVR32 hardware architecture.
It generates machine code for AVR32 processors from modules written in Oberon and stores it in corresponding object files.
For debugging purposes, it also creates a debugging information file as well as an assembly file containing a listing of the generated machine code.
In addition, it stores the interface of each module in a symbol file which is required when other modules import the module.
Programs generated with this compiler require additional runtime support that is stored in the \file{ob\-avr32\-run} library file.
\flowgraph{\resource{Oberon\\source code} \ar[r] & \toolbox{obavr32} \ar[r] \ar@/l/[d] \ar[rd] & \resource{object file} \\ \variable{ECSIMPORT} \ar[ru] & \resource{symbol\\files} \ar@/r/[u] & \resource{debugging\\information}}
\seeoberon\seeassembly\seeavrtt\seeobject\seedebugging
}

\providecommand{\obmabk}{
\toolsection{obm68k} is a compiler for the Oberon programming language targeting the M68000 hardware architecture.
It generates machine code for M68000 processors from modules written in Oberon and stores it in corresponding object files.
For debugging purposes, it also creates a debugging information file as well as an assembly file containing a listing of the generated machine code.
In addition, it stores the interface of each module in a symbol file which is required when other modules import the module.
Programs generated with this compiler require additional runtime support that is stored in the \file{ob\-m68k\-run} library file.
\flowgraph{\resource{Oberon\\source code} \ar[r] & \toolbox{obm68k} \ar[r] \ar@/l/[d] \ar[rd] & \resource{object file} \\ \variable{ECSIMPORT} \ar[ru] & \resource{symbol\\files} \ar@/r/[u] & \resource{debugging\\information}}
\seeoberon\seeassembly\seemabk\seeobject\seedebugging
}

\providecommand{\obmibl}{
\toolsection{obmibl} is a compiler for the Oberon programming language targeting the MicroBlaze hardware architecture.
It generates machine code for MicroBlaze processors from modules written in Oberon and stores it in corresponding object files.
For debugging purposes, it also creates a debugging information file as well as an assembly file containing a listing of the generated machine code.
In addition, it stores the interface of each module in a symbol file which is required when other modules import the module.
Programs generated with this compiler require additional runtime support that is stored in the \file{ob\-mibl\-run} library file.
\flowgraph{\resource{Oberon\\source code} \ar[r] & \toolbox{obmibl} \ar[r] \ar@/l/[d] \ar[rd] & \resource{object file} \\ \variable{ECSIMPORT} \ar[ru] & \resource{symbol\\files} \ar@/r/[u] & \resource{debugging\\information}}
\seeoberon\seeassembly\seemibl\seeobject\seedebugging
}

\providecommand{\obmipsa}{
\toolsection{obmips32} is a compiler for the Oberon programming language targeting the MIPS32 hardware architecture.
It generates machine code for MIPS32 processors from modules written in Oberon and stores it in corresponding object files.
For debugging purposes, it also creates a debugging information file as well as an assembly file containing a listing of the generated machine code.
In addition, it stores the interface of each module in a symbol file which is required when other modules import the module.
Programs generated with this compiler require additional runtime support that is stored in the \file{ob\-mips32\-run} library file.
\flowgraph{\resource{Oberon\\source code} \ar[r] & \toolbox{obmips32} \ar[r] \ar@/l/[d] \ar[rd] & \resource{object file} \\ \variable{ECSIMPORT} \ar[ru] & \resource{symbol\\files} \ar@/r/[u] & \resource{debugging\\information}}
\seeoberon\seeassembly\seemips\seeobject\seedebugging
}

\providecommand{\obmipsb}{
\toolsection{obmips64} is a compiler for the Oberon programming language targeting the MIPS64 hardware architecture.
It generates machine code for MIPS64 processors from modules written in Oberon and stores it in corresponding object files.
For debugging purposes, it also creates a debugging information file as well as an assembly file containing a listing of the generated machine code.
In addition, it stores the interface of each module in a symbol file which is required when other modules import the module.
Programs generated with this compiler require additional runtime support that is stored in the \file{ob\-mips64\-run} library file.
\flowgraph{\resource{Oberon\\source code} \ar[r] & \toolbox{obmips64} \ar[r] \ar@/l/[d] \ar[rd] & \resource{object file} \\ \variable{ECSIMPORT} \ar[ru] & \resource{symbol\\files} \ar@/r/[u] & \resource{debugging\\information}}
\seeoberon\seeassembly\seemips\seeobject\seedebugging
}

\providecommand{\obmmix}{
\toolsection{obmmix} is a compiler for the Oberon programming language targeting the MMIX hardware architecture.
It generates machine code for MMIX processors from modules written in Oberon and stores it in corresponding object files.
For debugging purposes, it also creates a debugging information file as well as an assembly file containing a listing of the generated machine code.
In addition, it stores the interface of each module in a symbol file which is required when other modules import the module.
Programs generated with this compiler require additional runtime support that is stored in the \file{ob\-mmix\-run} library file.
\flowgraph{\resource{Oberon\\source code} \ar[r] & \toolbox{obmmix} \ar[r] \ar@/l/[d] \ar[rd] & \resource{object file} \\ \variable{ECSIMPORT} \ar[ru] & \resource{symbol\\files} \ar@/r/[u] & \resource{debugging\\information}}
\seeoberon\seeassembly\seemmix\seeobject\seedebugging
}

\providecommand{\oborok}{
\toolsection{obor1k} is a compiler for the Oberon programming language targeting the OpenRISC 1000 hardware architecture.
It generates machine code for OpenRISC 1000 processors from modules written in Oberon and stores it in corresponding object files.
For debugging purposes, it also creates a debugging information file as well as an assembly file containing a listing of the generated machine code.
In addition, it stores the interface of each module in a symbol file which is required when other modules import the module.
Programs generated with this compiler require additional runtime support that is stored in the \file{ob\-or1k\-run} library file.
\flowgraph{\resource{Oberon\\source code} \ar[r] & \toolbox{obor1k} \ar[r] \ar@/l/[d] \ar[rd] & \resource{object file} \\ \variable{ECSIMPORT} \ar[ru] & \resource{symbol\\files} \ar@/r/[u] & \resource{debugging\\information}}
\seeoberon\seeassembly\seeorok\seeobject\seedebugging
}

\providecommand{\obppca}{
\toolsection{obppc32} is a compiler for the Oberon programming language targeting the PowerPC hardware architecture.
It generates machine code for PowerPC processors from modules written in Oberon and stores it in corresponding object files.
The compiler generates machine code for the 32-bit operating mode defined by the PowerPC architecture.
For debugging purposes, it also creates a debugging information file as well as an assembly file containing a listing of the generated machine code.
In addition, it stores the interface of each module in a symbol file which is required when other modules import the module.
Programs generated with this compiler require additional runtime support that is stored in the \file{ob\-ppc32\-run} library file.
\flowgraph{\resource{Oberon\\source code} \ar[r] & \toolbox{obppc32} \ar[r] \ar@/l/[d] \ar[rd] & \resource{object file} \\ \variable{ECSIMPORT} \ar[ru] & \resource{symbol\\files} \ar@/r/[u] & \resource{debugging\\information}}
\seeoberon\seeassembly\seeppc\seeobject\seedebugging
}

\providecommand{\obppcb}{
\toolsection{obppc64} is a compiler for the Oberon programming language targeting the PowerPC hardware architecture.
It generates machine code for PowerPC processors from modules written in Oberon and stores it in corresponding object files.
The compiler generates machine code for the 64-bit operating mode defined by the PowerPC architecture.
For debugging purposes, it also creates a debugging information file as well as an assembly file containing a listing of the generated machine code.
In addition, it stores the interface of each module in a symbol file which is required when other modules import the module.
Programs generated with this compiler require additional runtime support that is stored in the \file{ob\-ppc64\-run} library file.
\flowgraph{\resource{Oberon\\source code} \ar[r] & \toolbox{obppc64} \ar[r] \ar@/l/[d] \ar[rd] & \resource{object file} \\ \variable{ECSIMPORT} \ar[ru] & \resource{symbol\\files} \ar@/r/[u] & \resource{debugging\\information}}
\seeoberon\seeassembly\seeppc\seeobject\seedebugging
}

\providecommand{\obrisc}{
\toolsection{obrisc} is a compiler for the Oberon programming language targeting the RISC hardware architecture.
It generates machine code for RISC processors from modules written in Oberon and stores it in corresponding object files.
For debugging purposes, it also creates a debugging information file as well as an assembly file containing a listing of the generated machine code.
In addition, it stores the interface of each module in a symbol file which is required when other modules import the module.
Programs generated with this compiler require additional runtime support that is stored in the \file{ob\-risc\-run} library file.
\flowgraph{\resource{Oberon\\source code} \ar[r] & \toolbox{obrisc} \ar[r] \ar@/l/[d] \ar[rd] & \resource{object file} \\ \variable{ECSIMPORT} \ar[ru] & \resource{symbol\\files} \ar@/r/[u] & \resource{debugging\\information}}
\seeoberon\seeassembly\seerisc\seeobject\seedebugging
}

\providecommand{\obwasm}{
\toolsection{obwasm} is a compiler for the Oberon programming language targeting the WebAssembly architecture.
It generates machine code for WebAssembly targets from modules written in Oberon and stores it in corresponding object files.
For debugging purposes, it also creates a debugging information file as well as an assembly file containing a listing of the generated machine code.
In addition, it stores the interface of each module in a symbol file which is required when other modules import the module.
Programs generated with this compiler require additional runtime support that is stored in the \file{ob\-wasm\-run} library file.
\flowgraph{\resource{Oberon\\source code} \ar[r] & \toolbox{obwasm} \ar[r] \ar@/l/[d] \ar[rd] & \resource{object file} \\ \variable{ECSIMPORT} \ar[ru] & \resource{symbol\\files} \ar@/r/[u] & \resource{debugging\\information}}
\seeoberon\seeassembly\seewasm\seeobject\seedebugging
}

% converter tools

\providecommand{\dbgdwarf}{
\toolsection{dbgdwarf} is a DWARF debugging information converter tool.
It converts debugging information into the DWARF debugging data format and stores it in corresponding object files~\cite{dwarffile}.
The resulting debugging object files can be combined with runtime support that creates Executable and Linking Format (ELF) files~\cite{elffile}.
\flowgraph{\resource{debugging\\information} \ar[r] & \toolbox{dbgdwarf} \ar[r] & \resource{debugging\\object file}}
\seeobject\seedebugging
}

% assembler tools

\providecommand{\asmprint}{
\toolsection{asmprint} is a pretty printer for generic assembly code.
It reformats generic assembly code and writes it to the standard output stream.
\flowgraph{\resource{generic assembly\\source code} \ar[r] & \toolbox{asmprint} \ar[r] & \resource{reformatted\\source code}}
\seeassembly
}

\providecommand{\amdaasm}{
\toolsection{amd16asm} is an assembler for the AMD64 hardware architecture.
It translates assembly code into machine code for AMD64 processors and stores it in corresponding object files.
By default, the assembler generates machine code for the 16-bit operating mode defined by the AMD64 architecture.
\flowgraph{\resource{AMD16 assembly\\source code} \ar[r] & \toolbox{amd16asm} \ar[r] & \resource{object file}}
\seeassembly\seeamd\seeobject
}

\providecommand{\amdadism}{
\toolsection{amd16dism} is a disassembler for the AMD64 hardware architecture.
It translates machine code from object files targeting AMD64 processors into assembly code and writes it to the standard output stream.
It assumes that the machine code was generated for the 16-bit operating mode defined by the AMD64 architecture.
\flowgraph{\resource{object file} \ar[r] & \toolbox{amd16dism} \ar[r] & \resource{disassembly\\listing}}
\seeassembly\seeamd\seeobject
}

\providecommand{\amdbasm}{
\toolsection{amd32asm} is an assembler for the AMD64 hardware architecture.
It translates assembly code into machine code for AMD64 processors and stores it in corresponding object files.
By default, the assembler generates machine code for the 32-bit operating mode defined by the AMD64 architecture.
\flowgraph{\resource{AMD32 assembly\\source code} \ar[r] & \toolbox{amd32asm} \ar[r] & \resource{object file}}
\seeassembly\seeamd\seeobject
}

\providecommand{\amdbdism}{
\toolsection{amd32dism} is a disassembler for the AMD64 hardware architecture.
It translates machine code from object files targeting AMD64 processors into assembly code and writes it to the standard output stream.
It assumes that the machine code was generated for the 32-bit operating mode defined by the AMD64 architecture.
\flowgraph{\resource{object file} \ar[r] & \toolbox{amd32dism} \ar[r] & \resource{disassembly\\listing}}
\seeassembly\seeamd\seeobject
}

\providecommand{\amdcasm}{
\toolsection{amd64asm} is an assembler for the AMD64 hardware architecture.
It translates assembly code into machine code for AMD64 processors and stores it in corresponding object files.
By default, the assembler generates machine code for the 64-bit operating mode defined by the AMD64 architecture.
\flowgraph{\resource{AMD64 assembly\\source code} \ar[r] & \toolbox{amd64asm} \ar[r] & \resource{object file}}
\seeassembly\seeamd\seeobject
}

\providecommand{\amdcdism}{
\toolsection{amd64dism} is a disassembler for the AMD64 hardware architecture.
It translates machine code from object files targeting AMD64 processors into assembly code and writes it to the standard output stream.
It assumes that the machine code was generated for the 64-bit operating mode defined by the AMD64 architecture.
\flowgraph{\resource{object file} \ar[r] & \toolbox{amd64dism} \ar[r] & \resource{disassembly\\listing}}
\seeassembly\seeamd\seeobject
}

\providecommand{\armaasm}{
\toolsection{arma32asm} is an assembler for the ARM hardware architecture.
It translates assembly code into machine code for ARM processors executing A32 instructions and stores it in corresponding object files.
\flowgraph{\resource{ARM A32 assembly\\source code} \ar[r] & \toolbox{arma32asm} \ar[r] & \resource{object file}}
\seeassembly\seearm\seeobject
}

\providecommand{\armadism}{
\toolsection{arma32dism} is a disassembler for the ARM hardware architecture.
It translates machine code from object files targeting ARM processors executing A32 instructions into assembly code and writes it to the standard output stream.
\flowgraph{\resource{object file} \ar[r] & \toolbox{arma32dism} \ar[r] & \resource{disassembly\\listing}}
\seeassembly\seearm\seeobject
}

\providecommand{\armbasm}{
\toolsection{arma64asm} is an assembler for the ARM hardware architecture.
It translates assembly code into machine code for ARM processors executing A64 instructions and stores it in corresponding object files.
\flowgraph{\resource{ARM A64 assembly\\source code} \ar[r] & \toolbox{arma64asm} \ar[r] & \resource{object file}}
\seeassembly\seearm\seeobject
}

\providecommand{\armbdism}{
\toolsection{arma64dism} is a disassembler for the ARM hardware architecture.
It translates machine code from object files targeting ARM processors executing A64 instructions into assembly code and writes it to the standard output stream.
\flowgraph{\resource{object file} \ar[r] & \toolbox{arma64dism} \ar[r] & \resource{disassembly\\listing}}
\seeassembly\seearm\seeobject
}

\providecommand{\armcasm}{
\toolsection{armt32asm} is an assembler for the ARM hardware architecture.
It translates assembly code into machine code for ARM processors executing T32 instructions and stores it in corresponding object files.
\flowgraph{\resource{ARM T32 assembly\\source code} \ar[r] & \toolbox{armt32asm} \ar[r] & \resource{object file}}
\seeassembly\seearm\seeobject
}

\providecommand{\armcdism}{
\toolsection{armt32dism} is a disassembler for the ARM hardware architecture.
It translates machine code from object files targeting ARM processors executing T32 instructions into assembly code and writes it to the standard output stream.
\flowgraph{\resource{object file} \ar[r] & \toolbox{armt32dism} \ar[r] & \resource{disassembly\\listing}}
\seeassembly\seearm\seeobject
}

\providecommand{\avrasm}{
\toolsection{avrasm} is an assembler for the AVR hardware architecture.
It translates assembly code into machine code for AVR processors and stores it in corresponding object files.
The identifiers \texttt{RXL}, \texttt{RXH}, \texttt{RYL}, \texttt{RYH}, \texttt{RZL}, and \texttt{RZH} are predefined and name the corresponding registers.
The identifiers \texttt{SPL} and \texttt{SPH} are also predefined and evaluate to the address of the corresponding registers.
\flowgraph{\resource{AVR assembly\\source code} \ar[r] & \toolbox{avrasm} \ar[r] & \resource{object file}}
\seeassembly\seeavr\seeobject
}

\providecommand{\avrdism}{
\toolsection{avrdism} is a disassembler for the AVR hardware architecture.
It translates machine code from object files targeting AVR processors into assembly code and writes it to the standard output stream.
\flowgraph{\resource{object file} \ar[r] & \toolbox{avrdism} \ar[r] & \resource{disassembly\\listing}}
\seeassembly\seeavr\seeobject
}

\providecommand{\avrttasm}{
\toolsection{avr32asm} is an assembler for the AVR32 hardware architecture.
It translates assembly code into machine code for AVR32 processors and stores it in corresponding object files.
\flowgraph{\resource{AVR32 assembly\\source code} \ar[r] & \toolbox{avr32asm} \ar[r] & \resource{object file}}
\seeassembly\seeavrtt\seeobject
}

\providecommand{\avrttdism}{
\toolsection{avr32dism} is a disassembler for the AVR32 hardware architecture.
It translates machine code from object files targeting AVR32 processors into assembly code and writes it to the standard output stream.
\flowgraph{\resource{object file} \ar[r] & \toolbox{avr32dism} \ar[r] & \resource{disassembly\\listing}}
\seeassembly\seeavrtt\seeobject
}

\providecommand{\mabkasm}{
\toolsection{m68kasm} is an assembler for the M68000 hardware architecture.
It translates assembly code into machine code for M68000 processors and stores it in corresponding object files.
\flowgraph{\resource{68000 assembly\\source code} \ar[r] & \toolbox{m68kasm} \ar[r] & \resource{object file}}
\seeassembly\seemabk\seeobject
}

\providecommand{\mabkdism}{
\toolsection{m68kdism} is a disassembler for the M68000 hardware architecture.
It translates machine code from object files targeting M68000 processors into assembly code and writes it to the standard output stream.
\flowgraph{\resource{object file} \ar[r] & \toolbox{m68kdism} \ar[r] & \resource{disassembly\\listing}}
\seeassembly\seemabk\seeobject
}

\providecommand{\miblasm}{
\toolsection{miblasm} is an assembler for the MicroBlaze hardware architecture.
It translates assembly code into machine code for MicroBlaze processors and stores it in corresponding object files.
\flowgraph{\resource{MicroBlaze assembly\\source code} \ar[r] & \toolbox{miblasm} \ar[r] & \resource{object file}}
\seeassembly\seemibl\seeobject
}

\providecommand{\mibldism}{
\toolsection{mibldism} is a disassembler for the MicroBlaze hardware architecture.
It translates machine code from object files targeting MicroBlaze processors into assembly code and writes it to the standard output stream.
\flowgraph{\resource{object file} \ar[r] & \toolbox{mibldism} \ar[r] & \resource{disassembly\\listing}}
\seeassembly\seemibl\seeobject
}

\providecommand{\mipsaasm}{
\toolsection{mips32asm} is an assembler for the MIPS32 hardware architecture.
It translates assembly code into machine code for MIPS32 processors and stores it in corresponding object files.
\flowgraph{\resource{MIPS32 assembly\\source code} \ar[r] & \toolbox{mips32asm} \ar[r] & \resource{object file}}
\seeassembly\seemips\seeobject
}

\providecommand{\mipsadism}{
\toolsection{mips32dism} is a disassembler for the MIPS32 hardware architecture.
It translates machine code from object files targeting MIPS32 processors into assembly code and writes it to the standard output stream.
\flowgraph{\resource{object file} \ar[r] & \toolbox{mips32dism} \ar[r] & \resource{disassembly\\listing}}
\seeassembly\seemips\seeobject
}

\providecommand{\mipsbasm}{
\toolsection{mips64asm} is an assembler for the MIPS64 hardware architecture.
It translates assembly code into machine code for MIPS64 processors and stores it in corresponding object files.
\flowgraph{\resource{MIPS64 assembly\\source code} \ar[r] & \toolbox{mips64asm} \ar[r] & \resource{object file}}
\seeassembly\seemips\seeobject
}

\providecommand{\mipsbdism}{
\toolsection{mips64dism} is a disassembler for the MIPS64 hardware architecture.
It translates machine code from object files targeting MIPS64 processors into assembly code and writes it to the standard output stream.
\flowgraph{\resource{object file} \ar[r] & \toolbox{mips64dism} \ar[r] & \resource{disassembly\\listing}}
\seeassembly\seemips\seeobject
}

\providecommand{\mmixasm}{
\toolsection{mmixasm} is an assembler for the MMIX hardware architecture.
It translates assembly code into machine code for MMIX processors and stores it in corresponding object files.
The names of all special registers are predefined and evaluate to the corresponding number.
\flowgraph{\resource{MMIX assembly\\source code} \ar[r] & \toolbox{mmixasm} \ar[r] & \resource{object file}}
\seeassembly\seemmix\seeobject
}

\providecommand{\mmixdism}{
\toolsection{mmixdism} is a disassembler for the MMIX hardware architecture.
It translates machine code from object files targeting MMIX processors into assembly code and writes it to the standard output stream.
\flowgraph{\resource{object file} \ar[r] & \toolbox{mmixdism} \ar[r] & \resource{disassembly\\listing}}
\seeassembly\seemmix\seeobject
}

\providecommand{\orokasm}{
\toolsection{or1kasm} is an assembler for the OpenRISC 1000 hardware architecture.
It translates assembly code into machine code for OpenRISC 1000 processors and stores it in corresponding object files.
\flowgraph{\resource{OpenRISC 1000 assembly\\source code} \ar[r] & \toolbox{or1kasm} \ar[r] & \resource{object file}}
\seeassembly\seeorok\seeobject
}

\providecommand{\orokdism}{
\toolsection{or1kdism} is a disassembler for the OpenRISC 1000 hardware architecture.
It translates machine code from object files targeting OpenRISC 1000 processors into assembly code and writes it to the standard output stream.
\flowgraph{\resource{object file} \ar[r] & \toolbox{or1kdism} \ar[r] & \resource{disassembly\\listing}}
\seeassembly\seeorok\seeobject
}

\providecommand{\ppcaasm}{
\toolsection{ppc32asm} is an assembler for the PowerPC hardware architecture.
It translates assembly code into machine code for PowerPC processors and stores it in corresponding object files.
By default, the assembler generates machine code for the 32-bit operating mode defined by the PowerPC architecture.
\flowgraph{\resource{PowerPC assembly\\source code} \ar[r] & \toolbox{ppc32asm} \ar[r] & \resource{object file}}
\seeassembly\seeppc\seeobject
}

\providecommand{\ppcadism}{
\toolsection{ppc32dism} is a disassembler for the PowerPC hardware architecture.
It translates machine code from object files targeting PowerPC processors into assembly code and writes it to the standard output stream.
It assumes that the machine code was generated for the 32-bit operating mode defined by the PowerPC architecture.
\flowgraph{\resource{object file} \ar[r] & \toolbox{ppc32dism} \ar[r] & \resource{disassembly\\listing}}
\seeassembly\seeppc\seeobject
}

\providecommand{\ppcbasm}{
\toolsection{ppc64asm} is an assembler for the PowerPC hardware architecture.
It translates assembly code into machine code for PowerPC processors and stores it in corresponding object files.
By default, the assembler generates machine code for the 64-bit operating mode defined by the PowerPC architecture.
\flowgraph{\resource{PowerPC assembly\\source code} \ar[r] & \toolbox{ppc64asm} \ar[r] & \resource{object file}}
\seeassembly\seeppc\seeobject
}

\providecommand{\ppcbdism}{
\toolsection{ppc64dism} is a disassembler for the PowerPC hardware architecture.
It translates machine code from object files targeting PowerPC processors into assembly code and writes it to the standard output stream.
It assumes that the machine code was generated for the 64-bit operating mode defined by the PowerPC architecture.
\flowgraph{\resource{object file} \ar[r] & \toolbox{ppc64dism} \ar[r] & \resource{disassembly\\listing}}
\seeassembly\seeppc\seeobject
}

\providecommand{\riscasm}{
\toolsection{riscasm} is an assembler for the RISC hardware architecture.
It translates assembly code into machine code for RISC processors and stores it in corresponding object files.
The names of all special registers are predefined and evaluate to the corresponding number.
\flowgraph{\resource{RISC assembly\\source code} \ar[r] & \toolbox{riscasm} \ar[r] & \resource{object file}}
\seeassembly\seerisc\seeobject
}

\providecommand{\riscdism}{
\toolsection{riscdism} is a disassembler for the RISC hardware architecture.
It translates machine code from object files targeting RISC processors into assembly code and writes it to the standard output stream.
\flowgraph{\resource{object file} \ar[r] & \toolbox{riscdism} \ar[r] & \resource{disassembly\\listing}}
\seeassembly\seerisc\seeobject
}

\providecommand{\wasmasm}{
\toolsection{wasmasm} is an assembler for the WebAssembly architecture.
It translates assembly code into machine code for WebAssembly targets and stores it in corresponding object files.
The names of all special registers are predefined and evaluate to the corresponding number.
\flowgraph{\resource{WebAssembly assembly\\source code} \ar[r] & \toolbox{wasmasm} \ar[r] & \resource{object file}}
\seeassembly\seewasm\seeobject
}

\providecommand{\wasmdism}{
\toolsection{wasmdism} is a disassembler for the WebAssembly architecture.
It translates machine code from object files targeting WebAssembly targets into assembly code and writes it to the standard output stream.
\flowgraph{\resource{object file} \ar[r] & \toolbox{wasmdism} \ar[r] & \resource{disassembly\\listing}}
\seeassembly\seewasm\seeobject
}

% linker tools

\providecommand{\linklib}{
\toolsection{linklib} is an object file combiner.
It creates a static library file by combining all object files given to it into a single one.
\flowgraph{\resource{object files} \ar[r] & \toolbox{linklib} \ar[r] & \resource{library file}}
\seeobject
}

\providecommand{\linkbin}{
\toolsection{linkbin} is a linker for plain binary files.
It links all object files given to it into a single image and stores it in a binary file that begins with the first linked section.
It also creates a map file that lists the address, type, name and size of all used sections.
The filename extension of the resulting binary file can be specified by putting it into a constant data section called \texttt{\_extension}.
\flowgraph{\resource{object files} \ar[r] & \toolbox{linkbin} \ar[r] \ar[d] & \resource{binary file} \\ & \resource{map file}}
\seeobject
}

\providecommand{\linkmem}{
\toolsection{linkmem} is a linker for plain binary files partitioned into random-access and read-only memory.
It links all object files given to it into two distinct images, one for data sections and one for code and constant data sections, and stores each image in a binary file that begins with the first linked section of the corresponding type.
It also creates a map file that lists the address, type, name and size of all used sections.
\flowgraph{\resource{object files} \ar[r] & \toolbox{linkmem} \ar[r] \ar[d] & \resource{RAM file/\\ROM file} \\ & \resource{map file}}
\seeobject
}

\providecommand{\linkprg}{
\toolsection{linkprg} is a linker for GEMDOS executable files.
It links all object files given to it into a single image and stores the image in an Atari GEMDOS executable file~\cite{gemdosfile}.
It also creates a map file that lists the address relative to the text segment, type, name and size of all used sections.
The filename extension of the resulting executable file can be specified by putting it into a constant data section called \texttt{\_extension}.
The GEMDOS executable file format requires all patch patterns of absolute link patches to consist of four full bitmasks with descending offsets.
\flowgraph{\resource{object files} \ar[r] & \toolbox{linkprg} \ar[r] \ar[d] & \resource{executable file} \\ & \resource{map file}}
\seeobject
}

\providecommand{\linkhex}{
\toolsection{linkhex} is a linker for Intel HEX files.
It links all code sections of the object files given to it into single image and stores the image in an Intel HEX file~\cite{hexfile} that begins with the first linked section.
It also creates a map file that lists the address, type, name and size of all used sections.
\flowgraph{\resource{object files} \ar[r] & \toolbox{linkhex} \ar[r] \ar[d] & \resource{HEX file} \\ & \resource{map file}}
\seeobject
}

\providecommand{\mapsearch}{
\toolsection{mapsearch} is a debugging tool.
It searches map files generated by linker tools for the name of a binary section that encompasses a memory address read from the standard input stream.
If additionally provided with one or more object files, it also stores an excerpt thereof in a separate object file called map search result which only contains the identified binary section for disassembling purposes.
\flowgraph{& \resource{map files/\\object files} \ar[d] \\ \resource{memory\\address} \ar[r] & \toolbox{mapsearch} \ar[r] \ar[d] & \resource{section name/\\relative offset} \\ & \resource{object file\\excerpt}}
\seeobject
}


\startchapter{Presentation Material}{Presentation Material}{material}
{This \documentation{} is a compilation of presentation material that can be used for presenting the design, functionality, and various implementation details of the \ecs{}.
Each presentation consists of a set of slides and a detailed explanation of their contents.}

\ifbook
\newcommand{\thepresentation}{}\setlength{\marginparwidth}{15em}\setlength{\marginparsep}{-\marginparwidth}
\newcommand{\slide}[1]{\marginpar{\framebox{\pgfimage[width=\dimexpr\marginparwidth-2\fboxrule-2\fboxsep,page=#1]{\thepresentation}}}}
\newenvironment{presentation}[2]{\section{#1}\marginpar{}\renewcommand{\thepresentation}{#2}\begin{adjustwidth}{}{\marginparwidth+1em}}{\end{adjustwidth}}
\else
\newcommand{\thepresentation}{}
\newcommand{\slide}[1]{\begin{center}\framebox{\pgfimage[height=25ex,page=#1]{\thepresentation}}\end{center}}
\newenvironment{presentation}[2]{\bigskip\begin{multicols}{2}[\section{#1}]\renewcommand{\thepresentation}{#2}}{\end{multicols}}
\fi

\begin{presentation}{Overview}{overview}

\emph{\ecs{}} is the name of a free software collection of development tools.
This presentation gives a general overview over the \ecs{} by describing its features, design, functionality and status, and comparing it to related free software.

\slide{2}

\subsection{Features}

The \ecs{} is a toolchain for developing software that contains various tools like compilers, interpreters, assemblers, and linkers targeting a variety of programming languages and hardware architectures.
It implements each of its supported programming languages by providing tools like pretty printers, semantic checkers, interpreters, and front-ends for compilers.
Depending on the programming language, other tools like preprocessors, transpilers, and documentation generators are provided as well.
The \ecs{} additionally defines the generic assembly language that provides a common programming framework for all supported assemblers.
\ifbook\else\nocite{cpp}\nocite{false}\nocite{oberon}\nocite{assembly}\fi

\slide{3}

The \ecs{} supports various hardware architectures by providing tools like assemblers, disassemblers, and machine code generators for compilers.
Some of these architectures like MIPS, AMD64, and PowerPC define different operating modes or bit sizes in which case the \ecs{} provides a set of tools for each variant.
Some of the supported architectures like RISC, MMIX, MicroBlaze, and OpenRISC 1000 may only be available on simulators or FPGA implementations.

\slide{4}

Concerning its supported runtime environments, the \ecs{} provides several different linker tools that generate files like executables, bootloaders, or disk images for simulators.

\slide{5}

\subsection{Design}

The \ecs{} was primarily designed to be a simple but complete and self-contained toolchain.
In particular, all tools featured by the \ecs{} share the same easy-to-use command-line user interface and do not require complex installations or other prerequisites.
Another objective is to produce correct results as well as comprehensive diagnostic messages.
Optimizations and performance are explicitly secondary, as the implementation of the \ecs{} is mainly driven by correctness, reliability, and simplicity.
The \ecs{} is completely written using standard and portable programming language features in order to guarantee the portability of its source code.
Therefore, all compilers featured by the \ecs{} are cross compilers by design.
Furthermore, tools like compilers and assemblers shall enable interoperability in-between all programming languages supported by the \ecs{}.
Since all intermediate data is represented using a human-readable and machine-independent text format, programmers and maintainers are able to view, modify and even manually create all kinds of intermediate data.
Finally, the implementation of the \ecs{} shall provide generic abstractions in order to reuse code wherever possible while implementing these objectives.

\slide{6}

In order to ensure that all programming tools of the \ecs{} are reliable and produce correct results, several test and validation suites are provided which enable automatic regression testing.
Test Suites for programming languages are separated into compilation and execution tests in order to check for code that should or should not compile, as well as complete and valid programs that either generate a runtime error or run successfully to completion.
Regarding the generic assembly language, there exists a test suite for each supported hardware architecture that features at least one compilation test for each element of the corresponding instruction set.

\slide{7}

Each test suite tests one particular tool and is represented using a single text file that contains an arbitrary number of tests.
Each test is identified using a unique description for regression testing and is followed by the corresponding input for the tool under test.
A positive test requires the tool to succeed when given the corresponding input while a negative test expects a failure.
This textual representation is very compact and has proven itself to be very versatile because it allows testing any command-line tool.

\slide{8}

One goal of the \ecs{} is to provide compilers, assemblers, and linkers for all possible combinations of programming languages and target architectures.
For this reason, the \ecs{} uses an intermediate code representation that acts as a generic abstraction in-between \emph{front-ends} implementing programming languages and \emph{back-ends} targeting hardware architectures.
A single data flow of the shown diagram depicts a single compiler for a particular programming language targeting a particular hardware architecture.
\seecode

\slide{9}

Object files on the other hand are the key abstraction that enables interoperability between all compilers and assemblers of the \ecs{}.
All these tools generate the same kind of output called object files which can be processed by all linkers and disassemblers of the \ecs{}.
In addition, the use of object files enables separate compilation which allows compiling only those parts of a program that have actually changed.
\seeobject

\slide{10}

\subsection{Functionality}

Each tool of the \ecs{} provides the same command-line user interface which treats each command-line argument as the name of a textual input file.
Command-line options are not supported in order to make the contents of output files independent from the actual tool invocation.
For compiling programs for example, the compiler as to be invoked using the corresponding name of the source code file.
This results in a new object file for the provided source code which can be used to link the executable program.
Depending on the used compiler, the hardware architecture it targets, as well as the desired runtime environment, the corresponding runtime support files have to be linked as well.
The same result can also be achieved using a utility tool called \tool{ecsd} which automatically infers the required tools from the source file and the specified target environment and conveniently invokes them with the necessary runtime support.

\slide{11}

\subsection{Status}

Except for \cpp{} all programming language implementations provided by the \ecs{} are completed.
The \cpp{} implementation features a preprocessor, parser, and pretty printer but currently lacks proper support for the semantic checker, interpreter, documentation generator, and intermediate code emitter.
Once, \cpp{} is supported sufficiently, its test suite intended to cover all of the ISO \cpp{} Standard ISO/IEC 14882:2023~\cite{iso2023} can be completed as well.
The same holds for the currently still missing software-based floating-point support for all hardware architectures that do not have a native floating-point instruction set.
Some architectures like MIPS or MicroBlaze have not yet been tested on simulators or actual hardware.

\slide{12}

\subsection{Comparison}

In contrast to comparable proprietary compilers, the \ecs{} is free software which allows it to be studied, modified, and used for any purpose.
In comparison to other free software compiler suites like GCC or Clang however, the code base of the \ecs{} is much smaller and allows it to be maintained by a single person.
In addition, the \ecs{} is completely self-contained and requires no other software in order to build programs.
Furthermore, all of its tools are completely portable and therefore executable in any runtime environment.
In order to target different systems, it therefore often suffices to exchange the runtime support for the corresponding runtime environment.

\slide{13}

On the other hand however, the \ecs{} is not highly optimized and does not provide a debugger.
Since it defines its own object file format, the produced object files and binaries are not ABI compatible with existing formats.

\end{presentation}

\begin{presentation}{Implementation Details}{implementation}

The \ecs{} is a completely free and self-contained toolchain written in \cpp{}.
This presentation gives some background information about its implementation and is mainly intended for developers and maintainers of the \ecs{}.

\slide{2}

The \ecs{} provides a makefile which allows building all of its tools in environments like Windows, OS~X, and Linux-based operating systems.
Its source code however is does not depend on any particular operating system, underlying hardware architecture, or any tools other than a conforming \cpp{} compiler.

\subsection{Overall Design}

The tools of the \ecs{} are completely written in standard \cpp{}.
The only issue where its implementation currently depends on the execution environment is the handling of directory path delimiting characters.
Apart from that, the source code of the \ecs{} is completely portable and designed to be maintainable and self-documenting.
By convention, it uniformly uses a consistent naming convention and self-explanatory function predicates which purposely render a lot of comments unnecessary.

\slide{3}

The project itself has a simple structure with meaningful filenames that directly follows its logical layout.
From the beginning, all of the files contained therein have been plain text files and under revision control.
Quality assurance is achieved using an issue tracking system and build automation utilities.

\subsection{Code Structure}

The implementation of each major component of the \ecs{} is physically stored in one source and one header file.
Components of a typical programming language implementation are its syntax representation, the lexer, the parser, the semantic checker, and the intermediate code emitter.
The implementation of a hardware architecture on the other hand is composed of a representation of its instruction set, an assembler, a disassembler, and a machine code generator.
Other examples of major abstractions are the representations of intermediate code, object files, and debugging information.

\slide{4}

Each tool of the \ecs{} combines one or more of these components into an executable file.
For each tool there is a corresponding source file that contains the \texttt{main} function and is called a driver.
Examples of drivers are pretty printers, semantic checkers, interpreters, and compilers for various programming languages.
Furthermore there are one or more assemblers and disassemblers for each hardware architecture, as well as linkers and debugging tools.

\slide{5}

Finally there are common utilities that are used throughout the whole implementation of the \ecs{}.
These utilities are provided by header files and provide frameworks for writing drivers, or generic abstractions like character sets and diagnostic interfaces.

Logically, each programming language, hardware architecture, and major abstraction provides its components as classes in a distinct namespace.
These namespaces and all common utilities reside in a global namespace called \texttt{ECS}.
This allows any component of the \ecs{} to be used as a library in other projects without provoking name clashes.
Drivers are examples of how that library interface is used to combine various components of the \ecs{} into executable files.

\slide{6}

\subsection{Macro Definition Files}

In addition to source and header files, the implementations of almost all major components make use of so-called \emph{macro definition files}.
These are essentially header files that contain only sequential invocations of preprocessor macros typically known as X~macros.
This allows including the same macro definition file in various places using different macro definitions.
The idea is to repeat the sequence of macro invocations in various different contexts while maintaining the actual macro arguments in a single place.

\slide{7}

A typical use case of macro definition files is the definition of enumerations in header files where the macro is defined to provide the name of an enumerator.
The same macro definition file is then typically also included in the corresponding source file where the macro may for example be defined to provide the textual representation of each enumerator in an array of constant strings.
This is useful for maintaining a sorted list of the keywords of a programming language for example.
Since its lexer can compare scanned identifiers with elements of the constant string array, modifying the macro definition file is usually sufficient to add or remove keywords.

\slide{8}

Other components that profit from the same technique are instruction set implementations which can represent complete instruction set tables in a single macro definition file.
Using an appropriate macro definition, the corresponding set of instruction mnemonics can even be included in the documentation of the \ecs{}.
\Documentation{}~\documentationref{assembly}{Generic Assembly Language Specification} for example uses this technique to list all instruction sets supported by the \ecs{}.

\subsection{Context Classes}

Each major component described above is represented using a distinct class that provides an interface for processing or transforming some kind of intermediate representation.
A parser for example is a class that provides a function which takes a character stream as input and transforms it into an abstract syntax tree.
The implementation of that interface however is typically not provided by the component class itself but by a private and nested class.
These classes are conventionally called context classes and have several advantages over more traditional designs which implement the interface directly in the component class.

\slide{9}

Since the implementation of an interface is just forwarded to the context class, the latter can be defined and implemented exclusively in the source file.
All of the processing state that is necessary during an invocation of the interface can therefore be stored by the temporary instance of the context class rather than the component.
As a consequence, the only information that remains necessary to represent directly in the component class is some optional constant configuration.
This renders the actual definition of the component class in the header file light-weight, stable, and by design thread-safe.

\slide{10}

\subsection{Error Handling}

The \ecs{} provides a generic interface for reporting different kinds of diagnostic messages like errors, warnings, and notes.
For drivers, it provides an implementation thereof which prints consistent diagnostic messages to standard output streams.
In contrast to warnings and notes however which have merely informational purposes, errors additionally indicate failures to proceed.
After emitting an arbitrarily detailed diagnostic message, an error is therefore also reported by throwing an exception.
This technique conveniently allows programmers to prematurely abort any processing that results in an error without having to provide any information about the error in the exception itself.
The resulting code can therefore treat error conditions like assertions and does not actually need to handle errors after they have been reported.
As a consequence, most components of the \ecs{} typically abort after the first encounter of an error and do not recover without further precaution.
Usually however, subsequent errors are aftereffects anyway whereas independent exceptions can easily be batched together if necessary.

\slide{11}

\end{presentation}

\concludechapter

% Frequently Asked Questions
% Copyright (C) Florian Negele

% This file is part of the Eigen Compiler Suite.

% Permission is granted to copy, distribute and/or modify this document
% under the terms of the GNU Free Documentation License, Version 1.3
% or any later version published by the Free Software Foundation.

% You should have received a copy of the GNU Free Documentation License
% along with the ECS.  If not, see <https://www.gnu.org/licenses/>.

% Generic documentation utilities
% Copyright (C) Florian Negele

% This file is part of the Eigen Compiler Suite.

% Permission is granted to copy, distribute and/or modify this document
% under the terms of the GNU Free Documentation License, Version 1.3
% or any later version published by the Free Software Foundation.

% You should have received a copy of the GNU Free Documentation License
% along with the ECS.  If not, see <https://www.gnu.org/licenses/>.

\providecommand{\cpp}{C\texttt{++}}
\providecommand{\opt}{_\mathit{opt}}
\providecommand{\tool}[1]{\texttt{#1}}
\providecommand{\version}{Version 0.0.40}
\providecommand{\resource}[1]{*++\txt{#1}}
\providecommand{\ecs}{Eigen Compiler Suite}
\providecommand{\changed}[1]{\underline{#1}}
\providecommand{\toolbox}[1]{\converter{#1}}
\providecommand{\file}{}\renewcommand{\file}[1]{\texttt{#1}}
\providecommand{\alignright}{\hfill\linebreak[0]\hspace*{\fill}}
\providecommand{\converter}[1]{*++[F][F*:white][F,:gray]\txt{#1}}
\providecommand{\documentation}{\ifbook chapter\else document\fi}
\providecommand{\Documentation}{\ifbook Chapter\else Document\fi}
\providecommand{\variable}[1]{\resource{\texttt{\small#1}\\variable}}
\providecommand{\documentationref}[2]{\ifbook\ref{#1}\else``\href{#1}{#2}''~\cite{#1}\fi}
\providecommand{\objfile}[1]{\texttt{#1}\index[runtime]{#1 object file@\texttt{#1} object file}}
\providecommand{\libfile}[1]{\texttt{#1}\index[runtime]{#1 library file@\texttt{#1} library file}}
\providecommand{\epigraph}[2]{\ifbook\begin{quote}\flushright\textit{#1}\par--- #2\end{quote}\fi}
\providecommand{\environmentvariable}[1]{\texttt{#1}\index{Environment variables!#1@\texttt{#1}}}
\providecommand{\environment}[1]{\texttt{#1}\index[environment]{#1 environment@\texttt{#1} environment}}
\providecommand{\toolsection}{}\renewcommand{\toolsection}[1]{\subsection{#1}\label{\prefix:#1}\tool{#1}}
\providecommand{\instruction}{}\renewcommand{\instruction}[2]{\noindent\qquad\pdftooltip{\texttt{#1}}{#2}\refstepcounter{instruction}\par}
\providecommand{\flowgraph}{}\renewcommand{\flowgraph}[1]{\par\sffamily\begin{displaymath}\xymatrix@=4ex{#1}\end{displaymath}\normalfont\par}
\providecommand{\instructionset}{}\renewcommand{\instructionset}[4]{\setcounter{instruction}{0}\begin{multicols}{\ifbook#3\else#4\fi}[{\captionof{table}[#2]{#2 (\ref*{#1:instructions}~instructions)}\label{tab:#1set}\vspace{-2ex}}]\footnotesize\raggedcolumns\input{#1.set}\label{#1:instructions}\end{multicols}}

\providecommand{\gpl}{GNU General Public License}
\providecommand{\rse}{ECS Runtime Support Exception}
\providecommand{\fdl}{\href{https://www.gnu.org/licenses/fdl.html}{GNU Free Documentation License}}

\providecommand{\docbegin}{}
\providecommand{\docend}{}
\providecommand{\doclabel}[1]{\hypertarget{#1}}
\providecommand{\doclink}[2]{\hyperlink{#1}{#2}}
\providecommand{\docsection}[3]{\hypertarget{#1}{\subsection{#2}}\label{sec:#1}\index[library]{#2@#3}}
\providecommand{\docsectionstar}[1]{}
\providecommand{\docsubbegin}{\begin{description}}
\providecommand{\docsubend}{\end{description}}
\providecommand{\docsubsection}[3]{\item[\hypertarget{#1}{#2}]\index[library]{#2@#3}}
\providecommand{\docsubsectionstar}[1]{\smallskip}
\providecommand{\docsubsubsection}[3]{\docsubsection{#1}{#2}{#3}}
\providecommand{\docsubsubsectionstar}[1]{}
\providecommand{\docsubsubsubsection}[3]{}
\providecommand{\docsubsubsubsectionstar}[1]{}
\providecommand{\doctable}{}

\providecommand{\debuggingtool}{}\renewcommand{\debuggingtool}{This tool is provided for debugging purposes.
It allows exposing and modifying an internal data structure that is usually not accessible.
}

\providecommand{\interface}{All tools accept command-line arguments which are taken as names of plain text files containing the source code.
If no arguments are provided, the standard input stream is used instead.
Output files are generated in the current working directory and have the same name as the input file being processed whereas the filename extension gets replaced by an appropriate suffix.
\seeinterface
}

\providecommand{\license}{\noindent Copyright \copyright{} Florian Negele\par\medskip\noindent
Permission is granted to copy, distribute and/or modify this document under the terms of the
\fdl{}, Version 1.3 or any later version published by the \href{https://fsf.org/}{Free Software Foundation}.
}

\providecommand{\ecslogosurface}{
\fill[darkgray] (0,0,0) -- (0,0,3) -- (0,3,3) -- (0,3,1) -- (0,4,1) -- (0,4,3) -- (0,5,3) -- (0,5,0) -- (0,2,0) -- (0,2,2) -- (0,1,2) -- (0,1,0) -- cycle;
\fill[gray] (0,5,0) -- (0,5,3) -- (1,5,3) -- (1,5,1) -- (2,5,1) -- (2,5,3) -- (3,5,3) -- (3,5,0) -- cycle;
\fill[lightgray] (0,0,0) -- (0,1,0) -- (2,1,0) -- (2,4,0) -- (1,4,0) -- (1,3,0) -- (2,3,0) -- (2,2,0) -- (0,2,0) -- (0,5,0) -- (3,5,0) -- (3,0,0) -- cycle;
\begin{scope}[line width=0.5]
\begin{scope}[gray]
\draw (0,0,0) -- (0,1,0);
\draw (2,1,0) -- (2,2,0);
\draw (0,1,2) -- (0,2,2);
\draw (0,2,0) -- (0,5,0);
\draw (2,3,0) -- (2,4,0);
\end{scope}
\begin{scope}[lightgray]
\draw (0,1,0) -- (0,1,2);
\draw (0,3,1) -- (0,3,3);
\draw (0,5,0) -- (0,5,3);
\draw (2,5,1) -- (2,5,3);
\end{scope}
\begin{scope}[white]
\draw (0,1,0) -- (2,1,0);
\draw (1,3,0) -- (2,3,0);
\draw (0,5,0) -- (3,5,0);
\end{scope}
\end{scope}
}

\providecommand{\ecslogo}[1]{
\begin{tikzpicture}[scale={(#1)/((sin(45)+cos(45))*3cm)},x={({-cos(45)*1cm},{sin(45)*sin(30)*1cm})},y={({0cm},{(cos(30)*1cm})},z={({sin(45)*1cm},{cos(45)*sin(30)*1cm})}]
\begin{scope}[darkgray,line width=1]
\draw (0,0,0) -- (0,0,3) -- (0,3,3) -- (2,3,3) -- (2,5,3) -- (3,5,3) -- (3,5,0) -- (3,0,0) -- cycle;
\draw (0,3,1) -- (0,4,1) -- (0,4,3) -- (0,5,3) -- (1,5,3) -- (1,5,1) -- (2,5,1);
\draw (1,3,0) -- (1,4,0) -- (2,4,0);
\end{scope}
\fill[darkgray] (2,0,0) -- (2,0,3) -- (2,5,3) -- (2,5,1) -- (2,4,1) -- (2,4,0) -- cycle;
\fill[lightgray] (2,0,2) -- (0,0,2) -- (0,2,2) -- (2,2,2) -- cycle;
\fill[gray] (0,1,0) -- (2,1,0) -- (2,1,2) -- (0,1,2) -- cycle;
\fill[gray] (0,3,1) -- (0,3,3) -- (2,3,3) -- (2,3,0) -- (1,3,0) -- (1,3,1) -- cycle;
\ecslogosurface
\end{tikzpicture}
}

\providecommand{\shadowedecslogo}[3]{
\begin{tikzpicture}[scale={(#1)/((sin(#2)+cos(#2))*3cm)},x={({-cos(#2)*1cm},{sin(#2)*sin(#3)*1cm})},y={({0cm},{(cos(#3)*1cm})},z={({sin(#2)*1cm},{cos(#2)*sin(#3)*1cm})}]
\shade[top color=lightgray!50!white,bottom color=white,middle color=lightgray!50!white] (0,0,0) -- (3,0,0) -- (3,{-0.5-3*sin(#2)*sin(#3)/cos(#3)},0) -- (0,-0.5,0) -- cycle;
\shade[top color=darkgray!50!gray,bottom color=white,middle color=darkgray!50!white] (0,0,0) -- (0,0,3) -- (0,{-0.5-3*cos(#2)*sin(#3)/cos(#3)},3) -- (0,-0.5,0) -- cycle;
\begin{scope}[y={({(cos(#2)+sin(#2))*0.5cm},{(cos(#2)*sin(#3)-sin(#2)*sin(#3))*0.5cm})}]
\useasboundingbox (3,0,0) -- (0,0,0) -- (0,0,3);
\shade[left color=darkgray!80!black,right color=lightgray,middle color=gray] (0,0,0) -- (0,1,0) -- (0,1,0.5) -- (0,2,0) -- (0,5,0) -- (0,5,3) -- (1,5,3) -- (1,4,3) -- (1,4,2.5) -- (1,3,3) -- (2,5,3) -- (3,5,3) -- (3,0,3) -- cycle;
\clip (0,0,0) -- (0,0,3) -- ({-3*sin(#2)/cos(#2)},0,0) -- cycle;
\shade[left color=darkgray,right color=lightgray!50!gray] (0,0,0) -- (0,1,0) -- (0,1,0.5) -- (0,2,0) -- (0,5,0) -- (0,5,3) -- (1,5,3) -- (1,4,3) -- (1,4,2.5) -- (1,3,3) -- (2,5,3) -- (3,5,3) -- (3,0,3) -- cycle;
\end{scope}
\shade[left color=darkgray,right color=darkgray!80!black] (2,0,0) -- (2,0,3) -- (2,5,3) -- (2,5,1) -- (2,4,1) -- (2,4,0) -- cycle;
\shade[left color=darkgray!90!black,right color=gray!80!darkgray] (2,0,2) -- (0,0,2) -- (0,2,2) -- (2,2,2) -- cycle;
\shade[top color=darkgray!90!black,bottom color=gray!80!darkgray] (0,1,0) -- (2,1,0) -- (2,1,2) -- (0,1,2) -- cycle;
\shade[top color=darkgray!90!black,bottom color=gray!80!darkgray] (0,3,1) -- (0,3,3) -- (2,3,3) -- (2,3,0) -- (1,3,0) -- (1,3,1) -- cycle;
\fill[gray] (2,1,0) -- (1.5,1,0.5) -- (0,1,0.5) -- (0,1,0) -- cycle;
\fill[gray] (1,3,2) -- (0.5,3,2) -- (0.5,3,3) -- (1,3,3) -- cycle;
\fill[gray] (2,3,0) -- (1.5,3,0.5) -- (1,3,0.5) -- (1,3,0) -- cycle;
\ecslogosurface
\end{tikzpicture}
}

\providecommand{\cpplogo}[1]{
\begin{tikzpicture}[scale=(#1)/512em]
\fill[gray] (435.2794,398.7159) -- (247.1911,507.3075) .. controls (236.3563,513.5642) and (218.6240,513.5642) .. (207.7892,507.3075) -- (19.7009,398.7159) .. controls (8.8646,392.4606) and (0.0000,377.1043) .. (0.0000,364.5924) -- (0.0000,147.4076) .. controls (0.8430,132.8363) and (8.2856,120.7683) .. (19.7009,113.2842) -- (207.7892,4.6926) .. controls (218.6240,-1.5642) and (236.3564,-1.5642) .. (247.1911,4.6926) -- (435.2794,113.2842) .. controls (447.5273,121.4304) and (454.4987,133.6918) .. (454.9803,147.4076) -- (454.9803,364.5924) .. controls (454.5404,377.7571) and (446.6566,391.0351) .. (435.2794,398.7159) -- cycle(75.8301,255.9993) .. controls (74.9389,404.0881) and (273.2892,469.4783) .. (358.8263,331.8769) -- (293.1917,293.8965) .. controls (253.5702,359.4301) and (155.1909,335.9977) .. (151.6601,255.9993) .. controls (152.7204,182.2703) and (249.4137,148.0211) .. (293.1961,218.1065) -- (358.8308,180.1276) .. controls (283.4477,49.2645) and (79.6318,96.3470) .. (75.8301,255.9993) -- cycle(379.1503,247.5747) -- (362.2982,247.5747) -- (362.2982,230.7226) -- (345.4490,230.7226) -- (345.4490,247.5747) -- (328.5969,247.5747) -- (328.5969,264.4254) -- (345.4490,264.4254) -- (345.4490,281.2759) -- (362.2982,281.2759) -- (362.2982,264.4254) -- (379.1503,264.4254) -- cycle(442.3420,247.5747) -- (425.4899,247.5747) -- (425.4899,230.7226) -- (408.6408,230.7226) -- (408.6408,247.5747) -- (391.7886,247.5747) -- (391.7886,264.4254) -- (408.6408,264.4254) -- (408.6408,281.2759) -- (425.4899,281.2759) -- (425.4899,264.4254) -- (442.3420,264.4254) -- cycle;
\end{tikzpicture}
}

\providecommand{\fallogo}[1]{
\begin{tikzpicture}[scale=(#1)/512em]
\fill[gray] (185.7774,0.0000) .. controls (200.4486,15.9798) and (226.8966,8.7148) .. (235.0426,31.5836) .. controls (249.5297,58.0598) and (247.9581,97.9161) .. (280.3335,110.9762) .. controls (309.1690,120.3496) and (337.8406,104.2727) .. (366.5753,103.9379) .. controls (373.4449,111.5171) and (379.2885,128.2574) .. (383.9755,108.9744) .. controls (396.6979,102.5615) and (437.2808,107.6681) .. (426.9652,124.3252) .. controls (408.9822,121.0785) and (412.4742,146.0729) .. (426.5192,131.4996) .. controls (433.8413,120.8489) and (465.1541,126.5522) .. (441.9067,135.7950) .. controls (396.1879,157.7478) and (344.1112,161.5079) .. (298.5528,183.5702) .. controls (277.7471,193.5198) and (284.6941,218.7163) .. (285.2127,236.9640) .. controls (292.3599,316.2826) and (307.3929,394.6311) .. (317.1198,473.6154) .. controls (329.0637,505.4736) and (292.1195,528.5004) .. (265.9183,511.2761) .. controls (237.9284,499.2462) and (237.3684,465.2681) .. (230.9102,439.9421) .. controls (218.6692,374.3397) and (215.6307,306.9662) .. (198.1732,242.3977) .. controls (183.1379,232.7444) and (164.4245,256.0298) .. (149.0430,261.4799) .. controls (116.9328,279.2585) and (87.1822,308.5851) .. (48.2293,307.8914) .. controls (21.3220,306.9037) and (-15.9107,281.8761) .. (7.2921,252.7908) .. controls (29.7799,220.6177) and (67.5177,204.3028) .. (100.9287,185.9449) .. controls (130.8217,170.8906) and (161.1548,156.5903) .. (191.0278,141.5847) .. controls (196.1738,120.0520) and (186.6049,95.2409) .. (186.8382,72.4353) .. controls (185.5234,48.4204) and (183.1700,23.9341) .. (185.7774,0.0000) -- cycle;
\end{tikzpicture}
}

\providecommand{\oblogo}[1]{
\begin{tikzpicture}[scale=(#1)/512em]
\fill[gray] (160.3865,208.9117) .. controls (154.0879,214.6478) and (149.0735,221.2409) .. (145.4125,228.5384) .. controls (184.8790,248.4273) and (234.7122,269.8787) .. (297.5493,291.8782) .. controls (300.3943,281.4769) and (300.9552,268.7619) .. (300.4023,255.2389) .. controls (248.9909,244.7891) and (200.0310,225.9279) .. (160.3865,208.9117) -- cycle(225.7398,392.6996) .. controls (308.0209,392.1716) and (359.3326,345.9277) .. (368.7203,285.2098) .. controls (376.6742,197.1784) and (311.7194,141.3342) .. (205.4287,142.1456) .. controls (139.9485,141.4804) and (88.7155,166.1957) .. (73.5775,228.0086) .. controls (52.0297,320.3408) and (123.4078,391.0103) .. (225.7398,392.6996) -- cycle(216.0739,176.4733) .. controls (268.9183,179.2424) and (315.8292,206.5488) .. (312.7454,265.1139) .. controls (313.2769,315.6384) and (286.5993,353.4946) .. (216.6040,355.7934) .. controls (162.4657,355.7934) and (126.0914,317.5023) .. (126.0914,260.5103) .. controls (126.1733,214.2900) and (163.3363,176.2849) .. (216.0739,176.4733) -- cycle(76.4897,189.1754) .. controls (13.1586,147.5631) and (0.0000,119.4207) .. (0.0000,119.4207) -- (90.6499,170.1632) .. controls (85.3004,175.8497) and (80.5994,182.1633) .. (76.4897,189.1754) -- cycle(353.9486,119.3004) -- (402.9482,119.3004) .. controls (427.0025,137.0797) and (450.9893,162.7034) .. (474.9529,191.0213) .. controls (509.3540,228.5339) and (531.3391,294.2091) .. (487.8149,312.1206) .. controls (462.8165,324.7652) and (394.3874,316.8943) .. (373.8912,313.6651) .. controls (379.9291,297.7449) and (383.2899,278.4204) .. (381.4989,257.7214) .. controls (420.3069,248.0321) and (421.9610,218.3461) .. (407.7867,192.6417) .. controls (391.1113,162.4018) and (370.1114,132.9097) .. (353.9486,119.3004) -- cycle;
\end{tikzpicture}
}

\providecommand{\markuptable}{
\begin{table}
\sffamily\centering
\begin{tabular}{@{}lcl@{}}
\toprule
\texttt{//italics//} & $\rightarrow$ & \textit{italics} \\
\midrule
\texttt{**bold**} & $\rightarrow$ & \textbf{bold} \\
\midrule
\texttt{\# ordered list} & & 1 ordered list \\
\texttt{\# second item} & $\rightarrow$ & 2 second item \\
\texttt{\#\# sub item} & & \hspace{1em} 1 sub item \\
\midrule
\texttt{* unordered list} & & $\bullet$ unordered list \\
\texttt{* second item} & $\rightarrow$ & $\bullet$ second item \\
\texttt{** sub item} & & \hspace{1em} $\bullet$ sub item \\
\midrule
\texttt{link to [[label]]} & $\rightarrow$ & link to \underline{label} \\
\midrule
\texttt{<{}<label>{}> definition } & $\rightarrow$ & definition \\
\midrule
\texttt{[[url|link name]]} & $\rightarrow$ & \underline{link name} \\
\midrule\addlinespace
\texttt{= large heading} & & {\Large large heading} \smallskip \\
\texttt{== medium heading} & $\rightarrow$ & {\large medium heading} \\
\texttt{=== small heading} & & small heading \\
\midrule
\texttt{no line break} & & no line break for paragraphs \\
\texttt{for paragraphs} & $\rightarrow$ \\
& & use empty line \\
\texttt{use empty line} \\
\midrule
\texttt{force\textbackslash\textbackslash line break} & $\rightarrow$ & force \\
& & line break \\
\midrule
\texttt{horizontal line} & $\rightarrow$ & horizontal line \\
\texttt{----} & & \hrulefill \\
\midrule
\texttt{|=a|=table|=header} & & \underline{a \enspace table \enspace header} \\
\texttt{|a|table|row} & $\rightarrow$ & a \enspace table \enspace row \\
\texttt{|b|table|row} & & b \enspace table \enspace row \\
\midrule
\texttt{\{\{\{} \\
\texttt{unformatted} & $\rightarrow$ & \texttt{unformatted} \\
\texttt{code} & & \texttt{code} \\
\texttt{\}\}\}} \\
\midrule\addlinespace
\texttt{@ new article} & & {\Large 1.\ new article} \smallskip \\
\texttt{@ second article} & $\rightarrow$ & {\Large 2.\ second article} \smallskip \\
\texttt{@@ sub article} & & {\large 2.1.\ sub article} \\
\bottomrule
\end{tabular}
\normalfont\caption{Elements of the generic documentation markup language}
\label{tab:docmarkup}
\end{table}
}

\providecommand{\startchapter}[4]{
\documentclass[11pt,a4paper]{article}
\usepackage{booktabs}
\usepackage[format=hang,labelfont=bf]{caption}
\usepackage{changepage}
\usepackage[T1]{fontenc}
\usepackage[margin=2cm]{geometry}
\usepackage{hyperref}
\usepackage[american]{isodate}
\usepackage{lmodern}
\usepackage{longtable}
\usepackage{mathptmx}
\usepackage{microtype}
\usepackage[toc]{multitoc}
\usepackage{multirow}
\usepackage[all]{nowidow}
\usepackage{pdfcomment}
\usepackage{syntax}
\usepackage{tikz}
\usepackage[all]{xy}
\hypersetup{pdfborder={0 0 0},bookmarksnumbered=true,pdftitle={\ecs{}: #2},pdfauthor={Florian Negele},pdfsubject={\ecs{}},pdfkeywords={#1}}
\setlength{\grammarindent}{8em}\setlength{\grammarparsep}{0.2ex}
\setlength{\columnsep}{2em}
\newcommand{\prefix}{}
\newcounter{instruction}
\bibliographystyle{unsrt}
\renewcommand{\index}[2][]{}
\renewcommand{\arraystretch}{1.05}
\renewcommand{\floatpagefraction}{0.7}
\renewcommand{\syntleft}{\itshape}\renewcommand{\syntright}{}
\title{\vspace{-5ex}\Huge{\ecs{}}\medskip\hrule}
\author{\huge{#2}}
\date{\medskip\version}
\newif\ifbook\bookfalse
\pagestyle{headings}
\frenchspacing
\begin{document}
\maketitle\thispagestyle{empty}\noindent#4\setlength{\columnseprule}{0.4pt}\tableofcontents\setlength{\columnseprule}{0pt}\vfill\pagebreak[3]\null\vfill\bigskip\noindent
\parbox{\textwidth-4em}{\license The contents of this \documentation{} are part of the \href{manual}{\ecs{} User Manual}~\cite{manual} and correspond to Chapter ``\href{manual\##3}{#1}''.\alignright\mbox{\today}}
\parbox{4em}{\flushright\ecslogo{3em}}
\clearpage
}

\providecommand{\concludechapter}{
\vfill\pagebreak[3]\null\vfill
\thispagestyle{myheadings}\markright{REFERENCES}
\noindent\begin{minipage}{\textwidth}\begin{multicols}{2}[\section*{References}]
\renewcommand{\section}[2]{}\small\bibliography{references}
\end{multicols}\end{minipage}\end{document}
}

\providecommand{\startpresentation}[2]{
\documentclass[14pt,aspectratio=43,usepdftitle=false]{beamer}
\usepackage{booktabs}
\usepackage{etex}
\usepackage{multicol}
\usepackage{tikz}
\usepackage[all]{xy}
\bibliographystyle{unsrt}
\setlength{\columnsep}{1em}
\setlength{\leftmargini}{1em}
\setbeamercolor{title}{fg=black}
\setbeamercolor{structure}{fg=darkgray}
\setbeamercolor{bibliography item}{fg=darkgray}
\setbeamerfont{title}{series=\bfseries}
\setbeamerfont{subtitle}{series=\normalfont}
\setbeamerfont*{frametitle}{parent=title}
\setbeamerfont{block title}{series=\bfseries}
\setbeamerfont*{framesubtitle}{parent=subtitle}
\setbeamersize{text margin left=1em,text margin right=1em}
\setbeamertemplate{navigation symbols}{}
\setbeamertemplate{itemize item}[circle]{}
\setbeamertemplate{bibliography item}[triangle]{}
\setbeamertemplate{bibliography entry author}{\usebeamercolor[fg]{bibliography item}}
\setbeamertemplate{frametitle}{\medskip\usebeamerfont{frametitle}\color{gray}\raisebox{-2.5ex}[0ex][0ex]{\rule{0.1em}{4.5ex}}}
\addtobeamertemplate{frametitle}{}{\hspace{0.4em}\usebeamercolor[fg]{title}\insertframetitle\par\vspace{0.2ex}\hspace{0.5em}\usebeamerfont{framesubtitle}\insertframesubtitle}
\hypersetup{pdfborder={0 0 0},bookmarksnumbered=true,bookmarksopen=true,bookmarksopenlevel=0,pdftitle={\ecs{}: #1},pdfauthor={Florian Negele},pdfsubject={\ecs{}},pdfkeywords={#1}}
\renewcommand{\flowgraph}[1]{\resizebox{\textwidth}{!}{$$\xymatrix{##1}$$}}
\title{\ecs{}\medskip\hrule\medskip}
\institute{\shadowedecslogo{5em}{30}{15}}
\date{\version}
\subtitle{#1}
\begin{document}
\begin{frame}[plain]\titlepage\nocite{manual}\end{frame}
\begin{frame}{Contents}{#1}\begin{center}\tableofcontents\end{center}\end{frame}
}

\providecommand{\concludepresentation}{
\begin{frame}{References}\begin{footnotesize}\setlength{\columnseprule}{0.4pt}\begin{multicols}{2}\bibliography{references}\end{multicols}\end{footnotesize}\end{frame}
\end{document}
}

\providecommand{\startbook}[1]{
\documentclass[10pt,paper=17cm:24cm,DIV=13,twoside=semi,headings=normal,numbers=noendperiod,cleardoublepage=plain]{scrbook}
\usepackage{atveryend}
\usepackage{booktabs}
\usepackage{caption}
\usepackage{changepage}
\usepackage[T1]{fontenc}
\usepackage{imakeidx}
\usepackage{hyperref}
\usepackage[american]{isodate}
\usepackage{lmodern}
\usepackage{longtable}
\usepackage{mathptmx}
\usepackage[final]{microtype}
\usepackage{multicol}
\usepackage{multirow}
\usepackage[all]{nowidow}
\usepackage{pdfcomment}
\usepackage{scrlayer-scrpage}
\usepackage{setspace}
\usepackage{syntax}
\usepackage[eventxtindent=4pt,oddtxtexdent=4pt]{thumbs}
\usepackage{tikz}
\usepackage[all]{xy}
\hyphenation{Micro-Blaze Open-Cores Open-RISC Power-PC}
\hypersetup{pdfborder={0 0 0},bookmarksnumbered=true,bookmarksopen=true,bookmarksopenlevel=0,pdftitle={\ecs{}: #1},pdfauthor={Florian Negele},pdfsubject={\ecs{}},pdfkeywords={#1}}
\setlength{\grammarindent}{8em}\setlength{\grammarparsep}{0.7ex}
\setkomafont{captionlabel}{\usekomafont{descriptionlabel}}
\renewcommand{\arraystretch}{1.05}\setstretch{1.1}
\renewcommand{\chapterformat}{\thechapter\autodot\enskip\raisebox{-1ex}[0ex][0ex]{\color{gray}\rule{0.1em}{3.5ex}}\enskip}
\renewcommand{\startchapter}[4]{\hypertarget{##3}{\chapter{##1}}\label{##3}##4\addthumb{##1}{\LARGE\sffamily\bfseries\thechapter}{white}{gray}\renewcommand{\prefix}{##3}}
\renewcommand{\concludechapter}{\clearpage{\stopthumb\cleardoublepage}}
\renewcommand{\syntleft}{\itshape}\renewcommand{\syntright}{}
\renewcommand{\floatpagefraction}{0.7}
\renewcommand{\partheademptypage}{}
\DeclareMicrotypeAlias{lmss}{cmr}
\newcommand{\prefix}{}
\newcounter{instruction}
\bibliographystyle{unsrt}
\newif\ifbook\booktrue
\makeindex[intoc,title=Index]
\makeindex[intoc,name=tools,title=Index of Tools,columns=3]
\makeindex[intoc,name=library,title=Index of Library Names]
\makeindex[intoc,name=runtime,title=Index of Runtime Support]
\makeindex[intoc,name=environment,title=Index of Target Environments]
\indexsetup{toclevel=chapter,headers={\indexname}{\indexname}}
\frenchspacing
\begin{document}
\pagenumbering{alph}
\begin{titlepage}\centering
\huge\sffamily\null\vfill\textbf{\ecs{}}\bigskip\hrule\bigskip#1
\normalsize\normalfont\vfill\vfill\shadowedecslogo{10em}{30}{15}
\large\vfill\vfill\version
\end{titlepage}
\null\vfill
\thispagestyle{empty}
\noindent\today\par\medskip
\license A copy of this license is included in Appendix~\ref{fdl} on page~\pageref{fdl}.
All product names used herein are for identification purposes only and may be trademarks of their respective companies.
\concludechapter
\frontmatter
\setcounter{tocdepth}{1}
\tableofcontents
\setcounter{tocdepth}{2}
\concludechapter
\listoffigures
\concludechapter
\listoftables
\concludechapter
}

\providecommand{\concludebook}{
\backmatter
\addtocontents{toc}{\protect\setcounter{tocdepth}{-1}}
\phantomsection\addcontentsline{toc}{part}{Bibliography}
\bibliography{references}
\concludechapter
\phantomsection\addcontentsline{toc}{part}{Indexes}
\printindex
\concludechapter
\indexprologue{\label{idx:tools}}
\printindex[tools]
\concludechapter
\printindex[library]
\concludechapter
\indexprologue{\label{idx:runtime}}
\printindex[runtime]
\concludechapter
\indexprologue{\label{idx:environment}}
\printindex[environment]
\concludechapter
\pagestyle{empty}\pagenumbering{Alph}\null\clearpage
\null\vfill\centering\ecslogo{4em}\par\medskip\license
\end{document}
}

% chapter references

\providecommand{\seedocumentationref}{}\renewcommand{\seedocumentationref}[3]{#1, see \Documentation{}~\documentationref{#2}{#3}. }
\providecommand{\seeinterface}{}\renewcommand{\seeinterface}{\ifbook See \Documentation{}~\documentationref{interface}{User Interface} for more information about the common user interface of all of these tools. \fi}
\providecommand{\seeguide}{}\renewcommand{\seeguide}{\seedocumentationref{For basic examples of using some of these tools in practice}{guide}{User Guide}}
\providecommand{\seecpp}{}\renewcommand{\seecpp}{\seedocumentationref{For more information about the \cpp{} programming language and its implementation by the \ecs{}}{cpp}{User Manual for \cpp{}}}
\providecommand{\seefalse}{}\renewcommand{\seefalse}{\seedocumentationref{For more information about the FALSE programming language and its implementation by the \ecs{}}{false}{User Manual for FALSE}}
\providecommand{\seeoberon}{}\renewcommand{\seeoberon}{\seedocumentationref{For more information about the Oberon programming language and its implementation by the \ecs{}}{oberon}{User Manual for Oberon}}
\providecommand{\seeassembly}{}\renewcommand{\seeassembly}{\seedocumentationref{For more information about the generic assembly language and how to use it}{assembly}{Generic Assembly Language Specification}}
\providecommand{\seeamd}{}\renewcommand{\seeamd}{\seedocumentationref{For more information about how the \ecs{} supports the AMD64 hardware architecture}{amd64}{AMD64 Hardware Architecture Support}}
\providecommand{\seearm}{}\renewcommand{\seearm}{\seedocumentationref{For more information about how the \ecs{} supports the ARM hardware architecture}{arm}{ARM Hardware Architecture Support}}
\providecommand{\seeavr}{}\renewcommand{\seeavr}{\seedocumentationref{For more information about how the \ecs{} supports the AVR hardware architecture}{avr}{AVR Hardware Architecture Support}}
\providecommand{\seeavrtt}{}\renewcommand{\seeavrtt}{\seedocumentationref{For more information about how the \ecs{} supports the AVR32 hardware architecture}{avr32}{AVR32 Hardware Architecture Support}}
\providecommand{\seemabk}{}\renewcommand{\seemabk}{\seedocumentationref{For more information about how the \ecs{} supports the M68000 hardware architecture}{m68k}{M68000 Hardware Architecture Support}}
\providecommand{\seemibl}{}\renewcommand{\seemibl}{\seedocumentationref{For more information about how the \ecs{} supports the MicroBlaze hardware architecture}{mibl}{MicroBlaze Hardware Architecture Support}}
\providecommand{\seemips}{}\renewcommand{\seemips}{\seedocumentationref{For more information about how the \ecs{} supports the MIPS32 and MIPS64 hardware architectures}{mips}{MIPS Hardware Architecture Support}}
\providecommand{\seemmix}{}\renewcommand{\seemmix}{\seedocumentationref{For more information about how the \ecs{} supports the MMIX hardware architecture}{mmix}{MMIX Hardware Architecture Support}}
\providecommand{\seeorok}{}\renewcommand{\seeorok}{\seedocumentationref{For more information about how the \ecs{} supports the OpenRISC 1000 hardware architecture}{or1k}{OpenRISC 1000 Hardware Architecture Support}}
\providecommand{\seeppc}{}\renewcommand{\seeppc}{\seedocumentationref{For more information about how the \ecs{} supports the PowerPC hardware architecture}{ppc}{PowerPC Hardware Architecture Support}}
\providecommand{\seerisc}{}\renewcommand{\seerisc}{\seedocumentationref{For more information about how the \ecs{} supports the RISC hardware architecture}{risc}{RISC Hardware Architecture Support}}
\providecommand{\seewasm}{}\renewcommand{\seewasm}{\seedocumentationref{For more information about how the \ecs{} supports the WebAssembly architecture}{wasm}{WebAssembly Architecture Support}}
\providecommand{\seedocumentation}{}\renewcommand{\seedocumentation}{\seedocumentationref{For more information about generic documentations and their generation by the \ecs{}}{documentation}{Generic Documentation Generation}}
\providecommand{\seedebugging}{}\renewcommand{\seedebugging}{\seedocumentationref{For more information about debugging information and its representation}{debugging}{Debugging Information Representation}}
\providecommand{\seecode}{}\renewcommand{\seecode}{\seedocumentationref{For more information about intermediate code and its purpose}{code}{Intermediate Code Representation}}
\providecommand{\seeobject}{}\renewcommand{\seeobject}{\seedocumentationref{For more information about object files and their purpose}{object}{Object File Representation}}

% generic documentation tools

\providecommand{\docprint}{
\toolsection{docprint} is a pretty printer for generic documentations.
It reformats generic documentations and writes it to the standard output stream.
\debuggingtool
\flowgraph{\resource{generic\\documentation} \ar[r] & \toolbox{docprint} \ar[r] & \resource{generic\\documentation}}
\seedocumentation
}

\providecommand{\doccheck}{
\toolsection{doccheck} is a syntactic and semantic checker for generic documentations.
It just performs syntactic and semantic checks on generic documentations and writes its diagnostic messages to the standard error stream.
\debuggingtool
\flowgraph{\resource{generic\\documentation} \ar[r] & \toolbox{doccheck} \ar[r] & \resource{diagnostic\\messages}}
\seedocumentation
}

\providecommand{\dochtml}{
\toolsection{dochtml} is an HTML documentation generator for generic documentations.
It processes several generic documentations and assembles all information therein into an HTML document.
\debuggingtool
\flowgraph{\resource{generic\\documentation} \ar[r] & \toolbox{dochtml} \ar[r] & \resource{HTML\\document}}
\seedocumentation
}

\providecommand{\doclatex}{
\toolsection{doclatex} is a Latex documentation generator for generic documentations.
It processes several generic documentations and assembles all information therein into a Latex document.
\debuggingtool
\flowgraph{\resource{generic\\documentation} \ar[r] & \toolbox{doclatex} \ar[r] & \resource{Latex\\document}}
\seedocumentation
}

% intermediate code tools

\providecommand{\cdcheck}{
\toolsection{cdcheck} is a syntactic and semantic checker for intermediate code.
It just performs syntactic and semantic checks on programs written in intermediate code and writes its diagnostic messages to the standard error stream.
\debuggingtool
\flowgraph{\resource{intermediate\\code} \ar[r] & \toolbox{cdcheck} \ar[r] & \resource{diagnostic\\messages}}
\seeassembly\seecode
}

\providecommand{\cdopt}{
\toolsection{cdopt} is an optimizer for intermediate code.
It performs various optimizations on programs written in intermediate code and writes the result to the standard output stream.
\debuggingtool
\flowgraph{\resource{intermediate\\code} \ar[r] & \toolbox{cdopt} \ar[r] & \resource{optimized\\code}}
\seeassembly\seecode
}

\providecommand{\cdrun}{
\toolsection{cdrun} is an interpreter for intermediate code.
It processes and executes programs written in intermediate code.
The following code sections are predefined and have the usual semantics:
\texttt{abort}, \texttt{\_Exit}, \texttt{fflush}, \texttt{floor}, \texttt{fputc}, \texttt{free}, \texttt{getchar}, \texttt{malloc}, and \texttt{putchar}.
Diagnostic messages about invalid operations include the name of the executed code section and the index of the erroneous instruction.
\debuggingtool
\flowgraph{\resource{intermediate\\code} \ar[r] & \toolbox{cdrun} \ar@/u/[r] & \resource{input/\\output} \ar@/d/[l]}
\seeassembly\seecode
}

\providecommand{\cdamda}{
\toolsection{cdamd16} is a compiler for intermediate code targeting the AMD64 hardware architecture.
It generates machine code for AMD64 processors from programs written in intermediate code and stores it in corresponding object files.
The compiler generates machine code for the 16-bit operating mode defined by the AMD64 architecture.
It also creates a debugging information file as well as an assembly file containing a listing of the generated machine code.
\debuggingtool
\flowgraph{\resource{intermediate\\code} \ar[r] & \toolbox{cdamd16} \ar[r] \ar[d] \ar[rd] & \resource{object file} \\ & \resource{assembly\\listing} & \resource{debugging\\information}}
\seeassembly\seeamd\seeobject\seecode\seedebugging
}

\providecommand{\cdamdb}{
\toolsection{cdamd32} is a compiler for intermediate code targeting the AMD64 hardware architecture.
It generates machine code for AMD64 processors from programs written in intermediate code and stores it in corresponding object files.
The compiler generates machine code for the 32-bit operating mode defined by the AMD64 architecture.
It also creates a debugging information file as well as an assembly file containing a listing of the generated machine code.
\debuggingtool
\flowgraph{\resource{intermediate\\code} \ar[r] & \toolbox{cdamd32} \ar[r] \ar[d] \ar[rd] & \resource{object file} \\ & \resource{assembly\\listing} & \resource{debugging\\information}}
\seeassembly\seeamd\seeobject\seecode\seedebugging
}

\providecommand{\cdamdc}{
\toolsection{cdamd64} is a compiler for intermediate code targeting the AMD64 hardware architecture.
It generates machine code for AMD64 processors from programs written in intermediate code and stores it in corresponding object files.
The compiler generates machine code for the 64-bit operating mode defined by the AMD64 architecture.
It also creates a debugging information file as well as an assembly file containing a listing of the generated machine code.
\debuggingtool
\flowgraph{\resource{intermediate\\code} \ar[r] & \toolbox{cdamd64} \ar[r] \ar[d] \ar[rd] & \resource{object file} \\ & \resource{assembly\\listing} & \resource{debugging\\information}}
\seeassembly\seeamd\seeobject\seecode\seedebugging
}

\providecommand{\cdarma}{
\toolsection{cdarma32} is a compiler for intermediate code targeting the ARM hardware architecture.
It generates machine code for ARM processors executing A32 instructions from programs written in intermediate code and stores it in corresponding object files.
It also creates a debugging information file as well as an assembly file containing a listing of the generated machine code.
\debuggingtool
\flowgraph{\resource{intermediate\\code} \ar[r] & \toolbox{cdarma32} \ar[r] \ar[d] \ar[rd] & \resource{object file} \\ & \resource{assembly\\listing} & \resource{debugging\\information}}
\seeassembly\seearm\seeobject\seecode\seedebugging
}

\providecommand{\cdarmb}{
\toolsection{cdarma64} is a compiler for intermediate code targeting the ARM hardware architecture.
It generates machine code for ARM processors executing A64 instructions from programs written in intermediate code and stores it in corresponding object files.
It also creates a debugging information file as well as an assembly file containing a listing of the generated machine code.
\debuggingtool
\flowgraph{\resource{intermediate\\code} \ar[r] & \toolbox{cdarma64} \ar[r] \ar[d] \ar[rd] & \resource{object file} \\ & \resource{assembly\\listing} & \resource{debugging\\information}}
\seeassembly\seearm\seeobject\seecode\seedebugging
}

\providecommand{\cdarmc}{
\toolsection{cdarmt32} is a compiler for intermediate code targeting the ARM hardware architecture.
It generates machine code for ARM processors without floating-point extension executing T32 instructions from programs written in intermediate code and stores it in corresponding object files.
It also creates a debugging information file as well as an assembly file containing a listing of the generated machine code.
\debuggingtool
\flowgraph{\resource{intermediate\\code} \ar[r] & \toolbox{cdarmt32} \ar[r] \ar[d] \ar[rd] & \resource{object file} \\ & \resource{assembly\\listing} & \resource{debugging\\information}}
\seeassembly\seearm\seeobject\seecode\seedebugging
}

\providecommand{\cdarmcfpe}{
\toolsection{cdarmt32fpe} is a compiler for intermediate code targeting the ARM hardware architecture.
It generates machine code for ARM processors with floating-point extension executing T32 instructions from programs written in intermediate code and stores it in corresponding object files.
It also creates a debugging information file as well as an assembly file containing a listing of the generated machine code.
\debuggingtool
\flowgraph{\resource{intermediate\\code} \ar[r] & \toolbox{cdarmt32fpe} \ar[r] \ar[d] \ar[rd] & \resource{object file} \\ & \resource{assembly\\listing} & \resource{debugging\\information}}
\seeassembly\seearm\seeobject\seecode\seedebugging
}

\providecommand{\cdavr}{
\toolsection{cdavr} is a compiler for intermediate code targeting the AVR hardware architecture.
It generates machine code for AVR processors from programs written in intermediate code and stores it in corresponding object files.
It also creates a debugging information file as well as an assembly file containing a listing of the generated machine code.
\debuggingtool
\flowgraph{\resource{intermediate\\code} \ar[r] & \toolbox{cdavr} \ar[r] \ar[d] \ar[rd] & \resource{object file} \\ & \resource{assembly\\listing} & \resource{debugging\\information}}
\seeassembly\seeavr\seeobject\seecode\seedebugging
}

\providecommand{\cdavrtt}{
\toolsection{cdavr32} is a compiler for intermediate code targeting the AVR32 hardware architecture.
It generates machine code for AVR32 processors from programs written in intermediate code and stores it in corresponding object files.
It also creates a debugging information file as well as an assembly file containing a listing of the generated machine code.
\debuggingtool
\flowgraph{\resource{intermediate\\code} \ar[r] & \toolbox{cdavr32} \ar[r] \ar[d] \ar[rd] & \resource{object file} \\ & \resource{assembly\\listing} & \resource{debugging\\information}}
\seeassembly\seeavrtt\seeobject\seecode\seedebugging
}

\providecommand{\cdmabk}{
\toolsection{cdm68k} is a compiler for intermediate code targeting the M68000 hardware architecture.
It generates machine code for M68000 processors from programs written in intermediate code and stores it in corresponding object files.
It also creates a debugging information file as well as an assembly file containing a listing of the generated machine code.
\debuggingtool
\flowgraph{\resource{intermediate\\code} \ar[r] & \toolbox{cdm68k} \ar[r] \ar[d] \ar[rd] & \resource{object file} \\ & \resource{assembly\\listing} & \resource{debugging\\information}}
\seeassembly\seemabk\seeobject\seecode\seedebugging
}

\providecommand{\cdmibl}{
\toolsection{cdmibl} is a compiler for intermediate code targeting the MicroBlaze hardware architecture.
It generates machine code for MicroBlaze processors from programs written in intermediate code and stores it in corresponding object files.
It also creates a debugging information file as well as an assembly file containing a listing of the generated machine code.
\debuggingtool
\flowgraph{\resource{intermediate\\code} \ar[r] & \toolbox{cdmibl} \ar[r] \ar[d] \ar[rd] & \resource{object file} \\ & \resource{assembly\\listing} & \resource{debugging\\information}}
\seeassembly\seemibl\seeobject\seecode\seedebugging
}

\providecommand{\cdmipsa}{
\toolsection{cdmips32} is a compiler for intermediate code targeting the MIPS32 hardware architecture.
It generates machine code for MIPS32 processors from programs written in intermediate code and stores it in corresponding object files.
It also creates a debugging information file as well as an assembly file containing a listing of the generated machine code.
\debuggingtool
\flowgraph{\resource{intermediate\\code} \ar[r] & \toolbox{cdmips32} \ar[r] \ar[d] \ar[rd] & \resource{object file} \\ & \resource{assembly\\listing} & \resource{debugging\\information}}
\seeassembly\seemips\seeobject\seecode\seedebugging
}

\providecommand{\cdmipsb}{
\toolsection{cdmips64} is a compiler for intermediate code targeting the MIPS64 hardware architecture.
It generates machine code for MIPS64 processors from programs written in intermediate code and stores it in corresponding object files.
It also creates a debugging information file as well as an assembly file containing a listing of the generated machine code.
\debuggingtool
\flowgraph{\resource{intermediate\\code} \ar[r] & \toolbox{cdmips64} \ar[r] \ar[d] \ar[rd] & \resource{object file} \\ & \resource{assembly\\listing} & \resource{debugging\\information}}
\seeassembly\seemips\seeobject\seecode\seedebugging
}

\providecommand{\cdmmix}{
\toolsection{cdmmix} is a compiler for intermediate code targeting the MMIX hardware architecture.
It generates machine code for MMIX processors from programs written in intermediate code and stores it in corresponding object files.
It also creates a debugging information file as well as an assembly file containing a listing of the generated machine code.
\debuggingtool
\flowgraph{\resource{intermediate\\code} \ar[r] & \toolbox{cdmmix} \ar[r] \ar[d] \ar[rd] & \resource{object file} \\ & \resource{assembly\\listing} & \resource{debugging\\information}}
\seeassembly\seemmix\seeobject\seecode\seedebugging
}

\providecommand{\cdorok}{
\toolsection{cdor1k} is a compiler for intermediate code targeting the OpenRISC 1000 hardware architecture.
It generates machine code for OpenRISC 1000 processors from programs written in intermediate code and stores it in corresponding object files.
It also creates a debugging information file as well as an assembly file containing a listing of the generated machine code.
\debuggingtool
\flowgraph{\resource{intermediate\\code} \ar[r] & \toolbox{cdor1k} \ar[r] \ar[d] \ar[rd] & \resource{object file} \\ & \resource{assembly\\listing} & \resource{debugging\\information}}
\seeassembly\seeorok\seeobject\seecode\seedebugging
}

\providecommand{\cdppca}{
\toolsection{cdppc32} is a compiler for intermediate code targeting the PowerPC hardware architecture.
It generates machine code for PowerPC processors from programs written in intermediate code and stores it in corresponding object files.
The compiler generates machine code for the 32-bit operating mode defined by the PowerPC architecture.
It also creates a debugging information file as well as an assembly file containing a listing of the generated machine code.
\debuggingtool
\flowgraph{\resource{intermediate\\code} \ar[r] & \toolbox{cdppc32} \ar[r] \ar[d] \ar[rd] & \resource{object file} \\ & \resource{assembly\\listing} & \resource{debugging\\information}}
\seeassembly\seeppc\seeobject\seecode\seedebugging
}

\providecommand{\cdppcb}{
\toolsection{cdppc64} is a compiler for intermediate code targeting the PowerPC hardware architecture.
It generates machine code for PowerPC processors from programs written in intermediate code and stores it in corresponding object files.
The compiler generates machine code for the 64-bit operating mode defined by the PowerPC architecture.
It also creates a debugging information file as well as an assembly file containing a listing of the generated machine code.
\debuggingtool
\flowgraph{\resource{intermediate\\code} \ar[r] & \toolbox{cdppc64} \ar[r] \ar[d] \ar[rd] & \resource{object file} \\ & \resource{assembly\\listing} & \resource{debugging\\information}}
\seeassembly\seeppc\seeobject\seecode\seedebugging
}

\providecommand{\cdrisc}{
\toolsection{cdrisc} is a compiler for intermediate code targeting the RISC hardware architecture.
It generates machine code for RISC processors from programs written in intermediate code and stores it in corresponding object files.
It also creates a debugging information file as well as an assembly file containing a listing of the generated machine code.
\debuggingtool
\flowgraph{\resource{intermediate\\code} \ar[r] & \toolbox{cdrisc} \ar[r] \ar[d] \ar[rd] & \resource{object file} \\ & \resource{assembly\\listing} & \resource{debugging\\information}}
\seeassembly\seerisc\seeobject\seecode\seedebugging
}

\providecommand{\cdwasm}{
\toolsection{cdwasm} is a compiler for intermediate code targeting the WebAssembly architecture.
It generates machine code for WebAssembly targets from programs written in intermediate code and stores it in corresponding object files.
It also creates a debugging information file as well as an assembly file containing a listing of the generated machine code.
\debuggingtool
\flowgraph{\resource{intermediate\\code} \ar[r] & \toolbox{cdwasm} \ar[r] \ar[d] \ar[rd] & \resource{object file} \\ & \resource{assembly\\listing} & \resource{debugging\\information}}
\seeassembly\seewasm\seeobject\seecode\seedebugging
}

% C++ tools

\providecommand{\cppprep}{
\toolsection{cppprep} is a preprocessor for the \cpp{} programming language.
It preprocesses source code according to the rules of \cpp{} and writes it to the standard output stream.
Only the macro names \texttt{\_\_DATE\_\_}, \texttt{\_\_FILE\_\_}, \texttt{\_\_LINE\_\_}, and \texttt{\_\_TIME\_\_} are predefined.
\flowgraph{\resource{\cpp{} or other\\source code} \ar[r] & \toolbox{cppprep} \ar[r] & \resource{preprocessed\\source code} \\ & \variable{ECSINCLUDE} \ar[u]}
\seecpp
}

\providecommand{\cppprint}{
\toolsection{cppprint} is a pretty printer for the \cpp{} programming language.
It reformats the source code of \cpp{} programs and writes it to the standard output stream.
\flowgraph{\resource{\cpp{}\\source code} \ar[r] & \toolbox{cppprint} \ar[r] & \resource{reformatted\\source code} \\ & \variable{ECSINCLUDE} \ar[u]}
\seecpp
}

\providecommand{\cppcheck}{
\toolsection{cppcheck} is a syntactic and semantic checker for the \cpp{} programming language.
It just performs syntactic and semantic checks on \cpp{} programs and writes its diagnostic messages to the standard error stream.
\flowgraph{\resource{\cpp{}\\source code} \ar[r] & \toolbox{cppcheck} \ar[r] & \resource{diagnostic\\messages} \\ & \variable{ECSINCLUDE} \ar[u]}
\seecpp
}

\providecommand{\cppdump}{
\toolsection{cppdump} is a serializer for the \cpp{} programming language.
It dumps the complete internal representation of programs written in \cpp{} into an XML document.
\debuggingtool
\flowgraph{\resource{\cpp{}\\source code} \ar[r] & \toolbox{cppdump} \ar[r] & \resource{internal\\representation} \\ & \variable{ECSINCLUDE} \ar[u]}
\seecpp
}

\providecommand{\cpprun}{
\toolsection{cpprun} is an interpreter for the \cpp{} programming language.
It processes and executes programs written in \cpp{}.
The macro \texttt{\_\_run\_\_} is predefined in order to enable programmers to identify this tool while interpreting.
\flowgraph{\resource{\cpp{}\\source code} \ar[r] & \toolbox{cpprun} \ar@/u/[r] & \resource{input/\\output} \ar@/d/[l] \\ & \variable{ECSINCLUDE} \ar[u]}
\seecpp
}

\providecommand{\cppdoc}{
\toolsection{cppdoc} is a generic documentation generator for the \cpp{} programming language.
It processes several \cpp{} source files and assembles all information therein into a generic documentation.
\debuggingtool
\flowgraph{\resource{\cpp{}\\source code} \ar[r] & \toolbox{cppdoc} \ar[r] & \resource{generic\\documentation} \\ & \variable{ECSINCLUDE} \ar[u]}
\seecpp\seedocumentation
}

\providecommand{\cpphtml}{
\toolsection{cpphtml} is an HTML documentation generator for the \cpp{} programming language.
It processes several \cpp{} source files and assembles all information therein into an HTML document.
\flowgraph{\resource{\cpp{}\\source code} \ar[r] & \toolbox{cpphtml} \ar[r] & \resource{HTML\\document} \\ & \variable{ECSINCLUDE} \ar[u]}
\seecpp\seedocumentation
}

\providecommand{\cpplatex}{
\toolsection{cpplatex} is a Latex documentation generator for the \cpp{} programming language.
It processes several \cpp{} source files and assembles all information therein into a Latex document.
\flowgraph{\resource{\cpp{}\\source code} \ar[r] & \toolbox{cpplatex} \ar[r] & \resource{Latex\\document} \\ & \variable{ECSINCLUDE} \ar[u]}
\seecpp\seedocumentation
}

\providecommand{\cppcode}{
\toolsection{cppcode} is an intermediate code generator for the \cpp{} programming language.
It generates intermediate code from programs written in \cpp{} and stores it in corresponding assembly files.
The macro \texttt{\_\_code\_\_} is predefined in order to enable programmers to identify this tool while generating intermediate code.
Programs generated with this tool require additional runtime support that is stored in the \file{cpp\-code\-run} library file.
\debuggingtool
\flowgraph{\resource{\cpp{}\\source code} \ar[r] & \toolbox{cppcode} \ar[r] & \resource{intermediate\\code} \\ & \variable{ECSINCLUDE} \ar[u]}
\seecpp\seeassembly\seecode
}

\providecommand{\cppamda}{
\toolsection{cppamd16} is a compiler for the \cpp{} programming language targeting the AMD64 hardware architecture.
It generates machine code for AMD64 processors from programs written in \cpp{} and stores it in corresponding object files.
The compiler generates machine code for the 16-bit operating mode defined by the AMD64 architecture.
For debugging purposes, it also creates a debugging information file as well as an assembly file containing a listing of the generated machine code.
The macro \texttt{\_\_amd16\_\_} is predefined in order to enable programmers to identify this tool and its target architecture while compiling.
Programs generated with this compiler require additional runtime support that is stored in the \file{cpp\-amd16\-run} library file.
\flowgraph{\resource{\cpp{}\\source code} \ar[r] & \toolbox{cppamd16} \ar[r] \ar[d] \ar[rd] & \resource{object file} \\ \variable{ECSINCLUDE} \ar[ru] & \resource{debugging\\information} & \resource{assembly\\listing}}
\seecpp\seeassembly\seeamd\seeobject\seedebugging
}

\providecommand{\cppamdb}{
\toolsection{cppamd32} is a compiler for the \cpp{} programming language targeting the AMD64 hardware architecture.
It generates machine code for AMD64 processors from programs written in \cpp{} and stores it in corresponding object files.
The compiler generates machine code for the 32-bit operating mode defined by the AMD64 architecture.
For debugging purposes, it also creates a debugging information file as well as an assembly file containing a listing of the generated machine code.
The macro \texttt{\_\_amd32\_\_} is predefined in order to enable programmers to identify this tool and its target architecture while compiling.
Programs generated with this compiler require additional runtime support that is stored in the \file{cpp\-amd32\-run} library file.
\flowgraph{\resource{\cpp{}\\source code} \ar[r] & \toolbox{cppamd32} \ar[r] \ar[d] \ar[rd] & \resource{object file} \\ \variable{ECSINCLUDE} \ar[ru] & \resource{debugging\\information} & \resource{assembly\\listing}}
\seecpp\seeassembly\seeamd\seeobject\seedebugging
}

\providecommand{\cppamdc}{
\toolsection{cppamd64} is a compiler for the \cpp{} programming language targeting the AMD64 hardware architecture.
It generates machine code for AMD64 processors from programs written in \cpp{} and stores it in corresponding object files.
The compiler generates machine code for the 64-bit operating mode defined by the AMD64 architecture.
For debugging purposes, it also creates a debugging information file as well as an assembly file containing a listing of the generated machine code.
The macro \texttt{\_\_amd64\_\_} is predefined in order to enable programmers to identify this tool and its target architecture while compiling.
Programs generated with this compiler require additional runtime support that is stored in the \file{cpp\-amd64\-run} library file.
\flowgraph{\resource{\cpp{}\\source code} \ar[r] & \toolbox{cppamd64} \ar[r] \ar[d] \ar[rd] & \resource{object file} \\ \variable{ECSINCLUDE} \ar[ru] & \resource{debugging\\information} & \resource{assembly\\listing}}
\seecpp\seeassembly\seeamd\seeobject\seedebugging
}

\providecommand{\cpparma}{
\toolsection{cpparma32} is a compiler for the \cpp{} programming language targeting the ARM hardware architecture.
It generates machine code for ARM processors executing A32 instructions from programs written in \cpp{} and stores it in corresponding object files.
For debugging purposes, it also creates a debugging information file as well as an assembly file containing a listing of the generated machine code.
The macro \texttt{\_\_arma32\_\_} is predefined in order to enable programmers to identify this tool and its target architecture while compiling.
Programs generated with this compiler require additional runtime support that is stored in the \file{cpp\-arma32\-run} library file.
\flowgraph{\resource{\cpp{}\\source code} \ar[r] & \toolbox{cpparma32} \ar[r] \ar[d] \ar[rd] & \resource{object file} \\ \variable{ECSINCLUDE} \ar[ru] & \resource{debugging\\information} & \resource{assembly\\listing}}
\seecpp\seeassembly\seearm\seeobject\seedebugging
}

\providecommand{\cpparmb}{
\toolsection{cpparma64} is a compiler for the \cpp{} programming language targeting the ARM hardware architecture.
It generates machine code for ARM processors executing A64 instructions from programs written in \cpp{} and stores it in corresponding object files.
For debugging purposes, it also creates a debugging information file as well as an assembly file containing a listing of the generated machine code.
The macro \texttt{\_\_arma64\_\_} is predefined in order to enable programmers to identify this tool and its target architecture while compiling.
Programs generated with this compiler require additional runtime support that is stored in the \file{cpp\-arma64\-run} library file.
\flowgraph{\resource{\cpp{}\\source code} \ar[r] & \toolbox{cpparma64} \ar[r] \ar[d] \ar[rd] & \resource{object file} \\ \variable{ECSINCLUDE} \ar[ru] & \resource{debugging\\information} & \resource{assembly\\listing}}
\seecpp\seeassembly\seearm\seeobject\seedebugging
}

\providecommand{\cpparmc}{
\toolsection{cpparmt32} is a compiler for the \cpp{} programming language targeting the ARM hardware architecture.
It generates machine code for ARM processors without floating-point extension executing T32 instructions from programs written in \cpp{} and stores it in corresponding object files.
For debugging purposes, it also creates a debugging information file as well as an assembly file containing a listing of the generated machine code.
The macro \texttt{\_\_armt32\_\_} is predefined in order to enable programmers to identify this tool and its target architecture while compiling.
Programs generated with this compiler require additional runtime support that is stored in the \file{cpp\-armt32\-run} library file.
\flowgraph{\resource{\cpp{}\\source code} \ar[r] & \toolbox{cpparmt32} \ar[r] \ar[d] \ar[rd] & \resource{object file} \\ \variable{ECSINCLUDE} \ar[ru] & \resource{debugging\\information} & \resource{assembly\\listing}}
\seecpp\seeassembly\seearm\seeobject\seedebugging
}

\providecommand{\cpparmcfpe}{
\toolsection{cpparmt32fpe} is a compiler for the \cpp{} programming language targeting the ARM hardware architecture.
It generates machine code for ARM processors with floating-point extension executing T32 instructions from programs written in \cpp{} and stores it in corresponding object files.
For debugging purposes, it also creates a debugging information file as well as an assembly file containing a listing of the generated machine code.
The macro \texttt{\_\_armt32fpe\_\_} is predefined in order to enable programmers to identify this tool and its target architecture while compiling.
Programs generated with this compiler require additional runtime support that is stored in the \file{cpp\-armt32\-fpe\-run} library file.
\flowgraph{\resource{\cpp{}\\source code} \ar[r] & \toolbox{cpparmt32fpe} \ar[r] \ar[d] \ar[rd] & \resource{object file} \\ \variable{ECSINCLUDE} \ar[ru] & \resource{debugging\\information} & \resource{assembly\\listing}}
\seecpp\seeassembly\seearm\seeobject\seedebugging
}

\providecommand{\cppavr}{
\toolsection{cppavr} is a compiler for the \cpp{} programming language targeting the AVR hardware architecture.
It generates machine code for AVR processors from programs written in \cpp{} and stores it in corresponding object files.
For debugging purposes, it also creates a debugging information file as well as an assembly file containing a listing of the generated machine code.
The macro \texttt{\_\_avr\_\_} is predefined in order to enable programmers to identify this tool and its target architecture while compiling.
Programs generated with this compiler require additional runtime support that is stored in the \file{cpp\-avr\-run} library file.
\flowgraph{\resource{\cpp{}\\source code} \ar[r] & \toolbox{cppavr} \ar[r] \ar[d] \ar[rd] & \resource{object file} \\ \variable{ECSINCLUDE} \ar[ru] & \resource{debugging\\information} & \resource{assembly\\listing}}
\seecpp\seeassembly\seeavr\seeobject\seedebugging
}

\providecommand{\cppavrtt}{
\toolsection{cppavr32} is a compiler for the \cpp{} programming language targeting the AVR32 hardware architecture.
It generates machine code for AVR32 processors from programs written in \cpp{} and stores it in corresponding object files.
For debugging purposes, it also creates a debugging information file as well as an assembly file containing a listing of the generated machine code.
The macro \texttt{\_\_avr32\_\_} is predefined in order to enable programmers to identify this tool and its target architecture while compiling.
Programs generated with this compiler require additional runtime support that is stored in the \file{cpp\-avr32\-run} library file.
\flowgraph{\resource{\cpp{}\\source code} \ar[r] & \toolbox{cppavr32} \ar[r] \ar[d] \ar[rd] & \resource{object file} \\ \variable{ECSINCLUDE} \ar[ru] & \resource{debugging\\information} & \resource{assembly\\listing}}
\seecpp\seeassembly\seeavrtt\seeobject\seedebugging
}

\providecommand{\cppmabk}{
\toolsection{cppm68k} is a compiler for the \cpp{} programming language targeting the M68000 hardware architecture.
It generates machine code for M68000 processors from programs written in \cpp{} and stores it in corresponding object files.
For debugging purposes, it also creates a debugging information file as well as an assembly file containing a listing of the generated machine code.
The macro \texttt{\_\_m68k\_\_} is predefined in order to enable programmers to identify this tool and its target architecture while compiling.
Programs generated with this compiler require additional runtime support that is stored in the \file{cpp\-m68k\-run} library file.
\flowgraph{\resource{\cpp{}\\source code} \ar[r] & \toolbox{cppm68k} \ar[r] \ar[d] \ar[rd] & \resource{object file} \\ \variable{ECSINCLUDE} \ar[ru] & \resource{debugging\\information} & \resource{assembly\\listing}}
\seecpp\seeassembly\seemabk\seeobject\seedebugging
}

\providecommand{\cppmibl}{
\toolsection{cppmibl} is a compiler for the \cpp{} programming language targeting the MicroBlaze hardware architecture.
It generates machine code for MicroBlaze processors from programs written in \cpp{} and stores it in corresponding object files.
For debugging purposes, it also creates a debugging information file as well as an assembly file containing a listing of the generated machine code.
The macro \texttt{\_\_mibl\_\_} is predefined in order to enable programmers to identify this tool and its target architecture while compiling.
Programs generated with this compiler require additional runtime support that is stored in the \file{cpp\-mibl\-run} library file.
\flowgraph{\resource{\cpp{}\\source code} \ar[r] & \toolbox{cppmibl} \ar[r] \ar[d] \ar[rd] & \resource{object file} \\ \variable{ECSINCLUDE} \ar[ru] & \resource{debugging\\information} & \resource{assembly\\listing}}
\seecpp\seeassembly\seemibl\seeobject\seedebugging
}

\providecommand{\cppmipsa}{
\toolsection{cppmips32} is a compiler for the \cpp{} programming language targeting the MIPS32 hardware architecture.
It generates machine code for MIPS32 processors from programs written in \cpp{} and stores it in corresponding object files.
For debugging purposes, it also creates a debugging information file as well as an assembly file containing a listing of the generated machine code.
The macro \texttt{\_\_mips32\_\_} is predefined in order to enable programmers to identify this tool and its target architecture while compiling.
Programs generated with this compiler require additional runtime support that is stored in the \file{cpp\-mips32\-run} library file.
\flowgraph{\resource{\cpp{}\\source code} \ar[r] & \toolbox{cppmips32} \ar[r] \ar[d] \ar[rd] & \resource{object file} \\ \variable{ECSINCLUDE} \ar[ru] & \resource{debugging\\information} & \resource{assembly\\listing}}
\seecpp\seeassembly\seemips\seeobject\seedebugging
}

\providecommand{\cppmipsb}{
\toolsection{cppmips64} is a compiler for the \cpp{} programming language targeting the MIPS64 hardware architecture.
It generates machine code for MIPS64 processors from programs written in \cpp{} and stores it in corresponding object files.
For debugging purposes, it also creates a debugging information file as well as an assembly file containing a listing of the generated machine code.
The macro \texttt{\_\_mips64\_\_} is predefined in order to enable programmers to identify this tool and its target architecture while compiling.
Programs generated with this compiler require additional runtime support that is stored in the \file{cpp\-mips64\-run} library file.
\flowgraph{\resource{\cpp{}\\source code} \ar[r] & \toolbox{cppmips64} \ar[r] \ar[d] \ar[rd] & \resource{object file} \\ \variable{ECSINCLUDE} \ar[ru] & \resource{debugging\\information} & \resource{assembly\\listing}}
\seecpp\seeassembly\seemips\seeobject\seedebugging
}

\providecommand{\cppmmix}{
\toolsection{cppmmix} is a compiler for the \cpp{} programming language targeting the MMIX hardware architecture.
It generates machine code for MMIX processors from programs written in \cpp{} and stores it in corresponding object files.
For debugging purposes, it also creates a debugging information file as well as an assembly file containing a listing of the generated machine code.
The macro \texttt{\_\_mmix\_\_} is predefined in order to enable programmers to identify this tool and its target architecture while compiling.
Programs generated with this compiler require additional runtime support that is stored in the \file{cpp\-mmix\-run} library file.
\flowgraph{\resource{\cpp{}\\source code} \ar[r] & \toolbox{cppmmix} \ar[r] \ar[d] \ar[rd] & \resource{object file} \\ \variable{ECSINCLUDE} \ar[ru] & \resource{debugging\\information} & \resource{assembly\\listing}}
\seecpp\seeassembly\seemmix\seeobject\seedebugging
}

\providecommand{\cpporok}{
\toolsection{cppor1k} is a compiler for the \cpp{} programming language targeting the OpenRISC 1000 hardware architecture.
It generates machine code for OpenRISC 1000 processors from programs written in \cpp{} and stores it in corresponding object files.
For debugging purposes, it also creates a debugging information file as well as an assembly file containing a listing of the generated machine code.
The macro \texttt{\_\_or1k\_\_} is predefined in order to enable programmers to identify this tool and its target architecture while compiling.
Programs generated with this compiler require additional runtime support that is stored in the \file{cpp\-or1k\-run} library file.
\flowgraph{\resource{\cpp{}\\source code} \ar[r] & \toolbox{cppor1k} \ar[r] \ar[d] \ar[rd] & \resource{object file} \\ \variable{ECSINCLUDE} \ar[ru] & \resource{debugging\\information} & \resource{assembly\\listing}}
\seecpp\seeassembly\seeorok\seeobject\seedebugging
}

\providecommand{\cppppca}{
\toolsection{cppppc32} is a compiler for the \cpp{} programming language targeting the PowerPC hardware architecture.
It generates machine code for PowerPC processors from programs written in \cpp{} and stores it in corresponding object files.
The compiler generates machine code for the 32-bit operating mode defined by the PowerPC architecture.
For debugging purposes, it also creates a debugging information file as well as an assembly file containing a listing of the generated machine code.
The macro \texttt{\_\_ppc32\_\_} is predefined in order to enable programmers to identify this tool and its target architecture while compiling.
Programs generated with this compiler require additional runtime support that is stored in the \file{cpp\-ppc32\-run} library file.
\flowgraph{\resource{\cpp{}\\source code} \ar[r] & \toolbox{cppppc32} \ar[r] \ar[d] \ar[rd] & \resource{object file} \\ \variable{ECSINCLUDE} \ar[ru] & \resource{debugging\\information} & \resource{assembly\\listing}}
\seecpp\seeassembly\seeppc\seeobject\seedebugging
}

\providecommand{\cppppcb}{
\toolsection{cppppc64} is a compiler for the \cpp{} programming language targeting the PowerPC hardware architecture.
It generates machine code for PowerPC processors from programs written in \cpp{} and stores it in corresponding object files.
The compiler generates machine code for the 64-bit operating mode defined by the PowerPC architecture.
For debugging purposes, it also creates a debugging information file as well as an assembly file containing a listing of the generated machine code.
The macro \texttt{\_\_ppc64\_\_} is predefined in order to enable programmers to identify this tool and its target architecture while compiling.
Programs generated with this compiler require additional runtime support that is stored in the \file{cpp\-ppc64\-run} library file.
\flowgraph{\resource{\cpp{}\\source code} \ar[r] & \toolbox{cppppc64} \ar[r] \ar[d] \ar[rd] & \resource{object file} \\ \variable{ECSINCLUDE} \ar[ru] & \resource{debugging\\information} & \resource{assembly\\listing}}
\seecpp\seeassembly\seeppc\seeobject\seedebugging
}

\providecommand{\cpprisc}{
\toolsection{cpprisc} is a compiler for the \cpp{} programming language targeting the RISC hardware architecture.
It generates machine code for RISC processors from programs written in \cpp{} and stores it in corresponding object files.
For debugging purposes, it also creates a debugging information file as well as an assembly file containing a listing of the generated machine code.
The macro \texttt{\_\_risc\_\_} is predefined in order to enable programmers to identify this tool and its target architecture while compiling.
Programs generated with this compiler require additional runtime support that is stored in the \file{cpp\-risc\-run} library file.
\flowgraph{\resource{\cpp{}\\source code} \ar[r] & \toolbox{cpprisc} \ar[r] \ar[d] \ar[rd] & \resource{object file} \\ \variable{ECSINCLUDE} \ar[ru] & \resource{debugging\\information} & \resource{assembly\\listing}}
\seecpp\seeassembly\seerisc\seeobject\seedebugging
}

\providecommand{\cppwasm}{
\toolsection{cppwasm} is a compiler for the \cpp{} programming language targeting the WebAssembly architecture.
It generates machine code for WebAssembly targets from programs written in \cpp{} and stores it in corresponding object files.
For debugging purposes, it also creates a debugging information file as well as an assembly file containing a listing of the generated machine code.
The macro \texttt{\_\_wasm\_\_} is predefined in order to enable programmers to identify this tool and its target architecture while compiling.
Programs generated with this compiler require additional runtime support that is stored in the \file{cpp\-wasm\-run} library file.
\flowgraph{\resource{\cpp{}\\source code} \ar[r] & \toolbox{cppwasm} \ar[r] \ar[d] \ar[rd] & \resource{object file} \\ \variable{ECSINCLUDE} \ar[ru] & \resource{debugging\\information} & \resource{assembly\\listing}}
\seecpp\seeassembly\seewasm\seeobject\seedebugging
}

% FALSE tools

\providecommand{\falprint}{
\toolsection{falprint} is a pretty printer for the FALSE programming language.
It reformats the source code of FALSE programs and writes it to the standard output stream.
\flowgraph{\resource{FALSE\\source code} \ar[r] & \toolbox{falprint} \ar[r] & \resource{reformatted\\source code}}
\seefalse
}

\providecommand{\falcheck}{
\toolsection{falcheck} is a syntactic and semantic checker for the FALSE programming language.
It just performs syntactic and semantic checks on FALSE programs and writes its diagnostic messages to the standard error stream.
\flowgraph{\resource{FALSE\\source code} \ar[r] & \toolbox{falcheck} \ar[r] & \resource{diagnostic\\messages}}
\seefalse
}

\providecommand{\faldump}{
\toolsection{faldump} is a serializer for the FALSE programming language.
It dumps the complete internal representation of programs written in FALSE into an XML document.
\debuggingtool
\flowgraph{\resource{FALSE\\source code} \ar[r] & \toolbox{faldump} \ar[r] & \resource{internal\\representation}}
\seefalse
}

\providecommand{\falrun}{
\toolsection{falrun} is an interpreter for the FALSE programming language.
It processes and executes programs written in FALSE\@.
\flowgraph{\resource{FALSE\\source code} \ar[r] & \toolbox{falrun} \ar@/u/[r] & \resource{input/\\output} \ar@/d/[l]}
\seefalse
}

\providecommand{\falcpp}{
\toolsection{falcpp} is a transpiler for the FALSE programming language.
It translates programs written in FALSE into \cpp{} programs and stores them in corresponding source files.
\flowgraph{\resource{FALSE\\source code} \ar[r] & \toolbox{falcpp} \ar[r] & \resource{\cpp{}\\source file}}
\seefalse\seecpp
}

\providecommand{\falcode}{
\toolsection{falcode} is an intermediate code generator for the FALSE programming language.
It generates intermediate code from programs written in FALSE and stores it in corresponding assembly files.
\debuggingtool
\flowgraph{\resource{FALSE\\source code} \ar[r] & \toolbox{falcode} \ar[r] & \resource{intermediate\\code}}
\seefalse\seeassembly\seecode
}

\providecommand{\falamda}{
\toolsection{falamd16} is a compiler for the FALSE programming language targeting the AMD64 hardware architecture.
It generates machine code for AMD64 processors from programs written in FALSE and stores it in corresponding object files.
The compiler generates machine code for the 16-bit operating mode defined by the AMD64 architecture.
\flowgraph{\resource{FALSE\\source code} \ar[r] & \toolbox{falamd16} \ar[r] & \resource{object file}}
\seefalse\seeamd\seeobject
}

\providecommand{\falamdb}{
\toolsection{falamd32} is a compiler for the FALSE programming language targeting the AMD64 hardware architecture.
It generates machine code for AMD64 processors from programs written in FALSE and stores it in corresponding object files.
The compiler generates machine code for the 32-bit operating mode defined by the AMD64 architecture.
\flowgraph{\resource{FALSE\\source code} \ar[r] & \toolbox{falamd32} \ar[r] & \resource{object file}}
\seefalse\seeamd\seeobject
}

\providecommand{\falamdc}{
\toolsection{falamd64} is a compiler for the FALSE programming language targeting the AMD64 hardware architecture.
It generates machine code for AMD64 processors from programs written in FALSE and stores it in corresponding object files.
The compiler generates machine code for the 64-bit operating mode defined by the AMD64 architecture.
\flowgraph{\resource{FALSE\\source code} \ar[r] & \toolbox{falamd64} \ar[r] & \resource{object file}}
\seefalse\seeamd\seeobject
}

\providecommand{\falarma}{
\toolsection{falarma32} is a compiler for the FALSE programming language targeting the ARM hardware architecture.
It generates machine code for ARM processors executing A32 instructions from programs written in FALSE and stores it in corresponding object files.
\flowgraph{\resource{FALSE\\source code} \ar[r] & \toolbox{falarma32} \ar[r] & \resource{object file}}
\seefalse\seearm\seeobject
}

\providecommand{\falarmb}{
\toolsection{falarma64} is a compiler for the FALSE programming language targeting the ARM hardware architecture.
It generates machine code for ARM processors executing A64 instructions from programs written in FALSE and stores it in corresponding object files.
\flowgraph{\resource{FALSE\\source code} \ar[r] & \toolbox{falarma64} \ar[r] & \resource{object file}}
\seefalse\seearm\seeobject
}

\providecommand{\falarmc}{
\toolsection{falarmt32} is a compiler for the FALSE programming language targeting the ARM hardware architecture.
It generates machine code for ARM processors without floating-point extension executing T32 instructions from programs written in FALSE and stores it in corresponding object files.
\flowgraph{\resource{FALSE\\source code} \ar[r] & \toolbox{falarmt32} \ar[r] & \resource{object file}}
\seefalse\seearm\seeobject
}

\providecommand{\falarmcfpe}{
\toolsection{falarmt32fpe} is a compiler for the FALSE programming language targeting the ARM hardware architecture.
It generates machine code for ARM processors with floating-point extension executing T32 instructions from programs written in FALSE and stores it in corresponding object files.
\flowgraph{\resource{FALSE\\source code} \ar[r] & \toolbox{falarmt32fpe} \ar[r] & \resource{object file}}
\seefalse\seearm\seeobject
}

\providecommand{\falavr}{
\toolsection{falavr} is a compiler for the FALSE programming language targeting the AVR hardware architecture.
It generates machine code for AVR processors from programs written in FALSE and stores it in corresponding object files.
\flowgraph{\resource{FALSE\\source code} \ar[r] & \toolbox{falavr} \ar[r] & \resource{object file}}
\seefalse\seeavr\seeobject
}

\providecommand{\falavrtt}{
\toolsection{falavr32} is a compiler for the FALSE programming language targeting the AVR32 hardware architecture.
It generates machine code for AVR32 processors from programs written in FALSE and stores it in corresponding object files.
\flowgraph{\resource{FALSE\\source code} \ar[r] & \toolbox{falavr32} \ar[r] & \resource{object file}}
\seefalse\seeavrtt\seeobject
}

\providecommand{\falmabk}{
\toolsection{falm68k} is a compiler for the FALSE programming language targeting the M68000 hardware architecture.
It generates machine code for M68000 processors from programs written in FALSE and stores it in corresponding object files.
\flowgraph{\resource{FALSE\\source code} \ar[r] & \toolbox{falm68k} \ar[r] & \resource{object file}}
\seefalse\seemabk\seeobject
}

\providecommand{\falmibl}{
\toolsection{falmibl} is a compiler for the FALSE programming language targeting the MicroBlaze hardware architecture.
It generates machine code for MicroBlaze processors from programs written in FALSE and stores it in corresponding object files.
\flowgraph{\resource{FALSE\\source code} \ar[r] & \toolbox{falmibl} \ar[r] & \resource{object file}}
\seefalse\seemibl\seeobject
}

\providecommand{\falmipsa}{
\toolsection{falmips32} is a compiler for the FALSE programming language targeting the MIPS32 hardware architecture.
It generates machine code for MIPS32 processors from programs written in FALSE and stores it in corresponding object files.
\flowgraph{\resource{FALSE\\source code} \ar[r] & \toolbox{falmips32} \ar[r] & \resource{object file}}
\seefalse\seemips\seeobject
}

\providecommand{\falmipsb}{
\toolsection{falmips64} is a compiler for the FALSE programming language targeting the MIPS64 hardware architecture.
It generates machine code for MIPS64 processors from programs written in FALSE and stores it in corresponding object files.
\flowgraph{\resource{FALSE\\source code} \ar[r] & \toolbox{falmips64} \ar[r] & \resource{object file}}
\seefalse\seemips\seeobject
}

\providecommand{\falmmix}{
\toolsection{falmmix} is a compiler for the FALSE programming language targeting the MMIX hardware architecture.
It generates machine code for MMIX processors from programs written in FALSE and stores it in corresponding object files.
\flowgraph{\resource{FALSE\\source code} \ar[r] & \toolbox{falmmix} \ar[r] & \resource{object file}}
\seefalse\seemmix\seeobject
}

\providecommand{\falorok}{
\toolsection{falor1k} is a compiler for the FALSE programming language targeting the OpenRISC 1000 hardware architecture.
It generates machine code for OpenRISC 1000 processors from programs written in FALSE and stores it in corresponding object files.
\flowgraph{\resource{FALSE\\source code} \ar[r] & \toolbox{falor1k} \ar[r] & \resource{object file}}
\seefalse\seeorok\seeobject
}

\providecommand{\falppca}{
\toolsection{falppc32} is a compiler for the FALSE programming language targeting the PowerPC hardware architecture.
It generates machine code for PowerPC processors from programs written in FALSE and stores it in corresponding object files.
The compiler generates machine code for the 32-bit operating mode defined by the PowerPC architecture.
\flowgraph{\resource{FALSE\\source code} \ar[r] & \toolbox{falppc32} \ar[r] & \resource{object file}}
\seefalse\seeppc\seeobject
}

\providecommand{\falppcb}{
\toolsection{falppc64} is a compiler for the FALSE programming language targeting the PowerPC hardware architecture.
It generates machine code for PowerPC processors from programs written in FALSE and stores it in corresponding object files.
The compiler generates machine code for the 64-bit operating mode defined by the PowerPC architecture.
\flowgraph{\resource{FALSE\\source code} \ar[r] & \toolbox{falppc64} \ar[r] & \resource{object file}}
\seefalse\seeppc\seeobject
}

\providecommand{\falrisc}{
\toolsection{falrisc} is a compiler for the FALSE programming language targeting the RISC hardware architecture.
It generates machine code for RISC processors from programs written in FALSE and stores it in corresponding object files.
\flowgraph{\resource{FALSE\\source code} \ar[r] & \toolbox{falrisc} \ar[r] & \resource{object file}}
\seefalse\seerisc\seeobject
}

\providecommand{\falwasm}{
\toolsection{falwasm} is a compiler for the FALSE programming language targeting the WebAssembly architecture.
It generates machine code for WebAssembly targets from programs written in FALSE and stores it in corresponding object files.
\flowgraph{\resource{FALSE\\source code} \ar[r] & \toolbox{falwasm} \ar[r] & \resource{object file}}
\seefalse\seewasm\seeobject
}

% Oberon tools

\providecommand{\obprint}{
\toolsection{obprint} is a pretty printer for the Oberon programming language.
It reformats the source code of Oberon modules and writes it to the standard output stream.
\flowgraph{\resource{Oberon\\source code} \ar[r] & \toolbox{obprint} \ar[r] & \resource{reformatted\\source code}}
\seeoberon
}

\providecommand{\obcheck}{
\toolsection{obcheck} is a syntactic and semantic checker for the Oberon programming language.
It just performs syntactic and semantic checks on Oberon modules and writes its diagnostic messages to the standard error stream.
In addition, it stores the interface of each module in a symbol file which is required when other modules import the module.
\flowgraph{\resource{Oberon\\source code} \ar[r] & \toolbox{obcheck} \ar[r] \ar@/l/[d] & \resource{diagnostic\\messages} \\ \variable{ECSIMPORT} \ar[ru] & \resource{symbol\\files} \ar@/r/[u]}
\seeoberon
}

\providecommand{\obdump}{
\toolsection{obdump} is a serializer for the Oberon programming language.
It dumps the complete internal representation of modules written in Oberon into an XML document.
\debuggingtool
\flowgraph{\resource{Oberon\\source code} \ar[r] & \toolbox{obdump} \ar[r] \ar@/l/[d] & \resource{internal\\representation} \\ \variable{ECSIMPORT} \ar[ru] & \resource{symbol\\files} \ar@/r/[u]}
\seeoberon
}

\providecommand{\obrun}{
\toolsection{obrun} is an interpreter for the Oberon programming language.
It processes and executes modules written in Oberon.
This tool does neither generate nor process symbol files while interpreting modules.
If a module is imported by another one, its filename has to be named before the other one in the list of command-line arguments.
\flowgraph{\resource{Oberon\\source code} \ar[r] & \toolbox{obrun} \ar@/u/[r] & \resource{input/\\output} \ar@/d/[l]}
\seeoberon
}

\providecommand{\obcpp}{
\toolsection{obcpp} is a transpiler for the Oberon programming language.
It translates programs written in Oberon into \cpp{} programs and stores them in corresponding source and header files.
In addition, it stores the interface of each module in a symbol file which is required when other modules import the module.
The same interface is provided by the generated header file which can be used in other parts of the \cpp{} program.
\flowgraph{\resource{Oberon\\source code} \ar[r] & \toolbox{obcpp} \ar[r] \ar@/l/[d] \ar[rd] & \resource{\cpp{}\\source file} \\ \variable{ECSIMPORT} \ar[ru] & \resource{symbol\\files} \ar@/r/[u] & \resource{\cpp{}\\header file}}
\seeoberon\seecpp
}

\providecommand{\obdoc}{
\toolsection{obdoc} is a generic documentation generator for the Oberon programming language.
It processes several Oberon modules and assembles all information therein into a generic documentation.
In addition, it stores the interface of each module in a symbol file which is required when other modules import the module.
\debuggingtool
\flowgraph{\resource{Oberon\\source code} \ar[r] & \toolbox{obdoc} \ar[r] \ar@/l/[d] & \resource{generic\\documentation} \\ \variable{ECSIMPORT} \ar[ru] & \resource{symbol\\files} \ar@/r/[u]}
\seeoberon\seedocumentation
}

\providecommand{\obhtml}{
\toolsection{obhtml} is an HTML documentation generator for the Oberon programming language.
It processes several Oberon modules and assembles all information therein into an HTML document.
In addition, it stores the interface of each module in a symbol file which is required when other modules import the module.
\flowgraph{\resource{Oberon\\source code} \ar[r] & \toolbox{obhtml} \ar[r] \ar@/l/[d] & \resource{HTML\\document} \\ \variable{ECSIMPORT} \ar[ru] & \resource{symbol\\files} \ar@/r/[u]}
\seeoberon\seedocumentation
}

\providecommand{\oblatex}{
\toolsection{oblatex} is a Latex documentation generator for the Oberon programming language.
It processes several Oberon modules and assembles all information therein into a Latex document.
In addition, it stores the interface of each module in a symbol file which is required when other modules import the module.
\flowgraph{\resource{Oberon\\source code} \ar[r] & \toolbox{oblatex} \ar[r] \ar@/l/[d] & \resource{Latex\\document} \\ \variable{ECSIMPORT} \ar[ru] & \resource{symbol\\files} \ar@/r/[u]}
\seeoberon\seedocumentation
}

\providecommand{\obcode}{
\toolsection{obcode} is an intermediate code generator for the Oberon programming language.
It generates intermediate code from modules written in Oberon and stores it in corresponding assembly files.
In addition, it stores the interface of each module in a symbol file which is required when other modules import the module.
Programs generated with this tool require additional runtime support that is stored in the \file{ob\-code\-run} library file.
\debuggingtool
\flowgraph{\resource{Oberon\\source code} \ar[r] & \toolbox{obcode} \ar[r] \ar@/l/[d] & \resource{intermediate\\code} \\ \variable{ECSIMPORT} \ar[ru] & \resource{symbol\\files} \ar@/r/[u]}
\seeoberon\seeassembly\seecode
}

\providecommand{\obamda}{
\toolsection{obamd16} is a compiler for the Oberon programming language targeting the AMD64 hardware architecture.
It generates machine code for AMD64 processors from modules written in Oberon and stores it in corresponding object files.
The compiler generates machine code for the 16-bit operating mode defined by the AMD64 architecture.
For debugging purposes, it also creates a debugging information file as well as an assembly file containing a listing of the generated machine code.
In addition, it stores the interface of each module in a symbol file which is required when other modules import the module.
Programs generated with this compiler require additional runtime support that is stored in the \file{ob\-amd16\-run} library file.
\flowgraph{\resource{Oberon\\source code} \ar[r] & \toolbox{obamd16} \ar[r] \ar@/l/[d] \ar[rd] & \resource{object file} \\ \variable{ECSIMPORT} \ar[ru] & \resource{symbol\\files} \ar@/r/[u] & \resource{debugging\\information}}
\seeoberon\seeassembly\seeamd\seeobject\seedebugging
}

\providecommand{\obamdb}{
\toolsection{obamd32} is a compiler for the Oberon programming language targeting the AMD64 hardware architecture.
It generates machine code for AMD64 processors from modules written in Oberon and stores it in corresponding object files.
The compiler generates machine code for the 32-bit operating mode defined by the AMD64 architecture.
For debugging purposes, it also creates a debugging information file as well as an assembly file containing a listing of the generated machine code.
In addition, it stores the interface of each module in a symbol file which is required when other modules import the module.
Programs generated with this compiler require additional runtime support that is stored in the \file{ob\-amd32\-run} library file.
\flowgraph{\resource{Oberon\\source code} \ar[r] & \toolbox{obamd32} \ar[r] \ar@/l/[d] \ar[rd] & \resource{object file} \\ \variable{ECSIMPORT} \ar[ru] & \resource{symbol\\files} \ar@/r/[u] & \resource{debugging\\information}}
\seeoberon\seeassembly\seeamd\seeobject\seedebugging
}

\providecommand{\obamdc}{
\toolsection{obamd64} is a compiler for the Oberon programming language targeting the AMD64 hardware architecture.
It generates machine code for AMD64 processors from modules written in Oberon and stores it in corresponding object files.
The compiler generates machine code for the 64-bit operating mode defined by the AMD64 architecture.
For debugging purposes, it also creates a debugging information file as well as an assembly file containing a listing of the generated machine code.
In addition, it stores the interface of each module in a symbol file which is required when other modules import the module.
Programs generated with this compiler require additional runtime support that is stored in the \file{ob\-amd64\-run} library file.
\flowgraph{\resource{Oberon\\source code} \ar[r] & \toolbox{obamd64} \ar[r] \ar@/l/[d] \ar[rd] & \resource{object file} \\ \variable{ECSIMPORT} \ar[ru] & \resource{symbol\\files} \ar@/r/[u] & \resource{debugging\\information}}
\seeoberon\seeassembly\seeamd\seeobject\seedebugging
}

\providecommand{\obarma}{
\toolsection{obarma32} is a compiler for the Oberon programming language targeting the ARM hardware architecture.
It generates machine code for ARM processors executing A32 instructions from modules written in Oberon and stores it in corresponding object files.
For debugging purposes, it also creates a debugging information file as well as an assembly file containing a listing of the generated machine code.
In addition, it stores the interface of each module in a symbol file which is required when other modules import the module.
Programs generated with this compiler require additional runtime support that is stored in the \file{ob\-arma32\-run} library file.
\flowgraph{\resource{Oberon\\source code} \ar[r] & \toolbox{obarma32} \ar[r] \ar@/l/[d] \ar[rd] & \resource{object file} \\ \variable{ECSIMPORT} \ar[ru] & \resource{symbol\\files} \ar@/r/[u] & \resource{debugging\\information}}
\seeoberon\seeassembly\seearm\seeobject\seedebugging
}

\providecommand{\obarmb}{
\toolsection{obarma64} is a compiler for the Oberon programming language targeting the ARM hardware architecture.
It generates machine code for ARM processors executing A64 instructions from modules written in Oberon and stores it in corresponding object files.
For debugging purposes, it also creates a debugging information file as well as an assembly file containing a listing of the generated machine code.
In addition, it stores the interface of each module in a symbol file which is required when other modules import the module.
Programs generated with this compiler require additional runtime support that is stored in the \file{ob\-arma64\-run} library file.
\flowgraph{\resource{Oberon\\source code} \ar[r] & \toolbox{obarma64} \ar[r] \ar@/l/[d] \ar[rd] & \resource{object file} \\ \variable{ECSIMPORT} \ar[ru] & \resource{symbol\\files} \ar@/r/[u] & \resource{debugging\\information}}
\seeoberon\seeassembly\seearm\seeobject\seedebugging
}

\providecommand{\obarmc}{
\toolsection{obarmt32} is a compiler for the Oberon programming language targeting the ARM hardware architecture.
It generates machine code for ARM processors without floating-point extension executing T32 instructions from modules written in Oberon and stores it in corresponding object files.
For debugging purposes, it also creates a debugging information file as well as an assembly file containing a listing of the generated machine code.
In addition, it stores the interface of each module in a symbol file which is required when other modules import the module.
Programs generated with this compiler require additional runtime support that is stored in the \file{ob\-armt32\-run} library file.
\flowgraph{\resource{Oberon\\source code} \ar[r] & \toolbox{obarmt32} \ar[r] \ar@/l/[d] \ar[rd] & \resource{object file} \\ \variable{ECSIMPORT} \ar[ru] & \resource{symbol\\files} \ar@/r/[u] & \resource{debugging\\information}}
\seeoberon\seeassembly\seearm\seeobject\seedebugging
}

\providecommand{\obarmcfpe}{
\toolsection{obarmt32fpe} is a compiler for the Oberon programming language targeting the ARM hardware architecture.
It generates machine code for ARM processors with floating-point extension executing T32 instructions from modules written in Oberon and stores it in corresponding object files.
For debugging purposes, it also creates a debugging information file as well as an assembly file containing a listing of the generated machine code.
In addition, it stores the interface of each module in a symbol file which is required when other modules import the module.
Programs generated with this compiler require additional runtime support that is stored in the \file{ob\-armt32\-fpe\-run} library file.
\flowgraph{\resource{Oberon\\source code} \ar[r] & \toolbox{obarmt32fpe} \ar[r] \ar@/l/[d] \ar[rd] & \resource{object file} \\ \variable{ECSIMPORT} \ar[ru] & \resource{symbol\\files} \ar@/r/[u] & \resource{debugging\\information}}
\seeoberon\seeassembly\seearm\seeobject\seedebugging
}

\providecommand{\obavr}{
\toolsection{obavr} is a compiler for the Oberon programming language targeting the AVR hardware architecture.
It generates machine code for AVR processors from modules written in Oberon and stores it in corresponding object files.
For debugging purposes, it also creates a debugging information file as well as an assembly file containing a listing of the generated machine code.
In addition, it stores the interface of each module in a symbol file which is required when other modules import the module.
Programs generated with this compiler require additional runtime support that is stored in the \file{ob\-avr\-run} library file.
\flowgraph{\resource{Oberon\\source code} \ar[r] & \toolbox{obavr} \ar[r] \ar@/l/[d] \ar[rd] & \resource{object file} \\ \variable{ECSIMPORT} \ar[ru] & \resource{symbol\\files} \ar@/r/[u] & \resource{debugging\\information}}
\seeoberon\seeassembly\seeavr\seeobject\seedebugging
}

\providecommand{\obavrtt}{
\toolsection{obavr32} is a compiler for the Oberon programming language targeting the AVR32 hardware architecture.
It generates machine code for AVR32 processors from modules written in Oberon and stores it in corresponding object files.
For debugging purposes, it also creates a debugging information file as well as an assembly file containing a listing of the generated machine code.
In addition, it stores the interface of each module in a symbol file which is required when other modules import the module.
Programs generated with this compiler require additional runtime support that is stored in the \file{ob\-avr32\-run} library file.
\flowgraph{\resource{Oberon\\source code} \ar[r] & \toolbox{obavr32} \ar[r] \ar@/l/[d] \ar[rd] & \resource{object file} \\ \variable{ECSIMPORT} \ar[ru] & \resource{symbol\\files} \ar@/r/[u] & \resource{debugging\\information}}
\seeoberon\seeassembly\seeavrtt\seeobject\seedebugging
}

\providecommand{\obmabk}{
\toolsection{obm68k} is a compiler for the Oberon programming language targeting the M68000 hardware architecture.
It generates machine code for M68000 processors from modules written in Oberon and stores it in corresponding object files.
For debugging purposes, it also creates a debugging information file as well as an assembly file containing a listing of the generated machine code.
In addition, it stores the interface of each module in a symbol file which is required when other modules import the module.
Programs generated with this compiler require additional runtime support that is stored in the \file{ob\-m68k\-run} library file.
\flowgraph{\resource{Oberon\\source code} \ar[r] & \toolbox{obm68k} \ar[r] \ar@/l/[d] \ar[rd] & \resource{object file} \\ \variable{ECSIMPORT} \ar[ru] & \resource{symbol\\files} \ar@/r/[u] & \resource{debugging\\information}}
\seeoberon\seeassembly\seemabk\seeobject\seedebugging
}

\providecommand{\obmibl}{
\toolsection{obmibl} is a compiler for the Oberon programming language targeting the MicroBlaze hardware architecture.
It generates machine code for MicroBlaze processors from modules written in Oberon and stores it in corresponding object files.
For debugging purposes, it also creates a debugging information file as well as an assembly file containing a listing of the generated machine code.
In addition, it stores the interface of each module in a symbol file which is required when other modules import the module.
Programs generated with this compiler require additional runtime support that is stored in the \file{ob\-mibl\-run} library file.
\flowgraph{\resource{Oberon\\source code} \ar[r] & \toolbox{obmibl} \ar[r] \ar@/l/[d] \ar[rd] & \resource{object file} \\ \variable{ECSIMPORT} \ar[ru] & \resource{symbol\\files} \ar@/r/[u] & \resource{debugging\\information}}
\seeoberon\seeassembly\seemibl\seeobject\seedebugging
}

\providecommand{\obmipsa}{
\toolsection{obmips32} is a compiler for the Oberon programming language targeting the MIPS32 hardware architecture.
It generates machine code for MIPS32 processors from modules written in Oberon and stores it in corresponding object files.
For debugging purposes, it also creates a debugging information file as well as an assembly file containing a listing of the generated machine code.
In addition, it stores the interface of each module in a symbol file which is required when other modules import the module.
Programs generated with this compiler require additional runtime support that is stored in the \file{ob\-mips32\-run} library file.
\flowgraph{\resource{Oberon\\source code} \ar[r] & \toolbox{obmips32} \ar[r] \ar@/l/[d] \ar[rd] & \resource{object file} \\ \variable{ECSIMPORT} \ar[ru] & \resource{symbol\\files} \ar@/r/[u] & \resource{debugging\\information}}
\seeoberon\seeassembly\seemips\seeobject\seedebugging
}

\providecommand{\obmipsb}{
\toolsection{obmips64} is a compiler for the Oberon programming language targeting the MIPS64 hardware architecture.
It generates machine code for MIPS64 processors from modules written in Oberon and stores it in corresponding object files.
For debugging purposes, it also creates a debugging information file as well as an assembly file containing a listing of the generated machine code.
In addition, it stores the interface of each module in a symbol file which is required when other modules import the module.
Programs generated with this compiler require additional runtime support that is stored in the \file{ob\-mips64\-run} library file.
\flowgraph{\resource{Oberon\\source code} \ar[r] & \toolbox{obmips64} \ar[r] \ar@/l/[d] \ar[rd] & \resource{object file} \\ \variable{ECSIMPORT} \ar[ru] & \resource{symbol\\files} \ar@/r/[u] & \resource{debugging\\information}}
\seeoberon\seeassembly\seemips\seeobject\seedebugging
}

\providecommand{\obmmix}{
\toolsection{obmmix} is a compiler for the Oberon programming language targeting the MMIX hardware architecture.
It generates machine code for MMIX processors from modules written in Oberon and stores it in corresponding object files.
For debugging purposes, it also creates a debugging information file as well as an assembly file containing a listing of the generated machine code.
In addition, it stores the interface of each module in a symbol file which is required when other modules import the module.
Programs generated with this compiler require additional runtime support that is stored in the \file{ob\-mmix\-run} library file.
\flowgraph{\resource{Oberon\\source code} \ar[r] & \toolbox{obmmix} \ar[r] \ar@/l/[d] \ar[rd] & \resource{object file} \\ \variable{ECSIMPORT} \ar[ru] & \resource{symbol\\files} \ar@/r/[u] & \resource{debugging\\information}}
\seeoberon\seeassembly\seemmix\seeobject\seedebugging
}

\providecommand{\oborok}{
\toolsection{obor1k} is a compiler for the Oberon programming language targeting the OpenRISC 1000 hardware architecture.
It generates machine code for OpenRISC 1000 processors from modules written in Oberon and stores it in corresponding object files.
For debugging purposes, it also creates a debugging information file as well as an assembly file containing a listing of the generated machine code.
In addition, it stores the interface of each module in a symbol file which is required when other modules import the module.
Programs generated with this compiler require additional runtime support that is stored in the \file{ob\-or1k\-run} library file.
\flowgraph{\resource{Oberon\\source code} \ar[r] & \toolbox{obor1k} \ar[r] \ar@/l/[d] \ar[rd] & \resource{object file} \\ \variable{ECSIMPORT} \ar[ru] & \resource{symbol\\files} \ar@/r/[u] & \resource{debugging\\information}}
\seeoberon\seeassembly\seeorok\seeobject\seedebugging
}

\providecommand{\obppca}{
\toolsection{obppc32} is a compiler for the Oberon programming language targeting the PowerPC hardware architecture.
It generates machine code for PowerPC processors from modules written in Oberon and stores it in corresponding object files.
The compiler generates machine code for the 32-bit operating mode defined by the PowerPC architecture.
For debugging purposes, it also creates a debugging information file as well as an assembly file containing a listing of the generated machine code.
In addition, it stores the interface of each module in a symbol file which is required when other modules import the module.
Programs generated with this compiler require additional runtime support that is stored in the \file{ob\-ppc32\-run} library file.
\flowgraph{\resource{Oberon\\source code} \ar[r] & \toolbox{obppc32} \ar[r] \ar@/l/[d] \ar[rd] & \resource{object file} \\ \variable{ECSIMPORT} \ar[ru] & \resource{symbol\\files} \ar@/r/[u] & \resource{debugging\\information}}
\seeoberon\seeassembly\seeppc\seeobject\seedebugging
}

\providecommand{\obppcb}{
\toolsection{obppc64} is a compiler for the Oberon programming language targeting the PowerPC hardware architecture.
It generates machine code for PowerPC processors from modules written in Oberon and stores it in corresponding object files.
The compiler generates machine code for the 64-bit operating mode defined by the PowerPC architecture.
For debugging purposes, it also creates a debugging information file as well as an assembly file containing a listing of the generated machine code.
In addition, it stores the interface of each module in a symbol file which is required when other modules import the module.
Programs generated with this compiler require additional runtime support that is stored in the \file{ob\-ppc64\-run} library file.
\flowgraph{\resource{Oberon\\source code} \ar[r] & \toolbox{obppc64} \ar[r] \ar@/l/[d] \ar[rd] & \resource{object file} \\ \variable{ECSIMPORT} \ar[ru] & \resource{symbol\\files} \ar@/r/[u] & \resource{debugging\\information}}
\seeoberon\seeassembly\seeppc\seeobject\seedebugging
}

\providecommand{\obrisc}{
\toolsection{obrisc} is a compiler for the Oberon programming language targeting the RISC hardware architecture.
It generates machine code for RISC processors from modules written in Oberon and stores it in corresponding object files.
For debugging purposes, it also creates a debugging information file as well as an assembly file containing a listing of the generated machine code.
In addition, it stores the interface of each module in a symbol file which is required when other modules import the module.
Programs generated with this compiler require additional runtime support that is stored in the \file{ob\-risc\-run} library file.
\flowgraph{\resource{Oberon\\source code} \ar[r] & \toolbox{obrisc} \ar[r] \ar@/l/[d] \ar[rd] & \resource{object file} \\ \variable{ECSIMPORT} \ar[ru] & \resource{symbol\\files} \ar@/r/[u] & \resource{debugging\\information}}
\seeoberon\seeassembly\seerisc\seeobject\seedebugging
}

\providecommand{\obwasm}{
\toolsection{obwasm} is a compiler for the Oberon programming language targeting the WebAssembly architecture.
It generates machine code for WebAssembly targets from modules written in Oberon and stores it in corresponding object files.
For debugging purposes, it also creates a debugging information file as well as an assembly file containing a listing of the generated machine code.
In addition, it stores the interface of each module in a symbol file which is required when other modules import the module.
Programs generated with this compiler require additional runtime support that is stored in the \file{ob\-wasm\-run} library file.
\flowgraph{\resource{Oberon\\source code} \ar[r] & \toolbox{obwasm} \ar[r] \ar@/l/[d] \ar[rd] & \resource{object file} \\ \variable{ECSIMPORT} \ar[ru] & \resource{symbol\\files} \ar@/r/[u] & \resource{debugging\\information}}
\seeoberon\seeassembly\seewasm\seeobject\seedebugging
}

% converter tools

\providecommand{\dbgdwarf}{
\toolsection{dbgdwarf} is a DWARF debugging information converter tool.
It converts debugging information into the DWARF debugging data format and stores it in corresponding object files~\cite{dwarffile}.
The resulting debugging object files can be combined with runtime support that creates Executable and Linking Format (ELF) files~\cite{elffile}.
\flowgraph{\resource{debugging\\information} \ar[r] & \toolbox{dbgdwarf} \ar[r] & \resource{debugging\\object file}}
\seeobject\seedebugging
}

% assembler tools

\providecommand{\asmprint}{
\toolsection{asmprint} is a pretty printer for generic assembly code.
It reformats generic assembly code and writes it to the standard output stream.
\flowgraph{\resource{generic assembly\\source code} \ar[r] & \toolbox{asmprint} \ar[r] & \resource{reformatted\\source code}}
\seeassembly
}

\providecommand{\amdaasm}{
\toolsection{amd16asm} is an assembler for the AMD64 hardware architecture.
It translates assembly code into machine code for AMD64 processors and stores it in corresponding object files.
By default, the assembler generates machine code for the 16-bit operating mode defined by the AMD64 architecture.
\flowgraph{\resource{AMD16 assembly\\source code} \ar[r] & \toolbox{amd16asm} \ar[r] & \resource{object file}}
\seeassembly\seeamd\seeobject
}

\providecommand{\amdadism}{
\toolsection{amd16dism} is a disassembler for the AMD64 hardware architecture.
It translates machine code from object files targeting AMD64 processors into assembly code and writes it to the standard output stream.
It assumes that the machine code was generated for the 16-bit operating mode defined by the AMD64 architecture.
\flowgraph{\resource{object file} \ar[r] & \toolbox{amd16dism} \ar[r] & \resource{disassembly\\listing}}
\seeassembly\seeamd\seeobject
}

\providecommand{\amdbasm}{
\toolsection{amd32asm} is an assembler for the AMD64 hardware architecture.
It translates assembly code into machine code for AMD64 processors and stores it in corresponding object files.
By default, the assembler generates machine code for the 32-bit operating mode defined by the AMD64 architecture.
\flowgraph{\resource{AMD32 assembly\\source code} \ar[r] & \toolbox{amd32asm} \ar[r] & \resource{object file}}
\seeassembly\seeamd\seeobject
}

\providecommand{\amdbdism}{
\toolsection{amd32dism} is a disassembler for the AMD64 hardware architecture.
It translates machine code from object files targeting AMD64 processors into assembly code and writes it to the standard output stream.
It assumes that the machine code was generated for the 32-bit operating mode defined by the AMD64 architecture.
\flowgraph{\resource{object file} \ar[r] & \toolbox{amd32dism} \ar[r] & \resource{disassembly\\listing}}
\seeassembly\seeamd\seeobject
}

\providecommand{\amdcasm}{
\toolsection{amd64asm} is an assembler for the AMD64 hardware architecture.
It translates assembly code into machine code for AMD64 processors and stores it in corresponding object files.
By default, the assembler generates machine code for the 64-bit operating mode defined by the AMD64 architecture.
\flowgraph{\resource{AMD64 assembly\\source code} \ar[r] & \toolbox{amd64asm} \ar[r] & \resource{object file}}
\seeassembly\seeamd\seeobject
}

\providecommand{\amdcdism}{
\toolsection{amd64dism} is a disassembler for the AMD64 hardware architecture.
It translates machine code from object files targeting AMD64 processors into assembly code and writes it to the standard output stream.
It assumes that the machine code was generated for the 64-bit operating mode defined by the AMD64 architecture.
\flowgraph{\resource{object file} \ar[r] & \toolbox{amd64dism} \ar[r] & \resource{disassembly\\listing}}
\seeassembly\seeamd\seeobject
}

\providecommand{\armaasm}{
\toolsection{arma32asm} is an assembler for the ARM hardware architecture.
It translates assembly code into machine code for ARM processors executing A32 instructions and stores it in corresponding object files.
\flowgraph{\resource{ARM A32 assembly\\source code} \ar[r] & \toolbox{arma32asm} \ar[r] & \resource{object file}}
\seeassembly\seearm\seeobject
}

\providecommand{\armadism}{
\toolsection{arma32dism} is a disassembler for the ARM hardware architecture.
It translates machine code from object files targeting ARM processors executing A32 instructions into assembly code and writes it to the standard output stream.
\flowgraph{\resource{object file} \ar[r] & \toolbox{arma32dism} \ar[r] & \resource{disassembly\\listing}}
\seeassembly\seearm\seeobject
}

\providecommand{\armbasm}{
\toolsection{arma64asm} is an assembler for the ARM hardware architecture.
It translates assembly code into machine code for ARM processors executing A64 instructions and stores it in corresponding object files.
\flowgraph{\resource{ARM A64 assembly\\source code} \ar[r] & \toolbox{arma64asm} \ar[r] & \resource{object file}}
\seeassembly\seearm\seeobject
}

\providecommand{\armbdism}{
\toolsection{arma64dism} is a disassembler for the ARM hardware architecture.
It translates machine code from object files targeting ARM processors executing A64 instructions into assembly code and writes it to the standard output stream.
\flowgraph{\resource{object file} \ar[r] & \toolbox{arma64dism} \ar[r] & \resource{disassembly\\listing}}
\seeassembly\seearm\seeobject
}

\providecommand{\armcasm}{
\toolsection{armt32asm} is an assembler for the ARM hardware architecture.
It translates assembly code into machine code for ARM processors executing T32 instructions and stores it in corresponding object files.
\flowgraph{\resource{ARM T32 assembly\\source code} \ar[r] & \toolbox{armt32asm} \ar[r] & \resource{object file}}
\seeassembly\seearm\seeobject
}

\providecommand{\armcdism}{
\toolsection{armt32dism} is a disassembler for the ARM hardware architecture.
It translates machine code from object files targeting ARM processors executing T32 instructions into assembly code and writes it to the standard output stream.
\flowgraph{\resource{object file} \ar[r] & \toolbox{armt32dism} \ar[r] & \resource{disassembly\\listing}}
\seeassembly\seearm\seeobject
}

\providecommand{\avrasm}{
\toolsection{avrasm} is an assembler for the AVR hardware architecture.
It translates assembly code into machine code for AVR processors and stores it in corresponding object files.
The identifiers \texttt{RXL}, \texttt{RXH}, \texttt{RYL}, \texttt{RYH}, \texttt{RZL}, and \texttt{RZH} are predefined and name the corresponding registers.
The identifiers \texttt{SPL} and \texttt{SPH} are also predefined and evaluate to the address of the corresponding registers.
\flowgraph{\resource{AVR assembly\\source code} \ar[r] & \toolbox{avrasm} \ar[r] & \resource{object file}}
\seeassembly\seeavr\seeobject
}

\providecommand{\avrdism}{
\toolsection{avrdism} is a disassembler for the AVR hardware architecture.
It translates machine code from object files targeting AVR processors into assembly code and writes it to the standard output stream.
\flowgraph{\resource{object file} \ar[r] & \toolbox{avrdism} \ar[r] & \resource{disassembly\\listing}}
\seeassembly\seeavr\seeobject
}

\providecommand{\avrttasm}{
\toolsection{avr32asm} is an assembler for the AVR32 hardware architecture.
It translates assembly code into machine code for AVR32 processors and stores it in corresponding object files.
\flowgraph{\resource{AVR32 assembly\\source code} \ar[r] & \toolbox{avr32asm} \ar[r] & \resource{object file}}
\seeassembly\seeavrtt\seeobject
}

\providecommand{\avrttdism}{
\toolsection{avr32dism} is a disassembler for the AVR32 hardware architecture.
It translates machine code from object files targeting AVR32 processors into assembly code and writes it to the standard output stream.
\flowgraph{\resource{object file} \ar[r] & \toolbox{avr32dism} \ar[r] & \resource{disassembly\\listing}}
\seeassembly\seeavrtt\seeobject
}

\providecommand{\mabkasm}{
\toolsection{m68kasm} is an assembler for the M68000 hardware architecture.
It translates assembly code into machine code for M68000 processors and stores it in corresponding object files.
\flowgraph{\resource{68000 assembly\\source code} \ar[r] & \toolbox{m68kasm} \ar[r] & \resource{object file}}
\seeassembly\seemabk\seeobject
}

\providecommand{\mabkdism}{
\toolsection{m68kdism} is a disassembler for the M68000 hardware architecture.
It translates machine code from object files targeting M68000 processors into assembly code and writes it to the standard output stream.
\flowgraph{\resource{object file} \ar[r] & \toolbox{m68kdism} \ar[r] & \resource{disassembly\\listing}}
\seeassembly\seemabk\seeobject
}

\providecommand{\miblasm}{
\toolsection{miblasm} is an assembler for the MicroBlaze hardware architecture.
It translates assembly code into machine code for MicroBlaze processors and stores it in corresponding object files.
\flowgraph{\resource{MicroBlaze assembly\\source code} \ar[r] & \toolbox{miblasm} \ar[r] & \resource{object file}}
\seeassembly\seemibl\seeobject
}

\providecommand{\mibldism}{
\toolsection{mibldism} is a disassembler for the MicroBlaze hardware architecture.
It translates machine code from object files targeting MicroBlaze processors into assembly code and writes it to the standard output stream.
\flowgraph{\resource{object file} \ar[r] & \toolbox{mibldism} \ar[r] & \resource{disassembly\\listing}}
\seeassembly\seemibl\seeobject
}

\providecommand{\mipsaasm}{
\toolsection{mips32asm} is an assembler for the MIPS32 hardware architecture.
It translates assembly code into machine code for MIPS32 processors and stores it in corresponding object files.
\flowgraph{\resource{MIPS32 assembly\\source code} \ar[r] & \toolbox{mips32asm} \ar[r] & \resource{object file}}
\seeassembly\seemips\seeobject
}

\providecommand{\mipsadism}{
\toolsection{mips32dism} is a disassembler for the MIPS32 hardware architecture.
It translates machine code from object files targeting MIPS32 processors into assembly code and writes it to the standard output stream.
\flowgraph{\resource{object file} \ar[r] & \toolbox{mips32dism} \ar[r] & \resource{disassembly\\listing}}
\seeassembly\seemips\seeobject
}

\providecommand{\mipsbasm}{
\toolsection{mips64asm} is an assembler for the MIPS64 hardware architecture.
It translates assembly code into machine code for MIPS64 processors and stores it in corresponding object files.
\flowgraph{\resource{MIPS64 assembly\\source code} \ar[r] & \toolbox{mips64asm} \ar[r] & \resource{object file}}
\seeassembly\seemips\seeobject
}

\providecommand{\mipsbdism}{
\toolsection{mips64dism} is a disassembler for the MIPS64 hardware architecture.
It translates machine code from object files targeting MIPS64 processors into assembly code and writes it to the standard output stream.
\flowgraph{\resource{object file} \ar[r] & \toolbox{mips64dism} \ar[r] & \resource{disassembly\\listing}}
\seeassembly\seemips\seeobject
}

\providecommand{\mmixasm}{
\toolsection{mmixasm} is an assembler for the MMIX hardware architecture.
It translates assembly code into machine code for MMIX processors and stores it in corresponding object files.
The names of all special registers are predefined and evaluate to the corresponding number.
\flowgraph{\resource{MMIX assembly\\source code} \ar[r] & \toolbox{mmixasm} \ar[r] & \resource{object file}}
\seeassembly\seemmix\seeobject
}

\providecommand{\mmixdism}{
\toolsection{mmixdism} is a disassembler for the MMIX hardware architecture.
It translates machine code from object files targeting MMIX processors into assembly code and writes it to the standard output stream.
\flowgraph{\resource{object file} \ar[r] & \toolbox{mmixdism} \ar[r] & \resource{disassembly\\listing}}
\seeassembly\seemmix\seeobject
}

\providecommand{\orokasm}{
\toolsection{or1kasm} is an assembler for the OpenRISC 1000 hardware architecture.
It translates assembly code into machine code for OpenRISC 1000 processors and stores it in corresponding object files.
\flowgraph{\resource{OpenRISC 1000 assembly\\source code} \ar[r] & \toolbox{or1kasm} \ar[r] & \resource{object file}}
\seeassembly\seeorok\seeobject
}

\providecommand{\orokdism}{
\toolsection{or1kdism} is a disassembler for the OpenRISC 1000 hardware architecture.
It translates machine code from object files targeting OpenRISC 1000 processors into assembly code and writes it to the standard output stream.
\flowgraph{\resource{object file} \ar[r] & \toolbox{or1kdism} \ar[r] & \resource{disassembly\\listing}}
\seeassembly\seeorok\seeobject
}

\providecommand{\ppcaasm}{
\toolsection{ppc32asm} is an assembler for the PowerPC hardware architecture.
It translates assembly code into machine code for PowerPC processors and stores it in corresponding object files.
By default, the assembler generates machine code for the 32-bit operating mode defined by the PowerPC architecture.
\flowgraph{\resource{PowerPC assembly\\source code} \ar[r] & \toolbox{ppc32asm} \ar[r] & \resource{object file}}
\seeassembly\seeppc\seeobject
}

\providecommand{\ppcadism}{
\toolsection{ppc32dism} is a disassembler for the PowerPC hardware architecture.
It translates machine code from object files targeting PowerPC processors into assembly code and writes it to the standard output stream.
It assumes that the machine code was generated for the 32-bit operating mode defined by the PowerPC architecture.
\flowgraph{\resource{object file} \ar[r] & \toolbox{ppc32dism} \ar[r] & \resource{disassembly\\listing}}
\seeassembly\seeppc\seeobject
}

\providecommand{\ppcbasm}{
\toolsection{ppc64asm} is an assembler for the PowerPC hardware architecture.
It translates assembly code into machine code for PowerPC processors and stores it in corresponding object files.
By default, the assembler generates machine code for the 64-bit operating mode defined by the PowerPC architecture.
\flowgraph{\resource{PowerPC assembly\\source code} \ar[r] & \toolbox{ppc64asm} \ar[r] & \resource{object file}}
\seeassembly\seeppc\seeobject
}

\providecommand{\ppcbdism}{
\toolsection{ppc64dism} is a disassembler for the PowerPC hardware architecture.
It translates machine code from object files targeting PowerPC processors into assembly code and writes it to the standard output stream.
It assumes that the machine code was generated for the 64-bit operating mode defined by the PowerPC architecture.
\flowgraph{\resource{object file} \ar[r] & \toolbox{ppc64dism} \ar[r] & \resource{disassembly\\listing}}
\seeassembly\seeppc\seeobject
}

\providecommand{\riscasm}{
\toolsection{riscasm} is an assembler for the RISC hardware architecture.
It translates assembly code into machine code for RISC processors and stores it in corresponding object files.
The names of all special registers are predefined and evaluate to the corresponding number.
\flowgraph{\resource{RISC assembly\\source code} \ar[r] & \toolbox{riscasm} \ar[r] & \resource{object file}}
\seeassembly\seerisc\seeobject
}

\providecommand{\riscdism}{
\toolsection{riscdism} is a disassembler for the RISC hardware architecture.
It translates machine code from object files targeting RISC processors into assembly code and writes it to the standard output stream.
\flowgraph{\resource{object file} \ar[r] & \toolbox{riscdism} \ar[r] & \resource{disassembly\\listing}}
\seeassembly\seerisc\seeobject
}

\providecommand{\wasmasm}{
\toolsection{wasmasm} is an assembler for the WebAssembly architecture.
It translates assembly code into machine code for WebAssembly targets and stores it in corresponding object files.
The names of all special registers are predefined and evaluate to the corresponding number.
\flowgraph{\resource{WebAssembly assembly\\source code} \ar[r] & \toolbox{wasmasm} \ar[r] & \resource{object file}}
\seeassembly\seewasm\seeobject
}

\providecommand{\wasmdism}{
\toolsection{wasmdism} is a disassembler for the WebAssembly architecture.
It translates machine code from object files targeting WebAssembly targets into assembly code and writes it to the standard output stream.
\flowgraph{\resource{object file} \ar[r] & \toolbox{wasmdism} \ar[r] & \resource{disassembly\\listing}}
\seeassembly\seewasm\seeobject
}

% linker tools

\providecommand{\linklib}{
\toolsection{linklib} is an object file combiner.
It creates a static library file by combining all object files given to it into a single one.
\flowgraph{\resource{object files} \ar[r] & \toolbox{linklib} \ar[r] & \resource{library file}}
\seeobject
}

\providecommand{\linkbin}{
\toolsection{linkbin} is a linker for plain binary files.
It links all object files given to it into a single image and stores it in a binary file that begins with the first linked section.
It also creates a map file that lists the address, type, name and size of all used sections.
The filename extension of the resulting binary file can be specified by putting it into a constant data section called \texttt{\_extension}.
\flowgraph{\resource{object files} \ar[r] & \toolbox{linkbin} \ar[r] \ar[d] & \resource{binary file} \\ & \resource{map file}}
\seeobject
}

\providecommand{\linkmem}{
\toolsection{linkmem} is a linker for plain binary files partitioned into random-access and read-only memory.
It links all object files given to it into two distinct images, one for data sections and one for code and constant data sections, and stores each image in a binary file that begins with the first linked section of the corresponding type.
It also creates a map file that lists the address, type, name and size of all used sections.
\flowgraph{\resource{object files} \ar[r] & \toolbox{linkmem} \ar[r] \ar[d] & \resource{RAM file/\\ROM file} \\ & \resource{map file}}
\seeobject
}

\providecommand{\linkprg}{
\toolsection{linkprg} is a linker for GEMDOS executable files.
It links all object files given to it into a single image and stores the image in an Atari GEMDOS executable file~\cite{gemdosfile}.
It also creates a map file that lists the address relative to the text segment, type, name and size of all used sections.
The filename extension of the resulting executable file can be specified by putting it into a constant data section called \texttt{\_extension}.
The GEMDOS executable file format requires all patch patterns of absolute link patches to consist of four full bitmasks with descending offsets.
\flowgraph{\resource{object files} \ar[r] & \toolbox{linkprg} \ar[r] \ar[d] & \resource{executable file} \\ & \resource{map file}}
\seeobject
}

\providecommand{\linkhex}{
\toolsection{linkhex} is a linker for Intel HEX files.
It links all code sections of the object files given to it into single image and stores the image in an Intel HEX file~\cite{hexfile} that begins with the first linked section.
It also creates a map file that lists the address, type, name and size of all used sections.
\flowgraph{\resource{object files} \ar[r] & \toolbox{linkhex} \ar[r] \ar[d] & \resource{HEX file} \\ & \resource{map file}}
\seeobject
}

\providecommand{\mapsearch}{
\toolsection{mapsearch} is a debugging tool.
It searches map files generated by linker tools for the name of a binary section that encompasses a memory address read from the standard input stream.
If additionally provided with one or more object files, it also stores an excerpt thereof in a separate object file called map search result which only contains the identified binary section for disassembling purposes.
\flowgraph{& \resource{map files/\\object files} \ar[d] \\ \resource{memory\\address} \ar[r] & \toolbox{mapsearch} \ar[r] \ar[d] & \resource{section name/\\relative offset} \\ & \resource{object file\\excerpt}}
\seeobject
}


\startchapter{Questions and Answers}{Frequently Asked Questions}{faq}
{This \documentation{} answers frequently asked questions about the \ecs{} and how to use its tools to accomplish common tasks.}

\newcommand{\question}[1]{\subsection*{#1}}

\section{General Questions}

This section assembles questions frequently asked about the \ecs{} in general.

\question{What is the \ecs{}? Why does it exist?}

The \ecs{} is a completely self-contained collection of software development tools.
It exists to be recognized and adopted as a free development toolchain which is hopefully as useful and easy to use as its source code is intended to be approachable and comprehensible for developers and students wanting to learn, maintain, and customize a complete toolchain.

\question{What does the name \ecs{} mean?}

The \ecs{} provides all essential freedoms guaranteed by free software and is maintainable by a single person.
The term eigen in its name thus means that the \ecs{} is literally your own, regardless of whether you are a user or a developer.

\question{Does the \ecs{} support Unicode source files?}

The \ecs{} currently only supports the ASCII character encoding.
It will however eventually support Unicode source files encoded in UTF-8.

\section{How-Tos}

This section describes how to accomplish commonly requested tasks using the tools of the \ecs{} and therefore assumes some basic knowledge of its features and functionality.

\question{How to disassemble plain binary files?}

Since disassemblers can only process object files which are usually generated by other tools of the \ecs{}, a plain binary file must first be converted into an object file.
The simplest way to achieve this conversion is to assemble an assembly source file consisting of a single line of code which just copies the contents of the binary file using the embed binary file directive.

\concludechapter

\input{gpl}
\input{rse}
\input{fdl}

\concludebook
