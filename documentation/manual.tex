% User manual for the Eigen Compiler Suite
% Copyright (C) Florian Negele

% This file is part of the Eigen Compiler Suite.

% Permission is granted to copy, distribute and/or modify this document
% under the terms of the GNU Free Documentation License, Version 1.3
% or any later version published by the Free Software Foundation.

% You should have received a copy of the GNU Free Documentation License
% along with the ECS.  If not, see <https://www.gnu.org/licenses/>.

% Generic documentation utilities
% Copyright (C) Florian Negele

% This file is part of the Eigen Compiler Suite.

% Permission is granted to copy, distribute and/or modify this document
% under the terms of the GNU Free Documentation License, Version 1.3
% or any later version published by the Free Software Foundation.

% You should have received a copy of the GNU Free Documentation License
% along with the ECS.  If not, see <https://www.gnu.org/licenses/>.

\providecommand{\cpp}{C\texttt{++}}
\providecommand{\opt}{_\mathit{opt}}
\providecommand{\tool}[1]{\texttt{#1}}
\providecommand{\version}{Version 0.0.40}
\providecommand{\resource}[1]{*++\txt{#1}}
\providecommand{\ecs}{Eigen Compiler Suite}
\providecommand{\changed}[1]{\underline{#1}}
\providecommand{\toolbox}[1]{\converter{#1}}
\providecommand{\file}{}\renewcommand{\file}[1]{\texttt{#1}}
\providecommand{\alignright}{\hfill\linebreak[0]\hspace*{\fill}}
\providecommand{\converter}[1]{*++[F][F*:white][F,:gray]\txt{#1}}
\providecommand{\documentation}{\ifbook chapter\else document\fi}
\providecommand{\Documentation}{\ifbook Chapter\else Document\fi}
\providecommand{\variable}[1]{\resource{\texttt{\small#1}\\variable}}
\providecommand{\documentationref}[2]{\ifbook\ref{#1}\else``\href{#1}{#2}''~\cite{#1}\fi}
\providecommand{\objfile}[1]{\texttt{#1}\index[runtime]{#1 object file@\texttt{#1} object file}}
\providecommand{\libfile}[1]{\texttt{#1}\index[runtime]{#1 library file@\texttt{#1} library file}}
\providecommand{\epigraph}[2]{\ifbook\begin{quote}\flushright\textit{#1}\par--- #2\end{quote}\fi}
\providecommand{\environmentvariable}[1]{\texttt{#1}\index{Environment variables!#1@\texttt{#1}}}
\providecommand{\environment}[1]{\texttt{#1}\index[environment]{#1 environment@\texttt{#1} environment}}
\providecommand{\toolsection}{}\renewcommand{\toolsection}[1]{\subsection{#1}\label{\prefix:#1}\tool{#1}}
\providecommand{\instruction}{}\renewcommand{\instruction}[2]{\noindent\qquad\pdftooltip{\texttt{#1}}{#2}\refstepcounter{instruction}\par}
\providecommand{\flowgraph}{}\renewcommand{\flowgraph}[1]{\par\sffamily\begin{displaymath}\xymatrix@=4ex{#1}\end{displaymath}\normalfont\par}
\providecommand{\instructionset}{}\renewcommand{\instructionset}[4]{\setcounter{instruction}{0}\begin{multicols}{\ifbook#3\else#4\fi}[{\captionof{table}[#2]{#2 (\ref*{#1:instructions}~instructions)}\label{tab:#1set}\vspace{-2ex}}]\footnotesize\raggedcolumns\input{#1.set}\label{#1:instructions}\end{multicols}}

\providecommand{\gpl}{GNU General Public License}
\providecommand{\rse}{ECS Runtime Support Exception}
\providecommand{\fdl}{\href{https://www.gnu.org/licenses/fdl.html}{GNU Free Documentation License}}

\providecommand{\docbegin}{}
\providecommand{\docend}{}
\providecommand{\doclabel}[1]{\hypertarget{#1}}
\providecommand{\doclink}[2]{\hyperlink{#1}{#2}}
\providecommand{\docsection}[3]{\hypertarget{#1}{\subsection{#2}}\label{sec:#1}\index[library]{#2@#3}}
\providecommand{\docsectionstar}[1]{}
\providecommand{\docsubbegin}{\begin{description}}
\providecommand{\docsubend}{\end{description}}
\providecommand{\docsubsection}[3]{\item[\hypertarget{#1}{#2}]\index[library]{#2@#3}}
\providecommand{\docsubsectionstar}[1]{\smallskip}
\providecommand{\docsubsubsection}[3]{\docsubsection{#1}{#2}{#3}}
\providecommand{\docsubsubsectionstar}[1]{}
\providecommand{\docsubsubsubsection}[3]{}
\providecommand{\docsubsubsubsectionstar}[1]{}
\providecommand{\doctable}{}

\providecommand{\debuggingtool}{}\renewcommand{\debuggingtool}{This tool is provided for debugging purposes.
It allows exposing and modifying an internal data structure that is usually not accessible.
}

\providecommand{\interface}{All tools accept command-line arguments which are taken as names of plain text files containing the source code.
If no arguments are provided, the standard input stream is used instead.
Output files are generated in the current working directory and have the same name as the input file being processed whereas the filename extension gets replaced by an appropriate suffix.
\seeinterface
}

\providecommand{\license}{\noindent Copyright \copyright{} Florian Negele\par\medskip\noindent
Permission is granted to copy, distribute and/or modify this document under the terms of the
\fdl{}, Version 1.3 or any later version published by the \href{https://fsf.org/}{Free Software Foundation}.
}

\providecommand{\ecslogosurface}{
\fill[darkgray] (0,0,0) -- (0,0,3) -- (0,3,3) -- (0,3,1) -- (0,4,1) -- (0,4,3) -- (0,5,3) -- (0,5,0) -- (0,2,0) -- (0,2,2) -- (0,1,2) -- (0,1,0) -- cycle;
\fill[gray] (0,5,0) -- (0,5,3) -- (1,5,3) -- (1,5,1) -- (2,5,1) -- (2,5,3) -- (3,5,3) -- (3,5,0) -- cycle;
\fill[lightgray] (0,0,0) -- (0,1,0) -- (2,1,0) -- (2,4,0) -- (1,4,0) -- (1,3,0) -- (2,3,0) -- (2,2,0) -- (0,2,0) -- (0,5,0) -- (3,5,0) -- (3,0,0) -- cycle;
\begin{scope}[line width=0.5]
\begin{scope}[gray]
\draw (0,0,0) -- (0,1,0);
\draw (2,1,0) -- (2,2,0);
\draw (0,1,2) -- (0,2,2);
\draw (0,2,0) -- (0,5,0);
\draw (2,3,0) -- (2,4,0);
\end{scope}
\begin{scope}[lightgray]
\draw (0,1,0) -- (0,1,2);
\draw (0,3,1) -- (0,3,3);
\draw (0,5,0) -- (0,5,3);
\draw (2,5,1) -- (2,5,3);
\end{scope}
\begin{scope}[white]
\draw (0,1,0) -- (2,1,0);
\draw (1,3,0) -- (2,3,0);
\draw (0,5,0) -- (3,5,0);
\end{scope}
\end{scope}
}

\providecommand{\ecslogo}[1]{
\begin{tikzpicture}[scale={(#1)/((sin(45)+cos(45))*3cm)},x={({-cos(45)*1cm},{sin(45)*sin(30)*1cm})},y={({0cm},{(cos(30)*1cm})},z={({sin(45)*1cm},{cos(45)*sin(30)*1cm})}]
\begin{scope}[darkgray,line width=1]
\draw (0,0,0) -- (0,0,3) -- (0,3,3) -- (2,3,3) -- (2,5,3) -- (3,5,3) -- (3,5,0) -- (3,0,0) -- cycle;
\draw (0,3,1) -- (0,4,1) -- (0,4,3) -- (0,5,3) -- (1,5,3) -- (1,5,1) -- (2,5,1);
\draw (1,3,0) -- (1,4,0) -- (2,4,0);
\end{scope}
\fill[darkgray] (2,0,0) -- (2,0,3) -- (2,5,3) -- (2,5,1) -- (2,4,1) -- (2,4,0) -- cycle;
\fill[lightgray] (2,0,2) -- (0,0,2) -- (0,2,2) -- (2,2,2) -- cycle;
\fill[gray] (0,1,0) -- (2,1,0) -- (2,1,2) -- (0,1,2) -- cycle;
\fill[gray] (0,3,1) -- (0,3,3) -- (2,3,3) -- (2,3,0) -- (1,3,0) -- (1,3,1) -- cycle;
\ecslogosurface
\end{tikzpicture}
}

\providecommand{\shadowedecslogo}[3]{
\begin{tikzpicture}[scale={(#1)/((sin(#2)+cos(#2))*3cm)},x={({-cos(#2)*1cm},{sin(#2)*sin(#3)*1cm})},y={({0cm},{(cos(#3)*1cm})},z={({sin(#2)*1cm},{cos(#2)*sin(#3)*1cm})}]
\shade[top color=lightgray!50!white,bottom color=white,middle color=lightgray!50!white] (0,0,0) -- (3,0,0) -- (3,{-0.5-3*sin(#2)*sin(#3)/cos(#3)},0) -- (0,-0.5,0) -- cycle;
\shade[top color=darkgray!50!gray,bottom color=white,middle color=darkgray!50!white] (0,0,0) -- (0,0,3) -- (0,{-0.5-3*cos(#2)*sin(#3)/cos(#3)},3) -- (0,-0.5,0) -- cycle;
\begin{scope}[y={({(cos(#2)+sin(#2))*0.5cm},{(cos(#2)*sin(#3)-sin(#2)*sin(#3))*0.5cm})}]
\useasboundingbox (3,0,0) -- (0,0,0) -- (0,0,3);
\shade[left color=darkgray!80!black,right color=lightgray,middle color=gray] (0,0,0) -- (0,1,0) -- (0,1,0.5) -- (0,2,0) -- (0,5,0) -- (0,5,3) -- (1,5,3) -- (1,4,3) -- (1,4,2.5) -- (1,3,3) -- (2,5,3) -- (3,5,3) -- (3,0,3) -- cycle;
\clip (0,0,0) -- (0,0,3) -- ({-3*sin(#2)/cos(#2)},0,0) -- cycle;
\shade[left color=darkgray,right color=lightgray!50!gray] (0,0,0) -- (0,1,0) -- (0,1,0.5) -- (0,2,0) -- (0,5,0) -- (0,5,3) -- (1,5,3) -- (1,4,3) -- (1,4,2.5) -- (1,3,3) -- (2,5,3) -- (3,5,3) -- (3,0,3) -- cycle;
\end{scope}
\shade[left color=darkgray,right color=darkgray!80!black] (2,0,0) -- (2,0,3) -- (2,5,3) -- (2,5,1) -- (2,4,1) -- (2,4,0) -- cycle;
\shade[left color=darkgray!90!black,right color=gray!80!darkgray] (2,0,2) -- (0,0,2) -- (0,2,2) -- (2,2,2) -- cycle;
\shade[top color=darkgray!90!black,bottom color=gray!80!darkgray] (0,1,0) -- (2,1,0) -- (2,1,2) -- (0,1,2) -- cycle;
\shade[top color=darkgray!90!black,bottom color=gray!80!darkgray] (0,3,1) -- (0,3,3) -- (2,3,3) -- (2,3,0) -- (1,3,0) -- (1,3,1) -- cycle;
\fill[gray] (2,1,0) -- (1.5,1,0.5) -- (0,1,0.5) -- (0,1,0) -- cycle;
\fill[gray] (1,3,2) -- (0.5,3,2) -- (0.5,3,3) -- (1,3,3) -- cycle;
\fill[gray] (2,3,0) -- (1.5,3,0.5) -- (1,3,0.5) -- (1,3,0) -- cycle;
\ecslogosurface
\end{tikzpicture}
}

\providecommand{\cpplogo}[1]{
\begin{tikzpicture}[scale=(#1)/512em]
\fill[gray] (435.2794,398.7159) -- (247.1911,507.3075) .. controls (236.3563,513.5642) and (218.6240,513.5642) .. (207.7892,507.3075) -- (19.7009,398.7159) .. controls (8.8646,392.4606) and (0.0000,377.1043) .. (0.0000,364.5924) -- (0.0000,147.4076) .. controls (0.8430,132.8363) and (8.2856,120.7683) .. (19.7009,113.2842) -- (207.7892,4.6926) .. controls (218.6240,-1.5642) and (236.3564,-1.5642) .. (247.1911,4.6926) -- (435.2794,113.2842) .. controls (447.5273,121.4304) and (454.4987,133.6918) .. (454.9803,147.4076) -- (454.9803,364.5924) .. controls (454.5404,377.7571) and (446.6566,391.0351) .. (435.2794,398.7159) -- cycle(75.8301,255.9993) .. controls (74.9389,404.0881) and (273.2892,469.4783) .. (358.8263,331.8769) -- (293.1917,293.8965) .. controls (253.5702,359.4301) and (155.1909,335.9977) .. (151.6601,255.9993) .. controls (152.7204,182.2703) and (249.4137,148.0211) .. (293.1961,218.1065) -- (358.8308,180.1276) .. controls (283.4477,49.2645) and (79.6318,96.3470) .. (75.8301,255.9993) -- cycle(379.1503,247.5747) -- (362.2982,247.5747) -- (362.2982,230.7226) -- (345.4490,230.7226) -- (345.4490,247.5747) -- (328.5969,247.5747) -- (328.5969,264.4254) -- (345.4490,264.4254) -- (345.4490,281.2759) -- (362.2982,281.2759) -- (362.2982,264.4254) -- (379.1503,264.4254) -- cycle(442.3420,247.5747) -- (425.4899,247.5747) -- (425.4899,230.7226) -- (408.6408,230.7226) -- (408.6408,247.5747) -- (391.7886,247.5747) -- (391.7886,264.4254) -- (408.6408,264.4254) -- (408.6408,281.2759) -- (425.4899,281.2759) -- (425.4899,264.4254) -- (442.3420,264.4254) -- cycle;
\end{tikzpicture}
}

\providecommand{\fallogo}[1]{
\begin{tikzpicture}[scale=(#1)/512em]
\fill[gray] (185.7774,0.0000) .. controls (200.4486,15.9798) and (226.8966,8.7148) .. (235.0426,31.5836) .. controls (249.5297,58.0598) and (247.9581,97.9161) .. (280.3335,110.9762) .. controls (309.1690,120.3496) and (337.8406,104.2727) .. (366.5753,103.9379) .. controls (373.4449,111.5171) and (379.2885,128.2574) .. (383.9755,108.9744) .. controls (396.6979,102.5615) and (437.2808,107.6681) .. (426.9652,124.3252) .. controls (408.9822,121.0785) and (412.4742,146.0729) .. (426.5192,131.4996) .. controls (433.8413,120.8489) and (465.1541,126.5522) .. (441.9067,135.7950) .. controls (396.1879,157.7478) and (344.1112,161.5079) .. (298.5528,183.5702) .. controls (277.7471,193.5198) and (284.6941,218.7163) .. (285.2127,236.9640) .. controls (292.3599,316.2826) and (307.3929,394.6311) .. (317.1198,473.6154) .. controls (329.0637,505.4736) and (292.1195,528.5004) .. (265.9183,511.2761) .. controls (237.9284,499.2462) and (237.3684,465.2681) .. (230.9102,439.9421) .. controls (218.6692,374.3397) and (215.6307,306.9662) .. (198.1732,242.3977) .. controls (183.1379,232.7444) and (164.4245,256.0298) .. (149.0430,261.4799) .. controls (116.9328,279.2585) and (87.1822,308.5851) .. (48.2293,307.8914) .. controls (21.3220,306.9037) and (-15.9107,281.8761) .. (7.2921,252.7908) .. controls (29.7799,220.6177) and (67.5177,204.3028) .. (100.9287,185.9449) .. controls (130.8217,170.8906) and (161.1548,156.5903) .. (191.0278,141.5847) .. controls (196.1738,120.0520) and (186.6049,95.2409) .. (186.8382,72.4353) .. controls (185.5234,48.4204) and (183.1700,23.9341) .. (185.7774,0.0000) -- cycle;
\end{tikzpicture}
}

\providecommand{\oblogo}[1]{
\begin{tikzpicture}[scale=(#1)/512em]
\fill[gray] (160.3865,208.9117) .. controls (154.0879,214.6478) and (149.0735,221.2409) .. (145.4125,228.5384) .. controls (184.8790,248.4273) and (234.7122,269.8787) .. (297.5493,291.8782) .. controls (300.3943,281.4769) and (300.9552,268.7619) .. (300.4023,255.2389) .. controls (248.9909,244.7891) and (200.0310,225.9279) .. (160.3865,208.9117) -- cycle(225.7398,392.6996) .. controls (308.0209,392.1716) and (359.3326,345.9277) .. (368.7203,285.2098) .. controls (376.6742,197.1784) and (311.7194,141.3342) .. (205.4287,142.1456) .. controls (139.9485,141.4804) and (88.7155,166.1957) .. (73.5775,228.0086) .. controls (52.0297,320.3408) and (123.4078,391.0103) .. (225.7398,392.6996) -- cycle(216.0739,176.4733) .. controls (268.9183,179.2424) and (315.8292,206.5488) .. (312.7454,265.1139) .. controls (313.2769,315.6384) and (286.5993,353.4946) .. (216.6040,355.7934) .. controls (162.4657,355.7934) and (126.0914,317.5023) .. (126.0914,260.5103) .. controls (126.1733,214.2900) and (163.3363,176.2849) .. (216.0739,176.4733) -- cycle(76.4897,189.1754) .. controls (13.1586,147.5631) and (0.0000,119.4207) .. (0.0000,119.4207) -- (90.6499,170.1632) .. controls (85.3004,175.8497) and (80.5994,182.1633) .. (76.4897,189.1754) -- cycle(353.9486,119.3004) -- (402.9482,119.3004) .. controls (427.0025,137.0797) and (450.9893,162.7034) .. (474.9529,191.0213) .. controls (509.3540,228.5339) and (531.3391,294.2091) .. (487.8149,312.1206) .. controls (462.8165,324.7652) and (394.3874,316.8943) .. (373.8912,313.6651) .. controls (379.9291,297.7449) and (383.2899,278.4204) .. (381.4989,257.7214) .. controls (420.3069,248.0321) and (421.9610,218.3461) .. (407.7867,192.6417) .. controls (391.1113,162.4018) and (370.1114,132.9097) .. (353.9486,119.3004) -- cycle;
\end{tikzpicture}
}

\providecommand{\markuptable}{
\begin{table}
\sffamily\centering
\begin{tabular}{@{}lcl@{}}
\toprule
\texttt{//italics//} & $\rightarrow$ & \textit{italics} \\
\midrule
\texttt{**bold**} & $\rightarrow$ & \textbf{bold} \\
\midrule
\texttt{\# ordered list} & & 1 ordered list \\
\texttt{\# second item} & $\rightarrow$ & 2 second item \\
\texttt{\#\# sub item} & & \hspace{1em} 1 sub item \\
\midrule
\texttt{* unordered list} & & $\bullet$ unordered list \\
\texttt{* second item} & $\rightarrow$ & $\bullet$ second item \\
\texttt{** sub item} & & \hspace{1em} $\bullet$ sub item \\
\midrule
\texttt{link to [[label]]} & $\rightarrow$ & link to \underline{label} \\
\midrule
\texttt{<{}<label>{}> definition } & $\rightarrow$ & definition \\
\midrule
\texttt{[[url|link name]]} & $\rightarrow$ & \underline{link name} \\
\midrule\addlinespace
\texttt{= large heading} & & {\Large large heading} \smallskip \\
\texttt{== medium heading} & $\rightarrow$ & {\large medium heading} \\
\texttt{=== small heading} & & small heading \\
\midrule
\texttt{no line break} & & no line break for paragraphs \\
\texttt{for paragraphs} & $\rightarrow$ \\
& & use empty line \\
\texttt{use empty line} \\
\midrule
\texttt{force\textbackslash\textbackslash line break} & $\rightarrow$ & force \\
& & line break \\
\midrule
\texttt{horizontal line} & $\rightarrow$ & horizontal line \\
\texttt{----} & & \hrulefill \\
\midrule
\texttt{|=a|=table|=header} & & \underline{a \enspace table \enspace header} \\
\texttt{|a|table|row} & $\rightarrow$ & a \enspace table \enspace row \\
\texttt{|b|table|row} & & b \enspace table \enspace row \\
\midrule
\texttt{\{\{\{} \\
\texttt{unformatted} & $\rightarrow$ & \texttt{unformatted} \\
\texttt{code} & & \texttt{code} \\
\texttt{\}\}\}} \\
\midrule\addlinespace
\texttt{@ new article} & & {\Large 1.\ new article} \smallskip \\
\texttt{@ second article} & $\rightarrow$ & {\Large 2.\ second article} \smallskip \\
\texttt{@@ sub article} & & {\large 2.1.\ sub article} \\
\bottomrule
\end{tabular}
\normalfont\caption{Elements of the generic documentation markup language}
\label{tab:docmarkup}
\end{table}
}

\providecommand{\startchapter}[4]{
\documentclass[11pt,a4paper]{article}
\usepackage{booktabs}
\usepackage[format=hang,labelfont=bf]{caption}
\usepackage{changepage}
\usepackage[T1]{fontenc}
\usepackage[margin=2cm]{geometry}
\usepackage{hyperref}
\usepackage[american]{isodate}
\usepackage{lmodern}
\usepackage{longtable}
\usepackage{mathptmx}
\usepackage{microtype}
\usepackage[toc]{multitoc}
\usepackage{multirow}
\usepackage[all]{nowidow}
\usepackage{pdfcomment}
\usepackage{syntax}
\usepackage{tikz}
\usepackage[all]{xy}
\hypersetup{pdfborder={0 0 0},bookmarksnumbered=true,pdftitle={\ecs{}: #2},pdfauthor={Florian Negele},pdfsubject={\ecs{}},pdfkeywords={#1}}
\setlength{\grammarindent}{8em}\setlength{\grammarparsep}{0.2ex}
\setlength{\columnsep}{2em}
\newcommand{\prefix}{}
\newcounter{instruction}
\bibliographystyle{unsrt}
\renewcommand{\index}[2][]{}
\renewcommand{\arraystretch}{1.05}
\renewcommand{\floatpagefraction}{0.7}
\renewcommand{\syntleft}{\itshape}\renewcommand{\syntright}{}
\title{\vspace{-5ex}\Huge{\ecs{}}\medskip\hrule}
\author{\huge{#2}}
\date{\medskip\version}
\newif\ifbook\bookfalse
\pagestyle{headings}
\frenchspacing
\begin{document}
\maketitle\thispagestyle{empty}\noindent#4\setlength{\columnseprule}{0.4pt}\tableofcontents\setlength{\columnseprule}{0pt}\vfill\pagebreak[3]\null\vfill\bigskip\noindent
\parbox{\textwidth-4em}{\license The contents of this \documentation{} are part of the \href{manual}{\ecs{} User Manual}~\cite{manual} and correspond to Chapter ``\href{manual\##3}{#1}''.\alignright\mbox{\today}}
\parbox{4em}{\flushright\ecslogo{3em}}
\clearpage
}

\providecommand{\concludechapter}{
\vfill\pagebreak[3]\null\vfill
\thispagestyle{myheadings}\markright{REFERENCES}
\noindent\begin{minipage}{\textwidth}\begin{multicols}{2}[\section*{References}]
\renewcommand{\section}[2]{}\small\bibliography{references}
\end{multicols}\end{minipage}\end{document}
}

\providecommand{\startpresentation}[2]{
\documentclass[14pt,aspectratio=43,usepdftitle=false]{beamer}
\usepackage{booktabs}
\usepackage{etex}
\usepackage{multicol}
\usepackage{tikz}
\usepackage[all]{xy}
\bibliographystyle{unsrt}
\setlength{\columnsep}{1em}
\setlength{\leftmargini}{1em}
\setbeamercolor{title}{fg=black}
\setbeamercolor{structure}{fg=darkgray}
\setbeamercolor{bibliography item}{fg=darkgray}
\setbeamerfont{title}{series=\bfseries}
\setbeamerfont{subtitle}{series=\normalfont}
\setbeamerfont*{frametitle}{parent=title}
\setbeamerfont{block title}{series=\bfseries}
\setbeamerfont*{framesubtitle}{parent=subtitle}
\setbeamersize{text margin left=1em,text margin right=1em}
\setbeamertemplate{navigation symbols}{}
\setbeamertemplate{itemize item}[circle]{}
\setbeamertemplate{bibliography item}[triangle]{}
\setbeamertemplate{bibliography entry author}{\usebeamercolor[fg]{bibliography item}}
\setbeamertemplate{frametitle}{\medskip\usebeamerfont{frametitle}\color{gray}\raisebox{-2.5ex}[0ex][0ex]{\rule{0.1em}{4.5ex}}}
\addtobeamertemplate{frametitle}{}{\hspace{0.4em}\usebeamercolor[fg]{title}\insertframetitle\par\vspace{0.2ex}\hspace{0.5em}\usebeamerfont{framesubtitle}\insertframesubtitle}
\hypersetup{pdfborder={0 0 0},bookmarksnumbered=true,bookmarksopen=true,bookmarksopenlevel=0,pdftitle={\ecs{}: #1},pdfauthor={Florian Negele},pdfsubject={\ecs{}},pdfkeywords={#1}}
\renewcommand{\flowgraph}[1]{\resizebox{\textwidth}{!}{$$\xymatrix{##1}$$}}
\title{\ecs{}\medskip\hrule\medskip}
\institute{\shadowedecslogo{5em}{30}{15}}
\date{\version}
\subtitle{#1}
\begin{document}
\begin{frame}[plain]\titlepage\nocite{manual}\end{frame}
\begin{frame}{Contents}{#1}\begin{center}\tableofcontents\end{center}\end{frame}
}

\providecommand{\concludepresentation}{
\begin{frame}{References}\begin{footnotesize}\setlength{\columnseprule}{0.4pt}\begin{multicols}{2}\bibliography{references}\end{multicols}\end{footnotesize}\end{frame}
\end{document}
}

\providecommand{\startbook}[1]{
\documentclass[10pt,paper=17cm:24cm,DIV=13,twoside=semi,headings=normal,numbers=noendperiod,cleardoublepage=plain]{scrbook}
\usepackage{atveryend}
\usepackage{booktabs}
\usepackage{caption}
\usepackage{changepage}
\usepackage[T1]{fontenc}
\usepackage{imakeidx}
\usepackage{hyperref}
\usepackage[american]{isodate}
\usepackage{lmodern}
\usepackage{longtable}
\usepackage{mathptmx}
\usepackage[final]{microtype}
\usepackage{multicol}
\usepackage{multirow}
\usepackage[all]{nowidow}
\usepackage{pdfcomment}
\usepackage{scrlayer-scrpage}
\usepackage{setspace}
\usepackage{syntax}
\usepackage[eventxtindent=4pt,oddtxtexdent=4pt]{thumbs}
\usepackage{tikz}
\usepackage[all]{xy}
\hyphenation{Micro-Blaze Open-Cores Open-RISC Power-PC}
\hypersetup{pdfborder={0 0 0},bookmarksnumbered=true,bookmarksopen=true,bookmarksopenlevel=0,pdftitle={\ecs{}: #1},pdfauthor={Florian Negele},pdfsubject={\ecs{}},pdfkeywords={#1}}
\setlength{\grammarindent}{8em}\setlength{\grammarparsep}{0.7ex}
\setkomafont{captionlabel}{\usekomafont{descriptionlabel}}
\renewcommand{\arraystretch}{1.05}\setstretch{1.1}
\renewcommand{\chapterformat}{\thechapter\autodot\enskip\raisebox{-1ex}[0ex][0ex]{\color{gray}\rule{0.1em}{3.5ex}}\enskip}
\renewcommand{\startchapter}[4]{\hypertarget{##3}{\chapter{##1}}\label{##3}##4\addthumb{##1}{\LARGE\sffamily\bfseries\thechapter}{white}{gray}\renewcommand{\prefix}{##3}}
\renewcommand{\concludechapter}{\clearpage{\stopthumb\cleardoublepage}}
\renewcommand{\syntleft}{\itshape}\renewcommand{\syntright}{}
\renewcommand{\floatpagefraction}{0.7}
\renewcommand{\partheademptypage}{}
\DeclareMicrotypeAlias{lmss}{cmr}
\newcommand{\prefix}{}
\newcounter{instruction}
\bibliographystyle{unsrt}
\newif\ifbook\booktrue
\makeindex[intoc,title=Index]
\makeindex[intoc,name=tools,title=Index of Tools,columns=3]
\makeindex[intoc,name=library,title=Index of Library Names]
\makeindex[intoc,name=runtime,title=Index of Runtime Support]
\makeindex[intoc,name=environment,title=Index of Target Environments]
\indexsetup{toclevel=chapter,headers={\indexname}{\indexname}}
\frenchspacing
\begin{document}
\pagenumbering{alph}
\begin{titlepage}\centering
\huge\sffamily\null\vfill\textbf{\ecs{}}\bigskip\hrule\bigskip#1
\normalsize\normalfont\vfill\vfill\shadowedecslogo{10em}{30}{15}
\large\vfill\vfill\version
\end{titlepage}
\null\vfill
\thispagestyle{empty}
\noindent\today\par\medskip
\license A copy of this license is included in Appendix~\ref{fdl} on page~\pageref{fdl}.
All product names used herein are for identification purposes only and may be trademarks of their respective companies.
\concludechapter
\frontmatter
\setcounter{tocdepth}{1}
\tableofcontents
\setcounter{tocdepth}{2}
\concludechapter
\listoffigures
\concludechapter
\listoftables
\concludechapter
}

\providecommand{\concludebook}{
\backmatter
\addtocontents{toc}{\protect\setcounter{tocdepth}{-1}}
\phantomsection\addcontentsline{toc}{part}{Bibliography}
\bibliography{references}
\concludechapter
\phantomsection\addcontentsline{toc}{part}{Indexes}
\printindex
\concludechapter
\indexprologue{\label{idx:tools}}
\printindex[tools]
\concludechapter
\printindex[library]
\concludechapter
\indexprologue{\label{idx:runtime}}
\printindex[runtime]
\concludechapter
\indexprologue{\label{idx:environment}}
\printindex[environment]
\concludechapter
\pagestyle{empty}\pagenumbering{Alph}\null\clearpage
\null\vfill\centering\ecslogo{4em}\par\medskip\license
\end{document}
}

% chapter references

\providecommand{\seedocumentationref}{}\renewcommand{\seedocumentationref}[3]{#1, see \Documentation{}~\documentationref{#2}{#3}. }
\providecommand{\seeinterface}{}\renewcommand{\seeinterface}{\ifbook See \Documentation{}~\documentationref{interface}{User Interface} for more information about the common user interface of all of these tools. \fi}
\providecommand{\seeguide}{}\renewcommand{\seeguide}{\seedocumentationref{For basic examples of using some of these tools in practice}{guide}{User Guide}}
\providecommand{\seecpp}{}\renewcommand{\seecpp}{\seedocumentationref{For more information about the \cpp{} programming language and its implementation by the \ecs{}}{cpp}{User Manual for \cpp{}}}
\providecommand{\seefalse}{}\renewcommand{\seefalse}{\seedocumentationref{For more information about the FALSE programming language and its implementation by the \ecs{}}{false}{User Manual for FALSE}}
\providecommand{\seeoberon}{}\renewcommand{\seeoberon}{\seedocumentationref{For more information about the Oberon programming language and its implementation by the \ecs{}}{oberon}{User Manual for Oberon}}
\providecommand{\seeassembly}{}\renewcommand{\seeassembly}{\seedocumentationref{For more information about the generic assembly language and how to use it}{assembly}{Generic Assembly Language Specification}}
\providecommand{\seeamd}{}\renewcommand{\seeamd}{\seedocumentationref{For more information about how the \ecs{} supports the AMD64 hardware architecture}{amd64}{AMD64 Hardware Architecture Support}}
\providecommand{\seearm}{}\renewcommand{\seearm}{\seedocumentationref{For more information about how the \ecs{} supports the ARM hardware architecture}{arm}{ARM Hardware Architecture Support}}
\providecommand{\seeavr}{}\renewcommand{\seeavr}{\seedocumentationref{For more information about how the \ecs{} supports the AVR hardware architecture}{avr}{AVR Hardware Architecture Support}}
\providecommand{\seeavrtt}{}\renewcommand{\seeavrtt}{\seedocumentationref{For more information about how the \ecs{} supports the AVR32 hardware architecture}{avr32}{AVR32 Hardware Architecture Support}}
\providecommand{\seemabk}{}\renewcommand{\seemabk}{\seedocumentationref{For more information about how the \ecs{} supports the M68000 hardware architecture}{m68k}{M68000 Hardware Architecture Support}}
\providecommand{\seemibl}{}\renewcommand{\seemibl}{\seedocumentationref{For more information about how the \ecs{} supports the MicroBlaze hardware architecture}{mibl}{MicroBlaze Hardware Architecture Support}}
\providecommand{\seemips}{}\renewcommand{\seemips}{\seedocumentationref{For more information about how the \ecs{} supports the MIPS32 and MIPS64 hardware architectures}{mips}{MIPS Hardware Architecture Support}}
\providecommand{\seemmix}{}\renewcommand{\seemmix}{\seedocumentationref{For more information about how the \ecs{} supports the MMIX hardware architecture}{mmix}{MMIX Hardware Architecture Support}}
\providecommand{\seeorok}{}\renewcommand{\seeorok}{\seedocumentationref{For more information about how the \ecs{} supports the OpenRISC 1000 hardware architecture}{or1k}{OpenRISC 1000 Hardware Architecture Support}}
\providecommand{\seeppc}{}\renewcommand{\seeppc}{\seedocumentationref{For more information about how the \ecs{} supports the PowerPC hardware architecture}{ppc}{PowerPC Hardware Architecture Support}}
\providecommand{\seerisc}{}\renewcommand{\seerisc}{\seedocumentationref{For more information about how the \ecs{} supports the RISC hardware architecture}{risc}{RISC Hardware Architecture Support}}
\providecommand{\seewasm}{}\renewcommand{\seewasm}{\seedocumentationref{For more information about how the \ecs{} supports the WebAssembly architecture}{wasm}{WebAssembly Architecture Support}}
\providecommand{\seedocumentation}{}\renewcommand{\seedocumentation}{\seedocumentationref{For more information about generic documentations and their generation by the \ecs{}}{documentation}{Generic Documentation Generation}}
\providecommand{\seedebugging}{}\renewcommand{\seedebugging}{\seedocumentationref{For more information about debugging information and its representation}{debugging}{Debugging Information Representation}}
\providecommand{\seecode}{}\renewcommand{\seecode}{\seedocumentationref{For more information about intermediate code and its purpose}{code}{Intermediate Code Representation}}
\providecommand{\seeobject}{}\renewcommand{\seeobject}{\seedocumentationref{For more information about object files and their purpose}{object}{Object File Representation}}

% generic documentation tools

\providecommand{\docprint}{
\toolsection{docprint} is a pretty printer for generic documentations.
It reformats generic documentations and writes it to the standard output stream.
\debuggingtool
\flowgraph{\resource{generic\\documentation} \ar[r] & \toolbox{docprint} \ar[r] & \resource{generic\\documentation}}
\seedocumentation
}

\providecommand{\doccheck}{
\toolsection{doccheck} is a syntactic and semantic checker for generic documentations.
It just performs syntactic and semantic checks on generic documentations and writes its diagnostic messages to the standard error stream.
\debuggingtool
\flowgraph{\resource{generic\\documentation} \ar[r] & \toolbox{doccheck} \ar[r] & \resource{diagnostic\\messages}}
\seedocumentation
}

\providecommand{\dochtml}{
\toolsection{dochtml} is an HTML documentation generator for generic documentations.
It processes several generic documentations and assembles all information therein into an HTML document.
\debuggingtool
\flowgraph{\resource{generic\\documentation} \ar[r] & \toolbox{dochtml} \ar[r] & \resource{HTML\\document}}
\seedocumentation
}

\providecommand{\doclatex}{
\toolsection{doclatex} is a Latex documentation generator for generic documentations.
It processes several generic documentations and assembles all information therein into a Latex document.
\debuggingtool
\flowgraph{\resource{generic\\documentation} \ar[r] & \toolbox{doclatex} \ar[r] & \resource{Latex\\document}}
\seedocumentation
}

% intermediate code tools

\providecommand{\cdcheck}{
\toolsection{cdcheck} is a syntactic and semantic checker for intermediate code.
It just performs syntactic and semantic checks on programs written in intermediate code and writes its diagnostic messages to the standard error stream.
\debuggingtool
\flowgraph{\resource{intermediate\\code} \ar[r] & \toolbox{cdcheck} \ar[r] & \resource{diagnostic\\messages}}
\seeassembly\seecode
}

\providecommand{\cdopt}{
\toolsection{cdopt} is an optimizer for intermediate code.
It performs various optimizations on programs written in intermediate code and writes the result to the standard output stream.
\debuggingtool
\flowgraph{\resource{intermediate\\code} \ar[r] & \toolbox{cdopt} \ar[r] & \resource{optimized\\code}}
\seeassembly\seecode
}

\providecommand{\cdrun}{
\toolsection{cdrun} is an interpreter for intermediate code.
It processes and executes programs written in intermediate code.
The following code sections are predefined and have the usual semantics:
\texttt{abort}, \texttt{\_Exit}, \texttt{fflush}, \texttt{floor}, \texttt{fputc}, \texttt{free}, \texttt{getchar}, \texttt{malloc}, and \texttt{putchar}.
Diagnostic messages about invalid operations include the name of the executed code section and the index of the erroneous instruction.
\debuggingtool
\flowgraph{\resource{intermediate\\code} \ar[r] & \toolbox{cdrun} \ar@/u/[r] & \resource{input/\\output} \ar@/d/[l]}
\seeassembly\seecode
}

\providecommand{\cdamda}{
\toolsection{cdamd16} is a compiler for intermediate code targeting the AMD64 hardware architecture.
It generates machine code for AMD64 processors from programs written in intermediate code and stores it in corresponding object files.
The compiler generates machine code for the 16-bit operating mode defined by the AMD64 architecture.
It also creates a debugging information file as well as an assembly file containing a listing of the generated machine code.
\debuggingtool
\flowgraph{\resource{intermediate\\code} \ar[r] & \toolbox{cdamd16} \ar[r] \ar[d] \ar[rd] & \resource{object file} \\ & \resource{assembly\\listing} & \resource{debugging\\information}}
\seeassembly\seeamd\seeobject\seecode\seedebugging
}

\providecommand{\cdamdb}{
\toolsection{cdamd32} is a compiler for intermediate code targeting the AMD64 hardware architecture.
It generates machine code for AMD64 processors from programs written in intermediate code and stores it in corresponding object files.
The compiler generates machine code for the 32-bit operating mode defined by the AMD64 architecture.
It also creates a debugging information file as well as an assembly file containing a listing of the generated machine code.
\debuggingtool
\flowgraph{\resource{intermediate\\code} \ar[r] & \toolbox{cdamd32} \ar[r] \ar[d] \ar[rd] & \resource{object file} \\ & \resource{assembly\\listing} & \resource{debugging\\information}}
\seeassembly\seeamd\seeobject\seecode\seedebugging
}

\providecommand{\cdamdc}{
\toolsection{cdamd64} is a compiler for intermediate code targeting the AMD64 hardware architecture.
It generates machine code for AMD64 processors from programs written in intermediate code and stores it in corresponding object files.
The compiler generates machine code for the 64-bit operating mode defined by the AMD64 architecture.
It also creates a debugging information file as well as an assembly file containing a listing of the generated machine code.
\debuggingtool
\flowgraph{\resource{intermediate\\code} \ar[r] & \toolbox{cdamd64} \ar[r] \ar[d] \ar[rd] & \resource{object file} \\ & \resource{assembly\\listing} & \resource{debugging\\information}}
\seeassembly\seeamd\seeobject\seecode\seedebugging
}

\providecommand{\cdarma}{
\toolsection{cdarma32} is a compiler for intermediate code targeting the ARM hardware architecture.
It generates machine code for ARM processors executing A32 instructions from programs written in intermediate code and stores it in corresponding object files.
It also creates a debugging information file as well as an assembly file containing a listing of the generated machine code.
\debuggingtool
\flowgraph{\resource{intermediate\\code} \ar[r] & \toolbox{cdarma32} \ar[r] \ar[d] \ar[rd] & \resource{object file} \\ & \resource{assembly\\listing} & \resource{debugging\\information}}
\seeassembly\seearm\seeobject\seecode\seedebugging
}

\providecommand{\cdarmb}{
\toolsection{cdarma64} is a compiler for intermediate code targeting the ARM hardware architecture.
It generates machine code for ARM processors executing A64 instructions from programs written in intermediate code and stores it in corresponding object files.
It also creates a debugging information file as well as an assembly file containing a listing of the generated machine code.
\debuggingtool
\flowgraph{\resource{intermediate\\code} \ar[r] & \toolbox{cdarma64} \ar[r] \ar[d] \ar[rd] & \resource{object file} \\ & \resource{assembly\\listing} & \resource{debugging\\information}}
\seeassembly\seearm\seeobject\seecode\seedebugging
}

\providecommand{\cdarmc}{
\toolsection{cdarmt32} is a compiler for intermediate code targeting the ARM hardware architecture.
It generates machine code for ARM processors without floating-point extension executing T32 instructions from programs written in intermediate code and stores it in corresponding object files.
It also creates a debugging information file as well as an assembly file containing a listing of the generated machine code.
\debuggingtool
\flowgraph{\resource{intermediate\\code} \ar[r] & \toolbox{cdarmt32} \ar[r] \ar[d] \ar[rd] & \resource{object file} \\ & \resource{assembly\\listing} & \resource{debugging\\information}}
\seeassembly\seearm\seeobject\seecode\seedebugging
}

\providecommand{\cdarmcfpe}{
\toolsection{cdarmt32fpe} is a compiler for intermediate code targeting the ARM hardware architecture.
It generates machine code for ARM processors with floating-point extension executing T32 instructions from programs written in intermediate code and stores it in corresponding object files.
It also creates a debugging information file as well as an assembly file containing a listing of the generated machine code.
\debuggingtool
\flowgraph{\resource{intermediate\\code} \ar[r] & \toolbox{cdarmt32fpe} \ar[r] \ar[d] \ar[rd] & \resource{object file} \\ & \resource{assembly\\listing} & \resource{debugging\\information}}
\seeassembly\seearm\seeobject\seecode\seedebugging
}

\providecommand{\cdavr}{
\toolsection{cdavr} is a compiler for intermediate code targeting the AVR hardware architecture.
It generates machine code for AVR processors from programs written in intermediate code and stores it in corresponding object files.
It also creates a debugging information file as well as an assembly file containing a listing of the generated machine code.
\debuggingtool
\flowgraph{\resource{intermediate\\code} \ar[r] & \toolbox{cdavr} \ar[r] \ar[d] \ar[rd] & \resource{object file} \\ & \resource{assembly\\listing} & \resource{debugging\\information}}
\seeassembly\seeavr\seeobject\seecode\seedebugging
}

\providecommand{\cdavrtt}{
\toolsection{cdavr32} is a compiler for intermediate code targeting the AVR32 hardware architecture.
It generates machine code for AVR32 processors from programs written in intermediate code and stores it in corresponding object files.
It also creates a debugging information file as well as an assembly file containing a listing of the generated machine code.
\debuggingtool
\flowgraph{\resource{intermediate\\code} \ar[r] & \toolbox{cdavr32} \ar[r] \ar[d] \ar[rd] & \resource{object file} \\ & \resource{assembly\\listing} & \resource{debugging\\information}}
\seeassembly\seeavrtt\seeobject\seecode\seedebugging
}

\providecommand{\cdmabk}{
\toolsection{cdm68k} is a compiler for intermediate code targeting the M68000 hardware architecture.
It generates machine code for M68000 processors from programs written in intermediate code and stores it in corresponding object files.
It also creates a debugging information file as well as an assembly file containing a listing of the generated machine code.
\debuggingtool
\flowgraph{\resource{intermediate\\code} \ar[r] & \toolbox{cdm68k} \ar[r] \ar[d] \ar[rd] & \resource{object file} \\ & \resource{assembly\\listing} & \resource{debugging\\information}}
\seeassembly\seemabk\seeobject\seecode\seedebugging
}

\providecommand{\cdmibl}{
\toolsection{cdmibl} is a compiler for intermediate code targeting the MicroBlaze hardware architecture.
It generates machine code for MicroBlaze processors from programs written in intermediate code and stores it in corresponding object files.
It also creates a debugging information file as well as an assembly file containing a listing of the generated machine code.
\debuggingtool
\flowgraph{\resource{intermediate\\code} \ar[r] & \toolbox{cdmibl} \ar[r] \ar[d] \ar[rd] & \resource{object file} \\ & \resource{assembly\\listing} & \resource{debugging\\information}}
\seeassembly\seemibl\seeobject\seecode\seedebugging
}

\providecommand{\cdmipsa}{
\toolsection{cdmips32} is a compiler for intermediate code targeting the MIPS32 hardware architecture.
It generates machine code for MIPS32 processors from programs written in intermediate code and stores it in corresponding object files.
It also creates a debugging information file as well as an assembly file containing a listing of the generated machine code.
\debuggingtool
\flowgraph{\resource{intermediate\\code} \ar[r] & \toolbox{cdmips32} \ar[r] \ar[d] \ar[rd] & \resource{object file} \\ & \resource{assembly\\listing} & \resource{debugging\\information}}
\seeassembly\seemips\seeobject\seecode\seedebugging
}

\providecommand{\cdmipsb}{
\toolsection{cdmips64} is a compiler for intermediate code targeting the MIPS64 hardware architecture.
It generates machine code for MIPS64 processors from programs written in intermediate code and stores it in corresponding object files.
It also creates a debugging information file as well as an assembly file containing a listing of the generated machine code.
\debuggingtool
\flowgraph{\resource{intermediate\\code} \ar[r] & \toolbox{cdmips64} \ar[r] \ar[d] \ar[rd] & \resource{object file} \\ & \resource{assembly\\listing} & \resource{debugging\\information}}
\seeassembly\seemips\seeobject\seecode\seedebugging
}

\providecommand{\cdmmix}{
\toolsection{cdmmix} is a compiler for intermediate code targeting the MMIX hardware architecture.
It generates machine code for MMIX processors from programs written in intermediate code and stores it in corresponding object files.
It also creates a debugging information file as well as an assembly file containing a listing of the generated machine code.
\debuggingtool
\flowgraph{\resource{intermediate\\code} \ar[r] & \toolbox{cdmmix} \ar[r] \ar[d] \ar[rd] & \resource{object file} \\ & \resource{assembly\\listing} & \resource{debugging\\information}}
\seeassembly\seemmix\seeobject\seecode\seedebugging
}

\providecommand{\cdorok}{
\toolsection{cdor1k} is a compiler for intermediate code targeting the OpenRISC 1000 hardware architecture.
It generates machine code for OpenRISC 1000 processors from programs written in intermediate code and stores it in corresponding object files.
It also creates a debugging information file as well as an assembly file containing a listing of the generated machine code.
\debuggingtool
\flowgraph{\resource{intermediate\\code} \ar[r] & \toolbox{cdor1k} \ar[r] \ar[d] \ar[rd] & \resource{object file} \\ & \resource{assembly\\listing} & \resource{debugging\\information}}
\seeassembly\seeorok\seeobject\seecode\seedebugging
}

\providecommand{\cdppca}{
\toolsection{cdppc32} is a compiler for intermediate code targeting the PowerPC hardware architecture.
It generates machine code for PowerPC processors from programs written in intermediate code and stores it in corresponding object files.
The compiler generates machine code for the 32-bit operating mode defined by the PowerPC architecture.
It also creates a debugging information file as well as an assembly file containing a listing of the generated machine code.
\debuggingtool
\flowgraph{\resource{intermediate\\code} \ar[r] & \toolbox{cdppc32} \ar[r] \ar[d] \ar[rd] & \resource{object file} \\ & \resource{assembly\\listing} & \resource{debugging\\information}}
\seeassembly\seeppc\seeobject\seecode\seedebugging
}

\providecommand{\cdppcb}{
\toolsection{cdppc64} is a compiler for intermediate code targeting the PowerPC hardware architecture.
It generates machine code for PowerPC processors from programs written in intermediate code and stores it in corresponding object files.
The compiler generates machine code for the 64-bit operating mode defined by the PowerPC architecture.
It also creates a debugging information file as well as an assembly file containing a listing of the generated machine code.
\debuggingtool
\flowgraph{\resource{intermediate\\code} \ar[r] & \toolbox{cdppc64} \ar[r] \ar[d] \ar[rd] & \resource{object file} \\ & \resource{assembly\\listing} & \resource{debugging\\information}}
\seeassembly\seeppc\seeobject\seecode\seedebugging
}

\providecommand{\cdrisc}{
\toolsection{cdrisc} is a compiler for intermediate code targeting the RISC hardware architecture.
It generates machine code for RISC processors from programs written in intermediate code and stores it in corresponding object files.
It also creates a debugging information file as well as an assembly file containing a listing of the generated machine code.
\debuggingtool
\flowgraph{\resource{intermediate\\code} \ar[r] & \toolbox{cdrisc} \ar[r] \ar[d] \ar[rd] & \resource{object file} \\ & \resource{assembly\\listing} & \resource{debugging\\information}}
\seeassembly\seerisc\seeobject\seecode\seedebugging
}

\providecommand{\cdwasm}{
\toolsection{cdwasm} is a compiler for intermediate code targeting the WebAssembly architecture.
It generates machine code for WebAssembly targets from programs written in intermediate code and stores it in corresponding object files.
It also creates a debugging information file as well as an assembly file containing a listing of the generated machine code.
\debuggingtool
\flowgraph{\resource{intermediate\\code} \ar[r] & \toolbox{cdwasm} \ar[r] \ar[d] \ar[rd] & \resource{object file} \\ & \resource{assembly\\listing} & \resource{debugging\\information}}
\seeassembly\seewasm\seeobject\seecode\seedebugging
}

% C++ tools

\providecommand{\cppprep}{
\toolsection{cppprep} is a preprocessor for the \cpp{} programming language.
It preprocesses source code according to the rules of \cpp{} and writes it to the standard output stream.
Only the macro names \texttt{\_\_DATE\_\_}, \texttt{\_\_FILE\_\_}, \texttt{\_\_LINE\_\_}, and \texttt{\_\_TIME\_\_} are predefined.
\flowgraph{\resource{\cpp{} or other\\source code} \ar[r] & \toolbox{cppprep} \ar[r] & \resource{preprocessed\\source code} \\ & \variable{ECSINCLUDE} \ar[u]}
\seecpp
}

\providecommand{\cppprint}{
\toolsection{cppprint} is a pretty printer for the \cpp{} programming language.
It reformats the source code of \cpp{} programs and writes it to the standard output stream.
\flowgraph{\resource{\cpp{}\\source code} \ar[r] & \toolbox{cppprint} \ar[r] & \resource{reformatted\\source code} \\ & \variable{ECSINCLUDE} \ar[u]}
\seecpp
}

\providecommand{\cppcheck}{
\toolsection{cppcheck} is a syntactic and semantic checker for the \cpp{} programming language.
It just performs syntactic and semantic checks on \cpp{} programs and writes its diagnostic messages to the standard error stream.
\flowgraph{\resource{\cpp{}\\source code} \ar[r] & \toolbox{cppcheck} \ar[r] & \resource{diagnostic\\messages} \\ & \variable{ECSINCLUDE} \ar[u]}
\seecpp
}

\providecommand{\cppdump}{
\toolsection{cppdump} is a serializer for the \cpp{} programming language.
It dumps the complete internal representation of programs written in \cpp{} into an XML document.
\debuggingtool
\flowgraph{\resource{\cpp{}\\source code} \ar[r] & \toolbox{cppdump} \ar[r] & \resource{internal\\representation} \\ & \variable{ECSINCLUDE} \ar[u]}
\seecpp
}

\providecommand{\cpprun}{
\toolsection{cpprun} is an interpreter for the \cpp{} programming language.
It processes and executes programs written in \cpp{}.
The macro \texttt{\_\_run\_\_} is predefined in order to enable programmers to identify this tool while interpreting.
\flowgraph{\resource{\cpp{}\\source code} \ar[r] & \toolbox{cpprun} \ar@/u/[r] & \resource{input/\\output} \ar@/d/[l] \\ & \variable{ECSINCLUDE} \ar[u]}
\seecpp
}

\providecommand{\cppdoc}{
\toolsection{cppdoc} is a generic documentation generator for the \cpp{} programming language.
It processes several \cpp{} source files and assembles all information therein into a generic documentation.
\debuggingtool
\flowgraph{\resource{\cpp{}\\source code} \ar[r] & \toolbox{cppdoc} \ar[r] & \resource{generic\\documentation} \\ & \variable{ECSINCLUDE} \ar[u]}
\seecpp\seedocumentation
}

\providecommand{\cpphtml}{
\toolsection{cpphtml} is an HTML documentation generator for the \cpp{} programming language.
It processes several \cpp{} source files and assembles all information therein into an HTML document.
\flowgraph{\resource{\cpp{}\\source code} \ar[r] & \toolbox{cpphtml} \ar[r] & \resource{HTML\\document} \\ & \variable{ECSINCLUDE} \ar[u]}
\seecpp\seedocumentation
}

\providecommand{\cpplatex}{
\toolsection{cpplatex} is a Latex documentation generator for the \cpp{} programming language.
It processes several \cpp{} source files and assembles all information therein into a Latex document.
\flowgraph{\resource{\cpp{}\\source code} \ar[r] & \toolbox{cpplatex} \ar[r] & \resource{Latex\\document} \\ & \variable{ECSINCLUDE} \ar[u]}
\seecpp\seedocumentation
}

\providecommand{\cppcode}{
\toolsection{cppcode} is an intermediate code generator for the \cpp{} programming language.
It generates intermediate code from programs written in \cpp{} and stores it in corresponding assembly files.
The macro \texttt{\_\_code\_\_} is predefined in order to enable programmers to identify this tool while generating intermediate code.
Programs generated with this tool require additional runtime support that is stored in the \file{cpp\-code\-run} library file.
\debuggingtool
\flowgraph{\resource{\cpp{}\\source code} \ar[r] & \toolbox{cppcode} \ar[r] & \resource{intermediate\\code} \\ & \variable{ECSINCLUDE} \ar[u]}
\seecpp\seeassembly\seecode
}

\providecommand{\cppamda}{
\toolsection{cppamd16} is a compiler for the \cpp{} programming language targeting the AMD64 hardware architecture.
It generates machine code for AMD64 processors from programs written in \cpp{} and stores it in corresponding object files.
The compiler generates machine code for the 16-bit operating mode defined by the AMD64 architecture.
For debugging purposes, it also creates a debugging information file as well as an assembly file containing a listing of the generated machine code.
The macro \texttt{\_\_amd16\_\_} is predefined in order to enable programmers to identify this tool and its target architecture while compiling.
Programs generated with this compiler require additional runtime support that is stored in the \file{cpp\-amd16\-run} library file.
\flowgraph{\resource{\cpp{}\\source code} \ar[r] & \toolbox{cppamd16} \ar[r] \ar[d] \ar[rd] & \resource{object file} \\ \variable{ECSINCLUDE} \ar[ru] & \resource{debugging\\information} & \resource{assembly\\listing}}
\seecpp\seeassembly\seeamd\seeobject\seedebugging
}

\providecommand{\cppamdb}{
\toolsection{cppamd32} is a compiler for the \cpp{} programming language targeting the AMD64 hardware architecture.
It generates machine code for AMD64 processors from programs written in \cpp{} and stores it in corresponding object files.
The compiler generates machine code for the 32-bit operating mode defined by the AMD64 architecture.
For debugging purposes, it also creates a debugging information file as well as an assembly file containing a listing of the generated machine code.
The macro \texttt{\_\_amd32\_\_} is predefined in order to enable programmers to identify this tool and its target architecture while compiling.
Programs generated with this compiler require additional runtime support that is stored in the \file{cpp\-amd32\-run} library file.
\flowgraph{\resource{\cpp{}\\source code} \ar[r] & \toolbox{cppamd32} \ar[r] \ar[d] \ar[rd] & \resource{object file} \\ \variable{ECSINCLUDE} \ar[ru] & \resource{debugging\\information} & \resource{assembly\\listing}}
\seecpp\seeassembly\seeamd\seeobject\seedebugging
}

\providecommand{\cppamdc}{
\toolsection{cppamd64} is a compiler for the \cpp{} programming language targeting the AMD64 hardware architecture.
It generates machine code for AMD64 processors from programs written in \cpp{} and stores it in corresponding object files.
The compiler generates machine code for the 64-bit operating mode defined by the AMD64 architecture.
For debugging purposes, it also creates a debugging information file as well as an assembly file containing a listing of the generated machine code.
The macro \texttt{\_\_amd64\_\_} is predefined in order to enable programmers to identify this tool and its target architecture while compiling.
Programs generated with this compiler require additional runtime support that is stored in the \file{cpp\-amd64\-run} library file.
\flowgraph{\resource{\cpp{}\\source code} \ar[r] & \toolbox{cppamd64} \ar[r] \ar[d] \ar[rd] & \resource{object file} \\ \variable{ECSINCLUDE} \ar[ru] & \resource{debugging\\information} & \resource{assembly\\listing}}
\seecpp\seeassembly\seeamd\seeobject\seedebugging
}

\providecommand{\cpparma}{
\toolsection{cpparma32} is a compiler for the \cpp{} programming language targeting the ARM hardware architecture.
It generates machine code for ARM processors executing A32 instructions from programs written in \cpp{} and stores it in corresponding object files.
For debugging purposes, it also creates a debugging information file as well as an assembly file containing a listing of the generated machine code.
The macro \texttt{\_\_arma32\_\_} is predefined in order to enable programmers to identify this tool and its target architecture while compiling.
Programs generated with this compiler require additional runtime support that is stored in the \file{cpp\-arma32\-run} library file.
\flowgraph{\resource{\cpp{}\\source code} \ar[r] & \toolbox{cpparma32} \ar[r] \ar[d] \ar[rd] & \resource{object file} \\ \variable{ECSINCLUDE} \ar[ru] & \resource{debugging\\information} & \resource{assembly\\listing}}
\seecpp\seeassembly\seearm\seeobject\seedebugging
}

\providecommand{\cpparmb}{
\toolsection{cpparma64} is a compiler for the \cpp{} programming language targeting the ARM hardware architecture.
It generates machine code for ARM processors executing A64 instructions from programs written in \cpp{} and stores it in corresponding object files.
For debugging purposes, it also creates a debugging information file as well as an assembly file containing a listing of the generated machine code.
The macro \texttt{\_\_arma64\_\_} is predefined in order to enable programmers to identify this tool and its target architecture while compiling.
Programs generated with this compiler require additional runtime support that is stored in the \file{cpp\-arma64\-run} library file.
\flowgraph{\resource{\cpp{}\\source code} \ar[r] & \toolbox{cpparma64} \ar[r] \ar[d] \ar[rd] & \resource{object file} \\ \variable{ECSINCLUDE} \ar[ru] & \resource{debugging\\information} & \resource{assembly\\listing}}
\seecpp\seeassembly\seearm\seeobject\seedebugging
}

\providecommand{\cpparmc}{
\toolsection{cpparmt32} is a compiler for the \cpp{} programming language targeting the ARM hardware architecture.
It generates machine code for ARM processors without floating-point extension executing T32 instructions from programs written in \cpp{} and stores it in corresponding object files.
For debugging purposes, it also creates a debugging information file as well as an assembly file containing a listing of the generated machine code.
The macro \texttt{\_\_armt32\_\_} is predefined in order to enable programmers to identify this tool and its target architecture while compiling.
Programs generated with this compiler require additional runtime support that is stored in the \file{cpp\-armt32\-run} library file.
\flowgraph{\resource{\cpp{}\\source code} \ar[r] & \toolbox{cpparmt32} \ar[r] \ar[d] \ar[rd] & \resource{object file} \\ \variable{ECSINCLUDE} \ar[ru] & \resource{debugging\\information} & \resource{assembly\\listing}}
\seecpp\seeassembly\seearm\seeobject\seedebugging
}

\providecommand{\cpparmcfpe}{
\toolsection{cpparmt32fpe} is a compiler for the \cpp{} programming language targeting the ARM hardware architecture.
It generates machine code for ARM processors with floating-point extension executing T32 instructions from programs written in \cpp{} and stores it in corresponding object files.
For debugging purposes, it also creates a debugging information file as well as an assembly file containing a listing of the generated machine code.
The macro \texttt{\_\_armt32fpe\_\_} is predefined in order to enable programmers to identify this tool and its target architecture while compiling.
Programs generated with this compiler require additional runtime support that is stored in the \file{cpp\-armt32\-fpe\-run} library file.
\flowgraph{\resource{\cpp{}\\source code} \ar[r] & \toolbox{cpparmt32fpe} \ar[r] \ar[d] \ar[rd] & \resource{object file} \\ \variable{ECSINCLUDE} \ar[ru] & \resource{debugging\\information} & \resource{assembly\\listing}}
\seecpp\seeassembly\seearm\seeobject\seedebugging
}

\providecommand{\cppavr}{
\toolsection{cppavr} is a compiler for the \cpp{} programming language targeting the AVR hardware architecture.
It generates machine code for AVR processors from programs written in \cpp{} and stores it in corresponding object files.
For debugging purposes, it also creates a debugging information file as well as an assembly file containing a listing of the generated machine code.
The macro \texttt{\_\_avr\_\_} is predefined in order to enable programmers to identify this tool and its target architecture while compiling.
Programs generated with this compiler require additional runtime support that is stored in the \file{cpp\-avr\-run} library file.
\flowgraph{\resource{\cpp{}\\source code} \ar[r] & \toolbox{cppavr} \ar[r] \ar[d] \ar[rd] & \resource{object file} \\ \variable{ECSINCLUDE} \ar[ru] & \resource{debugging\\information} & \resource{assembly\\listing}}
\seecpp\seeassembly\seeavr\seeobject\seedebugging
}

\providecommand{\cppavrtt}{
\toolsection{cppavr32} is a compiler for the \cpp{} programming language targeting the AVR32 hardware architecture.
It generates machine code for AVR32 processors from programs written in \cpp{} and stores it in corresponding object files.
For debugging purposes, it also creates a debugging information file as well as an assembly file containing a listing of the generated machine code.
The macro \texttt{\_\_avr32\_\_} is predefined in order to enable programmers to identify this tool and its target architecture while compiling.
Programs generated with this compiler require additional runtime support that is stored in the \file{cpp\-avr32\-run} library file.
\flowgraph{\resource{\cpp{}\\source code} \ar[r] & \toolbox{cppavr32} \ar[r] \ar[d] \ar[rd] & \resource{object file} \\ \variable{ECSINCLUDE} \ar[ru] & \resource{debugging\\information} & \resource{assembly\\listing}}
\seecpp\seeassembly\seeavrtt\seeobject\seedebugging
}

\providecommand{\cppmabk}{
\toolsection{cppm68k} is a compiler for the \cpp{} programming language targeting the M68000 hardware architecture.
It generates machine code for M68000 processors from programs written in \cpp{} and stores it in corresponding object files.
For debugging purposes, it also creates a debugging information file as well as an assembly file containing a listing of the generated machine code.
The macro \texttt{\_\_m68k\_\_} is predefined in order to enable programmers to identify this tool and its target architecture while compiling.
Programs generated with this compiler require additional runtime support that is stored in the \file{cpp\-m68k\-run} library file.
\flowgraph{\resource{\cpp{}\\source code} \ar[r] & \toolbox{cppm68k} \ar[r] \ar[d] \ar[rd] & \resource{object file} \\ \variable{ECSINCLUDE} \ar[ru] & \resource{debugging\\information} & \resource{assembly\\listing}}
\seecpp\seeassembly\seemabk\seeobject\seedebugging
}

\providecommand{\cppmibl}{
\toolsection{cppmibl} is a compiler for the \cpp{} programming language targeting the MicroBlaze hardware architecture.
It generates machine code for MicroBlaze processors from programs written in \cpp{} and stores it in corresponding object files.
For debugging purposes, it also creates a debugging information file as well as an assembly file containing a listing of the generated machine code.
The macro \texttt{\_\_mibl\_\_} is predefined in order to enable programmers to identify this tool and its target architecture while compiling.
Programs generated with this compiler require additional runtime support that is stored in the \file{cpp\-mibl\-run} library file.
\flowgraph{\resource{\cpp{}\\source code} \ar[r] & \toolbox{cppmibl} \ar[r] \ar[d] \ar[rd] & \resource{object file} \\ \variable{ECSINCLUDE} \ar[ru] & \resource{debugging\\information} & \resource{assembly\\listing}}
\seecpp\seeassembly\seemibl\seeobject\seedebugging
}

\providecommand{\cppmipsa}{
\toolsection{cppmips32} is a compiler for the \cpp{} programming language targeting the MIPS32 hardware architecture.
It generates machine code for MIPS32 processors from programs written in \cpp{} and stores it in corresponding object files.
For debugging purposes, it also creates a debugging information file as well as an assembly file containing a listing of the generated machine code.
The macro \texttt{\_\_mips32\_\_} is predefined in order to enable programmers to identify this tool and its target architecture while compiling.
Programs generated with this compiler require additional runtime support that is stored in the \file{cpp\-mips32\-run} library file.
\flowgraph{\resource{\cpp{}\\source code} \ar[r] & \toolbox{cppmips32} \ar[r] \ar[d] \ar[rd] & \resource{object file} \\ \variable{ECSINCLUDE} \ar[ru] & \resource{debugging\\information} & \resource{assembly\\listing}}
\seecpp\seeassembly\seemips\seeobject\seedebugging
}

\providecommand{\cppmipsb}{
\toolsection{cppmips64} is a compiler for the \cpp{} programming language targeting the MIPS64 hardware architecture.
It generates machine code for MIPS64 processors from programs written in \cpp{} and stores it in corresponding object files.
For debugging purposes, it also creates a debugging information file as well as an assembly file containing a listing of the generated machine code.
The macro \texttt{\_\_mips64\_\_} is predefined in order to enable programmers to identify this tool and its target architecture while compiling.
Programs generated with this compiler require additional runtime support that is stored in the \file{cpp\-mips64\-run} library file.
\flowgraph{\resource{\cpp{}\\source code} \ar[r] & \toolbox{cppmips64} \ar[r] \ar[d] \ar[rd] & \resource{object file} \\ \variable{ECSINCLUDE} \ar[ru] & \resource{debugging\\information} & \resource{assembly\\listing}}
\seecpp\seeassembly\seemips\seeobject\seedebugging
}

\providecommand{\cppmmix}{
\toolsection{cppmmix} is a compiler for the \cpp{} programming language targeting the MMIX hardware architecture.
It generates machine code for MMIX processors from programs written in \cpp{} and stores it in corresponding object files.
For debugging purposes, it also creates a debugging information file as well as an assembly file containing a listing of the generated machine code.
The macro \texttt{\_\_mmix\_\_} is predefined in order to enable programmers to identify this tool and its target architecture while compiling.
Programs generated with this compiler require additional runtime support that is stored in the \file{cpp\-mmix\-run} library file.
\flowgraph{\resource{\cpp{}\\source code} \ar[r] & \toolbox{cppmmix} \ar[r] \ar[d] \ar[rd] & \resource{object file} \\ \variable{ECSINCLUDE} \ar[ru] & \resource{debugging\\information} & \resource{assembly\\listing}}
\seecpp\seeassembly\seemmix\seeobject\seedebugging
}

\providecommand{\cpporok}{
\toolsection{cppor1k} is a compiler for the \cpp{} programming language targeting the OpenRISC 1000 hardware architecture.
It generates machine code for OpenRISC 1000 processors from programs written in \cpp{} and stores it in corresponding object files.
For debugging purposes, it also creates a debugging information file as well as an assembly file containing a listing of the generated machine code.
The macro \texttt{\_\_or1k\_\_} is predefined in order to enable programmers to identify this tool and its target architecture while compiling.
Programs generated with this compiler require additional runtime support that is stored in the \file{cpp\-or1k\-run} library file.
\flowgraph{\resource{\cpp{}\\source code} \ar[r] & \toolbox{cppor1k} \ar[r] \ar[d] \ar[rd] & \resource{object file} \\ \variable{ECSINCLUDE} \ar[ru] & \resource{debugging\\information} & \resource{assembly\\listing}}
\seecpp\seeassembly\seeorok\seeobject\seedebugging
}

\providecommand{\cppppca}{
\toolsection{cppppc32} is a compiler for the \cpp{} programming language targeting the PowerPC hardware architecture.
It generates machine code for PowerPC processors from programs written in \cpp{} and stores it in corresponding object files.
The compiler generates machine code for the 32-bit operating mode defined by the PowerPC architecture.
For debugging purposes, it also creates a debugging information file as well as an assembly file containing a listing of the generated machine code.
The macro \texttt{\_\_ppc32\_\_} is predefined in order to enable programmers to identify this tool and its target architecture while compiling.
Programs generated with this compiler require additional runtime support that is stored in the \file{cpp\-ppc32\-run} library file.
\flowgraph{\resource{\cpp{}\\source code} \ar[r] & \toolbox{cppppc32} \ar[r] \ar[d] \ar[rd] & \resource{object file} \\ \variable{ECSINCLUDE} \ar[ru] & \resource{debugging\\information} & \resource{assembly\\listing}}
\seecpp\seeassembly\seeppc\seeobject\seedebugging
}

\providecommand{\cppppcb}{
\toolsection{cppppc64} is a compiler for the \cpp{} programming language targeting the PowerPC hardware architecture.
It generates machine code for PowerPC processors from programs written in \cpp{} and stores it in corresponding object files.
The compiler generates machine code for the 64-bit operating mode defined by the PowerPC architecture.
For debugging purposes, it also creates a debugging information file as well as an assembly file containing a listing of the generated machine code.
The macro \texttt{\_\_ppc64\_\_} is predefined in order to enable programmers to identify this tool and its target architecture while compiling.
Programs generated with this compiler require additional runtime support that is stored in the \file{cpp\-ppc64\-run} library file.
\flowgraph{\resource{\cpp{}\\source code} \ar[r] & \toolbox{cppppc64} \ar[r] \ar[d] \ar[rd] & \resource{object file} \\ \variable{ECSINCLUDE} \ar[ru] & \resource{debugging\\information} & \resource{assembly\\listing}}
\seecpp\seeassembly\seeppc\seeobject\seedebugging
}

\providecommand{\cpprisc}{
\toolsection{cpprisc} is a compiler for the \cpp{} programming language targeting the RISC hardware architecture.
It generates machine code for RISC processors from programs written in \cpp{} and stores it in corresponding object files.
For debugging purposes, it also creates a debugging information file as well as an assembly file containing a listing of the generated machine code.
The macro \texttt{\_\_risc\_\_} is predefined in order to enable programmers to identify this tool and its target architecture while compiling.
Programs generated with this compiler require additional runtime support that is stored in the \file{cpp\-risc\-run} library file.
\flowgraph{\resource{\cpp{}\\source code} \ar[r] & \toolbox{cpprisc} \ar[r] \ar[d] \ar[rd] & \resource{object file} \\ \variable{ECSINCLUDE} \ar[ru] & \resource{debugging\\information} & \resource{assembly\\listing}}
\seecpp\seeassembly\seerisc\seeobject\seedebugging
}

\providecommand{\cppwasm}{
\toolsection{cppwasm} is a compiler for the \cpp{} programming language targeting the WebAssembly architecture.
It generates machine code for WebAssembly targets from programs written in \cpp{} and stores it in corresponding object files.
For debugging purposes, it also creates a debugging information file as well as an assembly file containing a listing of the generated machine code.
The macro \texttt{\_\_wasm\_\_} is predefined in order to enable programmers to identify this tool and its target architecture while compiling.
Programs generated with this compiler require additional runtime support that is stored in the \file{cpp\-wasm\-run} library file.
\flowgraph{\resource{\cpp{}\\source code} \ar[r] & \toolbox{cppwasm} \ar[r] \ar[d] \ar[rd] & \resource{object file} \\ \variable{ECSINCLUDE} \ar[ru] & \resource{debugging\\information} & \resource{assembly\\listing}}
\seecpp\seeassembly\seewasm\seeobject\seedebugging
}

% FALSE tools

\providecommand{\falprint}{
\toolsection{falprint} is a pretty printer for the FALSE programming language.
It reformats the source code of FALSE programs and writes it to the standard output stream.
\flowgraph{\resource{FALSE\\source code} \ar[r] & \toolbox{falprint} \ar[r] & \resource{reformatted\\source code}}
\seefalse
}

\providecommand{\falcheck}{
\toolsection{falcheck} is a syntactic and semantic checker for the FALSE programming language.
It just performs syntactic and semantic checks on FALSE programs and writes its diagnostic messages to the standard error stream.
\flowgraph{\resource{FALSE\\source code} \ar[r] & \toolbox{falcheck} \ar[r] & \resource{diagnostic\\messages}}
\seefalse
}

\providecommand{\faldump}{
\toolsection{faldump} is a serializer for the FALSE programming language.
It dumps the complete internal representation of programs written in FALSE into an XML document.
\debuggingtool
\flowgraph{\resource{FALSE\\source code} \ar[r] & \toolbox{faldump} \ar[r] & \resource{internal\\representation}}
\seefalse
}

\providecommand{\falrun}{
\toolsection{falrun} is an interpreter for the FALSE programming language.
It processes and executes programs written in FALSE\@.
\flowgraph{\resource{FALSE\\source code} \ar[r] & \toolbox{falrun} \ar@/u/[r] & \resource{input/\\output} \ar@/d/[l]}
\seefalse
}

\providecommand{\falcpp}{
\toolsection{falcpp} is a transpiler for the FALSE programming language.
It translates programs written in FALSE into \cpp{} programs and stores them in corresponding source files.
\flowgraph{\resource{FALSE\\source code} \ar[r] & \toolbox{falcpp} \ar[r] & \resource{\cpp{}\\source file}}
\seefalse\seecpp
}

\providecommand{\falcode}{
\toolsection{falcode} is an intermediate code generator for the FALSE programming language.
It generates intermediate code from programs written in FALSE and stores it in corresponding assembly files.
\debuggingtool
\flowgraph{\resource{FALSE\\source code} \ar[r] & \toolbox{falcode} \ar[r] & \resource{intermediate\\code}}
\seefalse\seeassembly\seecode
}

\providecommand{\falamda}{
\toolsection{falamd16} is a compiler for the FALSE programming language targeting the AMD64 hardware architecture.
It generates machine code for AMD64 processors from programs written in FALSE and stores it in corresponding object files.
The compiler generates machine code for the 16-bit operating mode defined by the AMD64 architecture.
\flowgraph{\resource{FALSE\\source code} \ar[r] & \toolbox{falamd16} \ar[r] & \resource{object file}}
\seefalse\seeamd\seeobject
}

\providecommand{\falamdb}{
\toolsection{falamd32} is a compiler for the FALSE programming language targeting the AMD64 hardware architecture.
It generates machine code for AMD64 processors from programs written in FALSE and stores it in corresponding object files.
The compiler generates machine code for the 32-bit operating mode defined by the AMD64 architecture.
\flowgraph{\resource{FALSE\\source code} \ar[r] & \toolbox{falamd32} \ar[r] & \resource{object file}}
\seefalse\seeamd\seeobject
}

\providecommand{\falamdc}{
\toolsection{falamd64} is a compiler for the FALSE programming language targeting the AMD64 hardware architecture.
It generates machine code for AMD64 processors from programs written in FALSE and stores it in corresponding object files.
The compiler generates machine code for the 64-bit operating mode defined by the AMD64 architecture.
\flowgraph{\resource{FALSE\\source code} \ar[r] & \toolbox{falamd64} \ar[r] & \resource{object file}}
\seefalse\seeamd\seeobject
}

\providecommand{\falarma}{
\toolsection{falarma32} is a compiler for the FALSE programming language targeting the ARM hardware architecture.
It generates machine code for ARM processors executing A32 instructions from programs written in FALSE and stores it in corresponding object files.
\flowgraph{\resource{FALSE\\source code} \ar[r] & \toolbox{falarma32} \ar[r] & \resource{object file}}
\seefalse\seearm\seeobject
}

\providecommand{\falarmb}{
\toolsection{falarma64} is a compiler for the FALSE programming language targeting the ARM hardware architecture.
It generates machine code for ARM processors executing A64 instructions from programs written in FALSE and stores it in corresponding object files.
\flowgraph{\resource{FALSE\\source code} \ar[r] & \toolbox{falarma64} \ar[r] & \resource{object file}}
\seefalse\seearm\seeobject
}

\providecommand{\falarmc}{
\toolsection{falarmt32} is a compiler for the FALSE programming language targeting the ARM hardware architecture.
It generates machine code for ARM processors without floating-point extension executing T32 instructions from programs written in FALSE and stores it in corresponding object files.
\flowgraph{\resource{FALSE\\source code} \ar[r] & \toolbox{falarmt32} \ar[r] & \resource{object file}}
\seefalse\seearm\seeobject
}

\providecommand{\falarmcfpe}{
\toolsection{falarmt32fpe} is a compiler for the FALSE programming language targeting the ARM hardware architecture.
It generates machine code for ARM processors with floating-point extension executing T32 instructions from programs written in FALSE and stores it in corresponding object files.
\flowgraph{\resource{FALSE\\source code} \ar[r] & \toolbox{falarmt32fpe} \ar[r] & \resource{object file}}
\seefalse\seearm\seeobject
}

\providecommand{\falavr}{
\toolsection{falavr} is a compiler for the FALSE programming language targeting the AVR hardware architecture.
It generates machine code for AVR processors from programs written in FALSE and stores it in corresponding object files.
\flowgraph{\resource{FALSE\\source code} \ar[r] & \toolbox{falavr} \ar[r] & \resource{object file}}
\seefalse\seeavr\seeobject
}

\providecommand{\falavrtt}{
\toolsection{falavr32} is a compiler for the FALSE programming language targeting the AVR32 hardware architecture.
It generates machine code for AVR32 processors from programs written in FALSE and stores it in corresponding object files.
\flowgraph{\resource{FALSE\\source code} \ar[r] & \toolbox{falavr32} \ar[r] & \resource{object file}}
\seefalse\seeavrtt\seeobject
}

\providecommand{\falmabk}{
\toolsection{falm68k} is a compiler for the FALSE programming language targeting the M68000 hardware architecture.
It generates machine code for M68000 processors from programs written in FALSE and stores it in corresponding object files.
\flowgraph{\resource{FALSE\\source code} \ar[r] & \toolbox{falm68k} \ar[r] & \resource{object file}}
\seefalse\seemabk\seeobject
}

\providecommand{\falmibl}{
\toolsection{falmibl} is a compiler for the FALSE programming language targeting the MicroBlaze hardware architecture.
It generates machine code for MicroBlaze processors from programs written in FALSE and stores it in corresponding object files.
\flowgraph{\resource{FALSE\\source code} \ar[r] & \toolbox{falmibl} \ar[r] & \resource{object file}}
\seefalse\seemibl\seeobject
}

\providecommand{\falmipsa}{
\toolsection{falmips32} is a compiler for the FALSE programming language targeting the MIPS32 hardware architecture.
It generates machine code for MIPS32 processors from programs written in FALSE and stores it in corresponding object files.
\flowgraph{\resource{FALSE\\source code} \ar[r] & \toolbox{falmips32} \ar[r] & \resource{object file}}
\seefalse\seemips\seeobject
}

\providecommand{\falmipsb}{
\toolsection{falmips64} is a compiler for the FALSE programming language targeting the MIPS64 hardware architecture.
It generates machine code for MIPS64 processors from programs written in FALSE and stores it in corresponding object files.
\flowgraph{\resource{FALSE\\source code} \ar[r] & \toolbox{falmips64} \ar[r] & \resource{object file}}
\seefalse\seemips\seeobject
}

\providecommand{\falmmix}{
\toolsection{falmmix} is a compiler for the FALSE programming language targeting the MMIX hardware architecture.
It generates machine code for MMIX processors from programs written in FALSE and stores it in corresponding object files.
\flowgraph{\resource{FALSE\\source code} \ar[r] & \toolbox{falmmix} \ar[r] & \resource{object file}}
\seefalse\seemmix\seeobject
}

\providecommand{\falorok}{
\toolsection{falor1k} is a compiler for the FALSE programming language targeting the OpenRISC 1000 hardware architecture.
It generates machine code for OpenRISC 1000 processors from programs written in FALSE and stores it in corresponding object files.
\flowgraph{\resource{FALSE\\source code} \ar[r] & \toolbox{falor1k} \ar[r] & \resource{object file}}
\seefalse\seeorok\seeobject
}

\providecommand{\falppca}{
\toolsection{falppc32} is a compiler for the FALSE programming language targeting the PowerPC hardware architecture.
It generates machine code for PowerPC processors from programs written in FALSE and stores it in corresponding object files.
The compiler generates machine code for the 32-bit operating mode defined by the PowerPC architecture.
\flowgraph{\resource{FALSE\\source code} \ar[r] & \toolbox{falppc32} \ar[r] & \resource{object file}}
\seefalse\seeppc\seeobject
}

\providecommand{\falppcb}{
\toolsection{falppc64} is a compiler for the FALSE programming language targeting the PowerPC hardware architecture.
It generates machine code for PowerPC processors from programs written in FALSE and stores it in corresponding object files.
The compiler generates machine code for the 64-bit operating mode defined by the PowerPC architecture.
\flowgraph{\resource{FALSE\\source code} \ar[r] & \toolbox{falppc64} \ar[r] & \resource{object file}}
\seefalse\seeppc\seeobject
}

\providecommand{\falrisc}{
\toolsection{falrisc} is a compiler for the FALSE programming language targeting the RISC hardware architecture.
It generates machine code for RISC processors from programs written in FALSE and stores it in corresponding object files.
\flowgraph{\resource{FALSE\\source code} \ar[r] & \toolbox{falrisc} \ar[r] & \resource{object file}}
\seefalse\seerisc\seeobject
}

\providecommand{\falwasm}{
\toolsection{falwasm} is a compiler for the FALSE programming language targeting the WebAssembly architecture.
It generates machine code for WebAssembly targets from programs written in FALSE and stores it in corresponding object files.
\flowgraph{\resource{FALSE\\source code} \ar[r] & \toolbox{falwasm} \ar[r] & \resource{object file}}
\seefalse\seewasm\seeobject
}

% Oberon tools

\providecommand{\obprint}{
\toolsection{obprint} is a pretty printer for the Oberon programming language.
It reformats the source code of Oberon modules and writes it to the standard output stream.
\flowgraph{\resource{Oberon\\source code} \ar[r] & \toolbox{obprint} \ar[r] & \resource{reformatted\\source code}}
\seeoberon
}

\providecommand{\obcheck}{
\toolsection{obcheck} is a syntactic and semantic checker for the Oberon programming language.
It just performs syntactic and semantic checks on Oberon modules and writes its diagnostic messages to the standard error stream.
In addition, it stores the interface of each module in a symbol file which is required when other modules import the module.
\flowgraph{\resource{Oberon\\source code} \ar[r] & \toolbox{obcheck} \ar[r] \ar@/l/[d] & \resource{diagnostic\\messages} \\ \variable{ECSIMPORT} \ar[ru] & \resource{symbol\\files} \ar@/r/[u]}
\seeoberon
}

\providecommand{\obdump}{
\toolsection{obdump} is a serializer for the Oberon programming language.
It dumps the complete internal representation of modules written in Oberon into an XML document.
\debuggingtool
\flowgraph{\resource{Oberon\\source code} \ar[r] & \toolbox{obdump} \ar[r] \ar@/l/[d] & \resource{internal\\representation} \\ \variable{ECSIMPORT} \ar[ru] & \resource{symbol\\files} \ar@/r/[u]}
\seeoberon
}

\providecommand{\obrun}{
\toolsection{obrun} is an interpreter for the Oberon programming language.
It processes and executes modules written in Oberon.
This tool does neither generate nor process symbol files while interpreting modules.
If a module is imported by another one, its filename has to be named before the other one in the list of command-line arguments.
\flowgraph{\resource{Oberon\\source code} \ar[r] & \toolbox{obrun} \ar@/u/[r] & \resource{input/\\output} \ar@/d/[l]}
\seeoberon
}

\providecommand{\obcpp}{
\toolsection{obcpp} is a transpiler for the Oberon programming language.
It translates programs written in Oberon into \cpp{} programs and stores them in corresponding source and header files.
In addition, it stores the interface of each module in a symbol file which is required when other modules import the module.
The same interface is provided by the generated header file which can be used in other parts of the \cpp{} program.
\flowgraph{\resource{Oberon\\source code} \ar[r] & \toolbox{obcpp} \ar[r] \ar@/l/[d] \ar[rd] & \resource{\cpp{}\\source file} \\ \variable{ECSIMPORT} \ar[ru] & \resource{symbol\\files} \ar@/r/[u] & \resource{\cpp{}\\header file}}
\seeoberon\seecpp
}

\providecommand{\obdoc}{
\toolsection{obdoc} is a generic documentation generator for the Oberon programming language.
It processes several Oberon modules and assembles all information therein into a generic documentation.
In addition, it stores the interface of each module in a symbol file which is required when other modules import the module.
\debuggingtool
\flowgraph{\resource{Oberon\\source code} \ar[r] & \toolbox{obdoc} \ar[r] \ar@/l/[d] & \resource{generic\\documentation} \\ \variable{ECSIMPORT} \ar[ru] & \resource{symbol\\files} \ar@/r/[u]}
\seeoberon\seedocumentation
}

\providecommand{\obhtml}{
\toolsection{obhtml} is an HTML documentation generator for the Oberon programming language.
It processes several Oberon modules and assembles all information therein into an HTML document.
In addition, it stores the interface of each module in a symbol file which is required when other modules import the module.
\flowgraph{\resource{Oberon\\source code} \ar[r] & \toolbox{obhtml} \ar[r] \ar@/l/[d] & \resource{HTML\\document} \\ \variable{ECSIMPORT} \ar[ru] & \resource{symbol\\files} \ar@/r/[u]}
\seeoberon\seedocumentation
}

\providecommand{\oblatex}{
\toolsection{oblatex} is a Latex documentation generator for the Oberon programming language.
It processes several Oberon modules and assembles all information therein into a Latex document.
In addition, it stores the interface of each module in a symbol file which is required when other modules import the module.
\flowgraph{\resource{Oberon\\source code} \ar[r] & \toolbox{oblatex} \ar[r] \ar@/l/[d] & \resource{Latex\\document} \\ \variable{ECSIMPORT} \ar[ru] & \resource{symbol\\files} \ar@/r/[u]}
\seeoberon\seedocumentation
}

\providecommand{\obcode}{
\toolsection{obcode} is an intermediate code generator for the Oberon programming language.
It generates intermediate code from modules written in Oberon and stores it in corresponding assembly files.
In addition, it stores the interface of each module in a symbol file which is required when other modules import the module.
Programs generated with this tool require additional runtime support that is stored in the \file{ob\-code\-run} library file.
\debuggingtool
\flowgraph{\resource{Oberon\\source code} \ar[r] & \toolbox{obcode} \ar[r] \ar@/l/[d] & \resource{intermediate\\code} \\ \variable{ECSIMPORT} \ar[ru] & \resource{symbol\\files} \ar@/r/[u]}
\seeoberon\seeassembly\seecode
}

\providecommand{\obamda}{
\toolsection{obamd16} is a compiler for the Oberon programming language targeting the AMD64 hardware architecture.
It generates machine code for AMD64 processors from modules written in Oberon and stores it in corresponding object files.
The compiler generates machine code for the 16-bit operating mode defined by the AMD64 architecture.
For debugging purposes, it also creates a debugging information file as well as an assembly file containing a listing of the generated machine code.
In addition, it stores the interface of each module in a symbol file which is required when other modules import the module.
Programs generated with this compiler require additional runtime support that is stored in the \file{ob\-amd16\-run} library file.
\flowgraph{\resource{Oberon\\source code} \ar[r] & \toolbox{obamd16} \ar[r] \ar@/l/[d] \ar[rd] & \resource{object file} \\ \variable{ECSIMPORT} \ar[ru] & \resource{symbol\\files} \ar@/r/[u] & \resource{debugging\\information}}
\seeoberon\seeassembly\seeamd\seeobject\seedebugging
}

\providecommand{\obamdb}{
\toolsection{obamd32} is a compiler for the Oberon programming language targeting the AMD64 hardware architecture.
It generates machine code for AMD64 processors from modules written in Oberon and stores it in corresponding object files.
The compiler generates machine code for the 32-bit operating mode defined by the AMD64 architecture.
For debugging purposes, it also creates a debugging information file as well as an assembly file containing a listing of the generated machine code.
In addition, it stores the interface of each module in a symbol file which is required when other modules import the module.
Programs generated with this compiler require additional runtime support that is stored in the \file{ob\-amd32\-run} library file.
\flowgraph{\resource{Oberon\\source code} \ar[r] & \toolbox{obamd32} \ar[r] \ar@/l/[d] \ar[rd] & \resource{object file} \\ \variable{ECSIMPORT} \ar[ru] & \resource{symbol\\files} \ar@/r/[u] & \resource{debugging\\information}}
\seeoberon\seeassembly\seeamd\seeobject\seedebugging
}

\providecommand{\obamdc}{
\toolsection{obamd64} is a compiler for the Oberon programming language targeting the AMD64 hardware architecture.
It generates machine code for AMD64 processors from modules written in Oberon and stores it in corresponding object files.
The compiler generates machine code for the 64-bit operating mode defined by the AMD64 architecture.
For debugging purposes, it also creates a debugging information file as well as an assembly file containing a listing of the generated machine code.
In addition, it stores the interface of each module in a symbol file which is required when other modules import the module.
Programs generated with this compiler require additional runtime support that is stored in the \file{ob\-amd64\-run} library file.
\flowgraph{\resource{Oberon\\source code} \ar[r] & \toolbox{obamd64} \ar[r] \ar@/l/[d] \ar[rd] & \resource{object file} \\ \variable{ECSIMPORT} \ar[ru] & \resource{symbol\\files} \ar@/r/[u] & \resource{debugging\\information}}
\seeoberon\seeassembly\seeamd\seeobject\seedebugging
}

\providecommand{\obarma}{
\toolsection{obarma32} is a compiler for the Oberon programming language targeting the ARM hardware architecture.
It generates machine code for ARM processors executing A32 instructions from modules written in Oberon and stores it in corresponding object files.
For debugging purposes, it also creates a debugging information file as well as an assembly file containing a listing of the generated machine code.
In addition, it stores the interface of each module in a symbol file which is required when other modules import the module.
Programs generated with this compiler require additional runtime support that is stored in the \file{ob\-arma32\-run} library file.
\flowgraph{\resource{Oberon\\source code} \ar[r] & \toolbox{obarma32} \ar[r] \ar@/l/[d] \ar[rd] & \resource{object file} \\ \variable{ECSIMPORT} \ar[ru] & \resource{symbol\\files} \ar@/r/[u] & \resource{debugging\\information}}
\seeoberon\seeassembly\seearm\seeobject\seedebugging
}

\providecommand{\obarmb}{
\toolsection{obarma64} is a compiler for the Oberon programming language targeting the ARM hardware architecture.
It generates machine code for ARM processors executing A64 instructions from modules written in Oberon and stores it in corresponding object files.
For debugging purposes, it also creates a debugging information file as well as an assembly file containing a listing of the generated machine code.
In addition, it stores the interface of each module in a symbol file which is required when other modules import the module.
Programs generated with this compiler require additional runtime support that is stored in the \file{ob\-arma64\-run} library file.
\flowgraph{\resource{Oberon\\source code} \ar[r] & \toolbox{obarma64} \ar[r] \ar@/l/[d] \ar[rd] & \resource{object file} \\ \variable{ECSIMPORT} \ar[ru] & \resource{symbol\\files} \ar@/r/[u] & \resource{debugging\\information}}
\seeoberon\seeassembly\seearm\seeobject\seedebugging
}

\providecommand{\obarmc}{
\toolsection{obarmt32} is a compiler for the Oberon programming language targeting the ARM hardware architecture.
It generates machine code for ARM processors without floating-point extension executing T32 instructions from modules written in Oberon and stores it in corresponding object files.
For debugging purposes, it also creates a debugging information file as well as an assembly file containing a listing of the generated machine code.
In addition, it stores the interface of each module in a symbol file which is required when other modules import the module.
Programs generated with this compiler require additional runtime support that is stored in the \file{ob\-armt32\-run} library file.
\flowgraph{\resource{Oberon\\source code} \ar[r] & \toolbox{obarmt32} \ar[r] \ar@/l/[d] \ar[rd] & \resource{object file} \\ \variable{ECSIMPORT} \ar[ru] & \resource{symbol\\files} \ar@/r/[u] & \resource{debugging\\information}}
\seeoberon\seeassembly\seearm\seeobject\seedebugging
}

\providecommand{\obarmcfpe}{
\toolsection{obarmt32fpe} is a compiler for the Oberon programming language targeting the ARM hardware architecture.
It generates machine code for ARM processors with floating-point extension executing T32 instructions from modules written in Oberon and stores it in corresponding object files.
For debugging purposes, it also creates a debugging information file as well as an assembly file containing a listing of the generated machine code.
In addition, it stores the interface of each module in a symbol file which is required when other modules import the module.
Programs generated with this compiler require additional runtime support that is stored in the \file{ob\-armt32\-fpe\-run} library file.
\flowgraph{\resource{Oberon\\source code} \ar[r] & \toolbox{obarmt32fpe} \ar[r] \ar@/l/[d] \ar[rd] & \resource{object file} \\ \variable{ECSIMPORT} \ar[ru] & \resource{symbol\\files} \ar@/r/[u] & \resource{debugging\\information}}
\seeoberon\seeassembly\seearm\seeobject\seedebugging
}

\providecommand{\obavr}{
\toolsection{obavr} is a compiler for the Oberon programming language targeting the AVR hardware architecture.
It generates machine code for AVR processors from modules written in Oberon and stores it in corresponding object files.
For debugging purposes, it also creates a debugging information file as well as an assembly file containing a listing of the generated machine code.
In addition, it stores the interface of each module in a symbol file which is required when other modules import the module.
Programs generated with this compiler require additional runtime support that is stored in the \file{ob\-avr\-run} library file.
\flowgraph{\resource{Oberon\\source code} \ar[r] & \toolbox{obavr} \ar[r] \ar@/l/[d] \ar[rd] & \resource{object file} \\ \variable{ECSIMPORT} \ar[ru] & \resource{symbol\\files} \ar@/r/[u] & \resource{debugging\\information}}
\seeoberon\seeassembly\seeavr\seeobject\seedebugging
}

\providecommand{\obavrtt}{
\toolsection{obavr32} is a compiler for the Oberon programming language targeting the AVR32 hardware architecture.
It generates machine code for AVR32 processors from modules written in Oberon and stores it in corresponding object files.
For debugging purposes, it also creates a debugging information file as well as an assembly file containing a listing of the generated machine code.
In addition, it stores the interface of each module in a symbol file which is required when other modules import the module.
Programs generated with this compiler require additional runtime support that is stored in the \file{ob\-avr32\-run} library file.
\flowgraph{\resource{Oberon\\source code} \ar[r] & \toolbox{obavr32} \ar[r] \ar@/l/[d] \ar[rd] & \resource{object file} \\ \variable{ECSIMPORT} \ar[ru] & \resource{symbol\\files} \ar@/r/[u] & \resource{debugging\\information}}
\seeoberon\seeassembly\seeavrtt\seeobject\seedebugging
}

\providecommand{\obmabk}{
\toolsection{obm68k} is a compiler for the Oberon programming language targeting the M68000 hardware architecture.
It generates machine code for M68000 processors from modules written in Oberon and stores it in corresponding object files.
For debugging purposes, it also creates a debugging information file as well as an assembly file containing a listing of the generated machine code.
In addition, it stores the interface of each module in a symbol file which is required when other modules import the module.
Programs generated with this compiler require additional runtime support that is stored in the \file{ob\-m68k\-run} library file.
\flowgraph{\resource{Oberon\\source code} \ar[r] & \toolbox{obm68k} \ar[r] \ar@/l/[d] \ar[rd] & \resource{object file} \\ \variable{ECSIMPORT} \ar[ru] & \resource{symbol\\files} \ar@/r/[u] & \resource{debugging\\information}}
\seeoberon\seeassembly\seemabk\seeobject\seedebugging
}

\providecommand{\obmibl}{
\toolsection{obmibl} is a compiler for the Oberon programming language targeting the MicroBlaze hardware architecture.
It generates machine code for MicroBlaze processors from modules written in Oberon and stores it in corresponding object files.
For debugging purposes, it also creates a debugging information file as well as an assembly file containing a listing of the generated machine code.
In addition, it stores the interface of each module in a symbol file which is required when other modules import the module.
Programs generated with this compiler require additional runtime support that is stored in the \file{ob\-mibl\-run} library file.
\flowgraph{\resource{Oberon\\source code} \ar[r] & \toolbox{obmibl} \ar[r] \ar@/l/[d] \ar[rd] & \resource{object file} \\ \variable{ECSIMPORT} \ar[ru] & \resource{symbol\\files} \ar@/r/[u] & \resource{debugging\\information}}
\seeoberon\seeassembly\seemibl\seeobject\seedebugging
}

\providecommand{\obmipsa}{
\toolsection{obmips32} is a compiler for the Oberon programming language targeting the MIPS32 hardware architecture.
It generates machine code for MIPS32 processors from modules written in Oberon and stores it in corresponding object files.
For debugging purposes, it also creates a debugging information file as well as an assembly file containing a listing of the generated machine code.
In addition, it stores the interface of each module in a symbol file which is required when other modules import the module.
Programs generated with this compiler require additional runtime support that is stored in the \file{ob\-mips32\-run} library file.
\flowgraph{\resource{Oberon\\source code} \ar[r] & \toolbox{obmips32} \ar[r] \ar@/l/[d] \ar[rd] & \resource{object file} \\ \variable{ECSIMPORT} \ar[ru] & \resource{symbol\\files} \ar@/r/[u] & \resource{debugging\\information}}
\seeoberon\seeassembly\seemips\seeobject\seedebugging
}

\providecommand{\obmipsb}{
\toolsection{obmips64} is a compiler for the Oberon programming language targeting the MIPS64 hardware architecture.
It generates machine code for MIPS64 processors from modules written in Oberon and stores it in corresponding object files.
For debugging purposes, it also creates a debugging information file as well as an assembly file containing a listing of the generated machine code.
In addition, it stores the interface of each module in a symbol file which is required when other modules import the module.
Programs generated with this compiler require additional runtime support that is stored in the \file{ob\-mips64\-run} library file.
\flowgraph{\resource{Oberon\\source code} \ar[r] & \toolbox{obmips64} \ar[r] \ar@/l/[d] \ar[rd] & \resource{object file} \\ \variable{ECSIMPORT} \ar[ru] & \resource{symbol\\files} \ar@/r/[u] & \resource{debugging\\information}}
\seeoberon\seeassembly\seemips\seeobject\seedebugging
}

\providecommand{\obmmix}{
\toolsection{obmmix} is a compiler for the Oberon programming language targeting the MMIX hardware architecture.
It generates machine code for MMIX processors from modules written in Oberon and stores it in corresponding object files.
For debugging purposes, it also creates a debugging information file as well as an assembly file containing a listing of the generated machine code.
In addition, it stores the interface of each module in a symbol file which is required when other modules import the module.
Programs generated with this compiler require additional runtime support that is stored in the \file{ob\-mmix\-run} library file.
\flowgraph{\resource{Oberon\\source code} \ar[r] & \toolbox{obmmix} \ar[r] \ar@/l/[d] \ar[rd] & \resource{object file} \\ \variable{ECSIMPORT} \ar[ru] & \resource{symbol\\files} \ar@/r/[u] & \resource{debugging\\information}}
\seeoberon\seeassembly\seemmix\seeobject\seedebugging
}

\providecommand{\oborok}{
\toolsection{obor1k} is a compiler for the Oberon programming language targeting the OpenRISC 1000 hardware architecture.
It generates machine code for OpenRISC 1000 processors from modules written in Oberon and stores it in corresponding object files.
For debugging purposes, it also creates a debugging information file as well as an assembly file containing a listing of the generated machine code.
In addition, it stores the interface of each module in a symbol file which is required when other modules import the module.
Programs generated with this compiler require additional runtime support that is stored in the \file{ob\-or1k\-run} library file.
\flowgraph{\resource{Oberon\\source code} \ar[r] & \toolbox{obor1k} \ar[r] \ar@/l/[d] \ar[rd] & \resource{object file} \\ \variable{ECSIMPORT} \ar[ru] & \resource{symbol\\files} \ar@/r/[u] & \resource{debugging\\information}}
\seeoberon\seeassembly\seeorok\seeobject\seedebugging
}

\providecommand{\obppca}{
\toolsection{obppc32} is a compiler for the Oberon programming language targeting the PowerPC hardware architecture.
It generates machine code for PowerPC processors from modules written in Oberon and stores it in corresponding object files.
The compiler generates machine code for the 32-bit operating mode defined by the PowerPC architecture.
For debugging purposes, it also creates a debugging information file as well as an assembly file containing a listing of the generated machine code.
In addition, it stores the interface of each module in a symbol file which is required when other modules import the module.
Programs generated with this compiler require additional runtime support that is stored in the \file{ob\-ppc32\-run} library file.
\flowgraph{\resource{Oberon\\source code} \ar[r] & \toolbox{obppc32} \ar[r] \ar@/l/[d] \ar[rd] & \resource{object file} \\ \variable{ECSIMPORT} \ar[ru] & \resource{symbol\\files} \ar@/r/[u] & \resource{debugging\\information}}
\seeoberon\seeassembly\seeppc\seeobject\seedebugging
}

\providecommand{\obppcb}{
\toolsection{obppc64} is a compiler for the Oberon programming language targeting the PowerPC hardware architecture.
It generates machine code for PowerPC processors from modules written in Oberon and stores it in corresponding object files.
The compiler generates machine code for the 64-bit operating mode defined by the PowerPC architecture.
For debugging purposes, it also creates a debugging information file as well as an assembly file containing a listing of the generated machine code.
In addition, it stores the interface of each module in a symbol file which is required when other modules import the module.
Programs generated with this compiler require additional runtime support that is stored in the \file{ob\-ppc64\-run} library file.
\flowgraph{\resource{Oberon\\source code} \ar[r] & \toolbox{obppc64} \ar[r] \ar@/l/[d] \ar[rd] & \resource{object file} \\ \variable{ECSIMPORT} \ar[ru] & \resource{symbol\\files} \ar@/r/[u] & \resource{debugging\\information}}
\seeoberon\seeassembly\seeppc\seeobject\seedebugging
}

\providecommand{\obrisc}{
\toolsection{obrisc} is a compiler for the Oberon programming language targeting the RISC hardware architecture.
It generates machine code for RISC processors from modules written in Oberon and stores it in corresponding object files.
For debugging purposes, it also creates a debugging information file as well as an assembly file containing a listing of the generated machine code.
In addition, it stores the interface of each module in a symbol file which is required when other modules import the module.
Programs generated with this compiler require additional runtime support that is stored in the \file{ob\-risc\-run} library file.
\flowgraph{\resource{Oberon\\source code} \ar[r] & \toolbox{obrisc} \ar[r] \ar@/l/[d] \ar[rd] & \resource{object file} \\ \variable{ECSIMPORT} \ar[ru] & \resource{symbol\\files} \ar@/r/[u] & \resource{debugging\\information}}
\seeoberon\seeassembly\seerisc\seeobject\seedebugging
}

\providecommand{\obwasm}{
\toolsection{obwasm} is a compiler for the Oberon programming language targeting the WebAssembly architecture.
It generates machine code for WebAssembly targets from modules written in Oberon and stores it in corresponding object files.
For debugging purposes, it also creates a debugging information file as well as an assembly file containing a listing of the generated machine code.
In addition, it stores the interface of each module in a symbol file which is required when other modules import the module.
Programs generated with this compiler require additional runtime support that is stored in the \file{ob\-wasm\-run} library file.
\flowgraph{\resource{Oberon\\source code} \ar[r] & \toolbox{obwasm} \ar[r] \ar@/l/[d] \ar[rd] & \resource{object file} \\ \variable{ECSIMPORT} \ar[ru] & \resource{symbol\\files} \ar@/r/[u] & \resource{debugging\\information}}
\seeoberon\seeassembly\seewasm\seeobject\seedebugging
}

% converter tools

\providecommand{\dbgdwarf}{
\toolsection{dbgdwarf} is a DWARF debugging information converter tool.
It converts debugging information into the DWARF debugging data format and stores it in corresponding object files~\cite{dwarffile}.
The resulting debugging object files can be combined with runtime support that creates Executable and Linking Format (ELF) files~\cite{elffile}.
\flowgraph{\resource{debugging\\information} \ar[r] & \toolbox{dbgdwarf} \ar[r] & \resource{debugging\\object file}}
\seeobject\seedebugging
}

% assembler tools

\providecommand{\asmprint}{
\toolsection{asmprint} is a pretty printer for generic assembly code.
It reformats generic assembly code and writes it to the standard output stream.
\flowgraph{\resource{generic assembly\\source code} \ar[r] & \toolbox{asmprint} \ar[r] & \resource{reformatted\\source code}}
\seeassembly
}

\providecommand{\amdaasm}{
\toolsection{amd16asm} is an assembler for the AMD64 hardware architecture.
It translates assembly code into machine code for AMD64 processors and stores it in corresponding object files.
By default, the assembler generates machine code for the 16-bit operating mode defined by the AMD64 architecture.
\flowgraph{\resource{AMD16 assembly\\source code} \ar[r] & \toolbox{amd16asm} \ar[r] & \resource{object file}}
\seeassembly\seeamd\seeobject
}

\providecommand{\amdadism}{
\toolsection{amd16dism} is a disassembler for the AMD64 hardware architecture.
It translates machine code from object files targeting AMD64 processors into assembly code and writes it to the standard output stream.
It assumes that the machine code was generated for the 16-bit operating mode defined by the AMD64 architecture.
\flowgraph{\resource{object file} \ar[r] & \toolbox{amd16dism} \ar[r] & \resource{disassembly\\listing}}
\seeassembly\seeamd\seeobject
}

\providecommand{\amdbasm}{
\toolsection{amd32asm} is an assembler for the AMD64 hardware architecture.
It translates assembly code into machine code for AMD64 processors and stores it in corresponding object files.
By default, the assembler generates machine code for the 32-bit operating mode defined by the AMD64 architecture.
\flowgraph{\resource{AMD32 assembly\\source code} \ar[r] & \toolbox{amd32asm} \ar[r] & \resource{object file}}
\seeassembly\seeamd\seeobject
}

\providecommand{\amdbdism}{
\toolsection{amd32dism} is a disassembler for the AMD64 hardware architecture.
It translates machine code from object files targeting AMD64 processors into assembly code and writes it to the standard output stream.
It assumes that the machine code was generated for the 32-bit operating mode defined by the AMD64 architecture.
\flowgraph{\resource{object file} \ar[r] & \toolbox{amd32dism} \ar[r] & \resource{disassembly\\listing}}
\seeassembly\seeamd\seeobject
}

\providecommand{\amdcasm}{
\toolsection{amd64asm} is an assembler for the AMD64 hardware architecture.
It translates assembly code into machine code for AMD64 processors and stores it in corresponding object files.
By default, the assembler generates machine code for the 64-bit operating mode defined by the AMD64 architecture.
\flowgraph{\resource{AMD64 assembly\\source code} \ar[r] & \toolbox{amd64asm} \ar[r] & \resource{object file}}
\seeassembly\seeamd\seeobject
}

\providecommand{\amdcdism}{
\toolsection{amd64dism} is a disassembler for the AMD64 hardware architecture.
It translates machine code from object files targeting AMD64 processors into assembly code and writes it to the standard output stream.
It assumes that the machine code was generated for the 64-bit operating mode defined by the AMD64 architecture.
\flowgraph{\resource{object file} \ar[r] & \toolbox{amd64dism} \ar[r] & \resource{disassembly\\listing}}
\seeassembly\seeamd\seeobject
}

\providecommand{\armaasm}{
\toolsection{arma32asm} is an assembler for the ARM hardware architecture.
It translates assembly code into machine code for ARM processors executing A32 instructions and stores it in corresponding object files.
\flowgraph{\resource{ARM A32 assembly\\source code} \ar[r] & \toolbox{arma32asm} \ar[r] & \resource{object file}}
\seeassembly\seearm\seeobject
}

\providecommand{\armadism}{
\toolsection{arma32dism} is a disassembler for the ARM hardware architecture.
It translates machine code from object files targeting ARM processors executing A32 instructions into assembly code and writes it to the standard output stream.
\flowgraph{\resource{object file} \ar[r] & \toolbox{arma32dism} \ar[r] & \resource{disassembly\\listing}}
\seeassembly\seearm\seeobject
}

\providecommand{\armbasm}{
\toolsection{arma64asm} is an assembler for the ARM hardware architecture.
It translates assembly code into machine code for ARM processors executing A64 instructions and stores it in corresponding object files.
\flowgraph{\resource{ARM A64 assembly\\source code} \ar[r] & \toolbox{arma64asm} \ar[r] & \resource{object file}}
\seeassembly\seearm\seeobject
}

\providecommand{\armbdism}{
\toolsection{arma64dism} is a disassembler for the ARM hardware architecture.
It translates machine code from object files targeting ARM processors executing A64 instructions into assembly code and writes it to the standard output stream.
\flowgraph{\resource{object file} \ar[r] & \toolbox{arma64dism} \ar[r] & \resource{disassembly\\listing}}
\seeassembly\seearm\seeobject
}

\providecommand{\armcasm}{
\toolsection{armt32asm} is an assembler for the ARM hardware architecture.
It translates assembly code into machine code for ARM processors executing T32 instructions and stores it in corresponding object files.
\flowgraph{\resource{ARM T32 assembly\\source code} \ar[r] & \toolbox{armt32asm} \ar[r] & \resource{object file}}
\seeassembly\seearm\seeobject
}

\providecommand{\armcdism}{
\toolsection{armt32dism} is a disassembler for the ARM hardware architecture.
It translates machine code from object files targeting ARM processors executing T32 instructions into assembly code and writes it to the standard output stream.
\flowgraph{\resource{object file} \ar[r] & \toolbox{armt32dism} \ar[r] & \resource{disassembly\\listing}}
\seeassembly\seearm\seeobject
}

\providecommand{\avrasm}{
\toolsection{avrasm} is an assembler for the AVR hardware architecture.
It translates assembly code into machine code for AVR processors and stores it in corresponding object files.
The identifiers \texttt{RXL}, \texttt{RXH}, \texttt{RYL}, \texttt{RYH}, \texttt{RZL}, and \texttt{RZH} are predefined and name the corresponding registers.
The identifiers \texttt{SPL} and \texttt{SPH} are also predefined and evaluate to the address of the corresponding registers.
\flowgraph{\resource{AVR assembly\\source code} \ar[r] & \toolbox{avrasm} \ar[r] & \resource{object file}}
\seeassembly\seeavr\seeobject
}

\providecommand{\avrdism}{
\toolsection{avrdism} is a disassembler for the AVR hardware architecture.
It translates machine code from object files targeting AVR processors into assembly code and writes it to the standard output stream.
\flowgraph{\resource{object file} \ar[r] & \toolbox{avrdism} \ar[r] & \resource{disassembly\\listing}}
\seeassembly\seeavr\seeobject
}

\providecommand{\avrttasm}{
\toolsection{avr32asm} is an assembler for the AVR32 hardware architecture.
It translates assembly code into machine code for AVR32 processors and stores it in corresponding object files.
\flowgraph{\resource{AVR32 assembly\\source code} \ar[r] & \toolbox{avr32asm} \ar[r] & \resource{object file}}
\seeassembly\seeavrtt\seeobject
}

\providecommand{\avrttdism}{
\toolsection{avr32dism} is a disassembler for the AVR32 hardware architecture.
It translates machine code from object files targeting AVR32 processors into assembly code and writes it to the standard output stream.
\flowgraph{\resource{object file} \ar[r] & \toolbox{avr32dism} \ar[r] & \resource{disassembly\\listing}}
\seeassembly\seeavrtt\seeobject
}

\providecommand{\mabkasm}{
\toolsection{m68kasm} is an assembler for the M68000 hardware architecture.
It translates assembly code into machine code for M68000 processors and stores it in corresponding object files.
\flowgraph{\resource{68000 assembly\\source code} \ar[r] & \toolbox{m68kasm} \ar[r] & \resource{object file}}
\seeassembly\seemabk\seeobject
}

\providecommand{\mabkdism}{
\toolsection{m68kdism} is a disassembler for the M68000 hardware architecture.
It translates machine code from object files targeting M68000 processors into assembly code and writes it to the standard output stream.
\flowgraph{\resource{object file} \ar[r] & \toolbox{m68kdism} \ar[r] & \resource{disassembly\\listing}}
\seeassembly\seemabk\seeobject
}

\providecommand{\miblasm}{
\toolsection{miblasm} is an assembler for the MicroBlaze hardware architecture.
It translates assembly code into machine code for MicroBlaze processors and stores it in corresponding object files.
\flowgraph{\resource{MicroBlaze assembly\\source code} \ar[r] & \toolbox{miblasm} \ar[r] & \resource{object file}}
\seeassembly\seemibl\seeobject
}

\providecommand{\mibldism}{
\toolsection{mibldism} is a disassembler for the MicroBlaze hardware architecture.
It translates machine code from object files targeting MicroBlaze processors into assembly code and writes it to the standard output stream.
\flowgraph{\resource{object file} \ar[r] & \toolbox{mibldism} \ar[r] & \resource{disassembly\\listing}}
\seeassembly\seemibl\seeobject
}

\providecommand{\mipsaasm}{
\toolsection{mips32asm} is an assembler for the MIPS32 hardware architecture.
It translates assembly code into machine code for MIPS32 processors and stores it in corresponding object files.
\flowgraph{\resource{MIPS32 assembly\\source code} \ar[r] & \toolbox{mips32asm} \ar[r] & \resource{object file}}
\seeassembly\seemips\seeobject
}

\providecommand{\mipsadism}{
\toolsection{mips32dism} is a disassembler for the MIPS32 hardware architecture.
It translates machine code from object files targeting MIPS32 processors into assembly code and writes it to the standard output stream.
\flowgraph{\resource{object file} \ar[r] & \toolbox{mips32dism} \ar[r] & \resource{disassembly\\listing}}
\seeassembly\seemips\seeobject
}

\providecommand{\mipsbasm}{
\toolsection{mips64asm} is an assembler for the MIPS64 hardware architecture.
It translates assembly code into machine code for MIPS64 processors and stores it in corresponding object files.
\flowgraph{\resource{MIPS64 assembly\\source code} \ar[r] & \toolbox{mips64asm} \ar[r] & \resource{object file}}
\seeassembly\seemips\seeobject
}

\providecommand{\mipsbdism}{
\toolsection{mips64dism} is a disassembler for the MIPS64 hardware architecture.
It translates machine code from object files targeting MIPS64 processors into assembly code and writes it to the standard output stream.
\flowgraph{\resource{object file} \ar[r] & \toolbox{mips64dism} \ar[r] & \resource{disassembly\\listing}}
\seeassembly\seemips\seeobject
}

\providecommand{\mmixasm}{
\toolsection{mmixasm} is an assembler for the MMIX hardware architecture.
It translates assembly code into machine code for MMIX processors and stores it in corresponding object files.
The names of all special registers are predefined and evaluate to the corresponding number.
\flowgraph{\resource{MMIX assembly\\source code} \ar[r] & \toolbox{mmixasm} \ar[r] & \resource{object file}}
\seeassembly\seemmix\seeobject
}

\providecommand{\mmixdism}{
\toolsection{mmixdism} is a disassembler for the MMIX hardware architecture.
It translates machine code from object files targeting MMIX processors into assembly code and writes it to the standard output stream.
\flowgraph{\resource{object file} \ar[r] & \toolbox{mmixdism} \ar[r] & \resource{disassembly\\listing}}
\seeassembly\seemmix\seeobject
}

\providecommand{\orokasm}{
\toolsection{or1kasm} is an assembler for the OpenRISC 1000 hardware architecture.
It translates assembly code into machine code for OpenRISC 1000 processors and stores it in corresponding object files.
\flowgraph{\resource{OpenRISC 1000 assembly\\source code} \ar[r] & \toolbox{or1kasm} \ar[r] & \resource{object file}}
\seeassembly\seeorok\seeobject
}

\providecommand{\orokdism}{
\toolsection{or1kdism} is a disassembler for the OpenRISC 1000 hardware architecture.
It translates machine code from object files targeting OpenRISC 1000 processors into assembly code and writes it to the standard output stream.
\flowgraph{\resource{object file} \ar[r] & \toolbox{or1kdism} \ar[r] & \resource{disassembly\\listing}}
\seeassembly\seeorok\seeobject
}

\providecommand{\ppcaasm}{
\toolsection{ppc32asm} is an assembler for the PowerPC hardware architecture.
It translates assembly code into machine code for PowerPC processors and stores it in corresponding object files.
By default, the assembler generates machine code for the 32-bit operating mode defined by the PowerPC architecture.
\flowgraph{\resource{PowerPC assembly\\source code} \ar[r] & \toolbox{ppc32asm} \ar[r] & \resource{object file}}
\seeassembly\seeppc\seeobject
}

\providecommand{\ppcadism}{
\toolsection{ppc32dism} is a disassembler for the PowerPC hardware architecture.
It translates machine code from object files targeting PowerPC processors into assembly code and writes it to the standard output stream.
It assumes that the machine code was generated for the 32-bit operating mode defined by the PowerPC architecture.
\flowgraph{\resource{object file} \ar[r] & \toolbox{ppc32dism} \ar[r] & \resource{disassembly\\listing}}
\seeassembly\seeppc\seeobject
}

\providecommand{\ppcbasm}{
\toolsection{ppc64asm} is an assembler for the PowerPC hardware architecture.
It translates assembly code into machine code for PowerPC processors and stores it in corresponding object files.
By default, the assembler generates machine code for the 64-bit operating mode defined by the PowerPC architecture.
\flowgraph{\resource{PowerPC assembly\\source code} \ar[r] & \toolbox{ppc64asm} \ar[r] & \resource{object file}}
\seeassembly\seeppc\seeobject
}

\providecommand{\ppcbdism}{
\toolsection{ppc64dism} is a disassembler for the PowerPC hardware architecture.
It translates machine code from object files targeting PowerPC processors into assembly code and writes it to the standard output stream.
It assumes that the machine code was generated for the 64-bit operating mode defined by the PowerPC architecture.
\flowgraph{\resource{object file} \ar[r] & \toolbox{ppc64dism} \ar[r] & \resource{disassembly\\listing}}
\seeassembly\seeppc\seeobject
}

\providecommand{\riscasm}{
\toolsection{riscasm} is an assembler for the RISC hardware architecture.
It translates assembly code into machine code for RISC processors and stores it in corresponding object files.
The names of all special registers are predefined and evaluate to the corresponding number.
\flowgraph{\resource{RISC assembly\\source code} \ar[r] & \toolbox{riscasm} \ar[r] & \resource{object file}}
\seeassembly\seerisc\seeobject
}

\providecommand{\riscdism}{
\toolsection{riscdism} is a disassembler for the RISC hardware architecture.
It translates machine code from object files targeting RISC processors into assembly code and writes it to the standard output stream.
\flowgraph{\resource{object file} \ar[r] & \toolbox{riscdism} \ar[r] & \resource{disassembly\\listing}}
\seeassembly\seerisc\seeobject
}

\providecommand{\wasmasm}{
\toolsection{wasmasm} is an assembler for the WebAssembly architecture.
It translates assembly code into machine code for WebAssembly targets and stores it in corresponding object files.
The names of all special registers are predefined and evaluate to the corresponding number.
\flowgraph{\resource{WebAssembly assembly\\source code} \ar[r] & \toolbox{wasmasm} \ar[r] & \resource{object file}}
\seeassembly\seewasm\seeobject
}

\providecommand{\wasmdism}{
\toolsection{wasmdism} is a disassembler for the WebAssembly architecture.
It translates machine code from object files targeting WebAssembly targets into assembly code and writes it to the standard output stream.
\flowgraph{\resource{object file} \ar[r] & \toolbox{wasmdism} \ar[r] & \resource{disassembly\\listing}}
\seeassembly\seewasm\seeobject
}

% linker tools

\providecommand{\linklib}{
\toolsection{linklib} is an object file combiner.
It creates a static library file by combining all object files given to it into a single one.
\flowgraph{\resource{object files} \ar[r] & \toolbox{linklib} \ar[r] & \resource{library file}}
\seeobject
}

\providecommand{\linkbin}{
\toolsection{linkbin} is a linker for plain binary files.
It links all object files given to it into a single image and stores it in a binary file that begins with the first linked section.
It also creates a map file that lists the address, type, name and size of all used sections.
The filename extension of the resulting binary file can be specified by putting it into a constant data section called \texttt{\_extension}.
\flowgraph{\resource{object files} \ar[r] & \toolbox{linkbin} \ar[r] \ar[d] & \resource{binary file} \\ & \resource{map file}}
\seeobject
}

\providecommand{\linkmem}{
\toolsection{linkmem} is a linker for plain binary files partitioned into random-access and read-only memory.
It links all object files given to it into two distinct images, one for data sections and one for code and constant data sections, and stores each image in a binary file that begins with the first linked section of the corresponding type.
It also creates a map file that lists the address, type, name and size of all used sections.
\flowgraph{\resource{object files} \ar[r] & \toolbox{linkmem} \ar[r] \ar[d] & \resource{RAM file/\\ROM file} \\ & \resource{map file}}
\seeobject
}

\providecommand{\linkprg}{
\toolsection{linkprg} is a linker for GEMDOS executable files.
It links all object files given to it into a single image and stores the image in an Atari GEMDOS executable file~\cite{gemdosfile}.
It also creates a map file that lists the address relative to the text segment, type, name and size of all used sections.
The filename extension of the resulting executable file can be specified by putting it into a constant data section called \texttt{\_extension}.
The GEMDOS executable file format requires all patch patterns of absolute link patches to consist of four full bitmasks with descending offsets.
\flowgraph{\resource{object files} \ar[r] & \toolbox{linkprg} \ar[r] \ar[d] & \resource{executable file} \\ & \resource{map file}}
\seeobject
}

\providecommand{\linkhex}{
\toolsection{linkhex} is a linker for Intel HEX files.
It links all code sections of the object files given to it into single image and stores the image in an Intel HEX file~\cite{hexfile} that begins with the first linked section.
It also creates a map file that lists the address, type, name and size of all used sections.
\flowgraph{\resource{object files} \ar[r] & \toolbox{linkhex} \ar[r] \ar[d] & \resource{HEX file} \\ & \resource{map file}}
\seeobject
}

\providecommand{\mapsearch}{
\toolsection{mapsearch} is a debugging tool.
It searches map files generated by linker tools for the name of a binary section that encompasses a memory address read from the standard input stream.
If additionally provided with one or more object files, it also stores an excerpt thereof in a separate object file called map search result which only contains the identified binary section for disassembling purposes.
\flowgraph{& \resource{map files/\\object files} \ar[d] \\ \resource{memory\\address} \ar[r] & \toolbox{mapsearch} \ar[r] \ar[d] & \resource{section name/\\relative offset} \\ & \resource{object file\\excerpt}}
\seeobject
}


\startbook{User Manual}

\chapter*{Preface}\markboth{Preface}{Preface}

This user manual describes all components of a free and self-hosted software development toolchain called the \emph{\ecs{}} and explains how to use them.
It is partitioned into four parts.
The first part describes the overall mission, design, and the common user interface of the \ecs{}.
The second part specifies the programming languages supported by the \ecs{}, while the third part summarizes the hardware architectures and how they are supported.
The last part of this manual finally describes some internal functionality and explains how the \ecs{} can be extended in order to support additional programming languages, hardware architectures, and runtime environments.
The addendum elaborates on the implementation of the \ecs{} and includes its licenses.

\epigraph{Il semble que la perfection soit atteinte \\ non quand il n'y a plus rien \`a ajouter, \\ mais quand il n'y a plus rien \`a retrancher.}{Antoine de Saint-Exup\'ery}

\concludechapter
\mainmatter

\part{Using the \ecs{}}
% Introduction to the Eigen Compiler Suite
% Copyright (C) Florian Negele

% This file is part of the Eigen Compiler Suite.

% Permission is granted to copy, distribute and/or modify this document
% under the terms of the GNU Free Documentation License, Version 1.3
% or any later version published by the Free Software Foundation.

% You should have received a copy of the GNU Free Documentation License
% along with the ECS.  If not, see <https://www.gnu.org/licenses/>.

% Generic documentation utilities
% Copyright (C) Florian Negele

% This file is part of the Eigen Compiler Suite.

% Permission is granted to copy, distribute and/or modify this document
% under the terms of the GNU Free Documentation License, Version 1.3
% or any later version published by the Free Software Foundation.

% You should have received a copy of the GNU Free Documentation License
% along with the ECS.  If not, see <https://www.gnu.org/licenses/>.

\providecommand{\cpp}{C\texttt{++}}
\providecommand{\opt}{_\mathit{opt}}
\providecommand{\tool}[1]{\texttt{#1}}
\providecommand{\version}{Version 0.0.40}
\providecommand{\resource}[1]{*++\txt{#1}}
\providecommand{\ecs}{Eigen Compiler Suite}
\providecommand{\changed}[1]{\underline{#1}}
\providecommand{\toolbox}[1]{\converter{#1}}
\providecommand{\file}{}\renewcommand{\file}[1]{\texttt{#1}}
\providecommand{\alignright}{\hfill\linebreak[0]\hspace*{\fill}}
\providecommand{\converter}[1]{*++[F][F*:white][F,:gray]\txt{#1}}
\providecommand{\documentation}{\ifbook chapter\else document\fi}
\providecommand{\Documentation}{\ifbook Chapter\else Document\fi}
\providecommand{\variable}[1]{\resource{\texttt{\small#1}\\variable}}
\providecommand{\documentationref}[2]{\ifbook\ref{#1}\else``\href{#1}{#2}''~\cite{#1}\fi}
\providecommand{\objfile}[1]{\texttt{#1}\index[runtime]{#1 object file@\texttt{#1} object file}}
\providecommand{\libfile}[1]{\texttt{#1}\index[runtime]{#1 library file@\texttt{#1} library file}}
\providecommand{\epigraph}[2]{\ifbook\begin{quote}\flushright\textit{#1}\par--- #2\end{quote}\fi}
\providecommand{\environmentvariable}[1]{\texttt{#1}\index{Environment variables!#1@\texttt{#1}}}
\providecommand{\environment}[1]{\texttt{#1}\index[environment]{#1 environment@\texttt{#1} environment}}
\providecommand{\toolsection}{}\renewcommand{\toolsection}[1]{\subsection{#1}\label{\prefix:#1}\tool{#1}}
\providecommand{\instruction}{}\renewcommand{\instruction}[2]{\noindent\qquad\pdftooltip{\texttt{#1}}{#2}\refstepcounter{instruction}\par}
\providecommand{\flowgraph}{}\renewcommand{\flowgraph}[1]{\par\sffamily\begin{displaymath}\xymatrix@=4ex{#1}\end{displaymath}\normalfont\par}
\providecommand{\instructionset}{}\renewcommand{\instructionset}[4]{\setcounter{instruction}{0}\begin{multicols}{\ifbook#3\else#4\fi}[{\captionof{table}[#2]{#2 (\ref*{#1:instructions}~instructions)}\label{tab:#1set}\vspace{-2ex}}]\footnotesize\raggedcolumns\input{#1.set}\label{#1:instructions}\end{multicols}}

\providecommand{\gpl}{GNU General Public License}
\providecommand{\rse}{ECS Runtime Support Exception}
\providecommand{\fdl}{\href{https://www.gnu.org/licenses/fdl.html}{GNU Free Documentation License}}

\providecommand{\docbegin}{}
\providecommand{\docend}{}
\providecommand{\doclabel}[1]{\hypertarget{#1}}
\providecommand{\doclink}[2]{\hyperlink{#1}{#2}}
\providecommand{\docsection}[3]{\hypertarget{#1}{\subsection{#2}}\label{sec:#1}\index[library]{#2@#3}}
\providecommand{\docsectionstar}[1]{}
\providecommand{\docsubbegin}{\begin{description}}
\providecommand{\docsubend}{\end{description}}
\providecommand{\docsubsection}[3]{\item[\hypertarget{#1}{#2}]\index[library]{#2@#3}}
\providecommand{\docsubsectionstar}[1]{\smallskip}
\providecommand{\docsubsubsection}[3]{\docsubsection{#1}{#2}{#3}}
\providecommand{\docsubsubsectionstar}[1]{}
\providecommand{\docsubsubsubsection}[3]{}
\providecommand{\docsubsubsubsectionstar}[1]{}
\providecommand{\doctable}{}

\providecommand{\debuggingtool}{}\renewcommand{\debuggingtool}{This tool is provided for debugging purposes.
It allows exposing and modifying an internal data structure that is usually not accessible.
}

\providecommand{\interface}{All tools accept command-line arguments which are taken as names of plain text files containing the source code.
If no arguments are provided, the standard input stream is used instead.
Output files are generated in the current working directory and have the same name as the input file being processed whereas the filename extension gets replaced by an appropriate suffix.
\seeinterface
}

\providecommand{\license}{\noindent Copyright \copyright{} Florian Negele\par\medskip\noindent
Permission is granted to copy, distribute and/or modify this document under the terms of the
\fdl{}, Version 1.3 or any later version published by the \href{https://fsf.org/}{Free Software Foundation}.
}

\providecommand{\ecslogosurface}{
\fill[darkgray] (0,0,0) -- (0,0,3) -- (0,3,3) -- (0,3,1) -- (0,4,1) -- (0,4,3) -- (0,5,3) -- (0,5,0) -- (0,2,0) -- (0,2,2) -- (0,1,2) -- (0,1,0) -- cycle;
\fill[gray] (0,5,0) -- (0,5,3) -- (1,5,3) -- (1,5,1) -- (2,5,1) -- (2,5,3) -- (3,5,3) -- (3,5,0) -- cycle;
\fill[lightgray] (0,0,0) -- (0,1,0) -- (2,1,0) -- (2,4,0) -- (1,4,0) -- (1,3,0) -- (2,3,0) -- (2,2,0) -- (0,2,0) -- (0,5,0) -- (3,5,0) -- (3,0,0) -- cycle;
\begin{scope}[line width=0.5]
\begin{scope}[gray]
\draw (0,0,0) -- (0,1,0);
\draw (2,1,0) -- (2,2,0);
\draw (0,1,2) -- (0,2,2);
\draw (0,2,0) -- (0,5,0);
\draw (2,3,0) -- (2,4,0);
\end{scope}
\begin{scope}[lightgray]
\draw (0,1,0) -- (0,1,2);
\draw (0,3,1) -- (0,3,3);
\draw (0,5,0) -- (0,5,3);
\draw (2,5,1) -- (2,5,3);
\end{scope}
\begin{scope}[white]
\draw (0,1,0) -- (2,1,0);
\draw (1,3,0) -- (2,3,0);
\draw (0,5,0) -- (3,5,0);
\end{scope}
\end{scope}
}

\providecommand{\ecslogo}[1]{
\begin{tikzpicture}[scale={(#1)/((sin(45)+cos(45))*3cm)},x={({-cos(45)*1cm},{sin(45)*sin(30)*1cm})},y={({0cm},{(cos(30)*1cm})},z={({sin(45)*1cm},{cos(45)*sin(30)*1cm})}]
\begin{scope}[darkgray,line width=1]
\draw (0,0,0) -- (0,0,3) -- (0,3,3) -- (2,3,3) -- (2,5,3) -- (3,5,3) -- (3,5,0) -- (3,0,0) -- cycle;
\draw (0,3,1) -- (0,4,1) -- (0,4,3) -- (0,5,3) -- (1,5,3) -- (1,5,1) -- (2,5,1);
\draw (1,3,0) -- (1,4,0) -- (2,4,0);
\end{scope}
\fill[darkgray] (2,0,0) -- (2,0,3) -- (2,5,3) -- (2,5,1) -- (2,4,1) -- (2,4,0) -- cycle;
\fill[lightgray] (2,0,2) -- (0,0,2) -- (0,2,2) -- (2,2,2) -- cycle;
\fill[gray] (0,1,0) -- (2,1,0) -- (2,1,2) -- (0,1,2) -- cycle;
\fill[gray] (0,3,1) -- (0,3,3) -- (2,3,3) -- (2,3,0) -- (1,3,0) -- (1,3,1) -- cycle;
\ecslogosurface
\end{tikzpicture}
}

\providecommand{\shadowedecslogo}[3]{
\begin{tikzpicture}[scale={(#1)/((sin(#2)+cos(#2))*3cm)},x={({-cos(#2)*1cm},{sin(#2)*sin(#3)*1cm})},y={({0cm},{(cos(#3)*1cm})},z={({sin(#2)*1cm},{cos(#2)*sin(#3)*1cm})}]
\shade[top color=lightgray!50!white,bottom color=white,middle color=lightgray!50!white] (0,0,0) -- (3,0,0) -- (3,{-0.5-3*sin(#2)*sin(#3)/cos(#3)},0) -- (0,-0.5,0) -- cycle;
\shade[top color=darkgray!50!gray,bottom color=white,middle color=darkgray!50!white] (0,0,0) -- (0,0,3) -- (0,{-0.5-3*cos(#2)*sin(#3)/cos(#3)},3) -- (0,-0.5,0) -- cycle;
\begin{scope}[y={({(cos(#2)+sin(#2))*0.5cm},{(cos(#2)*sin(#3)-sin(#2)*sin(#3))*0.5cm})}]
\useasboundingbox (3,0,0) -- (0,0,0) -- (0,0,3);
\shade[left color=darkgray!80!black,right color=lightgray,middle color=gray] (0,0,0) -- (0,1,0) -- (0,1,0.5) -- (0,2,0) -- (0,5,0) -- (0,5,3) -- (1,5,3) -- (1,4,3) -- (1,4,2.5) -- (1,3,3) -- (2,5,3) -- (3,5,3) -- (3,0,3) -- cycle;
\clip (0,0,0) -- (0,0,3) -- ({-3*sin(#2)/cos(#2)},0,0) -- cycle;
\shade[left color=darkgray,right color=lightgray!50!gray] (0,0,0) -- (0,1,0) -- (0,1,0.5) -- (0,2,0) -- (0,5,0) -- (0,5,3) -- (1,5,3) -- (1,4,3) -- (1,4,2.5) -- (1,3,3) -- (2,5,3) -- (3,5,3) -- (3,0,3) -- cycle;
\end{scope}
\shade[left color=darkgray,right color=darkgray!80!black] (2,0,0) -- (2,0,3) -- (2,5,3) -- (2,5,1) -- (2,4,1) -- (2,4,0) -- cycle;
\shade[left color=darkgray!90!black,right color=gray!80!darkgray] (2,0,2) -- (0,0,2) -- (0,2,2) -- (2,2,2) -- cycle;
\shade[top color=darkgray!90!black,bottom color=gray!80!darkgray] (0,1,0) -- (2,1,0) -- (2,1,2) -- (0,1,2) -- cycle;
\shade[top color=darkgray!90!black,bottom color=gray!80!darkgray] (0,3,1) -- (0,3,3) -- (2,3,3) -- (2,3,0) -- (1,3,0) -- (1,3,1) -- cycle;
\fill[gray] (2,1,0) -- (1.5,1,0.5) -- (0,1,0.5) -- (0,1,0) -- cycle;
\fill[gray] (1,3,2) -- (0.5,3,2) -- (0.5,3,3) -- (1,3,3) -- cycle;
\fill[gray] (2,3,0) -- (1.5,3,0.5) -- (1,3,0.5) -- (1,3,0) -- cycle;
\ecslogosurface
\end{tikzpicture}
}

\providecommand{\cpplogo}[1]{
\begin{tikzpicture}[scale=(#1)/512em]
\fill[gray] (435.2794,398.7159) -- (247.1911,507.3075) .. controls (236.3563,513.5642) and (218.6240,513.5642) .. (207.7892,507.3075) -- (19.7009,398.7159) .. controls (8.8646,392.4606) and (0.0000,377.1043) .. (0.0000,364.5924) -- (0.0000,147.4076) .. controls (0.8430,132.8363) and (8.2856,120.7683) .. (19.7009,113.2842) -- (207.7892,4.6926) .. controls (218.6240,-1.5642) and (236.3564,-1.5642) .. (247.1911,4.6926) -- (435.2794,113.2842) .. controls (447.5273,121.4304) and (454.4987,133.6918) .. (454.9803,147.4076) -- (454.9803,364.5924) .. controls (454.5404,377.7571) and (446.6566,391.0351) .. (435.2794,398.7159) -- cycle(75.8301,255.9993) .. controls (74.9389,404.0881) and (273.2892,469.4783) .. (358.8263,331.8769) -- (293.1917,293.8965) .. controls (253.5702,359.4301) and (155.1909,335.9977) .. (151.6601,255.9993) .. controls (152.7204,182.2703) and (249.4137,148.0211) .. (293.1961,218.1065) -- (358.8308,180.1276) .. controls (283.4477,49.2645) and (79.6318,96.3470) .. (75.8301,255.9993) -- cycle(379.1503,247.5747) -- (362.2982,247.5747) -- (362.2982,230.7226) -- (345.4490,230.7226) -- (345.4490,247.5747) -- (328.5969,247.5747) -- (328.5969,264.4254) -- (345.4490,264.4254) -- (345.4490,281.2759) -- (362.2982,281.2759) -- (362.2982,264.4254) -- (379.1503,264.4254) -- cycle(442.3420,247.5747) -- (425.4899,247.5747) -- (425.4899,230.7226) -- (408.6408,230.7226) -- (408.6408,247.5747) -- (391.7886,247.5747) -- (391.7886,264.4254) -- (408.6408,264.4254) -- (408.6408,281.2759) -- (425.4899,281.2759) -- (425.4899,264.4254) -- (442.3420,264.4254) -- cycle;
\end{tikzpicture}
}

\providecommand{\fallogo}[1]{
\begin{tikzpicture}[scale=(#1)/512em]
\fill[gray] (185.7774,0.0000) .. controls (200.4486,15.9798) and (226.8966,8.7148) .. (235.0426,31.5836) .. controls (249.5297,58.0598) and (247.9581,97.9161) .. (280.3335,110.9762) .. controls (309.1690,120.3496) and (337.8406,104.2727) .. (366.5753,103.9379) .. controls (373.4449,111.5171) and (379.2885,128.2574) .. (383.9755,108.9744) .. controls (396.6979,102.5615) and (437.2808,107.6681) .. (426.9652,124.3252) .. controls (408.9822,121.0785) and (412.4742,146.0729) .. (426.5192,131.4996) .. controls (433.8413,120.8489) and (465.1541,126.5522) .. (441.9067,135.7950) .. controls (396.1879,157.7478) and (344.1112,161.5079) .. (298.5528,183.5702) .. controls (277.7471,193.5198) and (284.6941,218.7163) .. (285.2127,236.9640) .. controls (292.3599,316.2826) and (307.3929,394.6311) .. (317.1198,473.6154) .. controls (329.0637,505.4736) and (292.1195,528.5004) .. (265.9183,511.2761) .. controls (237.9284,499.2462) and (237.3684,465.2681) .. (230.9102,439.9421) .. controls (218.6692,374.3397) and (215.6307,306.9662) .. (198.1732,242.3977) .. controls (183.1379,232.7444) and (164.4245,256.0298) .. (149.0430,261.4799) .. controls (116.9328,279.2585) and (87.1822,308.5851) .. (48.2293,307.8914) .. controls (21.3220,306.9037) and (-15.9107,281.8761) .. (7.2921,252.7908) .. controls (29.7799,220.6177) and (67.5177,204.3028) .. (100.9287,185.9449) .. controls (130.8217,170.8906) and (161.1548,156.5903) .. (191.0278,141.5847) .. controls (196.1738,120.0520) and (186.6049,95.2409) .. (186.8382,72.4353) .. controls (185.5234,48.4204) and (183.1700,23.9341) .. (185.7774,0.0000) -- cycle;
\end{tikzpicture}
}

\providecommand{\oblogo}[1]{
\begin{tikzpicture}[scale=(#1)/512em]
\fill[gray] (160.3865,208.9117) .. controls (154.0879,214.6478) and (149.0735,221.2409) .. (145.4125,228.5384) .. controls (184.8790,248.4273) and (234.7122,269.8787) .. (297.5493,291.8782) .. controls (300.3943,281.4769) and (300.9552,268.7619) .. (300.4023,255.2389) .. controls (248.9909,244.7891) and (200.0310,225.9279) .. (160.3865,208.9117) -- cycle(225.7398,392.6996) .. controls (308.0209,392.1716) and (359.3326,345.9277) .. (368.7203,285.2098) .. controls (376.6742,197.1784) and (311.7194,141.3342) .. (205.4287,142.1456) .. controls (139.9485,141.4804) and (88.7155,166.1957) .. (73.5775,228.0086) .. controls (52.0297,320.3408) and (123.4078,391.0103) .. (225.7398,392.6996) -- cycle(216.0739,176.4733) .. controls (268.9183,179.2424) and (315.8292,206.5488) .. (312.7454,265.1139) .. controls (313.2769,315.6384) and (286.5993,353.4946) .. (216.6040,355.7934) .. controls (162.4657,355.7934) and (126.0914,317.5023) .. (126.0914,260.5103) .. controls (126.1733,214.2900) and (163.3363,176.2849) .. (216.0739,176.4733) -- cycle(76.4897,189.1754) .. controls (13.1586,147.5631) and (0.0000,119.4207) .. (0.0000,119.4207) -- (90.6499,170.1632) .. controls (85.3004,175.8497) and (80.5994,182.1633) .. (76.4897,189.1754) -- cycle(353.9486,119.3004) -- (402.9482,119.3004) .. controls (427.0025,137.0797) and (450.9893,162.7034) .. (474.9529,191.0213) .. controls (509.3540,228.5339) and (531.3391,294.2091) .. (487.8149,312.1206) .. controls (462.8165,324.7652) and (394.3874,316.8943) .. (373.8912,313.6651) .. controls (379.9291,297.7449) and (383.2899,278.4204) .. (381.4989,257.7214) .. controls (420.3069,248.0321) and (421.9610,218.3461) .. (407.7867,192.6417) .. controls (391.1113,162.4018) and (370.1114,132.9097) .. (353.9486,119.3004) -- cycle;
\end{tikzpicture}
}

\providecommand{\markuptable}{
\begin{table}
\sffamily\centering
\begin{tabular}{@{}lcl@{}}
\toprule
\texttt{//italics//} & $\rightarrow$ & \textit{italics} \\
\midrule
\texttt{**bold**} & $\rightarrow$ & \textbf{bold} \\
\midrule
\texttt{\# ordered list} & & 1 ordered list \\
\texttt{\# second item} & $\rightarrow$ & 2 second item \\
\texttt{\#\# sub item} & & \hspace{1em} 1 sub item \\
\midrule
\texttt{* unordered list} & & $\bullet$ unordered list \\
\texttt{* second item} & $\rightarrow$ & $\bullet$ second item \\
\texttt{** sub item} & & \hspace{1em} $\bullet$ sub item \\
\midrule
\texttt{link to [[label]]} & $\rightarrow$ & link to \underline{label} \\
\midrule
\texttt{<{}<label>{}> definition } & $\rightarrow$ & definition \\
\midrule
\texttt{[[url|link name]]} & $\rightarrow$ & \underline{link name} \\
\midrule\addlinespace
\texttt{= large heading} & & {\Large large heading} \smallskip \\
\texttt{== medium heading} & $\rightarrow$ & {\large medium heading} \\
\texttt{=== small heading} & & small heading \\
\midrule
\texttt{no line break} & & no line break for paragraphs \\
\texttt{for paragraphs} & $\rightarrow$ \\
& & use empty line \\
\texttt{use empty line} \\
\midrule
\texttt{force\textbackslash\textbackslash line break} & $\rightarrow$ & force \\
& & line break \\
\midrule
\texttt{horizontal line} & $\rightarrow$ & horizontal line \\
\texttt{----} & & \hrulefill \\
\midrule
\texttt{|=a|=table|=header} & & \underline{a \enspace table \enspace header} \\
\texttt{|a|table|row} & $\rightarrow$ & a \enspace table \enspace row \\
\texttt{|b|table|row} & & b \enspace table \enspace row \\
\midrule
\texttt{\{\{\{} \\
\texttt{unformatted} & $\rightarrow$ & \texttt{unformatted} \\
\texttt{code} & & \texttt{code} \\
\texttt{\}\}\}} \\
\midrule\addlinespace
\texttt{@ new article} & & {\Large 1.\ new article} \smallskip \\
\texttt{@ second article} & $\rightarrow$ & {\Large 2.\ second article} \smallskip \\
\texttt{@@ sub article} & & {\large 2.1.\ sub article} \\
\bottomrule
\end{tabular}
\normalfont\caption{Elements of the generic documentation markup language}
\label{tab:docmarkup}
\end{table}
}

\providecommand{\startchapter}[4]{
\documentclass[11pt,a4paper]{article}
\usepackage{booktabs}
\usepackage[format=hang,labelfont=bf]{caption}
\usepackage{changepage}
\usepackage[T1]{fontenc}
\usepackage[margin=2cm]{geometry}
\usepackage{hyperref}
\usepackage[american]{isodate}
\usepackage{lmodern}
\usepackage{longtable}
\usepackage{mathptmx}
\usepackage{microtype}
\usepackage[toc]{multitoc}
\usepackage{multirow}
\usepackage[all]{nowidow}
\usepackage{pdfcomment}
\usepackage{syntax}
\usepackage{tikz}
\usepackage[all]{xy}
\hypersetup{pdfborder={0 0 0},bookmarksnumbered=true,pdftitle={\ecs{}: #2},pdfauthor={Florian Negele},pdfsubject={\ecs{}},pdfkeywords={#1}}
\setlength{\grammarindent}{8em}\setlength{\grammarparsep}{0.2ex}
\setlength{\columnsep}{2em}
\newcommand{\prefix}{}
\newcounter{instruction}
\bibliographystyle{unsrt}
\renewcommand{\index}[2][]{}
\renewcommand{\arraystretch}{1.05}
\renewcommand{\floatpagefraction}{0.7}
\renewcommand{\syntleft}{\itshape}\renewcommand{\syntright}{}
\title{\vspace{-5ex}\Huge{\ecs{}}\medskip\hrule}
\author{\huge{#2}}
\date{\medskip\version}
\newif\ifbook\bookfalse
\pagestyle{headings}
\frenchspacing
\begin{document}
\maketitle\thispagestyle{empty}\noindent#4\setlength{\columnseprule}{0.4pt}\tableofcontents\setlength{\columnseprule}{0pt}\vfill\pagebreak[3]\null\vfill\bigskip\noindent
\parbox{\textwidth-4em}{\license The contents of this \documentation{} are part of the \href{manual}{\ecs{} User Manual}~\cite{manual} and correspond to Chapter ``\href{manual\##3}{#1}''.\alignright\mbox{\today}}
\parbox{4em}{\flushright\ecslogo{3em}}
\clearpage
}

\providecommand{\concludechapter}{
\vfill\pagebreak[3]\null\vfill
\thispagestyle{myheadings}\markright{REFERENCES}
\noindent\begin{minipage}{\textwidth}\begin{multicols}{2}[\section*{References}]
\renewcommand{\section}[2]{}\small\bibliography{references}
\end{multicols}\end{minipage}\end{document}
}

\providecommand{\startpresentation}[2]{
\documentclass[14pt,aspectratio=43,usepdftitle=false]{beamer}
\usepackage{booktabs}
\usepackage{etex}
\usepackage{multicol}
\usepackage{tikz}
\usepackage[all]{xy}
\bibliographystyle{unsrt}
\setlength{\columnsep}{1em}
\setlength{\leftmargini}{1em}
\setbeamercolor{title}{fg=black}
\setbeamercolor{structure}{fg=darkgray}
\setbeamercolor{bibliography item}{fg=darkgray}
\setbeamerfont{title}{series=\bfseries}
\setbeamerfont{subtitle}{series=\normalfont}
\setbeamerfont*{frametitle}{parent=title}
\setbeamerfont{block title}{series=\bfseries}
\setbeamerfont*{framesubtitle}{parent=subtitle}
\setbeamersize{text margin left=1em,text margin right=1em}
\setbeamertemplate{navigation symbols}{}
\setbeamertemplate{itemize item}[circle]{}
\setbeamertemplate{bibliography item}[triangle]{}
\setbeamertemplate{bibliography entry author}{\usebeamercolor[fg]{bibliography item}}
\setbeamertemplate{frametitle}{\medskip\usebeamerfont{frametitle}\color{gray}\raisebox{-2.5ex}[0ex][0ex]{\rule{0.1em}{4.5ex}}}
\addtobeamertemplate{frametitle}{}{\hspace{0.4em}\usebeamercolor[fg]{title}\insertframetitle\par\vspace{0.2ex}\hspace{0.5em}\usebeamerfont{framesubtitle}\insertframesubtitle}
\hypersetup{pdfborder={0 0 0},bookmarksnumbered=true,bookmarksopen=true,bookmarksopenlevel=0,pdftitle={\ecs{}: #1},pdfauthor={Florian Negele},pdfsubject={\ecs{}},pdfkeywords={#1}}
\renewcommand{\flowgraph}[1]{\resizebox{\textwidth}{!}{$$\xymatrix{##1}$$}}
\title{\ecs{}\medskip\hrule\medskip}
\institute{\shadowedecslogo{5em}{30}{15}}
\date{\version}
\subtitle{#1}
\begin{document}
\begin{frame}[plain]\titlepage\nocite{manual}\end{frame}
\begin{frame}{Contents}{#1}\begin{center}\tableofcontents\end{center}\end{frame}
}

\providecommand{\concludepresentation}{
\begin{frame}{References}\begin{footnotesize}\setlength{\columnseprule}{0.4pt}\begin{multicols}{2}\bibliography{references}\end{multicols}\end{footnotesize}\end{frame}
\end{document}
}

\providecommand{\startbook}[1]{
\documentclass[10pt,paper=17cm:24cm,DIV=13,twoside=semi,headings=normal,numbers=noendperiod,cleardoublepage=plain]{scrbook}
\usepackage{atveryend}
\usepackage{booktabs}
\usepackage{caption}
\usepackage{changepage}
\usepackage[T1]{fontenc}
\usepackage{imakeidx}
\usepackage{hyperref}
\usepackage[american]{isodate}
\usepackage{lmodern}
\usepackage{longtable}
\usepackage{mathptmx}
\usepackage[final]{microtype}
\usepackage{multicol}
\usepackage{multirow}
\usepackage[all]{nowidow}
\usepackage{pdfcomment}
\usepackage{scrlayer-scrpage}
\usepackage{setspace}
\usepackage{syntax}
\usepackage[eventxtindent=4pt,oddtxtexdent=4pt]{thumbs}
\usepackage{tikz}
\usepackage[all]{xy}
\hyphenation{Micro-Blaze Open-Cores Open-RISC Power-PC}
\hypersetup{pdfborder={0 0 0},bookmarksnumbered=true,bookmarksopen=true,bookmarksopenlevel=0,pdftitle={\ecs{}: #1},pdfauthor={Florian Negele},pdfsubject={\ecs{}},pdfkeywords={#1}}
\setlength{\grammarindent}{8em}\setlength{\grammarparsep}{0.7ex}
\setkomafont{captionlabel}{\usekomafont{descriptionlabel}}
\renewcommand{\arraystretch}{1.05}\setstretch{1.1}
\renewcommand{\chapterformat}{\thechapter\autodot\enskip\raisebox{-1ex}[0ex][0ex]{\color{gray}\rule{0.1em}{3.5ex}}\enskip}
\renewcommand{\startchapter}[4]{\hypertarget{##3}{\chapter{##1}}\label{##3}##4\addthumb{##1}{\LARGE\sffamily\bfseries\thechapter}{white}{gray}\renewcommand{\prefix}{##3}}
\renewcommand{\concludechapter}{\clearpage{\stopthumb\cleardoublepage}}
\renewcommand{\syntleft}{\itshape}\renewcommand{\syntright}{}
\renewcommand{\floatpagefraction}{0.7}
\renewcommand{\partheademptypage}{}
\DeclareMicrotypeAlias{lmss}{cmr}
\newcommand{\prefix}{}
\newcounter{instruction}
\bibliographystyle{unsrt}
\newif\ifbook\booktrue
\makeindex[intoc,title=Index]
\makeindex[intoc,name=tools,title=Index of Tools,columns=3]
\makeindex[intoc,name=library,title=Index of Library Names]
\makeindex[intoc,name=runtime,title=Index of Runtime Support]
\makeindex[intoc,name=environment,title=Index of Target Environments]
\indexsetup{toclevel=chapter,headers={\indexname}{\indexname}}
\frenchspacing
\begin{document}
\pagenumbering{alph}
\begin{titlepage}\centering
\huge\sffamily\null\vfill\textbf{\ecs{}}\bigskip\hrule\bigskip#1
\normalsize\normalfont\vfill\vfill\shadowedecslogo{10em}{30}{15}
\large\vfill\vfill\version
\end{titlepage}
\null\vfill
\thispagestyle{empty}
\noindent\today\par\medskip
\license A copy of this license is included in Appendix~\ref{fdl} on page~\pageref{fdl}.
All product names used herein are for identification purposes only and may be trademarks of their respective companies.
\concludechapter
\frontmatter
\setcounter{tocdepth}{1}
\tableofcontents
\setcounter{tocdepth}{2}
\concludechapter
\listoffigures
\concludechapter
\listoftables
\concludechapter
}

\providecommand{\concludebook}{
\backmatter
\addtocontents{toc}{\protect\setcounter{tocdepth}{-1}}
\phantomsection\addcontentsline{toc}{part}{Bibliography}
\bibliography{references}
\concludechapter
\phantomsection\addcontentsline{toc}{part}{Indexes}
\printindex
\concludechapter
\indexprologue{\label{idx:tools}}
\printindex[tools]
\concludechapter
\printindex[library]
\concludechapter
\indexprologue{\label{idx:runtime}}
\printindex[runtime]
\concludechapter
\indexprologue{\label{idx:environment}}
\printindex[environment]
\concludechapter
\pagestyle{empty}\pagenumbering{Alph}\null\clearpage
\null\vfill\centering\ecslogo{4em}\par\medskip\license
\end{document}
}

% chapter references

\providecommand{\seedocumentationref}{}\renewcommand{\seedocumentationref}[3]{#1, see \Documentation{}~\documentationref{#2}{#3}. }
\providecommand{\seeinterface}{}\renewcommand{\seeinterface}{\ifbook See \Documentation{}~\documentationref{interface}{User Interface} for more information about the common user interface of all of these tools. \fi}
\providecommand{\seeguide}{}\renewcommand{\seeguide}{\seedocumentationref{For basic examples of using some of these tools in practice}{guide}{User Guide}}
\providecommand{\seecpp}{}\renewcommand{\seecpp}{\seedocumentationref{For more information about the \cpp{} programming language and its implementation by the \ecs{}}{cpp}{User Manual for \cpp{}}}
\providecommand{\seefalse}{}\renewcommand{\seefalse}{\seedocumentationref{For more information about the FALSE programming language and its implementation by the \ecs{}}{false}{User Manual for FALSE}}
\providecommand{\seeoberon}{}\renewcommand{\seeoberon}{\seedocumentationref{For more information about the Oberon programming language and its implementation by the \ecs{}}{oberon}{User Manual for Oberon}}
\providecommand{\seeassembly}{}\renewcommand{\seeassembly}{\seedocumentationref{For more information about the generic assembly language and how to use it}{assembly}{Generic Assembly Language Specification}}
\providecommand{\seeamd}{}\renewcommand{\seeamd}{\seedocumentationref{For more information about how the \ecs{} supports the AMD64 hardware architecture}{amd64}{AMD64 Hardware Architecture Support}}
\providecommand{\seearm}{}\renewcommand{\seearm}{\seedocumentationref{For more information about how the \ecs{} supports the ARM hardware architecture}{arm}{ARM Hardware Architecture Support}}
\providecommand{\seeavr}{}\renewcommand{\seeavr}{\seedocumentationref{For more information about how the \ecs{} supports the AVR hardware architecture}{avr}{AVR Hardware Architecture Support}}
\providecommand{\seeavrtt}{}\renewcommand{\seeavrtt}{\seedocumentationref{For more information about how the \ecs{} supports the AVR32 hardware architecture}{avr32}{AVR32 Hardware Architecture Support}}
\providecommand{\seemabk}{}\renewcommand{\seemabk}{\seedocumentationref{For more information about how the \ecs{} supports the M68000 hardware architecture}{m68k}{M68000 Hardware Architecture Support}}
\providecommand{\seemibl}{}\renewcommand{\seemibl}{\seedocumentationref{For more information about how the \ecs{} supports the MicroBlaze hardware architecture}{mibl}{MicroBlaze Hardware Architecture Support}}
\providecommand{\seemips}{}\renewcommand{\seemips}{\seedocumentationref{For more information about how the \ecs{} supports the MIPS32 and MIPS64 hardware architectures}{mips}{MIPS Hardware Architecture Support}}
\providecommand{\seemmix}{}\renewcommand{\seemmix}{\seedocumentationref{For more information about how the \ecs{} supports the MMIX hardware architecture}{mmix}{MMIX Hardware Architecture Support}}
\providecommand{\seeorok}{}\renewcommand{\seeorok}{\seedocumentationref{For more information about how the \ecs{} supports the OpenRISC 1000 hardware architecture}{or1k}{OpenRISC 1000 Hardware Architecture Support}}
\providecommand{\seeppc}{}\renewcommand{\seeppc}{\seedocumentationref{For more information about how the \ecs{} supports the PowerPC hardware architecture}{ppc}{PowerPC Hardware Architecture Support}}
\providecommand{\seerisc}{}\renewcommand{\seerisc}{\seedocumentationref{For more information about how the \ecs{} supports the RISC hardware architecture}{risc}{RISC Hardware Architecture Support}}
\providecommand{\seewasm}{}\renewcommand{\seewasm}{\seedocumentationref{For more information about how the \ecs{} supports the WebAssembly architecture}{wasm}{WebAssembly Architecture Support}}
\providecommand{\seedocumentation}{}\renewcommand{\seedocumentation}{\seedocumentationref{For more information about generic documentations and their generation by the \ecs{}}{documentation}{Generic Documentation Generation}}
\providecommand{\seedebugging}{}\renewcommand{\seedebugging}{\seedocumentationref{For more information about debugging information and its representation}{debugging}{Debugging Information Representation}}
\providecommand{\seecode}{}\renewcommand{\seecode}{\seedocumentationref{For more information about intermediate code and its purpose}{code}{Intermediate Code Representation}}
\providecommand{\seeobject}{}\renewcommand{\seeobject}{\seedocumentationref{For more information about object files and their purpose}{object}{Object File Representation}}

% generic documentation tools

\providecommand{\docprint}{
\toolsection{docprint} is a pretty printer for generic documentations.
It reformats generic documentations and writes it to the standard output stream.
\debuggingtool
\flowgraph{\resource{generic\\documentation} \ar[r] & \toolbox{docprint} \ar[r] & \resource{generic\\documentation}}
\seedocumentation
}

\providecommand{\doccheck}{
\toolsection{doccheck} is a syntactic and semantic checker for generic documentations.
It just performs syntactic and semantic checks on generic documentations and writes its diagnostic messages to the standard error stream.
\debuggingtool
\flowgraph{\resource{generic\\documentation} \ar[r] & \toolbox{doccheck} \ar[r] & \resource{diagnostic\\messages}}
\seedocumentation
}

\providecommand{\dochtml}{
\toolsection{dochtml} is an HTML documentation generator for generic documentations.
It processes several generic documentations and assembles all information therein into an HTML document.
\debuggingtool
\flowgraph{\resource{generic\\documentation} \ar[r] & \toolbox{dochtml} \ar[r] & \resource{HTML\\document}}
\seedocumentation
}

\providecommand{\doclatex}{
\toolsection{doclatex} is a Latex documentation generator for generic documentations.
It processes several generic documentations and assembles all information therein into a Latex document.
\debuggingtool
\flowgraph{\resource{generic\\documentation} \ar[r] & \toolbox{doclatex} \ar[r] & \resource{Latex\\document}}
\seedocumentation
}

% intermediate code tools

\providecommand{\cdcheck}{
\toolsection{cdcheck} is a syntactic and semantic checker for intermediate code.
It just performs syntactic and semantic checks on programs written in intermediate code and writes its diagnostic messages to the standard error stream.
\debuggingtool
\flowgraph{\resource{intermediate\\code} \ar[r] & \toolbox{cdcheck} \ar[r] & \resource{diagnostic\\messages}}
\seeassembly\seecode
}

\providecommand{\cdopt}{
\toolsection{cdopt} is an optimizer for intermediate code.
It performs various optimizations on programs written in intermediate code and writes the result to the standard output stream.
\debuggingtool
\flowgraph{\resource{intermediate\\code} \ar[r] & \toolbox{cdopt} \ar[r] & \resource{optimized\\code}}
\seeassembly\seecode
}

\providecommand{\cdrun}{
\toolsection{cdrun} is an interpreter for intermediate code.
It processes and executes programs written in intermediate code.
The following code sections are predefined and have the usual semantics:
\texttt{abort}, \texttt{\_Exit}, \texttt{fflush}, \texttt{floor}, \texttt{fputc}, \texttt{free}, \texttt{getchar}, \texttt{malloc}, and \texttt{putchar}.
Diagnostic messages about invalid operations include the name of the executed code section and the index of the erroneous instruction.
\debuggingtool
\flowgraph{\resource{intermediate\\code} \ar[r] & \toolbox{cdrun} \ar@/u/[r] & \resource{input/\\output} \ar@/d/[l]}
\seeassembly\seecode
}

\providecommand{\cdamda}{
\toolsection{cdamd16} is a compiler for intermediate code targeting the AMD64 hardware architecture.
It generates machine code for AMD64 processors from programs written in intermediate code and stores it in corresponding object files.
The compiler generates machine code for the 16-bit operating mode defined by the AMD64 architecture.
It also creates a debugging information file as well as an assembly file containing a listing of the generated machine code.
\debuggingtool
\flowgraph{\resource{intermediate\\code} \ar[r] & \toolbox{cdamd16} \ar[r] \ar[d] \ar[rd] & \resource{object file} \\ & \resource{assembly\\listing} & \resource{debugging\\information}}
\seeassembly\seeamd\seeobject\seecode\seedebugging
}

\providecommand{\cdamdb}{
\toolsection{cdamd32} is a compiler for intermediate code targeting the AMD64 hardware architecture.
It generates machine code for AMD64 processors from programs written in intermediate code and stores it in corresponding object files.
The compiler generates machine code for the 32-bit operating mode defined by the AMD64 architecture.
It also creates a debugging information file as well as an assembly file containing a listing of the generated machine code.
\debuggingtool
\flowgraph{\resource{intermediate\\code} \ar[r] & \toolbox{cdamd32} \ar[r] \ar[d] \ar[rd] & \resource{object file} \\ & \resource{assembly\\listing} & \resource{debugging\\information}}
\seeassembly\seeamd\seeobject\seecode\seedebugging
}

\providecommand{\cdamdc}{
\toolsection{cdamd64} is a compiler for intermediate code targeting the AMD64 hardware architecture.
It generates machine code for AMD64 processors from programs written in intermediate code and stores it in corresponding object files.
The compiler generates machine code for the 64-bit operating mode defined by the AMD64 architecture.
It also creates a debugging information file as well as an assembly file containing a listing of the generated machine code.
\debuggingtool
\flowgraph{\resource{intermediate\\code} \ar[r] & \toolbox{cdamd64} \ar[r] \ar[d] \ar[rd] & \resource{object file} \\ & \resource{assembly\\listing} & \resource{debugging\\information}}
\seeassembly\seeamd\seeobject\seecode\seedebugging
}

\providecommand{\cdarma}{
\toolsection{cdarma32} is a compiler for intermediate code targeting the ARM hardware architecture.
It generates machine code for ARM processors executing A32 instructions from programs written in intermediate code and stores it in corresponding object files.
It also creates a debugging information file as well as an assembly file containing a listing of the generated machine code.
\debuggingtool
\flowgraph{\resource{intermediate\\code} \ar[r] & \toolbox{cdarma32} \ar[r] \ar[d] \ar[rd] & \resource{object file} \\ & \resource{assembly\\listing} & \resource{debugging\\information}}
\seeassembly\seearm\seeobject\seecode\seedebugging
}

\providecommand{\cdarmb}{
\toolsection{cdarma64} is a compiler for intermediate code targeting the ARM hardware architecture.
It generates machine code for ARM processors executing A64 instructions from programs written in intermediate code and stores it in corresponding object files.
It also creates a debugging information file as well as an assembly file containing a listing of the generated machine code.
\debuggingtool
\flowgraph{\resource{intermediate\\code} \ar[r] & \toolbox{cdarma64} \ar[r] \ar[d] \ar[rd] & \resource{object file} \\ & \resource{assembly\\listing} & \resource{debugging\\information}}
\seeassembly\seearm\seeobject\seecode\seedebugging
}

\providecommand{\cdarmc}{
\toolsection{cdarmt32} is a compiler for intermediate code targeting the ARM hardware architecture.
It generates machine code for ARM processors without floating-point extension executing T32 instructions from programs written in intermediate code and stores it in corresponding object files.
It also creates a debugging information file as well as an assembly file containing a listing of the generated machine code.
\debuggingtool
\flowgraph{\resource{intermediate\\code} \ar[r] & \toolbox{cdarmt32} \ar[r] \ar[d] \ar[rd] & \resource{object file} \\ & \resource{assembly\\listing} & \resource{debugging\\information}}
\seeassembly\seearm\seeobject\seecode\seedebugging
}

\providecommand{\cdarmcfpe}{
\toolsection{cdarmt32fpe} is a compiler for intermediate code targeting the ARM hardware architecture.
It generates machine code for ARM processors with floating-point extension executing T32 instructions from programs written in intermediate code and stores it in corresponding object files.
It also creates a debugging information file as well as an assembly file containing a listing of the generated machine code.
\debuggingtool
\flowgraph{\resource{intermediate\\code} \ar[r] & \toolbox{cdarmt32fpe} \ar[r] \ar[d] \ar[rd] & \resource{object file} \\ & \resource{assembly\\listing} & \resource{debugging\\information}}
\seeassembly\seearm\seeobject\seecode\seedebugging
}

\providecommand{\cdavr}{
\toolsection{cdavr} is a compiler for intermediate code targeting the AVR hardware architecture.
It generates machine code for AVR processors from programs written in intermediate code and stores it in corresponding object files.
It also creates a debugging information file as well as an assembly file containing a listing of the generated machine code.
\debuggingtool
\flowgraph{\resource{intermediate\\code} \ar[r] & \toolbox{cdavr} \ar[r] \ar[d] \ar[rd] & \resource{object file} \\ & \resource{assembly\\listing} & \resource{debugging\\information}}
\seeassembly\seeavr\seeobject\seecode\seedebugging
}

\providecommand{\cdavrtt}{
\toolsection{cdavr32} is a compiler for intermediate code targeting the AVR32 hardware architecture.
It generates machine code for AVR32 processors from programs written in intermediate code and stores it in corresponding object files.
It also creates a debugging information file as well as an assembly file containing a listing of the generated machine code.
\debuggingtool
\flowgraph{\resource{intermediate\\code} \ar[r] & \toolbox{cdavr32} \ar[r] \ar[d] \ar[rd] & \resource{object file} \\ & \resource{assembly\\listing} & \resource{debugging\\information}}
\seeassembly\seeavrtt\seeobject\seecode\seedebugging
}

\providecommand{\cdmabk}{
\toolsection{cdm68k} is a compiler for intermediate code targeting the M68000 hardware architecture.
It generates machine code for M68000 processors from programs written in intermediate code and stores it in corresponding object files.
It also creates a debugging information file as well as an assembly file containing a listing of the generated machine code.
\debuggingtool
\flowgraph{\resource{intermediate\\code} \ar[r] & \toolbox{cdm68k} \ar[r] \ar[d] \ar[rd] & \resource{object file} \\ & \resource{assembly\\listing} & \resource{debugging\\information}}
\seeassembly\seemabk\seeobject\seecode\seedebugging
}

\providecommand{\cdmibl}{
\toolsection{cdmibl} is a compiler for intermediate code targeting the MicroBlaze hardware architecture.
It generates machine code for MicroBlaze processors from programs written in intermediate code and stores it in corresponding object files.
It also creates a debugging information file as well as an assembly file containing a listing of the generated machine code.
\debuggingtool
\flowgraph{\resource{intermediate\\code} \ar[r] & \toolbox{cdmibl} \ar[r] \ar[d] \ar[rd] & \resource{object file} \\ & \resource{assembly\\listing} & \resource{debugging\\information}}
\seeassembly\seemibl\seeobject\seecode\seedebugging
}

\providecommand{\cdmipsa}{
\toolsection{cdmips32} is a compiler for intermediate code targeting the MIPS32 hardware architecture.
It generates machine code for MIPS32 processors from programs written in intermediate code and stores it in corresponding object files.
It also creates a debugging information file as well as an assembly file containing a listing of the generated machine code.
\debuggingtool
\flowgraph{\resource{intermediate\\code} \ar[r] & \toolbox{cdmips32} \ar[r] \ar[d] \ar[rd] & \resource{object file} \\ & \resource{assembly\\listing} & \resource{debugging\\information}}
\seeassembly\seemips\seeobject\seecode\seedebugging
}

\providecommand{\cdmipsb}{
\toolsection{cdmips64} is a compiler for intermediate code targeting the MIPS64 hardware architecture.
It generates machine code for MIPS64 processors from programs written in intermediate code and stores it in corresponding object files.
It also creates a debugging information file as well as an assembly file containing a listing of the generated machine code.
\debuggingtool
\flowgraph{\resource{intermediate\\code} \ar[r] & \toolbox{cdmips64} \ar[r] \ar[d] \ar[rd] & \resource{object file} \\ & \resource{assembly\\listing} & \resource{debugging\\information}}
\seeassembly\seemips\seeobject\seecode\seedebugging
}

\providecommand{\cdmmix}{
\toolsection{cdmmix} is a compiler for intermediate code targeting the MMIX hardware architecture.
It generates machine code for MMIX processors from programs written in intermediate code and stores it in corresponding object files.
It also creates a debugging information file as well as an assembly file containing a listing of the generated machine code.
\debuggingtool
\flowgraph{\resource{intermediate\\code} \ar[r] & \toolbox{cdmmix} \ar[r] \ar[d] \ar[rd] & \resource{object file} \\ & \resource{assembly\\listing} & \resource{debugging\\information}}
\seeassembly\seemmix\seeobject\seecode\seedebugging
}

\providecommand{\cdorok}{
\toolsection{cdor1k} is a compiler for intermediate code targeting the OpenRISC 1000 hardware architecture.
It generates machine code for OpenRISC 1000 processors from programs written in intermediate code and stores it in corresponding object files.
It also creates a debugging information file as well as an assembly file containing a listing of the generated machine code.
\debuggingtool
\flowgraph{\resource{intermediate\\code} \ar[r] & \toolbox{cdor1k} \ar[r] \ar[d] \ar[rd] & \resource{object file} \\ & \resource{assembly\\listing} & \resource{debugging\\information}}
\seeassembly\seeorok\seeobject\seecode\seedebugging
}

\providecommand{\cdppca}{
\toolsection{cdppc32} is a compiler for intermediate code targeting the PowerPC hardware architecture.
It generates machine code for PowerPC processors from programs written in intermediate code and stores it in corresponding object files.
The compiler generates machine code for the 32-bit operating mode defined by the PowerPC architecture.
It also creates a debugging information file as well as an assembly file containing a listing of the generated machine code.
\debuggingtool
\flowgraph{\resource{intermediate\\code} \ar[r] & \toolbox{cdppc32} \ar[r] \ar[d] \ar[rd] & \resource{object file} \\ & \resource{assembly\\listing} & \resource{debugging\\information}}
\seeassembly\seeppc\seeobject\seecode\seedebugging
}

\providecommand{\cdppcb}{
\toolsection{cdppc64} is a compiler for intermediate code targeting the PowerPC hardware architecture.
It generates machine code for PowerPC processors from programs written in intermediate code and stores it in corresponding object files.
The compiler generates machine code for the 64-bit operating mode defined by the PowerPC architecture.
It also creates a debugging information file as well as an assembly file containing a listing of the generated machine code.
\debuggingtool
\flowgraph{\resource{intermediate\\code} \ar[r] & \toolbox{cdppc64} \ar[r] \ar[d] \ar[rd] & \resource{object file} \\ & \resource{assembly\\listing} & \resource{debugging\\information}}
\seeassembly\seeppc\seeobject\seecode\seedebugging
}

\providecommand{\cdrisc}{
\toolsection{cdrisc} is a compiler for intermediate code targeting the RISC hardware architecture.
It generates machine code for RISC processors from programs written in intermediate code and stores it in corresponding object files.
It also creates a debugging information file as well as an assembly file containing a listing of the generated machine code.
\debuggingtool
\flowgraph{\resource{intermediate\\code} \ar[r] & \toolbox{cdrisc} \ar[r] \ar[d] \ar[rd] & \resource{object file} \\ & \resource{assembly\\listing} & \resource{debugging\\information}}
\seeassembly\seerisc\seeobject\seecode\seedebugging
}

\providecommand{\cdwasm}{
\toolsection{cdwasm} is a compiler for intermediate code targeting the WebAssembly architecture.
It generates machine code for WebAssembly targets from programs written in intermediate code and stores it in corresponding object files.
It also creates a debugging information file as well as an assembly file containing a listing of the generated machine code.
\debuggingtool
\flowgraph{\resource{intermediate\\code} \ar[r] & \toolbox{cdwasm} \ar[r] \ar[d] \ar[rd] & \resource{object file} \\ & \resource{assembly\\listing} & \resource{debugging\\information}}
\seeassembly\seewasm\seeobject\seecode\seedebugging
}

% C++ tools

\providecommand{\cppprep}{
\toolsection{cppprep} is a preprocessor for the \cpp{} programming language.
It preprocesses source code according to the rules of \cpp{} and writes it to the standard output stream.
Only the macro names \texttt{\_\_DATE\_\_}, \texttt{\_\_FILE\_\_}, \texttt{\_\_LINE\_\_}, and \texttt{\_\_TIME\_\_} are predefined.
\flowgraph{\resource{\cpp{} or other\\source code} \ar[r] & \toolbox{cppprep} \ar[r] & \resource{preprocessed\\source code} \\ & \variable{ECSINCLUDE} \ar[u]}
\seecpp
}

\providecommand{\cppprint}{
\toolsection{cppprint} is a pretty printer for the \cpp{} programming language.
It reformats the source code of \cpp{} programs and writes it to the standard output stream.
\flowgraph{\resource{\cpp{}\\source code} \ar[r] & \toolbox{cppprint} \ar[r] & \resource{reformatted\\source code} \\ & \variable{ECSINCLUDE} \ar[u]}
\seecpp
}

\providecommand{\cppcheck}{
\toolsection{cppcheck} is a syntactic and semantic checker for the \cpp{} programming language.
It just performs syntactic and semantic checks on \cpp{} programs and writes its diagnostic messages to the standard error stream.
\flowgraph{\resource{\cpp{}\\source code} \ar[r] & \toolbox{cppcheck} \ar[r] & \resource{diagnostic\\messages} \\ & \variable{ECSINCLUDE} \ar[u]}
\seecpp
}

\providecommand{\cppdump}{
\toolsection{cppdump} is a serializer for the \cpp{} programming language.
It dumps the complete internal representation of programs written in \cpp{} into an XML document.
\debuggingtool
\flowgraph{\resource{\cpp{}\\source code} \ar[r] & \toolbox{cppdump} \ar[r] & \resource{internal\\representation} \\ & \variable{ECSINCLUDE} \ar[u]}
\seecpp
}

\providecommand{\cpprun}{
\toolsection{cpprun} is an interpreter for the \cpp{} programming language.
It processes and executes programs written in \cpp{}.
The macro \texttt{\_\_run\_\_} is predefined in order to enable programmers to identify this tool while interpreting.
\flowgraph{\resource{\cpp{}\\source code} \ar[r] & \toolbox{cpprun} \ar@/u/[r] & \resource{input/\\output} \ar@/d/[l] \\ & \variable{ECSINCLUDE} \ar[u]}
\seecpp
}

\providecommand{\cppdoc}{
\toolsection{cppdoc} is a generic documentation generator for the \cpp{} programming language.
It processes several \cpp{} source files and assembles all information therein into a generic documentation.
\debuggingtool
\flowgraph{\resource{\cpp{}\\source code} \ar[r] & \toolbox{cppdoc} \ar[r] & \resource{generic\\documentation} \\ & \variable{ECSINCLUDE} \ar[u]}
\seecpp\seedocumentation
}

\providecommand{\cpphtml}{
\toolsection{cpphtml} is an HTML documentation generator for the \cpp{} programming language.
It processes several \cpp{} source files and assembles all information therein into an HTML document.
\flowgraph{\resource{\cpp{}\\source code} \ar[r] & \toolbox{cpphtml} \ar[r] & \resource{HTML\\document} \\ & \variable{ECSINCLUDE} \ar[u]}
\seecpp\seedocumentation
}

\providecommand{\cpplatex}{
\toolsection{cpplatex} is a Latex documentation generator for the \cpp{} programming language.
It processes several \cpp{} source files and assembles all information therein into a Latex document.
\flowgraph{\resource{\cpp{}\\source code} \ar[r] & \toolbox{cpplatex} \ar[r] & \resource{Latex\\document} \\ & \variable{ECSINCLUDE} \ar[u]}
\seecpp\seedocumentation
}

\providecommand{\cppcode}{
\toolsection{cppcode} is an intermediate code generator for the \cpp{} programming language.
It generates intermediate code from programs written in \cpp{} and stores it in corresponding assembly files.
The macro \texttt{\_\_code\_\_} is predefined in order to enable programmers to identify this tool while generating intermediate code.
Programs generated with this tool require additional runtime support that is stored in the \file{cpp\-code\-run} library file.
\debuggingtool
\flowgraph{\resource{\cpp{}\\source code} \ar[r] & \toolbox{cppcode} \ar[r] & \resource{intermediate\\code} \\ & \variable{ECSINCLUDE} \ar[u]}
\seecpp\seeassembly\seecode
}

\providecommand{\cppamda}{
\toolsection{cppamd16} is a compiler for the \cpp{} programming language targeting the AMD64 hardware architecture.
It generates machine code for AMD64 processors from programs written in \cpp{} and stores it in corresponding object files.
The compiler generates machine code for the 16-bit operating mode defined by the AMD64 architecture.
For debugging purposes, it also creates a debugging information file as well as an assembly file containing a listing of the generated machine code.
The macro \texttt{\_\_amd16\_\_} is predefined in order to enable programmers to identify this tool and its target architecture while compiling.
Programs generated with this compiler require additional runtime support that is stored in the \file{cpp\-amd16\-run} library file.
\flowgraph{\resource{\cpp{}\\source code} \ar[r] & \toolbox{cppamd16} \ar[r] \ar[d] \ar[rd] & \resource{object file} \\ \variable{ECSINCLUDE} \ar[ru] & \resource{debugging\\information} & \resource{assembly\\listing}}
\seecpp\seeassembly\seeamd\seeobject\seedebugging
}

\providecommand{\cppamdb}{
\toolsection{cppamd32} is a compiler for the \cpp{} programming language targeting the AMD64 hardware architecture.
It generates machine code for AMD64 processors from programs written in \cpp{} and stores it in corresponding object files.
The compiler generates machine code for the 32-bit operating mode defined by the AMD64 architecture.
For debugging purposes, it also creates a debugging information file as well as an assembly file containing a listing of the generated machine code.
The macro \texttt{\_\_amd32\_\_} is predefined in order to enable programmers to identify this tool and its target architecture while compiling.
Programs generated with this compiler require additional runtime support that is stored in the \file{cpp\-amd32\-run} library file.
\flowgraph{\resource{\cpp{}\\source code} \ar[r] & \toolbox{cppamd32} \ar[r] \ar[d] \ar[rd] & \resource{object file} \\ \variable{ECSINCLUDE} \ar[ru] & \resource{debugging\\information} & \resource{assembly\\listing}}
\seecpp\seeassembly\seeamd\seeobject\seedebugging
}

\providecommand{\cppamdc}{
\toolsection{cppamd64} is a compiler for the \cpp{} programming language targeting the AMD64 hardware architecture.
It generates machine code for AMD64 processors from programs written in \cpp{} and stores it in corresponding object files.
The compiler generates machine code for the 64-bit operating mode defined by the AMD64 architecture.
For debugging purposes, it also creates a debugging information file as well as an assembly file containing a listing of the generated machine code.
The macro \texttt{\_\_amd64\_\_} is predefined in order to enable programmers to identify this tool and its target architecture while compiling.
Programs generated with this compiler require additional runtime support that is stored in the \file{cpp\-amd64\-run} library file.
\flowgraph{\resource{\cpp{}\\source code} \ar[r] & \toolbox{cppamd64} \ar[r] \ar[d] \ar[rd] & \resource{object file} \\ \variable{ECSINCLUDE} \ar[ru] & \resource{debugging\\information} & \resource{assembly\\listing}}
\seecpp\seeassembly\seeamd\seeobject\seedebugging
}

\providecommand{\cpparma}{
\toolsection{cpparma32} is a compiler for the \cpp{} programming language targeting the ARM hardware architecture.
It generates machine code for ARM processors executing A32 instructions from programs written in \cpp{} and stores it in corresponding object files.
For debugging purposes, it also creates a debugging information file as well as an assembly file containing a listing of the generated machine code.
The macro \texttt{\_\_arma32\_\_} is predefined in order to enable programmers to identify this tool and its target architecture while compiling.
Programs generated with this compiler require additional runtime support that is stored in the \file{cpp\-arma32\-run} library file.
\flowgraph{\resource{\cpp{}\\source code} \ar[r] & \toolbox{cpparma32} \ar[r] \ar[d] \ar[rd] & \resource{object file} \\ \variable{ECSINCLUDE} \ar[ru] & \resource{debugging\\information} & \resource{assembly\\listing}}
\seecpp\seeassembly\seearm\seeobject\seedebugging
}

\providecommand{\cpparmb}{
\toolsection{cpparma64} is a compiler for the \cpp{} programming language targeting the ARM hardware architecture.
It generates machine code for ARM processors executing A64 instructions from programs written in \cpp{} and stores it in corresponding object files.
For debugging purposes, it also creates a debugging information file as well as an assembly file containing a listing of the generated machine code.
The macro \texttt{\_\_arma64\_\_} is predefined in order to enable programmers to identify this tool and its target architecture while compiling.
Programs generated with this compiler require additional runtime support that is stored in the \file{cpp\-arma64\-run} library file.
\flowgraph{\resource{\cpp{}\\source code} \ar[r] & \toolbox{cpparma64} \ar[r] \ar[d] \ar[rd] & \resource{object file} \\ \variable{ECSINCLUDE} \ar[ru] & \resource{debugging\\information} & \resource{assembly\\listing}}
\seecpp\seeassembly\seearm\seeobject\seedebugging
}

\providecommand{\cpparmc}{
\toolsection{cpparmt32} is a compiler for the \cpp{} programming language targeting the ARM hardware architecture.
It generates machine code for ARM processors without floating-point extension executing T32 instructions from programs written in \cpp{} and stores it in corresponding object files.
For debugging purposes, it also creates a debugging information file as well as an assembly file containing a listing of the generated machine code.
The macro \texttt{\_\_armt32\_\_} is predefined in order to enable programmers to identify this tool and its target architecture while compiling.
Programs generated with this compiler require additional runtime support that is stored in the \file{cpp\-armt32\-run} library file.
\flowgraph{\resource{\cpp{}\\source code} \ar[r] & \toolbox{cpparmt32} \ar[r] \ar[d] \ar[rd] & \resource{object file} \\ \variable{ECSINCLUDE} \ar[ru] & \resource{debugging\\information} & \resource{assembly\\listing}}
\seecpp\seeassembly\seearm\seeobject\seedebugging
}

\providecommand{\cpparmcfpe}{
\toolsection{cpparmt32fpe} is a compiler for the \cpp{} programming language targeting the ARM hardware architecture.
It generates machine code for ARM processors with floating-point extension executing T32 instructions from programs written in \cpp{} and stores it in corresponding object files.
For debugging purposes, it also creates a debugging information file as well as an assembly file containing a listing of the generated machine code.
The macro \texttt{\_\_armt32fpe\_\_} is predefined in order to enable programmers to identify this tool and its target architecture while compiling.
Programs generated with this compiler require additional runtime support that is stored in the \file{cpp\-armt32\-fpe\-run} library file.
\flowgraph{\resource{\cpp{}\\source code} \ar[r] & \toolbox{cpparmt32fpe} \ar[r] \ar[d] \ar[rd] & \resource{object file} \\ \variable{ECSINCLUDE} \ar[ru] & \resource{debugging\\information} & \resource{assembly\\listing}}
\seecpp\seeassembly\seearm\seeobject\seedebugging
}

\providecommand{\cppavr}{
\toolsection{cppavr} is a compiler for the \cpp{} programming language targeting the AVR hardware architecture.
It generates machine code for AVR processors from programs written in \cpp{} and stores it in corresponding object files.
For debugging purposes, it also creates a debugging information file as well as an assembly file containing a listing of the generated machine code.
The macro \texttt{\_\_avr\_\_} is predefined in order to enable programmers to identify this tool and its target architecture while compiling.
Programs generated with this compiler require additional runtime support that is stored in the \file{cpp\-avr\-run} library file.
\flowgraph{\resource{\cpp{}\\source code} \ar[r] & \toolbox{cppavr} \ar[r] \ar[d] \ar[rd] & \resource{object file} \\ \variable{ECSINCLUDE} \ar[ru] & \resource{debugging\\information} & \resource{assembly\\listing}}
\seecpp\seeassembly\seeavr\seeobject\seedebugging
}

\providecommand{\cppavrtt}{
\toolsection{cppavr32} is a compiler for the \cpp{} programming language targeting the AVR32 hardware architecture.
It generates machine code for AVR32 processors from programs written in \cpp{} and stores it in corresponding object files.
For debugging purposes, it also creates a debugging information file as well as an assembly file containing a listing of the generated machine code.
The macro \texttt{\_\_avr32\_\_} is predefined in order to enable programmers to identify this tool and its target architecture while compiling.
Programs generated with this compiler require additional runtime support that is stored in the \file{cpp\-avr32\-run} library file.
\flowgraph{\resource{\cpp{}\\source code} \ar[r] & \toolbox{cppavr32} \ar[r] \ar[d] \ar[rd] & \resource{object file} \\ \variable{ECSINCLUDE} \ar[ru] & \resource{debugging\\information} & \resource{assembly\\listing}}
\seecpp\seeassembly\seeavrtt\seeobject\seedebugging
}

\providecommand{\cppmabk}{
\toolsection{cppm68k} is a compiler for the \cpp{} programming language targeting the M68000 hardware architecture.
It generates machine code for M68000 processors from programs written in \cpp{} and stores it in corresponding object files.
For debugging purposes, it also creates a debugging information file as well as an assembly file containing a listing of the generated machine code.
The macro \texttt{\_\_m68k\_\_} is predefined in order to enable programmers to identify this tool and its target architecture while compiling.
Programs generated with this compiler require additional runtime support that is stored in the \file{cpp\-m68k\-run} library file.
\flowgraph{\resource{\cpp{}\\source code} \ar[r] & \toolbox{cppm68k} \ar[r] \ar[d] \ar[rd] & \resource{object file} \\ \variable{ECSINCLUDE} \ar[ru] & \resource{debugging\\information} & \resource{assembly\\listing}}
\seecpp\seeassembly\seemabk\seeobject\seedebugging
}

\providecommand{\cppmibl}{
\toolsection{cppmibl} is a compiler for the \cpp{} programming language targeting the MicroBlaze hardware architecture.
It generates machine code for MicroBlaze processors from programs written in \cpp{} and stores it in corresponding object files.
For debugging purposes, it also creates a debugging information file as well as an assembly file containing a listing of the generated machine code.
The macro \texttt{\_\_mibl\_\_} is predefined in order to enable programmers to identify this tool and its target architecture while compiling.
Programs generated with this compiler require additional runtime support that is stored in the \file{cpp\-mibl\-run} library file.
\flowgraph{\resource{\cpp{}\\source code} \ar[r] & \toolbox{cppmibl} \ar[r] \ar[d] \ar[rd] & \resource{object file} \\ \variable{ECSINCLUDE} \ar[ru] & \resource{debugging\\information} & \resource{assembly\\listing}}
\seecpp\seeassembly\seemibl\seeobject\seedebugging
}

\providecommand{\cppmipsa}{
\toolsection{cppmips32} is a compiler for the \cpp{} programming language targeting the MIPS32 hardware architecture.
It generates machine code for MIPS32 processors from programs written in \cpp{} and stores it in corresponding object files.
For debugging purposes, it also creates a debugging information file as well as an assembly file containing a listing of the generated machine code.
The macro \texttt{\_\_mips32\_\_} is predefined in order to enable programmers to identify this tool and its target architecture while compiling.
Programs generated with this compiler require additional runtime support that is stored in the \file{cpp\-mips32\-run} library file.
\flowgraph{\resource{\cpp{}\\source code} \ar[r] & \toolbox{cppmips32} \ar[r] \ar[d] \ar[rd] & \resource{object file} \\ \variable{ECSINCLUDE} \ar[ru] & \resource{debugging\\information} & \resource{assembly\\listing}}
\seecpp\seeassembly\seemips\seeobject\seedebugging
}

\providecommand{\cppmipsb}{
\toolsection{cppmips64} is a compiler for the \cpp{} programming language targeting the MIPS64 hardware architecture.
It generates machine code for MIPS64 processors from programs written in \cpp{} and stores it in corresponding object files.
For debugging purposes, it also creates a debugging information file as well as an assembly file containing a listing of the generated machine code.
The macro \texttt{\_\_mips64\_\_} is predefined in order to enable programmers to identify this tool and its target architecture while compiling.
Programs generated with this compiler require additional runtime support that is stored in the \file{cpp\-mips64\-run} library file.
\flowgraph{\resource{\cpp{}\\source code} \ar[r] & \toolbox{cppmips64} \ar[r] \ar[d] \ar[rd] & \resource{object file} \\ \variable{ECSINCLUDE} \ar[ru] & \resource{debugging\\information} & \resource{assembly\\listing}}
\seecpp\seeassembly\seemips\seeobject\seedebugging
}

\providecommand{\cppmmix}{
\toolsection{cppmmix} is a compiler for the \cpp{} programming language targeting the MMIX hardware architecture.
It generates machine code for MMIX processors from programs written in \cpp{} and stores it in corresponding object files.
For debugging purposes, it also creates a debugging information file as well as an assembly file containing a listing of the generated machine code.
The macro \texttt{\_\_mmix\_\_} is predefined in order to enable programmers to identify this tool and its target architecture while compiling.
Programs generated with this compiler require additional runtime support that is stored in the \file{cpp\-mmix\-run} library file.
\flowgraph{\resource{\cpp{}\\source code} \ar[r] & \toolbox{cppmmix} \ar[r] \ar[d] \ar[rd] & \resource{object file} \\ \variable{ECSINCLUDE} \ar[ru] & \resource{debugging\\information} & \resource{assembly\\listing}}
\seecpp\seeassembly\seemmix\seeobject\seedebugging
}

\providecommand{\cpporok}{
\toolsection{cppor1k} is a compiler for the \cpp{} programming language targeting the OpenRISC 1000 hardware architecture.
It generates machine code for OpenRISC 1000 processors from programs written in \cpp{} and stores it in corresponding object files.
For debugging purposes, it also creates a debugging information file as well as an assembly file containing a listing of the generated machine code.
The macro \texttt{\_\_or1k\_\_} is predefined in order to enable programmers to identify this tool and its target architecture while compiling.
Programs generated with this compiler require additional runtime support that is stored in the \file{cpp\-or1k\-run} library file.
\flowgraph{\resource{\cpp{}\\source code} \ar[r] & \toolbox{cppor1k} \ar[r] \ar[d] \ar[rd] & \resource{object file} \\ \variable{ECSINCLUDE} \ar[ru] & \resource{debugging\\information} & \resource{assembly\\listing}}
\seecpp\seeassembly\seeorok\seeobject\seedebugging
}

\providecommand{\cppppca}{
\toolsection{cppppc32} is a compiler for the \cpp{} programming language targeting the PowerPC hardware architecture.
It generates machine code for PowerPC processors from programs written in \cpp{} and stores it in corresponding object files.
The compiler generates machine code for the 32-bit operating mode defined by the PowerPC architecture.
For debugging purposes, it also creates a debugging information file as well as an assembly file containing a listing of the generated machine code.
The macro \texttt{\_\_ppc32\_\_} is predefined in order to enable programmers to identify this tool and its target architecture while compiling.
Programs generated with this compiler require additional runtime support that is stored in the \file{cpp\-ppc32\-run} library file.
\flowgraph{\resource{\cpp{}\\source code} \ar[r] & \toolbox{cppppc32} \ar[r] \ar[d] \ar[rd] & \resource{object file} \\ \variable{ECSINCLUDE} \ar[ru] & \resource{debugging\\information} & \resource{assembly\\listing}}
\seecpp\seeassembly\seeppc\seeobject\seedebugging
}

\providecommand{\cppppcb}{
\toolsection{cppppc64} is a compiler for the \cpp{} programming language targeting the PowerPC hardware architecture.
It generates machine code for PowerPC processors from programs written in \cpp{} and stores it in corresponding object files.
The compiler generates machine code for the 64-bit operating mode defined by the PowerPC architecture.
For debugging purposes, it also creates a debugging information file as well as an assembly file containing a listing of the generated machine code.
The macro \texttt{\_\_ppc64\_\_} is predefined in order to enable programmers to identify this tool and its target architecture while compiling.
Programs generated with this compiler require additional runtime support that is stored in the \file{cpp\-ppc64\-run} library file.
\flowgraph{\resource{\cpp{}\\source code} \ar[r] & \toolbox{cppppc64} \ar[r] \ar[d] \ar[rd] & \resource{object file} \\ \variable{ECSINCLUDE} \ar[ru] & \resource{debugging\\information} & \resource{assembly\\listing}}
\seecpp\seeassembly\seeppc\seeobject\seedebugging
}

\providecommand{\cpprisc}{
\toolsection{cpprisc} is a compiler for the \cpp{} programming language targeting the RISC hardware architecture.
It generates machine code for RISC processors from programs written in \cpp{} and stores it in corresponding object files.
For debugging purposes, it also creates a debugging information file as well as an assembly file containing a listing of the generated machine code.
The macro \texttt{\_\_risc\_\_} is predefined in order to enable programmers to identify this tool and its target architecture while compiling.
Programs generated with this compiler require additional runtime support that is stored in the \file{cpp\-risc\-run} library file.
\flowgraph{\resource{\cpp{}\\source code} \ar[r] & \toolbox{cpprisc} \ar[r] \ar[d] \ar[rd] & \resource{object file} \\ \variable{ECSINCLUDE} \ar[ru] & \resource{debugging\\information} & \resource{assembly\\listing}}
\seecpp\seeassembly\seerisc\seeobject\seedebugging
}

\providecommand{\cppwasm}{
\toolsection{cppwasm} is a compiler for the \cpp{} programming language targeting the WebAssembly architecture.
It generates machine code for WebAssembly targets from programs written in \cpp{} and stores it in corresponding object files.
For debugging purposes, it also creates a debugging information file as well as an assembly file containing a listing of the generated machine code.
The macro \texttt{\_\_wasm\_\_} is predefined in order to enable programmers to identify this tool and its target architecture while compiling.
Programs generated with this compiler require additional runtime support that is stored in the \file{cpp\-wasm\-run} library file.
\flowgraph{\resource{\cpp{}\\source code} \ar[r] & \toolbox{cppwasm} \ar[r] \ar[d] \ar[rd] & \resource{object file} \\ \variable{ECSINCLUDE} \ar[ru] & \resource{debugging\\information} & \resource{assembly\\listing}}
\seecpp\seeassembly\seewasm\seeobject\seedebugging
}

% FALSE tools

\providecommand{\falprint}{
\toolsection{falprint} is a pretty printer for the FALSE programming language.
It reformats the source code of FALSE programs and writes it to the standard output stream.
\flowgraph{\resource{FALSE\\source code} \ar[r] & \toolbox{falprint} \ar[r] & \resource{reformatted\\source code}}
\seefalse
}

\providecommand{\falcheck}{
\toolsection{falcheck} is a syntactic and semantic checker for the FALSE programming language.
It just performs syntactic and semantic checks on FALSE programs and writes its diagnostic messages to the standard error stream.
\flowgraph{\resource{FALSE\\source code} \ar[r] & \toolbox{falcheck} \ar[r] & \resource{diagnostic\\messages}}
\seefalse
}

\providecommand{\faldump}{
\toolsection{faldump} is a serializer for the FALSE programming language.
It dumps the complete internal representation of programs written in FALSE into an XML document.
\debuggingtool
\flowgraph{\resource{FALSE\\source code} \ar[r] & \toolbox{faldump} \ar[r] & \resource{internal\\representation}}
\seefalse
}

\providecommand{\falrun}{
\toolsection{falrun} is an interpreter for the FALSE programming language.
It processes and executes programs written in FALSE\@.
\flowgraph{\resource{FALSE\\source code} \ar[r] & \toolbox{falrun} \ar@/u/[r] & \resource{input/\\output} \ar@/d/[l]}
\seefalse
}

\providecommand{\falcpp}{
\toolsection{falcpp} is a transpiler for the FALSE programming language.
It translates programs written in FALSE into \cpp{} programs and stores them in corresponding source files.
\flowgraph{\resource{FALSE\\source code} \ar[r] & \toolbox{falcpp} \ar[r] & \resource{\cpp{}\\source file}}
\seefalse\seecpp
}

\providecommand{\falcode}{
\toolsection{falcode} is an intermediate code generator for the FALSE programming language.
It generates intermediate code from programs written in FALSE and stores it in corresponding assembly files.
\debuggingtool
\flowgraph{\resource{FALSE\\source code} \ar[r] & \toolbox{falcode} \ar[r] & \resource{intermediate\\code}}
\seefalse\seeassembly\seecode
}

\providecommand{\falamda}{
\toolsection{falamd16} is a compiler for the FALSE programming language targeting the AMD64 hardware architecture.
It generates machine code for AMD64 processors from programs written in FALSE and stores it in corresponding object files.
The compiler generates machine code for the 16-bit operating mode defined by the AMD64 architecture.
\flowgraph{\resource{FALSE\\source code} \ar[r] & \toolbox{falamd16} \ar[r] & \resource{object file}}
\seefalse\seeamd\seeobject
}

\providecommand{\falamdb}{
\toolsection{falamd32} is a compiler for the FALSE programming language targeting the AMD64 hardware architecture.
It generates machine code for AMD64 processors from programs written in FALSE and stores it in corresponding object files.
The compiler generates machine code for the 32-bit operating mode defined by the AMD64 architecture.
\flowgraph{\resource{FALSE\\source code} \ar[r] & \toolbox{falamd32} \ar[r] & \resource{object file}}
\seefalse\seeamd\seeobject
}

\providecommand{\falamdc}{
\toolsection{falamd64} is a compiler for the FALSE programming language targeting the AMD64 hardware architecture.
It generates machine code for AMD64 processors from programs written in FALSE and stores it in corresponding object files.
The compiler generates machine code for the 64-bit operating mode defined by the AMD64 architecture.
\flowgraph{\resource{FALSE\\source code} \ar[r] & \toolbox{falamd64} \ar[r] & \resource{object file}}
\seefalse\seeamd\seeobject
}

\providecommand{\falarma}{
\toolsection{falarma32} is a compiler for the FALSE programming language targeting the ARM hardware architecture.
It generates machine code for ARM processors executing A32 instructions from programs written in FALSE and stores it in corresponding object files.
\flowgraph{\resource{FALSE\\source code} \ar[r] & \toolbox{falarma32} \ar[r] & \resource{object file}}
\seefalse\seearm\seeobject
}

\providecommand{\falarmb}{
\toolsection{falarma64} is a compiler for the FALSE programming language targeting the ARM hardware architecture.
It generates machine code for ARM processors executing A64 instructions from programs written in FALSE and stores it in corresponding object files.
\flowgraph{\resource{FALSE\\source code} \ar[r] & \toolbox{falarma64} \ar[r] & \resource{object file}}
\seefalse\seearm\seeobject
}

\providecommand{\falarmc}{
\toolsection{falarmt32} is a compiler for the FALSE programming language targeting the ARM hardware architecture.
It generates machine code for ARM processors without floating-point extension executing T32 instructions from programs written in FALSE and stores it in corresponding object files.
\flowgraph{\resource{FALSE\\source code} \ar[r] & \toolbox{falarmt32} \ar[r] & \resource{object file}}
\seefalse\seearm\seeobject
}

\providecommand{\falarmcfpe}{
\toolsection{falarmt32fpe} is a compiler for the FALSE programming language targeting the ARM hardware architecture.
It generates machine code for ARM processors with floating-point extension executing T32 instructions from programs written in FALSE and stores it in corresponding object files.
\flowgraph{\resource{FALSE\\source code} \ar[r] & \toolbox{falarmt32fpe} \ar[r] & \resource{object file}}
\seefalse\seearm\seeobject
}

\providecommand{\falavr}{
\toolsection{falavr} is a compiler for the FALSE programming language targeting the AVR hardware architecture.
It generates machine code for AVR processors from programs written in FALSE and stores it in corresponding object files.
\flowgraph{\resource{FALSE\\source code} \ar[r] & \toolbox{falavr} \ar[r] & \resource{object file}}
\seefalse\seeavr\seeobject
}

\providecommand{\falavrtt}{
\toolsection{falavr32} is a compiler for the FALSE programming language targeting the AVR32 hardware architecture.
It generates machine code for AVR32 processors from programs written in FALSE and stores it in corresponding object files.
\flowgraph{\resource{FALSE\\source code} \ar[r] & \toolbox{falavr32} \ar[r] & \resource{object file}}
\seefalse\seeavrtt\seeobject
}

\providecommand{\falmabk}{
\toolsection{falm68k} is a compiler for the FALSE programming language targeting the M68000 hardware architecture.
It generates machine code for M68000 processors from programs written in FALSE and stores it in corresponding object files.
\flowgraph{\resource{FALSE\\source code} \ar[r] & \toolbox{falm68k} \ar[r] & \resource{object file}}
\seefalse\seemabk\seeobject
}

\providecommand{\falmibl}{
\toolsection{falmibl} is a compiler for the FALSE programming language targeting the MicroBlaze hardware architecture.
It generates machine code for MicroBlaze processors from programs written in FALSE and stores it in corresponding object files.
\flowgraph{\resource{FALSE\\source code} \ar[r] & \toolbox{falmibl} \ar[r] & \resource{object file}}
\seefalse\seemibl\seeobject
}

\providecommand{\falmipsa}{
\toolsection{falmips32} is a compiler for the FALSE programming language targeting the MIPS32 hardware architecture.
It generates machine code for MIPS32 processors from programs written in FALSE and stores it in corresponding object files.
\flowgraph{\resource{FALSE\\source code} \ar[r] & \toolbox{falmips32} \ar[r] & \resource{object file}}
\seefalse\seemips\seeobject
}

\providecommand{\falmipsb}{
\toolsection{falmips64} is a compiler for the FALSE programming language targeting the MIPS64 hardware architecture.
It generates machine code for MIPS64 processors from programs written in FALSE and stores it in corresponding object files.
\flowgraph{\resource{FALSE\\source code} \ar[r] & \toolbox{falmips64} \ar[r] & \resource{object file}}
\seefalse\seemips\seeobject
}

\providecommand{\falmmix}{
\toolsection{falmmix} is a compiler for the FALSE programming language targeting the MMIX hardware architecture.
It generates machine code for MMIX processors from programs written in FALSE and stores it in corresponding object files.
\flowgraph{\resource{FALSE\\source code} \ar[r] & \toolbox{falmmix} \ar[r] & \resource{object file}}
\seefalse\seemmix\seeobject
}

\providecommand{\falorok}{
\toolsection{falor1k} is a compiler for the FALSE programming language targeting the OpenRISC 1000 hardware architecture.
It generates machine code for OpenRISC 1000 processors from programs written in FALSE and stores it in corresponding object files.
\flowgraph{\resource{FALSE\\source code} \ar[r] & \toolbox{falor1k} \ar[r] & \resource{object file}}
\seefalse\seeorok\seeobject
}

\providecommand{\falppca}{
\toolsection{falppc32} is a compiler for the FALSE programming language targeting the PowerPC hardware architecture.
It generates machine code for PowerPC processors from programs written in FALSE and stores it in corresponding object files.
The compiler generates machine code for the 32-bit operating mode defined by the PowerPC architecture.
\flowgraph{\resource{FALSE\\source code} \ar[r] & \toolbox{falppc32} \ar[r] & \resource{object file}}
\seefalse\seeppc\seeobject
}

\providecommand{\falppcb}{
\toolsection{falppc64} is a compiler for the FALSE programming language targeting the PowerPC hardware architecture.
It generates machine code for PowerPC processors from programs written in FALSE and stores it in corresponding object files.
The compiler generates machine code for the 64-bit operating mode defined by the PowerPC architecture.
\flowgraph{\resource{FALSE\\source code} \ar[r] & \toolbox{falppc64} \ar[r] & \resource{object file}}
\seefalse\seeppc\seeobject
}

\providecommand{\falrisc}{
\toolsection{falrisc} is a compiler for the FALSE programming language targeting the RISC hardware architecture.
It generates machine code for RISC processors from programs written in FALSE and stores it in corresponding object files.
\flowgraph{\resource{FALSE\\source code} \ar[r] & \toolbox{falrisc} \ar[r] & \resource{object file}}
\seefalse\seerisc\seeobject
}

\providecommand{\falwasm}{
\toolsection{falwasm} is a compiler for the FALSE programming language targeting the WebAssembly architecture.
It generates machine code for WebAssembly targets from programs written in FALSE and stores it in corresponding object files.
\flowgraph{\resource{FALSE\\source code} \ar[r] & \toolbox{falwasm} \ar[r] & \resource{object file}}
\seefalse\seewasm\seeobject
}

% Oberon tools

\providecommand{\obprint}{
\toolsection{obprint} is a pretty printer for the Oberon programming language.
It reformats the source code of Oberon modules and writes it to the standard output stream.
\flowgraph{\resource{Oberon\\source code} \ar[r] & \toolbox{obprint} \ar[r] & \resource{reformatted\\source code}}
\seeoberon
}

\providecommand{\obcheck}{
\toolsection{obcheck} is a syntactic and semantic checker for the Oberon programming language.
It just performs syntactic and semantic checks on Oberon modules and writes its diagnostic messages to the standard error stream.
In addition, it stores the interface of each module in a symbol file which is required when other modules import the module.
\flowgraph{\resource{Oberon\\source code} \ar[r] & \toolbox{obcheck} \ar[r] \ar@/l/[d] & \resource{diagnostic\\messages} \\ \variable{ECSIMPORT} \ar[ru] & \resource{symbol\\files} \ar@/r/[u]}
\seeoberon
}

\providecommand{\obdump}{
\toolsection{obdump} is a serializer for the Oberon programming language.
It dumps the complete internal representation of modules written in Oberon into an XML document.
\debuggingtool
\flowgraph{\resource{Oberon\\source code} \ar[r] & \toolbox{obdump} \ar[r] \ar@/l/[d] & \resource{internal\\representation} \\ \variable{ECSIMPORT} \ar[ru] & \resource{symbol\\files} \ar@/r/[u]}
\seeoberon
}

\providecommand{\obrun}{
\toolsection{obrun} is an interpreter for the Oberon programming language.
It processes and executes modules written in Oberon.
This tool does neither generate nor process symbol files while interpreting modules.
If a module is imported by another one, its filename has to be named before the other one in the list of command-line arguments.
\flowgraph{\resource{Oberon\\source code} \ar[r] & \toolbox{obrun} \ar@/u/[r] & \resource{input/\\output} \ar@/d/[l]}
\seeoberon
}

\providecommand{\obcpp}{
\toolsection{obcpp} is a transpiler for the Oberon programming language.
It translates programs written in Oberon into \cpp{} programs and stores them in corresponding source and header files.
In addition, it stores the interface of each module in a symbol file which is required when other modules import the module.
The same interface is provided by the generated header file which can be used in other parts of the \cpp{} program.
\flowgraph{\resource{Oberon\\source code} \ar[r] & \toolbox{obcpp} \ar[r] \ar@/l/[d] \ar[rd] & \resource{\cpp{}\\source file} \\ \variable{ECSIMPORT} \ar[ru] & \resource{symbol\\files} \ar@/r/[u] & \resource{\cpp{}\\header file}}
\seeoberon\seecpp
}

\providecommand{\obdoc}{
\toolsection{obdoc} is a generic documentation generator for the Oberon programming language.
It processes several Oberon modules and assembles all information therein into a generic documentation.
In addition, it stores the interface of each module in a symbol file which is required when other modules import the module.
\debuggingtool
\flowgraph{\resource{Oberon\\source code} \ar[r] & \toolbox{obdoc} \ar[r] \ar@/l/[d] & \resource{generic\\documentation} \\ \variable{ECSIMPORT} \ar[ru] & \resource{symbol\\files} \ar@/r/[u]}
\seeoberon\seedocumentation
}

\providecommand{\obhtml}{
\toolsection{obhtml} is an HTML documentation generator for the Oberon programming language.
It processes several Oberon modules and assembles all information therein into an HTML document.
In addition, it stores the interface of each module in a symbol file which is required when other modules import the module.
\flowgraph{\resource{Oberon\\source code} \ar[r] & \toolbox{obhtml} \ar[r] \ar@/l/[d] & \resource{HTML\\document} \\ \variable{ECSIMPORT} \ar[ru] & \resource{symbol\\files} \ar@/r/[u]}
\seeoberon\seedocumentation
}

\providecommand{\oblatex}{
\toolsection{oblatex} is a Latex documentation generator for the Oberon programming language.
It processes several Oberon modules and assembles all information therein into a Latex document.
In addition, it stores the interface of each module in a symbol file which is required when other modules import the module.
\flowgraph{\resource{Oberon\\source code} \ar[r] & \toolbox{oblatex} \ar[r] \ar@/l/[d] & \resource{Latex\\document} \\ \variable{ECSIMPORT} \ar[ru] & \resource{symbol\\files} \ar@/r/[u]}
\seeoberon\seedocumentation
}

\providecommand{\obcode}{
\toolsection{obcode} is an intermediate code generator for the Oberon programming language.
It generates intermediate code from modules written in Oberon and stores it in corresponding assembly files.
In addition, it stores the interface of each module in a symbol file which is required when other modules import the module.
Programs generated with this tool require additional runtime support that is stored in the \file{ob\-code\-run} library file.
\debuggingtool
\flowgraph{\resource{Oberon\\source code} \ar[r] & \toolbox{obcode} \ar[r] \ar@/l/[d] & \resource{intermediate\\code} \\ \variable{ECSIMPORT} \ar[ru] & \resource{symbol\\files} \ar@/r/[u]}
\seeoberon\seeassembly\seecode
}

\providecommand{\obamda}{
\toolsection{obamd16} is a compiler for the Oberon programming language targeting the AMD64 hardware architecture.
It generates machine code for AMD64 processors from modules written in Oberon and stores it in corresponding object files.
The compiler generates machine code for the 16-bit operating mode defined by the AMD64 architecture.
For debugging purposes, it also creates a debugging information file as well as an assembly file containing a listing of the generated machine code.
In addition, it stores the interface of each module in a symbol file which is required when other modules import the module.
Programs generated with this compiler require additional runtime support that is stored in the \file{ob\-amd16\-run} library file.
\flowgraph{\resource{Oberon\\source code} \ar[r] & \toolbox{obamd16} \ar[r] \ar@/l/[d] \ar[rd] & \resource{object file} \\ \variable{ECSIMPORT} \ar[ru] & \resource{symbol\\files} \ar@/r/[u] & \resource{debugging\\information}}
\seeoberon\seeassembly\seeamd\seeobject\seedebugging
}

\providecommand{\obamdb}{
\toolsection{obamd32} is a compiler for the Oberon programming language targeting the AMD64 hardware architecture.
It generates machine code for AMD64 processors from modules written in Oberon and stores it in corresponding object files.
The compiler generates machine code for the 32-bit operating mode defined by the AMD64 architecture.
For debugging purposes, it also creates a debugging information file as well as an assembly file containing a listing of the generated machine code.
In addition, it stores the interface of each module in a symbol file which is required when other modules import the module.
Programs generated with this compiler require additional runtime support that is stored in the \file{ob\-amd32\-run} library file.
\flowgraph{\resource{Oberon\\source code} \ar[r] & \toolbox{obamd32} \ar[r] \ar@/l/[d] \ar[rd] & \resource{object file} \\ \variable{ECSIMPORT} \ar[ru] & \resource{symbol\\files} \ar@/r/[u] & \resource{debugging\\information}}
\seeoberon\seeassembly\seeamd\seeobject\seedebugging
}

\providecommand{\obamdc}{
\toolsection{obamd64} is a compiler for the Oberon programming language targeting the AMD64 hardware architecture.
It generates machine code for AMD64 processors from modules written in Oberon and stores it in corresponding object files.
The compiler generates machine code for the 64-bit operating mode defined by the AMD64 architecture.
For debugging purposes, it also creates a debugging information file as well as an assembly file containing a listing of the generated machine code.
In addition, it stores the interface of each module in a symbol file which is required when other modules import the module.
Programs generated with this compiler require additional runtime support that is stored in the \file{ob\-amd64\-run} library file.
\flowgraph{\resource{Oberon\\source code} \ar[r] & \toolbox{obamd64} \ar[r] \ar@/l/[d] \ar[rd] & \resource{object file} \\ \variable{ECSIMPORT} \ar[ru] & \resource{symbol\\files} \ar@/r/[u] & \resource{debugging\\information}}
\seeoberon\seeassembly\seeamd\seeobject\seedebugging
}

\providecommand{\obarma}{
\toolsection{obarma32} is a compiler for the Oberon programming language targeting the ARM hardware architecture.
It generates machine code for ARM processors executing A32 instructions from modules written in Oberon and stores it in corresponding object files.
For debugging purposes, it also creates a debugging information file as well as an assembly file containing a listing of the generated machine code.
In addition, it stores the interface of each module in a symbol file which is required when other modules import the module.
Programs generated with this compiler require additional runtime support that is stored in the \file{ob\-arma32\-run} library file.
\flowgraph{\resource{Oberon\\source code} \ar[r] & \toolbox{obarma32} \ar[r] \ar@/l/[d] \ar[rd] & \resource{object file} \\ \variable{ECSIMPORT} \ar[ru] & \resource{symbol\\files} \ar@/r/[u] & \resource{debugging\\information}}
\seeoberon\seeassembly\seearm\seeobject\seedebugging
}

\providecommand{\obarmb}{
\toolsection{obarma64} is a compiler for the Oberon programming language targeting the ARM hardware architecture.
It generates machine code for ARM processors executing A64 instructions from modules written in Oberon and stores it in corresponding object files.
For debugging purposes, it also creates a debugging information file as well as an assembly file containing a listing of the generated machine code.
In addition, it stores the interface of each module in a symbol file which is required when other modules import the module.
Programs generated with this compiler require additional runtime support that is stored in the \file{ob\-arma64\-run} library file.
\flowgraph{\resource{Oberon\\source code} \ar[r] & \toolbox{obarma64} \ar[r] \ar@/l/[d] \ar[rd] & \resource{object file} \\ \variable{ECSIMPORT} \ar[ru] & \resource{symbol\\files} \ar@/r/[u] & \resource{debugging\\information}}
\seeoberon\seeassembly\seearm\seeobject\seedebugging
}

\providecommand{\obarmc}{
\toolsection{obarmt32} is a compiler for the Oberon programming language targeting the ARM hardware architecture.
It generates machine code for ARM processors without floating-point extension executing T32 instructions from modules written in Oberon and stores it in corresponding object files.
For debugging purposes, it also creates a debugging information file as well as an assembly file containing a listing of the generated machine code.
In addition, it stores the interface of each module in a symbol file which is required when other modules import the module.
Programs generated with this compiler require additional runtime support that is stored in the \file{ob\-armt32\-run} library file.
\flowgraph{\resource{Oberon\\source code} \ar[r] & \toolbox{obarmt32} \ar[r] \ar@/l/[d] \ar[rd] & \resource{object file} \\ \variable{ECSIMPORT} \ar[ru] & \resource{symbol\\files} \ar@/r/[u] & \resource{debugging\\information}}
\seeoberon\seeassembly\seearm\seeobject\seedebugging
}

\providecommand{\obarmcfpe}{
\toolsection{obarmt32fpe} is a compiler for the Oberon programming language targeting the ARM hardware architecture.
It generates machine code for ARM processors with floating-point extension executing T32 instructions from modules written in Oberon and stores it in corresponding object files.
For debugging purposes, it also creates a debugging information file as well as an assembly file containing a listing of the generated machine code.
In addition, it stores the interface of each module in a symbol file which is required when other modules import the module.
Programs generated with this compiler require additional runtime support that is stored in the \file{ob\-armt32\-fpe\-run} library file.
\flowgraph{\resource{Oberon\\source code} \ar[r] & \toolbox{obarmt32fpe} \ar[r] \ar@/l/[d] \ar[rd] & \resource{object file} \\ \variable{ECSIMPORT} \ar[ru] & \resource{symbol\\files} \ar@/r/[u] & \resource{debugging\\information}}
\seeoberon\seeassembly\seearm\seeobject\seedebugging
}

\providecommand{\obavr}{
\toolsection{obavr} is a compiler for the Oberon programming language targeting the AVR hardware architecture.
It generates machine code for AVR processors from modules written in Oberon and stores it in corresponding object files.
For debugging purposes, it also creates a debugging information file as well as an assembly file containing a listing of the generated machine code.
In addition, it stores the interface of each module in a symbol file which is required when other modules import the module.
Programs generated with this compiler require additional runtime support that is stored in the \file{ob\-avr\-run} library file.
\flowgraph{\resource{Oberon\\source code} \ar[r] & \toolbox{obavr} \ar[r] \ar@/l/[d] \ar[rd] & \resource{object file} \\ \variable{ECSIMPORT} \ar[ru] & \resource{symbol\\files} \ar@/r/[u] & \resource{debugging\\information}}
\seeoberon\seeassembly\seeavr\seeobject\seedebugging
}

\providecommand{\obavrtt}{
\toolsection{obavr32} is a compiler for the Oberon programming language targeting the AVR32 hardware architecture.
It generates machine code for AVR32 processors from modules written in Oberon and stores it in corresponding object files.
For debugging purposes, it also creates a debugging information file as well as an assembly file containing a listing of the generated machine code.
In addition, it stores the interface of each module in a symbol file which is required when other modules import the module.
Programs generated with this compiler require additional runtime support that is stored in the \file{ob\-avr32\-run} library file.
\flowgraph{\resource{Oberon\\source code} \ar[r] & \toolbox{obavr32} \ar[r] \ar@/l/[d] \ar[rd] & \resource{object file} \\ \variable{ECSIMPORT} \ar[ru] & \resource{symbol\\files} \ar@/r/[u] & \resource{debugging\\information}}
\seeoberon\seeassembly\seeavrtt\seeobject\seedebugging
}

\providecommand{\obmabk}{
\toolsection{obm68k} is a compiler for the Oberon programming language targeting the M68000 hardware architecture.
It generates machine code for M68000 processors from modules written in Oberon and stores it in corresponding object files.
For debugging purposes, it also creates a debugging information file as well as an assembly file containing a listing of the generated machine code.
In addition, it stores the interface of each module in a symbol file which is required when other modules import the module.
Programs generated with this compiler require additional runtime support that is stored in the \file{ob\-m68k\-run} library file.
\flowgraph{\resource{Oberon\\source code} \ar[r] & \toolbox{obm68k} \ar[r] \ar@/l/[d] \ar[rd] & \resource{object file} \\ \variable{ECSIMPORT} \ar[ru] & \resource{symbol\\files} \ar@/r/[u] & \resource{debugging\\information}}
\seeoberon\seeassembly\seemabk\seeobject\seedebugging
}

\providecommand{\obmibl}{
\toolsection{obmibl} is a compiler for the Oberon programming language targeting the MicroBlaze hardware architecture.
It generates machine code for MicroBlaze processors from modules written in Oberon and stores it in corresponding object files.
For debugging purposes, it also creates a debugging information file as well as an assembly file containing a listing of the generated machine code.
In addition, it stores the interface of each module in a symbol file which is required when other modules import the module.
Programs generated with this compiler require additional runtime support that is stored in the \file{ob\-mibl\-run} library file.
\flowgraph{\resource{Oberon\\source code} \ar[r] & \toolbox{obmibl} \ar[r] \ar@/l/[d] \ar[rd] & \resource{object file} \\ \variable{ECSIMPORT} \ar[ru] & \resource{symbol\\files} \ar@/r/[u] & \resource{debugging\\information}}
\seeoberon\seeassembly\seemibl\seeobject\seedebugging
}

\providecommand{\obmipsa}{
\toolsection{obmips32} is a compiler for the Oberon programming language targeting the MIPS32 hardware architecture.
It generates machine code for MIPS32 processors from modules written in Oberon and stores it in corresponding object files.
For debugging purposes, it also creates a debugging information file as well as an assembly file containing a listing of the generated machine code.
In addition, it stores the interface of each module in a symbol file which is required when other modules import the module.
Programs generated with this compiler require additional runtime support that is stored in the \file{ob\-mips32\-run} library file.
\flowgraph{\resource{Oberon\\source code} \ar[r] & \toolbox{obmips32} \ar[r] \ar@/l/[d] \ar[rd] & \resource{object file} \\ \variable{ECSIMPORT} \ar[ru] & \resource{symbol\\files} \ar@/r/[u] & \resource{debugging\\information}}
\seeoberon\seeassembly\seemips\seeobject\seedebugging
}

\providecommand{\obmipsb}{
\toolsection{obmips64} is a compiler for the Oberon programming language targeting the MIPS64 hardware architecture.
It generates machine code for MIPS64 processors from modules written in Oberon and stores it in corresponding object files.
For debugging purposes, it also creates a debugging information file as well as an assembly file containing a listing of the generated machine code.
In addition, it stores the interface of each module in a symbol file which is required when other modules import the module.
Programs generated with this compiler require additional runtime support that is stored in the \file{ob\-mips64\-run} library file.
\flowgraph{\resource{Oberon\\source code} \ar[r] & \toolbox{obmips64} \ar[r] \ar@/l/[d] \ar[rd] & \resource{object file} \\ \variable{ECSIMPORT} \ar[ru] & \resource{symbol\\files} \ar@/r/[u] & \resource{debugging\\information}}
\seeoberon\seeassembly\seemips\seeobject\seedebugging
}

\providecommand{\obmmix}{
\toolsection{obmmix} is a compiler for the Oberon programming language targeting the MMIX hardware architecture.
It generates machine code for MMIX processors from modules written in Oberon and stores it in corresponding object files.
For debugging purposes, it also creates a debugging information file as well as an assembly file containing a listing of the generated machine code.
In addition, it stores the interface of each module in a symbol file which is required when other modules import the module.
Programs generated with this compiler require additional runtime support that is stored in the \file{ob\-mmix\-run} library file.
\flowgraph{\resource{Oberon\\source code} \ar[r] & \toolbox{obmmix} \ar[r] \ar@/l/[d] \ar[rd] & \resource{object file} \\ \variable{ECSIMPORT} \ar[ru] & \resource{symbol\\files} \ar@/r/[u] & \resource{debugging\\information}}
\seeoberon\seeassembly\seemmix\seeobject\seedebugging
}

\providecommand{\oborok}{
\toolsection{obor1k} is a compiler for the Oberon programming language targeting the OpenRISC 1000 hardware architecture.
It generates machine code for OpenRISC 1000 processors from modules written in Oberon and stores it in corresponding object files.
For debugging purposes, it also creates a debugging information file as well as an assembly file containing a listing of the generated machine code.
In addition, it stores the interface of each module in a symbol file which is required when other modules import the module.
Programs generated with this compiler require additional runtime support that is stored in the \file{ob\-or1k\-run} library file.
\flowgraph{\resource{Oberon\\source code} \ar[r] & \toolbox{obor1k} \ar[r] \ar@/l/[d] \ar[rd] & \resource{object file} \\ \variable{ECSIMPORT} \ar[ru] & \resource{symbol\\files} \ar@/r/[u] & \resource{debugging\\information}}
\seeoberon\seeassembly\seeorok\seeobject\seedebugging
}

\providecommand{\obppca}{
\toolsection{obppc32} is a compiler for the Oberon programming language targeting the PowerPC hardware architecture.
It generates machine code for PowerPC processors from modules written in Oberon and stores it in corresponding object files.
The compiler generates machine code for the 32-bit operating mode defined by the PowerPC architecture.
For debugging purposes, it also creates a debugging information file as well as an assembly file containing a listing of the generated machine code.
In addition, it stores the interface of each module in a symbol file which is required when other modules import the module.
Programs generated with this compiler require additional runtime support that is stored in the \file{ob\-ppc32\-run} library file.
\flowgraph{\resource{Oberon\\source code} \ar[r] & \toolbox{obppc32} \ar[r] \ar@/l/[d] \ar[rd] & \resource{object file} \\ \variable{ECSIMPORT} \ar[ru] & \resource{symbol\\files} \ar@/r/[u] & \resource{debugging\\information}}
\seeoberon\seeassembly\seeppc\seeobject\seedebugging
}

\providecommand{\obppcb}{
\toolsection{obppc64} is a compiler for the Oberon programming language targeting the PowerPC hardware architecture.
It generates machine code for PowerPC processors from modules written in Oberon and stores it in corresponding object files.
The compiler generates machine code for the 64-bit operating mode defined by the PowerPC architecture.
For debugging purposes, it also creates a debugging information file as well as an assembly file containing a listing of the generated machine code.
In addition, it stores the interface of each module in a symbol file which is required when other modules import the module.
Programs generated with this compiler require additional runtime support that is stored in the \file{ob\-ppc64\-run} library file.
\flowgraph{\resource{Oberon\\source code} \ar[r] & \toolbox{obppc64} \ar[r] \ar@/l/[d] \ar[rd] & \resource{object file} \\ \variable{ECSIMPORT} \ar[ru] & \resource{symbol\\files} \ar@/r/[u] & \resource{debugging\\information}}
\seeoberon\seeassembly\seeppc\seeobject\seedebugging
}

\providecommand{\obrisc}{
\toolsection{obrisc} is a compiler for the Oberon programming language targeting the RISC hardware architecture.
It generates machine code for RISC processors from modules written in Oberon and stores it in corresponding object files.
For debugging purposes, it also creates a debugging information file as well as an assembly file containing a listing of the generated machine code.
In addition, it stores the interface of each module in a symbol file which is required when other modules import the module.
Programs generated with this compiler require additional runtime support that is stored in the \file{ob\-risc\-run} library file.
\flowgraph{\resource{Oberon\\source code} \ar[r] & \toolbox{obrisc} \ar[r] \ar@/l/[d] \ar[rd] & \resource{object file} \\ \variable{ECSIMPORT} \ar[ru] & \resource{symbol\\files} \ar@/r/[u] & \resource{debugging\\information}}
\seeoberon\seeassembly\seerisc\seeobject\seedebugging
}

\providecommand{\obwasm}{
\toolsection{obwasm} is a compiler for the Oberon programming language targeting the WebAssembly architecture.
It generates machine code for WebAssembly targets from modules written in Oberon and stores it in corresponding object files.
For debugging purposes, it also creates a debugging information file as well as an assembly file containing a listing of the generated machine code.
In addition, it stores the interface of each module in a symbol file which is required when other modules import the module.
Programs generated with this compiler require additional runtime support that is stored in the \file{ob\-wasm\-run} library file.
\flowgraph{\resource{Oberon\\source code} \ar[r] & \toolbox{obwasm} \ar[r] \ar@/l/[d] \ar[rd] & \resource{object file} \\ \variable{ECSIMPORT} \ar[ru] & \resource{symbol\\files} \ar@/r/[u] & \resource{debugging\\information}}
\seeoberon\seeassembly\seewasm\seeobject\seedebugging
}

% converter tools

\providecommand{\dbgdwarf}{
\toolsection{dbgdwarf} is a DWARF debugging information converter tool.
It converts debugging information into the DWARF debugging data format and stores it in corresponding object files~\cite{dwarffile}.
The resulting debugging object files can be combined with runtime support that creates Executable and Linking Format (ELF) files~\cite{elffile}.
\flowgraph{\resource{debugging\\information} \ar[r] & \toolbox{dbgdwarf} \ar[r] & \resource{debugging\\object file}}
\seeobject\seedebugging
}

% assembler tools

\providecommand{\asmprint}{
\toolsection{asmprint} is a pretty printer for generic assembly code.
It reformats generic assembly code and writes it to the standard output stream.
\flowgraph{\resource{generic assembly\\source code} \ar[r] & \toolbox{asmprint} \ar[r] & \resource{reformatted\\source code}}
\seeassembly
}

\providecommand{\amdaasm}{
\toolsection{amd16asm} is an assembler for the AMD64 hardware architecture.
It translates assembly code into machine code for AMD64 processors and stores it in corresponding object files.
By default, the assembler generates machine code for the 16-bit operating mode defined by the AMD64 architecture.
\flowgraph{\resource{AMD16 assembly\\source code} \ar[r] & \toolbox{amd16asm} \ar[r] & \resource{object file}}
\seeassembly\seeamd\seeobject
}

\providecommand{\amdadism}{
\toolsection{amd16dism} is a disassembler for the AMD64 hardware architecture.
It translates machine code from object files targeting AMD64 processors into assembly code and writes it to the standard output stream.
It assumes that the machine code was generated for the 16-bit operating mode defined by the AMD64 architecture.
\flowgraph{\resource{object file} \ar[r] & \toolbox{amd16dism} \ar[r] & \resource{disassembly\\listing}}
\seeassembly\seeamd\seeobject
}

\providecommand{\amdbasm}{
\toolsection{amd32asm} is an assembler for the AMD64 hardware architecture.
It translates assembly code into machine code for AMD64 processors and stores it in corresponding object files.
By default, the assembler generates machine code for the 32-bit operating mode defined by the AMD64 architecture.
\flowgraph{\resource{AMD32 assembly\\source code} \ar[r] & \toolbox{amd32asm} \ar[r] & \resource{object file}}
\seeassembly\seeamd\seeobject
}

\providecommand{\amdbdism}{
\toolsection{amd32dism} is a disassembler for the AMD64 hardware architecture.
It translates machine code from object files targeting AMD64 processors into assembly code and writes it to the standard output stream.
It assumes that the machine code was generated for the 32-bit operating mode defined by the AMD64 architecture.
\flowgraph{\resource{object file} \ar[r] & \toolbox{amd32dism} \ar[r] & \resource{disassembly\\listing}}
\seeassembly\seeamd\seeobject
}

\providecommand{\amdcasm}{
\toolsection{amd64asm} is an assembler for the AMD64 hardware architecture.
It translates assembly code into machine code for AMD64 processors and stores it in corresponding object files.
By default, the assembler generates machine code for the 64-bit operating mode defined by the AMD64 architecture.
\flowgraph{\resource{AMD64 assembly\\source code} \ar[r] & \toolbox{amd64asm} \ar[r] & \resource{object file}}
\seeassembly\seeamd\seeobject
}

\providecommand{\amdcdism}{
\toolsection{amd64dism} is a disassembler for the AMD64 hardware architecture.
It translates machine code from object files targeting AMD64 processors into assembly code and writes it to the standard output stream.
It assumes that the machine code was generated for the 64-bit operating mode defined by the AMD64 architecture.
\flowgraph{\resource{object file} \ar[r] & \toolbox{amd64dism} \ar[r] & \resource{disassembly\\listing}}
\seeassembly\seeamd\seeobject
}

\providecommand{\armaasm}{
\toolsection{arma32asm} is an assembler for the ARM hardware architecture.
It translates assembly code into machine code for ARM processors executing A32 instructions and stores it in corresponding object files.
\flowgraph{\resource{ARM A32 assembly\\source code} \ar[r] & \toolbox{arma32asm} \ar[r] & \resource{object file}}
\seeassembly\seearm\seeobject
}

\providecommand{\armadism}{
\toolsection{arma32dism} is a disassembler for the ARM hardware architecture.
It translates machine code from object files targeting ARM processors executing A32 instructions into assembly code and writes it to the standard output stream.
\flowgraph{\resource{object file} \ar[r] & \toolbox{arma32dism} \ar[r] & \resource{disassembly\\listing}}
\seeassembly\seearm\seeobject
}

\providecommand{\armbasm}{
\toolsection{arma64asm} is an assembler for the ARM hardware architecture.
It translates assembly code into machine code for ARM processors executing A64 instructions and stores it in corresponding object files.
\flowgraph{\resource{ARM A64 assembly\\source code} \ar[r] & \toolbox{arma64asm} \ar[r] & \resource{object file}}
\seeassembly\seearm\seeobject
}

\providecommand{\armbdism}{
\toolsection{arma64dism} is a disassembler for the ARM hardware architecture.
It translates machine code from object files targeting ARM processors executing A64 instructions into assembly code and writes it to the standard output stream.
\flowgraph{\resource{object file} \ar[r] & \toolbox{arma64dism} \ar[r] & \resource{disassembly\\listing}}
\seeassembly\seearm\seeobject
}

\providecommand{\armcasm}{
\toolsection{armt32asm} is an assembler for the ARM hardware architecture.
It translates assembly code into machine code for ARM processors executing T32 instructions and stores it in corresponding object files.
\flowgraph{\resource{ARM T32 assembly\\source code} \ar[r] & \toolbox{armt32asm} \ar[r] & \resource{object file}}
\seeassembly\seearm\seeobject
}

\providecommand{\armcdism}{
\toolsection{armt32dism} is a disassembler for the ARM hardware architecture.
It translates machine code from object files targeting ARM processors executing T32 instructions into assembly code and writes it to the standard output stream.
\flowgraph{\resource{object file} \ar[r] & \toolbox{armt32dism} \ar[r] & \resource{disassembly\\listing}}
\seeassembly\seearm\seeobject
}

\providecommand{\avrasm}{
\toolsection{avrasm} is an assembler for the AVR hardware architecture.
It translates assembly code into machine code for AVR processors and stores it in corresponding object files.
The identifiers \texttt{RXL}, \texttt{RXH}, \texttt{RYL}, \texttt{RYH}, \texttt{RZL}, and \texttt{RZH} are predefined and name the corresponding registers.
The identifiers \texttt{SPL} and \texttt{SPH} are also predefined and evaluate to the address of the corresponding registers.
\flowgraph{\resource{AVR assembly\\source code} \ar[r] & \toolbox{avrasm} \ar[r] & \resource{object file}}
\seeassembly\seeavr\seeobject
}

\providecommand{\avrdism}{
\toolsection{avrdism} is a disassembler for the AVR hardware architecture.
It translates machine code from object files targeting AVR processors into assembly code and writes it to the standard output stream.
\flowgraph{\resource{object file} \ar[r] & \toolbox{avrdism} \ar[r] & \resource{disassembly\\listing}}
\seeassembly\seeavr\seeobject
}

\providecommand{\avrttasm}{
\toolsection{avr32asm} is an assembler for the AVR32 hardware architecture.
It translates assembly code into machine code for AVR32 processors and stores it in corresponding object files.
\flowgraph{\resource{AVR32 assembly\\source code} \ar[r] & \toolbox{avr32asm} \ar[r] & \resource{object file}}
\seeassembly\seeavrtt\seeobject
}

\providecommand{\avrttdism}{
\toolsection{avr32dism} is a disassembler for the AVR32 hardware architecture.
It translates machine code from object files targeting AVR32 processors into assembly code and writes it to the standard output stream.
\flowgraph{\resource{object file} \ar[r] & \toolbox{avr32dism} \ar[r] & \resource{disassembly\\listing}}
\seeassembly\seeavrtt\seeobject
}

\providecommand{\mabkasm}{
\toolsection{m68kasm} is an assembler for the M68000 hardware architecture.
It translates assembly code into machine code for M68000 processors and stores it in corresponding object files.
\flowgraph{\resource{68000 assembly\\source code} \ar[r] & \toolbox{m68kasm} \ar[r] & \resource{object file}}
\seeassembly\seemabk\seeobject
}

\providecommand{\mabkdism}{
\toolsection{m68kdism} is a disassembler for the M68000 hardware architecture.
It translates machine code from object files targeting M68000 processors into assembly code and writes it to the standard output stream.
\flowgraph{\resource{object file} \ar[r] & \toolbox{m68kdism} \ar[r] & \resource{disassembly\\listing}}
\seeassembly\seemabk\seeobject
}

\providecommand{\miblasm}{
\toolsection{miblasm} is an assembler for the MicroBlaze hardware architecture.
It translates assembly code into machine code for MicroBlaze processors and stores it in corresponding object files.
\flowgraph{\resource{MicroBlaze assembly\\source code} \ar[r] & \toolbox{miblasm} \ar[r] & \resource{object file}}
\seeassembly\seemibl\seeobject
}

\providecommand{\mibldism}{
\toolsection{mibldism} is a disassembler for the MicroBlaze hardware architecture.
It translates machine code from object files targeting MicroBlaze processors into assembly code and writes it to the standard output stream.
\flowgraph{\resource{object file} \ar[r] & \toolbox{mibldism} \ar[r] & \resource{disassembly\\listing}}
\seeassembly\seemibl\seeobject
}

\providecommand{\mipsaasm}{
\toolsection{mips32asm} is an assembler for the MIPS32 hardware architecture.
It translates assembly code into machine code for MIPS32 processors and stores it in corresponding object files.
\flowgraph{\resource{MIPS32 assembly\\source code} \ar[r] & \toolbox{mips32asm} \ar[r] & \resource{object file}}
\seeassembly\seemips\seeobject
}

\providecommand{\mipsadism}{
\toolsection{mips32dism} is a disassembler for the MIPS32 hardware architecture.
It translates machine code from object files targeting MIPS32 processors into assembly code and writes it to the standard output stream.
\flowgraph{\resource{object file} \ar[r] & \toolbox{mips32dism} \ar[r] & \resource{disassembly\\listing}}
\seeassembly\seemips\seeobject
}

\providecommand{\mipsbasm}{
\toolsection{mips64asm} is an assembler for the MIPS64 hardware architecture.
It translates assembly code into machine code for MIPS64 processors and stores it in corresponding object files.
\flowgraph{\resource{MIPS64 assembly\\source code} \ar[r] & \toolbox{mips64asm} \ar[r] & \resource{object file}}
\seeassembly\seemips\seeobject
}

\providecommand{\mipsbdism}{
\toolsection{mips64dism} is a disassembler for the MIPS64 hardware architecture.
It translates machine code from object files targeting MIPS64 processors into assembly code and writes it to the standard output stream.
\flowgraph{\resource{object file} \ar[r] & \toolbox{mips64dism} \ar[r] & \resource{disassembly\\listing}}
\seeassembly\seemips\seeobject
}

\providecommand{\mmixasm}{
\toolsection{mmixasm} is an assembler for the MMIX hardware architecture.
It translates assembly code into machine code for MMIX processors and stores it in corresponding object files.
The names of all special registers are predefined and evaluate to the corresponding number.
\flowgraph{\resource{MMIX assembly\\source code} \ar[r] & \toolbox{mmixasm} \ar[r] & \resource{object file}}
\seeassembly\seemmix\seeobject
}

\providecommand{\mmixdism}{
\toolsection{mmixdism} is a disassembler for the MMIX hardware architecture.
It translates machine code from object files targeting MMIX processors into assembly code and writes it to the standard output stream.
\flowgraph{\resource{object file} \ar[r] & \toolbox{mmixdism} \ar[r] & \resource{disassembly\\listing}}
\seeassembly\seemmix\seeobject
}

\providecommand{\orokasm}{
\toolsection{or1kasm} is an assembler for the OpenRISC 1000 hardware architecture.
It translates assembly code into machine code for OpenRISC 1000 processors and stores it in corresponding object files.
\flowgraph{\resource{OpenRISC 1000 assembly\\source code} \ar[r] & \toolbox{or1kasm} \ar[r] & \resource{object file}}
\seeassembly\seeorok\seeobject
}

\providecommand{\orokdism}{
\toolsection{or1kdism} is a disassembler for the OpenRISC 1000 hardware architecture.
It translates machine code from object files targeting OpenRISC 1000 processors into assembly code and writes it to the standard output stream.
\flowgraph{\resource{object file} \ar[r] & \toolbox{or1kdism} \ar[r] & \resource{disassembly\\listing}}
\seeassembly\seeorok\seeobject
}

\providecommand{\ppcaasm}{
\toolsection{ppc32asm} is an assembler for the PowerPC hardware architecture.
It translates assembly code into machine code for PowerPC processors and stores it in corresponding object files.
By default, the assembler generates machine code for the 32-bit operating mode defined by the PowerPC architecture.
\flowgraph{\resource{PowerPC assembly\\source code} \ar[r] & \toolbox{ppc32asm} \ar[r] & \resource{object file}}
\seeassembly\seeppc\seeobject
}

\providecommand{\ppcadism}{
\toolsection{ppc32dism} is a disassembler for the PowerPC hardware architecture.
It translates machine code from object files targeting PowerPC processors into assembly code and writes it to the standard output stream.
It assumes that the machine code was generated for the 32-bit operating mode defined by the PowerPC architecture.
\flowgraph{\resource{object file} \ar[r] & \toolbox{ppc32dism} \ar[r] & \resource{disassembly\\listing}}
\seeassembly\seeppc\seeobject
}

\providecommand{\ppcbasm}{
\toolsection{ppc64asm} is an assembler for the PowerPC hardware architecture.
It translates assembly code into machine code for PowerPC processors and stores it in corresponding object files.
By default, the assembler generates machine code for the 64-bit operating mode defined by the PowerPC architecture.
\flowgraph{\resource{PowerPC assembly\\source code} \ar[r] & \toolbox{ppc64asm} \ar[r] & \resource{object file}}
\seeassembly\seeppc\seeobject
}

\providecommand{\ppcbdism}{
\toolsection{ppc64dism} is a disassembler for the PowerPC hardware architecture.
It translates machine code from object files targeting PowerPC processors into assembly code and writes it to the standard output stream.
It assumes that the machine code was generated for the 64-bit operating mode defined by the PowerPC architecture.
\flowgraph{\resource{object file} \ar[r] & \toolbox{ppc64dism} \ar[r] & \resource{disassembly\\listing}}
\seeassembly\seeppc\seeobject
}

\providecommand{\riscasm}{
\toolsection{riscasm} is an assembler for the RISC hardware architecture.
It translates assembly code into machine code for RISC processors and stores it in corresponding object files.
The names of all special registers are predefined and evaluate to the corresponding number.
\flowgraph{\resource{RISC assembly\\source code} \ar[r] & \toolbox{riscasm} \ar[r] & \resource{object file}}
\seeassembly\seerisc\seeobject
}

\providecommand{\riscdism}{
\toolsection{riscdism} is a disassembler for the RISC hardware architecture.
It translates machine code from object files targeting RISC processors into assembly code and writes it to the standard output stream.
\flowgraph{\resource{object file} \ar[r] & \toolbox{riscdism} \ar[r] & \resource{disassembly\\listing}}
\seeassembly\seerisc\seeobject
}

\providecommand{\wasmasm}{
\toolsection{wasmasm} is an assembler for the WebAssembly architecture.
It translates assembly code into machine code for WebAssembly targets and stores it in corresponding object files.
The names of all special registers are predefined and evaluate to the corresponding number.
\flowgraph{\resource{WebAssembly assembly\\source code} \ar[r] & \toolbox{wasmasm} \ar[r] & \resource{object file}}
\seeassembly\seewasm\seeobject
}

\providecommand{\wasmdism}{
\toolsection{wasmdism} is a disassembler for the WebAssembly architecture.
It translates machine code from object files targeting WebAssembly targets into assembly code and writes it to the standard output stream.
\flowgraph{\resource{object file} \ar[r] & \toolbox{wasmdism} \ar[r] & \resource{disassembly\\listing}}
\seeassembly\seewasm\seeobject
}

% linker tools

\providecommand{\linklib}{
\toolsection{linklib} is an object file combiner.
It creates a static library file by combining all object files given to it into a single one.
\flowgraph{\resource{object files} \ar[r] & \toolbox{linklib} \ar[r] & \resource{library file}}
\seeobject
}

\providecommand{\linkbin}{
\toolsection{linkbin} is a linker for plain binary files.
It links all object files given to it into a single image and stores it in a binary file that begins with the first linked section.
It also creates a map file that lists the address, type, name and size of all used sections.
The filename extension of the resulting binary file can be specified by putting it into a constant data section called \texttt{\_extension}.
\flowgraph{\resource{object files} \ar[r] & \toolbox{linkbin} \ar[r] \ar[d] & \resource{binary file} \\ & \resource{map file}}
\seeobject
}

\providecommand{\linkmem}{
\toolsection{linkmem} is a linker for plain binary files partitioned into random-access and read-only memory.
It links all object files given to it into two distinct images, one for data sections and one for code and constant data sections, and stores each image in a binary file that begins with the first linked section of the corresponding type.
It also creates a map file that lists the address, type, name and size of all used sections.
\flowgraph{\resource{object files} \ar[r] & \toolbox{linkmem} \ar[r] \ar[d] & \resource{RAM file/\\ROM file} \\ & \resource{map file}}
\seeobject
}

\providecommand{\linkprg}{
\toolsection{linkprg} is a linker for GEMDOS executable files.
It links all object files given to it into a single image and stores the image in an Atari GEMDOS executable file~\cite{gemdosfile}.
It also creates a map file that lists the address relative to the text segment, type, name and size of all used sections.
The filename extension of the resulting executable file can be specified by putting it into a constant data section called \texttt{\_extension}.
The GEMDOS executable file format requires all patch patterns of absolute link patches to consist of four full bitmasks with descending offsets.
\flowgraph{\resource{object files} \ar[r] & \toolbox{linkprg} \ar[r] \ar[d] & \resource{executable file} \\ & \resource{map file}}
\seeobject
}

\providecommand{\linkhex}{
\toolsection{linkhex} is a linker for Intel HEX files.
It links all code sections of the object files given to it into single image and stores the image in an Intel HEX file~\cite{hexfile} that begins with the first linked section.
It also creates a map file that lists the address, type, name and size of all used sections.
\flowgraph{\resource{object files} \ar[r] & \toolbox{linkhex} \ar[r] \ar[d] & \resource{HEX file} \\ & \resource{map file}}
\seeobject
}

\providecommand{\mapsearch}{
\toolsection{mapsearch} is a debugging tool.
It searches map files generated by linker tools for the name of a binary section that encompasses a memory address read from the standard input stream.
If additionally provided with one or more object files, it also stores an excerpt thereof in a separate object file called map search result which only contains the identified binary section for disassembling purposes.
\flowgraph{& \resource{map files/\\object files} \ar[d] \\ \resource{memory\\address} \ar[r] & \toolbox{mapsearch} \ar[r] \ar[d] & \resource{section name/\\relative offset} \\ & \resource{object file\\excerpt}}
\seeobject
}


\startchapter{Introduction}{Introduction to the \ecs{}}{introduction}
{\emph{\ecs{}}\index{Eigen Compiler Suite} is the name of a free software collection of development tools like
compilers, assemblers, and linkers targeting different programming languages and different hardware architectures.
This \documentation{} gives a general overview over the \ecs{} and describes its features and design in detail.}

\epigraph{I will not follow where the path may lead, \\ instead I will go where there is no path \\ and leave a trail.}{Muriel Strode}

\section{Features}\index{Features, of the Eigen Compiler Suite}\index{Eigen Compiler Suite!Features}

The \ecs{} is a software development toolchain.
It contains tools like compilers, pretty printers, interpreters, assemblers, and linkers targeting a variety of programming languages and hardware architectures.
The \ecs{} features pretty printers, interpreters, and compilers for the following programming languages:

\begin{center}\cpplogo{1em}\fallogo{1em}\oblogo{2em}\end{center}

\begin{itemize}

\item \cpp{}\nopagebreak

\cpp{} is a general-purpose programming language with a bias toward systems programming.
\seecpp

\item FALSE\nopagebreak

FALSE is an esoteric stack-oriented programming language that provides lambda abstractions and is quite powerful for its size.
\seefalse

\item Oberon\nopagebreak

Oberon is a general-purpose programming language that supports type extension with type-bound procedures which makes it an object-oriented language.
\seeoberon

\end{itemize}

Besides the support for these programming languages, the \ecs{} also features assemblers, disassemblers, and compilers targeting the following hardware architectures:

\begin{itemize}

\item AMD64\nopagebreak

AMD64 is a 64-bit instruction set architecture developed by AMD\@.
\seeamd

\item ARM\nopagebreak

ARM is a 32-bit and 64-bit instruction set architecture developed by ARM Holdings.
\seearm

\item AVR\nopagebreak

AVR is the name of an 8-bit microcontroller architecture developed by Atmel.
\seeavr

\item AVR32\nopagebreak

AVR32 is the name of an 32-bit microcontroller architecture developed by Atmel.
\seeavrtt

\item M68000\nopagebreak

M68000 is a 16/32-bit instruction set architecture developed by Motorola.
\seemabk

\item MicroBlaze\nopagebreak

MicroBlaze is a 32-bit instruction set architecture developed by Xilinx.
\seemibl

\item MIPS\nopagebreak

MIPS32 and MIPS64 are 32-bit and 64-bit instruction set architectures developed by MIPS Technologies.
\seemips

\item MMIX\nopagebreak

MMIX is a 64-bit instruction set architecture designed by Donald~E.\ Knuth.
\seemmix

\item OpenRISC 1000\nopagebreak

OpenRISC 1000 is a 32/64-bit instruction set architecture developed by OpenCores.
\seeorok

\item PowerPC\nopagebreak

PowerPC is 32-bit and 64-bit instruction set architecture developed by AIM\@.
\seeppc

\item RISC\nopagebreak

RISC is a 32-bit instruction set architecture designed by Niklaus Wirth.
\seerisc

\item WebAssembly\nopagebreak

WebAssembly is a 32-bit instruction set architecture designed by the World Wide Web Consortium (W3C).
\seewasm

\item Xtensa\nopagebreak

Xtensa is a 32-bit instruction set architecture designed by Cadence Design Systems.
\seextensa

\end{itemize}

Finally, the \ecs{} features various linkers and runtime support\index{Runtime support} for the following runtime environments:

\begin{itemize}

\item Atari TOS\nopagebreak

The \ecs{} provides runtime support and a linker for programs that can be executed under Atari TOS\@.

\item AVR Microcontrollers\nopagebreak

The \ecs{} provides runtime support for AVR microcontrollers.
Additionally, it features a linker that generates Intel HEX files that can be used to program these microcontrollers.

\item BIOS\nopagebreak

The \ecs{} provides runtime support and a linker for bootloaders executed by the BIOS\@.

\item DOS\nopagebreak

The \ecs{} features runtime support and a linker for programs that are executed under DOS\@.

\item EFI\nopagebreak

The \ecs{} provides runtime support and a linker for 32-bit as well as 64-bit EFI applications that can be executed in an EFI boot console.

\item Linux\nopagebreak

The \ecs{} features runtime support and a linker for programs that can be executed under Linux-based operating systems.

\item MMIX Simulator\nopagebreak

The \ecs{} provides runtime support and a linker for object files that are executed within the MMIX simulator.

\item OpenRISC 1000 Simulator\nopagebreak

The \ecs{} provides runtime support and a linker for object files that are executed within the OpenRISC 1000 simulator.

\item OS~X\nopagebreak

The \ecs{} provides runtime support and a linker for 32-bit as well as 64-bit programs that can be executed under OS~X\@.

\item Raspberry Pi\nopagebreak

The \ecs{} provides runtime support and a linker for bootloaders running on the Raspberry Pi~2 Model~B.

\item RISC Microcontrollers\nopagebreak

The \ecs{} provides runtime support for RISC microcontrollers.

\item WebAssembly Environments\nopagebreak

The \ecs{} provides runtime support and a linker for WebAssembly modules that can be executed in corresponding web environments.

\item Windows\nopagebreak

The \ecs{} provides runtime support and a linker for 32-bit as well as 64-bit programs that can be executed under Windows.

\end{itemize}

The \ecs{} features additional linkers that are not mentioned above.
See \Documentation{}~\documentationref{object}{Object File Representation} for a list of all linkers provided by the \ecs{}.

\section{Design}\index{Design, of the Eigen Compiler Suite}\index{Eigen Compiler Suite!Design}

The \ecs{} was designed to be a simple and minimalistic but complete and self-contained toolchain for several different programming languages and hardware architectures.
While the main objective of the \ecs{} is to be self-hosting, its design strives for the following goals:

\begin{itemize}

\item Usability\index{Usability}\nopagebreak

All tools featured by the \ecs{} shall be easy to use and should not require complex installations or other prerequisites.
Their user-friendliness shall be enabled by a common basic user interface which does not demand any options or configurations of the user.
This design enforces stability and reproducibility because the result of executing any tool depends solely on the contents of its input files.

\item Reliability\index{Reliability}\nopagebreak

All programming tools of the \ecs{} shall be reliable and produce correct results as well as comprehensive diagnostic messages.
For this purpose, the \ecs{} contains several test and validation suites that enable automatic regression testing of its various tools.
In general, the implementation of the \ecs{} is mainly driven by correctness, robustness, and simplicity.
Therefore, optimizations and performance are explicitly secondary.

\item Portability\index{Portability}\nopagebreak

The \ecs{} is completely written using standard and portable programming language features in order to guarantee the portability of its source code.
All tools of the \ecs{} are therefore compilable and executable within different runtime environments.
Thus, all compilers featured by the \ecs{} are cross compilers by design.

\item Interoperability\index{Interoperability}\nopagebreak

Tools like compilers and assemblers shall enable interoperability in-between all programming languages supported by the \ecs{}.
Usually, programs are written using a single programming language.
Interoperability in this case means, that some parts of the program can also be implemented using a different programming language or with the help of an assembler.
In the end, all these parts work seamlessly together and constitute the complete program.

\item Textual Intermediate Representations\index{Intermediate representations}\nopagebreak

The various tools of the \ecs{} are typically used in a chain, such that the output of one tool becomes the input of the next one.
However, all data that is transported this way should be represented using a human-readable and machine-independent text format.
This design enables programmers and maintainers to view, modify and even manually create all kinds of intermediate data.
In addition to the tools themselves, also the temporary data they generate is therefore portable across different runtime environments.

\item Reuse of Generic Abstractions\index{Generic abstractions}\nopagebreak

The implementation of the \ecs{} shall provide generic abstractions that help to achieve all goals mentioned above.
Additionally, the abstractions shall enable a good code reuse within the implementation itself.
The most important abstractions provided by the \ecs{} are described in Section~\ref{sec:introabstractions}.

\item Complete and Consistent Documentation\nopagebreak

All features of the \ecs{} and especially the usage of its toolchain shall be documented in detail.
For all major components like implementations of a programming language or supported hardware architectures, there are consistent documentations available which describe the usage of the corresponding component and its implementation by the \ecs{}.
This user manual merges all of these documentations into a single document.

\end{itemize}

Some design guidelines like reliability and interoperability have also been incoorporated into the logo of the \ecs{} which combines the three self-contained letters of its abbreviation into a robust three-dimensional structure as shown in Figure~\ref{fig:intrologo}.

\begin{figure}
\centering\ecslogo{5em}
\caption{The logo of the \ecs{}}
\label{fig:intrologo}
\end{figure}

\section{Abstractions}\index{Abstractions}\label{sec:introabstractions}

The \ecs{} supports several different programming languages as well as several different target hardware architectures.
One goal of the \ecs{} is to provide compilers, assemblers, and linkers for all possible combinations of programming languages and target architectures.
For this reason, the \ecs{} defines the following generic abstractions:

\begin{itemize}

\item Object Files\nopagebreak

Object files are the key abstraction in order to enable interoperability between all compilers and assemblers of the \ecs{}.
All these tools generate the same kind of output called object files which can be processed by all linkers and disassemblers of the \ecs{}.
Figure~\ref{fig:introobject} shows how object files are used in-between some compilers, assemblers, linkers, and disassemblers for the AMD64 hardware architecture.
\seeobject

\begin{figure}
\flowgraph{
\resource{\cpp{}\\source code} \ar[d] & \resource{assembly\\source code} \ar[d] & \resource{Oberon\\source code} \ar[d] \\
\converter{cppamd64} \ar[rd] & \converter{amd64asm} \ar[d] & \converter{obamd64} \ar[ld] \\
& \resource{object files} \ar[ld] \ar[d] \ar[rd] \\
\converter{linkbin} \ar[d] & \converter{amd64dism} \ar[d] & \converter{linklib} \ar[d] \\
\resource{binary file} & \resource{disassembly\\listing} & \resource{library file} \\
}\caption{Some tools of the \ecs{} that process object files}
\label{fig:introobject}
\end{figure}

\item Intermediate Code\nopagebreak

Intermediate code is the generic abstraction in-between \emph{front-ends} implementing programming languages and \emph{back-ends} targeting hardware architectures.
Front-ends implement the actual translation of source code into intermediate code, whereas back-ends translate intermediate code into machine code for the respective architecture.
Figure~\ref{fig:introcode} shows some front-ends and back-ends supported by the \ecs{}.
Each data flow within this diagram depicts the input and output of a single compiler for a particular programming language targeting a particular hardware architecture.
As a consequence, programmers only have to provide a single front-end or back-end in order to establish all possible combinations with existing implementations of programming languages and target architectures.
\seecode

\begin{figure}
\flowgraph{
\resource{\cpp{}\\source code} \ar[d] & \resource{Oberon\\source code} \ar[d] & \resource{FALSE\\source code} \ar[d] \\
\converter{Front-End\\for \cpp{}} \ar[rd] & \converter{Front-End for\\Oberon} \ar[d] & \converter{Front-End\\for FALSE} \ar[ld] \\
& \resource{intermediate\\code} \ar[ld] \ar[d] \ar[rd] \\
\converter{Back-End\\for AVR} \ar[d] & \converter{Back-End\\for M68000} \ar[d] & \converter{Back-End\\for AMD64} \ar[d] \\
\resource{AVR\\machine code} & \resource{M68000\\machine code} & \resource{AMD64\\machine code} \\
}\caption{Some front-ends and back-ends provided by the \ecs{}}
\label{fig:introcode}
\end{figure}

\item Generic Assembly Language\nopagebreak

The generic assembly language is an abstraction for the common features of all assemblers provided by the \ecs{}.
It is capable of representing complete assembly programs including all the different instruction sets supported by the \ecs{}.
Concrete assembler tools only have to implement the translation of architecture specific instructions into their binary representation.
Everything else that is actually not dependent on the actual hardware architecture can therefore be translated separately.
\seeassembly

\item Debugging Information\nopagebreak

Debugging information is a generic abstraction of symbolic metadata about a program generated by compilers.
It represents all programming language constructs like functions, variables, data types, and statements compiled into an object file.
Converter tools convert debugging information into a binary debugging data format which decouples compilers from the debugger of the target runtime environment.
\seedebugging

\end{itemize}

Concrete information about how programmers can make use of these abstractions in order to extend the \ecs{} with additional support
for programming languages and hardware architectures is given in \Documentation{}~\documentationref{extensions}{Extensions to the \ecs{}}.

\concludechapter

% Common user interface
% Copyright (C) Florian Negele

% This file is part of the Eigen Compiler Suite.

% Permission is granted to copy, distribute and/or modify this document
% under the terms of the GNU Free Documentation License, Version 1.3
% or any later version published by the Free Software Foundation.

% You should have received a copy of the GNU Free Documentation License
% along with the ECS.  If not, see <https://www.gnu.org/licenses/>.

% Generic documentation utilities
% Copyright (C) Florian Negele

% This file is part of the Eigen Compiler Suite.

% Permission is granted to copy, distribute and/or modify this document
% under the terms of the GNU Free Documentation License, Version 1.3
% or any later version published by the Free Software Foundation.

% You should have received a copy of the GNU Free Documentation License
% along with the ECS.  If not, see <https://www.gnu.org/licenses/>.

\providecommand{\cpp}{C\texttt{++}}
\providecommand{\opt}{_\mathit{opt}}
\providecommand{\tool}[1]{\texttt{#1}}
\providecommand{\version}{Version 0.0.40}
\providecommand{\resource}[1]{*++\txt{#1}}
\providecommand{\ecs}{Eigen Compiler Suite}
\providecommand{\changed}[1]{\underline{#1}}
\providecommand{\toolbox}[1]{\converter{#1}}
\providecommand{\file}{}\renewcommand{\file}[1]{\texttt{#1}}
\providecommand{\alignright}{\hfill\linebreak[0]\hspace*{\fill}}
\providecommand{\converter}[1]{*++[F][F*:white][F,:gray]\txt{#1}}
\providecommand{\documentation}{\ifbook chapter\else document\fi}
\providecommand{\Documentation}{\ifbook Chapter\else Document\fi}
\providecommand{\variable}[1]{\resource{\texttt{\small#1}\\variable}}
\providecommand{\documentationref}[2]{\ifbook\ref{#1}\else``\href{#1}{#2}''~\cite{#1}\fi}
\providecommand{\objfile}[1]{\texttt{#1}\index[runtime]{#1 object file@\texttt{#1} object file}}
\providecommand{\libfile}[1]{\texttt{#1}\index[runtime]{#1 library file@\texttt{#1} library file}}
\providecommand{\epigraph}[2]{\ifbook\begin{quote}\flushright\textit{#1}\par--- #2\end{quote}\fi}
\providecommand{\environmentvariable}[1]{\texttt{#1}\index{Environment variables!#1@\texttt{#1}}}
\providecommand{\environment}[1]{\texttt{#1}\index[environment]{#1 environment@\texttt{#1} environment}}
\providecommand{\toolsection}{}\renewcommand{\toolsection}[1]{\subsection{#1}\label{\prefix:#1}\tool{#1}}
\providecommand{\instruction}{}\renewcommand{\instruction}[2]{\noindent\qquad\pdftooltip{\texttt{#1}}{#2}\refstepcounter{instruction}\par}
\providecommand{\flowgraph}{}\renewcommand{\flowgraph}[1]{\par\sffamily\begin{displaymath}\xymatrix@=4ex{#1}\end{displaymath}\normalfont\par}
\providecommand{\instructionset}{}\renewcommand{\instructionset}[4]{\setcounter{instruction}{0}\begin{multicols}{\ifbook#3\else#4\fi}[{\captionof{table}[#2]{#2 (\ref*{#1:instructions}~instructions)}\label{tab:#1set}\vspace{-2ex}}]\footnotesize\raggedcolumns\input{#1.set}\label{#1:instructions}\end{multicols}}

\providecommand{\gpl}{GNU General Public License}
\providecommand{\rse}{ECS Runtime Support Exception}
\providecommand{\fdl}{\href{https://www.gnu.org/licenses/fdl.html}{GNU Free Documentation License}}

\providecommand{\docbegin}{}
\providecommand{\docend}{}
\providecommand{\doclabel}[1]{\hypertarget{#1}}
\providecommand{\doclink}[2]{\hyperlink{#1}{#2}}
\providecommand{\docsection}[3]{\hypertarget{#1}{\subsection{#2}}\label{sec:#1}\index[library]{#2@#3}}
\providecommand{\docsectionstar}[1]{}
\providecommand{\docsubbegin}{\begin{description}}
\providecommand{\docsubend}{\end{description}}
\providecommand{\docsubsection}[3]{\item[\hypertarget{#1}{#2}]\index[library]{#2@#3}}
\providecommand{\docsubsectionstar}[1]{\smallskip}
\providecommand{\docsubsubsection}[3]{\docsubsection{#1}{#2}{#3}}
\providecommand{\docsubsubsectionstar}[1]{}
\providecommand{\docsubsubsubsection}[3]{}
\providecommand{\docsubsubsubsectionstar}[1]{}
\providecommand{\doctable}{}

\providecommand{\debuggingtool}{}\renewcommand{\debuggingtool}{This tool is provided for debugging purposes.
It allows exposing and modifying an internal data structure that is usually not accessible.
}

\providecommand{\interface}{All tools accept command-line arguments which are taken as names of plain text files containing the source code.
If no arguments are provided, the standard input stream is used instead.
Output files are generated in the current working directory and have the same name as the input file being processed whereas the filename extension gets replaced by an appropriate suffix.
\seeinterface
}

\providecommand{\license}{\noindent Copyright \copyright{} Florian Negele\par\medskip\noindent
Permission is granted to copy, distribute and/or modify this document under the terms of the
\fdl{}, Version 1.3 or any later version published by the \href{https://fsf.org/}{Free Software Foundation}.
}

\providecommand{\ecslogosurface}{
\fill[darkgray] (0,0,0) -- (0,0,3) -- (0,3,3) -- (0,3,1) -- (0,4,1) -- (0,4,3) -- (0,5,3) -- (0,5,0) -- (0,2,0) -- (0,2,2) -- (0,1,2) -- (0,1,0) -- cycle;
\fill[gray] (0,5,0) -- (0,5,3) -- (1,5,3) -- (1,5,1) -- (2,5,1) -- (2,5,3) -- (3,5,3) -- (3,5,0) -- cycle;
\fill[lightgray] (0,0,0) -- (0,1,0) -- (2,1,0) -- (2,4,0) -- (1,4,0) -- (1,3,0) -- (2,3,0) -- (2,2,0) -- (0,2,0) -- (0,5,0) -- (3,5,0) -- (3,0,0) -- cycle;
\begin{scope}[line width=0.5]
\begin{scope}[gray]
\draw (0,0,0) -- (0,1,0);
\draw (2,1,0) -- (2,2,0);
\draw (0,1,2) -- (0,2,2);
\draw (0,2,0) -- (0,5,0);
\draw (2,3,0) -- (2,4,0);
\end{scope}
\begin{scope}[lightgray]
\draw (0,1,0) -- (0,1,2);
\draw (0,3,1) -- (0,3,3);
\draw (0,5,0) -- (0,5,3);
\draw (2,5,1) -- (2,5,3);
\end{scope}
\begin{scope}[white]
\draw (0,1,0) -- (2,1,0);
\draw (1,3,0) -- (2,3,0);
\draw (0,5,0) -- (3,5,0);
\end{scope}
\end{scope}
}

\providecommand{\ecslogo}[1]{
\begin{tikzpicture}[scale={(#1)/((sin(45)+cos(45))*3cm)},x={({-cos(45)*1cm},{sin(45)*sin(30)*1cm})},y={({0cm},{(cos(30)*1cm})},z={({sin(45)*1cm},{cos(45)*sin(30)*1cm})}]
\begin{scope}[darkgray,line width=1]
\draw (0,0,0) -- (0,0,3) -- (0,3,3) -- (2,3,3) -- (2,5,3) -- (3,5,3) -- (3,5,0) -- (3,0,0) -- cycle;
\draw (0,3,1) -- (0,4,1) -- (0,4,3) -- (0,5,3) -- (1,5,3) -- (1,5,1) -- (2,5,1);
\draw (1,3,0) -- (1,4,0) -- (2,4,0);
\end{scope}
\fill[darkgray] (2,0,0) -- (2,0,3) -- (2,5,3) -- (2,5,1) -- (2,4,1) -- (2,4,0) -- cycle;
\fill[lightgray] (2,0,2) -- (0,0,2) -- (0,2,2) -- (2,2,2) -- cycle;
\fill[gray] (0,1,0) -- (2,1,0) -- (2,1,2) -- (0,1,2) -- cycle;
\fill[gray] (0,3,1) -- (0,3,3) -- (2,3,3) -- (2,3,0) -- (1,3,0) -- (1,3,1) -- cycle;
\ecslogosurface
\end{tikzpicture}
}

\providecommand{\shadowedecslogo}[3]{
\begin{tikzpicture}[scale={(#1)/((sin(#2)+cos(#2))*3cm)},x={({-cos(#2)*1cm},{sin(#2)*sin(#3)*1cm})},y={({0cm},{(cos(#3)*1cm})},z={({sin(#2)*1cm},{cos(#2)*sin(#3)*1cm})}]
\shade[top color=lightgray!50!white,bottom color=white,middle color=lightgray!50!white] (0,0,0) -- (3,0,0) -- (3,{-0.5-3*sin(#2)*sin(#3)/cos(#3)},0) -- (0,-0.5,0) -- cycle;
\shade[top color=darkgray!50!gray,bottom color=white,middle color=darkgray!50!white] (0,0,0) -- (0,0,3) -- (0,{-0.5-3*cos(#2)*sin(#3)/cos(#3)},3) -- (0,-0.5,0) -- cycle;
\begin{scope}[y={({(cos(#2)+sin(#2))*0.5cm},{(cos(#2)*sin(#3)-sin(#2)*sin(#3))*0.5cm})}]
\useasboundingbox (3,0,0) -- (0,0,0) -- (0,0,3);
\shade[left color=darkgray!80!black,right color=lightgray,middle color=gray] (0,0,0) -- (0,1,0) -- (0,1,0.5) -- (0,2,0) -- (0,5,0) -- (0,5,3) -- (1,5,3) -- (1,4,3) -- (1,4,2.5) -- (1,3,3) -- (2,5,3) -- (3,5,3) -- (3,0,3) -- cycle;
\clip (0,0,0) -- (0,0,3) -- ({-3*sin(#2)/cos(#2)},0,0) -- cycle;
\shade[left color=darkgray,right color=lightgray!50!gray] (0,0,0) -- (0,1,0) -- (0,1,0.5) -- (0,2,0) -- (0,5,0) -- (0,5,3) -- (1,5,3) -- (1,4,3) -- (1,4,2.5) -- (1,3,3) -- (2,5,3) -- (3,5,3) -- (3,0,3) -- cycle;
\end{scope}
\shade[left color=darkgray,right color=darkgray!80!black] (2,0,0) -- (2,0,3) -- (2,5,3) -- (2,5,1) -- (2,4,1) -- (2,4,0) -- cycle;
\shade[left color=darkgray!90!black,right color=gray!80!darkgray] (2,0,2) -- (0,0,2) -- (0,2,2) -- (2,2,2) -- cycle;
\shade[top color=darkgray!90!black,bottom color=gray!80!darkgray] (0,1,0) -- (2,1,0) -- (2,1,2) -- (0,1,2) -- cycle;
\shade[top color=darkgray!90!black,bottom color=gray!80!darkgray] (0,3,1) -- (0,3,3) -- (2,3,3) -- (2,3,0) -- (1,3,0) -- (1,3,1) -- cycle;
\fill[gray] (2,1,0) -- (1.5,1,0.5) -- (0,1,0.5) -- (0,1,0) -- cycle;
\fill[gray] (1,3,2) -- (0.5,3,2) -- (0.5,3,3) -- (1,3,3) -- cycle;
\fill[gray] (2,3,0) -- (1.5,3,0.5) -- (1,3,0.5) -- (1,3,0) -- cycle;
\ecslogosurface
\end{tikzpicture}
}

\providecommand{\cpplogo}[1]{
\begin{tikzpicture}[scale=(#1)/512em]
\fill[gray] (435.2794,398.7159) -- (247.1911,507.3075) .. controls (236.3563,513.5642) and (218.6240,513.5642) .. (207.7892,507.3075) -- (19.7009,398.7159) .. controls (8.8646,392.4606) and (0.0000,377.1043) .. (0.0000,364.5924) -- (0.0000,147.4076) .. controls (0.8430,132.8363) and (8.2856,120.7683) .. (19.7009,113.2842) -- (207.7892,4.6926) .. controls (218.6240,-1.5642) and (236.3564,-1.5642) .. (247.1911,4.6926) -- (435.2794,113.2842) .. controls (447.5273,121.4304) and (454.4987,133.6918) .. (454.9803,147.4076) -- (454.9803,364.5924) .. controls (454.5404,377.7571) and (446.6566,391.0351) .. (435.2794,398.7159) -- cycle(75.8301,255.9993) .. controls (74.9389,404.0881) and (273.2892,469.4783) .. (358.8263,331.8769) -- (293.1917,293.8965) .. controls (253.5702,359.4301) and (155.1909,335.9977) .. (151.6601,255.9993) .. controls (152.7204,182.2703) and (249.4137,148.0211) .. (293.1961,218.1065) -- (358.8308,180.1276) .. controls (283.4477,49.2645) and (79.6318,96.3470) .. (75.8301,255.9993) -- cycle(379.1503,247.5747) -- (362.2982,247.5747) -- (362.2982,230.7226) -- (345.4490,230.7226) -- (345.4490,247.5747) -- (328.5969,247.5747) -- (328.5969,264.4254) -- (345.4490,264.4254) -- (345.4490,281.2759) -- (362.2982,281.2759) -- (362.2982,264.4254) -- (379.1503,264.4254) -- cycle(442.3420,247.5747) -- (425.4899,247.5747) -- (425.4899,230.7226) -- (408.6408,230.7226) -- (408.6408,247.5747) -- (391.7886,247.5747) -- (391.7886,264.4254) -- (408.6408,264.4254) -- (408.6408,281.2759) -- (425.4899,281.2759) -- (425.4899,264.4254) -- (442.3420,264.4254) -- cycle;
\end{tikzpicture}
}

\providecommand{\fallogo}[1]{
\begin{tikzpicture}[scale=(#1)/512em]
\fill[gray] (185.7774,0.0000) .. controls (200.4486,15.9798) and (226.8966,8.7148) .. (235.0426,31.5836) .. controls (249.5297,58.0598) and (247.9581,97.9161) .. (280.3335,110.9762) .. controls (309.1690,120.3496) and (337.8406,104.2727) .. (366.5753,103.9379) .. controls (373.4449,111.5171) and (379.2885,128.2574) .. (383.9755,108.9744) .. controls (396.6979,102.5615) and (437.2808,107.6681) .. (426.9652,124.3252) .. controls (408.9822,121.0785) and (412.4742,146.0729) .. (426.5192,131.4996) .. controls (433.8413,120.8489) and (465.1541,126.5522) .. (441.9067,135.7950) .. controls (396.1879,157.7478) and (344.1112,161.5079) .. (298.5528,183.5702) .. controls (277.7471,193.5198) and (284.6941,218.7163) .. (285.2127,236.9640) .. controls (292.3599,316.2826) and (307.3929,394.6311) .. (317.1198,473.6154) .. controls (329.0637,505.4736) and (292.1195,528.5004) .. (265.9183,511.2761) .. controls (237.9284,499.2462) and (237.3684,465.2681) .. (230.9102,439.9421) .. controls (218.6692,374.3397) and (215.6307,306.9662) .. (198.1732,242.3977) .. controls (183.1379,232.7444) and (164.4245,256.0298) .. (149.0430,261.4799) .. controls (116.9328,279.2585) and (87.1822,308.5851) .. (48.2293,307.8914) .. controls (21.3220,306.9037) and (-15.9107,281.8761) .. (7.2921,252.7908) .. controls (29.7799,220.6177) and (67.5177,204.3028) .. (100.9287,185.9449) .. controls (130.8217,170.8906) and (161.1548,156.5903) .. (191.0278,141.5847) .. controls (196.1738,120.0520) and (186.6049,95.2409) .. (186.8382,72.4353) .. controls (185.5234,48.4204) and (183.1700,23.9341) .. (185.7774,0.0000) -- cycle;
\end{tikzpicture}
}

\providecommand{\oblogo}[1]{
\begin{tikzpicture}[scale=(#1)/512em]
\fill[gray] (160.3865,208.9117) .. controls (154.0879,214.6478) and (149.0735,221.2409) .. (145.4125,228.5384) .. controls (184.8790,248.4273) and (234.7122,269.8787) .. (297.5493,291.8782) .. controls (300.3943,281.4769) and (300.9552,268.7619) .. (300.4023,255.2389) .. controls (248.9909,244.7891) and (200.0310,225.9279) .. (160.3865,208.9117) -- cycle(225.7398,392.6996) .. controls (308.0209,392.1716) and (359.3326,345.9277) .. (368.7203,285.2098) .. controls (376.6742,197.1784) and (311.7194,141.3342) .. (205.4287,142.1456) .. controls (139.9485,141.4804) and (88.7155,166.1957) .. (73.5775,228.0086) .. controls (52.0297,320.3408) and (123.4078,391.0103) .. (225.7398,392.6996) -- cycle(216.0739,176.4733) .. controls (268.9183,179.2424) and (315.8292,206.5488) .. (312.7454,265.1139) .. controls (313.2769,315.6384) and (286.5993,353.4946) .. (216.6040,355.7934) .. controls (162.4657,355.7934) and (126.0914,317.5023) .. (126.0914,260.5103) .. controls (126.1733,214.2900) and (163.3363,176.2849) .. (216.0739,176.4733) -- cycle(76.4897,189.1754) .. controls (13.1586,147.5631) and (0.0000,119.4207) .. (0.0000,119.4207) -- (90.6499,170.1632) .. controls (85.3004,175.8497) and (80.5994,182.1633) .. (76.4897,189.1754) -- cycle(353.9486,119.3004) -- (402.9482,119.3004) .. controls (427.0025,137.0797) and (450.9893,162.7034) .. (474.9529,191.0213) .. controls (509.3540,228.5339) and (531.3391,294.2091) .. (487.8149,312.1206) .. controls (462.8165,324.7652) and (394.3874,316.8943) .. (373.8912,313.6651) .. controls (379.9291,297.7449) and (383.2899,278.4204) .. (381.4989,257.7214) .. controls (420.3069,248.0321) and (421.9610,218.3461) .. (407.7867,192.6417) .. controls (391.1113,162.4018) and (370.1114,132.9097) .. (353.9486,119.3004) -- cycle;
\end{tikzpicture}
}

\providecommand{\markuptable}{
\begin{table}
\sffamily\centering
\begin{tabular}{@{}lcl@{}}
\toprule
\texttt{//italics//} & $\rightarrow$ & \textit{italics} \\
\midrule
\texttt{**bold**} & $\rightarrow$ & \textbf{bold} \\
\midrule
\texttt{\# ordered list} & & 1 ordered list \\
\texttt{\# second item} & $\rightarrow$ & 2 second item \\
\texttt{\#\# sub item} & & \hspace{1em} 1 sub item \\
\midrule
\texttt{* unordered list} & & $\bullet$ unordered list \\
\texttt{* second item} & $\rightarrow$ & $\bullet$ second item \\
\texttt{** sub item} & & \hspace{1em} $\bullet$ sub item \\
\midrule
\texttt{link to [[label]]} & $\rightarrow$ & link to \underline{label} \\
\midrule
\texttt{<{}<label>{}> definition } & $\rightarrow$ & definition \\
\midrule
\texttt{[[url|link name]]} & $\rightarrow$ & \underline{link name} \\
\midrule\addlinespace
\texttt{= large heading} & & {\Large large heading} \smallskip \\
\texttt{== medium heading} & $\rightarrow$ & {\large medium heading} \\
\texttt{=== small heading} & & small heading \\
\midrule
\texttt{no line break} & & no line break for paragraphs \\
\texttt{for paragraphs} & $\rightarrow$ \\
& & use empty line \\
\texttt{use empty line} \\
\midrule
\texttt{force\textbackslash\textbackslash line break} & $\rightarrow$ & force \\
& & line break \\
\midrule
\texttt{horizontal line} & $\rightarrow$ & horizontal line \\
\texttt{----} & & \hrulefill \\
\midrule
\texttt{|=a|=table|=header} & & \underline{a \enspace table \enspace header} \\
\texttt{|a|table|row} & $\rightarrow$ & a \enspace table \enspace row \\
\texttt{|b|table|row} & & b \enspace table \enspace row \\
\midrule
\texttt{\{\{\{} \\
\texttt{unformatted} & $\rightarrow$ & \texttt{unformatted} \\
\texttt{code} & & \texttt{code} \\
\texttt{\}\}\}} \\
\midrule\addlinespace
\texttt{@ new article} & & {\Large 1.\ new article} \smallskip \\
\texttt{@ second article} & $\rightarrow$ & {\Large 2.\ second article} \smallskip \\
\texttt{@@ sub article} & & {\large 2.1.\ sub article} \\
\bottomrule
\end{tabular}
\normalfont\caption{Elements of the generic documentation markup language}
\label{tab:docmarkup}
\end{table}
}

\providecommand{\startchapter}[4]{
\documentclass[11pt,a4paper]{article}
\usepackage{booktabs}
\usepackage[format=hang,labelfont=bf]{caption}
\usepackage{changepage}
\usepackage[T1]{fontenc}
\usepackage[margin=2cm]{geometry}
\usepackage{hyperref}
\usepackage[american]{isodate}
\usepackage{lmodern}
\usepackage{longtable}
\usepackage{mathptmx}
\usepackage{microtype}
\usepackage[toc]{multitoc}
\usepackage{multirow}
\usepackage[all]{nowidow}
\usepackage{pdfcomment}
\usepackage{syntax}
\usepackage{tikz}
\usepackage[all]{xy}
\hypersetup{pdfborder={0 0 0},bookmarksnumbered=true,pdftitle={\ecs{}: #2},pdfauthor={Florian Negele},pdfsubject={\ecs{}},pdfkeywords={#1}}
\setlength{\grammarindent}{8em}\setlength{\grammarparsep}{0.2ex}
\setlength{\columnsep}{2em}
\newcommand{\prefix}{}
\newcounter{instruction}
\bibliographystyle{unsrt}
\renewcommand{\index}[2][]{}
\renewcommand{\arraystretch}{1.05}
\renewcommand{\floatpagefraction}{0.7}
\renewcommand{\syntleft}{\itshape}\renewcommand{\syntright}{}
\title{\vspace{-5ex}\Huge{\ecs{}}\medskip\hrule}
\author{\huge{#2}}
\date{\medskip\version}
\newif\ifbook\bookfalse
\pagestyle{headings}
\frenchspacing
\begin{document}
\maketitle\thispagestyle{empty}\noindent#4\setlength{\columnseprule}{0.4pt}\tableofcontents\setlength{\columnseprule}{0pt}\vfill\pagebreak[3]\null\vfill\bigskip\noindent
\parbox{\textwidth-4em}{\license The contents of this \documentation{} are part of the \href{manual}{\ecs{} User Manual}~\cite{manual} and correspond to Chapter ``\href{manual\##3}{#1}''.\alignright\mbox{\today}}
\parbox{4em}{\flushright\ecslogo{3em}}
\clearpage
}

\providecommand{\concludechapter}{
\vfill\pagebreak[3]\null\vfill
\thispagestyle{myheadings}\markright{REFERENCES}
\noindent\begin{minipage}{\textwidth}\begin{multicols}{2}[\section*{References}]
\renewcommand{\section}[2]{}\small\bibliography{references}
\end{multicols}\end{minipage}\end{document}
}

\providecommand{\startpresentation}[2]{
\documentclass[14pt,aspectratio=43,usepdftitle=false]{beamer}
\usepackage{booktabs}
\usepackage{etex}
\usepackage{multicol}
\usepackage{tikz}
\usepackage[all]{xy}
\bibliographystyle{unsrt}
\setlength{\columnsep}{1em}
\setlength{\leftmargini}{1em}
\setbeamercolor{title}{fg=black}
\setbeamercolor{structure}{fg=darkgray}
\setbeamercolor{bibliography item}{fg=darkgray}
\setbeamerfont{title}{series=\bfseries}
\setbeamerfont{subtitle}{series=\normalfont}
\setbeamerfont*{frametitle}{parent=title}
\setbeamerfont{block title}{series=\bfseries}
\setbeamerfont*{framesubtitle}{parent=subtitle}
\setbeamersize{text margin left=1em,text margin right=1em}
\setbeamertemplate{navigation symbols}{}
\setbeamertemplate{itemize item}[circle]{}
\setbeamertemplate{bibliography item}[triangle]{}
\setbeamertemplate{bibliography entry author}{\usebeamercolor[fg]{bibliography item}}
\setbeamertemplate{frametitle}{\medskip\usebeamerfont{frametitle}\color{gray}\raisebox{-2.5ex}[0ex][0ex]{\rule{0.1em}{4.5ex}}}
\addtobeamertemplate{frametitle}{}{\hspace{0.4em}\usebeamercolor[fg]{title}\insertframetitle\par\vspace{0.2ex}\hspace{0.5em}\usebeamerfont{framesubtitle}\insertframesubtitle}
\hypersetup{pdfborder={0 0 0},bookmarksnumbered=true,bookmarksopen=true,bookmarksopenlevel=0,pdftitle={\ecs{}: #1},pdfauthor={Florian Negele},pdfsubject={\ecs{}},pdfkeywords={#1}}
\renewcommand{\flowgraph}[1]{\resizebox{\textwidth}{!}{$$\xymatrix{##1}$$}}
\title{\ecs{}\medskip\hrule\medskip}
\institute{\shadowedecslogo{5em}{30}{15}}
\date{\version}
\subtitle{#1}
\begin{document}
\begin{frame}[plain]\titlepage\nocite{manual}\end{frame}
\begin{frame}{Contents}{#1}\begin{center}\tableofcontents\end{center}\end{frame}
}

\providecommand{\concludepresentation}{
\begin{frame}{References}\begin{footnotesize}\setlength{\columnseprule}{0.4pt}\begin{multicols}{2}\bibliography{references}\end{multicols}\end{footnotesize}\end{frame}
\end{document}
}

\providecommand{\startbook}[1]{
\documentclass[10pt,paper=17cm:24cm,DIV=13,twoside=semi,headings=normal,numbers=noendperiod,cleardoublepage=plain]{scrbook}
\usepackage{atveryend}
\usepackage{booktabs}
\usepackage{caption}
\usepackage{changepage}
\usepackage[T1]{fontenc}
\usepackage{imakeidx}
\usepackage{hyperref}
\usepackage[american]{isodate}
\usepackage{lmodern}
\usepackage{longtable}
\usepackage{mathptmx}
\usepackage[final]{microtype}
\usepackage{multicol}
\usepackage{multirow}
\usepackage[all]{nowidow}
\usepackage{pdfcomment}
\usepackage{scrlayer-scrpage}
\usepackage{setspace}
\usepackage{syntax}
\usepackage[eventxtindent=4pt,oddtxtexdent=4pt]{thumbs}
\usepackage{tikz}
\usepackage[all]{xy}
\hyphenation{Micro-Blaze Open-Cores Open-RISC Power-PC}
\hypersetup{pdfborder={0 0 0},bookmarksnumbered=true,bookmarksopen=true,bookmarksopenlevel=0,pdftitle={\ecs{}: #1},pdfauthor={Florian Negele},pdfsubject={\ecs{}},pdfkeywords={#1}}
\setlength{\grammarindent}{8em}\setlength{\grammarparsep}{0.7ex}
\setkomafont{captionlabel}{\usekomafont{descriptionlabel}}
\renewcommand{\arraystretch}{1.05}\setstretch{1.1}
\renewcommand{\chapterformat}{\thechapter\autodot\enskip\raisebox{-1ex}[0ex][0ex]{\color{gray}\rule{0.1em}{3.5ex}}\enskip}
\renewcommand{\startchapter}[4]{\hypertarget{##3}{\chapter{##1}}\label{##3}##4\addthumb{##1}{\LARGE\sffamily\bfseries\thechapter}{white}{gray}\renewcommand{\prefix}{##3}}
\renewcommand{\concludechapter}{\clearpage{\stopthumb\cleardoublepage}}
\renewcommand{\syntleft}{\itshape}\renewcommand{\syntright}{}
\renewcommand{\floatpagefraction}{0.7}
\renewcommand{\partheademptypage}{}
\DeclareMicrotypeAlias{lmss}{cmr}
\newcommand{\prefix}{}
\newcounter{instruction}
\bibliographystyle{unsrt}
\newif\ifbook\booktrue
\makeindex[intoc,title=Index]
\makeindex[intoc,name=tools,title=Index of Tools,columns=3]
\makeindex[intoc,name=library,title=Index of Library Names]
\makeindex[intoc,name=runtime,title=Index of Runtime Support]
\makeindex[intoc,name=environment,title=Index of Target Environments]
\indexsetup{toclevel=chapter,headers={\indexname}{\indexname}}
\frenchspacing
\begin{document}
\pagenumbering{alph}
\begin{titlepage}\centering
\huge\sffamily\null\vfill\textbf{\ecs{}}\bigskip\hrule\bigskip#1
\normalsize\normalfont\vfill\vfill\shadowedecslogo{10em}{30}{15}
\large\vfill\vfill\version
\end{titlepage}
\null\vfill
\thispagestyle{empty}
\noindent\today\par\medskip
\license A copy of this license is included in Appendix~\ref{fdl} on page~\pageref{fdl}.
All product names used herein are for identification purposes only and may be trademarks of their respective companies.
\concludechapter
\frontmatter
\setcounter{tocdepth}{1}
\tableofcontents
\setcounter{tocdepth}{2}
\concludechapter
\listoffigures
\concludechapter
\listoftables
\concludechapter
}

\providecommand{\concludebook}{
\backmatter
\addtocontents{toc}{\protect\setcounter{tocdepth}{-1}}
\phantomsection\addcontentsline{toc}{part}{Bibliography}
\bibliography{references}
\concludechapter
\phantomsection\addcontentsline{toc}{part}{Indexes}
\printindex
\concludechapter
\indexprologue{\label{idx:tools}}
\printindex[tools]
\concludechapter
\printindex[library]
\concludechapter
\indexprologue{\label{idx:runtime}}
\printindex[runtime]
\concludechapter
\indexprologue{\label{idx:environment}}
\printindex[environment]
\concludechapter
\pagestyle{empty}\pagenumbering{Alph}\null\clearpage
\null\vfill\centering\ecslogo{4em}\par\medskip\license
\end{document}
}

% chapter references

\providecommand{\seedocumentationref}{}\renewcommand{\seedocumentationref}[3]{#1, see \Documentation{}~\documentationref{#2}{#3}. }
\providecommand{\seeinterface}{}\renewcommand{\seeinterface}{\ifbook See \Documentation{}~\documentationref{interface}{User Interface} for more information about the common user interface of all of these tools. \fi}
\providecommand{\seeguide}{}\renewcommand{\seeguide}{\seedocumentationref{For basic examples of using some of these tools in practice}{guide}{User Guide}}
\providecommand{\seecpp}{}\renewcommand{\seecpp}{\seedocumentationref{For more information about the \cpp{} programming language and its implementation by the \ecs{}}{cpp}{User Manual for \cpp{}}}
\providecommand{\seefalse}{}\renewcommand{\seefalse}{\seedocumentationref{For more information about the FALSE programming language and its implementation by the \ecs{}}{false}{User Manual for FALSE}}
\providecommand{\seeoberon}{}\renewcommand{\seeoberon}{\seedocumentationref{For more information about the Oberon programming language and its implementation by the \ecs{}}{oberon}{User Manual for Oberon}}
\providecommand{\seeassembly}{}\renewcommand{\seeassembly}{\seedocumentationref{For more information about the generic assembly language and how to use it}{assembly}{Generic Assembly Language Specification}}
\providecommand{\seeamd}{}\renewcommand{\seeamd}{\seedocumentationref{For more information about how the \ecs{} supports the AMD64 hardware architecture}{amd64}{AMD64 Hardware Architecture Support}}
\providecommand{\seearm}{}\renewcommand{\seearm}{\seedocumentationref{For more information about how the \ecs{} supports the ARM hardware architecture}{arm}{ARM Hardware Architecture Support}}
\providecommand{\seeavr}{}\renewcommand{\seeavr}{\seedocumentationref{For more information about how the \ecs{} supports the AVR hardware architecture}{avr}{AVR Hardware Architecture Support}}
\providecommand{\seeavrtt}{}\renewcommand{\seeavrtt}{\seedocumentationref{For more information about how the \ecs{} supports the AVR32 hardware architecture}{avr32}{AVR32 Hardware Architecture Support}}
\providecommand{\seemabk}{}\renewcommand{\seemabk}{\seedocumentationref{For more information about how the \ecs{} supports the M68000 hardware architecture}{m68k}{M68000 Hardware Architecture Support}}
\providecommand{\seemibl}{}\renewcommand{\seemibl}{\seedocumentationref{For more information about how the \ecs{} supports the MicroBlaze hardware architecture}{mibl}{MicroBlaze Hardware Architecture Support}}
\providecommand{\seemips}{}\renewcommand{\seemips}{\seedocumentationref{For more information about how the \ecs{} supports the MIPS32 and MIPS64 hardware architectures}{mips}{MIPS Hardware Architecture Support}}
\providecommand{\seemmix}{}\renewcommand{\seemmix}{\seedocumentationref{For more information about how the \ecs{} supports the MMIX hardware architecture}{mmix}{MMIX Hardware Architecture Support}}
\providecommand{\seeorok}{}\renewcommand{\seeorok}{\seedocumentationref{For more information about how the \ecs{} supports the OpenRISC 1000 hardware architecture}{or1k}{OpenRISC 1000 Hardware Architecture Support}}
\providecommand{\seeppc}{}\renewcommand{\seeppc}{\seedocumentationref{For more information about how the \ecs{} supports the PowerPC hardware architecture}{ppc}{PowerPC Hardware Architecture Support}}
\providecommand{\seerisc}{}\renewcommand{\seerisc}{\seedocumentationref{For more information about how the \ecs{} supports the RISC hardware architecture}{risc}{RISC Hardware Architecture Support}}
\providecommand{\seewasm}{}\renewcommand{\seewasm}{\seedocumentationref{For more information about how the \ecs{} supports the WebAssembly architecture}{wasm}{WebAssembly Architecture Support}}
\providecommand{\seedocumentation}{}\renewcommand{\seedocumentation}{\seedocumentationref{For more information about generic documentations and their generation by the \ecs{}}{documentation}{Generic Documentation Generation}}
\providecommand{\seedebugging}{}\renewcommand{\seedebugging}{\seedocumentationref{For more information about debugging information and its representation}{debugging}{Debugging Information Representation}}
\providecommand{\seecode}{}\renewcommand{\seecode}{\seedocumentationref{For more information about intermediate code and its purpose}{code}{Intermediate Code Representation}}
\providecommand{\seeobject}{}\renewcommand{\seeobject}{\seedocumentationref{For more information about object files and their purpose}{object}{Object File Representation}}

% generic documentation tools

\providecommand{\docprint}{
\toolsection{docprint} is a pretty printer for generic documentations.
It reformats generic documentations and writes it to the standard output stream.
\debuggingtool
\flowgraph{\resource{generic\\documentation} \ar[r] & \toolbox{docprint} \ar[r] & \resource{generic\\documentation}}
\seedocumentation
}

\providecommand{\doccheck}{
\toolsection{doccheck} is a syntactic and semantic checker for generic documentations.
It just performs syntactic and semantic checks on generic documentations and writes its diagnostic messages to the standard error stream.
\debuggingtool
\flowgraph{\resource{generic\\documentation} \ar[r] & \toolbox{doccheck} \ar[r] & \resource{diagnostic\\messages}}
\seedocumentation
}

\providecommand{\dochtml}{
\toolsection{dochtml} is an HTML documentation generator for generic documentations.
It processes several generic documentations and assembles all information therein into an HTML document.
\debuggingtool
\flowgraph{\resource{generic\\documentation} \ar[r] & \toolbox{dochtml} \ar[r] & \resource{HTML\\document}}
\seedocumentation
}

\providecommand{\doclatex}{
\toolsection{doclatex} is a Latex documentation generator for generic documentations.
It processes several generic documentations and assembles all information therein into a Latex document.
\debuggingtool
\flowgraph{\resource{generic\\documentation} \ar[r] & \toolbox{doclatex} \ar[r] & \resource{Latex\\document}}
\seedocumentation
}

% intermediate code tools

\providecommand{\cdcheck}{
\toolsection{cdcheck} is a syntactic and semantic checker for intermediate code.
It just performs syntactic and semantic checks on programs written in intermediate code and writes its diagnostic messages to the standard error stream.
\debuggingtool
\flowgraph{\resource{intermediate\\code} \ar[r] & \toolbox{cdcheck} \ar[r] & \resource{diagnostic\\messages}}
\seeassembly\seecode
}

\providecommand{\cdopt}{
\toolsection{cdopt} is an optimizer for intermediate code.
It performs various optimizations on programs written in intermediate code and writes the result to the standard output stream.
\debuggingtool
\flowgraph{\resource{intermediate\\code} \ar[r] & \toolbox{cdopt} \ar[r] & \resource{optimized\\code}}
\seeassembly\seecode
}

\providecommand{\cdrun}{
\toolsection{cdrun} is an interpreter for intermediate code.
It processes and executes programs written in intermediate code.
The following code sections are predefined and have the usual semantics:
\texttt{abort}, \texttt{\_Exit}, \texttt{fflush}, \texttt{floor}, \texttt{fputc}, \texttt{free}, \texttt{getchar}, \texttt{malloc}, and \texttt{putchar}.
Diagnostic messages about invalid operations include the name of the executed code section and the index of the erroneous instruction.
\debuggingtool
\flowgraph{\resource{intermediate\\code} \ar[r] & \toolbox{cdrun} \ar@/u/[r] & \resource{input/\\output} \ar@/d/[l]}
\seeassembly\seecode
}

\providecommand{\cdamda}{
\toolsection{cdamd16} is a compiler for intermediate code targeting the AMD64 hardware architecture.
It generates machine code for AMD64 processors from programs written in intermediate code and stores it in corresponding object files.
The compiler generates machine code for the 16-bit operating mode defined by the AMD64 architecture.
It also creates a debugging information file as well as an assembly file containing a listing of the generated machine code.
\debuggingtool
\flowgraph{\resource{intermediate\\code} \ar[r] & \toolbox{cdamd16} \ar[r] \ar[d] \ar[rd] & \resource{object file} \\ & \resource{assembly\\listing} & \resource{debugging\\information}}
\seeassembly\seeamd\seeobject\seecode\seedebugging
}

\providecommand{\cdamdb}{
\toolsection{cdamd32} is a compiler for intermediate code targeting the AMD64 hardware architecture.
It generates machine code for AMD64 processors from programs written in intermediate code and stores it in corresponding object files.
The compiler generates machine code for the 32-bit operating mode defined by the AMD64 architecture.
It also creates a debugging information file as well as an assembly file containing a listing of the generated machine code.
\debuggingtool
\flowgraph{\resource{intermediate\\code} \ar[r] & \toolbox{cdamd32} \ar[r] \ar[d] \ar[rd] & \resource{object file} \\ & \resource{assembly\\listing} & \resource{debugging\\information}}
\seeassembly\seeamd\seeobject\seecode\seedebugging
}

\providecommand{\cdamdc}{
\toolsection{cdamd64} is a compiler for intermediate code targeting the AMD64 hardware architecture.
It generates machine code for AMD64 processors from programs written in intermediate code and stores it in corresponding object files.
The compiler generates machine code for the 64-bit operating mode defined by the AMD64 architecture.
It also creates a debugging information file as well as an assembly file containing a listing of the generated machine code.
\debuggingtool
\flowgraph{\resource{intermediate\\code} \ar[r] & \toolbox{cdamd64} \ar[r] \ar[d] \ar[rd] & \resource{object file} \\ & \resource{assembly\\listing} & \resource{debugging\\information}}
\seeassembly\seeamd\seeobject\seecode\seedebugging
}

\providecommand{\cdarma}{
\toolsection{cdarma32} is a compiler for intermediate code targeting the ARM hardware architecture.
It generates machine code for ARM processors executing A32 instructions from programs written in intermediate code and stores it in corresponding object files.
It also creates a debugging information file as well as an assembly file containing a listing of the generated machine code.
\debuggingtool
\flowgraph{\resource{intermediate\\code} \ar[r] & \toolbox{cdarma32} \ar[r] \ar[d] \ar[rd] & \resource{object file} \\ & \resource{assembly\\listing} & \resource{debugging\\information}}
\seeassembly\seearm\seeobject\seecode\seedebugging
}

\providecommand{\cdarmb}{
\toolsection{cdarma64} is a compiler for intermediate code targeting the ARM hardware architecture.
It generates machine code for ARM processors executing A64 instructions from programs written in intermediate code and stores it in corresponding object files.
It also creates a debugging information file as well as an assembly file containing a listing of the generated machine code.
\debuggingtool
\flowgraph{\resource{intermediate\\code} \ar[r] & \toolbox{cdarma64} \ar[r] \ar[d] \ar[rd] & \resource{object file} \\ & \resource{assembly\\listing} & \resource{debugging\\information}}
\seeassembly\seearm\seeobject\seecode\seedebugging
}

\providecommand{\cdarmc}{
\toolsection{cdarmt32} is a compiler for intermediate code targeting the ARM hardware architecture.
It generates machine code for ARM processors without floating-point extension executing T32 instructions from programs written in intermediate code and stores it in corresponding object files.
It also creates a debugging information file as well as an assembly file containing a listing of the generated machine code.
\debuggingtool
\flowgraph{\resource{intermediate\\code} \ar[r] & \toolbox{cdarmt32} \ar[r] \ar[d] \ar[rd] & \resource{object file} \\ & \resource{assembly\\listing} & \resource{debugging\\information}}
\seeassembly\seearm\seeobject\seecode\seedebugging
}

\providecommand{\cdarmcfpe}{
\toolsection{cdarmt32fpe} is a compiler for intermediate code targeting the ARM hardware architecture.
It generates machine code for ARM processors with floating-point extension executing T32 instructions from programs written in intermediate code and stores it in corresponding object files.
It also creates a debugging information file as well as an assembly file containing a listing of the generated machine code.
\debuggingtool
\flowgraph{\resource{intermediate\\code} \ar[r] & \toolbox{cdarmt32fpe} \ar[r] \ar[d] \ar[rd] & \resource{object file} \\ & \resource{assembly\\listing} & \resource{debugging\\information}}
\seeassembly\seearm\seeobject\seecode\seedebugging
}

\providecommand{\cdavr}{
\toolsection{cdavr} is a compiler for intermediate code targeting the AVR hardware architecture.
It generates machine code for AVR processors from programs written in intermediate code and stores it in corresponding object files.
It also creates a debugging information file as well as an assembly file containing a listing of the generated machine code.
\debuggingtool
\flowgraph{\resource{intermediate\\code} \ar[r] & \toolbox{cdavr} \ar[r] \ar[d] \ar[rd] & \resource{object file} \\ & \resource{assembly\\listing} & \resource{debugging\\information}}
\seeassembly\seeavr\seeobject\seecode\seedebugging
}

\providecommand{\cdavrtt}{
\toolsection{cdavr32} is a compiler for intermediate code targeting the AVR32 hardware architecture.
It generates machine code for AVR32 processors from programs written in intermediate code and stores it in corresponding object files.
It also creates a debugging information file as well as an assembly file containing a listing of the generated machine code.
\debuggingtool
\flowgraph{\resource{intermediate\\code} \ar[r] & \toolbox{cdavr32} \ar[r] \ar[d] \ar[rd] & \resource{object file} \\ & \resource{assembly\\listing} & \resource{debugging\\information}}
\seeassembly\seeavrtt\seeobject\seecode\seedebugging
}

\providecommand{\cdmabk}{
\toolsection{cdm68k} is a compiler for intermediate code targeting the M68000 hardware architecture.
It generates machine code for M68000 processors from programs written in intermediate code and stores it in corresponding object files.
It also creates a debugging information file as well as an assembly file containing a listing of the generated machine code.
\debuggingtool
\flowgraph{\resource{intermediate\\code} \ar[r] & \toolbox{cdm68k} \ar[r] \ar[d] \ar[rd] & \resource{object file} \\ & \resource{assembly\\listing} & \resource{debugging\\information}}
\seeassembly\seemabk\seeobject\seecode\seedebugging
}

\providecommand{\cdmibl}{
\toolsection{cdmibl} is a compiler for intermediate code targeting the MicroBlaze hardware architecture.
It generates machine code for MicroBlaze processors from programs written in intermediate code and stores it in corresponding object files.
It also creates a debugging information file as well as an assembly file containing a listing of the generated machine code.
\debuggingtool
\flowgraph{\resource{intermediate\\code} \ar[r] & \toolbox{cdmibl} \ar[r] \ar[d] \ar[rd] & \resource{object file} \\ & \resource{assembly\\listing} & \resource{debugging\\information}}
\seeassembly\seemibl\seeobject\seecode\seedebugging
}

\providecommand{\cdmipsa}{
\toolsection{cdmips32} is a compiler for intermediate code targeting the MIPS32 hardware architecture.
It generates machine code for MIPS32 processors from programs written in intermediate code and stores it in corresponding object files.
It also creates a debugging information file as well as an assembly file containing a listing of the generated machine code.
\debuggingtool
\flowgraph{\resource{intermediate\\code} \ar[r] & \toolbox{cdmips32} \ar[r] \ar[d] \ar[rd] & \resource{object file} \\ & \resource{assembly\\listing} & \resource{debugging\\information}}
\seeassembly\seemips\seeobject\seecode\seedebugging
}

\providecommand{\cdmipsb}{
\toolsection{cdmips64} is a compiler for intermediate code targeting the MIPS64 hardware architecture.
It generates machine code for MIPS64 processors from programs written in intermediate code and stores it in corresponding object files.
It also creates a debugging information file as well as an assembly file containing a listing of the generated machine code.
\debuggingtool
\flowgraph{\resource{intermediate\\code} \ar[r] & \toolbox{cdmips64} \ar[r] \ar[d] \ar[rd] & \resource{object file} \\ & \resource{assembly\\listing} & \resource{debugging\\information}}
\seeassembly\seemips\seeobject\seecode\seedebugging
}

\providecommand{\cdmmix}{
\toolsection{cdmmix} is a compiler for intermediate code targeting the MMIX hardware architecture.
It generates machine code for MMIX processors from programs written in intermediate code and stores it in corresponding object files.
It also creates a debugging information file as well as an assembly file containing a listing of the generated machine code.
\debuggingtool
\flowgraph{\resource{intermediate\\code} \ar[r] & \toolbox{cdmmix} \ar[r] \ar[d] \ar[rd] & \resource{object file} \\ & \resource{assembly\\listing} & \resource{debugging\\information}}
\seeassembly\seemmix\seeobject\seecode\seedebugging
}

\providecommand{\cdorok}{
\toolsection{cdor1k} is a compiler for intermediate code targeting the OpenRISC 1000 hardware architecture.
It generates machine code for OpenRISC 1000 processors from programs written in intermediate code and stores it in corresponding object files.
It also creates a debugging information file as well as an assembly file containing a listing of the generated machine code.
\debuggingtool
\flowgraph{\resource{intermediate\\code} \ar[r] & \toolbox{cdor1k} \ar[r] \ar[d] \ar[rd] & \resource{object file} \\ & \resource{assembly\\listing} & \resource{debugging\\information}}
\seeassembly\seeorok\seeobject\seecode\seedebugging
}

\providecommand{\cdppca}{
\toolsection{cdppc32} is a compiler for intermediate code targeting the PowerPC hardware architecture.
It generates machine code for PowerPC processors from programs written in intermediate code and stores it in corresponding object files.
The compiler generates machine code for the 32-bit operating mode defined by the PowerPC architecture.
It also creates a debugging information file as well as an assembly file containing a listing of the generated machine code.
\debuggingtool
\flowgraph{\resource{intermediate\\code} \ar[r] & \toolbox{cdppc32} \ar[r] \ar[d] \ar[rd] & \resource{object file} \\ & \resource{assembly\\listing} & \resource{debugging\\information}}
\seeassembly\seeppc\seeobject\seecode\seedebugging
}

\providecommand{\cdppcb}{
\toolsection{cdppc64} is a compiler for intermediate code targeting the PowerPC hardware architecture.
It generates machine code for PowerPC processors from programs written in intermediate code and stores it in corresponding object files.
The compiler generates machine code for the 64-bit operating mode defined by the PowerPC architecture.
It also creates a debugging information file as well as an assembly file containing a listing of the generated machine code.
\debuggingtool
\flowgraph{\resource{intermediate\\code} \ar[r] & \toolbox{cdppc64} \ar[r] \ar[d] \ar[rd] & \resource{object file} \\ & \resource{assembly\\listing} & \resource{debugging\\information}}
\seeassembly\seeppc\seeobject\seecode\seedebugging
}

\providecommand{\cdrisc}{
\toolsection{cdrisc} is a compiler for intermediate code targeting the RISC hardware architecture.
It generates machine code for RISC processors from programs written in intermediate code and stores it in corresponding object files.
It also creates a debugging information file as well as an assembly file containing a listing of the generated machine code.
\debuggingtool
\flowgraph{\resource{intermediate\\code} \ar[r] & \toolbox{cdrisc} \ar[r] \ar[d] \ar[rd] & \resource{object file} \\ & \resource{assembly\\listing} & \resource{debugging\\information}}
\seeassembly\seerisc\seeobject\seecode\seedebugging
}

\providecommand{\cdwasm}{
\toolsection{cdwasm} is a compiler for intermediate code targeting the WebAssembly architecture.
It generates machine code for WebAssembly targets from programs written in intermediate code and stores it in corresponding object files.
It also creates a debugging information file as well as an assembly file containing a listing of the generated machine code.
\debuggingtool
\flowgraph{\resource{intermediate\\code} \ar[r] & \toolbox{cdwasm} \ar[r] \ar[d] \ar[rd] & \resource{object file} \\ & \resource{assembly\\listing} & \resource{debugging\\information}}
\seeassembly\seewasm\seeobject\seecode\seedebugging
}

% C++ tools

\providecommand{\cppprep}{
\toolsection{cppprep} is a preprocessor for the \cpp{} programming language.
It preprocesses source code according to the rules of \cpp{} and writes it to the standard output stream.
Only the macro names \texttt{\_\_DATE\_\_}, \texttt{\_\_FILE\_\_}, \texttt{\_\_LINE\_\_}, and \texttt{\_\_TIME\_\_} are predefined.
\flowgraph{\resource{\cpp{} or other\\source code} \ar[r] & \toolbox{cppprep} \ar[r] & \resource{preprocessed\\source code} \\ & \variable{ECSINCLUDE} \ar[u]}
\seecpp
}

\providecommand{\cppprint}{
\toolsection{cppprint} is a pretty printer for the \cpp{} programming language.
It reformats the source code of \cpp{} programs and writes it to the standard output stream.
\flowgraph{\resource{\cpp{}\\source code} \ar[r] & \toolbox{cppprint} \ar[r] & \resource{reformatted\\source code} \\ & \variable{ECSINCLUDE} \ar[u]}
\seecpp
}

\providecommand{\cppcheck}{
\toolsection{cppcheck} is a syntactic and semantic checker for the \cpp{} programming language.
It just performs syntactic and semantic checks on \cpp{} programs and writes its diagnostic messages to the standard error stream.
\flowgraph{\resource{\cpp{}\\source code} \ar[r] & \toolbox{cppcheck} \ar[r] & \resource{diagnostic\\messages} \\ & \variable{ECSINCLUDE} \ar[u]}
\seecpp
}

\providecommand{\cppdump}{
\toolsection{cppdump} is a serializer for the \cpp{} programming language.
It dumps the complete internal representation of programs written in \cpp{} into an XML document.
\debuggingtool
\flowgraph{\resource{\cpp{}\\source code} \ar[r] & \toolbox{cppdump} \ar[r] & \resource{internal\\representation} \\ & \variable{ECSINCLUDE} \ar[u]}
\seecpp
}

\providecommand{\cpprun}{
\toolsection{cpprun} is an interpreter for the \cpp{} programming language.
It processes and executes programs written in \cpp{}.
The macro \texttt{\_\_run\_\_} is predefined in order to enable programmers to identify this tool while interpreting.
\flowgraph{\resource{\cpp{}\\source code} \ar[r] & \toolbox{cpprun} \ar@/u/[r] & \resource{input/\\output} \ar@/d/[l] \\ & \variable{ECSINCLUDE} \ar[u]}
\seecpp
}

\providecommand{\cppdoc}{
\toolsection{cppdoc} is a generic documentation generator for the \cpp{} programming language.
It processes several \cpp{} source files and assembles all information therein into a generic documentation.
\debuggingtool
\flowgraph{\resource{\cpp{}\\source code} \ar[r] & \toolbox{cppdoc} \ar[r] & \resource{generic\\documentation} \\ & \variable{ECSINCLUDE} \ar[u]}
\seecpp\seedocumentation
}

\providecommand{\cpphtml}{
\toolsection{cpphtml} is an HTML documentation generator for the \cpp{} programming language.
It processes several \cpp{} source files and assembles all information therein into an HTML document.
\flowgraph{\resource{\cpp{}\\source code} \ar[r] & \toolbox{cpphtml} \ar[r] & \resource{HTML\\document} \\ & \variable{ECSINCLUDE} \ar[u]}
\seecpp\seedocumentation
}

\providecommand{\cpplatex}{
\toolsection{cpplatex} is a Latex documentation generator for the \cpp{} programming language.
It processes several \cpp{} source files and assembles all information therein into a Latex document.
\flowgraph{\resource{\cpp{}\\source code} \ar[r] & \toolbox{cpplatex} \ar[r] & \resource{Latex\\document} \\ & \variable{ECSINCLUDE} \ar[u]}
\seecpp\seedocumentation
}

\providecommand{\cppcode}{
\toolsection{cppcode} is an intermediate code generator for the \cpp{} programming language.
It generates intermediate code from programs written in \cpp{} and stores it in corresponding assembly files.
The macro \texttt{\_\_code\_\_} is predefined in order to enable programmers to identify this tool while generating intermediate code.
Programs generated with this tool require additional runtime support that is stored in the \file{cpp\-code\-run} library file.
\debuggingtool
\flowgraph{\resource{\cpp{}\\source code} \ar[r] & \toolbox{cppcode} \ar[r] & \resource{intermediate\\code} \\ & \variable{ECSINCLUDE} \ar[u]}
\seecpp\seeassembly\seecode
}

\providecommand{\cppamda}{
\toolsection{cppamd16} is a compiler for the \cpp{} programming language targeting the AMD64 hardware architecture.
It generates machine code for AMD64 processors from programs written in \cpp{} and stores it in corresponding object files.
The compiler generates machine code for the 16-bit operating mode defined by the AMD64 architecture.
For debugging purposes, it also creates a debugging information file as well as an assembly file containing a listing of the generated machine code.
The macro \texttt{\_\_amd16\_\_} is predefined in order to enable programmers to identify this tool and its target architecture while compiling.
Programs generated with this compiler require additional runtime support that is stored in the \file{cpp\-amd16\-run} library file.
\flowgraph{\resource{\cpp{}\\source code} \ar[r] & \toolbox{cppamd16} \ar[r] \ar[d] \ar[rd] & \resource{object file} \\ \variable{ECSINCLUDE} \ar[ru] & \resource{debugging\\information} & \resource{assembly\\listing}}
\seecpp\seeassembly\seeamd\seeobject\seedebugging
}

\providecommand{\cppamdb}{
\toolsection{cppamd32} is a compiler for the \cpp{} programming language targeting the AMD64 hardware architecture.
It generates machine code for AMD64 processors from programs written in \cpp{} and stores it in corresponding object files.
The compiler generates machine code for the 32-bit operating mode defined by the AMD64 architecture.
For debugging purposes, it also creates a debugging information file as well as an assembly file containing a listing of the generated machine code.
The macro \texttt{\_\_amd32\_\_} is predefined in order to enable programmers to identify this tool and its target architecture while compiling.
Programs generated with this compiler require additional runtime support that is stored in the \file{cpp\-amd32\-run} library file.
\flowgraph{\resource{\cpp{}\\source code} \ar[r] & \toolbox{cppamd32} \ar[r] \ar[d] \ar[rd] & \resource{object file} \\ \variable{ECSINCLUDE} \ar[ru] & \resource{debugging\\information} & \resource{assembly\\listing}}
\seecpp\seeassembly\seeamd\seeobject\seedebugging
}

\providecommand{\cppamdc}{
\toolsection{cppamd64} is a compiler for the \cpp{} programming language targeting the AMD64 hardware architecture.
It generates machine code for AMD64 processors from programs written in \cpp{} and stores it in corresponding object files.
The compiler generates machine code for the 64-bit operating mode defined by the AMD64 architecture.
For debugging purposes, it also creates a debugging information file as well as an assembly file containing a listing of the generated machine code.
The macro \texttt{\_\_amd64\_\_} is predefined in order to enable programmers to identify this tool and its target architecture while compiling.
Programs generated with this compiler require additional runtime support that is stored in the \file{cpp\-amd64\-run} library file.
\flowgraph{\resource{\cpp{}\\source code} \ar[r] & \toolbox{cppamd64} \ar[r] \ar[d] \ar[rd] & \resource{object file} \\ \variable{ECSINCLUDE} \ar[ru] & \resource{debugging\\information} & \resource{assembly\\listing}}
\seecpp\seeassembly\seeamd\seeobject\seedebugging
}

\providecommand{\cpparma}{
\toolsection{cpparma32} is a compiler for the \cpp{} programming language targeting the ARM hardware architecture.
It generates machine code for ARM processors executing A32 instructions from programs written in \cpp{} and stores it in corresponding object files.
For debugging purposes, it also creates a debugging information file as well as an assembly file containing a listing of the generated machine code.
The macro \texttt{\_\_arma32\_\_} is predefined in order to enable programmers to identify this tool and its target architecture while compiling.
Programs generated with this compiler require additional runtime support that is stored in the \file{cpp\-arma32\-run} library file.
\flowgraph{\resource{\cpp{}\\source code} \ar[r] & \toolbox{cpparma32} \ar[r] \ar[d] \ar[rd] & \resource{object file} \\ \variable{ECSINCLUDE} \ar[ru] & \resource{debugging\\information} & \resource{assembly\\listing}}
\seecpp\seeassembly\seearm\seeobject\seedebugging
}

\providecommand{\cpparmb}{
\toolsection{cpparma64} is a compiler for the \cpp{} programming language targeting the ARM hardware architecture.
It generates machine code for ARM processors executing A64 instructions from programs written in \cpp{} and stores it in corresponding object files.
For debugging purposes, it also creates a debugging information file as well as an assembly file containing a listing of the generated machine code.
The macro \texttt{\_\_arma64\_\_} is predefined in order to enable programmers to identify this tool and its target architecture while compiling.
Programs generated with this compiler require additional runtime support that is stored in the \file{cpp\-arma64\-run} library file.
\flowgraph{\resource{\cpp{}\\source code} \ar[r] & \toolbox{cpparma64} \ar[r] \ar[d] \ar[rd] & \resource{object file} \\ \variable{ECSINCLUDE} \ar[ru] & \resource{debugging\\information} & \resource{assembly\\listing}}
\seecpp\seeassembly\seearm\seeobject\seedebugging
}

\providecommand{\cpparmc}{
\toolsection{cpparmt32} is a compiler for the \cpp{} programming language targeting the ARM hardware architecture.
It generates machine code for ARM processors without floating-point extension executing T32 instructions from programs written in \cpp{} and stores it in corresponding object files.
For debugging purposes, it also creates a debugging information file as well as an assembly file containing a listing of the generated machine code.
The macro \texttt{\_\_armt32\_\_} is predefined in order to enable programmers to identify this tool and its target architecture while compiling.
Programs generated with this compiler require additional runtime support that is stored in the \file{cpp\-armt32\-run} library file.
\flowgraph{\resource{\cpp{}\\source code} \ar[r] & \toolbox{cpparmt32} \ar[r] \ar[d] \ar[rd] & \resource{object file} \\ \variable{ECSINCLUDE} \ar[ru] & \resource{debugging\\information} & \resource{assembly\\listing}}
\seecpp\seeassembly\seearm\seeobject\seedebugging
}

\providecommand{\cpparmcfpe}{
\toolsection{cpparmt32fpe} is a compiler for the \cpp{} programming language targeting the ARM hardware architecture.
It generates machine code for ARM processors with floating-point extension executing T32 instructions from programs written in \cpp{} and stores it in corresponding object files.
For debugging purposes, it also creates a debugging information file as well as an assembly file containing a listing of the generated machine code.
The macro \texttt{\_\_armt32fpe\_\_} is predefined in order to enable programmers to identify this tool and its target architecture while compiling.
Programs generated with this compiler require additional runtime support that is stored in the \file{cpp\-armt32\-fpe\-run} library file.
\flowgraph{\resource{\cpp{}\\source code} \ar[r] & \toolbox{cpparmt32fpe} \ar[r] \ar[d] \ar[rd] & \resource{object file} \\ \variable{ECSINCLUDE} \ar[ru] & \resource{debugging\\information} & \resource{assembly\\listing}}
\seecpp\seeassembly\seearm\seeobject\seedebugging
}

\providecommand{\cppavr}{
\toolsection{cppavr} is a compiler for the \cpp{} programming language targeting the AVR hardware architecture.
It generates machine code for AVR processors from programs written in \cpp{} and stores it in corresponding object files.
For debugging purposes, it also creates a debugging information file as well as an assembly file containing a listing of the generated machine code.
The macro \texttt{\_\_avr\_\_} is predefined in order to enable programmers to identify this tool and its target architecture while compiling.
Programs generated with this compiler require additional runtime support that is stored in the \file{cpp\-avr\-run} library file.
\flowgraph{\resource{\cpp{}\\source code} \ar[r] & \toolbox{cppavr} \ar[r] \ar[d] \ar[rd] & \resource{object file} \\ \variable{ECSINCLUDE} \ar[ru] & \resource{debugging\\information} & \resource{assembly\\listing}}
\seecpp\seeassembly\seeavr\seeobject\seedebugging
}

\providecommand{\cppavrtt}{
\toolsection{cppavr32} is a compiler for the \cpp{} programming language targeting the AVR32 hardware architecture.
It generates machine code for AVR32 processors from programs written in \cpp{} and stores it in corresponding object files.
For debugging purposes, it also creates a debugging information file as well as an assembly file containing a listing of the generated machine code.
The macro \texttt{\_\_avr32\_\_} is predefined in order to enable programmers to identify this tool and its target architecture while compiling.
Programs generated with this compiler require additional runtime support that is stored in the \file{cpp\-avr32\-run} library file.
\flowgraph{\resource{\cpp{}\\source code} \ar[r] & \toolbox{cppavr32} \ar[r] \ar[d] \ar[rd] & \resource{object file} \\ \variable{ECSINCLUDE} \ar[ru] & \resource{debugging\\information} & \resource{assembly\\listing}}
\seecpp\seeassembly\seeavrtt\seeobject\seedebugging
}

\providecommand{\cppmabk}{
\toolsection{cppm68k} is a compiler for the \cpp{} programming language targeting the M68000 hardware architecture.
It generates machine code for M68000 processors from programs written in \cpp{} and stores it in corresponding object files.
For debugging purposes, it also creates a debugging information file as well as an assembly file containing a listing of the generated machine code.
The macro \texttt{\_\_m68k\_\_} is predefined in order to enable programmers to identify this tool and its target architecture while compiling.
Programs generated with this compiler require additional runtime support that is stored in the \file{cpp\-m68k\-run} library file.
\flowgraph{\resource{\cpp{}\\source code} \ar[r] & \toolbox{cppm68k} \ar[r] \ar[d] \ar[rd] & \resource{object file} \\ \variable{ECSINCLUDE} \ar[ru] & \resource{debugging\\information} & \resource{assembly\\listing}}
\seecpp\seeassembly\seemabk\seeobject\seedebugging
}

\providecommand{\cppmibl}{
\toolsection{cppmibl} is a compiler for the \cpp{} programming language targeting the MicroBlaze hardware architecture.
It generates machine code for MicroBlaze processors from programs written in \cpp{} and stores it in corresponding object files.
For debugging purposes, it also creates a debugging information file as well as an assembly file containing a listing of the generated machine code.
The macro \texttt{\_\_mibl\_\_} is predefined in order to enable programmers to identify this tool and its target architecture while compiling.
Programs generated with this compiler require additional runtime support that is stored in the \file{cpp\-mibl\-run} library file.
\flowgraph{\resource{\cpp{}\\source code} \ar[r] & \toolbox{cppmibl} \ar[r] \ar[d] \ar[rd] & \resource{object file} \\ \variable{ECSINCLUDE} \ar[ru] & \resource{debugging\\information} & \resource{assembly\\listing}}
\seecpp\seeassembly\seemibl\seeobject\seedebugging
}

\providecommand{\cppmipsa}{
\toolsection{cppmips32} is a compiler for the \cpp{} programming language targeting the MIPS32 hardware architecture.
It generates machine code for MIPS32 processors from programs written in \cpp{} and stores it in corresponding object files.
For debugging purposes, it also creates a debugging information file as well as an assembly file containing a listing of the generated machine code.
The macro \texttt{\_\_mips32\_\_} is predefined in order to enable programmers to identify this tool and its target architecture while compiling.
Programs generated with this compiler require additional runtime support that is stored in the \file{cpp\-mips32\-run} library file.
\flowgraph{\resource{\cpp{}\\source code} \ar[r] & \toolbox{cppmips32} \ar[r] \ar[d] \ar[rd] & \resource{object file} \\ \variable{ECSINCLUDE} \ar[ru] & \resource{debugging\\information} & \resource{assembly\\listing}}
\seecpp\seeassembly\seemips\seeobject\seedebugging
}

\providecommand{\cppmipsb}{
\toolsection{cppmips64} is a compiler for the \cpp{} programming language targeting the MIPS64 hardware architecture.
It generates machine code for MIPS64 processors from programs written in \cpp{} and stores it in corresponding object files.
For debugging purposes, it also creates a debugging information file as well as an assembly file containing a listing of the generated machine code.
The macro \texttt{\_\_mips64\_\_} is predefined in order to enable programmers to identify this tool and its target architecture while compiling.
Programs generated with this compiler require additional runtime support that is stored in the \file{cpp\-mips64\-run} library file.
\flowgraph{\resource{\cpp{}\\source code} \ar[r] & \toolbox{cppmips64} \ar[r] \ar[d] \ar[rd] & \resource{object file} \\ \variable{ECSINCLUDE} \ar[ru] & \resource{debugging\\information} & \resource{assembly\\listing}}
\seecpp\seeassembly\seemips\seeobject\seedebugging
}

\providecommand{\cppmmix}{
\toolsection{cppmmix} is a compiler for the \cpp{} programming language targeting the MMIX hardware architecture.
It generates machine code for MMIX processors from programs written in \cpp{} and stores it in corresponding object files.
For debugging purposes, it also creates a debugging information file as well as an assembly file containing a listing of the generated machine code.
The macro \texttt{\_\_mmix\_\_} is predefined in order to enable programmers to identify this tool and its target architecture while compiling.
Programs generated with this compiler require additional runtime support that is stored in the \file{cpp\-mmix\-run} library file.
\flowgraph{\resource{\cpp{}\\source code} \ar[r] & \toolbox{cppmmix} \ar[r] \ar[d] \ar[rd] & \resource{object file} \\ \variable{ECSINCLUDE} \ar[ru] & \resource{debugging\\information} & \resource{assembly\\listing}}
\seecpp\seeassembly\seemmix\seeobject\seedebugging
}

\providecommand{\cpporok}{
\toolsection{cppor1k} is a compiler for the \cpp{} programming language targeting the OpenRISC 1000 hardware architecture.
It generates machine code for OpenRISC 1000 processors from programs written in \cpp{} and stores it in corresponding object files.
For debugging purposes, it also creates a debugging information file as well as an assembly file containing a listing of the generated machine code.
The macro \texttt{\_\_or1k\_\_} is predefined in order to enable programmers to identify this tool and its target architecture while compiling.
Programs generated with this compiler require additional runtime support that is stored in the \file{cpp\-or1k\-run} library file.
\flowgraph{\resource{\cpp{}\\source code} \ar[r] & \toolbox{cppor1k} \ar[r] \ar[d] \ar[rd] & \resource{object file} \\ \variable{ECSINCLUDE} \ar[ru] & \resource{debugging\\information} & \resource{assembly\\listing}}
\seecpp\seeassembly\seeorok\seeobject\seedebugging
}

\providecommand{\cppppca}{
\toolsection{cppppc32} is a compiler for the \cpp{} programming language targeting the PowerPC hardware architecture.
It generates machine code for PowerPC processors from programs written in \cpp{} and stores it in corresponding object files.
The compiler generates machine code for the 32-bit operating mode defined by the PowerPC architecture.
For debugging purposes, it also creates a debugging information file as well as an assembly file containing a listing of the generated machine code.
The macro \texttt{\_\_ppc32\_\_} is predefined in order to enable programmers to identify this tool and its target architecture while compiling.
Programs generated with this compiler require additional runtime support that is stored in the \file{cpp\-ppc32\-run} library file.
\flowgraph{\resource{\cpp{}\\source code} \ar[r] & \toolbox{cppppc32} \ar[r] \ar[d] \ar[rd] & \resource{object file} \\ \variable{ECSINCLUDE} \ar[ru] & \resource{debugging\\information} & \resource{assembly\\listing}}
\seecpp\seeassembly\seeppc\seeobject\seedebugging
}

\providecommand{\cppppcb}{
\toolsection{cppppc64} is a compiler for the \cpp{} programming language targeting the PowerPC hardware architecture.
It generates machine code for PowerPC processors from programs written in \cpp{} and stores it in corresponding object files.
The compiler generates machine code for the 64-bit operating mode defined by the PowerPC architecture.
For debugging purposes, it also creates a debugging information file as well as an assembly file containing a listing of the generated machine code.
The macro \texttt{\_\_ppc64\_\_} is predefined in order to enable programmers to identify this tool and its target architecture while compiling.
Programs generated with this compiler require additional runtime support that is stored in the \file{cpp\-ppc64\-run} library file.
\flowgraph{\resource{\cpp{}\\source code} \ar[r] & \toolbox{cppppc64} \ar[r] \ar[d] \ar[rd] & \resource{object file} \\ \variable{ECSINCLUDE} \ar[ru] & \resource{debugging\\information} & \resource{assembly\\listing}}
\seecpp\seeassembly\seeppc\seeobject\seedebugging
}

\providecommand{\cpprisc}{
\toolsection{cpprisc} is a compiler for the \cpp{} programming language targeting the RISC hardware architecture.
It generates machine code for RISC processors from programs written in \cpp{} and stores it in corresponding object files.
For debugging purposes, it also creates a debugging information file as well as an assembly file containing a listing of the generated machine code.
The macro \texttt{\_\_risc\_\_} is predefined in order to enable programmers to identify this tool and its target architecture while compiling.
Programs generated with this compiler require additional runtime support that is stored in the \file{cpp\-risc\-run} library file.
\flowgraph{\resource{\cpp{}\\source code} \ar[r] & \toolbox{cpprisc} \ar[r] \ar[d] \ar[rd] & \resource{object file} \\ \variable{ECSINCLUDE} \ar[ru] & \resource{debugging\\information} & \resource{assembly\\listing}}
\seecpp\seeassembly\seerisc\seeobject\seedebugging
}

\providecommand{\cppwasm}{
\toolsection{cppwasm} is a compiler for the \cpp{} programming language targeting the WebAssembly architecture.
It generates machine code for WebAssembly targets from programs written in \cpp{} and stores it in corresponding object files.
For debugging purposes, it also creates a debugging information file as well as an assembly file containing a listing of the generated machine code.
The macro \texttt{\_\_wasm\_\_} is predefined in order to enable programmers to identify this tool and its target architecture while compiling.
Programs generated with this compiler require additional runtime support that is stored in the \file{cpp\-wasm\-run} library file.
\flowgraph{\resource{\cpp{}\\source code} \ar[r] & \toolbox{cppwasm} \ar[r] \ar[d] \ar[rd] & \resource{object file} \\ \variable{ECSINCLUDE} \ar[ru] & \resource{debugging\\information} & \resource{assembly\\listing}}
\seecpp\seeassembly\seewasm\seeobject\seedebugging
}

% FALSE tools

\providecommand{\falprint}{
\toolsection{falprint} is a pretty printer for the FALSE programming language.
It reformats the source code of FALSE programs and writes it to the standard output stream.
\flowgraph{\resource{FALSE\\source code} \ar[r] & \toolbox{falprint} \ar[r] & \resource{reformatted\\source code}}
\seefalse
}

\providecommand{\falcheck}{
\toolsection{falcheck} is a syntactic and semantic checker for the FALSE programming language.
It just performs syntactic and semantic checks on FALSE programs and writes its diagnostic messages to the standard error stream.
\flowgraph{\resource{FALSE\\source code} \ar[r] & \toolbox{falcheck} \ar[r] & \resource{diagnostic\\messages}}
\seefalse
}

\providecommand{\faldump}{
\toolsection{faldump} is a serializer for the FALSE programming language.
It dumps the complete internal representation of programs written in FALSE into an XML document.
\debuggingtool
\flowgraph{\resource{FALSE\\source code} \ar[r] & \toolbox{faldump} \ar[r] & \resource{internal\\representation}}
\seefalse
}

\providecommand{\falrun}{
\toolsection{falrun} is an interpreter for the FALSE programming language.
It processes and executes programs written in FALSE\@.
\flowgraph{\resource{FALSE\\source code} \ar[r] & \toolbox{falrun} \ar@/u/[r] & \resource{input/\\output} \ar@/d/[l]}
\seefalse
}

\providecommand{\falcpp}{
\toolsection{falcpp} is a transpiler for the FALSE programming language.
It translates programs written in FALSE into \cpp{} programs and stores them in corresponding source files.
\flowgraph{\resource{FALSE\\source code} \ar[r] & \toolbox{falcpp} \ar[r] & \resource{\cpp{}\\source file}}
\seefalse\seecpp
}

\providecommand{\falcode}{
\toolsection{falcode} is an intermediate code generator for the FALSE programming language.
It generates intermediate code from programs written in FALSE and stores it in corresponding assembly files.
\debuggingtool
\flowgraph{\resource{FALSE\\source code} \ar[r] & \toolbox{falcode} \ar[r] & \resource{intermediate\\code}}
\seefalse\seeassembly\seecode
}

\providecommand{\falamda}{
\toolsection{falamd16} is a compiler for the FALSE programming language targeting the AMD64 hardware architecture.
It generates machine code for AMD64 processors from programs written in FALSE and stores it in corresponding object files.
The compiler generates machine code for the 16-bit operating mode defined by the AMD64 architecture.
\flowgraph{\resource{FALSE\\source code} \ar[r] & \toolbox{falamd16} \ar[r] & \resource{object file}}
\seefalse\seeamd\seeobject
}

\providecommand{\falamdb}{
\toolsection{falamd32} is a compiler for the FALSE programming language targeting the AMD64 hardware architecture.
It generates machine code for AMD64 processors from programs written in FALSE and stores it in corresponding object files.
The compiler generates machine code for the 32-bit operating mode defined by the AMD64 architecture.
\flowgraph{\resource{FALSE\\source code} \ar[r] & \toolbox{falamd32} \ar[r] & \resource{object file}}
\seefalse\seeamd\seeobject
}

\providecommand{\falamdc}{
\toolsection{falamd64} is a compiler for the FALSE programming language targeting the AMD64 hardware architecture.
It generates machine code for AMD64 processors from programs written in FALSE and stores it in corresponding object files.
The compiler generates machine code for the 64-bit operating mode defined by the AMD64 architecture.
\flowgraph{\resource{FALSE\\source code} \ar[r] & \toolbox{falamd64} \ar[r] & \resource{object file}}
\seefalse\seeamd\seeobject
}

\providecommand{\falarma}{
\toolsection{falarma32} is a compiler for the FALSE programming language targeting the ARM hardware architecture.
It generates machine code for ARM processors executing A32 instructions from programs written in FALSE and stores it in corresponding object files.
\flowgraph{\resource{FALSE\\source code} \ar[r] & \toolbox{falarma32} \ar[r] & \resource{object file}}
\seefalse\seearm\seeobject
}

\providecommand{\falarmb}{
\toolsection{falarma64} is a compiler for the FALSE programming language targeting the ARM hardware architecture.
It generates machine code for ARM processors executing A64 instructions from programs written in FALSE and stores it in corresponding object files.
\flowgraph{\resource{FALSE\\source code} \ar[r] & \toolbox{falarma64} \ar[r] & \resource{object file}}
\seefalse\seearm\seeobject
}

\providecommand{\falarmc}{
\toolsection{falarmt32} is a compiler for the FALSE programming language targeting the ARM hardware architecture.
It generates machine code for ARM processors without floating-point extension executing T32 instructions from programs written in FALSE and stores it in corresponding object files.
\flowgraph{\resource{FALSE\\source code} \ar[r] & \toolbox{falarmt32} \ar[r] & \resource{object file}}
\seefalse\seearm\seeobject
}

\providecommand{\falarmcfpe}{
\toolsection{falarmt32fpe} is a compiler for the FALSE programming language targeting the ARM hardware architecture.
It generates machine code for ARM processors with floating-point extension executing T32 instructions from programs written in FALSE and stores it in corresponding object files.
\flowgraph{\resource{FALSE\\source code} \ar[r] & \toolbox{falarmt32fpe} \ar[r] & \resource{object file}}
\seefalse\seearm\seeobject
}

\providecommand{\falavr}{
\toolsection{falavr} is a compiler for the FALSE programming language targeting the AVR hardware architecture.
It generates machine code for AVR processors from programs written in FALSE and stores it in corresponding object files.
\flowgraph{\resource{FALSE\\source code} \ar[r] & \toolbox{falavr} \ar[r] & \resource{object file}}
\seefalse\seeavr\seeobject
}

\providecommand{\falavrtt}{
\toolsection{falavr32} is a compiler for the FALSE programming language targeting the AVR32 hardware architecture.
It generates machine code for AVR32 processors from programs written in FALSE and stores it in corresponding object files.
\flowgraph{\resource{FALSE\\source code} \ar[r] & \toolbox{falavr32} \ar[r] & \resource{object file}}
\seefalse\seeavrtt\seeobject
}

\providecommand{\falmabk}{
\toolsection{falm68k} is a compiler for the FALSE programming language targeting the M68000 hardware architecture.
It generates machine code for M68000 processors from programs written in FALSE and stores it in corresponding object files.
\flowgraph{\resource{FALSE\\source code} \ar[r] & \toolbox{falm68k} \ar[r] & \resource{object file}}
\seefalse\seemabk\seeobject
}

\providecommand{\falmibl}{
\toolsection{falmibl} is a compiler for the FALSE programming language targeting the MicroBlaze hardware architecture.
It generates machine code for MicroBlaze processors from programs written in FALSE and stores it in corresponding object files.
\flowgraph{\resource{FALSE\\source code} \ar[r] & \toolbox{falmibl} \ar[r] & \resource{object file}}
\seefalse\seemibl\seeobject
}

\providecommand{\falmipsa}{
\toolsection{falmips32} is a compiler for the FALSE programming language targeting the MIPS32 hardware architecture.
It generates machine code for MIPS32 processors from programs written in FALSE and stores it in corresponding object files.
\flowgraph{\resource{FALSE\\source code} \ar[r] & \toolbox{falmips32} \ar[r] & \resource{object file}}
\seefalse\seemips\seeobject
}

\providecommand{\falmipsb}{
\toolsection{falmips64} is a compiler for the FALSE programming language targeting the MIPS64 hardware architecture.
It generates machine code for MIPS64 processors from programs written in FALSE and stores it in corresponding object files.
\flowgraph{\resource{FALSE\\source code} \ar[r] & \toolbox{falmips64} \ar[r] & \resource{object file}}
\seefalse\seemips\seeobject
}

\providecommand{\falmmix}{
\toolsection{falmmix} is a compiler for the FALSE programming language targeting the MMIX hardware architecture.
It generates machine code for MMIX processors from programs written in FALSE and stores it in corresponding object files.
\flowgraph{\resource{FALSE\\source code} \ar[r] & \toolbox{falmmix} \ar[r] & \resource{object file}}
\seefalse\seemmix\seeobject
}

\providecommand{\falorok}{
\toolsection{falor1k} is a compiler for the FALSE programming language targeting the OpenRISC 1000 hardware architecture.
It generates machine code for OpenRISC 1000 processors from programs written in FALSE and stores it in corresponding object files.
\flowgraph{\resource{FALSE\\source code} \ar[r] & \toolbox{falor1k} \ar[r] & \resource{object file}}
\seefalse\seeorok\seeobject
}

\providecommand{\falppca}{
\toolsection{falppc32} is a compiler for the FALSE programming language targeting the PowerPC hardware architecture.
It generates machine code for PowerPC processors from programs written in FALSE and stores it in corresponding object files.
The compiler generates machine code for the 32-bit operating mode defined by the PowerPC architecture.
\flowgraph{\resource{FALSE\\source code} \ar[r] & \toolbox{falppc32} \ar[r] & \resource{object file}}
\seefalse\seeppc\seeobject
}

\providecommand{\falppcb}{
\toolsection{falppc64} is a compiler for the FALSE programming language targeting the PowerPC hardware architecture.
It generates machine code for PowerPC processors from programs written in FALSE and stores it in corresponding object files.
The compiler generates machine code for the 64-bit operating mode defined by the PowerPC architecture.
\flowgraph{\resource{FALSE\\source code} \ar[r] & \toolbox{falppc64} \ar[r] & \resource{object file}}
\seefalse\seeppc\seeobject
}

\providecommand{\falrisc}{
\toolsection{falrisc} is a compiler for the FALSE programming language targeting the RISC hardware architecture.
It generates machine code for RISC processors from programs written in FALSE and stores it in corresponding object files.
\flowgraph{\resource{FALSE\\source code} \ar[r] & \toolbox{falrisc} \ar[r] & \resource{object file}}
\seefalse\seerisc\seeobject
}

\providecommand{\falwasm}{
\toolsection{falwasm} is a compiler for the FALSE programming language targeting the WebAssembly architecture.
It generates machine code for WebAssembly targets from programs written in FALSE and stores it in corresponding object files.
\flowgraph{\resource{FALSE\\source code} \ar[r] & \toolbox{falwasm} \ar[r] & \resource{object file}}
\seefalse\seewasm\seeobject
}

% Oberon tools

\providecommand{\obprint}{
\toolsection{obprint} is a pretty printer for the Oberon programming language.
It reformats the source code of Oberon modules and writes it to the standard output stream.
\flowgraph{\resource{Oberon\\source code} \ar[r] & \toolbox{obprint} \ar[r] & \resource{reformatted\\source code}}
\seeoberon
}

\providecommand{\obcheck}{
\toolsection{obcheck} is a syntactic and semantic checker for the Oberon programming language.
It just performs syntactic and semantic checks on Oberon modules and writes its diagnostic messages to the standard error stream.
In addition, it stores the interface of each module in a symbol file which is required when other modules import the module.
\flowgraph{\resource{Oberon\\source code} \ar[r] & \toolbox{obcheck} \ar[r] \ar@/l/[d] & \resource{diagnostic\\messages} \\ \variable{ECSIMPORT} \ar[ru] & \resource{symbol\\files} \ar@/r/[u]}
\seeoberon
}

\providecommand{\obdump}{
\toolsection{obdump} is a serializer for the Oberon programming language.
It dumps the complete internal representation of modules written in Oberon into an XML document.
\debuggingtool
\flowgraph{\resource{Oberon\\source code} \ar[r] & \toolbox{obdump} \ar[r] \ar@/l/[d] & \resource{internal\\representation} \\ \variable{ECSIMPORT} \ar[ru] & \resource{symbol\\files} \ar@/r/[u]}
\seeoberon
}

\providecommand{\obrun}{
\toolsection{obrun} is an interpreter for the Oberon programming language.
It processes and executes modules written in Oberon.
This tool does neither generate nor process symbol files while interpreting modules.
If a module is imported by another one, its filename has to be named before the other one in the list of command-line arguments.
\flowgraph{\resource{Oberon\\source code} \ar[r] & \toolbox{obrun} \ar@/u/[r] & \resource{input/\\output} \ar@/d/[l]}
\seeoberon
}

\providecommand{\obcpp}{
\toolsection{obcpp} is a transpiler for the Oberon programming language.
It translates programs written in Oberon into \cpp{} programs and stores them in corresponding source and header files.
In addition, it stores the interface of each module in a symbol file which is required when other modules import the module.
The same interface is provided by the generated header file which can be used in other parts of the \cpp{} program.
\flowgraph{\resource{Oberon\\source code} \ar[r] & \toolbox{obcpp} \ar[r] \ar@/l/[d] \ar[rd] & \resource{\cpp{}\\source file} \\ \variable{ECSIMPORT} \ar[ru] & \resource{symbol\\files} \ar@/r/[u] & \resource{\cpp{}\\header file}}
\seeoberon\seecpp
}

\providecommand{\obdoc}{
\toolsection{obdoc} is a generic documentation generator for the Oberon programming language.
It processes several Oberon modules and assembles all information therein into a generic documentation.
In addition, it stores the interface of each module in a symbol file which is required when other modules import the module.
\debuggingtool
\flowgraph{\resource{Oberon\\source code} \ar[r] & \toolbox{obdoc} \ar[r] \ar@/l/[d] & \resource{generic\\documentation} \\ \variable{ECSIMPORT} \ar[ru] & \resource{symbol\\files} \ar@/r/[u]}
\seeoberon\seedocumentation
}

\providecommand{\obhtml}{
\toolsection{obhtml} is an HTML documentation generator for the Oberon programming language.
It processes several Oberon modules and assembles all information therein into an HTML document.
In addition, it stores the interface of each module in a symbol file which is required when other modules import the module.
\flowgraph{\resource{Oberon\\source code} \ar[r] & \toolbox{obhtml} \ar[r] \ar@/l/[d] & \resource{HTML\\document} \\ \variable{ECSIMPORT} \ar[ru] & \resource{symbol\\files} \ar@/r/[u]}
\seeoberon\seedocumentation
}

\providecommand{\oblatex}{
\toolsection{oblatex} is a Latex documentation generator for the Oberon programming language.
It processes several Oberon modules and assembles all information therein into a Latex document.
In addition, it stores the interface of each module in a symbol file which is required when other modules import the module.
\flowgraph{\resource{Oberon\\source code} \ar[r] & \toolbox{oblatex} \ar[r] \ar@/l/[d] & \resource{Latex\\document} \\ \variable{ECSIMPORT} \ar[ru] & \resource{symbol\\files} \ar@/r/[u]}
\seeoberon\seedocumentation
}

\providecommand{\obcode}{
\toolsection{obcode} is an intermediate code generator for the Oberon programming language.
It generates intermediate code from modules written in Oberon and stores it in corresponding assembly files.
In addition, it stores the interface of each module in a symbol file which is required when other modules import the module.
Programs generated with this tool require additional runtime support that is stored in the \file{ob\-code\-run} library file.
\debuggingtool
\flowgraph{\resource{Oberon\\source code} \ar[r] & \toolbox{obcode} \ar[r] \ar@/l/[d] & \resource{intermediate\\code} \\ \variable{ECSIMPORT} \ar[ru] & \resource{symbol\\files} \ar@/r/[u]}
\seeoberon\seeassembly\seecode
}

\providecommand{\obamda}{
\toolsection{obamd16} is a compiler for the Oberon programming language targeting the AMD64 hardware architecture.
It generates machine code for AMD64 processors from modules written in Oberon and stores it in corresponding object files.
The compiler generates machine code for the 16-bit operating mode defined by the AMD64 architecture.
For debugging purposes, it also creates a debugging information file as well as an assembly file containing a listing of the generated machine code.
In addition, it stores the interface of each module in a symbol file which is required when other modules import the module.
Programs generated with this compiler require additional runtime support that is stored in the \file{ob\-amd16\-run} library file.
\flowgraph{\resource{Oberon\\source code} \ar[r] & \toolbox{obamd16} \ar[r] \ar@/l/[d] \ar[rd] & \resource{object file} \\ \variable{ECSIMPORT} \ar[ru] & \resource{symbol\\files} \ar@/r/[u] & \resource{debugging\\information}}
\seeoberon\seeassembly\seeamd\seeobject\seedebugging
}

\providecommand{\obamdb}{
\toolsection{obamd32} is a compiler for the Oberon programming language targeting the AMD64 hardware architecture.
It generates machine code for AMD64 processors from modules written in Oberon and stores it in corresponding object files.
The compiler generates machine code for the 32-bit operating mode defined by the AMD64 architecture.
For debugging purposes, it also creates a debugging information file as well as an assembly file containing a listing of the generated machine code.
In addition, it stores the interface of each module in a symbol file which is required when other modules import the module.
Programs generated with this compiler require additional runtime support that is stored in the \file{ob\-amd32\-run} library file.
\flowgraph{\resource{Oberon\\source code} \ar[r] & \toolbox{obamd32} \ar[r] \ar@/l/[d] \ar[rd] & \resource{object file} \\ \variable{ECSIMPORT} \ar[ru] & \resource{symbol\\files} \ar@/r/[u] & \resource{debugging\\information}}
\seeoberon\seeassembly\seeamd\seeobject\seedebugging
}

\providecommand{\obamdc}{
\toolsection{obamd64} is a compiler for the Oberon programming language targeting the AMD64 hardware architecture.
It generates machine code for AMD64 processors from modules written in Oberon and stores it in corresponding object files.
The compiler generates machine code for the 64-bit operating mode defined by the AMD64 architecture.
For debugging purposes, it also creates a debugging information file as well as an assembly file containing a listing of the generated machine code.
In addition, it stores the interface of each module in a symbol file which is required when other modules import the module.
Programs generated with this compiler require additional runtime support that is stored in the \file{ob\-amd64\-run} library file.
\flowgraph{\resource{Oberon\\source code} \ar[r] & \toolbox{obamd64} \ar[r] \ar@/l/[d] \ar[rd] & \resource{object file} \\ \variable{ECSIMPORT} \ar[ru] & \resource{symbol\\files} \ar@/r/[u] & \resource{debugging\\information}}
\seeoberon\seeassembly\seeamd\seeobject\seedebugging
}

\providecommand{\obarma}{
\toolsection{obarma32} is a compiler for the Oberon programming language targeting the ARM hardware architecture.
It generates machine code for ARM processors executing A32 instructions from modules written in Oberon and stores it in corresponding object files.
For debugging purposes, it also creates a debugging information file as well as an assembly file containing a listing of the generated machine code.
In addition, it stores the interface of each module in a symbol file which is required when other modules import the module.
Programs generated with this compiler require additional runtime support that is stored in the \file{ob\-arma32\-run} library file.
\flowgraph{\resource{Oberon\\source code} \ar[r] & \toolbox{obarma32} \ar[r] \ar@/l/[d] \ar[rd] & \resource{object file} \\ \variable{ECSIMPORT} \ar[ru] & \resource{symbol\\files} \ar@/r/[u] & \resource{debugging\\information}}
\seeoberon\seeassembly\seearm\seeobject\seedebugging
}

\providecommand{\obarmb}{
\toolsection{obarma64} is a compiler for the Oberon programming language targeting the ARM hardware architecture.
It generates machine code for ARM processors executing A64 instructions from modules written in Oberon and stores it in corresponding object files.
For debugging purposes, it also creates a debugging information file as well as an assembly file containing a listing of the generated machine code.
In addition, it stores the interface of each module in a symbol file which is required when other modules import the module.
Programs generated with this compiler require additional runtime support that is stored in the \file{ob\-arma64\-run} library file.
\flowgraph{\resource{Oberon\\source code} \ar[r] & \toolbox{obarma64} \ar[r] \ar@/l/[d] \ar[rd] & \resource{object file} \\ \variable{ECSIMPORT} \ar[ru] & \resource{symbol\\files} \ar@/r/[u] & \resource{debugging\\information}}
\seeoberon\seeassembly\seearm\seeobject\seedebugging
}

\providecommand{\obarmc}{
\toolsection{obarmt32} is a compiler for the Oberon programming language targeting the ARM hardware architecture.
It generates machine code for ARM processors without floating-point extension executing T32 instructions from modules written in Oberon and stores it in corresponding object files.
For debugging purposes, it also creates a debugging information file as well as an assembly file containing a listing of the generated machine code.
In addition, it stores the interface of each module in a symbol file which is required when other modules import the module.
Programs generated with this compiler require additional runtime support that is stored in the \file{ob\-armt32\-run} library file.
\flowgraph{\resource{Oberon\\source code} \ar[r] & \toolbox{obarmt32} \ar[r] \ar@/l/[d] \ar[rd] & \resource{object file} \\ \variable{ECSIMPORT} \ar[ru] & \resource{symbol\\files} \ar@/r/[u] & \resource{debugging\\information}}
\seeoberon\seeassembly\seearm\seeobject\seedebugging
}

\providecommand{\obarmcfpe}{
\toolsection{obarmt32fpe} is a compiler for the Oberon programming language targeting the ARM hardware architecture.
It generates machine code for ARM processors with floating-point extension executing T32 instructions from modules written in Oberon and stores it in corresponding object files.
For debugging purposes, it also creates a debugging information file as well as an assembly file containing a listing of the generated machine code.
In addition, it stores the interface of each module in a symbol file which is required when other modules import the module.
Programs generated with this compiler require additional runtime support that is stored in the \file{ob\-armt32\-fpe\-run} library file.
\flowgraph{\resource{Oberon\\source code} \ar[r] & \toolbox{obarmt32fpe} \ar[r] \ar@/l/[d] \ar[rd] & \resource{object file} \\ \variable{ECSIMPORT} \ar[ru] & \resource{symbol\\files} \ar@/r/[u] & \resource{debugging\\information}}
\seeoberon\seeassembly\seearm\seeobject\seedebugging
}

\providecommand{\obavr}{
\toolsection{obavr} is a compiler for the Oberon programming language targeting the AVR hardware architecture.
It generates machine code for AVR processors from modules written in Oberon and stores it in corresponding object files.
For debugging purposes, it also creates a debugging information file as well as an assembly file containing a listing of the generated machine code.
In addition, it stores the interface of each module in a symbol file which is required when other modules import the module.
Programs generated with this compiler require additional runtime support that is stored in the \file{ob\-avr\-run} library file.
\flowgraph{\resource{Oberon\\source code} \ar[r] & \toolbox{obavr} \ar[r] \ar@/l/[d] \ar[rd] & \resource{object file} \\ \variable{ECSIMPORT} \ar[ru] & \resource{symbol\\files} \ar@/r/[u] & \resource{debugging\\information}}
\seeoberon\seeassembly\seeavr\seeobject\seedebugging
}

\providecommand{\obavrtt}{
\toolsection{obavr32} is a compiler for the Oberon programming language targeting the AVR32 hardware architecture.
It generates machine code for AVR32 processors from modules written in Oberon and stores it in corresponding object files.
For debugging purposes, it also creates a debugging information file as well as an assembly file containing a listing of the generated machine code.
In addition, it stores the interface of each module in a symbol file which is required when other modules import the module.
Programs generated with this compiler require additional runtime support that is stored in the \file{ob\-avr32\-run} library file.
\flowgraph{\resource{Oberon\\source code} \ar[r] & \toolbox{obavr32} \ar[r] \ar@/l/[d] \ar[rd] & \resource{object file} \\ \variable{ECSIMPORT} \ar[ru] & \resource{symbol\\files} \ar@/r/[u] & \resource{debugging\\information}}
\seeoberon\seeassembly\seeavrtt\seeobject\seedebugging
}

\providecommand{\obmabk}{
\toolsection{obm68k} is a compiler for the Oberon programming language targeting the M68000 hardware architecture.
It generates machine code for M68000 processors from modules written in Oberon and stores it in corresponding object files.
For debugging purposes, it also creates a debugging information file as well as an assembly file containing a listing of the generated machine code.
In addition, it stores the interface of each module in a symbol file which is required when other modules import the module.
Programs generated with this compiler require additional runtime support that is stored in the \file{ob\-m68k\-run} library file.
\flowgraph{\resource{Oberon\\source code} \ar[r] & \toolbox{obm68k} \ar[r] \ar@/l/[d] \ar[rd] & \resource{object file} \\ \variable{ECSIMPORT} \ar[ru] & \resource{symbol\\files} \ar@/r/[u] & \resource{debugging\\information}}
\seeoberon\seeassembly\seemabk\seeobject\seedebugging
}

\providecommand{\obmibl}{
\toolsection{obmibl} is a compiler for the Oberon programming language targeting the MicroBlaze hardware architecture.
It generates machine code for MicroBlaze processors from modules written in Oberon and stores it in corresponding object files.
For debugging purposes, it also creates a debugging information file as well as an assembly file containing a listing of the generated machine code.
In addition, it stores the interface of each module in a symbol file which is required when other modules import the module.
Programs generated with this compiler require additional runtime support that is stored in the \file{ob\-mibl\-run} library file.
\flowgraph{\resource{Oberon\\source code} \ar[r] & \toolbox{obmibl} \ar[r] \ar@/l/[d] \ar[rd] & \resource{object file} \\ \variable{ECSIMPORT} \ar[ru] & \resource{symbol\\files} \ar@/r/[u] & \resource{debugging\\information}}
\seeoberon\seeassembly\seemibl\seeobject\seedebugging
}

\providecommand{\obmipsa}{
\toolsection{obmips32} is a compiler for the Oberon programming language targeting the MIPS32 hardware architecture.
It generates machine code for MIPS32 processors from modules written in Oberon and stores it in corresponding object files.
For debugging purposes, it also creates a debugging information file as well as an assembly file containing a listing of the generated machine code.
In addition, it stores the interface of each module in a symbol file which is required when other modules import the module.
Programs generated with this compiler require additional runtime support that is stored in the \file{ob\-mips32\-run} library file.
\flowgraph{\resource{Oberon\\source code} \ar[r] & \toolbox{obmips32} \ar[r] \ar@/l/[d] \ar[rd] & \resource{object file} \\ \variable{ECSIMPORT} \ar[ru] & \resource{symbol\\files} \ar@/r/[u] & \resource{debugging\\information}}
\seeoberon\seeassembly\seemips\seeobject\seedebugging
}

\providecommand{\obmipsb}{
\toolsection{obmips64} is a compiler for the Oberon programming language targeting the MIPS64 hardware architecture.
It generates machine code for MIPS64 processors from modules written in Oberon and stores it in corresponding object files.
For debugging purposes, it also creates a debugging information file as well as an assembly file containing a listing of the generated machine code.
In addition, it stores the interface of each module in a symbol file which is required when other modules import the module.
Programs generated with this compiler require additional runtime support that is stored in the \file{ob\-mips64\-run} library file.
\flowgraph{\resource{Oberon\\source code} \ar[r] & \toolbox{obmips64} \ar[r] \ar@/l/[d] \ar[rd] & \resource{object file} \\ \variable{ECSIMPORT} \ar[ru] & \resource{symbol\\files} \ar@/r/[u] & \resource{debugging\\information}}
\seeoberon\seeassembly\seemips\seeobject\seedebugging
}

\providecommand{\obmmix}{
\toolsection{obmmix} is a compiler for the Oberon programming language targeting the MMIX hardware architecture.
It generates machine code for MMIX processors from modules written in Oberon and stores it in corresponding object files.
For debugging purposes, it also creates a debugging information file as well as an assembly file containing a listing of the generated machine code.
In addition, it stores the interface of each module in a symbol file which is required when other modules import the module.
Programs generated with this compiler require additional runtime support that is stored in the \file{ob\-mmix\-run} library file.
\flowgraph{\resource{Oberon\\source code} \ar[r] & \toolbox{obmmix} \ar[r] \ar@/l/[d] \ar[rd] & \resource{object file} \\ \variable{ECSIMPORT} \ar[ru] & \resource{symbol\\files} \ar@/r/[u] & \resource{debugging\\information}}
\seeoberon\seeassembly\seemmix\seeobject\seedebugging
}

\providecommand{\oborok}{
\toolsection{obor1k} is a compiler for the Oberon programming language targeting the OpenRISC 1000 hardware architecture.
It generates machine code for OpenRISC 1000 processors from modules written in Oberon and stores it in corresponding object files.
For debugging purposes, it also creates a debugging information file as well as an assembly file containing a listing of the generated machine code.
In addition, it stores the interface of each module in a symbol file which is required when other modules import the module.
Programs generated with this compiler require additional runtime support that is stored in the \file{ob\-or1k\-run} library file.
\flowgraph{\resource{Oberon\\source code} \ar[r] & \toolbox{obor1k} \ar[r] \ar@/l/[d] \ar[rd] & \resource{object file} \\ \variable{ECSIMPORT} \ar[ru] & \resource{symbol\\files} \ar@/r/[u] & \resource{debugging\\information}}
\seeoberon\seeassembly\seeorok\seeobject\seedebugging
}

\providecommand{\obppca}{
\toolsection{obppc32} is a compiler for the Oberon programming language targeting the PowerPC hardware architecture.
It generates machine code for PowerPC processors from modules written in Oberon and stores it in corresponding object files.
The compiler generates machine code for the 32-bit operating mode defined by the PowerPC architecture.
For debugging purposes, it also creates a debugging information file as well as an assembly file containing a listing of the generated machine code.
In addition, it stores the interface of each module in a symbol file which is required when other modules import the module.
Programs generated with this compiler require additional runtime support that is stored in the \file{ob\-ppc32\-run} library file.
\flowgraph{\resource{Oberon\\source code} \ar[r] & \toolbox{obppc32} \ar[r] \ar@/l/[d] \ar[rd] & \resource{object file} \\ \variable{ECSIMPORT} \ar[ru] & \resource{symbol\\files} \ar@/r/[u] & \resource{debugging\\information}}
\seeoberon\seeassembly\seeppc\seeobject\seedebugging
}

\providecommand{\obppcb}{
\toolsection{obppc64} is a compiler for the Oberon programming language targeting the PowerPC hardware architecture.
It generates machine code for PowerPC processors from modules written in Oberon and stores it in corresponding object files.
The compiler generates machine code for the 64-bit operating mode defined by the PowerPC architecture.
For debugging purposes, it also creates a debugging information file as well as an assembly file containing a listing of the generated machine code.
In addition, it stores the interface of each module in a symbol file which is required when other modules import the module.
Programs generated with this compiler require additional runtime support that is stored in the \file{ob\-ppc64\-run} library file.
\flowgraph{\resource{Oberon\\source code} \ar[r] & \toolbox{obppc64} \ar[r] \ar@/l/[d] \ar[rd] & \resource{object file} \\ \variable{ECSIMPORT} \ar[ru] & \resource{symbol\\files} \ar@/r/[u] & \resource{debugging\\information}}
\seeoberon\seeassembly\seeppc\seeobject\seedebugging
}

\providecommand{\obrisc}{
\toolsection{obrisc} is a compiler for the Oberon programming language targeting the RISC hardware architecture.
It generates machine code for RISC processors from modules written in Oberon and stores it in corresponding object files.
For debugging purposes, it also creates a debugging information file as well as an assembly file containing a listing of the generated machine code.
In addition, it stores the interface of each module in a symbol file which is required when other modules import the module.
Programs generated with this compiler require additional runtime support that is stored in the \file{ob\-risc\-run} library file.
\flowgraph{\resource{Oberon\\source code} \ar[r] & \toolbox{obrisc} \ar[r] \ar@/l/[d] \ar[rd] & \resource{object file} \\ \variable{ECSIMPORT} \ar[ru] & \resource{symbol\\files} \ar@/r/[u] & \resource{debugging\\information}}
\seeoberon\seeassembly\seerisc\seeobject\seedebugging
}

\providecommand{\obwasm}{
\toolsection{obwasm} is a compiler for the Oberon programming language targeting the WebAssembly architecture.
It generates machine code for WebAssembly targets from modules written in Oberon and stores it in corresponding object files.
For debugging purposes, it also creates a debugging information file as well as an assembly file containing a listing of the generated machine code.
In addition, it stores the interface of each module in a symbol file which is required when other modules import the module.
Programs generated with this compiler require additional runtime support that is stored in the \file{ob\-wasm\-run} library file.
\flowgraph{\resource{Oberon\\source code} \ar[r] & \toolbox{obwasm} \ar[r] \ar@/l/[d] \ar[rd] & \resource{object file} \\ \variable{ECSIMPORT} \ar[ru] & \resource{symbol\\files} \ar@/r/[u] & \resource{debugging\\information}}
\seeoberon\seeassembly\seewasm\seeobject\seedebugging
}

% converter tools

\providecommand{\dbgdwarf}{
\toolsection{dbgdwarf} is a DWARF debugging information converter tool.
It converts debugging information into the DWARF debugging data format and stores it in corresponding object files~\cite{dwarffile}.
The resulting debugging object files can be combined with runtime support that creates Executable and Linking Format (ELF) files~\cite{elffile}.
\flowgraph{\resource{debugging\\information} \ar[r] & \toolbox{dbgdwarf} \ar[r] & \resource{debugging\\object file}}
\seeobject\seedebugging
}

% assembler tools

\providecommand{\asmprint}{
\toolsection{asmprint} is a pretty printer for generic assembly code.
It reformats generic assembly code and writes it to the standard output stream.
\flowgraph{\resource{generic assembly\\source code} \ar[r] & \toolbox{asmprint} \ar[r] & \resource{reformatted\\source code}}
\seeassembly
}

\providecommand{\amdaasm}{
\toolsection{amd16asm} is an assembler for the AMD64 hardware architecture.
It translates assembly code into machine code for AMD64 processors and stores it in corresponding object files.
By default, the assembler generates machine code for the 16-bit operating mode defined by the AMD64 architecture.
\flowgraph{\resource{AMD16 assembly\\source code} \ar[r] & \toolbox{amd16asm} \ar[r] & \resource{object file}}
\seeassembly\seeamd\seeobject
}

\providecommand{\amdadism}{
\toolsection{amd16dism} is a disassembler for the AMD64 hardware architecture.
It translates machine code from object files targeting AMD64 processors into assembly code and writes it to the standard output stream.
It assumes that the machine code was generated for the 16-bit operating mode defined by the AMD64 architecture.
\flowgraph{\resource{object file} \ar[r] & \toolbox{amd16dism} \ar[r] & \resource{disassembly\\listing}}
\seeassembly\seeamd\seeobject
}

\providecommand{\amdbasm}{
\toolsection{amd32asm} is an assembler for the AMD64 hardware architecture.
It translates assembly code into machine code for AMD64 processors and stores it in corresponding object files.
By default, the assembler generates machine code for the 32-bit operating mode defined by the AMD64 architecture.
\flowgraph{\resource{AMD32 assembly\\source code} \ar[r] & \toolbox{amd32asm} \ar[r] & \resource{object file}}
\seeassembly\seeamd\seeobject
}

\providecommand{\amdbdism}{
\toolsection{amd32dism} is a disassembler for the AMD64 hardware architecture.
It translates machine code from object files targeting AMD64 processors into assembly code and writes it to the standard output stream.
It assumes that the machine code was generated for the 32-bit operating mode defined by the AMD64 architecture.
\flowgraph{\resource{object file} \ar[r] & \toolbox{amd32dism} \ar[r] & \resource{disassembly\\listing}}
\seeassembly\seeamd\seeobject
}

\providecommand{\amdcasm}{
\toolsection{amd64asm} is an assembler for the AMD64 hardware architecture.
It translates assembly code into machine code for AMD64 processors and stores it in corresponding object files.
By default, the assembler generates machine code for the 64-bit operating mode defined by the AMD64 architecture.
\flowgraph{\resource{AMD64 assembly\\source code} \ar[r] & \toolbox{amd64asm} \ar[r] & \resource{object file}}
\seeassembly\seeamd\seeobject
}

\providecommand{\amdcdism}{
\toolsection{amd64dism} is a disassembler for the AMD64 hardware architecture.
It translates machine code from object files targeting AMD64 processors into assembly code and writes it to the standard output stream.
It assumes that the machine code was generated for the 64-bit operating mode defined by the AMD64 architecture.
\flowgraph{\resource{object file} \ar[r] & \toolbox{amd64dism} \ar[r] & \resource{disassembly\\listing}}
\seeassembly\seeamd\seeobject
}

\providecommand{\armaasm}{
\toolsection{arma32asm} is an assembler for the ARM hardware architecture.
It translates assembly code into machine code for ARM processors executing A32 instructions and stores it in corresponding object files.
\flowgraph{\resource{ARM A32 assembly\\source code} \ar[r] & \toolbox{arma32asm} \ar[r] & \resource{object file}}
\seeassembly\seearm\seeobject
}

\providecommand{\armadism}{
\toolsection{arma32dism} is a disassembler for the ARM hardware architecture.
It translates machine code from object files targeting ARM processors executing A32 instructions into assembly code and writes it to the standard output stream.
\flowgraph{\resource{object file} \ar[r] & \toolbox{arma32dism} \ar[r] & \resource{disassembly\\listing}}
\seeassembly\seearm\seeobject
}

\providecommand{\armbasm}{
\toolsection{arma64asm} is an assembler for the ARM hardware architecture.
It translates assembly code into machine code for ARM processors executing A64 instructions and stores it in corresponding object files.
\flowgraph{\resource{ARM A64 assembly\\source code} \ar[r] & \toolbox{arma64asm} \ar[r] & \resource{object file}}
\seeassembly\seearm\seeobject
}

\providecommand{\armbdism}{
\toolsection{arma64dism} is a disassembler for the ARM hardware architecture.
It translates machine code from object files targeting ARM processors executing A64 instructions into assembly code and writes it to the standard output stream.
\flowgraph{\resource{object file} \ar[r] & \toolbox{arma64dism} \ar[r] & \resource{disassembly\\listing}}
\seeassembly\seearm\seeobject
}

\providecommand{\armcasm}{
\toolsection{armt32asm} is an assembler for the ARM hardware architecture.
It translates assembly code into machine code for ARM processors executing T32 instructions and stores it in corresponding object files.
\flowgraph{\resource{ARM T32 assembly\\source code} \ar[r] & \toolbox{armt32asm} \ar[r] & \resource{object file}}
\seeassembly\seearm\seeobject
}

\providecommand{\armcdism}{
\toolsection{armt32dism} is a disassembler for the ARM hardware architecture.
It translates machine code from object files targeting ARM processors executing T32 instructions into assembly code and writes it to the standard output stream.
\flowgraph{\resource{object file} \ar[r] & \toolbox{armt32dism} \ar[r] & \resource{disassembly\\listing}}
\seeassembly\seearm\seeobject
}

\providecommand{\avrasm}{
\toolsection{avrasm} is an assembler for the AVR hardware architecture.
It translates assembly code into machine code for AVR processors and stores it in corresponding object files.
The identifiers \texttt{RXL}, \texttt{RXH}, \texttt{RYL}, \texttt{RYH}, \texttt{RZL}, and \texttt{RZH} are predefined and name the corresponding registers.
The identifiers \texttt{SPL} and \texttt{SPH} are also predefined and evaluate to the address of the corresponding registers.
\flowgraph{\resource{AVR assembly\\source code} \ar[r] & \toolbox{avrasm} \ar[r] & \resource{object file}}
\seeassembly\seeavr\seeobject
}

\providecommand{\avrdism}{
\toolsection{avrdism} is a disassembler for the AVR hardware architecture.
It translates machine code from object files targeting AVR processors into assembly code and writes it to the standard output stream.
\flowgraph{\resource{object file} \ar[r] & \toolbox{avrdism} \ar[r] & \resource{disassembly\\listing}}
\seeassembly\seeavr\seeobject
}

\providecommand{\avrttasm}{
\toolsection{avr32asm} is an assembler for the AVR32 hardware architecture.
It translates assembly code into machine code for AVR32 processors and stores it in corresponding object files.
\flowgraph{\resource{AVR32 assembly\\source code} \ar[r] & \toolbox{avr32asm} \ar[r] & \resource{object file}}
\seeassembly\seeavrtt\seeobject
}

\providecommand{\avrttdism}{
\toolsection{avr32dism} is a disassembler for the AVR32 hardware architecture.
It translates machine code from object files targeting AVR32 processors into assembly code and writes it to the standard output stream.
\flowgraph{\resource{object file} \ar[r] & \toolbox{avr32dism} \ar[r] & \resource{disassembly\\listing}}
\seeassembly\seeavrtt\seeobject
}

\providecommand{\mabkasm}{
\toolsection{m68kasm} is an assembler for the M68000 hardware architecture.
It translates assembly code into machine code for M68000 processors and stores it in corresponding object files.
\flowgraph{\resource{68000 assembly\\source code} \ar[r] & \toolbox{m68kasm} \ar[r] & \resource{object file}}
\seeassembly\seemabk\seeobject
}

\providecommand{\mabkdism}{
\toolsection{m68kdism} is a disassembler for the M68000 hardware architecture.
It translates machine code from object files targeting M68000 processors into assembly code and writes it to the standard output stream.
\flowgraph{\resource{object file} \ar[r] & \toolbox{m68kdism} \ar[r] & \resource{disassembly\\listing}}
\seeassembly\seemabk\seeobject
}

\providecommand{\miblasm}{
\toolsection{miblasm} is an assembler for the MicroBlaze hardware architecture.
It translates assembly code into machine code for MicroBlaze processors and stores it in corresponding object files.
\flowgraph{\resource{MicroBlaze assembly\\source code} \ar[r] & \toolbox{miblasm} \ar[r] & \resource{object file}}
\seeassembly\seemibl\seeobject
}

\providecommand{\mibldism}{
\toolsection{mibldism} is a disassembler for the MicroBlaze hardware architecture.
It translates machine code from object files targeting MicroBlaze processors into assembly code and writes it to the standard output stream.
\flowgraph{\resource{object file} \ar[r] & \toolbox{mibldism} \ar[r] & \resource{disassembly\\listing}}
\seeassembly\seemibl\seeobject
}

\providecommand{\mipsaasm}{
\toolsection{mips32asm} is an assembler for the MIPS32 hardware architecture.
It translates assembly code into machine code for MIPS32 processors and stores it in corresponding object files.
\flowgraph{\resource{MIPS32 assembly\\source code} \ar[r] & \toolbox{mips32asm} \ar[r] & \resource{object file}}
\seeassembly\seemips\seeobject
}

\providecommand{\mipsadism}{
\toolsection{mips32dism} is a disassembler for the MIPS32 hardware architecture.
It translates machine code from object files targeting MIPS32 processors into assembly code and writes it to the standard output stream.
\flowgraph{\resource{object file} \ar[r] & \toolbox{mips32dism} \ar[r] & \resource{disassembly\\listing}}
\seeassembly\seemips\seeobject
}

\providecommand{\mipsbasm}{
\toolsection{mips64asm} is an assembler for the MIPS64 hardware architecture.
It translates assembly code into machine code for MIPS64 processors and stores it in corresponding object files.
\flowgraph{\resource{MIPS64 assembly\\source code} \ar[r] & \toolbox{mips64asm} \ar[r] & \resource{object file}}
\seeassembly\seemips\seeobject
}

\providecommand{\mipsbdism}{
\toolsection{mips64dism} is a disassembler for the MIPS64 hardware architecture.
It translates machine code from object files targeting MIPS64 processors into assembly code and writes it to the standard output stream.
\flowgraph{\resource{object file} \ar[r] & \toolbox{mips64dism} \ar[r] & \resource{disassembly\\listing}}
\seeassembly\seemips\seeobject
}

\providecommand{\mmixasm}{
\toolsection{mmixasm} is an assembler for the MMIX hardware architecture.
It translates assembly code into machine code for MMIX processors and stores it in corresponding object files.
The names of all special registers are predefined and evaluate to the corresponding number.
\flowgraph{\resource{MMIX assembly\\source code} \ar[r] & \toolbox{mmixasm} \ar[r] & \resource{object file}}
\seeassembly\seemmix\seeobject
}

\providecommand{\mmixdism}{
\toolsection{mmixdism} is a disassembler for the MMIX hardware architecture.
It translates machine code from object files targeting MMIX processors into assembly code and writes it to the standard output stream.
\flowgraph{\resource{object file} \ar[r] & \toolbox{mmixdism} \ar[r] & \resource{disassembly\\listing}}
\seeassembly\seemmix\seeobject
}

\providecommand{\orokasm}{
\toolsection{or1kasm} is an assembler for the OpenRISC 1000 hardware architecture.
It translates assembly code into machine code for OpenRISC 1000 processors and stores it in corresponding object files.
\flowgraph{\resource{OpenRISC 1000 assembly\\source code} \ar[r] & \toolbox{or1kasm} \ar[r] & \resource{object file}}
\seeassembly\seeorok\seeobject
}

\providecommand{\orokdism}{
\toolsection{or1kdism} is a disassembler for the OpenRISC 1000 hardware architecture.
It translates machine code from object files targeting OpenRISC 1000 processors into assembly code and writes it to the standard output stream.
\flowgraph{\resource{object file} \ar[r] & \toolbox{or1kdism} \ar[r] & \resource{disassembly\\listing}}
\seeassembly\seeorok\seeobject
}

\providecommand{\ppcaasm}{
\toolsection{ppc32asm} is an assembler for the PowerPC hardware architecture.
It translates assembly code into machine code for PowerPC processors and stores it in corresponding object files.
By default, the assembler generates machine code for the 32-bit operating mode defined by the PowerPC architecture.
\flowgraph{\resource{PowerPC assembly\\source code} \ar[r] & \toolbox{ppc32asm} \ar[r] & \resource{object file}}
\seeassembly\seeppc\seeobject
}

\providecommand{\ppcadism}{
\toolsection{ppc32dism} is a disassembler for the PowerPC hardware architecture.
It translates machine code from object files targeting PowerPC processors into assembly code and writes it to the standard output stream.
It assumes that the machine code was generated for the 32-bit operating mode defined by the PowerPC architecture.
\flowgraph{\resource{object file} \ar[r] & \toolbox{ppc32dism} \ar[r] & \resource{disassembly\\listing}}
\seeassembly\seeppc\seeobject
}

\providecommand{\ppcbasm}{
\toolsection{ppc64asm} is an assembler for the PowerPC hardware architecture.
It translates assembly code into machine code for PowerPC processors and stores it in corresponding object files.
By default, the assembler generates machine code for the 64-bit operating mode defined by the PowerPC architecture.
\flowgraph{\resource{PowerPC assembly\\source code} \ar[r] & \toolbox{ppc64asm} \ar[r] & \resource{object file}}
\seeassembly\seeppc\seeobject
}

\providecommand{\ppcbdism}{
\toolsection{ppc64dism} is a disassembler for the PowerPC hardware architecture.
It translates machine code from object files targeting PowerPC processors into assembly code and writes it to the standard output stream.
It assumes that the machine code was generated for the 64-bit operating mode defined by the PowerPC architecture.
\flowgraph{\resource{object file} \ar[r] & \toolbox{ppc64dism} \ar[r] & \resource{disassembly\\listing}}
\seeassembly\seeppc\seeobject
}

\providecommand{\riscasm}{
\toolsection{riscasm} is an assembler for the RISC hardware architecture.
It translates assembly code into machine code for RISC processors and stores it in corresponding object files.
The names of all special registers are predefined and evaluate to the corresponding number.
\flowgraph{\resource{RISC assembly\\source code} \ar[r] & \toolbox{riscasm} \ar[r] & \resource{object file}}
\seeassembly\seerisc\seeobject
}

\providecommand{\riscdism}{
\toolsection{riscdism} is a disassembler for the RISC hardware architecture.
It translates machine code from object files targeting RISC processors into assembly code and writes it to the standard output stream.
\flowgraph{\resource{object file} \ar[r] & \toolbox{riscdism} \ar[r] & \resource{disassembly\\listing}}
\seeassembly\seerisc\seeobject
}

\providecommand{\wasmasm}{
\toolsection{wasmasm} is an assembler for the WebAssembly architecture.
It translates assembly code into machine code for WebAssembly targets and stores it in corresponding object files.
The names of all special registers are predefined and evaluate to the corresponding number.
\flowgraph{\resource{WebAssembly assembly\\source code} \ar[r] & \toolbox{wasmasm} \ar[r] & \resource{object file}}
\seeassembly\seewasm\seeobject
}

\providecommand{\wasmdism}{
\toolsection{wasmdism} is a disassembler for the WebAssembly architecture.
It translates machine code from object files targeting WebAssembly targets into assembly code and writes it to the standard output stream.
\flowgraph{\resource{object file} \ar[r] & \toolbox{wasmdism} \ar[r] & \resource{disassembly\\listing}}
\seeassembly\seewasm\seeobject
}

% linker tools

\providecommand{\linklib}{
\toolsection{linklib} is an object file combiner.
It creates a static library file by combining all object files given to it into a single one.
\flowgraph{\resource{object files} \ar[r] & \toolbox{linklib} \ar[r] & \resource{library file}}
\seeobject
}

\providecommand{\linkbin}{
\toolsection{linkbin} is a linker for plain binary files.
It links all object files given to it into a single image and stores it in a binary file that begins with the first linked section.
It also creates a map file that lists the address, type, name and size of all used sections.
The filename extension of the resulting binary file can be specified by putting it into a constant data section called \texttt{\_extension}.
\flowgraph{\resource{object files} \ar[r] & \toolbox{linkbin} \ar[r] \ar[d] & \resource{binary file} \\ & \resource{map file}}
\seeobject
}

\providecommand{\linkmem}{
\toolsection{linkmem} is a linker for plain binary files partitioned into random-access and read-only memory.
It links all object files given to it into two distinct images, one for data sections and one for code and constant data sections, and stores each image in a binary file that begins with the first linked section of the corresponding type.
It also creates a map file that lists the address, type, name and size of all used sections.
\flowgraph{\resource{object files} \ar[r] & \toolbox{linkmem} \ar[r] \ar[d] & \resource{RAM file/\\ROM file} \\ & \resource{map file}}
\seeobject
}

\providecommand{\linkprg}{
\toolsection{linkprg} is a linker for GEMDOS executable files.
It links all object files given to it into a single image and stores the image in an Atari GEMDOS executable file~\cite{gemdosfile}.
It also creates a map file that lists the address relative to the text segment, type, name and size of all used sections.
The filename extension of the resulting executable file can be specified by putting it into a constant data section called \texttt{\_extension}.
The GEMDOS executable file format requires all patch patterns of absolute link patches to consist of four full bitmasks with descending offsets.
\flowgraph{\resource{object files} \ar[r] & \toolbox{linkprg} \ar[r] \ar[d] & \resource{executable file} \\ & \resource{map file}}
\seeobject
}

\providecommand{\linkhex}{
\toolsection{linkhex} is a linker for Intel HEX files.
It links all code sections of the object files given to it into single image and stores the image in an Intel HEX file~\cite{hexfile} that begins with the first linked section.
It also creates a map file that lists the address, type, name and size of all used sections.
\flowgraph{\resource{object files} \ar[r] & \toolbox{linkhex} \ar[r] \ar[d] & \resource{HEX file} \\ & \resource{map file}}
\seeobject
}

\providecommand{\mapsearch}{
\toolsection{mapsearch} is a debugging tool.
It searches map files generated by linker tools for the name of a binary section that encompasses a memory address read from the standard input stream.
If additionally provided with one or more object files, it also stores an excerpt thereof in a separate object file called map search result which only contains the identified binary section for disassembling purposes.
\flowgraph{& \resource{map files/\\object files} \ar[d] \\ \resource{memory\\address} \ar[r] & \toolbox{mapsearch} \ar[r] \ar[d] & \resource{section name/\\relative offset} \\ & \resource{object file\\excerpt}}
\seeobject
}


\startchapter{User Interface}{User Interface}{interface}
{The \ecs{} features a variety of different tools like compilers, assemblers, and linkers.
This \documentation{} describes their input and output and explains the common user interface that is shared across all of these tools.}

\epigraph{The trivial round, the common task, \\ would furnish all we ought to ask.}{John Keble}

\section{Introduction}

The \ecs{} is a toolchain that consists of several different programming tools targeting a variety of programming languages, hardware architectures, and runtime environments.
The following list categorizes all tools provided by the \ecs{} and visualizes their input and output:

\begin{itemize}

\item Preprocessors\index{Preprocessors}\nopagebreak

Preprocessors are tools that perform simple lexical substitutions like macro expansions and conditional inclusion on source code written in a programming language.
This very first phase of translating the source code is also part of all subsequent tools if the corresponding programming language supports preprocessing.
An example of a preprocessor tool is \tool{cpp\-prep}.

\flowgraph{\resource{source code} \ar[r] & \converter{Preprocessor} \ar[r] & \resource{preprocessed\\source code}}

\item Pretty Printers\index{Pretty printers}\nopagebreak

Pretty printers are tools that reformat source code written in a programming language by printing it again using a consistent layout.
Examples of pretty printer tools are \tool{asm\-print} and \tool{fal\-print}.

\flowgraph{\resource{source code} \ar[r] & \converter{Pretty Printer} \ar[r] & \resource{reformatted\\source code}}

\item Semantic Checkers\index{Semantic checkers}\nopagebreak

Semantic checkers perform syntactic and semantic checks on source code written in a programming language and print diagnostic messages.
Some checkers may have to process additional semantic information stored in separate files.
This phase of translating the source code is also part of all subsequent tools.
Examples of semantic checker tools are \tool{cpp\-check} and \tool{cd\-check}.

\flowgraph{\resource{source code} \ar[r] & \converter{Semantic\\Checker} \ar@/l/[d] \ar[r] & \resource{diagnostic\\messages} \\ & \resource{external\\information} \ar@/r/[u]}

\item Serializers\index{Serializers}\nopagebreak

Serializers are debugging tools that dump the complete internal representation of a program in a human readable format.
This potentially also allows external development tools to make use of the same internal program representation.
Examples of serialization tools are \tool{ob\-dump} and \tool{fal\-dump}.

\flowgraph{\resource{source code} \ar[r] & \converter{Serializer} \ar[r] & \resource{internal\\representation}}

\item Interpreters\index{Interpreters}\nopagebreak

Interpreters read and process source code for a programming language by emulating a runtime environment for this language and executing the program accordingly.
The actual interpreted program defines the input and output the interpreter processes and produces.
Examples of interpreter tools are \tool{cpp\-run} and \tool{cd\-run}.

\flowgraph{\resource{source code} \ar[r] & \converter{Interpreter} \ar@/u/[r] & \resource{input/\\output} \ar@/d/[l]}

\item Transpilers\index{Transpilers}\nopagebreak

Transpilers translate source code written in one programming language into source code for another programming language.
They store the resulting code in corresponding source code files.
Examples of transpiler tools are \tool{fal\-cpp} and \tool{ob\-cpp}.

\flowgraph{\resource{source code} \ar[r] & \converter{Transpiler} \ar[r] & \resource{translated\\source code}}

\item Documentation Generators\index{Documentation generators}\nopagebreak

Documentation generators extract the structure and comments of the source code in order to generate documentations about the program.
They store the resulting documentation in document files of various formats. \seedocumentation
Examples of documentation generator tools are \tool{cpp\-html} and \tool{ob\-latex}.

\flowgraph{\resource{source code} \ar[r] & \converter{Documentation\\Generator} \ar[r] & \resource{formatted\\document}}

\item Compilers\index{Compilers}\nopagebreak

Compilers translate programs written in a programming language into machine code for some hardware processors.
They store the resulting binary representation of the source code in object files. \seeobject
In addition, they also create debugging information and assembly code listings of the generated machine code.
Examples of compiler tools are \tool{cpp\-amd64} and \tool{ob\-avr}.

\flowgraph{\resource{source code} \ar[r] & \converter{Compiler} \ar[r] \ar[d] \ar[rd] & \resource{object file} \\ & \resource{debugging\\information} & \resource{assembly\\listing}}

\item Debugging Information Converters\index{Debugging information converters}\nopagebreak

Debugging information converters process debugging information files and generate debugging object files storing a binary representation thereof in a format suitable for the debugger of a specific runtime environment. \seedebugging
An example of a debugging information converter tool is \tool{dbg\-dwarf}.

\flowgraph{\resource{debugging\\information} \ar[r] & \toolbox{Converter} \ar[r] & \resource{debugging\\object file}}

\item Assemblers\index{Assemblers}\nopagebreak

Assemblers translate assembly code for some specific hardware architecture into equivalent binary machine code.
They store the resulting binary representation of the source assembly code in object files. \seeassembly
Examples of assembler tools are \tool{arma32\-asm} and \tool{ppc\-asm}.

\flowgraph{\resource{assembly\\source code} \ar[r] & \converter{Assembler} \ar[r] & \resource{object file}}

\item Disassemblers\index{Disassemblers}\nopagebreak

Disassemblers translate machine code for some specific hardware architecture stored in object files by printing it using a human-readable disassembly listing.
Examples of disassembler tools are \tool{mibl\-dism} and \tool{arma64\-dism}.

\flowgraph{\resource{object file} \ar[r] & \converter{Disassembler} \ar[r] & \resource{disassembly\\listing}}

\item Linkers\index{Linkers}\nopagebreak

Linkers assemble the binary representation of a program stored in object files to generate output files that are executable on some target platforms.
Examples of linker tools are \tool{link\-bin} and \tool{link\-prg}.

\flowgraph{\resource{object files} \ar[r] & \converter{Linker} \ar[r] & \resource{executable\\binary image}}

\end{itemize}

Although all of these tools process and produce different kinds of input and output, all of them present the same command-line interface to the user which is described in the following sections.

\section{Command-Line Arguments}\index{Command-line arguments}

Each tool of the \ecs{} provides a command-line user interface and accepts zero, one or more command-line arguments.
If there are one or more command-line arguments, each of them is treated as the name of an input file relative to the current working directory.
The notion of command-line options is not supported in order to make the contents of output files independent from the tool invocation.

Input files are plain text files containing the source code or other contents to be processed.
Output files are generated in the current working directory and have the same name as the input file being processed whereas the filename extension gets replaced by an appropriate suffix.
If the user does not provide any command-line argument, the tool reads the source code from the standard input stream.
Beforehand, it prints its version and a short copyright notice.

If there are more than one command-line argument, each input file is processed in sequence according to the given order of the arguments.
If there are errors during the processing of an input file, subsequent command-line arguments are ignored.
Some tools like assemblers and disassemblers do not depend on other input except for the actual source code.
In this case, the order of the command-line arguments is irrelevant and it would be equivalent to execute the tool once for every input file.

However, there are tools that depend on the actual order of the command-line arguments.
They even may behave differently if executed several times with the same command-line argument.
This may especially be the case for compilers which may store additional semantic information in separate files for every input file they compile.
In addition, interpreters may also depend on the actual order of the input files they have processed so far.

On the other hand, it is crucial for some tools to process several input files during the same execution of the tool.
Particularly linkers have to collect the information stored in several object files to generate an executable binary image.
In this case, the order of the arguments is not important, since all information is collected first before the linking process is started.
However, linkers generate output files named after the very first argument with the filename extension being replaced by an appropriate suffix.

\section{Diagnostic Messages}\index{Diagnostic messages}\index{Messages|see{Diagnostic messages}}

All tools of the \ecs{} use the same scheme for the diagnostic messages they generate.
The following list describes the different types of messages used by the \ecs{}:

\begin{itemize}

\item Errors\index{Error messages}\nopagebreak

Error messages indicate a problem within the contents of an input file that prohibits the tool from proceeding successfully.
Compilers for example use error messages to output syntactic or semantic errors that have to be fixed by the programmer.
In general, tools diagnose only the first few of the encountered problems but always yield a return code indicating the failure.

\item Fatal Errors\index{Fatal error messages}\nopagebreak

Fatal errors indicate a general failure of the tool that cannot be fixed by changing the contents of the input file.
This includes for example internal program errors, failures to open files, or critical system conditions like out of memory.

\item Warnings\index{Warnings}\nopagebreak

Warning messages indicate potential flaws within the contents of an input file.
They do not cause the tool to fail but they identify issues that may lead to unexpected behavior.
Therefore, warnings may be prevented by changing the contents of an input file without changing its semantics.

\item Notes\index{Notes}\nopagebreak

Notes give additional information about the result of processing an input file.
They are often used for debugging purposes or to diagnose violations of coding conventions.

\end{itemize}

All diagnostic messages are written to the standard error stream.
Each message consists of a short line of text and the name of the input file or the tool that caused the diagnostic message.
The message also contains the position\index{Position}\index{Diagnostic messages!Position} within the input file, if the problem can be located.
Since input files are text files in general, the position is usually given as the number of the text line containing the problem followed by the column therein.
For convenience, some tools like compilers also indicate the problematic position by reproducing the respective line of text from the input file.

\section{The \ecs{} Driver}\index{Eigen Compiler Suite!Driver}\index{Driver, Eigen Compiler Suite}

The \ecs{} provides a utility tool called \tool{ecsd} which autonomously drives its toolchain in order to build complete executable files for a specific runtime environment.
It infers the set of required tools like compilers, assemblers, and linkers from the type of its input files and automatically invokes one tool after the other with appropriate command-line arguments and environment variables\index{Environment variables}.

\flowgraph{\resource{input\\files} \ar[d] \ar[rd]|\hole \ar[rrd]|!{[r];[d]}\hole|\hole \ar@{-->}[r] & \converter{ecsd\vphantom{Compilers}} \ar@{~}[ld] \ar@{~}[d] \ar@{~}[rd] \ar@{-->}[r] & \resource{executable\\file} \\
\converter{Compilers} \ar[d] \ar[rd] & \converter{Assemblers\vphantom{Compilers}} \ar[d] & \converter{Linkers\vphantom{Compilers}} \ar[u] \\
\resource{debugging\\information\vphantom{support}} & \resource{object\vphantom{debugging}\\files\vphantom{support}} \ar[ru] & \resource{runtime\vphantom{debugging}\\support} \ar[u]}

In contrast to all other tools of the \ecs{}, this utility tool does support command-line options which influence how tools are identified and invoked.
The following list describes the form and behavior of the most important options:

\begin{description}

\item\syntax{"-b" <directory> $\mid$ "-""-base" <directory>}\alignright Use Specified Base Directory\nopagebreak

Allows overriding the base directory provided by the \environmentvariable{ECSBASE} environment variable which is used to locate tools and the necessary runtime support.
If omitted, the tool infers the base directory from its own location.

\item\syntax{"-c" $\mid$ "-""-compile"}\alignright Compile and Assemble Only\nopagebreak

Compiles and assembles input files without invoking the linker at the end.
The output is a set of object files rather than an executable file.

\item\syntax{"-d" $\mid$ "-""-disassemble"}\alignright Disassemble Object Files\nopagebreak

Invokes the disassembler instead of the linker at the end.
The output is a set of disassembly listings rather than an executable file.

\item\syntax{"-g" $\mid$ "-""-generate"}\alignright Generate Debugging Information\nopagebreak

Generates and includes debugging information during linking.
The resulting executable file can be executed and tested using a symbolic debugger.

\item\syntax{"-h" $\mid$ "-""-help"}\alignright Print Command-Line Help\nopagebreak

Prints information about all supported command-line arguments.
This especially also includes all supported source types and available target environments.

\item\syntax{"-i" <tool> $\mid$ "-""-invoke" <tool>}\alignright Invoke Specified Tool Only\nopagebreak

Processes either input files or the standard input stream using the specified tool.
The output is whatever that tool generates rather than an executable file.
\ifbook See page~\pageref{idx:tools} for an index of all available tools. \fi

\item\syntax{"-l" $\mid$ "-""-library"}\alignright Create Library File\nopagebreak

Combines input files into a single object file.
The output is a library file rather than an executable file.

\item\syntax{"-o" <format> $\mid$ "-""-output" <format>}\alignright Output Specified File Format\nopagebreak

Specifies the output file format of the linker.
The available file formats correspond to the suffixes of the linker tools listed in \Documentation{}~\documentationref{object}{Object File Representation}.
If omitted, the tool uses the default file format of the target environment.

\item\syntax{"-p" $\mid$ "-""-protect"}\alignright Protect Environment Variables\nopagebreak

Prevents the environment from being automatically modified when invoking tools that depend on environment variables.

\item\syntax{"-r" <support> $\mid$ "-""-runtime" <support>}\alignright Include Specified Runtime Support\nopagebreak

Includes additional runtime support stored in the specified object or library file during linking.
\ifbook See page~\pageref{idx:runtime} for an index of all available runtime support. \fi

\item\syntax{"-s" <type> $\mid$ "-""-source" <type>}\alignright Use Specified Source Type\nopagebreak

Allows defining the source type of input files for which the corresponding tool could not be inferred from the filename extension.

\item\syntax{"-t" <environment> $\mid$ "-""-target" <environment>}\alignright Target Specified Environment\nopagebreak

Specifies the target environment for cross compilations.
\ifbook See page~\pageref{idx:environment} for an index of all available target environments. \fi
If omitted, the tool tries to target its own runtime environment.

\item\syntax{"-v" $\mid$ "-""-verbose"}\alignright Print Command Line\nopagebreak

Enables verbose mode which prints the actual command line before invoking any tool.
This is helpful for examining how and which tools are invoked and what runtime support is provided.

\end{description}

The \tool{ecsd} driver tool is most useful when it is accessible via the \environmentvariable{PATH} environment variable of the runtime environment and is part of an appropriate installation of the \ecs{}.
See \Documentation{}~\documentationref{guide}{User Guide} for some examples of using this utility tool in practice.

\concludechapter

% User guide for the Eigen Compiler Suite
% Copyright (C) Florian Negele

% This file is part of the Eigen Compiler Suite.

% Permission is granted to copy, distribute and/or modify this document
% under the terms of the GNU Free Documentation License, Version 1.3
% or any later version published by the Free Software Foundation.

% You should have received a copy of the GNU Free Documentation License
% along with the ECS.  If not, see <https://www.gnu.org/licenses/>.

% Generic documentation utilities
% Copyright (C) Florian Negele

% This file is part of the Eigen Compiler Suite.

% Permission is granted to copy, distribute and/or modify this document
% under the terms of the GNU Free Documentation License, Version 1.3
% or any later version published by the Free Software Foundation.

% You should have received a copy of the GNU Free Documentation License
% along with the ECS.  If not, see <https://www.gnu.org/licenses/>.

\providecommand{\cpp}{C\texttt{++}}
\providecommand{\opt}{_\mathit{opt}}
\providecommand{\tool}[1]{\texttt{#1}}
\providecommand{\version}{Version 0.0.40}
\providecommand{\resource}[1]{*++\txt{#1}}
\providecommand{\ecs}{Eigen Compiler Suite}
\providecommand{\changed}[1]{\underline{#1}}
\providecommand{\toolbox}[1]{\converter{#1}}
\providecommand{\file}{}\renewcommand{\file}[1]{\texttt{#1}}
\providecommand{\alignright}{\hfill\linebreak[0]\hspace*{\fill}}
\providecommand{\converter}[1]{*++[F][F*:white][F,:gray]\txt{#1}}
\providecommand{\documentation}{\ifbook chapter\else document\fi}
\providecommand{\Documentation}{\ifbook Chapter\else Document\fi}
\providecommand{\variable}[1]{\resource{\texttt{\small#1}\\variable}}
\providecommand{\documentationref}[2]{\ifbook\ref{#1}\else``\href{#1}{#2}''~\cite{#1}\fi}
\providecommand{\objfile}[1]{\texttt{#1}\index[runtime]{#1 object file@\texttt{#1} object file}}
\providecommand{\libfile}[1]{\texttt{#1}\index[runtime]{#1 library file@\texttt{#1} library file}}
\providecommand{\epigraph}[2]{\ifbook\begin{quote}\flushright\textit{#1}\par--- #2\end{quote}\fi}
\providecommand{\environmentvariable}[1]{\texttt{#1}\index{Environment variables!#1@\texttt{#1}}}
\providecommand{\environment}[1]{\texttt{#1}\index[environment]{#1 environment@\texttt{#1} environment}}
\providecommand{\toolsection}{}\renewcommand{\toolsection}[1]{\subsection{#1}\label{\prefix:#1}\tool{#1}}
\providecommand{\instruction}{}\renewcommand{\instruction}[2]{\noindent\qquad\pdftooltip{\texttt{#1}}{#2}\refstepcounter{instruction}\par}
\providecommand{\flowgraph}{}\renewcommand{\flowgraph}[1]{\par\sffamily\begin{displaymath}\xymatrix@=4ex{#1}\end{displaymath}\normalfont\par}
\providecommand{\instructionset}{}\renewcommand{\instructionset}[4]{\setcounter{instruction}{0}\begin{multicols}{\ifbook#3\else#4\fi}[{\captionof{table}[#2]{#2 (\ref*{#1:instructions}~instructions)}\label{tab:#1set}\vspace{-2ex}}]\footnotesize\raggedcolumns\input{#1.set}\label{#1:instructions}\end{multicols}}

\providecommand{\gpl}{GNU General Public License}
\providecommand{\rse}{ECS Runtime Support Exception}
\providecommand{\fdl}{\href{https://www.gnu.org/licenses/fdl.html}{GNU Free Documentation License}}

\providecommand{\docbegin}{}
\providecommand{\docend}{}
\providecommand{\doclabel}[1]{\hypertarget{#1}}
\providecommand{\doclink}[2]{\hyperlink{#1}{#2}}
\providecommand{\docsection}[3]{\hypertarget{#1}{\subsection{#2}}\label{sec:#1}\index[library]{#2@#3}}
\providecommand{\docsectionstar}[1]{}
\providecommand{\docsubbegin}{\begin{description}}
\providecommand{\docsubend}{\end{description}}
\providecommand{\docsubsection}[3]{\item[\hypertarget{#1}{#2}]\index[library]{#2@#3}}
\providecommand{\docsubsectionstar}[1]{\smallskip}
\providecommand{\docsubsubsection}[3]{\docsubsection{#1}{#2}{#3}}
\providecommand{\docsubsubsectionstar}[1]{}
\providecommand{\docsubsubsubsection}[3]{}
\providecommand{\docsubsubsubsectionstar}[1]{}
\providecommand{\doctable}{}

\providecommand{\debuggingtool}{}\renewcommand{\debuggingtool}{This tool is provided for debugging purposes.
It allows exposing and modifying an internal data structure that is usually not accessible.
}

\providecommand{\interface}{All tools accept command-line arguments which are taken as names of plain text files containing the source code.
If no arguments are provided, the standard input stream is used instead.
Output files are generated in the current working directory and have the same name as the input file being processed whereas the filename extension gets replaced by an appropriate suffix.
\seeinterface
}

\providecommand{\license}{\noindent Copyright \copyright{} Florian Negele\par\medskip\noindent
Permission is granted to copy, distribute and/or modify this document under the terms of the
\fdl{}, Version 1.3 or any later version published by the \href{https://fsf.org/}{Free Software Foundation}.
}

\providecommand{\ecslogosurface}{
\fill[darkgray] (0,0,0) -- (0,0,3) -- (0,3,3) -- (0,3,1) -- (0,4,1) -- (0,4,3) -- (0,5,3) -- (0,5,0) -- (0,2,0) -- (0,2,2) -- (0,1,2) -- (0,1,0) -- cycle;
\fill[gray] (0,5,0) -- (0,5,3) -- (1,5,3) -- (1,5,1) -- (2,5,1) -- (2,5,3) -- (3,5,3) -- (3,5,0) -- cycle;
\fill[lightgray] (0,0,0) -- (0,1,0) -- (2,1,0) -- (2,4,0) -- (1,4,0) -- (1,3,0) -- (2,3,0) -- (2,2,0) -- (0,2,0) -- (0,5,0) -- (3,5,0) -- (3,0,0) -- cycle;
\begin{scope}[line width=0.5]
\begin{scope}[gray]
\draw (0,0,0) -- (0,1,0);
\draw (2,1,0) -- (2,2,0);
\draw (0,1,2) -- (0,2,2);
\draw (0,2,0) -- (0,5,0);
\draw (2,3,0) -- (2,4,0);
\end{scope}
\begin{scope}[lightgray]
\draw (0,1,0) -- (0,1,2);
\draw (0,3,1) -- (0,3,3);
\draw (0,5,0) -- (0,5,3);
\draw (2,5,1) -- (2,5,3);
\end{scope}
\begin{scope}[white]
\draw (0,1,0) -- (2,1,0);
\draw (1,3,0) -- (2,3,0);
\draw (0,5,0) -- (3,5,0);
\end{scope}
\end{scope}
}

\providecommand{\ecslogo}[1]{
\begin{tikzpicture}[scale={(#1)/((sin(45)+cos(45))*3cm)},x={({-cos(45)*1cm},{sin(45)*sin(30)*1cm})},y={({0cm},{(cos(30)*1cm})},z={({sin(45)*1cm},{cos(45)*sin(30)*1cm})}]
\begin{scope}[darkgray,line width=1]
\draw (0,0,0) -- (0,0,3) -- (0,3,3) -- (2,3,3) -- (2,5,3) -- (3,5,3) -- (3,5,0) -- (3,0,0) -- cycle;
\draw (0,3,1) -- (0,4,1) -- (0,4,3) -- (0,5,3) -- (1,5,3) -- (1,5,1) -- (2,5,1);
\draw (1,3,0) -- (1,4,0) -- (2,4,0);
\end{scope}
\fill[darkgray] (2,0,0) -- (2,0,3) -- (2,5,3) -- (2,5,1) -- (2,4,1) -- (2,4,0) -- cycle;
\fill[lightgray] (2,0,2) -- (0,0,2) -- (0,2,2) -- (2,2,2) -- cycle;
\fill[gray] (0,1,0) -- (2,1,0) -- (2,1,2) -- (0,1,2) -- cycle;
\fill[gray] (0,3,1) -- (0,3,3) -- (2,3,3) -- (2,3,0) -- (1,3,0) -- (1,3,1) -- cycle;
\ecslogosurface
\end{tikzpicture}
}

\providecommand{\shadowedecslogo}[3]{
\begin{tikzpicture}[scale={(#1)/((sin(#2)+cos(#2))*3cm)},x={({-cos(#2)*1cm},{sin(#2)*sin(#3)*1cm})},y={({0cm},{(cos(#3)*1cm})},z={({sin(#2)*1cm},{cos(#2)*sin(#3)*1cm})}]
\shade[top color=lightgray!50!white,bottom color=white,middle color=lightgray!50!white] (0,0,0) -- (3,0,0) -- (3,{-0.5-3*sin(#2)*sin(#3)/cos(#3)},0) -- (0,-0.5,0) -- cycle;
\shade[top color=darkgray!50!gray,bottom color=white,middle color=darkgray!50!white] (0,0,0) -- (0,0,3) -- (0,{-0.5-3*cos(#2)*sin(#3)/cos(#3)},3) -- (0,-0.5,0) -- cycle;
\begin{scope}[y={({(cos(#2)+sin(#2))*0.5cm},{(cos(#2)*sin(#3)-sin(#2)*sin(#3))*0.5cm})}]
\useasboundingbox (3,0,0) -- (0,0,0) -- (0,0,3);
\shade[left color=darkgray!80!black,right color=lightgray,middle color=gray] (0,0,0) -- (0,1,0) -- (0,1,0.5) -- (0,2,0) -- (0,5,0) -- (0,5,3) -- (1,5,3) -- (1,4,3) -- (1,4,2.5) -- (1,3,3) -- (2,5,3) -- (3,5,3) -- (3,0,3) -- cycle;
\clip (0,0,0) -- (0,0,3) -- ({-3*sin(#2)/cos(#2)},0,0) -- cycle;
\shade[left color=darkgray,right color=lightgray!50!gray] (0,0,0) -- (0,1,0) -- (0,1,0.5) -- (0,2,0) -- (0,5,0) -- (0,5,3) -- (1,5,3) -- (1,4,3) -- (1,4,2.5) -- (1,3,3) -- (2,5,3) -- (3,5,3) -- (3,0,3) -- cycle;
\end{scope}
\shade[left color=darkgray,right color=darkgray!80!black] (2,0,0) -- (2,0,3) -- (2,5,3) -- (2,5,1) -- (2,4,1) -- (2,4,0) -- cycle;
\shade[left color=darkgray!90!black,right color=gray!80!darkgray] (2,0,2) -- (0,0,2) -- (0,2,2) -- (2,2,2) -- cycle;
\shade[top color=darkgray!90!black,bottom color=gray!80!darkgray] (0,1,0) -- (2,1,0) -- (2,1,2) -- (0,1,2) -- cycle;
\shade[top color=darkgray!90!black,bottom color=gray!80!darkgray] (0,3,1) -- (0,3,3) -- (2,3,3) -- (2,3,0) -- (1,3,0) -- (1,3,1) -- cycle;
\fill[gray] (2,1,0) -- (1.5,1,0.5) -- (0,1,0.5) -- (0,1,0) -- cycle;
\fill[gray] (1,3,2) -- (0.5,3,2) -- (0.5,3,3) -- (1,3,3) -- cycle;
\fill[gray] (2,3,0) -- (1.5,3,0.5) -- (1,3,0.5) -- (1,3,0) -- cycle;
\ecslogosurface
\end{tikzpicture}
}

\providecommand{\cpplogo}[1]{
\begin{tikzpicture}[scale=(#1)/512em]
\fill[gray] (435.2794,398.7159) -- (247.1911,507.3075) .. controls (236.3563,513.5642) and (218.6240,513.5642) .. (207.7892,507.3075) -- (19.7009,398.7159) .. controls (8.8646,392.4606) and (0.0000,377.1043) .. (0.0000,364.5924) -- (0.0000,147.4076) .. controls (0.8430,132.8363) and (8.2856,120.7683) .. (19.7009,113.2842) -- (207.7892,4.6926) .. controls (218.6240,-1.5642) and (236.3564,-1.5642) .. (247.1911,4.6926) -- (435.2794,113.2842) .. controls (447.5273,121.4304) and (454.4987,133.6918) .. (454.9803,147.4076) -- (454.9803,364.5924) .. controls (454.5404,377.7571) and (446.6566,391.0351) .. (435.2794,398.7159) -- cycle(75.8301,255.9993) .. controls (74.9389,404.0881) and (273.2892,469.4783) .. (358.8263,331.8769) -- (293.1917,293.8965) .. controls (253.5702,359.4301) and (155.1909,335.9977) .. (151.6601,255.9993) .. controls (152.7204,182.2703) and (249.4137,148.0211) .. (293.1961,218.1065) -- (358.8308,180.1276) .. controls (283.4477,49.2645) and (79.6318,96.3470) .. (75.8301,255.9993) -- cycle(379.1503,247.5747) -- (362.2982,247.5747) -- (362.2982,230.7226) -- (345.4490,230.7226) -- (345.4490,247.5747) -- (328.5969,247.5747) -- (328.5969,264.4254) -- (345.4490,264.4254) -- (345.4490,281.2759) -- (362.2982,281.2759) -- (362.2982,264.4254) -- (379.1503,264.4254) -- cycle(442.3420,247.5747) -- (425.4899,247.5747) -- (425.4899,230.7226) -- (408.6408,230.7226) -- (408.6408,247.5747) -- (391.7886,247.5747) -- (391.7886,264.4254) -- (408.6408,264.4254) -- (408.6408,281.2759) -- (425.4899,281.2759) -- (425.4899,264.4254) -- (442.3420,264.4254) -- cycle;
\end{tikzpicture}
}

\providecommand{\fallogo}[1]{
\begin{tikzpicture}[scale=(#1)/512em]
\fill[gray] (185.7774,0.0000) .. controls (200.4486,15.9798) and (226.8966,8.7148) .. (235.0426,31.5836) .. controls (249.5297,58.0598) and (247.9581,97.9161) .. (280.3335,110.9762) .. controls (309.1690,120.3496) and (337.8406,104.2727) .. (366.5753,103.9379) .. controls (373.4449,111.5171) and (379.2885,128.2574) .. (383.9755,108.9744) .. controls (396.6979,102.5615) and (437.2808,107.6681) .. (426.9652,124.3252) .. controls (408.9822,121.0785) and (412.4742,146.0729) .. (426.5192,131.4996) .. controls (433.8413,120.8489) and (465.1541,126.5522) .. (441.9067,135.7950) .. controls (396.1879,157.7478) and (344.1112,161.5079) .. (298.5528,183.5702) .. controls (277.7471,193.5198) and (284.6941,218.7163) .. (285.2127,236.9640) .. controls (292.3599,316.2826) and (307.3929,394.6311) .. (317.1198,473.6154) .. controls (329.0637,505.4736) and (292.1195,528.5004) .. (265.9183,511.2761) .. controls (237.9284,499.2462) and (237.3684,465.2681) .. (230.9102,439.9421) .. controls (218.6692,374.3397) and (215.6307,306.9662) .. (198.1732,242.3977) .. controls (183.1379,232.7444) and (164.4245,256.0298) .. (149.0430,261.4799) .. controls (116.9328,279.2585) and (87.1822,308.5851) .. (48.2293,307.8914) .. controls (21.3220,306.9037) and (-15.9107,281.8761) .. (7.2921,252.7908) .. controls (29.7799,220.6177) and (67.5177,204.3028) .. (100.9287,185.9449) .. controls (130.8217,170.8906) and (161.1548,156.5903) .. (191.0278,141.5847) .. controls (196.1738,120.0520) and (186.6049,95.2409) .. (186.8382,72.4353) .. controls (185.5234,48.4204) and (183.1700,23.9341) .. (185.7774,0.0000) -- cycle;
\end{tikzpicture}
}

\providecommand{\oblogo}[1]{
\begin{tikzpicture}[scale=(#1)/512em]
\fill[gray] (160.3865,208.9117) .. controls (154.0879,214.6478) and (149.0735,221.2409) .. (145.4125,228.5384) .. controls (184.8790,248.4273) and (234.7122,269.8787) .. (297.5493,291.8782) .. controls (300.3943,281.4769) and (300.9552,268.7619) .. (300.4023,255.2389) .. controls (248.9909,244.7891) and (200.0310,225.9279) .. (160.3865,208.9117) -- cycle(225.7398,392.6996) .. controls (308.0209,392.1716) and (359.3326,345.9277) .. (368.7203,285.2098) .. controls (376.6742,197.1784) and (311.7194,141.3342) .. (205.4287,142.1456) .. controls (139.9485,141.4804) and (88.7155,166.1957) .. (73.5775,228.0086) .. controls (52.0297,320.3408) and (123.4078,391.0103) .. (225.7398,392.6996) -- cycle(216.0739,176.4733) .. controls (268.9183,179.2424) and (315.8292,206.5488) .. (312.7454,265.1139) .. controls (313.2769,315.6384) and (286.5993,353.4946) .. (216.6040,355.7934) .. controls (162.4657,355.7934) and (126.0914,317.5023) .. (126.0914,260.5103) .. controls (126.1733,214.2900) and (163.3363,176.2849) .. (216.0739,176.4733) -- cycle(76.4897,189.1754) .. controls (13.1586,147.5631) and (0.0000,119.4207) .. (0.0000,119.4207) -- (90.6499,170.1632) .. controls (85.3004,175.8497) and (80.5994,182.1633) .. (76.4897,189.1754) -- cycle(353.9486,119.3004) -- (402.9482,119.3004) .. controls (427.0025,137.0797) and (450.9893,162.7034) .. (474.9529,191.0213) .. controls (509.3540,228.5339) and (531.3391,294.2091) .. (487.8149,312.1206) .. controls (462.8165,324.7652) and (394.3874,316.8943) .. (373.8912,313.6651) .. controls (379.9291,297.7449) and (383.2899,278.4204) .. (381.4989,257.7214) .. controls (420.3069,248.0321) and (421.9610,218.3461) .. (407.7867,192.6417) .. controls (391.1113,162.4018) and (370.1114,132.9097) .. (353.9486,119.3004) -- cycle;
\end{tikzpicture}
}

\providecommand{\markuptable}{
\begin{table}
\sffamily\centering
\begin{tabular}{@{}lcl@{}}
\toprule
\texttt{//italics//} & $\rightarrow$ & \textit{italics} \\
\midrule
\texttt{**bold**} & $\rightarrow$ & \textbf{bold} \\
\midrule
\texttt{\# ordered list} & & 1 ordered list \\
\texttt{\# second item} & $\rightarrow$ & 2 second item \\
\texttt{\#\# sub item} & & \hspace{1em} 1 sub item \\
\midrule
\texttt{* unordered list} & & $\bullet$ unordered list \\
\texttt{* second item} & $\rightarrow$ & $\bullet$ second item \\
\texttt{** sub item} & & \hspace{1em} $\bullet$ sub item \\
\midrule
\texttt{link to [[label]]} & $\rightarrow$ & link to \underline{label} \\
\midrule
\texttt{<{}<label>{}> definition } & $\rightarrow$ & definition \\
\midrule
\texttt{[[url|link name]]} & $\rightarrow$ & \underline{link name} \\
\midrule\addlinespace
\texttt{= large heading} & & {\Large large heading} \smallskip \\
\texttt{== medium heading} & $\rightarrow$ & {\large medium heading} \\
\texttt{=== small heading} & & small heading \\
\midrule
\texttt{no line break} & & no line break for paragraphs \\
\texttt{for paragraphs} & $\rightarrow$ \\
& & use empty line \\
\texttt{use empty line} \\
\midrule
\texttt{force\textbackslash\textbackslash line break} & $\rightarrow$ & force \\
& & line break \\
\midrule
\texttt{horizontal line} & $\rightarrow$ & horizontal line \\
\texttt{----} & & \hrulefill \\
\midrule
\texttt{|=a|=table|=header} & & \underline{a \enspace table \enspace header} \\
\texttt{|a|table|row} & $\rightarrow$ & a \enspace table \enspace row \\
\texttt{|b|table|row} & & b \enspace table \enspace row \\
\midrule
\texttt{\{\{\{} \\
\texttt{unformatted} & $\rightarrow$ & \texttt{unformatted} \\
\texttt{code} & & \texttt{code} \\
\texttt{\}\}\}} \\
\midrule\addlinespace
\texttt{@ new article} & & {\Large 1.\ new article} \smallskip \\
\texttt{@ second article} & $\rightarrow$ & {\Large 2.\ second article} \smallskip \\
\texttt{@@ sub article} & & {\large 2.1.\ sub article} \\
\bottomrule
\end{tabular}
\normalfont\caption{Elements of the generic documentation markup language}
\label{tab:docmarkup}
\end{table}
}

\providecommand{\startchapter}[4]{
\documentclass[11pt,a4paper]{article}
\usepackage{booktabs}
\usepackage[format=hang,labelfont=bf]{caption}
\usepackage{changepage}
\usepackage[T1]{fontenc}
\usepackage[margin=2cm]{geometry}
\usepackage{hyperref}
\usepackage[american]{isodate}
\usepackage{lmodern}
\usepackage{longtable}
\usepackage{mathptmx}
\usepackage{microtype}
\usepackage[toc]{multitoc}
\usepackage{multirow}
\usepackage[all]{nowidow}
\usepackage{pdfcomment}
\usepackage{syntax}
\usepackage{tikz}
\usepackage[all]{xy}
\hypersetup{pdfborder={0 0 0},bookmarksnumbered=true,pdftitle={\ecs{}: #2},pdfauthor={Florian Negele},pdfsubject={\ecs{}},pdfkeywords={#1}}
\setlength{\grammarindent}{8em}\setlength{\grammarparsep}{0.2ex}
\setlength{\columnsep}{2em}
\newcommand{\prefix}{}
\newcounter{instruction}
\bibliographystyle{unsrt}
\renewcommand{\index}[2][]{}
\renewcommand{\arraystretch}{1.05}
\renewcommand{\floatpagefraction}{0.7}
\renewcommand{\syntleft}{\itshape}\renewcommand{\syntright}{}
\title{\vspace{-5ex}\Huge{\ecs{}}\medskip\hrule}
\author{\huge{#2}}
\date{\medskip\version}
\newif\ifbook\bookfalse
\pagestyle{headings}
\frenchspacing
\begin{document}
\maketitle\thispagestyle{empty}\noindent#4\setlength{\columnseprule}{0.4pt}\tableofcontents\setlength{\columnseprule}{0pt}\vfill\pagebreak[3]\null\vfill\bigskip\noindent
\parbox{\textwidth-4em}{\license The contents of this \documentation{} are part of the \href{manual}{\ecs{} User Manual}~\cite{manual} and correspond to Chapter ``\href{manual\##3}{#1}''.\alignright\mbox{\today}}
\parbox{4em}{\flushright\ecslogo{3em}}
\clearpage
}

\providecommand{\concludechapter}{
\vfill\pagebreak[3]\null\vfill
\thispagestyle{myheadings}\markright{REFERENCES}
\noindent\begin{minipage}{\textwidth}\begin{multicols}{2}[\section*{References}]
\renewcommand{\section}[2]{}\small\bibliography{references}
\end{multicols}\end{minipage}\end{document}
}

\providecommand{\startpresentation}[2]{
\documentclass[14pt,aspectratio=43,usepdftitle=false]{beamer}
\usepackage{booktabs}
\usepackage{etex}
\usepackage{multicol}
\usepackage{tikz}
\usepackage[all]{xy}
\bibliographystyle{unsrt}
\setlength{\columnsep}{1em}
\setlength{\leftmargini}{1em}
\setbeamercolor{title}{fg=black}
\setbeamercolor{structure}{fg=darkgray}
\setbeamercolor{bibliography item}{fg=darkgray}
\setbeamerfont{title}{series=\bfseries}
\setbeamerfont{subtitle}{series=\normalfont}
\setbeamerfont*{frametitle}{parent=title}
\setbeamerfont{block title}{series=\bfseries}
\setbeamerfont*{framesubtitle}{parent=subtitle}
\setbeamersize{text margin left=1em,text margin right=1em}
\setbeamertemplate{navigation symbols}{}
\setbeamertemplate{itemize item}[circle]{}
\setbeamertemplate{bibliography item}[triangle]{}
\setbeamertemplate{bibliography entry author}{\usebeamercolor[fg]{bibliography item}}
\setbeamertemplate{frametitle}{\medskip\usebeamerfont{frametitle}\color{gray}\raisebox{-2.5ex}[0ex][0ex]{\rule{0.1em}{4.5ex}}}
\addtobeamertemplate{frametitle}{}{\hspace{0.4em}\usebeamercolor[fg]{title}\insertframetitle\par\vspace{0.2ex}\hspace{0.5em}\usebeamerfont{framesubtitle}\insertframesubtitle}
\hypersetup{pdfborder={0 0 0},bookmarksnumbered=true,bookmarksopen=true,bookmarksopenlevel=0,pdftitle={\ecs{}: #1},pdfauthor={Florian Negele},pdfsubject={\ecs{}},pdfkeywords={#1}}
\renewcommand{\flowgraph}[1]{\resizebox{\textwidth}{!}{$$\xymatrix{##1}$$}}
\title{\ecs{}\medskip\hrule\medskip}
\institute{\shadowedecslogo{5em}{30}{15}}
\date{\version}
\subtitle{#1}
\begin{document}
\begin{frame}[plain]\titlepage\nocite{manual}\end{frame}
\begin{frame}{Contents}{#1}\begin{center}\tableofcontents\end{center}\end{frame}
}

\providecommand{\concludepresentation}{
\begin{frame}{References}\begin{footnotesize}\setlength{\columnseprule}{0.4pt}\begin{multicols}{2}\bibliography{references}\end{multicols}\end{footnotesize}\end{frame}
\end{document}
}

\providecommand{\startbook}[1]{
\documentclass[10pt,paper=17cm:24cm,DIV=13,twoside=semi,headings=normal,numbers=noendperiod,cleardoublepage=plain]{scrbook}
\usepackage{atveryend}
\usepackage{booktabs}
\usepackage{caption}
\usepackage{changepage}
\usepackage[T1]{fontenc}
\usepackage{imakeidx}
\usepackage{hyperref}
\usepackage[american]{isodate}
\usepackage{lmodern}
\usepackage{longtable}
\usepackage{mathptmx}
\usepackage[final]{microtype}
\usepackage{multicol}
\usepackage{multirow}
\usepackage[all]{nowidow}
\usepackage{pdfcomment}
\usepackage{scrlayer-scrpage}
\usepackage{setspace}
\usepackage{syntax}
\usepackage[eventxtindent=4pt,oddtxtexdent=4pt]{thumbs}
\usepackage{tikz}
\usepackage[all]{xy}
\hyphenation{Micro-Blaze Open-Cores Open-RISC Power-PC}
\hypersetup{pdfborder={0 0 0},bookmarksnumbered=true,bookmarksopen=true,bookmarksopenlevel=0,pdftitle={\ecs{}: #1},pdfauthor={Florian Negele},pdfsubject={\ecs{}},pdfkeywords={#1}}
\setlength{\grammarindent}{8em}\setlength{\grammarparsep}{0.7ex}
\setkomafont{captionlabel}{\usekomafont{descriptionlabel}}
\renewcommand{\arraystretch}{1.05}\setstretch{1.1}
\renewcommand{\chapterformat}{\thechapter\autodot\enskip\raisebox{-1ex}[0ex][0ex]{\color{gray}\rule{0.1em}{3.5ex}}\enskip}
\renewcommand{\startchapter}[4]{\hypertarget{##3}{\chapter{##1}}\label{##3}##4\addthumb{##1}{\LARGE\sffamily\bfseries\thechapter}{white}{gray}\renewcommand{\prefix}{##3}}
\renewcommand{\concludechapter}{\clearpage{\stopthumb\cleardoublepage}}
\renewcommand{\syntleft}{\itshape}\renewcommand{\syntright}{}
\renewcommand{\floatpagefraction}{0.7}
\renewcommand{\partheademptypage}{}
\DeclareMicrotypeAlias{lmss}{cmr}
\newcommand{\prefix}{}
\newcounter{instruction}
\bibliographystyle{unsrt}
\newif\ifbook\booktrue
\makeindex[intoc,title=Index]
\makeindex[intoc,name=tools,title=Index of Tools,columns=3]
\makeindex[intoc,name=library,title=Index of Library Names]
\makeindex[intoc,name=runtime,title=Index of Runtime Support]
\makeindex[intoc,name=environment,title=Index of Target Environments]
\indexsetup{toclevel=chapter,headers={\indexname}{\indexname}}
\frenchspacing
\begin{document}
\pagenumbering{alph}
\begin{titlepage}\centering
\huge\sffamily\null\vfill\textbf{\ecs{}}\bigskip\hrule\bigskip#1
\normalsize\normalfont\vfill\vfill\shadowedecslogo{10em}{30}{15}
\large\vfill\vfill\version
\end{titlepage}
\null\vfill
\thispagestyle{empty}
\noindent\today\par\medskip
\license A copy of this license is included in Appendix~\ref{fdl} on page~\pageref{fdl}.
All product names used herein are for identification purposes only and may be trademarks of their respective companies.
\concludechapter
\frontmatter
\setcounter{tocdepth}{1}
\tableofcontents
\setcounter{tocdepth}{2}
\concludechapter
\listoffigures
\concludechapter
\listoftables
\concludechapter
}

\providecommand{\concludebook}{
\backmatter
\addtocontents{toc}{\protect\setcounter{tocdepth}{-1}}
\phantomsection\addcontentsline{toc}{part}{Bibliography}
\bibliography{references}
\concludechapter
\phantomsection\addcontentsline{toc}{part}{Indexes}
\printindex
\concludechapter
\indexprologue{\label{idx:tools}}
\printindex[tools]
\concludechapter
\printindex[library]
\concludechapter
\indexprologue{\label{idx:runtime}}
\printindex[runtime]
\concludechapter
\indexprologue{\label{idx:environment}}
\printindex[environment]
\concludechapter
\pagestyle{empty}\pagenumbering{Alph}\null\clearpage
\null\vfill\centering\ecslogo{4em}\par\medskip\license
\end{document}
}

% chapter references

\providecommand{\seedocumentationref}{}\renewcommand{\seedocumentationref}[3]{#1, see \Documentation{}~\documentationref{#2}{#3}. }
\providecommand{\seeinterface}{}\renewcommand{\seeinterface}{\ifbook See \Documentation{}~\documentationref{interface}{User Interface} for more information about the common user interface of all of these tools. \fi}
\providecommand{\seeguide}{}\renewcommand{\seeguide}{\seedocumentationref{For basic examples of using some of these tools in practice}{guide}{User Guide}}
\providecommand{\seecpp}{}\renewcommand{\seecpp}{\seedocumentationref{For more information about the \cpp{} programming language and its implementation by the \ecs{}}{cpp}{User Manual for \cpp{}}}
\providecommand{\seefalse}{}\renewcommand{\seefalse}{\seedocumentationref{For more information about the FALSE programming language and its implementation by the \ecs{}}{false}{User Manual for FALSE}}
\providecommand{\seeoberon}{}\renewcommand{\seeoberon}{\seedocumentationref{For more information about the Oberon programming language and its implementation by the \ecs{}}{oberon}{User Manual for Oberon}}
\providecommand{\seeassembly}{}\renewcommand{\seeassembly}{\seedocumentationref{For more information about the generic assembly language and how to use it}{assembly}{Generic Assembly Language Specification}}
\providecommand{\seeamd}{}\renewcommand{\seeamd}{\seedocumentationref{For more information about how the \ecs{} supports the AMD64 hardware architecture}{amd64}{AMD64 Hardware Architecture Support}}
\providecommand{\seearm}{}\renewcommand{\seearm}{\seedocumentationref{For more information about how the \ecs{} supports the ARM hardware architecture}{arm}{ARM Hardware Architecture Support}}
\providecommand{\seeavr}{}\renewcommand{\seeavr}{\seedocumentationref{For more information about how the \ecs{} supports the AVR hardware architecture}{avr}{AVR Hardware Architecture Support}}
\providecommand{\seeavrtt}{}\renewcommand{\seeavrtt}{\seedocumentationref{For more information about how the \ecs{} supports the AVR32 hardware architecture}{avr32}{AVR32 Hardware Architecture Support}}
\providecommand{\seemabk}{}\renewcommand{\seemabk}{\seedocumentationref{For more information about how the \ecs{} supports the M68000 hardware architecture}{m68k}{M68000 Hardware Architecture Support}}
\providecommand{\seemibl}{}\renewcommand{\seemibl}{\seedocumentationref{For more information about how the \ecs{} supports the MicroBlaze hardware architecture}{mibl}{MicroBlaze Hardware Architecture Support}}
\providecommand{\seemips}{}\renewcommand{\seemips}{\seedocumentationref{For more information about how the \ecs{} supports the MIPS32 and MIPS64 hardware architectures}{mips}{MIPS Hardware Architecture Support}}
\providecommand{\seemmix}{}\renewcommand{\seemmix}{\seedocumentationref{For more information about how the \ecs{} supports the MMIX hardware architecture}{mmix}{MMIX Hardware Architecture Support}}
\providecommand{\seeorok}{}\renewcommand{\seeorok}{\seedocumentationref{For more information about how the \ecs{} supports the OpenRISC 1000 hardware architecture}{or1k}{OpenRISC 1000 Hardware Architecture Support}}
\providecommand{\seeppc}{}\renewcommand{\seeppc}{\seedocumentationref{For more information about how the \ecs{} supports the PowerPC hardware architecture}{ppc}{PowerPC Hardware Architecture Support}}
\providecommand{\seerisc}{}\renewcommand{\seerisc}{\seedocumentationref{For more information about how the \ecs{} supports the RISC hardware architecture}{risc}{RISC Hardware Architecture Support}}
\providecommand{\seewasm}{}\renewcommand{\seewasm}{\seedocumentationref{For more information about how the \ecs{} supports the WebAssembly architecture}{wasm}{WebAssembly Architecture Support}}
\providecommand{\seedocumentation}{}\renewcommand{\seedocumentation}{\seedocumentationref{For more information about generic documentations and their generation by the \ecs{}}{documentation}{Generic Documentation Generation}}
\providecommand{\seedebugging}{}\renewcommand{\seedebugging}{\seedocumentationref{For more information about debugging information and its representation}{debugging}{Debugging Information Representation}}
\providecommand{\seecode}{}\renewcommand{\seecode}{\seedocumentationref{For more information about intermediate code and its purpose}{code}{Intermediate Code Representation}}
\providecommand{\seeobject}{}\renewcommand{\seeobject}{\seedocumentationref{For more information about object files and their purpose}{object}{Object File Representation}}

% generic documentation tools

\providecommand{\docprint}{
\toolsection{docprint} is a pretty printer for generic documentations.
It reformats generic documentations and writes it to the standard output stream.
\debuggingtool
\flowgraph{\resource{generic\\documentation} \ar[r] & \toolbox{docprint} \ar[r] & \resource{generic\\documentation}}
\seedocumentation
}

\providecommand{\doccheck}{
\toolsection{doccheck} is a syntactic and semantic checker for generic documentations.
It just performs syntactic and semantic checks on generic documentations and writes its diagnostic messages to the standard error stream.
\debuggingtool
\flowgraph{\resource{generic\\documentation} \ar[r] & \toolbox{doccheck} \ar[r] & \resource{diagnostic\\messages}}
\seedocumentation
}

\providecommand{\dochtml}{
\toolsection{dochtml} is an HTML documentation generator for generic documentations.
It processes several generic documentations and assembles all information therein into an HTML document.
\debuggingtool
\flowgraph{\resource{generic\\documentation} \ar[r] & \toolbox{dochtml} \ar[r] & \resource{HTML\\document}}
\seedocumentation
}

\providecommand{\doclatex}{
\toolsection{doclatex} is a Latex documentation generator for generic documentations.
It processes several generic documentations and assembles all information therein into a Latex document.
\debuggingtool
\flowgraph{\resource{generic\\documentation} \ar[r] & \toolbox{doclatex} \ar[r] & \resource{Latex\\document}}
\seedocumentation
}

% intermediate code tools

\providecommand{\cdcheck}{
\toolsection{cdcheck} is a syntactic and semantic checker for intermediate code.
It just performs syntactic and semantic checks on programs written in intermediate code and writes its diagnostic messages to the standard error stream.
\debuggingtool
\flowgraph{\resource{intermediate\\code} \ar[r] & \toolbox{cdcheck} \ar[r] & \resource{diagnostic\\messages}}
\seeassembly\seecode
}

\providecommand{\cdopt}{
\toolsection{cdopt} is an optimizer for intermediate code.
It performs various optimizations on programs written in intermediate code and writes the result to the standard output stream.
\debuggingtool
\flowgraph{\resource{intermediate\\code} \ar[r] & \toolbox{cdopt} \ar[r] & \resource{optimized\\code}}
\seeassembly\seecode
}

\providecommand{\cdrun}{
\toolsection{cdrun} is an interpreter for intermediate code.
It processes and executes programs written in intermediate code.
The following code sections are predefined and have the usual semantics:
\texttt{abort}, \texttt{\_Exit}, \texttt{fflush}, \texttt{floor}, \texttt{fputc}, \texttt{free}, \texttt{getchar}, \texttt{malloc}, and \texttt{putchar}.
Diagnostic messages about invalid operations include the name of the executed code section and the index of the erroneous instruction.
\debuggingtool
\flowgraph{\resource{intermediate\\code} \ar[r] & \toolbox{cdrun} \ar@/u/[r] & \resource{input/\\output} \ar@/d/[l]}
\seeassembly\seecode
}

\providecommand{\cdamda}{
\toolsection{cdamd16} is a compiler for intermediate code targeting the AMD64 hardware architecture.
It generates machine code for AMD64 processors from programs written in intermediate code and stores it in corresponding object files.
The compiler generates machine code for the 16-bit operating mode defined by the AMD64 architecture.
It also creates a debugging information file as well as an assembly file containing a listing of the generated machine code.
\debuggingtool
\flowgraph{\resource{intermediate\\code} \ar[r] & \toolbox{cdamd16} \ar[r] \ar[d] \ar[rd] & \resource{object file} \\ & \resource{assembly\\listing} & \resource{debugging\\information}}
\seeassembly\seeamd\seeobject\seecode\seedebugging
}

\providecommand{\cdamdb}{
\toolsection{cdamd32} is a compiler for intermediate code targeting the AMD64 hardware architecture.
It generates machine code for AMD64 processors from programs written in intermediate code and stores it in corresponding object files.
The compiler generates machine code for the 32-bit operating mode defined by the AMD64 architecture.
It also creates a debugging information file as well as an assembly file containing a listing of the generated machine code.
\debuggingtool
\flowgraph{\resource{intermediate\\code} \ar[r] & \toolbox{cdamd32} \ar[r] \ar[d] \ar[rd] & \resource{object file} \\ & \resource{assembly\\listing} & \resource{debugging\\information}}
\seeassembly\seeamd\seeobject\seecode\seedebugging
}

\providecommand{\cdamdc}{
\toolsection{cdamd64} is a compiler for intermediate code targeting the AMD64 hardware architecture.
It generates machine code for AMD64 processors from programs written in intermediate code and stores it in corresponding object files.
The compiler generates machine code for the 64-bit operating mode defined by the AMD64 architecture.
It also creates a debugging information file as well as an assembly file containing a listing of the generated machine code.
\debuggingtool
\flowgraph{\resource{intermediate\\code} \ar[r] & \toolbox{cdamd64} \ar[r] \ar[d] \ar[rd] & \resource{object file} \\ & \resource{assembly\\listing} & \resource{debugging\\information}}
\seeassembly\seeamd\seeobject\seecode\seedebugging
}

\providecommand{\cdarma}{
\toolsection{cdarma32} is a compiler for intermediate code targeting the ARM hardware architecture.
It generates machine code for ARM processors executing A32 instructions from programs written in intermediate code and stores it in corresponding object files.
It also creates a debugging information file as well as an assembly file containing a listing of the generated machine code.
\debuggingtool
\flowgraph{\resource{intermediate\\code} \ar[r] & \toolbox{cdarma32} \ar[r] \ar[d] \ar[rd] & \resource{object file} \\ & \resource{assembly\\listing} & \resource{debugging\\information}}
\seeassembly\seearm\seeobject\seecode\seedebugging
}

\providecommand{\cdarmb}{
\toolsection{cdarma64} is a compiler for intermediate code targeting the ARM hardware architecture.
It generates machine code for ARM processors executing A64 instructions from programs written in intermediate code and stores it in corresponding object files.
It also creates a debugging information file as well as an assembly file containing a listing of the generated machine code.
\debuggingtool
\flowgraph{\resource{intermediate\\code} \ar[r] & \toolbox{cdarma64} \ar[r] \ar[d] \ar[rd] & \resource{object file} \\ & \resource{assembly\\listing} & \resource{debugging\\information}}
\seeassembly\seearm\seeobject\seecode\seedebugging
}

\providecommand{\cdarmc}{
\toolsection{cdarmt32} is a compiler for intermediate code targeting the ARM hardware architecture.
It generates machine code for ARM processors without floating-point extension executing T32 instructions from programs written in intermediate code and stores it in corresponding object files.
It also creates a debugging information file as well as an assembly file containing a listing of the generated machine code.
\debuggingtool
\flowgraph{\resource{intermediate\\code} \ar[r] & \toolbox{cdarmt32} \ar[r] \ar[d] \ar[rd] & \resource{object file} \\ & \resource{assembly\\listing} & \resource{debugging\\information}}
\seeassembly\seearm\seeobject\seecode\seedebugging
}

\providecommand{\cdarmcfpe}{
\toolsection{cdarmt32fpe} is a compiler for intermediate code targeting the ARM hardware architecture.
It generates machine code for ARM processors with floating-point extension executing T32 instructions from programs written in intermediate code and stores it in corresponding object files.
It also creates a debugging information file as well as an assembly file containing a listing of the generated machine code.
\debuggingtool
\flowgraph{\resource{intermediate\\code} \ar[r] & \toolbox{cdarmt32fpe} \ar[r] \ar[d] \ar[rd] & \resource{object file} \\ & \resource{assembly\\listing} & \resource{debugging\\information}}
\seeassembly\seearm\seeobject\seecode\seedebugging
}

\providecommand{\cdavr}{
\toolsection{cdavr} is a compiler for intermediate code targeting the AVR hardware architecture.
It generates machine code for AVR processors from programs written in intermediate code and stores it in corresponding object files.
It also creates a debugging information file as well as an assembly file containing a listing of the generated machine code.
\debuggingtool
\flowgraph{\resource{intermediate\\code} \ar[r] & \toolbox{cdavr} \ar[r] \ar[d] \ar[rd] & \resource{object file} \\ & \resource{assembly\\listing} & \resource{debugging\\information}}
\seeassembly\seeavr\seeobject\seecode\seedebugging
}

\providecommand{\cdavrtt}{
\toolsection{cdavr32} is a compiler for intermediate code targeting the AVR32 hardware architecture.
It generates machine code for AVR32 processors from programs written in intermediate code and stores it in corresponding object files.
It also creates a debugging information file as well as an assembly file containing a listing of the generated machine code.
\debuggingtool
\flowgraph{\resource{intermediate\\code} \ar[r] & \toolbox{cdavr32} \ar[r] \ar[d] \ar[rd] & \resource{object file} \\ & \resource{assembly\\listing} & \resource{debugging\\information}}
\seeassembly\seeavrtt\seeobject\seecode\seedebugging
}

\providecommand{\cdmabk}{
\toolsection{cdm68k} is a compiler for intermediate code targeting the M68000 hardware architecture.
It generates machine code for M68000 processors from programs written in intermediate code and stores it in corresponding object files.
It also creates a debugging information file as well as an assembly file containing a listing of the generated machine code.
\debuggingtool
\flowgraph{\resource{intermediate\\code} \ar[r] & \toolbox{cdm68k} \ar[r] \ar[d] \ar[rd] & \resource{object file} \\ & \resource{assembly\\listing} & \resource{debugging\\information}}
\seeassembly\seemabk\seeobject\seecode\seedebugging
}

\providecommand{\cdmibl}{
\toolsection{cdmibl} is a compiler for intermediate code targeting the MicroBlaze hardware architecture.
It generates machine code for MicroBlaze processors from programs written in intermediate code and stores it in corresponding object files.
It also creates a debugging information file as well as an assembly file containing a listing of the generated machine code.
\debuggingtool
\flowgraph{\resource{intermediate\\code} \ar[r] & \toolbox{cdmibl} \ar[r] \ar[d] \ar[rd] & \resource{object file} \\ & \resource{assembly\\listing} & \resource{debugging\\information}}
\seeassembly\seemibl\seeobject\seecode\seedebugging
}

\providecommand{\cdmipsa}{
\toolsection{cdmips32} is a compiler for intermediate code targeting the MIPS32 hardware architecture.
It generates machine code for MIPS32 processors from programs written in intermediate code and stores it in corresponding object files.
It also creates a debugging information file as well as an assembly file containing a listing of the generated machine code.
\debuggingtool
\flowgraph{\resource{intermediate\\code} \ar[r] & \toolbox{cdmips32} \ar[r] \ar[d] \ar[rd] & \resource{object file} \\ & \resource{assembly\\listing} & \resource{debugging\\information}}
\seeassembly\seemips\seeobject\seecode\seedebugging
}

\providecommand{\cdmipsb}{
\toolsection{cdmips64} is a compiler for intermediate code targeting the MIPS64 hardware architecture.
It generates machine code for MIPS64 processors from programs written in intermediate code and stores it in corresponding object files.
It also creates a debugging information file as well as an assembly file containing a listing of the generated machine code.
\debuggingtool
\flowgraph{\resource{intermediate\\code} \ar[r] & \toolbox{cdmips64} \ar[r] \ar[d] \ar[rd] & \resource{object file} \\ & \resource{assembly\\listing} & \resource{debugging\\information}}
\seeassembly\seemips\seeobject\seecode\seedebugging
}

\providecommand{\cdmmix}{
\toolsection{cdmmix} is a compiler for intermediate code targeting the MMIX hardware architecture.
It generates machine code for MMIX processors from programs written in intermediate code and stores it in corresponding object files.
It also creates a debugging information file as well as an assembly file containing a listing of the generated machine code.
\debuggingtool
\flowgraph{\resource{intermediate\\code} \ar[r] & \toolbox{cdmmix} \ar[r] \ar[d] \ar[rd] & \resource{object file} \\ & \resource{assembly\\listing} & \resource{debugging\\information}}
\seeassembly\seemmix\seeobject\seecode\seedebugging
}

\providecommand{\cdorok}{
\toolsection{cdor1k} is a compiler for intermediate code targeting the OpenRISC 1000 hardware architecture.
It generates machine code for OpenRISC 1000 processors from programs written in intermediate code and stores it in corresponding object files.
It also creates a debugging information file as well as an assembly file containing a listing of the generated machine code.
\debuggingtool
\flowgraph{\resource{intermediate\\code} \ar[r] & \toolbox{cdor1k} \ar[r] \ar[d] \ar[rd] & \resource{object file} \\ & \resource{assembly\\listing} & \resource{debugging\\information}}
\seeassembly\seeorok\seeobject\seecode\seedebugging
}

\providecommand{\cdppca}{
\toolsection{cdppc32} is a compiler for intermediate code targeting the PowerPC hardware architecture.
It generates machine code for PowerPC processors from programs written in intermediate code and stores it in corresponding object files.
The compiler generates machine code for the 32-bit operating mode defined by the PowerPC architecture.
It also creates a debugging information file as well as an assembly file containing a listing of the generated machine code.
\debuggingtool
\flowgraph{\resource{intermediate\\code} \ar[r] & \toolbox{cdppc32} \ar[r] \ar[d] \ar[rd] & \resource{object file} \\ & \resource{assembly\\listing} & \resource{debugging\\information}}
\seeassembly\seeppc\seeobject\seecode\seedebugging
}

\providecommand{\cdppcb}{
\toolsection{cdppc64} is a compiler for intermediate code targeting the PowerPC hardware architecture.
It generates machine code for PowerPC processors from programs written in intermediate code and stores it in corresponding object files.
The compiler generates machine code for the 64-bit operating mode defined by the PowerPC architecture.
It also creates a debugging information file as well as an assembly file containing a listing of the generated machine code.
\debuggingtool
\flowgraph{\resource{intermediate\\code} \ar[r] & \toolbox{cdppc64} \ar[r] \ar[d] \ar[rd] & \resource{object file} \\ & \resource{assembly\\listing} & \resource{debugging\\information}}
\seeassembly\seeppc\seeobject\seecode\seedebugging
}

\providecommand{\cdrisc}{
\toolsection{cdrisc} is a compiler for intermediate code targeting the RISC hardware architecture.
It generates machine code for RISC processors from programs written in intermediate code and stores it in corresponding object files.
It also creates a debugging information file as well as an assembly file containing a listing of the generated machine code.
\debuggingtool
\flowgraph{\resource{intermediate\\code} \ar[r] & \toolbox{cdrisc} \ar[r] \ar[d] \ar[rd] & \resource{object file} \\ & \resource{assembly\\listing} & \resource{debugging\\information}}
\seeassembly\seerisc\seeobject\seecode\seedebugging
}

\providecommand{\cdwasm}{
\toolsection{cdwasm} is a compiler for intermediate code targeting the WebAssembly architecture.
It generates machine code for WebAssembly targets from programs written in intermediate code and stores it in corresponding object files.
It also creates a debugging information file as well as an assembly file containing a listing of the generated machine code.
\debuggingtool
\flowgraph{\resource{intermediate\\code} \ar[r] & \toolbox{cdwasm} \ar[r] \ar[d] \ar[rd] & \resource{object file} \\ & \resource{assembly\\listing} & \resource{debugging\\information}}
\seeassembly\seewasm\seeobject\seecode\seedebugging
}

% C++ tools

\providecommand{\cppprep}{
\toolsection{cppprep} is a preprocessor for the \cpp{} programming language.
It preprocesses source code according to the rules of \cpp{} and writes it to the standard output stream.
Only the macro names \texttt{\_\_DATE\_\_}, \texttt{\_\_FILE\_\_}, \texttt{\_\_LINE\_\_}, and \texttt{\_\_TIME\_\_} are predefined.
\flowgraph{\resource{\cpp{} or other\\source code} \ar[r] & \toolbox{cppprep} \ar[r] & \resource{preprocessed\\source code} \\ & \variable{ECSINCLUDE} \ar[u]}
\seecpp
}

\providecommand{\cppprint}{
\toolsection{cppprint} is a pretty printer for the \cpp{} programming language.
It reformats the source code of \cpp{} programs and writes it to the standard output stream.
\flowgraph{\resource{\cpp{}\\source code} \ar[r] & \toolbox{cppprint} \ar[r] & \resource{reformatted\\source code} \\ & \variable{ECSINCLUDE} \ar[u]}
\seecpp
}

\providecommand{\cppcheck}{
\toolsection{cppcheck} is a syntactic and semantic checker for the \cpp{} programming language.
It just performs syntactic and semantic checks on \cpp{} programs and writes its diagnostic messages to the standard error stream.
\flowgraph{\resource{\cpp{}\\source code} \ar[r] & \toolbox{cppcheck} \ar[r] & \resource{diagnostic\\messages} \\ & \variable{ECSINCLUDE} \ar[u]}
\seecpp
}

\providecommand{\cppdump}{
\toolsection{cppdump} is a serializer for the \cpp{} programming language.
It dumps the complete internal representation of programs written in \cpp{} into an XML document.
\debuggingtool
\flowgraph{\resource{\cpp{}\\source code} \ar[r] & \toolbox{cppdump} \ar[r] & \resource{internal\\representation} \\ & \variable{ECSINCLUDE} \ar[u]}
\seecpp
}

\providecommand{\cpprun}{
\toolsection{cpprun} is an interpreter for the \cpp{} programming language.
It processes and executes programs written in \cpp{}.
The macro \texttt{\_\_run\_\_} is predefined in order to enable programmers to identify this tool while interpreting.
\flowgraph{\resource{\cpp{}\\source code} \ar[r] & \toolbox{cpprun} \ar@/u/[r] & \resource{input/\\output} \ar@/d/[l] \\ & \variable{ECSINCLUDE} \ar[u]}
\seecpp
}

\providecommand{\cppdoc}{
\toolsection{cppdoc} is a generic documentation generator for the \cpp{} programming language.
It processes several \cpp{} source files and assembles all information therein into a generic documentation.
\debuggingtool
\flowgraph{\resource{\cpp{}\\source code} \ar[r] & \toolbox{cppdoc} \ar[r] & \resource{generic\\documentation} \\ & \variable{ECSINCLUDE} \ar[u]}
\seecpp\seedocumentation
}

\providecommand{\cpphtml}{
\toolsection{cpphtml} is an HTML documentation generator for the \cpp{} programming language.
It processes several \cpp{} source files and assembles all information therein into an HTML document.
\flowgraph{\resource{\cpp{}\\source code} \ar[r] & \toolbox{cpphtml} \ar[r] & \resource{HTML\\document} \\ & \variable{ECSINCLUDE} \ar[u]}
\seecpp\seedocumentation
}

\providecommand{\cpplatex}{
\toolsection{cpplatex} is a Latex documentation generator for the \cpp{} programming language.
It processes several \cpp{} source files and assembles all information therein into a Latex document.
\flowgraph{\resource{\cpp{}\\source code} \ar[r] & \toolbox{cpplatex} \ar[r] & \resource{Latex\\document} \\ & \variable{ECSINCLUDE} \ar[u]}
\seecpp\seedocumentation
}

\providecommand{\cppcode}{
\toolsection{cppcode} is an intermediate code generator for the \cpp{} programming language.
It generates intermediate code from programs written in \cpp{} and stores it in corresponding assembly files.
The macro \texttt{\_\_code\_\_} is predefined in order to enable programmers to identify this tool while generating intermediate code.
Programs generated with this tool require additional runtime support that is stored in the \file{cpp\-code\-run} library file.
\debuggingtool
\flowgraph{\resource{\cpp{}\\source code} \ar[r] & \toolbox{cppcode} \ar[r] & \resource{intermediate\\code} \\ & \variable{ECSINCLUDE} \ar[u]}
\seecpp\seeassembly\seecode
}

\providecommand{\cppamda}{
\toolsection{cppamd16} is a compiler for the \cpp{} programming language targeting the AMD64 hardware architecture.
It generates machine code for AMD64 processors from programs written in \cpp{} and stores it in corresponding object files.
The compiler generates machine code for the 16-bit operating mode defined by the AMD64 architecture.
For debugging purposes, it also creates a debugging information file as well as an assembly file containing a listing of the generated machine code.
The macro \texttt{\_\_amd16\_\_} is predefined in order to enable programmers to identify this tool and its target architecture while compiling.
Programs generated with this compiler require additional runtime support that is stored in the \file{cpp\-amd16\-run} library file.
\flowgraph{\resource{\cpp{}\\source code} \ar[r] & \toolbox{cppamd16} \ar[r] \ar[d] \ar[rd] & \resource{object file} \\ \variable{ECSINCLUDE} \ar[ru] & \resource{debugging\\information} & \resource{assembly\\listing}}
\seecpp\seeassembly\seeamd\seeobject\seedebugging
}

\providecommand{\cppamdb}{
\toolsection{cppamd32} is a compiler for the \cpp{} programming language targeting the AMD64 hardware architecture.
It generates machine code for AMD64 processors from programs written in \cpp{} and stores it in corresponding object files.
The compiler generates machine code for the 32-bit operating mode defined by the AMD64 architecture.
For debugging purposes, it also creates a debugging information file as well as an assembly file containing a listing of the generated machine code.
The macro \texttt{\_\_amd32\_\_} is predefined in order to enable programmers to identify this tool and its target architecture while compiling.
Programs generated with this compiler require additional runtime support that is stored in the \file{cpp\-amd32\-run} library file.
\flowgraph{\resource{\cpp{}\\source code} \ar[r] & \toolbox{cppamd32} \ar[r] \ar[d] \ar[rd] & \resource{object file} \\ \variable{ECSINCLUDE} \ar[ru] & \resource{debugging\\information} & \resource{assembly\\listing}}
\seecpp\seeassembly\seeamd\seeobject\seedebugging
}

\providecommand{\cppamdc}{
\toolsection{cppamd64} is a compiler for the \cpp{} programming language targeting the AMD64 hardware architecture.
It generates machine code for AMD64 processors from programs written in \cpp{} and stores it in corresponding object files.
The compiler generates machine code for the 64-bit operating mode defined by the AMD64 architecture.
For debugging purposes, it also creates a debugging information file as well as an assembly file containing a listing of the generated machine code.
The macro \texttt{\_\_amd64\_\_} is predefined in order to enable programmers to identify this tool and its target architecture while compiling.
Programs generated with this compiler require additional runtime support that is stored in the \file{cpp\-amd64\-run} library file.
\flowgraph{\resource{\cpp{}\\source code} \ar[r] & \toolbox{cppamd64} \ar[r] \ar[d] \ar[rd] & \resource{object file} \\ \variable{ECSINCLUDE} \ar[ru] & \resource{debugging\\information} & \resource{assembly\\listing}}
\seecpp\seeassembly\seeamd\seeobject\seedebugging
}

\providecommand{\cpparma}{
\toolsection{cpparma32} is a compiler for the \cpp{} programming language targeting the ARM hardware architecture.
It generates machine code for ARM processors executing A32 instructions from programs written in \cpp{} and stores it in corresponding object files.
For debugging purposes, it also creates a debugging information file as well as an assembly file containing a listing of the generated machine code.
The macro \texttt{\_\_arma32\_\_} is predefined in order to enable programmers to identify this tool and its target architecture while compiling.
Programs generated with this compiler require additional runtime support that is stored in the \file{cpp\-arma32\-run} library file.
\flowgraph{\resource{\cpp{}\\source code} \ar[r] & \toolbox{cpparma32} \ar[r] \ar[d] \ar[rd] & \resource{object file} \\ \variable{ECSINCLUDE} \ar[ru] & \resource{debugging\\information} & \resource{assembly\\listing}}
\seecpp\seeassembly\seearm\seeobject\seedebugging
}

\providecommand{\cpparmb}{
\toolsection{cpparma64} is a compiler for the \cpp{} programming language targeting the ARM hardware architecture.
It generates machine code for ARM processors executing A64 instructions from programs written in \cpp{} and stores it in corresponding object files.
For debugging purposes, it also creates a debugging information file as well as an assembly file containing a listing of the generated machine code.
The macro \texttt{\_\_arma64\_\_} is predefined in order to enable programmers to identify this tool and its target architecture while compiling.
Programs generated with this compiler require additional runtime support that is stored in the \file{cpp\-arma64\-run} library file.
\flowgraph{\resource{\cpp{}\\source code} \ar[r] & \toolbox{cpparma64} \ar[r] \ar[d] \ar[rd] & \resource{object file} \\ \variable{ECSINCLUDE} \ar[ru] & \resource{debugging\\information} & \resource{assembly\\listing}}
\seecpp\seeassembly\seearm\seeobject\seedebugging
}

\providecommand{\cpparmc}{
\toolsection{cpparmt32} is a compiler for the \cpp{} programming language targeting the ARM hardware architecture.
It generates machine code for ARM processors without floating-point extension executing T32 instructions from programs written in \cpp{} and stores it in corresponding object files.
For debugging purposes, it also creates a debugging information file as well as an assembly file containing a listing of the generated machine code.
The macro \texttt{\_\_armt32\_\_} is predefined in order to enable programmers to identify this tool and its target architecture while compiling.
Programs generated with this compiler require additional runtime support that is stored in the \file{cpp\-armt32\-run} library file.
\flowgraph{\resource{\cpp{}\\source code} \ar[r] & \toolbox{cpparmt32} \ar[r] \ar[d] \ar[rd] & \resource{object file} \\ \variable{ECSINCLUDE} \ar[ru] & \resource{debugging\\information} & \resource{assembly\\listing}}
\seecpp\seeassembly\seearm\seeobject\seedebugging
}

\providecommand{\cpparmcfpe}{
\toolsection{cpparmt32fpe} is a compiler for the \cpp{} programming language targeting the ARM hardware architecture.
It generates machine code for ARM processors with floating-point extension executing T32 instructions from programs written in \cpp{} and stores it in corresponding object files.
For debugging purposes, it also creates a debugging information file as well as an assembly file containing a listing of the generated machine code.
The macro \texttt{\_\_armt32fpe\_\_} is predefined in order to enable programmers to identify this tool and its target architecture while compiling.
Programs generated with this compiler require additional runtime support that is stored in the \file{cpp\-armt32\-fpe\-run} library file.
\flowgraph{\resource{\cpp{}\\source code} \ar[r] & \toolbox{cpparmt32fpe} \ar[r] \ar[d] \ar[rd] & \resource{object file} \\ \variable{ECSINCLUDE} \ar[ru] & \resource{debugging\\information} & \resource{assembly\\listing}}
\seecpp\seeassembly\seearm\seeobject\seedebugging
}

\providecommand{\cppavr}{
\toolsection{cppavr} is a compiler for the \cpp{} programming language targeting the AVR hardware architecture.
It generates machine code for AVR processors from programs written in \cpp{} and stores it in corresponding object files.
For debugging purposes, it also creates a debugging information file as well as an assembly file containing a listing of the generated machine code.
The macro \texttt{\_\_avr\_\_} is predefined in order to enable programmers to identify this tool and its target architecture while compiling.
Programs generated with this compiler require additional runtime support that is stored in the \file{cpp\-avr\-run} library file.
\flowgraph{\resource{\cpp{}\\source code} \ar[r] & \toolbox{cppavr} \ar[r] \ar[d] \ar[rd] & \resource{object file} \\ \variable{ECSINCLUDE} \ar[ru] & \resource{debugging\\information} & \resource{assembly\\listing}}
\seecpp\seeassembly\seeavr\seeobject\seedebugging
}

\providecommand{\cppavrtt}{
\toolsection{cppavr32} is a compiler for the \cpp{} programming language targeting the AVR32 hardware architecture.
It generates machine code for AVR32 processors from programs written in \cpp{} and stores it in corresponding object files.
For debugging purposes, it also creates a debugging information file as well as an assembly file containing a listing of the generated machine code.
The macro \texttt{\_\_avr32\_\_} is predefined in order to enable programmers to identify this tool and its target architecture while compiling.
Programs generated with this compiler require additional runtime support that is stored in the \file{cpp\-avr32\-run} library file.
\flowgraph{\resource{\cpp{}\\source code} \ar[r] & \toolbox{cppavr32} \ar[r] \ar[d] \ar[rd] & \resource{object file} \\ \variable{ECSINCLUDE} \ar[ru] & \resource{debugging\\information} & \resource{assembly\\listing}}
\seecpp\seeassembly\seeavrtt\seeobject\seedebugging
}

\providecommand{\cppmabk}{
\toolsection{cppm68k} is a compiler for the \cpp{} programming language targeting the M68000 hardware architecture.
It generates machine code for M68000 processors from programs written in \cpp{} and stores it in corresponding object files.
For debugging purposes, it also creates a debugging information file as well as an assembly file containing a listing of the generated machine code.
The macro \texttt{\_\_m68k\_\_} is predefined in order to enable programmers to identify this tool and its target architecture while compiling.
Programs generated with this compiler require additional runtime support that is stored in the \file{cpp\-m68k\-run} library file.
\flowgraph{\resource{\cpp{}\\source code} \ar[r] & \toolbox{cppm68k} \ar[r] \ar[d] \ar[rd] & \resource{object file} \\ \variable{ECSINCLUDE} \ar[ru] & \resource{debugging\\information} & \resource{assembly\\listing}}
\seecpp\seeassembly\seemabk\seeobject\seedebugging
}

\providecommand{\cppmibl}{
\toolsection{cppmibl} is a compiler for the \cpp{} programming language targeting the MicroBlaze hardware architecture.
It generates machine code for MicroBlaze processors from programs written in \cpp{} and stores it in corresponding object files.
For debugging purposes, it also creates a debugging information file as well as an assembly file containing a listing of the generated machine code.
The macro \texttt{\_\_mibl\_\_} is predefined in order to enable programmers to identify this tool and its target architecture while compiling.
Programs generated with this compiler require additional runtime support that is stored in the \file{cpp\-mibl\-run} library file.
\flowgraph{\resource{\cpp{}\\source code} \ar[r] & \toolbox{cppmibl} \ar[r] \ar[d] \ar[rd] & \resource{object file} \\ \variable{ECSINCLUDE} \ar[ru] & \resource{debugging\\information} & \resource{assembly\\listing}}
\seecpp\seeassembly\seemibl\seeobject\seedebugging
}

\providecommand{\cppmipsa}{
\toolsection{cppmips32} is a compiler for the \cpp{} programming language targeting the MIPS32 hardware architecture.
It generates machine code for MIPS32 processors from programs written in \cpp{} and stores it in corresponding object files.
For debugging purposes, it also creates a debugging information file as well as an assembly file containing a listing of the generated machine code.
The macro \texttt{\_\_mips32\_\_} is predefined in order to enable programmers to identify this tool and its target architecture while compiling.
Programs generated with this compiler require additional runtime support that is stored in the \file{cpp\-mips32\-run} library file.
\flowgraph{\resource{\cpp{}\\source code} \ar[r] & \toolbox{cppmips32} \ar[r] \ar[d] \ar[rd] & \resource{object file} \\ \variable{ECSINCLUDE} \ar[ru] & \resource{debugging\\information} & \resource{assembly\\listing}}
\seecpp\seeassembly\seemips\seeobject\seedebugging
}

\providecommand{\cppmipsb}{
\toolsection{cppmips64} is a compiler for the \cpp{} programming language targeting the MIPS64 hardware architecture.
It generates machine code for MIPS64 processors from programs written in \cpp{} and stores it in corresponding object files.
For debugging purposes, it also creates a debugging information file as well as an assembly file containing a listing of the generated machine code.
The macro \texttt{\_\_mips64\_\_} is predefined in order to enable programmers to identify this tool and its target architecture while compiling.
Programs generated with this compiler require additional runtime support that is stored in the \file{cpp\-mips64\-run} library file.
\flowgraph{\resource{\cpp{}\\source code} \ar[r] & \toolbox{cppmips64} \ar[r] \ar[d] \ar[rd] & \resource{object file} \\ \variable{ECSINCLUDE} \ar[ru] & \resource{debugging\\information} & \resource{assembly\\listing}}
\seecpp\seeassembly\seemips\seeobject\seedebugging
}

\providecommand{\cppmmix}{
\toolsection{cppmmix} is a compiler for the \cpp{} programming language targeting the MMIX hardware architecture.
It generates machine code for MMIX processors from programs written in \cpp{} and stores it in corresponding object files.
For debugging purposes, it also creates a debugging information file as well as an assembly file containing a listing of the generated machine code.
The macro \texttt{\_\_mmix\_\_} is predefined in order to enable programmers to identify this tool and its target architecture while compiling.
Programs generated with this compiler require additional runtime support that is stored in the \file{cpp\-mmix\-run} library file.
\flowgraph{\resource{\cpp{}\\source code} \ar[r] & \toolbox{cppmmix} \ar[r] \ar[d] \ar[rd] & \resource{object file} \\ \variable{ECSINCLUDE} \ar[ru] & \resource{debugging\\information} & \resource{assembly\\listing}}
\seecpp\seeassembly\seemmix\seeobject\seedebugging
}

\providecommand{\cpporok}{
\toolsection{cppor1k} is a compiler for the \cpp{} programming language targeting the OpenRISC 1000 hardware architecture.
It generates machine code for OpenRISC 1000 processors from programs written in \cpp{} and stores it in corresponding object files.
For debugging purposes, it also creates a debugging information file as well as an assembly file containing a listing of the generated machine code.
The macro \texttt{\_\_or1k\_\_} is predefined in order to enable programmers to identify this tool and its target architecture while compiling.
Programs generated with this compiler require additional runtime support that is stored in the \file{cpp\-or1k\-run} library file.
\flowgraph{\resource{\cpp{}\\source code} \ar[r] & \toolbox{cppor1k} \ar[r] \ar[d] \ar[rd] & \resource{object file} \\ \variable{ECSINCLUDE} \ar[ru] & \resource{debugging\\information} & \resource{assembly\\listing}}
\seecpp\seeassembly\seeorok\seeobject\seedebugging
}

\providecommand{\cppppca}{
\toolsection{cppppc32} is a compiler for the \cpp{} programming language targeting the PowerPC hardware architecture.
It generates machine code for PowerPC processors from programs written in \cpp{} and stores it in corresponding object files.
The compiler generates machine code for the 32-bit operating mode defined by the PowerPC architecture.
For debugging purposes, it also creates a debugging information file as well as an assembly file containing a listing of the generated machine code.
The macro \texttt{\_\_ppc32\_\_} is predefined in order to enable programmers to identify this tool and its target architecture while compiling.
Programs generated with this compiler require additional runtime support that is stored in the \file{cpp\-ppc32\-run} library file.
\flowgraph{\resource{\cpp{}\\source code} \ar[r] & \toolbox{cppppc32} \ar[r] \ar[d] \ar[rd] & \resource{object file} \\ \variable{ECSINCLUDE} \ar[ru] & \resource{debugging\\information} & \resource{assembly\\listing}}
\seecpp\seeassembly\seeppc\seeobject\seedebugging
}

\providecommand{\cppppcb}{
\toolsection{cppppc64} is a compiler for the \cpp{} programming language targeting the PowerPC hardware architecture.
It generates machine code for PowerPC processors from programs written in \cpp{} and stores it in corresponding object files.
The compiler generates machine code for the 64-bit operating mode defined by the PowerPC architecture.
For debugging purposes, it also creates a debugging information file as well as an assembly file containing a listing of the generated machine code.
The macro \texttt{\_\_ppc64\_\_} is predefined in order to enable programmers to identify this tool and its target architecture while compiling.
Programs generated with this compiler require additional runtime support that is stored in the \file{cpp\-ppc64\-run} library file.
\flowgraph{\resource{\cpp{}\\source code} \ar[r] & \toolbox{cppppc64} \ar[r] \ar[d] \ar[rd] & \resource{object file} \\ \variable{ECSINCLUDE} \ar[ru] & \resource{debugging\\information} & \resource{assembly\\listing}}
\seecpp\seeassembly\seeppc\seeobject\seedebugging
}

\providecommand{\cpprisc}{
\toolsection{cpprisc} is a compiler for the \cpp{} programming language targeting the RISC hardware architecture.
It generates machine code for RISC processors from programs written in \cpp{} and stores it in corresponding object files.
For debugging purposes, it also creates a debugging information file as well as an assembly file containing a listing of the generated machine code.
The macro \texttt{\_\_risc\_\_} is predefined in order to enable programmers to identify this tool and its target architecture while compiling.
Programs generated with this compiler require additional runtime support that is stored in the \file{cpp\-risc\-run} library file.
\flowgraph{\resource{\cpp{}\\source code} \ar[r] & \toolbox{cpprisc} \ar[r] \ar[d] \ar[rd] & \resource{object file} \\ \variable{ECSINCLUDE} \ar[ru] & \resource{debugging\\information} & \resource{assembly\\listing}}
\seecpp\seeassembly\seerisc\seeobject\seedebugging
}

\providecommand{\cppwasm}{
\toolsection{cppwasm} is a compiler for the \cpp{} programming language targeting the WebAssembly architecture.
It generates machine code for WebAssembly targets from programs written in \cpp{} and stores it in corresponding object files.
For debugging purposes, it also creates a debugging information file as well as an assembly file containing a listing of the generated machine code.
The macro \texttt{\_\_wasm\_\_} is predefined in order to enable programmers to identify this tool and its target architecture while compiling.
Programs generated with this compiler require additional runtime support that is stored in the \file{cpp\-wasm\-run} library file.
\flowgraph{\resource{\cpp{}\\source code} \ar[r] & \toolbox{cppwasm} \ar[r] \ar[d] \ar[rd] & \resource{object file} \\ \variable{ECSINCLUDE} \ar[ru] & \resource{debugging\\information} & \resource{assembly\\listing}}
\seecpp\seeassembly\seewasm\seeobject\seedebugging
}

% FALSE tools

\providecommand{\falprint}{
\toolsection{falprint} is a pretty printer for the FALSE programming language.
It reformats the source code of FALSE programs and writes it to the standard output stream.
\flowgraph{\resource{FALSE\\source code} \ar[r] & \toolbox{falprint} \ar[r] & \resource{reformatted\\source code}}
\seefalse
}

\providecommand{\falcheck}{
\toolsection{falcheck} is a syntactic and semantic checker for the FALSE programming language.
It just performs syntactic and semantic checks on FALSE programs and writes its diagnostic messages to the standard error stream.
\flowgraph{\resource{FALSE\\source code} \ar[r] & \toolbox{falcheck} \ar[r] & \resource{diagnostic\\messages}}
\seefalse
}

\providecommand{\faldump}{
\toolsection{faldump} is a serializer for the FALSE programming language.
It dumps the complete internal representation of programs written in FALSE into an XML document.
\debuggingtool
\flowgraph{\resource{FALSE\\source code} \ar[r] & \toolbox{faldump} \ar[r] & \resource{internal\\representation}}
\seefalse
}

\providecommand{\falrun}{
\toolsection{falrun} is an interpreter for the FALSE programming language.
It processes and executes programs written in FALSE\@.
\flowgraph{\resource{FALSE\\source code} \ar[r] & \toolbox{falrun} \ar@/u/[r] & \resource{input/\\output} \ar@/d/[l]}
\seefalse
}

\providecommand{\falcpp}{
\toolsection{falcpp} is a transpiler for the FALSE programming language.
It translates programs written in FALSE into \cpp{} programs and stores them in corresponding source files.
\flowgraph{\resource{FALSE\\source code} \ar[r] & \toolbox{falcpp} \ar[r] & \resource{\cpp{}\\source file}}
\seefalse\seecpp
}

\providecommand{\falcode}{
\toolsection{falcode} is an intermediate code generator for the FALSE programming language.
It generates intermediate code from programs written in FALSE and stores it in corresponding assembly files.
\debuggingtool
\flowgraph{\resource{FALSE\\source code} \ar[r] & \toolbox{falcode} \ar[r] & \resource{intermediate\\code}}
\seefalse\seeassembly\seecode
}

\providecommand{\falamda}{
\toolsection{falamd16} is a compiler for the FALSE programming language targeting the AMD64 hardware architecture.
It generates machine code for AMD64 processors from programs written in FALSE and stores it in corresponding object files.
The compiler generates machine code for the 16-bit operating mode defined by the AMD64 architecture.
\flowgraph{\resource{FALSE\\source code} \ar[r] & \toolbox{falamd16} \ar[r] & \resource{object file}}
\seefalse\seeamd\seeobject
}

\providecommand{\falamdb}{
\toolsection{falamd32} is a compiler for the FALSE programming language targeting the AMD64 hardware architecture.
It generates machine code for AMD64 processors from programs written in FALSE and stores it in corresponding object files.
The compiler generates machine code for the 32-bit operating mode defined by the AMD64 architecture.
\flowgraph{\resource{FALSE\\source code} \ar[r] & \toolbox{falamd32} \ar[r] & \resource{object file}}
\seefalse\seeamd\seeobject
}

\providecommand{\falamdc}{
\toolsection{falamd64} is a compiler for the FALSE programming language targeting the AMD64 hardware architecture.
It generates machine code for AMD64 processors from programs written in FALSE and stores it in corresponding object files.
The compiler generates machine code for the 64-bit operating mode defined by the AMD64 architecture.
\flowgraph{\resource{FALSE\\source code} \ar[r] & \toolbox{falamd64} \ar[r] & \resource{object file}}
\seefalse\seeamd\seeobject
}

\providecommand{\falarma}{
\toolsection{falarma32} is a compiler for the FALSE programming language targeting the ARM hardware architecture.
It generates machine code for ARM processors executing A32 instructions from programs written in FALSE and stores it in corresponding object files.
\flowgraph{\resource{FALSE\\source code} \ar[r] & \toolbox{falarma32} \ar[r] & \resource{object file}}
\seefalse\seearm\seeobject
}

\providecommand{\falarmb}{
\toolsection{falarma64} is a compiler for the FALSE programming language targeting the ARM hardware architecture.
It generates machine code for ARM processors executing A64 instructions from programs written in FALSE and stores it in corresponding object files.
\flowgraph{\resource{FALSE\\source code} \ar[r] & \toolbox{falarma64} \ar[r] & \resource{object file}}
\seefalse\seearm\seeobject
}

\providecommand{\falarmc}{
\toolsection{falarmt32} is a compiler for the FALSE programming language targeting the ARM hardware architecture.
It generates machine code for ARM processors without floating-point extension executing T32 instructions from programs written in FALSE and stores it in corresponding object files.
\flowgraph{\resource{FALSE\\source code} \ar[r] & \toolbox{falarmt32} \ar[r] & \resource{object file}}
\seefalse\seearm\seeobject
}

\providecommand{\falarmcfpe}{
\toolsection{falarmt32fpe} is a compiler for the FALSE programming language targeting the ARM hardware architecture.
It generates machine code for ARM processors with floating-point extension executing T32 instructions from programs written in FALSE and stores it in corresponding object files.
\flowgraph{\resource{FALSE\\source code} \ar[r] & \toolbox{falarmt32fpe} \ar[r] & \resource{object file}}
\seefalse\seearm\seeobject
}

\providecommand{\falavr}{
\toolsection{falavr} is a compiler for the FALSE programming language targeting the AVR hardware architecture.
It generates machine code for AVR processors from programs written in FALSE and stores it in corresponding object files.
\flowgraph{\resource{FALSE\\source code} \ar[r] & \toolbox{falavr} \ar[r] & \resource{object file}}
\seefalse\seeavr\seeobject
}

\providecommand{\falavrtt}{
\toolsection{falavr32} is a compiler for the FALSE programming language targeting the AVR32 hardware architecture.
It generates machine code for AVR32 processors from programs written in FALSE and stores it in corresponding object files.
\flowgraph{\resource{FALSE\\source code} \ar[r] & \toolbox{falavr32} \ar[r] & \resource{object file}}
\seefalse\seeavrtt\seeobject
}

\providecommand{\falmabk}{
\toolsection{falm68k} is a compiler for the FALSE programming language targeting the M68000 hardware architecture.
It generates machine code for M68000 processors from programs written in FALSE and stores it in corresponding object files.
\flowgraph{\resource{FALSE\\source code} \ar[r] & \toolbox{falm68k} \ar[r] & \resource{object file}}
\seefalse\seemabk\seeobject
}

\providecommand{\falmibl}{
\toolsection{falmibl} is a compiler for the FALSE programming language targeting the MicroBlaze hardware architecture.
It generates machine code for MicroBlaze processors from programs written in FALSE and stores it in corresponding object files.
\flowgraph{\resource{FALSE\\source code} \ar[r] & \toolbox{falmibl} \ar[r] & \resource{object file}}
\seefalse\seemibl\seeobject
}

\providecommand{\falmipsa}{
\toolsection{falmips32} is a compiler for the FALSE programming language targeting the MIPS32 hardware architecture.
It generates machine code for MIPS32 processors from programs written in FALSE and stores it in corresponding object files.
\flowgraph{\resource{FALSE\\source code} \ar[r] & \toolbox{falmips32} \ar[r] & \resource{object file}}
\seefalse\seemips\seeobject
}

\providecommand{\falmipsb}{
\toolsection{falmips64} is a compiler for the FALSE programming language targeting the MIPS64 hardware architecture.
It generates machine code for MIPS64 processors from programs written in FALSE and stores it in corresponding object files.
\flowgraph{\resource{FALSE\\source code} \ar[r] & \toolbox{falmips64} \ar[r] & \resource{object file}}
\seefalse\seemips\seeobject
}

\providecommand{\falmmix}{
\toolsection{falmmix} is a compiler for the FALSE programming language targeting the MMIX hardware architecture.
It generates machine code for MMIX processors from programs written in FALSE and stores it in corresponding object files.
\flowgraph{\resource{FALSE\\source code} \ar[r] & \toolbox{falmmix} \ar[r] & \resource{object file}}
\seefalse\seemmix\seeobject
}

\providecommand{\falorok}{
\toolsection{falor1k} is a compiler for the FALSE programming language targeting the OpenRISC 1000 hardware architecture.
It generates machine code for OpenRISC 1000 processors from programs written in FALSE and stores it in corresponding object files.
\flowgraph{\resource{FALSE\\source code} \ar[r] & \toolbox{falor1k} \ar[r] & \resource{object file}}
\seefalse\seeorok\seeobject
}

\providecommand{\falppca}{
\toolsection{falppc32} is a compiler for the FALSE programming language targeting the PowerPC hardware architecture.
It generates machine code for PowerPC processors from programs written in FALSE and stores it in corresponding object files.
The compiler generates machine code for the 32-bit operating mode defined by the PowerPC architecture.
\flowgraph{\resource{FALSE\\source code} \ar[r] & \toolbox{falppc32} \ar[r] & \resource{object file}}
\seefalse\seeppc\seeobject
}

\providecommand{\falppcb}{
\toolsection{falppc64} is a compiler for the FALSE programming language targeting the PowerPC hardware architecture.
It generates machine code for PowerPC processors from programs written in FALSE and stores it in corresponding object files.
The compiler generates machine code for the 64-bit operating mode defined by the PowerPC architecture.
\flowgraph{\resource{FALSE\\source code} \ar[r] & \toolbox{falppc64} \ar[r] & \resource{object file}}
\seefalse\seeppc\seeobject
}

\providecommand{\falrisc}{
\toolsection{falrisc} is a compiler for the FALSE programming language targeting the RISC hardware architecture.
It generates machine code for RISC processors from programs written in FALSE and stores it in corresponding object files.
\flowgraph{\resource{FALSE\\source code} \ar[r] & \toolbox{falrisc} \ar[r] & \resource{object file}}
\seefalse\seerisc\seeobject
}

\providecommand{\falwasm}{
\toolsection{falwasm} is a compiler for the FALSE programming language targeting the WebAssembly architecture.
It generates machine code for WebAssembly targets from programs written in FALSE and stores it in corresponding object files.
\flowgraph{\resource{FALSE\\source code} \ar[r] & \toolbox{falwasm} \ar[r] & \resource{object file}}
\seefalse\seewasm\seeobject
}

% Oberon tools

\providecommand{\obprint}{
\toolsection{obprint} is a pretty printer for the Oberon programming language.
It reformats the source code of Oberon modules and writes it to the standard output stream.
\flowgraph{\resource{Oberon\\source code} \ar[r] & \toolbox{obprint} \ar[r] & \resource{reformatted\\source code}}
\seeoberon
}

\providecommand{\obcheck}{
\toolsection{obcheck} is a syntactic and semantic checker for the Oberon programming language.
It just performs syntactic and semantic checks on Oberon modules and writes its diagnostic messages to the standard error stream.
In addition, it stores the interface of each module in a symbol file which is required when other modules import the module.
\flowgraph{\resource{Oberon\\source code} \ar[r] & \toolbox{obcheck} \ar[r] \ar@/l/[d] & \resource{diagnostic\\messages} \\ \variable{ECSIMPORT} \ar[ru] & \resource{symbol\\files} \ar@/r/[u]}
\seeoberon
}

\providecommand{\obdump}{
\toolsection{obdump} is a serializer for the Oberon programming language.
It dumps the complete internal representation of modules written in Oberon into an XML document.
\debuggingtool
\flowgraph{\resource{Oberon\\source code} \ar[r] & \toolbox{obdump} \ar[r] \ar@/l/[d] & \resource{internal\\representation} \\ \variable{ECSIMPORT} \ar[ru] & \resource{symbol\\files} \ar@/r/[u]}
\seeoberon
}

\providecommand{\obrun}{
\toolsection{obrun} is an interpreter for the Oberon programming language.
It processes and executes modules written in Oberon.
This tool does neither generate nor process symbol files while interpreting modules.
If a module is imported by another one, its filename has to be named before the other one in the list of command-line arguments.
\flowgraph{\resource{Oberon\\source code} \ar[r] & \toolbox{obrun} \ar@/u/[r] & \resource{input/\\output} \ar@/d/[l]}
\seeoberon
}

\providecommand{\obcpp}{
\toolsection{obcpp} is a transpiler for the Oberon programming language.
It translates programs written in Oberon into \cpp{} programs and stores them in corresponding source and header files.
In addition, it stores the interface of each module in a symbol file which is required when other modules import the module.
The same interface is provided by the generated header file which can be used in other parts of the \cpp{} program.
\flowgraph{\resource{Oberon\\source code} \ar[r] & \toolbox{obcpp} \ar[r] \ar@/l/[d] \ar[rd] & \resource{\cpp{}\\source file} \\ \variable{ECSIMPORT} \ar[ru] & \resource{symbol\\files} \ar@/r/[u] & \resource{\cpp{}\\header file}}
\seeoberon\seecpp
}

\providecommand{\obdoc}{
\toolsection{obdoc} is a generic documentation generator for the Oberon programming language.
It processes several Oberon modules and assembles all information therein into a generic documentation.
In addition, it stores the interface of each module in a symbol file which is required when other modules import the module.
\debuggingtool
\flowgraph{\resource{Oberon\\source code} \ar[r] & \toolbox{obdoc} \ar[r] \ar@/l/[d] & \resource{generic\\documentation} \\ \variable{ECSIMPORT} \ar[ru] & \resource{symbol\\files} \ar@/r/[u]}
\seeoberon\seedocumentation
}

\providecommand{\obhtml}{
\toolsection{obhtml} is an HTML documentation generator for the Oberon programming language.
It processes several Oberon modules and assembles all information therein into an HTML document.
In addition, it stores the interface of each module in a symbol file which is required when other modules import the module.
\flowgraph{\resource{Oberon\\source code} \ar[r] & \toolbox{obhtml} \ar[r] \ar@/l/[d] & \resource{HTML\\document} \\ \variable{ECSIMPORT} \ar[ru] & \resource{symbol\\files} \ar@/r/[u]}
\seeoberon\seedocumentation
}

\providecommand{\oblatex}{
\toolsection{oblatex} is a Latex documentation generator for the Oberon programming language.
It processes several Oberon modules and assembles all information therein into a Latex document.
In addition, it stores the interface of each module in a symbol file which is required when other modules import the module.
\flowgraph{\resource{Oberon\\source code} \ar[r] & \toolbox{oblatex} \ar[r] \ar@/l/[d] & \resource{Latex\\document} \\ \variable{ECSIMPORT} \ar[ru] & \resource{symbol\\files} \ar@/r/[u]}
\seeoberon\seedocumentation
}

\providecommand{\obcode}{
\toolsection{obcode} is an intermediate code generator for the Oberon programming language.
It generates intermediate code from modules written in Oberon and stores it in corresponding assembly files.
In addition, it stores the interface of each module in a symbol file which is required when other modules import the module.
Programs generated with this tool require additional runtime support that is stored in the \file{ob\-code\-run} library file.
\debuggingtool
\flowgraph{\resource{Oberon\\source code} \ar[r] & \toolbox{obcode} \ar[r] \ar@/l/[d] & \resource{intermediate\\code} \\ \variable{ECSIMPORT} \ar[ru] & \resource{symbol\\files} \ar@/r/[u]}
\seeoberon\seeassembly\seecode
}

\providecommand{\obamda}{
\toolsection{obamd16} is a compiler for the Oberon programming language targeting the AMD64 hardware architecture.
It generates machine code for AMD64 processors from modules written in Oberon and stores it in corresponding object files.
The compiler generates machine code for the 16-bit operating mode defined by the AMD64 architecture.
For debugging purposes, it also creates a debugging information file as well as an assembly file containing a listing of the generated machine code.
In addition, it stores the interface of each module in a symbol file which is required when other modules import the module.
Programs generated with this compiler require additional runtime support that is stored in the \file{ob\-amd16\-run} library file.
\flowgraph{\resource{Oberon\\source code} \ar[r] & \toolbox{obamd16} \ar[r] \ar@/l/[d] \ar[rd] & \resource{object file} \\ \variable{ECSIMPORT} \ar[ru] & \resource{symbol\\files} \ar@/r/[u] & \resource{debugging\\information}}
\seeoberon\seeassembly\seeamd\seeobject\seedebugging
}

\providecommand{\obamdb}{
\toolsection{obamd32} is a compiler for the Oberon programming language targeting the AMD64 hardware architecture.
It generates machine code for AMD64 processors from modules written in Oberon and stores it in corresponding object files.
The compiler generates machine code for the 32-bit operating mode defined by the AMD64 architecture.
For debugging purposes, it also creates a debugging information file as well as an assembly file containing a listing of the generated machine code.
In addition, it stores the interface of each module in a symbol file which is required when other modules import the module.
Programs generated with this compiler require additional runtime support that is stored in the \file{ob\-amd32\-run} library file.
\flowgraph{\resource{Oberon\\source code} \ar[r] & \toolbox{obamd32} \ar[r] \ar@/l/[d] \ar[rd] & \resource{object file} \\ \variable{ECSIMPORT} \ar[ru] & \resource{symbol\\files} \ar@/r/[u] & \resource{debugging\\information}}
\seeoberon\seeassembly\seeamd\seeobject\seedebugging
}

\providecommand{\obamdc}{
\toolsection{obamd64} is a compiler for the Oberon programming language targeting the AMD64 hardware architecture.
It generates machine code for AMD64 processors from modules written in Oberon and stores it in corresponding object files.
The compiler generates machine code for the 64-bit operating mode defined by the AMD64 architecture.
For debugging purposes, it also creates a debugging information file as well as an assembly file containing a listing of the generated machine code.
In addition, it stores the interface of each module in a symbol file which is required when other modules import the module.
Programs generated with this compiler require additional runtime support that is stored in the \file{ob\-amd64\-run} library file.
\flowgraph{\resource{Oberon\\source code} \ar[r] & \toolbox{obamd64} \ar[r] \ar@/l/[d] \ar[rd] & \resource{object file} \\ \variable{ECSIMPORT} \ar[ru] & \resource{symbol\\files} \ar@/r/[u] & \resource{debugging\\information}}
\seeoberon\seeassembly\seeamd\seeobject\seedebugging
}

\providecommand{\obarma}{
\toolsection{obarma32} is a compiler for the Oberon programming language targeting the ARM hardware architecture.
It generates machine code for ARM processors executing A32 instructions from modules written in Oberon and stores it in corresponding object files.
For debugging purposes, it also creates a debugging information file as well as an assembly file containing a listing of the generated machine code.
In addition, it stores the interface of each module in a symbol file which is required when other modules import the module.
Programs generated with this compiler require additional runtime support that is stored in the \file{ob\-arma32\-run} library file.
\flowgraph{\resource{Oberon\\source code} \ar[r] & \toolbox{obarma32} \ar[r] \ar@/l/[d] \ar[rd] & \resource{object file} \\ \variable{ECSIMPORT} \ar[ru] & \resource{symbol\\files} \ar@/r/[u] & \resource{debugging\\information}}
\seeoberon\seeassembly\seearm\seeobject\seedebugging
}

\providecommand{\obarmb}{
\toolsection{obarma64} is a compiler for the Oberon programming language targeting the ARM hardware architecture.
It generates machine code for ARM processors executing A64 instructions from modules written in Oberon and stores it in corresponding object files.
For debugging purposes, it also creates a debugging information file as well as an assembly file containing a listing of the generated machine code.
In addition, it stores the interface of each module in a symbol file which is required when other modules import the module.
Programs generated with this compiler require additional runtime support that is stored in the \file{ob\-arma64\-run} library file.
\flowgraph{\resource{Oberon\\source code} \ar[r] & \toolbox{obarma64} \ar[r] \ar@/l/[d] \ar[rd] & \resource{object file} \\ \variable{ECSIMPORT} \ar[ru] & \resource{symbol\\files} \ar@/r/[u] & \resource{debugging\\information}}
\seeoberon\seeassembly\seearm\seeobject\seedebugging
}

\providecommand{\obarmc}{
\toolsection{obarmt32} is a compiler for the Oberon programming language targeting the ARM hardware architecture.
It generates machine code for ARM processors without floating-point extension executing T32 instructions from modules written in Oberon and stores it in corresponding object files.
For debugging purposes, it also creates a debugging information file as well as an assembly file containing a listing of the generated machine code.
In addition, it stores the interface of each module in a symbol file which is required when other modules import the module.
Programs generated with this compiler require additional runtime support that is stored in the \file{ob\-armt32\-run} library file.
\flowgraph{\resource{Oberon\\source code} \ar[r] & \toolbox{obarmt32} \ar[r] \ar@/l/[d] \ar[rd] & \resource{object file} \\ \variable{ECSIMPORT} \ar[ru] & \resource{symbol\\files} \ar@/r/[u] & \resource{debugging\\information}}
\seeoberon\seeassembly\seearm\seeobject\seedebugging
}

\providecommand{\obarmcfpe}{
\toolsection{obarmt32fpe} is a compiler for the Oberon programming language targeting the ARM hardware architecture.
It generates machine code for ARM processors with floating-point extension executing T32 instructions from modules written in Oberon and stores it in corresponding object files.
For debugging purposes, it also creates a debugging information file as well as an assembly file containing a listing of the generated machine code.
In addition, it stores the interface of each module in a symbol file which is required when other modules import the module.
Programs generated with this compiler require additional runtime support that is stored in the \file{ob\-armt32\-fpe\-run} library file.
\flowgraph{\resource{Oberon\\source code} \ar[r] & \toolbox{obarmt32fpe} \ar[r] \ar@/l/[d] \ar[rd] & \resource{object file} \\ \variable{ECSIMPORT} \ar[ru] & \resource{symbol\\files} \ar@/r/[u] & \resource{debugging\\information}}
\seeoberon\seeassembly\seearm\seeobject\seedebugging
}

\providecommand{\obavr}{
\toolsection{obavr} is a compiler for the Oberon programming language targeting the AVR hardware architecture.
It generates machine code for AVR processors from modules written in Oberon and stores it in corresponding object files.
For debugging purposes, it also creates a debugging information file as well as an assembly file containing a listing of the generated machine code.
In addition, it stores the interface of each module in a symbol file which is required when other modules import the module.
Programs generated with this compiler require additional runtime support that is stored in the \file{ob\-avr\-run} library file.
\flowgraph{\resource{Oberon\\source code} \ar[r] & \toolbox{obavr} \ar[r] \ar@/l/[d] \ar[rd] & \resource{object file} \\ \variable{ECSIMPORT} \ar[ru] & \resource{symbol\\files} \ar@/r/[u] & \resource{debugging\\information}}
\seeoberon\seeassembly\seeavr\seeobject\seedebugging
}

\providecommand{\obavrtt}{
\toolsection{obavr32} is a compiler for the Oberon programming language targeting the AVR32 hardware architecture.
It generates machine code for AVR32 processors from modules written in Oberon and stores it in corresponding object files.
For debugging purposes, it also creates a debugging information file as well as an assembly file containing a listing of the generated machine code.
In addition, it stores the interface of each module in a symbol file which is required when other modules import the module.
Programs generated with this compiler require additional runtime support that is stored in the \file{ob\-avr32\-run} library file.
\flowgraph{\resource{Oberon\\source code} \ar[r] & \toolbox{obavr32} \ar[r] \ar@/l/[d] \ar[rd] & \resource{object file} \\ \variable{ECSIMPORT} \ar[ru] & \resource{symbol\\files} \ar@/r/[u] & \resource{debugging\\information}}
\seeoberon\seeassembly\seeavrtt\seeobject\seedebugging
}

\providecommand{\obmabk}{
\toolsection{obm68k} is a compiler for the Oberon programming language targeting the M68000 hardware architecture.
It generates machine code for M68000 processors from modules written in Oberon and stores it in corresponding object files.
For debugging purposes, it also creates a debugging information file as well as an assembly file containing a listing of the generated machine code.
In addition, it stores the interface of each module in a symbol file which is required when other modules import the module.
Programs generated with this compiler require additional runtime support that is stored in the \file{ob\-m68k\-run} library file.
\flowgraph{\resource{Oberon\\source code} \ar[r] & \toolbox{obm68k} \ar[r] \ar@/l/[d] \ar[rd] & \resource{object file} \\ \variable{ECSIMPORT} \ar[ru] & \resource{symbol\\files} \ar@/r/[u] & \resource{debugging\\information}}
\seeoberon\seeassembly\seemabk\seeobject\seedebugging
}

\providecommand{\obmibl}{
\toolsection{obmibl} is a compiler for the Oberon programming language targeting the MicroBlaze hardware architecture.
It generates machine code for MicroBlaze processors from modules written in Oberon and stores it in corresponding object files.
For debugging purposes, it also creates a debugging information file as well as an assembly file containing a listing of the generated machine code.
In addition, it stores the interface of each module in a symbol file which is required when other modules import the module.
Programs generated with this compiler require additional runtime support that is stored in the \file{ob\-mibl\-run} library file.
\flowgraph{\resource{Oberon\\source code} \ar[r] & \toolbox{obmibl} \ar[r] \ar@/l/[d] \ar[rd] & \resource{object file} \\ \variable{ECSIMPORT} \ar[ru] & \resource{symbol\\files} \ar@/r/[u] & \resource{debugging\\information}}
\seeoberon\seeassembly\seemibl\seeobject\seedebugging
}

\providecommand{\obmipsa}{
\toolsection{obmips32} is a compiler for the Oberon programming language targeting the MIPS32 hardware architecture.
It generates machine code for MIPS32 processors from modules written in Oberon and stores it in corresponding object files.
For debugging purposes, it also creates a debugging information file as well as an assembly file containing a listing of the generated machine code.
In addition, it stores the interface of each module in a symbol file which is required when other modules import the module.
Programs generated with this compiler require additional runtime support that is stored in the \file{ob\-mips32\-run} library file.
\flowgraph{\resource{Oberon\\source code} \ar[r] & \toolbox{obmips32} \ar[r] \ar@/l/[d] \ar[rd] & \resource{object file} \\ \variable{ECSIMPORT} \ar[ru] & \resource{symbol\\files} \ar@/r/[u] & \resource{debugging\\information}}
\seeoberon\seeassembly\seemips\seeobject\seedebugging
}

\providecommand{\obmipsb}{
\toolsection{obmips64} is a compiler for the Oberon programming language targeting the MIPS64 hardware architecture.
It generates machine code for MIPS64 processors from modules written in Oberon and stores it in corresponding object files.
For debugging purposes, it also creates a debugging information file as well as an assembly file containing a listing of the generated machine code.
In addition, it stores the interface of each module in a symbol file which is required when other modules import the module.
Programs generated with this compiler require additional runtime support that is stored in the \file{ob\-mips64\-run} library file.
\flowgraph{\resource{Oberon\\source code} \ar[r] & \toolbox{obmips64} \ar[r] \ar@/l/[d] \ar[rd] & \resource{object file} \\ \variable{ECSIMPORT} \ar[ru] & \resource{symbol\\files} \ar@/r/[u] & \resource{debugging\\information}}
\seeoberon\seeassembly\seemips\seeobject\seedebugging
}

\providecommand{\obmmix}{
\toolsection{obmmix} is a compiler for the Oberon programming language targeting the MMIX hardware architecture.
It generates machine code for MMIX processors from modules written in Oberon and stores it in corresponding object files.
For debugging purposes, it also creates a debugging information file as well as an assembly file containing a listing of the generated machine code.
In addition, it stores the interface of each module in a symbol file which is required when other modules import the module.
Programs generated with this compiler require additional runtime support that is stored in the \file{ob\-mmix\-run} library file.
\flowgraph{\resource{Oberon\\source code} \ar[r] & \toolbox{obmmix} \ar[r] \ar@/l/[d] \ar[rd] & \resource{object file} \\ \variable{ECSIMPORT} \ar[ru] & \resource{symbol\\files} \ar@/r/[u] & \resource{debugging\\information}}
\seeoberon\seeassembly\seemmix\seeobject\seedebugging
}

\providecommand{\oborok}{
\toolsection{obor1k} is a compiler for the Oberon programming language targeting the OpenRISC 1000 hardware architecture.
It generates machine code for OpenRISC 1000 processors from modules written in Oberon and stores it in corresponding object files.
For debugging purposes, it also creates a debugging information file as well as an assembly file containing a listing of the generated machine code.
In addition, it stores the interface of each module in a symbol file which is required when other modules import the module.
Programs generated with this compiler require additional runtime support that is stored in the \file{ob\-or1k\-run} library file.
\flowgraph{\resource{Oberon\\source code} \ar[r] & \toolbox{obor1k} \ar[r] \ar@/l/[d] \ar[rd] & \resource{object file} \\ \variable{ECSIMPORT} \ar[ru] & \resource{symbol\\files} \ar@/r/[u] & \resource{debugging\\information}}
\seeoberon\seeassembly\seeorok\seeobject\seedebugging
}

\providecommand{\obppca}{
\toolsection{obppc32} is a compiler for the Oberon programming language targeting the PowerPC hardware architecture.
It generates machine code for PowerPC processors from modules written in Oberon and stores it in corresponding object files.
The compiler generates machine code for the 32-bit operating mode defined by the PowerPC architecture.
For debugging purposes, it also creates a debugging information file as well as an assembly file containing a listing of the generated machine code.
In addition, it stores the interface of each module in a symbol file which is required when other modules import the module.
Programs generated with this compiler require additional runtime support that is stored in the \file{ob\-ppc32\-run} library file.
\flowgraph{\resource{Oberon\\source code} \ar[r] & \toolbox{obppc32} \ar[r] \ar@/l/[d] \ar[rd] & \resource{object file} \\ \variable{ECSIMPORT} \ar[ru] & \resource{symbol\\files} \ar@/r/[u] & \resource{debugging\\information}}
\seeoberon\seeassembly\seeppc\seeobject\seedebugging
}

\providecommand{\obppcb}{
\toolsection{obppc64} is a compiler for the Oberon programming language targeting the PowerPC hardware architecture.
It generates machine code for PowerPC processors from modules written in Oberon and stores it in corresponding object files.
The compiler generates machine code for the 64-bit operating mode defined by the PowerPC architecture.
For debugging purposes, it also creates a debugging information file as well as an assembly file containing a listing of the generated machine code.
In addition, it stores the interface of each module in a symbol file which is required when other modules import the module.
Programs generated with this compiler require additional runtime support that is stored in the \file{ob\-ppc64\-run} library file.
\flowgraph{\resource{Oberon\\source code} \ar[r] & \toolbox{obppc64} \ar[r] \ar@/l/[d] \ar[rd] & \resource{object file} \\ \variable{ECSIMPORT} \ar[ru] & \resource{symbol\\files} \ar@/r/[u] & \resource{debugging\\information}}
\seeoberon\seeassembly\seeppc\seeobject\seedebugging
}

\providecommand{\obrisc}{
\toolsection{obrisc} is a compiler for the Oberon programming language targeting the RISC hardware architecture.
It generates machine code for RISC processors from modules written in Oberon and stores it in corresponding object files.
For debugging purposes, it also creates a debugging information file as well as an assembly file containing a listing of the generated machine code.
In addition, it stores the interface of each module in a symbol file which is required when other modules import the module.
Programs generated with this compiler require additional runtime support that is stored in the \file{ob\-risc\-run} library file.
\flowgraph{\resource{Oberon\\source code} \ar[r] & \toolbox{obrisc} \ar[r] \ar@/l/[d] \ar[rd] & \resource{object file} \\ \variable{ECSIMPORT} \ar[ru] & \resource{symbol\\files} \ar@/r/[u] & \resource{debugging\\information}}
\seeoberon\seeassembly\seerisc\seeobject\seedebugging
}

\providecommand{\obwasm}{
\toolsection{obwasm} is a compiler for the Oberon programming language targeting the WebAssembly architecture.
It generates machine code for WebAssembly targets from modules written in Oberon and stores it in corresponding object files.
For debugging purposes, it also creates a debugging information file as well as an assembly file containing a listing of the generated machine code.
In addition, it stores the interface of each module in a symbol file which is required when other modules import the module.
Programs generated with this compiler require additional runtime support that is stored in the \file{ob\-wasm\-run} library file.
\flowgraph{\resource{Oberon\\source code} \ar[r] & \toolbox{obwasm} \ar[r] \ar@/l/[d] \ar[rd] & \resource{object file} \\ \variable{ECSIMPORT} \ar[ru] & \resource{symbol\\files} \ar@/r/[u] & \resource{debugging\\information}}
\seeoberon\seeassembly\seewasm\seeobject\seedebugging
}

% converter tools

\providecommand{\dbgdwarf}{
\toolsection{dbgdwarf} is a DWARF debugging information converter tool.
It converts debugging information into the DWARF debugging data format and stores it in corresponding object files~\cite{dwarffile}.
The resulting debugging object files can be combined with runtime support that creates Executable and Linking Format (ELF) files~\cite{elffile}.
\flowgraph{\resource{debugging\\information} \ar[r] & \toolbox{dbgdwarf} \ar[r] & \resource{debugging\\object file}}
\seeobject\seedebugging
}

% assembler tools

\providecommand{\asmprint}{
\toolsection{asmprint} is a pretty printer for generic assembly code.
It reformats generic assembly code and writes it to the standard output stream.
\flowgraph{\resource{generic assembly\\source code} \ar[r] & \toolbox{asmprint} \ar[r] & \resource{reformatted\\source code}}
\seeassembly
}

\providecommand{\amdaasm}{
\toolsection{amd16asm} is an assembler for the AMD64 hardware architecture.
It translates assembly code into machine code for AMD64 processors and stores it in corresponding object files.
By default, the assembler generates machine code for the 16-bit operating mode defined by the AMD64 architecture.
\flowgraph{\resource{AMD16 assembly\\source code} \ar[r] & \toolbox{amd16asm} \ar[r] & \resource{object file}}
\seeassembly\seeamd\seeobject
}

\providecommand{\amdadism}{
\toolsection{amd16dism} is a disassembler for the AMD64 hardware architecture.
It translates machine code from object files targeting AMD64 processors into assembly code and writes it to the standard output stream.
It assumes that the machine code was generated for the 16-bit operating mode defined by the AMD64 architecture.
\flowgraph{\resource{object file} \ar[r] & \toolbox{amd16dism} \ar[r] & \resource{disassembly\\listing}}
\seeassembly\seeamd\seeobject
}

\providecommand{\amdbasm}{
\toolsection{amd32asm} is an assembler for the AMD64 hardware architecture.
It translates assembly code into machine code for AMD64 processors and stores it in corresponding object files.
By default, the assembler generates machine code for the 32-bit operating mode defined by the AMD64 architecture.
\flowgraph{\resource{AMD32 assembly\\source code} \ar[r] & \toolbox{amd32asm} \ar[r] & \resource{object file}}
\seeassembly\seeamd\seeobject
}

\providecommand{\amdbdism}{
\toolsection{amd32dism} is a disassembler for the AMD64 hardware architecture.
It translates machine code from object files targeting AMD64 processors into assembly code and writes it to the standard output stream.
It assumes that the machine code was generated for the 32-bit operating mode defined by the AMD64 architecture.
\flowgraph{\resource{object file} \ar[r] & \toolbox{amd32dism} \ar[r] & \resource{disassembly\\listing}}
\seeassembly\seeamd\seeobject
}

\providecommand{\amdcasm}{
\toolsection{amd64asm} is an assembler for the AMD64 hardware architecture.
It translates assembly code into machine code for AMD64 processors and stores it in corresponding object files.
By default, the assembler generates machine code for the 64-bit operating mode defined by the AMD64 architecture.
\flowgraph{\resource{AMD64 assembly\\source code} \ar[r] & \toolbox{amd64asm} \ar[r] & \resource{object file}}
\seeassembly\seeamd\seeobject
}

\providecommand{\amdcdism}{
\toolsection{amd64dism} is a disassembler for the AMD64 hardware architecture.
It translates machine code from object files targeting AMD64 processors into assembly code and writes it to the standard output stream.
It assumes that the machine code was generated for the 64-bit operating mode defined by the AMD64 architecture.
\flowgraph{\resource{object file} \ar[r] & \toolbox{amd64dism} \ar[r] & \resource{disassembly\\listing}}
\seeassembly\seeamd\seeobject
}

\providecommand{\armaasm}{
\toolsection{arma32asm} is an assembler for the ARM hardware architecture.
It translates assembly code into machine code for ARM processors executing A32 instructions and stores it in corresponding object files.
\flowgraph{\resource{ARM A32 assembly\\source code} \ar[r] & \toolbox{arma32asm} \ar[r] & \resource{object file}}
\seeassembly\seearm\seeobject
}

\providecommand{\armadism}{
\toolsection{arma32dism} is a disassembler for the ARM hardware architecture.
It translates machine code from object files targeting ARM processors executing A32 instructions into assembly code and writes it to the standard output stream.
\flowgraph{\resource{object file} \ar[r] & \toolbox{arma32dism} \ar[r] & \resource{disassembly\\listing}}
\seeassembly\seearm\seeobject
}

\providecommand{\armbasm}{
\toolsection{arma64asm} is an assembler for the ARM hardware architecture.
It translates assembly code into machine code for ARM processors executing A64 instructions and stores it in corresponding object files.
\flowgraph{\resource{ARM A64 assembly\\source code} \ar[r] & \toolbox{arma64asm} \ar[r] & \resource{object file}}
\seeassembly\seearm\seeobject
}

\providecommand{\armbdism}{
\toolsection{arma64dism} is a disassembler for the ARM hardware architecture.
It translates machine code from object files targeting ARM processors executing A64 instructions into assembly code and writes it to the standard output stream.
\flowgraph{\resource{object file} \ar[r] & \toolbox{arma64dism} \ar[r] & \resource{disassembly\\listing}}
\seeassembly\seearm\seeobject
}

\providecommand{\armcasm}{
\toolsection{armt32asm} is an assembler for the ARM hardware architecture.
It translates assembly code into machine code for ARM processors executing T32 instructions and stores it in corresponding object files.
\flowgraph{\resource{ARM T32 assembly\\source code} \ar[r] & \toolbox{armt32asm} \ar[r] & \resource{object file}}
\seeassembly\seearm\seeobject
}

\providecommand{\armcdism}{
\toolsection{armt32dism} is a disassembler for the ARM hardware architecture.
It translates machine code from object files targeting ARM processors executing T32 instructions into assembly code and writes it to the standard output stream.
\flowgraph{\resource{object file} \ar[r] & \toolbox{armt32dism} \ar[r] & \resource{disassembly\\listing}}
\seeassembly\seearm\seeobject
}

\providecommand{\avrasm}{
\toolsection{avrasm} is an assembler for the AVR hardware architecture.
It translates assembly code into machine code for AVR processors and stores it in corresponding object files.
The identifiers \texttt{RXL}, \texttt{RXH}, \texttt{RYL}, \texttt{RYH}, \texttt{RZL}, and \texttt{RZH} are predefined and name the corresponding registers.
The identifiers \texttt{SPL} and \texttt{SPH} are also predefined and evaluate to the address of the corresponding registers.
\flowgraph{\resource{AVR assembly\\source code} \ar[r] & \toolbox{avrasm} \ar[r] & \resource{object file}}
\seeassembly\seeavr\seeobject
}

\providecommand{\avrdism}{
\toolsection{avrdism} is a disassembler for the AVR hardware architecture.
It translates machine code from object files targeting AVR processors into assembly code and writes it to the standard output stream.
\flowgraph{\resource{object file} \ar[r] & \toolbox{avrdism} \ar[r] & \resource{disassembly\\listing}}
\seeassembly\seeavr\seeobject
}

\providecommand{\avrttasm}{
\toolsection{avr32asm} is an assembler for the AVR32 hardware architecture.
It translates assembly code into machine code for AVR32 processors and stores it in corresponding object files.
\flowgraph{\resource{AVR32 assembly\\source code} \ar[r] & \toolbox{avr32asm} \ar[r] & \resource{object file}}
\seeassembly\seeavrtt\seeobject
}

\providecommand{\avrttdism}{
\toolsection{avr32dism} is a disassembler for the AVR32 hardware architecture.
It translates machine code from object files targeting AVR32 processors into assembly code and writes it to the standard output stream.
\flowgraph{\resource{object file} \ar[r] & \toolbox{avr32dism} \ar[r] & \resource{disassembly\\listing}}
\seeassembly\seeavrtt\seeobject
}

\providecommand{\mabkasm}{
\toolsection{m68kasm} is an assembler for the M68000 hardware architecture.
It translates assembly code into machine code for M68000 processors and stores it in corresponding object files.
\flowgraph{\resource{68000 assembly\\source code} \ar[r] & \toolbox{m68kasm} \ar[r] & \resource{object file}}
\seeassembly\seemabk\seeobject
}

\providecommand{\mabkdism}{
\toolsection{m68kdism} is a disassembler for the M68000 hardware architecture.
It translates machine code from object files targeting M68000 processors into assembly code and writes it to the standard output stream.
\flowgraph{\resource{object file} \ar[r] & \toolbox{m68kdism} \ar[r] & \resource{disassembly\\listing}}
\seeassembly\seemabk\seeobject
}

\providecommand{\miblasm}{
\toolsection{miblasm} is an assembler for the MicroBlaze hardware architecture.
It translates assembly code into machine code for MicroBlaze processors and stores it in corresponding object files.
\flowgraph{\resource{MicroBlaze assembly\\source code} \ar[r] & \toolbox{miblasm} \ar[r] & \resource{object file}}
\seeassembly\seemibl\seeobject
}

\providecommand{\mibldism}{
\toolsection{mibldism} is a disassembler for the MicroBlaze hardware architecture.
It translates machine code from object files targeting MicroBlaze processors into assembly code and writes it to the standard output stream.
\flowgraph{\resource{object file} \ar[r] & \toolbox{mibldism} \ar[r] & \resource{disassembly\\listing}}
\seeassembly\seemibl\seeobject
}

\providecommand{\mipsaasm}{
\toolsection{mips32asm} is an assembler for the MIPS32 hardware architecture.
It translates assembly code into machine code for MIPS32 processors and stores it in corresponding object files.
\flowgraph{\resource{MIPS32 assembly\\source code} \ar[r] & \toolbox{mips32asm} \ar[r] & \resource{object file}}
\seeassembly\seemips\seeobject
}

\providecommand{\mipsadism}{
\toolsection{mips32dism} is a disassembler for the MIPS32 hardware architecture.
It translates machine code from object files targeting MIPS32 processors into assembly code and writes it to the standard output stream.
\flowgraph{\resource{object file} \ar[r] & \toolbox{mips32dism} \ar[r] & \resource{disassembly\\listing}}
\seeassembly\seemips\seeobject
}

\providecommand{\mipsbasm}{
\toolsection{mips64asm} is an assembler for the MIPS64 hardware architecture.
It translates assembly code into machine code for MIPS64 processors and stores it in corresponding object files.
\flowgraph{\resource{MIPS64 assembly\\source code} \ar[r] & \toolbox{mips64asm} \ar[r] & \resource{object file}}
\seeassembly\seemips\seeobject
}

\providecommand{\mipsbdism}{
\toolsection{mips64dism} is a disassembler for the MIPS64 hardware architecture.
It translates machine code from object files targeting MIPS64 processors into assembly code and writes it to the standard output stream.
\flowgraph{\resource{object file} \ar[r] & \toolbox{mips64dism} \ar[r] & \resource{disassembly\\listing}}
\seeassembly\seemips\seeobject
}

\providecommand{\mmixasm}{
\toolsection{mmixasm} is an assembler for the MMIX hardware architecture.
It translates assembly code into machine code for MMIX processors and stores it in corresponding object files.
The names of all special registers are predefined and evaluate to the corresponding number.
\flowgraph{\resource{MMIX assembly\\source code} \ar[r] & \toolbox{mmixasm} \ar[r] & \resource{object file}}
\seeassembly\seemmix\seeobject
}

\providecommand{\mmixdism}{
\toolsection{mmixdism} is a disassembler for the MMIX hardware architecture.
It translates machine code from object files targeting MMIX processors into assembly code and writes it to the standard output stream.
\flowgraph{\resource{object file} \ar[r] & \toolbox{mmixdism} \ar[r] & \resource{disassembly\\listing}}
\seeassembly\seemmix\seeobject
}

\providecommand{\orokasm}{
\toolsection{or1kasm} is an assembler for the OpenRISC 1000 hardware architecture.
It translates assembly code into machine code for OpenRISC 1000 processors and stores it in corresponding object files.
\flowgraph{\resource{OpenRISC 1000 assembly\\source code} \ar[r] & \toolbox{or1kasm} \ar[r] & \resource{object file}}
\seeassembly\seeorok\seeobject
}

\providecommand{\orokdism}{
\toolsection{or1kdism} is a disassembler for the OpenRISC 1000 hardware architecture.
It translates machine code from object files targeting OpenRISC 1000 processors into assembly code and writes it to the standard output stream.
\flowgraph{\resource{object file} \ar[r] & \toolbox{or1kdism} \ar[r] & \resource{disassembly\\listing}}
\seeassembly\seeorok\seeobject
}

\providecommand{\ppcaasm}{
\toolsection{ppc32asm} is an assembler for the PowerPC hardware architecture.
It translates assembly code into machine code for PowerPC processors and stores it in corresponding object files.
By default, the assembler generates machine code for the 32-bit operating mode defined by the PowerPC architecture.
\flowgraph{\resource{PowerPC assembly\\source code} \ar[r] & \toolbox{ppc32asm} \ar[r] & \resource{object file}}
\seeassembly\seeppc\seeobject
}

\providecommand{\ppcadism}{
\toolsection{ppc32dism} is a disassembler for the PowerPC hardware architecture.
It translates machine code from object files targeting PowerPC processors into assembly code and writes it to the standard output stream.
It assumes that the machine code was generated for the 32-bit operating mode defined by the PowerPC architecture.
\flowgraph{\resource{object file} \ar[r] & \toolbox{ppc32dism} \ar[r] & \resource{disassembly\\listing}}
\seeassembly\seeppc\seeobject
}

\providecommand{\ppcbasm}{
\toolsection{ppc64asm} is an assembler for the PowerPC hardware architecture.
It translates assembly code into machine code for PowerPC processors and stores it in corresponding object files.
By default, the assembler generates machine code for the 64-bit operating mode defined by the PowerPC architecture.
\flowgraph{\resource{PowerPC assembly\\source code} \ar[r] & \toolbox{ppc64asm} \ar[r] & \resource{object file}}
\seeassembly\seeppc\seeobject
}

\providecommand{\ppcbdism}{
\toolsection{ppc64dism} is a disassembler for the PowerPC hardware architecture.
It translates machine code from object files targeting PowerPC processors into assembly code and writes it to the standard output stream.
It assumes that the machine code was generated for the 64-bit operating mode defined by the PowerPC architecture.
\flowgraph{\resource{object file} \ar[r] & \toolbox{ppc64dism} \ar[r] & \resource{disassembly\\listing}}
\seeassembly\seeppc\seeobject
}

\providecommand{\riscasm}{
\toolsection{riscasm} is an assembler for the RISC hardware architecture.
It translates assembly code into machine code for RISC processors and stores it in corresponding object files.
The names of all special registers are predefined and evaluate to the corresponding number.
\flowgraph{\resource{RISC assembly\\source code} \ar[r] & \toolbox{riscasm} \ar[r] & \resource{object file}}
\seeassembly\seerisc\seeobject
}

\providecommand{\riscdism}{
\toolsection{riscdism} is a disassembler for the RISC hardware architecture.
It translates machine code from object files targeting RISC processors into assembly code and writes it to the standard output stream.
\flowgraph{\resource{object file} \ar[r] & \toolbox{riscdism} \ar[r] & \resource{disassembly\\listing}}
\seeassembly\seerisc\seeobject
}

\providecommand{\wasmasm}{
\toolsection{wasmasm} is an assembler for the WebAssembly architecture.
It translates assembly code into machine code for WebAssembly targets and stores it in corresponding object files.
The names of all special registers are predefined and evaluate to the corresponding number.
\flowgraph{\resource{WebAssembly assembly\\source code} \ar[r] & \toolbox{wasmasm} \ar[r] & \resource{object file}}
\seeassembly\seewasm\seeobject
}

\providecommand{\wasmdism}{
\toolsection{wasmdism} is a disassembler for the WebAssembly architecture.
It translates machine code from object files targeting WebAssembly targets into assembly code and writes it to the standard output stream.
\flowgraph{\resource{object file} \ar[r] & \toolbox{wasmdism} \ar[r] & \resource{disassembly\\listing}}
\seeassembly\seewasm\seeobject
}

% linker tools

\providecommand{\linklib}{
\toolsection{linklib} is an object file combiner.
It creates a static library file by combining all object files given to it into a single one.
\flowgraph{\resource{object files} \ar[r] & \toolbox{linklib} \ar[r] & \resource{library file}}
\seeobject
}

\providecommand{\linkbin}{
\toolsection{linkbin} is a linker for plain binary files.
It links all object files given to it into a single image and stores it in a binary file that begins with the first linked section.
It also creates a map file that lists the address, type, name and size of all used sections.
The filename extension of the resulting binary file can be specified by putting it into a constant data section called \texttt{\_extension}.
\flowgraph{\resource{object files} \ar[r] & \toolbox{linkbin} \ar[r] \ar[d] & \resource{binary file} \\ & \resource{map file}}
\seeobject
}

\providecommand{\linkmem}{
\toolsection{linkmem} is a linker for plain binary files partitioned into random-access and read-only memory.
It links all object files given to it into two distinct images, one for data sections and one for code and constant data sections, and stores each image in a binary file that begins with the first linked section of the corresponding type.
It also creates a map file that lists the address, type, name and size of all used sections.
\flowgraph{\resource{object files} \ar[r] & \toolbox{linkmem} \ar[r] \ar[d] & \resource{RAM file/\\ROM file} \\ & \resource{map file}}
\seeobject
}

\providecommand{\linkprg}{
\toolsection{linkprg} is a linker for GEMDOS executable files.
It links all object files given to it into a single image and stores the image in an Atari GEMDOS executable file~\cite{gemdosfile}.
It also creates a map file that lists the address relative to the text segment, type, name and size of all used sections.
The filename extension of the resulting executable file can be specified by putting it into a constant data section called \texttt{\_extension}.
The GEMDOS executable file format requires all patch patterns of absolute link patches to consist of four full bitmasks with descending offsets.
\flowgraph{\resource{object files} \ar[r] & \toolbox{linkprg} \ar[r] \ar[d] & \resource{executable file} \\ & \resource{map file}}
\seeobject
}

\providecommand{\linkhex}{
\toolsection{linkhex} is a linker for Intel HEX files.
It links all code sections of the object files given to it into single image and stores the image in an Intel HEX file~\cite{hexfile} that begins with the first linked section.
It also creates a map file that lists the address, type, name and size of all used sections.
\flowgraph{\resource{object files} \ar[r] & \toolbox{linkhex} \ar[r] \ar[d] & \resource{HEX file} \\ & \resource{map file}}
\seeobject
}

\providecommand{\mapsearch}{
\toolsection{mapsearch} is a debugging tool.
It searches map files generated by linker tools for the name of a binary section that encompasses a memory address read from the standard input stream.
If additionally provided with one or more object files, it also stores an excerpt thereof in a separate object file called map search result which only contains the identified binary section for disassembling purposes.
\flowgraph{& \resource{map files/\\object files} \ar[d] \\ \resource{memory\\address} \ar[r] & \toolbox{mapsearch} \ar[r] \ar[d] & \resource{section name/\\relative offset} \\ & \resource{object file\\excerpt}}
\seeobject
}

\renewcommand{\seeguide}{}

\startchapter{Getting Started}{User Guide}{guide}
{This \documentation{} provides a basic guide for novice users of the \ecs{}.
It explains how to get started with the toolchain and shows some typical use cases.
In the end, users will understand the functionality of the \ecs{} by building executable programs.}

\epigraph{Man f\"uhlt den Glanz von einer neuen Seite, \\ auf der noch alles werden kann.}{Rainer Maria Rilke}

\section{Introduction}

The \ecs{} features a set of development tools for several programming languages targeting a variety of hardware architectures.
This \documentation{} demonstrates how to use these tools in order to build executable applications.
The assumption is that the user already knows how to program in the programming languages presented in this \documentation{}.
For each programming language supported by the \ecs{} there are the following tools for processing and translating the source code:

\begin{itemize}
\item Pretty printers for reformatting the source code
\item Semantic checkers for analyzing the source code for semantic errors
\item Interpreters for executing the source code line by line
\item Compilers for translating the source code into executable machine code
\end{itemize}

All of these tools have the same functionality regardless of which programming language they actually implement.
Without loss of generality, the remainder of this guide therefore focuses exemplarily on the Oberon programming language.
\seeoberon
Regarding the hardware architectures on the other hand, the \ecs{} provides the following tools for each supported hardware architecture:

\begin{itemize}
\item Assemblers for translating assembly code into binary machine code
\item Disassemblers for translating machine code into human-readable text
\end{itemize}

The \ecs{} provides a couple of assemblers, all of which implement the same generic assembly language.
\seeassembly
In general, users can target any hardware architecture supported by the \ecs{} by just replacing the corresponding tools and their runtime support.
For instructional purposes, the remainder of this \documentation{} therefore assumes that the user targets the AMD64 hardware architecture.
\seeamd
Finally, for building executable programs generated by compilers and assemblers the \ecs{} provides the following tools:

\begin{itemize}
\item Linkers for combining binary data into an executable file
\item Debugging tools for processing debugging information
\end{itemize}

The debugging information generated by compilers and linkers allows debugging problematic programs and consists of a symbolic address mapping.
\seedebugging

\section{Prerequisites}

In order to process and build the examples shown in this guide, the user needs access to the following tools:

\begin{itemize}
\item \tool{obprint} for pretty printing Oberon modules
\item \tool{obcheck} for analyzing Oberon modules
\item \tool{obrun} for interpreting Oberon modules
\item \tool{obamd64} for compiling Oberon modules
\item \tool{amd64asm} for translating assembly code
\item \tool{amd64dism} for translating the generated machine code
\item \tool{linklib} for combining object files
\item \tool{linkbin} for linking an executable program
\item \tool{mapsearch} for debugging programs
\item \tool{dbgdwarf} for converting debugging information
\end{itemize}

The examples shown in Sections~\ref{sec:guidebasic} to~\ref{sec:guidedebugging} assume that all of these tools are executable from the current working directory.
The illustrated commands to invoke these tools may therefore have to be adjusted to their actual location and may additionally require a suffix for executable files.
Furthermore, all executable programs created using the compiler require runtime support which is provided by the following object and library files:

\begin{itemize}
\item \file{obamd64run.lib} required by the programming language
\item \file{amd64run.obf} required by the hardware architecture
\item \file{win64run.obf}, \file{osx64run.obf}, or \file{amd64linuxrun.obf} required by runtime environments like Windows, OS~X, or Linux-based operating systems
\end{itemize}

These files are assumed to be located in the current working directory as well and may therefore also have to be adjusted accordingly.
They store binary data such as the machine code of precompiled runtime support for the respective hardware architecture and runtime environment.
\seeobject

Installations of the \ecs{} typically feature a driver utility tool called \tool{ecsd} which is able to automatically locate and invoke all of the tools discussed in this section.
It also provides the necessary runtime support without requiring users to explicitly name any of the prerequisites listed above.
See Section~\ref{sec:guidedriver} for more information about how to run the examples described in Sections~\ref{sec:guidebasic} to~\ref{sec:guidedebugging} using the \ecs{} driver.
\ifbook For more information about this tool in general, see Chapter~\ref{interface}. \fi

The remainder of this section gives more detailed information about the required tools.
\interface

\renewcommand{\seeoberon}{}
\renewcommand{\seeassembly}{}
\renewcommand{\seeamd}{}
\renewcommand{\seedebugging}{}
\renewcommand{\seeobject}{}

\obprint
\obcheck
\obrun
\obamdc
\amdcasm
\amdcdism
\linklib
\linkbin
\mapsearch
\dbgdwarf

\section{Basic Processing}\label{sec:guidebasic}

The first example is a very simple Oberon module that defines a global variable, assigns a value to it and prints that value.
The corresponding source code looks as follows:

\begin{quote}\begin{verbatim}
MODULE Simple;
VAR variable: INTEGER;
BEGIN
  variable := 10;
  TRACE (variable);
END Simple.
\end{verbatim}\end{quote}

The remainder of this section assumes that these six lines of source code are stored in a plain text file called \file{simple.mod} in the current working directory.

\subsection{Pretty Printing}

Executing the pretty printer should yield the very same code in a slightly different layout and helps to check whether the source was syntactically recognized.
The corresponding command-line call looks like the following and produces the subsequent output:

\begin{quote}\begin{verbatim}
obprint simple.mod
MODULE Simple;

VAR
  variable: INTEGER;

BEGIN
  variable := 10;
  TRACE (variable);
END Simple.
\end{verbatim}\end{quote}

\subsection{Semantic Checking}

For checking whether the source code is also semantically valid one can execute the semantic checker using the following command-line call:

\begin{quote}\begin{verbatim}
obcheck simple.mod
\end{verbatim}\end{quote}

Since the input was semantically valid, the invocation of this tool does not generate any output message and succeeds.
However, if line 5 of the original source code was for example changed to \texttt{TRACE (x);} then the tool would identify a violation of the language rules:

\begin{quote}\begin{verbatim}
obcheck simple.mod
simple.mod:5:10: error: undeclared identifier 'x'
\end{verbatim}\end{quote}

If the semantic checker succeeds, it generates a plain text file called \file{simple.sym} in the current working directory which contains additional semantic information about the source code.
In the case of Oberon, this file is a symbol file containing the interface of the sample module which is required by other modules importing it.

\subsection{Interpreting}

Valid modules can be executed by the interpreter which simulates a runtime environment for the Oberon language.
The corresponding command-line call looks like the following and produces the subsequent output:

\begin{quote}\begin{verbatim}
obrun simple.mod
simple.mod:5:10: note: 'variable' = 10
\end{verbatim}\end{quote}

The interpreter does not generate any output files.
It just executes the source code line by line by first assigning the value 10 to the variable and then printing its value.

\section{Building Simple Programs}\label{sec:guidesimple}

Running code using the interpreter is quite slow compared to executing the same program directly on the machine.
This section shows how the same source code can be translated into an executable program.

\subsection{Compiling}

The compiler translates the source code of the sample module into machine code using the following command-line call:

\begin{quote}\begin{verbatim}
obamd64 simple.mod
\end{verbatim}\end{quote}

If its invocation is successful, the compiler generates four different plain text files for this module.
The first file \file{simple.sym} is the same symbol file as generated by the semantic checker.
The second file \file{simple.obf} is an object file which stores the binary encoding of the machine code translated from the source code.
The third file \file{simple.lst} is a human-readable assembly code listing of the generated machine code.
The fourth file \file{simple.dbg} contains additional symbolic debugging information about the module.
The last two files are actually not required for building the executable program but are generated for debugging purposes.
Invoking the assembler on the assembly code listing using the following command-line call yields exactly the same object file:

\begin{quote}\begin{verbatim}
amd64asm simple.lst
\end{verbatim}\end{quote}

\subsection{Linking}

Although the generated object file contains executable machine code, it is not executable by itself.
The machine code has first to be combined with the required runtime support for the programming language, the hardware architecture, as well as the runtime environment.
This process is called linking and can be executed by invoking the linker using the following command-line call.
The actual order of the command-line arguments is not important but the linker takes the filename of the first object file as a basis for its output files:

\begin{quote}\begin{verbatim}
linkbin simple.obf obamd64run.lib amd64run.obf win64run.obf
\end{verbatim}\end{quote}

If this call succeeds, the linker generates a binary image of an executable program and stores it in a file called \file{simple.exe}.
This file can be executed on a machine running the Windows operating system producing exactly the same output as the interpreter above.
For creating a similar executable file called \file{simple} for the OS~X operating system, only the last command-line argument has to be changed:

\begin{quote}\begin{verbatim}
linkbin simple.obf obamd64run.lib amd64run.obf osx64run.obf
\end{verbatim}\end{quote}

This executable program still needs the same runtime support for the programming language and hardware architecture.
The same holds for programs that can be executed on Linux-based operating systems.
Again only the last command-line argument has to be modified accordingly:

\begin{quote}\begin{verbatim}
linkbin simple.obf obamd64run.lib amd64run.obf amd64linuxrun.obf
\end{verbatim}\end{quote}

In all three cases, the linker also generates a plain text file called \file{simple.map} summarizing the mapping of symbol names in the source code to their linked addresses.
This information is provided for debugging purposes and used in Section~\ref{sec:guidedebugging}.

\section{Building Complex Programs}\label{sec:guidecomplex}

The second example makes use of the interoperability features of the \ecs{}.
It shows how code from one programming language can access code and data defined in another language and vice versa.

\subsection{Compiling}

Here is an extended version of the first example.
Instead of assigning the value to the global variable directly, this version calls a procedure that overwrites the value of the variable.
In the end however, it should generate the same output as the previous example:

\begin{quote}\begin{verbatim}
MODULE Complex;
IMPORT SYSTEM;
VAR variable: INTEGER;
PROCEDURE ^ Assign ["assign"] (value: INTEGER);
BEGIN
  Assign (10);
  TRACE (variable);
END Complex.
\end{verbatim}\end{quote}

The forward declaration of the procedure is marked as external which causes the compiler to refer to the procedure using the given external name instead of defining a new procedure.
If the source code is stored in a plain text file called \file{complex.mod}, the compiler can be invoked using the following command-line call:

\begin{quote}\begin{verbatim}
obamd64 complex.mod
\end{verbatim}\end{quote}

The resulting plain text files are called \file{complex.sym}, \file{complex.obf}, \file{complex.lst}, and \file{complex.dbg}.
However, linking the generated object file together with the runtime support for any runtime environment as before yields the following linker error:

\begin{quote}\begin{verbatim}
linkbin complex.obf obamd64run.lib amd64run.obf amd64linuxrun.obf
Complex._body: error: unresolved symbol 'assign'
\end{verbatim}\end{quote}

The diagnostic message states that the body of the module calls a procedure called \texttt{assign} which is defined nowhere.
This behavior is intended because all external procedures need to be defined by another part of the program.
The following section shows how to implement it using assembly language.

\subsection{Assembling}

The missing \texttt{assign} procedure is defined in the following assembly source code.
It contains instructions which copy the procedure argument into a register and from there into the global variable:

\begin{quote}\begin{verbatim}
.code assign
  mov  ebx, [rsp + 8]
  mov  [@Complex.variable], ebx
  ret
\end{verbatim}\end{quote}

Assuming these four lines of source code are stored in a plain text file called \file{assign.asm}, the assembler can be invoked using the following command-line call:

\begin{quote}\begin{verbatim}
amd64asm assign.asm
\end{verbatim}\end{quote}

This generates an object file called \file{assign.obf} which stores the binary representation of the corresponding machine code of this procedure.
Its contents can be listed by the disassembler using the following command-line call which produces the subsequent output:

\begin{quote}\begin{verbatim}
amd64dism assign.obf
.code assign
   0  8b5c2408        mov  ebx, dword [rsp + 8]
   4  891c2500000000  mov  dword [0], ebx  ; @Complex.variable
  11  c3              ret
  12
\end{verbatim}\end{quote}

The disassembler lists all three instructions which require twelve octets of code space in total.
Each instruction is prefixed with its relative offset and binary encoding.

\subsection{Combining}

The following step is optional but shows how the object files generated by the compiler and the assembler can be merged together.
The result is a new object file called a library file which combines the machine code of the Oberon module with the procedure defined in the assembly code.
It is generated using the following command-line call:

\begin{quote}\begin{verbatim}
linklib complex.obf assign.obf
\end{verbatim}\end{quote}

This call generates a new library file called \file{complex.lib} which contains the contents of both object files.
This file can be used to link the executable program in the next step.

\subsection{Linking}

Since the library file contains the machine code of both the module and the \texttt{assign} procedure, the linker should not fail any longer.
The corresponding command-line call looks like the following:

\begin{quote}\begin{verbatim}
linkbin complex.lib obamd64run.lib amd64run.obf amd64linuxrun.obf
\end{verbatim}\end{quote}

This call generates a new binary executable file called \file{complex} that produces the same output as before when executed.
The second generated file \file{complex.map} contains the mapping of symbol names in the source code to their linked addresses and includes the new \texttt{assign} procedure.

The linker could have also been called using the two object files generated by the assembler and the compiler instead of the single library file yielding exactly the same result.
This combining and linking works for arbitrary object files generated by the various development tools of the \ecs{} as long as they target the same hardware architecture.

\section{Debugging Programs}\label{sec:guidedebugging}

Compilers and linkers provided by the \ecs{} generate various files for debugging purposes.
This section shows how this symbolic debugging information can be used to debug problematic programs.

\subsection{Basic Debugging}

If an erroneous program fails, the corresponding error message of the runtime environment or debugger often mentions the address of the first problematic instruction.
The \ecs{} provides a debugging tool called \tool{mapsearch} which allows searching map files generated by linker tools for the section encompassing the reported address.
This is helpful for identifying the executed code section and all other active procedures in a stack trace.
The tool reads a single address from the standard input stream and prints the name of the identified section if it is listed in the specified map file.
For convenience, the address can be provided by a single command-line call as follows:

\begin{quote}\begin{verbatim}
echo 0x80482a7 | mapsearch complex.map
mapsearch: note: offset 4 in code section 'assign'
\end{verbatim}\end{quote}

The address in this particular example corresponds to the instruction at offset four relative to the \texttt{assign} procedure.
If additionally provided with the object file containing the identified section, the tool also generates a separate object file called \file{complex.msr} which can be used to disassemble only that section:

\begin{quote}\begin{verbatim}
echo 0x80482a7 | mapsearch complex.map complex.lib
amd64dism complex.msr
\end{verbatim}\end{quote}

Searching map files generated by linkers for problematic addresses is a rather limited debugging facility but always available.

\subsection{Advanced Debugging}

For runtime environments that support more spohisticated debuggers, the \ecs{} provides tools that convert the symbolic debugging information generated by its compilers alongside an object file into a suitable format.
This debugging information consists of an abstract representation of all programming language constructs compiled into the object file and can be converted as follows for example:

\begin{quote}\begin{verbatim}
dbgdwarf complex.dbg
\end{verbatim}\end{quote}

This generates a debugging object file called \file{complex.dbf} which can optionally be provided together with the corresponding object file when linking for Linux-based operating systems:

\begin{quote}\begin{verbatim}
linkbin complex.obf complex.dbf assign.obf
        obamd64run.lib amd64run.obf amd64linuxrun.obf
\end{verbatim}\end{quote}

The resulting binary executable file behaves as before but additionally contains debugging information that enables interactive program animation and memory inspection when loaded in an external debugger.

\section{Using the \ecs{} Driver}\label{sec:guidedriver}

In addition to invoking tools directly as shown above, users can also call the \tool{ecsd} driver utility tool which is typically provided when installing the \ecs{}.
It allows compiling and linking complete executable files for a specific runtime environment in one step and automatically provides the necessary runtime support.

\flowgraph{\resource{input\\files} \ar[r] & \converter{ecsd} \ar[r] & \resource{executable\\file}}

In contrast to all other tools of the \ecs{}, the driver tool does support the notion of command-line options to allow users to influence how tools are identified and invoked.
The following command-line call prints a list of all supported command-line arguments:

\begin{quote}\begin{verbatim}
ecsd -h
\end{verbatim}\end{quote}

The remainder of this section consistently uses a short notation for options although all of them have also a longer and more descriptive form prefixed by two dashes.
\ifbook For more information about the user interface of this tool, see Chapter~\ref{interface}. \fi

\subsection{Basic Processing}

All of the tools mentioned in Section~\ref{sec:guidebasic} can also be invoked using the \ecs{} driver by putting \texttt{ecsd~-i} in front of the corresponding command-line call.
This particular option prompts the driver to process the input files given as command-line arguments using the specified tool.
Provided that the corresponding tools are available, the command-line calls for pretty printing, semantic checking, and interpreting for example look like the following:

\begin{quote}\begin{verbatim}
ecsd -i obprint simple.mod
ecsd -i obcheck simple.mod
ecsd -i obrun simple.mod
\end{verbatim}\end{quote}

The actual command-line call used by the driver to invoke the respective tool can be shown by adding the \texttt{-v}~flag.
This enables verbose mode which is especially helpful when building programs as shown in the following section.

\subsection{Building Programs}

If the driver is called without command-line options, it builds complete executable files for a specific runtime environment as described in Section~\ref{sec:guidesimple}.
It infers the set of required tools like compilers, assemblers, and linkers from the type of its input files and automatically invokes one tool after the other with appropriate command-line arguments.
By default, it tries to target its own runtime environment:

\begin{quote}\begin{verbatim}
ecsd simple.mod
\end{verbatim}\end{quote}

In order to explicitly target a runtime environment like Windows, OS~X, or Linux-based operating systems for example, use the \texttt{-t}~option as follows.
A list of all available target environments is accessible using the \texttt{-h}~flag:

\begin{quote}\begin{verbatim}
ecsd -t win64 simple.mod
ecsd -t osx64 simple.mod
ecsd -t amd64linux simple.mod
\end{verbatim}\end{quote}

Using the driver to build programs still generates intermediate files like object files and assembly code listings as before.
Use the \texttt{-c}~flag to compile and assemble input files without invoking the linker at the end:

\begin{quote}\begin{verbatim}
ecsd -c simple.mod
\end{verbatim}\end{quote}

The driver can also be called using the generated intermediate files.
However, because the driver cannot infer which compiler was used to generate object files or assembly code listings in the first place, it must be explicitly told which additional runtime support to include this time.
In the case of the Oberon programming language, the required runtime support is made available using the \texttt{-O}~flag:

\begin{quote}\begin{verbatim}
ecsd -O simple.obf
ecsd -O simple.lst
\end{verbatim}\end{quote}

The driver is able to process several different input files at once which allows building complex programs as described in Section~\ref{sec:guidecomplex}:

\begin{quote}\begin{verbatim}
ecsd complex.mod assign.asm
\end{verbatim}\end{quote}

Combining input files into a single object file can be achieved using the \texttt{-l}~flag.
As before, the resulting library file requires additional runtime support to be specified explicitly when building the program:

\begin{quote}\begin{verbatim}
ecsd -l complex.mod assign.obf
ecsd -O complex.lib
\end{verbatim}\end{quote}

In order to disassemble source code or object files instead of linking them, the driver can be invoked using the \texttt{-d}~flag:

\begin{quote}\begin{verbatim}
ecsd -d assign.asm
ecsd -d assign.obf
\end{verbatim}\end{quote}

Generally, the type of an input file is inferred from its filename extension.
The \texttt{-s}~option allows users to explicitly specify the source type of unknown input files.
A list of all supported source types is accessible using the \texttt{-h}~flag.

\subsection{Debugging Programs}

The \ecs{} driver also allows processing map files generated by linker tools.
This is useful for identifying sections that contain problematic instructions in erroneous programs as described in Section~\ref{sec:guidedebugging}.
Building a complex program as described in the previous section for example generates a file called \file{complex.map}:

\begin{quote}\begin{verbatim}
ecsd complex.mod assign.asm
\end{verbatim}\end{quote}

This file can be passed to the driver which then invokes a debugging tool that reads an address from the standard input stream in order to search for a linked section that encompasses that address.
If the address is mapped, the tool prints the name of the corresponding section and the relative offset of the address:

\begin{quote}\begin{verbatim}
echo 0x4011fe | ecsd complex.map
mapsearch: note: offset 4 in code section 'assign'
\end{verbatim}\end{quote}

If additionally provided with the object file or runtime support containing the identified section, the driver subsequently disassembles the latter when using the \texttt{-d}~flag.
In this particular example, the input address points to the second instruction of the \texttt{assign} procedure written in assembly code:

\begin{quote}\begin{verbatim}
echo 0x4011fe | ecsd -d complex.map assign.obf
mapsearch: note: offset 4 in code section 'assign'
.code assign
   0  8b5c2408        mov  ebx, dword [rsp + 8]
   4  891c2500000000  mov  dword [0], ebx  ; @Complex.variable
  11  c3              ret
  12
\end{verbatim}\end{quote}

For more advanced debugging using external debuggers on runtime environments that support them, the \ecs{} driver automatically converts and includes the symbolic debugging information generated by compilers using the \texttt{-g} flag:

\begin{quote}\begin{verbatim}
ecsd -g simple.mod
ecsd -g complex.mod assign.asm
\end{verbatim}\end{quote}

This generates executable files that incorporate the address mapping of all symbols and instructions necessary for interactive program animation and memory inspection when loaded in a debugger.

\concludechapter

% Eigen Compiler Suite tool reference
% Copyright (C) Florian Negele

% This file is part of the Eigen Compiler Suite.

% Permission is granted to copy, distribute and/or modify this document
% under the terms of the GNU Free Documentation License, Version 1.3
% or any later version published by the Free Software Foundation.

% You should have received a copy of the GNU Free Documentation License
% along with the ECS.  If not, see <https://www.gnu.org/licenses/>.

% Generic documentation utilities
% Copyright (C) Florian Negele

% This file is part of the Eigen Compiler Suite.

% Permission is granted to copy, distribute and/or modify this document
% under the terms of the GNU Free Documentation License, Version 1.3
% or any later version published by the Free Software Foundation.

% You should have received a copy of the GNU Free Documentation License
% along with the ECS.  If not, see <https://www.gnu.org/licenses/>.

\providecommand{\cpp}{C\texttt{++}}
\providecommand{\opt}{_\mathit{opt}}
\providecommand{\tool}[1]{\texttt{#1}}
\providecommand{\version}{Version 0.0.40}
\providecommand{\resource}[1]{*++\txt{#1}}
\providecommand{\ecs}{Eigen Compiler Suite}
\providecommand{\changed}[1]{\underline{#1}}
\providecommand{\toolbox}[1]{\converter{#1}}
\providecommand{\file}{}\renewcommand{\file}[1]{\texttt{#1}}
\providecommand{\alignright}{\hfill\linebreak[0]\hspace*{\fill}}
\providecommand{\converter}[1]{*++[F][F*:white][F,:gray]\txt{#1}}
\providecommand{\documentation}{\ifbook chapter\else document\fi}
\providecommand{\Documentation}{\ifbook Chapter\else Document\fi}
\providecommand{\variable}[1]{\resource{\texttt{\small#1}\\variable}}
\providecommand{\documentationref}[2]{\ifbook\ref{#1}\else``\href{#1}{#2}''~\cite{#1}\fi}
\providecommand{\objfile}[1]{\texttt{#1}\index[runtime]{#1 object file@\texttt{#1} object file}}
\providecommand{\libfile}[1]{\texttt{#1}\index[runtime]{#1 library file@\texttt{#1} library file}}
\providecommand{\epigraph}[2]{\ifbook\begin{quote}\flushright\textit{#1}\par--- #2\end{quote}\fi}
\providecommand{\environmentvariable}[1]{\texttt{#1}\index{Environment variables!#1@\texttt{#1}}}
\providecommand{\environment}[1]{\texttt{#1}\index[environment]{#1 environment@\texttt{#1} environment}}
\providecommand{\toolsection}{}\renewcommand{\toolsection}[1]{\subsection{#1}\label{\prefix:#1}\tool{#1}}
\providecommand{\instruction}{}\renewcommand{\instruction}[2]{\noindent\qquad\pdftooltip{\texttt{#1}}{#2}\refstepcounter{instruction}\par}
\providecommand{\flowgraph}{}\renewcommand{\flowgraph}[1]{\par\sffamily\begin{displaymath}\xymatrix@=4ex{#1}\end{displaymath}\normalfont\par}
\providecommand{\instructionset}{}\renewcommand{\instructionset}[4]{\setcounter{instruction}{0}\begin{multicols}{\ifbook#3\else#4\fi}[{\captionof{table}[#2]{#2 (\ref*{#1:instructions}~instructions)}\label{tab:#1set}\vspace{-2ex}}]\footnotesize\raggedcolumns\input{#1.set}\label{#1:instructions}\end{multicols}}

\providecommand{\gpl}{GNU General Public License}
\providecommand{\rse}{ECS Runtime Support Exception}
\providecommand{\fdl}{\href{https://www.gnu.org/licenses/fdl.html}{GNU Free Documentation License}}

\providecommand{\docbegin}{}
\providecommand{\docend}{}
\providecommand{\doclabel}[1]{\hypertarget{#1}}
\providecommand{\doclink}[2]{\hyperlink{#1}{#2}}
\providecommand{\docsection}[3]{\hypertarget{#1}{\subsection{#2}}\label{sec:#1}\index[library]{#2@#3}}
\providecommand{\docsectionstar}[1]{}
\providecommand{\docsubbegin}{\begin{description}}
\providecommand{\docsubend}{\end{description}}
\providecommand{\docsubsection}[3]{\item[\hypertarget{#1}{#2}]\index[library]{#2@#3}}
\providecommand{\docsubsectionstar}[1]{\smallskip}
\providecommand{\docsubsubsection}[3]{\docsubsection{#1}{#2}{#3}}
\providecommand{\docsubsubsectionstar}[1]{}
\providecommand{\docsubsubsubsection}[3]{}
\providecommand{\docsubsubsubsectionstar}[1]{}
\providecommand{\doctable}{}

\providecommand{\debuggingtool}{}\renewcommand{\debuggingtool}{This tool is provided for debugging purposes.
It allows exposing and modifying an internal data structure that is usually not accessible.
}

\providecommand{\interface}{All tools accept command-line arguments which are taken as names of plain text files containing the source code.
If no arguments are provided, the standard input stream is used instead.
Output files are generated in the current working directory and have the same name as the input file being processed whereas the filename extension gets replaced by an appropriate suffix.
\seeinterface
}

\providecommand{\license}{\noindent Copyright \copyright{} Florian Negele\par\medskip\noindent
Permission is granted to copy, distribute and/or modify this document under the terms of the
\fdl{}, Version 1.3 or any later version published by the \href{https://fsf.org/}{Free Software Foundation}.
}

\providecommand{\ecslogosurface}{
\fill[darkgray] (0,0,0) -- (0,0,3) -- (0,3,3) -- (0,3,1) -- (0,4,1) -- (0,4,3) -- (0,5,3) -- (0,5,0) -- (0,2,0) -- (0,2,2) -- (0,1,2) -- (0,1,0) -- cycle;
\fill[gray] (0,5,0) -- (0,5,3) -- (1,5,3) -- (1,5,1) -- (2,5,1) -- (2,5,3) -- (3,5,3) -- (3,5,0) -- cycle;
\fill[lightgray] (0,0,0) -- (0,1,0) -- (2,1,0) -- (2,4,0) -- (1,4,0) -- (1,3,0) -- (2,3,0) -- (2,2,0) -- (0,2,0) -- (0,5,0) -- (3,5,0) -- (3,0,0) -- cycle;
\begin{scope}[line width=0.5]
\begin{scope}[gray]
\draw (0,0,0) -- (0,1,0);
\draw (2,1,0) -- (2,2,0);
\draw (0,1,2) -- (0,2,2);
\draw (0,2,0) -- (0,5,0);
\draw (2,3,0) -- (2,4,0);
\end{scope}
\begin{scope}[lightgray]
\draw (0,1,0) -- (0,1,2);
\draw (0,3,1) -- (0,3,3);
\draw (0,5,0) -- (0,5,3);
\draw (2,5,1) -- (2,5,3);
\end{scope}
\begin{scope}[white]
\draw (0,1,0) -- (2,1,0);
\draw (1,3,0) -- (2,3,0);
\draw (0,5,0) -- (3,5,0);
\end{scope}
\end{scope}
}

\providecommand{\ecslogo}[1]{
\begin{tikzpicture}[scale={(#1)/((sin(45)+cos(45))*3cm)},x={({-cos(45)*1cm},{sin(45)*sin(30)*1cm})},y={({0cm},{(cos(30)*1cm})},z={({sin(45)*1cm},{cos(45)*sin(30)*1cm})}]
\begin{scope}[darkgray,line width=1]
\draw (0,0,0) -- (0,0,3) -- (0,3,3) -- (2,3,3) -- (2,5,3) -- (3,5,3) -- (3,5,0) -- (3,0,0) -- cycle;
\draw (0,3,1) -- (0,4,1) -- (0,4,3) -- (0,5,3) -- (1,5,3) -- (1,5,1) -- (2,5,1);
\draw (1,3,0) -- (1,4,0) -- (2,4,0);
\end{scope}
\fill[darkgray] (2,0,0) -- (2,0,3) -- (2,5,3) -- (2,5,1) -- (2,4,1) -- (2,4,0) -- cycle;
\fill[lightgray] (2,0,2) -- (0,0,2) -- (0,2,2) -- (2,2,2) -- cycle;
\fill[gray] (0,1,0) -- (2,1,0) -- (2,1,2) -- (0,1,2) -- cycle;
\fill[gray] (0,3,1) -- (0,3,3) -- (2,3,3) -- (2,3,0) -- (1,3,0) -- (1,3,1) -- cycle;
\ecslogosurface
\end{tikzpicture}
}

\providecommand{\shadowedecslogo}[3]{
\begin{tikzpicture}[scale={(#1)/((sin(#2)+cos(#2))*3cm)},x={({-cos(#2)*1cm},{sin(#2)*sin(#3)*1cm})},y={({0cm},{(cos(#3)*1cm})},z={({sin(#2)*1cm},{cos(#2)*sin(#3)*1cm})}]
\shade[top color=lightgray!50!white,bottom color=white,middle color=lightgray!50!white] (0,0,0) -- (3,0,0) -- (3,{-0.5-3*sin(#2)*sin(#3)/cos(#3)},0) -- (0,-0.5,0) -- cycle;
\shade[top color=darkgray!50!gray,bottom color=white,middle color=darkgray!50!white] (0,0,0) -- (0,0,3) -- (0,{-0.5-3*cos(#2)*sin(#3)/cos(#3)},3) -- (0,-0.5,0) -- cycle;
\begin{scope}[y={({(cos(#2)+sin(#2))*0.5cm},{(cos(#2)*sin(#3)-sin(#2)*sin(#3))*0.5cm})}]
\useasboundingbox (3,0,0) -- (0,0,0) -- (0,0,3);
\shade[left color=darkgray!80!black,right color=lightgray,middle color=gray] (0,0,0) -- (0,1,0) -- (0,1,0.5) -- (0,2,0) -- (0,5,0) -- (0,5,3) -- (1,5,3) -- (1,4,3) -- (1,4,2.5) -- (1,3,3) -- (2,5,3) -- (3,5,3) -- (3,0,3) -- cycle;
\clip (0,0,0) -- (0,0,3) -- ({-3*sin(#2)/cos(#2)},0,0) -- cycle;
\shade[left color=darkgray,right color=lightgray!50!gray] (0,0,0) -- (0,1,0) -- (0,1,0.5) -- (0,2,0) -- (0,5,0) -- (0,5,3) -- (1,5,3) -- (1,4,3) -- (1,4,2.5) -- (1,3,3) -- (2,5,3) -- (3,5,3) -- (3,0,3) -- cycle;
\end{scope}
\shade[left color=darkgray,right color=darkgray!80!black] (2,0,0) -- (2,0,3) -- (2,5,3) -- (2,5,1) -- (2,4,1) -- (2,4,0) -- cycle;
\shade[left color=darkgray!90!black,right color=gray!80!darkgray] (2,0,2) -- (0,0,2) -- (0,2,2) -- (2,2,2) -- cycle;
\shade[top color=darkgray!90!black,bottom color=gray!80!darkgray] (0,1,0) -- (2,1,0) -- (2,1,2) -- (0,1,2) -- cycle;
\shade[top color=darkgray!90!black,bottom color=gray!80!darkgray] (0,3,1) -- (0,3,3) -- (2,3,3) -- (2,3,0) -- (1,3,0) -- (1,3,1) -- cycle;
\fill[gray] (2,1,0) -- (1.5,1,0.5) -- (0,1,0.5) -- (0,1,0) -- cycle;
\fill[gray] (1,3,2) -- (0.5,3,2) -- (0.5,3,3) -- (1,3,3) -- cycle;
\fill[gray] (2,3,0) -- (1.5,3,0.5) -- (1,3,0.5) -- (1,3,0) -- cycle;
\ecslogosurface
\end{tikzpicture}
}

\providecommand{\cpplogo}[1]{
\begin{tikzpicture}[scale=(#1)/512em]
\fill[gray] (435.2794,398.7159) -- (247.1911,507.3075) .. controls (236.3563,513.5642) and (218.6240,513.5642) .. (207.7892,507.3075) -- (19.7009,398.7159) .. controls (8.8646,392.4606) and (0.0000,377.1043) .. (0.0000,364.5924) -- (0.0000,147.4076) .. controls (0.8430,132.8363) and (8.2856,120.7683) .. (19.7009,113.2842) -- (207.7892,4.6926) .. controls (218.6240,-1.5642) and (236.3564,-1.5642) .. (247.1911,4.6926) -- (435.2794,113.2842) .. controls (447.5273,121.4304) and (454.4987,133.6918) .. (454.9803,147.4076) -- (454.9803,364.5924) .. controls (454.5404,377.7571) and (446.6566,391.0351) .. (435.2794,398.7159) -- cycle(75.8301,255.9993) .. controls (74.9389,404.0881) and (273.2892,469.4783) .. (358.8263,331.8769) -- (293.1917,293.8965) .. controls (253.5702,359.4301) and (155.1909,335.9977) .. (151.6601,255.9993) .. controls (152.7204,182.2703) and (249.4137,148.0211) .. (293.1961,218.1065) -- (358.8308,180.1276) .. controls (283.4477,49.2645) and (79.6318,96.3470) .. (75.8301,255.9993) -- cycle(379.1503,247.5747) -- (362.2982,247.5747) -- (362.2982,230.7226) -- (345.4490,230.7226) -- (345.4490,247.5747) -- (328.5969,247.5747) -- (328.5969,264.4254) -- (345.4490,264.4254) -- (345.4490,281.2759) -- (362.2982,281.2759) -- (362.2982,264.4254) -- (379.1503,264.4254) -- cycle(442.3420,247.5747) -- (425.4899,247.5747) -- (425.4899,230.7226) -- (408.6408,230.7226) -- (408.6408,247.5747) -- (391.7886,247.5747) -- (391.7886,264.4254) -- (408.6408,264.4254) -- (408.6408,281.2759) -- (425.4899,281.2759) -- (425.4899,264.4254) -- (442.3420,264.4254) -- cycle;
\end{tikzpicture}
}

\providecommand{\fallogo}[1]{
\begin{tikzpicture}[scale=(#1)/512em]
\fill[gray] (185.7774,0.0000) .. controls (200.4486,15.9798) and (226.8966,8.7148) .. (235.0426,31.5836) .. controls (249.5297,58.0598) and (247.9581,97.9161) .. (280.3335,110.9762) .. controls (309.1690,120.3496) and (337.8406,104.2727) .. (366.5753,103.9379) .. controls (373.4449,111.5171) and (379.2885,128.2574) .. (383.9755,108.9744) .. controls (396.6979,102.5615) and (437.2808,107.6681) .. (426.9652,124.3252) .. controls (408.9822,121.0785) and (412.4742,146.0729) .. (426.5192,131.4996) .. controls (433.8413,120.8489) and (465.1541,126.5522) .. (441.9067,135.7950) .. controls (396.1879,157.7478) and (344.1112,161.5079) .. (298.5528,183.5702) .. controls (277.7471,193.5198) and (284.6941,218.7163) .. (285.2127,236.9640) .. controls (292.3599,316.2826) and (307.3929,394.6311) .. (317.1198,473.6154) .. controls (329.0637,505.4736) and (292.1195,528.5004) .. (265.9183,511.2761) .. controls (237.9284,499.2462) and (237.3684,465.2681) .. (230.9102,439.9421) .. controls (218.6692,374.3397) and (215.6307,306.9662) .. (198.1732,242.3977) .. controls (183.1379,232.7444) and (164.4245,256.0298) .. (149.0430,261.4799) .. controls (116.9328,279.2585) and (87.1822,308.5851) .. (48.2293,307.8914) .. controls (21.3220,306.9037) and (-15.9107,281.8761) .. (7.2921,252.7908) .. controls (29.7799,220.6177) and (67.5177,204.3028) .. (100.9287,185.9449) .. controls (130.8217,170.8906) and (161.1548,156.5903) .. (191.0278,141.5847) .. controls (196.1738,120.0520) and (186.6049,95.2409) .. (186.8382,72.4353) .. controls (185.5234,48.4204) and (183.1700,23.9341) .. (185.7774,0.0000) -- cycle;
\end{tikzpicture}
}

\providecommand{\oblogo}[1]{
\begin{tikzpicture}[scale=(#1)/512em]
\fill[gray] (160.3865,208.9117) .. controls (154.0879,214.6478) and (149.0735,221.2409) .. (145.4125,228.5384) .. controls (184.8790,248.4273) and (234.7122,269.8787) .. (297.5493,291.8782) .. controls (300.3943,281.4769) and (300.9552,268.7619) .. (300.4023,255.2389) .. controls (248.9909,244.7891) and (200.0310,225.9279) .. (160.3865,208.9117) -- cycle(225.7398,392.6996) .. controls (308.0209,392.1716) and (359.3326,345.9277) .. (368.7203,285.2098) .. controls (376.6742,197.1784) and (311.7194,141.3342) .. (205.4287,142.1456) .. controls (139.9485,141.4804) and (88.7155,166.1957) .. (73.5775,228.0086) .. controls (52.0297,320.3408) and (123.4078,391.0103) .. (225.7398,392.6996) -- cycle(216.0739,176.4733) .. controls (268.9183,179.2424) and (315.8292,206.5488) .. (312.7454,265.1139) .. controls (313.2769,315.6384) and (286.5993,353.4946) .. (216.6040,355.7934) .. controls (162.4657,355.7934) and (126.0914,317.5023) .. (126.0914,260.5103) .. controls (126.1733,214.2900) and (163.3363,176.2849) .. (216.0739,176.4733) -- cycle(76.4897,189.1754) .. controls (13.1586,147.5631) and (0.0000,119.4207) .. (0.0000,119.4207) -- (90.6499,170.1632) .. controls (85.3004,175.8497) and (80.5994,182.1633) .. (76.4897,189.1754) -- cycle(353.9486,119.3004) -- (402.9482,119.3004) .. controls (427.0025,137.0797) and (450.9893,162.7034) .. (474.9529,191.0213) .. controls (509.3540,228.5339) and (531.3391,294.2091) .. (487.8149,312.1206) .. controls (462.8165,324.7652) and (394.3874,316.8943) .. (373.8912,313.6651) .. controls (379.9291,297.7449) and (383.2899,278.4204) .. (381.4989,257.7214) .. controls (420.3069,248.0321) and (421.9610,218.3461) .. (407.7867,192.6417) .. controls (391.1113,162.4018) and (370.1114,132.9097) .. (353.9486,119.3004) -- cycle;
\end{tikzpicture}
}

\providecommand{\markuptable}{
\begin{table}
\sffamily\centering
\begin{tabular}{@{}lcl@{}}
\toprule
\texttt{//italics//} & $\rightarrow$ & \textit{italics} \\
\midrule
\texttt{**bold**} & $\rightarrow$ & \textbf{bold} \\
\midrule
\texttt{\# ordered list} & & 1 ordered list \\
\texttt{\# second item} & $\rightarrow$ & 2 second item \\
\texttt{\#\# sub item} & & \hspace{1em} 1 sub item \\
\midrule
\texttt{* unordered list} & & $\bullet$ unordered list \\
\texttt{* second item} & $\rightarrow$ & $\bullet$ second item \\
\texttt{** sub item} & & \hspace{1em} $\bullet$ sub item \\
\midrule
\texttt{link to [[label]]} & $\rightarrow$ & link to \underline{label} \\
\midrule
\texttt{<{}<label>{}> definition } & $\rightarrow$ & definition \\
\midrule
\texttt{[[url|link name]]} & $\rightarrow$ & \underline{link name} \\
\midrule\addlinespace
\texttt{= large heading} & & {\Large large heading} \smallskip \\
\texttt{== medium heading} & $\rightarrow$ & {\large medium heading} \\
\texttt{=== small heading} & & small heading \\
\midrule
\texttt{no line break} & & no line break for paragraphs \\
\texttt{for paragraphs} & $\rightarrow$ \\
& & use empty line \\
\texttt{use empty line} \\
\midrule
\texttt{force\textbackslash\textbackslash line break} & $\rightarrow$ & force \\
& & line break \\
\midrule
\texttt{horizontal line} & $\rightarrow$ & horizontal line \\
\texttt{----} & & \hrulefill \\
\midrule
\texttt{|=a|=table|=header} & & \underline{a \enspace table \enspace header} \\
\texttt{|a|table|row} & $\rightarrow$ & a \enspace table \enspace row \\
\texttt{|b|table|row} & & b \enspace table \enspace row \\
\midrule
\texttt{\{\{\{} \\
\texttt{unformatted} & $\rightarrow$ & \texttt{unformatted} \\
\texttt{code} & & \texttt{code} \\
\texttt{\}\}\}} \\
\midrule\addlinespace
\texttt{@ new article} & & {\Large 1.\ new article} \smallskip \\
\texttt{@ second article} & $\rightarrow$ & {\Large 2.\ second article} \smallskip \\
\texttt{@@ sub article} & & {\large 2.1.\ sub article} \\
\bottomrule
\end{tabular}
\normalfont\caption{Elements of the generic documentation markup language}
\label{tab:docmarkup}
\end{table}
}

\providecommand{\startchapter}[4]{
\documentclass[11pt,a4paper]{article}
\usepackage{booktabs}
\usepackage[format=hang,labelfont=bf]{caption}
\usepackage{changepage}
\usepackage[T1]{fontenc}
\usepackage[margin=2cm]{geometry}
\usepackage{hyperref}
\usepackage[american]{isodate}
\usepackage{lmodern}
\usepackage{longtable}
\usepackage{mathptmx}
\usepackage{microtype}
\usepackage[toc]{multitoc}
\usepackage{multirow}
\usepackage[all]{nowidow}
\usepackage{pdfcomment}
\usepackage{syntax}
\usepackage{tikz}
\usepackage[all]{xy}
\hypersetup{pdfborder={0 0 0},bookmarksnumbered=true,pdftitle={\ecs{}: #2},pdfauthor={Florian Negele},pdfsubject={\ecs{}},pdfkeywords={#1}}
\setlength{\grammarindent}{8em}\setlength{\grammarparsep}{0.2ex}
\setlength{\columnsep}{2em}
\newcommand{\prefix}{}
\newcounter{instruction}
\bibliographystyle{unsrt}
\renewcommand{\index}[2][]{}
\renewcommand{\arraystretch}{1.05}
\renewcommand{\floatpagefraction}{0.7}
\renewcommand{\syntleft}{\itshape}\renewcommand{\syntright}{}
\title{\vspace{-5ex}\Huge{\ecs{}}\medskip\hrule}
\author{\huge{#2}}
\date{\medskip\version}
\newif\ifbook\bookfalse
\pagestyle{headings}
\frenchspacing
\begin{document}
\maketitle\thispagestyle{empty}\noindent#4\setlength{\columnseprule}{0.4pt}\tableofcontents\setlength{\columnseprule}{0pt}\vfill\pagebreak[3]\null\vfill\bigskip\noindent
\parbox{\textwidth-4em}{\license The contents of this \documentation{} are part of the \href{manual}{\ecs{} User Manual}~\cite{manual} and correspond to Chapter ``\href{manual\##3}{#1}''.\alignright\mbox{\today}}
\parbox{4em}{\flushright\ecslogo{3em}}
\clearpage
}

\providecommand{\concludechapter}{
\vfill\pagebreak[3]\null\vfill
\thispagestyle{myheadings}\markright{REFERENCES}
\noindent\begin{minipage}{\textwidth}\begin{multicols}{2}[\section*{References}]
\renewcommand{\section}[2]{}\small\bibliography{references}
\end{multicols}\end{minipage}\end{document}
}

\providecommand{\startpresentation}[2]{
\documentclass[14pt,aspectratio=43,usepdftitle=false]{beamer}
\usepackage{booktabs}
\usepackage{etex}
\usepackage{multicol}
\usepackage{tikz}
\usepackage[all]{xy}
\bibliographystyle{unsrt}
\setlength{\columnsep}{1em}
\setlength{\leftmargini}{1em}
\setbeamercolor{title}{fg=black}
\setbeamercolor{structure}{fg=darkgray}
\setbeamercolor{bibliography item}{fg=darkgray}
\setbeamerfont{title}{series=\bfseries}
\setbeamerfont{subtitle}{series=\normalfont}
\setbeamerfont*{frametitle}{parent=title}
\setbeamerfont{block title}{series=\bfseries}
\setbeamerfont*{framesubtitle}{parent=subtitle}
\setbeamersize{text margin left=1em,text margin right=1em}
\setbeamertemplate{navigation symbols}{}
\setbeamertemplate{itemize item}[circle]{}
\setbeamertemplate{bibliography item}[triangle]{}
\setbeamertemplate{bibliography entry author}{\usebeamercolor[fg]{bibliography item}}
\setbeamertemplate{frametitle}{\medskip\usebeamerfont{frametitle}\color{gray}\raisebox{-2.5ex}[0ex][0ex]{\rule{0.1em}{4.5ex}}}
\addtobeamertemplate{frametitle}{}{\hspace{0.4em}\usebeamercolor[fg]{title}\insertframetitle\par\vspace{0.2ex}\hspace{0.5em}\usebeamerfont{framesubtitle}\insertframesubtitle}
\hypersetup{pdfborder={0 0 0},bookmarksnumbered=true,bookmarksopen=true,bookmarksopenlevel=0,pdftitle={\ecs{}: #1},pdfauthor={Florian Negele},pdfsubject={\ecs{}},pdfkeywords={#1}}
\renewcommand{\flowgraph}[1]{\resizebox{\textwidth}{!}{$$\xymatrix{##1}$$}}
\title{\ecs{}\medskip\hrule\medskip}
\institute{\shadowedecslogo{5em}{30}{15}}
\date{\version}
\subtitle{#1}
\begin{document}
\begin{frame}[plain]\titlepage\nocite{manual}\end{frame}
\begin{frame}{Contents}{#1}\begin{center}\tableofcontents\end{center}\end{frame}
}

\providecommand{\concludepresentation}{
\begin{frame}{References}\begin{footnotesize}\setlength{\columnseprule}{0.4pt}\begin{multicols}{2}\bibliography{references}\end{multicols}\end{footnotesize}\end{frame}
\end{document}
}

\providecommand{\startbook}[1]{
\documentclass[10pt,paper=17cm:24cm,DIV=13,twoside=semi,headings=normal,numbers=noendperiod,cleardoublepage=plain]{scrbook}
\usepackage{atveryend}
\usepackage{booktabs}
\usepackage{caption}
\usepackage{changepage}
\usepackage[T1]{fontenc}
\usepackage{imakeidx}
\usepackage{hyperref}
\usepackage[american]{isodate}
\usepackage{lmodern}
\usepackage{longtable}
\usepackage{mathptmx}
\usepackage[final]{microtype}
\usepackage{multicol}
\usepackage{multirow}
\usepackage[all]{nowidow}
\usepackage{pdfcomment}
\usepackage{scrlayer-scrpage}
\usepackage{setspace}
\usepackage{syntax}
\usepackage[eventxtindent=4pt,oddtxtexdent=4pt]{thumbs}
\usepackage{tikz}
\usepackage[all]{xy}
\hyphenation{Micro-Blaze Open-Cores Open-RISC Power-PC}
\hypersetup{pdfborder={0 0 0},bookmarksnumbered=true,bookmarksopen=true,bookmarksopenlevel=0,pdftitle={\ecs{}: #1},pdfauthor={Florian Negele},pdfsubject={\ecs{}},pdfkeywords={#1}}
\setlength{\grammarindent}{8em}\setlength{\grammarparsep}{0.7ex}
\setkomafont{captionlabel}{\usekomafont{descriptionlabel}}
\renewcommand{\arraystretch}{1.05}\setstretch{1.1}
\renewcommand{\chapterformat}{\thechapter\autodot\enskip\raisebox{-1ex}[0ex][0ex]{\color{gray}\rule{0.1em}{3.5ex}}\enskip}
\renewcommand{\startchapter}[4]{\hypertarget{##3}{\chapter{##1}}\label{##3}##4\addthumb{##1}{\LARGE\sffamily\bfseries\thechapter}{white}{gray}\renewcommand{\prefix}{##3}}
\renewcommand{\concludechapter}{\clearpage{\stopthumb\cleardoublepage}}
\renewcommand{\syntleft}{\itshape}\renewcommand{\syntright}{}
\renewcommand{\floatpagefraction}{0.7}
\renewcommand{\partheademptypage}{}
\DeclareMicrotypeAlias{lmss}{cmr}
\newcommand{\prefix}{}
\newcounter{instruction}
\bibliographystyle{unsrt}
\newif\ifbook\booktrue
\makeindex[intoc,title=Index]
\makeindex[intoc,name=tools,title=Index of Tools,columns=3]
\makeindex[intoc,name=library,title=Index of Library Names]
\makeindex[intoc,name=runtime,title=Index of Runtime Support]
\makeindex[intoc,name=environment,title=Index of Target Environments]
\indexsetup{toclevel=chapter,headers={\indexname}{\indexname}}
\frenchspacing
\begin{document}
\pagenumbering{alph}
\begin{titlepage}\centering
\huge\sffamily\null\vfill\textbf{\ecs{}}\bigskip\hrule\bigskip#1
\normalsize\normalfont\vfill\vfill\shadowedecslogo{10em}{30}{15}
\large\vfill\vfill\version
\end{titlepage}
\null\vfill
\thispagestyle{empty}
\noindent\today\par\medskip
\license A copy of this license is included in Appendix~\ref{fdl} on page~\pageref{fdl}.
All product names used herein are for identification purposes only and may be trademarks of their respective companies.
\concludechapter
\frontmatter
\setcounter{tocdepth}{1}
\tableofcontents
\setcounter{tocdepth}{2}
\concludechapter
\listoffigures
\concludechapter
\listoftables
\concludechapter
}

\providecommand{\concludebook}{
\backmatter
\addtocontents{toc}{\protect\setcounter{tocdepth}{-1}}
\phantomsection\addcontentsline{toc}{part}{Bibliography}
\bibliography{references}
\concludechapter
\phantomsection\addcontentsline{toc}{part}{Indexes}
\printindex
\concludechapter
\indexprologue{\label{idx:tools}}
\printindex[tools]
\concludechapter
\printindex[library]
\concludechapter
\indexprologue{\label{idx:runtime}}
\printindex[runtime]
\concludechapter
\indexprologue{\label{idx:environment}}
\printindex[environment]
\concludechapter
\pagestyle{empty}\pagenumbering{Alph}\null\clearpage
\null\vfill\centering\ecslogo{4em}\par\medskip\license
\end{document}
}

% chapter references

\providecommand{\seedocumentationref}{}\renewcommand{\seedocumentationref}[3]{#1, see \Documentation{}~\documentationref{#2}{#3}. }
\providecommand{\seeinterface}{}\renewcommand{\seeinterface}{\ifbook See \Documentation{}~\documentationref{interface}{User Interface} for more information about the common user interface of all of these tools. \fi}
\providecommand{\seeguide}{}\renewcommand{\seeguide}{\seedocumentationref{For basic examples of using some of these tools in practice}{guide}{User Guide}}
\providecommand{\seecpp}{}\renewcommand{\seecpp}{\seedocumentationref{For more information about the \cpp{} programming language and its implementation by the \ecs{}}{cpp}{User Manual for \cpp{}}}
\providecommand{\seefalse}{}\renewcommand{\seefalse}{\seedocumentationref{For more information about the FALSE programming language and its implementation by the \ecs{}}{false}{User Manual for FALSE}}
\providecommand{\seeoberon}{}\renewcommand{\seeoberon}{\seedocumentationref{For more information about the Oberon programming language and its implementation by the \ecs{}}{oberon}{User Manual for Oberon}}
\providecommand{\seeassembly}{}\renewcommand{\seeassembly}{\seedocumentationref{For more information about the generic assembly language and how to use it}{assembly}{Generic Assembly Language Specification}}
\providecommand{\seeamd}{}\renewcommand{\seeamd}{\seedocumentationref{For more information about how the \ecs{} supports the AMD64 hardware architecture}{amd64}{AMD64 Hardware Architecture Support}}
\providecommand{\seearm}{}\renewcommand{\seearm}{\seedocumentationref{For more information about how the \ecs{} supports the ARM hardware architecture}{arm}{ARM Hardware Architecture Support}}
\providecommand{\seeavr}{}\renewcommand{\seeavr}{\seedocumentationref{For more information about how the \ecs{} supports the AVR hardware architecture}{avr}{AVR Hardware Architecture Support}}
\providecommand{\seeavrtt}{}\renewcommand{\seeavrtt}{\seedocumentationref{For more information about how the \ecs{} supports the AVR32 hardware architecture}{avr32}{AVR32 Hardware Architecture Support}}
\providecommand{\seemabk}{}\renewcommand{\seemabk}{\seedocumentationref{For more information about how the \ecs{} supports the M68000 hardware architecture}{m68k}{M68000 Hardware Architecture Support}}
\providecommand{\seemibl}{}\renewcommand{\seemibl}{\seedocumentationref{For more information about how the \ecs{} supports the MicroBlaze hardware architecture}{mibl}{MicroBlaze Hardware Architecture Support}}
\providecommand{\seemips}{}\renewcommand{\seemips}{\seedocumentationref{For more information about how the \ecs{} supports the MIPS32 and MIPS64 hardware architectures}{mips}{MIPS Hardware Architecture Support}}
\providecommand{\seemmix}{}\renewcommand{\seemmix}{\seedocumentationref{For more information about how the \ecs{} supports the MMIX hardware architecture}{mmix}{MMIX Hardware Architecture Support}}
\providecommand{\seeorok}{}\renewcommand{\seeorok}{\seedocumentationref{For more information about how the \ecs{} supports the OpenRISC 1000 hardware architecture}{or1k}{OpenRISC 1000 Hardware Architecture Support}}
\providecommand{\seeppc}{}\renewcommand{\seeppc}{\seedocumentationref{For more information about how the \ecs{} supports the PowerPC hardware architecture}{ppc}{PowerPC Hardware Architecture Support}}
\providecommand{\seerisc}{}\renewcommand{\seerisc}{\seedocumentationref{For more information about how the \ecs{} supports the RISC hardware architecture}{risc}{RISC Hardware Architecture Support}}
\providecommand{\seewasm}{}\renewcommand{\seewasm}{\seedocumentationref{For more information about how the \ecs{} supports the WebAssembly architecture}{wasm}{WebAssembly Architecture Support}}
\providecommand{\seedocumentation}{}\renewcommand{\seedocumentation}{\seedocumentationref{For more information about generic documentations and their generation by the \ecs{}}{documentation}{Generic Documentation Generation}}
\providecommand{\seedebugging}{}\renewcommand{\seedebugging}{\seedocumentationref{For more information about debugging information and its representation}{debugging}{Debugging Information Representation}}
\providecommand{\seecode}{}\renewcommand{\seecode}{\seedocumentationref{For more information about intermediate code and its purpose}{code}{Intermediate Code Representation}}
\providecommand{\seeobject}{}\renewcommand{\seeobject}{\seedocumentationref{For more information about object files and their purpose}{object}{Object File Representation}}

% generic documentation tools

\providecommand{\docprint}{
\toolsection{docprint} is a pretty printer for generic documentations.
It reformats generic documentations and writes it to the standard output stream.
\debuggingtool
\flowgraph{\resource{generic\\documentation} \ar[r] & \toolbox{docprint} \ar[r] & \resource{generic\\documentation}}
\seedocumentation
}

\providecommand{\doccheck}{
\toolsection{doccheck} is a syntactic and semantic checker for generic documentations.
It just performs syntactic and semantic checks on generic documentations and writes its diagnostic messages to the standard error stream.
\debuggingtool
\flowgraph{\resource{generic\\documentation} \ar[r] & \toolbox{doccheck} \ar[r] & \resource{diagnostic\\messages}}
\seedocumentation
}

\providecommand{\dochtml}{
\toolsection{dochtml} is an HTML documentation generator for generic documentations.
It processes several generic documentations and assembles all information therein into an HTML document.
\debuggingtool
\flowgraph{\resource{generic\\documentation} \ar[r] & \toolbox{dochtml} \ar[r] & \resource{HTML\\document}}
\seedocumentation
}

\providecommand{\doclatex}{
\toolsection{doclatex} is a Latex documentation generator for generic documentations.
It processes several generic documentations and assembles all information therein into a Latex document.
\debuggingtool
\flowgraph{\resource{generic\\documentation} \ar[r] & \toolbox{doclatex} \ar[r] & \resource{Latex\\document}}
\seedocumentation
}

% intermediate code tools

\providecommand{\cdcheck}{
\toolsection{cdcheck} is a syntactic and semantic checker for intermediate code.
It just performs syntactic and semantic checks on programs written in intermediate code and writes its diagnostic messages to the standard error stream.
\debuggingtool
\flowgraph{\resource{intermediate\\code} \ar[r] & \toolbox{cdcheck} \ar[r] & \resource{diagnostic\\messages}}
\seeassembly\seecode
}

\providecommand{\cdopt}{
\toolsection{cdopt} is an optimizer for intermediate code.
It performs various optimizations on programs written in intermediate code and writes the result to the standard output stream.
\debuggingtool
\flowgraph{\resource{intermediate\\code} \ar[r] & \toolbox{cdopt} \ar[r] & \resource{optimized\\code}}
\seeassembly\seecode
}

\providecommand{\cdrun}{
\toolsection{cdrun} is an interpreter for intermediate code.
It processes and executes programs written in intermediate code.
The following code sections are predefined and have the usual semantics:
\texttt{abort}, \texttt{\_Exit}, \texttt{fflush}, \texttt{floor}, \texttt{fputc}, \texttt{free}, \texttt{getchar}, \texttt{malloc}, and \texttt{putchar}.
Diagnostic messages about invalid operations include the name of the executed code section and the index of the erroneous instruction.
\debuggingtool
\flowgraph{\resource{intermediate\\code} \ar[r] & \toolbox{cdrun} \ar@/u/[r] & \resource{input/\\output} \ar@/d/[l]}
\seeassembly\seecode
}

\providecommand{\cdamda}{
\toolsection{cdamd16} is a compiler for intermediate code targeting the AMD64 hardware architecture.
It generates machine code for AMD64 processors from programs written in intermediate code and stores it in corresponding object files.
The compiler generates machine code for the 16-bit operating mode defined by the AMD64 architecture.
It also creates a debugging information file as well as an assembly file containing a listing of the generated machine code.
\debuggingtool
\flowgraph{\resource{intermediate\\code} \ar[r] & \toolbox{cdamd16} \ar[r] \ar[d] \ar[rd] & \resource{object file} \\ & \resource{assembly\\listing} & \resource{debugging\\information}}
\seeassembly\seeamd\seeobject\seecode\seedebugging
}

\providecommand{\cdamdb}{
\toolsection{cdamd32} is a compiler for intermediate code targeting the AMD64 hardware architecture.
It generates machine code for AMD64 processors from programs written in intermediate code and stores it in corresponding object files.
The compiler generates machine code for the 32-bit operating mode defined by the AMD64 architecture.
It also creates a debugging information file as well as an assembly file containing a listing of the generated machine code.
\debuggingtool
\flowgraph{\resource{intermediate\\code} \ar[r] & \toolbox{cdamd32} \ar[r] \ar[d] \ar[rd] & \resource{object file} \\ & \resource{assembly\\listing} & \resource{debugging\\information}}
\seeassembly\seeamd\seeobject\seecode\seedebugging
}

\providecommand{\cdamdc}{
\toolsection{cdamd64} is a compiler for intermediate code targeting the AMD64 hardware architecture.
It generates machine code for AMD64 processors from programs written in intermediate code and stores it in corresponding object files.
The compiler generates machine code for the 64-bit operating mode defined by the AMD64 architecture.
It also creates a debugging information file as well as an assembly file containing a listing of the generated machine code.
\debuggingtool
\flowgraph{\resource{intermediate\\code} \ar[r] & \toolbox{cdamd64} \ar[r] \ar[d] \ar[rd] & \resource{object file} \\ & \resource{assembly\\listing} & \resource{debugging\\information}}
\seeassembly\seeamd\seeobject\seecode\seedebugging
}

\providecommand{\cdarma}{
\toolsection{cdarma32} is a compiler for intermediate code targeting the ARM hardware architecture.
It generates machine code for ARM processors executing A32 instructions from programs written in intermediate code and stores it in corresponding object files.
It also creates a debugging information file as well as an assembly file containing a listing of the generated machine code.
\debuggingtool
\flowgraph{\resource{intermediate\\code} \ar[r] & \toolbox{cdarma32} \ar[r] \ar[d] \ar[rd] & \resource{object file} \\ & \resource{assembly\\listing} & \resource{debugging\\information}}
\seeassembly\seearm\seeobject\seecode\seedebugging
}

\providecommand{\cdarmb}{
\toolsection{cdarma64} is a compiler for intermediate code targeting the ARM hardware architecture.
It generates machine code for ARM processors executing A64 instructions from programs written in intermediate code and stores it in corresponding object files.
It also creates a debugging information file as well as an assembly file containing a listing of the generated machine code.
\debuggingtool
\flowgraph{\resource{intermediate\\code} \ar[r] & \toolbox{cdarma64} \ar[r] \ar[d] \ar[rd] & \resource{object file} \\ & \resource{assembly\\listing} & \resource{debugging\\information}}
\seeassembly\seearm\seeobject\seecode\seedebugging
}

\providecommand{\cdarmc}{
\toolsection{cdarmt32} is a compiler for intermediate code targeting the ARM hardware architecture.
It generates machine code for ARM processors without floating-point extension executing T32 instructions from programs written in intermediate code and stores it in corresponding object files.
It also creates a debugging information file as well as an assembly file containing a listing of the generated machine code.
\debuggingtool
\flowgraph{\resource{intermediate\\code} \ar[r] & \toolbox{cdarmt32} \ar[r] \ar[d] \ar[rd] & \resource{object file} \\ & \resource{assembly\\listing} & \resource{debugging\\information}}
\seeassembly\seearm\seeobject\seecode\seedebugging
}

\providecommand{\cdarmcfpe}{
\toolsection{cdarmt32fpe} is a compiler for intermediate code targeting the ARM hardware architecture.
It generates machine code for ARM processors with floating-point extension executing T32 instructions from programs written in intermediate code and stores it in corresponding object files.
It also creates a debugging information file as well as an assembly file containing a listing of the generated machine code.
\debuggingtool
\flowgraph{\resource{intermediate\\code} \ar[r] & \toolbox{cdarmt32fpe} \ar[r] \ar[d] \ar[rd] & \resource{object file} \\ & \resource{assembly\\listing} & \resource{debugging\\information}}
\seeassembly\seearm\seeobject\seecode\seedebugging
}

\providecommand{\cdavr}{
\toolsection{cdavr} is a compiler for intermediate code targeting the AVR hardware architecture.
It generates machine code for AVR processors from programs written in intermediate code and stores it in corresponding object files.
It also creates a debugging information file as well as an assembly file containing a listing of the generated machine code.
\debuggingtool
\flowgraph{\resource{intermediate\\code} \ar[r] & \toolbox{cdavr} \ar[r] \ar[d] \ar[rd] & \resource{object file} \\ & \resource{assembly\\listing} & \resource{debugging\\information}}
\seeassembly\seeavr\seeobject\seecode\seedebugging
}

\providecommand{\cdavrtt}{
\toolsection{cdavr32} is a compiler for intermediate code targeting the AVR32 hardware architecture.
It generates machine code for AVR32 processors from programs written in intermediate code and stores it in corresponding object files.
It also creates a debugging information file as well as an assembly file containing a listing of the generated machine code.
\debuggingtool
\flowgraph{\resource{intermediate\\code} \ar[r] & \toolbox{cdavr32} \ar[r] \ar[d] \ar[rd] & \resource{object file} \\ & \resource{assembly\\listing} & \resource{debugging\\information}}
\seeassembly\seeavrtt\seeobject\seecode\seedebugging
}

\providecommand{\cdmabk}{
\toolsection{cdm68k} is a compiler for intermediate code targeting the M68000 hardware architecture.
It generates machine code for M68000 processors from programs written in intermediate code and stores it in corresponding object files.
It also creates a debugging information file as well as an assembly file containing a listing of the generated machine code.
\debuggingtool
\flowgraph{\resource{intermediate\\code} \ar[r] & \toolbox{cdm68k} \ar[r] \ar[d] \ar[rd] & \resource{object file} \\ & \resource{assembly\\listing} & \resource{debugging\\information}}
\seeassembly\seemabk\seeobject\seecode\seedebugging
}

\providecommand{\cdmibl}{
\toolsection{cdmibl} is a compiler for intermediate code targeting the MicroBlaze hardware architecture.
It generates machine code for MicroBlaze processors from programs written in intermediate code and stores it in corresponding object files.
It also creates a debugging information file as well as an assembly file containing a listing of the generated machine code.
\debuggingtool
\flowgraph{\resource{intermediate\\code} \ar[r] & \toolbox{cdmibl} \ar[r] \ar[d] \ar[rd] & \resource{object file} \\ & \resource{assembly\\listing} & \resource{debugging\\information}}
\seeassembly\seemibl\seeobject\seecode\seedebugging
}

\providecommand{\cdmipsa}{
\toolsection{cdmips32} is a compiler for intermediate code targeting the MIPS32 hardware architecture.
It generates machine code for MIPS32 processors from programs written in intermediate code and stores it in corresponding object files.
It also creates a debugging information file as well as an assembly file containing a listing of the generated machine code.
\debuggingtool
\flowgraph{\resource{intermediate\\code} \ar[r] & \toolbox{cdmips32} \ar[r] \ar[d] \ar[rd] & \resource{object file} \\ & \resource{assembly\\listing} & \resource{debugging\\information}}
\seeassembly\seemips\seeobject\seecode\seedebugging
}

\providecommand{\cdmipsb}{
\toolsection{cdmips64} is a compiler for intermediate code targeting the MIPS64 hardware architecture.
It generates machine code for MIPS64 processors from programs written in intermediate code and stores it in corresponding object files.
It also creates a debugging information file as well as an assembly file containing a listing of the generated machine code.
\debuggingtool
\flowgraph{\resource{intermediate\\code} \ar[r] & \toolbox{cdmips64} \ar[r] \ar[d] \ar[rd] & \resource{object file} \\ & \resource{assembly\\listing} & \resource{debugging\\information}}
\seeassembly\seemips\seeobject\seecode\seedebugging
}

\providecommand{\cdmmix}{
\toolsection{cdmmix} is a compiler for intermediate code targeting the MMIX hardware architecture.
It generates machine code for MMIX processors from programs written in intermediate code and stores it in corresponding object files.
It also creates a debugging information file as well as an assembly file containing a listing of the generated machine code.
\debuggingtool
\flowgraph{\resource{intermediate\\code} \ar[r] & \toolbox{cdmmix} \ar[r] \ar[d] \ar[rd] & \resource{object file} \\ & \resource{assembly\\listing} & \resource{debugging\\information}}
\seeassembly\seemmix\seeobject\seecode\seedebugging
}

\providecommand{\cdorok}{
\toolsection{cdor1k} is a compiler for intermediate code targeting the OpenRISC 1000 hardware architecture.
It generates machine code for OpenRISC 1000 processors from programs written in intermediate code and stores it in corresponding object files.
It also creates a debugging information file as well as an assembly file containing a listing of the generated machine code.
\debuggingtool
\flowgraph{\resource{intermediate\\code} \ar[r] & \toolbox{cdor1k} \ar[r] \ar[d] \ar[rd] & \resource{object file} \\ & \resource{assembly\\listing} & \resource{debugging\\information}}
\seeassembly\seeorok\seeobject\seecode\seedebugging
}

\providecommand{\cdppca}{
\toolsection{cdppc32} is a compiler for intermediate code targeting the PowerPC hardware architecture.
It generates machine code for PowerPC processors from programs written in intermediate code and stores it in corresponding object files.
The compiler generates machine code for the 32-bit operating mode defined by the PowerPC architecture.
It also creates a debugging information file as well as an assembly file containing a listing of the generated machine code.
\debuggingtool
\flowgraph{\resource{intermediate\\code} \ar[r] & \toolbox{cdppc32} \ar[r] \ar[d] \ar[rd] & \resource{object file} \\ & \resource{assembly\\listing} & \resource{debugging\\information}}
\seeassembly\seeppc\seeobject\seecode\seedebugging
}

\providecommand{\cdppcb}{
\toolsection{cdppc64} is a compiler for intermediate code targeting the PowerPC hardware architecture.
It generates machine code for PowerPC processors from programs written in intermediate code and stores it in corresponding object files.
The compiler generates machine code for the 64-bit operating mode defined by the PowerPC architecture.
It also creates a debugging information file as well as an assembly file containing a listing of the generated machine code.
\debuggingtool
\flowgraph{\resource{intermediate\\code} \ar[r] & \toolbox{cdppc64} \ar[r] \ar[d] \ar[rd] & \resource{object file} \\ & \resource{assembly\\listing} & \resource{debugging\\information}}
\seeassembly\seeppc\seeobject\seecode\seedebugging
}

\providecommand{\cdrisc}{
\toolsection{cdrisc} is a compiler for intermediate code targeting the RISC hardware architecture.
It generates machine code for RISC processors from programs written in intermediate code and stores it in corresponding object files.
It also creates a debugging information file as well as an assembly file containing a listing of the generated machine code.
\debuggingtool
\flowgraph{\resource{intermediate\\code} \ar[r] & \toolbox{cdrisc} \ar[r] \ar[d] \ar[rd] & \resource{object file} \\ & \resource{assembly\\listing} & \resource{debugging\\information}}
\seeassembly\seerisc\seeobject\seecode\seedebugging
}

\providecommand{\cdwasm}{
\toolsection{cdwasm} is a compiler for intermediate code targeting the WebAssembly architecture.
It generates machine code for WebAssembly targets from programs written in intermediate code and stores it in corresponding object files.
It also creates a debugging information file as well as an assembly file containing a listing of the generated machine code.
\debuggingtool
\flowgraph{\resource{intermediate\\code} \ar[r] & \toolbox{cdwasm} \ar[r] \ar[d] \ar[rd] & \resource{object file} \\ & \resource{assembly\\listing} & \resource{debugging\\information}}
\seeassembly\seewasm\seeobject\seecode\seedebugging
}

% C++ tools

\providecommand{\cppprep}{
\toolsection{cppprep} is a preprocessor for the \cpp{} programming language.
It preprocesses source code according to the rules of \cpp{} and writes it to the standard output stream.
Only the macro names \texttt{\_\_DATE\_\_}, \texttt{\_\_FILE\_\_}, \texttt{\_\_LINE\_\_}, and \texttt{\_\_TIME\_\_} are predefined.
\flowgraph{\resource{\cpp{} or other\\source code} \ar[r] & \toolbox{cppprep} \ar[r] & \resource{preprocessed\\source code} \\ & \variable{ECSINCLUDE} \ar[u]}
\seecpp
}

\providecommand{\cppprint}{
\toolsection{cppprint} is a pretty printer for the \cpp{} programming language.
It reformats the source code of \cpp{} programs and writes it to the standard output stream.
\flowgraph{\resource{\cpp{}\\source code} \ar[r] & \toolbox{cppprint} \ar[r] & \resource{reformatted\\source code} \\ & \variable{ECSINCLUDE} \ar[u]}
\seecpp
}

\providecommand{\cppcheck}{
\toolsection{cppcheck} is a syntactic and semantic checker for the \cpp{} programming language.
It just performs syntactic and semantic checks on \cpp{} programs and writes its diagnostic messages to the standard error stream.
\flowgraph{\resource{\cpp{}\\source code} \ar[r] & \toolbox{cppcheck} \ar[r] & \resource{diagnostic\\messages} \\ & \variable{ECSINCLUDE} \ar[u]}
\seecpp
}

\providecommand{\cppdump}{
\toolsection{cppdump} is a serializer for the \cpp{} programming language.
It dumps the complete internal representation of programs written in \cpp{} into an XML document.
\debuggingtool
\flowgraph{\resource{\cpp{}\\source code} \ar[r] & \toolbox{cppdump} \ar[r] & \resource{internal\\representation} \\ & \variable{ECSINCLUDE} \ar[u]}
\seecpp
}

\providecommand{\cpprun}{
\toolsection{cpprun} is an interpreter for the \cpp{} programming language.
It processes and executes programs written in \cpp{}.
The macro \texttt{\_\_run\_\_} is predefined in order to enable programmers to identify this tool while interpreting.
\flowgraph{\resource{\cpp{}\\source code} \ar[r] & \toolbox{cpprun} \ar@/u/[r] & \resource{input/\\output} \ar@/d/[l] \\ & \variable{ECSINCLUDE} \ar[u]}
\seecpp
}

\providecommand{\cppdoc}{
\toolsection{cppdoc} is a generic documentation generator for the \cpp{} programming language.
It processes several \cpp{} source files and assembles all information therein into a generic documentation.
\debuggingtool
\flowgraph{\resource{\cpp{}\\source code} \ar[r] & \toolbox{cppdoc} \ar[r] & \resource{generic\\documentation} \\ & \variable{ECSINCLUDE} \ar[u]}
\seecpp\seedocumentation
}

\providecommand{\cpphtml}{
\toolsection{cpphtml} is an HTML documentation generator for the \cpp{} programming language.
It processes several \cpp{} source files and assembles all information therein into an HTML document.
\flowgraph{\resource{\cpp{}\\source code} \ar[r] & \toolbox{cpphtml} \ar[r] & \resource{HTML\\document} \\ & \variable{ECSINCLUDE} \ar[u]}
\seecpp\seedocumentation
}

\providecommand{\cpplatex}{
\toolsection{cpplatex} is a Latex documentation generator for the \cpp{} programming language.
It processes several \cpp{} source files and assembles all information therein into a Latex document.
\flowgraph{\resource{\cpp{}\\source code} \ar[r] & \toolbox{cpplatex} \ar[r] & \resource{Latex\\document} \\ & \variable{ECSINCLUDE} \ar[u]}
\seecpp\seedocumentation
}

\providecommand{\cppcode}{
\toolsection{cppcode} is an intermediate code generator for the \cpp{} programming language.
It generates intermediate code from programs written in \cpp{} and stores it in corresponding assembly files.
The macro \texttt{\_\_code\_\_} is predefined in order to enable programmers to identify this tool while generating intermediate code.
Programs generated with this tool require additional runtime support that is stored in the \file{cpp\-code\-run} library file.
\debuggingtool
\flowgraph{\resource{\cpp{}\\source code} \ar[r] & \toolbox{cppcode} \ar[r] & \resource{intermediate\\code} \\ & \variable{ECSINCLUDE} \ar[u]}
\seecpp\seeassembly\seecode
}

\providecommand{\cppamda}{
\toolsection{cppamd16} is a compiler for the \cpp{} programming language targeting the AMD64 hardware architecture.
It generates machine code for AMD64 processors from programs written in \cpp{} and stores it in corresponding object files.
The compiler generates machine code for the 16-bit operating mode defined by the AMD64 architecture.
For debugging purposes, it also creates a debugging information file as well as an assembly file containing a listing of the generated machine code.
The macro \texttt{\_\_amd16\_\_} is predefined in order to enable programmers to identify this tool and its target architecture while compiling.
Programs generated with this compiler require additional runtime support that is stored in the \file{cpp\-amd16\-run} library file.
\flowgraph{\resource{\cpp{}\\source code} \ar[r] & \toolbox{cppamd16} \ar[r] \ar[d] \ar[rd] & \resource{object file} \\ \variable{ECSINCLUDE} \ar[ru] & \resource{debugging\\information} & \resource{assembly\\listing}}
\seecpp\seeassembly\seeamd\seeobject\seedebugging
}

\providecommand{\cppamdb}{
\toolsection{cppamd32} is a compiler for the \cpp{} programming language targeting the AMD64 hardware architecture.
It generates machine code for AMD64 processors from programs written in \cpp{} and stores it in corresponding object files.
The compiler generates machine code for the 32-bit operating mode defined by the AMD64 architecture.
For debugging purposes, it also creates a debugging information file as well as an assembly file containing a listing of the generated machine code.
The macro \texttt{\_\_amd32\_\_} is predefined in order to enable programmers to identify this tool and its target architecture while compiling.
Programs generated with this compiler require additional runtime support that is stored in the \file{cpp\-amd32\-run} library file.
\flowgraph{\resource{\cpp{}\\source code} \ar[r] & \toolbox{cppamd32} \ar[r] \ar[d] \ar[rd] & \resource{object file} \\ \variable{ECSINCLUDE} \ar[ru] & \resource{debugging\\information} & \resource{assembly\\listing}}
\seecpp\seeassembly\seeamd\seeobject\seedebugging
}

\providecommand{\cppamdc}{
\toolsection{cppamd64} is a compiler for the \cpp{} programming language targeting the AMD64 hardware architecture.
It generates machine code for AMD64 processors from programs written in \cpp{} and stores it in corresponding object files.
The compiler generates machine code for the 64-bit operating mode defined by the AMD64 architecture.
For debugging purposes, it also creates a debugging information file as well as an assembly file containing a listing of the generated machine code.
The macro \texttt{\_\_amd64\_\_} is predefined in order to enable programmers to identify this tool and its target architecture while compiling.
Programs generated with this compiler require additional runtime support that is stored in the \file{cpp\-amd64\-run} library file.
\flowgraph{\resource{\cpp{}\\source code} \ar[r] & \toolbox{cppamd64} \ar[r] \ar[d] \ar[rd] & \resource{object file} \\ \variable{ECSINCLUDE} \ar[ru] & \resource{debugging\\information} & \resource{assembly\\listing}}
\seecpp\seeassembly\seeamd\seeobject\seedebugging
}

\providecommand{\cpparma}{
\toolsection{cpparma32} is a compiler for the \cpp{} programming language targeting the ARM hardware architecture.
It generates machine code for ARM processors executing A32 instructions from programs written in \cpp{} and stores it in corresponding object files.
For debugging purposes, it also creates a debugging information file as well as an assembly file containing a listing of the generated machine code.
The macro \texttt{\_\_arma32\_\_} is predefined in order to enable programmers to identify this tool and its target architecture while compiling.
Programs generated with this compiler require additional runtime support that is stored in the \file{cpp\-arma32\-run} library file.
\flowgraph{\resource{\cpp{}\\source code} \ar[r] & \toolbox{cpparma32} \ar[r] \ar[d] \ar[rd] & \resource{object file} \\ \variable{ECSINCLUDE} \ar[ru] & \resource{debugging\\information} & \resource{assembly\\listing}}
\seecpp\seeassembly\seearm\seeobject\seedebugging
}

\providecommand{\cpparmb}{
\toolsection{cpparma64} is a compiler for the \cpp{} programming language targeting the ARM hardware architecture.
It generates machine code for ARM processors executing A64 instructions from programs written in \cpp{} and stores it in corresponding object files.
For debugging purposes, it also creates a debugging information file as well as an assembly file containing a listing of the generated machine code.
The macro \texttt{\_\_arma64\_\_} is predefined in order to enable programmers to identify this tool and its target architecture while compiling.
Programs generated with this compiler require additional runtime support that is stored in the \file{cpp\-arma64\-run} library file.
\flowgraph{\resource{\cpp{}\\source code} \ar[r] & \toolbox{cpparma64} \ar[r] \ar[d] \ar[rd] & \resource{object file} \\ \variable{ECSINCLUDE} \ar[ru] & \resource{debugging\\information} & \resource{assembly\\listing}}
\seecpp\seeassembly\seearm\seeobject\seedebugging
}

\providecommand{\cpparmc}{
\toolsection{cpparmt32} is a compiler for the \cpp{} programming language targeting the ARM hardware architecture.
It generates machine code for ARM processors without floating-point extension executing T32 instructions from programs written in \cpp{} and stores it in corresponding object files.
For debugging purposes, it also creates a debugging information file as well as an assembly file containing a listing of the generated machine code.
The macro \texttt{\_\_armt32\_\_} is predefined in order to enable programmers to identify this tool and its target architecture while compiling.
Programs generated with this compiler require additional runtime support that is stored in the \file{cpp\-armt32\-run} library file.
\flowgraph{\resource{\cpp{}\\source code} \ar[r] & \toolbox{cpparmt32} \ar[r] \ar[d] \ar[rd] & \resource{object file} \\ \variable{ECSINCLUDE} \ar[ru] & \resource{debugging\\information} & \resource{assembly\\listing}}
\seecpp\seeassembly\seearm\seeobject\seedebugging
}

\providecommand{\cpparmcfpe}{
\toolsection{cpparmt32fpe} is a compiler for the \cpp{} programming language targeting the ARM hardware architecture.
It generates machine code for ARM processors with floating-point extension executing T32 instructions from programs written in \cpp{} and stores it in corresponding object files.
For debugging purposes, it also creates a debugging information file as well as an assembly file containing a listing of the generated machine code.
The macro \texttt{\_\_armt32fpe\_\_} is predefined in order to enable programmers to identify this tool and its target architecture while compiling.
Programs generated with this compiler require additional runtime support that is stored in the \file{cpp\-armt32\-fpe\-run} library file.
\flowgraph{\resource{\cpp{}\\source code} \ar[r] & \toolbox{cpparmt32fpe} \ar[r] \ar[d] \ar[rd] & \resource{object file} \\ \variable{ECSINCLUDE} \ar[ru] & \resource{debugging\\information} & \resource{assembly\\listing}}
\seecpp\seeassembly\seearm\seeobject\seedebugging
}

\providecommand{\cppavr}{
\toolsection{cppavr} is a compiler for the \cpp{} programming language targeting the AVR hardware architecture.
It generates machine code for AVR processors from programs written in \cpp{} and stores it in corresponding object files.
For debugging purposes, it also creates a debugging information file as well as an assembly file containing a listing of the generated machine code.
The macro \texttt{\_\_avr\_\_} is predefined in order to enable programmers to identify this tool and its target architecture while compiling.
Programs generated with this compiler require additional runtime support that is stored in the \file{cpp\-avr\-run} library file.
\flowgraph{\resource{\cpp{}\\source code} \ar[r] & \toolbox{cppavr} \ar[r] \ar[d] \ar[rd] & \resource{object file} \\ \variable{ECSINCLUDE} \ar[ru] & \resource{debugging\\information} & \resource{assembly\\listing}}
\seecpp\seeassembly\seeavr\seeobject\seedebugging
}

\providecommand{\cppavrtt}{
\toolsection{cppavr32} is a compiler for the \cpp{} programming language targeting the AVR32 hardware architecture.
It generates machine code for AVR32 processors from programs written in \cpp{} and stores it in corresponding object files.
For debugging purposes, it also creates a debugging information file as well as an assembly file containing a listing of the generated machine code.
The macro \texttt{\_\_avr32\_\_} is predefined in order to enable programmers to identify this tool and its target architecture while compiling.
Programs generated with this compiler require additional runtime support that is stored in the \file{cpp\-avr32\-run} library file.
\flowgraph{\resource{\cpp{}\\source code} \ar[r] & \toolbox{cppavr32} \ar[r] \ar[d] \ar[rd] & \resource{object file} \\ \variable{ECSINCLUDE} \ar[ru] & \resource{debugging\\information} & \resource{assembly\\listing}}
\seecpp\seeassembly\seeavrtt\seeobject\seedebugging
}

\providecommand{\cppmabk}{
\toolsection{cppm68k} is a compiler for the \cpp{} programming language targeting the M68000 hardware architecture.
It generates machine code for M68000 processors from programs written in \cpp{} and stores it in corresponding object files.
For debugging purposes, it also creates a debugging information file as well as an assembly file containing a listing of the generated machine code.
The macro \texttt{\_\_m68k\_\_} is predefined in order to enable programmers to identify this tool and its target architecture while compiling.
Programs generated with this compiler require additional runtime support that is stored in the \file{cpp\-m68k\-run} library file.
\flowgraph{\resource{\cpp{}\\source code} \ar[r] & \toolbox{cppm68k} \ar[r] \ar[d] \ar[rd] & \resource{object file} \\ \variable{ECSINCLUDE} \ar[ru] & \resource{debugging\\information} & \resource{assembly\\listing}}
\seecpp\seeassembly\seemabk\seeobject\seedebugging
}

\providecommand{\cppmibl}{
\toolsection{cppmibl} is a compiler for the \cpp{} programming language targeting the MicroBlaze hardware architecture.
It generates machine code for MicroBlaze processors from programs written in \cpp{} and stores it in corresponding object files.
For debugging purposes, it also creates a debugging information file as well as an assembly file containing a listing of the generated machine code.
The macro \texttt{\_\_mibl\_\_} is predefined in order to enable programmers to identify this tool and its target architecture while compiling.
Programs generated with this compiler require additional runtime support that is stored in the \file{cpp\-mibl\-run} library file.
\flowgraph{\resource{\cpp{}\\source code} \ar[r] & \toolbox{cppmibl} \ar[r] \ar[d] \ar[rd] & \resource{object file} \\ \variable{ECSINCLUDE} \ar[ru] & \resource{debugging\\information} & \resource{assembly\\listing}}
\seecpp\seeassembly\seemibl\seeobject\seedebugging
}

\providecommand{\cppmipsa}{
\toolsection{cppmips32} is a compiler for the \cpp{} programming language targeting the MIPS32 hardware architecture.
It generates machine code for MIPS32 processors from programs written in \cpp{} and stores it in corresponding object files.
For debugging purposes, it also creates a debugging information file as well as an assembly file containing a listing of the generated machine code.
The macro \texttt{\_\_mips32\_\_} is predefined in order to enable programmers to identify this tool and its target architecture while compiling.
Programs generated with this compiler require additional runtime support that is stored in the \file{cpp\-mips32\-run} library file.
\flowgraph{\resource{\cpp{}\\source code} \ar[r] & \toolbox{cppmips32} \ar[r] \ar[d] \ar[rd] & \resource{object file} \\ \variable{ECSINCLUDE} \ar[ru] & \resource{debugging\\information} & \resource{assembly\\listing}}
\seecpp\seeassembly\seemips\seeobject\seedebugging
}

\providecommand{\cppmipsb}{
\toolsection{cppmips64} is a compiler for the \cpp{} programming language targeting the MIPS64 hardware architecture.
It generates machine code for MIPS64 processors from programs written in \cpp{} and stores it in corresponding object files.
For debugging purposes, it also creates a debugging information file as well as an assembly file containing a listing of the generated machine code.
The macro \texttt{\_\_mips64\_\_} is predefined in order to enable programmers to identify this tool and its target architecture while compiling.
Programs generated with this compiler require additional runtime support that is stored in the \file{cpp\-mips64\-run} library file.
\flowgraph{\resource{\cpp{}\\source code} \ar[r] & \toolbox{cppmips64} \ar[r] \ar[d] \ar[rd] & \resource{object file} \\ \variable{ECSINCLUDE} \ar[ru] & \resource{debugging\\information} & \resource{assembly\\listing}}
\seecpp\seeassembly\seemips\seeobject\seedebugging
}

\providecommand{\cppmmix}{
\toolsection{cppmmix} is a compiler for the \cpp{} programming language targeting the MMIX hardware architecture.
It generates machine code for MMIX processors from programs written in \cpp{} and stores it in corresponding object files.
For debugging purposes, it also creates a debugging information file as well as an assembly file containing a listing of the generated machine code.
The macro \texttt{\_\_mmix\_\_} is predefined in order to enable programmers to identify this tool and its target architecture while compiling.
Programs generated with this compiler require additional runtime support that is stored in the \file{cpp\-mmix\-run} library file.
\flowgraph{\resource{\cpp{}\\source code} \ar[r] & \toolbox{cppmmix} \ar[r] \ar[d] \ar[rd] & \resource{object file} \\ \variable{ECSINCLUDE} \ar[ru] & \resource{debugging\\information} & \resource{assembly\\listing}}
\seecpp\seeassembly\seemmix\seeobject\seedebugging
}

\providecommand{\cpporok}{
\toolsection{cppor1k} is a compiler for the \cpp{} programming language targeting the OpenRISC 1000 hardware architecture.
It generates machine code for OpenRISC 1000 processors from programs written in \cpp{} and stores it in corresponding object files.
For debugging purposes, it also creates a debugging information file as well as an assembly file containing a listing of the generated machine code.
The macro \texttt{\_\_or1k\_\_} is predefined in order to enable programmers to identify this tool and its target architecture while compiling.
Programs generated with this compiler require additional runtime support that is stored in the \file{cpp\-or1k\-run} library file.
\flowgraph{\resource{\cpp{}\\source code} \ar[r] & \toolbox{cppor1k} \ar[r] \ar[d] \ar[rd] & \resource{object file} \\ \variable{ECSINCLUDE} \ar[ru] & \resource{debugging\\information} & \resource{assembly\\listing}}
\seecpp\seeassembly\seeorok\seeobject\seedebugging
}

\providecommand{\cppppca}{
\toolsection{cppppc32} is a compiler for the \cpp{} programming language targeting the PowerPC hardware architecture.
It generates machine code for PowerPC processors from programs written in \cpp{} and stores it in corresponding object files.
The compiler generates machine code for the 32-bit operating mode defined by the PowerPC architecture.
For debugging purposes, it also creates a debugging information file as well as an assembly file containing a listing of the generated machine code.
The macro \texttt{\_\_ppc32\_\_} is predefined in order to enable programmers to identify this tool and its target architecture while compiling.
Programs generated with this compiler require additional runtime support that is stored in the \file{cpp\-ppc32\-run} library file.
\flowgraph{\resource{\cpp{}\\source code} \ar[r] & \toolbox{cppppc32} \ar[r] \ar[d] \ar[rd] & \resource{object file} \\ \variable{ECSINCLUDE} \ar[ru] & \resource{debugging\\information} & \resource{assembly\\listing}}
\seecpp\seeassembly\seeppc\seeobject\seedebugging
}

\providecommand{\cppppcb}{
\toolsection{cppppc64} is a compiler for the \cpp{} programming language targeting the PowerPC hardware architecture.
It generates machine code for PowerPC processors from programs written in \cpp{} and stores it in corresponding object files.
The compiler generates machine code for the 64-bit operating mode defined by the PowerPC architecture.
For debugging purposes, it also creates a debugging information file as well as an assembly file containing a listing of the generated machine code.
The macro \texttt{\_\_ppc64\_\_} is predefined in order to enable programmers to identify this tool and its target architecture while compiling.
Programs generated with this compiler require additional runtime support that is stored in the \file{cpp\-ppc64\-run} library file.
\flowgraph{\resource{\cpp{}\\source code} \ar[r] & \toolbox{cppppc64} \ar[r] \ar[d] \ar[rd] & \resource{object file} \\ \variable{ECSINCLUDE} \ar[ru] & \resource{debugging\\information} & \resource{assembly\\listing}}
\seecpp\seeassembly\seeppc\seeobject\seedebugging
}

\providecommand{\cpprisc}{
\toolsection{cpprisc} is a compiler for the \cpp{} programming language targeting the RISC hardware architecture.
It generates machine code for RISC processors from programs written in \cpp{} and stores it in corresponding object files.
For debugging purposes, it also creates a debugging information file as well as an assembly file containing a listing of the generated machine code.
The macro \texttt{\_\_risc\_\_} is predefined in order to enable programmers to identify this tool and its target architecture while compiling.
Programs generated with this compiler require additional runtime support that is stored in the \file{cpp\-risc\-run} library file.
\flowgraph{\resource{\cpp{}\\source code} \ar[r] & \toolbox{cpprisc} \ar[r] \ar[d] \ar[rd] & \resource{object file} \\ \variable{ECSINCLUDE} \ar[ru] & \resource{debugging\\information} & \resource{assembly\\listing}}
\seecpp\seeassembly\seerisc\seeobject\seedebugging
}

\providecommand{\cppwasm}{
\toolsection{cppwasm} is a compiler for the \cpp{} programming language targeting the WebAssembly architecture.
It generates machine code for WebAssembly targets from programs written in \cpp{} and stores it in corresponding object files.
For debugging purposes, it also creates a debugging information file as well as an assembly file containing a listing of the generated machine code.
The macro \texttt{\_\_wasm\_\_} is predefined in order to enable programmers to identify this tool and its target architecture while compiling.
Programs generated with this compiler require additional runtime support that is stored in the \file{cpp\-wasm\-run} library file.
\flowgraph{\resource{\cpp{}\\source code} \ar[r] & \toolbox{cppwasm} \ar[r] \ar[d] \ar[rd] & \resource{object file} \\ \variable{ECSINCLUDE} \ar[ru] & \resource{debugging\\information} & \resource{assembly\\listing}}
\seecpp\seeassembly\seewasm\seeobject\seedebugging
}

% FALSE tools

\providecommand{\falprint}{
\toolsection{falprint} is a pretty printer for the FALSE programming language.
It reformats the source code of FALSE programs and writes it to the standard output stream.
\flowgraph{\resource{FALSE\\source code} \ar[r] & \toolbox{falprint} \ar[r] & \resource{reformatted\\source code}}
\seefalse
}

\providecommand{\falcheck}{
\toolsection{falcheck} is a syntactic and semantic checker for the FALSE programming language.
It just performs syntactic and semantic checks on FALSE programs and writes its diagnostic messages to the standard error stream.
\flowgraph{\resource{FALSE\\source code} \ar[r] & \toolbox{falcheck} \ar[r] & \resource{diagnostic\\messages}}
\seefalse
}

\providecommand{\faldump}{
\toolsection{faldump} is a serializer for the FALSE programming language.
It dumps the complete internal representation of programs written in FALSE into an XML document.
\debuggingtool
\flowgraph{\resource{FALSE\\source code} \ar[r] & \toolbox{faldump} \ar[r] & \resource{internal\\representation}}
\seefalse
}

\providecommand{\falrun}{
\toolsection{falrun} is an interpreter for the FALSE programming language.
It processes and executes programs written in FALSE\@.
\flowgraph{\resource{FALSE\\source code} \ar[r] & \toolbox{falrun} \ar@/u/[r] & \resource{input/\\output} \ar@/d/[l]}
\seefalse
}

\providecommand{\falcpp}{
\toolsection{falcpp} is a transpiler for the FALSE programming language.
It translates programs written in FALSE into \cpp{} programs and stores them in corresponding source files.
\flowgraph{\resource{FALSE\\source code} \ar[r] & \toolbox{falcpp} \ar[r] & \resource{\cpp{}\\source file}}
\seefalse\seecpp
}

\providecommand{\falcode}{
\toolsection{falcode} is an intermediate code generator for the FALSE programming language.
It generates intermediate code from programs written in FALSE and stores it in corresponding assembly files.
\debuggingtool
\flowgraph{\resource{FALSE\\source code} \ar[r] & \toolbox{falcode} \ar[r] & \resource{intermediate\\code}}
\seefalse\seeassembly\seecode
}

\providecommand{\falamda}{
\toolsection{falamd16} is a compiler for the FALSE programming language targeting the AMD64 hardware architecture.
It generates machine code for AMD64 processors from programs written in FALSE and stores it in corresponding object files.
The compiler generates machine code for the 16-bit operating mode defined by the AMD64 architecture.
\flowgraph{\resource{FALSE\\source code} \ar[r] & \toolbox{falamd16} \ar[r] & \resource{object file}}
\seefalse\seeamd\seeobject
}

\providecommand{\falamdb}{
\toolsection{falamd32} is a compiler for the FALSE programming language targeting the AMD64 hardware architecture.
It generates machine code for AMD64 processors from programs written in FALSE and stores it in corresponding object files.
The compiler generates machine code for the 32-bit operating mode defined by the AMD64 architecture.
\flowgraph{\resource{FALSE\\source code} \ar[r] & \toolbox{falamd32} \ar[r] & \resource{object file}}
\seefalse\seeamd\seeobject
}

\providecommand{\falamdc}{
\toolsection{falamd64} is a compiler for the FALSE programming language targeting the AMD64 hardware architecture.
It generates machine code for AMD64 processors from programs written in FALSE and stores it in corresponding object files.
The compiler generates machine code for the 64-bit operating mode defined by the AMD64 architecture.
\flowgraph{\resource{FALSE\\source code} \ar[r] & \toolbox{falamd64} \ar[r] & \resource{object file}}
\seefalse\seeamd\seeobject
}

\providecommand{\falarma}{
\toolsection{falarma32} is a compiler for the FALSE programming language targeting the ARM hardware architecture.
It generates machine code for ARM processors executing A32 instructions from programs written in FALSE and stores it in corresponding object files.
\flowgraph{\resource{FALSE\\source code} \ar[r] & \toolbox{falarma32} \ar[r] & \resource{object file}}
\seefalse\seearm\seeobject
}

\providecommand{\falarmb}{
\toolsection{falarma64} is a compiler for the FALSE programming language targeting the ARM hardware architecture.
It generates machine code for ARM processors executing A64 instructions from programs written in FALSE and stores it in corresponding object files.
\flowgraph{\resource{FALSE\\source code} \ar[r] & \toolbox{falarma64} \ar[r] & \resource{object file}}
\seefalse\seearm\seeobject
}

\providecommand{\falarmc}{
\toolsection{falarmt32} is a compiler for the FALSE programming language targeting the ARM hardware architecture.
It generates machine code for ARM processors without floating-point extension executing T32 instructions from programs written in FALSE and stores it in corresponding object files.
\flowgraph{\resource{FALSE\\source code} \ar[r] & \toolbox{falarmt32} \ar[r] & \resource{object file}}
\seefalse\seearm\seeobject
}

\providecommand{\falarmcfpe}{
\toolsection{falarmt32fpe} is a compiler for the FALSE programming language targeting the ARM hardware architecture.
It generates machine code for ARM processors with floating-point extension executing T32 instructions from programs written in FALSE and stores it in corresponding object files.
\flowgraph{\resource{FALSE\\source code} \ar[r] & \toolbox{falarmt32fpe} \ar[r] & \resource{object file}}
\seefalse\seearm\seeobject
}

\providecommand{\falavr}{
\toolsection{falavr} is a compiler for the FALSE programming language targeting the AVR hardware architecture.
It generates machine code for AVR processors from programs written in FALSE and stores it in corresponding object files.
\flowgraph{\resource{FALSE\\source code} \ar[r] & \toolbox{falavr} \ar[r] & \resource{object file}}
\seefalse\seeavr\seeobject
}

\providecommand{\falavrtt}{
\toolsection{falavr32} is a compiler for the FALSE programming language targeting the AVR32 hardware architecture.
It generates machine code for AVR32 processors from programs written in FALSE and stores it in corresponding object files.
\flowgraph{\resource{FALSE\\source code} \ar[r] & \toolbox{falavr32} \ar[r] & \resource{object file}}
\seefalse\seeavrtt\seeobject
}

\providecommand{\falmabk}{
\toolsection{falm68k} is a compiler for the FALSE programming language targeting the M68000 hardware architecture.
It generates machine code for M68000 processors from programs written in FALSE and stores it in corresponding object files.
\flowgraph{\resource{FALSE\\source code} \ar[r] & \toolbox{falm68k} \ar[r] & \resource{object file}}
\seefalse\seemabk\seeobject
}

\providecommand{\falmibl}{
\toolsection{falmibl} is a compiler for the FALSE programming language targeting the MicroBlaze hardware architecture.
It generates machine code for MicroBlaze processors from programs written in FALSE and stores it in corresponding object files.
\flowgraph{\resource{FALSE\\source code} \ar[r] & \toolbox{falmibl} \ar[r] & \resource{object file}}
\seefalse\seemibl\seeobject
}

\providecommand{\falmipsa}{
\toolsection{falmips32} is a compiler for the FALSE programming language targeting the MIPS32 hardware architecture.
It generates machine code for MIPS32 processors from programs written in FALSE and stores it in corresponding object files.
\flowgraph{\resource{FALSE\\source code} \ar[r] & \toolbox{falmips32} \ar[r] & \resource{object file}}
\seefalse\seemips\seeobject
}

\providecommand{\falmipsb}{
\toolsection{falmips64} is a compiler for the FALSE programming language targeting the MIPS64 hardware architecture.
It generates machine code for MIPS64 processors from programs written in FALSE and stores it in corresponding object files.
\flowgraph{\resource{FALSE\\source code} \ar[r] & \toolbox{falmips64} \ar[r] & \resource{object file}}
\seefalse\seemips\seeobject
}

\providecommand{\falmmix}{
\toolsection{falmmix} is a compiler for the FALSE programming language targeting the MMIX hardware architecture.
It generates machine code for MMIX processors from programs written in FALSE and stores it in corresponding object files.
\flowgraph{\resource{FALSE\\source code} \ar[r] & \toolbox{falmmix} \ar[r] & \resource{object file}}
\seefalse\seemmix\seeobject
}

\providecommand{\falorok}{
\toolsection{falor1k} is a compiler for the FALSE programming language targeting the OpenRISC 1000 hardware architecture.
It generates machine code for OpenRISC 1000 processors from programs written in FALSE and stores it in corresponding object files.
\flowgraph{\resource{FALSE\\source code} \ar[r] & \toolbox{falor1k} \ar[r] & \resource{object file}}
\seefalse\seeorok\seeobject
}

\providecommand{\falppca}{
\toolsection{falppc32} is a compiler for the FALSE programming language targeting the PowerPC hardware architecture.
It generates machine code for PowerPC processors from programs written in FALSE and stores it in corresponding object files.
The compiler generates machine code for the 32-bit operating mode defined by the PowerPC architecture.
\flowgraph{\resource{FALSE\\source code} \ar[r] & \toolbox{falppc32} \ar[r] & \resource{object file}}
\seefalse\seeppc\seeobject
}

\providecommand{\falppcb}{
\toolsection{falppc64} is a compiler for the FALSE programming language targeting the PowerPC hardware architecture.
It generates machine code for PowerPC processors from programs written in FALSE and stores it in corresponding object files.
The compiler generates machine code for the 64-bit operating mode defined by the PowerPC architecture.
\flowgraph{\resource{FALSE\\source code} \ar[r] & \toolbox{falppc64} \ar[r] & \resource{object file}}
\seefalse\seeppc\seeobject
}

\providecommand{\falrisc}{
\toolsection{falrisc} is a compiler for the FALSE programming language targeting the RISC hardware architecture.
It generates machine code for RISC processors from programs written in FALSE and stores it in corresponding object files.
\flowgraph{\resource{FALSE\\source code} \ar[r] & \toolbox{falrisc} \ar[r] & \resource{object file}}
\seefalse\seerisc\seeobject
}

\providecommand{\falwasm}{
\toolsection{falwasm} is a compiler for the FALSE programming language targeting the WebAssembly architecture.
It generates machine code for WebAssembly targets from programs written in FALSE and stores it in corresponding object files.
\flowgraph{\resource{FALSE\\source code} \ar[r] & \toolbox{falwasm} \ar[r] & \resource{object file}}
\seefalse\seewasm\seeobject
}

% Oberon tools

\providecommand{\obprint}{
\toolsection{obprint} is a pretty printer for the Oberon programming language.
It reformats the source code of Oberon modules and writes it to the standard output stream.
\flowgraph{\resource{Oberon\\source code} \ar[r] & \toolbox{obprint} \ar[r] & \resource{reformatted\\source code}}
\seeoberon
}

\providecommand{\obcheck}{
\toolsection{obcheck} is a syntactic and semantic checker for the Oberon programming language.
It just performs syntactic and semantic checks on Oberon modules and writes its diagnostic messages to the standard error stream.
In addition, it stores the interface of each module in a symbol file which is required when other modules import the module.
\flowgraph{\resource{Oberon\\source code} \ar[r] & \toolbox{obcheck} \ar[r] \ar@/l/[d] & \resource{diagnostic\\messages} \\ \variable{ECSIMPORT} \ar[ru] & \resource{symbol\\files} \ar@/r/[u]}
\seeoberon
}

\providecommand{\obdump}{
\toolsection{obdump} is a serializer for the Oberon programming language.
It dumps the complete internal representation of modules written in Oberon into an XML document.
\debuggingtool
\flowgraph{\resource{Oberon\\source code} \ar[r] & \toolbox{obdump} \ar[r] \ar@/l/[d] & \resource{internal\\representation} \\ \variable{ECSIMPORT} \ar[ru] & \resource{symbol\\files} \ar@/r/[u]}
\seeoberon
}

\providecommand{\obrun}{
\toolsection{obrun} is an interpreter for the Oberon programming language.
It processes and executes modules written in Oberon.
This tool does neither generate nor process symbol files while interpreting modules.
If a module is imported by another one, its filename has to be named before the other one in the list of command-line arguments.
\flowgraph{\resource{Oberon\\source code} \ar[r] & \toolbox{obrun} \ar@/u/[r] & \resource{input/\\output} \ar@/d/[l]}
\seeoberon
}

\providecommand{\obcpp}{
\toolsection{obcpp} is a transpiler for the Oberon programming language.
It translates programs written in Oberon into \cpp{} programs and stores them in corresponding source and header files.
In addition, it stores the interface of each module in a symbol file which is required when other modules import the module.
The same interface is provided by the generated header file which can be used in other parts of the \cpp{} program.
\flowgraph{\resource{Oberon\\source code} \ar[r] & \toolbox{obcpp} \ar[r] \ar@/l/[d] \ar[rd] & \resource{\cpp{}\\source file} \\ \variable{ECSIMPORT} \ar[ru] & \resource{symbol\\files} \ar@/r/[u] & \resource{\cpp{}\\header file}}
\seeoberon\seecpp
}

\providecommand{\obdoc}{
\toolsection{obdoc} is a generic documentation generator for the Oberon programming language.
It processes several Oberon modules and assembles all information therein into a generic documentation.
In addition, it stores the interface of each module in a symbol file which is required when other modules import the module.
\debuggingtool
\flowgraph{\resource{Oberon\\source code} \ar[r] & \toolbox{obdoc} \ar[r] \ar@/l/[d] & \resource{generic\\documentation} \\ \variable{ECSIMPORT} \ar[ru] & \resource{symbol\\files} \ar@/r/[u]}
\seeoberon\seedocumentation
}

\providecommand{\obhtml}{
\toolsection{obhtml} is an HTML documentation generator for the Oberon programming language.
It processes several Oberon modules and assembles all information therein into an HTML document.
In addition, it stores the interface of each module in a symbol file which is required when other modules import the module.
\flowgraph{\resource{Oberon\\source code} \ar[r] & \toolbox{obhtml} \ar[r] \ar@/l/[d] & \resource{HTML\\document} \\ \variable{ECSIMPORT} \ar[ru] & \resource{symbol\\files} \ar@/r/[u]}
\seeoberon\seedocumentation
}

\providecommand{\oblatex}{
\toolsection{oblatex} is a Latex documentation generator for the Oberon programming language.
It processes several Oberon modules and assembles all information therein into a Latex document.
In addition, it stores the interface of each module in a symbol file which is required when other modules import the module.
\flowgraph{\resource{Oberon\\source code} \ar[r] & \toolbox{oblatex} \ar[r] \ar@/l/[d] & \resource{Latex\\document} \\ \variable{ECSIMPORT} \ar[ru] & \resource{symbol\\files} \ar@/r/[u]}
\seeoberon\seedocumentation
}

\providecommand{\obcode}{
\toolsection{obcode} is an intermediate code generator for the Oberon programming language.
It generates intermediate code from modules written in Oberon and stores it in corresponding assembly files.
In addition, it stores the interface of each module in a symbol file which is required when other modules import the module.
Programs generated with this tool require additional runtime support that is stored in the \file{ob\-code\-run} library file.
\debuggingtool
\flowgraph{\resource{Oberon\\source code} \ar[r] & \toolbox{obcode} \ar[r] \ar@/l/[d] & \resource{intermediate\\code} \\ \variable{ECSIMPORT} \ar[ru] & \resource{symbol\\files} \ar@/r/[u]}
\seeoberon\seeassembly\seecode
}

\providecommand{\obamda}{
\toolsection{obamd16} is a compiler for the Oberon programming language targeting the AMD64 hardware architecture.
It generates machine code for AMD64 processors from modules written in Oberon and stores it in corresponding object files.
The compiler generates machine code for the 16-bit operating mode defined by the AMD64 architecture.
For debugging purposes, it also creates a debugging information file as well as an assembly file containing a listing of the generated machine code.
In addition, it stores the interface of each module in a symbol file which is required when other modules import the module.
Programs generated with this compiler require additional runtime support that is stored in the \file{ob\-amd16\-run} library file.
\flowgraph{\resource{Oberon\\source code} \ar[r] & \toolbox{obamd16} \ar[r] \ar@/l/[d] \ar[rd] & \resource{object file} \\ \variable{ECSIMPORT} \ar[ru] & \resource{symbol\\files} \ar@/r/[u] & \resource{debugging\\information}}
\seeoberon\seeassembly\seeamd\seeobject\seedebugging
}

\providecommand{\obamdb}{
\toolsection{obamd32} is a compiler for the Oberon programming language targeting the AMD64 hardware architecture.
It generates machine code for AMD64 processors from modules written in Oberon and stores it in corresponding object files.
The compiler generates machine code for the 32-bit operating mode defined by the AMD64 architecture.
For debugging purposes, it also creates a debugging information file as well as an assembly file containing a listing of the generated machine code.
In addition, it stores the interface of each module in a symbol file which is required when other modules import the module.
Programs generated with this compiler require additional runtime support that is stored in the \file{ob\-amd32\-run} library file.
\flowgraph{\resource{Oberon\\source code} \ar[r] & \toolbox{obamd32} \ar[r] \ar@/l/[d] \ar[rd] & \resource{object file} \\ \variable{ECSIMPORT} \ar[ru] & \resource{symbol\\files} \ar@/r/[u] & \resource{debugging\\information}}
\seeoberon\seeassembly\seeamd\seeobject\seedebugging
}

\providecommand{\obamdc}{
\toolsection{obamd64} is a compiler for the Oberon programming language targeting the AMD64 hardware architecture.
It generates machine code for AMD64 processors from modules written in Oberon and stores it in corresponding object files.
The compiler generates machine code for the 64-bit operating mode defined by the AMD64 architecture.
For debugging purposes, it also creates a debugging information file as well as an assembly file containing a listing of the generated machine code.
In addition, it stores the interface of each module in a symbol file which is required when other modules import the module.
Programs generated with this compiler require additional runtime support that is stored in the \file{ob\-amd64\-run} library file.
\flowgraph{\resource{Oberon\\source code} \ar[r] & \toolbox{obamd64} \ar[r] \ar@/l/[d] \ar[rd] & \resource{object file} \\ \variable{ECSIMPORT} \ar[ru] & \resource{symbol\\files} \ar@/r/[u] & \resource{debugging\\information}}
\seeoberon\seeassembly\seeamd\seeobject\seedebugging
}

\providecommand{\obarma}{
\toolsection{obarma32} is a compiler for the Oberon programming language targeting the ARM hardware architecture.
It generates machine code for ARM processors executing A32 instructions from modules written in Oberon and stores it in corresponding object files.
For debugging purposes, it also creates a debugging information file as well as an assembly file containing a listing of the generated machine code.
In addition, it stores the interface of each module in a symbol file which is required when other modules import the module.
Programs generated with this compiler require additional runtime support that is stored in the \file{ob\-arma32\-run} library file.
\flowgraph{\resource{Oberon\\source code} \ar[r] & \toolbox{obarma32} \ar[r] \ar@/l/[d] \ar[rd] & \resource{object file} \\ \variable{ECSIMPORT} \ar[ru] & \resource{symbol\\files} \ar@/r/[u] & \resource{debugging\\information}}
\seeoberon\seeassembly\seearm\seeobject\seedebugging
}

\providecommand{\obarmb}{
\toolsection{obarma64} is a compiler for the Oberon programming language targeting the ARM hardware architecture.
It generates machine code for ARM processors executing A64 instructions from modules written in Oberon and stores it in corresponding object files.
For debugging purposes, it also creates a debugging information file as well as an assembly file containing a listing of the generated machine code.
In addition, it stores the interface of each module in a symbol file which is required when other modules import the module.
Programs generated with this compiler require additional runtime support that is stored in the \file{ob\-arma64\-run} library file.
\flowgraph{\resource{Oberon\\source code} \ar[r] & \toolbox{obarma64} \ar[r] \ar@/l/[d] \ar[rd] & \resource{object file} \\ \variable{ECSIMPORT} \ar[ru] & \resource{symbol\\files} \ar@/r/[u] & \resource{debugging\\information}}
\seeoberon\seeassembly\seearm\seeobject\seedebugging
}

\providecommand{\obarmc}{
\toolsection{obarmt32} is a compiler for the Oberon programming language targeting the ARM hardware architecture.
It generates machine code for ARM processors without floating-point extension executing T32 instructions from modules written in Oberon and stores it in corresponding object files.
For debugging purposes, it also creates a debugging information file as well as an assembly file containing a listing of the generated machine code.
In addition, it stores the interface of each module in a symbol file which is required when other modules import the module.
Programs generated with this compiler require additional runtime support that is stored in the \file{ob\-armt32\-run} library file.
\flowgraph{\resource{Oberon\\source code} \ar[r] & \toolbox{obarmt32} \ar[r] \ar@/l/[d] \ar[rd] & \resource{object file} \\ \variable{ECSIMPORT} \ar[ru] & \resource{symbol\\files} \ar@/r/[u] & \resource{debugging\\information}}
\seeoberon\seeassembly\seearm\seeobject\seedebugging
}

\providecommand{\obarmcfpe}{
\toolsection{obarmt32fpe} is a compiler for the Oberon programming language targeting the ARM hardware architecture.
It generates machine code for ARM processors with floating-point extension executing T32 instructions from modules written in Oberon and stores it in corresponding object files.
For debugging purposes, it also creates a debugging information file as well as an assembly file containing a listing of the generated machine code.
In addition, it stores the interface of each module in a symbol file which is required when other modules import the module.
Programs generated with this compiler require additional runtime support that is stored in the \file{ob\-armt32\-fpe\-run} library file.
\flowgraph{\resource{Oberon\\source code} \ar[r] & \toolbox{obarmt32fpe} \ar[r] \ar@/l/[d] \ar[rd] & \resource{object file} \\ \variable{ECSIMPORT} \ar[ru] & \resource{symbol\\files} \ar@/r/[u] & \resource{debugging\\information}}
\seeoberon\seeassembly\seearm\seeobject\seedebugging
}

\providecommand{\obavr}{
\toolsection{obavr} is a compiler for the Oberon programming language targeting the AVR hardware architecture.
It generates machine code for AVR processors from modules written in Oberon and stores it in corresponding object files.
For debugging purposes, it also creates a debugging information file as well as an assembly file containing a listing of the generated machine code.
In addition, it stores the interface of each module in a symbol file which is required when other modules import the module.
Programs generated with this compiler require additional runtime support that is stored in the \file{ob\-avr\-run} library file.
\flowgraph{\resource{Oberon\\source code} \ar[r] & \toolbox{obavr} \ar[r] \ar@/l/[d] \ar[rd] & \resource{object file} \\ \variable{ECSIMPORT} \ar[ru] & \resource{symbol\\files} \ar@/r/[u] & \resource{debugging\\information}}
\seeoberon\seeassembly\seeavr\seeobject\seedebugging
}

\providecommand{\obavrtt}{
\toolsection{obavr32} is a compiler for the Oberon programming language targeting the AVR32 hardware architecture.
It generates machine code for AVR32 processors from modules written in Oberon and stores it in corresponding object files.
For debugging purposes, it also creates a debugging information file as well as an assembly file containing a listing of the generated machine code.
In addition, it stores the interface of each module in a symbol file which is required when other modules import the module.
Programs generated with this compiler require additional runtime support that is stored in the \file{ob\-avr32\-run} library file.
\flowgraph{\resource{Oberon\\source code} \ar[r] & \toolbox{obavr32} \ar[r] \ar@/l/[d] \ar[rd] & \resource{object file} \\ \variable{ECSIMPORT} \ar[ru] & \resource{symbol\\files} \ar@/r/[u] & \resource{debugging\\information}}
\seeoberon\seeassembly\seeavrtt\seeobject\seedebugging
}

\providecommand{\obmabk}{
\toolsection{obm68k} is a compiler for the Oberon programming language targeting the M68000 hardware architecture.
It generates machine code for M68000 processors from modules written in Oberon and stores it in corresponding object files.
For debugging purposes, it also creates a debugging information file as well as an assembly file containing a listing of the generated machine code.
In addition, it stores the interface of each module in a symbol file which is required when other modules import the module.
Programs generated with this compiler require additional runtime support that is stored in the \file{ob\-m68k\-run} library file.
\flowgraph{\resource{Oberon\\source code} \ar[r] & \toolbox{obm68k} \ar[r] \ar@/l/[d] \ar[rd] & \resource{object file} \\ \variable{ECSIMPORT} \ar[ru] & \resource{symbol\\files} \ar@/r/[u] & \resource{debugging\\information}}
\seeoberon\seeassembly\seemabk\seeobject\seedebugging
}

\providecommand{\obmibl}{
\toolsection{obmibl} is a compiler for the Oberon programming language targeting the MicroBlaze hardware architecture.
It generates machine code for MicroBlaze processors from modules written in Oberon and stores it in corresponding object files.
For debugging purposes, it also creates a debugging information file as well as an assembly file containing a listing of the generated machine code.
In addition, it stores the interface of each module in a symbol file which is required when other modules import the module.
Programs generated with this compiler require additional runtime support that is stored in the \file{ob\-mibl\-run} library file.
\flowgraph{\resource{Oberon\\source code} \ar[r] & \toolbox{obmibl} \ar[r] \ar@/l/[d] \ar[rd] & \resource{object file} \\ \variable{ECSIMPORT} \ar[ru] & \resource{symbol\\files} \ar@/r/[u] & \resource{debugging\\information}}
\seeoberon\seeassembly\seemibl\seeobject\seedebugging
}

\providecommand{\obmipsa}{
\toolsection{obmips32} is a compiler for the Oberon programming language targeting the MIPS32 hardware architecture.
It generates machine code for MIPS32 processors from modules written in Oberon and stores it in corresponding object files.
For debugging purposes, it also creates a debugging information file as well as an assembly file containing a listing of the generated machine code.
In addition, it stores the interface of each module in a symbol file which is required when other modules import the module.
Programs generated with this compiler require additional runtime support that is stored in the \file{ob\-mips32\-run} library file.
\flowgraph{\resource{Oberon\\source code} \ar[r] & \toolbox{obmips32} \ar[r] \ar@/l/[d] \ar[rd] & \resource{object file} \\ \variable{ECSIMPORT} \ar[ru] & \resource{symbol\\files} \ar@/r/[u] & \resource{debugging\\information}}
\seeoberon\seeassembly\seemips\seeobject\seedebugging
}

\providecommand{\obmipsb}{
\toolsection{obmips64} is a compiler for the Oberon programming language targeting the MIPS64 hardware architecture.
It generates machine code for MIPS64 processors from modules written in Oberon and stores it in corresponding object files.
For debugging purposes, it also creates a debugging information file as well as an assembly file containing a listing of the generated machine code.
In addition, it stores the interface of each module in a symbol file which is required when other modules import the module.
Programs generated with this compiler require additional runtime support that is stored in the \file{ob\-mips64\-run} library file.
\flowgraph{\resource{Oberon\\source code} \ar[r] & \toolbox{obmips64} \ar[r] \ar@/l/[d] \ar[rd] & \resource{object file} \\ \variable{ECSIMPORT} \ar[ru] & \resource{symbol\\files} \ar@/r/[u] & \resource{debugging\\information}}
\seeoberon\seeassembly\seemips\seeobject\seedebugging
}

\providecommand{\obmmix}{
\toolsection{obmmix} is a compiler for the Oberon programming language targeting the MMIX hardware architecture.
It generates machine code for MMIX processors from modules written in Oberon and stores it in corresponding object files.
For debugging purposes, it also creates a debugging information file as well as an assembly file containing a listing of the generated machine code.
In addition, it stores the interface of each module in a symbol file which is required when other modules import the module.
Programs generated with this compiler require additional runtime support that is stored in the \file{ob\-mmix\-run} library file.
\flowgraph{\resource{Oberon\\source code} \ar[r] & \toolbox{obmmix} \ar[r] \ar@/l/[d] \ar[rd] & \resource{object file} \\ \variable{ECSIMPORT} \ar[ru] & \resource{symbol\\files} \ar@/r/[u] & \resource{debugging\\information}}
\seeoberon\seeassembly\seemmix\seeobject\seedebugging
}

\providecommand{\oborok}{
\toolsection{obor1k} is a compiler for the Oberon programming language targeting the OpenRISC 1000 hardware architecture.
It generates machine code for OpenRISC 1000 processors from modules written in Oberon and stores it in corresponding object files.
For debugging purposes, it also creates a debugging information file as well as an assembly file containing a listing of the generated machine code.
In addition, it stores the interface of each module in a symbol file which is required when other modules import the module.
Programs generated with this compiler require additional runtime support that is stored in the \file{ob\-or1k\-run} library file.
\flowgraph{\resource{Oberon\\source code} \ar[r] & \toolbox{obor1k} \ar[r] \ar@/l/[d] \ar[rd] & \resource{object file} \\ \variable{ECSIMPORT} \ar[ru] & \resource{symbol\\files} \ar@/r/[u] & \resource{debugging\\information}}
\seeoberon\seeassembly\seeorok\seeobject\seedebugging
}

\providecommand{\obppca}{
\toolsection{obppc32} is a compiler for the Oberon programming language targeting the PowerPC hardware architecture.
It generates machine code for PowerPC processors from modules written in Oberon and stores it in corresponding object files.
The compiler generates machine code for the 32-bit operating mode defined by the PowerPC architecture.
For debugging purposes, it also creates a debugging information file as well as an assembly file containing a listing of the generated machine code.
In addition, it stores the interface of each module in a symbol file which is required when other modules import the module.
Programs generated with this compiler require additional runtime support that is stored in the \file{ob\-ppc32\-run} library file.
\flowgraph{\resource{Oberon\\source code} \ar[r] & \toolbox{obppc32} \ar[r] \ar@/l/[d] \ar[rd] & \resource{object file} \\ \variable{ECSIMPORT} \ar[ru] & \resource{symbol\\files} \ar@/r/[u] & \resource{debugging\\information}}
\seeoberon\seeassembly\seeppc\seeobject\seedebugging
}

\providecommand{\obppcb}{
\toolsection{obppc64} is a compiler for the Oberon programming language targeting the PowerPC hardware architecture.
It generates machine code for PowerPC processors from modules written in Oberon and stores it in corresponding object files.
The compiler generates machine code for the 64-bit operating mode defined by the PowerPC architecture.
For debugging purposes, it also creates a debugging information file as well as an assembly file containing a listing of the generated machine code.
In addition, it stores the interface of each module in a symbol file which is required when other modules import the module.
Programs generated with this compiler require additional runtime support that is stored in the \file{ob\-ppc64\-run} library file.
\flowgraph{\resource{Oberon\\source code} \ar[r] & \toolbox{obppc64} \ar[r] \ar@/l/[d] \ar[rd] & \resource{object file} \\ \variable{ECSIMPORT} \ar[ru] & \resource{symbol\\files} \ar@/r/[u] & \resource{debugging\\information}}
\seeoberon\seeassembly\seeppc\seeobject\seedebugging
}

\providecommand{\obrisc}{
\toolsection{obrisc} is a compiler for the Oberon programming language targeting the RISC hardware architecture.
It generates machine code for RISC processors from modules written in Oberon and stores it in corresponding object files.
For debugging purposes, it also creates a debugging information file as well as an assembly file containing a listing of the generated machine code.
In addition, it stores the interface of each module in a symbol file which is required when other modules import the module.
Programs generated with this compiler require additional runtime support that is stored in the \file{ob\-risc\-run} library file.
\flowgraph{\resource{Oberon\\source code} \ar[r] & \toolbox{obrisc} \ar[r] \ar@/l/[d] \ar[rd] & \resource{object file} \\ \variable{ECSIMPORT} \ar[ru] & \resource{symbol\\files} \ar@/r/[u] & \resource{debugging\\information}}
\seeoberon\seeassembly\seerisc\seeobject\seedebugging
}

\providecommand{\obwasm}{
\toolsection{obwasm} is a compiler for the Oberon programming language targeting the WebAssembly architecture.
It generates machine code for WebAssembly targets from modules written in Oberon and stores it in corresponding object files.
For debugging purposes, it also creates a debugging information file as well as an assembly file containing a listing of the generated machine code.
In addition, it stores the interface of each module in a symbol file which is required when other modules import the module.
Programs generated with this compiler require additional runtime support that is stored in the \file{ob\-wasm\-run} library file.
\flowgraph{\resource{Oberon\\source code} \ar[r] & \toolbox{obwasm} \ar[r] \ar@/l/[d] \ar[rd] & \resource{object file} \\ \variable{ECSIMPORT} \ar[ru] & \resource{symbol\\files} \ar@/r/[u] & \resource{debugging\\information}}
\seeoberon\seeassembly\seewasm\seeobject\seedebugging
}

% converter tools

\providecommand{\dbgdwarf}{
\toolsection{dbgdwarf} is a DWARF debugging information converter tool.
It converts debugging information into the DWARF debugging data format and stores it in corresponding object files~\cite{dwarffile}.
The resulting debugging object files can be combined with runtime support that creates Executable and Linking Format (ELF) files~\cite{elffile}.
\flowgraph{\resource{debugging\\information} \ar[r] & \toolbox{dbgdwarf} \ar[r] & \resource{debugging\\object file}}
\seeobject\seedebugging
}

% assembler tools

\providecommand{\asmprint}{
\toolsection{asmprint} is a pretty printer for generic assembly code.
It reformats generic assembly code and writes it to the standard output stream.
\flowgraph{\resource{generic assembly\\source code} \ar[r] & \toolbox{asmprint} \ar[r] & \resource{reformatted\\source code}}
\seeassembly
}

\providecommand{\amdaasm}{
\toolsection{amd16asm} is an assembler for the AMD64 hardware architecture.
It translates assembly code into machine code for AMD64 processors and stores it in corresponding object files.
By default, the assembler generates machine code for the 16-bit operating mode defined by the AMD64 architecture.
\flowgraph{\resource{AMD16 assembly\\source code} \ar[r] & \toolbox{amd16asm} \ar[r] & \resource{object file}}
\seeassembly\seeamd\seeobject
}

\providecommand{\amdadism}{
\toolsection{amd16dism} is a disassembler for the AMD64 hardware architecture.
It translates machine code from object files targeting AMD64 processors into assembly code and writes it to the standard output stream.
It assumes that the machine code was generated for the 16-bit operating mode defined by the AMD64 architecture.
\flowgraph{\resource{object file} \ar[r] & \toolbox{amd16dism} \ar[r] & \resource{disassembly\\listing}}
\seeassembly\seeamd\seeobject
}

\providecommand{\amdbasm}{
\toolsection{amd32asm} is an assembler for the AMD64 hardware architecture.
It translates assembly code into machine code for AMD64 processors and stores it in corresponding object files.
By default, the assembler generates machine code for the 32-bit operating mode defined by the AMD64 architecture.
\flowgraph{\resource{AMD32 assembly\\source code} \ar[r] & \toolbox{amd32asm} \ar[r] & \resource{object file}}
\seeassembly\seeamd\seeobject
}

\providecommand{\amdbdism}{
\toolsection{amd32dism} is a disassembler for the AMD64 hardware architecture.
It translates machine code from object files targeting AMD64 processors into assembly code and writes it to the standard output stream.
It assumes that the machine code was generated for the 32-bit operating mode defined by the AMD64 architecture.
\flowgraph{\resource{object file} \ar[r] & \toolbox{amd32dism} \ar[r] & \resource{disassembly\\listing}}
\seeassembly\seeamd\seeobject
}

\providecommand{\amdcasm}{
\toolsection{amd64asm} is an assembler for the AMD64 hardware architecture.
It translates assembly code into machine code for AMD64 processors and stores it in corresponding object files.
By default, the assembler generates machine code for the 64-bit operating mode defined by the AMD64 architecture.
\flowgraph{\resource{AMD64 assembly\\source code} \ar[r] & \toolbox{amd64asm} \ar[r] & \resource{object file}}
\seeassembly\seeamd\seeobject
}

\providecommand{\amdcdism}{
\toolsection{amd64dism} is a disassembler for the AMD64 hardware architecture.
It translates machine code from object files targeting AMD64 processors into assembly code and writes it to the standard output stream.
It assumes that the machine code was generated for the 64-bit operating mode defined by the AMD64 architecture.
\flowgraph{\resource{object file} \ar[r] & \toolbox{amd64dism} \ar[r] & \resource{disassembly\\listing}}
\seeassembly\seeamd\seeobject
}

\providecommand{\armaasm}{
\toolsection{arma32asm} is an assembler for the ARM hardware architecture.
It translates assembly code into machine code for ARM processors executing A32 instructions and stores it in corresponding object files.
\flowgraph{\resource{ARM A32 assembly\\source code} \ar[r] & \toolbox{arma32asm} \ar[r] & \resource{object file}}
\seeassembly\seearm\seeobject
}

\providecommand{\armadism}{
\toolsection{arma32dism} is a disassembler for the ARM hardware architecture.
It translates machine code from object files targeting ARM processors executing A32 instructions into assembly code and writes it to the standard output stream.
\flowgraph{\resource{object file} \ar[r] & \toolbox{arma32dism} \ar[r] & \resource{disassembly\\listing}}
\seeassembly\seearm\seeobject
}

\providecommand{\armbasm}{
\toolsection{arma64asm} is an assembler for the ARM hardware architecture.
It translates assembly code into machine code for ARM processors executing A64 instructions and stores it in corresponding object files.
\flowgraph{\resource{ARM A64 assembly\\source code} \ar[r] & \toolbox{arma64asm} \ar[r] & \resource{object file}}
\seeassembly\seearm\seeobject
}

\providecommand{\armbdism}{
\toolsection{arma64dism} is a disassembler for the ARM hardware architecture.
It translates machine code from object files targeting ARM processors executing A64 instructions into assembly code and writes it to the standard output stream.
\flowgraph{\resource{object file} \ar[r] & \toolbox{arma64dism} \ar[r] & \resource{disassembly\\listing}}
\seeassembly\seearm\seeobject
}

\providecommand{\armcasm}{
\toolsection{armt32asm} is an assembler for the ARM hardware architecture.
It translates assembly code into machine code for ARM processors executing T32 instructions and stores it in corresponding object files.
\flowgraph{\resource{ARM T32 assembly\\source code} \ar[r] & \toolbox{armt32asm} \ar[r] & \resource{object file}}
\seeassembly\seearm\seeobject
}

\providecommand{\armcdism}{
\toolsection{armt32dism} is a disassembler for the ARM hardware architecture.
It translates machine code from object files targeting ARM processors executing T32 instructions into assembly code and writes it to the standard output stream.
\flowgraph{\resource{object file} \ar[r] & \toolbox{armt32dism} \ar[r] & \resource{disassembly\\listing}}
\seeassembly\seearm\seeobject
}

\providecommand{\avrasm}{
\toolsection{avrasm} is an assembler for the AVR hardware architecture.
It translates assembly code into machine code for AVR processors and stores it in corresponding object files.
The identifiers \texttt{RXL}, \texttt{RXH}, \texttt{RYL}, \texttt{RYH}, \texttt{RZL}, and \texttt{RZH} are predefined and name the corresponding registers.
The identifiers \texttt{SPL} and \texttt{SPH} are also predefined and evaluate to the address of the corresponding registers.
\flowgraph{\resource{AVR assembly\\source code} \ar[r] & \toolbox{avrasm} \ar[r] & \resource{object file}}
\seeassembly\seeavr\seeobject
}

\providecommand{\avrdism}{
\toolsection{avrdism} is a disassembler for the AVR hardware architecture.
It translates machine code from object files targeting AVR processors into assembly code and writes it to the standard output stream.
\flowgraph{\resource{object file} \ar[r] & \toolbox{avrdism} \ar[r] & \resource{disassembly\\listing}}
\seeassembly\seeavr\seeobject
}

\providecommand{\avrttasm}{
\toolsection{avr32asm} is an assembler for the AVR32 hardware architecture.
It translates assembly code into machine code for AVR32 processors and stores it in corresponding object files.
\flowgraph{\resource{AVR32 assembly\\source code} \ar[r] & \toolbox{avr32asm} \ar[r] & \resource{object file}}
\seeassembly\seeavrtt\seeobject
}

\providecommand{\avrttdism}{
\toolsection{avr32dism} is a disassembler for the AVR32 hardware architecture.
It translates machine code from object files targeting AVR32 processors into assembly code and writes it to the standard output stream.
\flowgraph{\resource{object file} \ar[r] & \toolbox{avr32dism} \ar[r] & \resource{disassembly\\listing}}
\seeassembly\seeavrtt\seeobject
}

\providecommand{\mabkasm}{
\toolsection{m68kasm} is an assembler for the M68000 hardware architecture.
It translates assembly code into machine code for M68000 processors and stores it in corresponding object files.
\flowgraph{\resource{68000 assembly\\source code} \ar[r] & \toolbox{m68kasm} \ar[r] & \resource{object file}}
\seeassembly\seemabk\seeobject
}

\providecommand{\mabkdism}{
\toolsection{m68kdism} is a disassembler for the M68000 hardware architecture.
It translates machine code from object files targeting M68000 processors into assembly code and writes it to the standard output stream.
\flowgraph{\resource{object file} \ar[r] & \toolbox{m68kdism} \ar[r] & \resource{disassembly\\listing}}
\seeassembly\seemabk\seeobject
}

\providecommand{\miblasm}{
\toolsection{miblasm} is an assembler for the MicroBlaze hardware architecture.
It translates assembly code into machine code for MicroBlaze processors and stores it in corresponding object files.
\flowgraph{\resource{MicroBlaze assembly\\source code} \ar[r] & \toolbox{miblasm} \ar[r] & \resource{object file}}
\seeassembly\seemibl\seeobject
}

\providecommand{\mibldism}{
\toolsection{mibldism} is a disassembler for the MicroBlaze hardware architecture.
It translates machine code from object files targeting MicroBlaze processors into assembly code and writes it to the standard output stream.
\flowgraph{\resource{object file} \ar[r] & \toolbox{mibldism} \ar[r] & \resource{disassembly\\listing}}
\seeassembly\seemibl\seeobject
}

\providecommand{\mipsaasm}{
\toolsection{mips32asm} is an assembler for the MIPS32 hardware architecture.
It translates assembly code into machine code for MIPS32 processors and stores it in corresponding object files.
\flowgraph{\resource{MIPS32 assembly\\source code} \ar[r] & \toolbox{mips32asm} \ar[r] & \resource{object file}}
\seeassembly\seemips\seeobject
}

\providecommand{\mipsadism}{
\toolsection{mips32dism} is a disassembler for the MIPS32 hardware architecture.
It translates machine code from object files targeting MIPS32 processors into assembly code and writes it to the standard output stream.
\flowgraph{\resource{object file} \ar[r] & \toolbox{mips32dism} \ar[r] & \resource{disassembly\\listing}}
\seeassembly\seemips\seeobject
}

\providecommand{\mipsbasm}{
\toolsection{mips64asm} is an assembler for the MIPS64 hardware architecture.
It translates assembly code into machine code for MIPS64 processors and stores it in corresponding object files.
\flowgraph{\resource{MIPS64 assembly\\source code} \ar[r] & \toolbox{mips64asm} \ar[r] & \resource{object file}}
\seeassembly\seemips\seeobject
}

\providecommand{\mipsbdism}{
\toolsection{mips64dism} is a disassembler for the MIPS64 hardware architecture.
It translates machine code from object files targeting MIPS64 processors into assembly code and writes it to the standard output stream.
\flowgraph{\resource{object file} \ar[r] & \toolbox{mips64dism} \ar[r] & \resource{disassembly\\listing}}
\seeassembly\seemips\seeobject
}

\providecommand{\mmixasm}{
\toolsection{mmixasm} is an assembler for the MMIX hardware architecture.
It translates assembly code into machine code for MMIX processors and stores it in corresponding object files.
The names of all special registers are predefined and evaluate to the corresponding number.
\flowgraph{\resource{MMIX assembly\\source code} \ar[r] & \toolbox{mmixasm} \ar[r] & \resource{object file}}
\seeassembly\seemmix\seeobject
}

\providecommand{\mmixdism}{
\toolsection{mmixdism} is a disassembler for the MMIX hardware architecture.
It translates machine code from object files targeting MMIX processors into assembly code and writes it to the standard output stream.
\flowgraph{\resource{object file} \ar[r] & \toolbox{mmixdism} \ar[r] & \resource{disassembly\\listing}}
\seeassembly\seemmix\seeobject
}

\providecommand{\orokasm}{
\toolsection{or1kasm} is an assembler for the OpenRISC 1000 hardware architecture.
It translates assembly code into machine code for OpenRISC 1000 processors and stores it in corresponding object files.
\flowgraph{\resource{OpenRISC 1000 assembly\\source code} \ar[r] & \toolbox{or1kasm} \ar[r] & \resource{object file}}
\seeassembly\seeorok\seeobject
}

\providecommand{\orokdism}{
\toolsection{or1kdism} is a disassembler for the OpenRISC 1000 hardware architecture.
It translates machine code from object files targeting OpenRISC 1000 processors into assembly code and writes it to the standard output stream.
\flowgraph{\resource{object file} \ar[r] & \toolbox{or1kdism} \ar[r] & \resource{disassembly\\listing}}
\seeassembly\seeorok\seeobject
}

\providecommand{\ppcaasm}{
\toolsection{ppc32asm} is an assembler for the PowerPC hardware architecture.
It translates assembly code into machine code for PowerPC processors and stores it in corresponding object files.
By default, the assembler generates machine code for the 32-bit operating mode defined by the PowerPC architecture.
\flowgraph{\resource{PowerPC assembly\\source code} \ar[r] & \toolbox{ppc32asm} \ar[r] & \resource{object file}}
\seeassembly\seeppc\seeobject
}

\providecommand{\ppcadism}{
\toolsection{ppc32dism} is a disassembler for the PowerPC hardware architecture.
It translates machine code from object files targeting PowerPC processors into assembly code and writes it to the standard output stream.
It assumes that the machine code was generated for the 32-bit operating mode defined by the PowerPC architecture.
\flowgraph{\resource{object file} \ar[r] & \toolbox{ppc32dism} \ar[r] & \resource{disassembly\\listing}}
\seeassembly\seeppc\seeobject
}

\providecommand{\ppcbasm}{
\toolsection{ppc64asm} is an assembler for the PowerPC hardware architecture.
It translates assembly code into machine code for PowerPC processors and stores it in corresponding object files.
By default, the assembler generates machine code for the 64-bit operating mode defined by the PowerPC architecture.
\flowgraph{\resource{PowerPC assembly\\source code} \ar[r] & \toolbox{ppc64asm} \ar[r] & \resource{object file}}
\seeassembly\seeppc\seeobject
}

\providecommand{\ppcbdism}{
\toolsection{ppc64dism} is a disassembler for the PowerPC hardware architecture.
It translates machine code from object files targeting PowerPC processors into assembly code and writes it to the standard output stream.
It assumes that the machine code was generated for the 64-bit operating mode defined by the PowerPC architecture.
\flowgraph{\resource{object file} \ar[r] & \toolbox{ppc64dism} \ar[r] & \resource{disassembly\\listing}}
\seeassembly\seeppc\seeobject
}

\providecommand{\riscasm}{
\toolsection{riscasm} is an assembler for the RISC hardware architecture.
It translates assembly code into machine code for RISC processors and stores it in corresponding object files.
The names of all special registers are predefined and evaluate to the corresponding number.
\flowgraph{\resource{RISC assembly\\source code} \ar[r] & \toolbox{riscasm} \ar[r] & \resource{object file}}
\seeassembly\seerisc\seeobject
}

\providecommand{\riscdism}{
\toolsection{riscdism} is a disassembler for the RISC hardware architecture.
It translates machine code from object files targeting RISC processors into assembly code and writes it to the standard output stream.
\flowgraph{\resource{object file} \ar[r] & \toolbox{riscdism} \ar[r] & \resource{disassembly\\listing}}
\seeassembly\seerisc\seeobject
}

\providecommand{\wasmasm}{
\toolsection{wasmasm} is an assembler for the WebAssembly architecture.
It translates assembly code into machine code for WebAssembly targets and stores it in corresponding object files.
The names of all special registers are predefined and evaluate to the corresponding number.
\flowgraph{\resource{WebAssembly assembly\\source code} \ar[r] & \toolbox{wasmasm} \ar[r] & \resource{object file}}
\seeassembly\seewasm\seeobject
}

\providecommand{\wasmdism}{
\toolsection{wasmdism} is a disassembler for the WebAssembly architecture.
It translates machine code from object files targeting WebAssembly targets into assembly code and writes it to the standard output stream.
\flowgraph{\resource{object file} \ar[r] & \toolbox{wasmdism} \ar[r] & \resource{disassembly\\listing}}
\seeassembly\seewasm\seeobject
}

% linker tools

\providecommand{\linklib}{
\toolsection{linklib} is an object file combiner.
It creates a static library file by combining all object files given to it into a single one.
\flowgraph{\resource{object files} \ar[r] & \toolbox{linklib} \ar[r] & \resource{library file}}
\seeobject
}

\providecommand{\linkbin}{
\toolsection{linkbin} is a linker for plain binary files.
It links all object files given to it into a single image and stores it in a binary file that begins with the first linked section.
It also creates a map file that lists the address, type, name and size of all used sections.
The filename extension of the resulting binary file can be specified by putting it into a constant data section called \texttt{\_extension}.
\flowgraph{\resource{object files} \ar[r] & \toolbox{linkbin} \ar[r] \ar[d] & \resource{binary file} \\ & \resource{map file}}
\seeobject
}

\providecommand{\linkmem}{
\toolsection{linkmem} is a linker for plain binary files partitioned into random-access and read-only memory.
It links all object files given to it into two distinct images, one for data sections and one for code and constant data sections, and stores each image in a binary file that begins with the first linked section of the corresponding type.
It also creates a map file that lists the address, type, name and size of all used sections.
\flowgraph{\resource{object files} \ar[r] & \toolbox{linkmem} \ar[r] \ar[d] & \resource{RAM file/\\ROM file} \\ & \resource{map file}}
\seeobject
}

\providecommand{\linkprg}{
\toolsection{linkprg} is a linker for GEMDOS executable files.
It links all object files given to it into a single image and stores the image in an Atari GEMDOS executable file~\cite{gemdosfile}.
It also creates a map file that lists the address relative to the text segment, type, name and size of all used sections.
The filename extension of the resulting executable file can be specified by putting it into a constant data section called \texttt{\_extension}.
The GEMDOS executable file format requires all patch patterns of absolute link patches to consist of four full bitmasks with descending offsets.
\flowgraph{\resource{object files} \ar[r] & \toolbox{linkprg} \ar[r] \ar[d] & \resource{executable file} \\ & \resource{map file}}
\seeobject
}

\providecommand{\linkhex}{
\toolsection{linkhex} is a linker for Intel HEX files.
It links all code sections of the object files given to it into single image and stores the image in an Intel HEX file~\cite{hexfile} that begins with the first linked section.
It also creates a map file that lists the address, type, name and size of all used sections.
\flowgraph{\resource{object files} \ar[r] & \toolbox{linkhex} \ar[r] \ar[d] & \resource{HEX file} \\ & \resource{map file}}
\seeobject
}

\providecommand{\mapsearch}{
\toolsection{mapsearch} is a debugging tool.
It searches map files generated by linker tools for the name of a binary section that encompasses a memory address read from the standard input stream.
If additionally provided with one or more object files, it also stores an excerpt thereof in a separate object file called map search result which only contains the identified binary section for disassembling purposes.
\flowgraph{& \resource{map files/\\object files} \ar[d] \\ \resource{memory\\address} \ar[r] & \toolbox{mapsearch} \ar[r] \ar[d] & \resource{section name/\\relative offset} \\ & \resource{object file\\excerpt}}
\seeobject
}

\renewcommand{\flowgraph}[1]{}
\newcounter{tool}\newcounter{compiler}
\addtocontents{toc}{\protect\setcounter{tocdepth}{0}}
\lehead{\chaptermarkformat Tool Reference}\rohead{\topmark{ -- }\botmark}
\newcommand{\seeprefix}{}\renewcommand{\seedocumentationref}[3]{\seeprefix{}\documentationref{#2}{#3}\renewcommand{\seeprefix}{/}}
\renewcommand{\toolsection}[1]{\normalfont\par\medskip\noindent\textbf{#1}\refstepcounter{tool}\phantomsection\addcontentsline{toc}{section}{#1}\markboth{}{\ifodd\thepage#1\else\botmark\fi}\index[tools]{#1 tool@\tool{#1} tool}\renewcommand{\seeprefix}{\em\alignright\mbox{See Chapter }}}

\startchapter{Tool Reference}{}{tools}{}

This chapter lists all \ref*{tools:all}~tools provided by the \ecs{} in alphabetical order.
All of them share the same command-line user interface as detailed in Chapter~\ref{interface}.
The implementation-defined behavior of programming language-specific tools like pretty printers, semantic checkers, serializers, interpreters, and compilers is described in Chapters~\ref{cpp} to~\ref{oberon}.
For more information about assembler and disassembler tools, see Chapter~\ref{assembly}.
The hardware architectures targeted by all of these tools are described in Chapters~\ref{amd64} to~\ref{wasm} whereas Chapter~\ref{object} details linker tools and their functionality.
For more information about debugging tools and documentation generators, see Chapters~\ref{code} to~\ref{documentation}.

\epigraph{This world is but canvas to our imaginations.}{Henry David Thoreau}

\bigskip
\begin{small}
\begin{multicols}{2}
\amdaasm
\amdadism
\amdbasm
\amdbdism
\amdcasm
\amdcdism
\armaasm
\armadism
\armbasm
\armbdism
\armcasm
\armcdism
\asmprint
\avrasm
\avrdism
\avrttasm
\avrttdism
\cdamda
\cdamdb
\cdamdc
\cdarma
\cdarmb
\cdarmc
\cdarmcfpe
\cdavr
\cdavrtt
\cdcheck
\cdmabk
\cdmibl
\cdmipsa
\cdmipsb
\cdmmix
\cdopt
\cdorok
\cdppca
\cdppcb
\cdrisc
\cdrun
\cdwasm
\cppamda
\cppamdb
\cppamdc
\cpparma
\cpparmb
\cpparmc
\cpparmcfpe
\cppavr
\cppavrtt
\cppcheck
\cppcode
\cppdoc
\cppdump
\cpphtml
\cpplatex
\cppmabk
\cppmibl
\cppmipsa
\cppmipsb
\cppmmix
\cpporok
\cppppca
\cppppcb
\cppprep
\cppprint
\cpprisc
\cpprun
\cppwasm
\dbgdwarf
\doccheck
\dochtml
\doclatex
\docprint
\falamda
\falamdb
\falamdc
\falarma
\falarmb
\falarmc
\falarmcfpe
\falavr
\falavrtt
\falcheck
\falcode
\falcpp
\faldump
\falmabk
\falmibl
\falmipsa
\falmipsb
\falmmix
\falorok
\falppca
\falppcb
\falprint
\falrisc
\falrun
\falwasm
\linkbin
\linkhex
\linklib
\linkmem
\linkprg
\mabkasm
\mabkdism
\mapsearch
\miblasm
\mibldism
\mipsaasm
\mipsadism
\mipsbasm
\mipsbdism
\mmixasm
\mmixdism
\obamda
\obamdb
\obamdc
\obarma
\obarmb
\obarmc
\obarmcfpe
\obavr
\obavrtt
\obcheck
\obcode
\obcpp
\obdoc
\obdump
\obhtml
\oblatex
\obmabk
\obmibl
\obmipsa
\obmipsb
\obmmix
\oborok
\obppca
\obppcb
\obprint
\obrisc
\obrun
\obwasm
\orokasm
\orokdism
\ppcaasm
\ppcadism
\ppcbasm
\ppcbdism
\riscasm
\riscdism
\wasmasm
\wasmdism
\label{tools:all}
\end{multicols}
\end{small}

Table~\ref{tab:tools} lists all \ref*{tools:compilers}~compiler and assembler tools and shows their naming scheme consisting of common programming language prefixes and hardware architecture suffixes.

\newcommand{\compilerref}[2]{\emph{\ref{#1:#2}}}
\newcommand{\compiler}[1]{\tool{#1}\refstepcounter{compiler}}
\newcommand{\languageref}[1]{\emph{Chapter~\ref{#1}}}
\newcommand{\architectureref}[1]{\multicolumn{2}{@{}l}{\emph{See Chapter~\ref{#1}}}}
\newcommand{\backends}[1]{\compiler{cpp#1} & \compiler{fal#1} & \compiler{ob#1}}
\newcommand{\backendrefs}[1]{\compilerref{cpp}{cpp#1} & \compilerref{false}{fal#1} & \compilerref{oberon}{ob#1}}
\newcommand{\compilers}[1]{\backends{#1} & \compiler{#1asm}}
\newcommand{\compilerrefs}[1]{\backendrefs{#1} & \compilerref{assembly}{#1asm}}

\begin{table}
\centering\footnotesize
\begin{tabular}{@{}lrcccc@{}}
\toprule \multicolumn{2}{@{}r}{Programming} & & & & Generic \\ \multicolumn{2}{@{}r}{Language} & \cpp{} & FALSE & Oberon & Assembly \\
\multicolumn{2}{@{}l}{Hardware} & \languageref{cpp} & \languageref{false} & \languageref{oberon} & \languageref{assembly} \\ \multicolumn{2}{@{}l}{Architecture} \\
\midrule AMD64 & 16-bit & \compilers{amd16} \\ & & \compilerrefs{amd16} \\ & 32-bit & \compilers{amd32} \\ & & \compilerrefs{amd32} \\ & 64-bit & \compilers{amd64} \\ \architectureref{amd64} & \compilerrefs{amd64} \\
\midrule ARM & A32 & \compilers{arma32} \\ & & \compilerrefs{arma32} \\ & A64 & \compilers{arma64} \\ & & \compilerrefs{arma64} \\ & T32 & \compilers{armt32} \\ & & \compilerrefs{armt32} \\ & & \backends{armt32fpe} \\ \architectureref{arm} & \backendrefs{armt32fpe} \\
\midrule \multicolumn{2}{@{}l}{AVR} & \compilers{avr} \\ \architectureref{avr} & \compilerrefs{avr} \\
\midrule \multicolumn{2}{@{}l}{AVR32} & \compilers{avr32} \\ \architectureref{avr32} & \compilerrefs{avr32} \\
\midrule \multicolumn{2}{@{}l}{M68000} & \compilers{m68k} \\ \architectureref{m68k} & \compilerrefs{m68k} \\
\midrule \multicolumn{2}{@{}l}{MicroBlaze} & \compilers{mibl} \\ \architectureref{mibl} & \compilerrefs{mibl} \\
\midrule MIPS & 32-bit & \compilers{mips32} \\ & & \compilerrefs{mips32} \\ & 64-bit & \compilers{mips64} \\ \architectureref{mips} & \compilerrefs{mips64} \\
\midrule \multicolumn{2}{@{}l}{MMIX} & \compilers{mmix} \\ \architectureref{mmix} & \compilerrefs{mmix} \\
\midrule \multicolumn{2}{@{}l}{OpenRISC 1000} & \compilers{or1k} \\ \architectureref{or1k} & \compilerrefs{or1k} \\
\midrule PowerPC & 32-bit & \compilers{ppc32} \\ & & \compilerrefs{ppc32} \\ & 64-bit & \compilers{ppc64} \\ \architectureref{ppc} & \compilerrefs{ppc64} \\
\midrule \multicolumn{2}{@{}l}{RISC} & \compilers{risc} \\ \architectureref{risc} & \compilerrefs{risc} \\
\midrule \multicolumn{2}{@{}l}{WebAssembly} & \compilers{wasm} \label{tools:compilers} \\ \architectureref{wasm} & \compilerrefs{wasm} \\
\bottomrule
\end{tabular}
\caption{References to all \ref*{tools:compilers}~compiler and assembler tools}
\label{tab:tools}
\end{table}

\concludechapter
\lehead{\leftmark}\rohead{\rightmark}
\addtocontents{toc}{\protect\setcounter{tocdepth}{1}}


\part{Supported Programming Languages}
% User manual for C++
% Copyright (C) Florian Negele

% This file is part of the Eigen Compiler Suite.

% Permission is granted to copy, distribute and/or modify this document
% under the terms of the GNU Free Documentation License, Version 1.3
% or any later version published by the Free Software Foundation.

% You should have received a copy of the GNU Free Documentation License
% along with the ECS.  If not, see <https://www.gnu.org/licenses/>.

% Generic documentation utilities
% Copyright (C) Florian Negele

% This file is part of the Eigen Compiler Suite.

% Permission is granted to copy, distribute and/or modify this document
% under the terms of the GNU Free Documentation License, Version 1.3
% or any later version published by the Free Software Foundation.

% You should have received a copy of the GNU Free Documentation License
% along with the ECS.  If not, see <https://www.gnu.org/licenses/>.

\providecommand{\cpp}{C\texttt{++}}
\providecommand{\opt}{_\mathit{opt}}
\providecommand{\tool}[1]{\texttt{#1}}
\providecommand{\version}{Version 0.0.40}
\providecommand{\resource}[1]{*++\txt{#1}}
\providecommand{\ecs}{Eigen Compiler Suite}
\providecommand{\changed}[1]{\underline{#1}}
\providecommand{\toolbox}[1]{\converter{#1}}
\providecommand{\file}{}\renewcommand{\file}[1]{\texttt{#1}}
\providecommand{\alignright}{\hfill\linebreak[0]\hspace*{\fill}}
\providecommand{\converter}[1]{*++[F][F*:white][F,:gray]\txt{#1}}
\providecommand{\documentation}{\ifbook chapter\else document\fi}
\providecommand{\Documentation}{\ifbook Chapter\else Document\fi}
\providecommand{\variable}[1]{\resource{\texttt{\small#1}\\variable}}
\providecommand{\documentationref}[2]{\ifbook\ref{#1}\else``\href{#1}{#2}''~\cite{#1}\fi}
\providecommand{\objfile}[1]{\texttt{#1}\index[runtime]{#1 object file@\texttt{#1} object file}}
\providecommand{\libfile}[1]{\texttt{#1}\index[runtime]{#1 library file@\texttt{#1} library file}}
\providecommand{\epigraph}[2]{\ifbook\begin{quote}\flushright\textit{#1}\par--- #2\end{quote}\fi}
\providecommand{\environmentvariable}[1]{\texttt{#1}\index{Environment variables!#1@\texttt{#1}}}
\providecommand{\environment}[1]{\texttt{#1}\index[environment]{#1 environment@\texttt{#1} environment}}
\providecommand{\toolsection}{}\renewcommand{\toolsection}[1]{\subsection{#1}\label{\prefix:#1}\tool{#1}}
\providecommand{\instruction}{}\renewcommand{\instruction}[2]{\noindent\qquad\pdftooltip{\texttt{#1}}{#2}\refstepcounter{instruction}\par}
\providecommand{\flowgraph}{}\renewcommand{\flowgraph}[1]{\par\sffamily\begin{displaymath}\xymatrix@=4ex{#1}\end{displaymath}\normalfont\par}
\providecommand{\instructionset}{}\renewcommand{\instructionset}[4]{\setcounter{instruction}{0}\begin{multicols}{\ifbook#3\else#4\fi}[{\captionof{table}[#2]{#2 (\ref*{#1:instructions}~instructions)}\label{tab:#1set}\vspace{-2ex}}]\footnotesize\raggedcolumns\input{#1.set}\label{#1:instructions}\end{multicols}}

\providecommand{\gpl}{GNU General Public License}
\providecommand{\rse}{ECS Runtime Support Exception}
\providecommand{\fdl}{\href{https://www.gnu.org/licenses/fdl.html}{GNU Free Documentation License}}

\providecommand{\docbegin}{}
\providecommand{\docend}{}
\providecommand{\doclabel}[1]{\hypertarget{#1}}
\providecommand{\doclink}[2]{\hyperlink{#1}{#2}}
\providecommand{\docsection}[3]{\hypertarget{#1}{\subsection{#2}}\label{sec:#1}\index[library]{#2@#3}}
\providecommand{\docsectionstar}[1]{}
\providecommand{\docsubbegin}{\begin{description}}
\providecommand{\docsubend}{\end{description}}
\providecommand{\docsubsection}[3]{\item[\hypertarget{#1}{#2}]\index[library]{#2@#3}}
\providecommand{\docsubsectionstar}[1]{\smallskip}
\providecommand{\docsubsubsection}[3]{\docsubsection{#1}{#2}{#3}}
\providecommand{\docsubsubsectionstar}[1]{}
\providecommand{\docsubsubsubsection}[3]{}
\providecommand{\docsubsubsubsectionstar}[1]{}
\providecommand{\doctable}{}

\providecommand{\debuggingtool}{}\renewcommand{\debuggingtool}{This tool is provided for debugging purposes.
It allows exposing and modifying an internal data structure that is usually not accessible.
}

\providecommand{\interface}{All tools accept command-line arguments which are taken as names of plain text files containing the source code.
If no arguments are provided, the standard input stream is used instead.
Output files are generated in the current working directory and have the same name as the input file being processed whereas the filename extension gets replaced by an appropriate suffix.
\seeinterface
}

\providecommand{\license}{\noindent Copyright \copyright{} Florian Negele\par\medskip\noindent
Permission is granted to copy, distribute and/or modify this document under the terms of the
\fdl{}, Version 1.3 or any later version published by the \href{https://fsf.org/}{Free Software Foundation}.
}

\providecommand{\ecslogosurface}{
\fill[darkgray] (0,0,0) -- (0,0,3) -- (0,3,3) -- (0,3,1) -- (0,4,1) -- (0,4,3) -- (0,5,3) -- (0,5,0) -- (0,2,0) -- (0,2,2) -- (0,1,2) -- (0,1,0) -- cycle;
\fill[gray] (0,5,0) -- (0,5,3) -- (1,5,3) -- (1,5,1) -- (2,5,1) -- (2,5,3) -- (3,5,3) -- (3,5,0) -- cycle;
\fill[lightgray] (0,0,0) -- (0,1,0) -- (2,1,0) -- (2,4,0) -- (1,4,0) -- (1,3,0) -- (2,3,0) -- (2,2,0) -- (0,2,0) -- (0,5,0) -- (3,5,0) -- (3,0,0) -- cycle;
\begin{scope}[line width=0.5]
\begin{scope}[gray]
\draw (0,0,0) -- (0,1,0);
\draw (2,1,0) -- (2,2,0);
\draw (0,1,2) -- (0,2,2);
\draw (0,2,0) -- (0,5,0);
\draw (2,3,0) -- (2,4,0);
\end{scope}
\begin{scope}[lightgray]
\draw (0,1,0) -- (0,1,2);
\draw (0,3,1) -- (0,3,3);
\draw (0,5,0) -- (0,5,3);
\draw (2,5,1) -- (2,5,3);
\end{scope}
\begin{scope}[white]
\draw (0,1,0) -- (2,1,0);
\draw (1,3,0) -- (2,3,0);
\draw (0,5,0) -- (3,5,0);
\end{scope}
\end{scope}
}

\providecommand{\ecslogo}[1]{
\begin{tikzpicture}[scale={(#1)/((sin(45)+cos(45))*3cm)},x={({-cos(45)*1cm},{sin(45)*sin(30)*1cm})},y={({0cm},{(cos(30)*1cm})},z={({sin(45)*1cm},{cos(45)*sin(30)*1cm})}]
\begin{scope}[darkgray,line width=1]
\draw (0,0,0) -- (0,0,3) -- (0,3,3) -- (2,3,3) -- (2,5,3) -- (3,5,3) -- (3,5,0) -- (3,0,0) -- cycle;
\draw (0,3,1) -- (0,4,1) -- (0,4,3) -- (0,5,3) -- (1,5,3) -- (1,5,1) -- (2,5,1);
\draw (1,3,0) -- (1,4,0) -- (2,4,0);
\end{scope}
\fill[darkgray] (2,0,0) -- (2,0,3) -- (2,5,3) -- (2,5,1) -- (2,4,1) -- (2,4,0) -- cycle;
\fill[lightgray] (2,0,2) -- (0,0,2) -- (0,2,2) -- (2,2,2) -- cycle;
\fill[gray] (0,1,0) -- (2,1,0) -- (2,1,2) -- (0,1,2) -- cycle;
\fill[gray] (0,3,1) -- (0,3,3) -- (2,3,3) -- (2,3,0) -- (1,3,0) -- (1,3,1) -- cycle;
\ecslogosurface
\end{tikzpicture}
}

\providecommand{\shadowedecslogo}[3]{
\begin{tikzpicture}[scale={(#1)/((sin(#2)+cos(#2))*3cm)},x={({-cos(#2)*1cm},{sin(#2)*sin(#3)*1cm})},y={({0cm},{(cos(#3)*1cm})},z={({sin(#2)*1cm},{cos(#2)*sin(#3)*1cm})}]
\shade[top color=lightgray!50!white,bottom color=white,middle color=lightgray!50!white] (0,0,0) -- (3,0,0) -- (3,{-0.5-3*sin(#2)*sin(#3)/cos(#3)},0) -- (0,-0.5,0) -- cycle;
\shade[top color=darkgray!50!gray,bottom color=white,middle color=darkgray!50!white] (0,0,0) -- (0,0,3) -- (0,{-0.5-3*cos(#2)*sin(#3)/cos(#3)},3) -- (0,-0.5,0) -- cycle;
\begin{scope}[y={({(cos(#2)+sin(#2))*0.5cm},{(cos(#2)*sin(#3)-sin(#2)*sin(#3))*0.5cm})}]
\useasboundingbox (3,0,0) -- (0,0,0) -- (0,0,3);
\shade[left color=darkgray!80!black,right color=lightgray,middle color=gray] (0,0,0) -- (0,1,0) -- (0,1,0.5) -- (0,2,0) -- (0,5,0) -- (0,5,3) -- (1,5,3) -- (1,4,3) -- (1,4,2.5) -- (1,3,3) -- (2,5,3) -- (3,5,3) -- (3,0,3) -- cycle;
\clip (0,0,0) -- (0,0,3) -- ({-3*sin(#2)/cos(#2)},0,0) -- cycle;
\shade[left color=darkgray,right color=lightgray!50!gray] (0,0,0) -- (0,1,0) -- (0,1,0.5) -- (0,2,0) -- (0,5,0) -- (0,5,3) -- (1,5,3) -- (1,4,3) -- (1,4,2.5) -- (1,3,3) -- (2,5,3) -- (3,5,3) -- (3,0,3) -- cycle;
\end{scope}
\shade[left color=darkgray,right color=darkgray!80!black] (2,0,0) -- (2,0,3) -- (2,5,3) -- (2,5,1) -- (2,4,1) -- (2,4,0) -- cycle;
\shade[left color=darkgray!90!black,right color=gray!80!darkgray] (2,0,2) -- (0,0,2) -- (0,2,2) -- (2,2,2) -- cycle;
\shade[top color=darkgray!90!black,bottom color=gray!80!darkgray] (0,1,0) -- (2,1,0) -- (2,1,2) -- (0,1,2) -- cycle;
\shade[top color=darkgray!90!black,bottom color=gray!80!darkgray] (0,3,1) -- (0,3,3) -- (2,3,3) -- (2,3,0) -- (1,3,0) -- (1,3,1) -- cycle;
\fill[gray] (2,1,0) -- (1.5,1,0.5) -- (0,1,0.5) -- (0,1,0) -- cycle;
\fill[gray] (1,3,2) -- (0.5,3,2) -- (0.5,3,3) -- (1,3,3) -- cycle;
\fill[gray] (2,3,0) -- (1.5,3,0.5) -- (1,3,0.5) -- (1,3,0) -- cycle;
\ecslogosurface
\end{tikzpicture}
}

\providecommand{\cpplogo}[1]{
\begin{tikzpicture}[scale=(#1)/512em]
\fill[gray] (435.2794,398.7159) -- (247.1911,507.3075) .. controls (236.3563,513.5642) and (218.6240,513.5642) .. (207.7892,507.3075) -- (19.7009,398.7159) .. controls (8.8646,392.4606) and (0.0000,377.1043) .. (0.0000,364.5924) -- (0.0000,147.4076) .. controls (0.8430,132.8363) and (8.2856,120.7683) .. (19.7009,113.2842) -- (207.7892,4.6926) .. controls (218.6240,-1.5642) and (236.3564,-1.5642) .. (247.1911,4.6926) -- (435.2794,113.2842) .. controls (447.5273,121.4304) and (454.4987,133.6918) .. (454.9803,147.4076) -- (454.9803,364.5924) .. controls (454.5404,377.7571) and (446.6566,391.0351) .. (435.2794,398.7159) -- cycle(75.8301,255.9993) .. controls (74.9389,404.0881) and (273.2892,469.4783) .. (358.8263,331.8769) -- (293.1917,293.8965) .. controls (253.5702,359.4301) and (155.1909,335.9977) .. (151.6601,255.9993) .. controls (152.7204,182.2703) and (249.4137,148.0211) .. (293.1961,218.1065) -- (358.8308,180.1276) .. controls (283.4477,49.2645) and (79.6318,96.3470) .. (75.8301,255.9993) -- cycle(379.1503,247.5747) -- (362.2982,247.5747) -- (362.2982,230.7226) -- (345.4490,230.7226) -- (345.4490,247.5747) -- (328.5969,247.5747) -- (328.5969,264.4254) -- (345.4490,264.4254) -- (345.4490,281.2759) -- (362.2982,281.2759) -- (362.2982,264.4254) -- (379.1503,264.4254) -- cycle(442.3420,247.5747) -- (425.4899,247.5747) -- (425.4899,230.7226) -- (408.6408,230.7226) -- (408.6408,247.5747) -- (391.7886,247.5747) -- (391.7886,264.4254) -- (408.6408,264.4254) -- (408.6408,281.2759) -- (425.4899,281.2759) -- (425.4899,264.4254) -- (442.3420,264.4254) -- cycle;
\end{tikzpicture}
}

\providecommand{\fallogo}[1]{
\begin{tikzpicture}[scale=(#1)/512em]
\fill[gray] (185.7774,0.0000) .. controls (200.4486,15.9798) and (226.8966,8.7148) .. (235.0426,31.5836) .. controls (249.5297,58.0598) and (247.9581,97.9161) .. (280.3335,110.9762) .. controls (309.1690,120.3496) and (337.8406,104.2727) .. (366.5753,103.9379) .. controls (373.4449,111.5171) and (379.2885,128.2574) .. (383.9755,108.9744) .. controls (396.6979,102.5615) and (437.2808,107.6681) .. (426.9652,124.3252) .. controls (408.9822,121.0785) and (412.4742,146.0729) .. (426.5192,131.4996) .. controls (433.8413,120.8489) and (465.1541,126.5522) .. (441.9067,135.7950) .. controls (396.1879,157.7478) and (344.1112,161.5079) .. (298.5528,183.5702) .. controls (277.7471,193.5198) and (284.6941,218.7163) .. (285.2127,236.9640) .. controls (292.3599,316.2826) and (307.3929,394.6311) .. (317.1198,473.6154) .. controls (329.0637,505.4736) and (292.1195,528.5004) .. (265.9183,511.2761) .. controls (237.9284,499.2462) and (237.3684,465.2681) .. (230.9102,439.9421) .. controls (218.6692,374.3397) and (215.6307,306.9662) .. (198.1732,242.3977) .. controls (183.1379,232.7444) and (164.4245,256.0298) .. (149.0430,261.4799) .. controls (116.9328,279.2585) and (87.1822,308.5851) .. (48.2293,307.8914) .. controls (21.3220,306.9037) and (-15.9107,281.8761) .. (7.2921,252.7908) .. controls (29.7799,220.6177) and (67.5177,204.3028) .. (100.9287,185.9449) .. controls (130.8217,170.8906) and (161.1548,156.5903) .. (191.0278,141.5847) .. controls (196.1738,120.0520) and (186.6049,95.2409) .. (186.8382,72.4353) .. controls (185.5234,48.4204) and (183.1700,23.9341) .. (185.7774,0.0000) -- cycle;
\end{tikzpicture}
}

\providecommand{\oblogo}[1]{
\begin{tikzpicture}[scale=(#1)/512em]
\fill[gray] (160.3865,208.9117) .. controls (154.0879,214.6478) and (149.0735,221.2409) .. (145.4125,228.5384) .. controls (184.8790,248.4273) and (234.7122,269.8787) .. (297.5493,291.8782) .. controls (300.3943,281.4769) and (300.9552,268.7619) .. (300.4023,255.2389) .. controls (248.9909,244.7891) and (200.0310,225.9279) .. (160.3865,208.9117) -- cycle(225.7398,392.6996) .. controls (308.0209,392.1716) and (359.3326,345.9277) .. (368.7203,285.2098) .. controls (376.6742,197.1784) and (311.7194,141.3342) .. (205.4287,142.1456) .. controls (139.9485,141.4804) and (88.7155,166.1957) .. (73.5775,228.0086) .. controls (52.0297,320.3408) and (123.4078,391.0103) .. (225.7398,392.6996) -- cycle(216.0739,176.4733) .. controls (268.9183,179.2424) and (315.8292,206.5488) .. (312.7454,265.1139) .. controls (313.2769,315.6384) and (286.5993,353.4946) .. (216.6040,355.7934) .. controls (162.4657,355.7934) and (126.0914,317.5023) .. (126.0914,260.5103) .. controls (126.1733,214.2900) and (163.3363,176.2849) .. (216.0739,176.4733) -- cycle(76.4897,189.1754) .. controls (13.1586,147.5631) and (0.0000,119.4207) .. (0.0000,119.4207) -- (90.6499,170.1632) .. controls (85.3004,175.8497) and (80.5994,182.1633) .. (76.4897,189.1754) -- cycle(353.9486,119.3004) -- (402.9482,119.3004) .. controls (427.0025,137.0797) and (450.9893,162.7034) .. (474.9529,191.0213) .. controls (509.3540,228.5339) and (531.3391,294.2091) .. (487.8149,312.1206) .. controls (462.8165,324.7652) and (394.3874,316.8943) .. (373.8912,313.6651) .. controls (379.9291,297.7449) and (383.2899,278.4204) .. (381.4989,257.7214) .. controls (420.3069,248.0321) and (421.9610,218.3461) .. (407.7867,192.6417) .. controls (391.1113,162.4018) and (370.1114,132.9097) .. (353.9486,119.3004) -- cycle;
\end{tikzpicture}
}

\providecommand{\markuptable}{
\begin{table}
\sffamily\centering
\begin{tabular}{@{}lcl@{}}
\toprule
\texttt{//italics//} & $\rightarrow$ & \textit{italics} \\
\midrule
\texttt{**bold**} & $\rightarrow$ & \textbf{bold} \\
\midrule
\texttt{\# ordered list} & & 1 ordered list \\
\texttt{\# second item} & $\rightarrow$ & 2 second item \\
\texttt{\#\# sub item} & & \hspace{1em} 1 sub item \\
\midrule
\texttt{* unordered list} & & $\bullet$ unordered list \\
\texttt{* second item} & $\rightarrow$ & $\bullet$ second item \\
\texttt{** sub item} & & \hspace{1em} $\bullet$ sub item \\
\midrule
\texttt{link to [[label]]} & $\rightarrow$ & link to \underline{label} \\
\midrule
\texttt{<{}<label>{}> definition } & $\rightarrow$ & definition \\
\midrule
\texttt{[[url|link name]]} & $\rightarrow$ & \underline{link name} \\
\midrule\addlinespace
\texttt{= large heading} & & {\Large large heading} \smallskip \\
\texttt{== medium heading} & $\rightarrow$ & {\large medium heading} \\
\texttt{=== small heading} & & small heading \\
\midrule
\texttt{no line break} & & no line break for paragraphs \\
\texttt{for paragraphs} & $\rightarrow$ \\
& & use empty line \\
\texttt{use empty line} \\
\midrule
\texttt{force\textbackslash\textbackslash line break} & $\rightarrow$ & force \\
& & line break \\
\midrule
\texttt{horizontal line} & $\rightarrow$ & horizontal line \\
\texttt{----} & & \hrulefill \\
\midrule
\texttt{|=a|=table|=header} & & \underline{a \enspace table \enspace header} \\
\texttt{|a|table|row} & $\rightarrow$ & a \enspace table \enspace row \\
\texttt{|b|table|row} & & b \enspace table \enspace row \\
\midrule
\texttt{\{\{\{} \\
\texttt{unformatted} & $\rightarrow$ & \texttt{unformatted} \\
\texttt{code} & & \texttt{code} \\
\texttt{\}\}\}} \\
\midrule\addlinespace
\texttt{@ new article} & & {\Large 1.\ new article} \smallskip \\
\texttt{@ second article} & $\rightarrow$ & {\Large 2.\ second article} \smallskip \\
\texttt{@@ sub article} & & {\large 2.1.\ sub article} \\
\bottomrule
\end{tabular}
\normalfont\caption{Elements of the generic documentation markup language}
\label{tab:docmarkup}
\end{table}
}

\providecommand{\startchapter}[4]{
\documentclass[11pt,a4paper]{article}
\usepackage{booktabs}
\usepackage[format=hang,labelfont=bf]{caption}
\usepackage{changepage}
\usepackage[T1]{fontenc}
\usepackage[margin=2cm]{geometry}
\usepackage{hyperref}
\usepackage[american]{isodate}
\usepackage{lmodern}
\usepackage{longtable}
\usepackage{mathptmx}
\usepackage{microtype}
\usepackage[toc]{multitoc}
\usepackage{multirow}
\usepackage[all]{nowidow}
\usepackage{pdfcomment}
\usepackage{syntax}
\usepackage{tikz}
\usepackage[all]{xy}
\hypersetup{pdfborder={0 0 0},bookmarksnumbered=true,pdftitle={\ecs{}: #2},pdfauthor={Florian Negele},pdfsubject={\ecs{}},pdfkeywords={#1}}
\setlength{\grammarindent}{8em}\setlength{\grammarparsep}{0.2ex}
\setlength{\columnsep}{2em}
\newcommand{\prefix}{}
\newcounter{instruction}
\bibliographystyle{unsrt}
\renewcommand{\index}[2][]{}
\renewcommand{\arraystretch}{1.05}
\renewcommand{\floatpagefraction}{0.7}
\renewcommand{\syntleft}{\itshape}\renewcommand{\syntright}{}
\title{\vspace{-5ex}\Huge{\ecs{}}\medskip\hrule}
\author{\huge{#2}}
\date{\medskip\version}
\newif\ifbook\bookfalse
\pagestyle{headings}
\frenchspacing
\begin{document}
\maketitle\thispagestyle{empty}\noindent#4\setlength{\columnseprule}{0.4pt}\tableofcontents\setlength{\columnseprule}{0pt}\vfill\pagebreak[3]\null\vfill\bigskip\noindent
\parbox{\textwidth-4em}{\license The contents of this \documentation{} are part of the \href{manual}{\ecs{} User Manual}~\cite{manual} and correspond to Chapter ``\href{manual\##3}{#1}''.\alignright\mbox{\today}}
\parbox{4em}{\flushright\ecslogo{3em}}
\clearpage
}

\providecommand{\concludechapter}{
\vfill\pagebreak[3]\null\vfill
\thispagestyle{myheadings}\markright{REFERENCES}
\noindent\begin{minipage}{\textwidth}\begin{multicols}{2}[\section*{References}]
\renewcommand{\section}[2]{}\small\bibliography{references}
\end{multicols}\end{minipage}\end{document}
}

\providecommand{\startpresentation}[2]{
\documentclass[14pt,aspectratio=43,usepdftitle=false]{beamer}
\usepackage{booktabs}
\usepackage{etex}
\usepackage{multicol}
\usepackage{tikz}
\usepackage[all]{xy}
\bibliographystyle{unsrt}
\setlength{\columnsep}{1em}
\setlength{\leftmargini}{1em}
\setbeamercolor{title}{fg=black}
\setbeamercolor{structure}{fg=darkgray}
\setbeamercolor{bibliography item}{fg=darkgray}
\setbeamerfont{title}{series=\bfseries}
\setbeamerfont{subtitle}{series=\normalfont}
\setbeamerfont*{frametitle}{parent=title}
\setbeamerfont{block title}{series=\bfseries}
\setbeamerfont*{framesubtitle}{parent=subtitle}
\setbeamersize{text margin left=1em,text margin right=1em}
\setbeamertemplate{navigation symbols}{}
\setbeamertemplate{itemize item}[circle]{}
\setbeamertemplate{bibliography item}[triangle]{}
\setbeamertemplate{bibliography entry author}{\usebeamercolor[fg]{bibliography item}}
\setbeamertemplate{frametitle}{\medskip\usebeamerfont{frametitle}\color{gray}\raisebox{-2.5ex}[0ex][0ex]{\rule{0.1em}{4.5ex}}}
\addtobeamertemplate{frametitle}{}{\hspace{0.4em}\usebeamercolor[fg]{title}\insertframetitle\par\vspace{0.2ex}\hspace{0.5em}\usebeamerfont{framesubtitle}\insertframesubtitle}
\hypersetup{pdfborder={0 0 0},bookmarksnumbered=true,bookmarksopen=true,bookmarksopenlevel=0,pdftitle={\ecs{}: #1},pdfauthor={Florian Negele},pdfsubject={\ecs{}},pdfkeywords={#1}}
\renewcommand{\flowgraph}[1]{\resizebox{\textwidth}{!}{$$\xymatrix{##1}$$}}
\title{\ecs{}\medskip\hrule\medskip}
\institute{\shadowedecslogo{5em}{30}{15}}
\date{\version}
\subtitle{#1}
\begin{document}
\begin{frame}[plain]\titlepage\nocite{manual}\end{frame}
\begin{frame}{Contents}{#1}\begin{center}\tableofcontents\end{center}\end{frame}
}

\providecommand{\concludepresentation}{
\begin{frame}{References}\begin{footnotesize}\setlength{\columnseprule}{0.4pt}\begin{multicols}{2}\bibliography{references}\end{multicols}\end{footnotesize}\end{frame}
\end{document}
}

\providecommand{\startbook}[1]{
\documentclass[10pt,paper=17cm:24cm,DIV=13,twoside=semi,headings=normal,numbers=noendperiod,cleardoublepage=plain]{scrbook}
\usepackage{atveryend}
\usepackage{booktabs}
\usepackage{caption}
\usepackage{changepage}
\usepackage[T1]{fontenc}
\usepackage{imakeidx}
\usepackage{hyperref}
\usepackage[american]{isodate}
\usepackage{lmodern}
\usepackage{longtable}
\usepackage{mathptmx}
\usepackage[final]{microtype}
\usepackage{multicol}
\usepackage{multirow}
\usepackage[all]{nowidow}
\usepackage{pdfcomment}
\usepackage{scrlayer-scrpage}
\usepackage{setspace}
\usepackage{syntax}
\usepackage[eventxtindent=4pt,oddtxtexdent=4pt]{thumbs}
\usepackage{tikz}
\usepackage[all]{xy}
\hyphenation{Micro-Blaze Open-Cores Open-RISC Power-PC}
\hypersetup{pdfborder={0 0 0},bookmarksnumbered=true,bookmarksopen=true,bookmarksopenlevel=0,pdftitle={\ecs{}: #1},pdfauthor={Florian Negele},pdfsubject={\ecs{}},pdfkeywords={#1}}
\setlength{\grammarindent}{8em}\setlength{\grammarparsep}{0.7ex}
\setkomafont{captionlabel}{\usekomafont{descriptionlabel}}
\renewcommand{\arraystretch}{1.05}\setstretch{1.1}
\renewcommand{\chapterformat}{\thechapter\autodot\enskip\raisebox{-1ex}[0ex][0ex]{\color{gray}\rule{0.1em}{3.5ex}}\enskip}
\renewcommand{\startchapter}[4]{\hypertarget{##3}{\chapter{##1}}\label{##3}##4\addthumb{##1}{\LARGE\sffamily\bfseries\thechapter}{white}{gray}\renewcommand{\prefix}{##3}}
\renewcommand{\concludechapter}{\clearpage{\stopthumb\cleardoublepage}}
\renewcommand{\syntleft}{\itshape}\renewcommand{\syntright}{}
\renewcommand{\floatpagefraction}{0.7}
\renewcommand{\partheademptypage}{}
\DeclareMicrotypeAlias{lmss}{cmr}
\newcommand{\prefix}{}
\newcounter{instruction}
\bibliographystyle{unsrt}
\newif\ifbook\booktrue
\makeindex[intoc,title=Index]
\makeindex[intoc,name=tools,title=Index of Tools,columns=3]
\makeindex[intoc,name=library,title=Index of Library Names]
\makeindex[intoc,name=runtime,title=Index of Runtime Support]
\makeindex[intoc,name=environment,title=Index of Target Environments]
\indexsetup{toclevel=chapter,headers={\indexname}{\indexname}}
\frenchspacing
\begin{document}
\pagenumbering{alph}
\begin{titlepage}\centering
\huge\sffamily\null\vfill\textbf{\ecs{}}\bigskip\hrule\bigskip#1
\normalsize\normalfont\vfill\vfill\shadowedecslogo{10em}{30}{15}
\large\vfill\vfill\version
\end{titlepage}
\null\vfill
\thispagestyle{empty}
\noindent\today\par\medskip
\license A copy of this license is included in Appendix~\ref{fdl} on page~\pageref{fdl}.
All product names used herein are for identification purposes only and may be trademarks of their respective companies.
\concludechapter
\frontmatter
\setcounter{tocdepth}{1}
\tableofcontents
\setcounter{tocdepth}{2}
\concludechapter
\listoffigures
\concludechapter
\listoftables
\concludechapter
}

\providecommand{\concludebook}{
\backmatter
\addtocontents{toc}{\protect\setcounter{tocdepth}{-1}}
\phantomsection\addcontentsline{toc}{part}{Bibliography}
\bibliography{references}
\concludechapter
\phantomsection\addcontentsline{toc}{part}{Indexes}
\printindex
\concludechapter
\indexprologue{\label{idx:tools}}
\printindex[tools]
\concludechapter
\printindex[library]
\concludechapter
\indexprologue{\label{idx:runtime}}
\printindex[runtime]
\concludechapter
\indexprologue{\label{idx:environment}}
\printindex[environment]
\concludechapter
\pagestyle{empty}\pagenumbering{Alph}\null\clearpage
\null\vfill\centering\ecslogo{4em}\par\medskip\license
\end{document}
}

% chapter references

\providecommand{\seedocumentationref}{}\renewcommand{\seedocumentationref}[3]{#1, see \Documentation{}~\documentationref{#2}{#3}. }
\providecommand{\seeinterface}{}\renewcommand{\seeinterface}{\ifbook See \Documentation{}~\documentationref{interface}{User Interface} for more information about the common user interface of all of these tools. \fi}
\providecommand{\seeguide}{}\renewcommand{\seeguide}{\seedocumentationref{For basic examples of using some of these tools in practice}{guide}{User Guide}}
\providecommand{\seecpp}{}\renewcommand{\seecpp}{\seedocumentationref{For more information about the \cpp{} programming language and its implementation by the \ecs{}}{cpp}{User Manual for \cpp{}}}
\providecommand{\seefalse}{}\renewcommand{\seefalse}{\seedocumentationref{For more information about the FALSE programming language and its implementation by the \ecs{}}{false}{User Manual for FALSE}}
\providecommand{\seeoberon}{}\renewcommand{\seeoberon}{\seedocumentationref{For more information about the Oberon programming language and its implementation by the \ecs{}}{oberon}{User Manual for Oberon}}
\providecommand{\seeassembly}{}\renewcommand{\seeassembly}{\seedocumentationref{For more information about the generic assembly language and how to use it}{assembly}{Generic Assembly Language Specification}}
\providecommand{\seeamd}{}\renewcommand{\seeamd}{\seedocumentationref{For more information about how the \ecs{} supports the AMD64 hardware architecture}{amd64}{AMD64 Hardware Architecture Support}}
\providecommand{\seearm}{}\renewcommand{\seearm}{\seedocumentationref{For more information about how the \ecs{} supports the ARM hardware architecture}{arm}{ARM Hardware Architecture Support}}
\providecommand{\seeavr}{}\renewcommand{\seeavr}{\seedocumentationref{For more information about how the \ecs{} supports the AVR hardware architecture}{avr}{AVR Hardware Architecture Support}}
\providecommand{\seeavrtt}{}\renewcommand{\seeavrtt}{\seedocumentationref{For more information about how the \ecs{} supports the AVR32 hardware architecture}{avr32}{AVR32 Hardware Architecture Support}}
\providecommand{\seemabk}{}\renewcommand{\seemabk}{\seedocumentationref{For more information about how the \ecs{} supports the M68000 hardware architecture}{m68k}{M68000 Hardware Architecture Support}}
\providecommand{\seemibl}{}\renewcommand{\seemibl}{\seedocumentationref{For more information about how the \ecs{} supports the MicroBlaze hardware architecture}{mibl}{MicroBlaze Hardware Architecture Support}}
\providecommand{\seemips}{}\renewcommand{\seemips}{\seedocumentationref{For more information about how the \ecs{} supports the MIPS32 and MIPS64 hardware architectures}{mips}{MIPS Hardware Architecture Support}}
\providecommand{\seemmix}{}\renewcommand{\seemmix}{\seedocumentationref{For more information about how the \ecs{} supports the MMIX hardware architecture}{mmix}{MMIX Hardware Architecture Support}}
\providecommand{\seeorok}{}\renewcommand{\seeorok}{\seedocumentationref{For more information about how the \ecs{} supports the OpenRISC 1000 hardware architecture}{or1k}{OpenRISC 1000 Hardware Architecture Support}}
\providecommand{\seeppc}{}\renewcommand{\seeppc}{\seedocumentationref{For more information about how the \ecs{} supports the PowerPC hardware architecture}{ppc}{PowerPC Hardware Architecture Support}}
\providecommand{\seerisc}{}\renewcommand{\seerisc}{\seedocumentationref{For more information about how the \ecs{} supports the RISC hardware architecture}{risc}{RISC Hardware Architecture Support}}
\providecommand{\seewasm}{}\renewcommand{\seewasm}{\seedocumentationref{For more information about how the \ecs{} supports the WebAssembly architecture}{wasm}{WebAssembly Architecture Support}}
\providecommand{\seedocumentation}{}\renewcommand{\seedocumentation}{\seedocumentationref{For more information about generic documentations and their generation by the \ecs{}}{documentation}{Generic Documentation Generation}}
\providecommand{\seedebugging}{}\renewcommand{\seedebugging}{\seedocumentationref{For more information about debugging information and its representation}{debugging}{Debugging Information Representation}}
\providecommand{\seecode}{}\renewcommand{\seecode}{\seedocumentationref{For more information about intermediate code and its purpose}{code}{Intermediate Code Representation}}
\providecommand{\seeobject}{}\renewcommand{\seeobject}{\seedocumentationref{For more information about object files and their purpose}{object}{Object File Representation}}

% generic documentation tools

\providecommand{\docprint}{
\toolsection{docprint} is a pretty printer for generic documentations.
It reformats generic documentations and writes it to the standard output stream.
\debuggingtool
\flowgraph{\resource{generic\\documentation} \ar[r] & \toolbox{docprint} \ar[r] & \resource{generic\\documentation}}
\seedocumentation
}

\providecommand{\doccheck}{
\toolsection{doccheck} is a syntactic and semantic checker for generic documentations.
It just performs syntactic and semantic checks on generic documentations and writes its diagnostic messages to the standard error stream.
\debuggingtool
\flowgraph{\resource{generic\\documentation} \ar[r] & \toolbox{doccheck} \ar[r] & \resource{diagnostic\\messages}}
\seedocumentation
}

\providecommand{\dochtml}{
\toolsection{dochtml} is an HTML documentation generator for generic documentations.
It processes several generic documentations and assembles all information therein into an HTML document.
\debuggingtool
\flowgraph{\resource{generic\\documentation} \ar[r] & \toolbox{dochtml} \ar[r] & \resource{HTML\\document}}
\seedocumentation
}

\providecommand{\doclatex}{
\toolsection{doclatex} is a Latex documentation generator for generic documentations.
It processes several generic documentations and assembles all information therein into a Latex document.
\debuggingtool
\flowgraph{\resource{generic\\documentation} \ar[r] & \toolbox{doclatex} \ar[r] & \resource{Latex\\document}}
\seedocumentation
}

% intermediate code tools

\providecommand{\cdcheck}{
\toolsection{cdcheck} is a syntactic and semantic checker for intermediate code.
It just performs syntactic and semantic checks on programs written in intermediate code and writes its diagnostic messages to the standard error stream.
\debuggingtool
\flowgraph{\resource{intermediate\\code} \ar[r] & \toolbox{cdcheck} \ar[r] & \resource{diagnostic\\messages}}
\seeassembly\seecode
}

\providecommand{\cdopt}{
\toolsection{cdopt} is an optimizer for intermediate code.
It performs various optimizations on programs written in intermediate code and writes the result to the standard output stream.
\debuggingtool
\flowgraph{\resource{intermediate\\code} \ar[r] & \toolbox{cdopt} \ar[r] & \resource{optimized\\code}}
\seeassembly\seecode
}

\providecommand{\cdrun}{
\toolsection{cdrun} is an interpreter for intermediate code.
It processes and executes programs written in intermediate code.
The following code sections are predefined and have the usual semantics:
\texttt{abort}, \texttt{\_Exit}, \texttt{fflush}, \texttt{floor}, \texttt{fputc}, \texttt{free}, \texttt{getchar}, \texttt{malloc}, and \texttt{putchar}.
Diagnostic messages about invalid operations include the name of the executed code section and the index of the erroneous instruction.
\debuggingtool
\flowgraph{\resource{intermediate\\code} \ar[r] & \toolbox{cdrun} \ar@/u/[r] & \resource{input/\\output} \ar@/d/[l]}
\seeassembly\seecode
}

\providecommand{\cdamda}{
\toolsection{cdamd16} is a compiler for intermediate code targeting the AMD64 hardware architecture.
It generates machine code for AMD64 processors from programs written in intermediate code and stores it in corresponding object files.
The compiler generates machine code for the 16-bit operating mode defined by the AMD64 architecture.
It also creates a debugging information file as well as an assembly file containing a listing of the generated machine code.
\debuggingtool
\flowgraph{\resource{intermediate\\code} \ar[r] & \toolbox{cdamd16} \ar[r] \ar[d] \ar[rd] & \resource{object file} \\ & \resource{assembly\\listing} & \resource{debugging\\information}}
\seeassembly\seeamd\seeobject\seecode\seedebugging
}

\providecommand{\cdamdb}{
\toolsection{cdamd32} is a compiler for intermediate code targeting the AMD64 hardware architecture.
It generates machine code for AMD64 processors from programs written in intermediate code and stores it in corresponding object files.
The compiler generates machine code for the 32-bit operating mode defined by the AMD64 architecture.
It also creates a debugging information file as well as an assembly file containing a listing of the generated machine code.
\debuggingtool
\flowgraph{\resource{intermediate\\code} \ar[r] & \toolbox{cdamd32} \ar[r] \ar[d] \ar[rd] & \resource{object file} \\ & \resource{assembly\\listing} & \resource{debugging\\information}}
\seeassembly\seeamd\seeobject\seecode\seedebugging
}

\providecommand{\cdamdc}{
\toolsection{cdamd64} is a compiler for intermediate code targeting the AMD64 hardware architecture.
It generates machine code for AMD64 processors from programs written in intermediate code and stores it in corresponding object files.
The compiler generates machine code for the 64-bit operating mode defined by the AMD64 architecture.
It also creates a debugging information file as well as an assembly file containing a listing of the generated machine code.
\debuggingtool
\flowgraph{\resource{intermediate\\code} \ar[r] & \toolbox{cdamd64} \ar[r] \ar[d] \ar[rd] & \resource{object file} \\ & \resource{assembly\\listing} & \resource{debugging\\information}}
\seeassembly\seeamd\seeobject\seecode\seedebugging
}

\providecommand{\cdarma}{
\toolsection{cdarma32} is a compiler for intermediate code targeting the ARM hardware architecture.
It generates machine code for ARM processors executing A32 instructions from programs written in intermediate code and stores it in corresponding object files.
It also creates a debugging information file as well as an assembly file containing a listing of the generated machine code.
\debuggingtool
\flowgraph{\resource{intermediate\\code} \ar[r] & \toolbox{cdarma32} \ar[r] \ar[d] \ar[rd] & \resource{object file} \\ & \resource{assembly\\listing} & \resource{debugging\\information}}
\seeassembly\seearm\seeobject\seecode\seedebugging
}

\providecommand{\cdarmb}{
\toolsection{cdarma64} is a compiler for intermediate code targeting the ARM hardware architecture.
It generates machine code for ARM processors executing A64 instructions from programs written in intermediate code and stores it in corresponding object files.
It also creates a debugging information file as well as an assembly file containing a listing of the generated machine code.
\debuggingtool
\flowgraph{\resource{intermediate\\code} \ar[r] & \toolbox{cdarma64} \ar[r] \ar[d] \ar[rd] & \resource{object file} \\ & \resource{assembly\\listing} & \resource{debugging\\information}}
\seeassembly\seearm\seeobject\seecode\seedebugging
}

\providecommand{\cdarmc}{
\toolsection{cdarmt32} is a compiler for intermediate code targeting the ARM hardware architecture.
It generates machine code for ARM processors without floating-point extension executing T32 instructions from programs written in intermediate code and stores it in corresponding object files.
It also creates a debugging information file as well as an assembly file containing a listing of the generated machine code.
\debuggingtool
\flowgraph{\resource{intermediate\\code} \ar[r] & \toolbox{cdarmt32} \ar[r] \ar[d] \ar[rd] & \resource{object file} \\ & \resource{assembly\\listing} & \resource{debugging\\information}}
\seeassembly\seearm\seeobject\seecode\seedebugging
}

\providecommand{\cdarmcfpe}{
\toolsection{cdarmt32fpe} is a compiler for intermediate code targeting the ARM hardware architecture.
It generates machine code for ARM processors with floating-point extension executing T32 instructions from programs written in intermediate code and stores it in corresponding object files.
It also creates a debugging information file as well as an assembly file containing a listing of the generated machine code.
\debuggingtool
\flowgraph{\resource{intermediate\\code} \ar[r] & \toolbox{cdarmt32fpe} \ar[r] \ar[d] \ar[rd] & \resource{object file} \\ & \resource{assembly\\listing} & \resource{debugging\\information}}
\seeassembly\seearm\seeobject\seecode\seedebugging
}

\providecommand{\cdavr}{
\toolsection{cdavr} is a compiler for intermediate code targeting the AVR hardware architecture.
It generates machine code for AVR processors from programs written in intermediate code and stores it in corresponding object files.
It also creates a debugging information file as well as an assembly file containing a listing of the generated machine code.
\debuggingtool
\flowgraph{\resource{intermediate\\code} \ar[r] & \toolbox{cdavr} \ar[r] \ar[d] \ar[rd] & \resource{object file} \\ & \resource{assembly\\listing} & \resource{debugging\\information}}
\seeassembly\seeavr\seeobject\seecode\seedebugging
}

\providecommand{\cdavrtt}{
\toolsection{cdavr32} is a compiler for intermediate code targeting the AVR32 hardware architecture.
It generates machine code for AVR32 processors from programs written in intermediate code and stores it in corresponding object files.
It also creates a debugging information file as well as an assembly file containing a listing of the generated machine code.
\debuggingtool
\flowgraph{\resource{intermediate\\code} \ar[r] & \toolbox{cdavr32} \ar[r] \ar[d] \ar[rd] & \resource{object file} \\ & \resource{assembly\\listing} & \resource{debugging\\information}}
\seeassembly\seeavrtt\seeobject\seecode\seedebugging
}

\providecommand{\cdmabk}{
\toolsection{cdm68k} is a compiler for intermediate code targeting the M68000 hardware architecture.
It generates machine code for M68000 processors from programs written in intermediate code and stores it in corresponding object files.
It also creates a debugging information file as well as an assembly file containing a listing of the generated machine code.
\debuggingtool
\flowgraph{\resource{intermediate\\code} \ar[r] & \toolbox{cdm68k} \ar[r] \ar[d] \ar[rd] & \resource{object file} \\ & \resource{assembly\\listing} & \resource{debugging\\information}}
\seeassembly\seemabk\seeobject\seecode\seedebugging
}

\providecommand{\cdmibl}{
\toolsection{cdmibl} is a compiler for intermediate code targeting the MicroBlaze hardware architecture.
It generates machine code for MicroBlaze processors from programs written in intermediate code and stores it in corresponding object files.
It also creates a debugging information file as well as an assembly file containing a listing of the generated machine code.
\debuggingtool
\flowgraph{\resource{intermediate\\code} \ar[r] & \toolbox{cdmibl} \ar[r] \ar[d] \ar[rd] & \resource{object file} \\ & \resource{assembly\\listing} & \resource{debugging\\information}}
\seeassembly\seemibl\seeobject\seecode\seedebugging
}

\providecommand{\cdmipsa}{
\toolsection{cdmips32} is a compiler for intermediate code targeting the MIPS32 hardware architecture.
It generates machine code for MIPS32 processors from programs written in intermediate code and stores it in corresponding object files.
It also creates a debugging information file as well as an assembly file containing a listing of the generated machine code.
\debuggingtool
\flowgraph{\resource{intermediate\\code} \ar[r] & \toolbox{cdmips32} \ar[r] \ar[d] \ar[rd] & \resource{object file} \\ & \resource{assembly\\listing} & \resource{debugging\\information}}
\seeassembly\seemips\seeobject\seecode\seedebugging
}

\providecommand{\cdmipsb}{
\toolsection{cdmips64} is a compiler for intermediate code targeting the MIPS64 hardware architecture.
It generates machine code for MIPS64 processors from programs written in intermediate code and stores it in corresponding object files.
It also creates a debugging information file as well as an assembly file containing a listing of the generated machine code.
\debuggingtool
\flowgraph{\resource{intermediate\\code} \ar[r] & \toolbox{cdmips64} \ar[r] \ar[d] \ar[rd] & \resource{object file} \\ & \resource{assembly\\listing} & \resource{debugging\\information}}
\seeassembly\seemips\seeobject\seecode\seedebugging
}

\providecommand{\cdmmix}{
\toolsection{cdmmix} is a compiler for intermediate code targeting the MMIX hardware architecture.
It generates machine code for MMIX processors from programs written in intermediate code and stores it in corresponding object files.
It also creates a debugging information file as well as an assembly file containing a listing of the generated machine code.
\debuggingtool
\flowgraph{\resource{intermediate\\code} \ar[r] & \toolbox{cdmmix} \ar[r] \ar[d] \ar[rd] & \resource{object file} \\ & \resource{assembly\\listing} & \resource{debugging\\information}}
\seeassembly\seemmix\seeobject\seecode\seedebugging
}

\providecommand{\cdorok}{
\toolsection{cdor1k} is a compiler for intermediate code targeting the OpenRISC 1000 hardware architecture.
It generates machine code for OpenRISC 1000 processors from programs written in intermediate code and stores it in corresponding object files.
It also creates a debugging information file as well as an assembly file containing a listing of the generated machine code.
\debuggingtool
\flowgraph{\resource{intermediate\\code} \ar[r] & \toolbox{cdor1k} \ar[r] \ar[d] \ar[rd] & \resource{object file} \\ & \resource{assembly\\listing} & \resource{debugging\\information}}
\seeassembly\seeorok\seeobject\seecode\seedebugging
}

\providecommand{\cdppca}{
\toolsection{cdppc32} is a compiler for intermediate code targeting the PowerPC hardware architecture.
It generates machine code for PowerPC processors from programs written in intermediate code and stores it in corresponding object files.
The compiler generates machine code for the 32-bit operating mode defined by the PowerPC architecture.
It also creates a debugging information file as well as an assembly file containing a listing of the generated machine code.
\debuggingtool
\flowgraph{\resource{intermediate\\code} \ar[r] & \toolbox{cdppc32} \ar[r] \ar[d] \ar[rd] & \resource{object file} \\ & \resource{assembly\\listing} & \resource{debugging\\information}}
\seeassembly\seeppc\seeobject\seecode\seedebugging
}

\providecommand{\cdppcb}{
\toolsection{cdppc64} is a compiler for intermediate code targeting the PowerPC hardware architecture.
It generates machine code for PowerPC processors from programs written in intermediate code and stores it in corresponding object files.
The compiler generates machine code for the 64-bit operating mode defined by the PowerPC architecture.
It also creates a debugging information file as well as an assembly file containing a listing of the generated machine code.
\debuggingtool
\flowgraph{\resource{intermediate\\code} \ar[r] & \toolbox{cdppc64} \ar[r] \ar[d] \ar[rd] & \resource{object file} \\ & \resource{assembly\\listing} & \resource{debugging\\information}}
\seeassembly\seeppc\seeobject\seecode\seedebugging
}

\providecommand{\cdrisc}{
\toolsection{cdrisc} is a compiler for intermediate code targeting the RISC hardware architecture.
It generates machine code for RISC processors from programs written in intermediate code and stores it in corresponding object files.
It also creates a debugging information file as well as an assembly file containing a listing of the generated machine code.
\debuggingtool
\flowgraph{\resource{intermediate\\code} \ar[r] & \toolbox{cdrisc} \ar[r] \ar[d] \ar[rd] & \resource{object file} \\ & \resource{assembly\\listing} & \resource{debugging\\information}}
\seeassembly\seerisc\seeobject\seecode\seedebugging
}

\providecommand{\cdwasm}{
\toolsection{cdwasm} is a compiler for intermediate code targeting the WebAssembly architecture.
It generates machine code for WebAssembly targets from programs written in intermediate code and stores it in corresponding object files.
It also creates a debugging information file as well as an assembly file containing a listing of the generated machine code.
\debuggingtool
\flowgraph{\resource{intermediate\\code} \ar[r] & \toolbox{cdwasm} \ar[r] \ar[d] \ar[rd] & \resource{object file} \\ & \resource{assembly\\listing} & \resource{debugging\\information}}
\seeassembly\seewasm\seeobject\seecode\seedebugging
}

% C++ tools

\providecommand{\cppprep}{
\toolsection{cppprep} is a preprocessor for the \cpp{} programming language.
It preprocesses source code according to the rules of \cpp{} and writes it to the standard output stream.
Only the macro names \texttt{\_\_DATE\_\_}, \texttt{\_\_FILE\_\_}, \texttt{\_\_LINE\_\_}, and \texttt{\_\_TIME\_\_} are predefined.
\flowgraph{\resource{\cpp{} or other\\source code} \ar[r] & \toolbox{cppprep} \ar[r] & \resource{preprocessed\\source code} \\ & \variable{ECSINCLUDE} \ar[u]}
\seecpp
}

\providecommand{\cppprint}{
\toolsection{cppprint} is a pretty printer for the \cpp{} programming language.
It reformats the source code of \cpp{} programs and writes it to the standard output stream.
\flowgraph{\resource{\cpp{}\\source code} \ar[r] & \toolbox{cppprint} \ar[r] & \resource{reformatted\\source code} \\ & \variable{ECSINCLUDE} \ar[u]}
\seecpp
}

\providecommand{\cppcheck}{
\toolsection{cppcheck} is a syntactic and semantic checker for the \cpp{} programming language.
It just performs syntactic and semantic checks on \cpp{} programs and writes its diagnostic messages to the standard error stream.
\flowgraph{\resource{\cpp{}\\source code} \ar[r] & \toolbox{cppcheck} \ar[r] & \resource{diagnostic\\messages} \\ & \variable{ECSINCLUDE} \ar[u]}
\seecpp
}

\providecommand{\cppdump}{
\toolsection{cppdump} is a serializer for the \cpp{} programming language.
It dumps the complete internal representation of programs written in \cpp{} into an XML document.
\debuggingtool
\flowgraph{\resource{\cpp{}\\source code} \ar[r] & \toolbox{cppdump} \ar[r] & \resource{internal\\representation} \\ & \variable{ECSINCLUDE} \ar[u]}
\seecpp
}

\providecommand{\cpprun}{
\toolsection{cpprun} is an interpreter for the \cpp{} programming language.
It processes and executes programs written in \cpp{}.
The macro \texttt{\_\_run\_\_} is predefined in order to enable programmers to identify this tool while interpreting.
\flowgraph{\resource{\cpp{}\\source code} \ar[r] & \toolbox{cpprun} \ar@/u/[r] & \resource{input/\\output} \ar@/d/[l] \\ & \variable{ECSINCLUDE} \ar[u]}
\seecpp
}

\providecommand{\cppdoc}{
\toolsection{cppdoc} is a generic documentation generator for the \cpp{} programming language.
It processes several \cpp{} source files and assembles all information therein into a generic documentation.
\debuggingtool
\flowgraph{\resource{\cpp{}\\source code} \ar[r] & \toolbox{cppdoc} \ar[r] & \resource{generic\\documentation} \\ & \variable{ECSINCLUDE} \ar[u]}
\seecpp\seedocumentation
}

\providecommand{\cpphtml}{
\toolsection{cpphtml} is an HTML documentation generator for the \cpp{} programming language.
It processes several \cpp{} source files and assembles all information therein into an HTML document.
\flowgraph{\resource{\cpp{}\\source code} \ar[r] & \toolbox{cpphtml} \ar[r] & \resource{HTML\\document} \\ & \variable{ECSINCLUDE} \ar[u]}
\seecpp\seedocumentation
}

\providecommand{\cpplatex}{
\toolsection{cpplatex} is a Latex documentation generator for the \cpp{} programming language.
It processes several \cpp{} source files and assembles all information therein into a Latex document.
\flowgraph{\resource{\cpp{}\\source code} \ar[r] & \toolbox{cpplatex} \ar[r] & \resource{Latex\\document} \\ & \variable{ECSINCLUDE} \ar[u]}
\seecpp\seedocumentation
}

\providecommand{\cppcode}{
\toolsection{cppcode} is an intermediate code generator for the \cpp{} programming language.
It generates intermediate code from programs written in \cpp{} and stores it in corresponding assembly files.
The macro \texttt{\_\_code\_\_} is predefined in order to enable programmers to identify this tool while generating intermediate code.
Programs generated with this tool require additional runtime support that is stored in the \file{cpp\-code\-run} library file.
\debuggingtool
\flowgraph{\resource{\cpp{}\\source code} \ar[r] & \toolbox{cppcode} \ar[r] & \resource{intermediate\\code} \\ & \variable{ECSINCLUDE} \ar[u]}
\seecpp\seeassembly\seecode
}

\providecommand{\cppamda}{
\toolsection{cppamd16} is a compiler for the \cpp{} programming language targeting the AMD64 hardware architecture.
It generates machine code for AMD64 processors from programs written in \cpp{} and stores it in corresponding object files.
The compiler generates machine code for the 16-bit operating mode defined by the AMD64 architecture.
For debugging purposes, it also creates a debugging information file as well as an assembly file containing a listing of the generated machine code.
The macro \texttt{\_\_amd16\_\_} is predefined in order to enable programmers to identify this tool and its target architecture while compiling.
Programs generated with this compiler require additional runtime support that is stored in the \file{cpp\-amd16\-run} library file.
\flowgraph{\resource{\cpp{}\\source code} \ar[r] & \toolbox{cppamd16} \ar[r] \ar[d] \ar[rd] & \resource{object file} \\ \variable{ECSINCLUDE} \ar[ru] & \resource{debugging\\information} & \resource{assembly\\listing}}
\seecpp\seeassembly\seeamd\seeobject\seedebugging
}

\providecommand{\cppamdb}{
\toolsection{cppamd32} is a compiler for the \cpp{} programming language targeting the AMD64 hardware architecture.
It generates machine code for AMD64 processors from programs written in \cpp{} and stores it in corresponding object files.
The compiler generates machine code for the 32-bit operating mode defined by the AMD64 architecture.
For debugging purposes, it also creates a debugging information file as well as an assembly file containing a listing of the generated machine code.
The macro \texttt{\_\_amd32\_\_} is predefined in order to enable programmers to identify this tool and its target architecture while compiling.
Programs generated with this compiler require additional runtime support that is stored in the \file{cpp\-amd32\-run} library file.
\flowgraph{\resource{\cpp{}\\source code} \ar[r] & \toolbox{cppamd32} \ar[r] \ar[d] \ar[rd] & \resource{object file} \\ \variable{ECSINCLUDE} \ar[ru] & \resource{debugging\\information} & \resource{assembly\\listing}}
\seecpp\seeassembly\seeamd\seeobject\seedebugging
}

\providecommand{\cppamdc}{
\toolsection{cppamd64} is a compiler for the \cpp{} programming language targeting the AMD64 hardware architecture.
It generates machine code for AMD64 processors from programs written in \cpp{} and stores it in corresponding object files.
The compiler generates machine code for the 64-bit operating mode defined by the AMD64 architecture.
For debugging purposes, it also creates a debugging information file as well as an assembly file containing a listing of the generated machine code.
The macro \texttt{\_\_amd64\_\_} is predefined in order to enable programmers to identify this tool and its target architecture while compiling.
Programs generated with this compiler require additional runtime support that is stored in the \file{cpp\-amd64\-run} library file.
\flowgraph{\resource{\cpp{}\\source code} \ar[r] & \toolbox{cppamd64} \ar[r] \ar[d] \ar[rd] & \resource{object file} \\ \variable{ECSINCLUDE} \ar[ru] & \resource{debugging\\information} & \resource{assembly\\listing}}
\seecpp\seeassembly\seeamd\seeobject\seedebugging
}

\providecommand{\cpparma}{
\toolsection{cpparma32} is a compiler for the \cpp{} programming language targeting the ARM hardware architecture.
It generates machine code for ARM processors executing A32 instructions from programs written in \cpp{} and stores it in corresponding object files.
For debugging purposes, it also creates a debugging information file as well as an assembly file containing a listing of the generated machine code.
The macro \texttt{\_\_arma32\_\_} is predefined in order to enable programmers to identify this tool and its target architecture while compiling.
Programs generated with this compiler require additional runtime support that is stored in the \file{cpp\-arma32\-run} library file.
\flowgraph{\resource{\cpp{}\\source code} \ar[r] & \toolbox{cpparma32} \ar[r] \ar[d] \ar[rd] & \resource{object file} \\ \variable{ECSINCLUDE} \ar[ru] & \resource{debugging\\information} & \resource{assembly\\listing}}
\seecpp\seeassembly\seearm\seeobject\seedebugging
}

\providecommand{\cpparmb}{
\toolsection{cpparma64} is a compiler for the \cpp{} programming language targeting the ARM hardware architecture.
It generates machine code for ARM processors executing A64 instructions from programs written in \cpp{} and stores it in corresponding object files.
For debugging purposes, it also creates a debugging information file as well as an assembly file containing a listing of the generated machine code.
The macro \texttt{\_\_arma64\_\_} is predefined in order to enable programmers to identify this tool and its target architecture while compiling.
Programs generated with this compiler require additional runtime support that is stored in the \file{cpp\-arma64\-run} library file.
\flowgraph{\resource{\cpp{}\\source code} \ar[r] & \toolbox{cpparma64} \ar[r] \ar[d] \ar[rd] & \resource{object file} \\ \variable{ECSINCLUDE} \ar[ru] & \resource{debugging\\information} & \resource{assembly\\listing}}
\seecpp\seeassembly\seearm\seeobject\seedebugging
}

\providecommand{\cpparmc}{
\toolsection{cpparmt32} is a compiler for the \cpp{} programming language targeting the ARM hardware architecture.
It generates machine code for ARM processors without floating-point extension executing T32 instructions from programs written in \cpp{} and stores it in corresponding object files.
For debugging purposes, it also creates a debugging information file as well as an assembly file containing a listing of the generated machine code.
The macro \texttt{\_\_armt32\_\_} is predefined in order to enable programmers to identify this tool and its target architecture while compiling.
Programs generated with this compiler require additional runtime support that is stored in the \file{cpp\-armt32\-run} library file.
\flowgraph{\resource{\cpp{}\\source code} \ar[r] & \toolbox{cpparmt32} \ar[r] \ar[d] \ar[rd] & \resource{object file} \\ \variable{ECSINCLUDE} \ar[ru] & \resource{debugging\\information} & \resource{assembly\\listing}}
\seecpp\seeassembly\seearm\seeobject\seedebugging
}

\providecommand{\cpparmcfpe}{
\toolsection{cpparmt32fpe} is a compiler for the \cpp{} programming language targeting the ARM hardware architecture.
It generates machine code for ARM processors with floating-point extension executing T32 instructions from programs written in \cpp{} and stores it in corresponding object files.
For debugging purposes, it also creates a debugging information file as well as an assembly file containing a listing of the generated machine code.
The macro \texttt{\_\_armt32fpe\_\_} is predefined in order to enable programmers to identify this tool and its target architecture while compiling.
Programs generated with this compiler require additional runtime support that is stored in the \file{cpp\-armt32\-fpe\-run} library file.
\flowgraph{\resource{\cpp{}\\source code} \ar[r] & \toolbox{cpparmt32fpe} \ar[r] \ar[d] \ar[rd] & \resource{object file} \\ \variable{ECSINCLUDE} \ar[ru] & \resource{debugging\\information} & \resource{assembly\\listing}}
\seecpp\seeassembly\seearm\seeobject\seedebugging
}

\providecommand{\cppavr}{
\toolsection{cppavr} is a compiler for the \cpp{} programming language targeting the AVR hardware architecture.
It generates machine code for AVR processors from programs written in \cpp{} and stores it in corresponding object files.
For debugging purposes, it also creates a debugging information file as well as an assembly file containing a listing of the generated machine code.
The macro \texttt{\_\_avr\_\_} is predefined in order to enable programmers to identify this tool and its target architecture while compiling.
Programs generated with this compiler require additional runtime support that is stored in the \file{cpp\-avr\-run} library file.
\flowgraph{\resource{\cpp{}\\source code} \ar[r] & \toolbox{cppavr} \ar[r] \ar[d] \ar[rd] & \resource{object file} \\ \variable{ECSINCLUDE} \ar[ru] & \resource{debugging\\information} & \resource{assembly\\listing}}
\seecpp\seeassembly\seeavr\seeobject\seedebugging
}

\providecommand{\cppavrtt}{
\toolsection{cppavr32} is a compiler for the \cpp{} programming language targeting the AVR32 hardware architecture.
It generates machine code for AVR32 processors from programs written in \cpp{} and stores it in corresponding object files.
For debugging purposes, it also creates a debugging information file as well as an assembly file containing a listing of the generated machine code.
The macro \texttt{\_\_avr32\_\_} is predefined in order to enable programmers to identify this tool and its target architecture while compiling.
Programs generated with this compiler require additional runtime support that is stored in the \file{cpp\-avr32\-run} library file.
\flowgraph{\resource{\cpp{}\\source code} \ar[r] & \toolbox{cppavr32} \ar[r] \ar[d] \ar[rd] & \resource{object file} \\ \variable{ECSINCLUDE} \ar[ru] & \resource{debugging\\information} & \resource{assembly\\listing}}
\seecpp\seeassembly\seeavrtt\seeobject\seedebugging
}

\providecommand{\cppmabk}{
\toolsection{cppm68k} is a compiler for the \cpp{} programming language targeting the M68000 hardware architecture.
It generates machine code for M68000 processors from programs written in \cpp{} and stores it in corresponding object files.
For debugging purposes, it also creates a debugging information file as well as an assembly file containing a listing of the generated machine code.
The macro \texttt{\_\_m68k\_\_} is predefined in order to enable programmers to identify this tool and its target architecture while compiling.
Programs generated with this compiler require additional runtime support that is stored in the \file{cpp\-m68k\-run} library file.
\flowgraph{\resource{\cpp{}\\source code} \ar[r] & \toolbox{cppm68k} \ar[r] \ar[d] \ar[rd] & \resource{object file} \\ \variable{ECSINCLUDE} \ar[ru] & \resource{debugging\\information} & \resource{assembly\\listing}}
\seecpp\seeassembly\seemabk\seeobject\seedebugging
}

\providecommand{\cppmibl}{
\toolsection{cppmibl} is a compiler for the \cpp{} programming language targeting the MicroBlaze hardware architecture.
It generates machine code for MicroBlaze processors from programs written in \cpp{} and stores it in corresponding object files.
For debugging purposes, it also creates a debugging information file as well as an assembly file containing a listing of the generated machine code.
The macro \texttt{\_\_mibl\_\_} is predefined in order to enable programmers to identify this tool and its target architecture while compiling.
Programs generated with this compiler require additional runtime support that is stored in the \file{cpp\-mibl\-run} library file.
\flowgraph{\resource{\cpp{}\\source code} \ar[r] & \toolbox{cppmibl} \ar[r] \ar[d] \ar[rd] & \resource{object file} \\ \variable{ECSINCLUDE} \ar[ru] & \resource{debugging\\information} & \resource{assembly\\listing}}
\seecpp\seeassembly\seemibl\seeobject\seedebugging
}

\providecommand{\cppmipsa}{
\toolsection{cppmips32} is a compiler for the \cpp{} programming language targeting the MIPS32 hardware architecture.
It generates machine code for MIPS32 processors from programs written in \cpp{} and stores it in corresponding object files.
For debugging purposes, it also creates a debugging information file as well as an assembly file containing a listing of the generated machine code.
The macro \texttt{\_\_mips32\_\_} is predefined in order to enable programmers to identify this tool and its target architecture while compiling.
Programs generated with this compiler require additional runtime support that is stored in the \file{cpp\-mips32\-run} library file.
\flowgraph{\resource{\cpp{}\\source code} \ar[r] & \toolbox{cppmips32} \ar[r] \ar[d] \ar[rd] & \resource{object file} \\ \variable{ECSINCLUDE} \ar[ru] & \resource{debugging\\information} & \resource{assembly\\listing}}
\seecpp\seeassembly\seemips\seeobject\seedebugging
}

\providecommand{\cppmipsb}{
\toolsection{cppmips64} is a compiler for the \cpp{} programming language targeting the MIPS64 hardware architecture.
It generates machine code for MIPS64 processors from programs written in \cpp{} and stores it in corresponding object files.
For debugging purposes, it also creates a debugging information file as well as an assembly file containing a listing of the generated machine code.
The macro \texttt{\_\_mips64\_\_} is predefined in order to enable programmers to identify this tool and its target architecture while compiling.
Programs generated with this compiler require additional runtime support that is stored in the \file{cpp\-mips64\-run} library file.
\flowgraph{\resource{\cpp{}\\source code} \ar[r] & \toolbox{cppmips64} \ar[r] \ar[d] \ar[rd] & \resource{object file} \\ \variable{ECSINCLUDE} \ar[ru] & \resource{debugging\\information} & \resource{assembly\\listing}}
\seecpp\seeassembly\seemips\seeobject\seedebugging
}

\providecommand{\cppmmix}{
\toolsection{cppmmix} is a compiler for the \cpp{} programming language targeting the MMIX hardware architecture.
It generates machine code for MMIX processors from programs written in \cpp{} and stores it in corresponding object files.
For debugging purposes, it also creates a debugging information file as well as an assembly file containing a listing of the generated machine code.
The macro \texttt{\_\_mmix\_\_} is predefined in order to enable programmers to identify this tool and its target architecture while compiling.
Programs generated with this compiler require additional runtime support that is stored in the \file{cpp\-mmix\-run} library file.
\flowgraph{\resource{\cpp{}\\source code} \ar[r] & \toolbox{cppmmix} \ar[r] \ar[d] \ar[rd] & \resource{object file} \\ \variable{ECSINCLUDE} \ar[ru] & \resource{debugging\\information} & \resource{assembly\\listing}}
\seecpp\seeassembly\seemmix\seeobject\seedebugging
}

\providecommand{\cpporok}{
\toolsection{cppor1k} is a compiler for the \cpp{} programming language targeting the OpenRISC 1000 hardware architecture.
It generates machine code for OpenRISC 1000 processors from programs written in \cpp{} and stores it in corresponding object files.
For debugging purposes, it also creates a debugging information file as well as an assembly file containing a listing of the generated machine code.
The macro \texttt{\_\_or1k\_\_} is predefined in order to enable programmers to identify this tool and its target architecture while compiling.
Programs generated with this compiler require additional runtime support that is stored in the \file{cpp\-or1k\-run} library file.
\flowgraph{\resource{\cpp{}\\source code} \ar[r] & \toolbox{cppor1k} \ar[r] \ar[d] \ar[rd] & \resource{object file} \\ \variable{ECSINCLUDE} \ar[ru] & \resource{debugging\\information} & \resource{assembly\\listing}}
\seecpp\seeassembly\seeorok\seeobject\seedebugging
}

\providecommand{\cppppca}{
\toolsection{cppppc32} is a compiler for the \cpp{} programming language targeting the PowerPC hardware architecture.
It generates machine code for PowerPC processors from programs written in \cpp{} and stores it in corresponding object files.
The compiler generates machine code for the 32-bit operating mode defined by the PowerPC architecture.
For debugging purposes, it also creates a debugging information file as well as an assembly file containing a listing of the generated machine code.
The macro \texttt{\_\_ppc32\_\_} is predefined in order to enable programmers to identify this tool and its target architecture while compiling.
Programs generated with this compiler require additional runtime support that is stored in the \file{cpp\-ppc32\-run} library file.
\flowgraph{\resource{\cpp{}\\source code} \ar[r] & \toolbox{cppppc32} \ar[r] \ar[d] \ar[rd] & \resource{object file} \\ \variable{ECSINCLUDE} \ar[ru] & \resource{debugging\\information} & \resource{assembly\\listing}}
\seecpp\seeassembly\seeppc\seeobject\seedebugging
}

\providecommand{\cppppcb}{
\toolsection{cppppc64} is a compiler for the \cpp{} programming language targeting the PowerPC hardware architecture.
It generates machine code for PowerPC processors from programs written in \cpp{} and stores it in corresponding object files.
The compiler generates machine code for the 64-bit operating mode defined by the PowerPC architecture.
For debugging purposes, it also creates a debugging information file as well as an assembly file containing a listing of the generated machine code.
The macro \texttt{\_\_ppc64\_\_} is predefined in order to enable programmers to identify this tool and its target architecture while compiling.
Programs generated with this compiler require additional runtime support that is stored in the \file{cpp\-ppc64\-run} library file.
\flowgraph{\resource{\cpp{}\\source code} \ar[r] & \toolbox{cppppc64} \ar[r] \ar[d] \ar[rd] & \resource{object file} \\ \variable{ECSINCLUDE} \ar[ru] & \resource{debugging\\information} & \resource{assembly\\listing}}
\seecpp\seeassembly\seeppc\seeobject\seedebugging
}

\providecommand{\cpprisc}{
\toolsection{cpprisc} is a compiler for the \cpp{} programming language targeting the RISC hardware architecture.
It generates machine code for RISC processors from programs written in \cpp{} and stores it in corresponding object files.
For debugging purposes, it also creates a debugging information file as well as an assembly file containing a listing of the generated machine code.
The macro \texttt{\_\_risc\_\_} is predefined in order to enable programmers to identify this tool and its target architecture while compiling.
Programs generated with this compiler require additional runtime support that is stored in the \file{cpp\-risc\-run} library file.
\flowgraph{\resource{\cpp{}\\source code} \ar[r] & \toolbox{cpprisc} \ar[r] \ar[d] \ar[rd] & \resource{object file} \\ \variable{ECSINCLUDE} \ar[ru] & \resource{debugging\\information} & \resource{assembly\\listing}}
\seecpp\seeassembly\seerisc\seeobject\seedebugging
}

\providecommand{\cppwasm}{
\toolsection{cppwasm} is a compiler for the \cpp{} programming language targeting the WebAssembly architecture.
It generates machine code for WebAssembly targets from programs written in \cpp{} and stores it in corresponding object files.
For debugging purposes, it also creates a debugging information file as well as an assembly file containing a listing of the generated machine code.
The macro \texttt{\_\_wasm\_\_} is predefined in order to enable programmers to identify this tool and its target architecture while compiling.
Programs generated with this compiler require additional runtime support that is stored in the \file{cpp\-wasm\-run} library file.
\flowgraph{\resource{\cpp{}\\source code} \ar[r] & \toolbox{cppwasm} \ar[r] \ar[d] \ar[rd] & \resource{object file} \\ \variable{ECSINCLUDE} \ar[ru] & \resource{debugging\\information} & \resource{assembly\\listing}}
\seecpp\seeassembly\seewasm\seeobject\seedebugging
}

% FALSE tools

\providecommand{\falprint}{
\toolsection{falprint} is a pretty printer for the FALSE programming language.
It reformats the source code of FALSE programs and writes it to the standard output stream.
\flowgraph{\resource{FALSE\\source code} \ar[r] & \toolbox{falprint} \ar[r] & \resource{reformatted\\source code}}
\seefalse
}

\providecommand{\falcheck}{
\toolsection{falcheck} is a syntactic and semantic checker for the FALSE programming language.
It just performs syntactic and semantic checks on FALSE programs and writes its diagnostic messages to the standard error stream.
\flowgraph{\resource{FALSE\\source code} \ar[r] & \toolbox{falcheck} \ar[r] & \resource{diagnostic\\messages}}
\seefalse
}

\providecommand{\faldump}{
\toolsection{faldump} is a serializer for the FALSE programming language.
It dumps the complete internal representation of programs written in FALSE into an XML document.
\debuggingtool
\flowgraph{\resource{FALSE\\source code} \ar[r] & \toolbox{faldump} \ar[r] & \resource{internal\\representation}}
\seefalse
}

\providecommand{\falrun}{
\toolsection{falrun} is an interpreter for the FALSE programming language.
It processes and executes programs written in FALSE\@.
\flowgraph{\resource{FALSE\\source code} \ar[r] & \toolbox{falrun} \ar@/u/[r] & \resource{input/\\output} \ar@/d/[l]}
\seefalse
}

\providecommand{\falcpp}{
\toolsection{falcpp} is a transpiler for the FALSE programming language.
It translates programs written in FALSE into \cpp{} programs and stores them in corresponding source files.
\flowgraph{\resource{FALSE\\source code} \ar[r] & \toolbox{falcpp} \ar[r] & \resource{\cpp{}\\source file}}
\seefalse\seecpp
}

\providecommand{\falcode}{
\toolsection{falcode} is an intermediate code generator for the FALSE programming language.
It generates intermediate code from programs written in FALSE and stores it in corresponding assembly files.
\debuggingtool
\flowgraph{\resource{FALSE\\source code} \ar[r] & \toolbox{falcode} \ar[r] & \resource{intermediate\\code}}
\seefalse\seeassembly\seecode
}

\providecommand{\falamda}{
\toolsection{falamd16} is a compiler for the FALSE programming language targeting the AMD64 hardware architecture.
It generates machine code for AMD64 processors from programs written in FALSE and stores it in corresponding object files.
The compiler generates machine code for the 16-bit operating mode defined by the AMD64 architecture.
\flowgraph{\resource{FALSE\\source code} \ar[r] & \toolbox{falamd16} \ar[r] & \resource{object file}}
\seefalse\seeamd\seeobject
}

\providecommand{\falamdb}{
\toolsection{falamd32} is a compiler for the FALSE programming language targeting the AMD64 hardware architecture.
It generates machine code for AMD64 processors from programs written in FALSE and stores it in corresponding object files.
The compiler generates machine code for the 32-bit operating mode defined by the AMD64 architecture.
\flowgraph{\resource{FALSE\\source code} \ar[r] & \toolbox{falamd32} \ar[r] & \resource{object file}}
\seefalse\seeamd\seeobject
}

\providecommand{\falamdc}{
\toolsection{falamd64} is a compiler for the FALSE programming language targeting the AMD64 hardware architecture.
It generates machine code for AMD64 processors from programs written in FALSE and stores it in corresponding object files.
The compiler generates machine code for the 64-bit operating mode defined by the AMD64 architecture.
\flowgraph{\resource{FALSE\\source code} \ar[r] & \toolbox{falamd64} \ar[r] & \resource{object file}}
\seefalse\seeamd\seeobject
}

\providecommand{\falarma}{
\toolsection{falarma32} is a compiler for the FALSE programming language targeting the ARM hardware architecture.
It generates machine code for ARM processors executing A32 instructions from programs written in FALSE and stores it in corresponding object files.
\flowgraph{\resource{FALSE\\source code} \ar[r] & \toolbox{falarma32} \ar[r] & \resource{object file}}
\seefalse\seearm\seeobject
}

\providecommand{\falarmb}{
\toolsection{falarma64} is a compiler for the FALSE programming language targeting the ARM hardware architecture.
It generates machine code for ARM processors executing A64 instructions from programs written in FALSE and stores it in corresponding object files.
\flowgraph{\resource{FALSE\\source code} \ar[r] & \toolbox{falarma64} \ar[r] & \resource{object file}}
\seefalse\seearm\seeobject
}

\providecommand{\falarmc}{
\toolsection{falarmt32} is a compiler for the FALSE programming language targeting the ARM hardware architecture.
It generates machine code for ARM processors without floating-point extension executing T32 instructions from programs written in FALSE and stores it in corresponding object files.
\flowgraph{\resource{FALSE\\source code} \ar[r] & \toolbox{falarmt32} \ar[r] & \resource{object file}}
\seefalse\seearm\seeobject
}

\providecommand{\falarmcfpe}{
\toolsection{falarmt32fpe} is a compiler for the FALSE programming language targeting the ARM hardware architecture.
It generates machine code for ARM processors with floating-point extension executing T32 instructions from programs written in FALSE and stores it in corresponding object files.
\flowgraph{\resource{FALSE\\source code} \ar[r] & \toolbox{falarmt32fpe} \ar[r] & \resource{object file}}
\seefalse\seearm\seeobject
}

\providecommand{\falavr}{
\toolsection{falavr} is a compiler for the FALSE programming language targeting the AVR hardware architecture.
It generates machine code for AVR processors from programs written in FALSE and stores it in corresponding object files.
\flowgraph{\resource{FALSE\\source code} \ar[r] & \toolbox{falavr} \ar[r] & \resource{object file}}
\seefalse\seeavr\seeobject
}

\providecommand{\falavrtt}{
\toolsection{falavr32} is a compiler for the FALSE programming language targeting the AVR32 hardware architecture.
It generates machine code for AVR32 processors from programs written in FALSE and stores it in corresponding object files.
\flowgraph{\resource{FALSE\\source code} \ar[r] & \toolbox{falavr32} \ar[r] & \resource{object file}}
\seefalse\seeavrtt\seeobject
}

\providecommand{\falmabk}{
\toolsection{falm68k} is a compiler for the FALSE programming language targeting the M68000 hardware architecture.
It generates machine code for M68000 processors from programs written in FALSE and stores it in corresponding object files.
\flowgraph{\resource{FALSE\\source code} \ar[r] & \toolbox{falm68k} \ar[r] & \resource{object file}}
\seefalse\seemabk\seeobject
}

\providecommand{\falmibl}{
\toolsection{falmibl} is a compiler for the FALSE programming language targeting the MicroBlaze hardware architecture.
It generates machine code for MicroBlaze processors from programs written in FALSE and stores it in corresponding object files.
\flowgraph{\resource{FALSE\\source code} \ar[r] & \toolbox{falmibl} \ar[r] & \resource{object file}}
\seefalse\seemibl\seeobject
}

\providecommand{\falmipsa}{
\toolsection{falmips32} is a compiler for the FALSE programming language targeting the MIPS32 hardware architecture.
It generates machine code for MIPS32 processors from programs written in FALSE and stores it in corresponding object files.
\flowgraph{\resource{FALSE\\source code} \ar[r] & \toolbox{falmips32} \ar[r] & \resource{object file}}
\seefalse\seemips\seeobject
}

\providecommand{\falmipsb}{
\toolsection{falmips64} is a compiler for the FALSE programming language targeting the MIPS64 hardware architecture.
It generates machine code for MIPS64 processors from programs written in FALSE and stores it in corresponding object files.
\flowgraph{\resource{FALSE\\source code} \ar[r] & \toolbox{falmips64} \ar[r] & \resource{object file}}
\seefalse\seemips\seeobject
}

\providecommand{\falmmix}{
\toolsection{falmmix} is a compiler for the FALSE programming language targeting the MMIX hardware architecture.
It generates machine code for MMIX processors from programs written in FALSE and stores it in corresponding object files.
\flowgraph{\resource{FALSE\\source code} \ar[r] & \toolbox{falmmix} \ar[r] & \resource{object file}}
\seefalse\seemmix\seeobject
}

\providecommand{\falorok}{
\toolsection{falor1k} is a compiler for the FALSE programming language targeting the OpenRISC 1000 hardware architecture.
It generates machine code for OpenRISC 1000 processors from programs written in FALSE and stores it in corresponding object files.
\flowgraph{\resource{FALSE\\source code} \ar[r] & \toolbox{falor1k} \ar[r] & \resource{object file}}
\seefalse\seeorok\seeobject
}

\providecommand{\falppca}{
\toolsection{falppc32} is a compiler for the FALSE programming language targeting the PowerPC hardware architecture.
It generates machine code for PowerPC processors from programs written in FALSE and stores it in corresponding object files.
The compiler generates machine code for the 32-bit operating mode defined by the PowerPC architecture.
\flowgraph{\resource{FALSE\\source code} \ar[r] & \toolbox{falppc32} \ar[r] & \resource{object file}}
\seefalse\seeppc\seeobject
}

\providecommand{\falppcb}{
\toolsection{falppc64} is a compiler for the FALSE programming language targeting the PowerPC hardware architecture.
It generates machine code for PowerPC processors from programs written in FALSE and stores it in corresponding object files.
The compiler generates machine code for the 64-bit operating mode defined by the PowerPC architecture.
\flowgraph{\resource{FALSE\\source code} \ar[r] & \toolbox{falppc64} \ar[r] & \resource{object file}}
\seefalse\seeppc\seeobject
}

\providecommand{\falrisc}{
\toolsection{falrisc} is a compiler for the FALSE programming language targeting the RISC hardware architecture.
It generates machine code for RISC processors from programs written in FALSE and stores it in corresponding object files.
\flowgraph{\resource{FALSE\\source code} \ar[r] & \toolbox{falrisc} \ar[r] & \resource{object file}}
\seefalse\seerisc\seeobject
}

\providecommand{\falwasm}{
\toolsection{falwasm} is a compiler for the FALSE programming language targeting the WebAssembly architecture.
It generates machine code for WebAssembly targets from programs written in FALSE and stores it in corresponding object files.
\flowgraph{\resource{FALSE\\source code} \ar[r] & \toolbox{falwasm} \ar[r] & \resource{object file}}
\seefalse\seewasm\seeobject
}

% Oberon tools

\providecommand{\obprint}{
\toolsection{obprint} is a pretty printer for the Oberon programming language.
It reformats the source code of Oberon modules and writes it to the standard output stream.
\flowgraph{\resource{Oberon\\source code} \ar[r] & \toolbox{obprint} \ar[r] & \resource{reformatted\\source code}}
\seeoberon
}

\providecommand{\obcheck}{
\toolsection{obcheck} is a syntactic and semantic checker for the Oberon programming language.
It just performs syntactic and semantic checks on Oberon modules and writes its diagnostic messages to the standard error stream.
In addition, it stores the interface of each module in a symbol file which is required when other modules import the module.
\flowgraph{\resource{Oberon\\source code} \ar[r] & \toolbox{obcheck} \ar[r] \ar@/l/[d] & \resource{diagnostic\\messages} \\ \variable{ECSIMPORT} \ar[ru] & \resource{symbol\\files} \ar@/r/[u]}
\seeoberon
}

\providecommand{\obdump}{
\toolsection{obdump} is a serializer for the Oberon programming language.
It dumps the complete internal representation of modules written in Oberon into an XML document.
\debuggingtool
\flowgraph{\resource{Oberon\\source code} \ar[r] & \toolbox{obdump} \ar[r] \ar@/l/[d] & \resource{internal\\representation} \\ \variable{ECSIMPORT} \ar[ru] & \resource{symbol\\files} \ar@/r/[u]}
\seeoberon
}

\providecommand{\obrun}{
\toolsection{obrun} is an interpreter for the Oberon programming language.
It processes and executes modules written in Oberon.
This tool does neither generate nor process symbol files while interpreting modules.
If a module is imported by another one, its filename has to be named before the other one in the list of command-line arguments.
\flowgraph{\resource{Oberon\\source code} \ar[r] & \toolbox{obrun} \ar@/u/[r] & \resource{input/\\output} \ar@/d/[l]}
\seeoberon
}

\providecommand{\obcpp}{
\toolsection{obcpp} is a transpiler for the Oberon programming language.
It translates programs written in Oberon into \cpp{} programs and stores them in corresponding source and header files.
In addition, it stores the interface of each module in a symbol file which is required when other modules import the module.
The same interface is provided by the generated header file which can be used in other parts of the \cpp{} program.
\flowgraph{\resource{Oberon\\source code} \ar[r] & \toolbox{obcpp} \ar[r] \ar@/l/[d] \ar[rd] & \resource{\cpp{}\\source file} \\ \variable{ECSIMPORT} \ar[ru] & \resource{symbol\\files} \ar@/r/[u] & \resource{\cpp{}\\header file}}
\seeoberon\seecpp
}

\providecommand{\obdoc}{
\toolsection{obdoc} is a generic documentation generator for the Oberon programming language.
It processes several Oberon modules and assembles all information therein into a generic documentation.
In addition, it stores the interface of each module in a symbol file which is required when other modules import the module.
\debuggingtool
\flowgraph{\resource{Oberon\\source code} \ar[r] & \toolbox{obdoc} \ar[r] \ar@/l/[d] & \resource{generic\\documentation} \\ \variable{ECSIMPORT} \ar[ru] & \resource{symbol\\files} \ar@/r/[u]}
\seeoberon\seedocumentation
}

\providecommand{\obhtml}{
\toolsection{obhtml} is an HTML documentation generator for the Oberon programming language.
It processes several Oberon modules and assembles all information therein into an HTML document.
In addition, it stores the interface of each module in a symbol file which is required when other modules import the module.
\flowgraph{\resource{Oberon\\source code} \ar[r] & \toolbox{obhtml} \ar[r] \ar@/l/[d] & \resource{HTML\\document} \\ \variable{ECSIMPORT} \ar[ru] & \resource{symbol\\files} \ar@/r/[u]}
\seeoberon\seedocumentation
}

\providecommand{\oblatex}{
\toolsection{oblatex} is a Latex documentation generator for the Oberon programming language.
It processes several Oberon modules and assembles all information therein into a Latex document.
In addition, it stores the interface of each module in a symbol file which is required when other modules import the module.
\flowgraph{\resource{Oberon\\source code} \ar[r] & \toolbox{oblatex} \ar[r] \ar@/l/[d] & \resource{Latex\\document} \\ \variable{ECSIMPORT} \ar[ru] & \resource{symbol\\files} \ar@/r/[u]}
\seeoberon\seedocumentation
}

\providecommand{\obcode}{
\toolsection{obcode} is an intermediate code generator for the Oberon programming language.
It generates intermediate code from modules written in Oberon and stores it in corresponding assembly files.
In addition, it stores the interface of each module in a symbol file which is required when other modules import the module.
Programs generated with this tool require additional runtime support that is stored in the \file{ob\-code\-run} library file.
\debuggingtool
\flowgraph{\resource{Oberon\\source code} \ar[r] & \toolbox{obcode} \ar[r] \ar@/l/[d] & \resource{intermediate\\code} \\ \variable{ECSIMPORT} \ar[ru] & \resource{symbol\\files} \ar@/r/[u]}
\seeoberon\seeassembly\seecode
}

\providecommand{\obamda}{
\toolsection{obamd16} is a compiler for the Oberon programming language targeting the AMD64 hardware architecture.
It generates machine code for AMD64 processors from modules written in Oberon and stores it in corresponding object files.
The compiler generates machine code for the 16-bit operating mode defined by the AMD64 architecture.
For debugging purposes, it also creates a debugging information file as well as an assembly file containing a listing of the generated machine code.
In addition, it stores the interface of each module in a symbol file which is required when other modules import the module.
Programs generated with this compiler require additional runtime support that is stored in the \file{ob\-amd16\-run} library file.
\flowgraph{\resource{Oberon\\source code} \ar[r] & \toolbox{obamd16} \ar[r] \ar@/l/[d] \ar[rd] & \resource{object file} \\ \variable{ECSIMPORT} \ar[ru] & \resource{symbol\\files} \ar@/r/[u] & \resource{debugging\\information}}
\seeoberon\seeassembly\seeamd\seeobject\seedebugging
}

\providecommand{\obamdb}{
\toolsection{obamd32} is a compiler for the Oberon programming language targeting the AMD64 hardware architecture.
It generates machine code for AMD64 processors from modules written in Oberon and stores it in corresponding object files.
The compiler generates machine code for the 32-bit operating mode defined by the AMD64 architecture.
For debugging purposes, it also creates a debugging information file as well as an assembly file containing a listing of the generated machine code.
In addition, it stores the interface of each module in a symbol file which is required when other modules import the module.
Programs generated with this compiler require additional runtime support that is stored in the \file{ob\-amd32\-run} library file.
\flowgraph{\resource{Oberon\\source code} \ar[r] & \toolbox{obamd32} \ar[r] \ar@/l/[d] \ar[rd] & \resource{object file} \\ \variable{ECSIMPORT} \ar[ru] & \resource{symbol\\files} \ar@/r/[u] & \resource{debugging\\information}}
\seeoberon\seeassembly\seeamd\seeobject\seedebugging
}

\providecommand{\obamdc}{
\toolsection{obamd64} is a compiler for the Oberon programming language targeting the AMD64 hardware architecture.
It generates machine code for AMD64 processors from modules written in Oberon and stores it in corresponding object files.
The compiler generates machine code for the 64-bit operating mode defined by the AMD64 architecture.
For debugging purposes, it also creates a debugging information file as well as an assembly file containing a listing of the generated machine code.
In addition, it stores the interface of each module in a symbol file which is required when other modules import the module.
Programs generated with this compiler require additional runtime support that is stored in the \file{ob\-amd64\-run} library file.
\flowgraph{\resource{Oberon\\source code} \ar[r] & \toolbox{obamd64} \ar[r] \ar@/l/[d] \ar[rd] & \resource{object file} \\ \variable{ECSIMPORT} \ar[ru] & \resource{symbol\\files} \ar@/r/[u] & \resource{debugging\\information}}
\seeoberon\seeassembly\seeamd\seeobject\seedebugging
}

\providecommand{\obarma}{
\toolsection{obarma32} is a compiler for the Oberon programming language targeting the ARM hardware architecture.
It generates machine code for ARM processors executing A32 instructions from modules written in Oberon and stores it in corresponding object files.
For debugging purposes, it also creates a debugging information file as well as an assembly file containing a listing of the generated machine code.
In addition, it stores the interface of each module in a symbol file which is required when other modules import the module.
Programs generated with this compiler require additional runtime support that is stored in the \file{ob\-arma32\-run} library file.
\flowgraph{\resource{Oberon\\source code} \ar[r] & \toolbox{obarma32} \ar[r] \ar@/l/[d] \ar[rd] & \resource{object file} \\ \variable{ECSIMPORT} \ar[ru] & \resource{symbol\\files} \ar@/r/[u] & \resource{debugging\\information}}
\seeoberon\seeassembly\seearm\seeobject\seedebugging
}

\providecommand{\obarmb}{
\toolsection{obarma64} is a compiler for the Oberon programming language targeting the ARM hardware architecture.
It generates machine code for ARM processors executing A64 instructions from modules written in Oberon and stores it in corresponding object files.
For debugging purposes, it also creates a debugging information file as well as an assembly file containing a listing of the generated machine code.
In addition, it stores the interface of each module in a symbol file which is required when other modules import the module.
Programs generated with this compiler require additional runtime support that is stored in the \file{ob\-arma64\-run} library file.
\flowgraph{\resource{Oberon\\source code} \ar[r] & \toolbox{obarma64} \ar[r] \ar@/l/[d] \ar[rd] & \resource{object file} \\ \variable{ECSIMPORT} \ar[ru] & \resource{symbol\\files} \ar@/r/[u] & \resource{debugging\\information}}
\seeoberon\seeassembly\seearm\seeobject\seedebugging
}

\providecommand{\obarmc}{
\toolsection{obarmt32} is a compiler for the Oberon programming language targeting the ARM hardware architecture.
It generates machine code for ARM processors without floating-point extension executing T32 instructions from modules written in Oberon and stores it in corresponding object files.
For debugging purposes, it also creates a debugging information file as well as an assembly file containing a listing of the generated machine code.
In addition, it stores the interface of each module in a symbol file which is required when other modules import the module.
Programs generated with this compiler require additional runtime support that is stored in the \file{ob\-armt32\-run} library file.
\flowgraph{\resource{Oberon\\source code} \ar[r] & \toolbox{obarmt32} \ar[r] \ar@/l/[d] \ar[rd] & \resource{object file} \\ \variable{ECSIMPORT} \ar[ru] & \resource{symbol\\files} \ar@/r/[u] & \resource{debugging\\information}}
\seeoberon\seeassembly\seearm\seeobject\seedebugging
}

\providecommand{\obarmcfpe}{
\toolsection{obarmt32fpe} is a compiler for the Oberon programming language targeting the ARM hardware architecture.
It generates machine code for ARM processors with floating-point extension executing T32 instructions from modules written in Oberon and stores it in corresponding object files.
For debugging purposes, it also creates a debugging information file as well as an assembly file containing a listing of the generated machine code.
In addition, it stores the interface of each module in a symbol file which is required when other modules import the module.
Programs generated with this compiler require additional runtime support that is stored in the \file{ob\-armt32\-fpe\-run} library file.
\flowgraph{\resource{Oberon\\source code} \ar[r] & \toolbox{obarmt32fpe} \ar[r] \ar@/l/[d] \ar[rd] & \resource{object file} \\ \variable{ECSIMPORT} \ar[ru] & \resource{symbol\\files} \ar@/r/[u] & \resource{debugging\\information}}
\seeoberon\seeassembly\seearm\seeobject\seedebugging
}

\providecommand{\obavr}{
\toolsection{obavr} is a compiler for the Oberon programming language targeting the AVR hardware architecture.
It generates machine code for AVR processors from modules written in Oberon and stores it in corresponding object files.
For debugging purposes, it also creates a debugging information file as well as an assembly file containing a listing of the generated machine code.
In addition, it stores the interface of each module in a symbol file which is required when other modules import the module.
Programs generated with this compiler require additional runtime support that is stored in the \file{ob\-avr\-run} library file.
\flowgraph{\resource{Oberon\\source code} \ar[r] & \toolbox{obavr} \ar[r] \ar@/l/[d] \ar[rd] & \resource{object file} \\ \variable{ECSIMPORT} \ar[ru] & \resource{symbol\\files} \ar@/r/[u] & \resource{debugging\\information}}
\seeoberon\seeassembly\seeavr\seeobject\seedebugging
}

\providecommand{\obavrtt}{
\toolsection{obavr32} is a compiler for the Oberon programming language targeting the AVR32 hardware architecture.
It generates machine code for AVR32 processors from modules written in Oberon and stores it in corresponding object files.
For debugging purposes, it also creates a debugging information file as well as an assembly file containing a listing of the generated machine code.
In addition, it stores the interface of each module in a symbol file which is required when other modules import the module.
Programs generated with this compiler require additional runtime support that is stored in the \file{ob\-avr32\-run} library file.
\flowgraph{\resource{Oberon\\source code} \ar[r] & \toolbox{obavr32} \ar[r] \ar@/l/[d] \ar[rd] & \resource{object file} \\ \variable{ECSIMPORT} \ar[ru] & \resource{symbol\\files} \ar@/r/[u] & \resource{debugging\\information}}
\seeoberon\seeassembly\seeavrtt\seeobject\seedebugging
}

\providecommand{\obmabk}{
\toolsection{obm68k} is a compiler for the Oberon programming language targeting the M68000 hardware architecture.
It generates machine code for M68000 processors from modules written in Oberon and stores it in corresponding object files.
For debugging purposes, it also creates a debugging information file as well as an assembly file containing a listing of the generated machine code.
In addition, it stores the interface of each module in a symbol file which is required when other modules import the module.
Programs generated with this compiler require additional runtime support that is stored in the \file{ob\-m68k\-run} library file.
\flowgraph{\resource{Oberon\\source code} \ar[r] & \toolbox{obm68k} \ar[r] \ar@/l/[d] \ar[rd] & \resource{object file} \\ \variable{ECSIMPORT} \ar[ru] & \resource{symbol\\files} \ar@/r/[u] & \resource{debugging\\information}}
\seeoberon\seeassembly\seemabk\seeobject\seedebugging
}

\providecommand{\obmibl}{
\toolsection{obmibl} is a compiler for the Oberon programming language targeting the MicroBlaze hardware architecture.
It generates machine code for MicroBlaze processors from modules written in Oberon and stores it in corresponding object files.
For debugging purposes, it also creates a debugging information file as well as an assembly file containing a listing of the generated machine code.
In addition, it stores the interface of each module in a symbol file which is required when other modules import the module.
Programs generated with this compiler require additional runtime support that is stored in the \file{ob\-mibl\-run} library file.
\flowgraph{\resource{Oberon\\source code} \ar[r] & \toolbox{obmibl} \ar[r] \ar@/l/[d] \ar[rd] & \resource{object file} \\ \variable{ECSIMPORT} \ar[ru] & \resource{symbol\\files} \ar@/r/[u] & \resource{debugging\\information}}
\seeoberon\seeassembly\seemibl\seeobject\seedebugging
}

\providecommand{\obmipsa}{
\toolsection{obmips32} is a compiler for the Oberon programming language targeting the MIPS32 hardware architecture.
It generates machine code for MIPS32 processors from modules written in Oberon and stores it in corresponding object files.
For debugging purposes, it also creates a debugging information file as well as an assembly file containing a listing of the generated machine code.
In addition, it stores the interface of each module in a symbol file which is required when other modules import the module.
Programs generated with this compiler require additional runtime support that is stored in the \file{ob\-mips32\-run} library file.
\flowgraph{\resource{Oberon\\source code} \ar[r] & \toolbox{obmips32} \ar[r] \ar@/l/[d] \ar[rd] & \resource{object file} \\ \variable{ECSIMPORT} \ar[ru] & \resource{symbol\\files} \ar@/r/[u] & \resource{debugging\\information}}
\seeoberon\seeassembly\seemips\seeobject\seedebugging
}

\providecommand{\obmipsb}{
\toolsection{obmips64} is a compiler for the Oberon programming language targeting the MIPS64 hardware architecture.
It generates machine code for MIPS64 processors from modules written in Oberon and stores it in corresponding object files.
For debugging purposes, it also creates a debugging information file as well as an assembly file containing a listing of the generated machine code.
In addition, it stores the interface of each module in a symbol file which is required when other modules import the module.
Programs generated with this compiler require additional runtime support that is stored in the \file{ob\-mips64\-run} library file.
\flowgraph{\resource{Oberon\\source code} \ar[r] & \toolbox{obmips64} \ar[r] \ar@/l/[d] \ar[rd] & \resource{object file} \\ \variable{ECSIMPORT} \ar[ru] & \resource{symbol\\files} \ar@/r/[u] & \resource{debugging\\information}}
\seeoberon\seeassembly\seemips\seeobject\seedebugging
}

\providecommand{\obmmix}{
\toolsection{obmmix} is a compiler for the Oberon programming language targeting the MMIX hardware architecture.
It generates machine code for MMIX processors from modules written in Oberon and stores it in corresponding object files.
For debugging purposes, it also creates a debugging information file as well as an assembly file containing a listing of the generated machine code.
In addition, it stores the interface of each module in a symbol file which is required when other modules import the module.
Programs generated with this compiler require additional runtime support that is stored in the \file{ob\-mmix\-run} library file.
\flowgraph{\resource{Oberon\\source code} \ar[r] & \toolbox{obmmix} \ar[r] \ar@/l/[d] \ar[rd] & \resource{object file} \\ \variable{ECSIMPORT} \ar[ru] & \resource{symbol\\files} \ar@/r/[u] & \resource{debugging\\information}}
\seeoberon\seeassembly\seemmix\seeobject\seedebugging
}

\providecommand{\oborok}{
\toolsection{obor1k} is a compiler for the Oberon programming language targeting the OpenRISC 1000 hardware architecture.
It generates machine code for OpenRISC 1000 processors from modules written in Oberon and stores it in corresponding object files.
For debugging purposes, it also creates a debugging information file as well as an assembly file containing a listing of the generated machine code.
In addition, it stores the interface of each module in a symbol file which is required when other modules import the module.
Programs generated with this compiler require additional runtime support that is stored in the \file{ob\-or1k\-run} library file.
\flowgraph{\resource{Oberon\\source code} \ar[r] & \toolbox{obor1k} \ar[r] \ar@/l/[d] \ar[rd] & \resource{object file} \\ \variable{ECSIMPORT} \ar[ru] & \resource{symbol\\files} \ar@/r/[u] & \resource{debugging\\information}}
\seeoberon\seeassembly\seeorok\seeobject\seedebugging
}

\providecommand{\obppca}{
\toolsection{obppc32} is a compiler for the Oberon programming language targeting the PowerPC hardware architecture.
It generates machine code for PowerPC processors from modules written in Oberon and stores it in corresponding object files.
The compiler generates machine code for the 32-bit operating mode defined by the PowerPC architecture.
For debugging purposes, it also creates a debugging information file as well as an assembly file containing a listing of the generated machine code.
In addition, it stores the interface of each module in a symbol file which is required when other modules import the module.
Programs generated with this compiler require additional runtime support that is stored in the \file{ob\-ppc32\-run} library file.
\flowgraph{\resource{Oberon\\source code} \ar[r] & \toolbox{obppc32} \ar[r] \ar@/l/[d] \ar[rd] & \resource{object file} \\ \variable{ECSIMPORT} \ar[ru] & \resource{symbol\\files} \ar@/r/[u] & \resource{debugging\\information}}
\seeoberon\seeassembly\seeppc\seeobject\seedebugging
}

\providecommand{\obppcb}{
\toolsection{obppc64} is a compiler for the Oberon programming language targeting the PowerPC hardware architecture.
It generates machine code for PowerPC processors from modules written in Oberon and stores it in corresponding object files.
The compiler generates machine code for the 64-bit operating mode defined by the PowerPC architecture.
For debugging purposes, it also creates a debugging information file as well as an assembly file containing a listing of the generated machine code.
In addition, it stores the interface of each module in a symbol file which is required when other modules import the module.
Programs generated with this compiler require additional runtime support that is stored in the \file{ob\-ppc64\-run} library file.
\flowgraph{\resource{Oberon\\source code} \ar[r] & \toolbox{obppc64} \ar[r] \ar@/l/[d] \ar[rd] & \resource{object file} \\ \variable{ECSIMPORT} \ar[ru] & \resource{symbol\\files} \ar@/r/[u] & \resource{debugging\\information}}
\seeoberon\seeassembly\seeppc\seeobject\seedebugging
}

\providecommand{\obrisc}{
\toolsection{obrisc} is a compiler for the Oberon programming language targeting the RISC hardware architecture.
It generates machine code for RISC processors from modules written in Oberon and stores it in corresponding object files.
For debugging purposes, it also creates a debugging information file as well as an assembly file containing a listing of the generated machine code.
In addition, it stores the interface of each module in a symbol file which is required when other modules import the module.
Programs generated with this compiler require additional runtime support that is stored in the \file{ob\-risc\-run} library file.
\flowgraph{\resource{Oberon\\source code} \ar[r] & \toolbox{obrisc} \ar[r] \ar@/l/[d] \ar[rd] & \resource{object file} \\ \variable{ECSIMPORT} \ar[ru] & \resource{symbol\\files} \ar@/r/[u] & \resource{debugging\\information}}
\seeoberon\seeassembly\seerisc\seeobject\seedebugging
}

\providecommand{\obwasm}{
\toolsection{obwasm} is a compiler for the Oberon programming language targeting the WebAssembly architecture.
It generates machine code for WebAssembly targets from modules written in Oberon and stores it in corresponding object files.
For debugging purposes, it also creates a debugging information file as well as an assembly file containing a listing of the generated machine code.
In addition, it stores the interface of each module in a symbol file which is required when other modules import the module.
Programs generated with this compiler require additional runtime support that is stored in the \file{ob\-wasm\-run} library file.
\flowgraph{\resource{Oberon\\source code} \ar[r] & \toolbox{obwasm} \ar[r] \ar@/l/[d] \ar[rd] & \resource{object file} \\ \variable{ECSIMPORT} \ar[ru] & \resource{symbol\\files} \ar@/r/[u] & \resource{debugging\\information}}
\seeoberon\seeassembly\seewasm\seeobject\seedebugging
}

% converter tools

\providecommand{\dbgdwarf}{
\toolsection{dbgdwarf} is a DWARF debugging information converter tool.
It converts debugging information into the DWARF debugging data format and stores it in corresponding object files~\cite{dwarffile}.
The resulting debugging object files can be combined with runtime support that creates Executable and Linking Format (ELF) files~\cite{elffile}.
\flowgraph{\resource{debugging\\information} \ar[r] & \toolbox{dbgdwarf} \ar[r] & \resource{debugging\\object file}}
\seeobject\seedebugging
}

% assembler tools

\providecommand{\asmprint}{
\toolsection{asmprint} is a pretty printer for generic assembly code.
It reformats generic assembly code and writes it to the standard output stream.
\flowgraph{\resource{generic assembly\\source code} \ar[r] & \toolbox{asmprint} \ar[r] & \resource{reformatted\\source code}}
\seeassembly
}

\providecommand{\amdaasm}{
\toolsection{amd16asm} is an assembler for the AMD64 hardware architecture.
It translates assembly code into machine code for AMD64 processors and stores it in corresponding object files.
By default, the assembler generates machine code for the 16-bit operating mode defined by the AMD64 architecture.
\flowgraph{\resource{AMD16 assembly\\source code} \ar[r] & \toolbox{amd16asm} \ar[r] & \resource{object file}}
\seeassembly\seeamd\seeobject
}

\providecommand{\amdadism}{
\toolsection{amd16dism} is a disassembler for the AMD64 hardware architecture.
It translates machine code from object files targeting AMD64 processors into assembly code and writes it to the standard output stream.
It assumes that the machine code was generated for the 16-bit operating mode defined by the AMD64 architecture.
\flowgraph{\resource{object file} \ar[r] & \toolbox{amd16dism} \ar[r] & \resource{disassembly\\listing}}
\seeassembly\seeamd\seeobject
}

\providecommand{\amdbasm}{
\toolsection{amd32asm} is an assembler for the AMD64 hardware architecture.
It translates assembly code into machine code for AMD64 processors and stores it in corresponding object files.
By default, the assembler generates machine code for the 32-bit operating mode defined by the AMD64 architecture.
\flowgraph{\resource{AMD32 assembly\\source code} \ar[r] & \toolbox{amd32asm} \ar[r] & \resource{object file}}
\seeassembly\seeamd\seeobject
}

\providecommand{\amdbdism}{
\toolsection{amd32dism} is a disassembler for the AMD64 hardware architecture.
It translates machine code from object files targeting AMD64 processors into assembly code and writes it to the standard output stream.
It assumes that the machine code was generated for the 32-bit operating mode defined by the AMD64 architecture.
\flowgraph{\resource{object file} \ar[r] & \toolbox{amd32dism} \ar[r] & \resource{disassembly\\listing}}
\seeassembly\seeamd\seeobject
}

\providecommand{\amdcasm}{
\toolsection{amd64asm} is an assembler for the AMD64 hardware architecture.
It translates assembly code into machine code for AMD64 processors and stores it in corresponding object files.
By default, the assembler generates machine code for the 64-bit operating mode defined by the AMD64 architecture.
\flowgraph{\resource{AMD64 assembly\\source code} \ar[r] & \toolbox{amd64asm} \ar[r] & \resource{object file}}
\seeassembly\seeamd\seeobject
}

\providecommand{\amdcdism}{
\toolsection{amd64dism} is a disassembler for the AMD64 hardware architecture.
It translates machine code from object files targeting AMD64 processors into assembly code and writes it to the standard output stream.
It assumes that the machine code was generated for the 64-bit operating mode defined by the AMD64 architecture.
\flowgraph{\resource{object file} \ar[r] & \toolbox{amd64dism} \ar[r] & \resource{disassembly\\listing}}
\seeassembly\seeamd\seeobject
}

\providecommand{\armaasm}{
\toolsection{arma32asm} is an assembler for the ARM hardware architecture.
It translates assembly code into machine code for ARM processors executing A32 instructions and stores it in corresponding object files.
\flowgraph{\resource{ARM A32 assembly\\source code} \ar[r] & \toolbox{arma32asm} \ar[r] & \resource{object file}}
\seeassembly\seearm\seeobject
}

\providecommand{\armadism}{
\toolsection{arma32dism} is a disassembler for the ARM hardware architecture.
It translates machine code from object files targeting ARM processors executing A32 instructions into assembly code and writes it to the standard output stream.
\flowgraph{\resource{object file} \ar[r] & \toolbox{arma32dism} \ar[r] & \resource{disassembly\\listing}}
\seeassembly\seearm\seeobject
}

\providecommand{\armbasm}{
\toolsection{arma64asm} is an assembler for the ARM hardware architecture.
It translates assembly code into machine code for ARM processors executing A64 instructions and stores it in corresponding object files.
\flowgraph{\resource{ARM A64 assembly\\source code} \ar[r] & \toolbox{arma64asm} \ar[r] & \resource{object file}}
\seeassembly\seearm\seeobject
}

\providecommand{\armbdism}{
\toolsection{arma64dism} is a disassembler for the ARM hardware architecture.
It translates machine code from object files targeting ARM processors executing A64 instructions into assembly code and writes it to the standard output stream.
\flowgraph{\resource{object file} \ar[r] & \toolbox{arma64dism} \ar[r] & \resource{disassembly\\listing}}
\seeassembly\seearm\seeobject
}

\providecommand{\armcasm}{
\toolsection{armt32asm} is an assembler for the ARM hardware architecture.
It translates assembly code into machine code for ARM processors executing T32 instructions and stores it in corresponding object files.
\flowgraph{\resource{ARM T32 assembly\\source code} \ar[r] & \toolbox{armt32asm} \ar[r] & \resource{object file}}
\seeassembly\seearm\seeobject
}

\providecommand{\armcdism}{
\toolsection{armt32dism} is a disassembler for the ARM hardware architecture.
It translates machine code from object files targeting ARM processors executing T32 instructions into assembly code and writes it to the standard output stream.
\flowgraph{\resource{object file} \ar[r] & \toolbox{armt32dism} \ar[r] & \resource{disassembly\\listing}}
\seeassembly\seearm\seeobject
}

\providecommand{\avrasm}{
\toolsection{avrasm} is an assembler for the AVR hardware architecture.
It translates assembly code into machine code for AVR processors and stores it in corresponding object files.
The identifiers \texttt{RXL}, \texttt{RXH}, \texttt{RYL}, \texttt{RYH}, \texttt{RZL}, and \texttt{RZH} are predefined and name the corresponding registers.
The identifiers \texttt{SPL} and \texttt{SPH} are also predefined and evaluate to the address of the corresponding registers.
\flowgraph{\resource{AVR assembly\\source code} \ar[r] & \toolbox{avrasm} \ar[r] & \resource{object file}}
\seeassembly\seeavr\seeobject
}

\providecommand{\avrdism}{
\toolsection{avrdism} is a disassembler for the AVR hardware architecture.
It translates machine code from object files targeting AVR processors into assembly code and writes it to the standard output stream.
\flowgraph{\resource{object file} \ar[r] & \toolbox{avrdism} \ar[r] & \resource{disassembly\\listing}}
\seeassembly\seeavr\seeobject
}

\providecommand{\avrttasm}{
\toolsection{avr32asm} is an assembler for the AVR32 hardware architecture.
It translates assembly code into machine code for AVR32 processors and stores it in corresponding object files.
\flowgraph{\resource{AVR32 assembly\\source code} \ar[r] & \toolbox{avr32asm} \ar[r] & \resource{object file}}
\seeassembly\seeavrtt\seeobject
}

\providecommand{\avrttdism}{
\toolsection{avr32dism} is a disassembler for the AVR32 hardware architecture.
It translates machine code from object files targeting AVR32 processors into assembly code and writes it to the standard output stream.
\flowgraph{\resource{object file} \ar[r] & \toolbox{avr32dism} \ar[r] & \resource{disassembly\\listing}}
\seeassembly\seeavrtt\seeobject
}

\providecommand{\mabkasm}{
\toolsection{m68kasm} is an assembler for the M68000 hardware architecture.
It translates assembly code into machine code for M68000 processors and stores it in corresponding object files.
\flowgraph{\resource{68000 assembly\\source code} \ar[r] & \toolbox{m68kasm} \ar[r] & \resource{object file}}
\seeassembly\seemabk\seeobject
}

\providecommand{\mabkdism}{
\toolsection{m68kdism} is a disassembler for the M68000 hardware architecture.
It translates machine code from object files targeting M68000 processors into assembly code and writes it to the standard output stream.
\flowgraph{\resource{object file} \ar[r] & \toolbox{m68kdism} \ar[r] & \resource{disassembly\\listing}}
\seeassembly\seemabk\seeobject
}

\providecommand{\miblasm}{
\toolsection{miblasm} is an assembler for the MicroBlaze hardware architecture.
It translates assembly code into machine code for MicroBlaze processors and stores it in corresponding object files.
\flowgraph{\resource{MicroBlaze assembly\\source code} \ar[r] & \toolbox{miblasm} \ar[r] & \resource{object file}}
\seeassembly\seemibl\seeobject
}

\providecommand{\mibldism}{
\toolsection{mibldism} is a disassembler for the MicroBlaze hardware architecture.
It translates machine code from object files targeting MicroBlaze processors into assembly code and writes it to the standard output stream.
\flowgraph{\resource{object file} \ar[r] & \toolbox{mibldism} \ar[r] & \resource{disassembly\\listing}}
\seeassembly\seemibl\seeobject
}

\providecommand{\mipsaasm}{
\toolsection{mips32asm} is an assembler for the MIPS32 hardware architecture.
It translates assembly code into machine code for MIPS32 processors and stores it in corresponding object files.
\flowgraph{\resource{MIPS32 assembly\\source code} \ar[r] & \toolbox{mips32asm} \ar[r] & \resource{object file}}
\seeassembly\seemips\seeobject
}

\providecommand{\mipsadism}{
\toolsection{mips32dism} is a disassembler for the MIPS32 hardware architecture.
It translates machine code from object files targeting MIPS32 processors into assembly code and writes it to the standard output stream.
\flowgraph{\resource{object file} \ar[r] & \toolbox{mips32dism} \ar[r] & \resource{disassembly\\listing}}
\seeassembly\seemips\seeobject
}

\providecommand{\mipsbasm}{
\toolsection{mips64asm} is an assembler for the MIPS64 hardware architecture.
It translates assembly code into machine code for MIPS64 processors and stores it in corresponding object files.
\flowgraph{\resource{MIPS64 assembly\\source code} \ar[r] & \toolbox{mips64asm} \ar[r] & \resource{object file}}
\seeassembly\seemips\seeobject
}

\providecommand{\mipsbdism}{
\toolsection{mips64dism} is a disassembler for the MIPS64 hardware architecture.
It translates machine code from object files targeting MIPS64 processors into assembly code and writes it to the standard output stream.
\flowgraph{\resource{object file} \ar[r] & \toolbox{mips64dism} \ar[r] & \resource{disassembly\\listing}}
\seeassembly\seemips\seeobject
}

\providecommand{\mmixasm}{
\toolsection{mmixasm} is an assembler for the MMIX hardware architecture.
It translates assembly code into machine code for MMIX processors and stores it in corresponding object files.
The names of all special registers are predefined and evaluate to the corresponding number.
\flowgraph{\resource{MMIX assembly\\source code} \ar[r] & \toolbox{mmixasm} \ar[r] & \resource{object file}}
\seeassembly\seemmix\seeobject
}

\providecommand{\mmixdism}{
\toolsection{mmixdism} is a disassembler for the MMIX hardware architecture.
It translates machine code from object files targeting MMIX processors into assembly code and writes it to the standard output stream.
\flowgraph{\resource{object file} \ar[r] & \toolbox{mmixdism} \ar[r] & \resource{disassembly\\listing}}
\seeassembly\seemmix\seeobject
}

\providecommand{\orokasm}{
\toolsection{or1kasm} is an assembler for the OpenRISC 1000 hardware architecture.
It translates assembly code into machine code for OpenRISC 1000 processors and stores it in corresponding object files.
\flowgraph{\resource{OpenRISC 1000 assembly\\source code} \ar[r] & \toolbox{or1kasm} \ar[r] & \resource{object file}}
\seeassembly\seeorok\seeobject
}

\providecommand{\orokdism}{
\toolsection{or1kdism} is a disassembler for the OpenRISC 1000 hardware architecture.
It translates machine code from object files targeting OpenRISC 1000 processors into assembly code and writes it to the standard output stream.
\flowgraph{\resource{object file} \ar[r] & \toolbox{or1kdism} \ar[r] & \resource{disassembly\\listing}}
\seeassembly\seeorok\seeobject
}

\providecommand{\ppcaasm}{
\toolsection{ppc32asm} is an assembler for the PowerPC hardware architecture.
It translates assembly code into machine code for PowerPC processors and stores it in corresponding object files.
By default, the assembler generates machine code for the 32-bit operating mode defined by the PowerPC architecture.
\flowgraph{\resource{PowerPC assembly\\source code} \ar[r] & \toolbox{ppc32asm} \ar[r] & \resource{object file}}
\seeassembly\seeppc\seeobject
}

\providecommand{\ppcadism}{
\toolsection{ppc32dism} is a disassembler for the PowerPC hardware architecture.
It translates machine code from object files targeting PowerPC processors into assembly code and writes it to the standard output stream.
It assumes that the machine code was generated for the 32-bit operating mode defined by the PowerPC architecture.
\flowgraph{\resource{object file} \ar[r] & \toolbox{ppc32dism} \ar[r] & \resource{disassembly\\listing}}
\seeassembly\seeppc\seeobject
}

\providecommand{\ppcbasm}{
\toolsection{ppc64asm} is an assembler for the PowerPC hardware architecture.
It translates assembly code into machine code for PowerPC processors and stores it in corresponding object files.
By default, the assembler generates machine code for the 64-bit operating mode defined by the PowerPC architecture.
\flowgraph{\resource{PowerPC assembly\\source code} \ar[r] & \toolbox{ppc64asm} \ar[r] & \resource{object file}}
\seeassembly\seeppc\seeobject
}

\providecommand{\ppcbdism}{
\toolsection{ppc64dism} is a disassembler for the PowerPC hardware architecture.
It translates machine code from object files targeting PowerPC processors into assembly code and writes it to the standard output stream.
It assumes that the machine code was generated for the 64-bit operating mode defined by the PowerPC architecture.
\flowgraph{\resource{object file} \ar[r] & \toolbox{ppc64dism} \ar[r] & \resource{disassembly\\listing}}
\seeassembly\seeppc\seeobject
}

\providecommand{\riscasm}{
\toolsection{riscasm} is an assembler for the RISC hardware architecture.
It translates assembly code into machine code for RISC processors and stores it in corresponding object files.
The names of all special registers are predefined and evaluate to the corresponding number.
\flowgraph{\resource{RISC assembly\\source code} \ar[r] & \toolbox{riscasm} \ar[r] & \resource{object file}}
\seeassembly\seerisc\seeobject
}

\providecommand{\riscdism}{
\toolsection{riscdism} is a disassembler for the RISC hardware architecture.
It translates machine code from object files targeting RISC processors into assembly code and writes it to the standard output stream.
\flowgraph{\resource{object file} \ar[r] & \toolbox{riscdism} \ar[r] & \resource{disassembly\\listing}}
\seeassembly\seerisc\seeobject
}

\providecommand{\wasmasm}{
\toolsection{wasmasm} is an assembler for the WebAssembly architecture.
It translates assembly code into machine code for WebAssembly targets and stores it in corresponding object files.
The names of all special registers are predefined and evaluate to the corresponding number.
\flowgraph{\resource{WebAssembly assembly\\source code} \ar[r] & \toolbox{wasmasm} \ar[r] & \resource{object file}}
\seeassembly\seewasm\seeobject
}

\providecommand{\wasmdism}{
\toolsection{wasmdism} is a disassembler for the WebAssembly architecture.
It translates machine code from object files targeting WebAssembly targets into assembly code and writes it to the standard output stream.
\flowgraph{\resource{object file} \ar[r] & \toolbox{wasmdism} \ar[r] & \resource{disassembly\\listing}}
\seeassembly\seewasm\seeobject
}

% linker tools

\providecommand{\linklib}{
\toolsection{linklib} is an object file combiner.
It creates a static library file by combining all object files given to it into a single one.
\flowgraph{\resource{object files} \ar[r] & \toolbox{linklib} \ar[r] & \resource{library file}}
\seeobject
}

\providecommand{\linkbin}{
\toolsection{linkbin} is a linker for plain binary files.
It links all object files given to it into a single image and stores it in a binary file that begins with the first linked section.
It also creates a map file that lists the address, type, name and size of all used sections.
The filename extension of the resulting binary file can be specified by putting it into a constant data section called \texttt{\_extension}.
\flowgraph{\resource{object files} \ar[r] & \toolbox{linkbin} \ar[r] \ar[d] & \resource{binary file} \\ & \resource{map file}}
\seeobject
}

\providecommand{\linkmem}{
\toolsection{linkmem} is a linker for plain binary files partitioned into random-access and read-only memory.
It links all object files given to it into two distinct images, one for data sections and one for code and constant data sections, and stores each image in a binary file that begins with the first linked section of the corresponding type.
It also creates a map file that lists the address, type, name and size of all used sections.
\flowgraph{\resource{object files} \ar[r] & \toolbox{linkmem} \ar[r] \ar[d] & \resource{RAM file/\\ROM file} \\ & \resource{map file}}
\seeobject
}

\providecommand{\linkprg}{
\toolsection{linkprg} is a linker for GEMDOS executable files.
It links all object files given to it into a single image and stores the image in an Atari GEMDOS executable file~\cite{gemdosfile}.
It also creates a map file that lists the address relative to the text segment, type, name and size of all used sections.
The filename extension of the resulting executable file can be specified by putting it into a constant data section called \texttt{\_extension}.
The GEMDOS executable file format requires all patch patterns of absolute link patches to consist of four full bitmasks with descending offsets.
\flowgraph{\resource{object files} \ar[r] & \toolbox{linkprg} \ar[r] \ar[d] & \resource{executable file} \\ & \resource{map file}}
\seeobject
}

\providecommand{\linkhex}{
\toolsection{linkhex} is a linker for Intel HEX files.
It links all code sections of the object files given to it into single image and stores the image in an Intel HEX file~\cite{hexfile} that begins with the first linked section.
It also creates a map file that lists the address, type, name and size of all used sections.
\flowgraph{\resource{object files} \ar[r] & \toolbox{linkhex} \ar[r] \ar[d] & \resource{HEX file} \\ & \resource{map file}}
\seeobject
}

\providecommand{\mapsearch}{
\toolsection{mapsearch} is a debugging tool.
It searches map files generated by linker tools for the name of a binary section that encompasses a memory address read from the standard input stream.
If additionally provided with one or more object files, it also stores an excerpt thereof in a separate object file called map search result which only contains the identified binary section for disassembling purposes.
\flowgraph{& \resource{map files/\\object files} \ar[d] \\ \resource{memory\\address} \ar[r] & \toolbox{mapsearch} \ar[r] \ar[d] & \resource{section name/\\relative offset} \\ & \resource{object file\\excerpt}}
\seeobject
}

\renewcommand{\seecpp}{}

\startchapter{\cpp{}}{User Manual for \cpp{}}{cpp}
{\cpp{} is a general-purpose programming language with a bias toward systems programming.
It is based on the programming language C and enhances it by supporting data abstraction, object-oriented programming, and generic programming.
This \documentation{} describes the implementation of \cpp{} by the \ecs{}.}

\epigraph{Greatness lies not in being strong, \\ but in the using of strength.}{Henry Ward Beecher}

\section{Introduction}

The \ecs{} implements the \cpp{} programming language according to the ISO \cpp{} Standard ISO/IEC 14882:2023~\cite{iso2023}.
\cpp{} is a general-purpose programming language based on the C programming language as described in the C Standard ISO/IEC 9899:2018~\cite{iso2018}.

\begin{center}\cpplogo{2em}\end{center}

The remainder of this \documentation{} describes the implementation-defined behavior of the implementation by the \ecs{} as required by both of these international standards.

\section{Implementation-Defined Behavior}

The \cpp{} Standard requires all conforming implementations to include documentation describing its characteristics and behavior in areas designated as implementation-defined.
This section lists all implementation-defined behavior including conditionally-supported constructs and locale-specific characteristics along with the corresponding section and paragraph numbers of the \cpp{} Standard.

\newcommand{\cppref}[3]{\alignright#1~[#2]~#3\nopagebreak}
\newcommand{\cppsection}[3]{\subsection[#1]{#1\alignright#2~[#3]}}

\cppsection{Terms and Definitions}{3}{intro.defs}

\begin{itemize}

\item Diagnostic messages \cppref{3.8}{defns.diagnostic}{1}

The set of diagnostic messages consist of all output messages issued by the tools of the \ecs{}.
Diagnostic messages are either errors, warnings, or notes and are meant to be self-explanatory in the context they occur:
Errors indicate a violation of a diagnosable rule or an occurrence of a unsupported construct described in the \cpp{} Standard as conditionally-supported.
Fatal errors flag internal program errors or environmental problems like failures to open source files or critical system conditions like out of memory.
Warning messages identify issues that may lead to unexpected behavior, whereas notes diagnose violations of common coding conventions.

\end{itemize}

\cppsection{General Principles}{4}{intro}

\begin{itemize}

\item Number of contiguous bits in a byte \cppref{4.4}{intro.memory}{1}

The \ecs{} uniformly uses octets as representation for bytes.

\item Interactive devices \cppref{4.6}{intro.execution}{7}

A device is considered to be interactive when it has to wait for input from the user or some other entity in order to proceed the execution of a program.

\item Multiple threads of execution \cppref{4.7}{intro.multithread}{1}

A program can have more than one thread of execution in a freestanding environment.

\end{itemize}

\cppsection{Lexical Conventions}{5}{lex}

\begin{itemize}

\item Mapping of source file characters to the basic source character set \cppref{5.2}{lex.phases}{1}

The \ecs{} assumes source files to be written using the ASCII character set.
This set allows mapping most of the characters to the basic source character set.
Any other character not in the basic source character set is replaced by the universal character name that designates that character.

\item Nonempty sequences of white-space characters \cppref{5.2}{lex.phases}{1}

Nonempty sequences of white-space characters are replaced by one space character.

\item Conversion of source characters to execution characters \cppref{5.2}{lex.phases}{1}

All members of the source character set have a corresponding member in the execution character set.

\item Source of template definitions \cppref{5.2}{lex.phases}{1}

Instantiations of templates require the source of the translation units containing their definitions.

\item Appearance of special characters in header names \cppref{5.8}{lex.header}{2}

Quotation marks, backslashes, and the character sequences~\texttt{/*} or~\texttt{//} may appear in header names but bear no special meaning and are therefore treated like any other character.

\item Character literals containing multiple characters \cppref{5.13.3}{lex.ccon}{2/6}

Ordinary and wide character literals may contain more than one character.
However, all but the first character are ignored when determining the value of these literals.

\item Additional escape sequences \cppref{5.13.3}{lex.ccon}{7}

The \ecs{} does only support the standard escape sequences.

\item Character values outside valid range \cppref{5.13.3}{lex.ccon}{8/9}

The value of an escape sequence or a universal character name in a character literal must be representable using the underlying type of the literal.

\item String literal concatenations \cppref{5.13.5}{lex.string}{13}

Concatenations of adjacent string literals with different encoding prefixes are not supported.

\end{itemize}

\cppsection{Basic Concepts}{6}{basic}

\begin{itemize}

\item Definition of functions or variables with origin \cppref {6.2}{basic.def.odr}{4}

A non-inline function or variable does not require a definition if it is declared using the origin attribute, see Section~\ref{sec:cppdeclarations}.

\item Definition of \texttt{main} in freestanding environments \cppref{6.6.1}{basic.start.main}{1}

A program in a freestanding environment is required to define a \texttt{main} function.

\item Linkage of \texttt{main} \cppref{6.6.1}{basic.start.main}{3}

The \texttt{main} function has external linkage.

\item Use of invalid pointer values \cppref{6.7}{basic.stc}{4}

There are no restrictions for using an invalid pointer value except for passing it to a deallocation function and accessing the referenced region of storage.

\item Pointer safety \cppref{6.7.4.3}{basic.stc.dynamic.safety}{4}

The \ecs{} has relaxed pointer safety.
A pointer value is valid regardless of whether it is safely derived or not.

\item Extended integer types \cppref{6.9.1}{basic.fundamental}{2}

The \ecs{} does not provide any extended integer types.

\item Fundamental alignments \cppref{6.11}{basic.align}{2}

The alignment of a fundamental type depends on the size of the type as well as the execution environment.
Table~\ref{tab:cppfundamentalalignments} lists the alignment of fundamental types for each \cpp{} compiler and type size supported by the \ecs{}.
The sizes of fundamental types are shown in Table~\ref{tab:cppfundamentaltypes} and Table~\ref{tab:cppdependenttypes}.
The interpreter and all other tools reuse the respective fundamental alignments of their own execution environment instead.

\begin{table}
\centering
\begin{tabular}{@{}lrlc@{\qquad}c@{\qquad}c@{\qquad}c@{}}
\toprule \multicolumn{2}{@{}l}{Hardware} & \cpp{} & \multicolumn{4}{c@{}}{Type Size} \\ \multicolumn{2}{@{}l}{Architecture} & Compiler & 1 & 2 & 4 & 8 \\
\midrule AMD64 & 16-bit & \tool{cppamd16} & 1 & 2 & 4 & 4 \\ & 32-bit & \tool{cppamd32} & 1 & 2 & 4 & 4 \\ & 64-bit & \tool{cppamd64} & 1 & 2 & 4 & 8 \\
\midrule ARM & A32 & \tool{cpparma32} & 1 & 2 & 4 & 8 \\ & A64 & \tool{cpparma64} & 1 & 2 & 4 & 8 \\ & T32 & \tool{cpparmt32} & 1 & 2 & 4 & 8 \\ & & \tool{cpparmt32fpe} & 1 & 2 & 4 & 8 \\
\midrule \multicolumn{2}{@{}l}{AVR} & \tool{cppavr} & 1 & 1 & 1 & 1 \\
\midrule \multicolumn{2}{@{}l}{AVR32} & \tool{cppavr32} & 1 & 2 & 4 & 8 \\
\midrule \multicolumn{2}{@{}l}{M68000} & \tool{cppm68k} & 1 & 2 & 2 & 2 \\
\midrule \multicolumn{2}{@{}l}{MicroBlaze} & \tool{cppmibl} & 1 & 2 & 4 & 8 \\
\midrule MIPS & 32-bit & \tool{cppmips32} & 1 & 2 & 4 & 8 \\ & 64-bit & \tool{cppmips64} & 1 & 2 & 4 & 8 \\
\midrule \multicolumn{2}{@{}l}{MMIX} & \tool{cppmmix} & 1 & 2 & 4 & 8 \\
\midrule \multicolumn{2}{@{}l}{OpenRISC 1000} & \tool{cppor1k} & 1 & 2 & 4 & 4 \\
\midrule PowerPC & 32-bit & \tool{cppppc32} & 1 & 2 & 4 & 8 \\ & 64-bit & \tool{cppppc64} & 1 & 2 & 4 & 8 \\
\midrule \multicolumn{2}{@{}l}{RISC} & \tool{cpprisc} & 1 & 2 & 4 & 4 \\
\midrule \multicolumn{2}{@{}l}{WebAssembly} & \tool{cppwasm} & 1 & 2 & 4 & 8 \\
\bottomrule
\end{tabular}
\caption{Alignments of fundamental \cpp{} types}
\label{tab:cppfundamentalalignments}
\end{table}

\item Extended alignments \cppref{6.11}{basic.align}{3}

The \ecs{} supports extended alignments for variables with static storage duration.
The set of valid alignments consists of all non-negative integral powers of two representable in the type \texttt{std::size\_t}.

\end{itemize}

\cppsection{Standard Conversions}{7}{conv}

\begin{itemize}

\item Value of integral conversions \cppref{7.8}{conv.integral}{3}

If the destination type is signed, the resulting value of an integral conversion is the least signed integer congruent to the source integer modulo $2^n$ where $n$ is the number of bits used to represent the signed type.

\item Ranks of extended signed integer types \cppref{7.15}{conv.rank}{1}

The \ecs{} does not provide any extended integer types.

\end{itemize}

\cppsection{Expressions}{8}{expr}

\begin{itemize}

\item Sizes of fundamental types \cppref{8.3.3}{expr.sizeof}{1}

Table~\ref{tab:cppfundamentaltypes} lists the size and value range of each fundamental type as defined by the \ecs{}.
The sizes of the types \texttt{int}, \texttt{unsigned int}, and \texttt{double} depend on the execution environment and are listed in Table~\ref{tab:cppdependenttypes} for each \cpp{} compiler provided by the \ecs{}.
The interpreter and all other tools reuse the respective type sizes of their own execution environment instead.

\begin{table}
\centering
\begin{tabular}{@{}llcl@{}}
\toprule Category & Type & Size & Value Range \\
\midrule Boolean
& \texttt{bool} & 1 & \texttt{false} or \texttt{true} \\
\midrule Character
& \texttt{char} & 1 & $0$ to $255$ \\
& \texttt{signed char} & 1 & $-128$ to $+127$ \\
& \texttt{unsigned char} & 1 & $0$ to $255$ \\
& \texttt{char16_t} & 2 & $0$ to $65\,535$ \\
& \texttt{char32_t} & 4 & $0$ to $4\,294\,967\,295$ \\
& \texttt{wchar_t} & 4 & $0$ to $4\,294\,967\,295$ \\
\midrule Integer
& \texttt{short int} & 2 & $-2^{15}$ to $+2^{15}-1$ \\
& \texttt{unsigned short int} & 2 & $0$ to $2^{16}-1$ \\
& \texttt{int} & $2/4$ & \emph{See Table~\ref{tab:cppdependenttypes}} \\
& \texttt{unsigned int} & $2/4$ & \emph{See Table~\ref{tab:cppdependenttypes}} \\
& \texttt{long int} & 4 & $-2^{31}$ to $+2^{31}-1$ \\
& \texttt{unsigned long int} & 4 & $0$ to $2^{32}-1$ \\
& \texttt{long long int} & 8 & $-2^{63}$ to $+2^{63}-1$ \\
& \texttt{unsigned long long int} & 8 & $0$ to $2^{64}-1$ \\
\midrule Floating-Point
& \texttt{float} & 4 & $\pm 3.4028234 \times 10^{38}$ \\
& \texttt{double} & $4/8$ & \emph{See Table~\ref{tab:cppdependenttypes}} \\
& \texttt{long double} & 8 & $\pm 1.7976931348623157 \times 10^{308}$ \\
\bottomrule
\end{tabular}
\caption{Sizes and value ranges of fundamental \cpp{} types}
\label{tab:cppfundamentaltypes}
\end{table}

\begin{table}
\centering
\begin{tabular}{@{}lrlccc@{}}
\toprule \multicolumn{2}{@{}l}{Hardware} & \cpp{} & \texttt{int} & & pointer types \\ \multicolumn{2}{@{}l}{Architecture} & Compiler & \texttt{unsigned int} & \texttt{double} & \texttt{std::size\_t} \\
\midrule AMD64 & 16-bit & \tool{cppamd16} & 2 & 8 & 2 \\ & 32-bit & \tool{cppamd32} & 4 & 8 & 4 \\ & 64-bit & \tool{cppamd64} & 4 & 8 & 8 \\
\midrule ARM & A32 & \tool{cpparma32} & 4 & 8 & 4 \\ & A64 & \tool{cpparma64} & 4 & 8 & 8 \\ & T32 & \tool{cpparmt32} & 4 & 4 & 4 \\ & & \tool{cpparmt32fpe} & 4 & 8 & 4 \\
\midrule \multicolumn{2}{@{}l}{AVR} & \tool{cppavr} & 2 & 4 & 2 \\
\midrule \multicolumn{2}{@{}l}{AVR32} & \tool{cppavr32} & 4 & 4 & 4 \\
\midrule \multicolumn{2}{@{}l}{M68000} & \tool{cppm68k} & 2 & 4 & 4 \\
\midrule \multicolumn{2}{@{}l}{MicroBlaze} & \tool{cppmibl} & 4 & 4 & 4 \\
\midrule MIPS & 32-bit & \tool{cppmips32} & 4 & 8 & 4 \\ & 64-bit & \tool{cppmips64} & 4 & 8 & 8 \\
\midrule \multicolumn{2}{@{}l}{MMIX} & \tool{cppmmix} & 4 & 8 & 8 \\
\midrule \multicolumn{2}{@{}l}{OpenRISC 1000} & \tool{cppor1k} & 4 & 4 & 4 \\
\midrule PowerPC & 32-bit & \tool{cppppc32} & 4 & 4 & 4 \\ & 64-bit & \tool{cppppc64} & 4 & 4 & 4 \\
\midrule \multicolumn{2}{@{}l}{RISC} & \tool{cpprisc} & 4 & 4 & 4 \\
\midrule \multicolumn{2}{@{}l}{WebAssembly} & \tool{cppwasm} & 4 & 8 & 4 \\
\bottomrule
\end{tabular}
\caption{Sizes of hardware-dependent \cpp{} types}
\label{tab:cppdependenttypes}
\end{table}

\item Resulting type of pointer subtractions \cppref{8.7}{expr.add}{5}

The type of the result of subtracting two pointer values is the signed counterpart of \texttt{std::size_t}.

\item Result of right-shifting negative values \cppref{8.8}{expr.shift}{3}

Shift operations on signed types are arithmetic.
Shifting a negative value to the right extends its sign.

\end{itemize}

\cppsection{Declarations}{10}{dcl.dcl}\label{sec:cppdeclarations}

\begin{itemize}

\item Meaning of attribute declarations \cppref{10}{dcl.dcl}{3}

Attribute declarations appertain to the current translation unit and may contain the standard \texttt{deprecated} attribute which applies to included headers and source files.

\item The \texttt{asm} declaration \cppref{10.4}{dcl.asm}{1}

The \texttt{asm} declaration is supported and allows writing inline assembly code using one of the various compilers for the \cpp{} programming language.
All of them pass the string literal of the \texttt{asm} declaration to the assembler used to generate the machine code.
The available instruction set therefore depends on the actual compiler used to compile the source code.
\seeassembly
The interpreter does not support executing \texttt{asm} declarations.

The assembly code of an \texttt{asm} declaration appearing at namespace scope has unrestricted access to all features of the assembler.
It must therefore first create a code or data section before any other directive or instruction can be used.
At block scope, the inline assembly code is part of a code section predefined by the compiler which also provides constant definitions for all labels, variables, parameters, enumerations and their enumerators that are accessible from that scope.
The name of a label is predefined as the address of the corresponding labeled statement and can therefore be used as the target of branch instructions.
The names of all parameters and variables with automatic storage duration evaluate to the offset of the corresponding entity relative to the frame pointer.
The name of a variable with static storage duration evaluates to its address.
If a variable or parameter is declared using the register attribute, its name is an alias for the corresponding processor register.
The name of an enumerator or a non-volatile constant variable initialized with a constant expression of fundamental or pointer type is predefined to its value.
For debugging purposes, all names predefined in inline assembly code and their actual values are accessible using the expression evaluation directive.

\item Language linkages \cppref{10.5}{dcl.link}{2}

Besides the two language linkages \texttt{"C"} and \texttt{"C++"} required by the \cpp{} Standard, the \ecs{} also supports the \texttt{"Oberon"} language linkage.
It allows accessing global procedures and variables defined in Oberon modules.
In order to identify the containing module, the corresponding functions and objects declared with this language linkage must be members of a named namespace.
\seeoberon

\item Non-standard attributes \cppref{10.6.1}{dcl.attr.grammar}{6}

Besides the attributes specified in the \cpp{} Standard, the \ecs{} also recognizes the following non-standard attributes.
All of them may only appear once in an attribute list and cannot be used as pack expansions.
Except where otherwise noted, they apply only to functions and variables with static storage duration and do not accept arguments:

\begin{itemize}

\item Alias attribute\alignright\syntax{"ecs::alias" "(" <string-literal> ")"}\nopagebreak

The alias attribute specifies that the corresponding entity may also be accessible in assembly code or other programming languages using the name given in the non-empty string literal.

\item Code attribute\alignright\syntax{"ecs::code"}\nopagebreak

The code attribute applies to \texttt{asm} declarations and causes their string literal to be interpreted as intermediate code.
\seecode

\item Duplicable attribute\alignright\syntax{"ecs::duplicable"}\nopagebreak

The duplicable attribute specifies that linkers merge the definition of the corresponding entity with other entities that have the same definition.

\item Group attribute\alignright\syntax{"ecs::group" "(" <string-literal> ")"}\nopagebreak

The group attribute specifies that linkers place the corresponding entity adjacent to other entities that also belong to the group named in the non-empty string literal.

\item Origin attribute\alignright\syntax{"ecs::origin" "(" <constant-expression> ")"}\nopagebreak

The origin attribute specifies the desired address of the variable or function.
It requires an integral constant expression as argument that is convertible to \texttt{std::size\_t}.
This attribute may not be combined with alignment specifiers.

\item Register attribute\alignright\syntax{"ecs::register"}\nopagebreak

The register attribute applies to up to four non-volatile parameters and variables with automatic storage duration of fundamental or pointer type and specifies that compilers store the corresponding variable or parameter in a register.

\item Replaceable attribute\alignright\syntax{"ecs::replaceable"}\nopagebreak

The replaceable attribute specifies that linkers discard the definition of the corresponding entity if another entity with the same name exists.

\item Required attribute\alignright\syntax{"ecs::required"}\nopagebreak

The required attribute specifies that linkers do not omit the definition of the corresponding entity even if it is never actually used in a program.

\end{itemize}

Any entity declared with one of these attributes can later be redeclared without that attribute and vice versa.
However, redeclarations and declarations in other translation units using different forms of the same attribute are not allowed.

\item Noreturn \texttt{asm} declarations \cppref{9.12.10}{dcl.attr.noreturn}{1}

The noreturn attribute may be applied to \texttt{asm} declarations at block scope and specifies that the corresponding assembly code does not return.

\end{itemize}

\cppsection{Declarators}{11}{dcl.decl}

\begin{itemize}

\item String value of \texttt{\_\_func\_\_} \cppref{11.4.1}{dcl.fct.def.general}{8}

The string resulting from the function-local predefined variable \texttt{\_\_func\_\_} matches the fully qualified section name of the function as described in Section~\ref{sec:cppnamingconventions}.

\end{itemize}

\cppsection{Classes}{12}{class}

\begin{itemize}

\item Allocation and alignment of bit-fields \cppref{12.2.4}{class.bit}{1}

Bit-fields are allocated consecutively and packed adjacent to each other, as long as they do not straddle allocation units and the alignment of their respective types does not require padding bits.

\end{itemize}

\cppsection{Preprocessing Directives}{19}{cpp}

\begin{itemize}

\item Additional preprocessing directives \cppref{19}{cpp}{2}

The \ecs{} does not support additional preprocessing directives.

\item Interpretation of character literals \cppref{19.1}{cpp.cond}{10}

The numeric value for character literals within a \texttt{\#if} or \texttt{\#elif} directive is non-negative and matches the value obtained when an identical character literal occurs in an expression.

\item Source file inclusion \cppref{19.2}{cpp.include}{2/3}

Source files identified between a pair of~\texttt{"} delimiters in \texttt{\#include} directives are searched relative to the directory of the current source file.
Headers identified between the~\texttt{<} and~\texttt{>} delimiters are searched in the relative directory given by the \environmentvariable{ECSINCLUDE} environment variable.
Two or more directories can be separated by semicolons and are searched in order.
Each directory must include a trailing path separator.

\item Combination of preprocessing tokens into one header name token \cppref{19.2}{cpp.include}{4}

A pair of~\texttt{"} delimiters is treated as a string literal token.
All preprocessing tokens between~\texttt{<} and~\texttt{>} delimiters are concatenated to form a single header name token.
Identifiers therein are subject to macro replacement.

\item Pragma directive \cppref{19.6}{cpp.pragma}{1}

The \ecs{} recognizes the \texttt{\#pragma end} pragma directive which simulates the end of a source file.
The remaining part of the source file following this pragma directive is completely ignored.
All other pragma directives are ignored.

\item Predefined macro names \cppref{19.8}{cpp.predefined}{1}

If the date of translation is not available, the macros \texttt{\_\_DATE\_\_} and \texttt{\_\_TIME\_\_} are predefined as \texttt{"May 10 2023"} and \texttt{"00:00:00"} respectively.
The macro \texttt{\_\_STD\_HOSTED\_\_} is defined as~\texttt{1} since the \cpp{} implementation of the \ecs{} is hosted.

\item Conditionally-defined macro names \cppref{16.8}{cpp.predefined}{2}

The macros \texttt{\_\_STDC\_\_} and \texttt{\_\_STDC\_VERSION\_\_} are not predefined.

\item Implementation-defined macro names \cppref{16.8}{cpp.predefined}{4}

The macro \texttt{\_\_ecs\_\_} is predefined in order to enable programmers to detect the \ecs{} while processing compiler-specific source code.
The value of the predefined macro \texttt{\_\_sizeof\_}\textit{type}\texttt{\_\_} where \textit{type} is either \texttt{double}, \texttt{float}, \texttt{int}, \texttt{long}, \texttt{long_double}, \texttt{pointer}, or \texttt{short} is the size of the corresponding type.
The interpreter and all compilers predefine an additional macro name of the form \texttt{\_\_}\textit{target}\texttt{\_\_} for identifying the target environment where \textit{target} is either \texttt{amd16}, \texttt{amd32}, \texttt{amd64}, \texttt{arma32}, \texttt{arma64}, \texttt{armt32}, \texttt{armt32fpe}, \texttt{avr}, \texttt{avr32}, \texttt{code}, \texttt{m68k}, \texttt{mibl}, \texttt{mips32}, \texttt{mips64}, \texttt{mmix}, \texttt{or1k}, \texttt{ppc32}, \texttt{ppc64}, \texttt{risc}, \texttt{run}, or \texttt{wasm}.

\end{itemize}

\cppsection{Library Introduction}{20}{library}

\begin{itemize}

\item Declaration of additional functions from the standard C library \cppref{20.5.1.2}{headers}{9}

The functions described in Annex K of the C standard are not declared when including \cpp{} headers.

\item Freestanding implementations \cppref{20.5.1.3}{compliance}{2}

The \ecs{} provides an implementation of \cpp{} that can be used in hosted as well as freestanding environments.
In both cases, the complete set of headers described by the \cpp{} Standard is available.

\item Linkage of names from the standard C library \cppref{20.5.2.3}{using.linkage}{2}

Names from the C standard library declared with external linkage have \texttt{extern "C"} linkage.

\end{itemize}

\cppsection{Language Support Library}{21}{language.support}

\begin{itemize}

\item Definition of macro \texttt{NULL} \cppref{21.2.3}{support.types.nullptr}{2}

The macro \texttt{NULL} is defined as \texttt{nullptr}.

\end{itemize}

\cppsection{Implementation Quantities}{B}{implimits}

According the \cpp{} Standard, each implementation shall document the limitations of the programs they can successfully process.
This section lists all known limitations of the \ecs{}:

\begin{itemize}

\item Size of an object \cppref{6.9.2}{basic.compound}{2}

The size of an object is limited by the maximal value representable in the type \texttt{std::size\_t}.

\item Nesting levels for \texttt{\#include} files \cppref{19.2}{cpp.include}{6}

The nesting level of \texttt{\#include} directives has a limit of 256 in order to detect potentially infinite recursions.

\item Functions registered by \texttt{atexit()} \cppref{21.5}{support.start.term}{6}

The \ecs{} supports the registration of at most 32 functions.

\item Functions registered by \texttt{at_quick_exit()} \cppref{21.5}{support.start.term}{10}

The \ecs{} supports the registration of at most 32 functions.

\item Recursive \texttt{constexpr} function invocations \cppref{8.20}{expr.const}{2}

The total number of nested calls of \texttt{constexpr} functions is limited to 512.

\item Recursively nested template instantiations \cppref{17.7.1}{temp.inst}{15}

The total depth of recursive template instantiations, including substitution during template argument deduction, has a limit of 1024 in order to detect potentially infinite recursions.

\item Number of placeholders \cppref{23.14.11.4}{func.bind.place}{1}

The implementation-defined number of placeholders $M$ is 10.

\end{itemize}

All other quantities listed in Annex~B of the \cpp{} Standard that are not mentioned above have no intrinsic limit and are only restricted by available memory.
The actual limit depends therefore only on the execution environment of the tool used to process the \cpp{} source file.

\section{The Standard \cpp{} Library}

The \ecs{} provides its own implementation of the \cpp{} standard library called the \emph{Standard \cpp{} Library}\index{Standard C++ Library@Standard \cpp{} Library}\index{Libraries!Standard C++ Library@Standard \cpp{} Library}.
Its facilities are made available by including one or more of its headers and providing the necessary runtime support as described in Section~\ref{sec:cppruntimesupport}.
All of these headers are governed by the \rse{} which is an additional permission to the \gpl{} that allows users of the \ecs{} to create proprietary programs.
\ifbook Copies of these licenses are included in Appendices~\ref{gpl} and~\ref{rse} on pages~\pageref{gpl} and~\pageref{rse} respectively. \fi

\input{cpplibrary.doc}

\section{Documentation Generation}

The \ecs{} provides several tools that are able to extract the structure of programs written in \cpp{} and generate documentations for them.
This section describes the contents of the extracted information and explains how programmers can provide a user-defined description of it.

\ifbook\else\markuptable\fi

\section{Runtime Support}\label{sec:cppruntimesupport}\index{Runtime support!for C++@for \cpp{}}

Some language features of \cpp{} such as exceptions require some additional runtime support.
This runtime support is stored in library files which are collections of object files. \seeobject
The \ecs{} provides the required runtime support in one library file for each hardware architecture it supports.
The name of the corresponding library file consists of a leading \file{cpp}, the name of the target hardware architecture, and a trailing \file{run} as in \file{cpp\-amd64\-run}.

\section{\cpp{} Tools}

The \ecs{} provides several different tools that process source files written in \cpp{}.
\interface

The tools process \cpp{} translation units in several consecutive stages.
In each stage, the internal representation of the translation unit is changed and transformed.
Figure~\ref{fig:cppdataflow} shows all stages and the different representations.

\begin{figure}
\flowgraph{
& \resource{\cpp{}\\source code} \ar[d] \\
\variable{ECSINCLUDE} \ar[r] & \converter{Preprocessor} \ar[d] \ar[r] & \resource{preprocessed\\source code} \\
& \resource{tokens} \ar[d] \\
& \converter{Parser} \ar[d] \\
\converter{Serializer} \ar[d] & \resource{abstract\\syntax tree} \ar[l] \ar[d] \ar[r] & \converter{Pretty Printer} \ar[d] \\
\resource{internal\\representation} & \converter{Semantic\\Checker} \ar[d] & \resource{reformatted\\source code} \\
\converter{Interpreter} \ar@/l/[d] & \resource{attributed\\syntax tree} \ar[l] \ar[d] \ar[r] & \converter{Documentation\\Extractor} \ar[d] \\
\resource{input/\\output} \ar@/r/[u] & \converter{Intermediate\\Code Emitter} \ar[d] & \resource{generic\\documentation} \\
& \resource{intermediate\\code} \ar[d] \ar@/u/[r] & \converter{Optimizer} \ar@/d/[l] \\
\resource{assembly\\listing} & \converter{Machine Code\\Generator} \ar[l] \ar[d] \ar[r] & \resource{debugging\\information} \\
& \resource{object file} \\
}\caption{Data flow within the tools for \cpp{}}
\label{fig:cppdataflow}
\end{figure}

\cppprep
\cppprint
\cppcheck
\cppdump
\cpprun
\cppdoc
\cpphtml
\cpplatex
\cppcode
\cppamda
\cppamdb
\cppamdc
\cpparma
\cpparmb
\cpparmc
\cpparmcfpe
\cppavr
\cppavrtt
\cppmabk
\cppmibl
\cppmipsa
\cppmipsb
\cppmmix
\cpporok
\cppppca
\cppppcb
\cpprisc
\cppwasm

\section{Interoperability}

In accordance with the goal of the \ecs{} to enable interoperability between its implemented programming languages,
the compilers for \cpp{} provide different mechanisms to exchange data with programs written with other tools of the \ecs{}.
The interoperability is enabled by a common intermediate code representation and calling convention. \seecode

This section describes the naming conventions used to uniquely identify the intermediate code sections defined by the compilers for \cpp{}
as well as the ways of accessing sections defined by other compilers and assemblers.

\subsection{Naming Conventions}\label{sec:cppnamingconventions}

The compilers for the \cpp{} programming language emit a code section for each function definition, function bodies of lambda expressions, and non-local variable with dynamic initialization.
Additional data sections are defined for non-local variables and type informations required at runtime.

The name of each section equals to the name of the corresponding entity and is prefixed by the name of its containing scope in case the entity does not have C language linkage.
The names of entities and their scope are delimited by scope operators such that names resemble qualified identifiers and no name mangling is necessary.

\subsection{Accessing Sections}

\cpp{} as implemented by the \ecs{} allows two different ways of accessing sections defined by other compilers or assemblers.

\begin{itemize}

\item
The \texttt{asm} declaration allows writing inline assembly code which naturally enables arbitrary access to any section, see Section~\ref{sec:cppdeclarations}.
\seeassembly

\item
The \texttt{extern} specifier allows referring to data and code sections that are defined elsewhere.
In cases where the different language linkages supported by the \ecs{} do not automatically yield the required section name for a suitably declared function or variable,
the alias attribute can be used to provide an arbitrary section name by hand, see Section~\ref{sec:cppdeclarations}.

\end{itemize}

\subsection{Program Execution}

The entry point of a program is represented by the \texttt{main} function.
Unless provided by any other programming language, it must be defined in one of the linked \cpp{} translation units.
Any dynamic initialization of global variables with static storage duration is executed before the \texttt{main} function.
The destruction of variables with static storage duration on the other hand requires an explicit or implicit call of the standard \texttt{exit} function.

\concludechapter

% User manual for FALSE
% Copyright (C) Florian Negele

% This file is part of the Eigen Compiler Suite.

% Permission is granted to copy, distribute and/or modify this document
% under the terms of the GNU Free Documentation License, Version 1.3
% or any later version published by the Free Software Foundation.

% You should have received a copy of the GNU Free Documentation License
% along with the ECS.  If not, see <https://www.gnu.org/licenses/>.

% Generic documentation utilities
% Copyright (C) Florian Negele

% This file is part of the Eigen Compiler Suite.

% Permission is granted to copy, distribute and/or modify this document
% under the terms of the GNU Free Documentation License, Version 1.3
% or any later version published by the Free Software Foundation.

% You should have received a copy of the GNU Free Documentation License
% along with the ECS.  If not, see <https://www.gnu.org/licenses/>.

\providecommand{\cpp}{C\texttt{++}}
\providecommand{\opt}{_\mathit{opt}}
\providecommand{\tool}[1]{\texttt{#1}}
\providecommand{\version}{Version 0.0.40}
\providecommand{\resource}[1]{*++\txt{#1}}
\providecommand{\ecs}{Eigen Compiler Suite}
\providecommand{\changed}[1]{\underline{#1}}
\providecommand{\toolbox}[1]{\converter{#1}}
\providecommand{\file}{}\renewcommand{\file}[1]{\texttt{#1}}
\providecommand{\alignright}{\hfill\linebreak[0]\hspace*{\fill}}
\providecommand{\converter}[1]{*++[F][F*:white][F,:gray]\txt{#1}}
\providecommand{\documentation}{\ifbook chapter\else document\fi}
\providecommand{\Documentation}{\ifbook Chapter\else Document\fi}
\providecommand{\variable}[1]{\resource{\texttt{\small#1}\\variable}}
\providecommand{\documentationref}[2]{\ifbook\ref{#1}\else``\href{#1}{#2}''~\cite{#1}\fi}
\providecommand{\objfile}[1]{\texttt{#1}\index[runtime]{#1 object file@\texttt{#1} object file}}
\providecommand{\libfile}[1]{\texttt{#1}\index[runtime]{#1 library file@\texttt{#1} library file}}
\providecommand{\epigraph}[2]{\ifbook\begin{quote}\flushright\textit{#1}\par--- #2\end{quote}\fi}
\providecommand{\environmentvariable}[1]{\texttt{#1}\index{Environment variables!#1@\texttt{#1}}}
\providecommand{\environment}[1]{\texttt{#1}\index[environment]{#1 environment@\texttt{#1} environment}}
\providecommand{\toolsection}{}\renewcommand{\toolsection}[1]{\subsection{#1}\label{\prefix:#1}\tool{#1}}
\providecommand{\instruction}{}\renewcommand{\instruction}[2]{\noindent\qquad\pdftooltip{\texttt{#1}}{#2}\refstepcounter{instruction}\par}
\providecommand{\flowgraph}{}\renewcommand{\flowgraph}[1]{\par\sffamily\begin{displaymath}\xymatrix@=4ex{#1}\end{displaymath}\normalfont\par}
\providecommand{\instructionset}{}\renewcommand{\instructionset}[4]{\setcounter{instruction}{0}\begin{multicols}{\ifbook#3\else#4\fi}[{\captionof{table}[#2]{#2 (\ref*{#1:instructions}~instructions)}\label{tab:#1set}\vspace{-2ex}}]\footnotesize\raggedcolumns\input{#1.set}\label{#1:instructions}\end{multicols}}

\providecommand{\gpl}{GNU General Public License}
\providecommand{\rse}{ECS Runtime Support Exception}
\providecommand{\fdl}{\href{https://www.gnu.org/licenses/fdl.html}{GNU Free Documentation License}}

\providecommand{\docbegin}{}
\providecommand{\docend}{}
\providecommand{\doclabel}[1]{\hypertarget{#1}}
\providecommand{\doclink}[2]{\hyperlink{#1}{#2}}
\providecommand{\docsection}[3]{\hypertarget{#1}{\subsection{#2}}\label{sec:#1}\index[library]{#2@#3}}
\providecommand{\docsectionstar}[1]{}
\providecommand{\docsubbegin}{\begin{description}}
\providecommand{\docsubend}{\end{description}}
\providecommand{\docsubsection}[3]{\item[\hypertarget{#1}{#2}]\index[library]{#2@#3}}
\providecommand{\docsubsectionstar}[1]{\smallskip}
\providecommand{\docsubsubsection}[3]{\docsubsection{#1}{#2}{#3}}
\providecommand{\docsubsubsectionstar}[1]{}
\providecommand{\docsubsubsubsection}[3]{}
\providecommand{\docsubsubsubsectionstar}[1]{}
\providecommand{\doctable}{}

\providecommand{\debuggingtool}{}\renewcommand{\debuggingtool}{This tool is provided for debugging purposes.
It allows exposing and modifying an internal data structure that is usually not accessible.
}

\providecommand{\interface}{All tools accept command-line arguments which are taken as names of plain text files containing the source code.
If no arguments are provided, the standard input stream is used instead.
Output files are generated in the current working directory and have the same name as the input file being processed whereas the filename extension gets replaced by an appropriate suffix.
\seeinterface
}

\providecommand{\license}{\noindent Copyright \copyright{} Florian Negele\par\medskip\noindent
Permission is granted to copy, distribute and/or modify this document under the terms of the
\fdl{}, Version 1.3 or any later version published by the \href{https://fsf.org/}{Free Software Foundation}.
}

\providecommand{\ecslogosurface}{
\fill[darkgray] (0,0,0) -- (0,0,3) -- (0,3,3) -- (0,3,1) -- (0,4,1) -- (0,4,3) -- (0,5,3) -- (0,5,0) -- (0,2,0) -- (0,2,2) -- (0,1,2) -- (0,1,0) -- cycle;
\fill[gray] (0,5,0) -- (0,5,3) -- (1,5,3) -- (1,5,1) -- (2,5,1) -- (2,5,3) -- (3,5,3) -- (3,5,0) -- cycle;
\fill[lightgray] (0,0,0) -- (0,1,0) -- (2,1,0) -- (2,4,0) -- (1,4,0) -- (1,3,0) -- (2,3,0) -- (2,2,0) -- (0,2,0) -- (0,5,0) -- (3,5,0) -- (3,0,0) -- cycle;
\begin{scope}[line width=0.5]
\begin{scope}[gray]
\draw (0,0,0) -- (0,1,0);
\draw (2,1,0) -- (2,2,0);
\draw (0,1,2) -- (0,2,2);
\draw (0,2,0) -- (0,5,0);
\draw (2,3,0) -- (2,4,0);
\end{scope}
\begin{scope}[lightgray]
\draw (0,1,0) -- (0,1,2);
\draw (0,3,1) -- (0,3,3);
\draw (0,5,0) -- (0,5,3);
\draw (2,5,1) -- (2,5,3);
\end{scope}
\begin{scope}[white]
\draw (0,1,0) -- (2,1,0);
\draw (1,3,0) -- (2,3,0);
\draw (0,5,0) -- (3,5,0);
\end{scope}
\end{scope}
}

\providecommand{\ecslogo}[1]{
\begin{tikzpicture}[scale={(#1)/((sin(45)+cos(45))*3cm)},x={({-cos(45)*1cm},{sin(45)*sin(30)*1cm})},y={({0cm},{(cos(30)*1cm})},z={({sin(45)*1cm},{cos(45)*sin(30)*1cm})}]
\begin{scope}[darkgray,line width=1]
\draw (0,0,0) -- (0,0,3) -- (0,3,3) -- (2,3,3) -- (2,5,3) -- (3,5,3) -- (3,5,0) -- (3,0,0) -- cycle;
\draw (0,3,1) -- (0,4,1) -- (0,4,3) -- (0,5,3) -- (1,5,3) -- (1,5,1) -- (2,5,1);
\draw (1,3,0) -- (1,4,0) -- (2,4,0);
\end{scope}
\fill[darkgray] (2,0,0) -- (2,0,3) -- (2,5,3) -- (2,5,1) -- (2,4,1) -- (2,4,0) -- cycle;
\fill[lightgray] (2,0,2) -- (0,0,2) -- (0,2,2) -- (2,2,2) -- cycle;
\fill[gray] (0,1,0) -- (2,1,0) -- (2,1,2) -- (0,1,2) -- cycle;
\fill[gray] (0,3,1) -- (0,3,3) -- (2,3,3) -- (2,3,0) -- (1,3,0) -- (1,3,1) -- cycle;
\ecslogosurface
\end{tikzpicture}
}

\providecommand{\shadowedecslogo}[3]{
\begin{tikzpicture}[scale={(#1)/((sin(#2)+cos(#2))*3cm)},x={({-cos(#2)*1cm},{sin(#2)*sin(#3)*1cm})},y={({0cm},{(cos(#3)*1cm})},z={({sin(#2)*1cm},{cos(#2)*sin(#3)*1cm})}]
\shade[top color=lightgray!50!white,bottom color=white,middle color=lightgray!50!white] (0,0,0) -- (3,0,0) -- (3,{-0.5-3*sin(#2)*sin(#3)/cos(#3)},0) -- (0,-0.5,0) -- cycle;
\shade[top color=darkgray!50!gray,bottom color=white,middle color=darkgray!50!white] (0,0,0) -- (0,0,3) -- (0,{-0.5-3*cos(#2)*sin(#3)/cos(#3)},3) -- (0,-0.5,0) -- cycle;
\begin{scope}[y={({(cos(#2)+sin(#2))*0.5cm},{(cos(#2)*sin(#3)-sin(#2)*sin(#3))*0.5cm})}]
\useasboundingbox (3,0,0) -- (0,0,0) -- (0,0,3);
\shade[left color=darkgray!80!black,right color=lightgray,middle color=gray] (0,0,0) -- (0,1,0) -- (0,1,0.5) -- (0,2,0) -- (0,5,0) -- (0,5,3) -- (1,5,3) -- (1,4,3) -- (1,4,2.5) -- (1,3,3) -- (2,5,3) -- (3,5,3) -- (3,0,3) -- cycle;
\clip (0,0,0) -- (0,0,3) -- ({-3*sin(#2)/cos(#2)},0,0) -- cycle;
\shade[left color=darkgray,right color=lightgray!50!gray] (0,0,0) -- (0,1,0) -- (0,1,0.5) -- (0,2,0) -- (0,5,0) -- (0,5,3) -- (1,5,3) -- (1,4,3) -- (1,4,2.5) -- (1,3,3) -- (2,5,3) -- (3,5,3) -- (3,0,3) -- cycle;
\end{scope}
\shade[left color=darkgray,right color=darkgray!80!black] (2,0,0) -- (2,0,3) -- (2,5,3) -- (2,5,1) -- (2,4,1) -- (2,4,0) -- cycle;
\shade[left color=darkgray!90!black,right color=gray!80!darkgray] (2,0,2) -- (0,0,2) -- (0,2,2) -- (2,2,2) -- cycle;
\shade[top color=darkgray!90!black,bottom color=gray!80!darkgray] (0,1,0) -- (2,1,0) -- (2,1,2) -- (0,1,2) -- cycle;
\shade[top color=darkgray!90!black,bottom color=gray!80!darkgray] (0,3,1) -- (0,3,3) -- (2,3,3) -- (2,3,0) -- (1,3,0) -- (1,3,1) -- cycle;
\fill[gray] (2,1,0) -- (1.5,1,0.5) -- (0,1,0.5) -- (0,1,0) -- cycle;
\fill[gray] (1,3,2) -- (0.5,3,2) -- (0.5,3,3) -- (1,3,3) -- cycle;
\fill[gray] (2,3,0) -- (1.5,3,0.5) -- (1,3,0.5) -- (1,3,0) -- cycle;
\ecslogosurface
\end{tikzpicture}
}

\providecommand{\cpplogo}[1]{
\begin{tikzpicture}[scale=(#1)/512em]
\fill[gray] (435.2794,398.7159) -- (247.1911,507.3075) .. controls (236.3563,513.5642) and (218.6240,513.5642) .. (207.7892,507.3075) -- (19.7009,398.7159) .. controls (8.8646,392.4606) and (0.0000,377.1043) .. (0.0000,364.5924) -- (0.0000,147.4076) .. controls (0.8430,132.8363) and (8.2856,120.7683) .. (19.7009,113.2842) -- (207.7892,4.6926) .. controls (218.6240,-1.5642) and (236.3564,-1.5642) .. (247.1911,4.6926) -- (435.2794,113.2842) .. controls (447.5273,121.4304) and (454.4987,133.6918) .. (454.9803,147.4076) -- (454.9803,364.5924) .. controls (454.5404,377.7571) and (446.6566,391.0351) .. (435.2794,398.7159) -- cycle(75.8301,255.9993) .. controls (74.9389,404.0881) and (273.2892,469.4783) .. (358.8263,331.8769) -- (293.1917,293.8965) .. controls (253.5702,359.4301) and (155.1909,335.9977) .. (151.6601,255.9993) .. controls (152.7204,182.2703) and (249.4137,148.0211) .. (293.1961,218.1065) -- (358.8308,180.1276) .. controls (283.4477,49.2645) and (79.6318,96.3470) .. (75.8301,255.9993) -- cycle(379.1503,247.5747) -- (362.2982,247.5747) -- (362.2982,230.7226) -- (345.4490,230.7226) -- (345.4490,247.5747) -- (328.5969,247.5747) -- (328.5969,264.4254) -- (345.4490,264.4254) -- (345.4490,281.2759) -- (362.2982,281.2759) -- (362.2982,264.4254) -- (379.1503,264.4254) -- cycle(442.3420,247.5747) -- (425.4899,247.5747) -- (425.4899,230.7226) -- (408.6408,230.7226) -- (408.6408,247.5747) -- (391.7886,247.5747) -- (391.7886,264.4254) -- (408.6408,264.4254) -- (408.6408,281.2759) -- (425.4899,281.2759) -- (425.4899,264.4254) -- (442.3420,264.4254) -- cycle;
\end{tikzpicture}
}

\providecommand{\fallogo}[1]{
\begin{tikzpicture}[scale=(#1)/512em]
\fill[gray] (185.7774,0.0000) .. controls (200.4486,15.9798) and (226.8966,8.7148) .. (235.0426,31.5836) .. controls (249.5297,58.0598) and (247.9581,97.9161) .. (280.3335,110.9762) .. controls (309.1690,120.3496) and (337.8406,104.2727) .. (366.5753,103.9379) .. controls (373.4449,111.5171) and (379.2885,128.2574) .. (383.9755,108.9744) .. controls (396.6979,102.5615) and (437.2808,107.6681) .. (426.9652,124.3252) .. controls (408.9822,121.0785) and (412.4742,146.0729) .. (426.5192,131.4996) .. controls (433.8413,120.8489) and (465.1541,126.5522) .. (441.9067,135.7950) .. controls (396.1879,157.7478) and (344.1112,161.5079) .. (298.5528,183.5702) .. controls (277.7471,193.5198) and (284.6941,218.7163) .. (285.2127,236.9640) .. controls (292.3599,316.2826) and (307.3929,394.6311) .. (317.1198,473.6154) .. controls (329.0637,505.4736) and (292.1195,528.5004) .. (265.9183,511.2761) .. controls (237.9284,499.2462) and (237.3684,465.2681) .. (230.9102,439.9421) .. controls (218.6692,374.3397) and (215.6307,306.9662) .. (198.1732,242.3977) .. controls (183.1379,232.7444) and (164.4245,256.0298) .. (149.0430,261.4799) .. controls (116.9328,279.2585) and (87.1822,308.5851) .. (48.2293,307.8914) .. controls (21.3220,306.9037) and (-15.9107,281.8761) .. (7.2921,252.7908) .. controls (29.7799,220.6177) and (67.5177,204.3028) .. (100.9287,185.9449) .. controls (130.8217,170.8906) and (161.1548,156.5903) .. (191.0278,141.5847) .. controls (196.1738,120.0520) and (186.6049,95.2409) .. (186.8382,72.4353) .. controls (185.5234,48.4204) and (183.1700,23.9341) .. (185.7774,0.0000) -- cycle;
\end{tikzpicture}
}

\providecommand{\oblogo}[1]{
\begin{tikzpicture}[scale=(#1)/512em]
\fill[gray] (160.3865,208.9117) .. controls (154.0879,214.6478) and (149.0735,221.2409) .. (145.4125,228.5384) .. controls (184.8790,248.4273) and (234.7122,269.8787) .. (297.5493,291.8782) .. controls (300.3943,281.4769) and (300.9552,268.7619) .. (300.4023,255.2389) .. controls (248.9909,244.7891) and (200.0310,225.9279) .. (160.3865,208.9117) -- cycle(225.7398,392.6996) .. controls (308.0209,392.1716) and (359.3326,345.9277) .. (368.7203,285.2098) .. controls (376.6742,197.1784) and (311.7194,141.3342) .. (205.4287,142.1456) .. controls (139.9485,141.4804) and (88.7155,166.1957) .. (73.5775,228.0086) .. controls (52.0297,320.3408) and (123.4078,391.0103) .. (225.7398,392.6996) -- cycle(216.0739,176.4733) .. controls (268.9183,179.2424) and (315.8292,206.5488) .. (312.7454,265.1139) .. controls (313.2769,315.6384) and (286.5993,353.4946) .. (216.6040,355.7934) .. controls (162.4657,355.7934) and (126.0914,317.5023) .. (126.0914,260.5103) .. controls (126.1733,214.2900) and (163.3363,176.2849) .. (216.0739,176.4733) -- cycle(76.4897,189.1754) .. controls (13.1586,147.5631) and (0.0000,119.4207) .. (0.0000,119.4207) -- (90.6499,170.1632) .. controls (85.3004,175.8497) and (80.5994,182.1633) .. (76.4897,189.1754) -- cycle(353.9486,119.3004) -- (402.9482,119.3004) .. controls (427.0025,137.0797) and (450.9893,162.7034) .. (474.9529,191.0213) .. controls (509.3540,228.5339) and (531.3391,294.2091) .. (487.8149,312.1206) .. controls (462.8165,324.7652) and (394.3874,316.8943) .. (373.8912,313.6651) .. controls (379.9291,297.7449) and (383.2899,278.4204) .. (381.4989,257.7214) .. controls (420.3069,248.0321) and (421.9610,218.3461) .. (407.7867,192.6417) .. controls (391.1113,162.4018) and (370.1114,132.9097) .. (353.9486,119.3004) -- cycle;
\end{tikzpicture}
}

\providecommand{\markuptable}{
\begin{table}
\sffamily\centering
\begin{tabular}{@{}lcl@{}}
\toprule
\texttt{//italics//} & $\rightarrow$ & \textit{italics} \\
\midrule
\texttt{**bold**} & $\rightarrow$ & \textbf{bold} \\
\midrule
\texttt{\# ordered list} & & 1 ordered list \\
\texttt{\# second item} & $\rightarrow$ & 2 second item \\
\texttt{\#\# sub item} & & \hspace{1em} 1 sub item \\
\midrule
\texttt{* unordered list} & & $\bullet$ unordered list \\
\texttt{* second item} & $\rightarrow$ & $\bullet$ second item \\
\texttt{** sub item} & & \hspace{1em} $\bullet$ sub item \\
\midrule
\texttt{link to [[label]]} & $\rightarrow$ & link to \underline{label} \\
\midrule
\texttt{<{}<label>{}> definition } & $\rightarrow$ & definition \\
\midrule
\texttt{[[url|link name]]} & $\rightarrow$ & \underline{link name} \\
\midrule\addlinespace
\texttt{= large heading} & & {\Large large heading} \smallskip \\
\texttt{== medium heading} & $\rightarrow$ & {\large medium heading} \\
\texttt{=== small heading} & & small heading \\
\midrule
\texttt{no line break} & & no line break for paragraphs \\
\texttt{for paragraphs} & $\rightarrow$ \\
& & use empty line \\
\texttt{use empty line} \\
\midrule
\texttt{force\textbackslash\textbackslash line break} & $\rightarrow$ & force \\
& & line break \\
\midrule
\texttt{horizontal line} & $\rightarrow$ & horizontal line \\
\texttt{----} & & \hrulefill \\
\midrule
\texttt{|=a|=table|=header} & & \underline{a \enspace table \enspace header} \\
\texttt{|a|table|row} & $\rightarrow$ & a \enspace table \enspace row \\
\texttt{|b|table|row} & & b \enspace table \enspace row \\
\midrule
\texttt{\{\{\{} \\
\texttt{unformatted} & $\rightarrow$ & \texttt{unformatted} \\
\texttt{code} & & \texttt{code} \\
\texttt{\}\}\}} \\
\midrule\addlinespace
\texttt{@ new article} & & {\Large 1.\ new article} \smallskip \\
\texttt{@ second article} & $\rightarrow$ & {\Large 2.\ second article} \smallskip \\
\texttt{@@ sub article} & & {\large 2.1.\ sub article} \\
\bottomrule
\end{tabular}
\normalfont\caption{Elements of the generic documentation markup language}
\label{tab:docmarkup}
\end{table}
}

\providecommand{\startchapter}[4]{
\documentclass[11pt,a4paper]{article}
\usepackage{booktabs}
\usepackage[format=hang,labelfont=bf]{caption}
\usepackage{changepage}
\usepackage[T1]{fontenc}
\usepackage[margin=2cm]{geometry}
\usepackage{hyperref}
\usepackage[american]{isodate}
\usepackage{lmodern}
\usepackage{longtable}
\usepackage{mathptmx}
\usepackage{microtype}
\usepackage[toc]{multitoc}
\usepackage{multirow}
\usepackage[all]{nowidow}
\usepackage{pdfcomment}
\usepackage{syntax}
\usepackage{tikz}
\usepackage[all]{xy}
\hypersetup{pdfborder={0 0 0},bookmarksnumbered=true,pdftitle={\ecs{}: #2},pdfauthor={Florian Negele},pdfsubject={\ecs{}},pdfkeywords={#1}}
\setlength{\grammarindent}{8em}\setlength{\grammarparsep}{0.2ex}
\setlength{\columnsep}{2em}
\newcommand{\prefix}{}
\newcounter{instruction}
\bibliographystyle{unsrt}
\renewcommand{\index}[2][]{}
\renewcommand{\arraystretch}{1.05}
\renewcommand{\floatpagefraction}{0.7}
\renewcommand{\syntleft}{\itshape}\renewcommand{\syntright}{}
\title{\vspace{-5ex}\Huge{\ecs{}}\medskip\hrule}
\author{\huge{#2}}
\date{\medskip\version}
\newif\ifbook\bookfalse
\pagestyle{headings}
\frenchspacing
\begin{document}
\maketitle\thispagestyle{empty}\noindent#4\setlength{\columnseprule}{0.4pt}\tableofcontents\setlength{\columnseprule}{0pt}\vfill\pagebreak[3]\null\vfill\bigskip\noindent
\parbox{\textwidth-4em}{\license The contents of this \documentation{} are part of the \href{manual}{\ecs{} User Manual}~\cite{manual} and correspond to Chapter ``\href{manual\##3}{#1}''.\alignright\mbox{\today}}
\parbox{4em}{\flushright\ecslogo{3em}}
\clearpage
}

\providecommand{\concludechapter}{
\vfill\pagebreak[3]\null\vfill
\thispagestyle{myheadings}\markright{REFERENCES}
\noindent\begin{minipage}{\textwidth}\begin{multicols}{2}[\section*{References}]
\renewcommand{\section}[2]{}\small\bibliography{references}
\end{multicols}\end{minipage}\end{document}
}

\providecommand{\startpresentation}[2]{
\documentclass[14pt,aspectratio=43,usepdftitle=false]{beamer}
\usepackage{booktabs}
\usepackage{etex}
\usepackage{multicol}
\usepackage{tikz}
\usepackage[all]{xy}
\bibliographystyle{unsrt}
\setlength{\columnsep}{1em}
\setlength{\leftmargini}{1em}
\setbeamercolor{title}{fg=black}
\setbeamercolor{structure}{fg=darkgray}
\setbeamercolor{bibliography item}{fg=darkgray}
\setbeamerfont{title}{series=\bfseries}
\setbeamerfont{subtitle}{series=\normalfont}
\setbeamerfont*{frametitle}{parent=title}
\setbeamerfont{block title}{series=\bfseries}
\setbeamerfont*{framesubtitle}{parent=subtitle}
\setbeamersize{text margin left=1em,text margin right=1em}
\setbeamertemplate{navigation symbols}{}
\setbeamertemplate{itemize item}[circle]{}
\setbeamertemplate{bibliography item}[triangle]{}
\setbeamertemplate{bibliography entry author}{\usebeamercolor[fg]{bibliography item}}
\setbeamertemplate{frametitle}{\medskip\usebeamerfont{frametitle}\color{gray}\raisebox{-2.5ex}[0ex][0ex]{\rule{0.1em}{4.5ex}}}
\addtobeamertemplate{frametitle}{}{\hspace{0.4em}\usebeamercolor[fg]{title}\insertframetitle\par\vspace{0.2ex}\hspace{0.5em}\usebeamerfont{framesubtitle}\insertframesubtitle}
\hypersetup{pdfborder={0 0 0},bookmarksnumbered=true,bookmarksopen=true,bookmarksopenlevel=0,pdftitle={\ecs{}: #1},pdfauthor={Florian Negele},pdfsubject={\ecs{}},pdfkeywords={#1}}
\renewcommand{\flowgraph}[1]{\resizebox{\textwidth}{!}{$$\xymatrix{##1}$$}}
\title{\ecs{}\medskip\hrule\medskip}
\institute{\shadowedecslogo{5em}{30}{15}}
\date{\version}
\subtitle{#1}
\begin{document}
\begin{frame}[plain]\titlepage\nocite{manual}\end{frame}
\begin{frame}{Contents}{#1}\begin{center}\tableofcontents\end{center}\end{frame}
}

\providecommand{\concludepresentation}{
\begin{frame}{References}\begin{footnotesize}\setlength{\columnseprule}{0.4pt}\begin{multicols}{2}\bibliography{references}\end{multicols}\end{footnotesize}\end{frame}
\end{document}
}

\providecommand{\startbook}[1]{
\documentclass[10pt,paper=17cm:24cm,DIV=13,twoside=semi,headings=normal,numbers=noendperiod,cleardoublepage=plain]{scrbook}
\usepackage{atveryend}
\usepackage{booktabs}
\usepackage{caption}
\usepackage{changepage}
\usepackage[T1]{fontenc}
\usepackage{imakeidx}
\usepackage{hyperref}
\usepackage[american]{isodate}
\usepackage{lmodern}
\usepackage{longtable}
\usepackage{mathptmx}
\usepackage[final]{microtype}
\usepackage{multicol}
\usepackage{multirow}
\usepackage[all]{nowidow}
\usepackage{pdfcomment}
\usepackage{scrlayer-scrpage}
\usepackage{setspace}
\usepackage{syntax}
\usepackage[eventxtindent=4pt,oddtxtexdent=4pt]{thumbs}
\usepackage{tikz}
\usepackage[all]{xy}
\hyphenation{Micro-Blaze Open-Cores Open-RISC Power-PC}
\hypersetup{pdfborder={0 0 0},bookmarksnumbered=true,bookmarksopen=true,bookmarksopenlevel=0,pdftitle={\ecs{}: #1},pdfauthor={Florian Negele},pdfsubject={\ecs{}},pdfkeywords={#1}}
\setlength{\grammarindent}{8em}\setlength{\grammarparsep}{0.7ex}
\setkomafont{captionlabel}{\usekomafont{descriptionlabel}}
\renewcommand{\arraystretch}{1.05}\setstretch{1.1}
\renewcommand{\chapterformat}{\thechapter\autodot\enskip\raisebox{-1ex}[0ex][0ex]{\color{gray}\rule{0.1em}{3.5ex}}\enskip}
\renewcommand{\startchapter}[4]{\hypertarget{##3}{\chapter{##1}}\label{##3}##4\addthumb{##1}{\LARGE\sffamily\bfseries\thechapter}{white}{gray}\renewcommand{\prefix}{##3}}
\renewcommand{\concludechapter}{\clearpage{\stopthumb\cleardoublepage}}
\renewcommand{\syntleft}{\itshape}\renewcommand{\syntright}{}
\renewcommand{\floatpagefraction}{0.7}
\renewcommand{\partheademptypage}{}
\DeclareMicrotypeAlias{lmss}{cmr}
\newcommand{\prefix}{}
\newcounter{instruction}
\bibliographystyle{unsrt}
\newif\ifbook\booktrue
\makeindex[intoc,title=Index]
\makeindex[intoc,name=tools,title=Index of Tools,columns=3]
\makeindex[intoc,name=library,title=Index of Library Names]
\makeindex[intoc,name=runtime,title=Index of Runtime Support]
\makeindex[intoc,name=environment,title=Index of Target Environments]
\indexsetup{toclevel=chapter,headers={\indexname}{\indexname}}
\frenchspacing
\begin{document}
\pagenumbering{alph}
\begin{titlepage}\centering
\huge\sffamily\null\vfill\textbf{\ecs{}}\bigskip\hrule\bigskip#1
\normalsize\normalfont\vfill\vfill\shadowedecslogo{10em}{30}{15}
\large\vfill\vfill\version
\end{titlepage}
\null\vfill
\thispagestyle{empty}
\noindent\today\par\medskip
\license A copy of this license is included in Appendix~\ref{fdl} on page~\pageref{fdl}.
All product names used herein are for identification purposes only and may be trademarks of their respective companies.
\concludechapter
\frontmatter
\setcounter{tocdepth}{1}
\tableofcontents
\setcounter{tocdepth}{2}
\concludechapter
\listoffigures
\concludechapter
\listoftables
\concludechapter
}

\providecommand{\concludebook}{
\backmatter
\addtocontents{toc}{\protect\setcounter{tocdepth}{-1}}
\phantomsection\addcontentsline{toc}{part}{Bibliography}
\bibliography{references}
\concludechapter
\phantomsection\addcontentsline{toc}{part}{Indexes}
\printindex
\concludechapter
\indexprologue{\label{idx:tools}}
\printindex[tools]
\concludechapter
\printindex[library]
\concludechapter
\indexprologue{\label{idx:runtime}}
\printindex[runtime]
\concludechapter
\indexprologue{\label{idx:environment}}
\printindex[environment]
\concludechapter
\pagestyle{empty}\pagenumbering{Alph}\null\clearpage
\null\vfill\centering\ecslogo{4em}\par\medskip\license
\end{document}
}

% chapter references

\providecommand{\seedocumentationref}{}\renewcommand{\seedocumentationref}[3]{#1, see \Documentation{}~\documentationref{#2}{#3}. }
\providecommand{\seeinterface}{}\renewcommand{\seeinterface}{\ifbook See \Documentation{}~\documentationref{interface}{User Interface} for more information about the common user interface of all of these tools. \fi}
\providecommand{\seeguide}{}\renewcommand{\seeguide}{\seedocumentationref{For basic examples of using some of these tools in practice}{guide}{User Guide}}
\providecommand{\seecpp}{}\renewcommand{\seecpp}{\seedocumentationref{For more information about the \cpp{} programming language and its implementation by the \ecs{}}{cpp}{User Manual for \cpp{}}}
\providecommand{\seefalse}{}\renewcommand{\seefalse}{\seedocumentationref{For more information about the FALSE programming language and its implementation by the \ecs{}}{false}{User Manual for FALSE}}
\providecommand{\seeoberon}{}\renewcommand{\seeoberon}{\seedocumentationref{For more information about the Oberon programming language and its implementation by the \ecs{}}{oberon}{User Manual for Oberon}}
\providecommand{\seeassembly}{}\renewcommand{\seeassembly}{\seedocumentationref{For more information about the generic assembly language and how to use it}{assembly}{Generic Assembly Language Specification}}
\providecommand{\seeamd}{}\renewcommand{\seeamd}{\seedocumentationref{For more information about how the \ecs{} supports the AMD64 hardware architecture}{amd64}{AMD64 Hardware Architecture Support}}
\providecommand{\seearm}{}\renewcommand{\seearm}{\seedocumentationref{For more information about how the \ecs{} supports the ARM hardware architecture}{arm}{ARM Hardware Architecture Support}}
\providecommand{\seeavr}{}\renewcommand{\seeavr}{\seedocumentationref{For more information about how the \ecs{} supports the AVR hardware architecture}{avr}{AVR Hardware Architecture Support}}
\providecommand{\seeavrtt}{}\renewcommand{\seeavrtt}{\seedocumentationref{For more information about how the \ecs{} supports the AVR32 hardware architecture}{avr32}{AVR32 Hardware Architecture Support}}
\providecommand{\seemabk}{}\renewcommand{\seemabk}{\seedocumentationref{For more information about how the \ecs{} supports the M68000 hardware architecture}{m68k}{M68000 Hardware Architecture Support}}
\providecommand{\seemibl}{}\renewcommand{\seemibl}{\seedocumentationref{For more information about how the \ecs{} supports the MicroBlaze hardware architecture}{mibl}{MicroBlaze Hardware Architecture Support}}
\providecommand{\seemips}{}\renewcommand{\seemips}{\seedocumentationref{For more information about how the \ecs{} supports the MIPS32 and MIPS64 hardware architectures}{mips}{MIPS Hardware Architecture Support}}
\providecommand{\seemmix}{}\renewcommand{\seemmix}{\seedocumentationref{For more information about how the \ecs{} supports the MMIX hardware architecture}{mmix}{MMIX Hardware Architecture Support}}
\providecommand{\seeorok}{}\renewcommand{\seeorok}{\seedocumentationref{For more information about how the \ecs{} supports the OpenRISC 1000 hardware architecture}{or1k}{OpenRISC 1000 Hardware Architecture Support}}
\providecommand{\seeppc}{}\renewcommand{\seeppc}{\seedocumentationref{For more information about how the \ecs{} supports the PowerPC hardware architecture}{ppc}{PowerPC Hardware Architecture Support}}
\providecommand{\seerisc}{}\renewcommand{\seerisc}{\seedocumentationref{For more information about how the \ecs{} supports the RISC hardware architecture}{risc}{RISC Hardware Architecture Support}}
\providecommand{\seewasm}{}\renewcommand{\seewasm}{\seedocumentationref{For more information about how the \ecs{} supports the WebAssembly architecture}{wasm}{WebAssembly Architecture Support}}
\providecommand{\seedocumentation}{}\renewcommand{\seedocumentation}{\seedocumentationref{For more information about generic documentations and their generation by the \ecs{}}{documentation}{Generic Documentation Generation}}
\providecommand{\seedebugging}{}\renewcommand{\seedebugging}{\seedocumentationref{For more information about debugging information and its representation}{debugging}{Debugging Information Representation}}
\providecommand{\seecode}{}\renewcommand{\seecode}{\seedocumentationref{For more information about intermediate code and its purpose}{code}{Intermediate Code Representation}}
\providecommand{\seeobject}{}\renewcommand{\seeobject}{\seedocumentationref{For more information about object files and their purpose}{object}{Object File Representation}}

% generic documentation tools

\providecommand{\docprint}{
\toolsection{docprint} is a pretty printer for generic documentations.
It reformats generic documentations and writes it to the standard output stream.
\debuggingtool
\flowgraph{\resource{generic\\documentation} \ar[r] & \toolbox{docprint} \ar[r] & \resource{generic\\documentation}}
\seedocumentation
}

\providecommand{\doccheck}{
\toolsection{doccheck} is a syntactic and semantic checker for generic documentations.
It just performs syntactic and semantic checks on generic documentations and writes its diagnostic messages to the standard error stream.
\debuggingtool
\flowgraph{\resource{generic\\documentation} \ar[r] & \toolbox{doccheck} \ar[r] & \resource{diagnostic\\messages}}
\seedocumentation
}

\providecommand{\dochtml}{
\toolsection{dochtml} is an HTML documentation generator for generic documentations.
It processes several generic documentations and assembles all information therein into an HTML document.
\debuggingtool
\flowgraph{\resource{generic\\documentation} \ar[r] & \toolbox{dochtml} \ar[r] & \resource{HTML\\document}}
\seedocumentation
}

\providecommand{\doclatex}{
\toolsection{doclatex} is a Latex documentation generator for generic documentations.
It processes several generic documentations and assembles all information therein into a Latex document.
\debuggingtool
\flowgraph{\resource{generic\\documentation} \ar[r] & \toolbox{doclatex} \ar[r] & \resource{Latex\\document}}
\seedocumentation
}

% intermediate code tools

\providecommand{\cdcheck}{
\toolsection{cdcheck} is a syntactic and semantic checker for intermediate code.
It just performs syntactic and semantic checks on programs written in intermediate code and writes its diagnostic messages to the standard error stream.
\debuggingtool
\flowgraph{\resource{intermediate\\code} \ar[r] & \toolbox{cdcheck} \ar[r] & \resource{diagnostic\\messages}}
\seeassembly\seecode
}

\providecommand{\cdopt}{
\toolsection{cdopt} is an optimizer for intermediate code.
It performs various optimizations on programs written in intermediate code and writes the result to the standard output stream.
\debuggingtool
\flowgraph{\resource{intermediate\\code} \ar[r] & \toolbox{cdopt} \ar[r] & \resource{optimized\\code}}
\seeassembly\seecode
}

\providecommand{\cdrun}{
\toolsection{cdrun} is an interpreter for intermediate code.
It processes and executes programs written in intermediate code.
The following code sections are predefined and have the usual semantics:
\texttt{abort}, \texttt{\_Exit}, \texttt{fflush}, \texttt{floor}, \texttt{fputc}, \texttt{free}, \texttt{getchar}, \texttt{malloc}, and \texttt{putchar}.
Diagnostic messages about invalid operations include the name of the executed code section and the index of the erroneous instruction.
\debuggingtool
\flowgraph{\resource{intermediate\\code} \ar[r] & \toolbox{cdrun} \ar@/u/[r] & \resource{input/\\output} \ar@/d/[l]}
\seeassembly\seecode
}

\providecommand{\cdamda}{
\toolsection{cdamd16} is a compiler for intermediate code targeting the AMD64 hardware architecture.
It generates machine code for AMD64 processors from programs written in intermediate code and stores it in corresponding object files.
The compiler generates machine code for the 16-bit operating mode defined by the AMD64 architecture.
It also creates a debugging information file as well as an assembly file containing a listing of the generated machine code.
\debuggingtool
\flowgraph{\resource{intermediate\\code} \ar[r] & \toolbox{cdamd16} \ar[r] \ar[d] \ar[rd] & \resource{object file} \\ & \resource{assembly\\listing} & \resource{debugging\\information}}
\seeassembly\seeamd\seeobject\seecode\seedebugging
}

\providecommand{\cdamdb}{
\toolsection{cdamd32} is a compiler for intermediate code targeting the AMD64 hardware architecture.
It generates machine code for AMD64 processors from programs written in intermediate code and stores it in corresponding object files.
The compiler generates machine code for the 32-bit operating mode defined by the AMD64 architecture.
It also creates a debugging information file as well as an assembly file containing a listing of the generated machine code.
\debuggingtool
\flowgraph{\resource{intermediate\\code} \ar[r] & \toolbox{cdamd32} \ar[r] \ar[d] \ar[rd] & \resource{object file} \\ & \resource{assembly\\listing} & \resource{debugging\\information}}
\seeassembly\seeamd\seeobject\seecode\seedebugging
}

\providecommand{\cdamdc}{
\toolsection{cdamd64} is a compiler for intermediate code targeting the AMD64 hardware architecture.
It generates machine code for AMD64 processors from programs written in intermediate code and stores it in corresponding object files.
The compiler generates machine code for the 64-bit operating mode defined by the AMD64 architecture.
It also creates a debugging information file as well as an assembly file containing a listing of the generated machine code.
\debuggingtool
\flowgraph{\resource{intermediate\\code} \ar[r] & \toolbox{cdamd64} \ar[r] \ar[d] \ar[rd] & \resource{object file} \\ & \resource{assembly\\listing} & \resource{debugging\\information}}
\seeassembly\seeamd\seeobject\seecode\seedebugging
}

\providecommand{\cdarma}{
\toolsection{cdarma32} is a compiler for intermediate code targeting the ARM hardware architecture.
It generates machine code for ARM processors executing A32 instructions from programs written in intermediate code and stores it in corresponding object files.
It also creates a debugging information file as well as an assembly file containing a listing of the generated machine code.
\debuggingtool
\flowgraph{\resource{intermediate\\code} \ar[r] & \toolbox{cdarma32} \ar[r] \ar[d] \ar[rd] & \resource{object file} \\ & \resource{assembly\\listing} & \resource{debugging\\information}}
\seeassembly\seearm\seeobject\seecode\seedebugging
}

\providecommand{\cdarmb}{
\toolsection{cdarma64} is a compiler for intermediate code targeting the ARM hardware architecture.
It generates machine code for ARM processors executing A64 instructions from programs written in intermediate code and stores it in corresponding object files.
It also creates a debugging information file as well as an assembly file containing a listing of the generated machine code.
\debuggingtool
\flowgraph{\resource{intermediate\\code} \ar[r] & \toolbox{cdarma64} \ar[r] \ar[d] \ar[rd] & \resource{object file} \\ & \resource{assembly\\listing} & \resource{debugging\\information}}
\seeassembly\seearm\seeobject\seecode\seedebugging
}

\providecommand{\cdarmc}{
\toolsection{cdarmt32} is a compiler for intermediate code targeting the ARM hardware architecture.
It generates machine code for ARM processors without floating-point extension executing T32 instructions from programs written in intermediate code and stores it in corresponding object files.
It also creates a debugging information file as well as an assembly file containing a listing of the generated machine code.
\debuggingtool
\flowgraph{\resource{intermediate\\code} \ar[r] & \toolbox{cdarmt32} \ar[r] \ar[d] \ar[rd] & \resource{object file} \\ & \resource{assembly\\listing} & \resource{debugging\\information}}
\seeassembly\seearm\seeobject\seecode\seedebugging
}

\providecommand{\cdarmcfpe}{
\toolsection{cdarmt32fpe} is a compiler for intermediate code targeting the ARM hardware architecture.
It generates machine code for ARM processors with floating-point extension executing T32 instructions from programs written in intermediate code and stores it in corresponding object files.
It also creates a debugging information file as well as an assembly file containing a listing of the generated machine code.
\debuggingtool
\flowgraph{\resource{intermediate\\code} \ar[r] & \toolbox{cdarmt32fpe} \ar[r] \ar[d] \ar[rd] & \resource{object file} \\ & \resource{assembly\\listing} & \resource{debugging\\information}}
\seeassembly\seearm\seeobject\seecode\seedebugging
}

\providecommand{\cdavr}{
\toolsection{cdavr} is a compiler for intermediate code targeting the AVR hardware architecture.
It generates machine code for AVR processors from programs written in intermediate code and stores it in corresponding object files.
It also creates a debugging information file as well as an assembly file containing a listing of the generated machine code.
\debuggingtool
\flowgraph{\resource{intermediate\\code} \ar[r] & \toolbox{cdavr} \ar[r] \ar[d] \ar[rd] & \resource{object file} \\ & \resource{assembly\\listing} & \resource{debugging\\information}}
\seeassembly\seeavr\seeobject\seecode\seedebugging
}

\providecommand{\cdavrtt}{
\toolsection{cdavr32} is a compiler for intermediate code targeting the AVR32 hardware architecture.
It generates machine code for AVR32 processors from programs written in intermediate code and stores it in corresponding object files.
It also creates a debugging information file as well as an assembly file containing a listing of the generated machine code.
\debuggingtool
\flowgraph{\resource{intermediate\\code} \ar[r] & \toolbox{cdavr32} \ar[r] \ar[d] \ar[rd] & \resource{object file} \\ & \resource{assembly\\listing} & \resource{debugging\\information}}
\seeassembly\seeavrtt\seeobject\seecode\seedebugging
}

\providecommand{\cdmabk}{
\toolsection{cdm68k} is a compiler for intermediate code targeting the M68000 hardware architecture.
It generates machine code for M68000 processors from programs written in intermediate code and stores it in corresponding object files.
It also creates a debugging information file as well as an assembly file containing a listing of the generated machine code.
\debuggingtool
\flowgraph{\resource{intermediate\\code} \ar[r] & \toolbox{cdm68k} \ar[r] \ar[d] \ar[rd] & \resource{object file} \\ & \resource{assembly\\listing} & \resource{debugging\\information}}
\seeassembly\seemabk\seeobject\seecode\seedebugging
}

\providecommand{\cdmibl}{
\toolsection{cdmibl} is a compiler for intermediate code targeting the MicroBlaze hardware architecture.
It generates machine code for MicroBlaze processors from programs written in intermediate code and stores it in corresponding object files.
It also creates a debugging information file as well as an assembly file containing a listing of the generated machine code.
\debuggingtool
\flowgraph{\resource{intermediate\\code} \ar[r] & \toolbox{cdmibl} \ar[r] \ar[d] \ar[rd] & \resource{object file} \\ & \resource{assembly\\listing} & \resource{debugging\\information}}
\seeassembly\seemibl\seeobject\seecode\seedebugging
}

\providecommand{\cdmipsa}{
\toolsection{cdmips32} is a compiler for intermediate code targeting the MIPS32 hardware architecture.
It generates machine code for MIPS32 processors from programs written in intermediate code and stores it in corresponding object files.
It also creates a debugging information file as well as an assembly file containing a listing of the generated machine code.
\debuggingtool
\flowgraph{\resource{intermediate\\code} \ar[r] & \toolbox{cdmips32} \ar[r] \ar[d] \ar[rd] & \resource{object file} \\ & \resource{assembly\\listing} & \resource{debugging\\information}}
\seeassembly\seemips\seeobject\seecode\seedebugging
}

\providecommand{\cdmipsb}{
\toolsection{cdmips64} is a compiler for intermediate code targeting the MIPS64 hardware architecture.
It generates machine code for MIPS64 processors from programs written in intermediate code and stores it in corresponding object files.
It also creates a debugging information file as well as an assembly file containing a listing of the generated machine code.
\debuggingtool
\flowgraph{\resource{intermediate\\code} \ar[r] & \toolbox{cdmips64} \ar[r] \ar[d] \ar[rd] & \resource{object file} \\ & \resource{assembly\\listing} & \resource{debugging\\information}}
\seeassembly\seemips\seeobject\seecode\seedebugging
}

\providecommand{\cdmmix}{
\toolsection{cdmmix} is a compiler for intermediate code targeting the MMIX hardware architecture.
It generates machine code for MMIX processors from programs written in intermediate code and stores it in corresponding object files.
It also creates a debugging information file as well as an assembly file containing a listing of the generated machine code.
\debuggingtool
\flowgraph{\resource{intermediate\\code} \ar[r] & \toolbox{cdmmix} \ar[r] \ar[d] \ar[rd] & \resource{object file} \\ & \resource{assembly\\listing} & \resource{debugging\\information}}
\seeassembly\seemmix\seeobject\seecode\seedebugging
}

\providecommand{\cdorok}{
\toolsection{cdor1k} is a compiler for intermediate code targeting the OpenRISC 1000 hardware architecture.
It generates machine code for OpenRISC 1000 processors from programs written in intermediate code and stores it in corresponding object files.
It also creates a debugging information file as well as an assembly file containing a listing of the generated machine code.
\debuggingtool
\flowgraph{\resource{intermediate\\code} \ar[r] & \toolbox{cdor1k} \ar[r] \ar[d] \ar[rd] & \resource{object file} \\ & \resource{assembly\\listing} & \resource{debugging\\information}}
\seeassembly\seeorok\seeobject\seecode\seedebugging
}

\providecommand{\cdppca}{
\toolsection{cdppc32} is a compiler for intermediate code targeting the PowerPC hardware architecture.
It generates machine code for PowerPC processors from programs written in intermediate code and stores it in corresponding object files.
The compiler generates machine code for the 32-bit operating mode defined by the PowerPC architecture.
It also creates a debugging information file as well as an assembly file containing a listing of the generated machine code.
\debuggingtool
\flowgraph{\resource{intermediate\\code} \ar[r] & \toolbox{cdppc32} \ar[r] \ar[d] \ar[rd] & \resource{object file} \\ & \resource{assembly\\listing} & \resource{debugging\\information}}
\seeassembly\seeppc\seeobject\seecode\seedebugging
}

\providecommand{\cdppcb}{
\toolsection{cdppc64} is a compiler for intermediate code targeting the PowerPC hardware architecture.
It generates machine code for PowerPC processors from programs written in intermediate code and stores it in corresponding object files.
The compiler generates machine code for the 64-bit operating mode defined by the PowerPC architecture.
It also creates a debugging information file as well as an assembly file containing a listing of the generated machine code.
\debuggingtool
\flowgraph{\resource{intermediate\\code} \ar[r] & \toolbox{cdppc64} \ar[r] \ar[d] \ar[rd] & \resource{object file} \\ & \resource{assembly\\listing} & \resource{debugging\\information}}
\seeassembly\seeppc\seeobject\seecode\seedebugging
}

\providecommand{\cdrisc}{
\toolsection{cdrisc} is a compiler for intermediate code targeting the RISC hardware architecture.
It generates machine code for RISC processors from programs written in intermediate code and stores it in corresponding object files.
It also creates a debugging information file as well as an assembly file containing a listing of the generated machine code.
\debuggingtool
\flowgraph{\resource{intermediate\\code} \ar[r] & \toolbox{cdrisc} \ar[r] \ar[d] \ar[rd] & \resource{object file} \\ & \resource{assembly\\listing} & \resource{debugging\\information}}
\seeassembly\seerisc\seeobject\seecode\seedebugging
}

\providecommand{\cdwasm}{
\toolsection{cdwasm} is a compiler for intermediate code targeting the WebAssembly architecture.
It generates machine code for WebAssembly targets from programs written in intermediate code and stores it in corresponding object files.
It also creates a debugging information file as well as an assembly file containing a listing of the generated machine code.
\debuggingtool
\flowgraph{\resource{intermediate\\code} \ar[r] & \toolbox{cdwasm} \ar[r] \ar[d] \ar[rd] & \resource{object file} \\ & \resource{assembly\\listing} & \resource{debugging\\information}}
\seeassembly\seewasm\seeobject\seecode\seedebugging
}

% C++ tools

\providecommand{\cppprep}{
\toolsection{cppprep} is a preprocessor for the \cpp{} programming language.
It preprocesses source code according to the rules of \cpp{} and writes it to the standard output stream.
Only the macro names \texttt{\_\_DATE\_\_}, \texttt{\_\_FILE\_\_}, \texttt{\_\_LINE\_\_}, and \texttt{\_\_TIME\_\_} are predefined.
\flowgraph{\resource{\cpp{} or other\\source code} \ar[r] & \toolbox{cppprep} \ar[r] & \resource{preprocessed\\source code} \\ & \variable{ECSINCLUDE} \ar[u]}
\seecpp
}

\providecommand{\cppprint}{
\toolsection{cppprint} is a pretty printer for the \cpp{} programming language.
It reformats the source code of \cpp{} programs and writes it to the standard output stream.
\flowgraph{\resource{\cpp{}\\source code} \ar[r] & \toolbox{cppprint} \ar[r] & \resource{reformatted\\source code} \\ & \variable{ECSINCLUDE} \ar[u]}
\seecpp
}

\providecommand{\cppcheck}{
\toolsection{cppcheck} is a syntactic and semantic checker for the \cpp{} programming language.
It just performs syntactic and semantic checks on \cpp{} programs and writes its diagnostic messages to the standard error stream.
\flowgraph{\resource{\cpp{}\\source code} \ar[r] & \toolbox{cppcheck} \ar[r] & \resource{diagnostic\\messages} \\ & \variable{ECSINCLUDE} \ar[u]}
\seecpp
}

\providecommand{\cppdump}{
\toolsection{cppdump} is a serializer for the \cpp{} programming language.
It dumps the complete internal representation of programs written in \cpp{} into an XML document.
\debuggingtool
\flowgraph{\resource{\cpp{}\\source code} \ar[r] & \toolbox{cppdump} \ar[r] & \resource{internal\\representation} \\ & \variable{ECSINCLUDE} \ar[u]}
\seecpp
}

\providecommand{\cpprun}{
\toolsection{cpprun} is an interpreter for the \cpp{} programming language.
It processes and executes programs written in \cpp{}.
The macro \texttt{\_\_run\_\_} is predefined in order to enable programmers to identify this tool while interpreting.
\flowgraph{\resource{\cpp{}\\source code} \ar[r] & \toolbox{cpprun} \ar@/u/[r] & \resource{input/\\output} \ar@/d/[l] \\ & \variable{ECSINCLUDE} \ar[u]}
\seecpp
}

\providecommand{\cppdoc}{
\toolsection{cppdoc} is a generic documentation generator for the \cpp{} programming language.
It processes several \cpp{} source files and assembles all information therein into a generic documentation.
\debuggingtool
\flowgraph{\resource{\cpp{}\\source code} \ar[r] & \toolbox{cppdoc} \ar[r] & \resource{generic\\documentation} \\ & \variable{ECSINCLUDE} \ar[u]}
\seecpp\seedocumentation
}

\providecommand{\cpphtml}{
\toolsection{cpphtml} is an HTML documentation generator for the \cpp{} programming language.
It processes several \cpp{} source files and assembles all information therein into an HTML document.
\flowgraph{\resource{\cpp{}\\source code} \ar[r] & \toolbox{cpphtml} \ar[r] & \resource{HTML\\document} \\ & \variable{ECSINCLUDE} \ar[u]}
\seecpp\seedocumentation
}

\providecommand{\cpplatex}{
\toolsection{cpplatex} is a Latex documentation generator for the \cpp{} programming language.
It processes several \cpp{} source files and assembles all information therein into a Latex document.
\flowgraph{\resource{\cpp{}\\source code} \ar[r] & \toolbox{cpplatex} \ar[r] & \resource{Latex\\document} \\ & \variable{ECSINCLUDE} \ar[u]}
\seecpp\seedocumentation
}

\providecommand{\cppcode}{
\toolsection{cppcode} is an intermediate code generator for the \cpp{} programming language.
It generates intermediate code from programs written in \cpp{} and stores it in corresponding assembly files.
The macro \texttt{\_\_code\_\_} is predefined in order to enable programmers to identify this tool while generating intermediate code.
Programs generated with this tool require additional runtime support that is stored in the \file{cpp\-code\-run} library file.
\debuggingtool
\flowgraph{\resource{\cpp{}\\source code} \ar[r] & \toolbox{cppcode} \ar[r] & \resource{intermediate\\code} \\ & \variable{ECSINCLUDE} \ar[u]}
\seecpp\seeassembly\seecode
}

\providecommand{\cppamda}{
\toolsection{cppamd16} is a compiler for the \cpp{} programming language targeting the AMD64 hardware architecture.
It generates machine code for AMD64 processors from programs written in \cpp{} and stores it in corresponding object files.
The compiler generates machine code for the 16-bit operating mode defined by the AMD64 architecture.
For debugging purposes, it also creates a debugging information file as well as an assembly file containing a listing of the generated machine code.
The macro \texttt{\_\_amd16\_\_} is predefined in order to enable programmers to identify this tool and its target architecture while compiling.
Programs generated with this compiler require additional runtime support that is stored in the \file{cpp\-amd16\-run} library file.
\flowgraph{\resource{\cpp{}\\source code} \ar[r] & \toolbox{cppamd16} \ar[r] \ar[d] \ar[rd] & \resource{object file} \\ \variable{ECSINCLUDE} \ar[ru] & \resource{debugging\\information} & \resource{assembly\\listing}}
\seecpp\seeassembly\seeamd\seeobject\seedebugging
}

\providecommand{\cppamdb}{
\toolsection{cppamd32} is a compiler for the \cpp{} programming language targeting the AMD64 hardware architecture.
It generates machine code for AMD64 processors from programs written in \cpp{} and stores it in corresponding object files.
The compiler generates machine code for the 32-bit operating mode defined by the AMD64 architecture.
For debugging purposes, it also creates a debugging information file as well as an assembly file containing a listing of the generated machine code.
The macro \texttt{\_\_amd32\_\_} is predefined in order to enable programmers to identify this tool and its target architecture while compiling.
Programs generated with this compiler require additional runtime support that is stored in the \file{cpp\-amd32\-run} library file.
\flowgraph{\resource{\cpp{}\\source code} \ar[r] & \toolbox{cppamd32} \ar[r] \ar[d] \ar[rd] & \resource{object file} \\ \variable{ECSINCLUDE} \ar[ru] & \resource{debugging\\information} & \resource{assembly\\listing}}
\seecpp\seeassembly\seeamd\seeobject\seedebugging
}

\providecommand{\cppamdc}{
\toolsection{cppamd64} is a compiler for the \cpp{} programming language targeting the AMD64 hardware architecture.
It generates machine code for AMD64 processors from programs written in \cpp{} and stores it in corresponding object files.
The compiler generates machine code for the 64-bit operating mode defined by the AMD64 architecture.
For debugging purposes, it also creates a debugging information file as well as an assembly file containing a listing of the generated machine code.
The macro \texttt{\_\_amd64\_\_} is predefined in order to enable programmers to identify this tool and its target architecture while compiling.
Programs generated with this compiler require additional runtime support that is stored in the \file{cpp\-amd64\-run} library file.
\flowgraph{\resource{\cpp{}\\source code} \ar[r] & \toolbox{cppamd64} \ar[r] \ar[d] \ar[rd] & \resource{object file} \\ \variable{ECSINCLUDE} \ar[ru] & \resource{debugging\\information} & \resource{assembly\\listing}}
\seecpp\seeassembly\seeamd\seeobject\seedebugging
}

\providecommand{\cpparma}{
\toolsection{cpparma32} is a compiler for the \cpp{} programming language targeting the ARM hardware architecture.
It generates machine code for ARM processors executing A32 instructions from programs written in \cpp{} and stores it in corresponding object files.
For debugging purposes, it also creates a debugging information file as well as an assembly file containing a listing of the generated machine code.
The macro \texttt{\_\_arma32\_\_} is predefined in order to enable programmers to identify this tool and its target architecture while compiling.
Programs generated with this compiler require additional runtime support that is stored in the \file{cpp\-arma32\-run} library file.
\flowgraph{\resource{\cpp{}\\source code} \ar[r] & \toolbox{cpparma32} \ar[r] \ar[d] \ar[rd] & \resource{object file} \\ \variable{ECSINCLUDE} \ar[ru] & \resource{debugging\\information} & \resource{assembly\\listing}}
\seecpp\seeassembly\seearm\seeobject\seedebugging
}

\providecommand{\cpparmb}{
\toolsection{cpparma64} is a compiler for the \cpp{} programming language targeting the ARM hardware architecture.
It generates machine code for ARM processors executing A64 instructions from programs written in \cpp{} and stores it in corresponding object files.
For debugging purposes, it also creates a debugging information file as well as an assembly file containing a listing of the generated machine code.
The macro \texttt{\_\_arma64\_\_} is predefined in order to enable programmers to identify this tool and its target architecture while compiling.
Programs generated with this compiler require additional runtime support that is stored in the \file{cpp\-arma64\-run} library file.
\flowgraph{\resource{\cpp{}\\source code} \ar[r] & \toolbox{cpparma64} \ar[r] \ar[d] \ar[rd] & \resource{object file} \\ \variable{ECSINCLUDE} \ar[ru] & \resource{debugging\\information} & \resource{assembly\\listing}}
\seecpp\seeassembly\seearm\seeobject\seedebugging
}

\providecommand{\cpparmc}{
\toolsection{cpparmt32} is a compiler for the \cpp{} programming language targeting the ARM hardware architecture.
It generates machine code for ARM processors without floating-point extension executing T32 instructions from programs written in \cpp{} and stores it in corresponding object files.
For debugging purposes, it also creates a debugging information file as well as an assembly file containing a listing of the generated machine code.
The macro \texttt{\_\_armt32\_\_} is predefined in order to enable programmers to identify this tool and its target architecture while compiling.
Programs generated with this compiler require additional runtime support that is stored in the \file{cpp\-armt32\-run} library file.
\flowgraph{\resource{\cpp{}\\source code} \ar[r] & \toolbox{cpparmt32} \ar[r] \ar[d] \ar[rd] & \resource{object file} \\ \variable{ECSINCLUDE} \ar[ru] & \resource{debugging\\information} & \resource{assembly\\listing}}
\seecpp\seeassembly\seearm\seeobject\seedebugging
}

\providecommand{\cpparmcfpe}{
\toolsection{cpparmt32fpe} is a compiler for the \cpp{} programming language targeting the ARM hardware architecture.
It generates machine code for ARM processors with floating-point extension executing T32 instructions from programs written in \cpp{} and stores it in corresponding object files.
For debugging purposes, it also creates a debugging information file as well as an assembly file containing a listing of the generated machine code.
The macro \texttt{\_\_armt32fpe\_\_} is predefined in order to enable programmers to identify this tool and its target architecture while compiling.
Programs generated with this compiler require additional runtime support that is stored in the \file{cpp\-armt32\-fpe\-run} library file.
\flowgraph{\resource{\cpp{}\\source code} \ar[r] & \toolbox{cpparmt32fpe} \ar[r] \ar[d] \ar[rd] & \resource{object file} \\ \variable{ECSINCLUDE} \ar[ru] & \resource{debugging\\information} & \resource{assembly\\listing}}
\seecpp\seeassembly\seearm\seeobject\seedebugging
}

\providecommand{\cppavr}{
\toolsection{cppavr} is a compiler for the \cpp{} programming language targeting the AVR hardware architecture.
It generates machine code for AVR processors from programs written in \cpp{} and stores it in corresponding object files.
For debugging purposes, it also creates a debugging information file as well as an assembly file containing a listing of the generated machine code.
The macro \texttt{\_\_avr\_\_} is predefined in order to enable programmers to identify this tool and its target architecture while compiling.
Programs generated with this compiler require additional runtime support that is stored in the \file{cpp\-avr\-run} library file.
\flowgraph{\resource{\cpp{}\\source code} \ar[r] & \toolbox{cppavr} \ar[r] \ar[d] \ar[rd] & \resource{object file} \\ \variable{ECSINCLUDE} \ar[ru] & \resource{debugging\\information} & \resource{assembly\\listing}}
\seecpp\seeassembly\seeavr\seeobject\seedebugging
}

\providecommand{\cppavrtt}{
\toolsection{cppavr32} is a compiler for the \cpp{} programming language targeting the AVR32 hardware architecture.
It generates machine code for AVR32 processors from programs written in \cpp{} and stores it in corresponding object files.
For debugging purposes, it also creates a debugging information file as well as an assembly file containing a listing of the generated machine code.
The macro \texttt{\_\_avr32\_\_} is predefined in order to enable programmers to identify this tool and its target architecture while compiling.
Programs generated with this compiler require additional runtime support that is stored in the \file{cpp\-avr32\-run} library file.
\flowgraph{\resource{\cpp{}\\source code} \ar[r] & \toolbox{cppavr32} \ar[r] \ar[d] \ar[rd] & \resource{object file} \\ \variable{ECSINCLUDE} \ar[ru] & \resource{debugging\\information} & \resource{assembly\\listing}}
\seecpp\seeassembly\seeavrtt\seeobject\seedebugging
}

\providecommand{\cppmabk}{
\toolsection{cppm68k} is a compiler for the \cpp{} programming language targeting the M68000 hardware architecture.
It generates machine code for M68000 processors from programs written in \cpp{} and stores it in corresponding object files.
For debugging purposes, it also creates a debugging information file as well as an assembly file containing a listing of the generated machine code.
The macro \texttt{\_\_m68k\_\_} is predefined in order to enable programmers to identify this tool and its target architecture while compiling.
Programs generated with this compiler require additional runtime support that is stored in the \file{cpp\-m68k\-run} library file.
\flowgraph{\resource{\cpp{}\\source code} \ar[r] & \toolbox{cppm68k} \ar[r] \ar[d] \ar[rd] & \resource{object file} \\ \variable{ECSINCLUDE} \ar[ru] & \resource{debugging\\information} & \resource{assembly\\listing}}
\seecpp\seeassembly\seemabk\seeobject\seedebugging
}

\providecommand{\cppmibl}{
\toolsection{cppmibl} is a compiler for the \cpp{} programming language targeting the MicroBlaze hardware architecture.
It generates machine code for MicroBlaze processors from programs written in \cpp{} and stores it in corresponding object files.
For debugging purposes, it also creates a debugging information file as well as an assembly file containing a listing of the generated machine code.
The macro \texttt{\_\_mibl\_\_} is predefined in order to enable programmers to identify this tool and its target architecture while compiling.
Programs generated with this compiler require additional runtime support that is stored in the \file{cpp\-mibl\-run} library file.
\flowgraph{\resource{\cpp{}\\source code} \ar[r] & \toolbox{cppmibl} \ar[r] \ar[d] \ar[rd] & \resource{object file} \\ \variable{ECSINCLUDE} \ar[ru] & \resource{debugging\\information} & \resource{assembly\\listing}}
\seecpp\seeassembly\seemibl\seeobject\seedebugging
}

\providecommand{\cppmipsa}{
\toolsection{cppmips32} is a compiler for the \cpp{} programming language targeting the MIPS32 hardware architecture.
It generates machine code for MIPS32 processors from programs written in \cpp{} and stores it in corresponding object files.
For debugging purposes, it also creates a debugging information file as well as an assembly file containing a listing of the generated machine code.
The macro \texttt{\_\_mips32\_\_} is predefined in order to enable programmers to identify this tool and its target architecture while compiling.
Programs generated with this compiler require additional runtime support that is stored in the \file{cpp\-mips32\-run} library file.
\flowgraph{\resource{\cpp{}\\source code} \ar[r] & \toolbox{cppmips32} \ar[r] \ar[d] \ar[rd] & \resource{object file} \\ \variable{ECSINCLUDE} \ar[ru] & \resource{debugging\\information} & \resource{assembly\\listing}}
\seecpp\seeassembly\seemips\seeobject\seedebugging
}

\providecommand{\cppmipsb}{
\toolsection{cppmips64} is a compiler for the \cpp{} programming language targeting the MIPS64 hardware architecture.
It generates machine code for MIPS64 processors from programs written in \cpp{} and stores it in corresponding object files.
For debugging purposes, it also creates a debugging information file as well as an assembly file containing a listing of the generated machine code.
The macro \texttt{\_\_mips64\_\_} is predefined in order to enable programmers to identify this tool and its target architecture while compiling.
Programs generated with this compiler require additional runtime support that is stored in the \file{cpp\-mips64\-run} library file.
\flowgraph{\resource{\cpp{}\\source code} \ar[r] & \toolbox{cppmips64} \ar[r] \ar[d] \ar[rd] & \resource{object file} \\ \variable{ECSINCLUDE} \ar[ru] & \resource{debugging\\information} & \resource{assembly\\listing}}
\seecpp\seeassembly\seemips\seeobject\seedebugging
}

\providecommand{\cppmmix}{
\toolsection{cppmmix} is a compiler for the \cpp{} programming language targeting the MMIX hardware architecture.
It generates machine code for MMIX processors from programs written in \cpp{} and stores it in corresponding object files.
For debugging purposes, it also creates a debugging information file as well as an assembly file containing a listing of the generated machine code.
The macro \texttt{\_\_mmix\_\_} is predefined in order to enable programmers to identify this tool and its target architecture while compiling.
Programs generated with this compiler require additional runtime support that is stored in the \file{cpp\-mmix\-run} library file.
\flowgraph{\resource{\cpp{}\\source code} \ar[r] & \toolbox{cppmmix} \ar[r] \ar[d] \ar[rd] & \resource{object file} \\ \variable{ECSINCLUDE} \ar[ru] & \resource{debugging\\information} & \resource{assembly\\listing}}
\seecpp\seeassembly\seemmix\seeobject\seedebugging
}

\providecommand{\cpporok}{
\toolsection{cppor1k} is a compiler for the \cpp{} programming language targeting the OpenRISC 1000 hardware architecture.
It generates machine code for OpenRISC 1000 processors from programs written in \cpp{} and stores it in corresponding object files.
For debugging purposes, it also creates a debugging information file as well as an assembly file containing a listing of the generated machine code.
The macro \texttt{\_\_or1k\_\_} is predefined in order to enable programmers to identify this tool and its target architecture while compiling.
Programs generated with this compiler require additional runtime support that is stored in the \file{cpp\-or1k\-run} library file.
\flowgraph{\resource{\cpp{}\\source code} \ar[r] & \toolbox{cppor1k} \ar[r] \ar[d] \ar[rd] & \resource{object file} \\ \variable{ECSINCLUDE} \ar[ru] & \resource{debugging\\information} & \resource{assembly\\listing}}
\seecpp\seeassembly\seeorok\seeobject\seedebugging
}

\providecommand{\cppppca}{
\toolsection{cppppc32} is a compiler for the \cpp{} programming language targeting the PowerPC hardware architecture.
It generates machine code for PowerPC processors from programs written in \cpp{} and stores it in corresponding object files.
The compiler generates machine code for the 32-bit operating mode defined by the PowerPC architecture.
For debugging purposes, it also creates a debugging information file as well as an assembly file containing a listing of the generated machine code.
The macro \texttt{\_\_ppc32\_\_} is predefined in order to enable programmers to identify this tool and its target architecture while compiling.
Programs generated with this compiler require additional runtime support that is stored in the \file{cpp\-ppc32\-run} library file.
\flowgraph{\resource{\cpp{}\\source code} \ar[r] & \toolbox{cppppc32} \ar[r] \ar[d] \ar[rd] & \resource{object file} \\ \variable{ECSINCLUDE} \ar[ru] & \resource{debugging\\information} & \resource{assembly\\listing}}
\seecpp\seeassembly\seeppc\seeobject\seedebugging
}

\providecommand{\cppppcb}{
\toolsection{cppppc64} is a compiler for the \cpp{} programming language targeting the PowerPC hardware architecture.
It generates machine code for PowerPC processors from programs written in \cpp{} and stores it in corresponding object files.
The compiler generates machine code for the 64-bit operating mode defined by the PowerPC architecture.
For debugging purposes, it also creates a debugging information file as well as an assembly file containing a listing of the generated machine code.
The macro \texttt{\_\_ppc64\_\_} is predefined in order to enable programmers to identify this tool and its target architecture while compiling.
Programs generated with this compiler require additional runtime support that is stored in the \file{cpp\-ppc64\-run} library file.
\flowgraph{\resource{\cpp{}\\source code} \ar[r] & \toolbox{cppppc64} \ar[r] \ar[d] \ar[rd] & \resource{object file} \\ \variable{ECSINCLUDE} \ar[ru] & \resource{debugging\\information} & \resource{assembly\\listing}}
\seecpp\seeassembly\seeppc\seeobject\seedebugging
}

\providecommand{\cpprisc}{
\toolsection{cpprisc} is a compiler for the \cpp{} programming language targeting the RISC hardware architecture.
It generates machine code for RISC processors from programs written in \cpp{} and stores it in corresponding object files.
For debugging purposes, it also creates a debugging information file as well as an assembly file containing a listing of the generated machine code.
The macro \texttt{\_\_risc\_\_} is predefined in order to enable programmers to identify this tool and its target architecture while compiling.
Programs generated with this compiler require additional runtime support that is stored in the \file{cpp\-risc\-run} library file.
\flowgraph{\resource{\cpp{}\\source code} \ar[r] & \toolbox{cpprisc} \ar[r] \ar[d] \ar[rd] & \resource{object file} \\ \variable{ECSINCLUDE} \ar[ru] & \resource{debugging\\information} & \resource{assembly\\listing}}
\seecpp\seeassembly\seerisc\seeobject\seedebugging
}

\providecommand{\cppwasm}{
\toolsection{cppwasm} is a compiler for the \cpp{} programming language targeting the WebAssembly architecture.
It generates machine code for WebAssembly targets from programs written in \cpp{} and stores it in corresponding object files.
For debugging purposes, it also creates a debugging information file as well as an assembly file containing a listing of the generated machine code.
The macro \texttt{\_\_wasm\_\_} is predefined in order to enable programmers to identify this tool and its target architecture while compiling.
Programs generated with this compiler require additional runtime support that is stored in the \file{cpp\-wasm\-run} library file.
\flowgraph{\resource{\cpp{}\\source code} \ar[r] & \toolbox{cppwasm} \ar[r] \ar[d] \ar[rd] & \resource{object file} \\ \variable{ECSINCLUDE} \ar[ru] & \resource{debugging\\information} & \resource{assembly\\listing}}
\seecpp\seeassembly\seewasm\seeobject\seedebugging
}

% FALSE tools

\providecommand{\falprint}{
\toolsection{falprint} is a pretty printer for the FALSE programming language.
It reformats the source code of FALSE programs and writes it to the standard output stream.
\flowgraph{\resource{FALSE\\source code} \ar[r] & \toolbox{falprint} \ar[r] & \resource{reformatted\\source code}}
\seefalse
}

\providecommand{\falcheck}{
\toolsection{falcheck} is a syntactic and semantic checker for the FALSE programming language.
It just performs syntactic and semantic checks on FALSE programs and writes its diagnostic messages to the standard error stream.
\flowgraph{\resource{FALSE\\source code} \ar[r] & \toolbox{falcheck} \ar[r] & \resource{diagnostic\\messages}}
\seefalse
}

\providecommand{\faldump}{
\toolsection{faldump} is a serializer for the FALSE programming language.
It dumps the complete internal representation of programs written in FALSE into an XML document.
\debuggingtool
\flowgraph{\resource{FALSE\\source code} \ar[r] & \toolbox{faldump} \ar[r] & \resource{internal\\representation}}
\seefalse
}

\providecommand{\falrun}{
\toolsection{falrun} is an interpreter for the FALSE programming language.
It processes and executes programs written in FALSE\@.
\flowgraph{\resource{FALSE\\source code} \ar[r] & \toolbox{falrun} \ar@/u/[r] & \resource{input/\\output} \ar@/d/[l]}
\seefalse
}

\providecommand{\falcpp}{
\toolsection{falcpp} is a transpiler for the FALSE programming language.
It translates programs written in FALSE into \cpp{} programs and stores them in corresponding source files.
\flowgraph{\resource{FALSE\\source code} \ar[r] & \toolbox{falcpp} \ar[r] & \resource{\cpp{}\\source file}}
\seefalse\seecpp
}

\providecommand{\falcode}{
\toolsection{falcode} is an intermediate code generator for the FALSE programming language.
It generates intermediate code from programs written in FALSE and stores it in corresponding assembly files.
\debuggingtool
\flowgraph{\resource{FALSE\\source code} \ar[r] & \toolbox{falcode} \ar[r] & \resource{intermediate\\code}}
\seefalse\seeassembly\seecode
}

\providecommand{\falamda}{
\toolsection{falamd16} is a compiler for the FALSE programming language targeting the AMD64 hardware architecture.
It generates machine code for AMD64 processors from programs written in FALSE and stores it in corresponding object files.
The compiler generates machine code for the 16-bit operating mode defined by the AMD64 architecture.
\flowgraph{\resource{FALSE\\source code} \ar[r] & \toolbox{falamd16} \ar[r] & \resource{object file}}
\seefalse\seeamd\seeobject
}

\providecommand{\falamdb}{
\toolsection{falamd32} is a compiler for the FALSE programming language targeting the AMD64 hardware architecture.
It generates machine code for AMD64 processors from programs written in FALSE and stores it in corresponding object files.
The compiler generates machine code for the 32-bit operating mode defined by the AMD64 architecture.
\flowgraph{\resource{FALSE\\source code} \ar[r] & \toolbox{falamd32} \ar[r] & \resource{object file}}
\seefalse\seeamd\seeobject
}

\providecommand{\falamdc}{
\toolsection{falamd64} is a compiler for the FALSE programming language targeting the AMD64 hardware architecture.
It generates machine code for AMD64 processors from programs written in FALSE and stores it in corresponding object files.
The compiler generates machine code for the 64-bit operating mode defined by the AMD64 architecture.
\flowgraph{\resource{FALSE\\source code} \ar[r] & \toolbox{falamd64} \ar[r] & \resource{object file}}
\seefalse\seeamd\seeobject
}

\providecommand{\falarma}{
\toolsection{falarma32} is a compiler for the FALSE programming language targeting the ARM hardware architecture.
It generates machine code for ARM processors executing A32 instructions from programs written in FALSE and stores it in corresponding object files.
\flowgraph{\resource{FALSE\\source code} \ar[r] & \toolbox{falarma32} \ar[r] & \resource{object file}}
\seefalse\seearm\seeobject
}

\providecommand{\falarmb}{
\toolsection{falarma64} is a compiler for the FALSE programming language targeting the ARM hardware architecture.
It generates machine code for ARM processors executing A64 instructions from programs written in FALSE and stores it in corresponding object files.
\flowgraph{\resource{FALSE\\source code} \ar[r] & \toolbox{falarma64} \ar[r] & \resource{object file}}
\seefalse\seearm\seeobject
}

\providecommand{\falarmc}{
\toolsection{falarmt32} is a compiler for the FALSE programming language targeting the ARM hardware architecture.
It generates machine code for ARM processors without floating-point extension executing T32 instructions from programs written in FALSE and stores it in corresponding object files.
\flowgraph{\resource{FALSE\\source code} \ar[r] & \toolbox{falarmt32} \ar[r] & \resource{object file}}
\seefalse\seearm\seeobject
}

\providecommand{\falarmcfpe}{
\toolsection{falarmt32fpe} is a compiler for the FALSE programming language targeting the ARM hardware architecture.
It generates machine code for ARM processors with floating-point extension executing T32 instructions from programs written in FALSE and stores it in corresponding object files.
\flowgraph{\resource{FALSE\\source code} \ar[r] & \toolbox{falarmt32fpe} \ar[r] & \resource{object file}}
\seefalse\seearm\seeobject
}

\providecommand{\falavr}{
\toolsection{falavr} is a compiler for the FALSE programming language targeting the AVR hardware architecture.
It generates machine code for AVR processors from programs written in FALSE and stores it in corresponding object files.
\flowgraph{\resource{FALSE\\source code} \ar[r] & \toolbox{falavr} \ar[r] & \resource{object file}}
\seefalse\seeavr\seeobject
}

\providecommand{\falavrtt}{
\toolsection{falavr32} is a compiler for the FALSE programming language targeting the AVR32 hardware architecture.
It generates machine code for AVR32 processors from programs written in FALSE and stores it in corresponding object files.
\flowgraph{\resource{FALSE\\source code} \ar[r] & \toolbox{falavr32} \ar[r] & \resource{object file}}
\seefalse\seeavrtt\seeobject
}

\providecommand{\falmabk}{
\toolsection{falm68k} is a compiler for the FALSE programming language targeting the M68000 hardware architecture.
It generates machine code for M68000 processors from programs written in FALSE and stores it in corresponding object files.
\flowgraph{\resource{FALSE\\source code} \ar[r] & \toolbox{falm68k} \ar[r] & \resource{object file}}
\seefalse\seemabk\seeobject
}

\providecommand{\falmibl}{
\toolsection{falmibl} is a compiler for the FALSE programming language targeting the MicroBlaze hardware architecture.
It generates machine code for MicroBlaze processors from programs written in FALSE and stores it in corresponding object files.
\flowgraph{\resource{FALSE\\source code} \ar[r] & \toolbox{falmibl} \ar[r] & \resource{object file}}
\seefalse\seemibl\seeobject
}

\providecommand{\falmipsa}{
\toolsection{falmips32} is a compiler for the FALSE programming language targeting the MIPS32 hardware architecture.
It generates machine code for MIPS32 processors from programs written in FALSE and stores it in corresponding object files.
\flowgraph{\resource{FALSE\\source code} \ar[r] & \toolbox{falmips32} \ar[r] & \resource{object file}}
\seefalse\seemips\seeobject
}

\providecommand{\falmipsb}{
\toolsection{falmips64} is a compiler for the FALSE programming language targeting the MIPS64 hardware architecture.
It generates machine code for MIPS64 processors from programs written in FALSE and stores it in corresponding object files.
\flowgraph{\resource{FALSE\\source code} \ar[r] & \toolbox{falmips64} \ar[r] & \resource{object file}}
\seefalse\seemips\seeobject
}

\providecommand{\falmmix}{
\toolsection{falmmix} is a compiler for the FALSE programming language targeting the MMIX hardware architecture.
It generates machine code for MMIX processors from programs written in FALSE and stores it in corresponding object files.
\flowgraph{\resource{FALSE\\source code} \ar[r] & \toolbox{falmmix} \ar[r] & \resource{object file}}
\seefalse\seemmix\seeobject
}

\providecommand{\falorok}{
\toolsection{falor1k} is a compiler for the FALSE programming language targeting the OpenRISC 1000 hardware architecture.
It generates machine code for OpenRISC 1000 processors from programs written in FALSE and stores it in corresponding object files.
\flowgraph{\resource{FALSE\\source code} \ar[r] & \toolbox{falor1k} \ar[r] & \resource{object file}}
\seefalse\seeorok\seeobject
}

\providecommand{\falppca}{
\toolsection{falppc32} is a compiler for the FALSE programming language targeting the PowerPC hardware architecture.
It generates machine code for PowerPC processors from programs written in FALSE and stores it in corresponding object files.
The compiler generates machine code for the 32-bit operating mode defined by the PowerPC architecture.
\flowgraph{\resource{FALSE\\source code} \ar[r] & \toolbox{falppc32} \ar[r] & \resource{object file}}
\seefalse\seeppc\seeobject
}

\providecommand{\falppcb}{
\toolsection{falppc64} is a compiler for the FALSE programming language targeting the PowerPC hardware architecture.
It generates machine code for PowerPC processors from programs written in FALSE and stores it in corresponding object files.
The compiler generates machine code for the 64-bit operating mode defined by the PowerPC architecture.
\flowgraph{\resource{FALSE\\source code} \ar[r] & \toolbox{falppc64} \ar[r] & \resource{object file}}
\seefalse\seeppc\seeobject
}

\providecommand{\falrisc}{
\toolsection{falrisc} is a compiler for the FALSE programming language targeting the RISC hardware architecture.
It generates machine code for RISC processors from programs written in FALSE and stores it in corresponding object files.
\flowgraph{\resource{FALSE\\source code} \ar[r] & \toolbox{falrisc} \ar[r] & \resource{object file}}
\seefalse\seerisc\seeobject
}

\providecommand{\falwasm}{
\toolsection{falwasm} is a compiler for the FALSE programming language targeting the WebAssembly architecture.
It generates machine code for WebAssembly targets from programs written in FALSE and stores it in corresponding object files.
\flowgraph{\resource{FALSE\\source code} \ar[r] & \toolbox{falwasm} \ar[r] & \resource{object file}}
\seefalse\seewasm\seeobject
}

% Oberon tools

\providecommand{\obprint}{
\toolsection{obprint} is a pretty printer for the Oberon programming language.
It reformats the source code of Oberon modules and writes it to the standard output stream.
\flowgraph{\resource{Oberon\\source code} \ar[r] & \toolbox{obprint} \ar[r] & \resource{reformatted\\source code}}
\seeoberon
}

\providecommand{\obcheck}{
\toolsection{obcheck} is a syntactic and semantic checker for the Oberon programming language.
It just performs syntactic and semantic checks on Oberon modules and writes its diagnostic messages to the standard error stream.
In addition, it stores the interface of each module in a symbol file which is required when other modules import the module.
\flowgraph{\resource{Oberon\\source code} \ar[r] & \toolbox{obcheck} \ar[r] \ar@/l/[d] & \resource{diagnostic\\messages} \\ \variable{ECSIMPORT} \ar[ru] & \resource{symbol\\files} \ar@/r/[u]}
\seeoberon
}

\providecommand{\obdump}{
\toolsection{obdump} is a serializer for the Oberon programming language.
It dumps the complete internal representation of modules written in Oberon into an XML document.
\debuggingtool
\flowgraph{\resource{Oberon\\source code} \ar[r] & \toolbox{obdump} \ar[r] \ar@/l/[d] & \resource{internal\\representation} \\ \variable{ECSIMPORT} \ar[ru] & \resource{symbol\\files} \ar@/r/[u]}
\seeoberon
}

\providecommand{\obrun}{
\toolsection{obrun} is an interpreter for the Oberon programming language.
It processes and executes modules written in Oberon.
This tool does neither generate nor process symbol files while interpreting modules.
If a module is imported by another one, its filename has to be named before the other one in the list of command-line arguments.
\flowgraph{\resource{Oberon\\source code} \ar[r] & \toolbox{obrun} \ar@/u/[r] & \resource{input/\\output} \ar@/d/[l]}
\seeoberon
}

\providecommand{\obcpp}{
\toolsection{obcpp} is a transpiler for the Oberon programming language.
It translates programs written in Oberon into \cpp{} programs and stores them in corresponding source and header files.
In addition, it stores the interface of each module in a symbol file which is required when other modules import the module.
The same interface is provided by the generated header file which can be used in other parts of the \cpp{} program.
\flowgraph{\resource{Oberon\\source code} \ar[r] & \toolbox{obcpp} \ar[r] \ar@/l/[d] \ar[rd] & \resource{\cpp{}\\source file} \\ \variable{ECSIMPORT} \ar[ru] & \resource{symbol\\files} \ar@/r/[u] & \resource{\cpp{}\\header file}}
\seeoberon\seecpp
}

\providecommand{\obdoc}{
\toolsection{obdoc} is a generic documentation generator for the Oberon programming language.
It processes several Oberon modules and assembles all information therein into a generic documentation.
In addition, it stores the interface of each module in a symbol file which is required when other modules import the module.
\debuggingtool
\flowgraph{\resource{Oberon\\source code} \ar[r] & \toolbox{obdoc} \ar[r] \ar@/l/[d] & \resource{generic\\documentation} \\ \variable{ECSIMPORT} \ar[ru] & \resource{symbol\\files} \ar@/r/[u]}
\seeoberon\seedocumentation
}

\providecommand{\obhtml}{
\toolsection{obhtml} is an HTML documentation generator for the Oberon programming language.
It processes several Oberon modules and assembles all information therein into an HTML document.
In addition, it stores the interface of each module in a symbol file which is required when other modules import the module.
\flowgraph{\resource{Oberon\\source code} \ar[r] & \toolbox{obhtml} \ar[r] \ar@/l/[d] & \resource{HTML\\document} \\ \variable{ECSIMPORT} \ar[ru] & \resource{symbol\\files} \ar@/r/[u]}
\seeoberon\seedocumentation
}

\providecommand{\oblatex}{
\toolsection{oblatex} is a Latex documentation generator for the Oberon programming language.
It processes several Oberon modules and assembles all information therein into a Latex document.
In addition, it stores the interface of each module in a symbol file which is required when other modules import the module.
\flowgraph{\resource{Oberon\\source code} \ar[r] & \toolbox{oblatex} \ar[r] \ar@/l/[d] & \resource{Latex\\document} \\ \variable{ECSIMPORT} \ar[ru] & \resource{symbol\\files} \ar@/r/[u]}
\seeoberon\seedocumentation
}

\providecommand{\obcode}{
\toolsection{obcode} is an intermediate code generator for the Oberon programming language.
It generates intermediate code from modules written in Oberon and stores it in corresponding assembly files.
In addition, it stores the interface of each module in a symbol file which is required when other modules import the module.
Programs generated with this tool require additional runtime support that is stored in the \file{ob\-code\-run} library file.
\debuggingtool
\flowgraph{\resource{Oberon\\source code} \ar[r] & \toolbox{obcode} \ar[r] \ar@/l/[d] & \resource{intermediate\\code} \\ \variable{ECSIMPORT} \ar[ru] & \resource{symbol\\files} \ar@/r/[u]}
\seeoberon\seeassembly\seecode
}

\providecommand{\obamda}{
\toolsection{obamd16} is a compiler for the Oberon programming language targeting the AMD64 hardware architecture.
It generates machine code for AMD64 processors from modules written in Oberon and stores it in corresponding object files.
The compiler generates machine code for the 16-bit operating mode defined by the AMD64 architecture.
For debugging purposes, it also creates a debugging information file as well as an assembly file containing a listing of the generated machine code.
In addition, it stores the interface of each module in a symbol file which is required when other modules import the module.
Programs generated with this compiler require additional runtime support that is stored in the \file{ob\-amd16\-run} library file.
\flowgraph{\resource{Oberon\\source code} \ar[r] & \toolbox{obamd16} \ar[r] \ar@/l/[d] \ar[rd] & \resource{object file} \\ \variable{ECSIMPORT} \ar[ru] & \resource{symbol\\files} \ar@/r/[u] & \resource{debugging\\information}}
\seeoberon\seeassembly\seeamd\seeobject\seedebugging
}

\providecommand{\obamdb}{
\toolsection{obamd32} is a compiler for the Oberon programming language targeting the AMD64 hardware architecture.
It generates machine code for AMD64 processors from modules written in Oberon and stores it in corresponding object files.
The compiler generates machine code for the 32-bit operating mode defined by the AMD64 architecture.
For debugging purposes, it also creates a debugging information file as well as an assembly file containing a listing of the generated machine code.
In addition, it stores the interface of each module in a symbol file which is required when other modules import the module.
Programs generated with this compiler require additional runtime support that is stored in the \file{ob\-amd32\-run} library file.
\flowgraph{\resource{Oberon\\source code} \ar[r] & \toolbox{obamd32} \ar[r] \ar@/l/[d] \ar[rd] & \resource{object file} \\ \variable{ECSIMPORT} \ar[ru] & \resource{symbol\\files} \ar@/r/[u] & \resource{debugging\\information}}
\seeoberon\seeassembly\seeamd\seeobject\seedebugging
}

\providecommand{\obamdc}{
\toolsection{obamd64} is a compiler for the Oberon programming language targeting the AMD64 hardware architecture.
It generates machine code for AMD64 processors from modules written in Oberon and stores it in corresponding object files.
The compiler generates machine code for the 64-bit operating mode defined by the AMD64 architecture.
For debugging purposes, it also creates a debugging information file as well as an assembly file containing a listing of the generated machine code.
In addition, it stores the interface of each module in a symbol file which is required when other modules import the module.
Programs generated with this compiler require additional runtime support that is stored in the \file{ob\-amd64\-run} library file.
\flowgraph{\resource{Oberon\\source code} \ar[r] & \toolbox{obamd64} \ar[r] \ar@/l/[d] \ar[rd] & \resource{object file} \\ \variable{ECSIMPORT} \ar[ru] & \resource{symbol\\files} \ar@/r/[u] & \resource{debugging\\information}}
\seeoberon\seeassembly\seeamd\seeobject\seedebugging
}

\providecommand{\obarma}{
\toolsection{obarma32} is a compiler for the Oberon programming language targeting the ARM hardware architecture.
It generates machine code for ARM processors executing A32 instructions from modules written in Oberon and stores it in corresponding object files.
For debugging purposes, it also creates a debugging information file as well as an assembly file containing a listing of the generated machine code.
In addition, it stores the interface of each module in a symbol file which is required when other modules import the module.
Programs generated with this compiler require additional runtime support that is stored in the \file{ob\-arma32\-run} library file.
\flowgraph{\resource{Oberon\\source code} \ar[r] & \toolbox{obarma32} \ar[r] \ar@/l/[d] \ar[rd] & \resource{object file} \\ \variable{ECSIMPORT} \ar[ru] & \resource{symbol\\files} \ar@/r/[u] & \resource{debugging\\information}}
\seeoberon\seeassembly\seearm\seeobject\seedebugging
}

\providecommand{\obarmb}{
\toolsection{obarma64} is a compiler for the Oberon programming language targeting the ARM hardware architecture.
It generates machine code for ARM processors executing A64 instructions from modules written in Oberon and stores it in corresponding object files.
For debugging purposes, it also creates a debugging information file as well as an assembly file containing a listing of the generated machine code.
In addition, it stores the interface of each module in a symbol file which is required when other modules import the module.
Programs generated with this compiler require additional runtime support that is stored in the \file{ob\-arma64\-run} library file.
\flowgraph{\resource{Oberon\\source code} \ar[r] & \toolbox{obarma64} \ar[r] \ar@/l/[d] \ar[rd] & \resource{object file} \\ \variable{ECSIMPORT} \ar[ru] & \resource{symbol\\files} \ar@/r/[u] & \resource{debugging\\information}}
\seeoberon\seeassembly\seearm\seeobject\seedebugging
}

\providecommand{\obarmc}{
\toolsection{obarmt32} is a compiler for the Oberon programming language targeting the ARM hardware architecture.
It generates machine code for ARM processors without floating-point extension executing T32 instructions from modules written in Oberon and stores it in corresponding object files.
For debugging purposes, it also creates a debugging information file as well as an assembly file containing a listing of the generated machine code.
In addition, it stores the interface of each module in a symbol file which is required when other modules import the module.
Programs generated with this compiler require additional runtime support that is stored in the \file{ob\-armt32\-run} library file.
\flowgraph{\resource{Oberon\\source code} \ar[r] & \toolbox{obarmt32} \ar[r] \ar@/l/[d] \ar[rd] & \resource{object file} \\ \variable{ECSIMPORT} \ar[ru] & \resource{symbol\\files} \ar@/r/[u] & \resource{debugging\\information}}
\seeoberon\seeassembly\seearm\seeobject\seedebugging
}

\providecommand{\obarmcfpe}{
\toolsection{obarmt32fpe} is a compiler for the Oberon programming language targeting the ARM hardware architecture.
It generates machine code for ARM processors with floating-point extension executing T32 instructions from modules written in Oberon and stores it in corresponding object files.
For debugging purposes, it also creates a debugging information file as well as an assembly file containing a listing of the generated machine code.
In addition, it stores the interface of each module in a symbol file which is required when other modules import the module.
Programs generated with this compiler require additional runtime support that is stored in the \file{ob\-armt32\-fpe\-run} library file.
\flowgraph{\resource{Oberon\\source code} \ar[r] & \toolbox{obarmt32fpe} \ar[r] \ar@/l/[d] \ar[rd] & \resource{object file} \\ \variable{ECSIMPORT} \ar[ru] & \resource{symbol\\files} \ar@/r/[u] & \resource{debugging\\information}}
\seeoberon\seeassembly\seearm\seeobject\seedebugging
}

\providecommand{\obavr}{
\toolsection{obavr} is a compiler for the Oberon programming language targeting the AVR hardware architecture.
It generates machine code for AVR processors from modules written in Oberon and stores it in corresponding object files.
For debugging purposes, it also creates a debugging information file as well as an assembly file containing a listing of the generated machine code.
In addition, it stores the interface of each module in a symbol file which is required when other modules import the module.
Programs generated with this compiler require additional runtime support that is stored in the \file{ob\-avr\-run} library file.
\flowgraph{\resource{Oberon\\source code} \ar[r] & \toolbox{obavr} \ar[r] \ar@/l/[d] \ar[rd] & \resource{object file} \\ \variable{ECSIMPORT} \ar[ru] & \resource{symbol\\files} \ar@/r/[u] & \resource{debugging\\information}}
\seeoberon\seeassembly\seeavr\seeobject\seedebugging
}

\providecommand{\obavrtt}{
\toolsection{obavr32} is a compiler for the Oberon programming language targeting the AVR32 hardware architecture.
It generates machine code for AVR32 processors from modules written in Oberon and stores it in corresponding object files.
For debugging purposes, it also creates a debugging information file as well as an assembly file containing a listing of the generated machine code.
In addition, it stores the interface of each module in a symbol file which is required when other modules import the module.
Programs generated with this compiler require additional runtime support that is stored in the \file{ob\-avr32\-run} library file.
\flowgraph{\resource{Oberon\\source code} \ar[r] & \toolbox{obavr32} \ar[r] \ar@/l/[d] \ar[rd] & \resource{object file} \\ \variable{ECSIMPORT} \ar[ru] & \resource{symbol\\files} \ar@/r/[u] & \resource{debugging\\information}}
\seeoberon\seeassembly\seeavrtt\seeobject\seedebugging
}

\providecommand{\obmabk}{
\toolsection{obm68k} is a compiler for the Oberon programming language targeting the M68000 hardware architecture.
It generates machine code for M68000 processors from modules written in Oberon and stores it in corresponding object files.
For debugging purposes, it also creates a debugging information file as well as an assembly file containing a listing of the generated machine code.
In addition, it stores the interface of each module in a symbol file which is required when other modules import the module.
Programs generated with this compiler require additional runtime support that is stored in the \file{ob\-m68k\-run} library file.
\flowgraph{\resource{Oberon\\source code} \ar[r] & \toolbox{obm68k} \ar[r] \ar@/l/[d] \ar[rd] & \resource{object file} \\ \variable{ECSIMPORT} \ar[ru] & \resource{symbol\\files} \ar@/r/[u] & \resource{debugging\\information}}
\seeoberon\seeassembly\seemabk\seeobject\seedebugging
}

\providecommand{\obmibl}{
\toolsection{obmibl} is a compiler for the Oberon programming language targeting the MicroBlaze hardware architecture.
It generates machine code for MicroBlaze processors from modules written in Oberon and stores it in corresponding object files.
For debugging purposes, it also creates a debugging information file as well as an assembly file containing a listing of the generated machine code.
In addition, it stores the interface of each module in a symbol file which is required when other modules import the module.
Programs generated with this compiler require additional runtime support that is stored in the \file{ob\-mibl\-run} library file.
\flowgraph{\resource{Oberon\\source code} \ar[r] & \toolbox{obmibl} \ar[r] \ar@/l/[d] \ar[rd] & \resource{object file} \\ \variable{ECSIMPORT} \ar[ru] & \resource{symbol\\files} \ar@/r/[u] & \resource{debugging\\information}}
\seeoberon\seeassembly\seemibl\seeobject\seedebugging
}

\providecommand{\obmipsa}{
\toolsection{obmips32} is a compiler for the Oberon programming language targeting the MIPS32 hardware architecture.
It generates machine code for MIPS32 processors from modules written in Oberon and stores it in corresponding object files.
For debugging purposes, it also creates a debugging information file as well as an assembly file containing a listing of the generated machine code.
In addition, it stores the interface of each module in a symbol file which is required when other modules import the module.
Programs generated with this compiler require additional runtime support that is stored in the \file{ob\-mips32\-run} library file.
\flowgraph{\resource{Oberon\\source code} \ar[r] & \toolbox{obmips32} \ar[r] \ar@/l/[d] \ar[rd] & \resource{object file} \\ \variable{ECSIMPORT} \ar[ru] & \resource{symbol\\files} \ar@/r/[u] & \resource{debugging\\information}}
\seeoberon\seeassembly\seemips\seeobject\seedebugging
}

\providecommand{\obmipsb}{
\toolsection{obmips64} is a compiler for the Oberon programming language targeting the MIPS64 hardware architecture.
It generates machine code for MIPS64 processors from modules written in Oberon and stores it in corresponding object files.
For debugging purposes, it also creates a debugging information file as well as an assembly file containing a listing of the generated machine code.
In addition, it stores the interface of each module in a symbol file which is required when other modules import the module.
Programs generated with this compiler require additional runtime support that is stored in the \file{ob\-mips64\-run} library file.
\flowgraph{\resource{Oberon\\source code} \ar[r] & \toolbox{obmips64} \ar[r] \ar@/l/[d] \ar[rd] & \resource{object file} \\ \variable{ECSIMPORT} \ar[ru] & \resource{symbol\\files} \ar@/r/[u] & \resource{debugging\\information}}
\seeoberon\seeassembly\seemips\seeobject\seedebugging
}

\providecommand{\obmmix}{
\toolsection{obmmix} is a compiler for the Oberon programming language targeting the MMIX hardware architecture.
It generates machine code for MMIX processors from modules written in Oberon and stores it in corresponding object files.
For debugging purposes, it also creates a debugging information file as well as an assembly file containing a listing of the generated machine code.
In addition, it stores the interface of each module in a symbol file which is required when other modules import the module.
Programs generated with this compiler require additional runtime support that is stored in the \file{ob\-mmix\-run} library file.
\flowgraph{\resource{Oberon\\source code} \ar[r] & \toolbox{obmmix} \ar[r] \ar@/l/[d] \ar[rd] & \resource{object file} \\ \variable{ECSIMPORT} \ar[ru] & \resource{symbol\\files} \ar@/r/[u] & \resource{debugging\\information}}
\seeoberon\seeassembly\seemmix\seeobject\seedebugging
}

\providecommand{\oborok}{
\toolsection{obor1k} is a compiler for the Oberon programming language targeting the OpenRISC 1000 hardware architecture.
It generates machine code for OpenRISC 1000 processors from modules written in Oberon and stores it in corresponding object files.
For debugging purposes, it also creates a debugging information file as well as an assembly file containing a listing of the generated machine code.
In addition, it stores the interface of each module in a symbol file which is required when other modules import the module.
Programs generated with this compiler require additional runtime support that is stored in the \file{ob\-or1k\-run} library file.
\flowgraph{\resource{Oberon\\source code} \ar[r] & \toolbox{obor1k} \ar[r] \ar@/l/[d] \ar[rd] & \resource{object file} \\ \variable{ECSIMPORT} \ar[ru] & \resource{symbol\\files} \ar@/r/[u] & \resource{debugging\\information}}
\seeoberon\seeassembly\seeorok\seeobject\seedebugging
}

\providecommand{\obppca}{
\toolsection{obppc32} is a compiler for the Oberon programming language targeting the PowerPC hardware architecture.
It generates machine code for PowerPC processors from modules written in Oberon and stores it in corresponding object files.
The compiler generates machine code for the 32-bit operating mode defined by the PowerPC architecture.
For debugging purposes, it also creates a debugging information file as well as an assembly file containing a listing of the generated machine code.
In addition, it stores the interface of each module in a symbol file which is required when other modules import the module.
Programs generated with this compiler require additional runtime support that is stored in the \file{ob\-ppc32\-run} library file.
\flowgraph{\resource{Oberon\\source code} \ar[r] & \toolbox{obppc32} \ar[r] \ar@/l/[d] \ar[rd] & \resource{object file} \\ \variable{ECSIMPORT} \ar[ru] & \resource{symbol\\files} \ar@/r/[u] & \resource{debugging\\information}}
\seeoberon\seeassembly\seeppc\seeobject\seedebugging
}

\providecommand{\obppcb}{
\toolsection{obppc64} is a compiler for the Oberon programming language targeting the PowerPC hardware architecture.
It generates machine code for PowerPC processors from modules written in Oberon and stores it in corresponding object files.
The compiler generates machine code for the 64-bit operating mode defined by the PowerPC architecture.
For debugging purposes, it also creates a debugging information file as well as an assembly file containing a listing of the generated machine code.
In addition, it stores the interface of each module in a symbol file which is required when other modules import the module.
Programs generated with this compiler require additional runtime support that is stored in the \file{ob\-ppc64\-run} library file.
\flowgraph{\resource{Oberon\\source code} \ar[r] & \toolbox{obppc64} \ar[r] \ar@/l/[d] \ar[rd] & \resource{object file} \\ \variable{ECSIMPORT} \ar[ru] & \resource{symbol\\files} \ar@/r/[u] & \resource{debugging\\information}}
\seeoberon\seeassembly\seeppc\seeobject\seedebugging
}

\providecommand{\obrisc}{
\toolsection{obrisc} is a compiler for the Oberon programming language targeting the RISC hardware architecture.
It generates machine code for RISC processors from modules written in Oberon and stores it in corresponding object files.
For debugging purposes, it also creates a debugging information file as well as an assembly file containing a listing of the generated machine code.
In addition, it stores the interface of each module in a symbol file which is required when other modules import the module.
Programs generated with this compiler require additional runtime support that is stored in the \file{ob\-risc\-run} library file.
\flowgraph{\resource{Oberon\\source code} \ar[r] & \toolbox{obrisc} \ar[r] \ar@/l/[d] \ar[rd] & \resource{object file} \\ \variable{ECSIMPORT} \ar[ru] & \resource{symbol\\files} \ar@/r/[u] & \resource{debugging\\information}}
\seeoberon\seeassembly\seerisc\seeobject\seedebugging
}

\providecommand{\obwasm}{
\toolsection{obwasm} is a compiler for the Oberon programming language targeting the WebAssembly architecture.
It generates machine code for WebAssembly targets from modules written in Oberon and stores it in corresponding object files.
For debugging purposes, it also creates a debugging information file as well as an assembly file containing a listing of the generated machine code.
In addition, it stores the interface of each module in a symbol file which is required when other modules import the module.
Programs generated with this compiler require additional runtime support that is stored in the \file{ob\-wasm\-run} library file.
\flowgraph{\resource{Oberon\\source code} \ar[r] & \toolbox{obwasm} \ar[r] \ar@/l/[d] \ar[rd] & \resource{object file} \\ \variable{ECSIMPORT} \ar[ru] & \resource{symbol\\files} \ar@/r/[u] & \resource{debugging\\information}}
\seeoberon\seeassembly\seewasm\seeobject\seedebugging
}

% converter tools

\providecommand{\dbgdwarf}{
\toolsection{dbgdwarf} is a DWARF debugging information converter tool.
It converts debugging information into the DWARF debugging data format and stores it in corresponding object files~\cite{dwarffile}.
The resulting debugging object files can be combined with runtime support that creates Executable and Linking Format (ELF) files~\cite{elffile}.
\flowgraph{\resource{debugging\\information} \ar[r] & \toolbox{dbgdwarf} \ar[r] & \resource{debugging\\object file}}
\seeobject\seedebugging
}

% assembler tools

\providecommand{\asmprint}{
\toolsection{asmprint} is a pretty printer for generic assembly code.
It reformats generic assembly code and writes it to the standard output stream.
\flowgraph{\resource{generic assembly\\source code} \ar[r] & \toolbox{asmprint} \ar[r] & \resource{reformatted\\source code}}
\seeassembly
}

\providecommand{\amdaasm}{
\toolsection{amd16asm} is an assembler for the AMD64 hardware architecture.
It translates assembly code into machine code for AMD64 processors and stores it in corresponding object files.
By default, the assembler generates machine code for the 16-bit operating mode defined by the AMD64 architecture.
\flowgraph{\resource{AMD16 assembly\\source code} \ar[r] & \toolbox{amd16asm} \ar[r] & \resource{object file}}
\seeassembly\seeamd\seeobject
}

\providecommand{\amdadism}{
\toolsection{amd16dism} is a disassembler for the AMD64 hardware architecture.
It translates machine code from object files targeting AMD64 processors into assembly code and writes it to the standard output stream.
It assumes that the machine code was generated for the 16-bit operating mode defined by the AMD64 architecture.
\flowgraph{\resource{object file} \ar[r] & \toolbox{amd16dism} \ar[r] & \resource{disassembly\\listing}}
\seeassembly\seeamd\seeobject
}

\providecommand{\amdbasm}{
\toolsection{amd32asm} is an assembler for the AMD64 hardware architecture.
It translates assembly code into machine code for AMD64 processors and stores it in corresponding object files.
By default, the assembler generates machine code for the 32-bit operating mode defined by the AMD64 architecture.
\flowgraph{\resource{AMD32 assembly\\source code} \ar[r] & \toolbox{amd32asm} \ar[r] & \resource{object file}}
\seeassembly\seeamd\seeobject
}

\providecommand{\amdbdism}{
\toolsection{amd32dism} is a disassembler for the AMD64 hardware architecture.
It translates machine code from object files targeting AMD64 processors into assembly code and writes it to the standard output stream.
It assumes that the machine code was generated for the 32-bit operating mode defined by the AMD64 architecture.
\flowgraph{\resource{object file} \ar[r] & \toolbox{amd32dism} \ar[r] & \resource{disassembly\\listing}}
\seeassembly\seeamd\seeobject
}

\providecommand{\amdcasm}{
\toolsection{amd64asm} is an assembler for the AMD64 hardware architecture.
It translates assembly code into machine code for AMD64 processors and stores it in corresponding object files.
By default, the assembler generates machine code for the 64-bit operating mode defined by the AMD64 architecture.
\flowgraph{\resource{AMD64 assembly\\source code} \ar[r] & \toolbox{amd64asm} \ar[r] & \resource{object file}}
\seeassembly\seeamd\seeobject
}

\providecommand{\amdcdism}{
\toolsection{amd64dism} is a disassembler for the AMD64 hardware architecture.
It translates machine code from object files targeting AMD64 processors into assembly code and writes it to the standard output stream.
It assumes that the machine code was generated for the 64-bit operating mode defined by the AMD64 architecture.
\flowgraph{\resource{object file} \ar[r] & \toolbox{amd64dism} \ar[r] & \resource{disassembly\\listing}}
\seeassembly\seeamd\seeobject
}

\providecommand{\armaasm}{
\toolsection{arma32asm} is an assembler for the ARM hardware architecture.
It translates assembly code into machine code for ARM processors executing A32 instructions and stores it in corresponding object files.
\flowgraph{\resource{ARM A32 assembly\\source code} \ar[r] & \toolbox{arma32asm} \ar[r] & \resource{object file}}
\seeassembly\seearm\seeobject
}

\providecommand{\armadism}{
\toolsection{arma32dism} is a disassembler for the ARM hardware architecture.
It translates machine code from object files targeting ARM processors executing A32 instructions into assembly code and writes it to the standard output stream.
\flowgraph{\resource{object file} \ar[r] & \toolbox{arma32dism} \ar[r] & \resource{disassembly\\listing}}
\seeassembly\seearm\seeobject
}

\providecommand{\armbasm}{
\toolsection{arma64asm} is an assembler for the ARM hardware architecture.
It translates assembly code into machine code for ARM processors executing A64 instructions and stores it in corresponding object files.
\flowgraph{\resource{ARM A64 assembly\\source code} \ar[r] & \toolbox{arma64asm} \ar[r] & \resource{object file}}
\seeassembly\seearm\seeobject
}

\providecommand{\armbdism}{
\toolsection{arma64dism} is a disassembler for the ARM hardware architecture.
It translates machine code from object files targeting ARM processors executing A64 instructions into assembly code and writes it to the standard output stream.
\flowgraph{\resource{object file} \ar[r] & \toolbox{arma64dism} \ar[r] & \resource{disassembly\\listing}}
\seeassembly\seearm\seeobject
}

\providecommand{\armcasm}{
\toolsection{armt32asm} is an assembler for the ARM hardware architecture.
It translates assembly code into machine code for ARM processors executing T32 instructions and stores it in corresponding object files.
\flowgraph{\resource{ARM T32 assembly\\source code} \ar[r] & \toolbox{armt32asm} \ar[r] & \resource{object file}}
\seeassembly\seearm\seeobject
}

\providecommand{\armcdism}{
\toolsection{armt32dism} is a disassembler for the ARM hardware architecture.
It translates machine code from object files targeting ARM processors executing T32 instructions into assembly code and writes it to the standard output stream.
\flowgraph{\resource{object file} \ar[r] & \toolbox{armt32dism} \ar[r] & \resource{disassembly\\listing}}
\seeassembly\seearm\seeobject
}

\providecommand{\avrasm}{
\toolsection{avrasm} is an assembler for the AVR hardware architecture.
It translates assembly code into machine code for AVR processors and stores it in corresponding object files.
The identifiers \texttt{RXL}, \texttt{RXH}, \texttt{RYL}, \texttt{RYH}, \texttt{RZL}, and \texttt{RZH} are predefined and name the corresponding registers.
The identifiers \texttt{SPL} and \texttt{SPH} are also predefined and evaluate to the address of the corresponding registers.
\flowgraph{\resource{AVR assembly\\source code} \ar[r] & \toolbox{avrasm} \ar[r] & \resource{object file}}
\seeassembly\seeavr\seeobject
}

\providecommand{\avrdism}{
\toolsection{avrdism} is a disassembler for the AVR hardware architecture.
It translates machine code from object files targeting AVR processors into assembly code and writes it to the standard output stream.
\flowgraph{\resource{object file} \ar[r] & \toolbox{avrdism} \ar[r] & \resource{disassembly\\listing}}
\seeassembly\seeavr\seeobject
}

\providecommand{\avrttasm}{
\toolsection{avr32asm} is an assembler for the AVR32 hardware architecture.
It translates assembly code into machine code for AVR32 processors and stores it in corresponding object files.
\flowgraph{\resource{AVR32 assembly\\source code} \ar[r] & \toolbox{avr32asm} \ar[r] & \resource{object file}}
\seeassembly\seeavrtt\seeobject
}

\providecommand{\avrttdism}{
\toolsection{avr32dism} is a disassembler for the AVR32 hardware architecture.
It translates machine code from object files targeting AVR32 processors into assembly code and writes it to the standard output stream.
\flowgraph{\resource{object file} \ar[r] & \toolbox{avr32dism} \ar[r] & \resource{disassembly\\listing}}
\seeassembly\seeavrtt\seeobject
}

\providecommand{\mabkasm}{
\toolsection{m68kasm} is an assembler for the M68000 hardware architecture.
It translates assembly code into machine code for M68000 processors and stores it in corresponding object files.
\flowgraph{\resource{68000 assembly\\source code} \ar[r] & \toolbox{m68kasm} \ar[r] & \resource{object file}}
\seeassembly\seemabk\seeobject
}

\providecommand{\mabkdism}{
\toolsection{m68kdism} is a disassembler for the M68000 hardware architecture.
It translates machine code from object files targeting M68000 processors into assembly code and writes it to the standard output stream.
\flowgraph{\resource{object file} \ar[r] & \toolbox{m68kdism} \ar[r] & \resource{disassembly\\listing}}
\seeassembly\seemabk\seeobject
}

\providecommand{\miblasm}{
\toolsection{miblasm} is an assembler for the MicroBlaze hardware architecture.
It translates assembly code into machine code for MicroBlaze processors and stores it in corresponding object files.
\flowgraph{\resource{MicroBlaze assembly\\source code} \ar[r] & \toolbox{miblasm} \ar[r] & \resource{object file}}
\seeassembly\seemibl\seeobject
}

\providecommand{\mibldism}{
\toolsection{mibldism} is a disassembler for the MicroBlaze hardware architecture.
It translates machine code from object files targeting MicroBlaze processors into assembly code and writes it to the standard output stream.
\flowgraph{\resource{object file} \ar[r] & \toolbox{mibldism} \ar[r] & \resource{disassembly\\listing}}
\seeassembly\seemibl\seeobject
}

\providecommand{\mipsaasm}{
\toolsection{mips32asm} is an assembler for the MIPS32 hardware architecture.
It translates assembly code into machine code for MIPS32 processors and stores it in corresponding object files.
\flowgraph{\resource{MIPS32 assembly\\source code} \ar[r] & \toolbox{mips32asm} \ar[r] & \resource{object file}}
\seeassembly\seemips\seeobject
}

\providecommand{\mipsadism}{
\toolsection{mips32dism} is a disassembler for the MIPS32 hardware architecture.
It translates machine code from object files targeting MIPS32 processors into assembly code and writes it to the standard output stream.
\flowgraph{\resource{object file} \ar[r] & \toolbox{mips32dism} \ar[r] & \resource{disassembly\\listing}}
\seeassembly\seemips\seeobject
}

\providecommand{\mipsbasm}{
\toolsection{mips64asm} is an assembler for the MIPS64 hardware architecture.
It translates assembly code into machine code for MIPS64 processors and stores it in corresponding object files.
\flowgraph{\resource{MIPS64 assembly\\source code} \ar[r] & \toolbox{mips64asm} \ar[r] & \resource{object file}}
\seeassembly\seemips\seeobject
}

\providecommand{\mipsbdism}{
\toolsection{mips64dism} is a disassembler for the MIPS64 hardware architecture.
It translates machine code from object files targeting MIPS64 processors into assembly code and writes it to the standard output stream.
\flowgraph{\resource{object file} \ar[r] & \toolbox{mips64dism} \ar[r] & \resource{disassembly\\listing}}
\seeassembly\seemips\seeobject
}

\providecommand{\mmixasm}{
\toolsection{mmixasm} is an assembler for the MMIX hardware architecture.
It translates assembly code into machine code for MMIX processors and stores it in corresponding object files.
The names of all special registers are predefined and evaluate to the corresponding number.
\flowgraph{\resource{MMIX assembly\\source code} \ar[r] & \toolbox{mmixasm} \ar[r] & \resource{object file}}
\seeassembly\seemmix\seeobject
}

\providecommand{\mmixdism}{
\toolsection{mmixdism} is a disassembler for the MMIX hardware architecture.
It translates machine code from object files targeting MMIX processors into assembly code and writes it to the standard output stream.
\flowgraph{\resource{object file} \ar[r] & \toolbox{mmixdism} \ar[r] & \resource{disassembly\\listing}}
\seeassembly\seemmix\seeobject
}

\providecommand{\orokasm}{
\toolsection{or1kasm} is an assembler for the OpenRISC 1000 hardware architecture.
It translates assembly code into machine code for OpenRISC 1000 processors and stores it in corresponding object files.
\flowgraph{\resource{OpenRISC 1000 assembly\\source code} \ar[r] & \toolbox{or1kasm} \ar[r] & \resource{object file}}
\seeassembly\seeorok\seeobject
}

\providecommand{\orokdism}{
\toolsection{or1kdism} is a disassembler for the OpenRISC 1000 hardware architecture.
It translates machine code from object files targeting OpenRISC 1000 processors into assembly code and writes it to the standard output stream.
\flowgraph{\resource{object file} \ar[r] & \toolbox{or1kdism} \ar[r] & \resource{disassembly\\listing}}
\seeassembly\seeorok\seeobject
}

\providecommand{\ppcaasm}{
\toolsection{ppc32asm} is an assembler for the PowerPC hardware architecture.
It translates assembly code into machine code for PowerPC processors and stores it in corresponding object files.
By default, the assembler generates machine code for the 32-bit operating mode defined by the PowerPC architecture.
\flowgraph{\resource{PowerPC assembly\\source code} \ar[r] & \toolbox{ppc32asm} \ar[r] & \resource{object file}}
\seeassembly\seeppc\seeobject
}

\providecommand{\ppcadism}{
\toolsection{ppc32dism} is a disassembler for the PowerPC hardware architecture.
It translates machine code from object files targeting PowerPC processors into assembly code and writes it to the standard output stream.
It assumes that the machine code was generated for the 32-bit operating mode defined by the PowerPC architecture.
\flowgraph{\resource{object file} \ar[r] & \toolbox{ppc32dism} \ar[r] & \resource{disassembly\\listing}}
\seeassembly\seeppc\seeobject
}

\providecommand{\ppcbasm}{
\toolsection{ppc64asm} is an assembler for the PowerPC hardware architecture.
It translates assembly code into machine code for PowerPC processors and stores it in corresponding object files.
By default, the assembler generates machine code for the 64-bit operating mode defined by the PowerPC architecture.
\flowgraph{\resource{PowerPC assembly\\source code} \ar[r] & \toolbox{ppc64asm} \ar[r] & \resource{object file}}
\seeassembly\seeppc\seeobject
}

\providecommand{\ppcbdism}{
\toolsection{ppc64dism} is a disassembler for the PowerPC hardware architecture.
It translates machine code from object files targeting PowerPC processors into assembly code and writes it to the standard output stream.
It assumes that the machine code was generated for the 64-bit operating mode defined by the PowerPC architecture.
\flowgraph{\resource{object file} \ar[r] & \toolbox{ppc64dism} \ar[r] & \resource{disassembly\\listing}}
\seeassembly\seeppc\seeobject
}

\providecommand{\riscasm}{
\toolsection{riscasm} is an assembler for the RISC hardware architecture.
It translates assembly code into machine code for RISC processors and stores it in corresponding object files.
The names of all special registers are predefined and evaluate to the corresponding number.
\flowgraph{\resource{RISC assembly\\source code} \ar[r] & \toolbox{riscasm} \ar[r] & \resource{object file}}
\seeassembly\seerisc\seeobject
}

\providecommand{\riscdism}{
\toolsection{riscdism} is a disassembler for the RISC hardware architecture.
It translates machine code from object files targeting RISC processors into assembly code and writes it to the standard output stream.
\flowgraph{\resource{object file} \ar[r] & \toolbox{riscdism} \ar[r] & \resource{disassembly\\listing}}
\seeassembly\seerisc\seeobject
}

\providecommand{\wasmasm}{
\toolsection{wasmasm} is an assembler for the WebAssembly architecture.
It translates assembly code into machine code for WebAssembly targets and stores it in corresponding object files.
The names of all special registers are predefined and evaluate to the corresponding number.
\flowgraph{\resource{WebAssembly assembly\\source code} \ar[r] & \toolbox{wasmasm} \ar[r] & \resource{object file}}
\seeassembly\seewasm\seeobject
}

\providecommand{\wasmdism}{
\toolsection{wasmdism} is a disassembler for the WebAssembly architecture.
It translates machine code from object files targeting WebAssembly targets into assembly code and writes it to the standard output stream.
\flowgraph{\resource{object file} \ar[r] & \toolbox{wasmdism} \ar[r] & \resource{disassembly\\listing}}
\seeassembly\seewasm\seeobject
}

% linker tools

\providecommand{\linklib}{
\toolsection{linklib} is an object file combiner.
It creates a static library file by combining all object files given to it into a single one.
\flowgraph{\resource{object files} \ar[r] & \toolbox{linklib} \ar[r] & \resource{library file}}
\seeobject
}

\providecommand{\linkbin}{
\toolsection{linkbin} is a linker for plain binary files.
It links all object files given to it into a single image and stores it in a binary file that begins with the first linked section.
It also creates a map file that lists the address, type, name and size of all used sections.
The filename extension of the resulting binary file can be specified by putting it into a constant data section called \texttt{\_extension}.
\flowgraph{\resource{object files} \ar[r] & \toolbox{linkbin} \ar[r] \ar[d] & \resource{binary file} \\ & \resource{map file}}
\seeobject
}

\providecommand{\linkmem}{
\toolsection{linkmem} is a linker for plain binary files partitioned into random-access and read-only memory.
It links all object files given to it into two distinct images, one for data sections and one for code and constant data sections, and stores each image in a binary file that begins with the first linked section of the corresponding type.
It also creates a map file that lists the address, type, name and size of all used sections.
\flowgraph{\resource{object files} \ar[r] & \toolbox{linkmem} \ar[r] \ar[d] & \resource{RAM file/\\ROM file} \\ & \resource{map file}}
\seeobject
}

\providecommand{\linkprg}{
\toolsection{linkprg} is a linker for GEMDOS executable files.
It links all object files given to it into a single image and stores the image in an Atari GEMDOS executable file~\cite{gemdosfile}.
It also creates a map file that lists the address relative to the text segment, type, name and size of all used sections.
The filename extension of the resulting executable file can be specified by putting it into a constant data section called \texttt{\_extension}.
The GEMDOS executable file format requires all patch patterns of absolute link patches to consist of four full bitmasks with descending offsets.
\flowgraph{\resource{object files} \ar[r] & \toolbox{linkprg} \ar[r] \ar[d] & \resource{executable file} \\ & \resource{map file}}
\seeobject
}

\providecommand{\linkhex}{
\toolsection{linkhex} is a linker for Intel HEX files.
It links all code sections of the object files given to it into single image and stores the image in an Intel HEX file~\cite{hexfile} that begins with the first linked section.
It also creates a map file that lists the address, type, name and size of all used sections.
\flowgraph{\resource{object files} \ar[r] & \toolbox{linkhex} \ar[r] \ar[d] & \resource{HEX file} \\ & \resource{map file}}
\seeobject
}

\providecommand{\mapsearch}{
\toolsection{mapsearch} is a debugging tool.
It searches map files generated by linker tools for the name of a binary section that encompasses a memory address read from the standard input stream.
If additionally provided with one or more object files, it also stores an excerpt thereof in a separate object file called map search result which only contains the identified binary section for disassembling purposes.
\flowgraph{& \resource{map files/\\object files} \ar[d] \\ \resource{memory\\address} \ar[r] & \toolbox{mapsearch} \ar[r] \ar[d] & \resource{section name/\\relative offset} \\ & \resource{object file\\excerpt}}
\seeobject
}

\renewcommand{\seefalse}{}

\startchapter{FALSE}{User Manual for FALSE}{false}
{FALSE is a stack-oriented programming language that supports lambda abstractions and is quite powerful for its size.
Although FALSE is quite cryptic by design and therefore considered esoteric, it provides language features that enable interoperability with other programming languages.
This \documentation{} describes the language and its implementation by the \ecs{}.}

\epigraph{What is written without effort \\ is in general read without pleasure.}{Samuel Johnson}

\section{Introduction}

FALSE is an esoteric but still powerful programming language designed by Wouter van~Oortmerssen, named after his favorite truth value~\cite{oortmerssen1993}.

\begin{center}\fallogo{2em}\end{center}

Each FALSE program consists of comments enclosed in braces and expressions.
Expressions consist of symbols which are mostly represented as a single character.
Table~\ref{tab:falsymbols} lists all of these symbols and the way the operate on the stack and the 26 variables predefined by the FALSE programming language.
The notation $u, v \to s, t$ means that $u$ and $v$ are popped from the stack in that order and $s$ and $t$ are pushed onto it afterward.
An $\epsilon$ on either side of that notation denotes that the corresponding operation does not push or pop anything respectively.
The symbols listed under the category named special-purpose are actually implementation-defined and are explained in more detail in Section~\ref{sec:falimplementation}.

\begin{table}
\centering
\begin{tabular}{@{}llll@{}}
\toprule Category & Symbol & Description & Stack Operation \\
\midrule Literals
& \texttt{number} & Value of integer $n$ & $\epsilon \to n$ \\
& \texttt{'character} & Value of character $c$ & $\epsilon \to c$ \\
& \texttt{a\ldots z} & Address $a$ of variable & $\epsilon \to a$ \\
& \texttt{[\ldots ]} & Address $f$ of function & $\epsilon \to f$ \\
\midrule Arithmetic
& \texttt{\_} & Negation & $x \to -x$ \\
Operations
& \texttt{+} & Add & $y, x \to x + y$ \\
& \texttt{-} & Subtract & $y, x \rightarrow x - y$ \\
& \texttt{*} & Multiply & $y, x \rightarrow x \times y$ \\
& \texttt{/} & Divide & $y, x \rightarrow x \div y$ \\
\midrule Logical
& \texttt{\textasciitilde} & Complement & $x \to \neg x$ \\
Operations
& \texttt{\&} & And & $y, x \to x \land y$ \\
& \texttt{|} & Or & $y, x \to x \lor y$ \\
\midrule Comparison
& \texttt{=} & Test if equal & $y, x \to x = y$ \\
Operations
& \texttt{>} & Test if greater & $y, x \to x > y$ \\
\midrule Stack
& \texttt{\$} & Duplicate & $x \to x, x$ \\
Operations
& \texttt{\%} & Delete & $x \to \epsilon$ \\
& \texttt{\textbackslash} & Swap & $x, y \to x, y$ \\
& \texttt{@} & Rotate & $x, y, z \to y, x, z$ \\
& \texttt{\o} & Select & $n \to n^{\mathit th}$ element \\
\midrule Statements
& \texttt{:} & Write $x$ to address $a$ & $a, x \to \epsilon$ \\
& \texttt{;} & Read $x$ from address $a$ & $a \to x$ \\
& \texttt{!} & Call function $f$ & $f \to \epsilon$ \\
& \texttt{?} & Call function $f$ if $x$ is true & $f, x \to \epsilon$ \\
& \texttt{\#} & Call function $f$ while & $f, g \to \epsilon$ \\ & & function $g$ returns true \\
\midrule Input/
& \texttt{.} & Print $x$ as integer & $x \to \epsilon$ \\
Output
& \texttt{,} & Print $x$ as character & $x \to \epsilon$ \\
& \texttt{\^} & Read $x$ as character & $\epsilon \to x$ \\
& \texttt{"\ldots "} & Print string $s$ & $\epsilon \to \epsilon$ \\
& \texttt{\ss} & Flush output stream & $\epsilon \to \epsilon$ \\
\midrule Language
& \texttt{"\ldots "V} & Address $a$ of external variable $s$ & $\epsilon \to a$ \\
extensions
& \texttt{"\ldots "F} & Address $a$ of external function $s$ & $\epsilon \to a$ \\
& \texttt{"\ldots "\`} & Emit inline assembly & $\epsilon \to \epsilon$ \\
\bottomrule
\end{tabular}
\caption{The symbols of the FALSE programming language}
\label{tab:falsymbols}\index{Symbols, of FALSE}\index{FALSE!Symbols}
\end{table}

In general, there are three kinds of symbols which either push values onto the stack, change values on the stack, or just pop values from it in order to perform memory accesses or input and output.
The type of the actual value that is pushed onto or popped from the stack is defined by the operation performed by the symbol.
Arithmetic operations like \texttt{+} and \texttt{-} for example pop two integer values from the stack and push their result back onto the stack again.
A single character symbol like \texttt{a} or \texttt{b} on the other hand names a variable and pushes the address of that variable onto the stack.
The actual contents of the variable can be an integer or another address and is accessed using the dereference symbol \texttt{;}.
The third kind of symbols are statements like assignments that perform their operation without pushing any new values onto the stack.

\section{Implementation-Defined Behavior}\label{sec:falimplementation}

Some issues in the specification of the otherwise abstract programming language are either left unspecified or depend on its original implementation.
This section describes the concrete implementation specific behavior defined by the \ecs{}:

\begin{itemize}

\item
The actual size of the values stored on the stack and in variables is equal to the address size of the target hardware architecture.

\item
The logical and comparison operations generate Boolean values where false is represented using the integer zero and true is represented using the integer one.
For evaluations of Boolean values the integer zero denotes false while all other values denote true.

\item
The behavior of programs that pop more elements from the stack than were pushed onto it beforehand is undefined.
The same holds for programs that access memory using invalid addresses.

\item
After the execution of each program, the value on top of the stack is taken as the return code for the runtime environment.
In order to allow programs that do not explicitly push a return code, each program first pushes a value indicating successful execution.

\item
The original specification allowed emitting inline assembly for the Motorola M68000 family of microprocessors using integer numbers in the range 0 up to 65335 followed by an apostrophe.
The \ecs{} does not provide this form of code generation because it supports more than one hardware architecture.
All of them define their own instruction set encoding which affects the length of an instruction as well as endianness issues.

Instead, the \ecs{} allows emitting inline assembly by providing the actual inline assembly code as a string followed by an apostrophe.
This is a more generic solution which does not need the programmer to apply the instruction set encoding manually.
Nevertheless, representing binary code and data with plain integers is still possible using the double byte data directive.
\seeassembly\seemabk

\item
The \ecs{} defines two new operations that allow retrieving the address of variables and functions that are not defined by FALSE programs.
This enables interoperability with other languages as described in Section~\ref{sec:falinteroperability}.

\end{itemize}

Accessing external symbols as well as the emission of inline assembly is only available in the compilers tools provided by the \ecs{}.
The interpreter does not support these operations and issues a corresponding error message.

\section{FALSE Tools}

The \ecs{} provides several different tools that process programs written in FALSE\@.
\interface

The tools process FALSE programs in several consecutive stages.
In each stage, the internal representation of the program is changed and transformed.
Figure~\ref{fig:faldataflow} shows all stages and the different representations.

\begin{figure}
\flowgraph{
& \resource{FALSE\\source code} \ar[d] \\
& \converter{Lexer} \ar[d] \\
& \resource{tokens} \ar[d] \\
& \converter{Parser} \ar[d] \\
\converter{Serializer} \ar[d] & \resource{abstract\\syntax tree} \ar[l] \ar[d] \ar[r] & \converter{Pretty Printer} \ar[d] \\
\resource{internal\\representation} & \converter{Semantic\\Checker} \ar[d] & \resource{reformatted\\source code} \\
\converter{Interpreter} \ar@/l/[d] & \resource{attributed\\syntax tree} \ar[l] \ar[d] \ar[r] & \converter{Transpiler} \ar[d] \\
\resource{input/\\output} \ar@/r/[u] & \converter{Intermediate\\Code Emitter} \ar[d] & \resource{translated\\source code} \\
& \resource{intermediate\\code} \ar[d] \ar@/u/[r] & \converter{Optimizer} \ar@/d/[l] \\
& \converter{Machine Code\\Generator} \ar[d] \\
& \resource{object file} \\
}\caption{Data flow within the tools for FALSE}
\label{fig:faldataflow}
\end{figure}

\falprint
\falcheck
\faldump
\falrun
\falcpp
\falcode
\falamda
\falamdb
\falamdc
\falarma
\falarmb
\falarmc
\falarmcfpe
\falavr
\falavrtt
\falmabk
\falmibl
\falmipsa
\falmipsb
\falmmix
\falorok
\falppca
\falppcb
\falrisc
\falwasm

\section{Interoperability}\label{sec:falinteroperability}

The compilers for FALSE enable interoperability with other programming languages implemented by the \ecs{}.
The interoperability is enabled by a common intermediate code representation and calling convention. \seecode
The compilers define several intermediate code sections for each program and maintain the following naming convention.

The main program is defined in a code section called \texttt{main}.
For each function inside the main program, the compilers define a code section called \texttt{function} followed by an integer index.
The index is incremented with each function discovered lexicographically in the source code and begins with zero.
In addition, each variable that is actually used is defined in a data section with the same name.

Accessing code and data sections that are defined elsewhere is enabled by two special-purpose symbols that push the corresponding address onto the stack.
Since all stack operations operate on the actual call stack, it is even possible to provide arguments for external functions.
However, the arguments are always passed as addresses.
The return value of a function is pushed onto the stack after the call.

Functions defined by the compilers can also be called from other programs.
They first pop the return address from the stack such that the top of the stack corresponds to the last argument passed by the caller.
In the end, right before returning to the caller, the top of the stack is taken as return value of the function.

In addition, the compilers for the FALSE programming language also allow writing inline assembly code.
The corresponding symbol is described in Section~\ref{sec:falimplementation} and enables arbitrary access to any section.
\seeassembly

\concludechapter

// Oberon language definitions
// Copyright (C) Florian Negele

// This file is part of the Eigen Compiler Suite.

// The ECS is free software: you can redistribute it and/or modify
// it under the terms of the GNU General Public License as published by
// the Free Software Foundation, either version 3 of the License, or
// (at your option) any later version.

// The ECS is distributed in the hope that it will be useful,
// but WITHOUT ANY WARRANTY; without even the implied warranty of
// MERCHANTABILITY or FITNESS FOR A PARTICULAR PURPOSE.  See the
// GNU General Public License for more details.

// You should have received a copy of the GNU General Public License
// along with the ECS.  If not, see <https://www.gnu.org/licenses/>.

#ifndef CONSTANT
	#define CONSTANT(scope, constant, name, type, value) DECLARATION (Constant, scope, constant, name)
#endif

#ifndef DECLARATION
	#define DECLARATION(model, scope, declaration, name)
#endif

#ifndef DEFINITION
	#define DEFINITION(definition, name, result, first, second)
#endif

#ifndef FUNCTION
	#define FUNCTION(scope, function, name) PROCEDURE (scope, function, name)
#endif

#ifndef PROCEDURE
	#define PROCEDURE(scope, procedure, name) DECLARATION (Procedure, scope, procedure, name)
#endif

#ifndef SYMBOL
	#define SYMBOL(symbol, name)
#endif

#ifndef TYPE
	#define TYPE(scope, type, name, model, size) DECLARATION (Type, scope, type, name)
#endif

// keywords

SYMBOL (Array,      "ARRAY")
SYMBOL (Begin,      "BEGIN")
SYMBOL (By,         "BY")
SYMBOL (Case,       "CASE")
SYMBOL (Const,      "CONST")
SYMBOL (Div,        "DIV")
SYMBOL (Do,         "DO")
SYMBOL (Else,       "ELSE")
SYMBOL (Elsif,      "ELSIF")
SYMBOL (End,        "END")
SYMBOL (Exit,       "EXIT")
SYMBOL (For,        "FOR")
SYMBOL (If,         "IF")
SYMBOL (Import,     "IMPORT")
SYMBOL (In,         "IN")
SYMBOL (Is,         "IS")
SYMBOL (Loop,       "LOOP")
SYMBOL (Mod,        "MOD")
SYMBOL (Module,     "MODULE")
SYMBOL (Nil,        "NIL")
SYMBOL (Of,         "OF")
SYMBOL (Or,         "OR")
SYMBOL (Pointer,    "POINTER")
SYMBOL (Procedure,  "PROCEDURE")
SYMBOL (Record,     "RECORD")
SYMBOL (Repeat,     "REPEAT")
SYMBOL (Return,     "RETURN")
SYMBOL (Then,       "THEN")
SYMBOL (To,         "TO")
SYMBOL (Type,       "TYPE")
SYMBOL (Until,      "UNTIL")
SYMBOL (Var,        "VAR")
SYMBOL (While,      "WHILE")
SYMBOL (With,       "WITH")

// operators

SYMBOL (Plus,          "+")
SYMBOL (Minus,         "-")
SYMBOL (Asterisk,      "*")
SYMBOL (Slash,         "/")
SYMBOL (Not,           "~")
SYMBOL (And,           "&")
SYMBOL (Assign,        ":=")
SYMBOL (Arrow,         "^")
SYMBOL (Equal,         "=")
SYMBOL (Unequal,       "#")
SYMBOL (Less,          "<")
SYMBOL (Greater,       ">")
SYMBOL (LessEqual,     "<=")
SYMBOL (GreaterEqual,  ">=")
SYMBOL (Range,         "..")

// delimiters

SYMBOL (Bar,           "|")
SYMBOL (Colon,         ":")
SYMBOL (Comma,         ",")
SYMBOL (Dot,           ".")
SYMBOL (LeftBrace,     "{")
SYMBOL (LeftBracket,   "[")
SYMBOL (LeftParen,     "(")
SYMBOL (RightBrace,    "}")
SYMBOL (RightBracket,  "]")
SYMBOL (RightParen,    ")")
SYMBOL (Semicolon,     ";")

// literals

SYMBOL (Identifier,          "identifier")
SYMBOL (Integer,             "integer constant")
SYMBOL (BinaryInteger,       "binary integer constant")
SYMBOL (OctalInteger,        "octal integer constant")
SYMBOL (HexadecimalInteger,  "hexadecimal integer constant")
SYMBOL (Real,                "real constant")
SYMBOL (Character,           "character constant")
SYMBOL (SingleQuotedString,  "string")
SYMBOL (DoubleQuotedString,  "string")

// constants

CONSTANT (global,  False,  "FALSE",  Boolean,  false)
CONSTANT (global,  I,      "I",      Complex,  (Complex {0, 1}))
CONSTANT (global,  Inf,    "INF",    Real,     INFINITY)
CONSTANT (global,  Nan,    "NAN",    Real,     NAN)
CONSTANT (global,  True,   "TRUE",   Boolean,  true)

// types

TYPE (global,  Boolean,        "BOOLEAN",       Boolean,    1)
TYPE (global,  Cardinal,       "CARDINAL",      Unsigned,   layout.integer.size)
TYPE (global,  Character,      "CHAR",          Character,  1)
TYPE (global,  Complex,        "COMPLEX",       Complex,    layout.float_.size)
TYPE (global,  Complex32,      "COMPLEX32",     Complex,    4)
TYPE (global,  Complex64,      "COMPLEX64",     Complex,    8)
TYPE (global,  Generic,        "GENERIC",       Generic,    0)
TYPE (global,  HugeCardinal,   "HUGECARD",      Unsigned,   8)
TYPE (global,  HugeInteger,    "HUGEINT",       Signed,     8)
TYPE (global,  HugeSet,        "HUGESET",       Set,        8)
TYPE (global,  Integer,        "INTEGER",       Signed,     layout.integer.size)
TYPE (global,  Length,         "LENGTH",        Signed,     layout.pointer.size)
TYPE (global,  LongCardinal,   "LONGCARD",      Unsigned,   4)
TYPE (global,  LongComplex,    "LONGCOMPLEX",   Complex,    8)
TYPE (global,  LongInteger,    "LONGINT",       Signed,     4)
TYPE (global,  LongReal,       "LONGREAL",      Real,       8)
TYPE (global,  LongSet,        "LONGSET",       Set,        4)
TYPE (global,  Module,         "MODULE",        Module,     0)
TYPE (global,  Nil,            "NIL",           Nil,        0)
TYPE (global,  Procedure,      "PROCEDURE",     Procedure,  layout.function.size)
TYPE (global,  Real,           "REAL",          Real,       layout.float_.size)
TYPE (global,  Real32,         "REAL32",        Real,       4)
TYPE (global,  Real64,         "REAL64",        Real,       8)
TYPE (global,  Set,            "SET",           Set,        layout.integer.size)
TYPE (global,  Set16,          "SET16",         Set,        2)
TYPE (global,  Set32,          "SET32",         Set,        4)
TYPE (global,  Set64,          "SET64",         Set,        8)
TYPE (global,  Set8,           "SET8",          Set,        1)
TYPE (global,  ShortCardinal,  "SHORTCARD",     Unsigned,   2)
TYPE (global,  ShortComplex,   "SHORTCOMPLEX",  Complex,    4)
TYPE (global,  ShortInteger,   "SHORTINT",      Signed,     2)
TYPE (global,  ShortReal,      "SHORTREAL",     Real,       4)
TYPE (global,  ShortSet,       "SHORTSET",      Set,        2)
TYPE (global,  Signed16,       "SIGNED16",      Signed,     2)
TYPE (global,  Signed32,       "SIGNED32",      Signed,     4)
TYPE (global,  Signed64,       "SIGNED64",      Signed,     8)
TYPE (global,  Signed8,        "SIGNED8",       Signed,     1)
TYPE (global,  String,         "STRING",        String,     0)
TYPE (global,  Unsigned16,     "UNSIGNED16",    Unsigned,   2)
TYPE (global,  Unsigned32,     "UNSIGNED32",    Unsigned,   4)
TYPE (global,  Unsigned64,     "UNSIGNED64",    Unsigned,   8)
TYPE (global,  Unsigned8,      "UNSIGNED8",     Unsigned,   1)
TYPE (global,  Void,           "VOID",          Void,       0)

TYPE (system,  Address,    "ADDRESS",  Unsigned,  layout.pointer.size)
TYPE (system,  Byte,       "BYTE",     Byte,      1)
TYPE (system,  Pointer,    "PTR",      Any,       layout.pointer.size)
TYPE (system,  SystemSet,  "SET",      Set,       layout.pointer.size)

// function procedures

FUNCTION (global,  Abs,     "ABS")
FUNCTION (global,  Ash,     "ASH")
FUNCTION (global,  Cap,     "CAP")
FUNCTION (global,  Chr,     "CHR")
FUNCTION (global,  Entier,  "ENTIER")
FUNCTION (global,  Im,      "IM")
FUNCTION (global,  Len,     "LEN")
FUNCTION (global,  Long,    "LONG")
FUNCTION (global,  Max,     "MAX")
FUNCTION (global,  Min,     "MIN")
FUNCTION (global,  Odd,     "ODD")
FUNCTION (global,  Ord,     "ORD")
FUNCTION (global,  Ptr,     "PTR")
FUNCTION (global,  Re,      "RE")
FUNCTION (global,  Sel,     "SEL")
FUNCTION (global,  Short,   "SHORT")
FUNCTION (global,  Size,    "SIZE")

FUNCTION (system,  Adr,  "ADR")
FUNCTION (system,  Bit,  "BIT")
FUNCTION (system,  Lsh,  "LSH")
FUNCTION (system,  Rot,  "ROT")
FUNCTION (system,  Val,  "VAL")

// proper procedures

PROCEDURE (global,  Assert,   "ASSERT")
PROCEDURE (global,  Copy,     "COPY")
PROCEDURE (global,  Dec,      "DEC")
PROCEDURE (global,  Dispose,  "DISPOSE")
PROCEDURE (global,  Excl,     "EXCL")
PROCEDURE (global,  Halt,     "HALT")
PROCEDURE (global,  Ignore,   "IGNORE")
PROCEDURE (global,  Inc,      "INC")
PROCEDURE (global,  Incl,     "INCL")
PROCEDURE (global,  New,      "NEW")
PROCEDURE (global,  Trace,    "TRACE")

PROCEDURE (system,  Asm,            "ASM")
PROCEDURE (system,  Code,           "CODE")
PROCEDURE (system,  Get,            "GET")
PROCEDURE (system,  Move,           "MOVE")
PROCEDURE (system,  Put,            "PUT")
PROCEDURE (system,  SystemDispose,  "DISPOSE")
PROCEDURE (system,  SystemNew,      "NEW")

// external definitions

DEFINITION (Getchar,  "getchar",  globalIntegerType,  globalVoidType,     globalVoidType)
DEFINITION (Putchar,  "putchar",  globalIntegerType,  globalIntegerType,  globalVoidType)

#undef CONSTANT
#undef DECLARATION
#undef DEFINITION
#undef FUNCTION
#undef PROCEDURE
#undef SYMBOL
#undef TYPE

% Generic assembly language specification
% Copyright (C) Florian Negele

% This file is part of the Eigen Compiler Suite.

% Permission is granted to copy, distribute and/or modify this document
% under the terms of the GNU Free Documentation License, Version 1.3
% or any later version published by the Free Software Foundation.

% You should have received a copy of the GNU Free Documentation License
% along with the ECS.  If not, see <https://www.gnu.org/licenses/>.

% Generic documentation utilities
% Copyright (C) Florian Negele

% This file is part of the Eigen Compiler Suite.

% Permission is granted to copy, distribute and/or modify this document
% under the terms of the GNU Free Documentation License, Version 1.3
% or any later version published by the Free Software Foundation.

% You should have received a copy of the GNU Free Documentation License
% along with the ECS.  If not, see <https://www.gnu.org/licenses/>.

\providecommand{\cpp}{C\texttt{++}}
\providecommand{\opt}{_\mathit{opt}}
\providecommand{\tool}[1]{\texttt{#1}}
\providecommand{\version}{Version 0.0.40}
\providecommand{\resource}[1]{*++\txt{#1}}
\providecommand{\ecs}{Eigen Compiler Suite}
\providecommand{\changed}[1]{\underline{#1}}
\providecommand{\toolbox}[1]{\converter{#1}}
\providecommand{\file}{}\renewcommand{\file}[1]{\texttt{#1}}
\providecommand{\alignright}{\hfill\linebreak[0]\hspace*{\fill}}
\providecommand{\converter}[1]{*++[F][F*:white][F,:gray]\txt{#1}}
\providecommand{\documentation}{\ifbook chapter\else document\fi}
\providecommand{\Documentation}{\ifbook Chapter\else Document\fi}
\providecommand{\variable}[1]{\resource{\texttt{\small#1}\\variable}}
\providecommand{\documentationref}[2]{\ifbook\ref{#1}\else``\href{#1}{#2}''~\cite{#1}\fi}
\providecommand{\objfile}[1]{\texttt{#1}\index[runtime]{#1 object file@\texttt{#1} object file}}
\providecommand{\libfile}[1]{\texttt{#1}\index[runtime]{#1 library file@\texttt{#1} library file}}
\providecommand{\epigraph}[2]{\ifbook\begin{quote}\flushright\textit{#1}\par--- #2\end{quote}\fi}
\providecommand{\environmentvariable}[1]{\texttt{#1}\index{Environment variables!#1@\texttt{#1}}}
\providecommand{\environment}[1]{\texttt{#1}\index[environment]{#1 environment@\texttt{#1} environment}}
\providecommand{\toolsection}{}\renewcommand{\toolsection}[1]{\subsection{#1}\label{\prefix:#1}\tool{#1}}
\providecommand{\instruction}{}\renewcommand{\instruction}[2]{\noindent\qquad\pdftooltip{\texttt{#1}}{#2}\refstepcounter{instruction}\par}
\providecommand{\flowgraph}{}\renewcommand{\flowgraph}[1]{\par\sffamily\begin{displaymath}\xymatrix@=4ex{#1}\end{displaymath}\normalfont\par}
\providecommand{\instructionset}{}\renewcommand{\instructionset}[4]{\setcounter{instruction}{0}\begin{multicols}{\ifbook#3\else#4\fi}[{\captionof{table}[#2]{#2 (\ref*{#1:instructions}~instructions)}\label{tab:#1set}\vspace{-2ex}}]\footnotesize\raggedcolumns\input{#1.set}\label{#1:instructions}\end{multicols}}

\providecommand{\gpl}{GNU General Public License}
\providecommand{\rse}{ECS Runtime Support Exception}
\providecommand{\fdl}{\href{https://www.gnu.org/licenses/fdl.html}{GNU Free Documentation License}}

\providecommand{\docbegin}{}
\providecommand{\docend}{}
\providecommand{\doclabel}[1]{\hypertarget{#1}}
\providecommand{\doclink}[2]{\hyperlink{#1}{#2}}
\providecommand{\docsection}[3]{\hypertarget{#1}{\subsection{#2}}\label{sec:#1}\index[library]{#2@#3}}
\providecommand{\docsectionstar}[1]{}
\providecommand{\docsubbegin}{\begin{description}}
\providecommand{\docsubend}{\end{description}}
\providecommand{\docsubsection}[3]{\item[\hypertarget{#1}{#2}]\index[library]{#2@#3}}
\providecommand{\docsubsectionstar}[1]{\smallskip}
\providecommand{\docsubsubsection}[3]{\docsubsection{#1}{#2}{#3}}
\providecommand{\docsubsubsectionstar}[1]{}
\providecommand{\docsubsubsubsection}[3]{}
\providecommand{\docsubsubsubsectionstar}[1]{}
\providecommand{\doctable}{}

\providecommand{\debuggingtool}{}\renewcommand{\debuggingtool}{This tool is provided for debugging purposes.
It allows exposing and modifying an internal data structure that is usually not accessible.
}

\providecommand{\interface}{All tools accept command-line arguments which are taken as names of plain text files containing the source code.
If no arguments are provided, the standard input stream is used instead.
Output files are generated in the current working directory and have the same name as the input file being processed whereas the filename extension gets replaced by an appropriate suffix.
\seeinterface
}

\providecommand{\license}{\noindent Copyright \copyright{} Florian Negele\par\medskip\noindent
Permission is granted to copy, distribute and/or modify this document under the terms of the
\fdl{}, Version 1.3 or any later version published by the \href{https://fsf.org/}{Free Software Foundation}.
}

\providecommand{\ecslogosurface}{
\fill[darkgray] (0,0,0) -- (0,0,3) -- (0,3,3) -- (0,3,1) -- (0,4,1) -- (0,4,3) -- (0,5,3) -- (0,5,0) -- (0,2,0) -- (0,2,2) -- (0,1,2) -- (0,1,0) -- cycle;
\fill[gray] (0,5,0) -- (0,5,3) -- (1,5,3) -- (1,5,1) -- (2,5,1) -- (2,5,3) -- (3,5,3) -- (3,5,0) -- cycle;
\fill[lightgray] (0,0,0) -- (0,1,0) -- (2,1,0) -- (2,4,0) -- (1,4,0) -- (1,3,0) -- (2,3,0) -- (2,2,0) -- (0,2,0) -- (0,5,0) -- (3,5,0) -- (3,0,0) -- cycle;
\begin{scope}[line width=0.5]
\begin{scope}[gray]
\draw (0,0,0) -- (0,1,0);
\draw (2,1,0) -- (2,2,0);
\draw (0,1,2) -- (0,2,2);
\draw (0,2,0) -- (0,5,0);
\draw (2,3,0) -- (2,4,0);
\end{scope}
\begin{scope}[lightgray]
\draw (0,1,0) -- (0,1,2);
\draw (0,3,1) -- (0,3,3);
\draw (0,5,0) -- (0,5,3);
\draw (2,5,1) -- (2,5,3);
\end{scope}
\begin{scope}[white]
\draw (0,1,0) -- (2,1,0);
\draw (1,3,0) -- (2,3,0);
\draw (0,5,0) -- (3,5,0);
\end{scope}
\end{scope}
}

\providecommand{\ecslogo}[1]{
\begin{tikzpicture}[scale={(#1)/((sin(45)+cos(45))*3cm)},x={({-cos(45)*1cm},{sin(45)*sin(30)*1cm})},y={({0cm},{(cos(30)*1cm})},z={({sin(45)*1cm},{cos(45)*sin(30)*1cm})}]
\begin{scope}[darkgray,line width=1]
\draw (0,0,0) -- (0,0,3) -- (0,3,3) -- (2,3,3) -- (2,5,3) -- (3,5,3) -- (3,5,0) -- (3,0,0) -- cycle;
\draw (0,3,1) -- (0,4,1) -- (0,4,3) -- (0,5,3) -- (1,5,3) -- (1,5,1) -- (2,5,1);
\draw (1,3,0) -- (1,4,0) -- (2,4,0);
\end{scope}
\fill[darkgray] (2,0,0) -- (2,0,3) -- (2,5,3) -- (2,5,1) -- (2,4,1) -- (2,4,0) -- cycle;
\fill[lightgray] (2,0,2) -- (0,0,2) -- (0,2,2) -- (2,2,2) -- cycle;
\fill[gray] (0,1,0) -- (2,1,0) -- (2,1,2) -- (0,1,2) -- cycle;
\fill[gray] (0,3,1) -- (0,3,3) -- (2,3,3) -- (2,3,0) -- (1,3,0) -- (1,3,1) -- cycle;
\ecslogosurface
\end{tikzpicture}
}

\providecommand{\shadowedecslogo}[3]{
\begin{tikzpicture}[scale={(#1)/((sin(#2)+cos(#2))*3cm)},x={({-cos(#2)*1cm},{sin(#2)*sin(#3)*1cm})},y={({0cm},{(cos(#3)*1cm})},z={({sin(#2)*1cm},{cos(#2)*sin(#3)*1cm})}]
\shade[top color=lightgray!50!white,bottom color=white,middle color=lightgray!50!white] (0,0,0) -- (3,0,0) -- (3,{-0.5-3*sin(#2)*sin(#3)/cos(#3)},0) -- (0,-0.5,0) -- cycle;
\shade[top color=darkgray!50!gray,bottom color=white,middle color=darkgray!50!white] (0,0,0) -- (0,0,3) -- (0,{-0.5-3*cos(#2)*sin(#3)/cos(#3)},3) -- (0,-0.5,0) -- cycle;
\begin{scope}[y={({(cos(#2)+sin(#2))*0.5cm},{(cos(#2)*sin(#3)-sin(#2)*sin(#3))*0.5cm})}]
\useasboundingbox (3,0,0) -- (0,0,0) -- (0,0,3);
\shade[left color=darkgray!80!black,right color=lightgray,middle color=gray] (0,0,0) -- (0,1,0) -- (0,1,0.5) -- (0,2,0) -- (0,5,0) -- (0,5,3) -- (1,5,3) -- (1,4,3) -- (1,4,2.5) -- (1,3,3) -- (2,5,3) -- (3,5,3) -- (3,0,3) -- cycle;
\clip (0,0,0) -- (0,0,3) -- ({-3*sin(#2)/cos(#2)},0,0) -- cycle;
\shade[left color=darkgray,right color=lightgray!50!gray] (0,0,0) -- (0,1,0) -- (0,1,0.5) -- (0,2,0) -- (0,5,0) -- (0,5,3) -- (1,5,3) -- (1,4,3) -- (1,4,2.5) -- (1,3,3) -- (2,5,3) -- (3,5,3) -- (3,0,3) -- cycle;
\end{scope}
\shade[left color=darkgray,right color=darkgray!80!black] (2,0,0) -- (2,0,3) -- (2,5,3) -- (2,5,1) -- (2,4,1) -- (2,4,0) -- cycle;
\shade[left color=darkgray!90!black,right color=gray!80!darkgray] (2,0,2) -- (0,0,2) -- (0,2,2) -- (2,2,2) -- cycle;
\shade[top color=darkgray!90!black,bottom color=gray!80!darkgray] (0,1,0) -- (2,1,0) -- (2,1,2) -- (0,1,2) -- cycle;
\shade[top color=darkgray!90!black,bottom color=gray!80!darkgray] (0,3,1) -- (0,3,3) -- (2,3,3) -- (2,3,0) -- (1,3,0) -- (1,3,1) -- cycle;
\fill[gray] (2,1,0) -- (1.5,1,0.5) -- (0,1,0.5) -- (0,1,0) -- cycle;
\fill[gray] (1,3,2) -- (0.5,3,2) -- (0.5,3,3) -- (1,3,3) -- cycle;
\fill[gray] (2,3,0) -- (1.5,3,0.5) -- (1,3,0.5) -- (1,3,0) -- cycle;
\ecslogosurface
\end{tikzpicture}
}

\providecommand{\cpplogo}[1]{
\begin{tikzpicture}[scale=(#1)/512em]
\fill[gray] (435.2794,398.7159) -- (247.1911,507.3075) .. controls (236.3563,513.5642) and (218.6240,513.5642) .. (207.7892,507.3075) -- (19.7009,398.7159) .. controls (8.8646,392.4606) and (0.0000,377.1043) .. (0.0000,364.5924) -- (0.0000,147.4076) .. controls (0.8430,132.8363) and (8.2856,120.7683) .. (19.7009,113.2842) -- (207.7892,4.6926) .. controls (218.6240,-1.5642) and (236.3564,-1.5642) .. (247.1911,4.6926) -- (435.2794,113.2842) .. controls (447.5273,121.4304) and (454.4987,133.6918) .. (454.9803,147.4076) -- (454.9803,364.5924) .. controls (454.5404,377.7571) and (446.6566,391.0351) .. (435.2794,398.7159) -- cycle(75.8301,255.9993) .. controls (74.9389,404.0881) and (273.2892,469.4783) .. (358.8263,331.8769) -- (293.1917,293.8965) .. controls (253.5702,359.4301) and (155.1909,335.9977) .. (151.6601,255.9993) .. controls (152.7204,182.2703) and (249.4137,148.0211) .. (293.1961,218.1065) -- (358.8308,180.1276) .. controls (283.4477,49.2645) and (79.6318,96.3470) .. (75.8301,255.9993) -- cycle(379.1503,247.5747) -- (362.2982,247.5747) -- (362.2982,230.7226) -- (345.4490,230.7226) -- (345.4490,247.5747) -- (328.5969,247.5747) -- (328.5969,264.4254) -- (345.4490,264.4254) -- (345.4490,281.2759) -- (362.2982,281.2759) -- (362.2982,264.4254) -- (379.1503,264.4254) -- cycle(442.3420,247.5747) -- (425.4899,247.5747) -- (425.4899,230.7226) -- (408.6408,230.7226) -- (408.6408,247.5747) -- (391.7886,247.5747) -- (391.7886,264.4254) -- (408.6408,264.4254) -- (408.6408,281.2759) -- (425.4899,281.2759) -- (425.4899,264.4254) -- (442.3420,264.4254) -- cycle;
\end{tikzpicture}
}

\providecommand{\fallogo}[1]{
\begin{tikzpicture}[scale=(#1)/512em]
\fill[gray] (185.7774,0.0000) .. controls (200.4486,15.9798) and (226.8966,8.7148) .. (235.0426,31.5836) .. controls (249.5297,58.0598) and (247.9581,97.9161) .. (280.3335,110.9762) .. controls (309.1690,120.3496) and (337.8406,104.2727) .. (366.5753,103.9379) .. controls (373.4449,111.5171) and (379.2885,128.2574) .. (383.9755,108.9744) .. controls (396.6979,102.5615) and (437.2808,107.6681) .. (426.9652,124.3252) .. controls (408.9822,121.0785) and (412.4742,146.0729) .. (426.5192,131.4996) .. controls (433.8413,120.8489) and (465.1541,126.5522) .. (441.9067,135.7950) .. controls (396.1879,157.7478) and (344.1112,161.5079) .. (298.5528,183.5702) .. controls (277.7471,193.5198) and (284.6941,218.7163) .. (285.2127,236.9640) .. controls (292.3599,316.2826) and (307.3929,394.6311) .. (317.1198,473.6154) .. controls (329.0637,505.4736) and (292.1195,528.5004) .. (265.9183,511.2761) .. controls (237.9284,499.2462) and (237.3684,465.2681) .. (230.9102,439.9421) .. controls (218.6692,374.3397) and (215.6307,306.9662) .. (198.1732,242.3977) .. controls (183.1379,232.7444) and (164.4245,256.0298) .. (149.0430,261.4799) .. controls (116.9328,279.2585) and (87.1822,308.5851) .. (48.2293,307.8914) .. controls (21.3220,306.9037) and (-15.9107,281.8761) .. (7.2921,252.7908) .. controls (29.7799,220.6177) and (67.5177,204.3028) .. (100.9287,185.9449) .. controls (130.8217,170.8906) and (161.1548,156.5903) .. (191.0278,141.5847) .. controls (196.1738,120.0520) and (186.6049,95.2409) .. (186.8382,72.4353) .. controls (185.5234,48.4204) and (183.1700,23.9341) .. (185.7774,0.0000) -- cycle;
\end{tikzpicture}
}

\providecommand{\oblogo}[1]{
\begin{tikzpicture}[scale=(#1)/512em]
\fill[gray] (160.3865,208.9117) .. controls (154.0879,214.6478) and (149.0735,221.2409) .. (145.4125,228.5384) .. controls (184.8790,248.4273) and (234.7122,269.8787) .. (297.5493,291.8782) .. controls (300.3943,281.4769) and (300.9552,268.7619) .. (300.4023,255.2389) .. controls (248.9909,244.7891) and (200.0310,225.9279) .. (160.3865,208.9117) -- cycle(225.7398,392.6996) .. controls (308.0209,392.1716) and (359.3326,345.9277) .. (368.7203,285.2098) .. controls (376.6742,197.1784) and (311.7194,141.3342) .. (205.4287,142.1456) .. controls (139.9485,141.4804) and (88.7155,166.1957) .. (73.5775,228.0086) .. controls (52.0297,320.3408) and (123.4078,391.0103) .. (225.7398,392.6996) -- cycle(216.0739,176.4733) .. controls (268.9183,179.2424) and (315.8292,206.5488) .. (312.7454,265.1139) .. controls (313.2769,315.6384) and (286.5993,353.4946) .. (216.6040,355.7934) .. controls (162.4657,355.7934) and (126.0914,317.5023) .. (126.0914,260.5103) .. controls (126.1733,214.2900) and (163.3363,176.2849) .. (216.0739,176.4733) -- cycle(76.4897,189.1754) .. controls (13.1586,147.5631) and (0.0000,119.4207) .. (0.0000,119.4207) -- (90.6499,170.1632) .. controls (85.3004,175.8497) and (80.5994,182.1633) .. (76.4897,189.1754) -- cycle(353.9486,119.3004) -- (402.9482,119.3004) .. controls (427.0025,137.0797) and (450.9893,162.7034) .. (474.9529,191.0213) .. controls (509.3540,228.5339) and (531.3391,294.2091) .. (487.8149,312.1206) .. controls (462.8165,324.7652) and (394.3874,316.8943) .. (373.8912,313.6651) .. controls (379.9291,297.7449) and (383.2899,278.4204) .. (381.4989,257.7214) .. controls (420.3069,248.0321) and (421.9610,218.3461) .. (407.7867,192.6417) .. controls (391.1113,162.4018) and (370.1114,132.9097) .. (353.9486,119.3004) -- cycle;
\end{tikzpicture}
}

\providecommand{\markuptable}{
\begin{table}
\sffamily\centering
\begin{tabular}{@{}lcl@{}}
\toprule
\texttt{//italics//} & $\rightarrow$ & \textit{italics} \\
\midrule
\texttt{**bold**} & $\rightarrow$ & \textbf{bold} \\
\midrule
\texttt{\# ordered list} & & 1 ordered list \\
\texttt{\# second item} & $\rightarrow$ & 2 second item \\
\texttt{\#\# sub item} & & \hspace{1em} 1 sub item \\
\midrule
\texttt{* unordered list} & & $\bullet$ unordered list \\
\texttt{* second item} & $\rightarrow$ & $\bullet$ second item \\
\texttt{** sub item} & & \hspace{1em} $\bullet$ sub item \\
\midrule
\texttt{link to [[label]]} & $\rightarrow$ & link to \underline{label} \\
\midrule
\texttt{<{}<label>{}> definition } & $\rightarrow$ & definition \\
\midrule
\texttt{[[url|link name]]} & $\rightarrow$ & \underline{link name} \\
\midrule\addlinespace
\texttt{= large heading} & & {\Large large heading} \smallskip \\
\texttt{== medium heading} & $\rightarrow$ & {\large medium heading} \\
\texttt{=== small heading} & & small heading \\
\midrule
\texttt{no line break} & & no line break for paragraphs \\
\texttt{for paragraphs} & $\rightarrow$ \\
& & use empty line \\
\texttt{use empty line} \\
\midrule
\texttt{force\textbackslash\textbackslash line break} & $\rightarrow$ & force \\
& & line break \\
\midrule
\texttt{horizontal line} & $\rightarrow$ & horizontal line \\
\texttt{----} & & \hrulefill \\
\midrule
\texttt{|=a|=table|=header} & & \underline{a \enspace table \enspace header} \\
\texttt{|a|table|row} & $\rightarrow$ & a \enspace table \enspace row \\
\texttt{|b|table|row} & & b \enspace table \enspace row \\
\midrule
\texttt{\{\{\{} \\
\texttt{unformatted} & $\rightarrow$ & \texttt{unformatted} \\
\texttt{code} & & \texttt{code} \\
\texttt{\}\}\}} \\
\midrule\addlinespace
\texttt{@ new article} & & {\Large 1.\ new article} \smallskip \\
\texttt{@ second article} & $\rightarrow$ & {\Large 2.\ second article} \smallskip \\
\texttt{@@ sub article} & & {\large 2.1.\ sub article} \\
\bottomrule
\end{tabular}
\normalfont\caption{Elements of the generic documentation markup language}
\label{tab:docmarkup}
\end{table}
}

\providecommand{\startchapter}[4]{
\documentclass[11pt,a4paper]{article}
\usepackage{booktabs}
\usepackage[format=hang,labelfont=bf]{caption}
\usepackage{changepage}
\usepackage[T1]{fontenc}
\usepackage[margin=2cm]{geometry}
\usepackage{hyperref}
\usepackage[american]{isodate}
\usepackage{lmodern}
\usepackage{longtable}
\usepackage{mathptmx}
\usepackage{microtype}
\usepackage[toc]{multitoc}
\usepackage{multirow}
\usepackage[all]{nowidow}
\usepackage{pdfcomment}
\usepackage{syntax}
\usepackage{tikz}
\usepackage[all]{xy}
\hypersetup{pdfborder={0 0 0},bookmarksnumbered=true,pdftitle={\ecs{}: #2},pdfauthor={Florian Negele},pdfsubject={\ecs{}},pdfkeywords={#1}}
\setlength{\grammarindent}{8em}\setlength{\grammarparsep}{0.2ex}
\setlength{\columnsep}{2em}
\newcommand{\prefix}{}
\newcounter{instruction}
\bibliographystyle{unsrt}
\renewcommand{\index}[2][]{}
\renewcommand{\arraystretch}{1.05}
\renewcommand{\floatpagefraction}{0.7}
\renewcommand{\syntleft}{\itshape}\renewcommand{\syntright}{}
\title{\vspace{-5ex}\Huge{\ecs{}}\medskip\hrule}
\author{\huge{#2}}
\date{\medskip\version}
\newif\ifbook\bookfalse
\pagestyle{headings}
\frenchspacing
\begin{document}
\maketitle\thispagestyle{empty}\noindent#4\setlength{\columnseprule}{0.4pt}\tableofcontents\setlength{\columnseprule}{0pt}\vfill\pagebreak[3]\null\vfill\bigskip\noindent
\parbox{\textwidth-4em}{\license The contents of this \documentation{} are part of the \href{manual}{\ecs{} User Manual}~\cite{manual} and correspond to Chapter ``\href{manual\##3}{#1}''.\alignright\mbox{\today}}
\parbox{4em}{\flushright\ecslogo{3em}}
\clearpage
}

\providecommand{\concludechapter}{
\vfill\pagebreak[3]\null\vfill
\thispagestyle{myheadings}\markright{REFERENCES}
\noindent\begin{minipage}{\textwidth}\begin{multicols}{2}[\section*{References}]
\renewcommand{\section}[2]{}\small\bibliography{references}
\end{multicols}\end{minipage}\end{document}
}

\providecommand{\startpresentation}[2]{
\documentclass[14pt,aspectratio=43,usepdftitle=false]{beamer}
\usepackage{booktabs}
\usepackage{etex}
\usepackage{multicol}
\usepackage{tikz}
\usepackage[all]{xy}
\bibliographystyle{unsrt}
\setlength{\columnsep}{1em}
\setlength{\leftmargini}{1em}
\setbeamercolor{title}{fg=black}
\setbeamercolor{structure}{fg=darkgray}
\setbeamercolor{bibliography item}{fg=darkgray}
\setbeamerfont{title}{series=\bfseries}
\setbeamerfont{subtitle}{series=\normalfont}
\setbeamerfont*{frametitle}{parent=title}
\setbeamerfont{block title}{series=\bfseries}
\setbeamerfont*{framesubtitle}{parent=subtitle}
\setbeamersize{text margin left=1em,text margin right=1em}
\setbeamertemplate{navigation symbols}{}
\setbeamertemplate{itemize item}[circle]{}
\setbeamertemplate{bibliography item}[triangle]{}
\setbeamertemplate{bibliography entry author}{\usebeamercolor[fg]{bibliography item}}
\setbeamertemplate{frametitle}{\medskip\usebeamerfont{frametitle}\color{gray}\raisebox{-2.5ex}[0ex][0ex]{\rule{0.1em}{4.5ex}}}
\addtobeamertemplate{frametitle}{}{\hspace{0.4em}\usebeamercolor[fg]{title}\insertframetitle\par\vspace{0.2ex}\hspace{0.5em}\usebeamerfont{framesubtitle}\insertframesubtitle}
\hypersetup{pdfborder={0 0 0},bookmarksnumbered=true,bookmarksopen=true,bookmarksopenlevel=0,pdftitle={\ecs{}: #1},pdfauthor={Florian Negele},pdfsubject={\ecs{}},pdfkeywords={#1}}
\renewcommand{\flowgraph}[1]{\resizebox{\textwidth}{!}{$$\xymatrix{##1}$$}}
\title{\ecs{}\medskip\hrule\medskip}
\institute{\shadowedecslogo{5em}{30}{15}}
\date{\version}
\subtitle{#1}
\begin{document}
\begin{frame}[plain]\titlepage\nocite{manual}\end{frame}
\begin{frame}{Contents}{#1}\begin{center}\tableofcontents\end{center}\end{frame}
}

\providecommand{\concludepresentation}{
\begin{frame}{References}\begin{footnotesize}\setlength{\columnseprule}{0.4pt}\begin{multicols}{2}\bibliography{references}\end{multicols}\end{footnotesize}\end{frame}
\end{document}
}

\providecommand{\startbook}[1]{
\documentclass[10pt,paper=17cm:24cm,DIV=13,twoside=semi,headings=normal,numbers=noendperiod,cleardoublepage=plain]{scrbook}
\usepackage{atveryend}
\usepackage{booktabs}
\usepackage{caption}
\usepackage{changepage}
\usepackage[T1]{fontenc}
\usepackage{imakeidx}
\usepackage{hyperref}
\usepackage[american]{isodate}
\usepackage{lmodern}
\usepackage{longtable}
\usepackage{mathptmx}
\usepackage[final]{microtype}
\usepackage{multicol}
\usepackage{multirow}
\usepackage[all]{nowidow}
\usepackage{pdfcomment}
\usepackage{scrlayer-scrpage}
\usepackage{setspace}
\usepackage{syntax}
\usepackage[eventxtindent=4pt,oddtxtexdent=4pt]{thumbs}
\usepackage{tikz}
\usepackage[all]{xy}
\hyphenation{Micro-Blaze Open-Cores Open-RISC Power-PC}
\hypersetup{pdfborder={0 0 0},bookmarksnumbered=true,bookmarksopen=true,bookmarksopenlevel=0,pdftitle={\ecs{}: #1},pdfauthor={Florian Negele},pdfsubject={\ecs{}},pdfkeywords={#1}}
\setlength{\grammarindent}{8em}\setlength{\grammarparsep}{0.7ex}
\setkomafont{captionlabel}{\usekomafont{descriptionlabel}}
\renewcommand{\arraystretch}{1.05}\setstretch{1.1}
\renewcommand{\chapterformat}{\thechapter\autodot\enskip\raisebox{-1ex}[0ex][0ex]{\color{gray}\rule{0.1em}{3.5ex}}\enskip}
\renewcommand{\startchapter}[4]{\hypertarget{##3}{\chapter{##1}}\label{##3}##4\addthumb{##1}{\LARGE\sffamily\bfseries\thechapter}{white}{gray}\renewcommand{\prefix}{##3}}
\renewcommand{\concludechapter}{\clearpage{\stopthumb\cleardoublepage}}
\renewcommand{\syntleft}{\itshape}\renewcommand{\syntright}{}
\renewcommand{\floatpagefraction}{0.7}
\renewcommand{\partheademptypage}{}
\DeclareMicrotypeAlias{lmss}{cmr}
\newcommand{\prefix}{}
\newcounter{instruction}
\bibliographystyle{unsrt}
\newif\ifbook\booktrue
\makeindex[intoc,title=Index]
\makeindex[intoc,name=tools,title=Index of Tools,columns=3]
\makeindex[intoc,name=library,title=Index of Library Names]
\makeindex[intoc,name=runtime,title=Index of Runtime Support]
\makeindex[intoc,name=environment,title=Index of Target Environments]
\indexsetup{toclevel=chapter,headers={\indexname}{\indexname}}
\frenchspacing
\begin{document}
\pagenumbering{alph}
\begin{titlepage}\centering
\huge\sffamily\null\vfill\textbf{\ecs{}}\bigskip\hrule\bigskip#1
\normalsize\normalfont\vfill\vfill\shadowedecslogo{10em}{30}{15}
\large\vfill\vfill\version
\end{titlepage}
\null\vfill
\thispagestyle{empty}
\noindent\today\par\medskip
\license A copy of this license is included in Appendix~\ref{fdl} on page~\pageref{fdl}.
All product names used herein are for identification purposes only and may be trademarks of their respective companies.
\concludechapter
\frontmatter
\setcounter{tocdepth}{1}
\tableofcontents
\setcounter{tocdepth}{2}
\concludechapter
\listoffigures
\concludechapter
\listoftables
\concludechapter
}

\providecommand{\concludebook}{
\backmatter
\addtocontents{toc}{\protect\setcounter{tocdepth}{-1}}
\phantomsection\addcontentsline{toc}{part}{Bibliography}
\bibliography{references}
\concludechapter
\phantomsection\addcontentsline{toc}{part}{Indexes}
\printindex
\concludechapter
\indexprologue{\label{idx:tools}}
\printindex[tools]
\concludechapter
\printindex[library]
\concludechapter
\indexprologue{\label{idx:runtime}}
\printindex[runtime]
\concludechapter
\indexprologue{\label{idx:environment}}
\printindex[environment]
\concludechapter
\pagestyle{empty}\pagenumbering{Alph}\null\clearpage
\null\vfill\centering\ecslogo{4em}\par\medskip\license
\end{document}
}

% chapter references

\providecommand{\seedocumentationref}{}\renewcommand{\seedocumentationref}[3]{#1, see \Documentation{}~\documentationref{#2}{#3}. }
\providecommand{\seeinterface}{}\renewcommand{\seeinterface}{\ifbook See \Documentation{}~\documentationref{interface}{User Interface} for more information about the common user interface of all of these tools. \fi}
\providecommand{\seeguide}{}\renewcommand{\seeguide}{\seedocumentationref{For basic examples of using some of these tools in practice}{guide}{User Guide}}
\providecommand{\seecpp}{}\renewcommand{\seecpp}{\seedocumentationref{For more information about the \cpp{} programming language and its implementation by the \ecs{}}{cpp}{User Manual for \cpp{}}}
\providecommand{\seefalse}{}\renewcommand{\seefalse}{\seedocumentationref{For more information about the FALSE programming language and its implementation by the \ecs{}}{false}{User Manual for FALSE}}
\providecommand{\seeoberon}{}\renewcommand{\seeoberon}{\seedocumentationref{For more information about the Oberon programming language and its implementation by the \ecs{}}{oberon}{User Manual for Oberon}}
\providecommand{\seeassembly}{}\renewcommand{\seeassembly}{\seedocumentationref{For more information about the generic assembly language and how to use it}{assembly}{Generic Assembly Language Specification}}
\providecommand{\seeamd}{}\renewcommand{\seeamd}{\seedocumentationref{For more information about how the \ecs{} supports the AMD64 hardware architecture}{amd64}{AMD64 Hardware Architecture Support}}
\providecommand{\seearm}{}\renewcommand{\seearm}{\seedocumentationref{For more information about how the \ecs{} supports the ARM hardware architecture}{arm}{ARM Hardware Architecture Support}}
\providecommand{\seeavr}{}\renewcommand{\seeavr}{\seedocumentationref{For more information about how the \ecs{} supports the AVR hardware architecture}{avr}{AVR Hardware Architecture Support}}
\providecommand{\seeavrtt}{}\renewcommand{\seeavrtt}{\seedocumentationref{For more information about how the \ecs{} supports the AVR32 hardware architecture}{avr32}{AVR32 Hardware Architecture Support}}
\providecommand{\seemabk}{}\renewcommand{\seemabk}{\seedocumentationref{For more information about how the \ecs{} supports the M68000 hardware architecture}{m68k}{M68000 Hardware Architecture Support}}
\providecommand{\seemibl}{}\renewcommand{\seemibl}{\seedocumentationref{For more information about how the \ecs{} supports the MicroBlaze hardware architecture}{mibl}{MicroBlaze Hardware Architecture Support}}
\providecommand{\seemips}{}\renewcommand{\seemips}{\seedocumentationref{For more information about how the \ecs{} supports the MIPS32 and MIPS64 hardware architectures}{mips}{MIPS Hardware Architecture Support}}
\providecommand{\seemmix}{}\renewcommand{\seemmix}{\seedocumentationref{For more information about how the \ecs{} supports the MMIX hardware architecture}{mmix}{MMIX Hardware Architecture Support}}
\providecommand{\seeorok}{}\renewcommand{\seeorok}{\seedocumentationref{For more information about how the \ecs{} supports the OpenRISC 1000 hardware architecture}{or1k}{OpenRISC 1000 Hardware Architecture Support}}
\providecommand{\seeppc}{}\renewcommand{\seeppc}{\seedocumentationref{For more information about how the \ecs{} supports the PowerPC hardware architecture}{ppc}{PowerPC Hardware Architecture Support}}
\providecommand{\seerisc}{}\renewcommand{\seerisc}{\seedocumentationref{For more information about how the \ecs{} supports the RISC hardware architecture}{risc}{RISC Hardware Architecture Support}}
\providecommand{\seewasm}{}\renewcommand{\seewasm}{\seedocumentationref{For more information about how the \ecs{} supports the WebAssembly architecture}{wasm}{WebAssembly Architecture Support}}
\providecommand{\seedocumentation}{}\renewcommand{\seedocumentation}{\seedocumentationref{For more information about generic documentations and their generation by the \ecs{}}{documentation}{Generic Documentation Generation}}
\providecommand{\seedebugging}{}\renewcommand{\seedebugging}{\seedocumentationref{For more information about debugging information and its representation}{debugging}{Debugging Information Representation}}
\providecommand{\seecode}{}\renewcommand{\seecode}{\seedocumentationref{For more information about intermediate code and its purpose}{code}{Intermediate Code Representation}}
\providecommand{\seeobject}{}\renewcommand{\seeobject}{\seedocumentationref{For more information about object files and their purpose}{object}{Object File Representation}}

% generic documentation tools

\providecommand{\docprint}{
\toolsection{docprint} is a pretty printer for generic documentations.
It reformats generic documentations and writes it to the standard output stream.
\debuggingtool
\flowgraph{\resource{generic\\documentation} \ar[r] & \toolbox{docprint} \ar[r] & \resource{generic\\documentation}}
\seedocumentation
}

\providecommand{\doccheck}{
\toolsection{doccheck} is a syntactic and semantic checker for generic documentations.
It just performs syntactic and semantic checks on generic documentations and writes its diagnostic messages to the standard error stream.
\debuggingtool
\flowgraph{\resource{generic\\documentation} \ar[r] & \toolbox{doccheck} \ar[r] & \resource{diagnostic\\messages}}
\seedocumentation
}

\providecommand{\dochtml}{
\toolsection{dochtml} is an HTML documentation generator for generic documentations.
It processes several generic documentations and assembles all information therein into an HTML document.
\debuggingtool
\flowgraph{\resource{generic\\documentation} \ar[r] & \toolbox{dochtml} \ar[r] & \resource{HTML\\document}}
\seedocumentation
}

\providecommand{\doclatex}{
\toolsection{doclatex} is a Latex documentation generator for generic documentations.
It processes several generic documentations and assembles all information therein into a Latex document.
\debuggingtool
\flowgraph{\resource{generic\\documentation} \ar[r] & \toolbox{doclatex} \ar[r] & \resource{Latex\\document}}
\seedocumentation
}

% intermediate code tools

\providecommand{\cdcheck}{
\toolsection{cdcheck} is a syntactic and semantic checker for intermediate code.
It just performs syntactic and semantic checks on programs written in intermediate code and writes its diagnostic messages to the standard error stream.
\debuggingtool
\flowgraph{\resource{intermediate\\code} \ar[r] & \toolbox{cdcheck} \ar[r] & \resource{diagnostic\\messages}}
\seeassembly\seecode
}

\providecommand{\cdopt}{
\toolsection{cdopt} is an optimizer for intermediate code.
It performs various optimizations on programs written in intermediate code and writes the result to the standard output stream.
\debuggingtool
\flowgraph{\resource{intermediate\\code} \ar[r] & \toolbox{cdopt} \ar[r] & \resource{optimized\\code}}
\seeassembly\seecode
}

\providecommand{\cdrun}{
\toolsection{cdrun} is an interpreter for intermediate code.
It processes and executes programs written in intermediate code.
The following code sections are predefined and have the usual semantics:
\texttt{abort}, \texttt{\_Exit}, \texttt{fflush}, \texttt{floor}, \texttt{fputc}, \texttt{free}, \texttt{getchar}, \texttt{malloc}, and \texttt{putchar}.
Diagnostic messages about invalid operations include the name of the executed code section and the index of the erroneous instruction.
\debuggingtool
\flowgraph{\resource{intermediate\\code} \ar[r] & \toolbox{cdrun} \ar@/u/[r] & \resource{input/\\output} \ar@/d/[l]}
\seeassembly\seecode
}

\providecommand{\cdamda}{
\toolsection{cdamd16} is a compiler for intermediate code targeting the AMD64 hardware architecture.
It generates machine code for AMD64 processors from programs written in intermediate code and stores it in corresponding object files.
The compiler generates machine code for the 16-bit operating mode defined by the AMD64 architecture.
It also creates a debugging information file as well as an assembly file containing a listing of the generated machine code.
\debuggingtool
\flowgraph{\resource{intermediate\\code} \ar[r] & \toolbox{cdamd16} \ar[r] \ar[d] \ar[rd] & \resource{object file} \\ & \resource{assembly\\listing} & \resource{debugging\\information}}
\seeassembly\seeamd\seeobject\seecode\seedebugging
}

\providecommand{\cdamdb}{
\toolsection{cdamd32} is a compiler for intermediate code targeting the AMD64 hardware architecture.
It generates machine code for AMD64 processors from programs written in intermediate code and stores it in corresponding object files.
The compiler generates machine code for the 32-bit operating mode defined by the AMD64 architecture.
It also creates a debugging information file as well as an assembly file containing a listing of the generated machine code.
\debuggingtool
\flowgraph{\resource{intermediate\\code} \ar[r] & \toolbox{cdamd32} \ar[r] \ar[d] \ar[rd] & \resource{object file} \\ & \resource{assembly\\listing} & \resource{debugging\\information}}
\seeassembly\seeamd\seeobject\seecode\seedebugging
}

\providecommand{\cdamdc}{
\toolsection{cdamd64} is a compiler for intermediate code targeting the AMD64 hardware architecture.
It generates machine code for AMD64 processors from programs written in intermediate code and stores it in corresponding object files.
The compiler generates machine code for the 64-bit operating mode defined by the AMD64 architecture.
It also creates a debugging information file as well as an assembly file containing a listing of the generated machine code.
\debuggingtool
\flowgraph{\resource{intermediate\\code} \ar[r] & \toolbox{cdamd64} \ar[r] \ar[d] \ar[rd] & \resource{object file} \\ & \resource{assembly\\listing} & \resource{debugging\\information}}
\seeassembly\seeamd\seeobject\seecode\seedebugging
}

\providecommand{\cdarma}{
\toolsection{cdarma32} is a compiler for intermediate code targeting the ARM hardware architecture.
It generates machine code for ARM processors executing A32 instructions from programs written in intermediate code and stores it in corresponding object files.
It also creates a debugging information file as well as an assembly file containing a listing of the generated machine code.
\debuggingtool
\flowgraph{\resource{intermediate\\code} \ar[r] & \toolbox{cdarma32} \ar[r] \ar[d] \ar[rd] & \resource{object file} \\ & \resource{assembly\\listing} & \resource{debugging\\information}}
\seeassembly\seearm\seeobject\seecode\seedebugging
}

\providecommand{\cdarmb}{
\toolsection{cdarma64} is a compiler for intermediate code targeting the ARM hardware architecture.
It generates machine code for ARM processors executing A64 instructions from programs written in intermediate code and stores it in corresponding object files.
It also creates a debugging information file as well as an assembly file containing a listing of the generated machine code.
\debuggingtool
\flowgraph{\resource{intermediate\\code} \ar[r] & \toolbox{cdarma64} \ar[r] \ar[d] \ar[rd] & \resource{object file} \\ & \resource{assembly\\listing} & \resource{debugging\\information}}
\seeassembly\seearm\seeobject\seecode\seedebugging
}

\providecommand{\cdarmc}{
\toolsection{cdarmt32} is a compiler for intermediate code targeting the ARM hardware architecture.
It generates machine code for ARM processors without floating-point extension executing T32 instructions from programs written in intermediate code and stores it in corresponding object files.
It also creates a debugging information file as well as an assembly file containing a listing of the generated machine code.
\debuggingtool
\flowgraph{\resource{intermediate\\code} \ar[r] & \toolbox{cdarmt32} \ar[r] \ar[d] \ar[rd] & \resource{object file} \\ & \resource{assembly\\listing} & \resource{debugging\\information}}
\seeassembly\seearm\seeobject\seecode\seedebugging
}

\providecommand{\cdarmcfpe}{
\toolsection{cdarmt32fpe} is a compiler for intermediate code targeting the ARM hardware architecture.
It generates machine code for ARM processors with floating-point extension executing T32 instructions from programs written in intermediate code and stores it in corresponding object files.
It also creates a debugging information file as well as an assembly file containing a listing of the generated machine code.
\debuggingtool
\flowgraph{\resource{intermediate\\code} \ar[r] & \toolbox{cdarmt32fpe} \ar[r] \ar[d] \ar[rd] & \resource{object file} \\ & \resource{assembly\\listing} & \resource{debugging\\information}}
\seeassembly\seearm\seeobject\seecode\seedebugging
}

\providecommand{\cdavr}{
\toolsection{cdavr} is a compiler for intermediate code targeting the AVR hardware architecture.
It generates machine code for AVR processors from programs written in intermediate code and stores it in corresponding object files.
It also creates a debugging information file as well as an assembly file containing a listing of the generated machine code.
\debuggingtool
\flowgraph{\resource{intermediate\\code} \ar[r] & \toolbox{cdavr} \ar[r] \ar[d] \ar[rd] & \resource{object file} \\ & \resource{assembly\\listing} & \resource{debugging\\information}}
\seeassembly\seeavr\seeobject\seecode\seedebugging
}

\providecommand{\cdavrtt}{
\toolsection{cdavr32} is a compiler for intermediate code targeting the AVR32 hardware architecture.
It generates machine code for AVR32 processors from programs written in intermediate code and stores it in corresponding object files.
It also creates a debugging information file as well as an assembly file containing a listing of the generated machine code.
\debuggingtool
\flowgraph{\resource{intermediate\\code} \ar[r] & \toolbox{cdavr32} \ar[r] \ar[d] \ar[rd] & \resource{object file} \\ & \resource{assembly\\listing} & \resource{debugging\\information}}
\seeassembly\seeavrtt\seeobject\seecode\seedebugging
}

\providecommand{\cdmabk}{
\toolsection{cdm68k} is a compiler for intermediate code targeting the M68000 hardware architecture.
It generates machine code for M68000 processors from programs written in intermediate code and stores it in corresponding object files.
It also creates a debugging information file as well as an assembly file containing a listing of the generated machine code.
\debuggingtool
\flowgraph{\resource{intermediate\\code} \ar[r] & \toolbox{cdm68k} \ar[r] \ar[d] \ar[rd] & \resource{object file} \\ & \resource{assembly\\listing} & \resource{debugging\\information}}
\seeassembly\seemabk\seeobject\seecode\seedebugging
}

\providecommand{\cdmibl}{
\toolsection{cdmibl} is a compiler for intermediate code targeting the MicroBlaze hardware architecture.
It generates machine code for MicroBlaze processors from programs written in intermediate code and stores it in corresponding object files.
It also creates a debugging information file as well as an assembly file containing a listing of the generated machine code.
\debuggingtool
\flowgraph{\resource{intermediate\\code} \ar[r] & \toolbox{cdmibl} \ar[r] \ar[d] \ar[rd] & \resource{object file} \\ & \resource{assembly\\listing} & \resource{debugging\\information}}
\seeassembly\seemibl\seeobject\seecode\seedebugging
}

\providecommand{\cdmipsa}{
\toolsection{cdmips32} is a compiler for intermediate code targeting the MIPS32 hardware architecture.
It generates machine code for MIPS32 processors from programs written in intermediate code and stores it in corresponding object files.
It also creates a debugging information file as well as an assembly file containing a listing of the generated machine code.
\debuggingtool
\flowgraph{\resource{intermediate\\code} \ar[r] & \toolbox{cdmips32} \ar[r] \ar[d] \ar[rd] & \resource{object file} \\ & \resource{assembly\\listing} & \resource{debugging\\information}}
\seeassembly\seemips\seeobject\seecode\seedebugging
}

\providecommand{\cdmipsb}{
\toolsection{cdmips64} is a compiler for intermediate code targeting the MIPS64 hardware architecture.
It generates machine code for MIPS64 processors from programs written in intermediate code and stores it in corresponding object files.
It also creates a debugging information file as well as an assembly file containing a listing of the generated machine code.
\debuggingtool
\flowgraph{\resource{intermediate\\code} \ar[r] & \toolbox{cdmips64} \ar[r] \ar[d] \ar[rd] & \resource{object file} \\ & \resource{assembly\\listing} & \resource{debugging\\information}}
\seeassembly\seemips\seeobject\seecode\seedebugging
}

\providecommand{\cdmmix}{
\toolsection{cdmmix} is a compiler for intermediate code targeting the MMIX hardware architecture.
It generates machine code for MMIX processors from programs written in intermediate code and stores it in corresponding object files.
It also creates a debugging information file as well as an assembly file containing a listing of the generated machine code.
\debuggingtool
\flowgraph{\resource{intermediate\\code} \ar[r] & \toolbox{cdmmix} \ar[r] \ar[d] \ar[rd] & \resource{object file} \\ & \resource{assembly\\listing} & \resource{debugging\\information}}
\seeassembly\seemmix\seeobject\seecode\seedebugging
}

\providecommand{\cdorok}{
\toolsection{cdor1k} is a compiler for intermediate code targeting the OpenRISC 1000 hardware architecture.
It generates machine code for OpenRISC 1000 processors from programs written in intermediate code and stores it in corresponding object files.
It also creates a debugging information file as well as an assembly file containing a listing of the generated machine code.
\debuggingtool
\flowgraph{\resource{intermediate\\code} \ar[r] & \toolbox{cdor1k} \ar[r] \ar[d] \ar[rd] & \resource{object file} \\ & \resource{assembly\\listing} & \resource{debugging\\information}}
\seeassembly\seeorok\seeobject\seecode\seedebugging
}

\providecommand{\cdppca}{
\toolsection{cdppc32} is a compiler for intermediate code targeting the PowerPC hardware architecture.
It generates machine code for PowerPC processors from programs written in intermediate code and stores it in corresponding object files.
The compiler generates machine code for the 32-bit operating mode defined by the PowerPC architecture.
It also creates a debugging information file as well as an assembly file containing a listing of the generated machine code.
\debuggingtool
\flowgraph{\resource{intermediate\\code} \ar[r] & \toolbox{cdppc32} \ar[r] \ar[d] \ar[rd] & \resource{object file} \\ & \resource{assembly\\listing} & \resource{debugging\\information}}
\seeassembly\seeppc\seeobject\seecode\seedebugging
}

\providecommand{\cdppcb}{
\toolsection{cdppc64} is a compiler for intermediate code targeting the PowerPC hardware architecture.
It generates machine code for PowerPC processors from programs written in intermediate code and stores it in corresponding object files.
The compiler generates machine code for the 64-bit operating mode defined by the PowerPC architecture.
It also creates a debugging information file as well as an assembly file containing a listing of the generated machine code.
\debuggingtool
\flowgraph{\resource{intermediate\\code} \ar[r] & \toolbox{cdppc64} \ar[r] \ar[d] \ar[rd] & \resource{object file} \\ & \resource{assembly\\listing} & \resource{debugging\\information}}
\seeassembly\seeppc\seeobject\seecode\seedebugging
}

\providecommand{\cdrisc}{
\toolsection{cdrisc} is a compiler for intermediate code targeting the RISC hardware architecture.
It generates machine code for RISC processors from programs written in intermediate code and stores it in corresponding object files.
It also creates a debugging information file as well as an assembly file containing a listing of the generated machine code.
\debuggingtool
\flowgraph{\resource{intermediate\\code} \ar[r] & \toolbox{cdrisc} \ar[r] \ar[d] \ar[rd] & \resource{object file} \\ & \resource{assembly\\listing} & \resource{debugging\\information}}
\seeassembly\seerisc\seeobject\seecode\seedebugging
}

\providecommand{\cdwasm}{
\toolsection{cdwasm} is a compiler for intermediate code targeting the WebAssembly architecture.
It generates machine code for WebAssembly targets from programs written in intermediate code and stores it in corresponding object files.
It also creates a debugging information file as well as an assembly file containing a listing of the generated machine code.
\debuggingtool
\flowgraph{\resource{intermediate\\code} \ar[r] & \toolbox{cdwasm} \ar[r] \ar[d] \ar[rd] & \resource{object file} \\ & \resource{assembly\\listing} & \resource{debugging\\information}}
\seeassembly\seewasm\seeobject\seecode\seedebugging
}

% C++ tools

\providecommand{\cppprep}{
\toolsection{cppprep} is a preprocessor for the \cpp{} programming language.
It preprocesses source code according to the rules of \cpp{} and writes it to the standard output stream.
Only the macro names \texttt{\_\_DATE\_\_}, \texttt{\_\_FILE\_\_}, \texttt{\_\_LINE\_\_}, and \texttt{\_\_TIME\_\_} are predefined.
\flowgraph{\resource{\cpp{} or other\\source code} \ar[r] & \toolbox{cppprep} \ar[r] & \resource{preprocessed\\source code} \\ & \variable{ECSINCLUDE} \ar[u]}
\seecpp
}

\providecommand{\cppprint}{
\toolsection{cppprint} is a pretty printer for the \cpp{} programming language.
It reformats the source code of \cpp{} programs and writes it to the standard output stream.
\flowgraph{\resource{\cpp{}\\source code} \ar[r] & \toolbox{cppprint} \ar[r] & \resource{reformatted\\source code} \\ & \variable{ECSINCLUDE} \ar[u]}
\seecpp
}

\providecommand{\cppcheck}{
\toolsection{cppcheck} is a syntactic and semantic checker for the \cpp{} programming language.
It just performs syntactic and semantic checks on \cpp{} programs and writes its diagnostic messages to the standard error stream.
\flowgraph{\resource{\cpp{}\\source code} \ar[r] & \toolbox{cppcheck} \ar[r] & \resource{diagnostic\\messages} \\ & \variable{ECSINCLUDE} \ar[u]}
\seecpp
}

\providecommand{\cppdump}{
\toolsection{cppdump} is a serializer for the \cpp{} programming language.
It dumps the complete internal representation of programs written in \cpp{} into an XML document.
\debuggingtool
\flowgraph{\resource{\cpp{}\\source code} \ar[r] & \toolbox{cppdump} \ar[r] & \resource{internal\\representation} \\ & \variable{ECSINCLUDE} \ar[u]}
\seecpp
}

\providecommand{\cpprun}{
\toolsection{cpprun} is an interpreter for the \cpp{} programming language.
It processes and executes programs written in \cpp{}.
The macro \texttt{\_\_run\_\_} is predefined in order to enable programmers to identify this tool while interpreting.
\flowgraph{\resource{\cpp{}\\source code} \ar[r] & \toolbox{cpprun} \ar@/u/[r] & \resource{input/\\output} \ar@/d/[l] \\ & \variable{ECSINCLUDE} \ar[u]}
\seecpp
}

\providecommand{\cppdoc}{
\toolsection{cppdoc} is a generic documentation generator for the \cpp{} programming language.
It processes several \cpp{} source files and assembles all information therein into a generic documentation.
\debuggingtool
\flowgraph{\resource{\cpp{}\\source code} \ar[r] & \toolbox{cppdoc} \ar[r] & \resource{generic\\documentation} \\ & \variable{ECSINCLUDE} \ar[u]}
\seecpp\seedocumentation
}

\providecommand{\cpphtml}{
\toolsection{cpphtml} is an HTML documentation generator for the \cpp{} programming language.
It processes several \cpp{} source files and assembles all information therein into an HTML document.
\flowgraph{\resource{\cpp{}\\source code} \ar[r] & \toolbox{cpphtml} \ar[r] & \resource{HTML\\document} \\ & \variable{ECSINCLUDE} \ar[u]}
\seecpp\seedocumentation
}

\providecommand{\cpplatex}{
\toolsection{cpplatex} is a Latex documentation generator for the \cpp{} programming language.
It processes several \cpp{} source files and assembles all information therein into a Latex document.
\flowgraph{\resource{\cpp{}\\source code} \ar[r] & \toolbox{cpplatex} \ar[r] & \resource{Latex\\document} \\ & \variable{ECSINCLUDE} \ar[u]}
\seecpp\seedocumentation
}

\providecommand{\cppcode}{
\toolsection{cppcode} is an intermediate code generator for the \cpp{} programming language.
It generates intermediate code from programs written in \cpp{} and stores it in corresponding assembly files.
The macro \texttt{\_\_code\_\_} is predefined in order to enable programmers to identify this tool while generating intermediate code.
Programs generated with this tool require additional runtime support that is stored in the \file{cpp\-code\-run} library file.
\debuggingtool
\flowgraph{\resource{\cpp{}\\source code} \ar[r] & \toolbox{cppcode} \ar[r] & \resource{intermediate\\code} \\ & \variable{ECSINCLUDE} \ar[u]}
\seecpp\seeassembly\seecode
}

\providecommand{\cppamda}{
\toolsection{cppamd16} is a compiler for the \cpp{} programming language targeting the AMD64 hardware architecture.
It generates machine code for AMD64 processors from programs written in \cpp{} and stores it in corresponding object files.
The compiler generates machine code for the 16-bit operating mode defined by the AMD64 architecture.
For debugging purposes, it also creates a debugging information file as well as an assembly file containing a listing of the generated machine code.
The macro \texttt{\_\_amd16\_\_} is predefined in order to enable programmers to identify this tool and its target architecture while compiling.
Programs generated with this compiler require additional runtime support that is stored in the \file{cpp\-amd16\-run} library file.
\flowgraph{\resource{\cpp{}\\source code} \ar[r] & \toolbox{cppamd16} \ar[r] \ar[d] \ar[rd] & \resource{object file} \\ \variable{ECSINCLUDE} \ar[ru] & \resource{debugging\\information} & \resource{assembly\\listing}}
\seecpp\seeassembly\seeamd\seeobject\seedebugging
}

\providecommand{\cppamdb}{
\toolsection{cppamd32} is a compiler for the \cpp{} programming language targeting the AMD64 hardware architecture.
It generates machine code for AMD64 processors from programs written in \cpp{} and stores it in corresponding object files.
The compiler generates machine code for the 32-bit operating mode defined by the AMD64 architecture.
For debugging purposes, it also creates a debugging information file as well as an assembly file containing a listing of the generated machine code.
The macro \texttt{\_\_amd32\_\_} is predefined in order to enable programmers to identify this tool and its target architecture while compiling.
Programs generated with this compiler require additional runtime support that is stored in the \file{cpp\-amd32\-run} library file.
\flowgraph{\resource{\cpp{}\\source code} \ar[r] & \toolbox{cppamd32} \ar[r] \ar[d] \ar[rd] & \resource{object file} \\ \variable{ECSINCLUDE} \ar[ru] & \resource{debugging\\information} & \resource{assembly\\listing}}
\seecpp\seeassembly\seeamd\seeobject\seedebugging
}

\providecommand{\cppamdc}{
\toolsection{cppamd64} is a compiler for the \cpp{} programming language targeting the AMD64 hardware architecture.
It generates machine code for AMD64 processors from programs written in \cpp{} and stores it in corresponding object files.
The compiler generates machine code for the 64-bit operating mode defined by the AMD64 architecture.
For debugging purposes, it also creates a debugging information file as well as an assembly file containing a listing of the generated machine code.
The macro \texttt{\_\_amd64\_\_} is predefined in order to enable programmers to identify this tool and its target architecture while compiling.
Programs generated with this compiler require additional runtime support that is stored in the \file{cpp\-amd64\-run} library file.
\flowgraph{\resource{\cpp{}\\source code} \ar[r] & \toolbox{cppamd64} \ar[r] \ar[d] \ar[rd] & \resource{object file} \\ \variable{ECSINCLUDE} \ar[ru] & \resource{debugging\\information} & \resource{assembly\\listing}}
\seecpp\seeassembly\seeamd\seeobject\seedebugging
}

\providecommand{\cpparma}{
\toolsection{cpparma32} is a compiler for the \cpp{} programming language targeting the ARM hardware architecture.
It generates machine code for ARM processors executing A32 instructions from programs written in \cpp{} and stores it in corresponding object files.
For debugging purposes, it also creates a debugging information file as well as an assembly file containing a listing of the generated machine code.
The macro \texttt{\_\_arma32\_\_} is predefined in order to enable programmers to identify this tool and its target architecture while compiling.
Programs generated with this compiler require additional runtime support that is stored in the \file{cpp\-arma32\-run} library file.
\flowgraph{\resource{\cpp{}\\source code} \ar[r] & \toolbox{cpparma32} \ar[r] \ar[d] \ar[rd] & \resource{object file} \\ \variable{ECSINCLUDE} \ar[ru] & \resource{debugging\\information} & \resource{assembly\\listing}}
\seecpp\seeassembly\seearm\seeobject\seedebugging
}

\providecommand{\cpparmb}{
\toolsection{cpparma64} is a compiler for the \cpp{} programming language targeting the ARM hardware architecture.
It generates machine code for ARM processors executing A64 instructions from programs written in \cpp{} and stores it in corresponding object files.
For debugging purposes, it also creates a debugging information file as well as an assembly file containing a listing of the generated machine code.
The macro \texttt{\_\_arma64\_\_} is predefined in order to enable programmers to identify this tool and its target architecture while compiling.
Programs generated with this compiler require additional runtime support that is stored in the \file{cpp\-arma64\-run} library file.
\flowgraph{\resource{\cpp{}\\source code} \ar[r] & \toolbox{cpparma64} \ar[r] \ar[d] \ar[rd] & \resource{object file} \\ \variable{ECSINCLUDE} \ar[ru] & \resource{debugging\\information} & \resource{assembly\\listing}}
\seecpp\seeassembly\seearm\seeobject\seedebugging
}

\providecommand{\cpparmc}{
\toolsection{cpparmt32} is a compiler for the \cpp{} programming language targeting the ARM hardware architecture.
It generates machine code for ARM processors without floating-point extension executing T32 instructions from programs written in \cpp{} and stores it in corresponding object files.
For debugging purposes, it also creates a debugging information file as well as an assembly file containing a listing of the generated machine code.
The macro \texttt{\_\_armt32\_\_} is predefined in order to enable programmers to identify this tool and its target architecture while compiling.
Programs generated with this compiler require additional runtime support that is stored in the \file{cpp\-armt32\-run} library file.
\flowgraph{\resource{\cpp{}\\source code} \ar[r] & \toolbox{cpparmt32} \ar[r] \ar[d] \ar[rd] & \resource{object file} \\ \variable{ECSINCLUDE} \ar[ru] & \resource{debugging\\information} & \resource{assembly\\listing}}
\seecpp\seeassembly\seearm\seeobject\seedebugging
}

\providecommand{\cpparmcfpe}{
\toolsection{cpparmt32fpe} is a compiler for the \cpp{} programming language targeting the ARM hardware architecture.
It generates machine code for ARM processors with floating-point extension executing T32 instructions from programs written in \cpp{} and stores it in corresponding object files.
For debugging purposes, it also creates a debugging information file as well as an assembly file containing a listing of the generated machine code.
The macro \texttt{\_\_armt32fpe\_\_} is predefined in order to enable programmers to identify this tool and its target architecture while compiling.
Programs generated with this compiler require additional runtime support that is stored in the \file{cpp\-armt32\-fpe\-run} library file.
\flowgraph{\resource{\cpp{}\\source code} \ar[r] & \toolbox{cpparmt32fpe} \ar[r] \ar[d] \ar[rd] & \resource{object file} \\ \variable{ECSINCLUDE} \ar[ru] & \resource{debugging\\information} & \resource{assembly\\listing}}
\seecpp\seeassembly\seearm\seeobject\seedebugging
}

\providecommand{\cppavr}{
\toolsection{cppavr} is a compiler for the \cpp{} programming language targeting the AVR hardware architecture.
It generates machine code for AVR processors from programs written in \cpp{} and stores it in corresponding object files.
For debugging purposes, it also creates a debugging information file as well as an assembly file containing a listing of the generated machine code.
The macro \texttt{\_\_avr\_\_} is predefined in order to enable programmers to identify this tool and its target architecture while compiling.
Programs generated with this compiler require additional runtime support that is stored in the \file{cpp\-avr\-run} library file.
\flowgraph{\resource{\cpp{}\\source code} \ar[r] & \toolbox{cppavr} \ar[r] \ar[d] \ar[rd] & \resource{object file} \\ \variable{ECSINCLUDE} \ar[ru] & \resource{debugging\\information} & \resource{assembly\\listing}}
\seecpp\seeassembly\seeavr\seeobject\seedebugging
}

\providecommand{\cppavrtt}{
\toolsection{cppavr32} is a compiler for the \cpp{} programming language targeting the AVR32 hardware architecture.
It generates machine code for AVR32 processors from programs written in \cpp{} and stores it in corresponding object files.
For debugging purposes, it also creates a debugging information file as well as an assembly file containing a listing of the generated machine code.
The macro \texttt{\_\_avr32\_\_} is predefined in order to enable programmers to identify this tool and its target architecture while compiling.
Programs generated with this compiler require additional runtime support that is stored in the \file{cpp\-avr32\-run} library file.
\flowgraph{\resource{\cpp{}\\source code} \ar[r] & \toolbox{cppavr32} \ar[r] \ar[d] \ar[rd] & \resource{object file} \\ \variable{ECSINCLUDE} \ar[ru] & \resource{debugging\\information} & \resource{assembly\\listing}}
\seecpp\seeassembly\seeavrtt\seeobject\seedebugging
}

\providecommand{\cppmabk}{
\toolsection{cppm68k} is a compiler for the \cpp{} programming language targeting the M68000 hardware architecture.
It generates machine code for M68000 processors from programs written in \cpp{} and stores it in corresponding object files.
For debugging purposes, it also creates a debugging information file as well as an assembly file containing a listing of the generated machine code.
The macro \texttt{\_\_m68k\_\_} is predefined in order to enable programmers to identify this tool and its target architecture while compiling.
Programs generated with this compiler require additional runtime support that is stored in the \file{cpp\-m68k\-run} library file.
\flowgraph{\resource{\cpp{}\\source code} \ar[r] & \toolbox{cppm68k} \ar[r] \ar[d] \ar[rd] & \resource{object file} \\ \variable{ECSINCLUDE} \ar[ru] & \resource{debugging\\information} & \resource{assembly\\listing}}
\seecpp\seeassembly\seemabk\seeobject\seedebugging
}

\providecommand{\cppmibl}{
\toolsection{cppmibl} is a compiler for the \cpp{} programming language targeting the MicroBlaze hardware architecture.
It generates machine code for MicroBlaze processors from programs written in \cpp{} and stores it in corresponding object files.
For debugging purposes, it also creates a debugging information file as well as an assembly file containing a listing of the generated machine code.
The macro \texttt{\_\_mibl\_\_} is predefined in order to enable programmers to identify this tool and its target architecture while compiling.
Programs generated with this compiler require additional runtime support that is stored in the \file{cpp\-mibl\-run} library file.
\flowgraph{\resource{\cpp{}\\source code} \ar[r] & \toolbox{cppmibl} \ar[r] \ar[d] \ar[rd] & \resource{object file} \\ \variable{ECSINCLUDE} \ar[ru] & \resource{debugging\\information} & \resource{assembly\\listing}}
\seecpp\seeassembly\seemibl\seeobject\seedebugging
}

\providecommand{\cppmipsa}{
\toolsection{cppmips32} is a compiler for the \cpp{} programming language targeting the MIPS32 hardware architecture.
It generates machine code for MIPS32 processors from programs written in \cpp{} and stores it in corresponding object files.
For debugging purposes, it also creates a debugging information file as well as an assembly file containing a listing of the generated machine code.
The macro \texttt{\_\_mips32\_\_} is predefined in order to enable programmers to identify this tool and its target architecture while compiling.
Programs generated with this compiler require additional runtime support that is stored in the \file{cpp\-mips32\-run} library file.
\flowgraph{\resource{\cpp{}\\source code} \ar[r] & \toolbox{cppmips32} \ar[r] \ar[d] \ar[rd] & \resource{object file} \\ \variable{ECSINCLUDE} \ar[ru] & \resource{debugging\\information} & \resource{assembly\\listing}}
\seecpp\seeassembly\seemips\seeobject\seedebugging
}

\providecommand{\cppmipsb}{
\toolsection{cppmips64} is a compiler for the \cpp{} programming language targeting the MIPS64 hardware architecture.
It generates machine code for MIPS64 processors from programs written in \cpp{} and stores it in corresponding object files.
For debugging purposes, it also creates a debugging information file as well as an assembly file containing a listing of the generated machine code.
The macro \texttt{\_\_mips64\_\_} is predefined in order to enable programmers to identify this tool and its target architecture while compiling.
Programs generated with this compiler require additional runtime support that is stored in the \file{cpp\-mips64\-run} library file.
\flowgraph{\resource{\cpp{}\\source code} \ar[r] & \toolbox{cppmips64} \ar[r] \ar[d] \ar[rd] & \resource{object file} \\ \variable{ECSINCLUDE} \ar[ru] & \resource{debugging\\information} & \resource{assembly\\listing}}
\seecpp\seeassembly\seemips\seeobject\seedebugging
}

\providecommand{\cppmmix}{
\toolsection{cppmmix} is a compiler for the \cpp{} programming language targeting the MMIX hardware architecture.
It generates machine code for MMIX processors from programs written in \cpp{} and stores it in corresponding object files.
For debugging purposes, it also creates a debugging information file as well as an assembly file containing a listing of the generated machine code.
The macro \texttt{\_\_mmix\_\_} is predefined in order to enable programmers to identify this tool and its target architecture while compiling.
Programs generated with this compiler require additional runtime support that is stored in the \file{cpp\-mmix\-run} library file.
\flowgraph{\resource{\cpp{}\\source code} \ar[r] & \toolbox{cppmmix} \ar[r] \ar[d] \ar[rd] & \resource{object file} \\ \variable{ECSINCLUDE} \ar[ru] & \resource{debugging\\information} & \resource{assembly\\listing}}
\seecpp\seeassembly\seemmix\seeobject\seedebugging
}

\providecommand{\cpporok}{
\toolsection{cppor1k} is a compiler for the \cpp{} programming language targeting the OpenRISC 1000 hardware architecture.
It generates machine code for OpenRISC 1000 processors from programs written in \cpp{} and stores it in corresponding object files.
For debugging purposes, it also creates a debugging information file as well as an assembly file containing a listing of the generated machine code.
The macro \texttt{\_\_or1k\_\_} is predefined in order to enable programmers to identify this tool and its target architecture while compiling.
Programs generated with this compiler require additional runtime support that is stored in the \file{cpp\-or1k\-run} library file.
\flowgraph{\resource{\cpp{}\\source code} \ar[r] & \toolbox{cppor1k} \ar[r] \ar[d] \ar[rd] & \resource{object file} \\ \variable{ECSINCLUDE} \ar[ru] & \resource{debugging\\information} & \resource{assembly\\listing}}
\seecpp\seeassembly\seeorok\seeobject\seedebugging
}

\providecommand{\cppppca}{
\toolsection{cppppc32} is a compiler for the \cpp{} programming language targeting the PowerPC hardware architecture.
It generates machine code for PowerPC processors from programs written in \cpp{} and stores it in corresponding object files.
The compiler generates machine code for the 32-bit operating mode defined by the PowerPC architecture.
For debugging purposes, it also creates a debugging information file as well as an assembly file containing a listing of the generated machine code.
The macro \texttt{\_\_ppc32\_\_} is predefined in order to enable programmers to identify this tool and its target architecture while compiling.
Programs generated with this compiler require additional runtime support that is stored in the \file{cpp\-ppc32\-run} library file.
\flowgraph{\resource{\cpp{}\\source code} \ar[r] & \toolbox{cppppc32} \ar[r] \ar[d] \ar[rd] & \resource{object file} \\ \variable{ECSINCLUDE} \ar[ru] & \resource{debugging\\information} & \resource{assembly\\listing}}
\seecpp\seeassembly\seeppc\seeobject\seedebugging
}

\providecommand{\cppppcb}{
\toolsection{cppppc64} is a compiler for the \cpp{} programming language targeting the PowerPC hardware architecture.
It generates machine code for PowerPC processors from programs written in \cpp{} and stores it in corresponding object files.
The compiler generates machine code for the 64-bit operating mode defined by the PowerPC architecture.
For debugging purposes, it also creates a debugging information file as well as an assembly file containing a listing of the generated machine code.
The macro \texttt{\_\_ppc64\_\_} is predefined in order to enable programmers to identify this tool and its target architecture while compiling.
Programs generated with this compiler require additional runtime support that is stored in the \file{cpp\-ppc64\-run} library file.
\flowgraph{\resource{\cpp{}\\source code} \ar[r] & \toolbox{cppppc64} \ar[r] \ar[d] \ar[rd] & \resource{object file} \\ \variable{ECSINCLUDE} \ar[ru] & \resource{debugging\\information} & \resource{assembly\\listing}}
\seecpp\seeassembly\seeppc\seeobject\seedebugging
}

\providecommand{\cpprisc}{
\toolsection{cpprisc} is a compiler for the \cpp{} programming language targeting the RISC hardware architecture.
It generates machine code for RISC processors from programs written in \cpp{} and stores it in corresponding object files.
For debugging purposes, it also creates a debugging information file as well as an assembly file containing a listing of the generated machine code.
The macro \texttt{\_\_risc\_\_} is predefined in order to enable programmers to identify this tool and its target architecture while compiling.
Programs generated with this compiler require additional runtime support that is stored in the \file{cpp\-risc\-run} library file.
\flowgraph{\resource{\cpp{}\\source code} \ar[r] & \toolbox{cpprisc} \ar[r] \ar[d] \ar[rd] & \resource{object file} \\ \variable{ECSINCLUDE} \ar[ru] & \resource{debugging\\information} & \resource{assembly\\listing}}
\seecpp\seeassembly\seerisc\seeobject\seedebugging
}

\providecommand{\cppwasm}{
\toolsection{cppwasm} is a compiler for the \cpp{} programming language targeting the WebAssembly architecture.
It generates machine code for WebAssembly targets from programs written in \cpp{} and stores it in corresponding object files.
For debugging purposes, it also creates a debugging information file as well as an assembly file containing a listing of the generated machine code.
The macro \texttt{\_\_wasm\_\_} is predefined in order to enable programmers to identify this tool and its target architecture while compiling.
Programs generated with this compiler require additional runtime support that is stored in the \file{cpp\-wasm\-run} library file.
\flowgraph{\resource{\cpp{}\\source code} \ar[r] & \toolbox{cppwasm} \ar[r] \ar[d] \ar[rd] & \resource{object file} \\ \variable{ECSINCLUDE} \ar[ru] & \resource{debugging\\information} & \resource{assembly\\listing}}
\seecpp\seeassembly\seewasm\seeobject\seedebugging
}

% FALSE tools

\providecommand{\falprint}{
\toolsection{falprint} is a pretty printer for the FALSE programming language.
It reformats the source code of FALSE programs and writes it to the standard output stream.
\flowgraph{\resource{FALSE\\source code} \ar[r] & \toolbox{falprint} \ar[r] & \resource{reformatted\\source code}}
\seefalse
}

\providecommand{\falcheck}{
\toolsection{falcheck} is a syntactic and semantic checker for the FALSE programming language.
It just performs syntactic and semantic checks on FALSE programs and writes its diagnostic messages to the standard error stream.
\flowgraph{\resource{FALSE\\source code} \ar[r] & \toolbox{falcheck} \ar[r] & \resource{diagnostic\\messages}}
\seefalse
}

\providecommand{\faldump}{
\toolsection{faldump} is a serializer for the FALSE programming language.
It dumps the complete internal representation of programs written in FALSE into an XML document.
\debuggingtool
\flowgraph{\resource{FALSE\\source code} \ar[r] & \toolbox{faldump} \ar[r] & \resource{internal\\representation}}
\seefalse
}

\providecommand{\falrun}{
\toolsection{falrun} is an interpreter for the FALSE programming language.
It processes and executes programs written in FALSE\@.
\flowgraph{\resource{FALSE\\source code} \ar[r] & \toolbox{falrun} \ar@/u/[r] & \resource{input/\\output} \ar@/d/[l]}
\seefalse
}

\providecommand{\falcpp}{
\toolsection{falcpp} is a transpiler for the FALSE programming language.
It translates programs written in FALSE into \cpp{} programs and stores them in corresponding source files.
\flowgraph{\resource{FALSE\\source code} \ar[r] & \toolbox{falcpp} \ar[r] & \resource{\cpp{}\\source file}}
\seefalse\seecpp
}

\providecommand{\falcode}{
\toolsection{falcode} is an intermediate code generator for the FALSE programming language.
It generates intermediate code from programs written in FALSE and stores it in corresponding assembly files.
\debuggingtool
\flowgraph{\resource{FALSE\\source code} \ar[r] & \toolbox{falcode} \ar[r] & \resource{intermediate\\code}}
\seefalse\seeassembly\seecode
}

\providecommand{\falamda}{
\toolsection{falamd16} is a compiler for the FALSE programming language targeting the AMD64 hardware architecture.
It generates machine code for AMD64 processors from programs written in FALSE and stores it in corresponding object files.
The compiler generates machine code for the 16-bit operating mode defined by the AMD64 architecture.
\flowgraph{\resource{FALSE\\source code} \ar[r] & \toolbox{falamd16} \ar[r] & \resource{object file}}
\seefalse\seeamd\seeobject
}

\providecommand{\falamdb}{
\toolsection{falamd32} is a compiler for the FALSE programming language targeting the AMD64 hardware architecture.
It generates machine code for AMD64 processors from programs written in FALSE and stores it in corresponding object files.
The compiler generates machine code for the 32-bit operating mode defined by the AMD64 architecture.
\flowgraph{\resource{FALSE\\source code} \ar[r] & \toolbox{falamd32} \ar[r] & \resource{object file}}
\seefalse\seeamd\seeobject
}

\providecommand{\falamdc}{
\toolsection{falamd64} is a compiler for the FALSE programming language targeting the AMD64 hardware architecture.
It generates machine code for AMD64 processors from programs written in FALSE and stores it in corresponding object files.
The compiler generates machine code for the 64-bit operating mode defined by the AMD64 architecture.
\flowgraph{\resource{FALSE\\source code} \ar[r] & \toolbox{falamd64} \ar[r] & \resource{object file}}
\seefalse\seeamd\seeobject
}

\providecommand{\falarma}{
\toolsection{falarma32} is a compiler for the FALSE programming language targeting the ARM hardware architecture.
It generates machine code for ARM processors executing A32 instructions from programs written in FALSE and stores it in corresponding object files.
\flowgraph{\resource{FALSE\\source code} \ar[r] & \toolbox{falarma32} \ar[r] & \resource{object file}}
\seefalse\seearm\seeobject
}

\providecommand{\falarmb}{
\toolsection{falarma64} is a compiler for the FALSE programming language targeting the ARM hardware architecture.
It generates machine code for ARM processors executing A64 instructions from programs written in FALSE and stores it in corresponding object files.
\flowgraph{\resource{FALSE\\source code} \ar[r] & \toolbox{falarma64} \ar[r] & \resource{object file}}
\seefalse\seearm\seeobject
}

\providecommand{\falarmc}{
\toolsection{falarmt32} is a compiler for the FALSE programming language targeting the ARM hardware architecture.
It generates machine code for ARM processors without floating-point extension executing T32 instructions from programs written in FALSE and stores it in corresponding object files.
\flowgraph{\resource{FALSE\\source code} \ar[r] & \toolbox{falarmt32} \ar[r] & \resource{object file}}
\seefalse\seearm\seeobject
}

\providecommand{\falarmcfpe}{
\toolsection{falarmt32fpe} is a compiler for the FALSE programming language targeting the ARM hardware architecture.
It generates machine code for ARM processors with floating-point extension executing T32 instructions from programs written in FALSE and stores it in corresponding object files.
\flowgraph{\resource{FALSE\\source code} \ar[r] & \toolbox{falarmt32fpe} \ar[r] & \resource{object file}}
\seefalse\seearm\seeobject
}

\providecommand{\falavr}{
\toolsection{falavr} is a compiler for the FALSE programming language targeting the AVR hardware architecture.
It generates machine code for AVR processors from programs written in FALSE and stores it in corresponding object files.
\flowgraph{\resource{FALSE\\source code} \ar[r] & \toolbox{falavr} \ar[r] & \resource{object file}}
\seefalse\seeavr\seeobject
}

\providecommand{\falavrtt}{
\toolsection{falavr32} is a compiler for the FALSE programming language targeting the AVR32 hardware architecture.
It generates machine code for AVR32 processors from programs written in FALSE and stores it in corresponding object files.
\flowgraph{\resource{FALSE\\source code} \ar[r] & \toolbox{falavr32} \ar[r] & \resource{object file}}
\seefalse\seeavrtt\seeobject
}

\providecommand{\falmabk}{
\toolsection{falm68k} is a compiler for the FALSE programming language targeting the M68000 hardware architecture.
It generates machine code for M68000 processors from programs written in FALSE and stores it in corresponding object files.
\flowgraph{\resource{FALSE\\source code} \ar[r] & \toolbox{falm68k} \ar[r] & \resource{object file}}
\seefalse\seemabk\seeobject
}

\providecommand{\falmibl}{
\toolsection{falmibl} is a compiler for the FALSE programming language targeting the MicroBlaze hardware architecture.
It generates machine code for MicroBlaze processors from programs written in FALSE and stores it in corresponding object files.
\flowgraph{\resource{FALSE\\source code} \ar[r] & \toolbox{falmibl} \ar[r] & \resource{object file}}
\seefalse\seemibl\seeobject
}

\providecommand{\falmipsa}{
\toolsection{falmips32} is a compiler for the FALSE programming language targeting the MIPS32 hardware architecture.
It generates machine code for MIPS32 processors from programs written in FALSE and stores it in corresponding object files.
\flowgraph{\resource{FALSE\\source code} \ar[r] & \toolbox{falmips32} \ar[r] & \resource{object file}}
\seefalse\seemips\seeobject
}

\providecommand{\falmipsb}{
\toolsection{falmips64} is a compiler for the FALSE programming language targeting the MIPS64 hardware architecture.
It generates machine code for MIPS64 processors from programs written in FALSE and stores it in corresponding object files.
\flowgraph{\resource{FALSE\\source code} \ar[r] & \toolbox{falmips64} \ar[r] & \resource{object file}}
\seefalse\seemips\seeobject
}

\providecommand{\falmmix}{
\toolsection{falmmix} is a compiler for the FALSE programming language targeting the MMIX hardware architecture.
It generates machine code for MMIX processors from programs written in FALSE and stores it in corresponding object files.
\flowgraph{\resource{FALSE\\source code} \ar[r] & \toolbox{falmmix} \ar[r] & \resource{object file}}
\seefalse\seemmix\seeobject
}

\providecommand{\falorok}{
\toolsection{falor1k} is a compiler for the FALSE programming language targeting the OpenRISC 1000 hardware architecture.
It generates machine code for OpenRISC 1000 processors from programs written in FALSE and stores it in corresponding object files.
\flowgraph{\resource{FALSE\\source code} \ar[r] & \toolbox{falor1k} \ar[r] & \resource{object file}}
\seefalse\seeorok\seeobject
}

\providecommand{\falppca}{
\toolsection{falppc32} is a compiler for the FALSE programming language targeting the PowerPC hardware architecture.
It generates machine code for PowerPC processors from programs written in FALSE and stores it in corresponding object files.
The compiler generates machine code for the 32-bit operating mode defined by the PowerPC architecture.
\flowgraph{\resource{FALSE\\source code} \ar[r] & \toolbox{falppc32} \ar[r] & \resource{object file}}
\seefalse\seeppc\seeobject
}

\providecommand{\falppcb}{
\toolsection{falppc64} is a compiler for the FALSE programming language targeting the PowerPC hardware architecture.
It generates machine code for PowerPC processors from programs written in FALSE and stores it in corresponding object files.
The compiler generates machine code for the 64-bit operating mode defined by the PowerPC architecture.
\flowgraph{\resource{FALSE\\source code} \ar[r] & \toolbox{falppc64} \ar[r] & \resource{object file}}
\seefalse\seeppc\seeobject
}

\providecommand{\falrisc}{
\toolsection{falrisc} is a compiler for the FALSE programming language targeting the RISC hardware architecture.
It generates machine code for RISC processors from programs written in FALSE and stores it in corresponding object files.
\flowgraph{\resource{FALSE\\source code} \ar[r] & \toolbox{falrisc} \ar[r] & \resource{object file}}
\seefalse\seerisc\seeobject
}

\providecommand{\falwasm}{
\toolsection{falwasm} is a compiler for the FALSE programming language targeting the WebAssembly architecture.
It generates machine code for WebAssembly targets from programs written in FALSE and stores it in corresponding object files.
\flowgraph{\resource{FALSE\\source code} \ar[r] & \toolbox{falwasm} \ar[r] & \resource{object file}}
\seefalse\seewasm\seeobject
}

% Oberon tools

\providecommand{\obprint}{
\toolsection{obprint} is a pretty printer for the Oberon programming language.
It reformats the source code of Oberon modules and writes it to the standard output stream.
\flowgraph{\resource{Oberon\\source code} \ar[r] & \toolbox{obprint} \ar[r] & \resource{reformatted\\source code}}
\seeoberon
}

\providecommand{\obcheck}{
\toolsection{obcheck} is a syntactic and semantic checker for the Oberon programming language.
It just performs syntactic and semantic checks on Oberon modules and writes its diagnostic messages to the standard error stream.
In addition, it stores the interface of each module in a symbol file which is required when other modules import the module.
\flowgraph{\resource{Oberon\\source code} \ar[r] & \toolbox{obcheck} \ar[r] \ar@/l/[d] & \resource{diagnostic\\messages} \\ \variable{ECSIMPORT} \ar[ru] & \resource{symbol\\files} \ar@/r/[u]}
\seeoberon
}

\providecommand{\obdump}{
\toolsection{obdump} is a serializer for the Oberon programming language.
It dumps the complete internal representation of modules written in Oberon into an XML document.
\debuggingtool
\flowgraph{\resource{Oberon\\source code} \ar[r] & \toolbox{obdump} \ar[r] \ar@/l/[d] & \resource{internal\\representation} \\ \variable{ECSIMPORT} \ar[ru] & \resource{symbol\\files} \ar@/r/[u]}
\seeoberon
}

\providecommand{\obrun}{
\toolsection{obrun} is an interpreter for the Oberon programming language.
It processes and executes modules written in Oberon.
This tool does neither generate nor process symbol files while interpreting modules.
If a module is imported by another one, its filename has to be named before the other one in the list of command-line arguments.
\flowgraph{\resource{Oberon\\source code} \ar[r] & \toolbox{obrun} \ar@/u/[r] & \resource{input/\\output} \ar@/d/[l]}
\seeoberon
}

\providecommand{\obcpp}{
\toolsection{obcpp} is a transpiler for the Oberon programming language.
It translates programs written in Oberon into \cpp{} programs and stores them in corresponding source and header files.
In addition, it stores the interface of each module in a symbol file which is required when other modules import the module.
The same interface is provided by the generated header file which can be used in other parts of the \cpp{} program.
\flowgraph{\resource{Oberon\\source code} \ar[r] & \toolbox{obcpp} \ar[r] \ar@/l/[d] \ar[rd] & \resource{\cpp{}\\source file} \\ \variable{ECSIMPORT} \ar[ru] & \resource{symbol\\files} \ar@/r/[u] & \resource{\cpp{}\\header file}}
\seeoberon\seecpp
}

\providecommand{\obdoc}{
\toolsection{obdoc} is a generic documentation generator for the Oberon programming language.
It processes several Oberon modules and assembles all information therein into a generic documentation.
In addition, it stores the interface of each module in a symbol file which is required when other modules import the module.
\debuggingtool
\flowgraph{\resource{Oberon\\source code} \ar[r] & \toolbox{obdoc} \ar[r] \ar@/l/[d] & \resource{generic\\documentation} \\ \variable{ECSIMPORT} \ar[ru] & \resource{symbol\\files} \ar@/r/[u]}
\seeoberon\seedocumentation
}

\providecommand{\obhtml}{
\toolsection{obhtml} is an HTML documentation generator for the Oberon programming language.
It processes several Oberon modules and assembles all information therein into an HTML document.
In addition, it stores the interface of each module in a symbol file which is required when other modules import the module.
\flowgraph{\resource{Oberon\\source code} \ar[r] & \toolbox{obhtml} \ar[r] \ar@/l/[d] & \resource{HTML\\document} \\ \variable{ECSIMPORT} \ar[ru] & \resource{symbol\\files} \ar@/r/[u]}
\seeoberon\seedocumentation
}

\providecommand{\oblatex}{
\toolsection{oblatex} is a Latex documentation generator for the Oberon programming language.
It processes several Oberon modules and assembles all information therein into a Latex document.
In addition, it stores the interface of each module in a symbol file which is required when other modules import the module.
\flowgraph{\resource{Oberon\\source code} \ar[r] & \toolbox{oblatex} \ar[r] \ar@/l/[d] & \resource{Latex\\document} \\ \variable{ECSIMPORT} \ar[ru] & \resource{symbol\\files} \ar@/r/[u]}
\seeoberon\seedocumentation
}

\providecommand{\obcode}{
\toolsection{obcode} is an intermediate code generator for the Oberon programming language.
It generates intermediate code from modules written in Oberon and stores it in corresponding assembly files.
In addition, it stores the interface of each module in a symbol file which is required when other modules import the module.
Programs generated with this tool require additional runtime support that is stored in the \file{ob\-code\-run} library file.
\debuggingtool
\flowgraph{\resource{Oberon\\source code} \ar[r] & \toolbox{obcode} \ar[r] \ar@/l/[d] & \resource{intermediate\\code} \\ \variable{ECSIMPORT} \ar[ru] & \resource{symbol\\files} \ar@/r/[u]}
\seeoberon\seeassembly\seecode
}

\providecommand{\obamda}{
\toolsection{obamd16} is a compiler for the Oberon programming language targeting the AMD64 hardware architecture.
It generates machine code for AMD64 processors from modules written in Oberon and stores it in corresponding object files.
The compiler generates machine code for the 16-bit operating mode defined by the AMD64 architecture.
For debugging purposes, it also creates a debugging information file as well as an assembly file containing a listing of the generated machine code.
In addition, it stores the interface of each module in a symbol file which is required when other modules import the module.
Programs generated with this compiler require additional runtime support that is stored in the \file{ob\-amd16\-run} library file.
\flowgraph{\resource{Oberon\\source code} \ar[r] & \toolbox{obamd16} \ar[r] \ar@/l/[d] \ar[rd] & \resource{object file} \\ \variable{ECSIMPORT} \ar[ru] & \resource{symbol\\files} \ar@/r/[u] & \resource{debugging\\information}}
\seeoberon\seeassembly\seeamd\seeobject\seedebugging
}

\providecommand{\obamdb}{
\toolsection{obamd32} is a compiler for the Oberon programming language targeting the AMD64 hardware architecture.
It generates machine code for AMD64 processors from modules written in Oberon and stores it in corresponding object files.
The compiler generates machine code for the 32-bit operating mode defined by the AMD64 architecture.
For debugging purposes, it also creates a debugging information file as well as an assembly file containing a listing of the generated machine code.
In addition, it stores the interface of each module in a symbol file which is required when other modules import the module.
Programs generated with this compiler require additional runtime support that is stored in the \file{ob\-amd32\-run} library file.
\flowgraph{\resource{Oberon\\source code} \ar[r] & \toolbox{obamd32} \ar[r] \ar@/l/[d] \ar[rd] & \resource{object file} \\ \variable{ECSIMPORT} \ar[ru] & \resource{symbol\\files} \ar@/r/[u] & \resource{debugging\\information}}
\seeoberon\seeassembly\seeamd\seeobject\seedebugging
}

\providecommand{\obamdc}{
\toolsection{obamd64} is a compiler for the Oberon programming language targeting the AMD64 hardware architecture.
It generates machine code for AMD64 processors from modules written in Oberon and stores it in corresponding object files.
The compiler generates machine code for the 64-bit operating mode defined by the AMD64 architecture.
For debugging purposes, it also creates a debugging information file as well as an assembly file containing a listing of the generated machine code.
In addition, it stores the interface of each module in a symbol file which is required when other modules import the module.
Programs generated with this compiler require additional runtime support that is stored in the \file{ob\-amd64\-run} library file.
\flowgraph{\resource{Oberon\\source code} \ar[r] & \toolbox{obamd64} \ar[r] \ar@/l/[d] \ar[rd] & \resource{object file} \\ \variable{ECSIMPORT} \ar[ru] & \resource{symbol\\files} \ar@/r/[u] & \resource{debugging\\information}}
\seeoberon\seeassembly\seeamd\seeobject\seedebugging
}

\providecommand{\obarma}{
\toolsection{obarma32} is a compiler for the Oberon programming language targeting the ARM hardware architecture.
It generates machine code for ARM processors executing A32 instructions from modules written in Oberon and stores it in corresponding object files.
For debugging purposes, it also creates a debugging information file as well as an assembly file containing a listing of the generated machine code.
In addition, it stores the interface of each module in a symbol file which is required when other modules import the module.
Programs generated with this compiler require additional runtime support that is stored in the \file{ob\-arma32\-run} library file.
\flowgraph{\resource{Oberon\\source code} \ar[r] & \toolbox{obarma32} \ar[r] \ar@/l/[d] \ar[rd] & \resource{object file} \\ \variable{ECSIMPORT} \ar[ru] & \resource{symbol\\files} \ar@/r/[u] & \resource{debugging\\information}}
\seeoberon\seeassembly\seearm\seeobject\seedebugging
}

\providecommand{\obarmb}{
\toolsection{obarma64} is a compiler for the Oberon programming language targeting the ARM hardware architecture.
It generates machine code for ARM processors executing A64 instructions from modules written in Oberon and stores it in corresponding object files.
For debugging purposes, it also creates a debugging information file as well as an assembly file containing a listing of the generated machine code.
In addition, it stores the interface of each module in a symbol file which is required when other modules import the module.
Programs generated with this compiler require additional runtime support that is stored in the \file{ob\-arma64\-run} library file.
\flowgraph{\resource{Oberon\\source code} \ar[r] & \toolbox{obarma64} \ar[r] \ar@/l/[d] \ar[rd] & \resource{object file} \\ \variable{ECSIMPORT} \ar[ru] & \resource{symbol\\files} \ar@/r/[u] & \resource{debugging\\information}}
\seeoberon\seeassembly\seearm\seeobject\seedebugging
}

\providecommand{\obarmc}{
\toolsection{obarmt32} is a compiler for the Oberon programming language targeting the ARM hardware architecture.
It generates machine code for ARM processors without floating-point extension executing T32 instructions from modules written in Oberon and stores it in corresponding object files.
For debugging purposes, it also creates a debugging information file as well as an assembly file containing a listing of the generated machine code.
In addition, it stores the interface of each module in a symbol file which is required when other modules import the module.
Programs generated with this compiler require additional runtime support that is stored in the \file{ob\-armt32\-run} library file.
\flowgraph{\resource{Oberon\\source code} \ar[r] & \toolbox{obarmt32} \ar[r] \ar@/l/[d] \ar[rd] & \resource{object file} \\ \variable{ECSIMPORT} \ar[ru] & \resource{symbol\\files} \ar@/r/[u] & \resource{debugging\\information}}
\seeoberon\seeassembly\seearm\seeobject\seedebugging
}

\providecommand{\obarmcfpe}{
\toolsection{obarmt32fpe} is a compiler for the Oberon programming language targeting the ARM hardware architecture.
It generates machine code for ARM processors with floating-point extension executing T32 instructions from modules written in Oberon and stores it in corresponding object files.
For debugging purposes, it also creates a debugging information file as well as an assembly file containing a listing of the generated machine code.
In addition, it stores the interface of each module in a symbol file which is required when other modules import the module.
Programs generated with this compiler require additional runtime support that is stored in the \file{ob\-armt32\-fpe\-run} library file.
\flowgraph{\resource{Oberon\\source code} \ar[r] & \toolbox{obarmt32fpe} \ar[r] \ar@/l/[d] \ar[rd] & \resource{object file} \\ \variable{ECSIMPORT} \ar[ru] & \resource{symbol\\files} \ar@/r/[u] & \resource{debugging\\information}}
\seeoberon\seeassembly\seearm\seeobject\seedebugging
}

\providecommand{\obavr}{
\toolsection{obavr} is a compiler for the Oberon programming language targeting the AVR hardware architecture.
It generates machine code for AVR processors from modules written in Oberon and stores it in corresponding object files.
For debugging purposes, it also creates a debugging information file as well as an assembly file containing a listing of the generated machine code.
In addition, it stores the interface of each module in a symbol file which is required when other modules import the module.
Programs generated with this compiler require additional runtime support that is stored in the \file{ob\-avr\-run} library file.
\flowgraph{\resource{Oberon\\source code} \ar[r] & \toolbox{obavr} \ar[r] \ar@/l/[d] \ar[rd] & \resource{object file} \\ \variable{ECSIMPORT} \ar[ru] & \resource{symbol\\files} \ar@/r/[u] & \resource{debugging\\information}}
\seeoberon\seeassembly\seeavr\seeobject\seedebugging
}

\providecommand{\obavrtt}{
\toolsection{obavr32} is a compiler for the Oberon programming language targeting the AVR32 hardware architecture.
It generates machine code for AVR32 processors from modules written in Oberon and stores it in corresponding object files.
For debugging purposes, it also creates a debugging information file as well as an assembly file containing a listing of the generated machine code.
In addition, it stores the interface of each module in a symbol file which is required when other modules import the module.
Programs generated with this compiler require additional runtime support that is stored in the \file{ob\-avr32\-run} library file.
\flowgraph{\resource{Oberon\\source code} \ar[r] & \toolbox{obavr32} \ar[r] \ar@/l/[d] \ar[rd] & \resource{object file} \\ \variable{ECSIMPORT} \ar[ru] & \resource{symbol\\files} \ar@/r/[u] & \resource{debugging\\information}}
\seeoberon\seeassembly\seeavrtt\seeobject\seedebugging
}

\providecommand{\obmabk}{
\toolsection{obm68k} is a compiler for the Oberon programming language targeting the M68000 hardware architecture.
It generates machine code for M68000 processors from modules written in Oberon and stores it in corresponding object files.
For debugging purposes, it also creates a debugging information file as well as an assembly file containing a listing of the generated machine code.
In addition, it stores the interface of each module in a symbol file which is required when other modules import the module.
Programs generated with this compiler require additional runtime support that is stored in the \file{ob\-m68k\-run} library file.
\flowgraph{\resource{Oberon\\source code} \ar[r] & \toolbox{obm68k} \ar[r] \ar@/l/[d] \ar[rd] & \resource{object file} \\ \variable{ECSIMPORT} \ar[ru] & \resource{symbol\\files} \ar@/r/[u] & \resource{debugging\\information}}
\seeoberon\seeassembly\seemabk\seeobject\seedebugging
}

\providecommand{\obmibl}{
\toolsection{obmibl} is a compiler for the Oberon programming language targeting the MicroBlaze hardware architecture.
It generates machine code for MicroBlaze processors from modules written in Oberon and stores it in corresponding object files.
For debugging purposes, it also creates a debugging information file as well as an assembly file containing a listing of the generated machine code.
In addition, it stores the interface of each module in a symbol file which is required when other modules import the module.
Programs generated with this compiler require additional runtime support that is stored in the \file{ob\-mibl\-run} library file.
\flowgraph{\resource{Oberon\\source code} \ar[r] & \toolbox{obmibl} \ar[r] \ar@/l/[d] \ar[rd] & \resource{object file} \\ \variable{ECSIMPORT} \ar[ru] & \resource{symbol\\files} \ar@/r/[u] & \resource{debugging\\information}}
\seeoberon\seeassembly\seemibl\seeobject\seedebugging
}

\providecommand{\obmipsa}{
\toolsection{obmips32} is a compiler for the Oberon programming language targeting the MIPS32 hardware architecture.
It generates machine code for MIPS32 processors from modules written in Oberon and stores it in corresponding object files.
For debugging purposes, it also creates a debugging information file as well as an assembly file containing a listing of the generated machine code.
In addition, it stores the interface of each module in a symbol file which is required when other modules import the module.
Programs generated with this compiler require additional runtime support that is stored in the \file{ob\-mips32\-run} library file.
\flowgraph{\resource{Oberon\\source code} \ar[r] & \toolbox{obmips32} \ar[r] \ar@/l/[d] \ar[rd] & \resource{object file} \\ \variable{ECSIMPORT} \ar[ru] & \resource{symbol\\files} \ar@/r/[u] & \resource{debugging\\information}}
\seeoberon\seeassembly\seemips\seeobject\seedebugging
}

\providecommand{\obmipsb}{
\toolsection{obmips64} is a compiler for the Oberon programming language targeting the MIPS64 hardware architecture.
It generates machine code for MIPS64 processors from modules written in Oberon and stores it in corresponding object files.
For debugging purposes, it also creates a debugging information file as well as an assembly file containing a listing of the generated machine code.
In addition, it stores the interface of each module in a symbol file which is required when other modules import the module.
Programs generated with this compiler require additional runtime support that is stored in the \file{ob\-mips64\-run} library file.
\flowgraph{\resource{Oberon\\source code} \ar[r] & \toolbox{obmips64} \ar[r] \ar@/l/[d] \ar[rd] & \resource{object file} \\ \variable{ECSIMPORT} \ar[ru] & \resource{symbol\\files} \ar@/r/[u] & \resource{debugging\\information}}
\seeoberon\seeassembly\seemips\seeobject\seedebugging
}

\providecommand{\obmmix}{
\toolsection{obmmix} is a compiler for the Oberon programming language targeting the MMIX hardware architecture.
It generates machine code for MMIX processors from modules written in Oberon and stores it in corresponding object files.
For debugging purposes, it also creates a debugging information file as well as an assembly file containing a listing of the generated machine code.
In addition, it stores the interface of each module in a symbol file which is required when other modules import the module.
Programs generated with this compiler require additional runtime support that is stored in the \file{ob\-mmix\-run} library file.
\flowgraph{\resource{Oberon\\source code} \ar[r] & \toolbox{obmmix} \ar[r] \ar@/l/[d] \ar[rd] & \resource{object file} \\ \variable{ECSIMPORT} \ar[ru] & \resource{symbol\\files} \ar@/r/[u] & \resource{debugging\\information}}
\seeoberon\seeassembly\seemmix\seeobject\seedebugging
}

\providecommand{\oborok}{
\toolsection{obor1k} is a compiler for the Oberon programming language targeting the OpenRISC 1000 hardware architecture.
It generates machine code for OpenRISC 1000 processors from modules written in Oberon and stores it in corresponding object files.
For debugging purposes, it also creates a debugging information file as well as an assembly file containing a listing of the generated machine code.
In addition, it stores the interface of each module in a symbol file which is required when other modules import the module.
Programs generated with this compiler require additional runtime support that is stored in the \file{ob\-or1k\-run} library file.
\flowgraph{\resource{Oberon\\source code} \ar[r] & \toolbox{obor1k} \ar[r] \ar@/l/[d] \ar[rd] & \resource{object file} \\ \variable{ECSIMPORT} \ar[ru] & \resource{symbol\\files} \ar@/r/[u] & \resource{debugging\\information}}
\seeoberon\seeassembly\seeorok\seeobject\seedebugging
}

\providecommand{\obppca}{
\toolsection{obppc32} is a compiler for the Oberon programming language targeting the PowerPC hardware architecture.
It generates machine code for PowerPC processors from modules written in Oberon and stores it in corresponding object files.
The compiler generates machine code for the 32-bit operating mode defined by the PowerPC architecture.
For debugging purposes, it also creates a debugging information file as well as an assembly file containing a listing of the generated machine code.
In addition, it stores the interface of each module in a symbol file which is required when other modules import the module.
Programs generated with this compiler require additional runtime support that is stored in the \file{ob\-ppc32\-run} library file.
\flowgraph{\resource{Oberon\\source code} \ar[r] & \toolbox{obppc32} \ar[r] \ar@/l/[d] \ar[rd] & \resource{object file} \\ \variable{ECSIMPORT} \ar[ru] & \resource{symbol\\files} \ar@/r/[u] & \resource{debugging\\information}}
\seeoberon\seeassembly\seeppc\seeobject\seedebugging
}

\providecommand{\obppcb}{
\toolsection{obppc64} is a compiler for the Oberon programming language targeting the PowerPC hardware architecture.
It generates machine code for PowerPC processors from modules written in Oberon and stores it in corresponding object files.
The compiler generates machine code for the 64-bit operating mode defined by the PowerPC architecture.
For debugging purposes, it also creates a debugging information file as well as an assembly file containing a listing of the generated machine code.
In addition, it stores the interface of each module in a symbol file which is required when other modules import the module.
Programs generated with this compiler require additional runtime support that is stored in the \file{ob\-ppc64\-run} library file.
\flowgraph{\resource{Oberon\\source code} \ar[r] & \toolbox{obppc64} \ar[r] \ar@/l/[d] \ar[rd] & \resource{object file} \\ \variable{ECSIMPORT} \ar[ru] & \resource{symbol\\files} \ar@/r/[u] & \resource{debugging\\information}}
\seeoberon\seeassembly\seeppc\seeobject\seedebugging
}

\providecommand{\obrisc}{
\toolsection{obrisc} is a compiler for the Oberon programming language targeting the RISC hardware architecture.
It generates machine code for RISC processors from modules written in Oberon and stores it in corresponding object files.
For debugging purposes, it also creates a debugging information file as well as an assembly file containing a listing of the generated machine code.
In addition, it stores the interface of each module in a symbol file which is required when other modules import the module.
Programs generated with this compiler require additional runtime support that is stored in the \file{ob\-risc\-run} library file.
\flowgraph{\resource{Oberon\\source code} \ar[r] & \toolbox{obrisc} \ar[r] \ar@/l/[d] \ar[rd] & \resource{object file} \\ \variable{ECSIMPORT} \ar[ru] & \resource{symbol\\files} \ar@/r/[u] & \resource{debugging\\information}}
\seeoberon\seeassembly\seerisc\seeobject\seedebugging
}

\providecommand{\obwasm}{
\toolsection{obwasm} is a compiler for the Oberon programming language targeting the WebAssembly architecture.
It generates machine code for WebAssembly targets from modules written in Oberon and stores it in corresponding object files.
For debugging purposes, it also creates a debugging information file as well as an assembly file containing a listing of the generated machine code.
In addition, it stores the interface of each module in a symbol file which is required when other modules import the module.
Programs generated with this compiler require additional runtime support that is stored in the \file{ob\-wasm\-run} library file.
\flowgraph{\resource{Oberon\\source code} \ar[r] & \toolbox{obwasm} \ar[r] \ar@/l/[d] \ar[rd] & \resource{object file} \\ \variable{ECSIMPORT} \ar[ru] & \resource{symbol\\files} \ar@/r/[u] & \resource{debugging\\information}}
\seeoberon\seeassembly\seewasm\seeobject\seedebugging
}

% converter tools

\providecommand{\dbgdwarf}{
\toolsection{dbgdwarf} is a DWARF debugging information converter tool.
It converts debugging information into the DWARF debugging data format and stores it in corresponding object files~\cite{dwarffile}.
The resulting debugging object files can be combined with runtime support that creates Executable and Linking Format (ELF) files~\cite{elffile}.
\flowgraph{\resource{debugging\\information} \ar[r] & \toolbox{dbgdwarf} \ar[r] & \resource{debugging\\object file}}
\seeobject\seedebugging
}

% assembler tools

\providecommand{\asmprint}{
\toolsection{asmprint} is a pretty printer for generic assembly code.
It reformats generic assembly code and writes it to the standard output stream.
\flowgraph{\resource{generic assembly\\source code} \ar[r] & \toolbox{asmprint} \ar[r] & \resource{reformatted\\source code}}
\seeassembly
}

\providecommand{\amdaasm}{
\toolsection{amd16asm} is an assembler for the AMD64 hardware architecture.
It translates assembly code into machine code for AMD64 processors and stores it in corresponding object files.
By default, the assembler generates machine code for the 16-bit operating mode defined by the AMD64 architecture.
\flowgraph{\resource{AMD16 assembly\\source code} \ar[r] & \toolbox{amd16asm} \ar[r] & \resource{object file}}
\seeassembly\seeamd\seeobject
}

\providecommand{\amdadism}{
\toolsection{amd16dism} is a disassembler for the AMD64 hardware architecture.
It translates machine code from object files targeting AMD64 processors into assembly code and writes it to the standard output stream.
It assumes that the machine code was generated for the 16-bit operating mode defined by the AMD64 architecture.
\flowgraph{\resource{object file} \ar[r] & \toolbox{amd16dism} \ar[r] & \resource{disassembly\\listing}}
\seeassembly\seeamd\seeobject
}

\providecommand{\amdbasm}{
\toolsection{amd32asm} is an assembler for the AMD64 hardware architecture.
It translates assembly code into machine code for AMD64 processors and stores it in corresponding object files.
By default, the assembler generates machine code for the 32-bit operating mode defined by the AMD64 architecture.
\flowgraph{\resource{AMD32 assembly\\source code} \ar[r] & \toolbox{amd32asm} \ar[r] & \resource{object file}}
\seeassembly\seeamd\seeobject
}

\providecommand{\amdbdism}{
\toolsection{amd32dism} is a disassembler for the AMD64 hardware architecture.
It translates machine code from object files targeting AMD64 processors into assembly code and writes it to the standard output stream.
It assumes that the machine code was generated for the 32-bit operating mode defined by the AMD64 architecture.
\flowgraph{\resource{object file} \ar[r] & \toolbox{amd32dism} \ar[r] & \resource{disassembly\\listing}}
\seeassembly\seeamd\seeobject
}

\providecommand{\amdcasm}{
\toolsection{amd64asm} is an assembler for the AMD64 hardware architecture.
It translates assembly code into machine code for AMD64 processors and stores it in corresponding object files.
By default, the assembler generates machine code for the 64-bit operating mode defined by the AMD64 architecture.
\flowgraph{\resource{AMD64 assembly\\source code} \ar[r] & \toolbox{amd64asm} \ar[r] & \resource{object file}}
\seeassembly\seeamd\seeobject
}

\providecommand{\amdcdism}{
\toolsection{amd64dism} is a disassembler for the AMD64 hardware architecture.
It translates machine code from object files targeting AMD64 processors into assembly code and writes it to the standard output stream.
It assumes that the machine code was generated for the 64-bit operating mode defined by the AMD64 architecture.
\flowgraph{\resource{object file} \ar[r] & \toolbox{amd64dism} \ar[r] & \resource{disassembly\\listing}}
\seeassembly\seeamd\seeobject
}

\providecommand{\armaasm}{
\toolsection{arma32asm} is an assembler for the ARM hardware architecture.
It translates assembly code into machine code for ARM processors executing A32 instructions and stores it in corresponding object files.
\flowgraph{\resource{ARM A32 assembly\\source code} \ar[r] & \toolbox{arma32asm} \ar[r] & \resource{object file}}
\seeassembly\seearm\seeobject
}

\providecommand{\armadism}{
\toolsection{arma32dism} is a disassembler for the ARM hardware architecture.
It translates machine code from object files targeting ARM processors executing A32 instructions into assembly code and writes it to the standard output stream.
\flowgraph{\resource{object file} \ar[r] & \toolbox{arma32dism} \ar[r] & \resource{disassembly\\listing}}
\seeassembly\seearm\seeobject
}

\providecommand{\armbasm}{
\toolsection{arma64asm} is an assembler for the ARM hardware architecture.
It translates assembly code into machine code for ARM processors executing A64 instructions and stores it in corresponding object files.
\flowgraph{\resource{ARM A64 assembly\\source code} \ar[r] & \toolbox{arma64asm} \ar[r] & \resource{object file}}
\seeassembly\seearm\seeobject
}

\providecommand{\armbdism}{
\toolsection{arma64dism} is a disassembler for the ARM hardware architecture.
It translates machine code from object files targeting ARM processors executing A64 instructions into assembly code and writes it to the standard output stream.
\flowgraph{\resource{object file} \ar[r] & \toolbox{arma64dism} \ar[r] & \resource{disassembly\\listing}}
\seeassembly\seearm\seeobject
}

\providecommand{\armcasm}{
\toolsection{armt32asm} is an assembler for the ARM hardware architecture.
It translates assembly code into machine code for ARM processors executing T32 instructions and stores it in corresponding object files.
\flowgraph{\resource{ARM T32 assembly\\source code} \ar[r] & \toolbox{armt32asm} \ar[r] & \resource{object file}}
\seeassembly\seearm\seeobject
}

\providecommand{\armcdism}{
\toolsection{armt32dism} is a disassembler for the ARM hardware architecture.
It translates machine code from object files targeting ARM processors executing T32 instructions into assembly code and writes it to the standard output stream.
\flowgraph{\resource{object file} \ar[r] & \toolbox{armt32dism} \ar[r] & \resource{disassembly\\listing}}
\seeassembly\seearm\seeobject
}

\providecommand{\avrasm}{
\toolsection{avrasm} is an assembler for the AVR hardware architecture.
It translates assembly code into machine code for AVR processors and stores it in corresponding object files.
The identifiers \texttt{RXL}, \texttt{RXH}, \texttt{RYL}, \texttt{RYH}, \texttt{RZL}, and \texttt{RZH} are predefined and name the corresponding registers.
The identifiers \texttt{SPL} and \texttt{SPH} are also predefined and evaluate to the address of the corresponding registers.
\flowgraph{\resource{AVR assembly\\source code} \ar[r] & \toolbox{avrasm} \ar[r] & \resource{object file}}
\seeassembly\seeavr\seeobject
}

\providecommand{\avrdism}{
\toolsection{avrdism} is a disassembler for the AVR hardware architecture.
It translates machine code from object files targeting AVR processors into assembly code and writes it to the standard output stream.
\flowgraph{\resource{object file} \ar[r] & \toolbox{avrdism} \ar[r] & \resource{disassembly\\listing}}
\seeassembly\seeavr\seeobject
}

\providecommand{\avrttasm}{
\toolsection{avr32asm} is an assembler for the AVR32 hardware architecture.
It translates assembly code into machine code for AVR32 processors and stores it in corresponding object files.
\flowgraph{\resource{AVR32 assembly\\source code} \ar[r] & \toolbox{avr32asm} \ar[r] & \resource{object file}}
\seeassembly\seeavrtt\seeobject
}

\providecommand{\avrttdism}{
\toolsection{avr32dism} is a disassembler for the AVR32 hardware architecture.
It translates machine code from object files targeting AVR32 processors into assembly code and writes it to the standard output stream.
\flowgraph{\resource{object file} \ar[r] & \toolbox{avr32dism} \ar[r] & \resource{disassembly\\listing}}
\seeassembly\seeavrtt\seeobject
}

\providecommand{\mabkasm}{
\toolsection{m68kasm} is an assembler for the M68000 hardware architecture.
It translates assembly code into machine code for M68000 processors and stores it in corresponding object files.
\flowgraph{\resource{68000 assembly\\source code} \ar[r] & \toolbox{m68kasm} \ar[r] & \resource{object file}}
\seeassembly\seemabk\seeobject
}

\providecommand{\mabkdism}{
\toolsection{m68kdism} is a disassembler for the M68000 hardware architecture.
It translates machine code from object files targeting M68000 processors into assembly code and writes it to the standard output stream.
\flowgraph{\resource{object file} \ar[r] & \toolbox{m68kdism} \ar[r] & \resource{disassembly\\listing}}
\seeassembly\seemabk\seeobject
}

\providecommand{\miblasm}{
\toolsection{miblasm} is an assembler for the MicroBlaze hardware architecture.
It translates assembly code into machine code for MicroBlaze processors and stores it in corresponding object files.
\flowgraph{\resource{MicroBlaze assembly\\source code} \ar[r] & \toolbox{miblasm} \ar[r] & \resource{object file}}
\seeassembly\seemibl\seeobject
}

\providecommand{\mibldism}{
\toolsection{mibldism} is a disassembler for the MicroBlaze hardware architecture.
It translates machine code from object files targeting MicroBlaze processors into assembly code and writes it to the standard output stream.
\flowgraph{\resource{object file} \ar[r] & \toolbox{mibldism} \ar[r] & \resource{disassembly\\listing}}
\seeassembly\seemibl\seeobject
}

\providecommand{\mipsaasm}{
\toolsection{mips32asm} is an assembler for the MIPS32 hardware architecture.
It translates assembly code into machine code for MIPS32 processors and stores it in corresponding object files.
\flowgraph{\resource{MIPS32 assembly\\source code} \ar[r] & \toolbox{mips32asm} \ar[r] & \resource{object file}}
\seeassembly\seemips\seeobject
}

\providecommand{\mipsadism}{
\toolsection{mips32dism} is a disassembler for the MIPS32 hardware architecture.
It translates machine code from object files targeting MIPS32 processors into assembly code and writes it to the standard output stream.
\flowgraph{\resource{object file} \ar[r] & \toolbox{mips32dism} \ar[r] & \resource{disassembly\\listing}}
\seeassembly\seemips\seeobject
}

\providecommand{\mipsbasm}{
\toolsection{mips64asm} is an assembler for the MIPS64 hardware architecture.
It translates assembly code into machine code for MIPS64 processors and stores it in corresponding object files.
\flowgraph{\resource{MIPS64 assembly\\source code} \ar[r] & \toolbox{mips64asm} \ar[r] & \resource{object file}}
\seeassembly\seemips\seeobject
}

\providecommand{\mipsbdism}{
\toolsection{mips64dism} is a disassembler for the MIPS64 hardware architecture.
It translates machine code from object files targeting MIPS64 processors into assembly code and writes it to the standard output stream.
\flowgraph{\resource{object file} \ar[r] & \toolbox{mips64dism} \ar[r] & \resource{disassembly\\listing}}
\seeassembly\seemips\seeobject
}

\providecommand{\mmixasm}{
\toolsection{mmixasm} is an assembler for the MMIX hardware architecture.
It translates assembly code into machine code for MMIX processors and stores it in corresponding object files.
The names of all special registers are predefined and evaluate to the corresponding number.
\flowgraph{\resource{MMIX assembly\\source code} \ar[r] & \toolbox{mmixasm} \ar[r] & \resource{object file}}
\seeassembly\seemmix\seeobject
}

\providecommand{\mmixdism}{
\toolsection{mmixdism} is a disassembler for the MMIX hardware architecture.
It translates machine code from object files targeting MMIX processors into assembly code and writes it to the standard output stream.
\flowgraph{\resource{object file} \ar[r] & \toolbox{mmixdism} \ar[r] & \resource{disassembly\\listing}}
\seeassembly\seemmix\seeobject
}

\providecommand{\orokasm}{
\toolsection{or1kasm} is an assembler for the OpenRISC 1000 hardware architecture.
It translates assembly code into machine code for OpenRISC 1000 processors and stores it in corresponding object files.
\flowgraph{\resource{OpenRISC 1000 assembly\\source code} \ar[r] & \toolbox{or1kasm} \ar[r] & \resource{object file}}
\seeassembly\seeorok\seeobject
}

\providecommand{\orokdism}{
\toolsection{or1kdism} is a disassembler for the OpenRISC 1000 hardware architecture.
It translates machine code from object files targeting OpenRISC 1000 processors into assembly code and writes it to the standard output stream.
\flowgraph{\resource{object file} \ar[r] & \toolbox{or1kdism} \ar[r] & \resource{disassembly\\listing}}
\seeassembly\seeorok\seeobject
}

\providecommand{\ppcaasm}{
\toolsection{ppc32asm} is an assembler for the PowerPC hardware architecture.
It translates assembly code into machine code for PowerPC processors and stores it in corresponding object files.
By default, the assembler generates machine code for the 32-bit operating mode defined by the PowerPC architecture.
\flowgraph{\resource{PowerPC assembly\\source code} \ar[r] & \toolbox{ppc32asm} \ar[r] & \resource{object file}}
\seeassembly\seeppc\seeobject
}

\providecommand{\ppcadism}{
\toolsection{ppc32dism} is a disassembler for the PowerPC hardware architecture.
It translates machine code from object files targeting PowerPC processors into assembly code and writes it to the standard output stream.
It assumes that the machine code was generated for the 32-bit operating mode defined by the PowerPC architecture.
\flowgraph{\resource{object file} \ar[r] & \toolbox{ppc32dism} \ar[r] & \resource{disassembly\\listing}}
\seeassembly\seeppc\seeobject
}

\providecommand{\ppcbasm}{
\toolsection{ppc64asm} is an assembler for the PowerPC hardware architecture.
It translates assembly code into machine code for PowerPC processors and stores it in corresponding object files.
By default, the assembler generates machine code for the 64-bit operating mode defined by the PowerPC architecture.
\flowgraph{\resource{PowerPC assembly\\source code} \ar[r] & \toolbox{ppc64asm} \ar[r] & \resource{object file}}
\seeassembly\seeppc\seeobject
}

\providecommand{\ppcbdism}{
\toolsection{ppc64dism} is a disassembler for the PowerPC hardware architecture.
It translates machine code from object files targeting PowerPC processors into assembly code and writes it to the standard output stream.
It assumes that the machine code was generated for the 64-bit operating mode defined by the PowerPC architecture.
\flowgraph{\resource{object file} \ar[r] & \toolbox{ppc64dism} \ar[r] & \resource{disassembly\\listing}}
\seeassembly\seeppc\seeobject
}

\providecommand{\riscasm}{
\toolsection{riscasm} is an assembler for the RISC hardware architecture.
It translates assembly code into machine code for RISC processors and stores it in corresponding object files.
The names of all special registers are predefined and evaluate to the corresponding number.
\flowgraph{\resource{RISC assembly\\source code} \ar[r] & \toolbox{riscasm} \ar[r] & \resource{object file}}
\seeassembly\seerisc\seeobject
}

\providecommand{\riscdism}{
\toolsection{riscdism} is a disassembler for the RISC hardware architecture.
It translates machine code from object files targeting RISC processors into assembly code and writes it to the standard output stream.
\flowgraph{\resource{object file} \ar[r] & \toolbox{riscdism} \ar[r] & \resource{disassembly\\listing}}
\seeassembly\seerisc\seeobject
}

\providecommand{\wasmasm}{
\toolsection{wasmasm} is an assembler for the WebAssembly architecture.
It translates assembly code into machine code for WebAssembly targets and stores it in corresponding object files.
The names of all special registers are predefined and evaluate to the corresponding number.
\flowgraph{\resource{WebAssembly assembly\\source code} \ar[r] & \toolbox{wasmasm} \ar[r] & \resource{object file}}
\seeassembly\seewasm\seeobject
}

\providecommand{\wasmdism}{
\toolsection{wasmdism} is a disassembler for the WebAssembly architecture.
It translates machine code from object files targeting WebAssembly targets into assembly code and writes it to the standard output stream.
\flowgraph{\resource{object file} \ar[r] & \toolbox{wasmdism} \ar[r] & \resource{disassembly\\listing}}
\seeassembly\seewasm\seeobject
}

% linker tools

\providecommand{\linklib}{
\toolsection{linklib} is an object file combiner.
It creates a static library file by combining all object files given to it into a single one.
\flowgraph{\resource{object files} \ar[r] & \toolbox{linklib} \ar[r] & \resource{library file}}
\seeobject
}

\providecommand{\linkbin}{
\toolsection{linkbin} is a linker for plain binary files.
It links all object files given to it into a single image and stores it in a binary file that begins with the first linked section.
It also creates a map file that lists the address, type, name and size of all used sections.
The filename extension of the resulting binary file can be specified by putting it into a constant data section called \texttt{\_extension}.
\flowgraph{\resource{object files} \ar[r] & \toolbox{linkbin} \ar[r] \ar[d] & \resource{binary file} \\ & \resource{map file}}
\seeobject
}

\providecommand{\linkmem}{
\toolsection{linkmem} is a linker for plain binary files partitioned into random-access and read-only memory.
It links all object files given to it into two distinct images, one for data sections and one for code and constant data sections, and stores each image in a binary file that begins with the first linked section of the corresponding type.
It also creates a map file that lists the address, type, name and size of all used sections.
\flowgraph{\resource{object files} \ar[r] & \toolbox{linkmem} \ar[r] \ar[d] & \resource{RAM file/\\ROM file} \\ & \resource{map file}}
\seeobject
}

\providecommand{\linkprg}{
\toolsection{linkprg} is a linker for GEMDOS executable files.
It links all object files given to it into a single image and stores the image in an Atari GEMDOS executable file~\cite{gemdosfile}.
It also creates a map file that lists the address relative to the text segment, type, name and size of all used sections.
The filename extension of the resulting executable file can be specified by putting it into a constant data section called \texttt{\_extension}.
The GEMDOS executable file format requires all patch patterns of absolute link patches to consist of four full bitmasks with descending offsets.
\flowgraph{\resource{object files} \ar[r] & \toolbox{linkprg} \ar[r] \ar[d] & \resource{executable file} \\ & \resource{map file}}
\seeobject
}

\providecommand{\linkhex}{
\toolsection{linkhex} is a linker for Intel HEX files.
It links all code sections of the object files given to it into single image and stores the image in an Intel HEX file~\cite{hexfile} that begins with the first linked section.
It also creates a map file that lists the address, type, name and size of all used sections.
\flowgraph{\resource{object files} \ar[r] & \toolbox{linkhex} \ar[r] \ar[d] & \resource{HEX file} \\ & \resource{map file}}
\seeobject
}

\providecommand{\mapsearch}{
\toolsection{mapsearch} is a debugging tool.
It searches map files generated by linker tools for the name of a binary section that encompasses a memory address read from the standard input stream.
If additionally provided with one or more object files, it also stores an excerpt thereof in a separate object file called map search result which only contains the identified binary section for disassembling purposes.
\flowgraph{& \resource{map files/\\object files} \ar[d] \\ \resource{memory\\address} \ar[r] & \toolbox{mapsearch} \ar[r] \ar[d] & \resource{section name/\\relative offset} \\ & \resource{object file\\excerpt}}
\seeobject
}

\renewcommand{\seeassembly}{}

\startchapter{Generic Assembly Language}{Generic Assembly Language Specification}{assembly}
{The \ecs{} provides assemblers for several different hardware architectures.
All of these tools are based on a generic assembly language that supports all of their common features.
This \documentation{} describes the generic assembly language and its implementation by the \ecs{}.}

\epigraph{Die Wahrheit und Einfachheit der Natur \\ sind immer die letzten Grundlagen \\ einer bedeutenden Kunst.}{Paul Ernst}

\section{Introduction}

The most obvious task of an assembler is to translate the textual representation of some processor instructions into their binary form.
The target hardware architecture of the assembler specifies the instruction set and the actual encoding of its instructions.
Therefore, an assembler implementing one specific instruction set is in general not able to translate instructions of another instruction set.

There is some functionality however that all assemblers do have in common.
This includes for example the support for constant definitions, symbolic names for the target of branch instructions, or sections that group instructions together.
The \ecs{} defines a generic assembly language that supports all of these common features.
The generic assembly language is therefore an abstraction for all concrete assembly languages and instruction sets supported by the assemblers of the \ecs{}.
These assemblers only have to implement the translation of those parts of a program that are actually dependent on the hardware architecture.

\section{Representation}

The \ecs{} provides several different assemblers.
They target different hardware architectures and therefore implement the encoding of different instruction sets.
However, each assembler additionally implements the generic assembly language which is able to represent all these instruction sets in an abstract way.
This section describes how assembly programs are represented by the generic assembly language.

\subsection{Programs}

An assembly program expressed in generic assembly language consists of an arbitrary number of interconnected units called \emph{sections}.
Sections are used as containers for data and machine code.
Each section contains \emph{instructions} and \emph{directives} that textually describe the actual binary contents of the section.

An instruction is a single operation of a processor implementing the target hardware architecture.
The architecture defines the available set of instructions and their textual and binary representations.
An assembler for that architecture uses this encoding to generate the actual machine code for each instruction in the source code.
Directives on the other hand do usually not generate any machine code but rather influence the actual encoding and state of the assembler.

\subsection{Section Types}\index{Section types}\index{Types, of sections}\label{sec:asmsectiontypes}

Sections represent contiguous memory regions containing either data or executable machine code.
The \emph{type} of a section describes the intended purpose of this memory region:

\begin{itemize}

\item Code Sections\index{Code sections}\nopagebreak

Code sections contain the binary representation of machine code for a specific hardware architecture.
\emph{Standard code sections}\index{Standard code sections} are typically used to represent functions which are usually called by machine code contained in other code sections.
\emph{Initializing code sections}\index{Initializing code sections} on the other hand represent machine code that has to be executed automatically at the begin of a program.
\emph{Data initializing code sections}\index{Data initializing code sections} are executed prior to other initializing code sections.
Their purpose is to initialize data sections with the data needed by initializing code sections.

\item Data Sections\index{Data sections}\nopagebreak

Data sections are not executed but contain the binary representation of the global data of a program.
\emph{Standard data sections}\index{Standard data sections} usually contain the modifiable global data of a program like global variables.
Often times, standard data sections only reserve space for data because it cannot be statically initialized and has to be generated at runtime.
\emph{Constant data sections}\index{Constant data sections} on the other hand represent global data that is not variable and remains constant throughout the execution of a program.
Constant data sections are typically used to represent character strings.

\item Metadata Sections\index{Metadata sections}\nopagebreak

Metadata sections are not part of the actual program but contain metadata about it.
\emph{Heading metadata sections}\index{Heading metadata sections} are placed in front of all other sections while \emph{trailing metadata sections}\index{Trailing metadata sections} are placed behind all other sections.
This allows mimicking the layout of some specific binary file format when linking object files into a single binary file.

\end{itemize}

The type of a section influences its actual placement in memory.
Constant data sections for example may be placed in a read-only memory area.
Initializing code sections are placed in front of all other code sections in order to guarantee their automatic execution.
In general, the distinction between code and data sections enables supporting separate memories as required by Harvard architectures.

\subsection{Section Options}\index{Section Options}\index{Options, of sections}\label{sec:asmsectionoptions}

Each section has a symbolic name that represents its actual start address in memory.
In addition, optional aliases allow referring to specific offsets within a section by name.
The \emph{options} of a section describe what happens if two or more sections can be referred to by the same name or are never used.
Sections can have the following freely combinable options:

\begin{itemize}

\item Required Sections\index{Required sections}\nopagebreak

Sections marked as \emph{required} are always part of the resulting binary program.
All other sections are only used if actually referenced directly or indirectly by required sections.

\item Duplicable Sections\index{Duplicable sections}\nopagebreak

There may be two or more sections with the same name, if and only if both sections are marked as \emph{duplicable} and contain exactly the same binary data.
All but one of the duplicated sections are discarded and references to them are redirected to the remaining one.
Constant data sections representing character strings are typically marked as duplicable.

\item Replaceable Sections\index{Replaceable sections}\nopagebreak

Sections marked as \emph{replaceable} exist as long as there is no other section with the same name.
If there is one, the replaceable section is discarded and references to it are redirected to the other one.

\end{itemize}

Assembly programs that begin with instructions rather than section creation directives implicitly define a standard code section called \texttt{main}.
These sections represent entry points of programs and are therefore always implicitly required.
They are executed automatically after initializing code sections and have to eventually return the control of execution to the host environment.

\subsection{Syntax}\index{Syntax, of Generic Assembly Language}

Each line of code of a program written in generic assembly language contains some combination of the following three fields:

\begin{quote}\begin{grammar}
<Line> = <Label>$\opt$ <Statement>$\opt$ <Comment>$\opt$ \par
<Label> = <Identifier>":" \par
<Statement> = <Instruction> $\mid$ <Directive> \par
<Instruction> = <Mnemonic> <Expressions>$\opt$ \par
<Directive> = "."<Mnemonic> <Expressions>$\opt$ $\mid$ "#"<Mnemonic> <Expressions>$\opt$ \par
<Mnemonic> = <Identifier> \par
\end{grammar}\end{quote}

Each instruction and directives defining constants or binary data may be associated with a \emph{label}.
Labels are unique identifiers that are followed by a colon.
A label can be used to refer to the numerical offset of the labeled instruction within the current section using a unique symbolic name.
It can also be used to refer to the expression of a \emph{constant definition} within the current section.

The mnemonic is a symbolic name that identifies the actual instruction or directive.
Mnemonics of directives are prefixed with a dot or a number sign.
Most instructions and directives take one or more operand specifiers that are separated by commas.
Operands are arbitrary expressions that specify numeric values, registers, labels, or memory addresses.
Section~\ref{sec:asmexpressions} describes the capabilities and representation of expressions.
Sections~\ref{sec:asminstructions} and~\ref{sec:asmdirectives} give more information about the valid operands of instructions and directives.

Finally, the generic assembly language supports comments that are ignored during the translation and may therefore contain a human-readable description of the source code.
Comments are introduced with a semicolon, may contain any text, and extend to the end of the line.

\subsection{Translation}\label{sec:asmtranslation}

The generic syntax defined by the generic assembly language allows supporting any textual representation.
Assemblers implementing the generic assembly language translate the source code in two different stages:

\begin{enumerate}

\item
The source code is decomposed into an abstract syntax tree representing each line of the source code.
If a line contains a directive, it is executed according to Section~\ref{sec:asmdirectives} and the next line of code is translated.
If a line contains an instruction, its operands are resolved and the whole instruction is rewritten using these results.
Resolving in this context means evaluating subexpressions or mapping labels to offsets.
If the operand or its subexpressions contain identifiers or operations that cannot be evaluated, they are rewritten as they are.
This most often includes instruction mnemonics or the names of registers only the concrete assembler in use knows about.

\item
The rewritten instruction is passed to the concrete assembler in use.
This assembler in turn assembles one instruction at a time and returns its equivalent binary encoding.

\end{enumerate}

Decomposing the translation into two consecutive stages offers some advantages.
First, all of the common tasks of the assemblers can be performed in the first stage.
Features supported by this stage are automatically available in any assembler by design.
This includes managing sections, writing the resulting object file, or supporting forward referencing labels using a prior assembly pass that only records instruction offsets.
Second, the actual translation of the rewritten instruction in the second stage is much easier because all symbolic names and expressions are already resolved beforehand.
This enables a loose coupling between the generic and the concrete part of an assembler.
These two parts only interface by exchanging a simple textual representation of an instruction by its binary equivalent as shown in Figure~\ref{fig:asmdataflow}.

\section{Expressions}\label{sec:asmexpressions}

The generic assembly language is able to handle arbitrarily complex expressions that can be used as operands for instructions and directives.
Sections~\ref{sec:asminstructions} and~\ref{sec:asmdirectives} give more information about the valid operands of instructions and directives.
This section describes how expressions are evaluated according to the generic assembly language.
They are textually represented according to the following syntax:

\begin{quote}\begin{grammar}
<Expressions> = <Expression> $\mid$ <Expressions> "," <Expression> \par
<Expression> = <Character> $\mid$ <String> $\mid$ <Number> $\mid$ <Identifier> $\mid$ \\ <Address> $\mid$ <Unary-Operation> $\mid$ <Binary-Operation> \par
\end{grammar}\end{quote}

Oftentimes, composite expressions like unary and binary operations as well as some instructions and directives need numeric operands.
In these cases, expressions are evaluated yielding a constant signed integer number as result.
However, there are expressions that cannot be evaluated and do therefore not yield a result.
Such expressions are literally treated as text during the translation as described in Section~\ref{sec:asmtranslation}.

\subsection{Characters}

The generic assembly language treats any non-empty sequence of up to eight characters enclosed in single quotes as character literals.
A character literal can be evaluated and yields an integer number that corresponds to the character sequence as represented in memory of the target hardware architecture.
Special characters can be defined using the escape sequences as shown in Table~\ref{tab:asmescapesequences}.

\subsection{Strings}

Assemblers treat any character sequences enclosed in double quotes as string literals.
Strings cannot be evaluated, but assemblers are able to generate the binary data representation of each character in a string literal.
Special characters can be defined using the escape sequences as shown in Table~\ref{tab:asmescapesequences}.
The number of octets representing a single character depends on the actual directive used to generate the binary data.

\begin{table}
\centering
\begin{tabular}{@{}cl@{}}
\toprule Sequence & Description \\ \midrule
\texttt{\textbackslash 0} & null character \\
\texttt{\textbackslash a} & alert \\
\texttt{\textbackslash b} & backspace \\
\texttt{\textbackslash t} & horizontal tab \\
\texttt{\textbackslash n} & new-line \\
\texttt{\textbackslash v} & vertical tab \\
\texttt{\textbackslash f} & form feed \\
\texttt{\textbackslash r} & carriage return \\
\texttt{\textbackslash '} & single quote \\
\texttt{\textbackslash "} & double quote \\
\texttt{\textbackslash \textbackslash} & backslash \\
\texttt{\textbackslash ?} & question mark \\
\texttt{\textbackslash x}$n$ & literal character with ordinal value \\ & expressed by hexadecimal number $n$ \\
\bottomrule
\end{tabular}
\caption{Escape sequences supported by the generic assembly language}
\label{tab:asmescapesequences}
\end{table}

\subsection{Numbers}

The generic assembly language supports numeric literals for the evaluation of expressions.
It allows specifying integer numbers using different number bases according to the following syntax:

\begin{quote}\begin{grammar}
<Number> = <Binary> $\mid$ <Octal> $\mid$ <Decimal> $\mid$ <Hexadecimal> \par
<Binary> = "0b" <Digits> $\mid$ <Digits> "b" \par
<Octal> = "0o" <Digits> $\mid$ <Digits> "o" $\mid$ "0" <Digits> \par
<Decimal> = "0d" <Digits> $\mid$ <Digits> "d" $\mid$ <Integer> \par
<Hexadecimal> = "0h" <Digits> $\mid$ <Digits> "h" $\mid$ "0x" <Digits> \par
<Integer> = <Digits> \par
\end{grammar}\end{quote}

The digit sequence of a number may contain separating single quotes which are ignored when determining its value.
Although assemblers do recognize the syntax of real numbers, they cannot be evaluated and are rewritten identically.

\subsection{Identifiers}

An identifier is any sequence of characters like letters and digits that does not begin with a digit.
It either refers to a label, a section name, or to a predefined identifier of the actual instruction set.
This includes mnemonics and register names defined by the instruction set for example.

If the identifier names a label, its value is the offset of the label within the binary representation of the section.
Usually, this offset is expressed in octets.
If an instruction expects other units or a relative operand however, the offset is automatically modified accordingly.
If the identifier names a constant definition, its value corresponds to the expression of that definition.
The mnemonic of some directives can also be used in expressions and yield the value of their current setting, see Section~\ref{sec:asmspecialpurposeoperations}.
In all other cases, identifiers cannot be evaluated and are simply treated as text.

\subsection{Addresses}

Addresses are identifiers that are prefixed with an at sign and allow referencing code or data sections by name.
They do not yield a value, but they act as a placeholder for the future memory address of the referenced section.
Addresses are usually absolute and expressed in octets, but if there are instructions that expect operands with other units or relative references, they are automatically modified accordingly.
All expressions involving an address may additionally be incremented or decremented by a signed displacement and shifted to the right by a scale.

If the address identifier contains question marks, the actual referenced section is named behind the last question mark.
If none of the sections named in front of a question mark are used, the actual address evaluates either to zero or to the section named behind an optional colon.

\subsection{Arithmetic Operations}

The generic assembly language supports the following arithmetic operations.
The precedence of each operation is shown in Table~\ref{tab:asmoperatorprecedence}.
All unary operations have right-to-left associativity.
All binary operations have left-to-right associativity:

\newcommand{\asmoperatorref}[2]{& \texttt{#1} & #2 \\}

\begin{table}
\centering
\begin{tabular}{@{}ccl@{}}
\toprule Precedence & Operator & Operation \\
\midrule 1
\asmoperatorref{( )}{Grouping or memory access}
\asmoperatorref{[ ]}{Memory access}
\asmoperatorref{\{ \}}{Register list}
\midrule 2
\asmoperatorref{+}{Identity}
\asmoperatorref{-}{Negation}
\asmoperatorref{\textasciitilde}{Bitwise NOT}
\asmoperatorref{!}{Logical NOT}
\midrule 3
\asmoperatorref{*}{Multiplication}
\asmoperatorref{/}{Division}
\asmoperatorref{\%}{Modulo}
\midrule 4
\asmoperatorref{+}{Addition}
\asmoperatorref{-}{Subtraction}
\midrule 5
\asmoperatorref{<{}<}{Bitwise left shift}
\asmoperatorref{>{}>}{Bitwise right shift}
\midrule 6
\asmoperatorref{<}{Less-than comparison}
\asmoperatorref{<=}{Less-than-or-equal-to comparison}
\asmoperatorref{>}{Greater-than comparison}
\asmoperatorref{>=}{Greater-than-or-equal-to comparison}
\midrule 7
\asmoperatorref{==}{Equal comparison}
\asmoperatorref{!=}{Unequal comparison}
\midrule 8
\asmoperatorref{===}{Identical comparison}
\asmoperatorref{!==}{Unidentical comparison}
\midrule 9
\asmoperatorref{\&}{Bitwise AND}
\midrule 10
\asmoperatorref{\^}{Bitwise exclusive OR}
\midrule 11
\asmoperatorref{|}{Bitwise inclusive OR}
\midrule 12
\asmoperatorref{\&\&}{Logical AND}
\midrule 13
\asmoperatorref{||}{Logical OR}
\bottomrule
\end{tabular}
\caption{Operator precedence defined by the generic assembly language}
\label{tab:asmoperatorprecedence}
\end{table}

\newcommand{\asmoperator}[1]{\item #1\alignright\nopagebreak}

\begin{itemize}

\asmoperator{Identity}\syntax{"+" <Expression>}

The result of the identity operation is the value of its operand.

\asmoperator{Negation}\syntax{"-" <Expression>}

The negation operation negates the value of its operand.

\asmoperator{Addition}\syntax{<Expression> "+" <Expression>}

The addition operation computes the sum of the values of its operands.

\asmoperator{Subtraction}\syntax{<Expression> "-" <Expression>}

The subtraction operation computes the difference of the values of its operands.

\asmoperator{Multiplication}\syntax{<Expression> "*" <Expression>}

The multiplication operation computes the product of the values of its operands.

\asmoperator{Division}\syntax{<Expression> "/" <Expression>}

The division operation computes the integer quotient of the division of the value of its first operand by the value of the second value.

\asmoperator{Modulo}\syntax{<Expression> "\%" <Expression>}

The modulo operation computes the integer remainder of the division of the value of its first operand by the value of the second value.

\end{itemize}

\subsection{Bitwise Operations}

The generic assembly language supports the following bitwise operations.
The precedence of each operation is shown in Table~\ref{tab:asmoperatorprecedence}:

\begin{itemize}

\asmoperator{Bitwise NOT}\syntax{"~" <Expression>}

The bitwise NOT operation computes the one's complement of the value of its operand.

\asmoperator{Bitwise left shift}\syntax{<Expression> "<""<" <Expression>}

The bitwise left shift operation computes the value of its first operand shifted left by the value of its second operand.

\asmoperator{Bitwise right shift}\syntax{<Expression> ">"">" <Expression>}

The bitwise right shift operation computes the value of its first operand shifted right by the value of its second operand.

\asmoperator{Bitwise AND}\syntax{<Expression> "&" <Expression>}

The bitwise AND operation yields the bitwise AND function of the values of its operands.

\asmoperator{Bitwise inclusive OR}\syntax{<Expression> "|" <Expression>}

The bitwise inclusive OR operation yields the bitwise inclusive OR function of the values of its operands.

\asmoperator{Bitwise exclusive OR}\syntax{<Expression> "^" <Expression>}

The bitwise exclusive OR operation yields the bitwise exclusive OR function of the values of its operands.

\end{itemize}

\subsection{Logical Operations}

The generic assembly language supports the following logical operations.
The precedence of each operation is shown in Table~\ref{tab:asmoperatorprecedence}.
During all these logical operations, an evaluated value of zero denotes the Boolean value false.
All other values represent the Boolean value true.
Logical operations yield either the value one or the value zero representing the Boolean values true and false respectively:

\begin{itemize}

\asmoperator{Logical NOT}\syntax{"!" <Expression>}

The logical NOT operation yields the complement of the Boolean value of its operand.

\asmoperator{Logical AND}\syntax{<Expression> "&&" <Expression>}

The logical AND operation performs a logical conjunction of the Boolean values of its operands.
The second operand is not evaluated if the first operand evaluates to false.

\asmoperator{Logical OR}\syntax{<Expression> "||" <Expression>}

The logical OR operation performs a logical disjunction of the Boolean values of its operands.
The second operand is not evaluated if the first operand evaluates to true.

\end{itemize}

\subsection{Comparison Operations}

The generic assembly language supports the following comparison operations.
The precedence of each operation is shown in Table~\ref{tab:asmoperatorprecedence}.
Comparison operations yield either the value one or the value zero representing the Boolean values true and false respectively:

\begin{itemize}

\asmoperator{Equal comparison}\syntax{<Expression> "==" <Expression>}

The equal comparison operation returns a Boolean value indicating whether the values of both of its operands are equal.

\asmoperator{Unequal comparison}\syntax{<Expression> "!=" <Expression>}

The unequal comparison operation returns a Boolean value indicating whether the values of both of its operands are unequal.

\asmoperator{Less-than comparison}\syntax{<Expression> "<" <Expression>}

The less-than comparison operation returns a Boolean value indicating whether the value of its first operand is less than the value of its second operand.

\asmoperator{Less-than-or-equal comparison}\syntax{<Expression> "<=" <Expression>}

The less-than-or-equal comparison operation returns a Boolean value indicating whether the value of its first operand is less than or equal to the value of its second operand.

\asmoperator{Greater-than comparison}\syntax{<Expression> ">" <Expression>}

The greater-than comparison operation returns a Boolean value indicating whether the value of its first operand is greater than the value of its second operand.

\asmoperator{Greater-than-or-equal comparison}\syntax{<Expression> ">=" <Expression>}

The greater-than-or-equal comparison operation returns a Boolean value indicating whether the value of its first operand is greater than or equal to the value of its second operand.

\asmoperator{Identical comparison}\syntax{<Expression> "===" <Expression>}

The identical comparison operation returns a Boolean value indicating whether the lexical token sequences of both of its operands are identical.

\asmoperator{Unidentical comparison}\syntax{<Expression> "!==" <Expression>}

The unidentical comparison operation returns a Boolean value indicating whether the lexical token sequences of both of its operands are unidentical.

\end{itemize}

\subsection{Special-Purpose Operations}\label{sec:asmspecialpurposeoperations}

In addition to all arithmetic, bitwise, logical, and comparison operations, the generic assembly language supports the following special-purpose operations:

\begin{itemize}

\asmoperator{Bit Mode}\syntax{".bitmode"}

The bit mode operation returns the current bit mode that is used to translate instructions into machine code.
This operation returns the default bit mode of the target hardware architecture if available and left unchanged by the bit mode directive, see Section~\ref{sec:asmbitmode}.

\asmoperator{Little-Endian Mode}\syntax{".little"}

The little-endian mode operation returns a Boolean value indicating whether generated binary data is currently ordered least significant octet first.
This operation returns the default endianness of the target hardware architecture if left unchanged by endian mode directives, see Sections~\ref{sec:asmbig} and~\ref{sec:asmlittle}.

\asmoperator{Big-Endian Mode}\syntax{".big"}

The big-endian mode operation returns a Boolean value indicating whether generated binary data is currently ordered most significant octet first.
This operation returns the default endianness of the target hardware architecture if left unchanged by endian mode directives, see Sections~\ref{sec:asmbig} and~\ref{sec:asmlittle}.

\asmoperator{Section Alignment}\syntax{".alignment"}

The section alignment operation returns the alignment of the current section as specified by the section alignment directive, see Section~\ref{sec:asmalignment}.

\asmoperator{Section Origin}\syntax{".origin"}

The section origin operation returns the origin of the current section as specified by the section origin directive, see Section~\ref{sec:asmorigin}.

\asmoperator{Section Group}\syntax{".group"}

The section group operation returns the group name of the current section as specified by the section group directive, see Section~\ref{sec:asmgroup}.

\asmoperator{Required Section}\syntax{".required"}

The required section operation returns a Boolean value indicating whether the current section is required as marked by the required section directive, see Section~\ref{sec:asmrequired}.

\asmoperator{Duplicable Section}\syntax{".duplicable"}

The duplicable section operation returns a Boolean value indicating whether the current section is duplicable as marked by the duplicable section directive, see Section~\ref{sec:asmduplicable}.

\asmoperator{Replaceable Section}\syntax{".replaceable"}

The replaceable section operation returns a Boolean value indicating whether the current section is replaceable as marked by the replaceable section directive, see Section~\ref{sec:asmreplaceable}.

\end{itemize}

The following special-purpose operations have a functional syntax accepting a single argument and modify the result of referencing labels or sections by name:

\begin{itemize}

\asmoperator{Relative Offset}\syntax{"offset" "(" <Expression> ")"}

The relative offset operation returns the offset of the named label or section relative to the current position expressed in octets.

\asmoperator{Section Size}\syntax{"size" "(" <Expression> ")"}

The section size operation returns the binary size of the named section expressed in octets.

\asmoperator{Section Extent}\syntax{"extent" "(" <Expression> ")"}

The section extent operation returns the address of the named section plus its binary size.

\asmoperator{Member Position}\syntax{"position" "(" <Expression> ")"}

The member position operation returns the position of the named section relative to the beginning of its group expressed in octets.
See Section~\ref{sec:asmgroup} for more information about the section group directive.

\asmoperator{Member Index}\syntax{"index" "(" <Expression> ")"}

The member index operation returns the index of the named section within the sequence of sections of its group.
See Section~\ref{sec:asmgroup} for more information about the section group directive.

\asmoperator{Member Count}\syntax{"count" "(" <Expression> ")"}

The member count operation returns the total number of sections contained in the named group.
See Section~\ref{sec:asmgroup} for more information about the section group directive.

\end{itemize}

\section{Instructions}\index{Instructions}\label{sec:asminstructions}

The actually available set of instructions and its textual representation are defined by the target hardware architecture.
The generic assembly language therefore does not know anything about the syntax or semantics of the concrete instruction set in use.
For a complete description of the concrete instruction set and its usage, users have to refer to the documentation of the corresponding target hardware architecture.
The remainder of this section lists all instruction sets supported by \ecs{}.

\renewcommand{\instruction}[2]{\pdftooltip{\texttt{#1}}{#2}\space}
\renewcommand{\instructionset}[1]{\begin{quote}\sloppy\footnotesize\input{#1.set}\end{quote}}

\subsection{AMD64 Instruction Set}

The \ecs{} supports the AMD64 instruction set as listed below and uses the same assembly syntax as predefined by AMD~\cite{amd64:volume3,amd64:volume4,amd64:volume5}.
The only exception are instructions for far procedure calls and far jumps which take the selector and offset of the immediate far pointer as two separate operands.
The actual set of supported instructions depends on the operating mode that is used.
If there are two or more possible encodings of an instruction, the shortest one is used by default.
In order to specify the encoding of an instruction explicitly, the size of its operands can be overridden by prefixing them with one of the following identifiers:
\texttt{byte} for 8-bit, \texttt{word} for 16-bit, \texttt{dword} for 32-bit, \texttt{qword} for 64-bit, \texttt{oword} for 128-bit, and \texttt{hword} for 256-bit.
\seeamd

\instructionset{amd64}

\subsection{ARM Instruction Sets}

The \ecs{} supports the ARM A32 instruction set as listed below and uses the same assembly syntax as predefined by the ARM architecture~\cite{arm:instructionset}.
The only exception are immediate values which are not prefixed by a number sign.
\seearm

\instructionset{arma32}

The \ecs{} supports the ARM A64 instruction set as listed below and uses the same assembly syntax as predefined by the ARM architecture~\cite{arm:instructionset}.
The only exception are immediate values which are not prefixed by a number sign.

\instructionset{arma64}

The \ecs{} supports the ARM T32 instruction set as listed below and uses the same assembly syntax as predefined by the ARM architecture~\cite{arm:instructionset}.
The only exception are immediate values which are not prefixed by a number sign.

\instructionset{arm}

\subsection{AVR Instruction Set}

The \ecs{} supports the AVR instruction set as listed below and uses the same assembly syntax as predefined by Atmel~\cite{avr:instructionset}.
\seeavr

\instructionset{avr}

\subsection{AVR32 Instruction Set}

The \ecs{} supports the AVR32 instruction set as listed below and uses the same assembly syntax as predefined by Atmel~\cite{avr32:instructionset}.
Instructions that operate on register lists are not supported.
\seeavrtt

\instructionset{avr32}

\subsection{M68000 Instruction Set}

The \ecs{} supports the M68000 instruction set as listed below and uses the same assembly syntax as predefined by Motorola~\cite{m68k:instructionset}.
The only exception are immediate values which are not prefixed by a number sign.
\seemabk

\instructionset{m68k}

\subsection{MicroBlaze Instruction Set}

The \ecs{} supports the MicroBlaze instruction set as listed below and uses the same assembly syntax as predefined by Xilinx~\cite{mibl:instructionset}.
\seemibl

\instructionset{mibl}

\subsection{MIPS Instruction Set}

The \ecs{} supports the MIPS32 and MIPS64 instruction sets as listed below and uses the same assembly syntax as predefined by MIPS Technologies~\cite{mips:volume1,mips:volume2}.
\seemips

\instructionset{mips}

\subsection{MMIX Instruction Set}

The \ecs{} supports the MMIX instruction set as listed below and uses the same assembly syntax as predefined by Donald~E.\ Knuth~\cite{mmixware}.
The only exception are immediate values which are not prefixed by a number sign.
\seemmix

\instructionset{mmix}

\subsection{OpenRISC 1000 Instruction Set}

The \ecs{} supports the OpenRISC 1000 instruction set as listed below and uses the same assembly syntax as predefined by OpenCores~\cite{or1k:instructionset}.
\seeorok

\instructionset{or1k}

\subsection{PowerPC Instruction Set}

The \ecs{} supports the PowerPC instruction set as listed below and uses the same assembly syntax as predefined by IBM~\cite{ppc:instructionset}.
\seeppc

\instructionset{ppc}

\subsection{RISC Instruction Set}

The \ecs{} supports the RISC instruction set as listed below and uses the same assembly syntax as predefined by Niklaus Wirth~\cite{risc:instructionset}.
\seerisc

\instructionset{risc}

\subsection{WebAssembly Instruction Set}

The \ecs{} supports the WebAssembly instruction set as listed below and uses the same assembly syntax as predefined by the World Wide Web Consortium (W3C)~\cite{wasm:instructionset}.
The only exception are the addition of the pseudo instructions \texttt{i32}, \texttt{label}, \texttt{lane}, \texttt{s32}, \texttt{u32}, and \texttt{valtype}.
They encode an immediate value of the respective type with a fixed size and can be used as a replacement for operands of instructions that require immediate values of that type.
\seewasm

\instructionset{wasm}

\subsection{Xtensa Instruction Set}

The \ecs{} supports the Xtensa instruction set as listed below and uses the same assembly syntax as predefined by Cadence Design Systems~\cite{xtensa:instructionset}.
\seextensa

\instructionset{xtensa}

\subsection{Intermediate Code Instruction Set}

Intermediate code, as exposed by several tools of the \ecs{} for debugging purposes, can also be represented using the generic assembly language.
The corresponding instruction set is listed below.
\seecode

\instructionset{code}

\section{Directives}\index{Directives}\label{sec:asmdirectives}

The generic assembly language defines a common set of directives for all concrete assemblers.
In contrast to instructions, directives do not depend on the instruction set defined by the target hardware architecture.
Tables~\ref{tab:asmdirectives} and~\ref{tab:asmpreprocessingdirectives} summarize all of these directives and the category they belong to.

\newcommand{\asmdirectiveref}[2]{& \texttt{#1#2} & \ref{sec:asm#2} \\}

\begin{table}
\centering
\begin{tabular}{@{}llll@{}}
\toprule Category & Mnemonic & See Section \\
\midrule Section Creation
\asmdirectiveref{.}{code}
\asmdirectiveref{.}{initcode}
\asmdirectiveref{.}{initdata}
\asmdirectiveref{.}{data}
\asmdirectiveref{.}{const}
\asmdirectiveref{.}{header}
\asmdirectiveref{.}{trailer}
\midrule Section Options
\asmdirectiveref{.}{required}
\asmdirectiveref{.}{duplicable}
\asmdirectiveref{.}{replaceable}
\midrule Section Placement
\asmdirectiveref{.}{alignment}
\asmdirectiveref{.}{origin}
\asmdirectiveref{.}{group}
\midrule Data Definition
\asmdirectiveref{.}{byte}
\asmdirectiveref{.}{dbyte}
\asmdirectiveref{.}{tbyte}
\asmdirectiveref{.}{qbyte}
\asmdirectiveref{.}{obyte}
\asmdirectiveref{.}{embed}
\midrule Memory Layout
\asmdirectiveref{.}{pad}
\asmdirectiveref{.}{align}
\asmdirectiveref{.}{reserve}
\midrule Mode Selection
\asmdirectiveref{.}{big}
\asmdirectiveref{.}{little}
\asmdirectiveref{.}{bitmode}
\midrule Special Purpose
\asmdirectiveref{.}{alias}
\asmdirectiveref{.}{assert}
\asmdirectiveref{.}{equals}
\asmdirectiveref{.}{require}
\asmdirectiveref{.}{trace}
\bottomrule
\end{tabular}
\caption{Directives of the generic assembly language}
\label{tab:asmdirectives}
\end{table}

\begin{table}
\centering
\begin{tabular}{@{}llll@{}}
\toprule Category & Mnemonic & See Section \\
\midrule Code Control
\asmdirectiveref{\#}{end}
\asmdirectiveref{\#}{line}
\midrule Conditional Inclusion
\asmdirectiveref{\#}{if}
\asmdirectiveref{\#}{elif}
\asmdirectiveref{\#}{else}
\asmdirectiveref{\#}{endif}
\midrule Code Repetition
\asmdirectiveref{\#}{repeat}
\asmdirectiveref{\#}{endrep}
\midrule Macro Definition
\asmdirectiveref{\#}{define}
\asmdirectiveref{\#}{enddef}
\asmdirectiveref{\#}{undef}
\bottomrule
\end{tabular}
\caption{Preprocessing directives of the generic assembly language}
\label{tab:asmpreprocessingdirectives}
\end{table}

Directives usually do not generate any machine code, they rather modify the generation of binary code and data itself.
They also allow describing characteristics of the current section or creating new sections.
All settings and modifications performed by directives apply only to the current section and are reset afterward.
Preprocessing directives on the other hand allow controlling the handling of the source code itself like repeating or conditionally including some portions of code.

The \ecs{} features compilers for programming languages that allow users to write so-called \emph{inline assembly code}\index{Inline assembly code} within standard program code.
Behind the scenes, compilers usually generate a code section for each functional unit of the implemented programming language.
Inline assembly code allows users to describe a part of this code section using the generic assembly language.
Since the placement and options of the section are still managed by the compiler however, directives for modifying these properties or creating new sections are not available in inline assembly code.

\newcommand{\asmdirective}[4]{\subsection[#2]{\texttt{#1#2}\enskip\textnormal{\textit{#3}}\enskip\alignright#4}\label{sec:asm#2}}

\asmdirective{.}{alias}{Identifier}{Alias Name}

The alias name directive assigns an additional name for the data or code at the current position in the section.
This data or code can be referred to by the name of the current section plus its position or simply by the new alias name.
Alias names may not be duplicated and must differ from the name of the section.
This directive may not be labeled.

\asmdirective{.}{align}{Expression}{Data Alignment}

The data alignment directive allows padding binary data space in order to accommodate alignment constraints.
It advances the current position in the section to the next multiple of the alignment specified by the operand.
It is expressed in octets and has to be a positive power of two.
This directive may not be labeled.

\asmdirective{.}{alignment}{Expression}{Section Alignment}

The section alignment directive specifies the alignment of the section.
The actual address of the section is a multiple of its alignment which may only be specified once and only if the section has no origin, see Section~\ref{sec:asmorigin}.
It is expressed in multiples of octets and has to be a positive power of two.
This directive is not available in inline assembly code and may not be labeled.

\asmdirective{.}{assert}{Expression}{Static Assertion}

The static assertion directive allows asserting conditions during the translation of assembly code.
It tests the Boolean value of its operand and aborts the translation if the condition is not satisfied.
This directive may not be labeled.

\asmdirective{.}{big}{}{Big-Endian Mode}

The big-endian mode directive changes the ordering of generated binary data to most significant octet first.
Some hardware architectures allow changing the endianness of machine code, in which case this ordering also applies to the binary encoding of instructions.
This directive may not be labeled and must restore the original endianness mode if it was modified in inline assembly code.

\asmdirective{.}{bitmode}{Expression}{Bit Mode}

The bit mode directive allows changing the bit mode that is used to translate the instructions of the current section into machine code.
The set of valid bit modes depends on the hardware architecture in use, where some architectures do not define bit modes at all.
Users have to refer to the documentation of the respective target hardware architecture.
This directive may not be labeled and must restore the original bit mode if it was modified in inline assembly code.

\asmdirective{.}{byte}{Expressions}{Byte Data}

The byte data directive allows generating the binary representation of the value of one or more operands.
Each value is represented using a single octet.
If an operand for this directive is a string, each character of the string is evaluated individually.

\asmdirective{.}{code}{Identifier}{Standard Code Section}

The standard code section directive creates a new standard code section with the specified name.
See Section~\ref{sec:asmsectiontypes} for information about standard code sections.
All labels as well as instructions and directives following this directive belong to the new section.
This directive is not available in inline assembly code and may not be labeled.

\asmdirective{.}{const}{Identifier}{Constant Data Section}

The constant data section directive creates a new constant data section with the specified name.
See Section~\ref{sec:asmsectiontypes} for information about constant data sections.
All labels as well as instructions and directives following this directive belong to the new section.
This directive is not available in inline assembly code and may not be labeled.

\asmdirective{.}{data}{Identifier}{Standard Data Section}

The standard data section directive creates a new standard data section with the specified name.
See Section~\ref{sec:asmsectiontypes} for information about standard data sections.
All labels as well as instructions and directives following this directive belong to the new section.
This directive is not available in inline assembly code and may not be labeled.

\asmdirective{.}{dbyte}{Expressions}{Double Byte Data}

The double byte data directive allows generating the binary representation of the value of one or more operands.
Each value is represented using two octets, where the current endianness mode defines the ordering of the most and least significant one.
If an operand for this directive is a string, each character of the string is evaluated individually.

\asmdirective{\#}{define}{Identifier}{Begin Macro Definition}

The begin macro definition preprocessing directive allows defining pseudo instructions.
Macros have a unique name and contain any sequence of assembly code until the next end macro definition directive.
As long as a macro is defined, its name can be used as the mnemonic of a new pseudo instruction that takes up to ten operands separated by commas.
This instruction is then replaced by the code sequence encompassed by the corresponding macro in a process called macro expansion.
While expanding, number sign characters followed by a single decimal digit within any character or string literal, identifier, address, or label are replaced by the operand identified by that zero-based index.
Two consecutive sign characters are replaced by the decimal line number of the macro expansion and allow defining unique labels.
This directive may not be labeled and requires a subsequent end macro definition directive.

\asmdirective{.}{duplicable}{}{Duplicable Section}

The duplicable section directive marks the current section as duplicable.
See Section~\ref{sec:asmsectionoptions} for information about duplicable sections.
This directive may only be specified once, is not available in inline assembly code and may not be labeled.

\asmdirective{\#}{elif}{Expression}{Conditional Else-If}

The conditional else-if preprocessing directive allows conditional inclusions of instructions and directives during the translation of assembly code.
It tests the Boolean value of its operand and skips subsequent code until the next conditional else or end-if directive
if the condition is not satisfied or any preceding conditional if or else-if directives did not skip code.
This directive may not be labeled and requires a preceding conditional if directive and a subsequent conditional end-if directive.

\asmdirective{\#}{else}{}{Conditional Else}

The conditional else preprocessing directive allows excluding instructions and directives during the translation of assembly code.
It skips subsequent code until the next conditional end-if directive if any preceding conditional if or else-if directives did not skip code.
This directive may not be labeled and requires a preceding conditional if or else-if directive and a subsequent conditional end-if directive.

\asmdirective{.}{embed}{String}{Embed Binary File}

The embed binary file directive allows copying the contents of a binary file named by the string operand into the current section.

\asmdirective{\#}{end}{}{End Code}

The end code preprocessing directive stops the translation of the assembly code.
The remaining part of the source code is completely ignored.
This directive may not be labeled.

\asmdirective{\#}{enddef}{}{End Macro Definition}

The end macro definition preprocessing directive allows specifying the end of the assembly code sequence encompassed by the current macro definition.
This directive may not be labeled and requires a preceding begin macro definition directive.

\asmdirective{\#}{endif}{}{Conditional End-If}

The conditional end-if preprocessing directive allows ending the conditional inclusion of instructions and directives during the translation of assembly code.
This directive may not be labeled and requires a preceding conditional if, else-if, or else directive.

\asmdirective{\#}{endrep}{}{End Repetition}

The end repetition preprocessing directive allows specifying the end of the assembly code sequence encompassed by the current code repetition.
This directive may not be labeled and requires a preceding begin repetition directive.

\asmdirective{.}{equals}{Expression}{Constant Definition}

The constant definition directive allows defining symbolic constants within the current section.
Constant definitions require a label which can be used to refer to the corresponding constant expression by name.
Constant definitions with the same name are allowed if the lexical token sequences of their operands are also identical.

\asmdirective{.}{group}{Identifier}{Section Group}

The section group directive allows grouping several sections together by aligning them adjacent to each other.
The group may only be specified once and is treated as the name of a virtual section that contains all sections in that group.
Multiple groups of sections with the same type are placed consecutively in lexicographic order.
This directive is not available in inline assembly code and may not be labeled.

\asmdirective{.}{header}{Identifier}{Heading Metadata Section}

The heading metadata section directive creates a new heading metadata section with the specified name.
See Section~\ref{sec:asmsectiontypes} for information about heading metadata sections.
All labels as well as instructions and directives following this directive belong to the new section.
This directive is not available in inline assembly code and may not be labeled.

\asmdirective{\#}{if}{Expression}{Conditional If}

The conditional if preprocessing directive allows conditional inclusions of instructions and directives during the translation of assembly code.
It tests the Boolean value of its operand and skips subsequent code until the next conditional else-if, else, or end-if directive if the condition is not satisfied.
This directive may not be labeled and requires a subsequent conditional end-if directive.

\asmdirective{.}{initcode}{Identifier}{Initializing Code Section}

The initializing code section directive creates a new initializing code section with the specified name.
See Section~\ref{sec:asmsectiontypes} for information about initializing code sections.
All labels as well as instructions and directives following this directive belong to the new section.
This directive is not available in inline assembly code and may not be labeled.

\asmdirective{.}{initdata}{Identifier}{Data Initializing Code Section}

The data initializing code section directive creates a new data initializing code section with the specified name.
See Section~\ref{sec:asmsectiontypes} for information about data initializing code sections.
All labels as well as instructions and directives following this directive belong to the new section.
This directive is not available in inline assembly code and may not be labeled.

\asmdirective{\#}{line}{Integer String$\opt$}{Line Control}

The line control preprocessing directive allow overwriting the filename of the source code and the position therein used in diagnostic messages.
The second operand is optional and specifies the filename of the source code.
The remaining part of the source code is treated as if it would start in that file at the line specified by the first operand.

\asmdirective{.}{little}{}{Little-Endian Mode}

The little-endian mode directive changes the ordering of generated binary data to least significant octet first.
Some hardware architectures allow changing the endianness of machine code, in which case this ordering also applies to the binary encoding of instructions.
This directive may not be labeled and must restore the original endianness mode if it was modified in inline assembly code.

\asmdirective{.}{obyte}{Expressions}{Octuple Byte Data}

The octuple byte data directive allows generating the binary representation of the value of one or more operands.
Each value is represented using eight octets, where the current endianness mode defines the ordering of the most and least significant one.
If an operand for this directive is a string, each character of the string is evaluated individually.

\asmdirective{.}{origin}{Expression}{Section Origin}

The section origin directive specifies the origin of the section.
The origin defines the actual address of a section and may only be specified once and only if the section has no alignment, see Section~\ref{sec:asmalignment}.
It is expressed in multiples of octets and may not be less than zero.
This directive is not available in inline assembly code and may not be labeled.

\asmdirective{.}{pad}{Expression}{Data Padding}

The data padding directive allows padding binary data space in order to accommodate layout constraints.
It advances the current position in the section to the octet specified by the operand.
This directive may not be labeled.

\asmdirective{.}{qbyte}{Expressions}{Quadruple Byte Data}

The quadruple byte data directive allows generating the binary representation of the value of one or more operands.
Each value is represented using four octets, where the current endianness mode defines the ordering of the most and least significant one.
If an operand for this directive is a string, each character of the string is evaluated individually.

\asmdirective{\#}{repeat}{Integer}{Begin Repetition}

The begin repetition preprocessing directive allows translating any sequence of assembly code several times.
Its operand specifies the number of repetitions of the assembly code encompassed by this directive and the subsequent end repetition directive.
While repeating, two consecutive number sign characters within any character or string literal, identifier, address, or label are replaced by the decimal integer number of complete repetitions done so far.
This directive may not be labeled and requires a subsequent end repetition directive.

\asmdirective{.}{replaceable}{}{Replaceable Section}

The replaceable section directive marks the current section as replaceable.
See Section~\ref{sec:asmsectionoptions} for information about replaceable sections.
This directive may only be specified once, is not available in inline assembly code and may not be labeled.

\asmdirective{.}{require}{Identifier}{Name Requirement}

The name requirement directive allows specifying additional dependencies on data or code sections that are otherwise not explicitly referenced in the current section.
This directive may not be labeled.

\asmdirective{.}{required}{}{Required Section}

The required section directive marks the current section as required.
See Section~\ref{sec:asmsectionoptions} for information about required sections.
This directive may only be specified once, is not available in inline assembly code and may not be labeled.

\asmdirective{.}{reserve}{Expression}{Data Reservation}

The data reservation directive allows reserving binary data without having to initialize it.
The size of the reserved data space is expressed in multiples of octets and may not be less than zero.

\asmdirective{.}{tbyte}{Expressions}{Triple Byte Data}

The triple byte data directive allows generating the binary representation of the value of one or more operands.
Each value is represented using three octets, where the current endianness mode defines the ordering of the most and least significant one.
If an operand for this directive is a string, each character of the string is evaluated individually.

\asmdirective{.}{trace}{Expressions$\opt$}{Expression Evaluation}

The expression evaluation directive allows rewriting and evaluating arbitrary expressions during the translation of assembly code for debugging purposes.
If no operands are provided, all currently defined labels and constant definitions are evaluated instead.
This directive may not be labeled.

\asmdirective{.}{trailer}{Identifier}{Trailing Metadata Section}

The trailing metadata section directive creates a new trailing metadata section with the specified name.
See Section~\ref{sec:asmsectiontypes} for information about trailing metadata sections.
All labels as well as instructions and directives following this directive belong to the new section.
This directive is not available in inline assembly code and may not be labeled.

\asmdirective{\#}{undef}{Identifier}{Remove Macro Definition}

The remove macro definition preprocessing directive causes the specified identifier no longer to be defined as a macro and thus allows redefining macros with the same name.
This directive may not be labeled.

\section{Assembler Tools}

The \ecs{} provides assemblers for several different hardware architectures.
\interface\seeguide

The assemblers process assembly source code in several consecutive stages.
In each stage, the internal representation of the module is changed and transformed.
Figure~\ref{fig:asmdataflow} shows all stages and the different representations.

\begin{figure}
\flowgraph{
\resource{generic assembly\\source code} \ar[d] \\
\converter{Lexer} \ar[d] \\
\resource{tokens} \ar[d] \\
\converter{Parser} \ar[d] \\
\resource{abstract\\syntax tree} \ar[d] \ar[r] & \converter{Pretty Printer} \ar[r] & \resource{reformatted\\source code} \\
\converter{Generic\\Assembler} \ar[d] \ar@/u/[rr]^{\txt{rewritten instruction}} & \txt{\hphantom{binary encoding}} & \converter{Concrete\\Assembler} \ar@/d/[ll]^{\txt{binary encoding}} \\
\resource{object file} \\
}\caption{Data flow within assembler tools}
\label{fig:asmdataflow}
\end{figure}

\asmprint
\amdaasm
\amdbasm
\amdcasm
\armaasm
\armbasm
\armcasm
\avrasm
\avrttasm
\mabkasm
\miblasm
\mipsaasm
\mipsbasm
\mmixasm
\orokasm
\ppcaasm
\ppcbasm
\riscasm
\wasmasm
\xtensaasm

\section{Disassembler Tools}

Each assembler provided by the \ecs{} is accompanied by a corresponding disassembler tool which reverses its operation.
Disassemblers open object files and print a human-readable disassembly listing of the machine code contained therein as shown in Figure~\ref{fig:dismdataflow}.
They accept command-line arguments which are taken as names of object files.
If no arguments are provided, object files are read from the standard input stream.
\seeinterface\seeguide

\begin{figure}
\flowgraph{
\resource{object file} \ar[d] \\
\converter{Generic\\Disassembler} \ar[d] \ar@/u/[rr]^{\txt{machine code}} & \txt{\hphantom{machine code}} & \converter{Concrete\\Disassembler} \ar@/d/[ll]^{\txt{decoded instruction}} \\
\resource{disassembly\\listing} \\
}\caption{Data flow within disassembler tools}
\label{fig:dismdataflow}
\end{figure}

\amdadism
\amdbdism
\amdcdism
\armadism
\armbdism
\armcdism
\avrdism
\avrttdism
\mabkdism
\mibldism
\mipsadism
\mipsbdism
\mmixdism
\orokdism
\ppcadism
\ppcbdism
\riscdism
\wasmdism
\xtensadism

\concludechapter


\part{Supported Hardware Architectures}
// AMD64 instruction set definitions
// Copyright (C) Florian Negele

// This file is part of the Eigen Compiler Suite.

// The ECS is free software: you can redistribute it and/or modify
// it under the terms of the GNU General Public License as published by
// the Free Software Foundation, either version 3 of the License, or
// (at your option) any later version.

// The ECS is distributed in the hope that it will be useful,
// but WITHOUT ANY WARRANTY; without even the implied warranty of
// MERCHANTABILITY or FITNESS FOR A PARTICULAR PURPOSE.  See the
// GNU General Public License for more details.

// You should have received a copy of the GNU General Public License
// along with the ECS.  If not, see <https://www.gnu.org/licenses/>.

#ifndef CODE
	#define CODE(code)
#endif

#ifndef FLAG
	#define FLAG(flag, value)
#endif

#ifndef INSTR
	#define INSTR(mnem, type1, type2, type3, type4, exprefix, opcode, code1, code2, suffix, flags)
#endif

#ifndef MNEM
	#define MNEM(name, mnem, ...)
#endif

#ifndef PREFIX
	#define PREFIX(prefix, byte)
#endif

#ifndef REG
	#define REG(reg, name)
#endif

#ifndef TYPE
	#define TYPE(type)
#endif

// mnemonics

MNEM (aaa,               AAA,               ASCII Adjust After Addition)
MNEM (aad,               AAD,               ASCII Adjust Before Division)
MNEM (aam,               AAM,               ASCII Adjust After Multiply)
MNEM (aas,               AAS,               ASCII Adjust After Subtraction)
MNEM (adc,               ADC,               Add with Carry)
MNEM (adcx,              ADCX,              Unsigned Add with Carry Flag)
MNEM (add,               ADD,               Signed or Unsigned Add)
MNEM (addpd,             ADDPD,             Add Packed Double-Precision Floating-Point)
MNEM (addps,             ADDPS,             Add Packed Single-Precision Floating-Point)
MNEM (addsd,             ADDSD,             Add Scalar Double-Precision Floating-Point)
MNEM (addss,             ADDSS,             Add Scalar Single-Precision Floating-Point)
MNEM (addsubpd,          ADDSUBPD,          Add and Subtract Packed Double-Precision)
MNEM (addsubps,          ADDSUBPS,          Add and Subtract Packed Single-Precision)
MNEM (adox,              ADOX,              Unsigned Add with Overflow Flag)
MNEM (aesdec,            AESDEC,            AES Decryption Round)
MNEM (aesdeclast,        AESDECLAST,        AES Last Decryption Round)
MNEM (aesenc,            AESENC,            AES Encryption Round)
MNEM (aesenclast,        AESENCLAST,        AES Last Encryption Round)
MNEM (aesimc,            AESIMC,            AES InvMixColumn Transformation)
MNEM (aeskeygenassist,   AESKEYGENASSIST,   AES Assist Round Key Generation)
MNEM (and,               AND,               Logical AND)
MNEM (andn,              ANDN,              Logical And-Not)
MNEM (andnpd,            ANDNPD,            Logical Bitwise AND NOT Packed Double-Precision Floating-Point)
MNEM (andnps,            ANDNPS,            Logical Bitwise AND NOT Packed Single-Precision Floating-Point)
MNEM (andpd,             ANDPD,             Logical Bitwise AND Packed Double-Precision Floating-Point)
MNEM (andps,             ANDPS,             Logical Bitwise AND Packed Single-Precision Floating-Point)
MNEM (arpl,              ARPL,              Adjust Requestor Privilege Level)
MNEM (bextr,             BEXTR,             Bit Field Extract)
MNEM (blcfill,           BLCFILL,           Fill From Lowest Clear Bit)
MNEM (blci,              BLCI,              Isolate Lowest Clear Bit)
MNEM (blcic,             BLCIC,             Isolate Lowest Clear Bit and Complement)
MNEM (blcmsk,            BLCMSK,            Mask From Lowest Clear Bit)
MNEM (blcs,              BLCS,              Set Lowest Clear Bit)
MNEM (blendpd,           BLENDPD,           Blend Packed Double-Precision Floating-Point)
MNEM (blendps,           BLENDPS,           Blend Packed Single-Precision Floating-Point)
MNEM (blendvpd,          BLENDVPD,          Variable Blend Packed Double-Precision Floating-Point)
MNEM (blendvps,          BLENDVPS,          Variable Blend Packed Single-Precision Floating-Point)
MNEM (blsfill,           BLSFILL,           Fill From Lowest Set Bit)
MNEM (blsi,              BLSI,              Isolate Lowest Set Bit)
MNEM (blsic,             BLSIC,             Isolate Lowest Set Bit and Complement)
MNEM (blsmsk,            BLSMSK,            Mask From Lowest Set Bit)
MNEM (blsr,              BLSR,              Reset Lowest Set Bit)
MNEM (bound,             BOUND,             Check Array Bound)
MNEM (bsf,               BSF,               Bit Scan Forward)
MNEM (bsr,               BSR,               Bit Scan Reverse)
MNEM (bswap,             BSWAP,             Byte Swap)
MNEM (bt,                BT,                Bit Test)
MNEM (btc,               BTC,               Bit Test and Complement)
MNEM (btr,               BTR,               Bit Test and Reset)
MNEM (bts,               BTS,               Bit Test and Set)
MNEM (bzhi,              BZHI,              Zero High Bits)
MNEM (call,              CALL,              Near Procedure Call)
MNEM (callfar,           CALLFAR,           Far Procedure Call)
MNEM (cbw,               CBW,               Convert AL to Sign-Extended AX)
MNEM (cdq,               CDQ,               Convert EAX to Sign-Extended EDX:EAX)
MNEM (cdqe,              CDQE,              Convert EAX to Sign-Extended RAX)
MNEM (clac,              CLAC,              Clear Alignment Check Flag)
MNEM (clc,               CLC,               Clear Carry Flag)
MNEM (cld,               CLD,               Clear Direction Flag)
MNEM (clflush,           CLFLUSH,           Cache Line Flush)
MNEM (clflushopt,        CLFLUSHOPT,        Optimized Cache Line Flush)
MNEM (clgi,              CLGI,              Clear Global Interrupt Flag)
MNEM (cli,               CLI,               Clear Interrupt Flag)
MNEM (clrssbsy,          CLRSSBSY,          Clear Shadow Stack Busy)
MNEM (clts,              CLTS,              Clear Task-Switched Flag in CR0)
MNEM (clwb,              CLWB,              Cache Line Write Back and Retain)
MNEM (clzero,            CLZERO,            Zero Cache Line)
MNEM (cmc,               CMC,               Complement Carry Flag)
MNEM (cmova,             CMOVA,             Move if above (CF = 0 and ZF = 0))
MNEM (cmovae,            CMOVAE,            Move if above or equal (CF = 0))
MNEM (cmovb,             CMOVB,             Move if below (CF = 1))
MNEM (cmovbe,            CMOVBE,            Move if below or equal (CF = 1 or ZF = 1))
MNEM (cmovc,             CMOVC,             Move if carry (CF = 1))
MNEM (cmove,             CMOVE,             Move if equal (ZF =1))
MNEM (cmovg,             CMOVG,             Move if greater (ZF = 0 and SF = OF))
MNEM (cmovge,            CMOVGE,            Move if greater or equal (SF = OF))
MNEM (cmovl,             CMOVL,             Move if less (SF <> OF))
MNEM (cmovle,            CMOVLE,            Move if less or equal (ZF = 1 or SF <> OF))
MNEM (cmovna,            CMOVNA,            Move if not above (CF = 1 or ZF = 1))
MNEM (cmovnae,           CMOVNAE,           Move if not above or equal (CF = 1))
MNEM (cmovnb,            CMOVNB,            Move if not below (CF = 0))
MNEM (cmovnbe,           CMOVNBE,           Move if not below or equal (CF = 0 and ZF = 0))
MNEM (cmovnc,            CMOVNC,            Move if not carry (CF = 0))
MNEM (cmovne,            CMOVNE,            Move if not equal (ZF = 0))
MNEM (cmovng,            CMOVNG,            Move if not greater (ZF = 1 or SF <> OF))
MNEM (cmovnge,           CMOVNGE,           Move if not greater or equal (SF <> OF))
MNEM (cmovnl,            CMOVNL,            Move if not less (SF = OF))
MNEM (cmovnle,           CMOVNLE,           Move if not less or equal (ZF = 0 and SF = OF))
MNEM (cmovno,            CMOVNO,            Move if not overflow (OF = 0))
MNEM (cmovnp,            CMOVNP,            Move if not parity (PF = 0))
MNEM (cmovns,            CMOVNS,            Move if not sign (SF = 0))
MNEM (cmovnz,            CMOVNZ,            Move if not zero (ZF = 0))
MNEM (cmovo,             CMOVO,             Move if overflow (OF = 1))
MNEM (cmovp,             CMOVP,             Move if parity (PF = 1))
MNEM (cmovpe,            CMOVPE,            Move if parity even (PF = 1))
MNEM (cmovpo,            CMOVPO,            Move if parity odd (PF = 0))
MNEM (cmovs,             CMOVS,             Move if sign (SF =1))
MNEM (cmovz,             CMOVZ,             Move if zero (ZF = 1))
MNEM (cmp,               CMP,               Compare)
MNEM (cmpeqpd,           CMPEQPD,           Compare Packed Double-Precision Floating-Point Equal)
MNEM (cmpeqps,           CMPEQPS,           Compare Packed Single-Precision Floating-Point Equal)
MNEM (cmpeqsd,           CMPEQSD,           Compare Scalar Double-Precision Floating-Point Equal)
MNEM (cmpeqss,           CMPEQSS,           Compare Scalar Single-Precision Floating-Point Equal)
MNEM (cmplepd,           CMPLEPD,           Compare Packed Double-Precision Floating-Point Less or Equal)
MNEM (cmpleps,           CMPLEPS,           Compare Packed Single-Precision Floating-Point Less or Equal)
MNEM (cmplesd,           CMPLESD,           Compare Scalar Double-Precision Floating-Point Less or Equal)
MNEM (cmpless,           CMPLESS,           Compare Scalar Single-Precision Floating-Point Less or Equal)
MNEM (cmpltpd,           CMPLTPD,           Compare Packed Double-Precision Floating-Point Less Than)
MNEM (cmpltps,           CMPLTPS,           Compare Packed Single-Precision Floating-Point Less Than)
MNEM (cmpltsd,           CMPLTSD,           Compare Scalar Double-Precision Floating-Point Less Than)
MNEM (cmpltss,           CMPLTSS,           Compare Scalar Single-Precision Floating-Point Less Than)
MNEM (cmpneqpd,          CMPNEQPD,          Compare Packed Double-Precision Floating-Point Not Equal)
MNEM (cmpneqps,          CMPNEQPS,          Compare Packed Single-Precision Floating-Point Not Equal)
MNEM (cmpneqsd,          CMPNEQSD,          Compare Scalar Double-Precision Floating-Point Not Equal)
MNEM (cmpneqss,          CMPNEQSS,          Compare Scalar Single-Precision Floating-Point Not Equal)
MNEM (cmpnlepd,          CMPNLEPD,          Compare Packed Double-Precision Floating-Point Not Less or Equal)
MNEM (cmpnleps,          CMPNLEPS,          Compare Packed Single-Precision Floating-Point Not Less or Equal)
MNEM (cmpnlesd,          CMPNLESD,          Compare Scalar Double-Precision Floating-Point Not Less or Equal)
MNEM (cmpnless,          CMPNLESS,          Compare Scalar Single-Precision Floating-Point Not Less or Equal)
MNEM (cmpnltpd,          CMPNLTPD,          Compare Packed Double-Precision Floating-Point Not Less Than)
MNEM (cmpnltps,          CMPNLTPS,          Compare Packed Single-Precision Floating-Point Not Less Than)
MNEM (cmpnltsd,          CMPNLTSD,          Compare Scalar Double-Precision Floating-Point Not Less Than)
MNEM (cmpnltss,          CMPNLTSS,          Compare Scalar Single-Precision Floating-Point Not Less Than)
MNEM (cmpordpd,          CMPORDPD,          Compare Packed Double-Precision Floating-Point Ordered)
MNEM (cmpordps,          CMPORDPS,          Compare Packed Single-Precision Floating-Point Ordered)
MNEM (cmpordsd,          CMPORDSD,          Compare Scalar Double-Precision Floating-Point Ordered)
MNEM (cmpordss,          CMPORDSS,          Compare Scalar Single-Precision Floating-Point Ordered)
MNEM (cmppd,             CMPPD,             Compare Packed Double-Precision Floating-Point)
MNEM (cmpps,             CMPPS,             Compare Packed Single-Precision Floating-Point)
MNEM (cmpsb,             CMPSB,             Compare Bytes)
MNEM (cmpsd,             CMPSD,             Compare Doublewords)
MNEM (cmpsq,             CMPSQ,             Compare Quadwords)
MNEM (cmpss,             CMPSS,             Compare Scalar Single-Precision Floating-Point)
MNEM (cmpsw,             CMPSW,             Compare Words)
MNEM (cmpunordpd,        CMPUNORDPD,        Compare Packed Double-Precision Floating-Point Unordered)
MNEM (cmpunordps,        CMPUNORDPS,        Compare Packed Single-Precision Floating-Point Unordered)
MNEM (cmpunordsd,        CMPUNORDSD,        Compare Scalar Double-Precision Floating-Point Unordered)
MNEM (cmpunordss,        CMPUNORDSS,        Compare Scalar Single-Precision Floating-Point Unordered)
MNEM (cmpxchg,           CMPXCHG,           Compare and Exchange)
MNEM (cmpxchg16b,        CMPXCHG16B,        Compare and Exchange Sixteen Bytes)
MNEM (cmpxchg8b,         CMPXCHG8B,         Compare and Exchange Eight Bytes)
MNEM (comisd,            COMISD,            Compare Ordered Scalar Double-Precision Floating-Point)
MNEM (comiss,            COMISS,            Compare Ordered Scalar Single-Precision Floating-Point)
MNEM (cpuid,             CPUID,             Processor Identification)
MNEM (cqo,               CQO,               Convert RAX to Sign-Extended RDX:RAX)
MNEM (crc32,             CRC32,             CRC32 Cyclical Redundancy Check)
MNEM (cvtdq2pd,          CVTDQ2PD,          Convert Packed Doubleword Integers to Packed Double-Precision Floating-Point)
MNEM (cvtdq2ps,          CVTDQ2PS,          Convert Packed Doubleword Integers to Packed Single-Precision Floating-Point)
MNEM (cvtpd2dq,          CVTPD2DQ,          Convert Packed Double-Precision Floating-Point to Packed Doubleword Integers)
MNEM (cvtpd2pi,          CVTPD2PI,          Convert Packed Double-Precision Floating-Point to Packed Doubleword Integers)
MNEM (cvtpd2ps,          CVTPD2PS,          Convert Packed Double-Precision Floating-Point to Packed Single-Precision Floating-Point)
MNEM (cvtpi2pd,          CVTPI2PD,          Convert Packed Doubleword Integers to Packed Double-Precision Floating-Point)
MNEM (cvtpi2ps,          CVTPI2PS,          Convert Packed Doubleword Integers to Packed Single-Precision Floating-Point)
MNEM (cvtps2dq,          CVTPS2DQ,          Convert Packed Single-Precision Floating-Point to Packed Doubleword Integers)
MNEM (cvtps2pd,          CVTPS2PD,          Convert Packed Single-Precision Floating-Point to Packed Double-Precision Floating-Point)
MNEM (cvtps2pi,          CVTPS2PI,          Convert Packed Single-Precision Floating-Point to Packed Doubleword Integers)
MNEM (cvtsd2si,          CVTSD2SI,          Convert Scalar Double-Precision Floating-Point to Signed Doubleword or Quadword Integer)
MNEM (cvtsd2ss,          CVTSD2SS,          Convert Scalar Double-Precision Floating-Point to Scalar Single-Precision Floating-Point)
MNEM (cvtsi2sd,          CVTSI2SD,          Convert Signed Doubleword or Quadword Integer to Scalar Double-Precision Floating-Point)
MNEM (cvtsi2ss,          CVTSI2SS,          Convert Signed Doubleword or Quadword Integer to Scalar Single-Precision Floating-Point)
MNEM (cvtss2sd,          CVTSS2SD,          Convert Scalar Single-Precision Floating-Point to Scalar Double-Precision Floating-Point)
MNEM (cvtss2si,          CVTSS2SI,          Convert Scalar Single-Precision Floating-Point to Signed Doubleword or Quadword Integer)
MNEM (cvttpd2dq,         CVTTPD2DQ,         Convert Packed Double-Precision Floating-Point to Packed Doubleword Integers Truncated)
MNEM (cvttpd2pi,         CVTTPD2PI,         Convert Packed Double-Precision Floating-Point to Packed Doubleword Integers Truncated)
MNEM (cvttps2dq,         CVTTPS2DQ,         Convert Packed Single-Precision Floating-Point to Packed Doubleword Integers Truncated)
MNEM (cvttps2pi,         CVTTPS2PI,         Convert Packed Single-Precision Floating-Point to Packed Doubleword Integers Truncated)
MNEM (cvttsd2si,         CVTTSD2SI,         Convert Scalar Double-Precision Floating-Point to Signed Doubleword of Quadword Integer Truncated)
MNEM (cvttss2si,         CVTTSS2SI,         Convert Scalar Single-Precision Floating-Point to Signed Doubleword or Quadword Integer Truncated)
MNEM (cwd,               CWD,               Convert AX to Sign-Extended DX:AX)
MNEM (cwde,              CWDE,              Convert AX to Sign-Extended EAX)
MNEM (daa,               DAA,               Decimal Adjust after Addition)
MNEM (das,               DAS,               Decimal Adjust after Subtraction)
MNEM (dec,               DEC,               Decrement by 1)
MNEM (div,               DIV,               Unsigned Divide)
MNEM (divpd,             DIVPD,             Divide Packed Double-Precision Floating-Point)
MNEM (divps,             DIVPS,             Divide Packed Single-Precision Floating-Point)
MNEM (divsd,             DIVSD,             Divide Scalar Double-Precision Floating-Point)
MNEM (divss,             DIVSS,             Divide Scalar Single-Precision Floating-Point)
MNEM (dppd,              DPPD,              Dot Product Packed Double-Precision Floating-Point)
MNEM (dpps,              DPPS,              Dot Product Packed Single-Precision Floating-Point)
MNEM (emms,              EMMS,              Exit Multimedia State)
MNEM (enter,             ENTER,             Create Procedure Stack Frame)
MNEM (extractps,         EXTRACTPS,         Extract Packed Single-Precision Floating-Point)
MNEM (extrq,             EXTRQ,             Extract Field From Register)
MNEM (f2xm1,             F2XM1,             Floating-Point Compute 2 to the Power of x-1)
MNEM (fabs,              FABS,              Floating-Point Absolute Value)
MNEM (fadd,              FADD,              Floating-Point Add)
MNEM (faddp,             FADDP,             Floating-Point Add and Pop)
MNEM (fbld,              FBLD,              Floating-Point Load Binary-Coded Decimal)
MNEM (fbstp,             FBSTP,             Floating-Point Store Binary-Coded Decimal and Pop)
MNEM (fchs,              FCHS,              Floating-Point Change Sign)
MNEM (fcmovb,            FCMOVB,            Floating-Point Move if below (CF = 1))
MNEM (fcmovbe,           FCMOVBE,           Floating-Point Move if below or equal (CF = 1 or ZF = 1))
MNEM (fcmove,            FCMOVE,            Floating-Point Move if equal (ZF = 1))
MNEM (fcmovnb,           FCMOVNB,           Floating-Point Move if not below (CF = 0))
MNEM (fcmovnbe,          FCMOVNBE,          Floating-Point Move if not below or equal (CF = 0 and ZF = 0))
MNEM (fcmovne,           FCMOVNE,           Floating-Point Move if not equal (ZF = 0))
MNEM (fcmovnu,           FCMOVNU,           Floating-Point Move if not unordered (PF = 0))
MNEM (fcmovu,            FCMOVU,            Floating-Point Move if unordered (PF = 1))
MNEM (fcom,              FCOM,              Floating-Point Compare)
MNEM (fcomi,             FCOMI,             Floating-Point Compare and Set Flags)
MNEM (fcomip,            FCOMIP,            Floating-Point Compare and Set Flags and Pop)
MNEM (fcomp,             FCOMP,             Floating-Point Compare and Pop)
MNEM (fcompp,            FCOMPP,            Floating-Point Compare and Pop twice)
MNEM (fcos,              FCOS,              Floating-Point Cosine)
MNEM (fdecstp,           FDECSTP,           Floating-Point Decrement Stack-Top Pointer)
MNEM (fdiv,              FDIV,              Floating-Point Divide)
MNEM (fdivp,             FDIVP,             Floating-Point Divide and Pop)
MNEM (fdivr,             FDIVR,             Floating-Point Divide Reverse)
MNEM (fdivrp,            FDIVRP,            Floating-Point Divide Reverse and Pop)
MNEM (femms,             FEMMS,             Fast Exit Multimedia State)
MNEM (ffree,             FFREE,             Floating-Point Free Register)
MNEM (fiadd,             FIADD,             Floating-Point Integer Add)
MNEM (ficom,             FICOM,             Floating-Point Integer Compare)
MNEM (ficomp,            FICOMP,            Floating-Point Integer Compare and Pop)
MNEM (fidiv,             FIDIV,             Floating-Point Integer Divide)
MNEM (fidivr,            FIDIVR,            Floating-Point Integer Divide Reverse)
MNEM (fild,              FILD,              Floating-Point Load Integer)
MNEM (fimul,             FIMUL,             Floating-Point Integer Multiply)
MNEM (fincstp,           FINCSTP,           Floating-Point Increment Stack-Top Pointer)
MNEM (fist,              FIST,              Floating-Point Integer Store)
MNEM (fistp,             FISTP,             Floating-Point Integer Store and Pop)
MNEM (fisttp,            FISTTP,            Floating-Point Integer Truncate and Store)
MNEM (fisub,             FISUB,             Floating-Point Integer Subtract)
MNEM (fisubr,            FISUBR,            Floating-Point Integer Subtract Reverse)
MNEM (fld,               FLD,               Floating-Point Load)
MNEM (fld1,              FLD1,              Floating-Point Load +1.0)
MNEM (fldcw,             FLDCW,             Floating-Point Load x87 Control Word)
MNEM (fldenv,            FLDENV,            Floating-Point Load x87 Environment)
MNEM (fldl2e,            FLDL2E,            Floating-Point Load Log2(e))
MNEM (fldl2t,            FLDL2T,            Floating-Point Load Log2(10))
MNEM (fldlg2,            FLDLG2,            Floating-Point Load Log10(2))
MNEM (fldln2,            FLDLN2,            Floating-Point Load Ln(2))
MNEM (fldpi,             FLDPI,             Floating-Point Load Pi)
MNEM (fldz,              FLDZ,              Floating-Point Load +0.0)
MNEM (fmul,              FMUL,              Floating-Point Multiply)
MNEM (fmulp,             FMULP,             Floating-Point Multiply and Pop)
MNEM (fnclex,            FNCLEX,            Floating-Point Clear Flags)
MNEM (fninit,            FNINIT,            Floating-Point Initialize)
MNEM (fnop,              FNOP,              Floating-Point No Operation)
MNEM (fnsave,            FNSAVE,            Floating-Point Save x87 and MMX State)
MNEM (fnstcw,            FNSTCW,            Floating-Point Store Control Word)
MNEM (fnstenv,           FNSTENV,           Floating-Point Store Environment)
MNEM (fnstsw,            FNSTSW,            Floating-Point Store Status Word)
MNEM (fpatan,            FPATAN,            Floating-Point Partial Arctangent)
MNEM (fprem,             FPREM,             Floating-Point Partial Remainder)
MNEM (fprem1,            FPREM1,            Floating-Point Partial Remainder)
MNEM (fptan,             FPTAN,             Floating-Point Partial Tangent)
MNEM (frndint,           FRNDINT,           Floating-Point Round to Integer)
MNEM (frstor,            FRSTOR,            Floating-Point Restore x87 and MMX State)
MNEM (fscale,            FSCALE,            Floating-Point Scale)
MNEM (fsin,              FSIN,              Floating-Point Sine)
MNEM (fsincos,           FSINCOS,           Floating-Point Sine and Cosine)
MNEM (fsqrt,             FSQRT,             Floating-Point Square Root)
MNEM (fst,               FST,               Floating-Point Store Stack Top)
MNEM (fstp,              FSTP,              Floating-Point Store Stack Top and Pop)
MNEM (fsub,              FSUB,              Floating-Point Subtract)
MNEM (fsubp,             FSUBP,             Floating-Point Subtract and Pop)
MNEM (fsubr,             FSUBR,             Floating-Point Subtract Reverse)
MNEM (fsubrp,            FSUBRP,            Floating-Point Subtract Reverse and Pop)
MNEM (ftst,              FTST,              Floating-Point Test with Zero)
MNEM (fucom,             FUCOM,             Floating-Point Unordered Compare)
MNEM (fucomi,            FUCOMI,            Floating-Point Unordered Compare and Set Flags)
MNEM (fucomip,           FUCOMIP,           Floating-Point Unordered Compare and Set Flags and Pop)
MNEM (fucomp,            FUCOMP,            Floating-Point Unordered Compare and Pop)
MNEM (fucompp,           FUCOMPP,           Floating-Point Unordered Compare and Pop twice)
MNEM (fwait,             FWAIT,             Wait for Unmasked x87 Floating-Point Exceptions)
MNEM (fxam,              FXAM,              Floating-Point Examine)
MNEM (fxch,              FXCH,              Floating-Point Exchange)
MNEM (fxrstor,           FXRSTOR,           Restore XMM MMX and x87 State)
MNEM (fxsave,            FXSAVE,            Save XMM MMX and x87 State)
MNEM (fxtract,           FXTRACT,           Floating-Point Extract Exponent and Significand)
MNEM (fyl2x,             FYL2X,             Floating-Point y*Log2(x))
MNEM (fyl2xp1,           FYL2XP1,           Floating-Point y*Log2(x+1))
MNEM (haddpd,            HADDPD,            Horizontal Add Packed Double)
MNEM (haddps,            HADDPS,            Horizontal Add Packed Single)
MNEM (hlt,               HLT,               Halt)
MNEM (hsubpd,            HSUBPD,            Horizontal Sub Packed Double)
MNEM (hsubps,            HSUBPS,            Horizontal Sub Packed Single)
MNEM (idiv,              IDIV,              Signed Divide)
MNEM (imul,              IMUL,              Signed Multiply)
MNEM (in,                IN,                Input from Port)
MNEM (inc,               INC,               Increment by 1)
MNEM (incssp,            INCSSP,            Increment Shadow Stack Pointer)
MNEM (insb,              INSB,              Input Bytes)
MNEM (insd,              INSD,              Input Doublewords)
MNEM (insertps,          INSERTPS,          Insert Packed Single-Precision Floating-Point)
MNEM (insertq,           INSERTQ,           Insert Field)
MNEM (insw,              INSW,              Input Words)
MNEM (int,               INT,               Interrupt to Vector)
MNEM (int3,              INT3,              Interrupt to Debug Vector)
MNEM (into,              INTO,              Interrupt to Overflow Vector)
MNEM (invd,              INVD,              Invalidate Caches)
MNEM (invlpg,            INVLPG,            Invalidate TLB Entry)
MNEM (invlpga,           INVLPGA,           Invalidate TLB Entry in a Specified ASID)
MNEM (invlpgb,           INVLPGB,           Invalidate TLB Entry with Broadcast)
MNEM (invpcid,           INVPCID,           Invalidate TLB Entry in a Specified PCID)
MNEM (iret,              IRET,              Return from Interrupt)
MNEM (iretd,             IRETD,             Return from Interrupt)
MNEM (iretq,             IRETQ,             Return from Interrupt)
MNEM (ja,                JA,                Jump if above (CF = 0 and ZF = 0))
MNEM (jae,               JAE,               Jump if above or equal (CF = 0))
MNEM (jb,                JB,                Jump if below (CF = 1))
MNEM (jbe,               JBE,               Jump if below or equal (CF = 1 or ZF = 1))
MNEM (jc,                JC,                Jump if carry (CF = 1))
MNEM (jcxz,              JCXZ,              Jump if CX Zero)
MNEM (je,                JE,                Jump if equal (ZF =1))
MNEM (jecxz,             JECXZ,             Jump if ECX Zero)
MNEM (jg,                JG,                Jump if greater (ZF = 0 and SF = OF))
MNEM (jge,               JGE,               Jump if greater or equal (SF = OF))
MNEM (jl,                JL,                Jump if less (SF <> OF))
MNEM (jle,               JLE,               Jump if less or equal (ZF = 1 or SF <> OF))
MNEM (jmp,               JMP,               Near Jump)
MNEM (jmpfar,            JMPFAR,            Far Jump)
MNEM (jna,               JNA,               Jump if not above (CF = 1 or ZF = 1))
MNEM (jnae,              JNAE,              Jump if not above or equal (CF = 1))
MNEM (jnb,               JNB,               Jump if not below (CF = 0))
MNEM (jnbe,              JNBE,              Jump if not below or equal (CF = 0 and ZF = 0))
MNEM (jnc,               JNC,               Jump if not carry (CF = 0))
MNEM (jne,               JNE,               Jump if not equal (ZF = 0))
MNEM (jng,               JNG,               Jump if not greater (ZF = 1 or SF <> OF))
MNEM (jnge,              JNGE,              Jump if not greater or equal (SF <> OF))
MNEM (jnl,               JNL,               Jump if not less (SF = OF))
MNEM (jnle,              JNLE,              Jump if not less or equal (ZF = 0 and SF = OF))
MNEM (jno,               JNO,               Jump if not overflow (OF = 0))
MNEM (jnp,               JNP,               Jump if not parity (PF = 0))
MNEM (jns,               JNS,               Jump if not sign (SF = 0))
MNEM (jnz,               JNZ,               Jump if not zero (ZF = 0))
MNEM (jo,                JO,                Jump if overflow (OF = 1))
MNEM (jp,                JP,                Jump if parity (PF = 1))
MNEM (jpe,               JPE,               Jump if parity even (PF = 1))
MNEM (jpo,               JPO,               Jump if parity odd (PF = 0))
MNEM (jrcxz,             JRCXZ,             Jump if RCX Zero)
MNEM (js,                JS,                Jump if sign (SF =1))
MNEM (jz,                JZ,                Jump if zero (ZF = 1))
MNEM (lahf,              LAHF,              Load Status Flags into AH Register)
MNEM (lar,               LAR,               Load Access Rights Byte)
MNEM (lddqu,             LDDQU,             Load Unaligned Double Quadword)
MNEM (ldmxcsr,           LDMXCSR,           Load MXCSR Control/Status Register)
MNEM (lds,               LDS,               Load Far Pointer)
MNEM (lea,               LEA,               Load Effective Address)
MNEM (leave,             LEAVE,             Delete Procedure Stack Frame)
MNEM (les,               LES,               Load Far Pointer)
MNEM (lfence,            LFENCE,            Load Fence)
MNEM (lfs,               LFS,               Load Far Pointer)
MNEM (lgdt,              LGDT,              Load Global Descriptor Table Register)
MNEM (lgs,               LGS,               Load Far Pointer)
MNEM (lidt,              LIDT,              Load Interrupt Descriptor Table Register)
MNEM (lldt,              LLDT,              Load Local Descriptor Table Register)
MNEM (llwpcb,            LLWPCB,            Load Lightweight Profiling Control Block Address)
MNEM (lmsw,              LMSW,              Load Machine Status Word)
MNEM (lodsb,             LODSB,             Load Bytes)
MNEM (lodsd,             LODSD,             Load Doublewords)
MNEM (lodsq,             LODSQ,             Load Quadwords)
MNEM (lodsw,             LODSW,             Load Words)
MNEM (loop,              LOOP,              Loop if rCX is not 0)
MNEM (loope,             LOOPE,             Loop if rCX is not 0 and ZF is 1)
MNEM (loopne,            LOOPNE,            Loop if rCX is not 0 and ZF is 0)
MNEM (loopnz,            LOOPNZ,            Loop if rCX is not 0 and ZF is 0)
MNEM (loopz,             LOOPZ,             Loop if rCX is not 0 and ZF is 1)
MNEM (lsl,               LSL,               Load Segment Limit)
MNEM (lss,               LSS,               Load Far Pointer)
MNEM (ltr,               LTR,               Load Task Register)
MNEM (lwpins,            LWPINS,            Lightweight Profiling Insert Record)
MNEM (lwpval,            LWPVAL,            Lightweight Profiling Insert Value)
MNEM (lzcnt,             LZCNT,             Count Leading Zeros)
MNEM (maskmovdqu,        MASKMOVDQU,        Masked Move Double Quadword Unaligned)
MNEM (maskmovq,          MASKMOVQ,          Masked Move Quadword)
MNEM (maxpd,             MAXPD,             Maximum Packed Double-Precision Floating-Point)
MNEM (maxps,             MAXPS,             Maximum Packed Single-Precision Floating-Point)
MNEM (maxsd,             MAXSD,             Maximum Scalar Double-Precision Floating-Point)
MNEM (maxss,             MAXSS,             Maximum Scalar Single-Precision Floating-Point)
MNEM (mcommit,           MCOMMIT,           Commit Stores to Memory)
MNEM (mfence,            MFENCE,            Memory Fence)
MNEM (minpd,             MINPD,             Minimum Packed Double-Precision Floating-Point)
MNEM (minps,             MINPS,             Minimum Packed Single-Precision Floating-Point)
MNEM (minsd,             MINSD,             Minimum Scalar Double-Precision Floating-Point)
MNEM (minss,             MINSS,             Minimum Scalar Single-Precision Floating-Point)
MNEM (monitor,           MONITOR,           Setup Monitor Address)
MNEM (monitorx,          MONITORX,          Setup Monitor Address)
MNEM (mov,               MOV,               Move)
MNEM (movapd,            MOVAPD,            Move Aligned Packed Double-Precision Floating-Point)
MNEM (movaps,            MOVAPS,            Move Aligned Packed Single-Precision Floating-Point)
MNEM (movbe,             MOVBE,             Move Big Endian)
MNEM (movd,              MOVD,              Move Doubleword or Quadword)
MNEM (movddup,           MOVDDUP,           Move Double-Precision and Duplicate)
MNEM (movdq2q,           MOVDQ2Q,           Move Quadword to Quadword)
MNEM (movdqa,            MOVDQA,            Move Aligned Double Quadword)
MNEM (movdqu,            MOVDQU,            Move Unaligned Double Quadword)
MNEM (movhlps,           MOVHLPS,           Move Packed Single-Precision Floating-Point High to Low)
MNEM (movhpd,            MOVHPD,            Move High Packed Double-Precision Floating-Point)
MNEM (movhps,            MOVHPS,            Move High Packed Single-Precision Floating-Point)
MNEM (movlhps,           MOVLHPS,           Move Packed Single-Precision Floating-Point Low to High)
MNEM (movlpd,            MOVLPD,            Move Low Packed Double-Precision Floating-Point)
MNEM (movlps,            MOVLPS,            Move Low Packed Single-Precision Floating-Point)
MNEM (movmskpd,          MOVMSKPD,          Extract Packed Double-Precision Floating-Point Sign Mask)
MNEM (movmskps,          MOVMSKPS,          Extract Packed Single-Precision Floating-Point Sign Mask)
MNEM (movntdq,           MOVNTDQ,           Move Non-Temporal Double Quadword)
MNEM (movntdqa,          MOVNTDQA,          Move Non-Temporal Double Quadword Aligned)
MNEM (movnti,            MOVNTI,            Move Non-Temporal Doubleword or Quadword)
MNEM (movntpd,           MOVNTPD,           Move Non-Temporal Packed Double-Precision Floating-Point)
MNEM (movntps,           MOVNTPS,           Move Non-Temporal Packed Single-Precision Floating-Point)
MNEM (movntq,            MOVNTQ,            Move Non-Temporal Double Quadword)
MNEM (movntsd,           MOVNTSD,           Move Non-Temporal Scalar Double-Precision Floating-Point)
MNEM (movntss,           MOVNTSS,           Move Non-Temporal Scalar Single-Precision Floating-Point)
MNEM (movq,              MOVQ,              Move Quadword)
MNEM (movq2dq,           MOVQ2DQ,           Move Quadword to Quadword)
MNEM (movsb,             MOVSB,             Move Bytes)
MNEM (movsd,             MOVSD,             Move Doublewords)
MNEM (movshdup,          MOVSHDUP,          Move Single-Precision High and Duplicate)
MNEM (movsldup,          MOVSLDUP,          Move Single-Precision Low and Duplicate)
MNEM (movsq,             MOVSQ,             Move Quadwords)
MNEM (movss,             MOVSS,             Move Scalar Single-Precision Floating-Point)
MNEM (movsw,             MOVSW,             Move Words)
MNEM (movsx,             MOVSX,             Move with Sign-Extension)
MNEM (movsxd,            MOVSXD,            Move with Sign-Extend Doubleword)
MNEM (movupd,            MOVUPD,            Move Unaligned Packed Double-Precision Floating-Point)
MNEM (movups,            MOVUPS,            Move Unaligned Packed Single-Precision Floating-Point)
MNEM (movzx,             MOVZX,             Move with Zero-Extension)
MNEM (mpsadbw,           MPSADBW,           Multiple Sum of Absolute Differences)
MNEM (mul,               MUL,               Unsigned Multiply)
MNEM (mulpd,             MULPD,             Multiply Packed Double-Precision Floating-Point)
MNEM (mulps,             MULPS,             Multiply Packed Single-Precision Floating-Point)
MNEM (mulsd,             MULSD,             Multiply Scalar Double-Precision Floating-Point)
MNEM (mulss,             MULSS,             Multiply Scalar Single-Precision Floating-Point)
MNEM (mulx,              MULX,              Multiply Unsigned)
MNEM (mwait,             MWAIT,             Monitor Wait)
MNEM (mwaitx,            MWAITX,            Monitor Wait with Timeout)
MNEM (neg,               NEG,               Two`s Complement Negation)
MNEM (nop,               NOP,               No Operation)
MNEM (not,               NOT,               One`s Complement Negation)
MNEM (or,                OR,                Logical OR)
MNEM (orpd,              ORPD,              Logical Bitwise OR Packed Double-Precision Floating-Point)
MNEM (orps,              ORPS,              Logical Bitwise OR Packed Single-Precision Floating-Point)
MNEM (out,               OUT,               Output to Port)
MNEM (outsb,             OUTSB,             Output Bytes)
MNEM (outsd,             OUTSD,             Output Doublewords)
MNEM (outsw,             OUTSW,             Output Words)
MNEM (pabsb,             PABSB,             Packed Absolute Value Signed Byte)
MNEM (pabsd,             PABSD,             Packed Absolute Value Signed Doubleword)
MNEM (pabsw,             PABSW,             Packed Absolute Value Signed Word)
MNEM (packssdw,          PACKSSDW,          Pack with Saturation Signed Doubleword to Word)
MNEM (packsswb,          PACKSSWB,          Pack with Saturation Signed Word to Byte)
MNEM (packusdw,          PACKUSDW,          Pack with Saturation Signed Word to Unsigned Byte)
MNEM (packuswb,          PACKUSWB,          Pack with Saturation Signed Word to Unsigned Byte)
MNEM (paddb,             PADDB,             Packed Add Bytes)
MNEM (paddd,             PADDD,             Packed Add Doublewords)
MNEM (paddq,             PADDQ,             Packed Add Quadwords)
MNEM (paddsb,            PADDSB,            Packed Add Signed with Saturation Bytes)
MNEM (paddsw,            PADDSW,            Packed Add Signed with Saturation Words)
MNEM (paddusb,           PADDUSB,           Packed Add Unsigned with Saturation Bytes)
MNEM (paddusw,           PADDUSW,           Packed Add Unsigned with Saturation Words)
MNEM (paddw,             PADDW,             Packed Add Words)
MNEM (palignr,           PALIGNR,           Packed Align Right)
MNEM (pand,              PAND,              Packed Logical Bitwise AND)
MNEM (pandn,             PANDN,             Packed Logical Bitwise AND NOT)
MNEM (pause,             PAUSE,             Pause)
MNEM (pavgb,             PAVGB,             Packed Average Unsigned Bytes)
MNEM (pavgusb,           PAVGUSB,           Packed Average Unsigned Bytes)
MNEM (pavgw,             PAVGW,             Packed Average Unsigned Words)
MNEM (pblendvb,          PBLENDVB,          Variable Blend Packed Bytes)
MNEM (pblendw,           PBLENDW,           Blend Packed Words)
MNEM (pclmulqdq,         PCLMULQDQ,         Carry-less Multiply Quadwords)
MNEM (pcmpeqb,           PCMPEQB,           Packed Compare Equal Bytes)
MNEM (pcmpeqd,           PCMPEQD,           Packed Compare Equal Doublewords)
MNEM (pcmpeqq,           PCMPEQQ,           Packed Compare Equal Quadwords)
MNEM (pcmpeqw,           PCMPEQW,           Packed Compare Equal Words)
MNEM (pcmpestri,         PCMPESTRI,         Packed Compare Explicit Length Strings Return Index)
MNEM (pcmpestrm,         PCMPESTRM,         Packed Compare Explicit Length Strings Return Mask)
MNEM (pcmpgtb,           PCMPGTB,           Packed Compare Greater Than Signed Bytes)
MNEM (pcmpgtd,           PCMPGTD,           Packed Compare Greater Than Signed Doublewords)
MNEM (pcmpgtq,           PCMPGTQ,           Packed Compare Greater Than Signed Quadwords)
MNEM (pcmpgtw,           PCMPGTW,           Packed Compare Greater Than Signed Words)
MNEM (pcmpistri,         PCMPISTRI,         Packed Compare Implicit Length Strings Return Index)
MNEM (pcmpistrm,         PCMPISTRM,         Packed Compare Implicit Length Strings Return Mask)
MNEM (pdep,              PDEP,              Parallel Deposit Bits)
MNEM (pext,              PEXT,              Parallel Extract Bits)
MNEM (pextrb,            PEXTRB,            Extract Packed Byte)
MNEM (pextrd,            PEXTRD,            Extract Packed Doubleword)
MNEM (pextrq,            PEXTRQ,            Extract Packed Quadword)
MNEM (pextrw,            PEXTRW,            Extract Packed Word)
MNEM (pf2id,             PF2ID,             Packed Floating-Point to Integer Doubleword Conversion)
MNEM (pf2iw,             PF2IW,             Packed Floating-Point to Integer Word Conversion)
MNEM (pfacc,             PFACC,             Packed Floating-Point Accumulate)
MNEM (pfadd,             PFADD,             Packed Floating-Point Add)
MNEM (pfcmpeq,           PFCMPEQ,           Packed Floating-Point Compare Equal)
MNEM (pfcmpge,           PFCMPGE,           Packed Floating-Point Compare Greater or Equal)
MNEM (pfcmpgt,           PFCMPGT,           Packed Floating-Point Compare Greater Than)
MNEM (pfmax,             PFMAX,             Packed Single-Precision Floating-Point Maximum)
MNEM (pfmin,             PFMIN,             Packed Single-Precision Floating-Point Minimum)
MNEM (pfmul,             PFMUL,             Packed Floating-Point Multiply)
MNEM (pfnacc,            PFNACC,            Packed Floating-Point Negative Accumulate)
MNEM (pfpnacc,           PFPNACC,           Packed Floating-Point Positive-Negative Accumulate)
MNEM (pfrcp,             PFRCP,             Packed Floating-Point Positive-Negative Accumulate)
MNEM (pfrcpit1,          PFRCPIT1,          Packed Floating-Point Reciprocal Iteration 1)
MNEM (pfrcpit2,          PFRCPIT2,          Packed Floating-Point Reciprocal Iteration 2)
MNEM (pfrsqit1,          PFRSQIT1,          Packed Floating-Point Reciprocal Square Root Iteration 1)
MNEM (pfrsqrt,           PFRSQRT,           Packed Floating-Point Reciprocal Square Root Approximation)
MNEM (pfsub,             PFSUB,             Packed Floating-Point Subtract)
MNEM (pfsubr,            PFSUBR,            Packed Floating-Point Subtract Reverse)
MNEM (phaddd,            PHADDD,            Packed Horizontal Add Doubleword)
MNEM (phaddsw,           PHADDSW,           Packed Horizontal Add with Saturation Word)
MNEM (phaddw,            PHADDW,            Packed Horizontal Add Word)
MNEM (phminposuw,        PHMINPOSUW,        Horizontal Minimum and Position)
MNEM (phsubd,            PHSUBD,            Packed Horizontal Subtract Doubleword)
MNEM (phsubsw,           PHSUBSW,           Packed Horizontal Subtract with Saturation Word)
MNEM (phsubw,            PHSUBW,            Packed Horizontal Subtract Word)
MNEM (pi2fd,             PI2FD,             Packed Integer to Floating-Point Doubleword Conversion)
MNEM (pi2fw,             PI2FW,             Packed Integer to Floating-Point Word Conversion)
MNEM (pinsrb,            PINSRB,            Packed Insert Byte)
MNEM (pinsrd,            PINSRD,            Packed Insert Doubleword)
MNEM (pinsrq,            PINSRQ,            Packed Insert Quadword)
MNEM (pinsrw,            PINSRW,            Packed Insert Word)
MNEM (pmaddubsw,         PMADDUBSW,         Packed Multiply and Add Unsigned Byte to Signed Word)
MNEM (pmaddwd,           PMADDWD,           Packed Multiply Words and Add Doublewords)
MNEM (pmaxsb,            PMAXSB,            Packed Maximum Signed Bytes)
MNEM (pmaxsd,            PMAXSD,            Packed Maximum Signed Doublewords)
MNEM (pmaxsw,            PMAXSW,            Packed Maximum Signed Words)
MNEM (pmaxub,            PMAXUB,            Packed Maximum Unsigned Bytes)
MNEM (pmaxud,            PMAXUD,            Packed Maximum Unsigned Doublewords)
MNEM (pmaxuw,            PMAXUW,            Packed Maximum Unsigned Words)
MNEM (pminsb,            PMINSB,            Packed Minimum Signed Bytes)
MNEM (pminsd,            PMINSD,            Packed Minimum Signed Doublewords)
MNEM (pminsw,            PMINSW,            Packed Minimum Signed Words)
MNEM (pminub,            PMINUB,            Packed Minimum Unsigned Bytes)
MNEM (pminud,            PMINUD,            Packed Minimum Unsigned Doublewords)
MNEM (pminuw,            PMINUW,            Packed Minimum Unsigned Words)
MNEM (pmovmskb,          PMOVMSKB,          Packed Move Mask Byte)
MNEM (pmovsxbd,          PMOVSXBD,          Packed Move with Sign-Extension Byte to Doubleword)
MNEM (pmovsxbq,          PMOVSXBQ,          Packed Move with Sign-Extension Byte to Quadword)
MNEM (pmovsxbw,          PMOVSXBW,          Packed Move with Sign-Extension Byte to Word)
MNEM (pmovsxdq,          PMOVSXDQ,          Packed Move with Sign-Extension Doubleword to Quadword)
MNEM (pmovsxwd,          PMOVSXWD,          Packed Move with Sign-Extension Word to Doubleword)
MNEM (pmovsxwq,          PMOVSXWQ,          Packed Move with Sign-Extension Word to Quadword)
MNEM (pmovzxbd,          PMOVZXBD,          Packed Move with Zero-Extension Byte to Doubleword)
MNEM (pmovzxbq,          PMOVZXBQ,          Packed Move with Zero-Extension Byte to Quadword)
MNEM (pmovzxbw,          PMOVZXBW,          Packed Move with Zero-Extension Byte to Word)
MNEM (pmovzxdq,          PMOVZXDQ,          Packed Move with Zero-Extension Doubleword to Quadword)
MNEM (pmovzxwd,          PMOVZXWD,          Packed Move with Zero-Extension Word to Doubleword)
MNEM (pmovzxwq,          PMOVZXWQ,          Packed Move with Zero-Extension Word to Quadword)
MNEM (pmuldq,            PMULDQ,            Packed Multiply Signed Doubleword to Quadword)
MNEM (pmulhrsw,          PMULHRSW,          Packed Multiply High with Round and Scale Words)
MNEM (pmulhrw,           PMULHRW,           Packed Multiply High Rounded Word)
MNEM (pmulhuw,           PMULHUW,           Packed Multiply High Unsigned Word)
MNEM (pmulhw,            PMULHW,            Packed Multiply High Signed Word)
MNEM (pmulld,            PMULLD,            Packed Multiply Low Signed Doubleword)
MNEM (pmullw,            PMULLW,            Packed Multiply Low Signed Word)
MNEM (pmuludq,           PMULUDQ,           Packed Multiply Unsigned Doubleword and Store Quadword)
MNEM (pop,               POP,               Pop Stack)
MNEM (popa,              POPA,              POP All GPRs)
MNEM (popad,             POPAD,             POP All GPRs)
MNEM (popcnt,            POPCNT,            Bit Population Count)
MNEM (popf,              POPF,              POP to FLAGS)
MNEM (popfd,             POPFD,             POP to EFLAGS)
MNEM (popfq,             POPFQ,             POP to RFLAGS)
MNEM (por,               POR,               Packed Logical Bitwise OR)
MNEM (prefetch,          PREFETCH,          Prefetch L1 Data-Cache Line)
MNEM (prefetchnta,       PREFETCHNTA,       Prefetch Data to Cache Level NTA)
MNEM (prefetcht0,        PREFETCHT0,        Prefetch Data to Cache Level T0)
MNEM (prefetcht1,        PREFETCHT1,        Prefetch Data to Cache Level T1)
MNEM (prefetcht2,        PREFETCHT2,        Prefetch Data to Cache Level T2)
MNEM (prefetchw,         PREFETCHW,         Prefetch L1 Data-Cache Line)
MNEM (psadbw,            PSADBW,            Packed Sum of Absolute Differences of Bytes Into a Word)
MNEM (pshufb,            PSHUFB,            Packed Shuffle Byte)
MNEM (pshufd,            PSHUFD,            Packed Shuffle Doublewords)
MNEM (pshufhw,           PSHUFHW,           Packed Shuffle High Words)
MNEM (pshuflw,           PSHUFLW,           Packed Shuffle Low Words)
MNEM (pshufw,            PSHUFW,            Packed Shuffle Words)
MNEM (psignb,            PSIGNB,            Packed Sign Byte)
MNEM (psignd,            PSIGND,            Packed Sign Doubleword)
MNEM (psignw,            PSIGNW,            Packed Sign Word)
MNEM (pslld,             PSLLD,             Packed Shift Left Logical Doublewords)
MNEM (pslldq,            PSLLDQ,            Packed Shift Left Logical Double Quadword)
MNEM (psllq,             PSLLQ,             Packed Shift Left Logical Quadwords)
MNEM (psllw,             PSLLW,             Packed Shift Left Logical Words)
MNEM (psmash,            PSMASH,            Page Smash)
MNEM (psrad,             PSRAD,             Packed Shift Right Arithmetic Doublewords)
MNEM (psraw,             PSRAW,             Packed Shift Right Arithmetic Words)
MNEM (psrld,             PSRLD,             Packed Shift Right Logical Doublewords)
MNEM (psrldq,            PSRLDQ,            Packed Shift Right Logical Double Quadword)
MNEM (psrlq,             PSRLQ,             Packed Shift Right Logical Quadwords)
MNEM (psrlw,             PSRLW,             Packed Shift Right Logical Words)
MNEM (psubb,             PSUBB,             Packed Subtract Bytes)
MNEM (psubd,             PSUBD,             Packed Subtract Doublewords)
MNEM (psubq,             PSUBQ,             Packed Subtract Quadwords)
MNEM (psubsb,            PSUBSB,            Packed Subtract Signed with Saturation Bytes)
MNEM (psubsw,            PSUBSW,            Packed Subtract Signed with Saturation Words)
MNEM (psubusb,           PSUBUSB,           Packed Subtract Unsigned with Saturation Bytes)
MNEM (psubusw,           PSUBUSW,           Packed Subtract Unsigned with Saturation Words)
MNEM (psubw,             PSUBW,             Packed Subtract Words)
MNEM (pswapd,            PSWAPD,            Packed Swap Doubleword)
MNEM (ptest,             PTEST,             Packed Bit Test)
MNEM (punpckhbw,         PUNPCKHBW,         Unpack and Interleave High Bytes)
MNEM (punpckhdq,         PUNPCKHDQ,         Unpack and Interleave High Doublewords)
MNEM (punpckhqdq,        PUNPCKHQDQ,        Unpack and Interleave High Quadwords)
MNEM (punpckhwd,         PUNPCKHWD,         Unpack and Interleave High Words)
MNEM (punpcklbw,         PUNPCKLBW,         Unpack and Interleave Low Bytes)
MNEM (punpckldq,         PUNPCKLDQ,         Unpack and Interleave Low Doublewords)
MNEM (punpcklqdq,        PUNPCKLQDQ,        Unpack and Interleave Low Quadwords)
MNEM (punpcklwd,         PUNPCKLWD,         Unpack and Interleave Low Words)
MNEM (push,              PUSH,              Push onto Stack)
MNEM (pusha,             PUSHA,             Push All GPRs onto Stack)
MNEM (pushad,            PUSHAD,            Push All GPRs onto Stack)
MNEM (pushf,             PUSHF,             Push FLAGS onto Stack)
MNEM (pushfd,            PUSHFD,            Push EFLAGS onto Stack)
MNEM (pushfq,            PUSHFQ,            Push RFLAGS onto Stack)
MNEM (pvalidate,         PVALIDATE,         Page Validate)
MNEM (pxor,              PXOR,              Packed Logical Bitwise Exclusive OR)
MNEM (rcl,               RCL,               Rotate Through Carry Left)
MNEM (rcpps,             RCPPS,             Reciprocal Packed Single-Precision Floating-Point)
MNEM (rcpss,             RCPSS,             Reciprocal Scalar Single-Precision Floating-Point)
MNEM (rcr,               RCR,               Rotate Through Carry Right)
MNEM (rdfsbase,          RDFSBASE,          Read FS.base)
MNEM (rdgsbase,          RDGSBASE,          Read GS.base)
MNEM (rdmsr,             RDMSR,             Read Model-Specific Register)
MNEM (rdpid,             RDPID,             Read Processor ID)
MNEM (rdpkru,            RDPKRU,            Read Protection Key Rights)
MNEM (rdpmc,             RDPMC,             Read Performance-Monitoring Counter)
MNEM (rdpru,             RDPRU,             Read Processor Register)
MNEM (rdrand,            RDRAND,            Read Random)
MNEM (rdseed,            RDSEED,            Read Random Seed)
MNEM (rdsspd,            RDSSPD,            Read Shawdow Stack Pointer)
MNEM (rdsspq,            RDSSPQ,            Read Shawdow Stack Pointer)
MNEM (rdtsc,             RDTSC,             Read Time-Stamp Counter)
MNEM (rdtscp,            RDTSCP,            Read Time-Stamp Counter and Processor ID)
MNEM (ret,               RET,               Near Return from Called Procedure)
MNEM (retf,              RETF,              Far Return from Called Procedure)
MNEM (rmpadjust,         RMPADJUST,         Adjust RMP Permissions)
MNEM (rmpquery,          RMPQUERY,          Read RMP Permissions)
MNEM (rmpupdate,         RMPUPDATE,         Write RMP Entry)
MNEM (rol,               ROL,               Rotate Left)
MNEM (ror,               ROR,               Rotate Right)
MNEM (rorx,              RORX,              Rotate Right Extended)
MNEM (roundpd,           ROUNDPD,           Round Packed Double-Precision Floating-Point)
MNEM (roundps,           ROUNDPS,           Round Packed Single-Precision Floating-Point)
MNEM (roundsd,           ROUNDSD,           Round Scalar Double-Precision Floating-Point)
MNEM (roundss,           ROUNDSS,           Round Scalar Single-Precision Floating-Point)
MNEM (rsm,               RSM,               Resume from System Management Mode)
MNEM (rsqrtps,           RSQRTPS,           Reciprocal Square Root Packed Single-Precision Floating-Point)
MNEM (rsqrtss,           RSQRTSS,           Reciprocal Square Root Scalar Single-Precision Floating-Point)
MNEM (rstorssp,          RSTORSSP,          Restore Saved Shadow Stack Pointer)
MNEM (sahf,              SAHF,              Store AH into Flags)
MNEM (sal,               SAL,               Shift Arithmetic Left)
MNEM (sar,               SAR,               Shift Arithmetic Right)
MNEM (sarx,              SARX,              Shift Right Arithmetic Extended)
MNEM (saveprevssp,       SAVEPREVSSP,       Save Previous Shadow Stack Pointer)
MNEM (sbb,               SBB,               Subtract with Borrow)
MNEM (scasb,             SCASB,             Scan Bytes)
MNEM (scasd,             SCASD,             Scan Doublewords)
MNEM (scasq,             SCASQ,             Scan Quadwords)
MNEM (scasw,             SCASW,             Scan Words)
MNEM (seta,              SETA,              Set byte if above (CF = 0 and ZF = 0))
MNEM (setae,             SETAE,             Set byte if above or equal (CF = 0))
MNEM (setb,              SETB,              Set byte if below (CF = 1))
MNEM (setbe,             SETBE,             Set byte if below or equal (CF = 1 or ZF = 1))
MNEM (setc,              SETC,              Set byte if carry (CF = 1))
MNEM (sete,              SETE,              Set byte if equal (ZF = 1))
MNEM (setg,              SETG,              Set byte if greater (ZF = 0 and SF = OF))
MNEM (setge,             SETGE,             Set byte if greater or equal (SF = OF))
MNEM (setl,              SETL,              Set byte if less (SF <> OF))
MNEM (setle,             SETLE,             Set byte if less or equal (ZF = 1 or SF <> OF))
MNEM (setna,             SETNA,             Set byte if not above (CF = 1 or ZF = 1))
MNEM (setnae,            SETNAE,            Set byte if not above or equal (CF = 1))
MNEM (setnb,             SETNB,             Set byte if not below (CF = 0))
MNEM (setnbe,            SETNBE,            Set byte if not below or equal (CF = 0 and ZF = 0))
MNEM (setnc,             SETNC,             Set byte if not carry (CF = 0))
MNEM (setne,             SETNE,             Set byte if not equal (ZF = 0))
MNEM (setng,             SETNG,             Set byte if not greater (ZF = 1 or SF <> OF))
MNEM (setnge,            SETNGE,            Set byte if not greater or equal (SF <> OF))
MNEM (setnl,             SETNL,             Set byte if not less (SF = OF))
MNEM (setnle,            SETNLE,            Set byte if not less or equal (ZF = 0 and SF = OF))
MNEM (setno,             SETNO,             Set byte if not overflow (OF = 0))
MNEM (setnp,             SETNP,             Set byte if not parity (PF = 0))
MNEM (setns,             SETNS,             Set byte if not sign (SF = 0))
MNEM (setnz,             SETNZ,             Set byte if not zero (ZF = 0))
MNEM (seto,              SETO,              Set byte if overflow (OF = 1))
MNEM (setp,              SETP,              Set byte if parity (PF = 1))
MNEM (setpe,             SETPE,             Set byte if parity even (PF = 1))
MNEM (setpo,             SETPO,             Set byte if parity odd (PF = 0))
MNEM (sets,              SETS,              Set byte if sign (SF = 1))
MNEM (setssbsy,          SETSSBSY,          Set Shadow Stack Busy)
MNEM (setz,              SETZ,              Set byte if zero (ZF = 1))
MNEM (sfence,            SFENCE,            Store Fence)
MNEM (sgdt,              SGDT,              Store Global Descriptor Table Register)
MNEM (sha1msg1,          SHA1MSG1,          Perform an Intermediate Calculation for the Next Four SHA1 Message Doublewords)
MNEM (sha1msg2,          SHA1MSG2,          Perform a Final Calculation for the Next Four SHA1 Message Doublewords)
MNEM (sha1nexte,         SHA1NEXTE,         Calculate SHA1 State Variable E after Four Rounds)
MNEM (sha1rnds4,         SHA1RNDS4,         Perform Four Rounds of SHA1 Operation)
MNEM (sha256msg1,        SHA256MSG1,        Perform an Intermediate Calculation for the Next Four SHA256 Message Doublewords)
MNEM (sha256msg2,        SHA256MSG2,        Perform a Final Calculation for the Next Four SHA256 Message Doublewords)
MNEM (sha256rnds2,       SHA256RNDS2,       Perform Two Rounds of SHA256 Operation)
MNEM (shl,               SHL,               Shift Left)
MNEM (shld,              SHLD,              Shift Left Double)
MNEM (shlx,              SHLX,              Shift Left Logical Extended)
MNEM (shr,               SHR,               Shift Right)
MNEM (shrd,              SHRD,              Shift Right Double)
MNEM (shrx,              SHRX,              Shift Right Logical Extended)
MNEM (shufpd,            SHUFPD,            Shuffle Packed Double-Precision Floating-Point)
MNEM (shufps,            SHUFPS,            Shuffle Packed Single-Precision Floating-Point)
MNEM (sidt,              SIDT,              Store Interrupt Descriptor Table Register)
MNEM (skinit,            SKINIT,            Secure Init and Jump with Attestation)
MNEM (sldt,              SLDT,              Store Local Descriptor Table Register)
MNEM (slwpcb,            SLWPCB,            Store Lightweight Profiling Control Block Address)
MNEM (smsw,              SMSW,              Store Machine Status Word)
MNEM (sqrtpd,            SQRTPD,            Square Root Packed Double-Precision Floating-Point)
MNEM (sqrtps,            SQRTPS,            Square Root Packed Single-Precision Floating-Point)
MNEM (sqrtsd,            SQRTSD,            Square Root Scalar Double-Precision Floating-Point)
MNEM (sqrtss,            SQRTSS,            Square Root Scalar Single-Precision Floating-Point)
MNEM (stac,              STAC,              Set Alignment Check Flag)
MNEM (stc,               STC,               Set Carry Flag)
MNEM (std,               STD,               Set Direction Flag)
MNEM (stgi,              STGI,              Set Global Interrupt Flag)
MNEM (sti,               STI,               Set Interrupt Flag)
MNEM (stmxcsr,           STMXCSR,           Store MXCSR Control/Status Register)
MNEM (stosb,             STOSB,             Store Bytes)
MNEM (stosd,             STOSD,             Store Doublewords)
MNEM (stosq,             STOSQ,             Store Quadwords)
MNEM (stosw,             STOSW,             Store Words)
MNEM (str,               STR,               Store Task Register)
MNEM (sub,               SUB,               Subtract)
MNEM (subpd,             SUBPD,             Subtract Packed Double-Precision Floating-Point)
MNEM (subps,             SUBPS,             Subtract Packed Single-Precision Floating-Point)
MNEM (subsd,             SUBSD,             Subtract Scalar Double-Precision Floating-Point)
MNEM (subss,             SUBSS,             Subtract Scalar Single-Precision Floating-Point)
MNEM (swapgs,            SWAPGS,            Swap GS Register with KernelGSbase MSR)
MNEM (syscall,           SYSCALL,           Fast System Call)
MNEM (sysenter,          SYSENTER,          System Call)
MNEM (sysexit,           SYSEXIT,           System Return)
MNEM (sysret,            SYSRET,            Fast System Return)
MNEM (t1mskc,            T1MSKC,            Inverse Mask From Trailing Ones)
MNEM (test,              TEST,              Test Bits)
MNEM (tlbsync,           TLBSYNC,           Synchronize TLB Invalidations)
MNEM (tzcnt,             TZCNT,             Count Trailing Zeros)
MNEM (tzmsk,             TZMSK,             Mask From Trailing Zeros)
MNEM (ucomisd,           UCOMISD,           Unordered Compare Scalar Double-Precision Floating-Point)
MNEM (ucomiss,           UCOMISS,           Unordered Compare Scalar Single-Precision Floating-Point)
MNEM (ud0,               UD0,               Undefined Operation)
MNEM (ud1,               UD1,               Undefined Operation)
MNEM (ud2,               UD2,               Undefined Operation)
MNEM (unpckhpd,          UNPCKHPD,          Unpack High Double-Precision Floating-Point)
MNEM (unpckhps,          UNPCKHPS,          Unpack High Single-Precision Floating-Point)
MNEM (unpcklpd,          UNPCKLPD,          Unpack Low Double-Precision Floating-Point)
MNEM (unpcklps,          UNPCKLPS,          Unpack Low Single-Precision Floating-Point)
MNEM (vaddpd,            VADDPD,            Add Packed Double-Precision Floating-Point)
MNEM (vaddps,            VADDPS,            Add Packed Single-Precision Floating-Point)
MNEM (vaddsd,            VADDSD,            Add Scalar Double-Precision Floating-Point)
MNEM (vaddss,            VADDSS,            Add Scalar Single-Precision Floating-Point)
MNEM (vaddsubpd,         VADDSUBPD,         Add and Subtract Packed Double-Precision)
MNEM (vaddsubps,         VADDSUBPS,         Add and Subtract Packed Single-Precision)
MNEM (vaesdec,           VAESDEC,           AES Decryption Round)
MNEM (vaesdeclast,       VAESDECLAST,       AES Last Decryption Round)
MNEM (vaesenc,           VAESENC,           AES Encryption Round)
MNEM (vaesenclast,       VAESENCLAST,       AES Last Encryption Round)
MNEM (vaesimc,           VAESIMC,           AES InvMixColumn Transformation)
MNEM (vaeskeygenassist,  VAESKEYGENASSIST,  AES Assist Round Key Generation)
MNEM (vandnpd,           VANDNPD,           Logical Bitwise AND NOT Packed Double-Precision Floating-Point)
MNEM (vandnps,           VANDNPS,           Logical Bitwise AND NOT Packed Single-Precision Floating-Point)
MNEM (vandpd,            VANDPD,            Logical Bitwise AND Packed Double-Precision Floating-Point)
MNEM (vandps,            VANDPS,            Logical Bitwise AND Packed Single-Precision Floating-Point)
MNEM (vblendpd,          VBLENDPD,          Blend Packed Double-Precision Floating-Point)
MNEM (vblendps,          VBLENDPS,          Blend Packed Single-Precision Floating-Point)
MNEM (vblendvpd,         VBLENDVPD,         Variable Blend Packed Double-Precision Floating-Point)
MNEM (vblendvps,         VBLENDVPS,         Variable Blend Packed Single-Precision Floating-Point)
MNEM (vbroadcastf128,    VBROADCASTF128,    Load With Broadcast From 128-bit Memory Location)
MNEM (vbroadcasti128,    VBROADCASTI128,    Load With Broadcast Integer From 128-bit Memory Location)
MNEM (vbroadcastsd,      VBROADCASTSD,      Load With Broadcast From 64-bit Memory Location)
MNEM (vbroadcastss,      VBROADCASTSS,      Load With Broadcast From 32-bit Memory Location)
MNEM (vcmpeqpd,          VCMPEQPD,          Compare Packed Double-Precision Floating-Point Equal)
MNEM (vcmpeqps,          VCMPEQPS,          Compare Packed Single-Precision Floating-Point Equal)
MNEM (vcmpeqsd,          VCMPEQSD,          Compare Scalar Double-Precision Floating-Point Equal)
MNEM (vcmpeqss,          VCMPEQSS,          Compare Scalar Single-Precision Floating-Point Equal)
MNEM (vcmplepd,          VCMPLEPD,          Compare Packed Double-Precision Floating-Point Less or Equal)
MNEM (vcmpleps,          VCMPLEPS,          Compare Packed Single-Precision Floating-Point Less or Equal)
MNEM (vcmplesd,          VCMPLESD,          Compare Scalar Double-Precision Floating-Point Less or Equal)
MNEM (vcmpless,          VCMPLESS,          Compare Scalar Single-Precision Floating-Point Less or Equal)
MNEM (vcmpltpd,          VCMPLTPD,          Compare Packed Double-Precision Floating-Point Less Than)
MNEM (vcmpltps,          VCMPLTPS,          Compare Packed Single-Precision Floating-Point Less Than)
MNEM (vcmpltsd,          VCMPLTSD,          Compare Scalar Double-Precision Floating-Point Less Than)
MNEM (vcmpltss,          VCMPLTSS,          Compare Scalar Single-Precision Floating-Point Less Than)
MNEM (vcmpneqpd,         VCMPNEQPD,         Compare Packed Double-Precision Floating-Point Not Equal)
MNEM (vcmpneqps,         VCMPNEQPS,         Compare Packed Single-Precision Floating-Point Not Equal)
MNEM (vcmpneqsd,         VCMPNEQSD,         Compare Scalar Double-Precision Floating-Point Not Equal)
MNEM (vcmpneqss,         VCMPNEQSS,         Compare Scalar Single-Precision Floating-Point Not Equal)
MNEM (vcmpnlepd,         VCMPNLEPD,         Compare Packed Double-Precision Floating-Point Not Less or Equal)
MNEM (vcmpnleps,         VCMPNLEPS,         Compare Packed Single-Precision Floating-Point Not Less or Equal)
MNEM (vcmpnlesd,         VCMPNLESD,         Compare Scalar Double-Precision Floating-Point Not Less or Equal)
MNEM (vcmpnless,         VCMPNLESS,         Compare Scalar Single-Precision Floating-Point Not Less or Equal)
MNEM (vcmpnltpd,         VCMPNLTPD,         Compare Packed Double-Precision Floating-Point Not Less Than)
MNEM (vcmpnltps,         VCMPNLTPS,         Compare Packed Single-Precision Floating-Point Not Less Than)
MNEM (vcmpnltsd,         VCMPNLTSD,         Compare Scalar Double-Precision Floating-Point Not Less Than)
MNEM (vcmpnltss,         VCMPNLTSS,         Compare Scalar Single-Precision Floating-Point Not Less Than)
MNEM (vcmpordpd,         VCMPORDPD,         Compare Packed Double-Precision Floating-Point Ordered)
MNEM (vcmpordps,         VCMPORDPS,         Compare Packed Single-Precision Floating-Point Ordered)
MNEM (vcmpordsd,         VCMPORDSD,         Compare Scalar Double-Precision Floating-Point Ordered)
MNEM (vcmpordss,         VCMPORDSS,         Compare Scalar Single-Precision Floating-Point Ordered)
MNEM (vcmppd,            VCMPPD,            Compare Packed Double-Precision Floating-Point)
MNEM (vcmpps,            VCMPPS,            Compare Packed Single-Precision Floating-Point)
MNEM (vcmpsd,            VCMPSD,            Compare Scalar Double-Precision Floating-Point)
MNEM (vcmpss,            VCMPSS,            Compare Scalar Single-Precision Floating-Point)
MNEM (vcmpunordpd,       VCMPUNORDPD,       Compare Packed Double-Precision Floating-Point Unordered)
MNEM (vcmpunordps,       VCMPUNORDPS,       Compare Packed Single-Precision Floating-Point Unordered)
MNEM (vcmpunordsd,       VCMPUNORDSD,       Compare Scalar Double-Precision Floating-Point Unordered)
MNEM (vcmpunordss,       VCMPUNORDSS,       Compare Scalar Single-Precision Floating-Point Unordered)
MNEM (vcomisd,           VCOMISD,           Compare Ordered Scalar Double-Precision Floating-Point)
MNEM (vcomiss,           VCOMISS,           Compare Ordered Scalar Single-Precision Floating-Point)
MNEM (vcvtdq2pd,         VCVTDQ2PD,         Convert Packed Doubleword Integers to Packed Double-Precision Floating-Point)
MNEM (vcvtdq2ps,         VCVTDQ2PS,         Convert Packed Doubleword Integers to Packed Single-Precision Floating-Point)
MNEM (vcvtpd2dq,         VCVTPD2DQ,         Convert Packed Double-Precision Floating-Point to Packed Doubleword Integers)
MNEM (vcvtpd2ps,         VCVTPD2PS,         Convert Packed Double-Precision Floating-Point to Packed Single-Precision Floating-Point)
MNEM (vcvtph2ps,         VCVTPH2PS,         Convert Packed 16-Bit Floating-Point to Single-Precision Floating-Point)
MNEM (vcvtps2dq,         VCVTPS2DQ,         Convert Packed Single-Precision Floating-Point to Packed Doubleword Integers)
MNEM (vcvtps2pd,         VCVTPS2PD,         Convert Packed Single-Precision Floating-Point to Packed Double-Precision Floating-Point)
MNEM (vcvtps2ph,         VCVTPS2PH,         Convert Packed Single-Precision Floating-Point to 16-Bit Floating-Point)
MNEM (vcvtsd2si,         VCVTSD2SI,         Convert Scalar Double-Precision Floating-Point to Signed Doubleword or Quadword Integer)
MNEM (vcvtsd2ss,         VCVTSD2SS,         Convert Scalar Double-Precision Floating-Point to Scalar Single-Precision Floating-Point)
MNEM (vcvtsi2sd,         VCVTSI2SD,         Convert Signed Doubleword or Quadword Integer to Scalar Double-Precision Floating-Point)
MNEM (vcvtsi2ss,         VCVTSI2SS,         Convert Signed Doubleword or Quadword Integer to Scalar Single-Precision Floating-Point)
MNEM (vcvtss2sd,         VCVTSS2SD,         Convert Scalar Single-Precision Floating-Point to Scalar Double-Precision Floating-Point)
MNEM (vcvtss2si,         VCVTSS2SI,         Convert Scalar Single-Precision Floating-Point to Signed Doubleword or Quadword Integer)
MNEM (vcvttpd2dq,        VCVTTPD2DQ,        Convert Packed Double-Precision Floating-Point to Packed Doubleword Integers Truncated)
MNEM (vcvttps2dq,        VCVTTPS2DQ,        Convert Packed Single-Precision Floating-Point to Packed Doubleword Integers Truncated)
MNEM (vcvttsd2si,        VCVTTSD2SI,        Convert Scalar Double-Precision Floating-Point to Signed Doubleword of Quadword Integer Truncated)
MNEM (vcvttss2si,        VCVTTSS2SI,        Convert Scalar Single-Precision Floating-Point to Signed Doubleword or Quadword Integer Truncated)
MNEM (vdivpd,            VDIVPD,            Divide Packed Double-Precision Floating-Point)
MNEM (vdivps,            VDIVPS,            Divide Packed Single-Precision Floating-Point)
MNEM (vdivsd,            VDIVSD,            Divide Scalar Double-Precision Floating-Point)
MNEM (vdivss,            VDIVSS,            Divide Scalar Single-Precision Floating-Point)
MNEM (vdppd,             VDPPD,             Dot Product Packed Double-Precision Floating-Point)
MNEM (vdpps,             VDPPS,             Dot Product Packed Single-Precision Floating-Point)
MNEM (verr,              VERR,              Verify Segment for Reads)
MNEM (verw,              VERW,              Verify Segment for Write)
MNEM (vextractf128,      VEXTRACTF128,      Extract Packed Floating-Point Values)
MNEM (vextracti128,      VEXTRACTI128,      Extract 128-bit Integer)
MNEM (vextractps,        VEXTRACTPS,        Extract Packed Single-Precision Floating-Point)
MNEM (vfmadd132pd,       VFMADD132PD,       Multiply and Add Packed Double-Precision Floating-Point)
MNEM (vfmadd132ps,       VFMADD132PS,       Multiply and Add Packed Single-Precision Floating-Point)
MNEM (vfmadd132sd,       VFMADD132SD,       Multiply and Add Scalar Double-Precision Floating-Point)
MNEM (vfmadd132ss,       VFMADD132SS,       Multiply and Add Scalar Single-Precision Floating-Point)
MNEM (vfmadd213pd,       VFMADD213PD,       Multiply and Add Packed Double-Precision Floating-Point)
MNEM (vfmadd213ps,       VFMADD213PS,       Multiply and Add Packed Single-Precision Floating-Point)
MNEM (vfmadd213sd,       VFMADD213SD,       Multiply and Add Scalar Double-Precision Floating-Point)
MNEM (vfmadd213ss,       VFMADD213SS,       Multiply and Add Scalar Single-Precision Floating-Point)
MNEM (vfmadd231pd,       VFMADD231PD,       Multiply and Add Packed Double-Precision Floating-Point)
MNEM (vfmadd231ps,       VFMADD231PS,       Multiply and Add Packed Single-Precision Floating-Point)
MNEM (vfmadd231sd,       VFMADD231SD,       Multiply and Add Scalar Double-Precision Floating-Point)
MNEM (vfmadd231ss,       VFMADD231SS,       Multiply and Add Scalar Single-Precision Floating-Point)
MNEM (vfmaddpd,          VFMADDPD,          Multiply and Add Packed Double-Precision Floating-Point)
MNEM (vfmaddps,          VFMADDPS,          Multiply and Add Packed Single-Precision Floating-Point)
MNEM (vfmaddsd,          VFMADDSD,          Multiply and Add Scalar Double-Precision Floating-Point)
MNEM (vfmaddss,          VFMADDSS,          Multiply and Add Scalar Single-Precision Floating-Point)
MNEM (vfmaddsub132pd,    VFMADDSUB132PD,    Multiply with Alternating Add/Subtract Packed Double-Precision Floating-Point)
MNEM (vfmaddsub132ps,    VFMADDSUB132PS,    Multiply with Alternating Add/Subtract Packed Single-Precision Floating-Point)
MNEM (vfmaddsub213pd,    VFMADDSUB213PD,    Multiply with Alternating Add/Subtract Packed Double-Precision Floating-Point)
MNEM (vfmaddsub213ps,    VFMADDSUB213PS,    Multiply with Alternating Add/Subtract Packed Single-Precision Floating-Point)
MNEM (vfmaddsub231pd,    VFMADDSUB231PD,    Multiply with Alternating Add/Subtract Packed Double-Precision Floating-Point)
MNEM (vfmaddsub231ps,    VFMADDSUB231PS,    Multiply with Alternating Add/Subtract Packed Single-Precision Floating-Point)
MNEM (vfmaddsubpd,       VFMADDSUBPD,       Multiply with Alternating Add/Subtract Packed Double-Precision Floating-Point)
MNEM (vfmaddsubps,       VFMADDSUBPS,       Multiply with Alternating Add/Subtract Packed Single-Precision Floating-Point)
MNEM (vfmsub132pd,       VFMSUB132PD,       Multiply and Subtract Packed Double-Precision Floating-Point)
MNEM (vfmsub132ps,       VFMSUB132PS,       Multiply and Subtract Packed Single-Precision Floating-Point)
MNEM (vfmsub132sd,       VFMSUB132SD,       Multiply and Subtract Scalar Double-Precision Floating-Point)
MNEM (vfmsub132ss,       VFMSUB132SS,       Multiply and Subtract Scalar Single-Precision Floating-Point)
MNEM (vfmsub213pd,       VFMSUB213PD,       Multiply and Subtract Packed Double-Precision Floating-Point)
MNEM (vfmsub213ps,       VFMSUB213PS,       Multiply and Subtract Packed Single-Precision Floating-Point)
MNEM (vfmsub213sd,       VFMSUB213SD,       Multiply and Subtract Scalar Double-Precision Floating-Point)
MNEM (vfmsub213ss,       VFMSUB213SS,       Multiply and Subtract Scalar Single-Precision Floating-Point)
MNEM (vfmsub231pd,       VFMSUB231PD,       Multiply and Subtract Packed Double-Precision Floating-Point)
MNEM (vfmsub231ps,       VFMSUB231PS,       Multiply and Subtract Packed Single-Precision Floating-Point)
MNEM (vfmsub231sd,       VFMSUB231SD,       Multiply and Subtract Scalar Double-Precision Floating-Point)
MNEM (vfmsub231ss,       VFMSUB231SS,       Multiply and Subtract Scalar Single-Precision Floating-Point)
MNEM (vfmsubadd132pd,    VFMSUBADD132PD,    Multiply with Alternating Subtract/Add Packed Double-Precision Floating-Point)
MNEM (vfmsubadd132ps,    VFMSUBADD132PS,    Multiply with Alternating Subtract/Add Packed Single-Precision Floating-Point)
MNEM (vfmsubadd213pd,    VFMSUBADD213PD,    Multiply with Alternating Subtract/Add Packed Double-Precision Floating-Point)
MNEM (vfmsubadd213ps,    VFMSUBADD213PS,    Multiply with Alternating Subtract/Add Packed Single-Precision Floating-Point)
MNEM (vfmsubadd231pd,    VFMSUBADD231PD,    Multiply with Alternating Subtract/Add Packed Double-Precision Floating-Point)
MNEM (vfmsubadd231ps,    VFMSUBADD231PS,    Multiply with Alternating Subtract/Add Packed Single-Precision Floating-Point)
MNEM (vfmsubaddpd,       VFMSUBADDPD,       Multiply with Alternating Subtract/Add Packed Double-Precision Floating-Point)
MNEM (vfmsubaddps,       VFMSUBADDPS,       Multiply with Alternating Subtract/Add Packed Single-Precision Floating-Point)
MNEM (vfmsubpd,          VFMSUBPD,          Multiply and Subtract Packed Double-Precision Floating-Point)
MNEM (vfmsubps,          VFMSUBPS,          Multiply and Subtract Packed Single-Precision Floating-Point)
MNEM (vfmsubsd,          VFMSUBSD,          Multiply and Subtract Scalar Double-Precision Floating-Point)
MNEM (vfmsubss,          VFMSUBSS,          Multiply and Subtract Scalar Single-Precision Floating-Point)
MNEM (vfnmadd132pd,      VFNMADD132PD,      Negative Multiply and Add Packed Double-Precision Floating-Point)
MNEM (vfnmadd132ps,      VFNMADD132PS,      Negative Multiply and Add Packed Single-Precision Floating-Point)
MNEM (vfnmadd132sd,      VFNMADD132SD,      Negative Multiply and Add Scalar Double-Precision Floating-Point)
MNEM (vfnmadd132ss,      VFNMADD132SS,      Negative Multiply and Add Scalar Single-Precision Floating-Point)
MNEM (vfnmadd213pd,      VFNMADD213PD,      Negative Multiply and Add Packed Double-Precision Floating-Point)
MNEM (vfnmadd213ps,      VFNMADD213PS,      Negative Multiply and Add Packed Single-Precision Floating-Point)
MNEM (vfnmadd213sd,      VFNMADD213SD,      Negative Multiply and Add Scalar Double-Precision Floating-Point)
MNEM (vfnmadd213ss,      VFNMADD213SS,      Negative Multiply and Add Scalar Single-Precision Floating-Point)
MNEM (vfnmadd231pd,      VFNMADD231PD,      Negative Multiply and Add Packed Double-Precision Floating-Point)
MNEM (vfnmadd231ps,      VFNMADD231PS,      Negative Multiply and Add Packed Single-Precision Floating-Point)
MNEM (vfnmadd231sd,      VFNMADD231SD,      Negative Multiply and Add Scalar Double-Precision Floating-Point)
MNEM (vfnmadd231ss,      VFNMADD231SS,      Negative Multiply and Add Scalar Single-Precision Floating-Point)
MNEM (vfnmaddpd,         VFNMADDPD,         Negative Multiply and Add Packed Double-Precision Floating-Point)
MNEM (vfnmaddps,         VFNMADDPS,         Negative Multiply and Add Packed Single-Precision Floating-Point)
MNEM (vfnmaddsd,         VFNMADDSD,         Negative Multiply and Add Scalar Double-Precision Floating-Point)
MNEM (vfnmaddss,         VFNMADDSS,         Negative Multiply and Add Scalar Single-Precision Floating-Point)
MNEM (vfnmsub132pd,      VFNMSUB132PD,      Negative Multiply and Subtract Packed Double-Precision Floating-Point)
MNEM (vfnmsub132ps,      VFNMSUB132PS,      Negative Multiply and Subtract Scalar Double-Precision Floating-Point)
MNEM (vfnmsub132sd,      VFNMSUB132SD,      Negative Multiply and Subtract Scalar Double-Precision Floating-Point)
MNEM (vfnmsub132ss,      VFNMSUB132SS,      Negative Multiply and Subtract Scalar Single-Precision Floating-Point)
MNEM (vfnmsub213pd,      VFNMSUB213PD,      Negative Multiply and Subtract Packed Double-Precision Floating-Point)
MNEM (vfnmsub213ps,      VFNMSUB213PS,      Negative Multiply and Subtract Scalar Double-Precision Floating-Point)
MNEM (vfnmsub213sd,      VFNMSUB213SD,      Negative Multiply and Subtract Scalar Double-Precision Floating-Point)
MNEM (vfnmsub213ss,      VFNMSUB213SS,      Negative Multiply and Subtract Scalar Single-Precision Floating-Point)
MNEM (vfnmsub231pd,      VFNMSUB231PD,      Negative Multiply and Subtract Packed Double-Precision Floating-Point)
MNEM (vfnmsub231ps,      VFNMSUB231PS,      Negative Multiply and Subtract Scalar Double-Precision Floating-Point)
MNEM (vfnmsub231sd,      VFNMSUB231SD,      Negative Multiply and Subtract Scalar Double-Precision Floating-Point)
MNEM (vfnmsub231ss,      VFNMSUB231SS,      Negative Multiply and Subtract Scalar Double-Precision Floating-Point)
MNEM (vfnmsubpd,         VFNMSUBPD,         Negative Multiply and Subtract Packed Double-Precision Floating-Point)
MNEM (vfnmsubps,         VFNMSUBPS,         Negative Multiply and Subtract Packed Single-Precision Floating-Point)
MNEM (vfnmsubsd,         VFNMSUBSD,         Negative Multiply and Subtract Scalar Double-Precision Floating-Point)
MNEM (vfnmsubss,         VFNMSUBSS,         Negative Multiply and Subtract Scalar Single-Precision Floating-Point)
MNEM (vfrczpd,           VFRCZPD,           Extract Fraction Packed Double-Precision Floating-Point)
MNEM (vfrczps,           VFRCZPS,           Extract Fraction Packed Single-Precision Floating-Point)
MNEM (vfrczsd,           VFRCZSD,           Extract Fraction Scalar Double-Precision Floating-Point)
MNEM (vfrczss,           VFRCZSS,           Extract Fraction Scalar Single-Precision Floating-Point)
MNEM (vhaddpd,           VHADDPD,           Horizontal Add Packed Double)
MNEM (vhaddps,           VHADDPS,           Horizontal Add Packed Single)
MNEM (vhsubpd,           VHSUBPD,           Horizontal Sub Packed Double)
MNEM (vhsubps,           VHSUBPS,           Horizontal Sub Packed Single)
MNEM (vinsertf128,       VINSERTF128,       Insert Packed Floating-Point Values 128-bit)
MNEM (vinserti128,       VINSERTI128,       Insert Packed Integer Values 128-bit)
MNEM (vinsertps,         VINSERTPS,         Insert Packed Single-Precision Floating-Point)
MNEM (vlddqu,            VLDDQU,            Load Unaligned Double Quadword)
MNEM (vldmxcsr,          VLDMXCSR,          Load MXCSR Control/Status Register)
MNEM (vmaskmovdqu,       VMASKMOVDQU,       Masked Move Double Quadword Unaligned)
MNEM (vmaskmovpd,        VMASKMOVPD,        Masked Move Packed Double-Precision)
MNEM (vmaskmovps,        VMASKMOVPS,        Masked Move Packed Single-Precision)
MNEM (vmaxpd,            VMAXPD,            Maximum Packed Double-Precision Floating-Point)
MNEM (vmaxps,            VMAXPS,            Maximum Packed Single-Precision Floating-Point)
MNEM (vmaxsd,            VMAXSD,            Maximum Scalar Double-Precision Floating-Point)
MNEM (vmaxss,            VMAXSS,            Maximum Scalar Single-Precision Floating-Point)
MNEM (vmgexit,           VMGEXIT,           SEV-ES Exit to VMM)
MNEM (vminpd,            VMINPD,            Minimum Packed Double-Precision Floating-Point)
MNEM (vminps,            VMINPS,            Minimum Packed Single-Precision Floating-Point)
MNEM (vminsd,            VMINSD,            Minimum Scalar Double-Precision Floating-Point)
MNEM (vminss,            VMINSS,            Minimum Scalar Single-Precision Floating-Point)
MNEM (vmload,            VMLOAD,            Load State from VMCB)
MNEM (vmmcall,           VMMCALL,           Call VMM)
MNEM (vmovapd,           VMOVAPD,           Move Aligned Packed Double-Precision Floating-Point)
MNEM (vmovaps,           VMOVAPS,           Move Aligned Packed Single-Precision Floating-Point)
MNEM (vmovd,             VMOVD,             Move Doubleword or Quadword)
MNEM (vmovddup,          VMOVDDUP,          Move Double-Precision and Duplicate)
MNEM (vmovdqa,           VMOVDQA,           Move Aligned Double Quadword)
MNEM (vmovdqu,           VMOVDQU,           Move Unaligned Double Quadword)
MNEM (vmovhlps,          VMOVHLPS,          Move Packed Single-Precision Floating-Point High to Low)
MNEM (vmovhpd,           VMOVHPD,           Move High Packed Double-Precision Floating-Point)
MNEM (vmovhps,           VMOVHPS,           Move High Packed Single-Precision Floating-Point)
MNEM (vmovlhps,          VMOVLHPS,          Move Packed Single-Precision Floating-Point Low to High)
MNEM (vmovlpd,           VMOVLPD,           Move Low Packed Double-Precision Floating-Point)
MNEM (vmovlps,           VMOVLPS,           Move Low Packed Single-Precision Floating-Point)
MNEM (vmovmskpd,         VMOVMSKPD,         Extract Packed Double-Precision Floating-Point Sign Mask)
MNEM (vmovmskps,         VMOVMSKPS,         Extract Packed Single-Precision Floating-Point Sign Mask)
MNEM (vmovntdq,          VMOVNTDQ,          Move Non-Temporal Double Quadword)
MNEM (vmovntdqa,         VMOVNTDQA,         Move Non-Temporal Double Quadword Aligned)
MNEM (vmovntpd,          VMOVNTPD,          Move Non-Temporal Packed Double-Precision Floating-Point)
MNEM (vmovntps,          VMOVNTPS,          Move Non-Temporal Packed Single-Precision Floating-Point)
MNEM (vmovq,             VMOVQ,             Move Quadword)
MNEM (vmovsd,            VMOVSD,            Move Doublewords)
MNEM (vmovshdup,         VMOVSHDUP,         Move Single-Precision High and Duplicate)
MNEM (vmovsldup,         VMOVSLDUP,         Move Single-Precision Low and Duplicate)
MNEM (vmovss,            VMOVSS,            Move Scalar Single-Precision Floating-Point)
MNEM (vmovupd,           VMOVUPD,           Move Unaligned Packed Double-Precision Floating-Point)
MNEM (vmovups,           VMOVUPS,           Move Unaligned Packed Single-Precision Floating-Point)
MNEM (vmpsadbw,          VMPSADBW,          Multiple Sum of Absolute Differences)
MNEM (vmrun,             VMRUN,             Run Virtual Machine)
MNEM (vmsave,            VMSAVE,            Save State to VMCB)
MNEM (vmulpd,            VMULPD,            Multiply Packed Double-Precision Floating-Point)
MNEM (vmulps,            VMULPS,            Multiply Packed Single-Precision Floating-Point)
MNEM (vmulsd,            VMULSD,            Multiply Scalar Double-Precision Floating-Point)
MNEM (vmulss,            VMULSS,            Multiply Scalar Single-Precision Floating-Point)
MNEM (vorpd,             VORPD,             Logical Bitwise OR Packed Double-Precision Floating-Point)
MNEM (vorps,             VORPS,             Logical Bitwise OR Packed Single-Precision Floating-Point)
MNEM (vpabsb,            VPABSB,            Packed Absolute Value Signed Byte)
MNEM (vpabsd,            VPABSD,            Packed Absolute Value Signed Doubleword)
MNEM (vpabsw,            VPABSW,            Packed Absolute Value Signed Word)
MNEM (vpackssdw,         VPACKSSDW,         Pack with Saturation Signed Doubleword to Word)
MNEM (vpacksswb,         VPACKSSWB,         Pack with Saturation Signed Word to Byte)
MNEM (vpackusdw,         VPACKUSDW,         Pack with Saturation Signed Word to Unsigned Byte)
MNEM (vpackuswb,         VPACKUSWB,         Pack with Saturation Signed Word to Unsigned Byte)
MNEM (vpaddb,            VPADDB,            Packed Add Bytes)
MNEM (vpaddd,            VPADDD,            Packed Add Doublewords)
MNEM (vpaddq,            VPADDQ,            Packed Add Quadwords)
MNEM (vpaddsb,           VPADDSB,           Packed Add Signed with Saturation Bytes)
MNEM (vpaddsw,           VPADDSW,           Packed Add Signed with Saturation Words)
MNEM (vpaddusb,          VPADDUSB,          Packed Add Unsigned with Saturation Bytes)
MNEM (vpaddusw,          VPADDUSW,          Packed Add Unsigned with Saturation Words)
MNEM (vpaddw,            VPADDW,            Packed Add Words)
MNEM (vpalignr,          VPALIGNR,          Packed Align Right)
MNEM (vpand,             VPAND,             Packed Logical Bitwise AND)
MNEM (vpandn,            VPANDN,            Packed Logical Bitwise AND NOT)
MNEM (vpavgb,            VPAVGB,            Packed Average Unsigned Bytes)
MNEM (vpavgw,            VPAVGW,            Packed Average Unsigned Words)
MNEM (vpblendd,          VPBLENDD,          Blend Packed Doublewords)
MNEM (vpblendvb,         VPBLENDVB,         Variable Blend Packed Bytes)
MNEM (vpblendw,          VPBLENDW,          Blend Packed Words)
MNEM (vpbroadcastb,      VPBROADCASTB,      Broadcast Packed Byte)
MNEM (vpbroadcastd,      VPBROADCASTD,      Broadcast Packed Doubleword)
MNEM (vpbroadcastq,      VPBROADCASTQ,      Broadcast Packed Quadword)
MNEM (vpbroadcastw,      VPBROADCASTW,      Broadcast Packed Quadword)
MNEM (vpclmulqdq,        VPCLMULQDQ,        Carry-less Multiply Quadwords)
MNEM (vpcmov,            VPCMOV,            Vector Conditional Moves)
MNEM (vpcmpeqb,          VPCMPEQB,          Packed Compare Equal Bytes)
MNEM (vpcmpeqd,          VPCMPEQD,          Packed Compare Equal Doublewords)
MNEM (vpcmpeqq,          VPCMPEQQ,          Packed Compare Equal Quadwords)
MNEM (vpcmpeqw,          VPCMPEQW,          Packed Compare Equal Words)
MNEM (vpcmpestri,        VPCMPESTRI,        Packed Compare Explicit Length Strings Return Index)
MNEM (vpcmpestrm,        VPCMPESTRM,        Packed Compare Explicit Length Strings Return Mask)
MNEM (vpcmpgtb,          VPCMPGTB,          Packed Compare Greater Than Signed Bytes)
MNEM (vpcmpgtd,          VPCMPGTD,          Packed Compare Greater Than Signed Doublewords)
MNEM (vpcmpgtq,          VPCMPGTQ,          Packed Compare Greater Than Signed Quadwords)
MNEM (vpcmpgtw,          VPCMPGTW,          Packed Compare Greater Than Signed Words)
MNEM (vpcmpistri,        VPCMPISTRI,        Packed Compare Implicit Length Strings Return Index)
MNEM (vpcmpistrm,        VPCMPISTRM,        Packed Compare Implicit Length Strings Return Mask)
MNEM (vpcomb,            VPCOMB,            Compare Vector Signed Bytes)
MNEM (vpcomd,            VPCOMD,            Compare Vector Signed Doublewords)
MNEM (vpcomeqb,          VPCOMEQB,          Compare Vector Signed Bytes Equal)
MNEM (vpcomeqd,          VPCOMEQD,          Compare Vector Signed Doublewords Equal)
MNEM (vpcomeqq,          VPCOMEQQ,          Compare Vector Signed Quadwords Equal)
MNEM (vpcomequb,         VPCOMEQUB,         Compare Vector Unsigned Bytes Equal)
MNEM (vpcomequd,         VPCOMEQUD,         Compare Vector Unsigned Doublewords Equal)
MNEM (vpcomequq,         VPCOMEQUQ,         Compare Vector Unsigned Quadwords Equal)
MNEM (vpcomequw,         VPCOMEQUW,         Compare Vector Unsigned Words Equal)
MNEM (vpcomeqw,          VPCOMEQW,          Compare Vector Signed Words Equal)
MNEM (vpcomfalseb,       VPCOMFALSEB,       Compare Vector Signed Bytes False)
MNEM (vpcomfalsed,       VPCOMFALSED,       Compare Vector Signed Doublewords False)
MNEM (vpcomfalseq,       VPCOMFALSEQ,       Compare Vector Signed Quadwords False)
MNEM (vpcomfalseub,      VPCOMFALSEUB,      Compare Vector Unsigned Bytes False)
MNEM (vpcomfalseud,      VPCOMFALSEUD,      Compare Vector Unsigned Doublewords False)
MNEM (vpcomfalseuq,      VPCOMFALSEUQ,      Compare Vector Unsigned Quadwords False)
MNEM (vpcomfalseuw,      VPCOMFALSEUW,      Compare Vector Unsigned Words False)
MNEM (vpcomfalsew,       VPCOMFALSEW,       Compare Vector Signed Words False)
MNEM (vpcomgeb,          VPCOMGEB,          Compare Vector Signed Bytes Greater Than or Equal)
MNEM (vpcomged,          VPCOMGED,          Compare Vector Signed Doublewords Greater Than or Equal)
MNEM (vpcomgeq,          VPCOMGEQ,          Compare Vector Signed Quadwords Greater Than or Equal)
MNEM (vpcomgeub,         VPCOMGEUB,         Compare Vector Unsigned Bytes Greater Than or Equal)
MNEM (vpcomgeud,         VPCOMGEUD,         Compare Vector Unsigned Doublewords Greater Than or Equal)
MNEM (vpcomgeuq,         VPCOMGEUQ,         Compare Vector Unsigned Quadwords Greater Than or Equal)
MNEM (vpcomgeuw,         VPCOMGEUW,         Compare Vector Unsigned Words Greater Than or Equal)
MNEM (vpcomgew,          VPCOMGEW,          Compare Vector Signed Words Greater Than or Equal)
MNEM (vpcomgtb,          VPCOMGTB,          Compare Vector Signed Bytes Greater Than)
MNEM (vpcomgtd,          VPCOMGTD,          Compare Vector Signed Doublewords Greater Than)
MNEM (vpcomgtq,          VPCOMGTQ,          Compare Vector Signed Quadwords Greater Than)
MNEM (vpcomgtub,         VPCOMGTUB,         Compare Vector Unsigned Bytes Greater Than)
MNEM (vpcomgtud,         VPCOMGTUD,         Compare Vector Unsigned Doublewords Greater Than)
MNEM (vpcomgtuq,         VPCOMGTUQ,         Compare Vector Unsigned Quadwords Greater Than)
MNEM (vpcomgtuw,         VPCOMGTUW,         Compare Vector Unsigned Words Greater Than)
MNEM (vpcomgtw,          VPCOMGTW,          Compare Vector Signed Words Greater Than)
MNEM (vpcomleb,          VPCOMLEB,          Compare Vector Signed Bytes Less Than or Equal)
MNEM (vpcomled,          VPCOMLED,          Compare Vector Signed Doublewords Less Than or Equal)
MNEM (vpcomleq,          VPCOMLEQ,          Compare Vector Signed Quadwords Less Than or Equal)
MNEM (vpcomleub,         VPCOMLEUB,         Compare Vector Unsigned Bytes Less Than or Equal)
MNEM (vpcomleud,         VPCOMLEUD,         Compare Vector Unsigned Doublewords Less Than or Equal)
MNEM (vpcomleuq,         VPCOMLEUQ,         Compare Vector Unsigned Quadwords Less Than or Equal)
MNEM (vpcomleuw,         VPCOMLEUW,         Compare Vector Unsigned Words Less Than or Equal)
MNEM (vpcomlew,          VPCOMLEW,          Compare Vector Signed Words Less Than or Equal)
MNEM (vpcomltb,          VPCOMLTB,          Compare Vector Signed Bytes Less Than)
MNEM (vpcomltd,          VPCOMLTD,          Compare Vector Signed Doublewords Less Than)
MNEM (vpcomltq,          VPCOMLTQ,          Compare Vector Signed Quadwords Less Than)
MNEM (vpcomltub,         VPCOMLTUB,         Compare Vector Unsigned Bytes Less Than)
MNEM (vpcomltud,         VPCOMLTUD,         Compare Vector Unsigned Doublewords Less Than)
MNEM (vpcomltuq,         VPCOMLTUQ,         Compare Vector Unsigned Quadwords Less Than)
MNEM (vpcomltuw,         VPCOMLTUW,         Compare Vector Unsigned Words Less Than)
MNEM (vpcomltw,          VPCOMLTW,          Compare Vector Signed Words Less Than)
MNEM (vpcomneqb,         VPCOMNEQB,         Compare Vector Signed Bytes Not Equal)
MNEM (vpcomneqd,         VPCOMNEQD,         Compare Vector Signed Doublewords Not Equal)
MNEM (vpcomneqq,         VPCOMNEQQ,         Compare Vector Signed Quadwords Not Equal)
MNEM (vpcomnequb,        VPCOMNEQUB,        Compare Vector Unsigned Bytes Not Equal)
MNEM (vpcomnequd,        VPCOMNEQUD,        Compare Vector Unsigned Doublewords Not Equal)
MNEM (vpcomnequq,        VPCOMNEQUQ,        Compare Vector Unsigned Quadwords Not Equal)
MNEM (vpcomnequw,        VPCOMNEQUW,        Compare Vector Unsigned Words Not Equal)
MNEM (vpcomneqw,         VPCOMNEQW,         Compare Vector Signed Words Not Equal)
MNEM (vpcomq,            VPCOMQ,            Compare Vector Signed Quadwords)
MNEM (vpcomtrueb,        VPCOMTRUEB,        Compare Vector Signed Bytes true)
MNEM (vpcomtrued,        VPCOMTRUED,        Compare Vector Signed Doublewords true)
MNEM (vpcomtrueq,        VPCOMTRUEQ,        Compare Vector Signed Quadwords true)
MNEM (vpcomtrueub,       VPCOMTRUEUB,       Compare Vector Unsigned Bytes true)
MNEM (vpcomtrueud,       VPCOMTRUEUD,       Compare Vector Unsigned Doublewords true)
MNEM (vpcomtrueuq,       VPCOMTRUEUQ,       Compare Vector Unsigned Quadwords true)
MNEM (vpcomtrueuw,       VPCOMTRUEUW,       Compare Vector Unsigned Words true)
MNEM (vpcomtruew,        VPCOMTRUEW,        Compare Vector Signed Words true)
MNEM (vpcomub,           VPCOMUB,           Compare Vector Unsigned Bytes)
MNEM (vpcomud,           VPCOMUD,           Compare Vector Unsigned Doublewords)
MNEM (vpcomuq,           VPCOMUQ,           Compare Vector Unsigned Quadwords)
MNEM (vpcomuw,           VPCOMUW,           Compare Vector Unsigned Words)
MNEM (vpcomw,            VPCOMW,            Compare Vector Signed Words)
MNEM (vperm2f128,        VPERM2F128,        Permute Floating-Point 128-bit)
MNEM (vperm2i128,        VPERM2I128,        Permute Integer 128-bit)
MNEM (vpermd,            VPERMD,            Packed Permute Doubleword)
MNEM (vpermpd,           VPERMPD,           Packed Permute Double-Precision Floating-Point)
MNEM (vpermps,           VPERMPS,           Packed Permute Single-Precision Floating-Point)
MNEM (vpermq,            VPERMQ,            Packed Permute Quadword)
MNEM (vpextrb,           VPEXTRB,           Extract Packed Byte)
MNEM (vpextrd,           VPEXTRD,           Extract Packed Doubleword)
MNEM (vpextrq,           VPEXTRQ,           Extract Packed Quadword)
MNEM (vpextrw,           VPEXTRW,           Extract Packed Word)
MNEM (vphaddbd,          VPHADDBD,          Packed Horizontal Add Signed Byte to Signed Doubleword)
MNEM (vphaddbq,          VPHADDBQ,          Packed Horizontal Add Signed Byte to Signed Quadword)
MNEM (vphaddbw,          VPHADDBW,          Packed Horizontal Add Signed Byte to Signed Word)
MNEM (vphaddd,           VPHADDD,           Packed Horizontal Add Doubleword)
MNEM (vphadddq,          VPHADDDQ,          Packed Horizontal Add Signed Doubleword to Signed Quadword)
MNEM (vphaddsw,          VPHADDSW,          Packed Horizontal Add with Saturation Word)
MNEM (vphaddubd,         VPHADDUBD,         Packed Horizontal Add Unsigned Byte to Doubleword)
MNEM (vphaddubq,         VPHADDUBQ,         Packed Horizontal Add Unsigned Byte to Quadword)
MNEM (vphaddubw,         VPHADDUBW,         Packed Horizontal Add Unsigned Byte to Word)
MNEM (vphaddudq,         VPHADDUDQ,         Packed Horizontal Add Unsigned Doubleword to Quadword)
MNEM (vphadduwd,         VPHADDUWD,         Packed Horizontal Add Unsigned Word to Doubleword)
MNEM (vphadduwq,         VPHADDUWQ,         Packed Horizontal Add Unsigned Word to Quadword)
MNEM (vphaddw,           VPHADDW,           Packed Horizontal Add Word)
MNEM (vphaddwd,          VPHADDWD,          Packed Horizontal Add Signed Word to Signed Doubleword)
MNEM (vphaddwq,          VPHADDWQ,          Packed Horizontal Add Signed Word to Signed Quadword)
MNEM (vphminposuw,       VPHMINPOSUW,       Horizontal Minimum and Position)
MNEM (vphsubbw,          VPHSUBBW,          Packed Horizontal Subtract Signed Byte to Signed Word)
MNEM (vphsubd,           VPHSUBD,           Packed Horizontal Subtract Doubleword)
MNEM (vphsubdq,          VPHSUBDQ,          Packed Horizontal Subtract Signed Doubleword to Signed Quadword)
MNEM (vphsubsw,          VPHSUBSW,          Packed Horizontal Subtract with Saturation Word)
MNEM (vphsubw,           VPHSUBW,           Packed Horizontal Subtract Word)
MNEM (vphsubwd,          VPHSUBWD,          Packed Horizontal Subtract Signed Word to Signed Doubleword)
MNEM (vpinsrb,           VPINSRB,           Packed Insert Byte)
MNEM (vpinsrd,           VPINSRD,           Packed Insert Doubleword)
MNEM (vpinsrq,           VPINSRQ,           Packed Insert Quadword)
MNEM (vpinsrw,           VPINSRW,           Packed Insert Word)
MNEM (vpmacsdd,          VPMACSDD,          Packed Multiply Accumulate Signed Doubleword to Signed Doubleword)
MNEM (vpmacsdqh,         VPMACSDQH,         Packed Multiply Accumulate Signed High Doubleword to Signed Quadword)
MNEM (vpmacsdql,         VPMACSDQL,         Packed Multiply Accumulate Signed Low Doubleword to Signed Quadword)
MNEM (vpmacssdd,         VPMACSSDD,         Packed Multiply Accumulate Signed Doubleword to Signed Doubleword with Saturation)
MNEM (vpmacssdqh,        VPMACSSDQH,        Packed Multiply Accumulate Signed High Doubleword to Signed Quadword with Saturation)
MNEM (vpmacssdql,        VPMACSSDQL,        Packed Multiply Accumulate Signed Low Doubleword to Signed Quadword with Saturation)
MNEM (vpmacsswd,         VPMACSSWD,         Packed Multiply Accumulate Signed Word to Signed Doubleword with Saturation)
MNEM (vpmacssww,         VPMACSSWW,         Packed Multiply Accumulate Signed Word to Signed Word with Saturation)
MNEM (vpmacswd,          VPMACSWD,          Packed Multiply Accumulate Signed Word to Signed Doubleword)
MNEM (vpmacsww,          VPMACSWW,          Packed Multiply Accumulate Signed Word to Signed Word)
MNEM (vpmadcsswd,        VPMADCSSWD,        Packed Multiply Add and Accumulate Signed Word to Signed Doubleword with Saturation)
MNEM (vpmadcswd,         VPMADCSWD,         Packed Multiply Add and Accumulate Signed Word to Signed Doubleword)
MNEM (vpmaddubsw,        VPMADDUBSW,        Packed Multiply and Add Unsigned Byte to Signed Word)
MNEM (vpmaddwd,          VPMADDWD,          Packed Multiply Words and Add Doublewords)
MNEM (vpmaskmovd,        VPMASKMOVD,        Masked Move Packed Doubleword)
MNEM (vpmaskmovq,        VPMASKMOVQ,        Masked Move Packed Quadword)
MNEM (vpmaxsb,           VPMAXSB,           Packed Maximum Signed Bytes)
MNEM (vpmaxsd,           VPMAXSD,           Packed Maximum Signed Doublewords)
MNEM (vpmaxsw,           VPMAXSW,           Packed Maximum Signed Words)
MNEM (vpmaxub,           VPMAXUB,           Packed Maximum Unsigned Bytes)
MNEM (vpmaxud,           VPMAXUD,           Packed Maximum Unsigned Doublewords)
MNEM (vpmaxuw,           VPMAXUW,           Packed Maximum Unsigned Words)
MNEM (vpminsb,           VPMINSB,           Packed Minimum Signed Bytes)
MNEM (vpminsd,           VPMINSD,           Packed Minimum Signed Doublewords)
MNEM (vpminsw,           VPMINSW,           Packed Minimum Signed Words)
MNEM (vpminub,           VPMINUB,           Packed Minimum Unsigned Bytes)
MNEM (vpminud,           VPMINUD,           Packed Minimum Unsigned Doublewords)
MNEM (vpminuw,           VPMINUW,           Packed Minimum Unsigned Words)
MNEM (vpmovmskb,         VPMOVMSKB,         Packed Move Mask Byte)
MNEM (vpmovsxbd,         VPMOVSXBD,         Packed Move with Sign-Extension Byte to Doubleword)
MNEM (vpmovsxbq,         VPMOVSXBQ,         Packed Move with Sign-Extension Byte to Quadword)
MNEM (vpmovsxbw,         VPMOVSXBW,         Packed Move with Sign-Extension Byte to Word)
MNEM (vpmovsxdq,         VPMOVSXDQ,         Packed Move with Sign-Extension Doubleword to Quadword)
MNEM (vpmovsxwd,         VPMOVSXWD,         Packed Move with Sign-Extension Word to Doubleword)
MNEM (vpmovsxwq,         VPMOVSXWQ,         Packed Move with Sign-Extension Word to Quadword)
MNEM (vpmovzxbd,         VPMOVZXBD,         Packed Move with Zero-Extension Byte to Doubleword)
MNEM (vpmovzxbq,         VPMOVZXBQ,         Packed Move with Zero-Extension Byte to Quadword)
MNEM (vpmovzxbw,         VPMOVZXBW,         Packed Move with Zero-Extension Byte to Word)
MNEM (vpmovzxdq,         VPMOVZXDQ,         Packed Move with Zero-Extension Doubleword to Quadword)
MNEM (vpmovzxwd,         VPMOVZXWD,         Packed Move with Zero-Extension Word to Doubleword)
MNEM (vpmovzxwq,         VPMOVZXWQ,         Packed Move with Zero-Extension Word to Quadword)
MNEM (vpmuldq,           VPMULDQ,           Packed Multiply Signed Doubleword to Quadword)
MNEM (vpmulhrsw,         VPMULHRSW,         Packed Multiply High with Round and Scale Words)
MNEM (vpmulhuw,          VPMULHUW,          Packed Multiply High Unsigned Word)
MNEM (vpmulhw,           VPMULHW,           Packed Multiply High Signed Word)
MNEM (vpmulld,           VPMULLD,           Packed Multiply Low Signed Doubleword)
MNEM (vpmullw,           VPMULLW,           Packed Multiply Low Signed Word)
MNEM (vpmuludq,          VPMULUDQ,          Packed Multiply Unsigned Doubleword and Store Quadword)
MNEM (vpor,              VPOR,              Packed Logical Bitwise OR)
MNEM (vpperm,            VPPERM,            Packed Permute Bytes)
MNEM (vprotb,            VPROTB,            Packed Rotate Bytes)
MNEM (vprotd,            VPROTD,            Packed Rotate Doublewords)
MNEM (vprotq,            VPROTQ,            Packed Rotate Quadwords)
MNEM (vprotw,            VPROTW,            Packed Rotate Words)
MNEM (vpsadbw,           VPSADBW,           Packed Sum of Absolute Differences of Bytes Into a Word)
MNEM (vpshab,            VPSHAB,            Packed Shift Arithmetic Bytes)
MNEM (vpshad,            VPSHAD,            Packed Shift Arithmetic Doublewords)
MNEM (vpshaq,            VPSHAQ,            Packed Shift Arithmetic Quadwords)
MNEM (vpshaw,            VPSHAW,            Packed Shift Arithmetic Words)
MNEM (vpshlb,            VPSHLB,            Packed Shift Logical Bytes)
MNEM (vpshld,            VPSHLD,            Packed Shift Logical Doublewords)
MNEM (vpshlq,            VPSHLQ,            Packed Shift Logical Quadwords)
MNEM (vpshlw,            VPSHLW,            Packed Shift Logical Words)
MNEM (vpshufb,           VPSHUFB,           Packed Shuffle Byte)
MNEM (vpshufd,           VPSHUFD,           Packed Shuffle Doublewords)
MNEM (vpshufhw,          VPSHUFHW,          Packed Shuffle High Words)
MNEM (vpshuflw,          VPSHUFLW,          Packed Shuffle Low Words)
MNEM (vpsignb,           VPSIGNB,           Packed Sign Byte)
MNEM (vpsignd,           VPSIGND,           Packed Sign Doubleword)
MNEM (vpsignw,           VPSIGNW,           Packed Sign Word)
MNEM (vpslld,            VPSLLD,            Packed Shift Left Logical Doublewords)
MNEM (vpslldq,           VPSLLDQ,           Packed Shift Left Logical Double Quadword)
MNEM (vpsllq,            VPSLLQ,            Packed Shift Left Logical Quadwords)
MNEM (vpsllvd,           VPSLLVD,           Variable Shift Left Logical Doublewords)
MNEM (vpsllvq,           VPSLLVQ,           Variable Shift Left Logical Quadwords)
MNEM (vpsllw,            VPSLLW,            Packed Shift Left Logical Words)
MNEM (vpsrad,            VPSRAD,            Packed Shift Right Arithmetic Doublewords)
MNEM (vpsravd,           VPSRAVD,           Variable Shift Right Arithmetic Doublewords)
MNEM (vpsraw,            VPSRAW,            Packed Shift Right Arithmetic Words)
MNEM (vpsrld,            VPSRLD,            Packed Shift Right Logical Doublewords)
MNEM (vpsrldq,           VPSRLDQ,           Packed Shift Right Logical Double Quadword)
MNEM (vpsrlq,            VPSRLQ,            Packed Shift Right Logical Quadwords)
MNEM (vpsrlvd,           VPSRLVD,           Variable Shift Right Logical Doublewords)
MNEM (vpsrlvq,           VPSRLVQ,           Variable Shift Right Logical Quadwords)
MNEM (vpsrlw,            VPSRLW,            Packed Shift Right Logical Words)
MNEM (vpsubb,            VPSUBB,            Packed Subtract Bytes)
MNEM (vpsubd,            VPSUBD,            Packed Subtract Doublewords)
MNEM (vpsubq,            VPSUBQ,            Packed Subtract Quadwords)
MNEM (vpsubsb,           VPSUBSB,           Packed Subtract Signed with Saturation Bytes)
MNEM (vpsubsw,           VPSUBSW,           Packed Subtract Signed with Saturation Words)
MNEM (vpsubusb,          VPSUBUSB,          Packed Subtract Unsigned with Saturation Bytes)
MNEM (vpsubusw,          VPSUBUSW,          Packed Subtract Unsigned with Saturation Words)
MNEM (vpsubw,            VPSUBW,            Packed Subtract Words)
MNEM (vptest,            VPTEST,            Packed Bit Test)
MNEM (vpunpckhbw,        VPUNPCKHBW,        Unpack and Interleave High Bytes)
MNEM (vpunpckhdq,        VPUNPCKHDQ,        Unpack and Interleave High Doublewords)
MNEM (vpunpckhqdq,       VPUNPCKHQDQ,       Unpack and Interleave High Quadwords)
MNEM (vpunpckhwd,        VPUNPCKHWD,        Unpack and Interleave High Words)
MNEM (vpunpcklbw,        VPUNPCKLBW,        Unpack and Interleave Low Bytes)
MNEM (vpunpckldq,        VPUNPCKLDQ,        Unpack and Interleave Low Doublewords)
MNEM (vpunpcklqdq,       VPUNPCKLQDQ,       Unpack and Interleave Low Quadwords)
MNEM (vpunpcklwd,        VPUNPCKLWD,        Unpack and Interleave Low Words)
MNEM (vpxor,             VPXOR,             Packed Logical Bitwise Exclusive OR)
MNEM (vrcpps,            VRCPPS,            Reciprocal Packed Single-Precision Floating-Point)
MNEM (vrcpss,            VRCPSS,            Reciprocal Scalar Single-Precision Floating-Point)
MNEM (vroundpd,          VROUNDPD,          Round Packed Double-Precision Floating-Point)
MNEM (vroundps,          VROUNDPS,          Round Packed Single-Precision Floating-Point)
MNEM (vroundsd,          VROUNDSD,          Round Scalar Double-Precision Floating-Point)
MNEM (vroundss,          VROUNDSS,          Round Scalar Single-Precision Floating-Point)
MNEM (vrsqrtps,          VRSQRTPS,          Reciprocal Square Root Packed Single-Precision Floating-Point)
MNEM (vrsqrtss,          VRSQRTSS,          Reciprocal Square Root Scalar Single-Precision Floating-Point)
MNEM (vshufpd,           VSHUFPD,           Shuffle Packed Double-Precision Floating-Point)
MNEM (vshufps,           VSHUFPS,           Shuffle Packed Single-Precision Floating-Point)
MNEM (vsqrtpd,           VSQRTPD,           Square Root Packed Double-Precision Floating-Point)
MNEM (vsqrtps,           VSQRTPS,           Square Root Packed Single-Precision Floating-Point)
MNEM (vsqrtsd,           VSQRTSD,           Square Root Scalar Double-Precision Floating-Point)
MNEM (vsqrtss,           VSQRTSS,           Square Root Scalar Single-Precision Floating-Point)
MNEM (vstmxcsr,          VSTMXCSR,          Store MXCSR Control/Status Register)
MNEM (vsubpd,            VSUBPD,            Subtract Packed Double-Precision Floating-Point)
MNEM (vsubps,            VSUBPS,            Subtract Packed Single-Precision Floating-Point)
MNEM (vsubsd,            VSUBSD,            Subtract Scalar Double-Precision Floating-Point)
MNEM (vsubss,            VSUBSS,            Subtract Scalar Single-Precision Floating-Point)
MNEM (vtestpd,           VTESTPD,           Packed Bit Test)
MNEM (vtestps,           VTESTPS,           Packed Bit Test)
MNEM (vucomisd,          VUCOMISD,          Unordered Compare Scalar Double-Precision Floating-Point)
MNEM (vucomiss,          VUCOMISS,          Unordered Compare Scalar Single-Precision Floating-Point)
MNEM (vunpckhpd,         VUNPCKHPD,         Unpack High Double-Precision Floating-Point)
MNEM (vunpckhps,         VUNPCKHPS,         Unpack High Single-Precision Floating-Point)
MNEM (vunpcklpd,         VUNPCKLPD,         Unpack Low Double-Precision Floating-Point)
MNEM (vunpcklps,         VUNPCKLPS,         Unpack Low Single-Precision Floating-Point)
MNEM (vxorpd,            VXORPD,            Logical Bitwise Exclusive OR Packed Double-Precision Floating-Point)
MNEM (vxorps,            VXORPS,            Logical Bitwise Exclusive OR Packed Single-Precision Floating-Point)
MNEM (vzeroall,          VZEROALL,          Zero All YMM Registers)
MNEM (vzeroupper,        VZEROUPPER,        Zero All YMM Registers Upper)
MNEM (wbinvd,            WBINVD,            Writeback and Invalidate Caches)
MNEM (wbnoinvd,          WBNOINVD,          Writeback With No Invalidate)
MNEM (wrfsbase,          WRFSBASE,          Write FS.base)
MNEM (wrgsbase,          WRGSBASE,          Write GS.base)
MNEM (wrmsr,             WRMSR,             Write to Model-Specific Register)
MNEM (wrpkru,            WRPKRU,            Write Protection Key Rights)
MNEM (wrssd,             WRSSD,             Write to Shadow Stack)
MNEM (wrssq,             WRSSQ,             Write to Shadow Stack)
MNEM (wrussd,            WRUSSD,            Write to User Shadow Stack)
MNEM (wrussq,            WRUSSQ,            Write to User Shadow Stack)
MNEM (xadd,              XADD,              Exchange and Add)
MNEM (xchg,              XCHG,              Exchange)
MNEM (xgetbv,            XGETBV,            Get Extended Control Register Value)
MNEM (xlatb,             XLATB,             Translate Table Index)
MNEM (xor,               XOR,               Logical Exclusive OR)
MNEM (xorpd,             XORPD,             Logical Bitwise Exclusive OR Packed Double-Precision Floating-Point)
MNEM (xorps,             XORPS,             Logical Bitwise Exclusive OR Packed Single-Precision Floating-Point)
MNEM (xrstor,            XRSTOR,            Restore Extended States)
MNEM (xrstors,           XRSTORS,           Restore Extended States Supervisor)
MNEM (xsave,             XSAVE,             Save Extended States)
MNEM (xsavec,            XSAVEC,            Save Extended States in Compacted Form)
MNEM (xsaveopt,          XSAVEOPT,          Save Extended States Performance Optimized)
MNEM (xsaves,            XSAVES,            Save Extended States Supervisor)
MNEM (xsetbv,            XSETBV,            Set Extended Control Register Value)

// General-Purpose Instruction Reference

INSTR (AAA,               none,       none,       none,       none,       0,       0x37,      no,  no,  0,     I64)

INSTR (AAD,               imm8,       none,       none,       none,       0,       0xd5,      ib,  no,  0,     I64)
INSTR (AAD,               none,       none,       none,       none,       0,       0xd5,      no,  no,  0x0a,  I64 | SFX)

INSTR (AAM,               imm8,       none,       none,       none,       0,       0xd4,      ib,  no,  0,     I64)
INSTR (AAM,               none,       none,       none,       none,       0,       0xd4,      no,  no,  0x0a,  I64 | SFX)

INSTR (AAS,               none,       none,       none,       none,       0,       0x3f,      no,  no,  0,     I64)

INSTR (ADC,               al,         imm8,       none,       none,       0,       0x14,      ib,  no,  0,     0)
INSTR (ADC,               regmem8,    imm8,       none,       none,       0,       0x80,      r2,  ib,  0,     PLOCK)
INSTR (ADC,               regmem8,    imm8,       none,       none,       0,       0x82,      r2,  ib,  0,     I64 | PLOCK)
INSTR (ADC,               regmem16,   simm8,      none,       none,       0,       0x83,      r2,  ib,  0,     O16 | PLOCK)
INSTR (ADC,               regmem32,   simm8,      none,       none,       0,       0x83,      r2,  ib,  0,     O32 | PLOCK)
INSTR (ADC,               regmem64,   simm8,      none,       none,       0,       0x83,      r2,  ib,  0,     O64 | PLOCK)
INSTR (ADC,               ax,         imm16,      none,       none,       0,       0x15,      iw,  no,  0,     O16)
INSTR (ADC,               regmem16,   imm16,      none,       none,       0,       0x81,      r2,  iw,  0,     O16 | PLOCK)
INSTR (ADC,               eax,        imm32,      none,       none,       0,       0x15,      id,  no,  0,     O32)
INSTR (ADC,               regmem32,   imm32,      none,       none,       0,       0x81,      r2,  id,  0,     O32 | PLOCK)
INSTR (ADC,               rax,        simm32,     none,       none,       0,       0x15,      id,  no,  0,     O64)
INSTR (ADC,               regmem64,   simm32,     none,       none,       0,       0x81,      r2,  id,  0,     O64 | PLOCK)
INSTR (ADC,               regmem8,    reg8,       none,       none,       0,       0x10,      r,   no,  0,     PLOCK)
INSTR (ADC,               regmem16,   reg16,      none,       none,       0,       0x11,      r,   no,  0,     O16 | PLOCK)
INSTR (ADC,               regmem32,   reg32,      none,       none,       0,       0x11,      r,   no,  0,     O32 | PLOCK)
INSTR (ADC,               regmem64,   reg64,      none,       none,       0,       0x11,      r,   no,  0,     O64 | PLOCK)
INSTR (ADC,               reg8,       regmem8,    none,       none,       0,       0x12,      r,   no,  0,     0)
INSTR (ADC,               reg16,      regmem16,   none,       none,       0,       0x13,      r,   no,  0,     O16)
INSTR (ADC,               reg32,      regmem32,   none,       none,       0,       0x13,      r,   no,  0,     O32)
INSTR (ADC,               reg64,      regmem64,   none,       none,       0,       0x13,      r,   no,  0,     O64)

INSTR (ADCX,              reg32,      regmem32,   none,       none,       0,       0x0f38f6,  r,   no,  0,     O16 | O32 | P66)
INSTR (ADCX,              reg64,      regmem64,   none,       none,       0,       0x0f38f6,  r,   no,  0,     O64 | P66)

INSTR (ADD,               al,         imm8,       none,       none,       0,       0x04,      ib,  no,  0,     0)
INSTR (ADD,               regmem8,    imm8,       none,       none,       0,       0x80,      r0,  ib,  0,     PLOCK)
INSTR (ADD,               regmem8,    imm8,       none,       none,       0,       0x82,      r0,  ib,  0,     I64 | PLOCK)
INSTR (ADD,               regmem16,   simm8,      none,       none,       0,       0x83,      r0,  ib,  0,     O16 | PLOCK)
INSTR (ADD,               regmem32,   simm8,      none,       none,       0,       0x83,      r0,  ib,  0,     O32 | PLOCK)
INSTR (ADD,               regmem64,   simm8,      none,       none,       0,       0x83,      r0,  ib,  0,     O64 | PLOCK)
INSTR (ADD,               ax,         imm16,      none,       none,       0,       0x05,      iw,  no,  0,     O16)
INSTR (ADD,               regmem16,   imm16,      none,       none,       0,       0x81,      r0,  iw,  0,     O16 | PLOCK)
INSTR (ADD,               eax,        imm32,      none,       none,       0,       0x05,      id,  no,  0,     O32)
INSTR (ADD,               regmem32,   imm32,      none,       none,       0,       0x81,      r0,  id,  0,     O32 | PLOCK)
INSTR (ADD,               rax,        simm32,     none,       none,       0,       0x05,      id,  no,  0,     O64)
INSTR (ADD,               regmem64,   simm32,     none,       none,       0,       0x81,      r0,  id,  0,     O64 | PLOCK)
INSTR (ADD,               regmem8,    reg8,       none,       none,       0,       0x00,      r,   no,  0,     PLOCK)
INSTR (ADD,               regmem16,   reg16,      none,       none,       0,       0x01,      r,   no,  0,     O16 | PLOCK)
INSTR (ADD,               regmem32,   reg32,      none,       none,       0,       0x01,      r,   no,  0,     O32 | PLOCK)
INSTR (ADD,               regmem64,   reg64,      none,       none,       0,       0x01,      r,   no,  0,     O64 | PLOCK)
INSTR (ADD,               reg8,       regmem8,    none,       none,       0,       0x02,      r,   no,  0,     0)
INSTR (ADD,               reg16,      regmem16,   none,       none,       0,       0x03,      r,   no,  0,     O16)
INSTR (ADD,               reg32,      regmem32,   none,       none,       0,       0x03,      r,   no,  0,     O32)
INSTR (ADD,               reg64,      regmem64,   none,       none,       0,       0x03,      r,   no,  0,     O64)

INSTR (ADOX,              reg32,      regmem32,   none,       none,       0,       0x0f38f6,  r,   no,  0,     O16 | O32 | PF3)
INSTR (ADOX,              reg64,      regmem64,   none,       none,       0,       0x0f38f6,  r,   no,  0,     O64 | PF3)

INSTR (AND,               al,         imm8,       none,       none,       0,       0x24,      ib,  no,  0,     0)
INSTR (AND,               regmem8,    imm8,       none,       none,       0,       0x80,      r4,  ib,  0,     PLOCK)
INSTR (AND,               regmem8,    imm8,       none,       none,       0,       0x82,      r4,  ib,  0,     I64 | PLOCK)
INSTR (AND,               regmem16,   simm8,      none,       none,       0,       0x83,      r4,  ib,  0,     O16 | PLOCK)
INSTR (AND,               regmem32,   simm8,      none,       none,       0,       0x83,      r4,  ib,  0,     O32 | PLOCK)
INSTR (AND,               regmem64,   simm8,      none,       none,       0,       0x83,      r4,  ib,  0,     O64 | PLOCK)
INSTR (AND,               ax,         imm16,      none,       none,       0,       0x25,      iw,  no,  0,     O16)
INSTR (AND,               regmem16,   imm16,      none,       none,       0,       0x81,      r4,  iw,  0,     O16 | PLOCK)
INSTR (AND,               eax,        imm32,      none,       none,       0,       0x25,      id,  no,  0,     O32)
INSTR (AND,               regmem32,   imm32,      none,       none,       0,       0x81,      r4,  id,  0,     O32 | PLOCK)
INSTR (AND,               rax,        simm32,     none,       none,       0,       0x25,      id,  no,  0,     O64)
INSTR (AND,               regmem64,   simm32,     none,       none,       0,       0x81,      r4,  id,  0,     O64 | PLOCK)
INSTR (AND,               regmem8,    reg8,       none,       none,       0,       0x20,      r,   no,  0,     PLOCK)
INSTR (AND,               regmem16,   reg16,      none,       none,       0,       0x21,      r,   no,  0,     O16 | PLOCK)
INSTR (AND,               regmem32,   reg32,      none,       none,       0,       0x21,      r,   no,  0,     O32 | PLOCK)
INSTR (AND,               regmem64,   reg64,      none,       none,       0,       0x21,      r,   no,  0,     O64 | PLOCK)
INSTR (AND,               reg8,       regmem8,    none,       none,       0,       0x22,      r,   no,  0,     0)
INSTR (AND,               reg16,      regmem16,   none,       none,       0,       0x23,      r,   no,  0,     O16)
INSTR (AND,               reg32,      regmem32,   none,       none,       0,       0x23,      r,   no,  0,     O32)
INSTR (AND,               reg64,      regmem64,   none,       none,       0,       0x23,      r,   no,  0,     O64)

INSTR (BOUND,             reg16,      mem,        none,       none,       0,       0x62,      r,   no,  0,     O16 | I64)
INSTR (BOUND,             reg32,      mem,        none,       none,       0,       0x62,      r,   no,  0,     O32 | I64)

INSTR (BSF,               reg16,      regmem16,   none,       none,       0,       0x0fbc,    r,   no,  0,     O16)
INSTR (BSF,               reg32,      regmem32,   none,       none,       0,       0x0fbc,    r,   no,  0,     O32)
INSTR (BSF,               reg64,      regmem64,   none,       none,       0,       0x0fbc,    r,   no,  0,     O64)

INSTR (LZCNT,             reg16,      regmem16,   none,       none,       0,       0x0fbd,    r,   no,  0,     O16 | PF3)
INSTR (LZCNT,             reg32,      regmem32,   none,       none,       0,       0x0fbd,    r,   no,  0,     O32 | PF3)
INSTR (LZCNT,             reg64,      regmem64,   none,       none,       0,       0x0fbd,    r,   no,  0,     O64 | PF3)

INSTR (BSR,               reg16,      regmem16,   none,       none,       0,       0x0fbd,    r,   no,  0,     O16)
INSTR (BSR,               reg32,      regmem32,   none,       none,       0,       0x0fbd,    r,   no,  0,     O32)
INSTR (BSR,               reg64,      regmem64,   none,       none,       0,       0x0fbd,    r,   no,  0,     O64)

INSTR (BSWAP,             reg32,      none,       none,       none,       0,       0x0fc8,    rv,  no,  0,     O32)
INSTR (BSWAP,             reg64,      none,       none,       none,       0,       0x0fc8,    rv,  no,  0,     O64)

INSTR (BT,                regmem16,   reg16,      none,       none,       0,       0x0fa3,    r,   no,  0,     O16)
INSTR (BT,                regmem32,   reg32,      none,       none,       0,       0x0fa3,    r,   no,  0,     O32)
INSTR (BT,                regmem64,   reg64,      none,       none,       0,       0x0fa3,    r,   no,  0,     O64)
INSTR (BT,                regmem16,   imm8,       none,       none,       0,       0x0fba,    r4,  ib,  0,     O16)
INSTR (BT,                regmem32,   imm8,       none,       none,       0,       0x0fba,    r4,  ib,  0,     O32)
INSTR (BT,                regmem64,   imm8,       none,       none,       0,       0x0fba,    r4,  ib,  0,     O64)

INSTR (BTC,               regmem16,   reg16,      none,       none,       0,       0x0fbb,    r,   no,  0,     O16 | PLOCK)
INSTR (BTC,               regmem32,   reg32,      none,       none,       0,       0x0fbb,    r,   no,  0,     O32 | PLOCK)
INSTR (BTC,               regmem64,   reg64,      none,       none,       0,       0x0fbb,    r,   no,  0,     O64 | PLOCK)
INSTR (BTC,               regmem16,   imm8,       none,       none,       0,       0x0fba,    r7,  ib,  0,     O16 | PLOCK)
INSTR (BTC,               regmem32,   imm8,       none,       none,       0,       0x0fba,    r7,  ib,  0,     O32 | PLOCK)
INSTR (BTC,               regmem64,   imm8,       none,       none,       0,       0x0fba,    r7,  ib,  0,     O64 | PLOCK)

INSTR (BTR,               regmem16,   reg16,      none,       none,       0,       0x0fb3,    r,   no,  0,     O16 | PLOCK)
INSTR (BTR,               regmem32,   reg32,      none,       none,       0,       0x0fb3,    r,   no,  0,     O32 | PLOCK)
INSTR (BTR,               regmem64,   reg64,      none,       none,       0,       0x0fb3,    r,   no,  0,     O64 | PLOCK)
INSTR (BTR,               regmem16,   imm8,       none,       none,       0,       0x0fba,    r6,  ib,  0,     O16 | PLOCK)
INSTR (BTR,               regmem32,   imm8,       none,       none,       0,       0x0fba,    r6,  ib,  0,     O32 | PLOCK)
INSTR (BTR,               regmem64,   imm8,       none,       none,       0,       0x0fba,    r6,  ib,  0,     O64 | PLOCK)

INSTR (BTS,               regmem16,   reg16,      none,       none,       0,       0x0fab,    r,   no,  0,     O16 | PLOCK)
INSTR (BTS,               regmem32,   reg32,      none,       none,       0,       0x0fab,    r,   no,  0,     O32 | PLOCK)
INSTR (BTS,               regmem64,   reg64,      none,       none,       0,       0x0fab,    r,   no,  0,     O64 | PLOCK)
INSTR (BTS,               regmem16,   imm8,       none,       none,       0,       0x0fba,    r5,  ib,  0,     O16 | PLOCK)
INSTR (BTS,               regmem32,   imm8,       none,       none,       0,       0x0fba,    r5,  ib,  0,     O32 | PLOCK)
INSTR (BTS,               regmem64,   imm8,       none,       none,       0,       0x0fba,    r5,  ib,  0,     O64 | PLOCK)

INSTR (BZHI,              reg32,      regmem32,   reg32vvvv,  none,       0x0200,  0xf5,      r,   no,  0,     PVEX)
INSTR (BZHI,              reg64,      regmem64,   reg64vvvv,  none,       0x0280,  0xf5,      r,   no,  0,     PVEX)

INSTR (CALL,              rel16off,   none,       none,       none,       0,       0xe8,      cw,  no,  0,     O16)
INSTR (CALL,              rel32off,   none,       none,       none,       0,       0xe8,      cd,  no,  0,     O32)
INSTR (CALL,              regmem16,   none,       none,       none,       0,       0xff,      r2,  no,  0,     O16)
INSTR (CALL,              regmem32,   none,       none,       none,       0,       0xff,      r2,  no,  0,     O32 | I64)
INSTR (CALL,              regmem64,   none,       none,       none,       0,       0xff,      r2,  no,  0,     O64 | D64)

INSTR (CALLFAR,           imm16,      rel16off,   none,       none,       0,       0x9a,      cw,  iw,  0,     O16 | I64 | FAR)
INSTR (CALLFAR,           imm16,      rel32off,   none,       none,       0,       0x9a,      cd,  iw,  0,     O32 | I64 | FAR)
INSTR (CALLFAR,           mem16,      none,       none,       none,       0,       0xff,      r3,  no,  0,     O16)
INSTR (CALLFAR,           mem32,      none,       none,       none,       0,       0xff,      r3,  no,  0,     O32)

INSTR (CBW,               none,       none,       none,       none,       0,       0x98,      no,  no,  0,     O16)

INSTR (CWDE,              none,       none,       none,       none,       0,       0x98,      no,  no,  0,     O32)

INSTR (CDQE,              none,       none,       none,       none,       0,       0x98,      no,  no,  0,     O64)

INSTR (CWD,               none,       none,       none,       none,       0,       0x99,      no,  no,  0,     O16)

INSTR (CDQ,               none,       none,       none,       none,       0,       0x99,      no,  no,  0,     O32)

INSTR (CQO,               none,       none,       none,       none,       0,       0x99,      no,  no,  0,     O64)

INSTR (CLAC,              none,       none,       none,       none,       0,       0x0f01,    no,  no,  0xca,  SFX)

INSTR (CLC,               none,       none,       none,       none,       0,       0xf8,      no,  no,  0,     0)

INSTR (CLD,               none,       none,       none,       none,       0,       0xfc,      no,  no,  0,     0)

INSTR (XSAVE,             mem,        none,       none,       none,       0,       0x0fae,    r4,  no,  0,     0)

INSTR (FXSAVE,            mem,        none,       none,       none,       0,       0x0fae,    r0,  no,  0,     0)

INSTR (XSAVEC,            mem,        none,       none,       none,       0,       0x0fc7,    r4,  no,  0,     0)

INSTR (CLWB,              mem,        none,       none,       none,       0,       0x0fae,    r6,  no,  0,     P66)

INSTR (XSAVEOPT,          mem,        none,       none,       none,       0,       0x0fae,    r6,  no,  0,     0)

INSTR (XSAVES,            mem,        none,       none,       none,       0,       0x0fc7,    r5,  no,  0,     0)

INSTR (XRSTOR,            mem,        none,       none,       none,       0,       0x0fae,    r5,  no,  0,     0)

INSTR (FXRSTOR,           mem,        none,       none,       none,       0,       0x0fae,    r1,  no,  0,     0)

INSTR (XRSTORS,           mem,        none,       none,       none,       0,       0x0fc7,    r3,  no,  0,     0)

INSTR (LDMXCSR,           mem32,      none,       none,       none,       0,       0x0fae,    r2,  no,  0,     0)

INSTR (VLDMXCSR,          mem32,      none,       none,       none,       0x01e8,  0xae,      r2,  no,  0,     PVEX)

INSTR (STMXCSR,           mem32,      none,       none,       none,       0,       0x0fae,    r3,  no,  0,     0)

INSTR (VSTMXCSR,          mem32,      none,       none,       none,       0x01e8,  0xae,      r3,  no,  0,     PVEX)

INSTR (CLFLUSHOPT,        mem,        none,       none,       none,       0,       0x0fae,    r7,  no,  0,     P66)

INSTR (CLFLUSH,           mem,        none,       none,       none,       0,       0x0fae,    r7,  no,  0,     0)

INSTR (CMC,               none,       none,       none,       none,       0,       0xf5,      no,  no,  0,     0)

INSTR (CMOVO,             reg16,      regmem16,   none,       none,       0,       0x0f40,    r,   no,  0,     O16)
INSTR (CMOVO,             reg32,      regmem32,   none,       none,       0,       0x0f40,    r,   no,  0,     O32)
INSTR (CMOVO,             reg64,      regmem64,   none,       none,       0,       0x0f40,    r,   no,  0,     O64)

INSTR (CMOVNO,            reg16,      regmem16,   none,       none,       0,       0x0f41,    r,   no,  0,     O16)
INSTR (CMOVNO,            reg32,      regmem32,   none,       none,       0,       0x0f41,    r,   no,  0,     O32)
INSTR (CMOVNO,            reg64,      regmem64,   none,       none,       0,       0x0f41,    r,   no,  0,     O64)

INSTR (CMOVC,             reg16,      regmem16,   none,       none,       0,       0x0f42,    r,   no,  0,     O16)
INSTR (CMOVC,             reg32,      regmem32,   none,       none,       0,       0x0f42,    r,   no,  0,     O32)
INSTR (CMOVC,             reg64,      regmem64,   none,       none,       0,       0x0f42,    r,   no,  0,     O64)

INSTR (CMOVB,             reg16,      regmem16,   none,       none,       0,       0x0f42,    r,   no,  0,     O16)
INSTR (CMOVB,             reg32,      regmem32,   none,       none,       0,       0x0f42,    r,   no,  0,     O32)
INSTR (CMOVB,             reg64,      regmem64,   none,       none,       0,       0x0f42,    r,   no,  0,     O64)

INSTR (CMOVNAE,           reg16,      regmem16,   none,       none,       0,       0x0f42,    r,   no,  0,     O16)
INSTR (CMOVNAE,           reg32,      regmem32,   none,       none,       0,       0x0f42,    r,   no,  0,     O32)
INSTR (CMOVNAE,           reg64,      regmem64,   none,       none,       0,       0x0f42,    r,   no,  0,     O64)

INSTR (CMOVNC,            reg16,      regmem16,   none,       none,       0,       0x0f43,    r,   no,  0,     O16)
INSTR (CMOVNC,            reg32,      regmem32,   none,       none,       0,       0x0f43,    r,   no,  0,     O32)
INSTR (CMOVNC,            reg64,      regmem64,   none,       none,       0,       0x0f43,    r,   no,  0,     O64)

INSTR (CMOVNB,            reg16,      regmem16,   none,       none,       0,       0x0f43,    r,   no,  0,     O16)
INSTR (CMOVNB,            reg32,      regmem32,   none,       none,       0,       0x0f43,    r,   no,  0,     O32)
INSTR (CMOVNB,            reg64,      regmem64,   none,       none,       0,       0x0f43,    r,   no,  0,     O64)

INSTR (CMOVAE,            reg16,      regmem16,   none,       none,       0,       0x0f43,    r,   no,  0,     O16)
INSTR (CMOVAE,            reg32,      regmem32,   none,       none,       0,       0x0f43,    r,   no,  0,     O32)
INSTR (CMOVAE,            reg64,      regmem64,   none,       none,       0,       0x0f43,    r,   no,  0,     O64)

INSTR (CMOVZ,             reg16,      regmem16,   none,       none,       0,       0x0f44,    r,   no,  0,     O16)
INSTR (CMOVZ,             reg32,      regmem32,   none,       none,       0,       0x0f44,    r,   no,  0,     O32)
INSTR (CMOVZ,             reg64,      regmem64,   none,       none,       0,       0x0f44,    r,   no,  0,     O64)

INSTR (CMOVE,             reg16,      regmem16,   none,       none,       0,       0x0f44,    r,   no,  0,     O16)
INSTR (CMOVE,             reg32,      regmem32,   none,       none,       0,       0x0f44,    r,   no,  0,     O32)
INSTR (CMOVE,             reg64,      regmem64,   none,       none,       0,       0x0f44,    r,   no,  0,     O64)

INSTR (CMOVNZ,            reg16,      regmem16,   none,       none,       0,       0x0f45,    r,   no,  0,     O16)
INSTR (CMOVNZ,            reg32,      regmem32,   none,       none,       0,       0x0f45,    r,   no,  0,     O32)
INSTR (CMOVNZ,            reg64,      regmem64,   none,       none,       0,       0x0f45,    r,   no,  0,     O64)

INSTR (CMOVNE,            reg16,      regmem16,   none,       none,       0,       0x0f45,    r,   no,  0,     O16)
INSTR (CMOVNE,            reg32,      regmem32,   none,       none,       0,       0x0f45,    r,   no,  0,     O32)
INSTR (CMOVNE,            reg64,      regmem64,   none,       none,       0,       0x0f45,    r,   no,  0,     O64)

INSTR (CMOVNA,            reg16,      regmem16,   none,       none,       0,       0x0f46,    r,   no,  0,     O16)
INSTR (CMOVNA,            reg32,      regmem32,   none,       none,       0,       0x0f46,    r,   no,  0,     O32)
INSTR (CMOVNA,            reg64,      regmem64,   none,       none,       0,       0x0f46,    r,   no,  0,     O64)

INSTR (CMOVBE,            reg16,      regmem16,   none,       none,       0,       0x0f46,    r,   no,  0,     O16)
INSTR (CMOVBE,            reg32,      regmem32,   none,       none,       0,       0x0f46,    r,   no,  0,     O32)
INSTR (CMOVBE,            reg64,      regmem64,   none,       none,       0,       0x0f46,    r,   no,  0,     O64)

INSTR (CMOVA,             reg16,      regmem16,   none,       none,       0,       0x0f47,    r,   no,  0,     O16)
INSTR (CMOVA,             reg32,      regmem32,   none,       none,       0,       0x0f47,    r,   no,  0,     O32)
INSTR (CMOVA,             reg64,      regmem64,   none,       none,       0,       0x0f47,    r,   no,  0,     O64)

INSTR (CMOVNBE,           reg16,      regmem16,   none,       none,       0,       0x0f47,    r,   no,  0,     O16)
INSTR (CMOVNBE,           reg32,      regmem32,   none,       none,       0,       0x0f47,    r,   no,  0,     O32)
INSTR (CMOVNBE,           reg64,      regmem64,   none,       none,       0,       0x0f47,    r,   no,  0,     O64)

INSTR (CMOVS,             reg16,      regmem16,   none,       none,       0,       0x0f48,    r,   no,  0,     O16)
INSTR (CMOVS,             reg32,      regmem32,   none,       none,       0,       0x0f48,    r,   no,  0,     O32)
INSTR (CMOVS,             reg64,      regmem64,   none,       none,       0,       0x0f48,    r,   no,  0,     O64)

INSTR (CMOVNS,            reg16,      regmem16,   none,       none,       0,       0x0f49,    r,   no,  0,     O16)
INSTR (CMOVNS,            reg32,      regmem32,   none,       none,       0,       0x0f49,    r,   no,  0,     O32)
INSTR (CMOVNS,            reg64,      regmem64,   none,       none,       0,       0x0f49,    r,   no,  0,     O64)

INSTR (CMOVPE,            reg16,      regmem16,   none,       none,       0,       0x0f4a,    r,   no,  0,     O16)
INSTR (CMOVPE,            reg32,      regmem32,   none,       none,       0,       0x0f4a,    r,   no,  0,     O32)
INSTR (CMOVPE,            reg64,      regmem64,   none,       none,       0,       0x0f4a,    r,   no,  0,     O64)

INSTR (CMOVP,             reg16,      regmem16,   none,       none,       0,       0x0f4a,    r,   no,  0,     O16)
INSTR (CMOVP,             reg32,      regmem32,   none,       none,       0,       0x0f4a,    r,   no,  0,     O32)
INSTR (CMOVP,             reg64,      regmem64,   none,       none,       0,       0x0f4a,    r,   no,  0,     O64)

INSTR (CMOVPO,            reg16,      regmem16,   none,       none,       0,       0x0f4b,    r,   no,  0,     O16)
INSTR (CMOVPO,            reg32,      regmem32,   none,       none,       0,       0x0f4b,    r,   no,  0,     O32)
INSTR (CMOVPO,            reg64,      regmem64,   none,       none,       0,       0x0f4b,    r,   no,  0,     O64)

INSTR (CMOVNP,            reg16,      regmem16,   none,       none,       0,       0x0f4b,    r,   no,  0,     O16)
INSTR (CMOVNP,            reg32,      regmem32,   none,       none,       0,       0x0f4b,    r,   no,  0,     O32)
INSTR (CMOVNP,            reg64,      regmem64,   none,       none,       0,       0x0f4b,    r,   no,  0,     O64)

INSTR (CMOVL,             reg16,      regmem16,   none,       none,       0,       0x0f4c,    r,   no,  0,     O16)
INSTR (CMOVL,             reg32,      regmem32,   none,       none,       0,       0x0f4c,    r,   no,  0,     O32)
INSTR (CMOVL,             reg64,      regmem64,   none,       none,       0,       0x0f4c,    r,   no,  0,     O64)

INSTR (CMOVNGE,           reg16,      regmem16,   none,       none,       0,       0x0f4c,    r,   no,  0,     O16)
INSTR (CMOVNGE,           reg32,      regmem32,   none,       none,       0,       0x0f4c,    r,   no,  0,     O32)
INSTR (CMOVNGE,           reg64,      regmem64,   none,       none,       0,       0x0f4c,    r,   no,  0,     O64)

INSTR (CMOVNL,            reg16,      regmem16,   none,       none,       0,       0x0f4d,    r,   no,  0,     O16)
INSTR (CMOVNL,            reg32,      regmem32,   none,       none,       0,       0x0f4d,    r,   no,  0,     O32)
INSTR (CMOVNL,            reg64,      regmem64,   none,       none,       0,       0x0f4d,    r,   no,  0,     O64)

INSTR (CMOVGE,            reg16,      regmem16,   none,       none,       0,       0x0f4d,    r,   no,  0,     O16)
INSTR (CMOVGE,            reg32,      regmem32,   none,       none,       0,       0x0f4d,    r,   no,  0,     O32)
INSTR (CMOVGE,            reg64,      regmem64,   none,       none,       0,       0x0f4d,    r,   no,  0,     O64)

INSTR (CMOVNG,            reg16,      regmem16,   none,       none,       0,       0x0f4e,    r,   no,  0,     O16)
INSTR (CMOVNG,            reg32,      regmem32,   none,       none,       0,       0x0f4e,    r,   no,  0,     O32)
INSTR (CMOVNG,            reg64,      regmem64,   none,       none,       0,       0x0f4e,    r,   no,  0,     O64)

INSTR (CMOVLE,            reg16,      regmem16,   none,       none,       0,       0x0f4e,    r,   no,  0,     O16)
INSTR (CMOVLE,            reg32,      regmem32,   none,       none,       0,       0x0f4e,    r,   no,  0,     O32)
INSTR (CMOVLE,            reg64,      regmem64,   none,       none,       0,       0x0f4e,    r,   no,  0,     O64)

INSTR (CMOVG,             reg16,      regmem16,   none,       none,       0,       0x0f4f,    r,   no,  0,     O16)
INSTR (CMOVG,             reg32,      regmem32,   none,       none,       0,       0x0f4f,    r,   no,  0,     O32)
INSTR (CMOVG,             reg64,      regmem64,   none,       none,       0,       0x0f4f,    r,   no,  0,     O64)

INSTR (CMOVNLE,           reg16,      regmem16,   none,       none,       0,       0x0f4f,    r,   no,  0,     O16)
INSTR (CMOVNLE,           reg32,      regmem32,   none,       none,       0,       0x0f4f,    r,   no,  0,     O32)
INSTR (CMOVNLE,           reg64,      regmem64,   none,       none,       0,       0x0f4f,    r,   no,  0,     O64)

INSTR (CMP,               al,         imm8,       none,       none,       0,       0x3c,      ib,  no,  0,     0)
INSTR (CMP,               regmem8,    imm8,       none,       none,       0,       0x80,      r7,  ib,  0,     0)
INSTR (CMP,               regmem8,    imm8,       none,       none,       0,       0x82,      r7,  ib,  0,     I64)
INSTR (CMP,               regmem16,   simm8,      none,       none,       0,       0x83,      r7,  ib,  0,     O16)
INSTR (CMP,               regmem32,   simm8,      none,       none,       0,       0x83,      r7,  ib,  0,     O32)
INSTR (CMP,               regmem64,   simm8,      none,       none,       0,       0x83,      r7,  ib,  0,     O64)
INSTR (CMP,               ax,         imm16,      none,       none,       0,       0x3d,      iw,  no,  0,     O16)
INSTR (CMP,               regmem16,   imm16,      none,       none,       0,       0x81,      r7,  iw,  0,     O16)
INSTR (CMP,               eax,        imm32,      none,       none,       0,       0x3d,      id,  no,  0,     O32)
INSTR (CMP,               regmem32,   imm32,      none,       none,       0,       0x81,      r7,  id,  0,     O32)
INSTR (CMP,               rax,        imm32,      none,       none,       0,       0x3d,      id,  no,  0,     O64)
INSTR (CMP,               regmem64,   simm32,     none,       none,       0,       0x81,      r7,  id,  0,     O64)
INSTR (CMP,               regmem8,    reg8,       none,       none,       0,       0x38,      r,   no,  0,     0)
INSTR (CMP,               regmem16,   reg16,      none,       none,       0,       0x39,      r,   no,  0,     O16)
INSTR (CMP,               regmem32,   reg32,      none,       none,       0,       0x39,      r,   no,  0,     O32)
INSTR (CMP,               regmem64,   reg64,      none,       none,       0,       0x39,      r,   no,  0,     O64)
INSTR (CMP,               reg8,       regmem8,    none,       none,       0,       0x3a,      r,   no,  0,     0)
INSTR (CMP,               reg16,      regmem16,   none,       none,       0,       0x3b,      r,   no,  0,     O16)
INSTR (CMP,               reg32,      regmem32,   none,       none,       0,       0x3b,      r,   no,  0,     O32)
INSTR (CMP,               reg64,      regmem64,   none,       none,       0,       0x3b,      r,   no,  0,     O64)

INSTR (CMPSB,             none,       none,       none,       none,       0,       0xa6,      no,  no,  0,     PREPE | PREPNE)

INSTR (CMPSW,             none,       none,       none,       none,       0,       0xa7,      no,  no,  0,     O16 | PREPE | PREPNE)

INSTR (CMPSD,             none,       none,       none,       none,       0,       0xa7,      no,  no,  0,     O32 | PREPE | PREPNE)
INSTR (CMPSD,             xmm,        xmmmem64,   imm8,       none,       0,       0x0fc2,    r,   ib,  0,     PF2)

INSTR (CMPSQ,             none,       none,       none,       none,       0,       0xa7,      no,  no,  0,     O64 | PREPE | PREPNE)

INSTR (CMPXCHG,           regmem8,    reg8,       none,       none,       0,       0x0fb0,    r,   no,  0,     PLOCK)
INSTR (CMPXCHG,           regmem16,   reg16,      none,       none,       0,       0x0fb1,    r,   no,  0,     O16 | PLOCK)
INSTR (CMPXCHG,           regmem32,   reg32,      none,       none,       0,       0x0fb1,    r,   no,  0,     O32 | PLOCK)
INSTR (CMPXCHG,           regmem64,   reg64,      none,       none,       0,       0x0fb1,    r,   no,  0,     O64 | PLOCK)

INSTR (CMPXCHG8B,         mem,        none,       none,       none,       0,       0x0fc7,    r1,  no,  0,     O32 | PLOCK)

INSTR (CMPXCHG16B,        mem,        none,       none,       none,       0,       0x0fc7,    r1,  no,  0,     O64 | PLOCK)

INSTR (CPUID,             none,       none,       none,       none,       0,       0x0fa2,    no,  no,  0,     0)

INSTR (CRC32,             reg32,      regmem8,    none,       none,       0,       0x0f38f0,  r,   no,  0,     PF2)
INSTR (CRC32,             reg32,      regmem16,   none,       none,       0,       0x0f38f1,  r,   no,  0,     O16 | PF2)
INSTR (CRC32,             reg32,      regmem32,   none,       none,       0,       0x0f38f1,  r,   no,  0,     O32 | PF2)
INSTR (CRC32,             reg64,      regmem8,    none,       none,       0,       0x0f38f0,  r,   no,  0,     PF2)
INSTR (CRC32,             reg64,      regmem64,   none,       none,       0,       0x0f38f1,  r,   no,  0,     O64 | PF2)

INSTR (DAA,               none,       none,       none,       none,       0,       0x27,      no,  no,  0,     I64)

INSTR (DAS,               none,       none,       none,       none,       0,       0x2f,      no,  no,  0,     I64)

INSTR (DEC,               reg16,      none,       none,       none,       0,       0x48,      rv,  no,  0,     O16 | I64)
INSTR (DEC,               reg32,      none,       none,       none,       0,       0x48,      rv,  no,  0,     O32 | I64)
INSTR (DEC,               regmem8,    none,       none,       none,       0,       0xfe,      r1,  no,  0,     PLOCK)
INSTR (DEC,               regmem16,   none,       none,       none,       0,       0xff,      r1,  no,  0,     O16 | PLOCK)
INSTR (DEC,               regmem32,   none,       none,       none,       0,       0xff,      r1,  no,  0,     O32 | PLOCK)
INSTR (DEC,               regmem64,   none,       none,       none,       0,       0xff,      r1,  no,  0,     O64 | PLOCK)

INSTR (ENTER,             rel16off,   imm8,       none,       none,       0,       0xc8,      cw,  ib,  0,     0)

INSTR (DIV,               regmem8,    none,       none,       none,       0,       0xf6,      r6,  no,  0,     0)
INSTR (DIV,               regmem16,   none,       none,       none,       0,       0xf7,      r6,  no,  0,     O16)
INSTR (DIV,               regmem32,   none,       none,       none,       0,       0xf7,      r6,  no,  0,     O32)
INSTR (DIV,               regmem64,   none,       none,       none,       0,       0xf7,      r6,  no,  0,     O64)

INSTR (IDIV,              regmem8,    none,       none,       none,       0,       0xf6,      r7,  no,  0,     0)
INSTR (IDIV,              regmem16,   none,       none,       none,       0,       0xf7,      r7,  no,  0,     O16)
INSTR (IDIV,              regmem32,   none,       none,       none,       0,       0xf7,      r7,  no,  0,     O32)
INSTR (IDIV,              regmem64,   none,       none,       none,       0,       0xf7,      r7,  no,  0,     O64)

INSTR (IMUL,              regmem8,    none,       none,       none,       0,       0xf6,      r5,  no,  0,     0)
INSTR (IMUL,              regmem16,   none,       none,       none,       0,       0xf7,      r5,  no,  0,     O16)
INSTR (IMUL,              regmem32,   none,       none,       none,       0,       0xf7,      r5,  no,  0,     O32)
INSTR (IMUL,              regmem64,   none,       none,       none,       0,       0xf7,      r5,  no,  0,     O64)
INSTR (IMUL,              reg16,      regmem16,   none,       none,       0,       0x0faf,    r,   no,  0,     O16)
INSTR (IMUL,              reg32,      regmem32,   none,       none,       0,       0x0faf,    r,   no,  0,     O32)
INSTR (IMUL,              reg64,      regmem64,   none,       none,       0,       0x0faf,    r,   no,  0,     O64)
INSTR (IMUL,              reg16,      regmem16,   simm8,      none,       0,       0x6b,      r,   ib,  0,     O16)
INSTR (IMUL,              reg32,      regmem32,   simm8,      none,       0,       0x6b,      r,   ib,  0,     O32)
INSTR (IMUL,              reg64,      regmem64,   simm8,      none,       0,       0x6b,      r,   ib,  0,     O64)
INSTR (IMUL,              reg16,      regmem16,   simm16,     none,       0,       0x69,      r,   iw,  0,     O16)
INSTR (IMUL,              reg32,      regmem32,   simm32,     none,       0,       0x69,      r,   id,  0,     O32)
INSTR (IMUL,              reg64,      regmem64,   simm32,     none,       0,       0x69,      r,   id,  0,     O64)

INSTR (IN,                al,         imm8,       none,       none,       0,       0xe4,      ib,  no,  0,     0)
INSTR (IN,                ax,         imm8,       none,       none,       0,       0xe5,      ib,  no,  0,     O16)
INSTR (IN,                eax,        imm8,       none,       none,       0,       0xe5,      ib,  no,  0,     O32)
INSTR (IN,                al,         dx,         none,       none,       0,       0xec,      no,  no,  0,     0)
INSTR (IN,                ax,         dx,         none,       none,       0,       0xed,      no,  no,  0,     O16)
INSTR (IN,                eax,        dx,         none,       none,       0,       0xed,      no,  no,  0,     O32)

INSTR (INC,               reg16,      none,       none,       none,       0,       0x40,      rv,  no,  0,     O16 | I64)
INSTR (INC,               reg32,      none,       none,       none,       0,       0x40,      rv,  no,  0,     O32 | I64)
INSTR (INC,               regmem8,    none,       none,       none,       0,       0xfe,      r0,  no,  0,     PLOCK)
INSTR (INC,               regmem16,   none,       none,       none,       0,       0xff,      r0,  no,  0,     O16 | PLOCK)
INSTR (INC,               regmem32,   none,       none,       none,       0,       0xff,      r0,  no,  0,     O32 | PLOCK)
INSTR (INC,               regmem64,   none,       none,       none,       0,       0xff,      r0,  no,  0,     O64 | PLOCK)

INSTR (INSB,              none,       none,       none,       none,       0,       0x6c,      no,  no,  0,     PREP)

INSTR (INSW,              none,       none,       none,       none,       0,       0x6d,      no,  no,  0,     O16 | PREP)

INSTR (INSD,              none,       none,       none,       none,       0,       0x6d,      no,  no,  0,     O32 | PREP)

INSTR (INT,               imm8,       none,       none,       none,       0,       0xcd,      ib,  no,  0,     0)

INSTR (INTO,              none,       none,       none,       none,       0,       0xce,      no,  no,  0,     I64)

INSTR (JO,                rel8off,    none,       none,       none,       0,       0x70,      cb,  no,  0,     0)
INSTR (JO,                rel16off,   none,       none,       none,       0,       0x0f80,    cw,  no,  0,     O16)
INSTR (JO,                rel32off,   none,       none,       none,       0,       0x0f80,    cd,  no,  0,     O32)

INSTR (JNO,               rel8off,    none,       none,       none,       0,       0x71,      cb,  no,  0,     0)
INSTR (JNO,               rel16off,   none,       none,       none,       0,       0x0f81,    cw,  no,  0,     O16)
INSTR (JNO,               rel32off,   none,       none,       none,       0,       0x0f81,    cd,  no,  0,     O32)

INSTR (JC,                rel8off,    none,       none,       none,       0,       0x72,      cb,  no,  0,     0)
INSTR (JC,                rel16off,   none,       none,       none,       0,       0x0f82,    cw,  no,  0,     O16)
INSTR (JC,                rel32off,   none,       none,       none,       0,       0x0f82,    cd,  no,  0,     O32)

INSTR (JB,                rel8off,    none,       none,       none,       0,       0x72,      cb,  no,  0,     0)
INSTR (JB,                rel16off,   none,       none,       none,       0,       0x0f82,    cw,  no,  0,     O16)
INSTR (JB,                rel32off,   none,       none,       none,       0,       0x0f82,    cd,  no,  0,     O32)

INSTR (JNAE,              rel8off,    none,       none,       none,       0,       0x72,      cb,  no,  0,     0)
INSTR (JNAE,              rel16off,   none,       none,       none,       0,       0x0f82,    cw,  no,  0,     O16)
INSTR (JNAE,              rel32off,   none,       none,       none,       0,       0x0f82,    cd,  no,  0,     O32)

INSTR (JNC,               rel8off,    none,       none,       none,       0,       0x73,      cb,  no,  0,     0)
INSTR (JNC,               rel16off,   none,       none,       none,       0,       0x0f83,    cw,  no,  0,     O16)
INSTR (JNC,               rel32off,   none,       none,       none,       0,       0x0f83,    cd,  no,  0,     O32)

INSTR (JNB,               rel8off,    none,       none,       none,       0,       0x73,      cb,  no,  0,     0)
INSTR (JNB,               rel16off,   none,       none,       none,       0,       0x0f83,    cw,  no,  0,     O16)
INSTR (JNB,               rel32off,   none,       none,       none,       0,       0x0f83,    cd,  no,  0,     O32)

INSTR (JAE,               rel8off,    none,       none,       none,       0,       0x73,      cb,  no,  0,     0)
INSTR (JAE,               rel16off,   none,       none,       none,       0,       0x0f83,    cw,  no,  0,     O16)
INSTR (JAE,               rel32off,   none,       none,       none,       0,       0x0f83,    cd,  no,  0,     O32)

INSTR (JZ,                rel8off,    none,       none,       none,       0,       0x74,      cb,  no,  0,     0)
INSTR (JZ,                rel16off,   none,       none,       none,       0,       0x0f84,    cw,  no,  0,     O16)
INSTR (JZ,                rel32off,   none,       none,       none,       0,       0x0f84,    cd,  no,  0,     O32)

INSTR (JE,                rel8off,    none,       none,       none,       0,       0x74,      cb,  no,  0,     0)
INSTR (JE,                rel16off,   none,       none,       none,       0,       0x0f84,    cw,  no,  0,     O16)
INSTR (JE,                rel32off,   none,       none,       none,       0,       0x0f84,    cd,  no,  0,     O32)

INSTR (JNZ,               rel8off,    none,       none,       none,       0,       0x75,      cb,  no,  0,     0)
INSTR (JNZ,               rel16off,   none,       none,       none,       0,       0x0f85,    cw,  no,  0,     O16)
INSTR (JNZ,               rel32off,   none,       none,       none,       0,       0x0f85,    cd,  no,  0,     O32)

INSTR (JNE,               rel8off,    none,       none,       none,       0,       0x75,      cb,  no,  0,     0)
INSTR (JNE,               rel16off,   none,       none,       none,       0,       0x0f85,    cw,  no,  0,     O16)
INSTR (JNE,               rel32off,   none,       none,       none,       0,       0x0f85,    cd,  no,  0,     O32)

INSTR (JNA,               rel8off,    none,       none,       none,       0,       0x76,      cb,  no,  0,     0)
INSTR (JNA,               rel16off,   none,       none,       none,       0,       0x0f86,    cw,  no,  0,     O16)
INSTR (JNA,               rel32off,   none,       none,       none,       0,       0x0f86,    cd,  no,  0,     O32)

INSTR (JBE,               rel8off,    none,       none,       none,       0,       0x76,      cb,  no,  0,     0)
INSTR (JBE,               rel16off,   none,       none,       none,       0,       0x0f86,    cw,  no,  0,     O16)
INSTR (JBE,               rel32off,   none,       none,       none,       0,       0x0f86,    cd,  no,  0,     O32)

INSTR (JA,                rel8off,    none,       none,       none,       0,       0x77,      cb,  no,  0,     0)
INSTR (JA,                rel16off,   none,       none,       none,       0,       0x0f87,    cw,  no,  0,     O16)
INSTR (JA,                rel32off,   none,       none,       none,       0,       0x0f87,    cd,  no,  0,     O32)

INSTR (JNBE,              rel8off,    none,       none,       none,       0,       0x77,      cb,  no,  0,     0)
INSTR (JNBE,              rel16off,   none,       none,       none,       0,       0x0f87,    cw,  no,  0,     O16)
INSTR (JNBE,              rel32off,   none,       none,       none,       0,       0x0f87,    cd,  no,  0,     O32)

INSTR (JS,                rel8off,    none,       none,       none,       0,       0x78,      cb,  no,  0,     0)
INSTR (JS,                rel16off,   none,       none,       none,       0,       0x0f88,    cw,  no,  0,     O16)
INSTR (JS,                rel32off,   none,       none,       none,       0,       0x0f88,    cd,  no,  0,     O32)

INSTR (JNS,               rel8off,    none,       none,       none,       0,       0x79,      cb,  no,  0,     0)
INSTR (JNS,               rel16off,   none,       none,       none,       0,       0x0f89,    cw,  no,  0,     O16)
INSTR (JNS,               rel32off,   none,       none,       none,       0,       0x0f89,    cd,  no,  0,     O32)

INSTR (JPE,               rel8off,    none,       none,       none,       0,       0x7a,      cb,  no,  0,     0)
INSTR (JPE,               rel16off,   none,       none,       none,       0,       0x0f8a,    cw,  no,  0,     O16)
INSTR (JPE,               rel32off,   none,       none,       none,       0,       0x0f8a,    cd,  no,  0,     O32)

INSTR (JP,                rel8off,    none,       none,       none,       0,       0x7a,      cb,  no,  0,     0)
INSTR (JP,                rel16off,   none,       none,       none,       0,       0x0f8a,    cw,  no,  0,     O16)
INSTR (JP,                rel32off,   none,       none,       none,       0,       0x0f8a,    cd,  no,  0,     O32)

INSTR (JPO,               rel8off,    none,       none,       none,       0,       0x7b,      cb,  no,  0,     0)
INSTR (JPO,               rel16off,   none,       none,       none,       0,       0x0f8b,    cw,  no,  0,     O16)
INSTR (JPO,               rel32off,   none,       none,       none,       0,       0x0f8b,    cd,  no,  0,     O32)

INSTR (JNP,               rel8off,    none,       none,       none,       0,       0x7b,      cb,  no,  0,     0)
INSTR (JNP,               rel16off,   none,       none,       none,       0,       0x0f8b,    cw,  no,  0,     O16)
INSTR (JNP,               rel32off,   none,       none,       none,       0,       0x0f8b,    cd,  no,  0,     O32)

INSTR (JL,                rel8off,    none,       none,       none,       0,       0x7c,      cb,  no,  0,     0)
INSTR (JL,                rel16off,   none,       none,       none,       0,       0x0f8c,    cw,  no,  0,     O16)
INSTR (JL,                rel32off,   none,       none,       none,       0,       0x0f8c,    cd,  no,  0,     O32)

INSTR (JNGE,              rel8off,    none,       none,       none,       0,       0x7c,      cb,  no,  0,     0)
INSTR (JNGE,              rel16off,   none,       none,       none,       0,       0x0f8c,    cw,  no,  0,     O16)
INSTR (JNGE,              rel32off,   none,       none,       none,       0,       0x0f8c,    cd,  no,  0,     O32)

INSTR (JNL,               rel8off,    none,       none,       none,       0,       0x7d,      cb,  no,  0,     0)
INSTR (JNL,               rel16off,   none,       none,       none,       0,       0x0f8d,    cw,  no,  0,     O16)
INSTR (JNL,               rel32off,   none,       none,       none,       0,       0x0f8d,    cd,  no,  0,     O32)

INSTR (JGE,               rel8off,    none,       none,       none,       0,       0x7d,      cb,  no,  0,     0)
INSTR (JGE,               rel16off,   none,       none,       none,       0,       0x0f8d,    cw,  no,  0,     O16)
INSTR (JGE,               rel32off,   none,       none,       none,       0,       0x0f8d,    cd,  no,  0,     O32)

INSTR (JNG,               rel8off,    none,       none,       none,       0,       0x7e,      cb,  no,  0,     0)
INSTR (JNG,               rel16off,   none,       none,       none,       0,       0x0f8e,    cw,  no,  0,     O16)
INSTR (JNG,               rel32off,   none,       none,       none,       0,       0x0f8e,    cd,  no,  0,     O32)

INSTR (JLE,               rel8off,    none,       none,       none,       0,       0x7e,      cb,  no,  0,     0)
INSTR (JLE,               rel16off,   none,       none,       none,       0,       0x0f8e,    cw,  no,  0,     O16)
INSTR (JLE,               rel32off,   none,       none,       none,       0,       0x0f8e,    cd,  no,  0,     O32)

INSTR (JG,                rel8off,    none,       none,       none,       0,       0x7f,      cb,  no,  0,     0)
INSTR (JG,                rel16off,   none,       none,       none,       0,       0x0f8f,    cw,  no,  0,     O16)
INSTR (JG,                rel32off,   none,       none,       none,       0,       0x0f8f,    cd,  no,  0,     O32)

INSTR (JNLE,              rel8off,    none,       none,       none,       0,       0x7f,      cb,  no,  0,     0)
INSTR (JNLE,              rel16off,   none,       none,       none,       0,       0x0f8f,    cw,  no,  0,     O16)
INSTR (JNLE,              rel32off,   none,       none,       none,       0,       0x0f8f,    cd,  no,  0,     O32)

INSTR (JCXZ,              rel8off,    none,       none,       none,       0,       0xe3,      cb,  no,  0,     O16 | I64)

INSTR (JECXZ,             rel8off,    none,       none,       none,       0,       0xe3,      cb,  no,  0,     O32)

INSTR (JRCXZ,             rel8off,    none,       none,       none,       0,       0xe3,      cb,  no,  0,     O64 | D64)

INSTR (JMP,               rel8off,    none,       none,       none,       0,       0xeb,      cb,  no,  0,     0)
INSTR (JMP,               rel16off,   none,       none,       none,       0,       0xe9,      cw,  no,  0,     O16)
INSTR (JMP,               rel32off,   none,       none,       none,       0,       0xe9,      cd,  no,  0,     O32)
INSTR (JMP,               regmem16,   none,       none,       none,       0,       0xff,      r4,  no,  0,     O16)
INSTR (JMP,               regmem32,   none,       none,       none,       0,       0xff,      r4,  no,  0,     O32 | I64)
INSTR (JMP,               regmem64,   none,       none,       none,       0,       0xff,      r4,  no,  0,     O64 | D64)

INSTR (JMPFAR,            imm16,      rel16off,   none,       none,       0,       0xea,      cw,  iw,  0,     O16 | I64 | FAR)
INSTR (JMPFAR,            imm16,      rel32off,   none,       none,       0,       0xea,      cd,  iw,  0,     O32 | I64 | FAR)
INSTR (JMPFAR,            mem16,      none,       none,       none,       0,       0xff,      r5,  no,  0,     O16)
INSTR (JMPFAR,            mem32,      none,       none,       none,       0,       0xff,      r5,  no,  0,     O32)

INSTR (LAHF,              none,       none,       none,       none,       0,       0x9f,      no,  no,  0,     0)

INSTR (LDS,               reg16,      mem,        none,       none,       0,       0xc5,      r,   no,  0,     O16 | I64)
INSTR (LDS,               reg32,      mem,        none,       none,       0,       0xc5,      r,   no,  0,     O32 | I64)

INSTR (LES,               reg16,      mem,        none,       none,       0,       0xc4,      r,   no,  0,     O16 | I64)
INSTR (LES,               reg32,      mem,        none,       none,       0,       0xc4,      r,   no,  0,     O32 | I64)

INSTR (LFS,               reg16,      mem,        none,       none,       0,       0x0fb4,    r,   no,  0,     O16)
INSTR (LFS,               reg32,      mem,        none,       none,       0,       0x0fb4,    r,   no,  0,     O32)

INSTR (LGS,               reg16,      mem,        none,       none,       0,       0x0fb5,    r,   no,  0,     O16)
INSTR (LGS,               reg32,      mem,        none,       none,       0,       0x0fb5,    r,   no,  0,     O32)

INSTR (LSS,               reg16,      mem,        none,       none,       0,       0x0fb2,    r,   no,  0,     O16)
INSTR (LSS,               reg32,      mem,        none,       none,       0,       0x0fb2,    r,   no,  0,     O32)

INSTR (LEA,               reg16,      mem,        none,       none,       0,       0x8d,      r,   no,  0,     O16)
INSTR (LEA,               reg32,      mem,        none,       none,       0,       0x8d,      r,   no,  0,     O32)
INSTR (LEA,               reg64,      mem,        none,       none,       0,       0x8d,      r,   no,  0,     O64)

INSTR (LEAVE,             none,       none,       none,       none,       0,       0xc9,      no,  no,  0,     0)

INSTR (LFENCE,            none,       none,       none,       none,       0,       0x0fae,    no,  no,  0xe8,  SFX)

INSTR (LLWPCB,            reg32rm,    none,       none,       none,       0x0978,  0x12,      r0,  no,  0,     PXOP)
INSTR (LLWPCB,            reg64rm,    none,       none,       none,       0x09f8,  0x12,      r0,  no,  0,     PXOP)

INSTR (LODSB,             none,       none,       none,       none,       0,       0xac,      no,  no,  0,     PREP)

INSTR (LODSW,             none,       none,       none,       none,       0,       0xad,      no,  no,  0,     O16 | PREP)

INSTR (LODSD,             none,       none,       none,       none,       0,       0xad,      no,  no,  0,     O32 | PREP)

INSTR (LODSQ,             none,       none,       none,       none,       0,       0xad,      no,  no,  0,     O64 | PREP)

INSTR (LOOP,              rel8off,    none,       none,       none,       0,       0xe2,      cb,  no,  0,     0)

INSTR (LOOPE,             rel8off,    none,       none,       none,       0,       0xe1,      cb,  no,  0,     0)

INSTR (LOOPNE,            rel8off,    none,       none,       none,       0,       0xe0,      cb,  no,  0,     0)

INSTR (LOOPNZ,            rel8off,    none,       none,       none,       0,       0xe0,      cb,  no,  0,     0)

INSTR (LOOPZ,             rel8off,    none,       none,       none,       0,       0xe1,      cb,  no,  0,     0)

INSTR (MFENCE,            none,       none,       none,       none,       0,       0x0fae,    no,  no,  0xf0,  SFX)

INSTR (LWPINS,            reg32vvvv,  regmem32,   imm32,      none,       0x0a10,  0x12,      r0,  id,  0,     PXOP)
INSTR (LWPINS,            reg64vvvv,  regmem32,   imm32,      none,       0x0a90,  0x12,      r0,  id,  0,     PXOP)

INSTR (LWPVAL,            reg32vvvv,  regmem32,   imm32,      none,       0x0a10,  0x12,      r1,  id,  0,     PXOP)
INSTR (LWPVAL,            reg64vvvv,  regmem32,   imm32,      none,       0x0a90,  0x12,      r1,  id,  0,     PXOP)

INSTR (RSTORSSP,          mem64,      none,       none,       none,       0,       0x0f01,    r5,  no,  0,     PF3)

INSTR (MCOMMIT,           none,       none,       none,       none,       0,       0x0f01,    no,  no,  0xfa,  PF3 | SFX)

INSTR (MOV,               al,         moffset,    none,       none,       0,       0xa0,      no,  no,  0,     0)
INSTR (MOV,               ax,         moffset,    none,       none,       0,       0xa1,      no,  no,  0,     O16)
INSTR (MOV,               eax,        moffset,    none,       none,       0,       0xa1,      no,  no,  0,     O32)
INSTR (MOV,               rax,        moffset,    none,       none,       0,       0xa1,      no,  no,  0,     O64)
INSTR (MOV,               moffset,    al,         none,       none,       0,       0xa2,      no,  no,  0,     0)
INSTR (MOV,               moffset,    ax,         none,       none,       0,       0xa3,      no,  no,  0,     O16)
INSTR (MOV,               moffset,    eax,        none,       none,       0,       0xa3,      no,  no,  0,     O32)
INSTR (MOV,               moffset,    rax,        none,       none,       0,       0xa3,      no,  no,  0,     O64)
INSTR (MOV,               regmem8,    reg8,       none,       none,       0,       0x88,      r,   no,  0,     0)
INSTR (MOV,               regmem16,   reg16,      none,       none,       0,       0x89,      r,   no,  0,     O16)
INSTR (MOV,               regmem32,   reg32,      none,       none,       0,       0x89,      r,   no,  0,     O32)
INSTR (MOV,               regmem64,   reg64,      none,       none,       0,       0x89,      r,   no,  0,     O64)
INSTR (MOV,               reg8,       regmem8,    none,       none,       0,       0x8a,      r,   no,  0,     0)
INSTR (MOV,               reg16,      regmem16,   none,       none,       0,       0x8b,      r,   no,  0,     O16)
INSTR (MOV,               reg32,      regmem32,   none,       none,       0,       0x8b,      r,   no,  0,     O32)
INSTR (MOV,               reg64,      regmem64,   none,       none,       0,       0x8b,      r,   no,  0,     O64)
INSTR (MOV,               regmem16,   segreg,     none,       none,       0,       0x8c,      r,   no,  0,     O16)
INSTR (MOV,               regmem32,   segreg,     none,       none,       0,       0x8c,      r,   no,  0,     O32)
INSTR (MOV,               regmem64,   segreg,     none,       none,       0,       0x8c,      r,   no,  0,     O64)
INSTR (MOV,               segreg,     regmem16,   none,       none,       0,       0x8e,      r,   no,  0,     0)
INSTR (MOV,               reg8,       imm8,       none,       none,       0,       0xb0,      rv,  ib,  0,     0)
INSTR (MOV,               reg16,      imm16,      none,       none,       0,       0xb8,      rv,  iw,  0,     O16)
INSTR (MOV,               reg32,      imm32,      none,       none,       0,       0xb8,      rv,  id,  0,     O32)
INSTR (MOV,               regmem8,    imm8,       none,       none,       0,       0xc6,      r0,  ib,  0,     0)
INSTR (MOV,               regmem16,   imm16,      none,       none,       0,       0xc7,      r0,  iw,  0,     O16)
INSTR (MOV,               regmem32,   imm32,      none,       none,       0,       0xc7,      r0,  id,  0,     O32)
INSTR (MOV,               regmem64,   simm32,     none,       none,       0,       0xc7,      r0,  id,  0,     O64)
INSTR (MOV,               reg64,      imm64,      none,       none,       0,       0xb8,      rv,  iq,  0,     O64)
INSTR (MOV,               cr8,        regmem32,   none,       none,       0,       0x0f22,    r,   no,  0,     I64 | PF0)
INSTR (MOV,               cr8,        regmem64,   none,       none,       0,       0x0f22,    r,   no,  0,     D64 | PF0)
INSTR (MOV,               regmem32,   cr8,        none,       none,       0,       0x0f20,    r,   no,  0,     I64 | PF0)
INSTR (MOV,               regmem64,   cr8,        none,       none,       0,       0x0f20,    r,   no,  0,     D64 | PF0)
INSTR (MOV,               cr,         regmem32,   none,       none,       0,       0x0f22,    r,   no,  0,     I64)
INSTR (MOV,               cr,         regmem64,   none,       none,       0,       0x0f22,    r,   no,  0,     D64)
INSTR (MOV,               regmem32,   cr,         none,       none,       0,       0x0f20,    r,   no,  0,     I64)
INSTR (MOV,               regmem64,   cr,         none,       none,       0,       0x0f20,    r,   no,  0,     D64)
INSTR (MOV,               dr,         regmem32,   none,       none,       0,       0x0f21,    r,   no,  0,     I64)
INSTR (MOV,               dr,         regmem64,   none,       none,       0,       0x0f21,    r,   no,  0,     D64)
INSTR (MOV,               regmem32,   dr,         none,       none,       0,       0x0f23,    r,   no,  0,     I64)
INSTR (MOV,               regmem64,   dr,         none,       none,       0,       0x0f23,    r,   no,  0,     D64)

INSTR (MOVBE,             reg16,      mem16,      none,       none,       0,       0x0f38f0,  r,   no,  0,     O16)
INSTR (MOVBE,             reg32,      mem32,      none,       none,       0,       0x0f38f0,  r,   no,  0,     O32)
INSTR (MOVBE,             reg64,      mem64,      none,       none,       0,       0x0f38f0,  r,   no,  0,     O64)
INSTR (MOVBE,             mem16,      reg16,      none,       none,       0,       0x0f38f1,  r,   no,  0,     O16)
INSTR (MOVBE,             mem32,      reg32,      none,       none,       0,       0x0f38f1,  r,   no,  0,     O32)
INSTR (MOVBE,             mem64,      reg64,      none,       none,       0,       0x0f38f1,  r,   no,  0,     O64)

INSTR (MOVDQA,            xmm,        xmmmem128,  none,       none,       0,       0x0f6f,    r,   no,  0,     P66)
INSTR (MOVDQA,            xmmmem128,  xmm,        none,       none,       0,       0x0f7f,    r,   no,  0,     P66)

INSTR (VMOVDQA,           xmm,        xmmmem128,  none,       none,       0x0179,  0x6f,      r,   no,  0,     PVEX)
INSTR (VMOVDQA,           xmmmem128,  xmm,        none,       none,       0x0179,  0x7f,      r,   no,  0,     PVEX)
INSTR (VMOVDQA,           ymm,        ymmmem256,  none,       none,       0x017d,  0x6f,      r,   no,  0,     PVEX)
INSTR (VMOVDQA,           ymmmem256,  ymm,        none,       none,       0x017d,  0x7f,      r,   no,  0,     PVEX)

INSTR (MOVDQU,            xmm,        xmmmem128,  none,       none,       0,       0x0f6f,    r,   no,  0,     PF3)
INSTR (MOVDQU,            xmmmem128,  xmm,        none,       none,       0,       0x0f7f,    r,   no,  0,     PF3)

INSTR (VMOVDQU,           xmm,        xmmmem128,  none,       none,       0x017a,  0x6f,      r,   no,  0,     PVEX)
INSTR (VMOVDQU,           xmmmem128,  xmm,        none,       none,       0x017a,  0x7f,      r,   no,  0,     PVEX)
INSTR (VMOVDQU,           ymm,        ymmmem256,  none,       none,       0x017e,  0x6f,      r,   no,  0,     PVEX)
INSTR (VMOVDQU,           ymmmem256,  ymm,        none,       none,       0x017e,  0x7f,      r,   no,  0,     PVEX)

INSTR (MOVQ,              xmm,        xmmmem64,   none,       none,       0,       0x0f7e,    r,   no,  0,     PF3)
INSTR (MOVQ,              xmmmem64,   xmm,        none,       none,       0,       0x0fd6,    r,   no,  0,     P66)
INSTR (MOVQ,              mmx,        mmxmem64,   none,       none,       0,       0x0f6f,    r,   no,  0,     0)
INSTR (MOVQ,              mmxmem64,   mmx,        none,       none,       0,       0x0f7f,    r,   no,  0,     0)

INSTR (VMOVQ,             xmm,        regmem64,   none,       none,       0x01f9,  0x6e,      r,   no,  0,     PVEX)
INSTR (VMOVQ,             regmem64,   xmm,        none,       none,       0x01fd,  0x7e,      r,   no,  0,     PVEX)
INSTR (VMOVQ,             xmm,        xmmmem64,   none,       none,       0x017a,  0x7e,      r,   no,  0,     PVEX)
INSTR (VMOVQ,             xmmmem64,   xmm,        none,       none,       0x0179,  0xd6,      r,   no,  0,     PVEX)

INSTR (MOVD,              xmm,        regmem32,   none,       none,       0,       0x0f6e,    r,   no,  0,     O16 | O32 | P66)
INSTR (MOVD,              xmm,        regmem64,   none,       none,       0,       0x0f6e,    r,   no,  0,     O64 | P66)
INSTR (MOVD,              regmem32,   xmm,        none,       none,       0,       0x0f7e,    r,   no,  0,     O16 | O32 | P66)
INSTR (MOVD,              regmem64,   xmm,        none,       none,       0,       0x0f7e,    r,   no,  0,     O64 | P66)
INSTR (MOVD,              mmx,        regmem32,   none,       none,       0,       0x0f6e,    r,   no,  0,     O16 | O32)
INSTR (MOVD,              mmx,        regmem64,   none,       none,       0,       0x0f6e,    r,   no,  0,     O64)
INSTR (MOVD,              regmem32,   mmx,        none,       none,       0,       0x0f7e,    r,   no,  0,     O16 | O32)
INSTR (MOVD,              regmem64,   mmx,        none,       none,       0,       0x0f7e,    r,   no,  0,     O64)

INSTR (VMOVD,             xmm,        regmem32,   none,       none,       0x0179,  0x6e,      r,   no,  0,     PVEX)
INSTR (VMOVD,             regmem32,   xmm,        none,       none,       0x0179,  0x7e,      r,   no,  0,     PVEX)

INSTR (MOVMSKPD,          reg32,      xmmmem32,   none,       none,       0,       0x0f50,    r,   no,  0,     O16 | O32 | P66)
INSTR (MOVMSKPD,          reg64,      xmmmem32,   none,       none,       0,       0x0f50,    r,   no,  0,     O64 | P66)

INSTR (VMOVMSKPD,         reg32,      xmmmem32,   none,       none,       0x0179,  0x50,      r,   no,  0,     PVEX)
INSTR (VMOVMSKPD,         reg64,      xmmmem64,   none,       none,       0x017d,  0x50,      r,   no,  0,     PVEX)

INSTR (MOVMSKPS,          reg32,      xmmmem32,   none,       none,       0,       0x0f50,    r,   no,  0,     O16 | O32)
INSTR (MOVMSKPS,          reg64,      xmmmem32,   none,       none,       0,       0x0f50,    r,   no,  0,     O64)

INSTR (VMOVMSKPS,         reg32,      xmmmem32,   none,       none,       0x0178,  0x50,      r,   no,  0,     PVEX)
INSTR (VMOVMSKPS,         reg64,      xmmmem64,   none,       none,       0x017c,  0x50,      r,   no,  0,     PVEX)

INSTR (MOVNTI,            mem32,      reg32,      none,       none,       0,       0x0fc3,    r,   no,  0,     O16 | O32)
INSTR (MOVNTI,            mem64,      reg64,      none,       none,       0,       0x0fc3,    r,   no,  0,     O64)

INSTR (MOVSB,             none,       none,       none,       none,       0,       0xa4,      no,  no,  0,     PREP)

INSTR (MOVSW,             none,       none,       none,       none,       0,       0xa5,      no,  no,  0,     O16 | PREP)

INSTR (MOVSD,             none,       none,       none,       none,       0,       0xa5,      no,  no,  0,     O32 | PREP)
INSTR (MOVSD,             xmm,        mem64,      none,       none,       0,       0x0f10,    r,   no,  0,     PF2)
INSTR (MOVSD,             xmmmem64,   xmm,        none,       none,       0,       0x0f11,    r,   no,  0,     PF2)

INSTR (VMOVSD,            xmm,        mem64,      none,       none,       0x017b,  0x10,      r,   no,  0,     PVEX)
INSTR (VMOVSD,            mem64,      xmm,        none,       none,       0x017b,  0x11,      r,   no,  0,     PVEX)
INSTR (VMOVSD,            xmm,        xvvvv,      xmmmem64,   none,       0x0107,  0x10,      r,   no,  0,     PVEX)
INSTR (VMOVSD,            xmm,        xvvvv,      xmmmem64,   none,       0x0107,  0x11,      r,   no,  0,     PVEX)

INSTR (MOVSQ,             none,       none,       none,       none,       0,       0xa5,      no,  no,  0,     O64 | PREP)

INSTR (MOVSX,             reg16,      regmem8,    none,       none,       0,       0x0fbe,    r,   no,  0,     O16)
INSTR (MOVSX,             reg32,      regmem8,    none,       none,       0,       0x0fbe,    r,   no,  0,     O32)
INSTR (MOVSX,             reg64,      regmem8,    none,       none,       0,       0x0fbe,    r,   no,  0,     O64)
INSTR (MOVSX,             reg32,      regmem16,   none,       none,       0,       0x0fbf,    r,   no,  0,     O32)
INSTR (MOVSX,             reg64,      regmem16,   none,       none,       0,       0x0fbf,    r,   no,  0,     O64)

INSTR (MOVSXD,            reg64,      regmem32,   none,       none,       0,       0x63,      r,   no,  0,     O64)

INSTR (MOVZX,             reg16,      regmem8,    none,       none,       0,       0x0fb6,    r,   no,  0,     O16)
INSTR (MOVZX,             reg32,      regmem8,    none,       none,       0,       0x0fb6,    r,   no,  0,     O32)
INSTR (MOVZX,             reg64,      regmem8,    none,       none,       0,       0x0fb6,    r,   no,  0,     O64)
INSTR (MOVZX,             reg32,      regmem16,   none,       none,       0,       0x0fb7,    r,   no,  0,     O32)
INSTR (MOVZX,             reg64,      regmem16,   none,       none,       0,       0x0fb7,    r,   no,  0,     O64)

INSTR (MUL,               regmem8,    none,       none,       none,       0,       0xf6,      r4,  no,  0,     0)
INSTR (MUL,               regmem16,   none,       none,       none,       0,       0xf7,      r4,  no,  0,     O16)
INSTR (MUL,               regmem32,   none,       none,       none,       0,       0xf7,      r4,  no,  0,     O32)
INSTR (MUL,               regmem64,   none,       none,       none,       0,       0xf7,      r4,  no,  0,     O64)

INSTR (MULX,              reg32,      reg32vvvv,  regmem32,   none,       0x0203,  0xf6,      r,   no,  0,     PVEX)
INSTR (MULX,              reg64,      reg64vvvv,  regmem64,   none,       0x0283,  0xf6,      r,   no,  0,     PVEX)

INSTR (NEG,               regmem8,    none,       none,       none,       0,       0xf6,      r3,  no,  0,     PLOCK)
INSTR (NEG,               regmem16,   none,       none,       none,       0,       0xf7,      r3,  no,  0,     O16 | PLOCK)
INSTR (NEG,               regmem32,   none,       none,       none,       0,       0xf7,      r3,  no,  0,     O32 | PLOCK)
INSTR (NEG,               regmem64,   none,       none,       none,       0,       0xf7,      r3,  no,  0,     O64 | PLOCK)

INSTR (PAUSE,             none,       none,       none,       none,       0,       0x90,      no,  no,  0,     PF3)

INSTR (NOP,               none,       none,       none,       none,       0,       0x90,      no,  no,  0,     0)
INSTR (NOP,               regmem16,   none,       none,       none,       0,       0x0f1f,    r0,  no,  0,     O16)
INSTR (NOP,               regmem32,   none,       none,       none,       0,       0x0f1f,    r0,  no,  0,     O32)
INSTR (NOP,               regmem64,   none,       none,       none,       0,       0x0f1f,    r0,  no,  0,     O64)

INSTR (NOT,               regmem8,    none,       none,       none,       0,       0xf6,      r2,  no,  0,     PLOCK)
INSTR (NOT,               regmem16,   none,       none,       none,       0,       0xf7,      r2,  no,  0,     O16 | PLOCK)
INSTR (NOT,               regmem32,   none,       none,       none,       0,       0xf7,      r2,  no,  0,     O32 | PLOCK)
INSTR (NOT,               regmem64,   none,       none,       none,       0,       0xf7,      r2,  no,  0,     O64 | PLOCK)

INSTR (OR,                al,         imm8,       none,       none,       0,       0x0c,      ib,  no,  0,     0)
INSTR (OR,                regmem8,    imm8,       none,       none,       0,       0x80,      r1,  ib,  0,     PLOCK)
INSTR (OR,                regmem8,    imm8,       none,       none,       0,       0x82,      r1,  ib,  0,     I64 | PLOCK)
INSTR (OR,                regmem16,   simm8,      none,       none,       0,       0x83,      r1,  ib,  0,     O16 | PLOCK)
INSTR (OR,                regmem32,   simm8,      none,       none,       0,       0x83,      r1,  ib,  0,     O32 | PLOCK)
INSTR (OR,                regmem64,   simm8,      none,       none,       0,       0x83,      r1,  ib,  0,     O64 | PLOCK)
INSTR (OR,                ax,         imm16,      none,       none,       0,       0x0d,      iw,  no,  0,     O16)
INSTR (OR,                regmem16,   imm16,      none,       none,       0,       0x81,      r1,  iw,  0,     O16 | PLOCK)
INSTR (OR,                eax,        imm32,      none,       none,       0,       0x0d,      id,  no,  0,     O32)
INSTR (OR,                regmem32,   imm32,      none,       none,       0,       0x81,      r1,  id,  0,     O32 | PLOCK)
INSTR (OR,                rax,        simm32,     none,       none,       0,       0x0d,      id,  no,  0,     O64)
INSTR (OR,                regmem64,   simm32,     none,       none,       0,       0x81,      r1,  id,  0,     O64 | PLOCK)
INSTR (OR,                regmem8,    reg8,       none,       none,       0,       0x08,      r,   no,  0,     PLOCK)
INSTR (OR,                regmem16,   reg16,      none,       none,       0,       0x09,      r,   no,  0,     O16 | PLOCK)
INSTR (OR,                regmem32,   reg32,      none,       none,       0,       0x09,      r,   no,  0,     O32 | PLOCK)
INSTR (OR,                regmem64,   reg64,      none,       none,       0,       0x09,      r,   no,  0,     O64 | PLOCK)
INSTR (OR,                reg8,       regmem8,    none,       none,       0,       0x0a,      r,   no,  0,     0)
INSTR (OR,                reg16,      regmem16,   none,       none,       0,       0x0b,      r,   no,  0,     O16)
INSTR (OR,                reg32,      regmem32,   none,       none,       0,       0x0b,      r,   no,  0,     O32)
INSTR (OR,                reg64,      regmem64,   none,       none,       0,       0x0b,      r,   no,  0,     O64)

INSTR (OUT,               imm8,       al,         none,       none,       0,       0xe6,      ib,  no,  0,     0)
INSTR (OUT,               imm8,       ax,         none,       none,       0,       0xe7,      ib,  no,  0,     O16)
INSTR (OUT,               imm8,       eax,        none,       none,       0,       0xe7,      ib,  no,  0,     O32)
INSTR (OUT,               dx,         al,         none,       none,       0,       0xee,      no,  no,  0,     0)
INSTR (OUT,               dx,         ax,         none,       none,       0,       0xef,      no,  no,  0,     O16)
INSTR (OUT,               dx,         eax,        none,       none,       0,       0xef,      no,  no,  0,     O32)

INSTR (OUTSB,             none,       none,       none,       none,       0,       0x6e,      no,  no,  0,     PREP)

INSTR (OUTSW,             none,       none,       none,       none,       0,       0x6f,      no,  no,  0,     O16 | PREP)

INSTR (OUTSD,             none,       none,       none,       none,       0,       0x6f,      no,  no,  0,     O32 | PREP)

INSTR (PDEP,              reg32,      reg32vvvv,  regmem32,   none,       0x0203,  0xf5,      r,   no,  0,     PVEX)
INSTR (PDEP,              reg64,      reg64vvvv,  regmem64,   none,       0x0283,  0xf5,      r,   no,  0,     PVEX)

INSTR (PEXT,              reg32,      reg32vvvv,  regmem32,   none,       0x0202,  0xf5,      r,   no,  0,     PVEX)
INSTR (PEXT,              reg64,      reg64vvvv,  regmem64,   none,       0x0282,  0xf5,      r,   no,  0,     PVEX)

INSTR (POP,               reg16,      none,       none,       none,       0,       0x58,      rv,  no,  0,     O16)
INSTR (POP,               reg32,      none,       none,       none,       0,       0x58,      rv,  no,  0,     O32 | I64)
INSTR (POP,               reg64,      none,       none,       none,       0,       0x58,      rv,  no,  0,     O64 | D64)
INSTR (POP,               regmem16,   none,       none,       none,       0,       0x8f,      r0,  no,  0,     O16)
INSTR (POP,               regmem32,   none,       none,       none,       0,       0x8f,      r0,  no,  0,     O32 | I64)
INSTR (POP,               regmem64,   none,       none,       none,       0,       0x8f,      r0,  no,  0,     O64 | D64)
INSTR (POP,               ds,         none,       none,       none,       0,       0x1f,      no,  no,  0,     I64)
INSTR (POP,               es,         none,       none,       none,       0,       0x07,      no,  no,  0,     I64)
INSTR (POP,               ss,         none,       none,       none,       0,       0x17,      no,  no,  0,     I64)
INSTR (POP,               fs,         none,       none,       none,       0,       0x0fa1,    no,  no,  0,     0)
INSTR (POP,               gs,         none,       none,       none,       0,       0x0fa9,    no,  no,  0,     0)

INSTR (POPA,              none,       none,       none,       none,       0,       0x61,      no,  no,  0,     O16 | I64)

INSTR (POPAD,             none,       none,       none,       none,       0,       0x61,      no,  no,  0,     O32 | I64)

INSTR (POPCNT,            reg16,      regmem16,   none,       none,       0,       0x0fb8,    r,   no,  0,     O16 | PF3)
INSTR (POPCNT,            reg32,      regmem32,   none,       none,       0,       0x0fb8,    r,   no,  0,     O32 | PF3)
INSTR (POPCNT,            reg64,      regmem64,   none,       none,       0,       0x0fb8,    r,   no,  0,     O64 | PF3)

INSTR (POPF,              none,       none,       none,       none,       0,       0x9d,      no,  no,  0,     O16)

INSTR (POPFD,             none,       none,       none,       none,       0,       0x9d,      no,  no,  0,     O32 | I64)

INSTR (POPFQ,             none,       none,       none,       none,       0,       0x9d,      no,  no,  0,     O64 | D64)

INSTR (PREFETCH,          mem,        none,       none,       none,       0,       0x0f0d,    r0,  no,  0,     0)

INSTR (PREFETCHW,         mem,        none,       none,       none,       0,       0x0f0d,    r1,  no,  0,     0)

INSTR (PREFETCHNTA,       mem,        none,       none,       none,       0,       0x0f18,    r0,  no,  0,     0)

INSTR (PREFETCHT0,        mem,        none,       none,       none,       0,       0x0f18,    r1,  no,  0,     0)

INSTR (PREFETCHT1,        mem,        none,       none,       none,       0,       0x0f18,    r2,  no,  0,     0)

INSTR (PREFETCHT2,        mem,        none,       none,       none,       0,       0x0f18,    r3,  no,  0,     0)

INSTR (PUSH,              reg16,      none,       none,       none,       0,       0x50,      rv,  no,  0,     O16)
INSTR (PUSH,              reg32,      none,       none,       none,       0,       0x50,      rv,  no,  0,     O32 | I64)
INSTR (PUSH,              reg64,      none,       none,       none,       0,       0x50,      rv,  no,  0,     O64 | D64)
INSTR (PUSH,              regmem16,   none,       none,       none,       0,       0xff,      r6,  no,  0,     O16)
INSTR (PUSH,              regmem32,   none,       none,       none,       0,       0xff,      r6,  no,  0,     O32 | I64)
INSTR (PUSH,              regmem64,   none,       none,       none,       0,       0xff,      r6,  no,  0,     O64 | D64)
INSTR (PUSH,              simm8,      none,       none,       none,       0,       0x6a,      ib,  no,  0,     0)
INSTR (PUSH,              imm16,      none,       none,       none,       0,       0x68,      iw,  no,  0,     O16)
INSTR (PUSH,              imm32,      none,       none,       none,       0,       0x68,      id,  no,  0,     O32 | I64)
INSTR (PUSH,              simm32,     none,       none,       none,       0,       0x68,      id,  no,  0,     O64 | D64)
INSTR (PUSH,              cs,         none,       none,       none,       0,       0x0e,      no,  no,  0,     I64)
INSTR (PUSH,              ss,         none,       none,       none,       0,       0x16,      no,  no,  0,     I64)
INSTR (PUSH,              ds,         none,       none,       none,       0,       0x1e,      no,  no,  0,     I64)
INSTR (PUSH,              es,         none,       none,       none,       0,       0x06,      no,  no,  0,     I64)
INSTR (PUSH,              fs,         none,       none,       none,       0,       0x0fa0,    no,  no,  0,     0)
INSTR (PUSH,              gs,         none,       none,       none,       0,       0x0fa8,    no,  no,  0,     0)

INSTR (PUSHA,             none,       none,       none,       none,       0,       0x60,      no,  no,  0,     O16 | I64)

INSTR (PUSHAD,            none,       none,       none,       none,       0,       0x60,      no,  no,  0,     O32 | I64)

INSTR (PUSHF,             none,       none,       none,       none,       0,       0x9c,      no,  no,  0,     O16)

INSTR (PUSHFD,            none,       none,       none,       none,       0,       0x9c,      no,  no,  0,     O32 | I64)

INSTR (PUSHFQ,            none,       none,       none,       none,       0,       0x9c,      no,  no,  0,     O64 | D64)

INSTR (RCL,               regmem8,    one,        none,       none,       0,       0xd0,      r2,  no,  0,     0)
INSTR (RCL,               regmem8,    cl,         none,       none,       0,       0xd2,      r2,  no,  0,     0)
INSTR (RCL,               regmem8,    imm8,       none,       none,       0,       0xc0,      r2,  ib,  0,     0)
INSTR (RCL,               regmem16,   one,        none,       none,       0,       0xd1,      r2,  no,  0,     O16)
INSTR (RCL,               regmem16,   cl,         none,       none,       0,       0xd3,      r2,  no,  0,     O16)
INSTR (RCL,               regmem16,   imm8,       none,       none,       0,       0xc1,      r2,  ib,  0,     O16)
INSTR (RCL,               regmem32,   one,        none,       none,       0,       0xd1,      r2,  no,  0,     O32)
INSTR (RCL,               regmem32,   cl,         none,       none,       0,       0xd3,      r2,  no,  0,     O32)
INSTR (RCL,               regmem32,   imm8,       none,       none,       0,       0xc1,      r2,  ib,  0,     O32)
INSTR (RCL,               regmem64,   one,        none,       none,       0,       0xd1,      r2,  no,  0,     O64)
INSTR (RCL,               regmem64,   cl,         none,       none,       0,       0xd3,      r2,  no,  0,     O64)
INSTR (RCL,               regmem64,   imm8,       none,       none,       0,       0xc1,      r2,  ib,  0,     O64)

INSTR (RCR,               regmem8,    one,        none,       none,       0,       0xd0,      r3,  no,  0,     0)
INSTR (RCR,               regmem8,    cl,         none,       none,       0,       0xd2,      r3,  no,  0,     0)
INSTR (RCR,               regmem8,    imm8,       none,       none,       0,       0xc0,      r3,  ib,  0,     0)
INSTR (RCR,               regmem16,   one,        none,       none,       0,       0xd1,      r3,  no,  0,     O16)
INSTR (RCR,               regmem16,   cl,         none,       none,       0,       0xd3,      r3,  no,  0,     O16)
INSTR (RCR,               regmem16,   imm8,       none,       none,       0,       0xc1,      r3,  ib,  0,     O16)
INSTR (RCR,               regmem32,   one,        none,       none,       0,       0xd1,      r3,  no,  0,     O32)
INSTR (RCR,               regmem32,   cl,         none,       none,       0,       0xd3,      r3,  no,  0,     O32)
INSTR (RCR,               regmem32,   imm8,       none,       none,       0,       0xc1,      r3,  ib,  0,     O32)
INSTR (RCR,               regmem64,   one,        none,       none,       0,       0xd1,      r3,  no,  0,     O64)
INSTR (RCR,               regmem64,   cl,         none,       none,       0,       0xd3,      r3,  no,  0,     O64)
INSTR (RCR,               regmem64,   imm8,       none,       none,       0,       0xc1,      r3,  ib,  0,     O64)

INSTR (RDFSBASE,          reg32rm,    none,       none,       none,       0,       0x0fae,    r0,  no,  0,     I16 | I32 | O32 | PF3)
INSTR (RDFSBASE,          reg64rm,    none,       none,       none,       0,       0x0fae,    r0,  no,  0,     I16 | I32 | O64 | PF3)

INSTR (RDGSBASE,          reg32rm,    none,       none,       none,       0,       0x0fae,    r1,  no,  0,     I16 | I32 | O32 | PF3)
INSTR (RDGSBASE,          reg64rm,    none,       none,       none,       0,       0x0fae,    r1,  no,  0,     I16 | I32 | O64 | PF3)

INSTR (RDPID,             reg32rm,    none,       none,       none,       0,       0x0fc7,    r7,  no,  0,     O32 | I64 | PF3)
INSTR (RDPID,             reg64rm,    none,       none,       none,       0,       0x0fc7,    r7,  no,  0,     O64 | D64 | PF3)

INSTR (RDPRU,             none,       none,       none,       none,       0,       0x0f01,    no,  no,  0xfd,  SFX)

INSTR (RDRAND,            reg16rm,    none,       none,       none,       0,       0x0fc7,    r6,  no,  0,     O16)
INSTR (RDRAND,            reg32rm,    none,       none,       none,       0,       0x0fc7,    r6,  no,  0,     O32)
INSTR (RDRAND,            reg64rm,    none,       none,       none,       0,       0x0fc7,    r6,  no,  0,     O64)

INSTR (RDSEED,            reg16rm,    none,       none,       none,       0,       0x0fc7,    r7,  no,  0,     O16)
INSTR (RDSEED,            reg32rm,    none,       none,       none,       0,       0x0fc7,    r7,  no,  0,     O32)
INSTR (RDSEED,            reg64rm,    none,       none,       none,       0,       0x0fc7,    r7,  no,  0,     O64)

INSTR (RET,               none,       none,       none,       none,       0,       0xc3,      no,  no,  0,     0)
INSTR (RET,               imm16,      none,       none,       none,       0,       0xc2,      iw,  no,  0,     0)

INSTR (RETF,              none,       none,       none,       none,       0,       0xcb,      no,  no,  0,     0)
INSTR (RETF,              imm16,      none,       none,       none,       0,       0xca,      iw,  no,  0,     0)

INSTR (ROL,               regmem8,    one,        none,       none,       0,       0xd0,      r0,  no,  0,     0)
INSTR (ROL,               regmem8,    cl,         none,       none,       0,       0xd2,      r0,  no,  0,     0)
INSTR (ROL,               regmem8,    imm8,       none,       none,       0,       0xc0,      r0,  ib,  0,     0)
INSTR (ROL,               regmem16,   one,        none,       none,       0,       0xd1,      r0,  no,  0,     O16)
INSTR (ROL,               regmem16,   cl,         none,       none,       0,       0xd3,      r0,  no,  0,     O16)
INSTR (ROL,               regmem16,   imm8,       none,       none,       0,       0xc1,      r0,  ib,  0,     O16)
INSTR (ROL,               regmem32,   one,        none,       none,       0,       0xd1,      r0,  no,  0,     O32)
INSTR (ROL,               regmem32,   cl,         none,       none,       0,       0xd3,      r0,  no,  0,     O32)
INSTR (ROL,               regmem32,   imm8,       none,       none,       0,       0xc1,      r0,  ib,  0,     O32)
INSTR (ROL,               regmem64,   one,        none,       none,       0,       0xd1,      r0,  no,  0,     O64)
INSTR (ROL,               regmem64,   cl,         none,       none,       0,       0xd3,      r0,  no,  0,     O64)
INSTR (ROL,               regmem64,   imm8,       none,       none,       0,       0xc1,      r0,  ib,  0,     O64)

INSTR (ROR,               regmem8,    one,        none,       none,       0,       0xd0,      r1,  no,  0,     0)
INSTR (ROR,               regmem8,    cl,         none,       none,       0,       0xd2,      r1,  no,  0,     0)
INSTR (ROR,               regmem8,    imm8,       none,       none,       0,       0xc0,      r1,  ib,  0,     0)
INSTR (ROR,               regmem16,   one,        none,       none,       0,       0xd1,      r1,  no,  0,     O16)
INSTR (ROR,               regmem16,   cl,         none,       none,       0,       0xd3,      r1,  no,  0,     O16)
INSTR (ROR,               regmem16,   imm8,       none,       none,       0,       0xc1,      r1,  ib,  0,     O16)
INSTR (ROR,               regmem32,   one,        none,       none,       0,       0xd1,      r1,  no,  0,     O32)
INSTR (ROR,               regmem32,   cl,         none,       none,       0,       0xd3,      r1,  no,  0,     O32)
INSTR (ROR,               regmem32,   imm8,       none,       none,       0,       0xc1,      r1,  ib,  0,     O32)
INSTR (ROR,               regmem64,   one,        none,       none,       0,       0xd1,      r1,  no,  0,     O64)
INSTR (ROR,               regmem64,   cl,         none,       none,       0,       0xd3,      r1,  no,  0,     O64)
INSTR (ROR,               regmem64,   imm8,       none,       none,       0,       0xc1,      r1,  ib,  0,     O64)

INSTR (RORX,              reg32,      regmem32,   imm8,       none,       0x037b,  0xf0,      r,   ib,  0,     PVEX)
INSTR (RORX,              reg64,      regmem64,   imm8,       none,       0x03fb,  0xf0,      r,   ib,  0,     PVEX)

INSTR (SAHF,              none,       none,       none,       none,       0,       0x9e,      no,  no,  0,     0)

INSTR (SHL,               regmem8,    one,        none,       none,       0,       0xd0,      r4,  no,  0,     0)
INSTR (SHL,               regmem8,    cl,         none,       none,       0,       0xd2,      r4,  no,  0,     0)
INSTR (SHL,               regmem8,    imm8,       none,       none,       0,       0xc0,      r4,  ib,  0,     0)
INSTR (SHL,               regmem16,   one,        none,       none,       0,       0xd1,      r4,  no,  0,     O16)
INSTR (SHL,               regmem16,   cl,         none,       none,       0,       0xd3,      r4,  no,  0,     O16)
INSTR (SHL,               regmem16,   imm8,       none,       none,       0,       0xc1,      r4,  ib,  0,     O16)
INSTR (SHL,               regmem32,   one,        none,       none,       0,       0xd1,      r4,  no,  0,     O32)
INSTR (SHL,               regmem32,   cl,         none,       none,       0,       0xd3,      r4,  no,  0,     O32)
INSTR (SHL,               regmem32,   imm8,       none,       none,       0,       0xc1,      r4,  ib,  0,     O32)
INSTR (SHL,               regmem64,   one,        none,       none,       0,       0xd1,      r4,  no,  0,     O64)
INSTR (SHL,               regmem64,   cl,         none,       none,       0,       0xd3,      r4,  no,  0,     O64)
INSTR (SHL,               regmem64,   imm8,       none,       none,       0,       0xc1,      r4,  ib,  0,     O64)

INSTR (SAL,               regmem8,    one,        none,       none,       0,       0xd0,      r4,  no,  0,     0)
INSTR (SAL,               regmem8,    cl,         none,       none,       0,       0xd2,      r4,  no,  0,     0)
INSTR (SAL,               regmem8,    imm8,       none,       none,       0,       0xc0,      r4,  ib,  0,     0)
INSTR (SAL,               regmem16,   one,        none,       none,       0,       0xd1,      r4,  no,  0,     O16)
INSTR (SAL,               regmem16,   cl,         none,       none,       0,       0xd3,      r4,  no,  0,     O16)
INSTR (SAL,               regmem16,   imm8,       none,       none,       0,       0xc1,      r4,  ib,  0,     O16)
INSTR (SAL,               regmem32,   one,        none,       none,       0,       0xd1,      r4,  no,  0,     O32)
INSTR (SAL,               regmem32,   cl,         none,       none,       0,       0xd3,      r4,  no,  0,     O32)
INSTR (SAL,               regmem32,   imm8,       none,       none,       0,       0xc1,      r4,  ib,  0,     O32)
INSTR (SAL,               regmem64,   one,        none,       none,       0,       0xd1,      r4,  no,  0,     O64)
INSTR (SAL,               regmem64,   cl,         none,       none,       0,       0xd3,      r4,  no,  0,     O64)
INSTR (SAL,               regmem64,   imm8,       none,       none,       0,       0xc1,      r4,  ib,  0,     O64)

INSTR (SAR,               regmem8,    one,        none,       none,       0,       0xd0,      r7,  no,  0,     0)
INSTR (SAR,               regmem8,    cl,         none,       none,       0,       0xd2,      r7,  no,  0,     0)
INSTR (SAR,               regmem8,    imm8,       none,       none,       0,       0xc0,      r7,  ib,  0,     0)
INSTR (SAR,               regmem16,   one,        none,       none,       0,       0xd1,      r7,  no,  0,     O16)
INSTR (SAR,               regmem16,   cl,         none,       none,       0,       0xd3,      r7,  no,  0,     O16)
INSTR (SAR,               regmem16,   imm8,       none,       none,       0,       0xc1,      r7,  ib,  0,     O16)
INSTR (SAR,               regmem32,   one,        none,       none,       0,       0xd1,      r7,  no,  0,     O32)
INSTR (SAR,               regmem32,   cl,         none,       none,       0,       0xd3,      r7,  no,  0,     O32)
INSTR (SAR,               regmem32,   imm8,       none,       none,       0,       0xc1,      r7,  ib,  0,     O32)
INSTR (SAR,               regmem64,   one,        none,       none,       0,       0xd1,      r7,  no,  0,     O64)
INSTR (SAR,               regmem64,   cl,         none,       none,       0,       0xd3,      r7,  no,  0,     O64)
INSTR (SAR,               regmem64,   imm8,       none,       none,       0,       0xc1,      r7,  ib,  0,     O64)

INSTR (SARX,              reg32,      regmem32,   reg32vvvv,  none,       0x0202,  0xf7,      r,   no,  0,     PVEX)
INSTR (SARX,              reg64,      regmem64,   reg64vvvv,  none,       0x0282,  0xf7,      r,   no,  0,     PVEX)

INSTR (SBB,               al,         imm8,       none,       none,       0,       0x1c,      ib,  no,  0,     0)
INSTR (SBB,               regmem8,    imm8,       none,       none,       0,       0x80,      r3,  ib,  0,     PLOCK)
INSTR (SBB,               regmem8,    imm8,       none,       none,       0,       0x82,      r3,  ib,  0,     I64 | PLOCK)
INSTR (SBB,               regmem16,   simm8,      none,       none,       0,       0x83,      r3,  ib,  0,     O16 | PLOCK)
INSTR (SBB,               regmem32,   simm8,      none,       none,       0,       0x83,      r3,  ib,  0,     O32 | PLOCK)
INSTR (SBB,               regmem64,   simm8,      none,       none,       0,       0x83,      r3,  ib,  0,     O64 | PLOCK)
INSTR (SBB,               ax,         imm16,      none,       none,       0,       0x1d,      iw,  no,  0,     O16)
INSTR (SBB,               regmem16,   imm16,      none,       none,       0,       0x81,      r3,  iw,  0,     O16 | PLOCK)
INSTR (SBB,               eax,        imm32,      none,       none,       0,       0x1d,      id,  no,  0,     O32)
INSTR (SBB,               regmem32,   imm32,      none,       none,       0,       0x81,      r3,  id,  0,     O32 | PLOCK)
INSTR (SBB,               rax,        simm32,     none,       none,       0,       0x1d,      id,  no,  0,     O64)
INSTR (SBB,               regmem64,   simm32,     none,       none,       0,       0x81,      r3,  id,  0,     O64 | PLOCK)
INSTR (SBB,               regmem8,    reg8,       none,       none,       0,       0x18,      r,   no,  0,     PLOCK)
INSTR (SBB,               regmem16,   reg16,      none,       none,       0,       0x19,      r,   no,  0,     O16 | PLOCK)
INSTR (SBB,               regmem32,   reg32,      none,       none,       0,       0x19,      r,   no,  0,     O32 | PLOCK)
INSTR (SBB,               regmem64,   reg64,      none,       none,       0,       0x19,      r,   no,  0,     O64 | PLOCK)
INSTR (SBB,               reg8,       regmem8,    none,       none,       0,       0x1a,      r,   no,  0,     0)
INSTR (SBB,               reg16,      regmem16,   none,       none,       0,       0x1b,      r,   no,  0,     O16)
INSTR (SBB,               reg32,      regmem32,   none,       none,       0,       0x1b,      r,   no,  0,     O32)
INSTR (SBB,               reg64,      regmem64,   none,       none,       0,       0x1b,      r,   no,  0,     O64)

INSTR (SCASB,             none,       none,       none,       none,       0,       0xae,      no,  no,  0,     PREPE | PREPNE)

INSTR (SCASW,             none,       none,       none,       none,       0,       0xaf,      no,  no,  0,     O16 | PREPE | PREPNE)

INSTR (SCASD,             none,       none,       none,       none,       0,       0xaf,      no,  no,  0,     O32 | PREPE | PREPNE)

INSTR (SCASQ,             none,       none,       none,       none,       0,       0xaf,      no,  no,  0,     O64 | PREPE | PREPNE)

INSTR (SETO,              regmem8,    none,       none,       none,       0,       0x0f90,    r0,  no,  0,     0)

INSTR (SETNO,             regmem8,    none,       none,       none,       0,       0x0f91,    r0,  no,  0,     0)

INSTR (SETC,              regmem8,    none,       none,       none,       0,       0x0f92,    r0,  no,  0,     0)

INSTR (SETB,              regmem8,    none,       none,       none,       0,       0x0f92,    r0,  no,  0,     0)

INSTR (SETNAE,            regmem8,    none,       none,       none,       0,       0x0f92,    r0,  no,  0,     0)

INSTR (SETNC,             regmem8,    none,       none,       none,       0,       0x0f93,    r0,  no,  0,     0)

INSTR (SETNB,             regmem8,    none,       none,       none,       0,       0x0f93,    r0,  no,  0,     0)

INSTR (SETAE,             regmem8,    none,       none,       none,       0,       0x0f93,    r0,  no,  0,     0)

INSTR (SETZ,              regmem8,    none,       none,       none,       0,       0x0f94,    r0,  no,  0,     0)

INSTR (SETE,              regmem8,    none,       none,       none,       0,       0x0f94,    r0,  no,  0,     0)

INSTR (SETNZ,             regmem8,    none,       none,       none,       0,       0x0f95,    r0,  no,  0,     0)

INSTR (SETNE,             regmem8,    none,       none,       none,       0,       0x0f95,    r0,  no,  0,     0)

INSTR (SETNA,             regmem8,    none,       none,       none,       0,       0x0f96,    r0,  no,  0,     0)

INSTR (SETBE,             regmem8,    none,       none,       none,       0,       0x0f96,    r0,  no,  0,     0)

INSTR (SETA,              regmem8,    none,       none,       none,       0,       0x0f97,    r0,  no,  0,     0)

INSTR (SETNBE,            regmem8,    none,       none,       none,       0,       0x0f97,    r0,  no,  0,     0)

INSTR (SETS,              regmem8,    none,       none,       none,       0,       0x0f98,    r0,  no,  0,     0)

INSTR (SETNS,             regmem8,    none,       none,       none,       0,       0x0f99,    r0,  no,  0,     0)

INSTR (SETPE,             regmem8,    none,       none,       none,       0,       0x0f9a,    r0,  no,  0,     0)

INSTR (SETP,              regmem8,    none,       none,       none,       0,       0x0f9a,    r0,  no,  0,     0)

INSTR (SETPO,             regmem8,    none,       none,       none,       0,       0x0f9b,    r0,  no,  0,     0)

INSTR (SETNP,             regmem8,    none,       none,       none,       0,       0x0f9b,    r0,  no,  0,     0)

INSTR (SETL,              regmem8,    none,       none,       none,       0,       0x0f9c,    r0,  no,  0,     0)

INSTR (SETNGE,            regmem8,    none,       none,       none,       0,       0x0f9c,    r0,  no,  0,     0)

INSTR (SETNL,             regmem8,    none,       none,       none,       0,       0x0f9d,    r0,  no,  0,     0)

INSTR (SETGE,             regmem8,    none,       none,       none,       0,       0x0f9d,    r0,  no,  0,     0)

INSTR (SETNG,             regmem8,    none,       none,       none,       0,       0x0f9e,    r0,  no,  0,     0)

INSTR (SETLE,             regmem8,    none,       none,       none,       0,       0x0f9e,    r0,  no,  0,     0)

INSTR (SETG,              regmem8,    none,       none,       none,       0,       0x0f9f,    r0,  no,  0,     0)

INSTR (SETNLE,            regmem8,    none,       none,       none,       0,       0x0f9f,    r0,  no,  0,     0)

INSTR (SFENCE,            none,       none,       none,       none,       0,       0x0fae,    no,  no,  0xf8,  SFX)

INSTR (SHLD,              regmem16,   reg16,      imm8,       none,       0,       0x0fa4,    r,   ib,  0,     O16)
INSTR (SHLD,              regmem16,   reg16,      cl,         none,       0,       0x0fa5,    r,   no,  0,     O16)
INSTR (SHLD,              regmem32,   reg32,      imm8,       none,       0,       0x0fa4,    r,   ib,  0,     O32)
INSTR (SHLD,              regmem32,   reg32,      cl,         none,       0,       0x0fa5,    r,   no,  0,     O32)
INSTR (SHLD,              regmem64,   reg64,      imm8,       none,       0,       0x0fa4,    r,   ib,  0,     O64)
INSTR (SHLD,              regmem64,   reg64,      cl,         none,       0,       0x0fa5,    r,   no,  0,     O64)

INSTR (SHLX,              reg32,      regmem32,   reg32vvvv,  none,       0x0201,  0xf7,      r,   no,  0,     PVEX)
INSTR (SHLX,              reg64,      regmem64,   reg64vvvv,  none,       0x0281,  0xf7,      r,   no,  0,     PVEX)

INSTR (SHR,               regmem8,    one,        none,       none,       0,       0xd0,      r5,  no,  0,     0)
INSTR (SHR,               regmem8,    cl,         none,       none,       0,       0xd2,      r5,  no,  0,     0)
INSTR (SHR,               regmem8,    imm8,       none,       none,       0,       0xc0,      r5,  ib,  0,     0)
INSTR (SHR,               regmem16,   one,        none,       none,       0,       0xd1,      r5,  no,  0,     O16)
INSTR (SHR,               regmem16,   cl,         none,       none,       0,       0xd3,      r5,  no,  0,     O16)
INSTR (SHR,               regmem16,   imm8,       none,       none,       0,       0xc1,      r5,  ib,  0,     O16)
INSTR (SHR,               regmem32,   one,        none,       none,       0,       0xd1,      r5,  no,  0,     O32)
INSTR (SHR,               regmem32,   cl,         none,       none,       0,       0xd3,      r5,  no,  0,     O32)
INSTR (SHR,               regmem32,   imm8,       none,       none,       0,       0xc1,      r5,  ib,  0,     O32)
INSTR (SHR,               regmem64,   one,        none,       none,       0,       0xd1,      r5,  no,  0,     O64)
INSTR (SHR,               regmem64,   cl,         none,       none,       0,       0xd3,      r5,  no,  0,     O64)
INSTR (SHR,               regmem64,   imm8,       none,       none,       0,       0xc1,      r5,  ib,  0,     O64)

INSTR (SHRD,              regmem16,   reg16,      imm8,       none,       0,       0x0fac,    r,   ib,  0,     O16)
INSTR (SHRD,              regmem16,   reg16,      cl,         none,       0,       0x0fad,    r,   no,  0,     O16)
INSTR (SHRD,              regmem32,   reg32,      imm8,       none,       0,       0x0fac,    r,   ib,  0,     O32)
INSTR (SHRD,              regmem32,   reg32,      cl,         none,       0,       0x0fad,    r,   no,  0,     O32)
INSTR (SHRD,              regmem64,   reg64,      imm8,       none,       0,       0x0fac,    r,   ib,  0,     O64)
INSTR (SHRD,              regmem64,   reg64,      cl,         none,       0,       0x0fad,    r,   no,  0,     O64)

INSTR (SHRX,              reg32,      regmem32,   reg32vvvv,  none,       0x0203,  0xf7,      r,   no,  0,     PVEX)
INSTR (SHRX,              reg64,      regmem64,   reg64vvvv,  none,       0x0283,  0xf7,      r,   no,  0,     PVEX)

INSTR (SLWPCB,            reg32rm,    none,       none,       none,       0x0978,  0x12,      r1,  no,  0,     PXOP)
INSTR (SLWPCB,            reg64rm,    none,       none,       none,       0x09f8,  0x12,      r1,  no,  0,     PXOP)

INSTR (STAC,              none,       none,       none,       none,       0,       0x0f01,    no,  no,  0xcb,  SFX)

INSTR (STC,               none,       none,       none,       none,       0,       0xf9,      no,  no,  0,     0)

INSTR (STD,               none,       none,       none,       none,       0,       0xfd,      no,  no,  0,     0)

INSTR (STOSB,             none,       none,       none,       none,       0,       0xaa,      no,  no,  0,     PREP)

INSTR (STOSW,             none,       none,       none,       none,       0,       0xab,      no,  no,  0,     O16 | PREP)

INSTR (STOSD,             none,       none,       none,       none,       0,       0xab,      no,  no,  0,     O32 | PREP)

INSTR (STOSQ,             none,       none,       none,       none,       0,       0xab,      no,  no,  0,     O64 | PREP)

INSTR (SUB,               al,         imm8,       none,       none,       0,       0x2c,      ib,  no,  0,     0)
INSTR (SUB,               regmem8,    imm8,       none,       none,       0,       0x80,      r5,  ib,  0,     PLOCK)
INSTR (SUB,               regmem8,    imm8,       none,       none,       0,       0x82,      r5,  ib,  0,     I64 | PLOCK)
INSTR (SUB,               regmem16,   simm8,      none,       none,       0,       0x83,      r5,  ib,  0,     O16 | PLOCK)
INSTR (SUB,               regmem32,   simm8,      none,       none,       0,       0x83,      r5,  ib,  0,     O32 | PLOCK)
INSTR (SUB,               regmem64,   simm8,      none,       none,       0,       0x83,      r5,  ib,  0,     O64 | PLOCK)
INSTR (SUB,               ax,         imm16,      none,       none,       0,       0x2d,      iw,  no,  0,     O16)
INSTR (SUB,               regmem16,   imm16,      none,       none,       0,       0x81,      r5,  iw,  0,     O16 | PLOCK)
INSTR (SUB,               eax,        imm32,      none,       none,       0,       0x2d,      id,  no,  0,     O32)
INSTR (SUB,               regmem32,   imm32,      none,       none,       0,       0x81,      r5,  id,  0,     O32 | PLOCK)
INSTR (SUB,               rax,        simm32,     none,       none,       0,       0x2d,      id,  no,  0,     O64)
INSTR (SUB,               regmem64,   simm32,     none,       none,       0,       0x81,      r5,  id,  0,     O64 | PLOCK)
INSTR (SUB,               regmem8,    reg8,       none,       none,       0,       0x28,      r,   no,  0,     PLOCK)
INSTR (SUB,               regmem16,   reg16,      none,       none,       0,       0x29,      r,   no,  0,     O16 | PLOCK)
INSTR (SUB,               regmem32,   reg32,      none,       none,       0,       0x29,      r,   no,  0,     O32 | PLOCK)
INSTR (SUB,               regmem64,   reg64,      none,       none,       0,       0x29,      r,   no,  0,     O64 | PLOCK)
INSTR (SUB,               reg8,       regmem8,    none,       none,       0,       0x2a,      r,   no,  0,     0)
INSTR (SUB,               reg16,      regmem16,   none,       none,       0,       0x2b,      r,   no,  0,     O16)
INSTR (SUB,               reg32,      regmem32,   none,       none,       0,       0x2b,      r,   no,  0,     O32)
INSTR (SUB,               reg64,      regmem64,   none,       none,       0,       0x2b,      r,   no,  0,     O64)

INSTR (TEST,              al,         imm8,       none,       none,       0,       0xa8,      ib,  no,  0,     0)
INSTR (TEST,              regmem8,    imm8,       none,       none,       0,       0xf6,      r0,  ib,  0,     0)
INSTR (TEST,              regmem8,    imm8,       none,       none,       0,       0xf6,      r1,  ib,  0,     0)
INSTR (TEST,              ax,         imm16,      none,       none,       0,       0xa9,      iw,  no,  0,     O16)
INSTR (TEST,              regmem16,   imm16,      none,       none,       0,       0xf7,      r0,  iw,  0,     O16)
INSTR (TEST,              regmem16,   imm16,      none,       none,       0,       0xf7,      r1,  iw,  0,     O16)
INSTR (TEST,              eax,        imm32,      none,       none,       0,       0xa9,      id,  no,  0,     O32)
INSTR (TEST,              regmem32,   imm32,      none,       none,       0,       0xf7,      r0,  id,  0,     O32)
INSTR (TEST,              regmem32,   imm32,      none,       none,       0,       0xf7,      r1,  id,  0,     O32)
INSTR (TEST,              rax,        simm32,     none,       none,       0,       0xa9,      id,  no,  0,     O64)
INSTR (TEST,              regmem64,   simm32,     none,       none,       0,       0xf7,      r0,  id,  0,     O64)
INSTR (TEST,              regmem64,   simm32,     none,       none,       0,       0xf7,      r1,  id,  0,     O64)
INSTR (TEST,              regmem8,    reg8,       none,       none,       0,       0x84,      r,   no,  0,     0)
INSTR (TEST,              regmem16,   reg16,      none,       none,       0,       0x85,      r,   no,  0,     O16)
INSTR (TEST,              regmem32,   reg32,      none,       none,       0,       0x85,      r,   no,  0,     O32)
INSTR (TEST,              regmem64,   reg64,      none,       none,       0,       0x85,      r,   no,  0,     O64)

INSTR (WRFSBASE,          reg32rm,    none,       none,       none,       0,       0x0fae,    r2,  no,  0,     I16 | I32 | O32 | PF3)
INSTR (WRFSBASE,          reg64rm,    none,       none,       none,       0,       0x0fae,    r2,  no,  0,     I16 | I32 | O64 | PF3)

INSTR (WRGSBASE,          reg32rm,    none,       none,       none,       0,       0x0fae,    r3,  no,  0,     I16 | I32 | O32 | PF3)
INSTR (WRGSBASE,          reg64rm,    none,       none,       none,       0,       0x0fae,    r3,  no,  0,     I16 | I32 | O64 | PF3)

INSTR (XADD,              regmem8,    reg8,       none,       none,       0,       0x0fc0,    r,   no,  0,     PLOCK)
INSTR (XADD,              regmem16,   reg16,      none,       none,       0,       0x0fc1,    r,   no,  0,     O16 | PLOCK)
INSTR (XADD,              regmem32,   reg32,      none,       none,       0,       0x0fc1,    r,   no,  0,     O32 | PLOCK)
INSTR (XADD,              regmem64,   reg64,      none,       none,       0,       0x0fc1,    r,   no,  0,     O64 | PLOCK)

INSTR (XCHG,              ax,         reg16,      none,       none,       0,       0x90,      rv,  no,  0,     O16)
INSTR (XCHG,              reg16,      ax,         none,       none,       0,       0x90,      rv,  no,  0,     O16)
INSTR (XCHG,              eax,        reg32,      none,       none,       0,       0x90,      rv,  no,  0,     O32)
INSTR (XCHG,              reg32,      eax,        none,       none,       0,       0x90,      rv,  no,  0,     O32)
INSTR (XCHG,              rax,        reg64,      none,       none,       0,       0x90,      rv,  no,  0,     O64)
INSTR (XCHG,              reg64,      rax,        none,       none,       0,       0x90,      rv,  no,  0,     O64)
INSTR (XCHG,              reg8,       regmem8,    none,       none,       0,       0x86,      r,   no,  0,     0)
INSTR (XCHG,              regmem8,    reg8,       none,       none,       0,       0x86,      r,   no,  0,     PLOCK)
INSTR (XCHG,              reg16,      regmem16,   none,       none,       0,       0x87,      r,   no,  0,     O16)
INSTR (XCHG,              regmem16,   reg16,      none,       none,       0,       0x87,      r,   no,  0,     O16 | PLOCK)
INSTR (XCHG,              reg32,      regmem32,   none,       none,       0,       0x87,      r,   no,  0,     O32)
INSTR (XCHG,              regmem32,   reg32,      none,       none,       0,       0x87,      r,   no,  0,     O32 | PLOCK)
INSTR (XCHG,              reg64,      regmem64,   none,       none,       0,       0x87,      r,   no,  0,     O64)
INSTR (XCHG,              regmem64,   reg64,      none,       none,       0,       0x87,      r,   no,  0,     O64 | PLOCK)

INSTR (XLATB,             none,       none,       none,       none,       0,       0xd7,      no,  no,  0,     0)

INSTR (XOR,               al,         imm8,       none,       none,       0,       0x34,      ib,  no,  0,     0)
INSTR (XOR,               regmem8,    imm8,       none,       none,       0,       0x80,      r6,  ib,  0,     PLOCK)
INSTR (XOR,               regmem8,    imm8,       none,       none,       0,       0x82,      r6,  ib,  0,     I64 | PLOCK)
INSTR (XOR,               regmem16,   simm8,      none,       none,       0,       0x83,      r6,  ib,  0,     O16 | PLOCK)
INSTR (XOR,               regmem32,   simm8,      none,       none,       0,       0x83,      r6,  ib,  0,     O32 | PLOCK)
INSTR (XOR,               regmem64,   simm8,      none,       none,       0,       0x83,      r6,  ib,  0,     O64 | PLOCK)
INSTR (XOR,               ax,         imm16,      none,       none,       0,       0x35,      iw,  no,  0,     O16)
INSTR (XOR,               regmem16,   imm16,      none,       none,       0,       0x81,      r6,  iw,  0,     O16 | PLOCK)
INSTR (XOR,               eax,        imm32,      none,       none,       0,       0x35,      id,  no,  0,     O32)
INSTR (XOR,               regmem32,   imm32,      none,       none,       0,       0x81,      r6,  id,  0,     O32 | PLOCK)
INSTR (XOR,               rax,        simm32,     none,       none,       0,       0x35,      id,  no,  0,     O64)
INSTR (XOR,               regmem64,   simm32,     none,       none,       0,       0x81,      r6,  id,  0,     O64 | PLOCK)
INSTR (XOR,               regmem8,    reg8,       none,       none,       0,       0x30,      r,   no,  0,     PLOCK)
INSTR (XOR,               regmem16,   reg16,      none,       none,       0,       0x31,      r,   no,  0,     O16 | PLOCK)
INSTR (XOR,               regmem32,   reg32,      none,       none,       0,       0x31,      r,   no,  0,     O32 | PLOCK)
INSTR (XOR,               regmem64,   reg64,      none,       none,       0,       0x31,      r,   no,  0,     O64 | PLOCK)
INSTR (XOR,               reg8,       regmem8,    none,       none,       0,       0x32,      r,   no,  0,     0)
INSTR (XOR,               reg16,      regmem16,   none,       none,       0,       0x33,      r,   no,  0,     O16)
INSTR (XOR,               reg32,      regmem32,   none,       none,       0,       0x33,      r,   no,  0,     O32)
INSTR (XOR,               reg64,      regmem64,   none,       none,       0,       0x33,      r,   no,  0,     O64)

// System Instruction Reference

INSTR (ARPL,              regmem16,   reg16,      none,       none,       0,       0x63,      r,   no,  0,     I64)

INSTR (CLI,               none,       none,       none,       none,       0,       0xfa,      no,  no,  0,     0)

INSTR (CLTS,              none,       none,       none,       none,       0,       0x0f06,    no,  no,  0,     0)

INSTR (CLRSSBSY,          mem64,      none,       none,       none,       0,       0x0fae,    r6,  no,  0,     PF3)

INSTR (HLT,               none,       none,       none,       none,       0,       0xf4,      no,  no,  0,     0)

INSTR (INCSSP,            reg32rm,    none,       none,       none,       0,       0x0fae,    r5,  no,  0,     O16 | O32 | PF3)
INSTR (INCSSP,            reg64rm,    none,       none,       none,       0,       0x0fae,    r5,  no,  0,     O64 | PF3)

INSTR (INT3,              none,       none,       none,       none,       0,       0xcc,      no,  no,  0,     0)

INSTR (INVD,              none,       none,       none,       none,       0,       0x0f08,    no,  no,  0,     0)

INSTR (INVLPG,            mem,        none,       none,       none,       0,       0x0f01,    r7,  no,  0,     0)

INSTR (IRET,              none,       none,       none,       none,       0,       0xcf,      no,  no,  0,     O16)

INSTR (IRETD,             none,       none,       none,       none,       0,       0xcf,      no,  no,  0,     O32)

INSTR (IRETQ,             none,       none,       none,       none,       0,       0xcf,      no,  no,  0,     O64)

INSTR (LAR,               reg16,      regmem16,   none,       none,       0,       0x0f02,    r,   no,  0,     O16)
INSTR (LAR,               reg32,      regmem16,   none,       none,       0,       0x0f02,    r,   no,  0,     O32)
INSTR (LAR,               reg64,      regmem16,   none,       none,       0,       0x0f02,    r,   no,  0,     O64)

INSTR (LGDT,              mem,        none,       none,       none,       0,       0x0f01,    r2,  no,  0,     0)

INSTR (LIDT,              mem,        none,       none,       none,       0,       0x0f01,    r3,  no,  0,     0)

INSTR (LLDT,              regmem16,   none,       none,       none,       0,       0x0f00,    r2,  no,  0,     0)

INSTR (LMSW,              regmem16,   none,       none,       none,       0,       0x0f01,    r6,  no,  0,     0)

INSTR (LSL,               reg16,      regmem16,   none,       none,       0,       0x0f03,    r,   no,  0,     O16)
INSTR (LSL,               reg32,      regmem16,   none,       none,       0,       0x0f03,    r,   no,  0,     O32)
INSTR (LSL,               reg64,      regmem16,   none,       none,       0,       0x0f03,    r,   no,  0,     O64)

INSTR (LTR,               regmem16,   none,       none,       none,       0,       0x0f00,    r3,  no,  0,     0)

INSTR (RDMSR,             none,       none,       none,       none,       0,       0x0f32,    no,  no,  0,     0)

INSTR (RDPKRU,            none,       none,       none,       none,       0,       0x0f01,    no,  no,  0xee,  SFX)

INSTR (RDPMC,             none,       none,       none,       none,       0,       0x0f33,    no,  no,  0,     0)

INSTR (RDSSPD,            reg32rm,    none,       none,       none,       0,       0x0f1e,    r1,  no,  0,     O16 | O32 | PF3)

INSTR (RDSSPQ,            reg64rm,    none,       none,       none,       0,       0x0f1e,    r1,  no,  0,     O64 | PF3)

INSTR (RDTSC,             none,       none,       none,       none,       0,       0x0f31,    no,  no,  0,     0)

INSTR (RSM,               none,       none,       none,       none,       0,       0x0faa,    no,  no,  0,     0)

INSTR (SAVEPREVSSP,       none,       none,       none,       none,       0,       0x0f01,    no,  no,  0xea,  PF3 | SFX)

INSTR (SETSSBSY,          none,       none,       none,       none,       0,       0x0f01,    no,  no,  0xe8,  PF3 | SFX)

INSTR (SGDT,              mem,        none,       none,       none,       0,       0x0f01,    r0,  no,  0,     0)

INSTR (SIDT,              mem,        none,       none,       none,       0,       0x0f01,    r1,  no,  0,     0)

INSTR (SLDT,              regmem16,   none,       none,       none,       0,       0x0f00,    r0,  no,  0,     0)

INSTR (SMSW,              regmem16,   none,       none,       none,       0,       0x0f01,    r4,  no,  0,     0)

INSTR (STI,               none,       none,       none,       none,       0,       0xfb,      no,  no,  0,     0)

INSTR (STR,               regmem16,   none,       none,       none,       0,       0x0f00,    r1,  no,  0,     0)

INSTR (SYSCALL,           none,       none,       none,       none,       0,       0x0f05,    no,  no,  0,     0)

INSTR (SYSENTER,          none,       none,       none,       none,       0,       0x0f34,    no,  no,  0,     0)

INSTR (SYSEXIT,           none,       none,       none,       none,       0,       0x0f35,    no,  no,  0,     0)

INSTR (SYSRET,            none,       none,       none,       none,       0,       0x0f07,    no,  no,  0,     0)

INSTR (TLBSYNC,           none,       none,       none,       none,       0,       0x0f01,    no,  no,  0xff,  SFX)

INSTR (UD0,               none,       none,       none,       none,       0,       0x0fff,    no,  no,  0,     0)

INSTR (UD1,               none,       none,       none,       none,       0,       0x0fb9,    r,   no,  0,     0)

INSTR (UD2,               none,       none,       none,       none,       0,       0x0f0b,    no,  no,  0,     0)

INSTR (VERR,              regmem16,   none,       none,       none,       0,       0x0f00,    r4,  no,  0,     0)

INSTR (VERW,              regmem16,   none,       none,       none,       0,       0x0f00,    r5,  no,  0,     0)

INSTR (WBINVD,            none,       none,       none,       none,       0,       0x0f09,    no,  no,  0,     0)

INSTR (WBNOINVD,          none,       none,       none,       none,       0,       0x0f09,    no,  no,  0,     PF3)

INSTR (WRMSR,             none,       none,       none,       none,       0,       0x0f30,    no,  no,  0,     0)

INSTR (WRPKRU,            none,       none,       none,       none,       0,       0x0f01,    no,  no,  0xef,  SFX)

INSTR (CLGI,              none,       none,       none,       none,       0,       0x0f01,    no,  no,  0xdd,  SFX)

INSTR (CLZERO,            none,       none,       none,       none,       0,       0x0f01,    no,  no,  0xfc,  SFX)

INSTR (INVLPGA,           ax,         ecx,        none,       none,       0,       0x0f01,    no,  no,  0xdf,  O16 | SFX)
INSTR (INVLPGA,           eax,        ecx,        none,       none,       0,       0x0f01,    no,  no,  0xdf,  O32 | SFX)
INSTR (INVLPGA,           rax,        ecx,        none,       none,       0,       0x0f01,    no,  no,  0xdf,  O64 | SFX)

INSTR (INVLPGB,           none,       none,       none,       none,       0,       0x0f01,    no,  no,  0xfe,  SFX)

INSTR (INVPCID,           reg32,      mem128,     none,       none,       0,       0x0f3882,  r,   no,  0,     O16 | O32 | P66)
INSTR (INVPCID,           reg64,      mem128,     none,       none,       0,       0x0f3882,  r,   no,  0,     O64 | P66)

INSTR (MONITOR,           none,       none,       none,       none,       0,       0x0f01,    no,  no,  0xc8,  SFX)

INSTR (MONITORX,          none,       none,       none,       none,       0,       0x0f01,    no,  no,  0xfa,  SFX)

INSTR (MWAIT,             none,       none,       none,       none,       0,       0x0f01,    no,  no,  0xc9,  SFX)

INSTR (MWAITX,            none,       none,       none,       none,       0,       0x0f01,    no,  no,  0xfb,  SFX)

INSTR (PSMASH,            none,       none,       none,       none,       0,       0x0f01,    no,  no,  0xff,  PF3 | SFX)

INSTR (PVALIDATE,         none,       none,       none,       none,       0,       0x0f01,    no,  no,  0xff,  PF2 | SFX)

INSTR (RDTSCP,            none,       none,       none,       none,       0,       0x0f01,    no,  no,  0xf9,  SFX)

INSTR (RMPADJUST,         none,       none,       none,       none,       0,       0x0f01,    no,  no,  0xfe,  PF3 | SFX)

INSTR (RMPQUERY,          none,       none,       none,       none,       0,       0x0f01,    no,  no,  0xfd,  PF3 | SFX)

INSTR (RMPUPDATE,         none,       none,       none,       none,       0,       0x0f01,    no,  no,  0xfe,  PF2 | SFX)

INSTR (SKINIT,            eax,        none,       none,       none,       0,       0x0f01,    no,  no,  0xde,  SFX)

INSTR (STGI,              none,       none,       none,       none,       0,       0x0f01,    no,  no,  0xdc,  SFX)

INSTR (SWAPGS,            none,       none,       none,       none,       0,       0x0f01,    no,  no,  0xf8,  I16 | I32 | SFX)

INSTR (VMLOAD,            ax,         none,       none,       none,       0,       0x0f01,    no,  no,  0xda,  O16 | SFX)
INSTR (VMLOAD,            eax,        none,       none,       none,       0,       0x0f01,    no,  no,  0xda,  O32 | SFX)
INSTR (VMLOAD,            rax,        none,       none,       none,       0,       0x0f01,    no,  no,  0xda,  O64 | SFX)

INSTR (VMMCALL,           none,       none,       none,       none,       0,       0x0f01,    no,  no,  0xd9,  SFX)

INSTR (VMGEXIT,           none,       none,       none,       none,       0,       0x0f01,    no,  no,  0xd9,  PF2 | SFX)
INSTR (VMGEXIT,           none,       none,       none,       none,       0,       0x0f01,    no,  no,  0xd9,  PF3 | SFX)

INSTR (VMRUN,             ax,         none,       none,       none,       0,       0x0f01,    no,  no,  0xd8,  O16 | SFX)
INSTR (VMRUN,             eax,        none,       none,       none,       0,       0x0f01,    no,  no,  0xd8,  O32 | SFX)
INSTR (VMRUN,             rax,        none,       none,       none,       0,       0x0f01,    no,  no,  0xd8,  O64 | SFX)

INSTR (VMSAVE,            ax,         none,       none,       none,       0,       0x0f01,    no,  no,  0xdb,  O16 | SFX)
INSTR (VMSAVE,            eax,        none,       none,       none,       0,       0x0f01,    no,  no,  0xdb,  O32 | SFX)
INSTR (VMSAVE,            rax,        none,       none,       none,       0,       0x0f01,    no,  no,  0xdb,  O64 | SFX)

INSTR (WRSSD,             mem32,      reg32,      none,       none,       0,       0x0f38f6,  r,   no,  0,     O16 | O32)
INSTR (WRSSQ,             mem64,      reg64,      none,       none,       0,       0x0f38f6,  r,   no,  0,     O64)

INSTR (WRUSSD,            mem32,      reg32,      none,       none,       0,       0x0f38f5,  r,   no,  0,     O16 | O32 | P66)
INSTR (WRUSSQ,            mem64,      reg64,      none,       none,       0,       0x0f38f5,  r,   no,  0,     O64 | P66)

// 128-Bit and 256-Bit Media Instruction Reference

INSTR (ADDPD,             xmm,        xmmmem128,  none,       none,       0,       0x0f58,    r,   no,  0,     P66)

INSTR (VADDPD,            xmm,        xvvvv,      xmmmem128,  none,       0x0101,  0x58,      r,   no,  0,     PVEX)
INSTR (VADDPD,            ymm,        yvvvv,      ymmmem256,  none,       0x0105,  0x58,      r,   no,  0,     PVEX)

INSTR (ADDSD,             xmm,        xmmmem64,   none,       none,       0,       0x0f58,    r,   no,  0,     PF2)

INSTR (VADDSD,            xmm,        xvvvv,      xmmmem64,   none,       0x0103,  0x58,      r,   no,  0,     PVEX)

INSTR (ADDSS,             xmm,        xmmmem32,   none,       none,       0,       0x0f58,    r,   no,  0,     PF3)

INSTR (VADDSS,            xmm,        xvvvv,      xmmmem32,   none,       0x0102,  0x58,      r,   no,  0,     PVEX)

INSTR (ADDPS,             xmm,        xmmmem128,  none,       none,       0,       0x0f58,    r,   no,  0,     0)

INSTR (VADDPS,            xmm,        xvvvv,      xmmmem128,  none,       0x0100,  0x58,      r,   no,  0,     PVEX)
INSTR (VADDPS,            ymm,        yvvvv,      ymmmem256,  none,       0x0104,  0x58,      r,   no,  0,     PVEX)

INSTR (ADDSUBPD,          xmm,        xmmmem128,  none,       none,       0,       0x0fd0,    r,   no,  0,     P66)

INSTR (VADDSUBPD,         xmm,        xvvvv,      xmmmem128,  none,       0x0101,  0xd0,      r,   no,  0,     PVEX)
INSTR (VADDSUBPD,         ymm,        yvvvv,      ymmmem256,  none,       0x0105,  0xd0,      r,   no,  0,     PVEX)

INSTR (ADDSUBPS,          xmm,        xmmmem128,  none,       none,       0,       0x0fd0,    r,   no,  0,     PF2)

INSTR (VADDSUBPS,         xmm,        xvvvv,      xmmmem128,  none,       0x0103,  0xd0,      r,   no,  0,     PVEX)
INSTR (VADDSUBPS,         ymm,        yvvvv,      ymmmem256,  none,       0x0107,  0xd0,      r,   no,  0,     PVEX)

INSTR (AESDEC,            xmm,        xmmmem128,  none,       none,       0,       0x0f38de,  r,   no,  0,     P66)

INSTR (VAESDEC,           xmm,        xvvvv,      xmmmem128,  none,       0x0201,  0xde,      r,   no,  0,     PVEX)

INSTR (AESDECLAST,        xmm,        xmmmem128,  none,       none,       0,       0x0f38df,  r,   no,  0,     P66)

INSTR (VAESDECLAST,       xmm,        xvvvv,      xmmmem128,  none,       0x0201,  0xdf,      r,   no,  0,     PVEX)

INSTR (AESENC,            xmm,        xmmmem128,  none,       none,       0,       0x0f38dc,  r,   no,  0,     P66)

INSTR (VAESENC,           xmm,        xvvvv,      xmmmem128,  none,       0x0201,  0xdc,      r,   no,  0,     PVEX)

INSTR (AESENCLAST,        xmm,        xmmmem128,  none,       none,       0,       0x0f38dd,  r,   no,  0,     P66)

INSTR (VAESENCLAST,       xmm,        xvvvv,      xmmmem128,  none,       0x0201,  0xdd,      r,   no,  0,     PVEX)

INSTR (AESIMC,            xmm,        xmmmem128,  none,       none,       0,       0x0f38db,  r,   no,  0,     P66)

INSTR (VAESIMC,           xmm,        xmmmem128,  none,       none,       0x0201,  0xdb,      r,   no,  0,     PVEX)

INSTR (AESKEYGENASSIST,   xmm,        xmmmem128,  imm8,       none,       0,       0x0f3adf,  r,   ib,  0,     P66)

INSTR (VAESKEYGENASSIST,  xmm,        xmmmem128,  imm8,       none,       0x0301,  0xdf,      r,   ib,  0,     PVEX)

INSTR (ANDN,              reg32,      reg32vvvv,  regmem32,   none,       0x0200,  0xf2,      r,   no,  0,     PVEX)
INSTR (ANDN,              reg64,      reg64vvvv,  regmem64,   none,       0x0280,  0xf2,      r,   no,  0,     PVEX)

INSTR (ANDNPD,            xmm,        xmmmem128,  none,       none,       0,       0x0f55,    r,   no,  0,     P66)

INSTR (VANDNPD,           xmm,        xvvvv,      xmmmem128,  none,       0x0101,  0x55,      r,   no,  0,     PVEX)
INSTR (VANDNPD,           ymm,        yvvvv,      ymmmem256,  none,       0x0105,  0x55,      r,   no,  0,     PVEX)

INSTR (ANDNPS,            xmm,        xmmmem128,  none,       none,       0,       0x0f55,    r,   no,  0,     0)

INSTR (VANDNPS,           xmm,        xvvvv,      xmmmem128,  none,       0x0100,  0x55,      r,   no,  0,     PVEX)
INSTR (VANDNPS,           ymm,        yvvvv,      ymmmem256,  none,       0x0104,  0x55,      r,   no,  0,     PVEX)

INSTR (ANDPD,             xmm,        xmmmem128,  none,       none,       0,       0x0f54,    r,   no,  0,     P66)

INSTR (VANDPD,            xmm,        xvvvv,      xmmmem128,  none,       0x0101,  0x54,      r,   no,  0,     PVEX)
INSTR (VANDPD,            ymm,        yvvvv,      ymmmem256,  none,       0x0105,  0x54,      r,   no,  0,     PVEX)

INSTR (ANDPS,             xmm,        xmmmem128,  none,       none,       0,       0x0f54,    r,   no,  0,     0)

INSTR (VANDPS,            xmm,        xvvvv,      xmmmem128,  none,       0x0100,  0x54,      r,   no,  0,     PVEX)
INSTR (VANDPS,            ymm,        yvvvv,      ymmmem256,  none,       0x0104,  0x54,      r,   no,  0,     PVEX)

INSTR (BEXTR,             reg32,      regmem32,   reg32vvvv,  none,       0x0200,  0xf7,      r,   no,  0,     PVEX)
INSTR (BEXTR,             reg64,      regmem64,   reg64vvvv,  none,       0x0280,  0xf7,      r,   no,  0,     PVEX)
INSTR (BEXTR,             reg32,      regmem32,   imm32,      none,       0x0a78,  0x10,      r,   id,  0,     PXOP)
INSTR (BEXTR,             reg64,      regmem64,   imm32,      none,       0x0af8,  0x10,      r,   id,  0,     PXOP)

INSTR (BLCFILL,           reg32vvvv,  regmem32,   none,       none,       0x0900,  0x01,      r1,  no,  0,     PXOP)
INSTR (BLCFILL,           reg64vvvv,  regmem64,   none,       none,       0x0980,  0x01,      r1,  no,  0,     PXOP)

INSTR (BLCI,              reg32vvvv,  regmem32,   none,       none,       0x0900,  0x02,      r6,  no,  0,     PXOP)
INSTR (BLCI,              reg64vvvv,  regmem64,   none,       none,       0x0980,  0x02,      r6,  no,  0,     PXOP)

INSTR (BLCIC,             reg32vvvv,  regmem32,   none,       none,       0x0900,  0x01,      r5,  no,  0,     PXOP)
INSTR (BLCIC,             reg64vvvv,  regmem64,   none,       none,       0x0980,  0x01,      r5,  no,  0,     PXOP)

INSTR (BLCMSK,            reg32vvvv,  regmem32,   none,       none,       0x0900,  0x02,      r1,  no,  0,     PXOP)
INSTR (BLCMSK,            reg64vvvv,  regmem64,   none,       none,       0x0980,  0x02,      r1,  no,  0,     PXOP)

INSTR (BLCS,              reg32vvvv,  regmem32,   none,       none,       0x0900,  0x01,      r3,  no,  0,     PXOP)
INSTR (BLCS,              reg64vvvv,  regmem64,   none,       none,       0x0980,  0x01,      r3,  no,  0,     PXOP)

INSTR (BLENDPD,           xmm,        xmmmem128,  imm8,       none,       0,       0x0f3a0d,  r,   ib,  0,     P66)

INSTR (VBLENDPD,          xmm,        xvvvv,      xmmmem128,  imm8,       0x0301,  0x0d,      r,   ib,  0,     PVEX)
INSTR (VBLENDPD,          ymm,        yvvvv,      ymmmem256,  imm8,       0x0305,  0x0d,      r,   ib,  0,     PVEX)

INSTR (BLENDPS,           xmm,        xmmmem128,  imm8,       none,       0,       0x0f3a0c,  r,   ib,  0,     P66)

INSTR (VBLENDPS,          xmm,        xvvvv,      xmmmem128,  imm8,       0x0301,  0x0c,      r,   ib,  0,     PVEX)
INSTR (VBLENDPS,          ymm,        yvvvv,      ymmmem256,  imm8,       0x0305,  0x0c,      r,   ib,  0,     PVEX)

INSTR (BLENDVPD,          xmm,        xmmmem128,  none,       none,       0,       0x0f3815,  r,   no,  0,     P66)

INSTR (VBLENDVPD,         xmm,        xvvvv,      xmmmem128,  ximm,       0x0301,  0x4b,      r,   is,  0,     PVEX)
INSTR (VBLENDVPD,         ymm,        yvvvv,      ymmmem256,  yimm,       0x0305,  0x4b,      r,   is,  0,     PVEX)

INSTR (BLENDVPS,          xmm,        xmmmem128,  none,       none,       0,       0x0f3814,  r,   no,  0,     P66)

INSTR (VBLENDVPS,         xmm,        xvvvv,      xmmmem128,  ximm,       0x0301,  0x4a,      r,   is,  0,     PVEX)
INSTR (VBLENDVPS,         ymm,        yvvvv,      ymmmem256,  yimm,       0x0305,  0x4a,      r,   is,  0,     PVEX)

INSTR (BLSFILL,           reg32vvvv,  regmem32,   none,       none,       0x0900,  0x01,      r2,  no,  0,     PXOP)
INSTR (BLSFILL,           reg64vvvv,  regmem64,   none,       none,       0x0980,  0x01,      r2,  no,  0,     PXOP)

INSTR (BLSI,              reg32vvvv,  regmem32,   none,       none,       0x0200,  0xf3,      r3,  no,  0,     PXOP)
INSTR (BLSI,              reg64vvvv,  regmem64,   none,       none,       0x0280,  0xf3,      r3,  no,  0,     PXOP)

INSTR (BLSIC,             reg32vvvv,  regmem32,   none,       none,       0x0900,  0x01,      r6,  no,  0,     PXOP)
INSTR (BLSIC,             reg64vvvv,  regmem64,   none,       none,       0x0980,  0x01,      r6,  no,  0,     PXOP)

INSTR (BLSMSK,            reg32vvvv,  regmem32,   none,       none,       0x0200,  0xf3,      r2,  no,  0,     PXOP)
INSTR (BLSMSK,            reg64vvvv,  regmem64,   none,       none,       0x0280,  0xf3,      r2,  no,  0,     PXOP)

INSTR (BLSR,              reg32vvvv,  regmem32,   none,       none,       0x0200,  0xf3,      r1,  no,  0,     PXOP)
INSTR (BLSR,              reg64vvvv,  regmem64,   none,       none,       0x0280,  0xf3,      r1,  no,  0,     PXOP)

INSTR (CMPPD,             xmm,        xmmmem128,  imm8,       none,       0,       0x0fc2,    r,   ib,  0,     P66)

INSTR (VCMPPD,            xmm,        xvvvv,      xmmmem128,  imm8,       0x0101,  0xc2,      r,   ib,  0,     PVEX)
INSTR (VCMPPD,            ymm,        yvvvv,      ymmmem256,  imm8,       0x0105,  0xc2,      r,   ib,  0,     PVEX)

INSTR (CMPEQPD,           xmm,        xmmmem128,  none,       none,       0,       0x0fc2,    r,   no,  0x00,  P66 | SFX)

INSTR (VCMPEQPD,          xmm,        xvvvv,      xmmmem128,  none,       0x0101,  0xc2,      r,   no,  0x00,  PVEX | SFX)
INSTR (VCMPEQPD,          ymm,        yvvvv,      ymmmem256,  none,       0x0105,  0xc2,      r,   no,  0x00,  PVEX | SFX)

INSTR (CMPLTPD,           xmm,        xmmmem128,  none,       none,       0,       0x0fc2,    r,   no,  0x01,  P66 | SFX)

INSTR (VCMPLTPD,          xmm,        xvvvv,      xmmmem128,  none,       0x0101,  0xc2,      r,   no,  0x01,  PVEX | SFX)
INSTR (VCMPLTPD,          ymm,        yvvvv,      ymmmem256,  none,       0x0105,  0xc2,      r,   no,  0x01,  PVEX | SFX)

INSTR (CMPLEPD,           xmm,        xmmmem128,  none,       none,       0,       0x0fc2,    r,   no,  0x02,  P66 | SFX)

INSTR (VCMPLEPD,          xmm,        xvvvv,      xmmmem128,  none,       0x0101,  0xc2,      r,   no,  0x02,  PVEX | SFX)
INSTR (VCMPLEPD,          ymm,        yvvvv,      ymmmem256,  none,       0x0105,  0xc2,      r,   no,  0x02,  PVEX | SFX)

INSTR (CMPUNORDPD,        xmm,        xmmmem128,  none,       none,       0,       0x0fc2,    r,   no,  0x03,  P66 | SFX)

INSTR (VCMPUNORDPD,       xmm,        xvvvv,      xmmmem128,  none,       0x0101,  0xc2,      r,   no,  0x03,  PVEX | SFX)
INSTR (VCMPUNORDPD,       ymm,        yvvvv,      ymmmem256,  none,       0x0105,  0xc2,      r,   no,  0x03,  PVEX | SFX)

INSTR (CMPNEQPD,          xmm,        xmmmem128,  none,       none,       0,       0x0fc2,    r,   no,  0x04,  P66 | SFX)

INSTR (VCMPNEQPD,         xmm,        xvvvv,      xmmmem128,  none,       0x0101,  0xc2,      r,   no,  0x04,  PVEX | SFX)
INSTR (VCMPNEQPD,         ymm,        yvvvv,      ymmmem256,  none,       0x0105,  0xc2,      r,   no,  0x04,  PVEX | SFX)

INSTR (CMPNLTPD,          xmm,        xmmmem128,  none,       none,       0,       0x0fc2,    r,   no,  0x05,  P66 | SFX)

INSTR (VCMPNLTPD,         xmm,        xvvvv,      xmmmem128,  none,       0x0101,  0xc2,      r,   no,  0x05,  PVEX | SFX)
INSTR (VCMPNLTPD,         ymm,        yvvvv,      ymmmem256,  none,       0x0105,  0xc2,      r,   no,  0x05,  PVEX | SFX)

INSTR (CMPNLEPD,          xmm,        xmmmem128,  none,       none,       0,       0x0fc2,    r,   no,  0x06,  P66 | SFX)

INSTR (VCMPNLEPD,         xmm,        xvvvv,      xmmmem128,  none,       0x0101,  0xc2,      r,   no,  0x06,  PVEX | SFX)
INSTR (VCMPNLEPD,         ymm,        yvvvv,      ymmmem256,  none,       0x0105,  0xc2,      r,   no,  0x06,  PVEX | SFX)

INSTR (CMPORDPD,          xmm,        xmmmem128,  none,       none,       0,       0x0fc2,    r,   no,  0x07,  P66 | SFX)

INSTR (VCMPORDPD,         xmm,        xvvvv,      xmmmem128,  none,       0x0101,  0xc2,      r,   no,  0x07,  PVEX | SFX)
INSTR (VCMPORDPD,         ymm,        yvvvv,      ymmmem256,  none,       0x0105,  0xc2,      r,   no,  0x07,  PVEX | SFX)

INSTR (VCMPSD,            xmm,        xvvvv,      xmmmem64,   imm8,       0x0103,  0xc2,      r,   ib,  0,     PVEX)

INSTR (CMPEQSD,           xmm,        xmmmem64,   none,       none,       0,       0x0fc2,    r,   no,  0x00,  PF2 | SFX)

INSTR (VCMPEQSD,          xmm,        xvvvv,      xmmmem64,   none,       0x0103,  0xc2,      r,   no,  0x00,  PVEX | SFX)

INSTR (CMPLTSD,           xmm,        xmmmem64,   none,       none,       0,       0x0fc2,    r,   no,  0x01,  PF2 | SFX)

INSTR (VCMPLTSD,          xmm,        xvvvv,      xmmmem64,   none,       0x0103,  0xc2,      r,   no,  0x01,  PVEX | SFX)

INSTR (CMPLESD,           xmm,        xmmmem64,   none,       none,       0,       0x0fc2,    r,   no,  0x02,  PF2 | SFX)

INSTR (VCMPLESD,          xmm,        xvvvv,      xmmmem64,   none,       0x0103,  0xc2,      r,   no,  0x02,  PVEX | SFX)

INSTR (CMPUNORDSD,        xmm,        xmmmem64,   none,       none,       0,       0x0fc2,    r,   no,  0x03,  PF2 | SFX)

INSTR (VCMPUNORDSD,       xmm,        xvvvv,      xmmmem64,   none,       0x0103,  0xc2,      r,   no,  0x03,  PVEX | SFX)

INSTR (CMPNEQSD,          xmm,        xmmmem64,   none,       none,       0,       0x0fc2,    r,   no,  0x04,  PF2 | SFX)

INSTR (VCMPNEQSD,         xmm,        xvvvv,      xmmmem64,   none,       0x0103,  0xc2,      r,   no,  0x04,  PVEX | SFX)

INSTR (CMPNLTSD,          xmm,        xmmmem64,   none,       none,       0,       0x0fc2,    r,   no,  0x05,  PF2 | SFX)

INSTR (VCMPNLTSD,         xmm,        xvvvv,      xmmmem64,   none,       0x0103,  0xc2,      r,   no,  0x05,  PVEX | SFX)

INSTR (CMPNLESD,          xmm,        xmmmem64,   none,       none,       0,       0x0fc2,    r,   no,  0x06,  PF2 | SFX)

INSTR (VCMPNLESD,         xmm,        xvvvv,      xmmmem64,   none,       0x0103,  0xc2,      r,   no,  0x06,  PVEX | SFX)

INSTR (CMPORDSD,          xmm,        xmmmem64,   none,       none,       0,       0x0fc2,    r,   no,  0x07,  PF2 | SFX)

INSTR (VCMPORDSD,         xmm,        xvvvv,      xmmmem64,   none,       0x0103,  0xc2,      r,   no,  0x07,  PVEX | SFX)

INSTR (CMPSS,             xmm,        xmmmem32,   imm8,       none,       0,       0x0fc2,    r,   ib,  0,     PF3)

INSTR (VCMPSS,            xmm,        xvvvv,      xmmmem32,   imm8,       0x0102,  0xc2,      r,   ib,  0,     PVEX)

INSTR (CMPEQSS,           xmm,        xmmmem32,   none,       none,       0,       0x0fc2,    r,   no,  0x00,  PF3 | SFX)

INSTR (VCMPEQSS,          xmm,        xvvvv,      xmmmem32,   none,       0x0102,  0xc2,      r,   no,  0x00,  PVEX | SFX)

INSTR (CMPLTSS,           xmm,        xmmmem32,   none,       none,       0,       0x0fc2,    r,   no,  0x01,  PF3 | SFX)

INSTR (VCMPLTSS,          xmm,        xvvvv,      xmmmem32,   none,       0x0102,  0xc2,      r,   no,  0x01,  PVEX | SFX)

INSTR (CMPLESS,           xmm,        xmmmem32,   none,       none,       0,       0x0fc2,    r,   no,  0x02,  PF3 | SFX)

INSTR (VCMPLESS,          xmm,        xvvvv,      xmmmem32,   none,       0x0102,  0xc2,      r,   no,  0x02,  PVEX | SFX)

INSTR (CMPUNORDSS,        xmm,        xmmmem32,   none,       none,       0,       0x0fc2,    r,   no,  0x03,  PF3 | SFX)

INSTR (VCMPUNORDSS,       xmm,        xvvvv,      xmmmem32,   none,       0x0102,  0xc2,      r,   no,  0x03,  PVEX | SFX)

INSTR (CMPNEQSS,          xmm,        xmmmem32,   none,       none,       0,       0x0fc2,    r,   no,  0x04,  PF3 | SFX)

INSTR (VCMPNEQSS,         xmm,        xvvvv,      xmmmem32,   none,       0x0102,  0xc2,      r,   no,  0x04,  PVEX | SFX)

INSTR (CMPNLTSS,          xmm,        xmmmem32,   none,       none,       0,       0x0fc2,    r,   no,  0x05,  PF3 | SFX)

INSTR (VCMPNLTSS,         xmm,        xvvvv,      xmmmem32,   none,       0x0102,  0xc2,      r,   no,  0x05,  PVEX | SFX)

INSTR (CMPNLESS,          xmm,        xmmmem32,   none,       none,       0,       0x0fc2,    r,   no,  0x06,  PF3 | SFX)

INSTR (VCMPNLESS,         xmm,        xvvvv,      xmmmem32,   none,       0x0102,  0xc2,      r,   no,  0x06,  PVEX | SFX)

INSTR (CMPORDSS,          xmm,        xmmmem32,   none,       none,       0,       0x0fc2,    r,   no,  0x07,  PF3 | SFX)

INSTR (VCMPORDSS,         xmm,        xvvvv,      xmmmem32,   none,       0x0102,  0xc2,      r,   no,  0x07,  PVEX | SFX)

INSTR (CMPPS,             xmm,        xmmmem128,  imm8,       none,       0,       0x0fc2,    r,   ib,  0,     0)

INSTR (VCMPPS,            xmm,        xvvvv,      xmmmem128,  imm8,       0x0100,  0xc2,      r,   ib,  0,     PVEX)

INSTR (CMPEQPS,           xmm,        xmmmem128,  none,       none,       0,       0x0fc2,    r,   no,  0x00,  SFX)

INSTR (VCMPEQPS,          xmm,        xvvvv,      xmmmem128,  none,       0x0100,  0xc2,      r,   no,  0x00,  PVEX | SFX)

INSTR (CMPLTPS,           xmm,        xmmmem128,  none,       none,       0,       0x0fc2,    r,   no,  0x01,  SFX)

INSTR (VCMPLTPS,          xmm,        xvvvv,      xmmmem128,  none,       0x0100,  0xc2,      r,   no,  0x01,  PVEX | SFX)

INSTR (CMPLEPS,           xmm,        xmmmem128,  none,       none,       0,       0x0fc2,    r,   no,  0x02,  SFX)

INSTR (VCMPLEPS,          xmm,        xvvvv,      xmmmem128,  none,       0x0100,  0xc2,      r,   no,  0x02,  PVEX | SFX)

INSTR (CMPUNORDPS,        xmm,        xmmmem128,  none,       none,       0,       0x0fc2,    r,   no,  0x03,  SFX)

INSTR (VCMPUNORDPS,       xmm,        xvvvv,      xmmmem128,  none,       0x0100,  0xc2,      r,   no,  0x03,  PVEX | SFX)

INSTR (CMPNEQPS,          xmm,        xmmmem128,  none,       none,       0,       0x0fc2,    r,   no,  0x04,  SFX)

INSTR (VCMPNEQPS,         xmm,        xvvvv,      xmmmem128,  none,       0x0100,  0xc2,      r,   no,  0x04,  PVEX | SFX)

INSTR (CMPNLTPS,          xmm,        xmmmem128,  none,       none,       0,       0x0fc2,    r,   no,  0x05,  SFX)

INSTR (VCMPNLTPS,         xmm,        xvvvv,      xmmmem128,  none,       0x0100,  0xc2,      r,   no,  0x05,  PVEX | SFX)

INSTR (CMPNLEPS,          xmm,        xmmmem128,  none,       none,       0,       0x0fc2,    r,   no,  0x06,  SFX)

INSTR (VCMPNLEPS,         xmm,        xvvvv,      xmmmem128,  none,       0x0100,  0xc2,      r,   no,  0x06,  PVEX | SFX)

INSTR (CMPORDPS,          xmm,        xmmmem128,  none,       none,       0,       0x0fc2,    r,   no,  0x07,  SFX)

INSTR (VCMPORDPS,         xmm,        xvvvv,      xmmmem128,  none,       0x0100,  0xc2,      r,   no,  0x07,  PVEX | SFX)

INSTR (COMISD,            xmm,        xmmmem64,   none,       none,       0,       0x0f2f,    r,   no,  0,     P66)

INSTR (VCOMISD,           xmm,        xmmmem64,   none,       none,       0x0101,  0x2f,      r,   no,  0,     PVEX)

INSTR (COMISS,            xmm,        xmmmem32,   none,       none,       0,       0x0f2f,    r,   no,  0,     0)

INSTR (VCOMISS,           xmm,        xmmmem32,   none,       none,       0x0100,  0x2f,      r,   no,  0,     PVEX)

INSTR (CVTTPD2DQ,         xmm,        xmmmem128,  none,       none,       0,       0x0fe6,    r,   no,  0,     P66)

INSTR (VCVTTPD2DQ,        xmm,        xmmmem128,  none,       none,       0x0179,  0xe6,      r,   no,  0,     PVEX)
INSTR (VCVTTPD2DQ,        xmm,        ymmmem256,  none,       none,       0x017d,  0xe6,      r,   no,  0,     PVEX)

INSTR (CVTPD2DQ,          xmm,        xmmmem128,  none,       none,       0,       0x0fe6,    r,   no,  0,     PF2)

INSTR (VCVTPD2DQ,         xmm,        xmmmem128,  none,       none,       0x017b,  0xe6,      r,   no,  0,     PVEX)
INSTR (VCVTPD2DQ,         xmm,        ymmmem256,  none,       none,       0x017f,  0xe6,      r,   no,  0,     PVEX)

INSTR (CVTDQ2PD,          xmm,        xmmmem64,   none,       none,       0,       0x0fe6,    r,   no,  0,     PF3)

INSTR (VCVTDQ2PD,         xmm,        xmmmem64,   none,       none,       0x017a,  0xe6,      r,   no,  0,     PVEX)
INSTR (VCVTDQ2PD,         ymm,        ymmmem128,  none,       none,       0x017e,  0xe6,      r,   no,  0,     PVEX)

INSTR (CVTPS2DQ,          xmm,        xmmmem128,  none,       none,       0,       0x0f5b,    r,   no,  0,     P66)

INSTR (VCVTPS2DQ,         xmm,        xmmmem128,  none,       none,       0x0179,  0x5b,      r,   no,  0,     PVEX)
INSTR (VCVTPS2DQ,         ymm,        ymmmem256,  none,       none,       0x017d,  0x5b,      r,   no,  0,     PVEX)

INSTR (CVTTPS2DQ,         xmm,        xmmmem128,  none,       none,       0,       0x0f5b,    r,   no,  0,     PF3)

INSTR (VCVTTPS2DQ,        xmm,        xmmmem128,  none,       none,       0x017a,  0x5b,      r,   no,  0,     PVEX)
INSTR (VCVTTPS2DQ,        ymm,        ymmmem256,  none,       none,       0x017e,  0x5b,      r,   no,  0,     PVEX)

INSTR (CVTDQ2PS,          xmm,        xmmmem128,  none,       none,       0,       0x0f5b,    r,   no,  0,     0)

INSTR (VCVTDQ2PS,         xmm,        xmmmem128,  none,       none,       0x0178,  0x5b,      r,   no,  0,     PVEX)
INSTR (VCVTDQ2PS,         ymm,        ymmmem256,  none,       none,       0x017c,  0x5b,      r,   no,  0,     PVEX)

INSTR (CVTPD2PI,          mmx,        xmmmem128,  none,       none,       0,       0x0f2d,    r,   no,  0,     P66)

INSTR (CVTSD2SI,          reg32,      xmmmem64,   none,       none,       0,       0x0f2d,    r,   no,  0,     O16 | O32 | PF2)
INSTR (CVTSD2SI,          reg64,      xmmmem64,   none,       none,       0,       0x0f2d,    r,   no,  0,     O64 | PF2)

INSTR (VCVTSD2SI,         reg32,      xmmmem64,   none,       none,       0x017b,  0x2d,      r,   no,  0,     PVEX)
INSTR (VCVTSD2SI,         reg64,      xmmmem64,   none,       none,       0x01fb,  0x2d,      r,   no,  0,     PVEX)

INSTR (CVTSS2SI,          reg32,      xmmmem32,   none,       none,       0,       0x0f2d,    r,   no,  0,     O16 | O32 | PF3)
INSTR (CVTSS2SI,          reg64,      xmmmem32,   none,       none,       0,       0x0f2d,    r,   no,  0,     O64 | PF3)

INSTR (VCVTSS2SI,         reg32,      xmmmem32,   none,       none,       0x017a,  0x2d,      r,   no,  0,     PVEX)
INSTR (VCVTSS2SI,         reg64,      xmmmem32,   none,       none,       0x01fa,  0x2d,      r,   no,  0,     PVEX)

INSTR (CVTPS2PI,          mmx,        xmmmem64,   none,       none,       0,       0x0f2d,    r,   no,  0,     0)

INSTR (CVTPI2PD,          xmm,        mmxmem64,   none,       none,       0,       0x0f2a,    r,   no,  0,     P66)

INSTR (CVTSI2SD,          xmm,        regmem32,   none,       none,       0,       0x0f2a,    r,   no,  0,     O16 | O32 | PF2)
INSTR (CVTSI2SD,          xmm,        regmem64,   none,       none,       0,       0x0f2a,    r,   no,  0,     O64 | PF2)

INSTR (VCVTSI2SD,         xmm,        xvvvv,      regmem32,   none,       0x0103,  0x2a,      r,   no,  0,     PVEX)
INSTR (VCVTSI2SD,         xmm,        xvvvv,      regmem64,   none,       0x0183,  0x2a,      r,   no,  0,     PVEX)

INSTR (CVTSI2SS,          xmm,        regmem32,   none,       none,       0,       0x0f2a,    r,   no,  0,     O16 | O32 | PF3)
INSTR (CVTSI2SS,          xmm,        regmem64,   none,       none,       0,       0x0f2a,    r,   no,  0,     O64 | PF3)

INSTR (VCVTSI2SS,         xmm,        xvvvv,      regmem32,   none,       0x0102,  0x2a,      r,   no,  0,     PVEX)
INSTR (VCVTSI2SS,         xmm,        xvvvv,      regmem64,   none,       0x0182,  0x2a,      r,   no,  0,     PVEX)

INSTR (CVTPI2PS,          xmm,        mmxmem64,   none,       none,       0,       0x0f2a,    r,   no,  0,     0)

INSTR (CVTPD2PS,          xmm,        xmmmem128,  none,       none,       0,       0x0f5a,    r,   no,  0,     P66)

INSTR (VCVTPD2PS,         xmm,        xmmmem128,  none,       none,       0x0179,  0x5a,      r,   no,  0,     PVEX)
INSTR (VCVTPD2PS,         xmm,        ymmmem256,  none,       none,       0x017d,  0x5a,      r,   no,  0,     PVEX)

INSTR (CVTSD2SS,          xmm,        xmmmem64,   none,       none,       0,       0x0f5a,    r,   no,  0,     PF2)

INSTR (VCVTSD2SS,         xmm,        xvvvv,      xmmmem64,   none,       0x0103,  0x5a,      r,   no,  0,     PVEX)

INSTR (CVTSS2SD,          xmm,        xmmmem32,   none,       none,       0,       0x0f5a,    r,   no,  0,     PF3)

INSTR (VCVTSS2SD,         xmm,        xvvvv,      xmmmem32,   none,       0x0102,  0x5a,      r,   no,  0,     PVEX)

INSTR (CVTPS2PD,          xmm,        xmmmem64,   none,       none,       0,       0x0f5a,    r,   no,  0,     0)

INSTR (VCVTPS2PD,         xmm,        xmmmem64,   none,       none,       0x0178,  0x5a,      r,   no,  0,     PVEX)
INSTR (VCVTPS2PD,         ymm,        ymmmem128,  none,       none,       0x017c,  0x5a,      r,   no,  0,     PVEX)

INSTR (CVTTPD2PI,         mmx,        xmmmem128,  none,       none,       0,       0x0f2c,    r,   no,  0,     P66)

INSTR (CVTTSD2SI,         reg32,      xmmmem64,   none,       none,       0,       0x0f2c,    r,   no,  0,     O16 | O32 | PF2)
INSTR (CVTTSD2SI,         reg64,      xmmmem64,   none,       none,       0,       0x0f2c,    r,   no,  0,     O64 | PF2)

INSTR (VCVTTSD2SI,        reg32,      xmmmem64,   none,       none,       0x017b,  0x2c,      r,   no,  0,     PVEX)
INSTR (VCVTTSD2SI,        reg64,      xmmmem64,   none,       none,       0x01fb,  0x2c,      r,   no,  0,     PVEX)

INSTR (CVTTSS2SI,         reg32,      xmmmem32,   none,       none,       0,       0x0f2c,    r,   no,  0,     O16 | O32 | PF3)
INSTR (CVTTSS2SI,         reg64,      xmmmem32,   none,       none,       0,       0x0f2c,    r,   no,  0,     O64 | PF3)

INSTR (VCVTTSS2SI,        reg32,      xmmmem32,   none,       none,       0x017a,  0x2c,      r,   no,  0,     PVEX)
INSTR (VCVTTSS2SI,        reg64,      xmmmem32,   none,       none,       0x01fa,  0x2c,      r,   no,  0,     PVEX)

INSTR (CVTTPS2PI,         mmx,        xmmmem64,   none,       none,       0,       0x0f2c,    r,   no,  0,     0)

INSTR (DIVPD,             xmm,        xmmmem128,  none,       none,       0,       0x0f5e,    r,   no,  0,     P66)

INSTR (VDIVPD,            xmm,        xvvvv,      xmmmem128,  none,       0x0101,  0x5e,      r,   no,  0,     PVEX)
INSTR (VDIVPD,            ymm,        yvvvv,      ymmmem256,  none,       0x0105,  0x5e,      r,   no,  0,     PVEX)

INSTR (DIVSD,             xmm,        xmmmem64,   none,       none,       0,       0x0f5e,    r,   no,  0,     PF2)

INSTR (VDIVSD,            xmm,        xvvvv,      xmmmem64,   none,       0x0103,  0x5e,      r,   no,  0,     PVEX)

INSTR (DIVSS,             xmm,        xmmmem32,   none,       none,       0,       0x0f5e,    r,   no,  0,     PF3)

INSTR (VDIVSS,            xmm,        xvvvv,      xmmmem32,   none,       0x0102,  0x5e,      r,   no,  0,     PVEX)

INSTR (DIVPS,             xmm,        xmmmem128,  none,       none,       0,       0x0f5e,    r,   no,  0,     0)

INSTR (VDIVPS,            xmm,        xvvvv,      xmmmem128,  none,       0x0100,  0x5e,      r,   no,  0,     PVEX)
INSTR (VDIVPS,            ymm,        yvvvv,      ymmmem256,  none,       0x0104,  0x5e,      r,   no,  0,     PVEX)

INSTR (DPPD,              xmm,        xmmmem128,  imm8,       none,       0,       0x0f3a41,  r,   ib,  0,     P66)

INSTR (VDPPD,             xmm,        xvvvv,      xmmmem128,  imm8,       0x0301,  0x41,      r,   ib,  0,     PVEX)

INSTR (DPPS,              xmm,        xmmmem128,  imm8,       none,       0,       0x0f3a40,  r,   ib,  0,     P66)

INSTR (VDPPS,             xmm,        xvvvv,      xmmmem128,  imm8,       0x0301,  0x40,      r,   ib,  0,     PVEX)
INSTR (VDPPS,             ymm,        yvvvv,      ymmmem256,  imm8,       0x0305,  0x40,      r,   ib,  0,     PVEX)

INSTR (EXTRACTPS,         regmem32,   xmm,        imm8,       none,       0,       0x0f3a17,  r,   ib,  0,     P66)

INSTR (VEXTRACTPS,        regmem32,   xmm,        imm8,       none,       0x0379,  0x17,      r,   ib,  0,     PVEX)

INSTR (EXTRQ,             xmmmem64,   imm16,      none,       none,       0,       0x0f78,    r0,  iw,  0,     P66)
INSTR (EXTRQ,             xmm,        xmmmem64,   none,       none,       0,       0x0f79,    r,   no,  0,     P66)

INSTR (HADDPD,            xmm,        xmmmem128,  none,       none,       0,       0x0f7c,    r,   no,  0,     P66)

INSTR (VHADDPD,           xmm,        xvvvv,      xmmmem128,  none,       0x0101,  0x7c,      r,   no,  0,     PVEX)
INSTR (VHADDPD,           ymm,        yvvvv,      ymmmem256,  none,       0x0105,  0x7c,      r,   no,  0,     PVEX)

INSTR (HADDPS,            xmm,        xmmmem128,  none,       none,       0,       0x0f7c,    r,   no,  0,     PF2)

INSTR (VHADDPS,           xmm,        xvvvv,      xmmmem128,  none,       0x0103,  0x7c,      r,   no,  0,     PVEX)
INSTR (VHADDPS,           ymm,        yvvvv,      ymmmem256,  none,       0x0107,  0x7c,      r,   no,  0,     PVEX)

INSTR (HSUBPD,            xmm,        xmmmem128,  none,       none,       0,       0x0f7d,    r,   no,  0,     P66)

INSTR (VHSUBPD,           xmm,        xvvvv,      xmmmem128,  none,       0x0101,  0x7d,      r,   no,  0,     PVEX)
INSTR (VHSUBPD,           ymm,        yvvvv,      ymmmem256,  none,       0x0105,  0x7d,      r,   no,  0,     PVEX)

INSTR (HSUBPS,            xmm,        xmmmem128,  none,       none,       0,       0x0f7d,    r,   no,  0,     PF2)

INSTR (VHSUBPS,           xmm,        xvvvv,      xmmmem128,  none,       0x0103,  0x7d,      r,   no,  0,     PVEX)
INSTR (VHSUBPS,           ymm,        yvvvv,      ymmmem256,  none,       0x0107,  0x7d,      r,   no,  0,     PVEX)

INSTR (INSERTPS,          xmm,        xmmmem128,  imm8,       none,       0,       0x0f3a21,  r,   ib,  0,     P66)

INSTR (VINSERTPS,         xmm,        xvvvv,      xmmmem128,  imm8,       0x0301,  0x21,      r,   ib,  0,     PVEX)

INSTR (INSERTQ,           xmm,        xmmmem64,   imm16,      none,       0,       0x0f78,    r,   iw,  0,     PF2)

INSTR (LDDQU,             xmm,        mem128,     none,       none,       0,       0x0ff0,    r,   no,  0,     PF2)

INSTR (VLDDQU,            xmm,        xmmmem128,  none,       none,       0x017b,  0xf0,      r,   no,  0,     PVEX)
INSTR (VLDDQU,            ymm,        ymmmem256,  none,       none,       0x017f,  0xf0,      r,   no,  0,     PVEX)

INSTR (MASKMOVDQU,        xmm,        xmmmem128,  none,       none,       0,       0x0ff7,    r,   no,  0,     P66)

INSTR (VMASKMOVDQU,       xmm,        xmmmem128,  none,       none,       0x0179,  0xf7,      r,   no,  0,     PVEX)

INSTR (MAXPD,             xmm,        xmmmem128,  none,       none,       0,       0x0f5f,    r,   no,  0,     P66)

INSTR (VMAXPD,            xmm,        xvvvv,      xmmmem128,  none,       0x0101,  0x5f,      r,   no,  0,     PVEX)
INSTR (VMAXPD,            ymm,        yvvvv,      ymmmem256,  none,       0x0105,  0x5f,      r,   no,  0,     PVEX)

INSTR (MAXSD,             xmm,        xmmmem64,   none,       none,       0,       0x0f5f,    r,   no,  0,     PF2)

INSTR (VMAXSD,            xmm,        xvvvv,      xmmmem64,   none,       0x0103,  0x5f,      r,   no,  0,     PVEX)

INSTR (MAXSS,             xmm,        xmmmem32,   none,       none,       0,       0x0f5f,    r,   no,  0,     PF3)

INSTR (VMAXSS,            xmm,        xvvvv,      xmmmem32,   none,       0x0102,  0x5f,      r,   no,  0,     PVEX)

INSTR (MAXPS,             xmm,        xmmmem128,  none,       none,       0,       0x0f5f,    r,   no,  0,     0)

INSTR (VMAXPS,            xmm,        xvvvv,      xmmmem128,  none,       0x0100,  0x5f,      r,   no,  0,     PVEX)
INSTR (VMAXPS,            ymm,        yvvvv,      ymmmem256,  none,       0x0104,  0x5f,      r,   no,  0,     PVEX)

INSTR (MINPD,             xmm,        xmmmem128,  none,       none,       0,       0x0f5d,    r,   no,  0,     P66)

INSTR (VMINPD,            xmm,        xvvvv,      xmmmem128,  none,       0x0101,  0x5d,      r,   no,  0,     PVEX)
INSTR (VMINPD,            ymm,        yvvvv,      ymmmem256,  none,       0x0105,  0x5d,      r,   no,  0,     PVEX)

INSTR (MINSD,             xmm,        xmmmem64,   none,       none,       0,       0x0f5d,    r,   no,  0,     PF2)

INSTR (VMINSD,            xmm,        xvvvv,      xmmmem64,   none,       0x0103,  0x5d,      r,   no,  0,     PVEX)

INSTR (MINSS,             xmm,        xmmmem32,   none,       none,       0,       0x0f5d,    r,   no,  0,     PF3)

INSTR (VMINSS,            xmm,        xvvvv,      xmmmem32,   none,       0x0102,  0x5d,      r,   no,  0,     PVEX)

INSTR (MINPS,             xmm,        xmmmem128,  none,       none,       0,       0x0f5d,    r,   no,  0,     0)

INSTR (VMINPS,            xmm,        xvvvv,      xmmmem128,  none,       0x0100,  0x5d,      r,   no,  0,     PVEX)
INSTR (VMINPS,            ymm,        yvvvv,      ymmmem256,  none,       0x0104,  0x5d,      r,   no,  0,     PVEX)

INSTR (MOVAPD,            xmm,        xmmmem128,  none,       none,       0,       0x0f28,    r,   no,  0,     P66)
INSTR (MOVAPD,            xmmmem128,  xmm,        none,       none,       0,       0x0f29,    r,   no,  0,     P66)

INSTR (VMOVAPD,           xmm,        xmmmem128,  none,       none,       0x0179,  0x28,      r,   no,  0,     PVEX)
INSTR (VMOVAPD,           xmmmem128,  xmm,        none,       none,       0x0179,  0x29,      r,   no,  0,     PVEX)
INSTR (VMOVAPD,           ymm,        ymmmem256,  none,       none,       0x017d,  0x28,      r,   no,  0,     PVEX)
INSTR (VMOVAPD,           ymmmem256,  ymm,        none,       none,       0x017d,  0x29,      r,   no,  0,     PVEX)

INSTR (MOVAPS,            xmm,        xmmmem128,  none,       none,       0,       0x0f28,    r,   no,  0,     0)
INSTR (MOVAPS,            xmmmem128,  xmm,        none,       none,       0,       0x0f29,    r,   no,  0,     0)

INSTR (VMOVAPS,           xmm,        xmmmem128,  none,       none,       0x0178,  0x28,      r,   no,  0,     PVEX)
INSTR (VMOVAPS,           xmmmem128,  xmm,        none,       none,       0x0178,  0x29,      r,   no,  0,     PVEX)
INSTR (VMOVAPS,           ymm,        ymmmem256,  none,       none,       0x017c,  0x28,      r,   no,  0,     PVEX)
INSTR (VMOVAPS,           ymmmem256,  ymm,        none,       none,       0x017c,  0x29,      r,   no,  0,     PVEX)

INSTR (MOVLPD,            xmm,        mem64,      none,       none,       0,       0x0f12,    r,   no,  0,     P66)
INSTR (MOVLPD,            mem64,      xmm,        none,       none,       0,       0x0f13,    r,   no,  0,     P66)

INSTR (VMOVLPD,           xmm,        xvvvv,      mem64,      none,       0x0101,  0x12,      r,   no,  0,     PVEX)
INSTR (VMOVLPD,           mem64,      xmm,        none,       none,       0x0179,  0x13,      r,   no,  0,     PVEX)

INSTR (MOVDDUP,           xmm,        xmmmem64,   none,       none,       0,       0x0f12,    r,   no,  0,     PF2)

INSTR (VMOVDDUP,          xmm,        xmmmem64,   none,       none,       0x017b,  0x12,      r,   no,  0,     PVEX)
INSTR (VMOVDDUP,          ymm,        ymmmem256,  none,       none,       0x017f,  0x12,      r,   no,  0,     PVEX)

INSTR (MOVSLDUP,          xmm,        xmmmem128,  none,       none,       0,       0x0f12,    r,   no,  0,     PF3)

INSTR (VMOVSLDUP,         xmm,        xmmmem128,  none,       none,       0x017a,  0x12,      r,   no,  0,     PVEX)
INSTR (VMOVSLDUP,         ymm,        ymmmem256,  none,       none,       0x017e,  0x12,      r,   no,  0,     PVEX)

INSTR (MOVLPS,            xmm,        mem64,      none,       none,       0,       0x0f12,    r,   no,  0,     0)
INSTR (MOVLPS,            mem64,      xmm,        none,       none,       0,       0x0f13,    r,   no,  0,     0)

INSTR (VMOVLPS,           xmm,        xvvvv,      mem64,      none,       0x0100,  0x12,      r,   no,  0,     PVEX)
INSTR (VMOVLPS,           mem64,      xmm,        none,       none,       0x0178,  0x13,      r,   no,  0,     PVEX)

INSTR (MOVHLPS,           xmm,        xmmmem128,  none,       none,       0,       0x0f12,    r,   no,  0,     0)

INSTR (VMOVHLPS,          xmm,        xvvvv,      xmmmem128,  none,       0x0100,  0x12,      r,   no,  0,     PVEX)

INSTR (MOVDQ2Q,           mmx,        xmmmem64,   none,       none,       0,       0x0fd6,    r,   no,  0,     PF2)

INSTR (MOVQ2DQ,           xmm,        mmxmem64,   none,       none,       0,       0x0fd6,    r,   no,  0,     PF3)

INSTR (MOVHPD,            xmm,        mem64,      none,       none,       0,       0x0f16,    r,   no,  0,     P66)
INSTR (MOVHPD,            mem64,      xmm,        none,       none,       0,       0x0f17,    r,   no,  0,     P66)

INSTR (VMOVHPD,           xmm,        xvvvv,      mem64,      none,       0x0101,  0x16,      r,   no,  0,     PVEX)
INSTR (VMOVHPD,           mem64,      xmm,        none,       none,       0x0179,  0x17,      r,   no,  0,     PVEX)

INSTR (MOVSHDUP,          xmm,        xmmmem128,  none,       none,       0,       0x0f16,    r,   no,  0,     PF3)

INSTR (VMOVSHDUP,         xmm,        xmmmem128,  none,       none,       0x017a,  0x16,      r,   no,  0,     PVEX)
INSTR (VMOVSHDUP,         ymm,        ymmmem256,  none,       none,       0x017e,  0x16,      r,   no,  0,     PVEX)

INSTR (MOVHPS,            xmm,        mem64,      none,       none,       0,       0x0f16,    r,   no,  0,     0)
INSTR (MOVHPS,            mem64,      xmm,        none,       none,       0,       0x0f17,    r,   no,  0,     0)

INSTR (VMOVHPS,           xmm,        xvvvv,      mem64,      none,       0x0100,  0x16,      r,   no,  0,     PVEX)
INSTR (VMOVHPS,           mem64,      xmm,        none,       none,       0x0178,  0x17,      r,   no,  0,     PVEX)

INSTR (MOVLHPS,           xmm,        xmmmem128,  none,       none,       0,       0x0f16,    r,   no,  0,     0)

INSTR (VMOVLHPS,          xmm,        xvvvv,      xmmmem128,  none,       0x0100,  0x16,      r,   no,  0,     PVEX)

INSTR (MOVNTDQ,           mem128,     xmm,        none,       none,       0,       0x0fe7,    r,   no,  0,     P66)

INSTR (VMOVNTDQ,          mem128,     xmm,        none,       none,       0x0179,  0xe7,      r,   no,  0,     PVEX)
INSTR (VMOVNTDQ,          mem256,     ymm,        none,       none,       0x017d,  0xe7,      r,   no,  0,     PVEX)

INSTR (MOVNTDQA,          xmm,        mem128,     none,       none,       0,       0x0f382a,  r,   no,  0,     P66)

INSTR (VMOVNTDQA,         xmm,        mem128,     none,       none,       0x0279,  0x2a,      r,   no,  0,     PVEX)
INSTR (VMOVNTDQA,         ymm,        mem256,     none,       none,       0x027d,  0x2a,      r,   no,  0,     PVEX)

INSTR (MOVNTPD,           mem128,     xmm,        none,       none,       0,       0x0f2b,    r,   no,  0,     P66)

INSTR (VMOVNTPD,          mem128,     xmm,        none,       none,       0x0179,  0x2b,      r,   no,  0,     PVEX)
INSTR (VMOVNTPD,          mem256,     ymm,        none,       none,       0x017d,  0x2b,      r,   no,  0,     PVEX)

INSTR (MOVNTPS,           mem128,     xmm,        none,       none,       0,       0x0f2b,    r,   no,  0,     0)

INSTR (VMOVNTPS,          mem128,     xmm,        none,       none,       0x0178,  0x2b,      r,   no,  0,     PVEX)
INSTR (VMOVNTPS,          mem256,     ymm,        none,       none,       0x017c,  0x2b,      r,   no,  0,     PVEX)

INSTR (MOVNTSD,           mem64,      xmm,        none,       none,       0,       0x0f2b,    r,   no,  0,     PF2)

INSTR (MOVNTSS,           mem32,      xmm,        none,       none,       0,       0x0f2b,    r,   no,  0,     PF3)

INSTR (MOVUPD,            xmm,        xmmmem128,  none,       none,       0,       0x0f10,    r,   no,  0,     P66)
INSTR (MOVUPD,            xmmmem128,  xmm,        none,       none,       0,       0x0f11,    r,   no,  0,     P66)

INSTR (VMOVUPD,           xmm,        xmmmem128,  none,       none,       0x0179,  0x10,      r,   no,  0,     PVEX)
INSTR (VMOVUPD,           xmmmem128,  xmm,        none,       none,       0x0179,  0x11,      r,   no,  0,     PVEX)
INSTR (VMOVUPD,           ymm,        ymmmem256,  none,       none,       0x017d,  0x10,      r,   no,  0,     PVEX)
INSTR (VMOVUPD,           ymmmem256,  ymm,        none,       none,       0x017d,  0x11,      r,   no,  0,     PVEX)

INSTR (MOVSS,             xmm,        xmmmem32,   none,       none,       0,       0x0f10,    r,   no,  0,     PF3)
INSTR (MOVSS,             xmmmem32,   xmm,        none,       none,       0,       0x0f11,    r,   no,  0,     PF3)

INSTR (VMOVSS,            xmm,        mem32,      none,       none,       0x017a,  0x10,      r,   no,  0,     PVEX)
INSTR (VMOVSS,            mem32,      xmm,        none,       none,       0x017a,  0x11,      r,   no,  0,     PVEX)
INSTR (VMOVSS,            xmm,        xvvvv,      xmmmem32,   none,       0x0102,  0x10,      r,   no,  0,     PVEX)
INSTR (VMOVSS,            xmm,        xvvvv,      xmmmem32,   none,       0x0102,  0x11,      r,   no,  0,     PVEX)

INSTR (MOVUPS,            xmm,        xmmmem128,  none,       none,       0,       0x0f10,    r,   no,  0,     0)
INSTR (MOVUPS,            xmmmem128,  xmm,        none,       none,       0,       0x0f11,    r,   no,  0,     0)

INSTR (VMOVUPS,           xmm,        xmmmem128,  none,       none,       0x0178,  0x10,      r,   no,  0,     PVEX)
INSTR (VMOVUPS,           xmmmem128,  xmm,        none,       none,       0x0178,  0x11,      r,   no,  0,     PVEX)
INSTR (VMOVUPS,           ymm,        ymmmem256,  none,       none,       0x017c,  0x10,      r,   no,  0,     PVEX)
INSTR (VMOVUPS,           ymmmem256,  ymm,        none,       none,       0x017c,  0x11,      r,   no,  0,     PVEX)

INSTR (MPSADBW,           xmm,        xmmmem128,  imm8,       none,       0,       0x0f3a42,  r,   ib,  0,     P66)

INSTR (VMPSADBW,          xmm,        xvvvv,      xmmmem128,  imm8,       0x0301,  0x42,      r,   ib,  0,     PVEX)
INSTR (VMPSADBW,          ymm,        yvvvv,      ymmmem256,  imm8,       0x0305,  0x42,      r,   ib,  0,     PVEX)

INSTR (MULPD,             xmm,        xmmmem128,  none,       none,       0,       0x0f59,    r,   no,  0,     P66)

INSTR (VMULPD,            xmm,        xvvvv,      xmmmem128,  none,       0x0101,  0x59,      r,   no,  0,     PVEX)
INSTR (VMULPD,            ymm,        yvvvv,      ymmmem256,  none,       0x0105,  0x59,      r,   no,  0,     PVEX)

INSTR (MULSD,             xmm,        xmmmem64,   none,       none,       0,       0x0f59,    r,   no,  0,     PF2)

INSTR (VMULSD,            xmm,        xvvvv,      xmmmem64,   none,       0x0103,  0x59,      r,   no,  0,     PVEX)

INSTR (MULSS,             xmm,        xmmmem32,   none,       none,       0,       0x0f59,    r,   no,  0,     PF3)

INSTR (VMULSS,            xmm,        xvvvv,      xmmmem32,   none,       0x0102,  0x59,      r,   no,  0,     PVEX)

INSTR (MULPS,             xmm,        xmmmem128,  none,       none,       0,       0x0f59,    r,   no,  0,     0)

INSTR (VMULPS,            xmm,        xvvvv,      xmmmem128,  none,       0x0100,  0x59,      r,   no,  0,     PVEX)
INSTR (VMULPS,            ymm,        yvvvv,      ymmmem256,  none,       0x0104,  0x59,      r,   no,  0,     PVEX)

INSTR (ORPD,              xmm,        xmmmem128,  none,       none,       0,       0x0f56,    r,   no,  0,     P66)

INSTR (VORPD,             xmm,        xvvvv,      xmmmem128,  none,       0x0101,  0x56,      r,   no,  0,     PVEX)
INSTR (VORPD,             ymm,        yvvvv,      ymmmem256,  none,       0x0105,  0x56,      r,   no,  0,     PVEX)

INSTR (ORPS,              xmm,        xmmmem128,  none,       none,       0,       0x0f56,    r,   no,  0,     0)

INSTR (VORPS,             xmm,        xvvvv,      xmmmem128,  none,       0x0100,  0x56,      r,   no,  0,     PVEX)
INSTR (VORPS,             ymm,        yvvvv,      ymmmem256,  none,       0x0104,  0x56,      r,   no,  0,     PVEX)

INSTR (PABSB,             xmm,        xmmmem128,  none,       none,       0,       0x0f381c,  r,   no,  0,     0)

INSTR (VPABSB,            xmm,        xmmmem128,  none,       none,       0x0279,  0x1c,      r,   no,  0,     PVEX)
INSTR (VPABSB,            ymm,        ymmmem256,  none,       none,       0x027d,  0x1c,      r,   no,  0,     PVEX)

INSTR (PABSD,             xmm,        xmmmem128,  none,       none,       0,       0x0f381e,  r,   no,  0,     0)

INSTR (VPABSD,            xmm,        xmmmem128,  none,       none,       0x0279,  0x1e,      r,   no,  0,     PVEX)
INSTR (VPABSD,            ymm,        ymmmem256,  none,       none,       0x027d,  0x1e,      r,   no,  0,     PVEX)

INSTR (PABSW,             xmm,        xmmmem128,  none,       none,       0,       0x0f381d,  r,   no,  0,     0)

INSTR (VPABSW,            xmm,        xmmmem128,  none,       none,       0x0279,  0x1d,      r,   no,  0,     PVEX)
INSTR (VPABSW,            ymm,        ymmmem256,  none,       none,       0x027d,  0x1d,      r,   no,  0,     PVEX)

INSTR (PACKSSDW,          xmm,        xmmmem128,  none,       none,       0,       0x0f6b,    r,   no,  0,     P66)
INSTR (PACKSSDW,          mmx,        mmxmem64,   none,       none,       0,       0x0f6b,    r,   no,  0,     0)

INSTR (VPACKSSDW,         xmm,        xvvvv,      xmmmem128,  none,       0x0101,  0x6b,      r,   no,  0,     PVEX)
INSTR (VPACKSSDW,         ymm,        yvvvv,      ymmmem256,  none,       0x0105,  0x6b,      r,   no,  0,     PVEX)

INSTR (PACKSSWB,          xmm,        xmmmem128,  none,       none,       0,       0x0f63,    r,   no,  0,     P66)
INSTR (PACKSSWB,          mmx,        mmxmem64,   none,       none,       0,       0x0f63,    r,   no,  0,     0)

INSTR (VPACKSSWB,         xmm,        xvvvv,      xmmmem128,  none,       0x0101,  0x63,      r,   no,  0,     PVEX)
INSTR (VPACKSSWB,         ymm,        yvvvv,      ymmmem256,  none,       0x0105,  0x63,      r,   no,  0,     PVEX)

INSTR (PACKUSDW,          xmm,        xmmmem128,  none,       none,       0,       0x0f382b,  r,   no,  0,     P66)

INSTR (VPACKUSDW,         xmm,        xvvvv,      xmmmem128,  none,       0x0201,  0x2b,      r,   no,  0,     PVEX)
INSTR (VPACKUSDW,         ymm,        yvvvv,      ymmmem256,  none,       0x0205,  0x2b,      r,   no,  0,     PVEX)

INSTR (PACKUSWB,          xmm,        xmmmem128,  none,       none,       0,       0x0f67,    r,   no,  0,     P66)
INSTR (PACKUSWB,          mmx,        mmxmem64,   none,       none,       0,       0x0f67,    r,   no,  0,     0)

INSTR (VPACKUSWB,         xmm,        xvvvv,      xmmmem128,  none,       0x0101,  0x67,      r,   no,  0,     PVEX)
INSTR (VPACKUSWB,         ymm,        yvvvv,      ymmmem256,  none,       0x0105,  0x67,      r,   no,  0,     PVEX)

INSTR (PADDB,             xmm,        xmmmem128,  none,       none,       0,       0x0ffc,    r,   no,  0,     P66)
INSTR (PADDB,             mmx,        mmxmem64,   none,       none,       0,       0x0ffc,    r,   no,  0,     0)

INSTR (VPADDB,            xmm,        xvvvv,      xmmmem128,  none,       0x0101,  0xfc,      r,   no,  0,     PVEX)
INSTR (VPADDB,            ymm,        yvvvv,      ymmmem256,  none,       0x0105,  0xfc,      r,   no,  0,     PVEX)

INSTR (PADDD,             xmm,        xmmmem128,  none,       none,       0,       0x0ffe,    r,   no,  0,     P66)
INSTR (PADDD,             mmx,        mmxmem64,   none,       none,       0,       0x0ffe,    r,   no,  0,     0)

INSTR (VPADDD,            xmm,        xvvvv,      xmmmem128,  none,       0x0101,  0xfe,      r,   no,  0,     PVEX)
INSTR (VPADDD,            ymm,        yvvvv,      ymmmem256,  none,       0x0105,  0xfe,      r,   no,  0,     PVEX)

INSTR (PADDQ,             xmm,        xmmmem128,  none,       none,       0,       0x0fd4,    r,   no,  0,     P66)
INSTR (PADDQ,             mmx,        mmxmem64,   none,       none,       0,       0x0fd4,    r,   no,  0,     0)

INSTR (VPADDQ,            xmm,        xvvvv,      xmmmem128,  none,       0x0101,  0xd4,      r,   no,  0,     PVEX)
INSTR (VPADDQ,            ymm,        yvvvv,      ymmmem256,  none,       0x0105,  0xd4,      r,   no,  0,     PVEX)

INSTR (PADDSB,            xmm,        xmmmem128,  none,       none,       0,       0x0fec,    r,   no,  0,     P66)
INSTR (PADDSB,            mmx,        mmxmem64,   none,       none,       0,       0x0fec,    r,   no,  0,     0)

INSTR (VPADDSB,           xmm,        xvvvv,      xmmmem128,  none,       0x0101,  0xec,      r,   no,  0,     PVEX)
INSTR (VPADDSB,           ymm,        yvvvv,      ymmmem256,  none,       0x0105,  0xec,      r,   no,  0,     PVEX)

INSTR (PADDSW,            xmm,        xmmmem128,  none,       none,       0,       0x0fed,    r,   no,  0,     P66)
INSTR (PADDSW,            mmx,        mmxmem64,   none,       none,       0,       0x0fed,    r,   no,  0,     0)

INSTR (VPADDSW,           xmm,        xvvvv,      xmmmem128,  none,       0x0101,  0xed,      r,   no,  0,     PVEX)
INSTR (VPADDSW,           ymm,        yvvvv,      ymmmem256,  none,       0x0105,  0xed,      r,   no,  0,     PVEX)

INSTR (PADDUSB,           xmm,        xmmmem128,  none,       none,       0,       0x0fdc,    r,   no,  0,     P66)
INSTR (PADDUSB,           mmx,        mmxmem64,   none,       none,       0,       0x0fdc,    r,   no,  0,     0)

INSTR (VPADDUSB,          xmm,        xvvvv,      xmmmem128,  none,       0x0101,  0xdc,      r,   no,  0,     PVEX)
INSTR (VPADDUSB,          ymm,        yvvvv,      ymmmem256,  none,       0x0105,  0xdc,      r,   no,  0,     PVEX)

INSTR (PADDUSW,           xmm,        xmmmem128,  none,       none,       0,       0x0fdd,    r,   no,  0,     P66)
INSTR (PADDUSW,           mmx,        mmxmem64,   none,       none,       0,       0x0fdd,    r,   no,  0,     0)

INSTR (VPADDUSW,          xmm,        xvvvv,      xmmmem128,  none,       0x0101,  0xdd,      r,   no,  0,     PVEX)
INSTR (VPADDUSW,          ymm,        yvvvv,      ymmmem256,  none,       0x0105,  0xdd,      r,   no,  0,     PVEX)

INSTR (PADDW,             xmm,        xmmmem128,  none,       none,       0,       0x0ffd,    r,   no,  0,     P66)
INSTR (PADDW,             mmx,        mmxmem64,   none,       none,       0,       0x0ffd,    r,   no,  0,     0)

INSTR (VPADDW,            xmm,        xvvvv,      xmmmem128,  none,       0x0101,  0xfd,      r,   no,  0,     PVEX)
INSTR (VPADDW,            ymm,        yvvvv,      ymmmem256,  none,       0x0105,  0xfd,      r,   no,  0,     PVEX)

INSTR (PALIGNR,           xmm,        xmmmem128,  imm8,       none,       0,       0x0f3a0f,  r,   ib,  0,     P66)

INSTR (VPALIGNR,          xmm,        xvvvv,      xmmmem128,  imm8,       0x0301,  0x0f,      r,   ib,  0,     PVEX)
INSTR (VPALIGNR,          ymm,        yvvvv,      ymmmem256,  imm8,       0x0305,  0x0f,      r,   ib,  0,     PVEX)

INSTR (PAND,              xmm,        xmmmem128,  none,       none,       0,       0x0fdb,    r,   no,  0,     P66)
INSTR (PAND,              mmx,        mmxmem64,   none,       none,       0,       0x0fdb,    r,   no,  0,     0)

INSTR (VPAND,             xmm,        xvvvv,      xmmmem128,  none,       0x0101,  0xdb,      r,   no,  0,     PVEX)
INSTR (VPAND,             ymm,        yvvvv,      ymmmem256,  none,       0x0105,  0xdb,      r,   no,  0,     PVEX)

INSTR (PANDN,             xmm,        xmmmem128,  none,       none,       0,       0x0fdf,    r,   no,  0,     P66)
INSTR (PANDN,             mmx,        mmxmem64,   none,       none,       0,       0x0fdf,    r,   no,  0,     0)

INSTR (VPANDN,            xmm,        xvvvv,      xmmmem128,  none,       0x0101,  0xdf,      r,   no,  0,     PVEX)
INSTR (VPANDN,            ymm,        yvvvv,      ymmmem256,  none,       0x0105,  0xdf,      r,   no,  0,     PVEX)

INSTR (PAVGB,             xmm,        xmmmem128,  none,       none,       0,       0x0fe0,    r,   no,  0,     P66)
INSTR (PAVGB,             mmx,        mmxmem64,   none,       none,       0,       0x0fe0,    r,   no,  0,     0)

INSTR (VPAVGB,            xmm,        xvvvv,      xmmmem128,  none,       0x0101,  0xe0,      r,   no,  0,     PVEX)
INSTR (VPAVGB,            ymm,        yvvvv,      ymmmem256,  none,       0x0105,  0xe0,      r,   no,  0,     PVEX)

INSTR (PAVGW,             xmm,        xmmmem128,  none,       none,       0,       0x0fe3,    r,   no,  0,     P66)
INSTR (PAVGW,             mmx,        mmxmem64,   none,       none,       0,       0x0fe3,    r,   no,  0,     0)

INSTR (VPAVGW,            xmm,        xvvvv,      xmmmem128,  none,       0x0101,  0xe3,      r,   no,  0,     PVEX)
INSTR (VPAVGW,            ymm,        yvvvv,      ymmmem256,  none,       0x0105,  0xe3,      r,   no,  0,     PVEX)

INSTR (PBLENDVB,          xmm,        xmmmem128,  none,       none,       0,       0x0f3810,  r,   no,  0,     P66)

INSTR (VPBLENDVB,         xmm,        xvvvv,      xmmmem128,  ximm,       0x0301,  0x4c,      r,   is,  0,     PVEX)
INSTR (VPBLENDVB,         ymm,        yvvvv,      ymmmem256,  yimm,       0x0305,  0x4c,      r,   is,  0,     PVEX)

INSTR (PBLENDW,           xmm,        xmmmem128,  imm8,       none,       0,       0x0f3a0e,  r,   ib,  0,     P66)

INSTR (VPBLENDW,          xmm,        xvvvv,      xmmmem128,  imm8,       0x0301,  0x0e,      r,   ib,  0,     PVEX)
INSTR (VPBLENDW,          ymm,        yvvvv,      ymmmem256,  imm8,       0x0305,  0x0e,      r,   ib,  0,     PVEX)

INSTR (PCLMULQDQ,         xmm,        xmmmem128,  imm8,       none,       0,       0x0f3a44,  r,   ib,  0,     P66)

INSTR (VPCLMULQDQ,        xmm,        xvvvv,      xmmmem128,  imm8,       0x0301,  0x44,      r,   ib,  0,     PVEX)

INSTR (PCMPEQB,           xmm,        xmmmem128,  none,       none,       0,       0x0f74,    r,   no,  0,     P66)
INSTR (PCMPEQB,           mmx,        mmxmem64,   none,       none,       0,       0x0f74,    r,   no,  0,     0)

INSTR (VPCMPEQB,          xmm,        xvvvv,      xmmmem128,  none,       0x0101,  0x74,      r,   no,  0,     PVEX)
INSTR (VPCMPEQB,          ymm,        yvvvv,      ymmmem256,  none,       0x0105,  0x74,      r,   no,  0,     PVEX)

INSTR (PCMPEQD,           xmm,        xmmmem128,  none,       none,       0,       0x0f76,    r,   no,  0,     P66)
INSTR (PCMPEQD,           mmx,        mmxmem64,   none,       none,       0,       0x0f76,    r,   no,  0,     0)

INSTR (VPCMPEQD,          xmm,        xvvvv,      xmmmem128,  none,       0x0101,  0x76,      r,   no,  0,     PVEX)
INSTR (VPCMPEQD,          ymm,        yvvvv,      ymmmem256,  none,       0x0105,  0x76,      r,   no,  0,     PVEX)

INSTR (PCMPEQQ,           xmm,        xmmmem128,  none,       none,       0,       0x0f3829,  r,   no,  0,     P66)

INSTR (VPCMPEQQ,          xmm,        xvvvv,      xmmmem128,  none,       0x0201,  0x29,      r,   no,  0,     PVEX)
INSTR (VPCMPEQQ,          ymm,        yvvvv,      ymmmem256,  none,       0x0205,  0x29,      r,   no,  0,     PVEX)

INSTR (PCMPEQW,           xmm,        xmmmem128,  none,       none,       0,       0x0f75,    r,   no,  0,     P66)
INSTR (PCMPEQW,           mmx,        mmxmem64,   none,       none,       0,       0x0f75,    r,   no,  0,     0)

INSTR (VPCMPEQW,          xmm,        xvvvv,      xmmmem128,  none,       0x0101,  0x75,      r,   no,  0,     PVEX)
INSTR (VPCMPEQW,          ymm,        yvvvv,      ymmmem256,  none,       0x0105,  0x75,      r,   no,  0,     PVEX)

INSTR (PCMPESTRI,         xmm,        xmmmem128,  imm8,       none,       0,       0x0f3a61,  r,   ib,  0,     P66)

INSTR (VPCMPESTRI,        xmm,        xmmmem128,  imm8,       none,       0x0379,  0x61,      r,   ib,  0,     PVEX)

INSTR (PCMPESTRM,         xmm,        xmmmem128,  imm8,       none,       0,       0x0f3a60,  r,   ib,  0,     P66)

INSTR (VPCMPESTRM,        xmm,        xmmmem128,  imm8,       none,       0x0379,  0x60,      r,   ib,  0,     PVEX)

INSTR (PCMPGTB,           xmm,        xmmmem128,  none,       none,       0,       0x0f64,    r,   no,  0,     P66)
INSTR (PCMPGTB,           mmx,        mmxmem64,   none,       none,       0,       0x0f64,    r,   no,  0,     0)

INSTR (VPCMPGTB,          xmm,        xvvvv,      xmmmem128,  none,       0x0101,  0x64,      r,   no,  0,     PVEX)
INSTR (VPCMPGTB,          ymm,        yvvvv,      ymmmem256,  none,       0x0105,  0x64,      r,   no,  0,     PVEX)

INSTR (PCMPGTD,           xmm,        xmmmem128,  none,       none,       0,       0x0f66,    r,   no,  0,     P66)
INSTR (PCMPGTD,           mmx,        mmxmem64,   none,       none,       0,       0x0f66,    r,   no,  0,     0)

INSTR (VPCMPGTD,          xmm,        xvvvv,      xmmmem128,  none,       0x0101,  0x66,      r,   no,  0,     PVEX)
INSTR (VPCMPGTD,          ymm,        yvvvv,      ymmmem256,  none,       0x0105,  0x66,      r,   no,  0,     PVEX)

INSTR (PCMPGTQ,           xmm,        xmmmem128,  none,       none,       0,       0x0f3837,  r,   no,  0,     P66)

INSTR (VPCMPGTQ,          xmm,        xvvvv,      xmmmem128,  none,       0x0201,  0x37,      r,   no,  0,     PVEX)
INSTR (VPCMPGTQ,          ymm,        yvvvv,      ymmmem256,  none,       0x0205,  0x37,      r,   no,  0,     PVEX)

INSTR (PCMPGTW,           xmm,        xmmmem128,  none,       none,       0,       0x0f65,    r,   no,  0,     P66)
INSTR (PCMPGTW,           mmx,        mmxmem64,   none,       none,       0,       0x0f65,    r,   no,  0,     0)

INSTR (VPCMPGTW,          xmm,        xvvvv,      xmmmem128,  none,       0x0101,  0x65,      r,   no,  0,     PVEX)
INSTR (VPCMPGTW,          ymm,        yvvvv,      ymmmem256,  none,       0x0105,  0x65,      r,   no,  0,     PVEX)

INSTR (PCMPISTRI,         xmm,        xmmmem128,  imm8,       none,       0,       0x0f3a63,  r,   ib,  0,     P66)

INSTR (VPCMPISTRI,        xmm,        xmmmem128,  imm8,       none,       0x0379,  0x63,      r,   ib,  0,     PVEX)

INSTR (PCMPISTRM,         xmm,        xmmmem128,  imm8,       none,       0,       0x0f3a62,  r,   ib,  0,     P66)

INSTR (VPCMPISTRM,        xmm,        xmmmem128,  imm8,       none,       0x0379,  0x62,      r,   ib,  0,     PVEX)

INSTR (PEXTRB,            regmem8,    xmm,        imm8,       none,       0,       0x0f3a14,  r,   ib,  0,     P66)

INSTR (VPEXTRB,           regmem8,    xmm,        imm8,       none,       0x0379,  0x14,      r,   ib,  0,     PVEX)

INSTR (PEXTRD,            regmem32,   xmm,        imm8,       none,       0,       0x0f3a16,  r,   ib,  0,     O16 | O32 | P66)

INSTR (VPEXTRD,           regmem32,   xmm,        imm8,       none,       0x0379,  0x16,      r,   ib,  0,     PVEX)

INSTR (PEXTRQ,            regmem64,   xmm,        imm8,       none,       0,       0x0f3a16,  r,   ib,  0,     O64 | P66)

INSTR (VPEXTRQ,           regmem64,   xmm,        imm8,       none,       0x03f9,  0x16,      r,   ib,  0,     PVEX)

INSTR (PEXTRW,            reg32,      xmmmem128,  imm8,       none,       0,       0x0fc5,    r,   ib,  0,     P66)
INSTR (PEXTRW,            reg32,      mmxmem64,   imm8,       none,       0,       0x0fc5,    r,   ib,  0,     0)
INSTR (PEXTRW,            regmem16,   xmm,        imm8,       none,       0,       0x0f3a15,  r,   ib,  0,     P66)

INSTR (VPEXTRW,           reg32,      xmmmem128,  imm8,       none,       0x0179,  0xc5,      r,   ib,  0,     PVEX)
INSTR (VPEXTRW,           regmem16,   xmm,        imm8,       none,       0x0379,  0x15,      r,   ib,  0,     PVEX)

INSTR (PHADDD,            xmm,        xmmmem128,  none,       none,       0,       0x0f3802,  r,   no,  0,     P66)

INSTR (VPHADDD,           xmm,        xvvvv,      xmmmem128,  none,       0x0201,  0x02,      r,   no,  0,     PVEX)
INSTR (VPHADDD,           ymm,        yvvvv,      ymmmem256,  none,       0x0205,  0x02,      r,   no,  0,     PVEX)

INSTR (PHADDSW,           xmm,        xmmmem128,  none,       none,       0,       0x0f3803,  r,   no,  0,     P66)

INSTR (VPHADDSW,          xmm,        xvvvv,      xmmmem128,  none,       0x0201,  0x03,      r,   no,  0,     PVEX)
INSTR (VPHADDSW,          ymm,        yvvvv,      ymmmem256,  none,       0x0205,  0x03,      r,   no,  0,     PVEX)

INSTR (PHADDW,            xmm,        xmmmem128,  none,       none,       0,       0x0f3801,  r,   no,  0,     P66)

INSTR (VPHADDW,           xmm,        xvvvv,      xmmmem128,  none,       0x0201,  0x01,      r,   no,  0,     PVEX)
INSTR (VPHADDW,           ymm,        yvvvv,      ymmmem256,  none,       0x0205,  0x01,      r,   no,  0,     PVEX)

INSTR (PHMINPOSUW,        xmm,        xmmmem128,  none,       none,       0,       0x0f3841,  r,   no,  0,     P66)

INSTR (VPHMINPOSUW,       xmm,        xmmmem128,  none,       none,       0x0279,  0x41,      r,   no,  0,     PVEX)

INSTR (PHSUBD,            xmm,        xmmmem128,  none,       none,       0,       0x0f3806,  r,   no,  0,     P66)

INSTR (VPHSUBD,           xmm,        xvvvv,      xmmmem128,  none,       0x0201,  0x06,      r,   no,  0,     PVEX)
INSTR (VPHSUBD,           ymm,        yvvvv,      ymmmem256,  none,       0x0205,  0x06,      r,   no,  0,     PVEX)

INSTR (PHSUBSW,           xmm,        xmmmem128,  none,       none,       0,       0x0f3807,  r,   no,  0,     P66)

INSTR (VPHSUBSW,          xmm,        xvvvv,      xmmmem128,  none,       0x0201,  0x07,      r,   no,  0,     PVEX)
INSTR (VPHSUBSW,          ymm,        yvvvv,      ymmmem256,  none,       0x0205,  0x07,      r,   no,  0,     PVEX)

INSTR (PHSUBW,            xmm,        xmmmem128,  none,       none,       0,       0x0f3805,  r,   no,  0,     P66)

INSTR (VPHSUBW,           xmm,        xvvvv,      xmmmem128,  none,       0x0201,  0x05,      r,   no,  0,     PVEX)
INSTR (VPHSUBW,           ymm,        yvvvv,      ymmmem256,  none,       0x0205,  0x05,      r,   no,  0,     PVEX)

INSTR (PINSRB,            xmm,        regmem8,    imm8,       none,       0,       0x0f3a20,  r,   ib,  0,     P66)

INSTR (VPINSRB,           xmm,        regmem8,    xvvvv,      imm8,       0x0301,  0x20,      r,   ib,  0,     PVEX)

INSTR (PINSRD,            xmm,        regmem32,   imm8,       none,       0,       0x0f3a22,  r,   ib,  0,     O16 | O32 | P66)

INSTR (VPINSRD,           xmm,        regmem32,   xvvvv,      imm8,       0x0301,  0x22,      r,   ib,  0,     PVEX)

INSTR (PINSRQ,            xmm,        regmem64,   imm8,       none,       0,       0x0f3a22,  r,   ib,  0,     O64 | P66)

INSTR (VPINSRQ,           xmm,        regmem64,   xvvvv,      imm8,       0x0301,  0x22,      r,   ib,  0,     PVEX)

INSTR (PINSRW,            xmm,        regmem32,   imm8,       none,       0,       0x0fc4,    r,   ib,  0,     P66)
INSTR (PINSRW,            mmx,        regmem32,   imm8,       none,       0,       0x0fc4,    r,   ib,  0,     0)

INSTR (VPINSRW,           xmm,        regmem32,   xvvvv,      imm8,       0x0101,  0xc4,      r,   ib,  0,     PVEX)

INSTR (PMADDUBSW,         xmm,        xmmmem128,  none,       none,       0,       0x0f3804,  r,   no,  0,     P66)

INSTR (VPMADDUBSW,        xmm,        xvvvv,      xmmmem128,  none,       0x0201,  0x04,      r,   no,  0,     PVEX)
INSTR (VPMADDUBSW,        ymm,        yvvvv,      ymmmem256,  none,       0x0205,  0x04,      r,   no,  0,     PVEX)

INSTR (PMADDWD,           xmm,        xmmmem128,  none,       none,       0,       0x0ff5,    r,   no,  0,     P66)
INSTR (PMADDWD,           mmx,        mmxmem64,   none,       none,       0,       0x0ff5,    r,   no,  0,     0)

INSTR (VPMADDWD,          xmm,        xvvvv,      xmmmem128,  none,       0x0101,  0xf5,      r,   no,  0,     PVEX)
INSTR (VPMADDWD,          ymm,        yvvvv,      ymmmem256,  none,       0x0105,  0xf5,      r,   no,  0,     PVEX)

INSTR (PMAXSB,            xmm,        xmmmem128,  none,       none,       0,       0x0f383c,  r,   no,  0,     P66)

INSTR (VPMAXSB,           xmm,        xvvvv,      xmmmem128,  none,       0x0201,  0x3c,      r,   no,  0,     PVEX)
INSTR (VPMAXSB,           ymm,        yvvvv,      ymmmem256,  none,       0x0205,  0x3c,      r,   no,  0,     PVEX)

INSTR (PMAXSD,            xmm,        xmmmem128,  none,       none,       0,       0x0f383d,  r,   no,  0,     P66)

INSTR (VPMAXSD,           xmm,        xvvvv,      xmmmem128,  none,       0x0201,  0x3d,      r,   no,  0,     PVEX)
INSTR (VPMAXSD,           ymm,        yvvvv,      ymmmem256,  none,       0x0205,  0x3d,      r,   no,  0,     PVEX)

INSTR (PMAXSW,            xmm,        xmmmem128,  none,       none,       0,       0x0fee,    r,   no,  0,     P66)
INSTR (PMAXSW,            mmx,        mmxmem64,   none,       none,       0,       0x0fee,    r,   no,  0,     0)

INSTR (VPMAXSW,           xmm,        xvvvv,      xmmmem128,  none,       0x0101,  0xee,      r,   no,  0,     PVEX)
INSTR (VPMAXSW,           ymm,        yvvvv,      ymmmem256,  none,       0x0105,  0xee,      r,   no,  0,     PVEX)

INSTR (PMAXUB,            xmm,        xmmmem128,  none,       none,       0,       0x0fde,    r,   no,  0,     P66)
INSTR (PMAXUB,            mmx,        mmxmem64,   none,       none,       0,       0x0fde,    r,   no,  0,     0)

INSTR (VPMAXUB,           xmm,        xvvvv,      xmmmem128,  none,       0x0101,  0xde,      r,   no,  0,     PVEX)
INSTR (VPMAXUB,           ymm,        yvvvv,      ymmmem256,  none,       0x0105,  0xde,      r,   no,  0,     PVEX)

INSTR (PMAXUD,            xmm,        xmmmem128,  none,       none,       0,       0x0f383f,  r,   no,  0,     P66)

INSTR (VPMAXUD,           xmm,        xvvvv,      xmmmem128,  none,       0x0201,  0x3f,      r,   no,  0,     PVEX)
INSTR (VPMAXUD,           ymm,        yvvvv,      ymmmem256,  none,       0x0205,  0x3f,      r,   no,  0,     PVEX)

INSTR (PMAXUW,            xmm,        xmmmem128,  none,       none,       0,       0x0f383e,  r,   no,  0,     P66)

INSTR (VPMAXUW,           xmm,        xvvvv,      xmmmem128,  none,       0x0201,  0x3e,      r,   no,  0,     PVEX)
INSTR (VPMAXUW,           ymm,        yvvvv,      ymmmem256,  none,       0x0205,  0x3e,      r,   no,  0,     PVEX)

INSTR (PMINSB,            xmm,        xmmmem128,  none,       none,       0,       0x0f3838,  r,   no,  0,     P66)

INSTR (VPMINSB,           xmm,        xvvvv,      xmmmem128,  none,       0x0201,  0x38,      r,   no,  0,     PVEX)
INSTR (VPMINSB,           ymm,        yvvvv,      ymmmem256,  none,       0x0205,  0x38,      r,   no,  0,     PVEX)

INSTR (PMINSD,            xmm,        xmmmem128,  none,       none,       0,       0x0f3839,  r,   no,  0,     P66)

INSTR (VPMINSD,           xmm,        xvvvv,      xmmmem128,  none,       0x0201,  0x39,      r,   no,  0,     PVEX)
INSTR (VPMINSD,           ymm,        yvvvv,      ymmmem256,  none,       0x0205,  0x39,      r,   no,  0,     PVEX)

INSTR (PMINSW,            xmm,        xmmmem128,  none,       none,       0,       0x0fea,    r,   no,  0,     P66)
INSTR (PMINSW,            mmx,        mmxmem64,   none,       none,       0,       0x0fea,    r,   no,  0,     0)

INSTR (VPMINSW,           xmm,        xvvvv,      xmmmem128,  none,       0x0101,  0xea,      r,   no,  0,     PVEX)
INSTR (VPMINSW,           ymm,        yvvvv,      ymmmem256,  none,       0x0105,  0xea,      r,   no,  0,     PVEX)

INSTR (PMINUB,            xmm,        xmmmem128,  none,       none,       0,       0x0fda,    r,   no,  0,     P66)
INSTR (PMINUB,            mmx,        mmxmem64,   none,       none,       0,       0x0fda,    r,   no,  0,     0)

INSTR (VPMINUB,           xmm,        xvvvv,      xmmmem128,  none,       0x0101,  0xda,      r,   no,  0,     PVEX)
INSTR (VPMINUB,           ymm,        yvvvv,      ymmmem256,  none,       0x0105,  0xda,      r,   no,  0,     PVEX)

INSTR (PMINUD,            xmm,        xmmmem128,  none,       none,       0,       0x0f383b,  r,   no,  0,     P66)

INSTR (VPMINUD,           xmm,        xvvvv,      xmmmem128,  none,       0x0201,  0x3b,      r,   no,  0,     PVEX)
INSTR (VPMINUD,           ymm,        yvvvv,      ymmmem256,  none,       0x0205,  0x3b,      r,   no,  0,     PVEX)

INSTR (PMINUW,            xmm,        xmmmem128,  none,       none,       0,       0x0f383a,  r,   no,  0,     P66)

INSTR (VPMINUW,           xmm,        xvvvv,      xmmmem128,  none,       0x0201,  0x3a,      r,   no,  0,     PVEX)
INSTR (VPMINUW,           ymm,        yvvvv,      ymmmem256,  none,       0x0205,  0x3a,      r,   no,  0,     PVEX)

INSTR (PMOVMSKB,          reg32,      xmmmem128,  none,       none,       0,       0x0fd7,    r,   no,  0,     P66)
INSTR (PMOVMSKB,          reg32,      mmxmem64,   none,       none,       0,       0x0fd7,    r,   no,  0,     0)

INSTR (VPMOVMSKB,         reg64,      xmmmem128,  none,       none,       0x0179,  0xd7,      r,   no,  0,     PVEX)
INSTR (VPMOVMSKB,         reg64,      ymmmem256,  none,       none,       0x017d,  0xd7,      r,   no,  0,     PVEX)

INSTR (PMOVSXBD,          xmm,        xmmmem32,   none,       none,       0,       0x0f3821,  r,   no,  0,     P66)

INSTR (VPMOVSXBD,         xmm,        xmmmem32,   none,       none,       0x0279,  0x21,      r,   no,  0,     PVEX)
INSTR (VPMOVSXBD,         ymm,        xmmmem64,   none,       none,       0x027d,  0x21,      r,   no,  0,     PVEX)

INSTR (PMOVSXBQ,          xmm,        xmmmem16,   none,       none,       0,       0x0f3822,  r,   no,  0,     P66)

INSTR (VPMOVSXBQ,         xmm,        xmmmem16,   none,       none,       0x0279,  0x22,      r,   no,  0,     PVEX)
INSTR (VPMOVSXBQ,         ymm,        xmmmem32,   none,       none,       0x027d,  0x22,      r,   no,  0,     PVEX)

INSTR (PMOVSXBW,          xmm,        xmmmem64,   none,       none,       0,       0x0f3820,  r,   no,  0,     P66)

INSTR (VPMOVSXBW,         xmm,        xmmmem64,   none,       none,       0x0279,  0x20,      r,   no,  0,     PVEX)
INSTR (VPMOVSXBW,         ymm,        xmmmem128,  none,       none,       0x027d,  0x20,      r,   no,  0,     PVEX)

INSTR (PMOVSXDQ,          xmm,        xmmmem64,   none,       none,       0,       0x0f3825,  r,   no,  0,     P66)

INSTR (VPMOVSXDQ,         xmm,        xmmmem64,   none,       none,       0x0279,  0x25,      r,   no,  0,     PVEX)
INSTR (VPMOVSXDQ,         ymm,        xmmmem128,  none,       none,       0x027d,  0x25,      r,   no,  0,     PVEX)

INSTR (PMOVSXWD,          xmm,        xmmmem64,   none,       none,       0,       0x0f3823,  r,   no,  0,     P66)

INSTR (VPMOVSXWD,         xmm,        xmmmem64,   none,       none,       0x0279,  0x23,      r,   no,  0,     PVEX)
INSTR (VPMOVSXWD,         ymm,        xmmmem128,  none,       none,       0x027d,  0x23,      r,   no,  0,     PVEX)

INSTR (PMOVSXWQ,          xmm,        xmmmem32,   none,       none,       0,       0x0f3824,  r,   no,  0,     P66)

INSTR (VPMOVSXWQ,         xmm,        xmmmem32,   none,       none,       0x0279,  0x24,      r,   no,  0,     PVEX)
INSTR (VPMOVSXWQ,         ymm,        xmmmem64,   none,       none,       0x027d,  0x24,      r,   no,  0,     PVEX)

INSTR (PMOVZXBD,          xmm,        xmmmem32,   none,       none,       0,       0x0f3831,  r,   no,  0,     P66)

INSTR (VPMOVZXBD,         xmm,        xmmmem32,   none,       none,       0x0279,  0x31,      r,   no,  0,     PVEX)
INSTR (VPMOVZXBD,         ymm,        xmmmem64,   none,       none,       0x027d,  0x31,      r,   no,  0,     PVEX)

INSTR (PMOVZXBQ,          xmm,        xmmmem16,   none,       none,       0,       0x0f3832,  r,   no,  0,     P66)

INSTR (VPMOVZXBQ,         xmm,        xmmmem16,   none,       none,       0x0279,  0x32,      r,   no,  0,     PVEX)
INSTR (VPMOVZXBQ,         ymm,        xmmmem32,   none,       none,       0x027d,  0x32,      r,   no,  0,     PVEX)

INSTR (PMOVZXBW,          xmm,        xmmmem64,   none,       none,       0,       0x0f3830,  r,   no,  0,     P66)

INSTR (VPMOVZXBW,         xmm,        xmmmem64,   none,       none,       0x0279,  0x30,      r,   no,  0,     PVEX)
INSTR (VPMOVZXBW,         ymm,        xmmmem128,  none,       none,       0x027d,  0x30,      r,   no,  0,     PVEX)

INSTR (PMOVZXDQ,          xmm,        xmmmem64,   none,       none,       0,       0x0f3835,  r,   no,  0,     P66)

INSTR (VPMOVZXDQ,         xmm,        xmmmem64,   none,       none,       0x0279,  0x35,      r,   no,  0,     PVEX)
INSTR (VPMOVZXDQ,         ymm,        xmmmem128,  none,       none,       0x027d,  0x35,      r,   no,  0,     PVEX)

INSTR (PMOVZXWD,          xmm,        xmmmem64,   none,       none,       0,       0x0f3833,  r,   no,  0,     P66)

INSTR (VPMOVZXWD,         xmm,        xmmmem64,   none,       none,       0x0279,  0x33,      r,   no,  0,     PVEX)
INSTR (VPMOVZXWD,         ymm,        xmmmem128,  none,       none,       0x027d,  0x33,      r,   no,  0,     PVEX)

INSTR (PMOVZXWQ,          xmm,        xmmmem32,   none,       none,       0,       0x0f3834,  r,   no,  0,     P66)

INSTR (VPMOVZXWQ,         xmm,        xmmmem32,   none,       none,       0x0279,  0x34,      r,   no,  0,     PVEX)
INSTR (VPMOVZXWQ,         ymm,        xmmmem64,   none,       none,       0x027d,  0x34,      r,   no,  0,     PVEX)

INSTR (PMULDQ,            xmm,        xmmmem128,  none,       none,       0,       0x0f3828,  r,   no,  0,     P66)

INSTR (VPMULDQ,           xmm,        xvvvv,      xmmmem128,  none,       0x0201,  0x28,      r,   no,  0,     PVEX)
INSTR (VPMULDQ,           ymm,        yvvvv,      ymmmem256,  none,       0x0205,  0x28,      r,   no,  0,     PVEX)

INSTR (PMULHRSW,          xmm,        xmmmem128,  none,       none,       0,       0x0f380b,  r,   no,  0,     P66)

INSTR (VPMULHRSW,         xmm,        xvvvv,      xmmmem128,  none,       0x0201,  0x0b,      r,   no,  0,     PVEX)
INSTR (VPMULHRSW,         ymm,        yvvvv,      ymmmem256,  none,       0x0205,  0x0b,      r,   no,  0,     PVEX)

INSTR (PMULHUW,           xmm,        xmmmem128,  none,       none,       0,       0x0fe4,    r,   no,  0,     P66)
INSTR (PMULHUW,           mmx,        mmxmem64,   none,       none,       0,       0x0fe4,    r,   no,  0,     0)

INSTR (VPMULHUW,          xmm,        xvvvv,      xmmmem128,  none,       0x0101,  0xe4,      r,   no,  0,     PVEX)
INSTR (VPMULHUW,          ymm,        yvvvv,      ymmmem256,  none,       0x0105,  0xe4,      r,   no,  0,     PVEX)

INSTR (PMULHW,            xmm,        xmmmem128,  none,       none,       0,       0x0fe5,    r,   no,  0,     P66)
INSTR (PMULHW,            mmx,        mmxmem64,   none,       none,       0,       0x0fe5,    r,   no,  0,     0)

INSTR (VPMULHW,           xmm,        xvvvv,      xmmmem128,  none,       0x0101,  0xe5,      r,   no,  0,     PVEX)
INSTR (VPMULHW,           ymm,        yvvvv,      ymmmem256,  none,       0x0105,  0xe5,      r,   no,  0,     PVEX)

INSTR (PMULLD,            xmm,        xmmmem128,  none,       none,       0,       0x0f3840,  r,   no,  0,     P66)

INSTR (VPMULLD,           xmm,        xvvvv,      xmmmem128,  none,       0x0201,  0x40,      r,   no,  0,     PVEX)
INSTR (VPMULLD,           ymm,        yvvvv,      ymmmem256,  none,       0x0205,  0x40,      r,   no,  0,     PVEX)

INSTR (PMULLW,            xmm,        xmmmem128,  none,       none,       0,       0x0fd5,    r,   no,  0,     P66)
INSTR (PMULLW,            mmx,        mmxmem64,   none,       none,       0,       0x0fd5,    r,   no,  0,     0)

INSTR (VPMULLW,           xmm,        xvvvv,      xmmmem128,  none,       0x0101,  0xd5,      r,   no,  0,     PVEX)
INSTR (VPMULLW,           ymm,        yvvvv,      ymmmem256,  none,       0x0105,  0xd5,      r,   no,  0,     PVEX)

INSTR (PMULUDQ,           xmm,        xmmmem128,  none,       none,       0,       0x0ff4,    r,   no,  0,     P66)
INSTR (PMULUDQ,           mmx,        mmxmem64,   none,       none,       0,       0x0ff4,    r,   no,  0,     0)

INSTR (VPMULUDQ,          xmm,        xvvvv,      xmmmem128,  none,       0x0101,  0xf4,      r,   no,  0,     PVEX)
INSTR (VPMULUDQ,          ymm,        yvvvv,      ymmmem256,  none,       0x0105,  0xf4,      r,   no,  0,     PVEX)

INSTR (POR,               xmm,        xmmmem128,  none,       none,       0,       0x0feb,    r,   no,  0,     P66)
INSTR (POR,               mmx,        mmxmem64,   none,       none,       0,       0x0feb,    r,   no,  0,     0)

INSTR (VPOR,              xmm,        xvvvv,      xmmmem128,  none,       0x0101,  0xeb,      r,   no,  0,     PVEX)
INSTR (VPOR,              ymm,        yvvvv,      ymmmem256,  none,       0x0105,  0xeb,      r,   no,  0,     PVEX)

INSTR (PSADBW,            xmm,        xmmmem128,  none,       none,       0,       0x0ff6,    r,   no,  0,     P66)
INSTR (PSADBW,            mmx,        mmxmem64,   none,       none,       0,       0x0ff6,    r,   no,  0,     0)

INSTR (VPSADBW,           xmm,        xvvvv,      xmmmem128,  none,       0x0101,  0xf6,      r,   no,  0,     PVEX)
INSTR (VPSADBW,           ymm,        yvvvv,      ymmmem256,  none,       0x0105,  0xf6,      r,   no,  0,     PVEX)

INSTR (PSHUFB,            xmm,        xmmmem128,  none,       none,       0,       0x0f3800,  r,   no,  0,     P66)

INSTR (VPSHUFB,           xmm,        xvvvv,      xmmmem128,  none,       0x0201,  0x00,      r,   no,  0,     PVEX)
INSTR (VPSHUFB,           ymm,        yvvvv,      ymmmem256,  none,       0x0205,  0x00,      r,   no,  0,     PVEX)

INSTR (PSHUFD,            xmm,        xmmmem128,  imm8,       none,       0,       0x0f70,    r,   ib,  0,     P66)

INSTR (VPSHUFD,           xmm,        xmmmem128,  imm8,       none,       0x0179,  0x70,      r,   ib,  0,     PVEX)
INSTR (VPSHUFD,           ymm,        ymmmem256,  imm8,       none,       0x017d,  0x70,      r,   ib,  0,     PVEX)

INSTR (PSHUFLW,           xmm,        xmmmem128,  imm8,       none,       0,       0x0f70,    r,   ib,  0,     PF2)

INSTR (VPSHUFLW,          xmm,        xmmmem128,  imm8,       none,       0x017b,  0x70,      r,   ib,  0,     PVEX)
INSTR (VPSHUFLW,          ymm,        ymmmem256,  imm8,       none,       0x017f,  0x70,      r,   ib,  0,     PVEX)

INSTR (PSHUFHW,           xmm,        xmmmem128,  imm8,       none,       0,       0x0f70,    r,   ib,  0,     PF3)

INSTR (VPSHUFHW,          xmm,        xmmmem128,  imm8,       none,       0x017a,  0x70,      r,   ib,  0,     PVEX)
INSTR (VPSHUFHW,          ymm,        ymmmem256,  imm8,       none,       0x017e,  0x70,      r,   ib,  0,     PVEX)

INSTR (PSIGNB,            xmm,        xmmmem128,  none,       none,       0,       0x0f3808,  r,   no,  0,     P66)

INSTR (VPSIGNB,           xmm,        xvvvv,      xmmmem128,  none,       0x0201,  0x08,      r,   no,  0,     PVEX)
INSTR (VPSIGNB,           ymm,        yvvvv,      ymmmem256,  none,       0x0205,  0x08,      r,   no,  0,     PVEX)

INSTR (PSIGND,            xmm,        xmmmem128,  none,       none,       0,       0x0f380a,  r,   no,  0,     P66)

INSTR (VPSIGND,           xmm,        xvvvv,      xmmmem128,  none,       0x0201,  0x0a,      r,   no,  0,     PVEX)
INSTR (VPSIGND,           ymm,        yvvvv,      ymmmem256,  none,       0x0205,  0x0a,      r,   no,  0,     PVEX)

INSTR (PSIGNW,            xmm,        xmmmem128,  none,       none,       0,       0x0f3809,  r,   no,  0,     P66)

INSTR (VPSIGNW,           xmm,        xvvvv,      xmmmem128,  none,       0x0201,  0x09,      r,   no,  0,     PVEX)
INSTR (VPSIGNW,           ymm,        yvvvv,      ymmmem256,  none,       0x0205,  0x09,      r,   no,  0,     PVEX)

INSTR (PSLLD,             xmm,        xmmmem128,  none,       none,       0,       0x0ff2,    r,   no,  0,     P66)
INSTR (PSLLD,             mmx,        mmxmem64,   none,       none,       0,       0x0ff2,    r,   no,  0,     0)
INSTR (PSLLD,             xmmmem128,  imm8,       none,       none,       0,       0x0f72,    r6,  ib,  0,     P66)
INSTR (PSLLD,             mmxmem64,   imm8,       none,       none,       0,       0x0f72,    r6,  ib,  0,     0)

INSTR (VPSLLD,            xmm,        xvvvv,      xmmmem128,  none,       0x0101,  0xf2,      r,   no,  0,     PVEX)
INSTR (VPSLLD,            xvvvv,      xmmmem128,  imm8,       none,       0x0101,  0x72,      r6,  ib,  0,     PVEX)
INSTR (VPSLLD,            ymm,        yvvvv,      xmmmem128,  none,       0x0105,  0xf2,      r,   no,  0,     PVEX)
INSTR (VPSLLD,            yvvvv,      ymmmem256,  imm8,       none,       0x0105,  0x72,      r6,  ib,  0,     PVEX)

INSTR (PSLLDQ,            xmmmem128,  imm8,       none,       none,       0,       0x0f73,    r7,  ib,  0,     P66)

INSTR (VPSLLDQ,           xvvvv,      xmmmem128,  imm8,       none,       0x0101,  0x73,      r7,  ib,  0,     PVEX)
INSTR (VPSLLDQ,           yvvvv,      ymmmem256,  imm8,       none,       0x0105,  0x73,      r7,  ib,  0,     PVEX)

INSTR (PSLLQ,             xmm,        xmmmem128,  none,       none,       0,       0x0ff3,    r,   no,  0,     P66)
INSTR (PSLLQ,             mmx,        mmxmem64,   none,       none,       0,       0x0ff3,    r,   no,  0,     0)
INSTR (PSLLQ,             xmmmem128,  imm8,       none,       none,       0,       0x0f73,    r6,  ib,  0,     P66)
INSTR (PSLLQ,             mmxmem64,   imm8,       none,       none,       0,       0x0f73,    r6,  ib,  0,     0)

INSTR (VPSLLQ,            xmm,        xvvvv,      xmmmem128,  none,       0x0101,  0xf3,      r,   no,  0,     PVEX)
INSTR (VPSLLQ,            xvvvv,      xmmmem128,  imm8,       none,       0x0101,  0x73,      r6,  ib,  0,     PVEX)
INSTR (VPSLLQ,            ymm,        yvvvv,      xmmmem128,  none,       0x0105,  0xf3,      r,   no,  0,     PVEX)
INSTR (VPSLLQ,            yvvvv,      ymmmem256,  imm8,       none,       0x0105,  0x73,      r6,  ib,  0,     PVEX)

INSTR (PSLLW,             xmm,        xmmmem128,  none,       none,       0,       0x0ff1,    r,   no,  0,     P66)
INSTR (PSLLW,             mmx,        mmxmem64,   none,       none,       0,       0x0ff1,    r,   no,  0,     0)
INSTR (PSLLW,             xmmmem128,  imm8,       none,       none,       0,       0x0f71,    r6,  ib,  0,     P66)
INSTR (PSLLW,             mmxmem64,   imm8,       none,       none,       0,       0x0f71,    r6,  ib,  0,     0)

INSTR (VPSLLW,            xmm,        xvvvv,      xmmmem128,  none,       0x0101,  0xf1,      r,   no,  0,     PVEX)
INSTR (VPSLLW,            xvvvv,      xmmmem128,  imm8,       none,       0x0101,  0x71,      r6,  ib,  0,     PVEX)
INSTR (VPSLLW,            ymm,        yvvvv,      xmmmem128,  none,       0x0105,  0xf1,      r,   no,  0,     PVEX)
INSTR (VPSLLW,            yvvvv,      ymmmem256,  imm8,       none,       0x0105,  0x71,      r6,  ib,  0,     PVEX)

INSTR (PSRAD,             xmm,        xmmmem128,  none,       none,       0,       0x0fe2,    r,   no,  0,     P66)
INSTR (PSRAD,             mmx,        mmxmem64,   none,       none,       0,       0x0fe2,    r,   no,  0,     0)
INSTR (PSRAD,             xmmmem128,  imm8,       none,       none,       0,       0x0f72,    r4,  ib,  0,     P66)
INSTR (PSRAD,             mmxmem64,   imm8,       none,       none,       0,       0x0f72,    r4,  ib,  0,     0)

INSTR (VPSRAD,            xmm,        xvvvv,      xmmmem128,  none,       0x0101,  0xe2,      r,   no,  0,     PVEX)
INSTR (VPSRAD,            xvvvv,      xmmmem128,  imm8,       none,       0x0101,  0x72,      r4,  ib,  0,     PVEX)
INSTR (VPSRAD,            ymm,        yvvvv,      xmmmem128,  none,       0x0105,  0xe2,      r,   no,  0,     PVEX)
INSTR (VPSRAD,            yvvvv,      ymmmem256,  imm8,       none,       0x0105,  0x72,      r4,  ib,  0,     PVEX)

INSTR (PSRAW,             xmm,        xmmmem128,  none,       none,       0,       0x0fe1,    r,   no,  0,     P66)
INSTR (PSRAW,             mmx,        mmxmem64,   none,       none,       0,       0x0fe1,    r,   no,  0,     0)
INSTR (PSRAW,             xmmmem128,  imm8,       none,       none,       0,       0x0f71,    r4,  ib,  0,     P66)
INSTR (PSRAW,             mmxmem64,   imm8,       none,       none,       0,       0x0f71,    r4,  ib,  0,     0)

INSTR (VPSRAW,            xmm,        xvvvv,      xmmmem128,  none,       0x0101,  0xe1,      r,   no,  0,     PVEX)
INSTR (VPSRAW,            xvvvv,      xmmmem128,  imm8,       none,       0x0101,  0x71,      r4,  ib,  0,     PVEX)
INSTR (VPSRAW,            ymm,        yvvvv,      xmmmem128,  none,       0x0105,  0xe1,      r,   no,  0,     PVEX)
INSTR (VPSRAW,            yvvvv,      ymmmem256,  imm8,       none,       0x0105,  0x71,      r4,  ib,  0,     PVEX)

INSTR (PSRLD,             xmm,        xmmmem128,  none,       none,       0,       0x0fd2,    r,   no,  0,     P66)
INSTR (PSRLD,             mmx,        mmxmem64,   none,       none,       0,       0x0fd2,    r,   no,  0,     0)
INSTR (PSRLD,             xmmmem128,  imm8,       none,       none,       0,       0x0f72,    r2,  ib,  0,     P66)
INSTR (PSRLD,             mmxmem64,   imm8,       none,       none,       0,       0x0f72,    r2,  ib,  0,     0)

INSTR (VPSRLD,            xmm,        xvvvv,      xmmmem128,  none,       0x0101,  0xd2,      r,   no,  0,     PVEX)
INSTR (VPSRLD,            xvvvv,      xmmmem128,  imm8,       none,       0x0101,  0x72,      r2,  ib,  0,     PVEX)
INSTR (VPSRLD,            ymm,        yvvvv,      xmmmem128,  none,       0x0105,  0xd2,      r,   no,  0,     PVEX)
INSTR (VPSRLD,            yvvvv,      ymmmem256,  imm8,       none,       0x0105,  0x72,      r2,  ib,  0,     PVEX)

INSTR (PSRLDQ,            xmmmem128,  imm8,       none,       none,       0,       0x0f73,    r3,  ib,  0,     P66)

INSTR (VPSRLDQ,           xvvvv,      xmmmem128,  imm8,       none,       0x0101,  0x73,      r3,  ib,  0,     PVEX)
INSTR (VPSRLDQ,           yvvvv,      ymmmem256,  imm8,       none,       0x0105,  0x73,      r3,  ib,  0,     PVEX)

INSTR (PSRLQ,             xmm,        xmmmem128,  none,       none,       0,       0x0fd3,    r,   no,  0,     P66)
INSTR (PSRLQ,             mmx,        mmxmem64,   none,       none,       0,       0x0fd3,    r,   no,  0,     0)
INSTR (PSRLQ,             xmmmem128,  imm8,       none,       none,       0,       0x0f73,    r2,  ib,  0,     P66)
INSTR (PSRLQ,             mmxmem64,   imm8,       none,       none,       0,       0x0f73,    r2,  ib,  0,     0)

INSTR (VPSRLQ,            xmm,        xvvvv,      xmmmem128,  none,       0x0101,  0xd3,      r,   no,  0,     PVEX)
INSTR (VPSRLQ,            xvvvv,      xmmmem128,  imm8,       none,       0x0101,  0x73,      r2,  ib,  0,     PVEX)
INSTR (VPSRLQ,            ymm,        yvvvv,      xmmmem128,  none,       0x0105,  0xd3,      r,   no,  0,     PVEX)
INSTR (VPSRLQ,            yvvvv,      ymmmem256,  imm8,       none,       0x0105,  0x73,      r2,  ib,  0,     PVEX)

INSTR (PSRLW,             xmm,        xmmmem128,  none,       none,       0,       0x0fd1,    r,   no,  0,     P66)
INSTR (PSRLW,             mmx,        mmxmem64,   none,       none,       0,       0x0fd1,    r,   no,  0,     0)
INSTR (PSRLW,             xmmmem128,  imm8,       none,       none,       0,       0x0f71,    r2,  ib,  0,     P66)
INSTR (PSRLW,             mmxmem64,   imm8,       none,       none,       0,       0x0f71,    r2,  ib,  0,     0)

INSTR (VPSRLW,            xmm,        xvvvv,      xmmmem128,  none,       0x0101,  0xd1,      r,   no,  0,     PVEX)
INSTR (VPSRLW,            xvvvv,      xmmmem128,  imm8,       none,       0x0101,  0x71,      r2,  ib,  0,     PVEX)
INSTR (VPSRLW,            ymm,        yvvvv,      xmmmem128,  none,       0x0105,  0xd1,      r,   no,  0,     PVEX)
INSTR (VPSRLW,            yvvvv,      ymmmem256,  imm8,       none,       0x0105,  0x71,      r2,  ib,  0,     PVEX)

INSTR (PSUBB,             xmm,        xmmmem128,  none,       none,       0,       0x0ff8,    r,   no,  0,     P66)
INSTR (PSUBB,             mmx,        mmxmem64,   none,       none,       0,       0x0ff8,    r,   no,  0,     0)

INSTR (VPSUBB,            xmm,        xvvvv,      xmmmem128,  none,       0x0101,  0xf8,      r,   no,  0,     PVEX)
INSTR (VPSUBB,            ymm,        yvvvv,      ymmmem256,  none,       0x0105,  0xf8,      r,   no,  0,     PVEX)

INSTR (PSUBD,             xmm,        xmmmem128,  none,       none,       0,       0x0ffa,    r,   no,  0,     P66)
INSTR (PSUBD,             mmx,        mmxmem64,   none,       none,       0,       0x0ffa,    r,   no,  0,     0)

INSTR (VPSUBD,            xmm,        xvvvv,      xmmmem128,  none,       0x0101,  0xfa,      r,   no,  0,     PVEX)
INSTR (VPSUBD,            ymm,        yvvvv,      ymmmem256,  none,       0x0105,  0xfa,      r,   no,  0,     PVEX)

INSTR (PSUBQ,             xmm,        xmmmem128,  none,       none,       0,       0x0ffb,    r,   no,  0,     P66)
INSTR (PSUBQ,             mmx,        mmxmem64,   none,       none,       0,       0x0ffb,    r,   no,  0,     0)

INSTR (VPSUBQ,            xmm,        xvvvv,      xmmmem128,  none,       0x0101,  0xfb,      r,   no,  0,     PVEX)
INSTR (VPSUBQ,            ymm,        yvvvv,      ymmmem256,  none,       0x0105,  0xfb,      r,   no,  0,     PVEX)

INSTR (PSUBSB,            xmm,        xmmmem128,  none,       none,       0,       0x0fe8,    r,   no,  0,     P66)
INSTR (PSUBSB,            mmx,        mmxmem64,   none,       none,       0,       0x0fe8,    r,   no,  0,     0)

INSTR (VPSUBSB,           xmm,        xvvvv,      xmmmem128,  none,       0x0101,  0xe8,      r,   no,  0,     PVEX)
INSTR (VPSUBSB,           ymm,        yvvvv,      ymmmem256,  none,       0x0105,  0xe8,      r,   no,  0,     PVEX)

INSTR (PSUBSW,            xmm,        xmmmem128,  none,       none,       0,       0x0fe9,    r,   no,  0,     P66)
INSTR (PSUBSW,            mmx,        mmxmem64,   none,       none,       0,       0x0fe9,    r,   no,  0,     0)

INSTR (VPSUBSW,           xmm,        xvvvv,      xmmmem128,  none,       0x0101,  0xe9,      r,   no,  0,     PVEX)
INSTR (VPSUBSW,           ymm,        yvvvv,      ymmmem256,  none,       0x0105,  0xe9,      r,   no,  0,     PVEX)

INSTR (PSUBUSB,           xmm,        xmmmem128,  none,       none,       0,       0x0fd8,    r,   no,  0,     P66)
INSTR (PSUBUSB,           mmx,        mmxmem64,   none,       none,       0,       0x0fd8,    r,   no,  0,     0)

INSTR (VPSUBUSB,          xmm,        xvvvv,      xmmmem128,  none,       0x0101,  0xd8,      r,   no,  0,     PVEX)
INSTR (VPSUBUSB,          ymm,        yvvvv,      ymmmem256,  none,       0x0105,  0xd8,      r,   no,  0,     PVEX)

INSTR (PSUBUSW,           xmm,        xmmmem128,  none,       none,       0,       0x0fd9,    r,   no,  0,     P66)
INSTR (PSUBUSW,           mmx,        mmxmem64,   none,       none,       0,       0x0fd9,    r,   no,  0,     0)

INSTR (VPSUBUSW,          xmm,        xvvvv,      xmmmem128,  none,       0x0101,  0xd9,      r,   no,  0,     PVEX)
INSTR (VPSUBUSW,          ymm,        yvvvv,      ymmmem256,  none,       0x0105,  0xd9,      r,   no,  0,     PVEX)

INSTR (PSUBW,             xmm,        xmmmem128,  none,       none,       0,       0x0ff9,    r,   no,  0,     P66)
INSTR (PSUBW,             mmx,        mmxmem64,   none,       none,       0,       0x0ff9,    r,   no,  0,     0)

INSTR (VPSUBW,            xmm,        xvvvv,      xmmmem128,  none,       0x0101,  0xf9,      r,   no,  0,     PVEX)
INSTR (VPSUBW,            ymm,        yvvvv,      ymmmem256,  none,       0x0105,  0xf9,      r,   no,  0,     PVEX)

INSTR (PTEST,             xmm,        xmmmem128,  none,       none,       0,       0x0f3817,  r,   no,  0,     P66)

INSTR (VPTEST,            xmm,        xmmmem128,  none,       none,       0x0279,  0x17,      r,   no,  0,     PVEX)
INSTR (VPTEST,            ymm,        ymmmem256,  none,       none,       0x027d,  0x17,      r,   no,  0,     PVEX)

INSTR (PUNPCKHBW,         xmm,        xmmmem128,  none,       none,       0,       0x0f68,    r,   no,  0,     P66)
INSTR (PUNPCKHBW,         mmx,        mmxmem64,   none,       none,       0,       0x0f68,    r,   no,  0,     0)

INSTR (VPUNPCKHBW,        xmm,        xvvvv,      xmmmem128,  none,       0x0101,  0x68,      r,   no,  0,     PVEX)
INSTR (VPUNPCKHBW,        ymm,        yvvvv,      ymmmem256,  none,       0x0105,  0x68,      r,   no,  0,     PVEX)

INSTR (PUNPCKHDQ,         xmm,        xmmmem128,  none,       none,       0,       0x0f6a,    r,   no,  0,     P66)
INSTR (PUNPCKHDQ,         mmx,        mmxmem64,   none,       none,       0,       0x0f6a,    r,   no,  0,     0)

INSTR (VPUNPCKHDQ,        xmm,        xvvvv,      xmmmem128,  none,       0x0101,  0x6a,      r,   no,  0,     PVEX)
INSTR (VPUNPCKHDQ,        ymm,        yvvvv,      ymmmem256,  none,       0x0105,  0x6a,      r,   no,  0,     PVEX)

INSTR (PUNPCKHQDQ,        xmm,        xmmmem128,  none,       none,       0,       0x0f6d,    r,   no,  0,     P66)

INSTR (VPUNPCKHQDQ,       xmm,        xvvvv,      xmmmem128,  none,       0x0101,  0x6d,      r,   no,  0,     PVEX)
INSTR (VPUNPCKHQDQ,       ymm,        yvvvv,      ymmmem256,  none,       0x0105,  0x6d,      r,   no,  0,     PVEX)

INSTR (PUNPCKHWD,         xmm,        xmmmem128,  none,       none,       0,       0x0f69,    r,   no,  0,     P66)
INSTR (PUNPCKHWD,         mmx,        mmxmem64,   none,       none,       0,       0x0f69,    r,   no,  0,     0)

INSTR (VPUNPCKHWD,        xmm,        xvvvv,      xmmmem128,  none,       0x0101,  0x69,      r,   no,  0,     PVEX)
INSTR (VPUNPCKHWD,        ymm,        yvvvv,      ymmmem256,  none,       0x0105,  0x69,      r,   no,  0,     PVEX)

INSTR (PUNPCKLBW,         xmm,        xmmmem128,  none,       none,       0,       0x0f60,    r,   no,  0,     P66)
INSTR (PUNPCKLBW,         mmx,        mmxmem32,   none,       none,       0,       0x0f60,    r,   no,  0,     0)

INSTR (VPUNPCKLBW,        xmm,        xvvvv,      xmmmem128,  none,       0x0101,  0x60,      r,   no,  0,     PVEX)
INSTR (VPUNPCKLBW,        ymm,        yvvvv,      ymmmem256,  none,       0x0105,  0x60,      r,   no,  0,     PVEX)

INSTR (PUNPCKLDQ,         xmm,        xmmmem128,  none,       none,       0,       0x0f62,    r,   no,  0,     P66)
INSTR (PUNPCKLDQ,         mmx,        mmxmem32,   none,       none,       0,       0x0f62,    r,   no,  0,     0)

INSTR (VPUNPCKLDQ,        xmm,        xvvvv,      xmmmem128,  none,       0x0101,  0x62,      r,   no,  0,     PVEX)
INSTR (VPUNPCKLDQ,        ymm,        yvvvv,      ymmmem256,  none,       0x0105,  0x62,      r,   no,  0,     PVEX)

INSTR (PUNPCKLQDQ,        xmm,        xmmmem128,  none,       none,       0,       0x0f6c,    r,   no,  0,     P66)

INSTR (VPUNPCKLQDQ,       xmm,        xvvvv,      xmmmem128,  none,       0x0101,  0x6c,      r,   no,  0,     PVEX)
INSTR (VPUNPCKLQDQ,       ymm,        yvvvv,      ymmmem256,  none,       0x0105,  0x6c,      r,   no,  0,     PVEX)

INSTR (PUNPCKLWD,         xmm,        xmmmem128,  none,       none,       0,       0x0f61,    r,   no,  0,     P66)
INSTR (PUNPCKLWD,         mmx,        mmxmem32,   none,       none,       0,       0x0f61,    r,   no,  0,     0)

INSTR (VPUNPCKLWD,        xmm,        xvvvv,      xmmmem128,  none,       0x0101,  0x61,      r,   no,  0,     PVEX)
INSTR (VPUNPCKLWD,        ymm,        yvvvv,      ymmmem256,  none,       0x0105,  0x61,      r,   no,  0,     PVEX)

INSTR (PXOR,              xmm,        xmmmem128,  none,       none,       0,       0x0fef,    r,   no,  0,     P66)
INSTR (PXOR,              mmx,        mmxmem64,   none,       none,       0,       0x0fef,    r,   no,  0,     0)

INSTR (VPXOR,             xmm,        xvvvv,      xmmmem128,  none,       0x0101,  0xef,      r,   no,  0,     PVEX)
INSTR (VPXOR,             ymm,        yvvvv,      ymmmem256,  none,       0x0105,  0xef,      r,   no,  0,     PVEX)

INSTR (RCPPS,             xmm,        xmmmem128,  none,       none,       0,       0x0f53,    r,   no,  0,     0)

INSTR (VRCPPS,            xmm,        xmmmem128,  none,       none,       0x0178,  0x53,      r,   no,  0,     PVEX)
INSTR (VRCPPS,            ymm,        ymmmem256,  none,       none,       0x017c,  0x53,      r,   no,  0,     PVEX)

INSTR (RCPSS,             xmm,        xmmmem32,   none,       none,       0,       0x0f53,    r,   no,  0,     PF3)

INSTR (VRCPSS,            xmm,        xvvvv,      xmmmem128,  none,       0x0102,  0x53,      r,   no,  0,     PVEX)

INSTR (ROUNDPD,           xmm,        xmmmem128,  imm8,       none,       0,       0x0f3a09,  r,   ib,  0,     P66)

INSTR (VROUNDPD,          xmm,        xmmmem128,  imm8,       none,       0x0301,  0x09,      r,   ib,  0,     PVEX)
INSTR (VROUNDPD,          ymm,        ymmmem256,  imm8,       none,       0x0305,  0x09,      r,   ib,  0,     PVEX)

INSTR (ROUNDPS,           xmm,        xmmmem128,  imm8,       none,       0,       0x0f3a08,  r,   ib,  0,     P66)

INSTR (VROUNDPS,          xmm,        xmmmem128,  imm8,       none,       0x0301,  0x08,      r,   ib,  0,     PVEX)
INSTR (VROUNDPS,          ymm,        ymmmem256,  imm8,       none,       0x0305,  0x08,      r,   ib,  0,     PVEX)

INSTR (ROUNDSD,           xmm,        xmmmem64,   imm8,       none,       0,       0x0f3a0b,  r,   ib,  0,     P66)

INSTR (VROUNDSD,          xmm,        xmmmem64,   imm8,       none,       0x0301,  0x0b,      r,   ib,  0,     PVEX)

INSTR (ROUNDSS,           xmm,        xmmmem32,   imm8,       none,       0,       0x0f3a0a,  r,   ib,  0,     P66)

INSTR (VROUNDSS,          xmm,        xmmmem32,   imm8,       none,       0x0301,  0x0a,      r,   ib,  0,     PVEX)

INSTR (SHA1MSG1,          xmm,        xmmmem128,  none,       none,       0,       0x0f38c9,  r,   no,  0,     0)

INSTR (SHA1MSG2,          xmm,        xmmmem128,  none,       none,       0,       0x0f38ca,  r,   no,  0,     0)

INSTR (SHA1NEXTE,         xmm,        xmmmem128,  none,       none,       0,       0x0f38c8,  r,   no,  0,     0)

INSTR (SHA1RNDS4,         xmm,        xmmmem128,  imm8,       none,       0,       0x0f3acc,  r,   ib,  0,     0)

INSTR (SHA256MSG1,        xmm,        xmmmem128,  none,       none,       0,       0x0f38cc,  r,   no,  0,     0)

INSTR (SHA256MSG2,        xmm,        xmmmem128,  none,       none,       0,       0x0f38cd,  r,   no,  0,     0)

INSTR (SHA256RNDS2,       xmm,        xmmmem128,  none,       none,       0,       0x0f38cb,  r,   no,  0,     0)

INSTR (RSQRTSS,           xmm,        xmmmem32,   none,       none,       0,       0x0f52,    r,   no,  0,     PF3)

INSTR (VRSQRTSS,          xmm,        xvvvv,      xmmmem32,   none,       0x0102,  0x52,      r,   no,  0,     PVEX)

INSTR (RSQRTPS,           xmm,        xmmmem128,  none,       none,       0,       0x0f52,    r,   no,  0,     0)

INSTR (VRSQRTPS,          xmm,        xmmmem128,  none,       none,       0x0178,  0x52,      r,   no,  0,     PVEX)
INSTR (VRSQRTPS,          ymm,        ymmmem256,  none,       none,       0x017c,  0x52,      r,   no,  0,     PVEX)

INSTR (SHUFPD,            xmm,        xmmmem128,  imm8,       none,       0,       0x0fc6,    r,   ib,  0,     P66)

INSTR (VSHUFPD,           xmm,        xvvvv,      xmmmem128,  imm8,       0x0101,  0xc6,      r,   ib,  0,     PVEX)
INSTR (VSHUFPD,           ymm,        yvvvv,      ymmmem256,  imm8,       0x0105,  0xc6,      r,   ib,  0,     PVEX)

INSTR (SHUFPS,            xmm,        xmmmem128,  imm8,       none,       0,       0x0fc6,    r,   ib,  0,     0)

INSTR (VSHUFPS,           xmm,        xvvvv,      xmmmem128,  imm8,       0x0100,  0xc6,      r,   ib,  0,     PVEX)
INSTR (VSHUFPS,           ymm,        yvvvv,      ymmmem256,  imm8,       0x0104,  0xc6,      r,   ib,  0,     PVEX)

INSTR (SQRTPD,            xmm,        xmmmem128,  none,       none,       0,       0x0f51,    r,   no,  0,     P66)

INSTR (VSQRTPD,           xmm,        xmmmem128,  none,       none,       0x0179,  0x51,      r,   no,  0,     PVEX)
INSTR (VSQRTPD,           ymm,        ymmmem256,  none,       none,       0x017d,  0x51,      r,   no,  0,     PVEX)

INSTR (SQRTSD,            xmm,        xmmmem64,   none,       none,       0,       0x0f51,    r,   no,  0,     PF2)

INSTR (VSQRTSD,           xmm,        xvvvv,      xmmmem64,   none,       0x0103,  0x51,      r,   no,  0,     PVEX)

INSTR (SQRTSS,            xmm,        xmmmem32,   none,       none,       0,       0x0f51,    r,   no,  0,     PF3)

INSTR (VSQRTSS,           xmm,        xvvvv,      xmmmem32,   none,       0x0102,  0x51,      r,   no,  0,     PVEX)

INSTR (SQRTPS,            xmm,        xmmmem128,  none,       none,       0,       0x0f51,    r,   no,  0,     0)

INSTR (VSQRTPS,           xmm,        xmmmem128,  none,       none,       0x0178,  0x51,      r,   no,  0,     PVEX)
INSTR (VSQRTPS,           ymm,        ymmmem256,  none,       none,       0x017c,  0x51,      r,   no,  0,     PVEX)

INSTR (SUBPD,             xmm,        xmmmem128,  none,       none,       0,       0x0f5c,    r,   no,  0,     P66)

INSTR (VSUBPD,            xmm,        xvvvv,      xmmmem128,  none,       0x0101,  0x5c,      r,   no,  0,     PVEX)
INSTR (VSUBPD,            ymm,        yvvvv,      ymmmem256,  none,       0x0105,  0x5c,      r,   no,  0,     PVEX)

INSTR (SUBSD,             xmm,        xmmmem64,   none,       none,       0,       0x0f5c,    r,   no,  0,     PF2)

INSTR (VSUBSD,            xmm,        xvvvv,      xmmmem64,   none,       0x0103,  0x5c,      r,   no,  0,     PVEX)

INSTR (SUBSS,             xmm,        xmmmem32,   none,       none,       0,       0x0f5c,    r,   no,  0,     PF3)

INSTR (VSUBSS,            xmm,        xvvvv,      xmmmem32,   none,       0x0102,  0x5c,      r,   no,  0,     PVEX)

INSTR (SUBPS,             xmm,        xmmmem128,  none,       none,       0,       0x0f5c,    r,   no,  0,     0)

INSTR (VSUBPS,            xmm,        xvvvv,      xmmmem128,  none,       0x0100,  0x5c,      r,   no,  0,     PVEX)
INSTR (VSUBPS,            ymm,        yvvvv,      ymmmem256,  none,       0x0104,  0x5c,      r,   no,  0,     PVEX)

INSTR (T1MSKC,            reg32vvvv,  regmem32,   none,       none,       0x0900,  0x01,      r7,  no,  0,     PXOP)
INSTR (T1MSKC,            reg64vvvv,  regmem64,   none,       none,       0x0980,  0x01,      r7,  no,  0,     PXOP)

INSTR (TZCNT,             reg16,      regmem16,   none,       none,       0,       0x0fbc,    r,   no,  0,     O16 | PF3)
INSTR (TZCNT,             reg32,      regmem32,   none,       none,       0,       0x0fbc,    r,   no,  0,     O32 | PF3)
INSTR (TZCNT,             reg64,      regmem64,   none,       none,       0,       0x0fbc,    r,   no,  0,     O64 | PF3)

INSTR (TZMSK,             reg32vvvv,  regmem32,   none,       none,       0x0900,  0x01,      r4,  no,  0,     PXOP)
INSTR (TZMSK,             reg64vvvv,  regmem64,   none,       none,       0x0980,  0x01,      r4,  no,  0,     PXOP)

INSTR (UCOMISD,           xmm,        xmmmem64,   none,       none,       0,       0x0f2e,    r,   no,  0,     P66)

INSTR (VUCOMISD,          xmm,        xmmmem64,   none,       none,       0x0179,  0x2e,      r,   no,  0,     PVEX)

INSTR (UCOMISS,           xmm,        xmmmem32,   none,       none,       0,       0x0f2e,    r,   no,  0,     0)

INSTR (VUCOMISS,          xmm,        xmmmem32,   none,       none,       0x0178,  0x2e,      r,   no,  0,     PVEX)

INSTR (UNPCKHPD,          xmm,        xmmmem128,  none,       none,       0,       0x0f15,    r,   no,  0,     P66)

INSTR (VUNPCKHPD,         xmm,        xvvvv,      xmmmem128,  none,       0x0101,  0x15,      r,   no,  0,     PVEX)
INSTR (VUNPCKHPD,         ymm,        yvvvv,      ymmmem256,  none,       0x0105,  0x15,      r,   no,  0,     PVEX)

INSTR (UNPCKHPS,          xmm,        xmmmem128,  none,       none,       0,       0x0f15,    r,   no,  0,     0)

INSTR (VUNPCKHPS,         xmm,        xvvvv,      xmmmem128,  none,       0x0100,  0x15,      r,   no,  0,     PVEX)
INSTR (VUNPCKHPS,         ymm,        yvvvv,      ymmmem256,  none,       0x0104,  0x15,      r,   no,  0,     PVEX)

INSTR (UNPCKLPD,          xmm,        xmmmem128,  none,       none,       0,       0x0f14,    r,   no,  0,     P66)

INSTR (VUNPCKLPD,         xmm,        xvvvv,      xmmmem128,  none,       0x0101,  0x14,      r,   no,  0,     PVEX)
INSTR (VUNPCKLPD,         ymm,        yvvvv,      ymmmem256,  none,       0x0105,  0x14,      r,   no,  0,     PVEX)

INSTR (UNPCKLPS,          xmm,        xmmmem128,  none,       none,       0,       0x0f14,    r,   no,  0,     0)

INSTR (VUNPCKLPS,         xmm,        xvvvv,      xmmmem128,  none,       0x0100,  0x14,      r,   no,  0,     PVEX)
INSTR (VUNPCKLPS,         ymm,        yvvvv,      ymmmem256,  none,       0x0104,  0x14,      r,   no,  0,     PVEX)

INSTR (VBROADCASTF128,    ymm,        mem128,     none,       none,       0x027d,  0x1a,      r,   no,  0,     PVEX)

INSTR (VBROADCASTI128,    ymm,        mem128,     none,       none,       0x027d,  0x5a,      r,   no,  0,     PVEX)

INSTR (VBROADCASTSD,      ymm,        xmmmem64,   none,       none,       0x027d,  0x19,      r,   no,  0,     PVEX)

INSTR (VBROADCASTSS,      xmm,        xmmmem32,   none,       none,       0x0279,  0x18,      r,   no,  0,     PVEX)
INSTR (VBROADCASTSS,      ymm,        xmmmem32,   none,       none,       0x027d,  0x18,      r,   no,  0,     PVEX)

INSTR (VCVTPH2PS,         xmm,        xmmmem64,   none,       none,       0x0279,  0x13,      r,   no,  0,     PVEX)
INSTR (VCVTPH2PS,         ymm,        xmmmem128,  none,       none,       0x027d,  0x13,      r,   no,  0,     PVEX)

INSTR (VCVTPS2PH,         xmmmem64,   xmm,        imm8,       none,       0x0379,  0x1d,      r,   ib,  0,     PVEX)
INSTR (VCVTPS2PH,         xmmmem128,  ymm,        imm8,       none,       0x037d,  0x1d,      r,   ib,  0,     PVEX)

INSTR (VEXTRACTF128,      xmmmem128,  ymm,        imm8,       none,       0x037d,  0x19,      r,   ib,  0,     PVEX)

INSTR (VEXTRACTI128,      xmmmem128,  ymm,        imm8,       none,       0x037d,  0x39,      r,   ib,  0,     PVEX)

INSTR (VFMADDPD,          xmm,        xvvvv,      xmmmem128,  ximm,       0x0301,  0x69,      r,   is,  0,     PVEX)
INSTR (VFMADDPD,          ymm,        yvvvv,      ymmmem256,  yimm,       0x0305,  0x69,      r,   is,  0,     PVEX)
INSTR (VFMADDPD,          xmm,        xvvvv,      ximm,       xmmmem128,  0x0381,  0x69,      r,   is,  0,     PVEX)
INSTR (VFMADDPD,          ymm,        yvvvv,      yimm,       ymmmem256,  0x0385,  0x69,      r,   is,  0,     PVEX)

INSTR (VFMADD132PD,       xmm,        xvvvv,      xmmmem128,  none,       0x0281,  0x98,      r,   no,  0,     PVEX)
INSTR (VFMADD132PD,       ymm,        yvvvv,      ymmmem256,  none,       0x0285,  0x98,      r,   no,  0,     PVEX)

INSTR (VFMADD213PD,       xmm,        xvvvv,      xmmmem128,  none,       0x0281,  0xa8,      r,   no,  0,     PVEX)
INSTR (VFMADD213PD,       ymm,        yvvvv,      ymmmem256,  none,       0x0285,  0xa8,      r,   no,  0,     PVEX)

INSTR (VFMADD231PD,       xmm,        xvvvv,      xmmmem128,  none,       0x0281,  0xb8,      r,   no,  0,     PVEX)
INSTR (VFMADD231PD,       ymm,        yvvvv,      ymmmem256,  none,       0x0285,  0xb8,      r,   no,  0,     PVEX)

INSTR (VFMADDPS,          xmm,        xvvvv,      xmmmem128,  ximm,       0x0301,  0x68,      r,   is,  0,     PVEX)
INSTR (VFMADDPS,          ymm,        yvvvv,      ymmmem256,  yimm,       0x0305,  0x68,      r,   is,  0,     PVEX)
INSTR (VFMADDPS,          xmm,        xvvvv,      ximm,       xmmmem128,  0x0381,  0x68,      r,   is,  0,     PVEX)
INSTR (VFMADDPS,          ymm,        yvvvv,      yimm,       ymmmem256,  0x0385,  0x68,      r,   is,  0,     PVEX)

INSTR (VFMADD132PS,       xmm,        xvvvv,      xmmmem128,  none,       0x0201,  0x98,      r,   no,  0,     PVEX)
INSTR (VFMADD132PS,       ymm,        yvvvv,      ymmmem256,  none,       0x0205,  0x98,      r,   no,  0,     PVEX)

INSTR (VFMADD213PS,       xmm,        xvvvv,      xmmmem128,  none,       0x0201,  0xa8,      r,   no,  0,     PVEX)
INSTR (VFMADD213PS,       ymm,        yvvvv,      ymmmem256,  none,       0x0205,  0xa8,      r,   no,  0,     PVEX)

INSTR (VFMADD231PS,       xmm,        xvvvv,      xmmmem128,  none,       0x0201,  0xb8,      r,   no,  0,     PVEX)
INSTR (VFMADD231PS,       ymm,        yvvvv,      ymmmem256,  none,       0x0205,  0xb8,      r,   no,  0,     PVEX)

INSTR (VFMADDSD,          xmm,        xvvvv,      xmmmem64,   ximm,       0x0301,  0x6b,      r,   is,  0,     PVEX)
INSTR (VFMADDSD,          xmm,        xvvvv,      ximm,       xmmmem64,   0x0381,  0x6b,      r,   is,  0,     PVEX)

INSTR (VFMADD132SD,       xmm,        xvvvv,      xmmmem64,   none,       0x0281,  0x99,      r,   no,  0,     PVEX)

INSTR (VFMADD213SD,       xmm,        xvvvv,      xmmmem64,   none,       0x0281,  0xa9,      r,   no,  0,     PVEX)

INSTR (VFMADD231SD,       xmm,        xvvvv,      xmmmem64,   none,       0x0281,  0xb9,      r,   no,  0,     PVEX)

INSTR (VFMADDSS,          xmm,        xvvvv,      xmmmem32,   ximm,       0x0301,  0x6a,      r,   is,  0,     PVEX)
INSTR (VFMADDSS,          xmm,        xvvvv,      ximm,       xmmmem32,   0x0381,  0x6a,      r,   is,  0,     PVEX)

INSTR (VFMADD132SS,       xmm,        xvvvv,      xmmmem32,   none,       0x0201,  0x99,      r,   no,  0,     PVEX)

INSTR (VFMADD213SS,       xmm,        xvvvv,      xmmmem32,   none,       0x0201,  0xa9,      r,   no,  0,     PVEX)

INSTR (VFMADD231SS,       xmm,        xvvvv,      xmmmem32,   none,       0x0201,  0xb9,      r,   no,  0,     PVEX)

INSTR (VFMADDSUBPD,       xmm,        xvvvv,      xmmmem128,  ximm,       0x0301,  0x5d,      r,   is,  0,     PVEX)
INSTR (VFMADDSUBPD,       ymm,        yvvvv,      ymmmem256,  yimm,       0x0305,  0x5d,      r,   is,  0,     PVEX)
INSTR (VFMADDSUBPD,       xmm,        xvvvv,      ximm,       xmmmem128,  0x0381,  0x5d,      r,   is,  0,     PVEX)
INSTR (VFMADDSUBPD,       ymm,        yvvvv,      yimm,       ymmmem256,  0x0385,  0x5d,      r,   is,  0,     PVEX)

INSTR (VFMADDSUB132PD,    xmm,        xvvvv,      xmmmem128,  none,       0x0281,  0x96,      r,   no,  0,     PVEX)
INSTR (VFMADDSUB132PD,    ymm,        yvvvv,      ymmmem256,  none,       0x0285,  0x96,      r,   no,  0,     PVEX)

INSTR (VFMADDSUB213PD,    xmm,        xvvvv,      xmmmem128,  none,       0x0281,  0xa6,      r,   no,  0,     PVEX)
INSTR (VFMADDSUB213PD,    ymm,        yvvvv,      ymmmem256,  none,       0x0285,  0xa6,      r,   no,  0,     PVEX)

INSTR (VFMADDSUB231PD,    xmm,        xvvvv,      xmmmem128,  none,       0x0281,  0xb6,      r,   no,  0,     PVEX)
INSTR (VFMADDSUB231PD,    ymm,        yvvvv,      ymmmem256,  none,       0x0285,  0xb6,      r,   no,  0,     PVEX)

INSTR (VFMADDSUBPS,       xmm,        xvvvv,      xmmmem128,  ximm,       0x0301,  0x5c,      r,   is,  0,     PVEX)
INSTR (VFMADDSUBPS,       ymm,        yvvvv,      ymmmem256,  yimm,       0x0305,  0x5c,      r,   is,  0,     PVEX)
INSTR (VFMADDSUBPS,       xmm,        xvvvv,      ximm,       xmmmem128,  0x0381,  0x5c,      r,   is,  0,     PVEX)
INSTR (VFMADDSUBPS,       ymm,        yvvvv,      yimm,       ymmmem256,  0x0385,  0x5c,      r,   is,  0,     PVEX)

INSTR (VFMADDSUB132PS,    xmm,        xvvvv,      xmmmem128,  none,       0x0201,  0x96,      r,   no,  0,     PVEX)
INSTR (VFMADDSUB132PS,    ymm,        yvvvv,      ymmmem256,  none,       0x0205,  0x96,      r,   no,  0,     PVEX)

INSTR (VFMADDSUB213PS,    xmm,        xvvvv,      xmmmem128,  none,       0x0201,  0xa6,      r,   no,  0,     PVEX)
INSTR (VFMADDSUB213PS,    ymm,        yvvvv,      ymmmem256,  none,       0x0205,  0xa6,      r,   no,  0,     PVEX)

INSTR (VFMADDSUB231PS,    xmm,        xvvvv,      xmmmem128,  none,       0x0201,  0xb6,      r,   no,  0,     PVEX)
INSTR (VFMADDSUB231PS,    ymm,        yvvvv,      ymmmem256,  none,       0x0205,  0xb6,      r,   no,  0,     PVEX)

INSTR (VFMSUBADDPD,       xmm,        xvvvv,      xmmmem128,  ximm,       0x0301,  0x5f,      r,   is,  0,     PVEX)
INSTR (VFMSUBADDPD,       ymm,        yvvvv,      ymmmem256,  yimm,       0x0305,  0x5f,      r,   is,  0,     PVEX)
INSTR (VFMSUBADDPD,       xmm,        xvvvv,      ximm,       xmmmem128,  0x0381,  0x5f,      r,   is,  0,     PVEX)
INSTR (VFMSUBADDPD,       ymm,        yvvvv,      yimm,       ymmmem256,  0x0385,  0x5f,      r,   is,  0,     PVEX)

INSTR (VFMSUBADD132PD,    xmm,        xvvvv,      xmmmem128,  none,       0x0281,  0x97,      r,   no,  0,     PVEX)
INSTR (VFMSUBADD132PD,    ymm,        yvvvv,      ymmmem256,  none,       0x0285,  0x97,      r,   no,  0,     PVEX)

INSTR (VFMSUBADD213PD,    xmm,        xvvvv,      xmmmem128,  none,       0x0281,  0xa7,      r,   no,  0,     PVEX)
INSTR (VFMSUBADD213PD,    ymm,        yvvvv,      ymmmem256,  none,       0x0285,  0xa7,      r,   no,  0,     PVEX)

INSTR (VFMSUBADD231PD,    xmm,        xvvvv,      xmmmem128,  none,       0x0281,  0xb7,      r,   no,  0,     PVEX)
INSTR (VFMSUBADD231PD,    ymm,        yvvvv,      ymmmem256,  none,       0x0285,  0xb7,      r,   no,  0,     PVEX)

INSTR (VFMSUBADDPS,       xmm,        xvvvv,      xmmmem128,  ximm,       0x0301,  0x5e,      r,   is,  0,     PVEX)
INSTR (VFMSUBADDPS,       ymm,        yvvvv,      ymmmem256,  yimm,       0x0305,  0x5e,      r,   is,  0,     PVEX)
INSTR (VFMSUBADDPS,       xmm,        xvvvv,      ximm,       xmmmem128,  0x0381,  0x5e,      r,   is,  0,     PVEX)
INSTR (VFMSUBADDPS,       ymm,        yvvvv,      yimm,       ymmmem256,  0x0385,  0x5e,      r,   is,  0,     PVEX)

INSTR (VFMSUBADD132PS,    xmm,        xvvvv,      xmmmem128,  none,       0x0201,  0x97,      r,   no,  0,     PVEX)
INSTR (VFMSUBADD132PS,    ymm,        yvvvv,      ymmmem256,  none,       0x0205,  0x97,      r,   no,  0,     PVEX)

INSTR (VFMSUBADD213PS,    xmm,        xvvvv,      xmmmem128,  none,       0x0201,  0xa7,      r,   no,  0,     PVEX)
INSTR (VFMSUBADD213PS,    ymm,        yvvvv,      ymmmem256,  none,       0x0205,  0xa7,      r,   no,  0,     PVEX)

INSTR (VFMSUBADD231PS,    xmm,        xvvvv,      xmmmem128,  none,       0x0201,  0xb7,      r,   no,  0,     PVEX)
INSTR (VFMSUBADD231PS,    ymm,        yvvvv,      ymmmem256,  none,       0x0205,  0xb7,      r,   no,  0,     PVEX)

INSTR (VFMSUBPD,          xmm,        xvvvv,      xmmmem128,  ximm,       0x0301,  0x6d,      r,   is,  0,     PVEX)
INSTR (VFMSUBPD,          ymm,        yvvvv,      ymmmem256,  yimm,       0x0305,  0x6d,      r,   is,  0,     PVEX)
INSTR (VFMSUBPD,          xmm,        xvvvv,      ximm,       xmmmem128,  0x0381,  0x6d,      r,   is,  0,     PVEX)
INSTR (VFMSUBPD,          ymm,        yvvvv,      yimm,       ymmmem256,  0x0385,  0x6d,      r,   is,  0,     PVEX)

INSTR (VFMSUB132PD,       xmm,        xvvvv,      xmmmem128,  none,       0x0281,  0x9a,      r,   no,  0,     PVEX)
INSTR (VFMSUB132PD,       ymm,        yvvvv,      ymmmem256,  none,       0x0285,  0x9a,      r,   no,  0,     PVEX)

INSTR (VFMSUB213PD,       xmm,        xvvvv,      xmmmem128,  none,       0x0281,  0xaa,      r,   no,  0,     PVEX)
INSTR (VFMSUB213PD,       ymm,        yvvvv,      ymmmem256,  none,       0x0285,  0xaa,      r,   no,  0,     PVEX)

INSTR (VFMSUB231PD,       xmm,        xvvvv,      xmmmem128,  none,       0x0281,  0xba,      r,   no,  0,     PVEX)
INSTR (VFMSUB231PD,       ymm,        yvvvv,      ymmmem256,  none,       0x0285,  0xba,      r,   no,  0,     PVEX)

INSTR (VFMSUBPS,          xmm,        xvvvv,      xmmmem128,  ximm,       0x0301,  0x6c,      r,   is,  0,     PVEX)
INSTR (VFMSUBPS,          ymm,        yvvvv,      ymmmem256,  yimm,       0x0305,  0x6c,      r,   is,  0,     PVEX)
INSTR (VFMSUBPS,          xmm,        xvvvv,      ximm,       xmmmem128,  0x0381,  0x6c,      r,   is,  0,     PVEX)
INSTR (VFMSUBPS,          ymm,        yvvvv,      yimm,       ymmmem256,  0x0385,  0x6c,      r,   is,  0,     PVEX)

INSTR (VFMSUB132PS,       xmm,        xvvvv,      xmmmem128,  none,       0x0201,  0x9a,      r,   no,  0,     PVEX)
INSTR (VFMSUB132PS,       ymm,        yvvvv,      ymmmem256,  none,       0x0205,  0x9a,      r,   no,  0,     PVEX)

INSTR (VFMSUB213PS,       xmm,        xvvvv,      xmmmem128,  none,       0x0201,  0xaa,      r,   no,  0,     PVEX)
INSTR (VFMSUB213PS,       ymm,        yvvvv,      ymmmem256,  none,       0x0205,  0xaa,      r,   no,  0,     PVEX)

INSTR (VFMSUB231PS,       xmm,        xvvvv,      xmmmem128,  none,       0x0201,  0xba,      r,   no,  0,     PVEX)
INSTR (VFMSUB231PS,       ymm,        yvvvv,      ymmmem256,  none,       0x0205,  0xba,      r,   no,  0,     PVEX)

INSTR (VFMSUBSD,          xmm,        xvvvv,      xmmmem64,   ximm,       0x0301,  0x6f,      r,   is,  0,     PVEX)
INSTR (VFMSUBSD,          xmm,        xvvvv,      ximm,       xmmmem64,   0x0381,  0x6f,      r,   is,  0,     PVEX)

INSTR (VFMSUB132SD,       xmm,        xvvvv,      xmmmem64,   none,       0x0281,  0x9b,      r,   no,  0,     PVEX)

INSTR (VFMSUB213SD,       xmm,        xvvvv,      xmmmem64,   none,       0x0281,  0xab,      r,   no,  0,     PVEX)

INSTR (VFMSUB231SD,       xmm,        xvvvv,      xmmmem64,   none,       0x0281,  0xbb,      r,   no,  0,     PVEX)

INSTR (VFMSUBSS,          xmm,        xvvvv,      xmmmem32,   ximm,       0x0301,  0x6e,      r,   is,  0,     PVEX)
INSTR (VFMSUBSS,          xmm,        xvvvv,      ximm,       xmmmem32,   0x0381,  0x6e,      r,   is,  0,     PVEX)

INSTR (VFMSUB132SS,       xmm,        xvvvv,      xmmmem32,   none,       0x0201,  0x9b,      r,   no,  0,     PVEX)

INSTR (VFMSUB213SS,       xmm,        xvvvv,      xmmmem32,   none,       0x0201,  0xab,      r,   no,  0,     PVEX)

INSTR (VFMSUB231SS,       xmm,        xvvvv,      xmmmem32,   none,       0x0201,  0xbb,      r,   no,  0,     PVEX)

INSTR (VFNMADDPD,         xmm,        xvvvv,      xmmmem128,  ximm,       0x0301,  0x79,      r,   is,  0,     PVEX)
INSTR (VFNMADDPD,         ymm,        yvvvv,      ymmmem256,  yimm,       0x0305,  0x79,      r,   is,  0,     PVEX)
INSTR (VFNMADDPD,         xmm,        xvvvv,      ximm,       xmmmem128,  0x0381,  0x79,      r,   is,  0,     PVEX)
INSTR (VFNMADDPD,         ymm,        yvvvv,      yimm,       ymmmem256,  0x0385,  0x79,      r,   is,  0,     PVEX)

INSTR (VFNMADD132PD,      xmm,        xvvvv,      xmmmem128,  none,       0x0281,  0x9c,      r,   no,  0,     PVEX)
INSTR (VFNMADD132PD,      ymm,        yvvvv,      ymmmem256,  none,       0x0285,  0x9c,      r,   no,  0,     PVEX)

INSTR (VFNMADD213PD,      xmm,        xvvvv,      xmmmem128,  none,       0x0281,  0xac,      r,   no,  0,     PVEX)
INSTR (VFNMADD213PD,      ymm,        yvvvv,      ymmmem256,  none,       0x0285,  0xac,      r,   no,  0,     PVEX)

INSTR (VFNMADD231PD,      xmm,        xvvvv,      xmmmem128,  none,       0x0281,  0xbc,      r,   no,  0,     PVEX)
INSTR (VFNMADD231PD,      ymm,        yvvvv,      ymmmem256,  none,       0x0285,  0xbc,      r,   no,  0,     PVEX)

INSTR (VFNMADDPS,         xmm,        xvvvv,      xmmmem128,  ximm,       0x0301,  0x78,      r,   is,  0,     PVEX)
INSTR (VFNMADDPS,         ymm,        yvvvv,      ymmmem256,  yimm,       0x0305,  0x78,      r,   is,  0,     PVEX)
INSTR (VFNMADDPS,         xmm,        xvvvv,      ximm,       xmmmem128,  0x0381,  0x78,      r,   is,  0,     PVEX)
INSTR (VFNMADDPS,         ymm,        yvvvv,      yimm,       ymmmem256,  0x0385,  0x78,      r,   is,  0,     PVEX)

INSTR (VFNMADD132PS,      xmm,        xvvvv,      xmmmem128,  none,       0x0201,  0x9c,      r,   no,  0,     PVEX)
INSTR (VFNMADD132PS,      ymm,        yvvvv,      ymmmem256,  none,       0x0205,  0x9c,      r,   no,  0,     PVEX)

INSTR (VFNMADD213PS,      xmm,        xvvvv,      xmmmem128,  none,       0x0201,  0xac,      r,   no,  0,     PVEX)
INSTR (VFNMADD213PS,      ymm,        yvvvv,      ymmmem256,  none,       0x0205,  0xac,      r,   no,  0,     PVEX)

INSTR (VFNMADD231PS,      xmm,        xvvvv,      xmmmem128,  none,       0x0201,  0xbc,      r,   no,  0,     PVEX)
INSTR (VFNMADD231PS,      ymm,        yvvvv,      ymmmem256,  none,       0x0205,  0xbc,      r,   no,  0,     PVEX)

INSTR (VFNMADDSD,         xmm,        xvvvv,      xmmmem64,   ximm,       0x0301,  0x7b,      r,   is,  0,     PVEX)
INSTR (VFNMADDSD,         xmm,        xvvvv,      ximm,       xmmmem64,   0x0381,  0x7b,      r,   is,  0,     PVEX)

INSTR (VFNMADD132SD,      xmm,        xvvvv,      xmmmem64,   none,       0x0281,  0x9d,      r,   no,  0,     PVEX)

INSTR (VFNMADD213SD,      xmm,        xvvvv,      xmmmem64,   none,       0x0281,  0xad,      r,   no,  0,     PVEX)

INSTR (VFNMADD231SD,      xmm,        xvvvv,      xmmmem64,   none,       0x0281,  0xbd,      r,   no,  0,     PVEX)

INSTR (VFNMADDSS,         xmm,        xvvvv,      xmmmem32,   ximm,       0x0301,  0x7a,      r,   is,  0,     PVEX)
INSTR (VFNMADDSS,         xmm,        xvvvv,      ximm,       xmmmem32,   0x0381,  0x7a,      r,   is,  0,     PVEX)

INSTR (VFNMADD132SS,      xmm,        xvvvv,      xmmmem32,   none,       0x0201,  0x9d,      r,   no,  0,     PVEX)

INSTR (VFNMADD213SS,      xmm,        xvvvv,      xmmmem32,   none,       0x0201,  0xad,      r,   no,  0,     PVEX)

INSTR (VFNMADD231SS,      xmm,        xvvvv,      xmmmem32,   none,       0x0201,  0xbd,      r,   no,  0,     PVEX)

INSTR (VFNMSUBPD,         xmm,        xvvvv,      xmmmem128,  ximm,       0x0301,  0x7d,      r,   is,  0,     PVEX)
INSTR (VFNMSUBPD,         ymm,        yvvvv,      ymmmem256,  yimm,       0x0305,  0x7d,      r,   is,  0,     PVEX)
INSTR (VFNMSUBPD,         xmm,        xvvvv,      ximm,       xmmmem128,  0x0381,  0x7d,      r,   is,  0,     PVEX)
INSTR (VFNMSUBPD,         ymm,        yvvvv,      yimm,       ymmmem256,  0x0385,  0x7d,      r,   is,  0,     PVEX)

INSTR (VFNMSUB132PD,      xmm,        xvvvv,      xmmmem128,  none,       0x0281,  0x9e,      r,   no,  0,     PVEX)
INSTR (VFNMSUB132PD,      ymm,        yvvvv,      ymmmem256,  none,       0x0285,  0x9e,      r,   no,  0,     PVEX)

INSTR (VFNMSUB213PD,      xmm,        xvvvv,      xmmmem128,  none,       0x0281,  0xae,      r,   no,  0,     PVEX)
INSTR (VFNMSUB213PD,      ymm,        yvvvv,      ymmmem256,  none,       0x0285,  0xae,      r,   no,  0,     PVEX)

INSTR (VFNMSUB231PD,      xmm,        xvvvv,      xmmmem128,  none,       0x0281,  0xbe,      r,   no,  0,     PVEX)
INSTR (VFNMSUB231PD,      ymm,        yvvvv,      ymmmem256,  none,       0x0285,  0xbe,      r,   no,  0,     PVEX)

INSTR (VFNMSUBPS,         xmm,        xvvvv,      xmmmem128,  ximm,       0x0301,  0x7c,      r,   is,  0,     PVEX)
INSTR (VFNMSUBPS,         ymm,        yvvvv,      ymmmem256,  yimm,       0x0305,  0x7c,      r,   is,  0,     PVEX)
INSTR (VFNMSUBPS,         xmm,        xvvvv,      ximm,       xmmmem128,  0x0381,  0x7c,      r,   is,  0,     PVEX)
INSTR (VFNMSUBPS,         ymm,        yvvvv,      yimm,       ymmmem256,  0x0385,  0x7c,      r,   is,  0,     PVEX)

INSTR (VFNMSUB132PS,      xmm,        xvvvv,      xmmmem128,  none,       0x0201,  0x9e,      r,   no,  0,     PVEX)
INSTR (VFNMSUB132PS,      ymm,        yvvvv,      ymmmem256,  none,       0x0205,  0x9e,      r,   no,  0,     PVEX)

INSTR (VFNMSUB213PS,      xmm,        xvvvv,      xmmmem128,  none,       0x0201,  0xae,      r,   no,  0,     PVEX)
INSTR (VFNMSUB213PS,      ymm,        yvvvv,      ymmmem256,  none,       0x0205,  0xae,      r,   no,  0,     PVEX)

INSTR (VFNMSUB231PS,      xmm,        xvvvv,      xmmmem128,  none,       0x0201,  0xbe,      r,   no,  0,     PVEX)
INSTR (VFNMSUB231PS,      ymm,        yvvvv,      ymmmem256,  none,       0x0205,  0xbe,      r,   no,  0,     PVEX)

INSTR (VFNMSUBSD,         xmm,        xvvvv,      xmmmem64,   ximm,       0x0301,  0x7f,      r,   is,  0,     PVEX)
INSTR (VFNMSUBSD,         xmm,        xvvvv,      ximm,       xmmmem64,   0x0381,  0x7f,      r,   is,  0,     PVEX)

INSTR (VFNMSUB132SD,      xmm,        xvvvv,      xmmmem64,   none,       0x0281,  0x9f,      r,   no,  0,     PVEX)

INSTR (VFNMSUB213SD,      xmm,        xvvvv,      xmmmem64,   none,       0x0281,  0xaf,      r,   no,  0,     PVEX)

INSTR (VFNMSUB231SD,      xmm,        xvvvv,      xmmmem64,   none,       0x0281,  0xbf,      r,   no,  0,     PVEX)

INSTR (VFNMSUBSS,         xmm,        xvvvv,      xmmmem32,   ximm,       0x0301,  0x7e,      r,   is,  0,     PVEX)
INSTR (VFNMSUBSS,         xmm,        xvvvv,      ximm,       xmmmem32,   0x0381,  0x7e,      r,   is,  0,     PVEX)

INSTR (VFNMSUB132SS,      xmm,        xvvvv,      xmmmem32,   none,       0x0201,  0x9f,      r,   no,  0,     PVEX)

INSTR (VFNMSUB213SS,      xmm,        xvvvv,      xmmmem32,   none,       0x0201,  0xaf,      r,   no,  0,     PVEX)

INSTR (VFNMSUB231SS,      xmm,        xvvvv,      xmmmem32,   none,       0x0201,  0xbf,      r,   no,  0,     PVEX)

INSTR (VFRCZPD,           xmm,        xmmmem128,  none,       none,       0x0978,  0x81,      r,   no,  0,     PXOP)
INSTR (VFRCZPD,           ymm,        ymmmem256,  none,       none,       0x097c,  0x81,      r,   no,  0,     PXOP)

INSTR (VFRCZPS,           xmm,        xmmmem128,  none,       none,       0x0978,  0x80,      r,   no,  0,     PXOP)
INSTR (VFRCZPS,           ymm,        ymmmem256,  none,       none,       0x097c,  0x80,      r,   no,  0,     PXOP)

INSTR (VFRCZSD,           xmm,        xmmmem64,   none,       none,       0x0978,  0x83,      r,   no,  0,     PXOP)

INSTR (VFRCZSS,           xmm,        xmmmem32,   none,       none,       0x0978,  0x82,      r,   no,  0,     PXOP)

INSTR (VINSERTF128,       ymm,        yvvvv,      xmmmem128,  imm8,       0x0305,  0x18,      r,   ib,  0,     PVEX)

INSTR (VINSERTI128,       ymm,        yvvvv,      xmmmem128,  imm8,       0x0305,  0x38,      r,   ib,  0,     PVEX)

INSTR (VMASKMOVPD,        xmm,        xvvvv,      mem128,     none,       0x0201,  0x2d,      r,   no,  0,     PVEX)
INSTR (VMASKMOVPD,        ymm,        yvvvv,      mem256,     none,       0x0205,  0x2d,      r,   no,  0,     PVEX)
INSTR (VMASKMOVPD,        mem128,     xvvvv,      xmm,        none,       0x0201,  0x2f,      r,   no,  0,     PVEX)
INSTR (VMASKMOVPD,        mem256,     yvvvv,      ymm,        none,       0x0205,  0x2f,      r,   no,  0,     PVEX)

INSTR (VMASKMOVPS,        xmm,        xvvvv,      mem128,     none,       0x0201,  0x2c,      r,   no,  0,     PVEX)
INSTR (VMASKMOVPS,        ymm,        yvvvv,      mem256,     none,       0x0205,  0x2c,      r,   no,  0,     PVEX)
INSTR (VMASKMOVPS,        mem128,     xvvvv,      xmm,        none,       0x0201,  0x2e,      r,   no,  0,     PVEX)
INSTR (VMASKMOVPS,        mem256,     yvvvv,      ymm,        none,       0x0205,  0x2e,      r,   no,  0,     PVEX)

INSTR (VPBLENDD,          xmm,        xvvvv,      xmmmem128,  imm8,       0x0301,  0x02,      r,   ib,  0,     PVEX)
INSTR (VPBLENDD,          ymm,        yvvvv,      ymmmem256,  imm8,       0x0305,  0x02,      r,   ib,  0,     PVEX)

INSTR (VPBROADCASTB,      xmm,        xmmmem8,    none,       none,       0x0279,  0x78,      r,   no,  0,     PVEX)
INSTR (VPBROADCASTB,      ymm,        xmmmem8,    none,       none,       0x027d,  0x78,      r,   no,  0,     PVEX)

INSTR (VPBROADCASTD,      xmm,        xmmmem32,   none,       none,       0x0279,  0x58,      r,   no,  0,     PVEX)
INSTR (VPBROADCASTD,      ymm,        xmmmem32,   none,       none,       0x027d,  0x58,      r,   no,  0,     PVEX)

INSTR (VPBROADCASTQ,      xmm,        xmmmem64,   none,       none,       0x0279,  0x59,      r,   no,  0,     PVEX)
INSTR (VPBROADCASTQ,      ymm,        xmmmem64,   none,       none,       0x027d,  0x59,      r,   no,  0,     PVEX)

INSTR (VPBROADCASTW,      xmm,        xmmmem16,   none,       none,       0x0279,  0x79,      r,   no,  0,     PVEX)
INSTR (VPBROADCASTW,      ymm,        xmmmem16,   none,       none,       0x027d,  0x79,      r,   no,  0,     PVEX)

INSTR (VPCMOV,            xmm,        xvvvv,      xmmmem128,  ximm,       0x0800,  0xa2,      r,   is,  0,     PXOP)
INSTR (VPCMOV,            ymm,        yvvvv,      ymmmem256,  yimm,       0x0804,  0xa2,      r,   is,  0,     PXOP)
INSTR (VPCMOV,            xmm,        xvvvv,      ximm,       xmmmem128,  0x0880,  0xa2,      r,   is,  0,     PXOP)
INSTR (VPCMOV,            ymm,        yvvvv,      yimm,       ymmmem256,  0x0884,  0xa2,      r,   is,  0,     PXOP)

INSTR (VPCOMB,            xmm,        xvvvv,      xmmmem128,  imm8,       0x0800,  0xcc,      r,   ib,  0,     PXOP)

INSTR (VPCOMLTB,          xmm,        xvvvv,      xmmmem128,  none,       0x0800,  0xcc,      r,   no,  0x00,  PXOP | SFX)

INSTR (VPCOMLEB,          xmm,        xvvvv,      xmmmem128,  none,       0x0800,  0xcc,      r,   no,  0x01,  PXOP | SFX)

INSTR (VPCOMGTB,          xmm,        xvvvv,      xmmmem128,  none,       0x0800,  0xcc,      r,   no,  0x02,  PXOP | SFX)

INSTR (VPCOMGEB,          xmm,        xvvvv,      xmmmem128,  none,       0x0800,  0xcc,      r,   no,  0x03,  PXOP | SFX)

INSTR (VPCOMEQB,          xmm,        xvvvv,      xmmmem128,  none,       0x0800,  0xcc,      r,   no,  0x04,  PXOP | SFX)

INSTR (VPCOMNEQB,         xmm,        xvvvv,      xmmmem128,  none,       0x0800,  0xcc,      r,   no,  0x05,  PXOP | SFX)

INSTR (VPCOMFALSEB,       xmm,        xvvvv,      xmmmem128,  none,       0x0800,  0xcc,      r,   no,  0x06,  PXOP | SFX)

INSTR (VPCOMTRUEB,        xmm,        xvvvv,      xmmmem128,  none,       0x0800,  0xcc,      r,   no,  0x07,  PXOP | SFX)

INSTR (VPCOMD,            xmm,        xvvvv,      xmmmem128,  imm8,       0x0800,  0xce,      r,   ib,  0,     PXOP)

INSTR (VPCOMLTD,          xmm,        xvvvv,      xmmmem128,  none,       0x0800,  0xce,      r,   no,  0x00,  PXOP | SFX)

INSTR (VPCOMLED,          xmm,        xvvvv,      xmmmem128,  none,       0x0800,  0xce,      r,   no,  0x01,  PXOP | SFX)

INSTR (VPCOMGTD,          xmm,        xvvvv,      xmmmem128,  none,       0x0800,  0xce,      r,   no,  0x02,  PXOP | SFX)

INSTR (VPCOMGED,          xmm,        xvvvv,      xmmmem128,  none,       0x0800,  0xce,      r,   no,  0x03,  PXOP | SFX)

INSTR (VPCOMEQD,          xmm,        xvvvv,      xmmmem128,  none,       0x0800,  0xce,      r,   no,  0x04,  PXOP | SFX)

INSTR (VPCOMNEQD,         xmm,        xvvvv,      xmmmem128,  none,       0x0800,  0xce,      r,   no,  0x05,  PXOP | SFX)

INSTR (VPCOMFALSED,       xmm,        xvvvv,      xmmmem128,  none,       0x0800,  0xce,      r,   no,  0x06,  PXOP | SFX)

INSTR (VPCOMTRUED,        xmm,        xvvvv,      xmmmem128,  none,       0x0800,  0xce,      r,   no,  0x07,  PXOP | SFX)

INSTR (VPCOMQ,            xmm,        xvvvv,      xmmmem128,  imm8,       0x0800,  0xcf,      r,   ib,  0,     PXOP)

INSTR (VPCOMLTQ,          xmm,        xvvvv,      xmmmem128,  none,       0x0800,  0xcf,      r,   no,  0x00,  PXOP | SFX)

INSTR (VPCOMLEQ,          xmm,        xvvvv,      xmmmem128,  none,       0x0800,  0xcf,      r,   no,  0x01,  PXOP | SFX)

INSTR (VPCOMGTQ,          xmm,        xvvvv,      xmmmem128,  none,       0x0800,  0xcf,      r,   no,  0x02,  PXOP | SFX)

INSTR (VPCOMGEQ,          xmm,        xvvvv,      xmmmem128,  none,       0x0800,  0xcf,      r,   no,  0x03,  PXOP | SFX)

INSTR (VPCOMEQQ,          xmm,        xvvvv,      xmmmem128,  none,       0x0800,  0xcf,      r,   no,  0x04,  PXOP | SFX)

INSTR (VPCOMNEQQ,         xmm,        xvvvv,      xmmmem128,  none,       0x0800,  0xcf,      r,   no,  0x05,  PXOP | SFX)

INSTR (VPCOMFALSEQ,       xmm,        xvvvv,      xmmmem128,  none,       0x0800,  0xcf,      r,   no,  0x06,  PXOP | SFX)

INSTR (VPCOMTRUEQ,        xmm,        xvvvv,      xmmmem128,  none,       0x0800,  0xcf,      r,   no,  0x07,  PXOP | SFX)

INSTR (VPCOMUB,           xmm,        xvvvv,      xmmmem128,  imm8,       0x0800,  0xec,      r,   ib,  0,     PXOP)

INSTR (VPCOMLTUB,         xmm,        xvvvv,      xmmmem128,  none,       0x0800,  0xec,      r,   no,  0x00,  PXOP | SFX)

INSTR (VPCOMLEUB,         xmm,        xvvvv,      xmmmem128,  none,       0x0800,  0xec,      r,   no,  0x01,  PXOP | SFX)

INSTR (VPCOMGTUB,         xmm,        xvvvv,      xmmmem128,  none,       0x0800,  0xec,      r,   no,  0x02,  PXOP | SFX)

INSTR (VPCOMGEUB,         xmm,        xvvvv,      xmmmem128,  none,       0x0800,  0xec,      r,   no,  0x03,  PXOP | SFX)

INSTR (VPCOMEQUB,         xmm,        xvvvv,      xmmmem128,  none,       0x0800,  0xec,      r,   no,  0x04,  PXOP | SFX)

INSTR (VPCOMNEQUB,        xmm,        xvvvv,      xmmmem128,  none,       0x0800,  0xec,      r,   no,  0x05,  PXOP | SFX)

INSTR (VPCOMFALSEUB,      xmm,        xvvvv,      xmmmem128,  none,       0x0800,  0xec,      r,   no,  0x06,  PXOP | SFX)

INSTR (VPCOMTRUEUB,       xmm,        xvvvv,      xmmmem128,  none,       0x0800,  0xec,      r,   no,  0x07,  PXOP | SFX)

INSTR (VPCOMUD,           xmm,        xvvvv,      xmmmem128,  imm8,       0x0800,  0xee,      r,   ib,  0,     PXOP)

INSTR (VPCOMLTUD,         xmm,        xvvvv,      xmmmem128,  none,       0x0800,  0xee,      r,   no,  0x00,  PXOP | SFX)

INSTR (VPCOMLEUD,         xmm,        xvvvv,      xmmmem128,  none,       0x0800,  0xee,      r,   no,  0x01,  PXOP | SFX)

INSTR (VPCOMGTUD,         xmm,        xvvvv,      xmmmem128,  none,       0x0800,  0xee,      r,   no,  0x02,  PXOP | SFX)

INSTR (VPCOMGEUD,         xmm,        xvvvv,      xmmmem128,  none,       0x0800,  0xee,      r,   no,  0x03,  PXOP | SFX)

INSTR (VPCOMEQUD,         xmm,        xvvvv,      xmmmem128,  none,       0x0800,  0xee,      r,   no,  0x04,  PXOP | SFX)

INSTR (VPCOMNEQUD,        xmm,        xvvvv,      xmmmem128,  none,       0x0800,  0xee,      r,   no,  0x05,  PXOP | SFX)

INSTR (VPCOMFALSEUD,      xmm,        xvvvv,      xmmmem128,  none,       0x0800,  0xee,      r,   no,  0x06,  PXOP | SFX)

INSTR (VPCOMTRUEUD,       xmm,        xvvvv,      xmmmem128,  none,       0x0800,  0xee,      r,   no,  0x07,  PXOP | SFX)

INSTR (VPCOMUQ,           xmm,        xvvvv,      xmmmem128,  imm8,       0x0800,  0xef,      r,   ib,  0,     PXOP)

INSTR (VPCOMLTUQ,         xmm,        xvvvv,      xmmmem128,  none,       0x0800,  0xef,      r,   no,  0x00,  PXOP | SFX)

INSTR (VPCOMLEUQ,         xmm,        xvvvv,      xmmmem128,  none,       0x0800,  0xef,      r,   no,  0x01,  PXOP | SFX)

INSTR (VPCOMGTUQ,         xmm,        xvvvv,      xmmmem128,  none,       0x0800,  0xef,      r,   no,  0x02,  PXOP | SFX)

INSTR (VPCOMGEUQ,         xmm,        xvvvv,      xmmmem128,  none,       0x0800,  0xef,      r,   no,  0x03,  PXOP | SFX)

INSTR (VPCOMEQUQ,         xmm,        xvvvv,      xmmmem128,  none,       0x0800,  0xef,      r,   no,  0x04,  PXOP | SFX)

INSTR (VPCOMNEQUQ,        xmm,        xvvvv,      xmmmem128,  none,       0x0800,  0xef,      r,   no,  0x05,  PXOP | SFX)

INSTR (VPCOMFALSEUQ,      xmm,        xvvvv,      xmmmem128,  none,       0x0800,  0xef,      r,   no,  0x06,  PXOP | SFX)

INSTR (VPCOMTRUEUQ,       xmm,        xvvvv,      xmmmem128,  none,       0x0800,  0xef,      r,   no,  0x07,  PXOP | SFX)

INSTR (VPCOMUW,           xmm,        xvvvv,      xmmmem128,  imm8,       0x0800,  0xed,      r,   ib,  0,     PXOP)

INSTR (VPCOMLTUW,         xmm,        xvvvv,      xmmmem128,  none,       0x0800,  0xed,      r,   no,  0x00,  PXOP | SFX)

INSTR (VPCOMLEUW,         xmm,        xvvvv,      xmmmem128,  none,       0x0800,  0xed,      r,   no,  0x01,  PXOP | SFX)

INSTR (VPCOMGTUW,         xmm,        xvvvv,      xmmmem128,  none,       0x0800,  0xed,      r,   no,  0x02,  PXOP | SFX)

INSTR (VPCOMGEUW,         xmm,        xvvvv,      xmmmem128,  none,       0x0800,  0xed,      r,   no,  0x03,  PXOP | SFX)

INSTR (VPCOMEQUW,         xmm,        xvvvv,      xmmmem128,  none,       0x0800,  0xed,      r,   no,  0x04,  PXOP | SFX)

INSTR (VPCOMNEQUW,        xmm,        xvvvv,      xmmmem128,  none,       0x0800,  0xed,      r,   no,  0x05,  PXOP | SFX)

INSTR (VPCOMFALSEUW,      xmm,        xvvvv,      xmmmem128,  none,       0x0800,  0xed,      r,   no,  0x06,  PXOP | SFX)

INSTR (VPCOMTRUEUW,       xmm,        xvvvv,      xmmmem128,  none,       0x0800,  0xed,      r,   no,  0x07,  PXOP | SFX)

INSTR (VPCOMW,            xmm,        xvvvv,      xmmmem128,  imm8,       0x0800,  0xcd,      r,   ib,  0,     PXOP)

INSTR (VPCOMLTW,          xmm,        xvvvv,      xmmmem128,  none,       0x0800,  0xcd,      r,   no,  0x00,  PXOP | SFX)

INSTR (VPCOMLEW,          xmm,        xvvvv,      xmmmem128,  none,       0x0800,  0xcd,      r,   no,  0x01,  PXOP | SFX)

INSTR (VPCOMGTW,          xmm,        xvvvv,      xmmmem128,  none,       0x0800,  0xcd,      r,   no,  0x02,  PXOP | SFX)

INSTR (VPCOMGEW,          xmm,        xvvvv,      xmmmem128,  none,       0x0800,  0xcd,      r,   no,  0x03,  PXOP | SFX)

INSTR (VPCOMEQW,          xmm,        xvvvv,      xmmmem128,  none,       0x0800,  0xcd,      r,   no,  0x04,  PXOP | SFX)

INSTR (VPCOMNEQW,         xmm,        xvvvv,      xmmmem128,  none,       0x0800,  0xcd,      r,   no,  0x05,  PXOP | SFX)

INSTR (VPCOMFALSEW,       xmm,        xvvvv,      xmmmem128,  none,       0x0800,  0xcd,      r,   no,  0x06,  PXOP | SFX)

INSTR (VPCOMTRUEW,        xmm,        xvvvv,      xmmmem128,  none,       0x0800,  0xcd,      r,   no,  0x07,  PXOP | SFX)

INSTR (VPERM2F128,        ymm,        yvvvv,      ymmmem256,  imm8,       0x0305,  0x06,      r,   ib,  0,     PVEX)

INSTR (VPERM2I128,        ymm,        yvvvv,      ymmmem256,  imm8,       0x0305,  0x46,      r,   ib,  0,     PVEX)

INSTR (VPERMD,            ymm,        yvvvv,      ymmmem256,  none,       0x0205,  0x36,      r,   no,  0,     PVEX)

INSTR (VPERMPD,           ymm,        ymmmem256,  imm8,       none,       0x03fd,  0x01,      r,   ib,  0,     PVEX)

INSTR (VPERMPS,           ymm,        yvvvv,      ymmmem256,  none,       0x0205,  0x16,      r,   no,  0,     PVEX)

INSTR (VPERMQ,            ymm,        ymmmem256,  imm8,       none,       0x03fd,  0x00,      r,   ib,  0,     PVEX)

INSTR (VPHADDBD,          xmm,        xmmmem128,  none,       none,       0x0978,  0xc2,      r,   no,  0,     PXOP)

INSTR (VPHADDBQ,          xmm,        xmmmem128,  none,       none,       0x0978,  0xc3,      r,   no,  0,     PXOP)

INSTR (VPHADDBW,          xmm,        xmmmem128,  none,       none,       0x0978,  0xc1,      r,   no,  0,     PXOP)

INSTR (VPHADDDQ,          xmm,        xmmmem128,  none,       none,       0x0978,  0xcb,      r,   no,  0,     PXOP)

INSTR (VPHADDUBD,         xmm,        xmmmem128,  none,       none,       0x0978,  0xd2,      r,   no,  0,     PXOP)

INSTR (VPHADDUBQ,         xmm,        xmmmem128,  none,       none,       0x0978,  0xd3,      r,   no,  0,     PXOP)

INSTR (VPHADDUBW,         xmm,        xmmmem128,  none,       none,       0x0978,  0xd1,      r,   no,  0,     PXOP)

INSTR (VPHADDUDQ,         xmm,        xmmmem128,  none,       none,       0x0978,  0xdb,      r,   no,  0,     PXOP)

INSTR (VPHADDUWD,         xmm,        xmmmem128,  none,       none,       0x0978,  0xd6,      r,   no,  0,     PXOP)

INSTR (VPHADDUWQ,         xmm,        xmmmem128,  none,       none,       0x0978,  0xd7,      r,   no,  0,     PXOP)

INSTR (VPHADDWD,          xmm,        xmmmem128,  none,       none,       0x0978,  0xc6,      r,   no,  0,     PXOP)

INSTR (VPHADDWQ,          xmm,        xmmmem128,  none,       none,       0x0978,  0xc7,      r,   no,  0,     PXOP)

INSTR (VPHSUBBW,          xmm,        xmmmem128,  none,       none,       0x0978,  0xe1,      r,   no,  0,     PXOP)

INSTR (VPHSUBDQ,          xmm,        xmmmem128,  none,       none,       0x0978,  0xe3,      r,   no,  0,     PXOP)

INSTR (VPHSUBWD,          xmm,        xmmmem128,  none,       none,       0x0978,  0xe2,      r,   no,  0,     PXOP)

INSTR (VPMACSDD,          xmm,        xvvvv,      xmmmem128,  ximm,       0x0800,  0x9e,      r,   is,  0,     PXOP)

INSTR (VPMACSDQH,         xmm,        xvvvv,      xmmmem128,  ximm,       0x0800,  0x9f,      r,   is,  0,     PXOP)

INSTR (VPMACSDQL,         xmm,        xvvvv,      xmmmem128,  ximm,       0x0800,  0x97,      r,   is,  0,     PXOP)

INSTR (VPMACSSDD,         xmm,        xvvvv,      xmmmem128,  ximm,       0x0800,  0x8e,      r,   is,  0,     PXOP)

INSTR (VPMACSSDQH,        xmm,        xvvvv,      xmmmem128,  ximm,       0x0800,  0x8f,      r,   is,  0,     PXOP)

INSTR (VPMACSSDQL,        xmm,        xvvvv,      xmmmem128,  ximm,       0x0800,  0x87,      r,   is,  0,     PXOP)

INSTR (VPMACSSWD,         xmm,        xvvvv,      xmmmem128,  ximm,       0x0800,  0x86,      r,   is,  0,     PXOP)

INSTR (VPMACSSWW,         xmm,        xvvvv,      xmmmem128,  ximm,       0x0800,  0x85,      r,   is,  0,     PXOP)

INSTR (VPMACSWD,          xmm,        xvvvv,      xmmmem128,  ximm,       0x0800,  0x96,      r,   is,  0,     PXOP)

INSTR (VPMACSWW,          xmm,        xvvvv,      xmmmem128,  ximm,       0x0800,  0x95,      r,   is,  0,     PXOP)

INSTR (VPMADCSSWD,        xmm,        xvvvv,      xmmmem128,  ximm,       0x0800,  0xa6,      r,   is,  0,     PXOP)

INSTR (VPMADCSWD,         xmm,        xvvvv,      xmmmem128,  ximm,       0x0800,  0xb6,      r,   is,  0,     PXOP)

INSTR (VPMASKMOVD,        xmm,        xvvvv,      mem128,     none,       0x0201,  0x8c,      r,   no,  0,     PVEX)
INSTR (VPMASKMOVD,        ymm,        yvvvv,      mem256,     none,       0x0205,  0x8c,      r,   no,  0,     PVEX)
INSTR (VPMASKMOVD,        mem128,     xvvvv,      xmm,        none,       0x0201,  0x8e,      r,   no,  0,     PVEX)
INSTR (VPMASKMOVD,        mem256,     yvvvv,      ymm,        none,       0x0205,  0x8e,      r,   no,  0,     PVEX)

INSTR (VPMASKMOVQ,        xmm,        xvvvv,      mem128,     none,       0x0281,  0x8c,      r,   no,  0,     PVEX)
INSTR (VPMASKMOVQ,        ymm,        yvvvv,      mem256,     none,       0x0285,  0x8c,      r,   no,  0,     PVEX)
INSTR (VPMASKMOVQ,        mem128,     xvvvv,      xmm,        none,       0x0281,  0x8e,      r,   no,  0,     PVEX)
INSTR (VPMASKMOVQ,        mem256,     yvvvv,      ymm,        none,       0x0285,  0x8e,      r,   no,  0,     PVEX)

INSTR (VPPERM,            xmm,        xvvvv,      ximm,       xmmmem128,  0x0880,  0xa3,      r,   is,  0,     PXOP)
INSTR (VPPERM,            xmm,        xvvvv,      xmmmem128,  ximm,       0x0800,  0xa3,      r,   is,  0,     PXOP)

INSTR (VPROTB,            xmm,        xmmmem128,  xvvvv,      none,       0x0900,  0x90,      r,   no,  0,     PXOP)
INSTR (VPROTB,            xmm,        xvvvv,      xmmmem128,  none,       0x0980,  0x90,      r,   no,  0,     PXOP)
INSTR (VPROTB,            xmm,        xmmmem128,  imm8,       none,       0x0878,  0xc0,      r,   ib,  0,     PXOP)

INSTR (VPROTD,            xmm,        xmmmem128,  xvvvv,      none,       0x0900,  0x92,      r,   no,  0,     PXOP)
INSTR (VPROTD,            xmm,        xvvvv,      xmmmem128,  none,       0x0980,  0x92,      r,   no,  0,     PXOP)
INSTR (VPROTD,            xmm,        xmmmem128,  imm8,       none,       0x0878,  0xc2,      r,   ib,  0,     PXOP)

INSTR (VPROTQ,            xmm,        xmmmem128,  xvvvv,      none,       0x0900,  0x93,      r,   no,  0,     PXOP)
INSTR (VPROTQ,            xmm,        xvvvv,      xmmmem128,  none,       0x0980,  0x93,      r,   no,  0,     PXOP)
INSTR (VPROTQ,            xmm,        xmmmem128,  imm8,       none,       0x0878,  0xc3,      r,   ib,  0,     PXOP)

INSTR (VPROTW,            xmm,        xmmmem128,  xvvvv,      none,       0x0900,  0x91,      r,   no,  0,     PXOP)
INSTR (VPROTW,            xmm,        xvvvv,      xmmmem128,  none,       0x0980,  0x91,      r,   no,  0,     PXOP)
INSTR (VPROTW,            xmm,        xmmmem128,  imm8,       none,       0x0878,  0xc1,      r,   ib,  0,     PXOP)

INSTR (VPSHAB,            xmm,        xmmmem128,  xvvvv,      none,       0x0900,  0x98,      r,   no,  0,     PXOP)
INSTR (VPSHAB,            xmm,        xvvvv,      xmmmem128,  none,       0x0980,  0x98,      r,   no,  0,     PXOP)

INSTR (VPSHAD,            xmm,        xmmmem128,  xvvvv,      none,       0x0900,  0x9a,      r,   no,  0,     PXOP)
INSTR (VPSHAD,            xmm,        xvvvv,      xmmmem128,  none,       0x0980,  0x9a,      r,   no,  0,     PXOP)

INSTR (VPSHAQ,            xmm,        xmmmem128,  xvvvv,      none,       0x0900,  0x9b,      r,   no,  0,     PXOP)
INSTR (VPSHAQ,            xmm,        xvvvv,      xmmmem128,  none,       0x0980,  0x9b,      r,   no,  0,     PXOP)

INSTR (VPSHAW,            xmm,        xmmmem128,  xvvvv,      none,       0x0900,  0x99,      r,   no,  0,     PXOP)
INSTR (VPSHAW,            xmm,        xvvvv,      xmmmem128,  none,       0x0980,  0x99,      r,   no,  0,     PXOP)

INSTR (VPSHLB,            xmm,        xmmmem128,  xvvvv,      none,       0x0900,  0x94,      r,   no,  0,     PXOP)
INSTR (VPSHLB,            xmm,        xvvvv,      xmmmem128,  none,       0x0980,  0x94,      r,   no,  0,     PXOP)

INSTR (VPSHLD,            xmm,        xmmmem128,  xvvvv,      none,       0x0900,  0x96,      r,   no,  0,     PXOP)
INSTR (VPSHLD,            xmm,        xvvvv,      xmmmem128,  none,       0x0980,  0x96,      r,   no,  0,     PXOP)

INSTR (VPSHLQ,            xmm,        xmmmem128,  xvvvv,      none,       0x0900,  0x97,      r,   no,  0,     PXOP)
INSTR (VPSHLQ,            xmm,        xvvvv,      xmmmem128,  none,       0x0980,  0x97,      r,   no,  0,     PXOP)

INSTR (VPSHLW,            xmm,        xmmmem128,  xvvvv,      none,       0x0900,  0x95,      r,   no,  0,     PXOP)
INSTR (VPSHLW,            xmm,        xvvvv,      xmmmem128,  none,       0x0980,  0x95,      r,   no,  0,     PXOP)

INSTR (VPSLLVD,           xmm,        xvvvv,      xmmmem128,  none,       0x0201,  0x47,      r,   no,  0,     PVEX)
INSTR (VPSLLVD,           ymm,        yvvvv,      ymmmem256,  none,       0x0205,  0x47,      r,   no,  0,     PVEX)

INSTR (VPSLLVQ,           xmm,        xvvvv,      xmmmem128,  none,       0x0281,  0x47,      r,   no,  0,     PVEX)
INSTR (VPSLLVQ,           ymm,        yvvvv,      ymmmem256,  none,       0x0285,  0x47,      r,   no,  0,     PVEX)

INSTR (VPSRAVD,           xmm,        xvvvv,      xmmmem128,  none,       0x0201,  0x46,      r,   no,  0,     PVEX)
INSTR (VPSRAVD,           ymm,        yvvvv,      ymmmem256,  none,       0x0205,  0x46,      r,   no,  0,     PVEX)

INSTR (VPSRLVD,           xmm,        xvvvv,      xmmmem128,  none,       0x0201,  0x45,      r,   no,  0,     PVEX)
INSTR (VPSRLVD,           ymm,        yvvvv,      ymmmem256,  none,       0x0205,  0x45,      r,   no,  0,     PVEX)

INSTR (VPSRLVQ,           xmm,        xvvvv,      xmmmem128,  none,       0x0281,  0x45,      r,   no,  0,     PVEX)
INSTR (VPSRLVQ,           ymm,        yvvvv,      ymmmem256,  none,       0x0285,  0x45,      r,   no,  0,     PVEX)

INSTR (VTESTPD,           xmm,        xmmmem128,  none,       none,       0x0279,  0x0f,      r,   no,  0,     PVEX)
INSTR (VTESTPD,           ymm,        ymmmem256,  none,       none,       0x027d,  0x0f,      r,   no,  0,     PVEX)

INSTR (VTESTPS,           xmm,        xmmmem128,  none,       none,       0x0279,  0x0e,      r,   no,  0,     PVEX)
INSTR (VTESTPS,           ymm,        ymmmem256,  none,       none,       0x027d,  0x0e,      r,   no,  0,     PVEX)

INSTR (VZEROALL,          none,       none,       none,       none,       0x017c,  0x77,      no,  no,  0,     PVEX)

INSTR (VZEROUPPER,        none,       none,       none,       none,       0x0178,  0x77,      no,  no,  0,     PVEX)

INSTR (XORPD,             xmm,        xmmmem128,  none,       none,       0,       0x0f57,    r,   no,  0,     P66)

INSTR (VXORPD,            xmm,        xvvvv,      xmmmem128,  none,       0x0101,  0x57,      r,   no,  0,     PVEX)
INSTR (VXORPD,            ymm,        yvvvv,      ymmmem256,  none,       0x0105,  0x57,      r,   no,  0,     PVEX)

INSTR (XORPS,             xmm,        xmmmem128,  none,       none,       0,       0x0f57,    r,   no,  0,     0)

INSTR (VXORPS,            xmm,        xvvvv,      xmmmem128,  none,       0x0100,  0x57,      r,   no,  0,     PVEX)
INSTR (VXORPS,            ymm,        yvvvv,      ymmmem256,  none,       0x0104,  0x57,      r,   no,  0,     PVEX)

INSTR (XGETBV,            none,       none,       none,       none,       0,       0x0f01,    no,  no,  0xd0,  SFX)

INSTR (XSETBV,            none,       none,       none,       none,       0,       0x0f01,    no,  no,  0xd1,  SFX)

// 64-Bit Media Instruction Reference

INSTR (EMMS,              none,       none,       none,       none,       0,       0x0f77,    no,  no,  0,     0)

INSTR (FEMMS,             none,       none,       none,       none,       0,       0x0f0e,    no,  no,  0,     0)

INSTR (MASKMOVQ,          mmx,        mmxmem64,   none,       none,       0,       0x0ff7,    r,   no,  0,     0)

INSTR (MOVNTQ,            mem64,      mmx,        none,       none,       0,       0x0fe7,    r,   no,  0,     0)

INSTR (PAVGUSB,           mmx,        mmxmem64,   none,       none,       0,       0x0f0f,    r,   no,  0xbf,  SFX)

INSTR (PF2ID,             mmx,        mmxmem64,   none,       none,       0,       0x0f0f,    r,   no,  0x1d,  SFX)

INSTR (PF2IW,             mmx,        mmxmem64,   none,       none,       0,       0x0f0f,    r,   no,  0x1c,  SFX)

INSTR (PFACC,             mmx,        mmxmem64,   none,       none,       0,       0x0f0f,    r,   no,  0xae,  SFX)

INSTR (PFADD,             mmx,        mmxmem64,   none,       none,       0,       0x0f0f,    r,   no,  0x9e,  SFX)

INSTR (PFCMPEQ,           mmx,        mmxmem64,   none,       none,       0,       0x0f0f,    r,   no,  0xb0,  SFX)

INSTR (PFCMPGE,           mmx,        mmxmem64,   none,       none,       0,       0x0f0f,    r,   no,  0x90,  SFX)

INSTR (PFCMPGT,           mmx,        mmxmem64,   none,       none,       0,       0x0f0f,    r,   no,  0xa0,  SFX)

INSTR (PFMAX,             mmx,        mmxmem64,   none,       none,       0,       0x0f0f,    r,   no,  0xa4,  SFX)

INSTR (PFMIN,             mmx,        mmxmem64,   none,       none,       0,       0x0f0f,    r,   no,  0x94,  SFX)

INSTR (PFMUL,             mmx,        mmxmem64,   none,       none,       0,       0x0f0f,    r,   no,  0xb4,  SFX)

INSTR (PFNACC,            mmx,        mmxmem64,   none,       none,       0,       0x0f0f,    r,   no,  0x8a,  SFX)

INSTR (PFPNACC,           mmx,        mmxmem64,   none,       none,       0,       0x0f0f,    r,   no,  0x8e,  SFX)

INSTR (PFRCP,             mmx,        mmxmem64,   none,       none,       0,       0x0f0f,    r,   no,  0x96,  SFX)

INSTR (PFRCPIT1,          mmx,        mmxmem64,   none,       none,       0,       0x0f0f,    r,   no,  0xa6,  SFX)

INSTR (PFRCPIT2,          mmx,        mmxmem64,   none,       none,       0,       0x0f0f,    r,   no,  0xb6,  SFX)

INSTR (PFRSQIT1,          mmx,        mmxmem64,   none,       none,       0,       0x0f0f,    r,   no,  0xa7,  SFX)

INSTR (PFRSQRT,           mmx,        mmxmem64,   none,       none,       0,       0x0f0f,    r,   no,  0x97,  SFX)

INSTR (PFSUB,             mmx,        mmxmem64,   none,       none,       0,       0x0f0f,    r,   no,  0x9a,  SFX)

INSTR (PFSUBR,            mmx,        mmxmem64,   none,       none,       0,       0x0f0f,    r,   no,  0xaa,  SFX)

INSTR (PI2FD,             mmx,        mmxmem64,   none,       none,       0,       0x0f0f,    r,   no,  0x0d,  SFX)

INSTR (PI2FW,             mmx,        mmxmem64,   none,       none,       0,       0x0f0f,    r,   no,  0x0c,  SFX)

INSTR (PMULHRW,           mmx,        mmxmem64,   none,       none,       0,       0x0f0f,    r,   no,  0xb7,  SFX)

INSTR (PSHUFW,            mmx,        mmxmem64,   imm8,       none,       0,       0x0f70,    r,   ib,  0,     0)

INSTR (PSWAPD,            mmx,        mmxmem64,   none,       none,       0,       0x0f0f,    r,   no,  0xbb,  SFX)

// x87 Floating-Point Instruction Reference

INSTR (F2XM1,             none,       none,       none,       none,       0,       0xd9f0,    no,  no,  0,     0)

INSTR (FABS,              none,       none,       none,       none,       0,       0xd9e1,    no,  no,  0,     0)

INSTR (FADD,              st0,        sti,        none,       none,       0,       0xd8c0,    rv,  no,  0,     0)
INSTR (FADD,              sti,        st0,        none,       none,       0,       0xdcc0,    rv,  no,  0,     0)
INSTR (FADD,              mem32,      none,       none,       none,       0,       0xd8,      r0,  no,  0,     0)
INSTR (FADD,              mem64,      none,       none,       none,       0,       0xdc,      r0,  no,  0,     0)

INSTR (FADDP,             none,       none,       none,       none,       0,       0xdec1,    no,  no,  0,     0)
INSTR (FADDP,             sti,        st0,        none,       none,       0,       0xdec0,    rv,  no,  0,     0)

INSTR (FIADD,             mem16,      none,       none,       none,       0,       0xde,      r0,  no,  0,     0)
INSTR (FIADD,             mem32,      none,       none,       none,       0,       0xda,      r0,  no,  0,     0)

INSTR (FBLD,              mem,        none,       none,       none,       0,       0xdf,      r4,  no,  0,     0)

INSTR (FBSTP,             mem,        none,       none,       none,       0,       0xdf,      r6,  no,  0,     0)

INSTR (FCHS,              none,       none,       none,       none,       0,       0xd9e0,    no,  no,  0,     0)

INSTR (FNCLEX,            none,       none,       none,       none,       0,       0xdbe2,    no,  no,  0,     0)

INSTR (FCMOVB,            st0,        sti,        none,       none,       0,       0xdac0,    rv,  no,  0,     0)

INSTR (FCMOVBE,           st0,        sti,        none,       none,       0,       0xdad0,    rv,  no,  0,     0)

INSTR (FCMOVE,            st0,        sti,        none,       none,       0,       0xdac8,    rv,  no,  0,     0)

INSTR (FCMOVNB,           st0,        sti,        none,       none,       0,       0xdbc0,    rv,  no,  0,     0)

INSTR (FCMOVNBE,          st0,        sti,        none,       none,       0,       0xdbd0,    rv,  no,  0,     0)

INSTR (FCMOVNE,           st0,        sti,        none,       none,       0,       0xdbc8,    rv,  no,  0,     0)

INSTR (FCMOVNU,           st0,        sti,        none,       none,       0,       0xdbd8,    rv,  no,  0,     0)

INSTR (FCMOVU,            st0,        sti,        none,       none,       0,       0xdad8,    rv,  no,  0,     0)

INSTR (FCOM,              none,       none,       none,       none,       0,       0xd8d1,    no,  no,  0,     0)
INSTR (FCOM,              sti,        none,       none,       none,       0,       0xd8d0,    rv,  no,  0,     0)
INSTR (FCOM,              mem32,      none,       none,       none,       0,       0xd8,      r2,  no,  0,     0)
INSTR (FCOM,              mem64,      none,       none,       none,       0,       0xdc,      r2,  no,  0,     0)

INSTR (FCOMP,             none,       none,       none,       none,       0,       0xd8d9,    no,  no,  0,     0)
INSTR (FCOMP,             sti,        none,       none,       none,       0,       0xd8d8,    rv,  no,  0,     0)
INSTR (FCOMP,             mem32,      none,       none,       none,       0,       0xd8,      r3,  no,  0,     0)
INSTR (FCOMP,             mem64,      none,       none,       none,       0,       0xdc,      r3,  no,  0,     0)

INSTR (FCOMPP,            none,       none,       none,       none,       0,       0xded9,    no,  no,  0,     0)

INSTR (FCOMI,             st0,        sti,        none,       none,       0,       0xdbf0,    rv,  no,  0,     0)

INSTR (FCOMIP,            st0,        sti,        none,       none,       0,       0xdff0,    rv,  no,  0,     0)

INSTR (FCOS,              none,       none,       none,       none,       0,       0xd9ff,    no,  no,  0,     0)

INSTR (FDECSTP,           none,       none,       none,       none,       0,       0xd9f6,    no,  no,  0,     0)

INSTR (FDIV,              st0,        sti,        none,       none,       0,       0xd8f0,    rv,  no,  0,     0)
INSTR (FDIV,              sti,        st0,        none,       none,       0,       0xdcf8,    rv,  no,  0,     0)
INSTR (FDIV,              mem32,      none,       none,       none,       0,       0xd8,      r6,  no,  0,     0)
INSTR (FDIV,              mem64,      none,       none,       none,       0,       0xdc,      r6,  no,  0,     0)

INSTR (FDIVP,             none,       none,       none,       none,       0,       0xdef9,    no,  no,  0,     0)
INSTR (FDIVP,             sti,        st0,        none,       none,       0,       0xdef8,    rv,  no,  0,     0)

INSTR (FIDIV,             mem16,      none,       none,       none,       0,       0xde,      r6,  no,  0,     0)
INSTR (FIDIV,             mem32,      none,       none,       none,       0,       0xda,      r6,  no,  0,     0)

INSTR (FDIVR,             st0,        sti,        none,       none,       0,       0xd8f8,    rv,  no,  0,     0)
INSTR (FDIVR,             sti,        st0,        none,       none,       0,       0xdcf0,    rv,  no,  0,     0)
INSTR (FDIVR,             mem32,      none,       none,       none,       0,       0xd8,      r7,  no,  0,     0)
INSTR (FDIVR,             mem64,      none,       none,       none,       0,       0xdc,      r7,  no,  0,     0)

INSTR (FDIVRP,            none,       none,       none,       none,       0,       0xdef1,    no,  no,  0,     0)
INSTR (FDIVRP,            sti,        st0,        none,       none,       0,       0xdef0,    rv,  no,  0,     0)

INSTR (FIDIVR,            mem16,      none,       none,       none,       0,       0xde,      r7,  no,  0,     0)
INSTR (FIDIVR,            mem32,      none,       none,       none,       0,       0xda,      r7,  no,  0,     0)

INSTR (FFREE,             sti,        none,       none,       none,       0,       0xddc0,    rv,  no,  0,     0)

INSTR (FICOM,             mem16,      none,       none,       none,       0,       0xde,      r2,  no,  0,     0)
INSTR (FICOM,             mem32,      none,       none,       none,       0,       0xda,      r2,  no,  0,     0)

INSTR (FICOMP,            mem16,      none,       none,       none,       0,       0xde,      r3,  no,  0,     0)
INSTR (FICOMP,            mem32,      none,       none,       none,       0,       0xda,      r3,  no,  0,     0)

INSTR (FILD,              mem16,      none,       none,       none,       0,       0xdf,      r0,  no,  0,     0)
INSTR (FILD,              mem32,      none,       none,       none,       0,       0xdb,      r0,  no,  0,     0)
INSTR (FILD,              mem64,      none,       none,       none,       0,       0xdf,      r5,  no,  0,     0)

INSTR (FINCSTP,           none,       none,       none,       none,       0,       0xd9f7,    no,  no,  0,     0)

INSTR (FNINIT,            none,       none,       none,       none,       0,       0xdbe3,    no,  no,  0,     0)

INSTR (FIST,              mem16,      none,       none,       none,       0,       0xdf,      r2,  no,  0,     0)
INSTR (FIST,              mem32,      none,       none,       none,       0,       0xdb,      r2,  no,  0,     0)

INSTR (FISTP,             mem16,      none,       none,       none,       0,       0xdf,      r3,  no,  0,     0)
INSTR (FISTP,             mem32,      none,       none,       none,       0,       0xdb,      r3,  no,  0,     0)
INSTR (FISTP,             mem64,      none,       none,       none,       0,       0xdf,      r7,  no,  0,     0)

INSTR (FISTTP,            mem16,      none,       none,       none,       0,       0xdf,      r1,  no,  0,     0)
INSTR (FISTTP,            mem32,      none,       none,       none,       0,       0xdb,      r1,  no,  0,     0)
INSTR (FISTTP,            mem64,      none,       none,       none,       0,       0xdd,      r1,  no,  0,     0)

INSTR (FLD,               sti,        none,       none,       none,       0,       0xd9c0,    rv,  no,  0,     0)
INSTR (FLD,               mem32,      none,       none,       none,       0,       0xd9,      r0,  no,  0,     0)
INSTR (FLD,               mem64,      none,       none,       none,       0,       0xdd,      r0,  no,  0,     0)

INSTR (FLD1,              none,       none,       none,       none,       0,       0xd9e8,    no,  no,  0,     0)

INSTR (FLDCW,             mem,        none,       none,       none,       0,       0xd9,      r5,  no,  0,     0)

INSTR (FLDENV,            mem,        none,       none,       none,       0,       0xd9,      r4,  no,  0,     0)

INSTR (FLDL2E,            none,       none,       none,       none,       0,       0xd9ea,    no,  no,  0,     0)

INSTR (FLDL2T,            none,       none,       none,       none,       0,       0xd9e9,    no,  no,  0,     0)

INSTR (FLDLG2,            none,       none,       none,       none,       0,       0xd9ec,    no,  no,  0,     0)

INSTR (FLDLN2,            none,       none,       none,       none,       0,       0xd9ed,    no,  no,  0,     0)

INSTR (FLDPI,             none,       none,       none,       none,       0,       0xd9eb,    no,  no,  0,     0)

INSTR (FLDZ,              none,       none,       none,       none,       0,       0xd9ee,    no,  no,  0,     0)

INSTR (FMUL,              st0,        sti,        none,       none,       0,       0xd8c8,    rv,  no,  0,     0)
INSTR (FMUL,              sti,        st0,        none,       none,       0,       0xdcc8,    rv,  no,  0,     0)
INSTR (FMUL,              mem32,      none,       none,       none,       0,       0xd8,      r1,  no,  0,     0)
INSTR (FMUL,              mem64,      none,       none,       none,       0,       0xdc,      r1,  no,  0,     0)

INSTR (FMULP,             none,       none,       none,       none,       0,       0xdec9,    no,  no,  0,     0)
INSTR (FMULP,             sti,        st0,        none,       none,       0,       0xdec8,    rv,  no,  0,     0)

INSTR (FIMUL,             mem16,      none,       none,       none,       0,       0xde,      r1,  no,  0,     0)
INSTR (FIMUL,             mem32,      none,       none,       none,       0,       0xda,      r1,  no,  0,     0)

INSTR (FNOP,              none,       none,       none,       none,       0,       0xd9d0,    no,  no,  0,     0)

INSTR (FPATAN,            none,       none,       none,       none,       0,       0xd9f3,    no,  no,  0,     0)

INSTR (FPREM,             none,       none,       none,       none,       0,       0xd9f8,    no,  no,  0,     0)

INSTR (FPREM1,            none,       none,       none,       none,       0,       0xd9f5,    no,  no,  0,     0)

INSTR (FPTAN,             none,       none,       none,       none,       0,       0xd9f2,    no,  no,  0,     0)

INSTR (FRNDINT,           none,       none,       none,       none,       0,       0xd9fc,    no,  no,  0,     0)

INSTR (FRSTOR,            mem,        none,       none,       none,       0,       0xdd,      r4,  no,  0,     0)

INSTR (FNSAVE,            mem,        none,       none,       none,       0,       0xdd,      r6,  no,  0,     0)

INSTR (FSCALE,            none,       none,       none,       none,       0,       0xd9fd,    no,  no,  0,     0)

INSTR (FSIN,              none,       none,       none,       none,       0,       0xd9fe,    no,  no,  0,     0)

INSTR (FSINCOS,           none,       none,       none,       none,       0,       0xd9fb,    no,  no,  0,     0)

INSTR (FSQRT,             none,       none,       none,       none,       0,       0xd9fa,    no,  no,  0,     0)

INSTR (FST,               sti,        none,       none,       none,       0,       0xddd0,    rv,  no,  0,     0)
INSTR (FST,               mem32,      none,       none,       none,       0,       0xd9,      r2,  no,  0,     0)
INSTR (FST,               mem64,      none,       none,       none,       0,       0xdd,      r2,  no,  0,     0)

INSTR (FSTP,              sti,        none,       none,       none,       0,       0xddd8,    rv,  no,  0,     0)
INSTR (FSTP,              mem32,      none,       none,       none,       0,       0xd9,      r3,  no,  0,     0)
INSTR (FSTP,              mem64,      none,       none,       none,       0,       0xdd,      r3,  no,  0,     0)

INSTR (FNSTCW,            mem,        none,       none,       none,       0,       0xd9,      r7,  no,  0,     0)

INSTR (FNSTENV,           mem,        none,       none,       none,       0,       0xd9,      r6,  no,  0,     0)

INSTR (FNSTSW,            ax,         none,       none,       none,       0,       0xdfe0,    no,  no,  0,     0)
INSTR (FNSTSW,            mem,        none,       none,       none,       0,       0xdd,      r7,  no,  0,     0)

INSTR (FSUB,              st0,        sti,        none,       none,       0,       0xd8e0,    rv,  no,  0,     0)
INSTR (FSUB,              sti,        st0,        none,       none,       0,       0xdce8,    rv,  no,  0,     0)
INSTR (FSUB,              mem32,      none,       none,       none,       0,       0xd8,      r4,  no,  0,     0)
INSTR (FSUB,              mem64,      none,       none,       none,       0,       0xdc,      r4,  no,  0,     0)

INSTR (FSUBP,             none,       none,       none,       none,       0,       0xdee9,    no,  no,  0,     0)
INSTR (FSUBP,             sti,        st0,        none,       none,       0,       0xdee8,    rv,  no,  0,     0)

INSTR (FISUB,             mem16,      none,       none,       none,       0,       0xde,      r4,  no,  0,     0)
INSTR (FISUB,             mem32,      none,       none,       none,       0,       0xda,      r4,  no,  0,     0)

INSTR (FSUBR,             st0,        sti,        none,       none,       0,       0xd8e8,    rv,  no,  0,     0)
INSTR (FSUBR,             sti,        st0,        none,       none,       0,       0xdce0,    rv,  no,  0,     0)
INSTR (FSUBR,             mem32,      none,       none,       none,       0,       0xd8,      r5,  no,  0,     0)
INSTR (FSUBR,             mem64,      none,       none,       none,       0,       0xdc,      r5,  no,  0,     0)

INSTR (FSUBRP,            none,       none,       none,       none,       0,       0xdee1,    no,  no,  0,     0)
INSTR (FSUBRP,            sti,        st0,        none,       none,       0,       0xdee0,    rv,  no,  0,     0)

INSTR (FISUBR,            mem16,      none,       none,       none,       0,       0xde,      r5,  no,  0,     0)
INSTR (FISUBR,            mem32,      none,       none,       none,       0,       0xda,      r5,  no,  0,     0)

INSTR (FTST,              none,       none,       none,       none,       0,       0xd9e4,    no,  no,  0,     0)

INSTR (FUCOM,             none,       none,       none,       none,       0,       0xdde1,    no,  no,  0,     0)
INSTR (FUCOM,             sti,        none,       none,       none,       0,       0xdde0,    rv,  no,  0,     0)

INSTR (FUCOMP,            none,       none,       none,       none,       0,       0xdde9,    no,  no,  0,     0)
INSTR (FUCOMP,            sti,        none,       none,       none,       0,       0xdde8,    rv,  no,  0,     0)

INSTR (FUCOMPP,           none,       none,       none,       none,       0,       0xdae9,    no,  no,  0,     0)

INSTR (FUCOMI,            st0,        sti,        none,       none,       0,       0xdbe8,    rv,  no,  0,     0)

INSTR (FUCOMIP,           st0,        sti,        none,       none,       0,       0xdfe8,    rv,  no,  0,     0)

INSTR (FWAIT,             none,       none,       none,       none,       0,       0x9b,      no,  no,  0,     0)

INSTR (FXAM,              none,       none,       none,       none,       0,       0xd9e5,    no,  no,  0,     0)

INSTR (FXCH,              none,       none,       none,       none,       0,       0xd9c9,    no,  no,  0,     0)
INSTR (FXCH,              sti,        none,       none,       none,       0,       0xd9c8,    rv,  no,  0,     0)

INSTR (FXTRACT,           none,       none,       none,       none,       0,       0xd9f4,    no,  no,  0,     0)

INSTR (FYL2X,             none,       none,       none,       none,       0,       0xd9f1,    no,  no,  0,     0)

INSTR (FYL2XP1,           none,       none,       none,       none,       0,       0xd9f9,    no,  no,  0,     0)

// operand types

TYPE (one)
TYPE (imm8)
TYPE (imm16)
TYPE (imm32)
TYPE (imm64)
TYPE (simm8)
TYPE (simm16)
TYPE (simm32)
TYPE (rel8off)
TYPE (rel16off)
TYPE (rel32off)
TYPE (moffset)
TYPE (al)
TYPE (cl)
TYPE (ax)
TYPE (dx)
TYPE (eax)
TYPE (ecx)
TYPE (rax)
TYPE (es)
TYPE (cs)
TYPE (ss)
TYPE (ds)
TYPE (fs)
TYPE (gs)
TYPE (cr8)
TYPE (reg8)
TYPE (reg16)
TYPE (reg16rm)
TYPE (reg32)
TYPE (reg32rm)
TYPE (reg32vvvv)
TYPE (reg64)
TYPE (reg64rm)
TYPE (reg64vvvv)
TYPE (mmx)
TYPE (xmm)
TYPE (ximm)
TYPE (xvvvv)
TYPE (ymm)
TYPE (yimm)
TYPE (yvvvv)
TYPE (segreg)
TYPE (cr)
TYPE (dr)
TYPE (st0)
TYPE (sti)
TYPE (mem)
TYPE (mem16)
TYPE (mem32)
TYPE (mem64)
TYPE (mem128)
TYPE (mem256)
TYPE (regmem8)
TYPE (regmem16)
TYPE (regmem32)
TYPE (regmem64)
TYPE (mmxmem32)
TYPE (mmxmem64)
TYPE (xmmmem8)
TYPE (xmmmem16)
TYPE (xmmmem32)
TYPE (xmmmem64)
TYPE (xmmmem128)
TYPE (ymmmem128)
TYPE (ymmmem256)

// operand codes

CODE (ib)
CODE (iw)
CODE (id)
CODE (iq)
CODE (is)
CODE (cb)
CODE (cw)
CODE (cd)
CODE (r)
CODE (r0)
CODE (r1)
CODE (r2)
CODE (r3)
CODE (r4)
CODE (r5)
CODE (r6)
CODE (r7)
CODE (rv)

// instruction flags

FLAG (I16,     0x1)
FLAG (I32,     0x2)
FLAG (I64,     0x4)
FLAG (D64,     0x8)
FLAG (O16,     0x10)
FLAG (O32,     0x20)
FLAG (O64,     0x40)
FLAG (PLOCK,   0x80)
FLAG (PREP,    0x100)
FLAG (PREPE,   0x100)
FLAG (PREPNE,  0x200)
FLAG (P66,     0x400)
FLAG (PF0,     0x800)
FLAG (PF2,     0x1000)
FLAG (PF3,     0x2000)
FLAG (PXOP,    0x4000)
FLAG (PVEX,    0x8000)
FLAG (SFX,     0x10000)
FLAG (FAR,     0x20000)

// registers

REG (AL,     al)
REG (CL,     cl)
REG (DL,     dl)
REG (BL,     bl)
REG (AH,     ah)
REG (CH,     ch)
REG (DH,     dh)
REG (BH,     bh)
REG (R8B,    r8b)
REG (R9B,    r9b)
REG (R10B,   r10b)
REG (R11B,   r11b)
REG (R12B,   r12b)
REG (R13B,   r13b)
REG (R14B,   r14b)
REG (R15B,   r15b)

REG (SPL,    spl)
REG (BPL,    bpl)
REG (SIL,    sil)
REG (DIL,    dil)

REG (AX,     ax)
REG (CX,     cx)
REG (DX,     dx)
REG (BX,     bx)
REG (SP,     sp)
REG (BP,     bp)
REG (SI,     si)
REG (DI,     di)
REG (R8W,    r8w)
REG (R9W,    r9w)
REG (R10W,   r10w)
REG (R11W,   r11w)
REG (R12W,   r12w)
REG (R13W,   r13w)
REG (R14W,   r14w)
REG (R15W,   r15w)

REG (EAX,    eax)
REG (ECX,    ecx)
REG (EDX,    edx)
REG (EBX,    ebx)
REG (ESP,    esp)
REG (EBP,    ebp)
REG (ESI,    esi)
REG (EDI,    edi)
REG (R8D,    r8d)
REG (R9D,    r9d)
REG (R10D,   r10d)
REG (R11D,   r11d)
REG (R12D,   r12d)
REG (R13D,   r13d)
REG (R14D,   r14d)
REG (R15D,   r15d)

REG (RAX,    rax)
REG (RCX,    rcx)
REG (RDX,    rdx)
REG (RBX,    rbx)
REG (RSP,    rsp)
REG (RBP,    rbp)
REG (RSI,    rsi)
REG (RDI,    rdi)
REG (R8,     r8)
REG (R9,     r9)
REG (R10,    r10)
REG (R11,    r11)
REG (R12,    r12)
REG (R13,    r13)
REG (R14,    r14)
REG (R15,    r15)

REG (MMX0,   mmx0)
REG (MMX1,   mmx1)
REG (MMX2,   mmx2)
REG (MMX3,   mmx3)
REG (MMX4,   mmx4)
REG (MMX5,   mmx5)
REG (MMX6,   mmx6)
REG (MMX7,   mmx7)
REG (MMX8,   mmx8)
REG (MMX9,   mmx9)
REG (MMX10,  mmx10)
REG (MMX11,  mmx11)
REG (MMX12,  mmx12)
REG (MMX13,  mmx13)
REG (MMX14,  mmx14)
REG (MMX15,  mmx15)

REG (XMM0,   xmm0)
REG (XMM1,   xmm1)
REG (XMM2,   xmm2)
REG (XMM3,   xmm3)
REG (XMM4,   xmm4)
REG (XMM5,   xmm5)
REG (XMM6,   xmm6)
REG (XMM7,   xmm7)
REG (XMM8,   xmm8)
REG (XMM9,   xmm9)
REG (XMM10,  xmm10)
REG (XMM11,  xmm11)
REG (XMM12,  xmm12)
REG (XMM13,  xmm13)
REG (XMM14,  xmm14)
REG (XMM15,  xmm15)

REG (YMM0,   ymm0)
REG (YMM1,   ymm1)
REG (YMM2,   ymm2)
REG (YMM3,   ymm3)
REG (YMM4,   ymm4)
REG (YMM5,   ymm5)
REG (YMM6,   ymm6)
REG (YMM7,   ymm7)
REG (YMM8,   ymm8)
REG (YMM9,   ymm9)
REG (YMM10,  ymm10)
REG (YMM11,  ymm11)
REG (YMM12,  ymm12)
REG (YMM13,  ymm13)
REG (YMM14,  ymm14)
REG (YMM15,  ymm15)

REG (ES,     es)
REG (CS,     cs)
REG (SS,     ss)
REG (DS,     ds)
REG (FS,     fs)
REG (GS,     gs)

REG (CR0,    cr0)
REG (CR1,    cr1)
REG (CR2,    cr2)
REG (CR3,    cr3)
REG (CR4,    cr4)
REG (CR5,    cr5)
REG (CR6,    cr6)
REG (CR7,    cr7)
REG (CR8,    cr8)
REG (CR9,    cr9)
REG (CR10,   cr10)
REG (CR11,   cr11)
REG (CR12,   cr12)
REG (CR13,   cr13)
REG (CR14,   cr14)
REG (CR15,   cr15)

REG (DR0,    dr0)
REG (DR1,    dr1)
REG (DR2,    dr2)
REG (DR3,    dr3)
REG (DR4,    dr4)
REG (DR5,    dr5)
REG (DR6,    dr6)
REG (DR7,    dr7)
REG (DR8,    dr8)
REG (DR9,    dr9)
REG (DR10,   dr10)
REG (DR11,   dr11)
REG (DR12,   dr12)
REG (DR13,   dr13)
REG (DR14,   dr14)
REG (DR15,   dr15)

REG (ST0,    st0)
REG (ST1,    st1)
REG (ST2,    st2)
REG (ST3,    st3)
REG (ST4,    st4)
REG (ST5,    st5)
REG (ST6,    st6)
REG (ST7,    st7)

REG (EIP,    eip)
REG (RIP,    rip)

// prefixes

PREFIX (OS,     0x66)
PREFIX (AS,     0x67)
PREFIX (ES,     0x26)
PREFIX (CS,     0x2e)
PREFIX (SS,     0x36)
PREFIX (DS,     0x3e)
PREFIX (FS,     0x64)
PREFIX (GS,     0x65)
PREFIX (LOCK,   0xf0)
PREFIX (REP,    0xf3)
PREFIX (REPNE,  0xf2)

PREFIX (REX,    0x40)
PREFIX (REXB,   0x41)
PREFIX (REXX,   0x42)
PREFIX (REXR,   0x44)
PREFIX (REXW,   0x48)

PREFIX (XOP,    0x8f)
PREFIX (VEX,    0xc4)
PREFIX (VEX2,   0xc5)

#undef CODE
#undef FLAG
#undef INSTR
#undef MNEM
#undef PREFIX
#undef REG
#undef TYPE

% ARM architecture documentation
% Copyright (C) Florian Negele

% This file is part of the Eigen Compiler Suite.

% Permission is granted to copy, distribute and/or modify this document
% under the terms of the GNU Free Documentation License, Version 1.3
% or any later version published by the Free Software Foundation.

% You should have received a copy of the GNU Free Documentation License
% along with the ECS.  If not, see <https://www.gnu.org/licenses/>.

% Generic documentation utilities
% Copyright (C) Florian Negele

% This file is part of the Eigen Compiler Suite.

% Permission is granted to copy, distribute and/or modify this document
% under the terms of the GNU Free Documentation License, Version 1.3
% or any later version published by the Free Software Foundation.

% You should have received a copy of the GNU Free Documentation License
% along with the ECS.  If not, see <https://www.gnu.org/licenses/>.

\providecommand{\cpp}{C\texttt{++}}
\providecommand{\opt}{_\mathit{opt}}
\providecommand{\tool}[1]{\texttt{#1}}
\providecommand{\version}{Version 0.0.40}
\providecommand{\resource}[1]{*++\txt{#1}}
\providecommand{\ecs}{Eigen Compiler Suite}
\providecommand{\changed}[1]{\underline{#1}}
\providecommand{\toolbox}[1]{\converter{#1}}
\providecommand{\file}{}\renewcommand{\file}[1]{\texttt{#1}}
\providecommand{\alignright}{\hfill\linebreak[0]\hspace*{\fill}}
\providecommand{\converter}[1]{*++[F][F*:white][F,:gray]\txt{#1}}
\providecommand{\documentation}{\ifbook chapter\else document\fi}
\providecommand{\Documentation}{\ifbook Chapter\else Document\fi}
\providecommand{\variable}[1]{\resource{\texttt{\small#1}\\variable}}
\providecommand{\documentationref}[2]{\ifbook\ref{#1}\else``\href{#1}{#2}''~\cite{#1}\fi}
\providecommand{\objfile}[1]{\texttt{#1}\index[runtime]{#1 object file@\texttt{#1} object file}}
\providecommand{\libfile}[1]{\texttt{#1}\index[runtime]{#1 library file@\texttt{#1} library file}}
\providecommand{\epigraph}[2]{\ifbook\begin{quote}\flushright\textit{#1}\par--- #2\end{quote}\fi}
\providecommand{\environmentvariable}[1]{\texttt{#1}\index{Environment variables!#1@\texttt{#1}}}
\providecommand{\environment}[1]{\texttt{#1}\index[environment]{#1 environment@\texttt{#1} environment}}
\providecommand{\toolsection}{}\renewcommand{\toolsection}[1]{\subsection{#1}\label{\prefix:#1}\tool{#1}}
\providecommand{\instruction}{}\renewcommand{\instruction}[2]{\noindent\qquad\pdftooltip{\texttt{#1}}{#2}\refstepcounter{instruction}\par}
\providecommand{\flowgraph}{}\renewcommand{\flowgraph}[1]{\par\sffamily\begin{displaymath}\xymatrix@=4ex{#1}\end{displaymath}\normalfont\par}
\providecommand{\instructionset}{}\renewcommand{\instructionset}[4]{\setcounter{instruction}{0}\begin{multicols}{\ifbook#3\else#4\fi}[{\captionof{table}[#2]{#2 (\ref*{#1:instructions}~instructions)}\label{tab:#1set}\vspace{-2ex}}]\footnotesize\raggedcolumns\input{#1.set}\label{#1:instructions}\end{multicols}}

\providecommand{\gpl}{GNU General Public License}
\providecommand{\rse}{ECS Runtime Support Exception}
\providecommand{\fdl}{\href{https://www.gnu.org/licenses/fdl.html}{GNU Free Documentation License}}

\providecommand{\docbegin}{}
\providecommand{\docend}{}
\providecommand{\doclabel}[1]{\hypertarget{#1}}
\providecommand{\doclink}[2]{\hyperlink{#1}{#2}}
\providecommand{\docsection}[3]{\hypertarget{#1}{\subsection{#2}}\label{sec:#1}\index[library]{#2@#3}}
\providecommand{\docsectionstar}[1]{}
\providecommand{\docsubbegin}{\begin{description}}
\providecommand{\docsubend}{\end{description}}
\providecommand{\docsubsection}[3]{\item[\hypertarget{#1}{#2}]\index[library]{#2@#3}}
\providecommand{\docsubsectionstar}[1]{\smallskip}
\providecommand{\docsubsubsection}[3]{\docsubsection{#1}{#2}{#3}}
\providecommand{\docsubsubsectionstar}[1]{}
\providecommand{\docsubsubsubsection}[3]{}
\providecommand{\docsubsubsubsectionstar}[1]{}
\providecommand{\doctable}{}

\providecommand{\debuggingtool}{}\renewcommand{\debuggingtool}{This tool is provided for debugging purposes.
It allows exposing and modifying an internal data structure that is usually not accessible.
}

\providecommand{\interface}{All tools accept command-line arguments which are taken as names of plain text files containing the source code.
If no arguments are provided, the standard input stream is used instead.
Output files are generated in the current working directory and have the same name as the input file being processed whereas the filename extension gets replaced by an appropriate suffix.
\seeinterface
}

\providecommand{\license}{\noindent Copyright \copyright{} Florian Negele\par\medskip\noindent
Permission is granted to copy, distribute and/or modify this document under the terms of the
\fdl{}, Version 1.3 or any later version published by the \href{https://fsf.org/}{Free Software Foundation}.
}

\providecommand{\ecslogosurface}{
\fill[darkgray] (0,0,0) -- (0,0,3) -- (0,3,3) -- (0,3,1) -- (0,4,1) -- (0,4,3) -- (0,5,3) -- (0,5,0) -- (0,2,0) -- (0,2,2) -- (0,1,2) -- (0,1,0) -- cycle;
\fill[gray] (0,5,0) -- (0,5,3) -- (1,5,3) -- (1,5,1) -- (2,5,1) -- (2,5,3) -- (3,5,3) -- (3,5,0) -- cycle;
\fill[lightgray] (0,0,0) -- (0,1,0) -- (2,1,0) -- (2,4,0) -- (1,4,0) -- (1,3,0) -- (2,3,0) -- (2,2,0) -- (0,2,0) -- (0,5,0) -- (3,5,0) -- (3,0,0) -- cycle;
\begin{scope}[line width=0.5]
\begin{scope}[gray]
\draw (0,0,0) -- (0,1,0);
\draw (2,1,0) -- (2,2,0);
\draw (0,1,2) -- (0,2,2);
\draw (0,2,0) -- (0,5,0);
\draw (2,3,0) -- (2,4,0);
\end{scope}
\begin{scope}[lightgray]
\draw (0,1,0) -- (0,1,2);
\draw (0,3,1) -- (0,3,3);
\draw (0,5,0) -- (0,5,3);
\draw (2,5,1) -- (2,5,3);
\end{scope}
\begin{scope}[white]
\draw (0,1,0) -- (2,1,0);
\draw (1,3,0) -- (2,3,0);
\draw (0,5,0) -- (3,5,0);
\end{scope}
\end{scope}
}

\providecommand{\ecslogo}[1]{
\begin{tikzpicture}[scale={(#1)/((sin(45)+cos(45))*3cm)},x={({-cos(45)*1cm},{sin(45)*sin(30)*1cm})},y={({0cm},{(cos(30)*1cm})},z={({sin(45)*1cm},{cos(45)*sin(30)*1cm})}]
\begin{scope}[darkgray,line width=1]
\draw (0,0,0) -- (0,0,3) -- (0,3,3) -- (2,3,3) -- (2,5,3) -- (3,5,3) -- (3,5,0) -- (3,0,0) -- cycle;
\draw (0,3,1) -- (0,4,1) -- (0,4,3) -- (0,5,3) -- (1,5,3) -- (1,5,1) -- (2,5,1);
\draw (1,3,0) -- (1,4,0) -- (2,4,0);
\end{scope}
\fill[darkgray] (2,0,0) -- (2,0,3) -- (2,5,3) -- (2,5,1) -- (2,4,1) -- (2,4,0) -- cycle;
\fill[lightgray] (2,0,2) -- (0,0,2) -- (0,2,2) -- (2,2,2) -- cycle;
\fill[gray] (0,1,0) -- (2,1,0) -- (2,1,2) -- (0,1,2) -- cycle;
\fill[gray] (0,3,1) -- (0,3,3) -- (2,3,3) -- (2,3,0) -- (1,3,0) -- (1,3,1) -- cycle;
\ecslogosurface
\end{tikzpicture}
}

\providecommand{\shadowedecslogo}[3]{
\begin{tikzpicture}[scale={(#1)/((sin(#2)+cos(#2))*3cm)},x={({-cos(#2)*1cm},{sin(#2)*sin(#3)*1cm})},y={({0cm},{(cos(#3)*1cm})},z={({sin(#2)*1cm},{cos(#2)*sin(#3)*1cm})}]
\shade[top color=lightgray!50!white,bottom color=white,middle color=lightgray!50!white] (0,0,0) -- (3,0,0) -- (3,{-0.5-3*sin(#2)*sin(#3)/cos(#3)},0) -- (0,-0.5,0) -- cycle;
\shade[top color=darkgray!50!gray,bottom color=white,middle color=darkgray!50!white] (0,0,0) -- (0,0,3) -- (0,{-0.5-3*cos(#2)*sin(#3)/cos(#3)},3) -- (0,-0.5,0) -- cycle;
\begin{scope}[y={({(cos(#2)+sin(#2))*0.5cm},{(cos(#2)*sin(#3)-sin(#2)*sin(#3))*0.5cm})}]
\useasboundingbox (3,0,0) -- (0,0,0) -- (0,0,3);
\shade[left color=darkgray!80!black,right color=lightgray,middle color=gray] (0,0,0) -- (0,1,0) -- (0,1,0.5) -- (0,2,0) -- (0,5,0) -- (0,5,3) -- (1,5,3) -- (1,4,3) -- (1,4,2.5) -- (1,3,3) -- (2,5,3) -- (3,5,3) -- (3,0,3) -- cycle;
\clip (0,0,0) -- (0,0,3) -- ({-3*sin(#2)/cos(#2)},0,0) -- cycle;
\shade[left color=darkgray,right color=lightgray!50!gray] (0,0,0) -- (0,1,0) -- (0,1,0.5) -- (0,2,0) -- (0,5,0) -- (0,5,3) -- (1,5,3) -- (1,4,3) -- (1,4,2.5) -- (1,3,3) -- (2,5,3) -- (3,5,3) -- (3,0,3) -- cycle;
\end{scope}
\shade[left color=darkgray,right color=darkgray!80!black] (2,0,0) -- (2,0,3) -- (2,5,3) -- (2,5,1) -- (2,4,1) -- (2,4,0) -- cycle;
\shade[left color=darkgray!90!black,right color=gray!80!darkgray] (2,0,2) -- (0,0,2) -- (0,2,2) -- (2,2,2) -- cycle;
\shade[top color=darkgray!90!black,bottom color=gray!80!darkgray] (0,1,0) -- (2,1,0) -- (2,1,2) -- (0,1,2) -- cycle;
\shade[top color=darkgray!90!black,bottom color=gray!80!darkgray] (0,3,1) -- (0,3,3) -- (2,3,3) -- (2,3,0) -- (1,3,0) -- (1,3,1) -- cycle;
\fill[gray] (2,1,0) -- (1.5,1,0.5) -- (0,1,0.5) -- (0,1,0) -- cycle;
\fill[gray] (1,3,2) -- (0.5,3,2) -- (0.5,3,3) -- (1,3,3) -- cycle;
\fill[gray] (2,3,0) -- (1.5,3,0.5) -- (1,3,0.5) -- (1,3,0) -- cycle;
\ecslogosurface
\end{tikzpicture}
}

\providecommand{\cpplogo}[1]{
\begin{tikzpicture}[scale=(#1)/512em]
\fill[gray] (435.2794,398.7159) -- (247.1911,507.3075) .. controls (236.3563,513.5642) and (218.6240,513.5642) .. (207.7892,507.3075) -- (19.7009,398.7159) .. controls (8.8646,392.4606) and (0.0000,377.1043) .. (0.0000,364.5924) -- (0.0000,147.4076) .. controls (0.8430,132.8363) and (8.2856,120.7683) .. (19.7009,113.2842) -- (207.7892,4.6926) .. controls (218.6240,-1.5642) and (236.3564,-1.5642) .. (247.1911,4.6926) -- (435.2794,113.2842) .. controls (447.5273,121.4304) and (454.4987,133.6918) .. (454.9803,147.4076) -- (454.9803,364.5924) .. controls (454.5404,377.7571) and (446.6566,391.0351) .. (435.2794,398.7159) -- cycle(75.8301,255.9993) .. controls (74.9389,404.0881) and (273.2892,469.4783) .. (358.8263,331.8769) -- (293.1917,293.8965) .. controls (253.5702,359.4301) and (155.1909,335.9977) .. (151.6601,255.9993) .. controls (152.7204,182.2703) and (249.4137,148.0211) .. (293.1961,218.1065) -- (358.8308,180.1276) .. controls (283.4477,49.2645) and (79.6318,96.3470) .. (75.8301,255.9993) -- cycle(379.1503,247.5747) -- (362.2982,247.5747) -- (362.2982,230.7226) -- (345.4490,230.7226) -- (345.4490,247.5747) -- (328.5969,247.5747) -- (328.5969,264.4254) -- (345.4490,264.4254) -- (345.4490,281.2759) -- (362.2982,281.2759) -- (362.2982,264.4254) -- (379.1503,264.4254) -- cycle(442.3420,247.5747) -- (425.4899,247.5747) -- (425.4899,230.7226) -- (408.6408,230.7226) -- (408.6408,247.5747) -- (391.7886,247.5747) -- (391.7886,264.4254) -- (408.6408,264.4254) -- (408.6408,281.2759) -- (425.4899,281.2759) -- (425.4899,264.4254) -- (442.3420,264.4254) -- cycle;
\end{tikzpicture}
}

\providecommand{\fallogo}[1]{
\begin{tikzpicture}[scale=(#1)/512em]
\fill[gray] (185.7774,0.0000) .. controls (200.4486,15.9798) and (226.8966,8.7148) .. (235.0426,31.5836) .. controls (249.5297,58.0598) and (247.9581,97.9161) .. (280.3335,110.9762) .. controls (309.1690,120.3496) and (337.8406,104.2727) .. (366.5753,103.9379) .. controls (373.4449,111.5171) and (379.2885,128.2574) .. (383.9755,108.9744) .. controls (396.6979,102.5615) and (437.2808,107.6681) .. (426.9652,124.3252) .. controls (408.9822,121.0785) and (412.4742,146.0729) .. (426.5192,131.4996) .. controls (433.8413,120.8489) and (465.1541,126.5522) .. (441.9067,135.7950) .. controls (396.1879,157.7478) and (344.1112,161.5079) .. (298.5528,183.5702) .. controls (277.7471,193.5198) and (284.6941,218.7163) .. (285.2127,236.9640) .. controls (292.3599,316.2826) and (307.3929,394.6311) .. (317.1198,473.6154) .. controls (329.0637,505.4736) and (292.1195,528.5004) .. (265.9183,511.2761) .. controls (237.9284,499.2462) and (237.3684,465.2681) .. (230.9102,439.9421) .. controls (218.6692,374.3397) and (215.6307,306.9662) .. (198.1732,242.3977) .. controls (183.1379,232.7444) and (164.4245,256.0298) .. (149.0430,261.4799) .. controls (116.9328,279.2585) and (87.1822,308.5851) .. (48.2293,307.8914) .. controls (21.3220,306.9037) and (-15.9107,281.8761) .. (7.2921,252.7908) .. controls (29.7799,220.6177) and (67.5177,204.3028) .. (100.9287,185.9449) .. controls (130.8217,170.8906) and (161.1548,156.5903) .. (191.0278,141.5847) .. controls (196.1738,120.0520) and (186.6049,95.2409) .. (186.8382,72.4353) .. controls (185.5234,48.4204) and (183.1700,23.9341) .. (185.7774,0.0000) -- cycle;
\end{tikzpicture}
}

\providecommand{\oblogo}[1]{
\begin{tikzpicture}[scale=(#1)/512em]
\fill[gray] (160.3865,208.9117) .. controls (154.0879,214.6478) and (149.0735,221.2409) .. (145.4125,228.5384) .. controls (184.8790,248.4273) and (234.7122,269.8787) .. (297.5493,291.8782) .. controls (300.3943,281.4769) and (300.9552,268.7619) .. (300.4023,255.2389) .. controls (248.9909,244.7891) and (200.0310,225.9279) .. (160.3865,208.9117) -- cycle(225.7398,392.6996) .. controls (308.0209,392.1716) and (359.3326,345.9277) .. (368.7203,285.2098) .. controls (376.6742,197.1784) and (311.7194,141.3342) .. (205.4287,142.1456) .. controls (139.9485,141.4804) and (88.7155,166.1957) .. (73.5775,228.0086) .. controls (52.0297,320.3408) and (123.4078,391.0103) .. (225.7398,392.6996) -- cycle(216.0739,176.4733) .. controls (268.9183,179.2424) and (315.8292,206.5488) .. (312.7454,265.1139) .. controls (313.2769,315.6384) and (286.5993,353.4946) .. (216.6040,355.7934) .. controls (162.4657,355.7934) and (126.0914,317.5023) .. (126.0914,260.5103) .. controls (126.1733,214.2900) and (163.3363,176.2849) .. (216.0739,176.4733) -- cycle(76.4897,189.1754) .. controls (13.1586,147.5631) and (0.0000,119.4207) .. (0.0000,119.4207) -- (90.6499,170.1632) .. controls (85.3004,175.8497) and (80.5994,182.1633) .. (76.4897,189.1754) -- cycle(353.9486,119.3004) -- (402.9482,119.3004) .. controls (427.0025,137.0797) and (450.9893,162.7034) .. (474.9529,191.0213) .. controls (509.3540,228.5339) and (531.3391,294.2091) .. (487.8149,312.1206) .. controls (462.8165,324.7652) and (394.3874,316.8943) .. (373.8912,313.6651) .. controls (379.9291,297.7449) and (383.2899,278.4204) .. (381.4989,257.7214) .. controls (420.3069,248.0321) and (421.9610,218.3461) .. (407.7867,192.6417) .. controls (391.1113,162.4018) and (370.1114,132.9097) .. (353.9486,119.3004) -- cycle;
\end{tikzpicture}
}

\providecommand{\markuptable}{
\begin{table}
\sffamily\centering
\begin{tabular}{@{}lcl@{}}
\toprule
\texttt{//italics//} & $\rightarrow$ & \textit{italics} \\
\midrule
\texttt{**bold**} & $\rightarrow$ & \textbf{bold} \\
\midrule
\texttt{\# ordered list} & & 1 ordered list \\
\texttt{\# second item} & $\rightarrow$ & 2 second item \\
\texttt{\#\# sub item} & & \hspace{1em} 1 sub item \\
\midrule
\texttt{* unordered list} & & $\bullet$ unordered list \\
\texttt{* second item} & $\rightarrow$ & $\bullet$ second item \\
\texttt{** sub item} & & \hspace{1em} $\bullet$ sub item \\
\midrule
\texttt{link to [[label]]} & $\rightarrow$ & link to \underline{label} \\
\midrule
\texttt{<{}<label>{}> definition } & $\rightarrow$ & definition \\
\midrule
\texttt{[[url|link name]]} & $\rightarrow$ & \underline{link name} \\
\midrule\addlinespace
\texttt{= large heading} & & {\Large large heading} \smallskip \\
\texttt{== medium heading} & $\rightarrow$ & {\large medium heading} \\
\texttt{=== small heading} & & small heading \\
\midrule
\texttt{no line break} & & no line break for paragraphs \\
\texttt{for paragraphs} & $\rightarrow$ \\
& & use empty line \\
\texttt{use empty line} \\
\midrule
\texttt{force\textbackslash\textbackslash line break} & $\rightarrow$ & force \\
& & line break \\
\midrule
\texttt{horizontal line} & $\rightarrow$ & horizontal line \\
\texttt{----} & & \hrulefill \\
\midrule
\texttt{|=a|=table|=header} & & \underline{a \enspace table \enspace header} \\
\texttt{|a|table|row} & $\rightarrow$ & a \enspace table \enspace row \\
\texttt{|b|table|row} & & b \enspace table \enspace row \\
\midrule
\texttt{\{\{\{} \\
\texttt{unformatted} & $\rightarrow$ & \texttt{unformatted} \\
\texttt{code} & & \texttt{code} \\
\texttt{\}\}\}} \\
\midrule\addlinespace
\texttt{@ new article} & & {\Large 1.\ new article} \smallskip \\
\texttt{@ second article} & $\rightarrow$ & {\Large 2.\ second article} \smallskip \\
\texttt{@@ sub article} & & {\large 2.1.\ sub article} \\
\bottomrule
\end{tabular}
\normalfont\caption{Elements of the generic documentation markup language}
\label{tab:docmarkup}
\end{table}
}

\providecommand{\startchapter}[4]{
\documentclass[11pt,a4paper]{article}
\usepackage{booktabs}
\usepackage[format=hang,labelfont=bf]{caption}
\usepackage{changepage}
\usepackage[T1]{fontenc}
\usepackage[margin=2cm]{geometry}
\usepackage{hyperref}
\usepackage[american]{isodate}
\usepackage{lmodern}
\usepackage{longtable}
\usepackage{mathptmx}
\usepackage{microtype}
\usepackage[toc]{multitoc}
\usepackage{multirow}
\usepackage[all]{nowidow}
\usepackage{pdfcomment}
\usepackage{syntax}
\usepackage{tikz}
\usepackage[all]{xy}
\hypersetup{pdfborder={0 0 0},bookmarksnumbered=true,pdftitle={\ecs{}: #2},pdfauthor={Florian Negele},pdfsubject={\ecs{}},pdfkeywords={#1}}
\setlength{\grammarindent}{8em}\setlength{\grammarparsep}{0.2ex}
\setlength{\columnsep}{2em}
\newcommand{\prefix}{}
\newcounter{instruction}
\bibliographystyle{unsrt}
\renewcommand{\index}[2][]{}
\renewcommand{\arraystretch}{1.05}
\renewcommand{\floatpagefraction}{0.7}
\renewcommand{\syntleft}{\itshape}\renewcommand{\syntright}{}
\title{\vspace{-5ex}\Huge{\ecs{}}\medskip\hrule}
\author{\huge{#2}}
\date{\medskip\version}
\newif\ifbook\bookfalse
\pagestyle{headings}
\frenchspacing
\begin{document}
\maketitle\thispagestyle{empty}\noindent#4\setlength{\columnseprule}{0.4pt}\tableofcontents\setlength{\columnseprule}{0pt}\vfill\pagebreak[3]\null\vfill\bigskip\noindent
\parbox{\textwidth-4em}{\license The contents of this \documentation{} are part of the \href{manual}{\ecs{} User Manual}~\cite{manual} and correspond to Chapter ``\href{manual\##3}{#1}''.\alignright\mbox{\today}}
\parbox{4em}{\flushright\ecslogo{3em}}
\clearpage
}

\providecommand{\concludechapter}{
\vfill\pagebreak[3]\null\vfill
\thispagestyle{myheadings}\markright{REFERENCES}
\noindent\begin{minipage}{\textwidth}\begin{multicols}{2}[\section*{References}]
\renewcommand{\section}[2]{}\small\bibliography{references}
\end{multicols}\end{minipage}\end{document}
}

\providecommand{\startpresentation}[2]{
\documentclass[14pt,aspectratio=43,usepdftitle=false]{beamer}
\usepackage{booktabs}
\usepackage{etex}
\usepackage{multicol}
\usepackage{tikz}
\usepackage[all]{xy}
\bibliographystyle{unsrt}
\setlength{\columnsep}{1em}
\setlength{\leftmargini}{1em}
\setbeamercolor{title}{fg=black}
\setbeamercolor{structure}{fg=darkgray}
\setbeamercolor{bibliography item}{fg=darkgray}
\setbeamerfont{title}{series=\bfseries}
\setbeamerfont{subtitle}{series=\normalfont}
\setbeamerfont*{frametitle}{parent=title}
\setbeamerfont{block title}{series=\bfseries}
\setbeamerfont*{framesubtitle}{parent=subtitle}
\setbeamersize{text margin left=1em,text margin right=1em}
\setbeamertemplate{navigation symbols}{}
\setbeamertemplate{itemize item}[circle]{}
\setbeamertemplate{bibliography item}[triangle]{}
\setbeamertemplate{bibliography entry author}{\usebeamercolor[fg]{bibliography item}}
\setbeamertemplate{frametitle}{\medskip\usebeamerfont{frametitle}\color{gray}\raisebox{-2.5ex}[0ex][0ex]{\rule{0.1em}{4.5ex}}}
\addtobeamertemplate{frametitle}{}{\hspace{0.4em}\usebeamercolor[fg]{title}\insertframetitle\par\vspace{0.2ex}\hspace{0.5em}\usebeamerfont{framesubtitle}\insertframesubtitle}
\hypersetup{pdfborder={0 0 0},bookmarksnumbered=true,bookmarksopen=true,bookmarksopenlevel=0,pdftitle={\ecs{}: #1},pdfauthor={Florian Negele},pdfsubject={\ecs{}},pdfkeywords={#1}}
\renewcommand{\flowgraph}[1]{\resizebox{\textwidth}{!}{$$\xymatrix{##1}$$}}
\title{\ecs{}\medskip\hrule\medskip}
\institute{\shadowedecslogo{5em}{30}{15}}
\date{\version}
\subtitle{#1}
\begin{document}
\begin{frame}[plain]\titlepage\nocite{manual}\end{frame}
\begin{frame}{Contents}{#1}\begin{center}\tableofcontents\end{center}\end{frame}
}

\providecommand{\concludepresentation}{
\begin{frame}{References}\begin{footnotesize}\setlength{\columnseprule}{0.4pt}\begin{multicols}{2}\bibliography{references}\end{multicols}\end{footnotesize}\end{frame}
\end{document}
}

\providecommand{\startbook}[1]{
\documentclass[10pt,paper=17cm:24cm,DIV=13,twoside=semi,headings=normal,numbers=noendperiod,cleardoublepage=plain]{scrbook}
\usepackage{atveryend}
\usepackage{booktabs}
\usepackage{caption}
\usepackage{changepage}
\usepackage[T1]{fontenc}
\usepackage{imakeidx}
\usepackage{hyperref}
\usepackage[american]{isodate}
\usepackage{lmodern}
\usepackage{longtable}
\usepackage{mathptmx}
\usepackage[final]{microtype}
\usepackage{multicol}
\usepackage{multirow}
\usepackage[all]{nowidow}
\usepackage{pdfcomment}
\usepackage{scrlayer-scrpage}
\usepackage{setspace}
\usepackage{syntax}
\usepackage[eventxtindent=4pt,oddtxtexdent=4pt]{thumbs}
\usepackage{tikz}
\usepackage[all]{xy}
\hyphenation{Micro-Blaze Open-Cores Open-RISC Power-PC}
\hypersetup{pdfborder={0 0 0},bookmarksnumbered=true,bookmarksopen=true,bookmarksopenlevel=0,pdftitle={\ecs{}: #1},pdfauthor={Florian Negele},pdfsubject={\ecs{}},pdfkeywords={#1}}
\setlength{\grammarindent}{8em}\setlength{\grammarparsep}{0.7ex}
\setkomafont{captionlabel}{\usekomafont{descriptionlabel}}
\renewcommand{\arraystretch}{1.05}\setstretch{1.1}
\renewcommand{\chapterformat}{\thechapter\autodot\enskip\raisebox{-1ex}[0ex][0ex]{\color{gray}\rule{0.1em}{3.5ex}}\enskip}
\renewcommand{\startchapter}[4]{\hypertarget{##3}{\chapter{##1}}\label{##3}##4\addthumb{##1}{\LARGE\sffamily\bfseries\thechapter}{white}{gray}\renewcommand{\prefix}{##3}}
\renewcommand{\concludechapter}{\clearpage{\stopthumb\cleardoublepage}}
\renewcommand{\syntleft}{\itshape}\renewcommand{\syntright}{}
\renewcommand{\floatpagefraction}{0.7}
\renewcommand{\partheademptypage}{}
\DeclareMicrotypeAlias{lmss}{cmr}
\newcommand{\prefix}{}
\newcounter{instruction}
\bibliographystyle{unsrt}
\newif\ifbook\booktrue
\makeindex[intoc,title=Index]
\makeindex[intoc,name=tools,title=Index of Tools,columns=3]
\makeindex[intoc,name=library,title=Index of Library Names]
\makeindex[intoc,name=runtime,title=Index of Runtime Support]
\makeindex[intoc,name=environment,title=Index of Target Environments]
\indexsetup{toclevel=chapter,headers={\indexname}{\indexname}}
\frenchspacing
\begin{document}
\pagenumbering{alph}
\begin{titlepage}\centering
\huge\sffamily\null\vfill\textbf{\ecs{}}\bigskip\hrule\bigskip#1
\normalsize\normalfont\vfill\vfill\shadowedecslogo{10em}{30}{15}
\large\vfill\vfill\version
\end{titlepage}
\null\vfill
\thispagestyle{empty}
\noindent\today\par\medskip
\license A copy of this license is included in Appendix~\ref{fdl} on page~\pageref{fdl}.
All product names used herein are for identification purposes only and may be trademarks of their respective companies.
\concludechapter
\frontmatter
\setcounter{tocdepth}{1}
\tableofcontents
\setcounter{tocdepth}{2}
\concludechapter
\listoffigures
\concludechapter
\listoftables
\concludechapter
}

\providecommand{\concludebook}{
\backmatter
\addtocontents{toc}{\protect\setcounter{tocdepth}{-1}}
\phantomsection\addcontentsline{toc}{part}{Bibliography}
\bibliography{references}
\concludechapter
\phantomsection\addcontentsline{toc}{part}{Indexes}
\printindex
\concludechapter
\indexprologue{\label{idx:tools}}
\printindex[tools]
\concludechapter
\printindex[library]
\concludechapter
\indexprologue{\label{idx:runtime}}
\printindex[runtime]
\concludechapter
\indexprologue{\label{idx:environment}}
\printindex[environment]
\concludechapter
\pagestyle{empty}\pagenumbering{Alph}\null\clearpage
\null\vfill\centering\ecslogo{4em}\par\medskip\license
\end{document}
}

% chapter references

\providecommand{\seedocumentationref}{}\renewcommand{\seedocumentationref}[3]{#1, see \Documentation{}~\documentationref{#2}{#3}. }
\providecommand{\seeinterface}{}\renewcommand{\seeinterface}{\ifbook See \Documentation{}~\documentationref{interface}{User Interface} for more information about the common user interface of all of these tools. \fi}
\providecommand{\seeguide}{}\renewcommand{\seeguide}{\seedocumentationref{For basic examples of using some of these tools in practice}{guide}{User Guide}}
\providecommand{\seecpp}{}\renewcommand{\seecpp}{\seedocumentationref{For more information about the \cpp{} programming language and its implementation by the \ecs{}}{cpp}{User Manual for \cpp{}}}
\providecommand{\seefalse}{}\renewcommand{\seefalse}{\seedocumentationref{For more information about the FALSE programming language and its implementation by the \ecs{}}{false}{User Manual for FALSE}}
\providecommand{\seeoberon}{}\renewcommand{\seeoberon}{\seedocumentationref{For more information about the Oberon programming language and its implementation by the \ecs{}}{oberon}{User Manual for Oberon}}
\providecommand{\seeassembly}{}\renewcommand{\seeassembly}{\seedocumentationref{For more information about the generic assembly language and how to use it}{assembly}{Generic Assembly Language Specification}}
\providecommand{\seeamd}{}\renewcommand{\seeamd}{\seedocumentationref{For more information about how the \ecs{} supports the AMD64 hardware architecture}{amd64}{AMD64 Hardware Architecture Support}}
\providecommand{\seearm}{}\renewcommand{\seearm}{\seedocumentationref{For more information about how the \ecs{} supports the ARM hardware architecture}{arm}{ARM Hardware Architecture Support}}
\providecommand{\seeavr}{}\renewcommand{\seeavr}{\seedocumentationref{For more information about how the \ecs{} supports the AVR hardware architecture}{avr}{AVR Hardware Architecture Support}}
\providecommand{\seeavrtt}{}\renewcommand{\seeavrtt}{\seedocumentationref{For more information about how the \ecs{} supports the AVR32 hardware architecture}{avr32}{AVR32 Hardware Architecture Support}}
\providecommand{\seemabk}{}\renewcommand{\seemabk}{\seedocumentationref{For more information about how the \ecs{} supports the M68000 hardware architecture}{m68k}{M68000 Hardware Architecture Support}}
\providecommand{\seemibl}{}\renewcommand{\seemibl}{\seedocumentationref{For more information about how the \ecs{} supports the MicroBlaze hardware architecture}{mibl}{MicroBlaze Hardware Architecture Support}}
\providecommand{\seemips}{}\renewcommand{\seemips}{\seedocumentationref{For more information about how the \ecs{} supports the MIPS32 and MIPS64 hardware architectures}{mips}{MIPS Hardware Architecture Support}}
\providecommand{\seemmix}{}\renewcommand{\seemmix}{\seedocumentationref{For more information about how the \ecs{} supports the MMIX hardware architecture}{mmix}{MMIX Hardware Architecture Support}}
\providecommand{\seeorok}{}\renewcommand{\seeorok}{\seedocumentationref{For more information about how the \ecs{} supports the OpenRISC 1000 hardware architecture}{or1k}{OpenRISC 1000 Hardware Architecture Support}}
\providecommand{\seeppc}{}\renewcommand{\seeppc}{\seedocumentationref{For more information about how the \ecs{} supports the PowerPC hardware architecture}{ppc}{PowerPC Hardware Architecture Support}}
\providecommand{\seerisc}{}\renewcommand{\seerisc}{\seedocumentationref{For more information about how the \ecs{} supports the RISC hardware architecture}{risc}{RISC Hardware Architecture Support}}
\providecommand{\seewasm}{}\renewcommand{\seewasm}{\seedocumentationref{For more information about how the \ecs{} supports the WebAssembly architecture}{wasm}{WebAssembly Architecture Support}}
\providecommand{\seedocumentation}{}\renewcommand{\seedocumentation}{\seedocumentationref{For more information about generic documentations and their generation by the \ecs{}}{documentation}{Generic Documentation Generation}}
\providecommand{\seedebugging}{}\renewcommand{\seedebugging}{\seedocumentationref{For more information about debugging information and its representation}{debugging}{Debugging Information Representation}}
\providecommand{\seecode}{}\renewcommand{\seecode}{\seedocumentationref{For more information about intermediate code and its purpose}{code}{Intermediate Code Representation}}
\providecommand{\seeobject}{}\renewcommand{\seeobject}{\seedocumentationref{For more information about object files and their purpose}{object}{Object File Representation}}

% generic documentation tools

\providecommand{\docprint}{
\toolsection{docprint} is a pretty printer for generic documentations.
It reformats generic documentations and writes it to the standard output stream.
\debuggingtool
\flowgraph{\resource{generic\\documentation} \ar[r] & \toolbox{docprint} \ar[r] & \resource{generic\\documentation}}
\seedocumentation
}

\providecommand{\doccheck}{
\toolsection{doccheck} is a syntactic and semantic checker for generic documentations.
It just performs syntactic and semantic checks on generic documentations and writes its diagnostic messages to the standard error stream.
\debuggingtool
\flowgraph{\resource{generic\\documentation} \ar[r] & \toolbox{doccheck} \ar[r] & \resource{diagnostic\\messages}}
\seedocumentation
}

\providecommand{\dochtml}{
\toolsection{dochtml} is an HTML documentation generator for generic documentations.
It processes several generic documentations and assembles all information therein into an HTML document.
\debuggingtool
\flowgraph{\resource{generic\\documentation} \ar[r] & \toolbox{dochtml} \ar[r] & \resource{HTML\\document}}
\seedocumentation
}

\providecommand{\doclatex}{
\toolsection{doclatex} is a Latex documentation generator for generic documentations.
It processes several generic documentations and assembles all information therein into a Latex document.
\debuggingtool
\flowgraph{\resource{generic\\documentation} \ar[r] & \toolbox{doclatex} \ar[r] & \resource{Latex\\document}}
\seedocumentation
}

% intermediate code tools

\providecommand{\cdcheck}{
\toolsection{cdcheck} is a syntactic and semantic checker for intermediate code.
It just performs syntactic and semantic checks on programs written in intermediate code and writes its diagnostic messages to the standard error stream.
\debuggingtool
\flowgraph{\resource{intermediate\\code} \ar[r] & \toolbox{cdcheck} \ar[r] & \resource{diagnostic\\messages}}
\seeassembly\seecode
}

\providecommand{\cdopt}{
\toolsection{cdopt} is an optimizer for intermediate code.
It performs various optimizations on programs written in intermediate code and writes the result to the standard output stream.
\debuggingtool
\flowgraph{\resource{intermediate\\code} \ar[r] & \toolbox{cdopt} \ar[r] & \resource{optimized\\code}}
\seeassembly\seecode
}

\providecommand{\cdrun}{
\toolsection{cdrun} is an interpreter for intermediate code.
It processes and executes programs written in intermediate code.
The following code sections are predefined and have the usual semantics:
\texttt{abort}, \texttt{\_Exit}, \texttt{fflush}, \texttt{floor}, \texttt{fputc}, \texttt{free}, \texttt{getchar}, \texttt{malloc}, and \texttt{putchar}.
Diagnostic messages about invalid operations include the name of the executed code section and the index of the erroneous instruction.
\debuggingtool
\flowgraph{\resource{intermediate\\code} \ar[r] & \toolbox{cdrun} \ar@/u/[r] & \resource{input/\\output} \ar@/d/[l]}
\seeassembly\seecode
}

\providecommand{\cdamda}{
\toolsection{cdamd16} is a compiler for intermediate code targeting the AMD64 hardware architecture.
It generates machine code for AMD64 processors from programs written in intermediate code and stores it in corresponding object files.
The compiler generates machine code for the 16-bit operating mode defined by the AMD64 architecture.
It also creates a debugging information file as well as an assembly file containing a listing of the generated machine code.
\debuggingtool
\flowgraph{\resource{intermediate\\code} \ar[r] & \toolbox{cdamd16} \ar[r] \ar[d] \ar[rd] & \resource{object file} \\ & \resource{assembly\\listing} & \resource{debugging\\information}}
\seeassembly\seeamd\seeobject\seecode\seedebugging
}

\providecommand{\cdamdb}{
\toolsection{cdamd32} is a compiler for intermediate code targeting the AMD64 hardware architecture.
It generates machine code for AMD64 processors from programs written in intermediate code and stores it in corresponding object files.
The compiler generates machine code for the 32-bit operating mode defined by the AMD64 architecture.
It also creates a debugging information file as well as an assembly file containing a listing of the generated machine code.
\debuggingtool
\flowgraph{\resource{intermediate\\code} \ar[r] & \toolbox{cdamd32} \ar[r] \ar[d] \ar[rd] & \resource{object file} \\ & \resource{assembly\\listing} & \resource{debugging\\information}}
\seeassembly\seeamd\seeobject\seecode\seedebugging
}

\providecommand{\cdamdc}{
\toolsection{cdamd64} is a compiler for intermediate code targeting the AMD64 hardware architecture.
It generates machine code for AMD64 processors from programs written in intermediate code and stores it in corresponding object files.
The compiler generates machine code for the 64-bit operating mode defined by the AMD64 architecture.
It also creates a debugging information file as well as an assembly file containing a listing of the generated machine code.
\debuggingtool
\flowgraph{\resource{intermediate\\code} \ar[r] & \toolbox{cdamd64} \ar[r] \ar[d] \ar[rd] & \resource{object file} \\ & \resource{assembly\\listing} & \resource{debugging\\information}}
\seeassembly\seeamd\seeobject\seecode\seedebugging
}

\providecommand{\cdarma}{
\toolsection{cdarma32} is a compiler for intermediate code targeting the ARM hardware architecture.
It generates machine code for ARM processors executing A32 instructions from programs written in intermediate code and stores it in corresponding object files.
It also creates a debugging information file as well as an assembly file containing a listing of the generated machine code.
\debuggingtool
\flowgraph{\resource{intermediate\\code} \ar[r] & \toolbox{cdarma32} \ar[r] \ar[d] \ar[rd] & \resource{object file} \\ & \resource{assembly\\listing} & \resource{debugging\\information}}
\seeassembly\seearm\seeobject\seecode\seedebugging
}

\providecommand{\cdarmb}{
\toolsection{cdarma64} is a compiler for intermediate code targeting the ARM hardware architecture.
It generates machine code for ARM processors executing A64 instructions from programs written in intermediate code and stores it in corresponding object files.
It also creates a debugging information file as well as an assembly file containing a listing of the generated machine code.
\debuggingtool
\flowgraph{\resource{intermediate\\code} \ar[r] & \toolbox{cdarma64} \ar[r] \ar[d] \ar[rd] & \resource{object file} \\ & \resource{assembly\\listing} & \resource{debugging\\information}}
\seeassembly\seearm\seeobject\seecode\seedebugging
}

\providecommand{\cdarmc}{
\toolsection{cdarmt32} is a compiler for intermediate code targeting the ARM hardware architecture.
It generates machine code for ARM processors without floating-point extension executing T32 instructions from programs written in intermediate code and stores it in corresponding object files.
It also creates a debugging information file as well as an assembly file containing a listing of the generated machine code.
\debuggingtool
\flowgraph{\resource{intermediate\\code} \ar[r] & \toolbox{cdarmt32} \ar[r] \ar[d] \ar[rd] & \resource{object file} \\ & \resource{assembly\\listing} & \resource{debugging\\information}}
\seeassembly\seearm\seeobject\seecode\seedebugging
}

\providecommand{\cdarmcfpe}{
\toolsection{cdarmt32fpe} is a compiler for intermediate code targeting the ARM hardware architecture.
It generates machine code for ARM processors with floating-point extension executing T32 instructions from programs written in intermediate code and stores it in corresponding object files.
It also creates a debugging information file as well as an assembly file containing a listing of the generated machine code.
\debuggingtool
\flowgraph{\resource{intermediate\\code} \ar[r] & \toolbox{cdarmt32fpe} \ar[r] \ar[d] \ar[rd] & \resource{object file} \\ & \resource{assembly\\listing} & \resource{debugging\\information}}
\seeassembly\seearm\seeobject\seecode\seedebugging
}

\providecommand{\cdavr}{
\toolsection{cdavr} is a compiler for intermediate code targeting the AVR hardware architecture.
It generates machine code for AVR processors from programs written in intermediate code and stores it in corresponding object files.
It also creates a debugging information file as well as an assembly file containing a listing of the generated machine code.
\debuggingtool
\flowgraph{\resource{intermediate\\code} \ar[r] & \toolbox{cdavr} \ar[r] \ar[d] \ar[rd] & \resource{object file} \\ & \resource{assembly\\listing} & \resource{debugging\\information}}
\seeassembly\seeavr\seeobject\seecode\seedebugging
}

\providecommand{\cdavrtt}{
\toolsection{cdavr32} is a compiler for intermediate code targeting the AVR32 hardware architecture.
It generates machine code for AVR32 processors from programs written in intermediate code and stores it in corresponding object files.
It also creates a debugging information file as well as an assembly file containing a listing of the generated machine code.
\debuggingtool
\flowgraph{\resource{intermediate\\code} \ar[r] & \toolbox{cdavr32} \ar[r] \ar[d] \ar[rd] & \resource{object file} \\ & \resource{assembly\\listing} & \resource{debugging\\information}}
\seeassembly\seeavrtt\seeobject\seecode\seedebugging
}

\providecommand{\cdmabk}{
\toolsection{cdm68k} is a compiler for intermediate code targeting the M68000 hardware architecture.
It generates machine code for M68000 processors from programs written in intermediate code and stores it in corresponding object files.
It also creates a debugging information file as well as an assembly file containing a listing of the generated machine code.
\debuggingtool
\flowgraph{\resource{intermediate\\code} \ar[r] & \toolbox{cdm68k} \ar[r] \ar[d] \ar[rd] & \resource{object file} \\ & \resource{assembly\\listing} & \resource{debugging\\information}}
\seeassembly\seemabk\seeobject\seecode\seedebugging
}

\providecommand{\cdmibl}{
\toolsection{cdmibl} is a compiler for intermediate code targeting the MicroBlaze hardware architecture.
It generates machine code for MicroBlaze processors from programs written in intermediate code and stores it in corresponding object files.
It also creates a debugging information file as well as an assembly file containing a listing of the generated machine code.
\debuggingtool
\flowgraph{\resource{intermediate\\code} \ar[r] & \toolbox{cdmibl} \ar[r] \ar[d] \ar[rd] & \resource{object file} \\ & \resource{assembly\\listing} & \resource{debugging\\information}}
\seeassembly\seemibl\seeobject\seecode\seedebugging
}

\providecommand{\cdmipsa}{
\toolsection{cdmips32} is a compiler for intermediate code targeting the MIPS32 hardware architecture.
It generates machine code for MIPS32 processors from programs written in intermediate code and stores it in corresponding object files.
It also creates a debugging information file as well as an assembly file containing a listing of the generated machine code.
\debuggingtool
\flowgraph{\resource{intermediate\\code} \ar[r] & \toolbox{cdmips32} \ar[r] \ar[d] \ar[rd] & \resource{object file} \\ & \resource{assembly\\listing} & \resource{debugging\\information}}
\seeassembly\seemips\seeobject\seecode\seedebugging
}

\providecommand{\cdmipsb}{
\toolsection{cdmips64} is a compiler for intermediate code targeting the MIPS64 hardware architecture.
It generates machine code for MIPS64 processors from programs written in intermediate code and stores it in corresponding object files.
It also creates a debugging information file as well as an assembly file containing a listing of the generated machine code.
\debuggingtool
\flowgraph{\resource{intermediate\\code} \ar[r] & \toolbox{cdmips64} \ar[r] \ar[d] \ar[rd] & \resource{object file} \\ & \resource{assembly\\listing} & \resource{debugging\\information}}
\seeassembly\seemips\seeobject\seecode\seedebugging
}

\providecommand{\cdmmix}{
\toolsection{cdmmix} is a compiler for intermediate code targeting the MMIX hardware architecture.
It generates machine code for MMIX processors from programs written in intermediate code and stores it in corresponding object files.
It also creates a debugging information file as well as an assembly file containing a listing of the generated machine code.
\debuggingtool
\flowgraph{\resource{intermediate\\code} \ar[r] & \toolbox{cdmmix} \ar[r] \ar[d] \ar[rd] & \resource{object file} \\ & \resource{assembly\\listing} & \resource{debugging\\information}}
\seeassembly\seemmix\seeobject\seecode\seedebugging
}

\providecommand{\cdorok}{
\toolsection{cdor1k} is a compiler for intermediate code targeting the OpenRISC 1000 hardware architecture.
It generates machine code for OpenRISC 1000 processors from programs written in intermediate code and stores it in corresponding object files.
It also creates a debugging information file as well as an assembly file containing a listing of the generated machine code.
\debuggingtool
\flowgraph{\resource{intermediate\\code} \ar[r] & \toolbox{cdor1k} \ar[r] \ar[d] \ar[rd] & \resource{object file} \\ & \resource{assembly\\listing} & \resource{debugging\\information}}
\seeassembly\seeorok\seeobject\seecode\seedebugging
}

\providecommand{\cdppca}{
\toolsection{cdppc32} is a compiler for intermediate code targeting the PowerPC hardware architecture.
It generates machine code for PowerPC processors from programs written in intermediate code and stores it in corresponding object files.
The compiler generates machine code for the 32-bit operating mode defined by the PowerPC architecture.
It also creates a debugging information file as well as an assembly file containing a listing of the generated machine code.
\debuggingtool
\flowgraph{\resource{intermediate\\code} \ar[r] & \toolbox{cdppc32} \ar[r] \ar[d] \ar[rd] & \resource{object file} \\ & \resource{assembly\\listing} & \resource{debugging\\information}}
\seeassembly\seeppc\seeobject\seecode\seedebugging
}

\providecommand{\cdppcb}{
\toolsection{cdppc64} is a compiler for intermediate code targeting the PowerPC hardware architecture.
It generates machine code for PowerPC processors from programs written in intermediate code and stores it in corresponding object files.
The compiler generates machine code for the 64-bit operating mode defined by the PowerPC architecture.
It also creates a debugging information file as well as an assembly file containing a listing of the generated machine code.
\debuggingtool
\flowgraph{\resource{intermediate\\code} \ar[r] & \toolbox{cdppc64} \ar[r] \ar[d] \ar[rd] & \resource{object file} \\ & \resource{assembly\\listing} & \resource{debugging\\information}}
\seeassembly\seeppc\seeobject\seecode\seedebugging
}

\providecommand{\cdrisc}{
\toolsection{cdrisc} is a compiler for intermediate code targeting the RISC hardware architecture.
It generates machine code for RISC processors from programs written in intermediate code and stores it in corresponding object files.
It also creates a debugging information file as well as an assembly file containing a listing of the generated machine code.
\debuggingtool
\flowgraph{\resource{intermediate\\code} \ar[r] & \toolbox{cdrisc} \ar[r] \ar[d] \ar[rd] & \resource{object file} \\ & \resource{assembly\\listing} & \resource{debugging\\information}}
\seeassembly\seerisc\seeobject\seecode\seedebugging
}

\providecommand{\cdwasm}{
\toolsection{cdwasm} is a compiler for intermediate code targeting the WebAssembly architecture.
It generates machine code for WebAssembly targets from programs written in intermediate code and stores it in corresponding object files.
It also creates a debugging information file as well as an assembly file containing a listing of the generated machine code.
\debuggingtool
\flowgraph{\resource{intermediate\\code} \ar[r] & \toolbox{cdwasm} \ar[r] \ar[d] \ar[rd] & \resource{object file} \\ & \resource{assembly\\listing} & \resource{debugging\\information}}
\seeassembly\seewasm\seeobject\seecode\seedebugging
}

% C++ tools

\providecommand{\cppprep}{
\toolsection{cppprep} is a preprocessor for the \cpp{} programming language.
It preprocesses source code according to the rules of \cpp{} and writes it to the standard output stream.
Only the macro names \texttt{\_\_DATE\_\_}, \texttt{\_\_FILE\_\_}, \texttt{\_\_LINE\_\_}, and \texttt{\_\_TIME\_\_} are predefined.
\flowgraph{\resource{\cpp{} or other\\source code} \ar[r] & \toolbox{cppprep} \ar[r] & \resource{preprocessed\\source code} \\ & \variable{ECSINCLUDE} \ar[u]}
\seecpp
}

\providecommand{\cppprint}{
\toolsection{cppprint} is a pretty printer for the \cpp{} programming language.
It reformats the source code of \cpp{} programs and writes it to the standard output stream.
\flowgraph{\resource{\cpp{}\\source code} \ar[r] & \toolbox{cppprint} \ar[r] & \resource{reformatted\\source code} \\ & \variable{ECSINCLUDE} \ar[u]}
\seecpp
}

\providecommand{\cppcheck}{
\toolsection{cppcheck} is a syntactic and semantic checker for the \cpp{} programming language.
It just performs syntactic and semantic checks on \cpp{} programs and writes its diagnostic messages to the standard error stream.
\flowgraph{\resource{\cpp{}\\source code} \ar[r] & \toolbox{cppcheck} \ar[r] & \resource{diagnostic\\messages} \\ & \variable{ECSINCLUDE} \ar[u]}
\seecpp
}

\providecommand{\cppdump}{
\toolsection{cppdump} is a serializer for the \cpp{} programming language.
It dumps the complete internal representation of programs written in \cpp{} into an XML document.
\debuggingtool
\flowgraph{\resource{\cpp{}\\source code} \ar[r] & \toolbox{cppdump} \ar[r] & \resource{internal\\representation} \\ & \variable{ECSINCLUDE} \ar[u]}
\seecpp
}

\providecommand{\cpprun}{
\toolsection{cpprun} is an interpreter for the \cpp{} programming language.
It processes and executes programs written in \cpp{}.
The macro \texttt{\_\_run\_\_} is predefined in order to enable programmers to identify this tool while interpreting.
\flowgraph{\resource{\cpp{}\\source code} \ar[r] & \toolbox{cpprun} \ar@/u/[r] & \resource{input/\\output} \ar@/d/[l] \\ & \variable{ECSINCLUDE} \ar[u]}
\seecpp
}

\providecommand{\cppdoc}{
\toolsection{cppdoc} is a generic documentation generator for the \cpp{} programming language.
It processes several \cpp{} source files and assembles all information therein into a generic documentation.
\debuggingtool
\flowgraph{\resource{\cpp{}\\source code} \ar[r] & \toolbox{cppdoc} \ar[r] & \resource{generic\\documentation} \\ & \variable{ECSINCLUDE} \ar[u]}
\seecpp\seedocumentation
}

\providecommand{\cpphtml}{
\toolsection{cpphtml} is an HTML documentation generator for the \cpp{} programming language.
It processes several \cpp{} source files and assembles all information therein into an HTML document.
\flowgraph{\resource{\cpp{}\\source code} \ar[r] & \toolbox{cpphtml} \ar[r] & \resource{HTML\\document} \\ & \variable{ECSINCLUDE} \ar[u]}
\seecpp\seedocumentation
}

\providecommand{\cpplatex}{
\toolsection{cpplatex} is a Latex documentation generator for the \cpp{} programming language.
It processes several \cpp{} source files and assembles all information therein into a Latex document.
\flowgraph{\resource{\cpp{}\\source code} \ar[r] & \toolbox{cpplatex} \ar[r] & \resource{Latex\\document} \\ & \variable{ECSINCLUDE} \ar[u]}
\seecpp\seedocumentation
}

\providecommand{\cppcode}{
\toolsection{cppcode} is an intermediate code generator for the \cpp{} programming language.
It generates intermediate code from programs written in \cpp{} and stores it in corresponding assembly files.
The macro \texttt{\_\_code\_\_} is predefined in order to enable programmers to identify this tool while generating intermediate code.
Programs generated with this tool require additional runtime support that is stored in the \file{cpp\-code\-run} library file.
\debuggingtool
\flowgraph{\resource{\cpp{}\\source code} \ar[r] & \toolbox{cppcode} \ar[r] & \resource{intermediate\\code} \\ & \variable{ECSINCLUDE} \ar[u]}
\seecpp\seeassembly\seecode
}

\providecommand{\cppamda}{
\toolsection{cppamd16} is a compiler for the \cpp{} programming language targeting the AMD64 hardware architecture.
It generates machine code for AMD64 processors from programs written in \cpp{} and stores it in corresponding object files.
The compiler generates machine code for the 16-bit operating mode defined by the AMD64 architecture.
For debugging purposes, it also creates a debugging information file as well as an assembly file containing a listing of the generated machine code.
The macro \texttt{\_\_amd16\_\_} is predefined in order to enable programmers to identify this tool and its target architecture while compiling.
Programs generated with this compiler require additional runtime support that is stored in the \file{cpp\-amd16\-run} library file.
\flowgraph{\resource{\cpp{}\\source code} \ar[r] & \toolbox{cppamd16} \ar[r] \ar[d] \ar[rd] & \resource{object file} \\ \variable{ECSINCLUDE} \ar[ru] & \resource{debugging\\information} & \resource{assembly\\listing}}
\seecpp\seeassembly\seeamd\seeobject\seedebugging
}

\providecommand{\cppamdb}{
\toolsection{cppamd32} is a compiler for the \cpp{} programming language targeting the AMD64 hardware architecture.
It generates machine code for AMD64 processors from programs written in \cpp{} and stores it in corresponding object files.
The compiler generates machine code for the 32-bit operating mode defined by the AMD64 architecture.
For debugging purposes, it also creates a debugging information file as well as an assembly file containing a listing of the generated machine code.
The macro \texttt{\_\_amd32\_\_} is predefined in order to enable programmers to identify this tool and its target architecture while compiling.
Programs generated with this compiler require additional runtime support that is stored in the \file{cpp\-amd32\-run} library file.
\flowgraph{\resource{\cpp{}\\source code} \ar[r] & \toolbox{cppamd32} \ar[r] \ar[d] \ar[rd] & \resource{object file} \\ \variable{ECSINCLUDE} \ar[ru] & \resource{debugging\\information} & \resource{assembly\\listing}}
\seecpp\seeassembly\seeamd\seeobject\seedebugging
}

\providecommand{\cppamdc}{
\toolsection{cppamd64} is a compiler for the \cpp{} programming language targeting the AMD64 hardware architecture.
It generates machine code for AMD64 processors from programs written in \cpp{} and stores it in corresponding object files.
The compiler generates machine code for the 64-bit operating mode defined by the AMD64 architecture.
For debugging purposes, it also creates a debugging information file as well as an assembly file containing a listing of the generated machine code.
The macro \texttt{\_\_amd64\_\_} is predefined in order to enable programmers to identify this tool and its target architecture while compiling.
Programs generated with this compiler require additional runtime support that is stored in the \file{cpp\-amd64\-run} library file.
\flowgraph{\resource{\cpp{}\\source code} \ar[r] & \toolbox{cppamd64} \ar[r] \ar[d] \ar[rd] & \resource{object file} \\ \variable{ECSINCLUDE} \ar[ru] & \resource{debugging\\information} & \resource{assembly\\listing}}
\seecpp\seeassembly\seeamd\seeobject\seedebugging
}

\providecommand{\cpparma}{
\toolsection{cpparma32} is a compiler for the \cpp{} programming language targeting the ARM hardware architecture.
It generates machine code for ARM processors executing A32 instructions from programs written in \cpp{} and stores it in corresponding object files.
For debugging purposes, it also creates a debugging information file as well as an assembly file containing a listing of the generated machine code.
The macro \texttt{\_\_arma32\_\_} is predefined in order to enable programmers to identify this tool and its target architecture while compiling.
Programs generated with this compiler require additional runtime support that is stored in the \file{cpp\-arma32\-run} library file.
\flowgraph{\resource{\cpp{}\\source code} \ar[r] & \toolbox{cpparma32} \ar[r] \ar[d] \ar[rd] & \resource{object file} \\ \variable{ECSINCLUDE} \ar[ru] & \resource{debugging\\information} & \resource{assembly\\listing}}
\seecpp\seeassembly\seearm\seeobject\seedebugging
}

\providecommand{\cpparmb}{
\toolsection{cpparma64} is a compiler for the \cpp{} programming language targeting the ARM hardware architecture.
It generates machine code for ARM processors executing A64 instructions from programs written in \cpp{} and stores it in corresponding object files.
For debugging purposes, it also creates a debugging information file as well as an assembly file containing a listing of the generated machine code.
The macro \texttt{\_\_arma64\_\_} is predefined in order to enable programmers to identify this tool and its target architecture while compiling.
Programs generated with this compiler require additional runtime support that is stored in the \file{cpp\-arma64\-run} library file.
\flowgraph{\resource{\cpp{}\\source code} \ar[r] & \toolbox{cpparma64} \ar[r] \ar[d] \ar[rd] & \resource{object file} \\ \variable{ECSINCLUDE} \ar[ru] & \resource{debugging\\information} & \resource{assembly\\listing}}
\seecpp\seeassembly\seearm\seeobject\seedebugging
}

\providecommand{\cpparmc}{
\toolsection{cpparmt32} is a compiler for the \cpp{} programming language targeting the ARM hardware architecture.
It generates machine code for ARM processors without floating-point extension executing T32 instructions from programs written in \cpp{} and stores it in corresponding object files.
For debugging purposes, it also creates a debugging information file as well as an assembly file containing a listing of the generated machine code.
The macro \texttt{\_\_armt32\_\_} is predefined in order to enable programmers to identify this tool and its target architecture while compiling.
Programs generated with this compiler require additional runtime support that is stored in the \file{cpp\-armt32\-run} library file.
\flowgraph{\resource{\cpp{}\\source code} \ar[r] & \toolbox{cpparmt32} \ar[r] \ar[d] \ar[rd] & \resource{object file} \\ \variable{ECSINCLUDE} \ar[ru] & \resource{debugging\\information} & \resource{assembly\\listing}}
\seecpp\seeassembly\seearm\seeobject\seedebugging
}

\providecommand{\cpparmcfpe}{
\toolsection{cpparmt32fpe} is a compiler for the \cpp{} programming language targeting the ARM hardware architecture.
It generates machine code for ARM processors with floating-point extension executing T32 instructions from programs written in \cpp{} and stores it in corresponding object files.
For debugging purposes, it also creates a debugging information file as well as an assembly file containing a listing of the generated machine code.
The macro \texttt{\_\_armt32fpe\_\_} is predefined in order to enable programmers to identify this tool and its target architecture while compiling.
Programs generated with this compiler require additional runtime support that is stored in the \file{cpp\-armt32\-fpe\-run} library file.
\flowgraph{\resource{\cpp{}\\source code} \ar[r] & \toolbox{cpparmt32fpe} \ar[r] \ar[d] \ar[rd] & \resource{object file} \\ \variable{ECSINCLUDE} \ar[ru] & \resource{debugging\\information} & \resource{assembly\\listing}}
\seecpp\seeassembly\seearm\seeobject\seedebugging
}

\providecommand{\cppavr}{
\toolsection{cppavr} is a compiler for the \cpp{} programming language targeting the AVR hardware architecture.
It generates machine code for AVR processors from programs written in \cpp{} and stores it in corresponding object files.
For debugging purposes, it also creates a debugging information file as well as an assembly file containing a listing of the generated machine code.
The macro \texttt{\_\_avr\_\_} is predefined in order to enable programmers to identify this tool and its target architecture while compiling.
Programs generated with this compiler require additional runtime support that is stored in the \file{cpp\-avr\-run} library file.
\flowgraph{\resource{\cpp{}\\source code} \ar[r] & \toolbox{cppavr} \ar[r] \ar[d] \ar[rd] & \resource{object file} \\ \variable{ECSINCLUDE} \ar[ru] & \resource{debugging\\information} & \resource{assembly\\listing}}
\seecpp\seeassembly\seeavr\seeobject\seedebugging
}

\providecommand{\cppavrtt}{
\toolsection{cppavr32} is a compiler for the \cpp{} programming language targeting the AVR32 hardware architecture.
It generates machine code for AVR32 processors from programs written in \cpp{} and stores it in corresponding object files.
For debugging purposes, it also creates a debugging information file as well as an assembly file containing a listing of the generated machine code.
The macro \texttt{\_\_avr32\_\_} is predefined in order to enable programmers to identify this tool and its target architecture while compiling.
Programs generated with this compiler require additional runtime support that is stored in the \file{cpp\-avr32\-run} library file.
\flowgraph{\resource{\cpp{}\\source code} \ar[r] & \toolbox{cppavr32} \ar[r] \ar[d] \ar[rd] & \resource{object file} \\ \variable{ECSINCLUDE} \ar[ru] & \resource{debugging\\information} & \resource{assembly\\listing}}
\seecpp\seeassembly\seeavrtt\seeobject\seedebugging
}

\providecommand{\cppmabk}{
\toolsection{cppm68k} is a compiler for the \cpp{} programming language targeting the M68000 hardware architecture.
It generates machine code for M68000 processors from programs written in \cpp{} and stores it in corresponding object files.
For debugging purposes, it also creates a debugging information file as well as an assembly file containing a listing of the generated machine code.
The macro \texttt{\_\_m68k\_\_} is predefined in order to enable programmers to identify this tool and its target architecture while compiling.
Programs generated with this compiler require additional runtime support that is stored in the \file{cpp\-m68k\-run} library file.
\flowgraph{\resource{\cpp{}\\source code} \ar[r] & \toolbox{cppm68k} \ar[r] \ar[d] \ar[rd] & \resource{object file} \\ \variable{ECSINCLUDE} \ar[ru] & \resource{debugging\\information} & \resource{assembly\\listing}}
\seecpp\seeassembly\seemabk\seeobject\seedebugging
}

\providecommand{\cppmibl}{
\toolsection{cppmibl} is a compiler for the \cpp{} programming language targeting the MicroBlaze hardware architecture.
It generates machine code for MicroBlaze processors from programs written in \cpp{} and stores it in corresponding object files.
For debugging purposes, it also creates a debugging information file as well as an assembly file containing a listing of the generated machine code.
The macro \texttt{\_\_mibl\_\_} is predefined in order to enable programmers to identify this tool and its target architecture while compiling.
Programs generated with this compiler require additional runtime support that is stored in the \file{cpp\-mibl\-run} library file.
\flowgraph{\resource{\cpp{}\\source code} \ar[r] & \toolbox{cppmibl} \ar[r] \ar[d] \ar[rd] & \resource{object file} \\ \variable{ECSINCLUDE} \ar[ru] & \resource{debugging\\information} & \resource{assembly\\listing}}
\seecpp\seeassembly\seemibl\seeobject\seedebugging
}

\providecommand{\cppmipsa}{
\toolsection{cppmips32} is a compiler for the \cpp{} programming language targeting the MIPS32 hardware architecture.
It generates machine code for MIPS32 processors from programs written in \cpp{} and stores it in corresponding object files.
For debugging purposes, it also creates a debugging information file as well as an assembly file containing a listing of the generated machine code.
The macro \texttt{\_\_mips32\_\_} is predefined in order to enable programmers to identify this tool and its target architecture while compiling.
Programs generated with this compiler require additional runtime support that is stored in the \file{cpp\-mips32\-run} library file.
\flowgraph{\resource{\cpp{}\\source code} \ar[r] & \toolbox{cppmips32} \ar[r] \ar[d] \ar[rd] & \resource{object file} \\ \variable{ECSINCLUDE} \ar[ru] & \resource{debugging\\information} & \resource{assembly\\listing}}
\seecpp\seeassembly\seemips\seeobject\seedebugging
}

\providecommand{\cppmipsb}{
\toolsection{cppmips64} is a compiler for the \cpp{} programming language targeting the MIPS64 hardware architecture.
It generates machine code for MIPS64 processors from programs written in \cpp{} and stores it in corresponding object files.
For debugging purposes, it also creates a debugging information file as well as an assembly file containing a listing of the generated machine code.
The macro \texttt{\_\_mips64\_\_} is predefined in order to enable programmers to identify this tool and its target architecture while compiling.
Programs generated with this compiler require additional runtime support that is stored in the \file{cpp\-mips64\-run} library file.
\flowgraph{\resource{\cpp{}\\source code} \ar[r] & \toolbox{cppmips64} \ar[r] \ar[d] \ar[rd] & \resource{object file} \\ \variable{ECSINCLUDE} \ar[ru] & \resource{debugging\\information} & \resource{assembly\\listing}}
\seecpp\seeassembly\seemips\seeobject\seedebugging
}

\providecommand{\cppmmix}{
\toolsection{cppmmix} is a compiler for the \cpp{} programming language targeting the MMIX hardware architecture.
It generates machine code for MMIX processors from programs written in \cpp{} and stores it in corresponding object files.
For debugging purposes, it also creates a debugging information file as well as an assembly file containing a listing of the generated machine code.
The macro \texttt{\_\_mmix\_\_} is predefined in order to enable programmers to identify this tool and its target architecture while compiling.
Programs generated with this compiler require additional runtime support that is stored in the \file{cpp\-mmix\-run} library file.
\flowgraph{\resource{\cpp{}\\source code} \ar[r] & \toolbox{cppmmix} \ar[r] \ar[d] \ar[rd] & \resource{object file} \\ \variable{ECSINCLUDE} \ar[ru] & \resource{debugging\\information} & \resource{assembly\\listing}}
\seecpp\seeassembly\seemmix\seeobject\seedebugging
}

\providecommand{\cpporok}{
\toolsection{cppor1k} is a compiler for the \cpp{} programming language targeting the OpenRISC 1000 hardware architecture.
It generates machine code for OpenRISC 1000 processors from programs written in \cpp{} and stores it in corresponding object files.
For debugging purposes, it also creates a debugging information file as well as an assembly file containing a listing of the generated machine code.
The macro \texttt{\_\_or1k\_\_} is predefined in order to enable programmers to identify this tool and its target architecture while compiling.
Programs generated with this compiler require additional runtime support that is stored in the \file{cpp\-or1k\-run} library file.
\flowgraph{\resource{\cpp{}\\source code} \ar[r] & \toolbox{cppor1k} \ar[r] \ar[d] \ar[rd] & \resource{object file} \\ \variable{ECSINCLUDE} \ar[ru] & \resource{debugging\\information} & \resource{assembly\\listing}}
\seecpp\seeassembly\seeorok\seeobject\seedebugging
}

\providecommand{\cppppca}{
\toolsection{cppppc32} is a compiler for the \cpp{} programming language targeting the PowerPC hardware architecture.
It generates machine code for PowerPC processors from programs written in \cpp{} and stores it in corresponding object files.
The compiler generates machine code for the 32-bit operating mode defined by the PowerPC architecture.
For debugging purposes, it also creates a debugging information file as well as an assembly file containing a listing of the generated machine code.
The macro \texttt{\_\_ppc32\_\_} is predefined in order to enable programmers to identify this tool and its target architecture while compiling.
Programs generated with this compiler require additional runtime support that is stored in the \file{cpp\-ppc32\-run} library file.
\flowgraph{\resource{\cpp{}\\source code} \ar[r] & \toolbox{cppppc32} \ar[r] \ar[d] \ar[rd] & \resource{object file} \\ \variable{ECSINCLUDE} \ar[ru] & \resource{debugging\\information} & \resource{assembly\\listing}}
\seecpp\seeassembly\seeppc\seeobject\seedebugging
}

\providecommand{\cppppcb}{
\toolsection{cppppc64} is a compiler for the \cpp{} programming language targeting the PowerPC hardware architecture.
It generates machine code for PowerPC processors from programs written in \cpp{} and stores it in corresponding object files.
The compiler generates machine code for the 64-bit operating mode defined by the PowerPC architecture.
For debugging purposes, it also creates a debugging information file as well as an assembly file containing a listing of the generated machine code.
The macro \texttt{\_\_ppc64\_\_} is predefined in order to enable programmers to identify this tool and its target architecture while compiling.
Programs generated with this compiler require additional runtime support that is stored in the \file{cpp\-ppc64\-run} library file.
\flowgraph{\resource{\cpp{}\\source code} \ar[r] & \toolbox{cppppc64} \ar[r] \ar[d] \ar[rd] & \resource{object file} \\ \variable{ECSINCLUDE} \ar[ru] & \resource{debugging\\information} & \resource{assembly\\listing}}
\seecpp\seeassembly\seeppc\seeobject\seedebugging
}

\providecommand{\cpprisc}{
\toolsection{cpprisc} is a compiler for the \cpp{} programming language targeting the RISC hardware architecture.
It generates machine code for RISC processors from programs written in \cpp{} and stores it in corresponding object files.
For debugging purposes, it also creates a debugging information file as well as an assembly file containing a listing of the generated machine code.
The macro \texttt{\_\_risc\_\_} is predefined in order to enable programmers to identify this tool and its target architecture while compiling.
Programs generated with this compiler require additional runtime support that is stored in the \file{cpp\-risc\-run} library file.
\flowgraph{\resource{\cpp{}\\source code} \ar[r] & \toolbox{cpprisc} \ar[r] \ar[d] \ar[rd] & \resource{object file} \\ \variable{ECSINCLUDE} \ar[ru] & \resource{debugging\\information} & \resource{assembly\\listing}}
\seecpp\seeassembly\seerisc\seeobject\seedebugging
}

\providecommand{\cppwasm}{
\toolsection{cppwasm} is a compiler for the \cpp{} programming language targeting the WebAssembly architecture.
It generates machine code for WebAssembly targets from programs written in \cpp{} and stores it in corresponding object files.
For debugging purposes, it also creates a debugging information file as well as an assembly file containing a listing of the generated machine code.
The macro \texttt{\_\_wasm\_\_} is predefined in order to enable programmers to identify this tool and its target architecture while compiling.
Programs generated with this compiler require additional runtime support that is stored in the \file{cpp\-wasm\-run} library file.
\flowgraph{\resource{\cpp{}\\source code} \ar[r] & \toolbox{cppwasm} \ar[r] \ar[d] \ar[rd] & \resource{object file} \\ \variable{ECSINCLUDE} \ar[ru] & \resource{debugging\\information} & \resource{assembly\\listing}}
\seecpp\seeassembly\seewasm\seeobject\seedebugging
}

% FALSE tools

\providecommand{\falprint}{
\toolsection{falprint} is a pretty printer for the FALSE programming language.
It reformats the source code of FALSE programs and writes it to the standard output stream.
\flowgraph{\resource{FALSE\\source code} \ar[r] & \toolbox{falprint} \ar[r] & \resource{reformatted\\source code}}
\seefalse
}

\providecommand{\falcheck}{
\toolsection{falcheck} is a syntactic and semantic checker for the FALSE programming language.
It just performs syntactic and semantic checks on FALSE programs and writes its diagnostic messages to the standard error stream.
\flowgraph{\resource{FALSE\\source code} \ar[r] & \toolbox{falcheck} \ar[r] & \resource{diagnostic\\messages}}
\seefalse
}

\providecommand{\faldump}{
\toolsection{faldump} is a serializer for the FALSE programming language.
It dumps the complete internal representation of programs written in FALSE into an XML document.
\debuggingtool
\flowgraph{\resource{FALSE\\source code} \ar[r] & \toolbox{faldump} \ar[r] & \resource{internal\\representation}}
\seefalse
}

\providecommand{\falrun}{
\toolsection{falrun} is an interpreter for the FALSE programming language.
It processes and executes programs written in FALSE\@.
\flowgraph{\resource{FALSE\\source code} \ar[r] & \toolbox{falrun} \ar@/u/[r] & \resource{input/\\output} \ar@/d/[l]}
\seefalse
}

\providecommand{\falcpp}{
\toolsection{falcpp} is a transpiler for the FALSE programming language.
It translates programs written in FALSE into \cpp{} programs and stores them in corresponding source files.
\flowgraph{\resource{FALSE\\source code} \ar[r] & \toolbox{falcpp} \ar[r] & \resource{\cpp{}\\source file}}
\seefalse\seecpp
}

\providecommand{\falcode}{
\toolsection{falcode} is an intermediate code generator for the FALSE programming language.
It generates intermediate code from programs written in FALSE and stores it in corresponding assembly files.
\debuggingtool
\flowgraph{\resource{FALSE\\source code} \ar[r] & \toolbox{falcode} \ar[r] & \resource{intermediate\\code}}
\seefalse\seeassembly\seecode
}

\providecommand{\falamda}{
\toolsection{falamd16} is a compiler for the FALSE programming language targeting the AMD64 hardware architecture.
It generates machine code for AMD64 processors from programs written in FALSE and stores it in corresponding object files.
The compiler generates machine code for the 16-bit operating mode defined by the AMD64 architecture.
\flowgraph{\resource{FALSE\\source code} \ar[r] & \toolbox{falamd16} \ar[r] & \resource{object file}}
\seefalse\seeamd\seeobject
}

\providecommand{\falamdb}{
\toolsection{falamd32} is a compiler for the FALSE programming language targeting the AMD64 hardware architecture.
It generates machine code for AMD64 processors from programs written in FALSE and stores it in corresponding object files.
The compiler generates machine code for the 32-bit operating mode defined by the AMD64 architecture.
\flowgraph{\resource{FALSE\\source code} \ar[r] & \toolbox{falamd32} \ar[r] & \resource{object file}}
\seefalse\seeamd\seeobject
}

\providecommand{\falamdc}{
\toolsection{falamd64} is a compiler for the FALSE programming language targeting the AMD64 hardware architecture.
It generates machine code for AMD64 processors from programs written in FALSE and stores it in corresponding object files.
The compiler generates machine code for the 64-bit operating mode defined by the AMD64 architecture.
\flowgraph{\resource{FALSE\\source code} \ar[r] & \toolbox{falamd64} \ar[r] & \resource{object file}}
\seefalse\seeamd\seeobject
}

\providecommand{\falarma}{
\toolsection{falarma32} is a compiler for the FALSE programming language targeting the ARM hardware architecture.
It generates machine code for ARM processors executing A32 instructions from programs written in FALSE and stores it in corresponding object files.
\flowgraph{\resource{FALSE\\source code} \ar[r] & \toolbox{falarma32} \ar[r] & \resource{object file}}
\seefalse\seearm\seeobject
}

\providecommand{\falarmb}{
\toolsection{falarma64} is a compiler for the FALSE programming language targeting the ARM hardware architecture.
It generates machine code for ARM processors executing A64 instructions from programs written in FALSE and stores it in corresponding object files.
\flowgraph{\resource{FALSE\\source code} \ar[r] & \toolbox{falarma64} \ar[r] & \resource{object file}}
\seefalse\seearm\seeobject
}

\providecommand{\falarmc}{
\toolsection{falarmt32} is a compiler for the FALSE programming language targeting the ARM hardware architecture.
It generates machine code for ARM processors without floating-point extension executing T32 instructions from programs written in FALSE and stores it in corresponding object files.
\flowgraph{\resource{FALSE\\source code} \ar[r] & \toolbox{falarmt32} \ar[r] & \resource{object file}}
\seefalse\seearm\seeobject
}

\providecommand{\falarmcfpe}{
\toolsection{falarmt32fpe} is a compiler for the FALSE programming language targeting the ARM hardware architecture.
It generates machine code for ARM processors with floating-point extension executing T32 instructions from programs written in FALSE and stores it in corresponding object files.
\flowgraph{\resource{FALSE\\source code} \ar[r] & \toolbox{falarmt32fpe} \ar[r] & \resource{object file}}
\seefalse\seearm\seeobject
}

\providecommand{\falavr}{
\toolsection{falavr} is a compiler for the FALSE programming language targeting the AVR hardware architecture.
It generates machine code for AVR processors from programs written in FALSE and stores it in corresponding object files.
\flowgraph{\resource{FALSE\\source code} \ar[r] & \toolbox{falavr} \ar[r] & \resource{object file}}
\seefalse\seeavr\seeobject
}

\providecommand{\falavrtt}{
\toolsection{falavr32} is a compiler for the FALSE programming language targeting the AVR32 hardware architecture.
It generates machine code for AVR32 processors from programs written in FALSE and stores it in corresponding object files.
\flowgraph{\resource{FALSE\\source code} \ar[r] & \toolbox{falavr32} \ar[r] & \resource{object file}}
\seefalse\seeavrtt\seeobject
}

\providecommand{\falmabk}{
\toolsection{falm68k} is a compiler for the FALSE programming language targeting the M68000 hardware architecture.
It generates machine code for M68000 processors from programs written in FALSE and stores it in corresponding object files.
\flowgraph{\resource{FALSE\\source code} \ar[r] & \toolbox{falm68k} \ar[r] & \resource{object file}}
\seefalse\seemabk\seeobject
}

\providecommand{\falmibl}{
\toolsection{falmibl} is a compiler for the FALSE programming language targeting the MicroBlaze hardware architecture.
It generates machine code for MicroBlaze processors from programs written in FALSE and stores it in corresponding object files.
\flowgraph{\resource{FALSE\\source code} \ar[r] & \toolbox{falmibl} \ar[r] & \resource{object file}}
\seefalse\seemibl\seeobject
}

\providecommand{\falmipsa}{
\toolsection{falmips32} is a compiler for the FALSE programming language targeting the MIPS32 hardware architecture.
It generates machine code for MIPS32 processors from programs written in FALSE and stores it in corresponding object files.
\flowgraph{\resource{FALSE\\source code} \ar[r] & \toolbox{falmips32} \ar[r] & \resource{object file}}
\seefalse\seemips\seeobject
}

\providecommand{\falmipsb}{
\toolsection{falmips64} is a compiler for the FALSE programming language targeting the MIPS64 hardware architecture.
It generates machine code for MIPS64 processors from programs written in FALSE and stores it in corresponding object files.
\flowgraph{\resource{FALSE\\source code} \ar[r] & \toolbox{falmips64} \ar[r] & \resource{object file}}
\seefalse\seemips\seeobject
}

\providecommand{\falmmix}{
\toolsection{falmmix} is a compiler for the FALSE programming language targeting the MMIX hardware architecture.
It generates machine code for MMIX processors from programs written in FALSE and stores it in corresponding object files.
\flowgraph{\resource{FALSE\\source code} \ar[r] & \toolbox{falmmix} \ar[r] & \resource{object file}}
\seefalse\seemmix\seeobject
}

\providecommand{\falorok}{
\toolsection{falor1k} is a compiler for the FALSE programming language targeting the OpenRISC 1000 hardware architecture.
It generates machine code for OpenRISC 1000 processors from programs written in FALSE and stores it in corresponding object files.
\flowgraph{\resource{FALSE\\source code} \ar[r] & \toolbox{falor1k} \ar[r] & \resource{object file}}
\seefalse\seeorok\seeobject
}

\providecommand{\falppca}{
\toolsection{falppc32} is a compiler for the FALSE programming language targeting the PowerPC hardware architecture.
It generates machine code for PowerPC processors from programs written in FALSE and stores it in corresponding object files.
The compiler generates machine code for the 32-bit operating mode defined by the PowerPC architecture.
\flowgraph{\resource{FALSE\\source code} \ar[r] & \toolbox{falppc32} \ar[r] & \resource{object file}}
\seefalse\seeppc\seeobject
}

\providecommand{\falppcb}{
\toolsection{falppc64} is a compiler for the FALSE programming language targeting the PowerPC hardware architecture.
It generates machine code for PowerPC processors from programs written in FALSE and stores it in corresponding object files.
The compiler generates machine code for the 64-bit operating mode defined by the PowerPC architecture.
\flowgraph{\resource{FALSE\\source code} \ar[r] & \toolbox{falppc64} \ar[r] & \resource{object file}}
\seefalse\seeppc\seeobject
}

\providecommand{\falrisc}{
\toolsection{falrisc} is a compiler for the FALSE programming language targeting the RISC hardware architecture.
It generates machine code for RISC processors from programs written in FALSE and stores it in corresponding object files.
\flowgraph{\resource{FALSE\\source code} \ar[r] & \toolbox{falrisc} \ar[r] & \resource{object file}}
\seefalse\seerisc\seeobject
}

\providecommand{\falwasm}{
\toolsection{falwasm} is a compiler for the FALSE programming language targeting the WebAssembly architecture.
It generates machine code for WebAssembly targets from programs written in FALSE and stores it in corresponding object files.
\flowgraph{\resource{FALSE\\source code} \ar[r] & \toolbox{falwasm} \ar[r] & \resource{object file}}
\seefalse\seewasm\seeobject
}

% Oberon tools

\providecommand{\obprint}{
\toolsection{obprint} is a pretty printer for the Oberon programming language.
It reformats the source code of Oberon modules and writes it to the standard output stream.
\flowgraph{\resource{Oberon\\source code} \ar[r] & \toolbox{obprint} \ar[r] & \resource{reformatted\\source code}}
\seeoberon
}

\providecommand{\obcheck}{
\toolsection{obcheck} is a syntactic and semantic checker for the Oberon programming language.
It just performs syntactic and semantic checks on Oberon modules and writes its diagnostic messages to the standard error stream.
In addition, it stores the interface of each module in a symbol file which is required when other modules import the module.
\flowgraph{\resource{Oberon\\source code} \ar[r] & \toolbox{obcheck} \ar[r] \ar@/l/[d] & \resource{diagnostic\\messages} \\ \variable{ECSIMPORT} \ar[ru] & \resource{symbol\\files} \ar@/r/[u]}
\seeoberon
}

\providecommand{\obdump}{
\toolsection{obdump} is a serializer for the Oberon programming language.
It dumps the complete internal representation of modules written in Oberon into an XML document.
\debuggingtool
\flowgraph{\resource{Oberon\\source code} \ar[r] & \toolbox{obdump} \ar[r] \ar@/l/[d] & \resource{internal\\representation} \\ \variable{ECSIMPORT} \ar[ru] & \resource{symbol\\files} \ar@/r/[u]}
\seeoberon
}

\providecommand{\obrun}{
\toolsection{obrun} is an interpreter for the Oberon programming language.
It processes and executes modules written in Oberon.
This tool does neither generate nor process symbol files while interpreting modules.
If a module is imported by another one, its filename has to be named before the other one in the list of command-line arguments.
\flowgraph{\resource{Oberon\\source code} \ar[r] & \toolbox{obrun} \ar@/u/[r] & \resource{input/\\output} \ar@/d/[l]}
\seeoberon
}

\providecommand{\obcpp}{
\toolsection{obcpp} is a transpiler for the Oberon programming language.
It translates programs written in Oberon into \cpp{} programs and stores them in corresponding source and header files.
In addition, it stores the interface of each module in a symbol file which is required when other modules import the module.
The same interface is provided by the generated header file which can be used in other parts of the \cpp{} program.
\flowgraph{\resource{Oberon\\source code} \ar[r] & \toolbox{obcpp} \ar[r] \ar@/l/[d] \ar[rd] & \resource{\cpp{}\\source file} \\ \variable{ECSIMPORT} \ar[ru] & \resource{symbol\\files} \ar@/r/[u] & \resource{\cpp{}\\header file}}
\seeoberon\seecpp
}

\providecommand{\obdoc}{
\toolsection{obdoc} is a generic documentation generator for the Oberon programming language.
It processes several Oberon modules and assembles all information therein into a generic documentation.
In addition, it stores the interface of each module in a symbol file which is required when other modules import the module.
\debuggingtool
\flowgraph{\resource{Oberon\\source code} \ar[r] & \toolbox{obdoc} \ar[r] \ar@/l/[d] & \resource{generic\\documentation} \\ \variable{ECSIMPORT} \ar[ru] & \resource{symbol\\files} \ar@/r/[u]}
\seeoberon\seedocumentation
}

\providecommand{\obhtml}{
\toolsection{obhtml} is an HTML documentation generator for the Oberon programming language.
It processes several Oberon modules and assembles all information therein into an HTML document.
In addition, it stores the interface of each module in a symbol file which is required when other modules import the module.
\flowgraph{\resource{Oberon\\source code} \ar[r] & \toolbox{obhtml} \ar[r] \ar@/l/[d] & \resource{HTML\\document} \\ \variable{ECSIMPORT} \ar[ru] & \resource{symbol\\files} \ar@/r/[u]}
\seeoberon\seedocumentation
}

\providecommand{\oblatex}{
\toolsection{oblatex} is a Latex documentation generator for the Oberon programming language.
It processes several Oberon modules and assembles all information therein into a Latex document.
In addition, it stores the interface of each module in a symbol file which is required when other modules import the module.
\flowgraph{\resource{Oberon\\source code} \ar[r] & \toolbox{oblatex} \ar[r] \ar@/l/[d] & \resource{Latex\\document} \\ \variable{ECSIMPORT} \ar[ru] & \resource{symbol\\files} \ar@/r/[u]}
\seeoberon\seedocumentation
}

\providecommand{\obcode}{
\toolsection{obcode} is an intermediate code generator for the Oberon programming language.
It generates intermediate code from modules written in Oberon and stores it in corresponding assembly files.
In addition, it stores the interface of each module in a symbol file which is required when other modules import the module.
Programs generated with this tool require additional runtime support that is stored in the \file{ob\-code\-run} library file.
\debuggingtool
\flowgraph{\resource{Oberon\\source code} \ar[r] & \toolbox{obcode} \ar[r] \ar@/l/[d] & \resource{intermediate\\code} \\ \variable{ECSIMPORT} \ar[ru] & \resource{symbol\\files} \ar@/r/[u]}
\seeoberon\seeassembly\seecode
}

\providecommand{\obamda}{
\toolsection{obamd16} is a compiler for the Oberon programming language targeting the AMD64 hardware architecture.
It generates machine code for AMD64 processors from modules written in Oberon and stores it in corresponding object files.
The compiler generates machine code for the 16-bit operating mode defined by the AMD64 architecture.
For debugging purposes, it also creates a debugging information file as well as an assembly file containing a listing of the generated machine code.
In addition, it stores the interface of each module in a symbol file which is required when other modules import the module.
Programs generated with this compiler require additional runtime support that is stored in the \file{ob\-amd16\-run} library file.
\flowgraph{\resource{Oberon\\source code} \ar[r] & \toolbox{obamd16} \ar[r] \ar@/l/[d] \ar[rd] & \resource{object file} \\ \variable{ECSIMPORT} \ar[ru] & \resource{symbol\\files} \ar@/r/[u] & \resource{debugging\\information}}
\seeoberon\seeassembly\seeamd\seeobject\seedebugging
}

\providecommand{\obamdb}{
\toolsection{obamd32} is a compiler for the Oberon programming language targeting the AMD64 hardware architecture.
It generates machine code for AMD64 processors from modules written in Oberon and stores it in corresponding object files.
The compiler generates machine code for the 32-bit operating mode defined by the AMD64 architecture.
For debugging purposes, it also creates a debugging information file as well as an assembly file containing a listing of the generated machine code.
In addition, it stores the interface of each module in a symbol file which is required when other modules import the module.
Programs generated with this compiler require additional runtime support that is stored in the \file{ob\-amd32\-run} library file.
\flowgraph{\resource{Oberon\\source code} \ar[r] & \toolbox{obamd32} \ar[r] \ar@/l/[d] \ar[rd] & \resource{object file} \\ \variable{ECSIMPORT} \ar[ru] & \resource{symbol\\files} \ar@/r/[u] & \resource{debugging\\information}}
\seeoberon\seeassembly\seeamd\seeobject\seedebugging
}

\providecommand{\obamdc}{
\toolsection{obamd64} is a compiler for the Oberon programming language targeting the AMD64 hardware architecture.
It generates machine code for AMD64 processors from modules written in Oberon and stores it in corresponding object files.
The compiler generates machine code for the 64-bit operating mode defined by the AMD64 architecture.
For debugging purposes, it also creates a debugging information file as well as an assembly file containing a listing of the generated machine code.
In addition, it stores the interface of each module in a symbol file which is required when other modules import the module.
Programs generated with this compiler require additional runtime support that is stored in the \file{ob\-amd64\-run} library file.
\flowgraph{\resource{Oberon\\source code} \ar[r] & \toolbox{obamd64} \ar[r] \ar@/l/[d] \ar[rd] & \resource{object file} \\ \variable{ECSIMPORT} \ar[ru] & \resource{symbol\\files} \ar@/r/[u] & \resource{debugging\\information}}
\seeoberon\seeassembly\seeamd\seeobject\seedebugging
}

\providecommand{\obarma}{
\toolsection{obarma32} is a compiler for the Oberon programming language targeting the ARM hardware architecture.
It generates machine code for ARM processors executing A32 instructions from modules written in Oberon and stores it in corresponding object files.
For debugging purposes, it also creates a debugging information file as well as an assembly file containing a listing of the generated machine code.
In addition, it stores the interface of each module in a symbol file which is required when other modules import the module.
Programs generated with this compiler require additional runtime support that is stored in the \file{ob\-arma32\-run} library file.
\flowgraph{\resource{Oberon\\source code} \ar[r] & \toolbox{obarma32} \ar[r] \ar@/l/[d] \ar[rd] & \resource{object file} \\ \variable{ECSIMPORT} \ar[ru] & \resource{symbol\\files} \ar@/r/[u] & \resource{debugging\\information}}
\seeoberon\seeassembly\seearm\seeobject\seedebugging
}

\providecommand{\obarmb}{
\toolsection{obarma64} is a compiler for the Oberon programming language targeting the ARM hardware architecture.
It generates machine code for ARM processors executing A64 instructions from modules written in Oberon and stores it in corresponding object files.
For debugging purposes, it also creates a debugging information file as well as an assembly file containing a listing of the generated machine code.
In addition, it stores the interface of each module in a symbol file which is required when other modules import the module.
Programs generated with this compiler require additional runtime support that is stored in the \file{ob\-arma64\-run} library file.
\flowgraph{\resource{Oberon\\source code} \ar[r] & \toolbox{obarma64} \ar[r] \ar@/l/[d] \ar[rd] & \resource{object file} \\ \variable{ECSIMPORT} \ar[ru] & \resource{symbol\\files} \ar@/r/[u] & \resource{debugging\\information}}
\seeoberon\seeassembly\seearm\seeobject\seedebugging
}

\providecommand{\obarmc}{
\toolsection{obarmt32} is a compiler for the Oberon programming language targeting the ARM hardware architecture.
It generates machine code for ARM processors without floating-point extension executing T32 instructions from modules written in Oberon and stores it in corresponding object files.
For debugging purposes, it also creates a debugging information file as well as an assembly file containing a listing of the generated machine code.
In addition, it stores the interface of each module in a symbol file which is required when other modules import the module.
Programs generated with this compiler require additional runtime support that is stored in the \file{ob\-armt32\-run} library file.
\flowgraph{\resource{Oberon\\source code} \ar[r] & \toolbox{obarmt32} \ar[r] \ar@/l/[d] \ar[rd] & \resource{object file} \\ \variable{ECSIMPORT} \ar[ru] & \resource{symbol\\files} \ar@/r/[u] & \resource{debugging\\information}}
\seeoberon\seeassembly\seearm\seeobject\seedebugging
}

\providecommand{\obarmcfpe}{
\toolsection{obarmt32fpe} is a compiler for the Oberon programming language targeting the ARM hardware architecture.
It generates machine code for ARM processors with floating-point extension executing T32 instructions from modules written in Oberon and stores it in corresponding object files.
For debugging purposes, it also creates a debugging information file as well as an assembly file containing a listing of the generated machine code.
In addition, it stores the interface of each module in a symbol file which is required when other modules import the module.
Programs generated with this compiler require additional runtime support that is stored in the \file{ob\-armt32\-fpe\-run} library file.
\flowgraph{\resource{Oberon\\source code} \ar[r] & \toolbox{obarmt32fpe} \ar[r] \ar@/l/[d] \ar[rd] & \resource{object file} \\ \variable{ECSIMPORT} \ar[ru] & \resource{symbol\\files} \ar@/r/[u] & \resource{debugging\\information}}
\seeoberon\seeassembly\seearm\seeobject\seedebugging
}

\providecommand{\obavr}{
\toolsection{obavr} is a compiler for the Oberon programming language targeting the AVR hardware architecture.
It generates machine code for AVR processors from modules written in Oberon and stores it in corresponding object files.
For debugging purposes, it also creates a debugging information file as well as an assembly file containing a listing of the generated machine code.
In addition, it stores the interface of each module in a symbol file which is required when other modules import the module.
Programs generated with this compiler require additional runtime support that is stored in the \file{ob\-avr\-run} library file.
\flowgraph{\resource{Oberon\\source code} \ar[r] & \toolbox{obavr} \ar[r] \ar@/l/[d] \ar[rd] & \resource{object file} \\ \variable{ECSIMPORT} \ar[ru] & \resource{symbol\\files} \ar@/r/[u] & \resource{debugging\\information}}
\seeoberon\seeassembly\seeavr\seeobject\seedebugging
}

\providecommand{\obavrtt}{
\toolsection{obavr32} is a compiler for the Oberon programming language targeting the AVR32 hardware architecture.
It generates machine code for AVR32 processors from modules written in Oberon and stores it in corresponding object files.
For debugging purposes, it also creates a debugging information file as well as an assembly file containing a listing of the generated machine code.
In addition, it stores the interface of each module in a symbol file which is required when other modules import the module.
Programs generated with this compiler require additional runtime support that is stored in the \file{ob\-avr32\-run} library file.
\flowgraph{\resource{Oberon\\source code} \ar[r] & \toolbox{obavr32} \ar[r] \ar@/l/[d] \ar[rd] & \resource{object file} \\ \variable{ECSIMPORT} \ar[ru] & \resource{symbol\\files} \ar@/r/[u] & \resource{debugging\\information}}
\seeoberon\seeassembly\seeavrtt\seeobject\seedebugging
}

\providecommand{\obmabk}{
\toolsection{obm68k} is a compiler for the Oberon programming language targeting the M68000 hardware architecture.
It generates machine code for M68000 processors from modules written in Oberon and stores it in corresponding object files.
For debugging purposes, it also creates a debugging information file as well as an assembly file containing a listing of the generated machine code.
In addition, it stores the interface of each module in a symbol file which is required when other modules import the module.
Programs generated with this compiler require additional runtime support that is stored in the \file{ob\-m68k\-run} library file.
\flowgraph{\resource{Oberon\\source code} \ar[r] & \toolbox{obm68k} \ar[r] \ar@/l/[d] \ar[rd] & \resource{object file} \\ \variable{ECSIMPORT} \ar[ru] & \resource{symbol\\files} \ar@/r/[u] & \resource{debugging\\information}}
\seeoberon\seeassembly\seemabk\seeobject\seedebugging
}

\providecommand{\obmibl}{
\toolsection{obmibl} is a compiler for the Oberon programming language targeting the MicroBlaze hardware architecture.
It generates machine code for MicroBlaze processors from modules written in Oberon and stores it in corresponding object files.
For debugging purposes, it also creates a debugging information file as well as an assembly file containing a listing of the generated machine code.
In addition, it stores the interface of each module in a symbol file which is required when other modules import the module.
Programs generated with this compiler require additional runtime support that is stored in the \file{ob\-mibl\-run} library file.
\flowgraph{\resource{Oberon\\source code} \ar[r] & \toolbox{obmibl} \ar[r] \ar@/l/[d] \ar[rd] & \resource{object file} \\ \variable{ECSIMPORT} \ar[ru] & \resource{symbol\\files} \ar@/r/[u] & \resource{debugging\\information}}
\seeoberon\seeassembly\seemibl\seeobject\seedebugging
}

\providecommand{\obmipsa}{
\toolsection{obmips32} is a compiler for the Oberon programming language targeting the MIPS32 hardware architecture.
It generates machine code for MIPS32 processors from modules written in Oberon and stores it in corresponding object files.
For debugging purposes, it also creates a debugging information file as well as an assembly file containing a listing of the generated machine code.
In addition, it stores the interface of each module in a symbol file which is required when other modules import the module.
Programs generated with this compiler require additional runtime support that is stored in the \file{ob\-mips32\-run} library file.
\flowgraph{\resource{Oberon\\source code} \ar[r] & \toolbox{obmips32} \ar[r] \ar@/l/[d] \ar[rd] & \resource{object file} \\ \variable{ECSIMPORT} \ar[ru] & \resource{symbol\\files} \ar@/r/[u] & \resource{debugging\\information}}
\seeoberon\seeassembly\seemips\seeobject\seedebugging
}

\providecommand{\obmipsb}{
\toolsection{obmips64} is a compiler for the Oberon programming language targeting the MIPS64 hardware architecture.
It generates machine code for MIPS64 processors from modules written in Oberon and stores it in corresponding object files.
For debugging purposes, it also creates a debugging information file as well as an assembly file containing a listing of the generated machine code.
In addition, it stores the interface of each module in a symbol file which is required when other modules import the module.
Programs generated with this compiler require additional runtime support that is stored in the \file{ob\-mips64\-run} library file.
\flowgraph{\resource{Oberon\\source code} \ar[r] & \toolbox{obmips64} \ar[r] \ar@/l/[d] \ar[rd] & \resource{object file} \\ \variable{ECSIMPORT} \ar[ru] & \resource{symbol\\files} \ar@/r/[u] & \resource{debugging\\information}}
\seeoberon\seeassembly\seemips\seeobject\seedebugging
}

\providecommand{\obmmix}{
\toolsection{obmmix} is a compiler for the Oberon programming language targeting the MMIX hardware architecture.
It generates machine code for MMIX processors from modules written in Oberon and stores it in corresponding object files.
For debugging purposes, it also creates a debugging information file as well as an assembly file containing a listing of the generated machine code.
In addition, it stores the interface of each module in a symbol file which is required when other modules import the module.
Programs generated with this compiler require additional runtime support that is stored in the \file{ob\-mmix\-run} library file.
\flowgraph{\resource{Oberon\\source code} \ar[r] & \toolbox{obmmix} \ar[r] \ar@/l/[d] \ar[rd] & \resource{object file} \\ \variable{ECSIMPORT} \ar[ru] & \resource{symbol\\files} \ar@/r/[u] & \resource{debugging\\information}}
\seeoberon\seeassembly\seemmix\seeobject\seedebugging
}

\providecommand{\oborok}{
\toolsection{obor1k} is a compiler for the Oberon programming language targeting the OpenRISC 1000 hardware architecture.
It generates machine code for OpenRISC 1000 processors from modules written in Oberon and stores it in corresponding object files.
For debugging purposes, it also creates a debugging information file as well as an assembly file containing a listing of the generated machine code.
In addition, it stores the interface of each module in a symbol file which is required when other modules import the module.
Programs generated with this compiler require additional runtime support that is stored in the \file{ob\-or1k\-run} library file.
\flowgraph{\resource{Oberon\\source code} \ar[r] & \toolbox{obor1k} \ar[r] \ar@/l/[d] \ar[rd] & \resource{object file} \\ \variable{ECSIMPORT} \ar[ru] & \resource{symbol\\files} \ar@/r/[u] & \resource{debugging\\information}}
\seeoberon\seeassembly\seeorok\seeobject\seedebugging
}

\providecommand{\obppca}{
\toolsection{obppc32} is a compiler for the Oberon programming language targeting the PowerPC hardware architecture.
It generates machine code for PowerPC processors from modules written in Oberon and stores it in corresponding object files.
The compiler generates machine code for the 32-bit operating mode defined by the PowerPC architecture.
For debugging purposes, it also creates a debugging information file as well as an assembly file containing a listing of the generated machine code.
In addition, it stores the interface of each module in a symbol file which is required when other modules import the module.
Programs generated with this compiler require additional runtime support that is stored in the \file{ob\-ppc32\-run} library file.
\flowgraph{\resource{Oberon\\source code} \ar[r] & \toolbox{obppc32} \ar[r] \ar@/l/[d] \ar[rd] & \resource{object file} \\ \variable{ECSIMPORT} \ar[ru] & \resource{symbol\\files} \ar@/r/[u] & \resource{debugging\\information}}
\seeoberon\seeassembly\seeppc\seeobject\seedebugging
}

\providecommand{\obppcb}{
\toolsection{obppc64} is a compiler for the Oberon programming language targeting the PowerPC hardware architecture.
It generates machine code for PowerPC processors from modules written in Oberon and stores it in corresponding object files.
The compiler generates machine code for the 64-bit operating mode defined by the PowerPC architecture.
For debugging purposes, it also creates a debugging information file as well as an assembly file containing a listing of the generated machine code.
In addition, it stores the interface of each module in a symbol file which is required when other modules import the module.
Programs generated with this compiler require additional runtime support that is stored in the \file{ob\-ppc64\-run} library file.
\flowgraph{\resource{Oberon\\source code} \ar[r] & \toolbox{obppc64} \ar[r] \ar@/l/[d] \ar[rd] & \resource{object file} \\ \variable{ECSIMPORT} \ar[ru] & \resource{symbol\\files} \ar@/r/[u] & \resource{debugging\\information}}
\seeoberon\seeassembly\seeppc\seeobject\seedebugging
}

\providecommand{\obrisc}{
\toolsection{obrisc} is a compiler for the Oberon programming language targeting the RISC hardware architecture.
It generates machine code for RISC processors from modules written in Oberon and stores it in corresponding object files.
For debugging purposes, it also creates a debugging information file as well as an assembly file containing a listing of the generated machine code.
In addition, it stores the interface of each module in a symbol file which is required when other modules import the module.
Programs generated with this compiler require additional runtime support that is stored in the \file{ob\-risc\-run} library file.
\flowgraph{\resource{Oberon\\source code} \ar[r] & \toolbox{obrisc} \ar[r] \ar@/l/[d] \ar[rd] & \resource{object file} \\ \variable{ECSIMPORT} \ar[ru] & \resource{symbol\\files} \ar@/r/[u] & \resource{debugging\\information}}
\seeoberon\seeassembly\seerisc\seeobject\seedebugging
}

\providecommand{\obwasm}{
\toolsection{obwasm} is a compiler for the Oberon programming language targeting the WebAssembly architecture.
It generates machine code for WebAssembly targets from modules written in Oberon and stores it in corresponding object files.
For debugging purposes, it also creates a debugging information file as well as an assembly file containing a listing of the generated machine code.
In addition, it stores the interface of each module in a symbol file which is required when other modules import the module.
Programs generated with this compiler require additional runtime support that is stored in the \file{ob\-wasm\-run} library file.
\flowgraph{\resource{Oberon\\source code} \ar[r] & \toolbox{obwasm} \ar[r] \ar@/l/[d] \ar[rd] & \resource{object file} \\ \variable{ECSIMPORT} \ar[ru] & \resource{symbol\\files} \ar@/r/[u] & \resource{debugging\\information}}
\seeoberon\seeassembly\seewasm\seeobject\seedebugging
}

% converter tools

\providecommand{\dbgdwarf}{
\toolsection{dbgdwarf} is a DWARF debugging information converter tool.
It converts debugging information into the DWARF debugging data format and stores it in corresponding object files~\cite{dwarffile}.
The resulting debugging object files can be combined with runtime support that creates Executable and Linking Format (ELF) files~\cite{elffile}.
\flowgraph{\resource{debugging\\information} \ar[r] & \toolbox{dbgdwarf} \ar[r] & \resource{debugging\\object file}}
\seeobject\seedebugging
}

% assembler tools

\providecommand{\asmprint}{
\toolsection{asmprint} is a pretty printer for generic assembly code.
It reformats generic assembly code and writes it to the standard output stream.
\flowgraph{\resource{generic assembly\\source code} \ar[r] & \toolbox{asmprint} \ar[r] & \resource{reformatted\\source code}}
\seeassembly
}

\providecommand{\amdaasm}{
\toolsection{amd16asm} is an assembler for the AMD64 hardware architecture.
It translates assembly code into machine code for AMD64 processors and stores it in corresponding object files.
By default, the assembler generates machine code for the 16-bit operating mode defined by the AMD64 architecture.
\flowgraph{\resource{AMD16 assembly\\source code} \ar[r] & \toolbox{amd16asm} \ar[r] & \resource{object file}}
\seeassembly\seeamd\seeobject
}

\providecommand{\amdadism}{
\toolsection{amd16dism} is a disassembler for the AMD64 hardware architecture.
It translates machine code from object files targeting AMD64 processors into assembly code and writes it to the standard output stream.
It assumes that the machine code was generated for the 16-bit operating mode defined by the AMD64 architecture.
\flowgraph{\resource{object file} \ar[r] & \toolbox{amd16dism} \ar[r] & \resource{disassembly\\listing}}
\seeassembly\seeamd\seeobject
}

\providecommand{\amdbasm}{
\toolsection{amd32asm} is an assembler for the AMD64 hardware architecture.
It translates assembly code into machine code for AMD64 processors and stores it in corresponding object files.
By default, the assembler generates machine code for the 32-bit operating mode defined by the AMD64 architecture.
\flowgraph{\resource{AMD32 assembly\\source code} \ar[r] & \toolbox{amd32asm} \ar[r] & \resource{object file}}
\seeassembly\seeamd\seeobject
}

\providecommand{\amdbdism}{
\toolsection{amd32dism} is a disassembler for the AMD64 hardware architecture.
It translates machine code from object files targeting AMD64 processors into assembly code and writes it to the standard output stream.
It assumes that the machine code was generated for the 32-bit operating mode defined by the AMD64 architecture.
\flowgraph{\resource{object file} \ar[r] & \toolbox{amd32dism} \ar[r] & \resource{disassembly\\listing}}
\seeassembly\seeamd\seeobject
}

\providecommand{\amdcasm}{
\toolsection{amd64asm} is an assembler for the AMD64 hardware architecture.
It translates assembly code into machine code for AMD64 processors and stores it in corresponding object files.
By default, the assembler generates machine code for the 64-bit operating mode defined by the AMD64 architecture.
\flowgraph{\resource{AMD64 assembly\\source code} \ar[r] & \toolbox{amd64asm} \ar[r] & \resource{object file}}
\seeassembly\seeamd\seeobject
}

\providecommand{\amdcdism}{
\toolsection{amd64dism} is a disassembler for the AMD64 hardware architecture.
It translates machine code from object files targeting AMD64 processors into assembly code and writes it to the standard output stream.
It assumes that the machine code was generated for the 64-bit operating mode defined by the AMD64 architecture.
\flowgraph{\resource{object file} \ar[r] & \toolbox{amd64dism} \ar[r] & \resource{disassembly\\listing}}
\seeassembly\seeamd\seeobject
}

\providecommand{\armaasm}{
\toolsection{arma32asm} is an assembler for the ARM hardware architecture.
It translates assembly code into machine code for ARM processors executing A32 instructions and stores it in corresponding object files.
\flowgraph{\resource{ARM A32 assembly\\source code} \ar[r] & \toolbox{arma32asm} \ar[r] & \resource{object file}}
\seeassembly\seearm\seeobject
}

\providecommand{\armadism}{
\toolsection{arma32dism} is a disassembler for the ARM hardware architecture.
It translates machine code from object files targeting ARM processors executing A32 instructions into assembly code and writes it to the standard output stream.
\flowgraph{\resource{object file} \ar[r] & \toolbox{arma32dism} \ar[r] & \resource{disassembly\\listing}}
\seeassembly\seearm\seeobject
}

\providecommand{\armbasm}{
\toolsection{arma64asm} is an assembler for the ARM hardware architecture.
It translates assembly code into machine code for ARM processors executing A64 instructions and stores it in corresponding object files.
\flowgraph{\resource{ARM A64 assembly\\source code} \ar[r] & \toolbox{arma64asm} \ar[r] & \resource{object file}}
\seeassembly\seearm\seeobject
}

\providecommand{\armbdism}{
\toolsection{arma64dism} is a disassembler for the ARM hardware architecture.
It translates machine code from object files targeting ARM processors executing A64 instructions into assembly code and writes it to the standard output stream.
\flowgraph{\resource{object file} \ar[r] & \toolbox{arma64dism} \ar[r] & \resource{disassembly\\listing}}
\seeassembly\seearm\seeobject
}

\providecommand{\armcasm}{
\toolsection{armt32asm} is an assembler for the ARM hardware architecture.
It translates assembly code into machine code for ARM processors executing T32 instructions and stores it in corresponding object files.
\flowgraph{\resource{ARM T32 assembly\\source code} \ar[r] & \toolbox{armt32asm} \ar[r] & \resource{object file}}
\seeassembly\seearm\seeobject
}

\providecommand{\armcdism}{
\toolsection{armt32dism} is a disassembler for the ARM hardware architecture.
It translates machine code from object files targeting ARM processors executing T32 instructions into assembly code and writes it to the standard output stream.
\flowgraph{\resource{object file} \ar[r] & \toolbox{armt32dism} \ar[r] & \resource{disassembly\\listing}}
\seeassembly\seearm\seeobject
}

\providecommand{\avrasm}{
\toolsection{avrasm} is an assembler for the AVR hardware architecture.
It translates assembly code into machine code for AVR processors and stores it in corresponding object files.
The identifiers \texttt{RXL}, \texttt{RXH}, \texttt{RYL}, \texttt{RYH}, \texttt{RZL}, and \texttt{RZH} are predefined and name the corresponding registers.
The identifiers \texttt{SPL} and \texttt{SPH} are also predefined and evaluate to the address of the corresponding registers.
\flowgraph{\resource{AVR assembly\\source code} \ar[r] & \toolbox{avrasm} \ar[r] & \resource{object file}}
\seeassembly\seeavr\seeobject
}

\providecommand{\avrdism}{
\toolsection{avrdism} is a disassembler for the AVR hardware architecture.
It translates machine code from object files targeting AVR processors into assembly code and writes it to the standard output stream.
\flowgraph{\resource{object file} \ar[r] & \toolbox{avrdism} \ar[r] & \resource{disassembly\\listing}}
\seeassembly\seeavr\seeobject
}

\providecommand{\avrttasm}{
\toolsection{avr32asm} is an assembler for the AVR32 hardware architecture.
It translates assembly code into machine code for AVR32 processors and stores it in corresponding object files.
\flowgraph{\resource{AVR32 assembly\\source code} \ar[r] & \toolbox{avr32asm} \ar[r] & \resource{object file}}
\seeassembly\seeavrtt\seeobject
}

\providecommand{\avrttdism}{
\toolsection{avr32dism} is a disassembler for the AVR32 hardware architecture.
It translates machine code from object files targeting AVR32 processors into assembly code and writes it to the standard output stream.
\flowgraph{\resource{object file} \ar[r] & \toolbox{avr32dism} \ar[r] & \resource{disassembly\\listing}}
\seeassembly\seeavrtt\seeobject
}

\providecommand{\mabkasm}{
\toolsection{m68kasm} is an assembler for the M68000 hardware architecture.
It translates assembly code into machine code for M68000 processors and stores it in corresponding object files.
\flowgraph{\resource{68000 assembly\\source code} \ar[r] & \toolbox{m68kasm} \ar[r] & \resource{object file}}
\seeassembly\seemabk\seeobject
}

\providecommand{\mabkdism}{
\toolsection{m68kdism} is a disassembler for the M68000 hardware architecture.
It translates machine code from object files targeting M68000 processors into assembly code and writes it to the standard output stream.
\flowgraph{\resource{object file} \ar[r] & \toolbox{m68kdism} \ar[r] & \resource{disassembly\\listing}}
\seeassembly\seemabk\seeobject
}

\providecommand{\miblasm}{
\toolsection{miblasm} is an assembler for the MicroBlaze hardware architecture.
It translates assembly code into machine code for MicroBlaze processors and stores it in corresponding object files.
\flowgraph{\resource{MicroBlaze assembly\\source code} \ar[r] & \toolbox{miblasm} \ar[r] & \resource{object file}}
\seeassembly\seemibl\seeobject
}

\providecommand{\mibldism}{
\toolsection{mibldism} is a disassembler for the MicroBlaze hardware architecture.
It translates machine code from object files targeting MicroBlaze processors into assembly code and writes it to the standard output stream.
\flowgraph{\resource{object file} \ar[r] & \toolbox{mibldism} \ar[r] & \resource{disassembly\\listing}}
\seeassembly\seemibl\seeobject
}

\providecommand{\mipsaasm}{
\toolsection{mips32asm} is an assembler for the MIPS32 hardware architecture.
It translates assembly code into machine code for MIPS32 processors and stores it in corresponding object files.
\flowgraph{\resource{MIPS32 assembly\\source code} \ar[r] & \toolbox{mips32asm} \ar[r] & \resource{object file}}
\seeassembly\seemips\seeobject
}

\providecommand{\mipsadism}{
\toolsection{mips32dism} is a disassembler for the MIPS32 hardware architecture.
It translates machine code from object files targeting MIPS32 processors into assembly code and writes it to the standard output stream.
\flowgraph{\resource{object file} \ar[r] & \toolbox{mips32dism} \ar[r] & \resource{disassembly\\listing}}
\seeassembly\seemips\seeobject
}

\providecommand{\mipsbasm}{
\toolsection{mips64asm} is an assembler for the MIPS64 hardware architecture.
It translates assembly code into machine code for MIPS64 processors and stores it in corresponding object files.
\flowgraph{\resource{MIPS64 assembly\\source code} \ar[r] & \toolbox{mips64asm} \ar[r] & \resource{object file}}
\seeassembly\seemips\seeobject
}

\providecommand{\mipsbdism}{
\toolsection{mips64dism} is a disassembler for the MIPS64 hardware architecture.
It translates machine code from object files targeting MIPS64 processors into assembly code and writes it to the standard output stream.
\flowgraph{\resource{object file} \ar[r] & \toolbox{mips64dism} \ar[r] & \resource{disassembly\\listing}}
\seeassembly\seemips\seeobject
}

\providecommand{\mmixasm}{
\toolsection{mmixasm} is an assembler for the MMIX hardware architecture.
It translates assembly code into machine code for MMIX processors and stores it in corresponding object files.
The names of all special registers are predefined and evaluate to the corresponding number.
\flowgraph{\resource{MMIX assembly\\source code} \ar[r] & \toolbox{mmixasm} \ar[r] & \resource{object file}}
\seeassembly\seemmix\seeobject
}

\providecommand{\mmixdism}{
\toolsection{mmixdism} is a disassembler for the MMIX hardware architecture.
It translates machine code from object files targeting MMIX processors into assembly code and writes it to the standard output stream.
\flowgraph{\resource{object file} \ar[r] & \toolbox{mmixdism} \ar[r] & \resource{disassembly\\listing}}
\seeassembly\seemmix\seeobject
}

\providecommand{\orokasm}{
\toolsection{or1kasm} is an assembler for the OpenRISC 1000 hardware architecture.
It translates assembly code into machine code for OpenRISC 1000 processors and stores it in corresponding object files.
\flowgraph{\resource{OpenRISC 1000 assembly\\source code} \ar[r] & \toolbox{or1kasm} \ar[r] & \resource{object file}}
\seeassembly\seeorok\seeobject
}

\providecommand{\orokdism}{
\toolsection{or1kdism} is a disassembler for the OpenRISC 1000 hardware architecture.
It translates machine code from object files targeting OpenRISC 1000 processors into assembly code and writes it to the standard output stream.
\flowgraph{\resource{object file} \ar[r] & \toolbox{or1kdism} \ar[r] & \resource{disassembly\\listing}}
\seeassembly\seeorok\seeobject
}

\providecommand{\ppcaasm}{
\toolsection{ppc32asm} is an assembler for the PowerPC hardware architecture.
It translates assembly code into machine code for PowerPC processors and stores it in corresponding object files.
By default, the assembler generates machine code for the 32-bit operating mode defined by the PowerPC architecture.
\flowgraph{\resource{PowerPC assembly\\source code} \ar[r] & \toolbox{ppc32asm} \ar[r] & \resource{object file}}
\seeassembly\seeppc\seeobject
}

\providecommand{\ppcadism}{
\toolsection{ppc32dism} is a disassembler for the PowerPC hardware architecture.
It translates machine code from object files targeting PowerPC processors into assembly code and writes it to the standard output stream.
It assumes that the machine code was generated for the 32-bit operating mode defined by the PowerPC architecture.
\flowgraph{\resource{object file} \ar[r] & \toolbox{ppc32dism} \ar[r] & \resource{disassembly\\listing}}
\seeassembly\seeppc\seeobject
}

\providecommand{\ppcbasm}{
\toolsection{ppc64asm} is an assembler for the PowerPC hardware architecture.
It translates assembly code into machine code for PowerPC processors and stores it in corresponding object files.
By default, the assembler generates machine code for the 64-bit operating mode defined by the PowerPC architecture.
\flowgraph{\resource{PowerPC assembly\\source code} \ar[r] & \toolbox{ppc64asm} \ar[r] & \resource{object file}}
\seeassembly\seeppc\seeobject
}

\providecommand{\ppcbdism}{
\toolsection{ppc64dism} is a disassembler for the PowerPC hardware architecture.
It translates machine code from object files targeting PowerPC processors into assembly code and writes it to the standard output stream.
It assumes that the machine code was generated for the 64-bit operating mode defined by the PowerPC architecture.
\flowgraph{\resource{object file} \ar[r] & \toolbox{ppc64dism} \ar[r] & \resource{disassembly\\listing}}
\seeassembly\seeppc\seeobject
}

\providecommand{\riscasm}{
\toolsection{riscasm} is an assembler for the RISC hardware architecture.
It translates assembly code into machine code for RISC processors and stores it in corresponding object files.
The names of all special registers are predefined and evaluate to the corresponding number.
\flowgraph{\resource{RISC assembly\\source code} \ar[r] & \toolbox{riscasm} \ar[r] & \resource{object file}}
\seeassembly\seerisc\seeobject
}

\providecommand{\riscdism}{
\toolsection{riscdism} is a disassembler for the RISC hardware architecture.
It translates machine code from object files targeting RISC processors into assembly code and writes it to the standard output stream.
\flowgraph{\resource{object file} \ar[r] & \toolbox{riscdism} \ar[r] & \resource{disassembly\\listing}}
\seeassembly\seerisc\seeobject
}

\providecommand{\wasmasm}{
\toolsection{wasmasm} is an assembler for the WebAssembly architecture.
It translates assembly code into machine code for WebAssembly targets and stores it in corresponding object files.
The names of all special registers are predefined and evaluate to the corresponding number.
\flowgraph{\resource{WebAssembly assembly\\source code} \ar[r] & \toolbox{wasmasm} \ar[r] & \resource{object file}}
\seeassembly\seewasm\seeobject
}

\providecommand{\wasmdism}{
\toolsection{wasmdism} is a disassembler for the WebAssembly architecture.
It translates machine code from object files targeting WebAssembly targets into assembly code and writes it to the standard output stream.
\flowgraph{\resource{object file} \ar[r] & \toolbox{wasmdism} \ar[r] & \resource{disassembly\\listing}}
\seeassembly\seewasm\seeobject
}

% linker tools

\providecommand{\linklib}{
\toolsection{linklib} is an object file combiner.
It creates a static library file by combining all object files given to it into a single one.
\flowgraph{\resource{object files} \ar[r] & \toolbox{linklib} \ar[r] & \resource{library file}}
\seeobject
}

\providecommand{\linkbin}{
\toolsection{linkbin} is a linker for plain binary files.
It links all object files given to it into a single image and stores it in a binary file that begins with the first linked section.
It also creates a map file that lists the address, type, name and size of all used sections.
The filename extension of the resulting binary file can be specified by putting it into a constant data section called \texttt{\_extension}.
\flowgraph{\resource{object files} \ar[r] & \toolbox{linkbin} \ar[r] \ar[d] & \resource{binary file} \\ & \resource{map file}}
\seeobject
}

\providecommand{\linkmem}{
\toolsection{linkmem} is a linker for plain binary files partitioned into random-access and read-only memory.
It links all object files given to it into two distinct images, one for data sections and one for code and constant data sections, and stores each image in a binary file that begins with the first linked section of the corresponding type.
It also creates a map file that lists the address, type, name and size of all used sections.
\flowgraph{\resource{object files} \ar[r] & \toolbox{linkmem} \ar[r] \ar[d] & \resource{RAM file/\\ROM file} \\ & \resource{map file}}
\seeobject
}

\providecommand{\linkprg}{
\toolsection{linkprg} is a linker for GEMDOS executable files.
It links all object files given to it into a single image and stores the image in an Atari GEMDOS executable file~\cite{gemdosfile}.
It also creates a map file that lists the address relative to the text segment, type, name and size of all used sections.
The filename extension of the resulting executable file can be specified by putting it into a constant data section called \texttt{\_extension}.
The GEMDOS executable file format requires all patch patterns of absolute link patches to consist of four full bitmasks with descending offsets.
\flowgraph{\resource{object files} \ar[r] & \toolbox{linkprg} \ar[r] \ar[d] & \resource{executable file} \\ & \resource{map file}}
\seeobject
}

\providecommand{\linkhex}{
\toolsection{linkhex} is a linker for Intel HEX files.
It links all code sections of the object files given to it into single image and stores the image in an Intel HEX file~\cite{hexfile} that begins with the first linked section.
It also creates a map file that lists the address, type, name and size of all used sections.
\flowgraph{\resource{object files} \ar[r] & \toolbox{linkhex} \ar[r] \ar[d] & \resource{HEX file} \\ & \resource{map file}}
\seeobject
}

\providecommand{\mapsearch}{
\toolsection{mapsearch} is a debugging tool.
It searches map files generated by linker tools for the name of a binary section that encompasses a memory address read from the standard input stream.
If additionally provided with one or more object files, it also stores an excerpt thereof in a separate object file called map search result which only contains the identified binary section for disassembling purposes.
\flowgraph{& \resource{map files/\\object files} \ar[d] \\ \resource{memory\\address} \ar[r] & \toolbox{mapsearch} \ar[r] \ar[d] & \resource{section name/\\relative offset} \\ & \resource{object file\\excerpt}}
\seeobject
}

\renewcommand{\seearm}{}

\startchapter{ARM}{ARM Hardware Architecture Support}{arm}
{This \documentation{} describes how the \ecs{} supports the ARM hardware architecture.
This includes information about the assemblers, disassemblers, and the various compilers featured by the \ecs{} as well as the interoperability between these tools.}

\section{Introduction}

The \ecs{} features various compilers, assemblers, and disassemblers that target the ARM hardware architecture by ARM Limited.
Figure~\ref{fig:armdataflow} shows the data flow in-between these tools.

\begin{figure}
\flowgraph{
\resource{intermediate\\code} \ar[d] & & \resource{assembly\\source code} \ar[d] \\
\converter{A32/A64/T32\\Generator} \ar[r] \ar[rd] \ar[d] & \resource{assembly\\listing} \ar[r] & \converter{A32/A64/T32\\Assembler} \ar[ld] \\
\resource{debugging\\information} & \resource{object file} \ar[d] \\
& \converter{A32/A64/T32\\Disassembler} \ar[d] \\
& \resource{disassembly\\listing} \\
}\caption{Data flow within the tools targeting the ARM architecture}
\label{fig:armdataflow}
\end{figure}

All compilers targeting the ARM architecture translate their programs using an intermediate code representation.
The A32, A64 and T32 generators are able to translate the intermediate code representation of a program into machine code for ARM processors executing the respective instruction set.
All of them store the resulting binary code and data in so-called object files.
Additionally, all generators are able to create an assembly code listing of the machine code for debugging purposes.
This assembly code listing can also be processed by the corresponding assembler yielding exactly the same object file.
The corresponding disassembler is able to open object files and print a human-readable disassembly listing of their contents.
\seeobject\seecode

\section{Instruction Sets}

Tools targeting the ARM architecture support the instruction sets listed in Tables~\ref{tab:arma32set} to~\ref{tab:armset} and use the same assembly syntax as predefined by the ARM architecture~\cite{arm:instructionset}.
The only exception are immediate values which are not prefixed by a number sign.
\seeassembly

\instructionset{arma32}{Supported ARM A32 instruction set}{5}{6}
\instructionset{arma64}{Supported ARM A64 instruction set}{5}{6}
\instructionset{arm}{Supported ARM T32 instruction set}{5}{6}

The A64 and T32 assemblers allow users to temporarily switch to the A32 instruction set by passing 32 as operand to the bit mode directive.
The original instruction set can be restored using bit modes 64 and 16 respectively.

\section{Calling Convention}\index{Calling convention!of ARM}

The machine code generators and runtime support for the ARM architecture as provided by the \ecs{} use the following calling convention in order to enable interoperability.

\subsection{Stack Operations}

Arguments for functions are in general passed using the stack according to the intermediate code specification.
See \Documentation{}~\documentationref{code}{Intermediate Code Representation} for more information about the role of the stack.
Function arguments are pushed on the stack in reverse order and cleaned by the caller.

\subsection{Floating-Point Support}

The ARM architecture optionally supports floating-point operations.
The A32 and A64 generators are able to generate native floating-point operations for processors that do support them.

\subsection{Register Mapping}

The special-purpose registers defined by the intermediate code representation are mapped to their corresponding physical registers in the following way:

\begin{itemize}

\item Result Register\alignright\texttt{\$res}\nopagebreak

The intermediate code result register \texttt{\$res} is mapped to registers \texttt{r0} and \texttt{r1} in 32-bit mode or registers \texttt{w0} or \texttt{x0} in 64-bit mode depending on the size of the actual return type.
If supported natively, floating-point results are stored in registers \texttt{s0} or \texttt{d0} depending on their size.

\item Stack Pointer Register\alignright\texttt{\$sp}\nopagebreak

The intermediate code stack pointer register \texttt{\$sp} is mapped to register \texttt{sp}.

\item Frame Pointer Register\alignright\texttt{\$fp}\nopagebreak

The intermediate code frame pointer register \texttt{\$fp} is mapped to register \texttt{r11} in 32-bit mode or \texttt{x29} in 64-bit mode.

\item Link Register\alignright\texttt{\$lnk}\nopagebreak

The intermediate code link register \texttt{\$lnk} is supported and mapped to register \texttt{lr} in 32-bit mode or register \texttt{x30} in 64-bit mode.

\end{itemize}

All other intermediate code registers are mapped as needed to the remaining physical registers.
Their contents and mapping are therefore considered volatile across function calls.

\section{Runtime Support}\index{Runtime support!for ARM}

The \ecs{} provides runtime support for the ARM architecture and runtime environments based on this hardware architecture in object files.
Users targeting a specific runtime environment have to use an appropriate linker together with these object files in order create an executable program.
This section gives information about all supported runtime environments based on the ARM hardware architecture as well as the required combination of linker and object files.

Basic architectural runtime support is provided by the object files \objfile{arma32\-run}, \objfile{arma64\-run}, \objfile{armt32\-run} and \objfile{armt32\-fpe\-run}.
Users should always include one of these object files during linking regardless of the actual target runtime environment.
All other object files given to the linker should target the same hardware architecture.

Programs written in \cpp{} need additional runtime support stored in the \libfile{cpp\-arma32\-run}, \libfile{cpp\-arma64\-run}, \libfile{cpp\-armt32\-run} and \libfile{cpp\-armt32\-fpe\-run} library files respectively.
Programs written in Oberon need additional runtime support stored in the \libfile{ob\-arma32\-run}, \libfile{ob\-arma64\-run}, \libfile{ob\-armt32\-run} and \libfile{ob\-armt32\-fpe\-run} library files respectively.
\seecpp\seeoberon

\subsection{Linux}

Programs targeting Linux-based operating systems are created using the \tool{link\-bin} linker tool.
It creates either 32-bit or 64-bit Executable and Linking Format (ELF) files~\cite{elffile} if provided with the runtime support stored in the \objfile{arma32\-linux\-run}, \objfile{arma64\-linux\-run}, \objfile{armt32\-linux\-run} and \objfile{armt32\-fpe\-linux\-run} object files respectively.
Calling the \tool{ecsd} utility tool using the \environment{arma32\-linux}, \environment{arma64\-linux}, \environment{armt32\-linux} or \environment{armt32\-fpe\-linux} target environment achieves the same result.
External libraries are available using additional runtime support stored in the following object files: \objfile{arma32\-libdl}, \objfile{arma32\-libpthread}, \objfile{arma32\-libsdl}, \objfile{arma64\-libdl}, \objfile{arma64\-libpthread}, \objfile{arma64\-libsdl}, \objfile{armt32\-libdl}, \objfile{armt32\-libpthread}, \objfile{armt32\-libsdl}, \objfile{armt32\-fpe\-libdl}, \objfile{armt32\-fpe\-libpthread}, and \objfile{armt32\-fpe\-libsdl}.

\subsection{Raspberry Pi}

The \ecs{} provides basic runtime support for a native Raspberry Pi~2 Model~B computer environment stored in the \objfile{rpi2b\-run} object file.
Users creating their own systems may use the \tool{link\-bin} linker together with this object file in order create an executable bootloader.
Calling the \tool{ecsd} utility tool using the \environment{rpi2b} target environment achieves the same result.
The UART communication device is configured to use a baud rate of 115200~bps with eight data bits, no parity, and one stop bit.

\section{ARM Tools}

The \ecs{} provides the following tools that are able to process object files targeting the ARM hardware architecture.
\interface

\cdarma
\cdarmb
\cdarmc
\cdarmcfpe
\cpparma
\cpparmb
\cpparmc
\cpparmcfpe
\falarma
\falarmb
\falarmc
\falarmcfpe
\obarma
\obarmb
\obarmc
\obarmcfpe
\armaasm
\armbasm
\armcasm
\armadism
\armbdism
\armcdism
\linkbin
\linkmem

\concludechapter

% AVR architecture documentation
% Copyright (C) Florian Negele

% This file is part of the Eigen Compiler Suite.

% Permission is granted to copy, distribute and/or modify this document
% under the terms of the GNU Free Documentation License, Version 1.3
% or any later version published by the Free Software Foundation.

% You should have received a copy of the GNU Free Documentation License
% along with the ECS.  If not, see <https://www.gnu.org/licenses/>.

% Generic documentation utilities
% Copyright (C) Florian Negele

% This file is part of the Eigen Compiler Suite.

% Permission is granted to copy, distribute and/or modify this document
% under the terms of the GNU Free Documentation License, Version 1.3
% or any later version published by the Free Software Foundation.

% You should have received a copy of the GNU Free Documentation License
% along with the ECS.  If not, see <https://www.gnu.org/licenses/>.

\providecommand{\cpp}{C\texttt{++}}
\providecommand{\opt}{_\mathit{opt}}
\providecommand{\tool}[1]{\texttt{#1}}
\providecommand{\version}{Version 0.0.40}
\providecommand{\resource}[1]{*++\txt{#1}}
\providecommand{\ecs}{Eigen Compiler Suite}
\providecommand{\changed}[1]{\underline{#1}}
\providecommand{\toolbox}[1]{\converter{#1}}
\providecommand{\file}{}\renewcommand{\file}[1]{\texttt{#1}}
\providecommand{\alignright}{\hfill\linebreak[0]\hspace*{\fill}}
\providecommand{\converter}[1]{*++[F][F*:white][F,:gray]\txt{#1}}
\providecommand{\documentation}{\ifbook chapter\else document\fi}
\providecommand{\Documentation}{\ifbook Chapter\else Document\fi}
\providecommand{\variable}[1]{\resource{\texttt{\small#1}\\variable}}
\providecommand{\documentationref}[2]{\ifbook\ref{#1}\else``\href{#1}{#2}''~\cite{#1}\fi}
\providecommand{\objfile}[1]{\texttt{#1}\index[runtime]{#1 object file@\texttt{#1} object file}}
\providecommand{\libfile}[1]{\texttt{#1}\index[runtime]{#1 library file@\texttt{#1} library file}}
\providecommand{\epigraph}[2]{\ifbook\begin{quote}\flushright\textit{#1}\par--- #2\end{quote}\fi}
\providecommand{\environmentvariable}[1]{\texttt{#1}\index{Environment variables!#1@\texttt{#1}}}
\providecommand{\environment}[1]{\texttt{#1}\index[environment]{#1 environment@\texttt{#1} environment}}
\providecommand{\toolsection}{}\renewcommand{\toolsection}[1]{\subsection{#1}\label{\prefix:#1}\tool{#1}}
\providecommand{\instruction}{}\renewcommand{\instruction}[2]{\noindent\qquad\pdftooltip{\texttt{#1}}{#2}\refstepcounter{instruction}\par}
\providecommand{\flowgraph}{}\renewcommand{\flowgraph}[1]{\par\sffamily\begin{displaymath}\xymatrix@=4ex{#1}\end{displaymath}\normalfont\par}
\providecommand{\instructionset}{}\renewcommand{\instructionset}[4]{\setcounter{instruction}{0}\begin{multicols}{\ifbook#3\else#4\fi}[{\captionof{table}[#2]{#2 (\ref*{#1:instructions}~instructions)}\label{tab:#1set}\vspace{-2ex}}]\footnotesize\raggedcolumns\input{#1.set}\label{#1:instructions}\end{multicols}}

\providecommand{\gpl}{GNU General Public License}
\providecommand{\rse}{ECS Runtime Support Exception}
\providecommand{\fdl}{\href{https://www.gnu.org/licenses/fdl.html}{GNU Free Documentation License}}

\providecommand{\docbegin}{}
\providecommand{\docend}{}
\providecommand{\doclabel}[1]{\hypertarget{#1}}
\providecommand{\doclink}[2]{\hyperlink{#1}{#2}}
\providecommand{\docsection}[3]{\hypertarget{#1}{\subsection{#2}}\label{sec:#1}\index[library]{#2@#3}}
\providecommand{\docsectionstar}[1]{}
\providecommand{\docsubbegin}{\begin{description}}
\providecommand{\docsubend}{\end{description}}
\providecommand{\docsubsection}[3]{\item[\hypertarget{#1}{#2}]\index[library]{#2@#3}}
\providecommand{\docsubsectionstar}[1]{\smallskip}
\providecommand{\docsubsubsection}[3]{\docsubsection{#1}{#2}{#3}}
\providecommand{\docsubsubsectionstar}[1]{}
\providecommand{\docsubsubsubsection}[3]{}
\providecommand{\docsubsubsubsectionstar}[1]{}
\providecommand{\doctable}{}

\providecommand{\debuggingtool}{}\renewcommand{\debuggingtool}{This tool is provided for debugging purposes.
It allows exposing and modifying an internal data structure that is usually not accessible.
}

\providecommand{\interface}{All tools accept command-line arguments which are taken as names of plain text files containing the source code.
If no arguments are provided, the standard input stream is used instead.
Output files are generated in the current working directory and have the same name as the input file being processed whereas the filename extension gets replaced by an appropriate suffix.
\seeinterface
}

\providecommand{\license}{\noindent Copyright \copyright{} Florian Negele\par\medskip\noindent
Permission is granted to copy, distribute and/or modify this document under the terms of the
\fdl{}, Version 1.3 or any later version published by the \href{https://fsf.org/}{Free Software Foundation}.
}

\providecommand{\ecslogosurface}{
\fill[darkgray] (0,0,0) -- (0,0,3) -- (0,3,3) -- (0,3,1) -- (0,4,1) -- (0,4,3) -- (0,5,3) -- (0,5,0) -- (0,2,0) -- (0,2,2) -- (0,1,2) -- (0,1,0) -- cycle;
\fill[gray] (0,5,0) -- (0,5,3) -- (1,5,3) -- (1,5,1) -- (2,5,1) -- (2,5,3) -- (3,5,3) -- (3,5,0) -- cycle;
\fill[lightgray] (0,0,0) -- (0,1,0) -- (2,1,0) -- (2,4,0) -- (1,4,0) -- (1,3,0) -- (2,3,0) -- (2,2,0) -- (0,2,0) -- (0,5,0) -- (3,5,0) -- (3,0,0) -- cycle;
\begin{scope}[line width=0.5]
\begin{scope}[gray]
\draw (0,0,0) -- (0,1,0);
\draw (2,1,0) -- (2,2,0);
\draw (0,1,2) -- (0,2,2);
\draw (0,2,0) -- (0,5,0);
\draw (2,3,0) -- (2,4,0);
\end{scope}
\begin{scope}[lightgray]
\draw (0,1,0) -- (0,1,2);
\draw (0,3,1) -- (0,3,3);
\draw (0,5,0) -- (0,5,3);
\draw (2,5,1) -- (2,5,3);
\end{scope}
\begin{scope}[white]
\draw (0,1,0) -- (2,1,0);
\draw (1,3,0) -- (2,3,0);
\draw (0,5,0) -- (3,5,0);
\end{scope}
\end{scope}
}

\providecommand{\ecslogo}[1]{
\begin{tikzpicture}[scale={(#1)/((sin(45)+cos(45))*3cm)},x={({-cos(45)*1cm},{sin(45)*sin(30)*1cm})},y={({0cm},{(cos(30)*1cm})},z={({sin(45)*1cm},{cos(45)*sin(30)*1cm})}]
\begin{scope}[darkgray,line width=1]
\draw (0,0,0) -- (0,0,3) -- (0,3,3) -- (2,3,3) -- (2,5,3) -- (3,5,3) -- (3,5,0) -- (3,0,0) -- cycle;
\draw (0,3,1) -- (0,4,1) -- (0,4,3) -- (0,5,3) -- (1,5,3) -- (1,5,1) -- (2,5,1);
\draw (1,3,0) -- (1,4,0) -- (2,4,0);
\end{scope}
\fill[darkgray] (2,0,0) -- (2,0,3) -- (2,5,3) -- (2,5,1) -- (2,4,1) -- (2,4,0) -- cycle;
\fill[lightgray] (2,0,2) -- (0,0,2) -- (0,2,2) -- (2,2,2) -- cycle;
\fill[gray] (0,1,0) -- (2,1,0) -- (2,1,2) -- (0,1,2) -- cycle;
\fill[gray] (0,3,1) -- (0,3,3) -- (2,3,3) -- (2,3,0) -- (1,3,0) -- (1,3,1) -- cycle;
\ecslogosurface
\end{tikzpicture}
}

\providecommand{\shadowedecslogo}[3]{
\begin{tikzpicture}[scale={(#1)/((sin(#2)+cos(#2))*3cm)},x={({-cos(#2)*1cm},{sin(#2)*sin(#3)*1cm})},y={({0cm},{(cos(#3)*1cm})},z={({sin(#2)*1cm},{cos(#2)*sin(#3)*1cm})}]
\shade[top color=lightgray!50!white,bottom color=white,middle color=lightgray!50!white] (0,0,0) -- (3,0,0) -- (3,{-0.5-3*sin(#2)*sin(#3)/cos(#3)},0) -- (0,-0.5,0) -- cycle;
\shade[top color=darkgray!50!gray,bottom color=white,middle color=darkgray!50!white] (0,0,0) -- (0,0,3) -- (0,{-0.5-3*cos(#2)*sin(#3)/cos(#3)},3) -- (0,-0.5,0) -- cycle;
\begin{scope}[y={({(cos(#2)+sin(#2))*0.5cm},{(cos(#2)*sin(#3)-sin(#2)*sin(#3))*0.5cm})}]
\useasboundingbox (3,0,0) -- (0,0,0) -- (0,0,3);
\shade[left color=darkgray!80!black,right color=lightgray,middle color=gray] (0,0,0) -- (0,1,0) -- (0,1,0.5) -- (0,2,0) -- (0,5,0) -- (0,5,3) -- (1,5,3) -- (1,4,3) -- (1,4,2.5) -- (1,3,3) -- (2,5,3) -- (3,5,3) -- (3,0,3) -- cycle;
\clip (0,0,0) -- (0,0,3) -- ({-3*sin(#2)/cos(#2)},0,0) -- cycle;
\shade[left color=darkgray,right color=lightgray!50!gray] (0,0,0) -- (0,1,0) -- (0,1,0.5) -- (0,2,0) -- (0,5,0) -- (0,5,3) -- (1,5,3) -- (1,4,3) -- (1,4,2.5) -- (1,3,3) -- (2,5,3) -- (3,5,3) -- (3,0,3) -- cycle;
\end{scope}
\shade[left color=darkgray,right color=darkgray!80!black] (2,0,0) -- (2,0,3) -- (2,5,3) -- (2,5,1) -- (2,4,1) -- (2,4,0) -- cycle;
\shade[left color=darkgray!90!black,right color=gray!80!darkgray] (2,0,2) -- (0,0,2) -- (0,2,2) -- (2,2,2) -- cycle;
\shade[top color=darkgray!90!black,bottom color=gray!80!darkgray] (0,1,0) -- (2,1,0) -- (2,1,2) -- (0,1,2) -- cycle;
\shade[top color=darkgray!90!black,bottom color=gray!80!darkgray] (0,3,1) -- (0,3,3) -- (2,3,3) -- (2,3,0) -- (1,3,0) -- (1,3,1) -- cycle;
\fill[gray] (2,1,0) -- (1.5,1,0.5) -- (0,1,0.5) -- (0,1,0) -- cycle;
\fill[gray] (1,3,2) -- (0.5,3,2) -- (0.5,3,3) -- (1,3,3) -- cycle;
\fill[gray] (2,3,0) -- (1.5,3,0.5) -- (1,3,0.5) -- (1,3,0) -- cycle;
\ecslogosurface
\end{tikzpicture}
}

\providecommand{\cpplogo}[1]{
\begin{tikzpicture}[scale=(#1)/512em]
\fill[gray] (435.2794,398.7159) -- (247.1911,507.3075) .. controls (236.3563,513.5642) and (218.6240,513.5642) .. (207.7892,507.3075) -- (19.7009,398.7159) .. controls (8.8646,392.4606) and (0.0000,377.1043) .. (0.0000,364.5924) -- (0.0000,147.4076) .. controls (0.8430,132.8363) and (8.2856,120.7683) .. (19.7009,113.2842) -- (207.7892,4.6926) .. controls (218.6240,-1.5642) and (236.3564,-1.5642) .. (247.1911,4.6926) -- (435.2794,113.2842) .. controls (447.5273,121.4304) and (454.4987,133.6918) .. (454.9803,147.4076) -- (454.9803,364.5924) .. controls (454.5404,377.7571) and (446.6566,391.0351) .. (435.2794,398.7159) -- cycle(75.8301,255.9993) .. controls (74.9389,404.0881) and (273.2892,469.4783) .. (358.8263,331.8769) -- (293.1917,293.8965) .. controls (253.5702,359.4301) and (155.1909,335.9977) .. (151.6601,255.9993) .. controls (152.7204,182.2703) and (249.4137,148.0211) .. (293.1961,218.1065) -- (358.8308,180.1276) .. controls (283.4477,49.2645) and (79.6318,96.3470) .. (75.8301,255.9993) -- cycle(379.1503,247.5747) -- (362.2982,247.5747) -- (362.2982,230.7226) -- (345.4490,230.7226) -- (345.4490,247.5747) -- (328.5969,247.5747) -- (328.5969,264.4254) -- (345.4490,264.4254) -- (345.4490,281.2759) -- (362.2982,281.2759) -- (362.2982,264.4254) -- (379.1503,264.4254) -- cycle(442.3420,247.5747) -- (425.4899,247.5747) -- (425.4899,230.7226) -- (408.6408,230.7226) -- (408.6408,247.5747) -- (391.7886,247.5747) -- (391.7886,264.4254) -- (408.6408,264.4254) -- (408.6408,281.2759) -- (425.4899,281.2759) -- (425.4899,264.4254) -- (442.3420,264.4254) -- cycle;
\end{tikzpicture}
}

\providecommand{\fallogo}[1]{
\begin{tikzpicture}[scale=(#1)/512em]
\fill[gray] (185.7774,0.0000) .. controls (200.4486,15.9798) and (226.8966,8.7148) .. (235.0426,31.5836) .. controls (249.5297,58.0598) and (247.9581,97.9161) .. (280.3335,110.9762) .. controls (309.1690,120.3496) and (337.8406,104.2727) .. (366.5753,103.9379) .. controls (373.4449,111.5171) and (379.2885,128.2574) .. (383.9755,108.9744) .. controls (396.6979,102.5615) and (437.2808,107.6681) .. (426.9652,124.3252) .. controls (408.9822,121.0785) and (412.4742,146.0729) .. (426.5192,131.4996) .. controls (433.8413,120.8489) and (465.1541,126.5522) .. (441.9067,135.7950) .. controls (396.1879,157.7478) and (344.1112,161.5079) .. (298.5528,183.5702) .. controls (277.7471,193.5198) and (284.6941,218.7163) .. (285.2127,236.9640) .. controls (292.3599,316.2826) and (307.3929,394.6311) .. (317.1198,473.6154) .. controls (329.0637,505.4736) and (292.1195,528.5004) .. (265.9183,511.2761) .. controls (237.9284,499.2462) and (237.3684,465.2681) .. (230.9102,439.9421) .. controls (218.6692,374.3397) and (215.6307,306.9662) .. (198.1732,242.3977) .. controls (183.1379,232.7444) and (164.4245,256.0298) .. (149.0430,261.4799) .. controls (116.9328,279.2585) and (87.1822,308.5851) .. (48.2293,307.8914) .. controls (21.3220,306.9037) and (-15.9107,281.8761) .. (7.2921,252.7908) .. controls (29.7799,220.6177) and (67.5177,204.3028) .. (100.9287,185.9449) .. controls (130.8217,170.8906) and (161.1548,156.5903) .. (191.0278,141.5847) .. controls (196.1738,120.0520) and (186.6049,95.2409) .. (186.8382,72.4353) .. controls (185.5234,48.4204) and (183.1700,23.9341) .. (185.7774,0.0000) -- cycle;
\end{tikzpicture}
}

\providecommand{\oblogo}[1]{
\begin{tikzpicture}[scale=(#1)/512em]
\fill[gray] (160.3865,208.9117) .. controls (154.0879,214.6478) and (149.0735,221.2409) .. (145.4125,228.5384) .. controls (184.8790,248.4273) and (234.7122,269.8787) .. (297.5493,291.8782) .. controls (300.3943,281.4769) and (300.9552,268.7619) .. (300.4023,255.2389) .. controls (248.9909,244.7891) and (200.0310,225.9279) .. (160.3865,208.9117) -- cycle(225.7398,392.6996) .. controls (308.0209,392.1716) and (359.3326,345.9277) .. (368.7203,285.2098) .. controls (376.6742,197.1784) and (311.7194,141.3342) .. (205.4287,142.1456) .. controls (139.9485,141.4804) and (88.7155,166.1957) .. (73.5775,228.0086) .. controls (52.0297,320.3408) and (123.4078,391.0103) .. (225.7398,392.6996) -- cycle(216.0739,176.4733) .. controls (268.9183,179.2424) and (315.8292,206.5488) .. (312.7454,265.1139) .. controls (313.2769,315.6384) and (286.5993,353.4946) .. (216.6040,355.7934) .. controls (162.4657,355.7934) and (126.0914,317.5023) .. (126.0914,260.5103) .. controls (126.1733,214.2900) and (163.3363,176.2849) .. (216.0739,176.4733) -- cycle(76.4897,189.1754) .. controls (13.1586,147.5631) and (0.0000,119.4207) .. (0.0000,119.4207) -- (90.6499,170.1632) .. controls (85.3004,175.8497) and (80.5994,182.1633) .. (76.4897,189.1754) -- cycle(353.9486,119.3004) -- (402.9482,119.3004) .. controls (427.0025,137.0797) and (450.9893,162.7034) .. (474.9529,191.0213) .. controls (509.3540,228.5339) and (531.3391,294.2091) .. (487.8149,312.1206) .. controls (462.8165,324.7652) and (394.3874,316.8943) .. (373.8912,313.6651) .. controls (379.9291,297.7449) and (383.2899,278.4204) .. (381.4989,257.7214) .. controls (420.3069,248.0321) and (421.9610,218.3461) .. (407.7867,192.6417) .. controls (391.1113,162.4018) and (370.1114,132.9097) .. (353.9486,119.3004) -- cycle;
\end{tikzpicture}
}

\providecommand{\markuptable}{
\begin{table}
\sffamily\centering
\begin{tabular}{@{}lcl@{}}
\toprule
\texttt{//italics//} & $\rightarrow$ & \textit{italics} \\
\midrule
\texttt{**bold**} & $\rightarrow$ & \textbf{bold} \\
\midrule
\texttt{\# ordered list} & & 1 ordered list \\
\texttt{\# second item} & $\rightarrow$ & 2 second item \\
\texttt{\#\# sub item} & & \hspace{1em} 1 sub item \\
\midrule
\texttt{* unordered list} & & $\bullet$ unordered list \\
\texttt{* second item} & $\rightarrow$ & $\bullet$ second item \\
\texttt{** sub item} & & \hspace{1em} $\bullet$ sub item \\
\midrule
\texttt{link to [[label]]} & $\rightarrow$ & link to \underline{label} \\
\midrule
\texttt{<{}<label>{}> definition } & $\rightarrow$ & definition \\
\midrule
\texttt{[[url|link name]]} & $\rightarrow$ & \underline{link name} \\
\midrule\addlinespace
\texttt{= large heading} & & {\Large large heading} \smallskip \\
\texttt{== medium heading} & $\rightarrow$ & {\large medium heading} \\
\texttt{=== small heading} & & small heading \\
\midrule
\texttt{no line break} & & no line break for paragraphs \\
\texttt{for paragraphs} & $\rightarrow$ \\
& & use empty line \\
\texttt{use empty line} \\
\midrule
\texttt{force\textbackslash\textbackslash line break} & $\rightarrow$ & force \\
& & line break \\
\midrule
\texttt{horizontal line} & $\rightarrow$ & horizontal line \\
\texttt{----} & & \hrulefill \\
\midrule
\texttt{|=a|=table|=header} & & \underline{a \enspace table \enspace header} \\
\texttt{|a|table|row} & $\rightarrow$ & a \enspace table \enspace row \\
\texttt{|b|table|row} & & b \enspace table \enspace row \\
\midrule
\texttt{\{\{\{} \\
\texttt{unformatted} & $\rightarrow$ & \texttt{unformatted} \\
\texttt{code} & & \texttt{code} \\
\texttt{\}\}\}} \\
\midrule\addlinespace
\texttt{@ new article} & & {\Large 1.\ new article} \smallskip \\
\texttt{@ second article} & $\rightarrow$ & {\Large 2.\ second article} \smallskip \\
\texttt{@@ sub article} & & {\large 2.1.\ sub article} \\
\bottomrule
\end{tabular}
\normalfont\caption{Elements of the generic documentation markup language}
\label{tab:docmarkup}
\end{table}
}

\providecommand{\startchapter}[4]{
\documentclass[11pt,a4paper]{article}
\usepackage{booktabs}
\usepackage[format=hang,labelfont=bf]{caption}
\usepackage{changepage}
\usepackage[T1]{fontenc}
\usepackage[margin=2cm]{geometry}
\usepackage{hyperref}
\usepackage[american]{isodate}
\usepackage{lmodern}
\usepackage{longtable}
\usepackage{mathptmx}
\usepackage{microtype}
\usepackage[toc]{multitoc}
\usepackage{multirow}
\usepackage[all]{nowidow}
\usepackage{pdfcomment}
\usepackage{syntax}
\usepackage{tikz}
\usepackage[all]{xy}
\hypersetup{pdfborder={0 0 0},bookmarksnumbered=true,pdftitle={\ecs{}: #2},pdfauthor={Florian Negele},pdfsubject={\ecs{}},pdfkeywords={#1}}
\setlength{\grammarindent}{8em}\setlength{\grammarparsep}{0.2ex}
\setlength{\columnsep}{2em}
\newcommand{\prefix}{}
\newcounter{instruction}
\bibliographystyle{unsrt}
\renewcommand{\index}[2][]{}
\renewcommand{\arraystretch}{1.05}
\renewcommand{\floatpagefraction}{0.7}
\renewcommand{\syntleft}{\itshape}\renewcommand{\syntright}{}
\title{\vspace{-5ex}\Huge{\ecs{}}\medskip\hrule}
\author{\huge{#2}}
\date{\medskip\version}
\newif\ifbook\bookfalse
\pagestyle{headings}
\frenchspacing
\begin{document}
\maketitle\thispagestyle{empty}\noindent#4\setlength{\columnseprule}{0.4pt}\tableofcontents\setlength{\columnseprule}{0pt}\vfill\pagebreak[3]\null\vfill\bigskip\noindent
\parbox{\textwidth-4em}{\license The contents of this \documentation{} are part of the \href{manual}{\ecs{} User Manual}~\cite{manual} and correspond to Chapter ``\href{manual\##3}{#1}''.\alignright\mbox{\today}}
\parbox{4em}{\flushright\ecslogo{3em}}
\clearpage
}

\providecommand{\concludechapter}{
\vfill\pagebreak[3]\null\vfill
\thispagestyle{myheadings}\markright{REFERENCES}
\noindent\begin{minipage}{\textwidth}\begin{multicols}{2}[\section*{References}]
\renewcommand{\section}[2]{}\small\bibliography{references}
\end{multicols}\end{minipage}\end{document}
}

\providecommand{\startpresentation}[2]{
\documentclass[14pt,aspectratio=43,usepdftitle=false]{beamer}
\usepackage{booktabs}
\usepackage{etex}
\usepackage{multicol}
\usepackage{tikz}
\usepackage[all]{xy}
\bibliographystyle{unsrt}
\setlength{\columnsep}{1em}
\setlength{\leftmargini}{1em}
\setbeamercolor{title}{fg=black}
\setbeamercolor{structure}{fg=darkgray}
\setbeamercolor{bibliography item}{fg=darkgray}
\setbeamerfont{title}{series=\bfseries}
\setbeamerfont{subtitle}{series=\normalfont}
\setbeamerfont*{frametitle}{parent=title}
\setbeamerfont{block title}{series=\bfseries}
\setbeamerfont*{framesubtitle}{parent=subtitle}
\setbeamersize{text margin left=1em,text margin right=1em}
\setbeamertemplate{navigation symbols}{}
\setbeamertemplate{itemize item}[circle]{}
\setbeamertemplate{bibliography item}[triangle]{}
\setbeamertemplate{bibliography entry author}{\usebeamercolor[fg]{bibliography item}}
\setbeamertemplate{frametitle}{\medskip\usebeamerfont{frametitle}\color{gray}\raisebox{-2.5ex}[0ex][0ex]{\rule{0.1em}{4.5ex}}}
\addtobeamertemplate{frametitle}{}{\hspace{0.4em}\usebeamercolor[fg]{title}\insertframetitle\par\vspace{0.2ex}\hspace{0.5em}\usebeamerfont{framesubtitle}\insertframesubtitle}
\hypersetup{pdfborder={0 0 0},bookmarksnumbered=true,bookmarksopen=true,bookmarksopenlevel=0,pdftitle={\ecs{}: #1},pdfauthor={Florian Negele},pdfsubject={\ecs{}},pdfkeywords={#1}}
\renewcommand{\flowgraph}[1]{\resizebox{\textwidth}{!}{$$\xymatrix{##1}$$}}
\title{\ecs{}\medskip\hrule\medskip}
\institute{\shadowedecslogo{5em}{30}{15}}
\date{\version}
\subtitle{#1}
\begin{document}
\begin{frame}[plain]\titlepage\nocite{manual}\end{frame}
\begin{frame}{Contents}{#1}\begin{center}\tableofcontents\end{center}\end{frame}
}

\providecommand{\concludepresentation}{
\begin{frame}{References}\begin{footnotesize}\setlength{\columnseprule}{0.4pt}\begin{multicols}{2}\bibliography{references}\end{multicols}\end{footnotesize}\end{frame}
\end{document}
}

\providecommand{\startbook}[1]{
\documentclass[10pt,paper=17cm:24cm,DIV=13,twoside=semi,headings=normal,numbers=noendperiod,cleardoublepage=plain]{scrbook}
\usepackage{atveryend}
\usepackage{booktabs}
\usepackage{caption}
\usepackage{changepage}
\usepackage[T1]{fontenc}
\usepackage{imakeidx}
\usepackage{hyperref}
\usepackage[american]{isodate}
\usepackage{lmodern}
\usepackage{longtable}
\usepackage{mathptmx}
\usepackage[final]{microtype}
\usepackage{multicol}
\usepackage{multirow}
\usepackage[all]{nowidow}
\usepackage{pdfcomment}
\usepackage{scrlayer-scrpage}
\usepackage{setspace}
\usepackage{syntax}
\usepackage[eventxtindent=4pt,oddtxtexdent=4pt]{thumbs}
\usepackage{tikz}
\usepackage[all]{xy}
\hyphenation{Micro-Blaze Open-Cores Open-RISC Power-PC}
\hypersetup{pdfborder={0 0 0},bookmarksnumbered=true,bookmarksopen=true,bookmarksopenlevel=0,pdftitle={\ecs{}: #1},pdfauthor={Florian Negele},pdfsubject={\ecs{}},pdfkeywords={#1}}
\setlength{\grammarindent}{8em}\setlength{\grammarparsep}{0.7ex}
\setkomafont{captionlabel}{\usekomafont{descriptionlabel}}
\renewcommand{\arraystretch}{1.05}\setstretch{1.1}
\renewcommand{\chapterformat}{\thechapter\autodot\enskip\raisebox{-1ex}[0ex][0ex]{\color{gray}\rule{0.1em}{3.5ex}}\enskip}
\renewcommand{\startchapter}[4]{\hypertarget{##3}{\chapter{##1}}\label{##3}##4\addthumb{##1}{\LARGE\sffamily\bfseries\thechapter}{white}{gray}\renewcommand{\prefix}{##3}}
\renewcommand{\concludechapter}{\clearpage{\stopthumb\cleardoublepage}}
\renewcommand{\syntleft}{\itshape}\renewcommand{\syntright}{}
\renewcommand{\floatpagefraction}{0.7}
\renewcommand{\partheademptypage}{}
\DeclareMicrotypeAlias{lmss}{cmr}
\newcommand{\prefix}{}
\newcounter{instruction}
\bibliographystyle{unsrt}
\newif\ifbook\booktrue
\makeindex[intoc,title=Index]
\makeindex[intoc,name=tools,title=Index of Tools,columns=3]
\makeindex[intoc,name=library,title=Index of Library Names]
\makeindex[intoc,name=runtime,title=Index of Runtime Support]
\makeindex[intoc,name=environment,title=Index of Target Environments]
\indexsetup{toclevel=chapter,headers={\indexname}{\indexname}}
\frenchspacing
\begin{document}
\pagenumbering{alph}
\begin{titlepage}\centering
\huge\sffamily\null\vfill\textbf{\ecs{}}\bigskip\hrule\bigskip#1
\normalsize\normalfont\vfill\vfill\shadowedecslogo{10em}{30}{15}
\large\vfill\vfill\version
\end{titlepage}
\null\vfill
\thispagestyle{empty}
\noindent\today\par\medskip
\license A copy of this license is included in Appendix~\ref{fdl} on page~\pageref{fdl}.
All product names used herein are for identification purposes only and may be trademarks of their respective companies.
\concludechapter
\frontmatter
\setcounter{tocdepth}{1}
\tableofcontents
\setcounter{tocdepth}{2}
\concludechapter
\listoffigures
\concludechapter
\listoftables
\concludechapter
}

\providecommand{\concludebook}{
\backmatter
\addtocontents{toc}{\protect\setcounter{tocdepth}{-1}}
\phantomsection\addcontentsline{toc}{part}{Bibliography}
\bibliography{references}
\concludechapter
\phantomsection\addcontentsline{toc}{part}{Indexes}
\printindex
\concludechapter
\indexprologue{\label{idx:tools}}
\printindex[tools]
\concludechapter
\printindex[library]
\concludechapter
\indexprologue{\label{idx:runtime}}
\printindex[runtime]
\concludechapter
\indexprologue{\label{idx:environment}}
\printindex[environment]
\concludechapter
\pagestyle{empty}\pagenumbering{Alph}\null\clearpage
\null\vfill\centering\ecslogo{4em}\par\medskip\license
\end{document}
}

% chapter references

\providecommand{\seedocumentationref}{}\renewcommand{\seedocumentationref}[3]{#1, see \Documentation{}~\documentationref{#2}{#3}. }
\providecommand{\seeinterface}{}\renewcommand{\seeinterface}{\ifbook See \Documentation{}~\documentationref{interface}{User Interface} for more information about the common user interface of all of these tools. \fi}
\providecommand{\seeguide}{}\renewcommand{\seeguide}{\seedocumentationref{For basic examples of using some of these tools in practice}{guide}{User Guide}}
\providecommand{\seecpp}{}\renewcommand{\seecpp}{\seedocumentationref{For more information about the \cpp{} programming language and its implementation by the \ecs{}}{cpp}{User Manual for \cpp{}}}
\providecommand{\seefalse}{}\renewcommand{\seefalse}{\seedocumentationref{For more information about the FALSE programming language and its implementation by the \ecs{}}{false}{User Manual for FALSE}}
\providecommand{\seeoberon}{}\renewcommand{\seeoberon}{\seedocumentationref{For more information about the Oberon programming language and its implementation by the \ecs{}}{oberon}{User Manual for Oberon}}
\providecommand{\seeassembly}{}\renewcommand{\seeassembly}{\seedocumentationref{For more information about the generic assembly language and how to use it}{assembly}{Generic Assembly Language Specification}}
\providecommand{\seeamd}{}\renewcommand{\seeamd}{\seedocumentationref{For more information about how the \ecs{} supports the AMD64 hardware architecture}{amd64}{AMD64 Hardware Architecture Support}}
\providecommand{\seearm}{}\renewcommand{\seearm}{\seedocumentationref{For more information about how the \ecs{} supports the ARM hardware architecture}{arm}{ARM Hardware Architecture Support}}
\providecommand{\seeavr}{}\renewcommand{\seeavr}{\seedocumentationref{For more information about how the \ecs{} supports the AVR hardware architecture}{avr}{AVR Hardware Architecture Support}}
\providecommand{\seeavrtt}{}\renewcommand{\seeavrtt}{\seedocumentationref{For more information about how the \ecs{} supports the AVR32 hardware architecture}{avr32}{AVR32 Hardware Architecture Support}}
\providecommand{\seemabk}{}\renewcommand{\seemabk}{\seedocumentationref{For more information about how the \ecs{} supports the M68000 hardware architecture}{m68k}{M68000 Hardware Architecture Support}}
\providecommand{\seemibl}{}\renewcommand{\seemibl}{\seedocumentationref{For more information about how the \ecs{} supports the MicroBlaze hardware architecture}{mibl}{MicroBlaze Hardware Architecture Support}}
\providecommand{\seemips}{}\renewcommand{\seemips}{\seedocumentationref{For more information about how the \ecs{} supports the MIPS32 and MIPS64 hardware architectures}{mips}{MIPS Hardware Architecture Support}}
\providecommand{\seemmix}{}\renewcommand{\seemmix}{\seedocumentationref{For more information about how the \ecs{} supports the MMIX hardware architecture}{mmix}{MMIX Hardware Architecture Support}}
\providecommand{\seeorok}{}\renewcommand{\seeorok}{\seedocumentationref{For more information about how the \ecs{} supports the OpenRISC 1000 hardware architecture}{or1k}{OpenRISC 1000 Hardware Architecture Support}}
\providecommand{\seeppc}{}\renewcommand{\seeppc}{\seedocumentationref{For more information about how the \ecs{} supports the PowerPC hardware architecture}{ppc}{PowerPC Hardware Architecture Support}}
\providecommand{\seerisc}{}\renewcommand{\seerisc}{\seedocumentationref{For more information about how the \ecs{} supports the RISC hardware architecture}{risc}{RISC Hardware Architecture Support}}
\providecommand{\seewasm}{}\renewcommand{\seewasm}{\seedocumentationref{For more information about how the \ecs{} supports the WebAssembly architecture}{wasm}{WebAssembly Architecture Support}}
\providecommand{\seedocumentation}{}\renewcommand{\seedocumentation}{\seedocumentationref{For more information about generic documentations and their generation by the \ecs{}}{documentation}{Generic Documentation Generation}}
\providecommand{\seedebugging}{}\renewcommand{\seedebugging}{\seedocumentationref{For more information about debugging information and its representation}{debugging}{Debugging Information Representation}}
\providecommand{\seecode}{}\renewcommand{\seecode}{\seedocumentationref{For more information about intermediate code and its purpose}{code}{Intermediate Code Representation}}
\providecommand{\seeobject}{}\renewcommand{\seeobject}{\seedocumentationref{For more information about object files and their purpose}{object}{Object File Representation}}

% generic documentation tools

\providecommand{\docprint}{
\toolsection{docprint} is a pretty printer for generic documentations.
It reformats generic documentations and writes it to the standard output stream.
\debuggingtool
\flowgraph{\resource{generic\\documentation} \ar[r] & \toolbox{docprint} \ar[r] & \resource{generic\\documentation}}
\seedocumentation
}

\providecommand{\doccheck}{
\toolsection{doccheck} is a syntactic and semantic checker for generic documentations.
It just performs syntactic and semantic checks on generic documentations and writes its diagnostic messages to the standard error stream.
\debuggingtool
\flowgraph{\resource{generic\\documentation} \ar[r] & \toolbox{doccheck} \ar[r] & \resource{diagnostic\\messages}}
\seedocumentation
}

\providecommand{\dochtml}{
\toolsection{dochtml} is an HTML documentation generator for generic documentations.
It processes several generic documentations and assembles all information therein into an HTML document.
\debuggingtool
\flowgraph{\resource{generic\\documentation} \ar[r] & \toolbox{dochtml} \ar[r] & \resource{HTML\\document}}
\seedocumentation
}

\providecommand{\doclatex}{
\toolsection{doclatex} is a Latex documentation generator for generic documentations.
It processes several generic documentations and assembles all information therein into a Latex document.
\debuggingtool
\flowgraph{\resource{generic\\documentation} \ar[r] & \toolbox{doclatex} \ar[r] & \resource{Latex\\document}}
\seedocumentation
}

% intermediate code tools

\providecommand{\cdcheck}{
\toolsection{cdcheck} is a syntactic and semantic checker for intermediate code.
It just performs syntactic and semantic checks on programs written in intermediate code and writes its diagnostic messages to the standard error stream.
\debuggingtool
\flowgraph{\resource{intermediate\\code} \ar[r] & \toolbox{cdcheck} \ar[r] & \resource{diagnostic\\messages}}
\seeassembly\seecode
}

\providecommand{\cdopt}{
\toolsection{cdopt} is an optimizer for intermediate code.
It performs various optimizations on programs written in intermediate code and writes the result to the standard output stream.
\debuggingtool
\flowgraph{\resource{intermediate\\code} \ar[r] & \toolbox{cdopt} \ar[r] & \resource{optimized\\code}}
\seeassembly\seecode
}

\providecommand{\cdrun}{
\toolsection{cdrun} is an interpreter for intermediate code.
It processes and executes programs written in intermediate code.
The following code sections are predefined and have the usual semantics:
\texttt{abort}, \texttt{\_Exit}, \texttt{fflush}, \texttt{floor}, \texttt{fputc}, \texttt{free}, \texttt{getchar}, \texttt{malloc}, and \texttt{putchar}.
Diagnostic messages about invalid operations include the name of the executed code section and the index of the erroneous instruction.
\debuggingtool
\flowgraph{\resource{intermediate\\code} \ar[r] & \toolbox{cdrun} \ar@/u/[r] & \resource{input/\\output} \ar@/d/[l]}
\seeassembly\seecode
}

\providecommand{\cdamda}{
\toolsection{cdamd16} is a compiler for intermediate code targeting the AMD64 hardware architecture.
It generates machine code for AMD64 processors from programs written in intermediate code and stores it in corresponding object files.
The compiler generates machine code for the 16-bit operating mode defined by the AMD64 architecture.
It also creates a debugging information file as well as an assembly file containing a listing of the generated machine code.
\debuggingtool
\flowgraph{\resource{intermediate\\code} \ar[r] & \toolbox{cdamd16} \ar[r] \ar[d] \ar[rd] & \resource{object file} \\ & \resource{assembly\\listing} & \resource{debugging\\information}}
\seeassembly\seeamd\seeobject\seecode\seedebugging
}

\providecommand{\cdamdb}{
\toolsection{cdamd32} is a compiler for intermediate code targeting the AMD64 hardware architecture.
It generates machine code for AMD64 processors from programs written in intermediate code and stores it in corresponding object files.
The compiler generates machine code for the 32-bit operating mode defined by the AMD64 architecture.
It also creates a debugging information file as well as an assembly file containing a listing of the generated machine code.
\debuggingtool
\flowgraph{\resource{intermediate\\code} \ar[r] & \toolbox{cdamd32} \ar[r] \ar[d] \ar[rd] & \resource{object file} \\ & \resource{assembly\\listing} & \resource{debugging\\information}}
\seeassembly\seeamd\seeobject\seecode\seedebugging
}

\providecommand{\cdamdc}{
\toolsection{cdamd64} is a compiler for intermediate code targeting the AMD64 hardware architecture.
It generates machine code for AMD64 processors from programs written in intermediate code and stores it in corresponding object files.
The compiler generates machine code for the 64-bit operating mode defined by the AMD64 architecture.
It also creates a debugging information file as well as an assembly file containing a listing of the generated machine code.
\debuggingtool
\flowgraph{\resource{intermediate\\code} \ar[r] & \toolbox{cdamd64} \ar[r] \ar[d] \ar[rd] & \resource{object file} \\ & \resource{assembly\\listing} & \resource{debugging\\information}}
\seeassembly\seeamd\seeobject\seecode\seedebugging
}

\providecommand{\cdarma}{
\toolsection{cdarma32} is a compiler for intermediate code targeting the ARM hardware architecture.
It generates machine code for ARM processors executing A32 instructions from programs written in intermediate code and stores it in corresponding object files.
It also creates a debugging information file as well as an assembly file containing a listing of the generated machine code.
\debuggingtool
\flowgraph{\resource{intermediate\\code} \ar[r] & \toolbox{cdarma32} \ar[r] \ar[d] \ar[rd] & \resource{object file} \\ & \resource{assembly\\listing} & \resource{debugging\\information}}
\seeassembly\seearm\seeobject\seecode\seedebugging
}

\providecommand{\cdarmb}{
\toolsection{cdarma64} is a compiler for intermediate code targeting the ARM hardware architecture.
It generates machine code for ARM processors executing A64 instructions from programs written in intermediate code and stores it in corresponding object files.
It also creates a debugging information file as well as an assembly file containing a listing of the generated machine code.
\debuggingtool
\flowgraph{\resource{intermediate\\code} \ar[r] & \toolbox{cdarma64} \ar[r] \ar[d] \ar[rd] & \resource{object file} \\ & \resource{assembly\\listing} & \resource{debugging\\information}}
\seeassembly\seearm\seeobject\seecode\seedebugging
}

\providecommand{\cdarmc}{
\toolsection{cdarmt32} is a compiler for intermediate code targeting the ARM hardware architecture.
It generates machine code for ARM processors without floating-point extension executing T32 instructions from programs written in intermediate code and stores it in corresponding object files.
It also creates a debugging information file as well as an assembly file containing a listing of the generated machine code.
\debuggingtool
\flowgraph{\resource{intermediate\\code} \ar[r] & \toolbox{cdarmt32} \ar[r] \ar[d] \ar[rd] & \resource{object file} \\ & \resource{assembly\\listing} & \resource{debugging\\information}}
\seeassembly\seearm\seeobject\seecode\seedebugging
}

\providecommand{\cdarmcfpe}{
\toolsection{cdarmt32fpe} is a compiler for intermediate code targeting the ARM hardware architecture.
It generates machine code for ARM processors with floating-point extension executing T32 instructions from programs written in intermediate code and stores it in corresponding object files.
It also creates a debugging information file as well as an assembly file containing a listing of the generated machine code.
\debuggingtool
\flowgraph{\resource{intermediate\\code} \ar[r] & \toolbox{cdarmt32fpe} \ar[r] \ar[d] \ar[rd] & \resource{object file} \\ & \resource{assembly\\listing} & \resource{debugging\\information}}
\seeassembly\seearm\seeobject\seecode\seedebugging
}

\providecommand{\cdavr}{
\toolsection{cdavr} is a compiler for intermediate code targeting the AVR hardware architecture.
It generates machine code for AVR processors from programs written in intermediate code and stores it in corresponding object files.
It also creates a debugging information file as well as an assembly file containing a listing of the generated machine code.
\debuggingtool
\flowgraph{\resource{intermediate\\code} \ar[r] & \toolbox{cdavr} \ar[r] \ar[d] \ar[rd] & \resource{object file} \\ & \resource{assembly\\listing} & \resource{debugging\\information}}
\seeassembly\seeavr\seeobject\seecode\seedebugging
}

\providecommand{\cdavrtt}{
\toolsection{cdavr32} is a compiler for intermediate code targeting the AVR32 hardware architecture.
It generates machine code for AVR32 processors from programs written in intermediate code and stores it in corresponding object files.
It also creates a debugging information file as well as an assembly file containing a listing of the generated machine code.
\debuggingtool
\flowgraph{\resource{intermediate\\code} \ar[r] & \toolbox{cdavr32} \ar[r] \ar[d] \ar[rd] & \resource{object file} \\ & \resource{assembly\\listing} & \resource{debugging\\information}}
\seeassembly\seeavrtt\seeobject\seecode\seedebugging
}

\providecommand{\cdmabk}{
\toolsection{cdm68k} is a compiler for intermediate code targeting the M68000 hardware architecture.
It generates machine code for M68000 processors from programs written in intermediate code and stores it in corresponding object files.
It also creates a debugging information file as well as an assembly file containing a listing of the generated machine code.
\debuggingtool
\flowgraph{\resource{intermediate\\code} \ar[r] & \toolbox{cdm68k} \ar[r] \ar[d] \ar[rd] & \resource{object file} \\ & \resource{assembly\\listing} & \resource{debugging\\information}}
\seeassembly\seemabk\seeobject\seecode\seedebugging
}

\providecommand{\cdmibl}{
\toolsection{cdmibl} is a compiler for intermediate code targeting the MicroBlaze hardware architecture.
It generates machine code for MicroBlaze processors from programs written in intermediate code and stores it in corresponding object files.
It also creates a debugging information file as well as an assembly file containing a listing of the generated machine code.
\debuggingtool
\flowgraph{\resource{intermediate\\code} \ar[r] & \toolbox{cdmibl} \ar[r] \ar[d] \ar[rd] & \resource{object file} \\ & \resource{assembly\\listing} & \resource{debugging\\information}}
\seeassembly\seemibl\seeobject\seecode\seedebugging
}

\providecommand{\cdmipsa}{
\toolsection{cdmips32} is a compiler for intermediate code targeting the MIPS32 hardware architecture.
It generates machine code for MIPS32 processors from programs written in intermediate code and stores it in corresponding object files.
It also creates a debugging information file as well as an assembly file containing a listing of the generated machine code.
\debuggingtool
\flowgraph{\resource{intermediate\\code} \ar[r] & \toolbox{cdmips32} \ar[r] \ar[d] \ar[rd] & \resource{object file} \\ & \resource{assembly\\listing} & \resource{debugging\\information}}
\seeassembly\seemips\seeobject\seecode\seedebugging
}

\providecommand{\cdmipsb}{
\toolsection{cdmips64} is a compiler for intermediate code targeting the MIPS64 hardware architecture.
It generates machine code for MIPS64 processors from programs written in intermediate code and stores it in corresponding object files.
It also creates a debugging information file as well as an assembly file containing a listing of the generated machine code.
\debuggingtool
\flowgraph{\resource{intermediate\\code} \ar[r] & \toolbox{cdmips64} \ar[r] \ar[d] \ar[rd] & \resource{object file} \\ & \resource{assembly\\listing} & \resource{debugging\\information}}
\seeassembly\seemips\seeobject\seecode\seedebugging
}

\providecommand{\cdmmix}{
\toolsection{cdmmix} is a compiler for intermediate code targeting the MMIX hardware architecture.
It generates machine code for MMIX processors from programs written in intermediate code and stores it in corresponding object files.
It also creates a debugging information file as well as an assembly file containing a listing of the generated machine code.
\debuggingtool
\flowgraph{\resource{intermediate\\code} \ar[r] & \toolbox{cdmmix} \ar[r] \ar[d] \ar[rd] & \resource{object file} \\ & \resource{assembly\\listing} & \resource{debugging\\information}}
\seeassembly\seemmix\seeobject\seecode\seedebugging
}

\providecommand{\cdorok}{
\toolsection{cdor1k} is a compiler for intermediate code targeting the OpenRISC 1000 hardware architecture.
It generates machine code for OpenRISC 1000 processors from programs written in intermediate code and stores it in corresponding object files.
It also creates a debugging information file as well as an assembly file containing a listing of the generated machine code.
\debuggingtool
\flowgraph{\resource{intermediate\\code} \ar[r] & \toolbox{cdor1k} \ar[r] \ar[d] \ar[rd] & \resource{object file} \\ & \resource{assembly\\listing} & \resource{debugging\\information}}
\seeassembly\seeorok\seeobject\seecode\seedebugging
}

\providecommand{\cdppca}{
\toolsection{cdppc32} is a compiler for intermediate code targeting the PowerPC hardware architecture.
It generates machine code for PowerPC processors from programs written in intermediate code and stores it in corresponding object files.
The compiler generates machine code for the 32-bit operating mode defined by the PowerPC architecture.
It also creates a debugging information file as well as an assembly file containing a listing of the generated machine code.
\debuggingtool
\flowgraph{\resource{intermediate\\code} \ar[r] & \toolbox{cdppc32} \ar[r] \ar[d] \ar[rd] & \resource{object file} \\ & \resource{assembly\\listing} & \resource{debugging\\information}}
\seeassembly\seeppc\seeobject\seecode\seedebugging
}

\providecommand{\cdppcb}{
\toolsection{cdppc64} is a compiler for intermediate code targeting the PowerPC hardware architecture.
It generates machine code for PowerPC processors from programs written in intermediate code and stores it in corresponding object files.
The compiler generates machine code for the 64-bit operating mode defined by the PowerPC architecture.
It also creates a debugging information file as well as an assembly file containing a listing of the generated machine code.
\debuggingtool
\flowgraph{\resource{intermediate\\code} \ar[r] & \toolbox{cdppc64} \ar[r] \ar[d] \ar[rd] & \resource{object file} \\ & \resource{assembly\\listing} & \resource{debugging\\information}}
\seeassembly\seeppc\seeobject\seecode\seedebugging
}

\providecommand{\cdrisc}{
\toolsection{cdrisc} is a compiler for intermediate code targeting the RISC hardware architecture.
It generates machine code for RISC processors from programs written in intermediate code and stores it in corresponding object files.
It also creates a debugging information file as well as an assembly file containing a listing of the generated machine code.
\debuggingtool
\flowgraph{\resource{intermediate\\code} \ar[r] & \toolbox{cdrisc} \ar[r] \ar[d] \ar[rd] & \resource{object file} \\ & \resource{assembly\\listing} & \resource{debugging\\information}}
\seeassembly\seerisc\seeobject\seecode\seedebugging
}

\providecommand{\cdwasm}{
\toolsection{cdwasm} is a compiler for intermediate code targeting the WebAssembly architecture.
It generates machine code for WebAssembly targets from programs written in intermediate code and stores it in corresponding object files.
It also creates a debugging information file as well as an assembly file containing a listing of the generated machine code.
\debuggingtool
\flowgraph{\resource{intermediate\\code} \ar[r] & \toolbox{cdwasm} \ar[r] \ar[d] \ar[rd] & \resource{object file} \\ & \resource{assembly\\listing} & \resource{debugging\\information}}
\seeassembly\seewasm\seeobject\seecode\seedebugging
}

% C++ tools

\providecommand{\cppprep}{
\toolsection{cppprep} is a preprocessor for the \cpp{} programming language.
It preprocesses source code according to the rules of \cpp{} and writes it to the standard output stream.
Only the macro names \texttt{\_\_DATE\_\_}, \texttt{\_\_FILE\_\_}, \texttt{\_\_LINE\_\_}, and \texttt{\_\_TIME\_\_} are predefined.
\flowgraph{\resource{\cpp{} or other\\source code} \ar[r] & \toolbox{cppprep} \ar[r] & \resource{preprocessed\\source code} \\ & \variable{ECSINCLUDE} \ar[u]}
\seecpp
}

\providecommand{\cppprint}{
\toolsection{cppprint} is a pretty printer for the \cpp{} programming language.
It reformats the source code of \cpp{} programs and writes it to the standard output stream.
\flowgraph{\resource{\cpp{}\\source code} \ar[r] & \toolbox{cppprint} \ar[r] & \resource{reformatted\\source code} \\ & \variable{ECSINCLUDE} \ar[u]}
\seecpp
}

\providecommand{\cppcheck}{
\toolsection{cppcheck} is a syntactic and semantic checker for the \cpp{} programming language.
It just performs syntactic and semantic checks on \cpp{} programs and writes its diagnostic messages to the standard error stream.
\flowgraph{\resource{\cpp{}\\source code} \ar[r] & \toolbox{cppcheck} \ar[r] & \resource{diagnostic\\messages} \\ & \variable{ECSINCLUDE} \ar[u]}
\seecpp
}

\providecommand{\cppdump}{
\toolsection{cppdump} is a serializer for the \cpp{} programming language.
It dumps the complete internal representation of programs written in \cpp{} into an XML document.
\debuggingtool
\flowgraph{\resource{\cpp{}\\source code} \ar[r] & \toolbox{cppdump} \ar[r] & \resource{internal\\representation} \\ & \variable{ECSINCLUDE} \ar[u]}
\seecpp
}

\providecommand{\cpprun}{
\toolsection{cpprun} is an interpreter for the \cpp{} programming language.
It processes and executes programs written in \cpp{}.
The macro \texttt{\_\_run\_\_} is predefined in order to enable programmers to identify this tool while interpreting.
\flowgraph{\resource{\cpp{}\\source code} \ar[r] & \toolbox{cpprun} \ar@/u/[r] & \resource{input/\\output} \ar@/d/[l] \\ & \variable{ECSINCLUDE} \ar[u]}
\seecpp
}

\providecommand{\cppdoc}{
\toolsection{cppdoc} is a generic documentation generator for the \cpp{} programming language.
It processes several \cpp{} source files and assembles all information therein into a generic documentation.
\debuggingtool
\flowgraph{\resource{\cpp{}\\source code} \ar[r] & \toolbox{cppdoc} \ar[r] & \resource{generic\\documentation} \\ & \variable{ECSINCLUDE} \ar[u]}
\seecpp\seedocumentation
}

\providecommand{\cpphtml}{
\toolsection{cpphtml} is an HTML documentation generator for the \cpp{} programming language.
It processes several \cpp{} source files and assembles all information therein into an HTML document.
\flowgraph{\resource{\cpp{}\\source code} \ar[r] & \toolbox{cpphtml} \ar[r] & \resource{HTML\\document} \\ & \variable{ECSINCLUDE} \ar[u]}
\seecpp\seedocumentation
}

\providecommand{\cpplatex}{
\toolsection{cpplatex} is a Latex documentation generator for the \cpp{} programming language.
It processes several \cpp{} source files and assembles all information therein into a Latex document.
\flowgraph{\resource{\cpp{}\\source code} \ar[r] & \toolbox{cpplatex} \ar[r] & \resource{Latex\\document} \\ & \variable{ECSINCLUDE} \ar[u]}
\seecpp\seedocumentation
}

\providecommand{\cppcode}{
\toolsection{cppcode} is an intermediate code generator for the \cpp{} programming language.
It generates intermediate code from programs written in \cpp{} and stores it in corresponding assembly files.
The macro \texttt{\_\_code\_\_} is predefined in order to enable programmers to identify this tool while generating intermediate code.
Programs generated with this tool require additional runtime support that is stored in the \file{cpp\-code\-run} library file.
\debuggingtool
\flowgraph{\resource{\cpp{}\\source code} \ar[r] & \toolbox{cppcode} \ar[r] & \resource{intermediate\\code} \\ & \variable{ECSINCLUDE} \ar[u]}
\seecpp\seeassembly\seecode
}

\providecommand{\cppamda}{
\toolsection{cppamd16} is a compiler for the \cpp{} programming language targeting the AMD64 hardware architecture.
It generates machine code for AMD64 processors from programs written in \cpp{} and stores it in corresponding object files.
The compiler generates machine code for the 16-bit operating mode defined by the AMD64 architecture.
For debugging purposes, it also creates a debugging information file as well as an assembly file containing a listing of the generated machine code.
The macro \texttt{\_\_amd16\_\_} is predefined in order to enable programmers to identify this tool and its target architecture while compiling.
Programs generated with this compiler require additional runtime support that is stored in the \file{cpp\-amd16\-run} library file.
\flowgraph{\resource{\cpp{}\\source code} \ar[r] & \toolbox{cppamd16} \ar[r] \ar[d] \ar[rd] & \resource{object file} \\ \variable{ECSINCLUDE} \ar[ru] & \resource{debugging\\information} & \resource{assembly\\listing}}
\seecpp\seeassembly\seeamd\seeobject\seedebugging
}

\providecommand{\cppamdb}{
\toolsection{cppamd32} is a compiler for the \cpp{} programming language targeting the AMD64 hardware architecture.
It generates machine code for AMD64 processors from programs written in \cpp{} and stores it in corresponding object files.
The compiler generates machine code for the 32-bit operating mode defined by the AMD64 architecture.
For debugging purposes, it also creates a debugging information file as well as an assembly file containing a listing of the generated machine code.
The macro \texttt{\_\_amd32\_\_} is predefined in order to enable programmers to identify this tool and its target architecture while compiling.
Programs generated with this compiler require additional runtime support that is stored in the \file{cpp\-amd32\-run} library file.
\flowgraph{\resource{\cpp{}\\source code} \ar[r] & \toolbox{cppamd32} \ar[r] \ar[d] \ar[rd] & \resource{object file} \\ \variable{ECSINCLUDE} \ar[ru] & \resource{debugging\\information} & \resource{assembly\\listing}}
\seecpp\seeassembly\seeamd\seeobject\seedebugging
}

\providecommand{\cppamdc}{
\toolsection{cppamd64} is a compiler for the \cpp{} programming language targeting the AMD64 hardware architecture.
It generates machine code for AMD64 processors from programs written in \cpp{} and stores it in corresponding object files.
The compiler generates machine code for the 64-bit operating mode defined by the AMD64 architecture.
For debugging purposes, it also creates a debugging information file as well as an assembly file containing a listing of the generated machine code.
The macro \texttt{\_\_amd64\_\_} is predefined in order to enable programmers to identify this tool and its target architecture while compiling.
Programs generated with this compiler require additional runtime support that is stored in the \file{cpp\-amd64\-run} library file.
\flowgraph{\resource{\cpp{}\\source code} \ar[r] & \toolbox{cppamd64} \ar[r] \ar[d] \ar[rd] & \resource{object file} \\ \variable{ECSINCLUDE} \ar[ru] & \resource{debugging\\information} & \resource{assembly\\listing}}
\seecpp\seeassembly\seeamd\seeobject\seedebugging
}

\providecommand{\cpparma}{
\toolsection{cpparma32} is a compiler for the \cpp{} programming language targeting the ARM hardware architecture.
It generates machine code for ARM processors executing A32 instructions from programs written in \cpp{} and stores it in corresponding object files.
For debugging purposes, it also creates a debugging information file as well as an assembly file containing a listing of the generated machine code.
The macro \texttt{\_\_arma32\_\_} is predefined in order to enable programmers to identify this tool and its target architecture while compiling.
Programs generated with this compiler require additional runtime support that is stored in the \file{cpp\-arma32\-run} library file.
\flowgraph{\resource{\cpp{}\\source code} \ar[r] & \toolbox{cpparma32} \ar[r] \ar[d] \ar[rd] & \resource{object file} \\ \variable{ECSINCLUDE} \ar[ru] & \resource{debugging\\information} & \resource{assembly\\listing}}
\seecpp\seeassembly\seearm\seeobject\seedebugging
}

\providecommand{\cpparmb}{
\toolsection{cpparma64} is a compiler for the \cpp{} programming language targeting the ARM hardware architecture.
It generates machine code for ARM processors executing A64 instructions from programs written in \cpp{} and stores it in corresponding object files.
For debugging purposes, it also creates a debugging information file as well as an assembly file containing a listing of the generated machine code.
The macro \texttt{\_\_arma64\_\_} is predefined in order to enable programmers to identify this tool and its target architecture while compiling.
Programs generated with this compiler require additional runtime support that is stored in the \file{cpp\-arma64\-run} library file.
\flowgraph{\resource{\cpp{}\\source code} \ar[r] & \toolbox{cpparma64} \ar[r] \ar[d] \ar[rd] & \resource{object file} \\ \variable{ECSINCLUDE} \ar[ru] & \resource{debugging\\information} & \resource{assembly\\listing}}
\seecpp\seeassembly\seearm\seeobject\seedebugging
}

\providecommand{\cpparmc}{
\toolsection{cpparmt32} is a compiler for the \cpp{} programming language targeting the ARM hardware architecture.
It generates machine code for ARM processors without floating-point extension executing T32 instructions from programs written in \cpp{} and stores it in corresponding object files.
For debugging purposes, it also creates a debugging information file as well as an assembly file containing a listing of the generated machine code.
The macro \texttt{\_\_armt32\_\_} is predefined in order to enable programmers to identify this tool and its target architecture while compiling.
Programs generated with this compiler require additional runtime support that is stored in the \file{cpp\-armt32\-run} library file.
\flowgraph{\resource{\cpp{}\\source code} \ar[r] & \toolbox{cpparmt32} \ar[r] \ar[d] \ar[rd] & \resource{object file} \\ \variable{ECSINCLUDE} \ar[ru] & \resource{debugging\\information} & \resource{assembly\\listing}}
\seecpp\seeassembly\seearm\seeobject\seedebugging
}

\providecommand{\cpparmcfpe}{
\toolsection{cpparmt32fpe} is a compiler for the \cpp{} programming language targeting the ARM hardware architecture.
It generates machine code for ARM processors with floating-point extension executing T32 instructions from programs written in \cpp{} and stores it in corresponding object files.
For debugging purposes, it also creates a debugging information file as well as an assembly file containing a listing of the generated machine code.
The macro \texttt{\_\_armt32fpe\_\_} is predefined in order to enable programmers to identify this tool and its target architecture while compiling.
Programs generated with this compiler require additional runtime support that is stored in the \file{cpp\-armt32\-fpe\-run} library file.
\flowgraph{\resource{\cpp{}\\source code} \ar[r] & \toolbox{cpparmt32fpe} \ar[r] \ar[d] \ar[rd] & \resource{object file} \\ \variable{ECSINCLUDE} \ar[ru] & \resource{debugging\\information} & \resource{assembly\\listing}}
\seecpp\seeassembly\seearm\seeobject\seedebugging
}

\providecommand{\cppavr}{
\toolsection{cppavr} is a compiler for the \cpp{} programming language targeting the AVR hardware architecture.
It generates machine code for AVR processors from programs written in \cpp{} and stores it in corresponding object files.
For debugging purposes, it also creates a debugging information file as well as an assembly file containing a listing of the generated machine code.
The macro \texttt{\_\_avr\_\_} is predefined in order to enable programmers to identify this tool and its target architecture while compiling.
Programs generated with this compiler require additional runtime support that is stored in the \file{cpp\-avr\-run} library file.
\flowgraph{\resource{\cpp{}\\source code} \ar[r] & \toolbox{cppavr} \ar[r] \ar[d] \ar[rd] & \resource{object file} \\ \variable{ECSINCLUDE} \ar[ru] & \resource{debugging\\information} & \resource{assembly\\listing}}
\seecpp\seeassembly\seeavr\seeobject\seedebugging
}

\providecommand{\cppavrtt}{
\toolsection{cppavr32} is a compiler for the \cpp{} programming language targeting the AVR32 hardware architecture.
It generates machine code for AVR32 processors from programs written in \cpp{} and stores it in corresponding object files.
For debugging purposes, it also creates a debugging information file as well as an assembly file containing a listing of the generated machine code.
The macro \texttt{\_\_avr32\_\_} is predefined in order to enable programmers to identify this tool and its target architecture while compiling.
Programs generated with this compiler require additional runtime support that is stored in the \file{cpp\-avr32\-run} library file.
\flowgraph{\resource{\cpp{}\\source code} \ar[r] & \toolbox{cppavr32} \ar[r] \ar[d] \ar[rd] & \resource{object file} \\ \variable{ECSINCLUDE} \ar[ru] & \resource{debugging\\information} & \resource{assembly\\listing}}
\seecpp\seeassembly\seeavrtt\seeobject\seedebugging
}

\providecommand{\cppmabk}{
\toolsection{cppm68k} is a compiler for the \cpp{} programming language targeting the M68000 hardware architecture.
It generates machine code for M68000 processors from programs written in \cpp{} and stores it in corresponding object files.
For debugging purposes, it also creates a debugging information file as well as an assembly file containing a listing of the generated machine code.
The macro \texttt{\_\_m68k\_\_} is predefined in order to enable programmers to identify this tool and its target architecture while compiling.
Programs generated with this compiler require additional runtime support that is stored in the \file{cpp\-m68k\-run} library file.
\flowgraph{\resource{\cpp{}\\source code} \ar[r] & \toolbox{cppm68k} \ar[r] \ar[d] \ar[rd] & \resource{object file} \\ \variable{ECSINCLUDE} \ar[ru] & \resource{debugging\\information} & \resource{assembly\\listing}}
\seecpp\seeassembly\seemabk\seeobject\seedebugging
}

\providecommand{\cppmibl}{
\toolsection{cppmibl} is a compiler for the \cpp{} programming language targeting the MicroBlaze hardware architecture.
It generates machine code for MicroBlaze processors from programs written in \cpp{} and stores it in corresponding object files.
For debugging purposes, it also creates a debugging information file as well as an assembly file containing a listing of the generated machine code.
The macro \texttt{\_\_mibl\_\_} is predefined in order to enable programmers to identify this tool and its target architecture while compiling.
Programs generated with this compiler require additional runtime support that is stored in the \file{cpp\-mibl\-run} library file.
\flowgraph{\resource{\cpp{}\\source code} \ar[r] & \toolbox{cppmibl} \ar[r] \ar[d] \ar[rd] & \resource{object file} \\ \variable{ECSINCLUDE} \ar[ru] & \resource{debugging\\information} & \resource{assembly\\listing}}
\seecpp\seeassembly\seemibl\seeobject\seedebugging
}

\providecommand{\cppmipsa}{
\toolsection{cppmips32} is a compiler for the \cpp{} programming language targeting the MIPS32 hardware architecture.
It generates machine code for MIPS32 processors from programs written in \cpp{} and stores it in corresponding object files.
For debugging purposes, it also creates a debugging information file as well as an assembly file containing a listing of the generated machine code.
The macro \texttt{\_\_mips32\_\_} is predefined in order to enable programmers to identify this tool and its target architecture while compiling.
Programs generated with this compiler require additional runtime support that is stored in the \file{cpp\-mips32\-run} library file.
\flowgraph{\resource{\cpp{}\\source code} \ar[r] & \toolbox{cppmips32} \ar[r] \ar[d] \ar[rd] & \resource{object file} \\ \variable{ECSINCLUDE} \ar[ru] & \resource{debugging\\information} & \resource{assembly\\listing}}
\seecpp\seeassembly\seemips\seeobject\seedebugging
}

\providecommand{\cppmipsb}{
\toolsection{cppmips64} is a compiler for the \cpp{} programming language targeting the MIPS64 hardware architecture.
It generates machine code for MIPS64 processors from programs written in \cpp{} and stores it in corresponding object files.
For debugging purposes, it also creates a debugging information file as well as an assembly file containing a listing of the generated machine code.
The macro \texttt{\_\_mips64\_\_} is predefined in order to enable programmers to identify this tool and its target architecture while compiling.
Programs generated with this compiler require additional runtime support that is stored in the \file{cpp\-mips64\-run} library file.
\flowgraph{\resource{\cpp{}\\source code} \ar[r] & \toolbox{cppmips64} \ar[r] \ar[d] \ar[rd] & \resource{object file} \\ \variable{ECSINCLUDE} \ar[ru] & \resource{debugging\\information} & \resource{assembly\\listing}}
\seecpp\seeassembly\seemips\seeobject\seedebugging
}

\providecommand{\cppmmix}{
\toolsection{cppmmix} is a compiler for the \cpp{} programming language targeting the MMIX hardware architecture.
It generates machine code for MMIX processors from programs written in \cpp{} and stores it in corresponding object files.
For debugging purposes, it also creates a debugging information file as well as an assembly file containing a listing of the generated machine code.
The macro \texttt{\_\_mmix\_\_} is predefined in order to enable programmers to identify this tool and its target architecture while compiling.
Programs generated with this compiler require additional runtime support that is stored in the \file{cpp\-mmix\-run} library file.
\flowgraph{\resource{\cpp{}\\source code} \ar[r] & \toolbox{cppmmix} \ar[r] \ar[d] \ar[rd] & \resource{object file} \\ \variable{ECSINCLUDE} \ar[ru] & \resource{debugging\\information} & \resource{assembly\\listing}}
\seecpp\seeassembly\seemmix\seeobject\seedebugging
}

\providecommand{\cpporok}{
\toolsection{cppor1k} is a compiler for the \cpp{} programming language targeting the OpenRISC 1000 hardware architecture.
It generates machine code for OpenRISC 1000 processors from programs written in \cpp{} and stores it in corresponding object files.
For debugging purposes, it also creates a debugging information file as well as an assembly file containing a listing of the generated machine code.
The macro \texttt{\_\_or1k\_\_} is predefined in order to enable programmers to identify this tool and its target architecture while compiling.
Programs generated with this compiler require additional runtime support that is stored in the \file{cpp\-or1k\-run} library file.
\flowgraph{\resource{\cpp{}\\source code} \ar[r] & \toolbox{cppor1k} \ar[r] \ar[d] \ar[rd] & \resource{object file} \\ \variable{ECSINCLUDE} \ar[ru] & \resource{debugging\\information} & \resource{assembly\\listing}}
\seecpp\seeassembly\seeorok\seeobject\seedebugging
}

\providecommand{\cppppca}{
\toolsection{cppppc32} is a compiler for the \cpp{} programming language targeting the PowerPC hardware architecture.
It generates machine code for PowerPC processors from programs written in \cpp{} and stores it in corresponding object files.
The compiler generates machine code for the 32-bit operating mode defined by the PowerPC architecture.
For debugging purposes, it also creates a debugging information file as well as an assembly file containing a listing of the generated machine code.
The macro \texttt{\_\_ppc32\_\_} is predefined in order to enable programmers to identify this tool and its target architecture while compiling.
Programs generated with this compiler require additional runtime support that is stored in the \file{cpp\-ppc32\-run} library file.
\flowgraph{\resource{\cpp{}\\source code} \ar[r] & \toolbox{cppppc32} \ar[r] \ar[d] \ar[rd] & \resource{object file} \\ \variable{ECSINCLUDE} \ar[ru] & \resource{debugging\\information} & \resource{assembly\\listing}}
\seecpp\seeassembly\seeppc\seeobject\seedebugging
}

\providecommand{\cppppcb}{
\toolsection{cppppc64} is a compiler for the \cpp{} programming language targeting the PowerPC hardware architecture.
It generates machine code for PowerPC processors from programs written in \cpp{} and stores it in corresponding object files.
The compiler generates machine code for the 64-bit operating mode defined by the PowerPC architecture.
For debugging purposes, it also creates a debugging information file as well as an assembly file containing a listing of the generated machine code.
The macro \texttt{\_\_ppc64\_\_} is predefined in order to enable programmers to identify this tool and its target architecture while compiling.
Programs generated with this compiler require additional runtime support that is stored in the \file{cpp\-ppc64\-run} library file.
\flowgraph{\resource{\cpp{}\\source code} \ar[r] & \toolbox{cppppc64} \ar[r] \ar[d] \ar[rd] & \resource{object file} \\ \variable{ECSINCLUDE} \ar[ru] & \resource{debugging\\information} & \resource{assembly\\listing}}
\seecpp\seeassembly\seeppc\seeobject\seedebugging
}

\providecommand{\cpprisc}{
\toolsection{cpprisc} is a compiler for the \cpp{} programming language targeting the RISC hardware architecture.
It generates machine code for RISC processors from programs written in \cpp{} and stores it in corresponding object files.
For debugging purposes, it also creates a debugging information file as well as an assembly file containing a listing of the generated machine code.
The macro \texttt{\_\_risc\_\_} is predefined in order to enable programmers to identify this tool and its target architecture while compiling.
Programs generated with this compiler require additional runtime support that is stored in the \file{cpp\-risc\-run} library file.
\flowgraph{\resource{\cpp{}\\source code} \ar[r] & \toolbox{cpprisc} \ar[r] \ar[d] \ar[rd] & \resource{object file} \\ \variable{ECSINCLUDE} \ar[ru] & \resource{debugging\\information} & \resource{assembly\\listing}}
\seecpp\seeassembly\seerisc\seeobject\seedebugging
}

\providecommand{\cppwasm}{
\toolsection{cppwasm} is a compiler for the \cpp{} programming language targeting the WebAssembly architecture.
It generates machine code for WebAssembly targets from programs written in \cpp{} and stores it in corresponding object files.
For debugging purposes, it also creates a debugging information file as well as an assembly file containing a listing of the generated machine code.
The macro \texttt{\_\_wasm\_\_} is predefined in order to enable programmers to identify this tool and its target architecture while compiling.
Programs generated with this compiler require additional runtime support that is stored in the \file{cpp\-wasm\-run} library file.
\flowgraph{\resource{\cpp{}\\source code} \ar[r] & \toolbox{cppwasm} \ar[r] \ar[d] \ar[rd] & \resource{object file} \\ \variable{ECSINCLUDE} \ar[ru] & \resource{debugging\\information} & \resource{assembly\\listing}}
\seecpp\seeassembly\seewasm\seeobject\seedebugging
}

% FALSE tools

\providecommand{\falprint}{
\toolsection{falprint} is a pretty printer for the FALSE programming language.
It reformats the source code of FALSE programs and writes it to the standard output stream.
\flowgraph{\resource{FALSE\\source code} \ar[r] & \toolbox{falprint} \ar[r] & \resource{reformatted\\source code}}
\seefalse
}

\providecommand{\falcheck}{
\toolsection{falcheck} is a syntactic and semantic checker for the FALSE programming language.
It just performs syntactic and semantic checks on FALSE programs and writes its diagnostic messages to the standard error stream.
\flowgraph{\resource{FALSE\\source code} \ar[r] & \toolbox{falcheck} \ar[r] & \resource{diagnostic\\messages}}
\seefalse
}

\providecommand{\faldump}{
\toolsection{faldump} is a serializer for the FALSE programming language.
It dumps the complete internal representation of programs written in FALSE into an XML document.
\debuggingtool
\flowgraph{\resource{FALSE\\source code} \ar[r] & \toolbox{faldump} \ar[r] & \resource{internal\\representation}}
\seefalse
}

\providecommand{\falrun}{
\toolsection{falrun} is an interpreter for the FALSE programming language.
It processes and executes programs written in FALSE\@.
\flowgraph{\resource{FALSE\\source code} \ar[r] & \toolbox{falrun} \ar@/u/[r] & \resource{input/\\output} \ar@/d/[l]}
\seefalse
}

\providecommand{\falcpp}{
\toolsection{falcpp} is a transpiler for the FALSE programming language.
It translates programs written in FALSE into \cpp{} programs and stores them in corresponding source files.
\flowgraph{\resource{FALSE\\source code} \ar[r] & \toolbox{falcpp} \ar[r] & \resource{\cpp{}\\source file}}
\seefalse\seecpp
}

\providecommand{\falcode}{
\toolsection{falcode} is an intermediate code generator for the FALSE programming language.
It generates intermediate code from programs written in FALSE and stores it in corresponding assembly files.
\debuggingtool
\flowgraph{\resource{FALSE\\source code} \ar[r] & \toolbox{falcode} \ar[r] & \resource{intermediate\\code}}
\seefalse\seeassembly\seecode
}

\providecommand{\falamda}{
\toolsection{falamd16} is a compiler for the FALSE programming language targeting the AMD64 hardware architecture.
It generates machine code for AMD64 processors from programs written in FALSE and stores it in corresponding object files.
The compiler generates machine code for the 16-bit operating mode defined by the AMD64 architecture.
\flowgraph{\resource{FALSE\\source code} \ar[r] & \toolbox{falamd16} \ar[r] & \resource{object file}}
\seefalse\seeamd\seeobject
}

\providecommand{\falamdb}{
\toolsection{falamd32} is a compiler for the FALSE programming language targeting the AMD64 hardware architecture.
It generates machine code for AMD64 processors from programs written in FALSE and stores it in corresponding object files.
The compiler generates machine code for the 32-bit operating mode defined by the AMD64 architecture.
\flowgraph{\resource{FALSE\\source code} \ar[r] & \toolbox{falamd32} \ar[r] & \resource{object file}}
\seefalse\seeamd\seeobject
}

\providecommand{\falamdc}{
\toolsection{falamd64} is a compiler for the FALSE programming language targeting the AMD64 hardware architecture.
It generates machine code for AMD64 processors from programs written in FALSE and stores it in corresponding object files.
The compiler generates machine code for the 64-bit operating mode defined by the AMD64 architecture.
\flowgraph{\resource{FALSE\\source code} \ar[r] & \toolbox{falamd64} \ar[r] & \resource{object file}}
\seefalse\seeamd\seeobject
}

\providecommand{\falarma}{
\toolsection{falarma32} is a compiler for the FALSE programming language targeting the ARM hardware architecture.
It generates machine code for ARM processors executing A32 instructions from programs written in FALSE and stores it in corresponding object files.
\flowgraph{\resource{FALSE\\source code} \ar[r] & \toolbox{falarma32} \ar[r] & \resource{object file}}
\seefalse\seearm\seeobject
}

\providecommand{\falarmb}{
\toolsection{falarma64} is a compiler for the FALSE programming language targeting the ARM hardware architecture.
It generates machine code for ARM processors executing A64 instructions from programs written in FALSE and stores it in corresponding object files.
\flowgraph{\resource{FALSE\\source code} \ar[r] & \toolbox{falarma64} \ar[r] & \resource{object file}}
\seefalse\seearm\seeobject
}

\providecommand{\falarmc}{
\toolsection{falarmt32} is a compiler for the FALSE programming language targeting the ARM hardware architecture.
It generates machine code for ARM processors without floating-point extension executing T32 instructions from programs written in FALSE and stores it in corresponding object files.
\flowgraph{\resource{FALSE\\source code} \ar[r] & \toolbox{falarmt32} \ar[r] & \resource{object file}}
\seefalse\seearm\seeobject
}

\providecommand{\falarmcfpe}{
\toolsection{falarmt32fpe} is a compiler for the FALSE programming language targeting the ARM hardware architecture.
It generates machine code for ARM processors with floating-point extension executing T32 instructions from programs written in FALSE and stores it in corresponding object files.
\flowgraph{\resource{FALSE\\source code} \ar[r] & \toolbox{falarmt32fpe} \ar[r] & \resource{object file}}
\seefalse\seearm\seeobject
}

\providecommand{\falavr}{
\toolsection{falavr} is a compiler for the FALSE programming language targeting the AVR hardware architecture.
It generates machine code for AVR processors from programs written in FALSE and stores it in corresponding object files.
\flowgraph{\resource{FALSE\\source code} \ar[r] & \toolbox{falavr} \ar[r] & \resource{object file}}
\seefalse\seeavr\seeobject
}

\providecommand{\falavrtt}{
\toolsection{falavr32} is a compiler for the FALSE programming language targeting the AVR32 hardware architecture.
It generates machine code for AVR32 processors from programs written in FALSE and stores it in corresponding object files.
\flowgraph{\resource{FALSE\\source code} \ar[r] & \toolbox{falavr32} \ar[r] & \resource{object file}}
\seefalse\seeavrtt\seeobject
}

\providecommand{\falmabk}{
\toolsection{falm68k} is a compiler for the FALSE programming language targeting the M68000 hardware architecture.
It generates machine code for M68000 processors from programs written in FALSE and stores it in corresponding object files.
\flowgraph{\resource{FALSE\\source code} \ar[r] & \toolbox{falm68k} \ar[r] & \resource{object file}}
\seefalse\seemabk\seeobject
}

\providecommand{\falmibl}{
\toolsection{falmibl} is a compiler for the FALSE programming language targeting the MicroBlaze hardware architecture.
It generates machine code for MicroBlaze processors from programs written in FALSE and stores it in corresponding object files.
\flowgraph{\resource{FALSE\\source code} \ar[r] & \toolbox{falmibl} \ar[r] & \resource{object file}}
\seefalse\seemibl\seeobject
}

\providecommand{\falmipsa}{
\toolsection{falmips32} is a compiler for the FALSE programming language targeting the MIPS32 hardware architecture.
It generates machine code for MIPS32 processors from programs written in FALSE and stores it in corresponding object files.
\flowgraph{\resource{FALSE\\source code} \ar[r] & \toolbox{falmips32} \ar[r] & \resource{object file}}
\seefalse\seemips\seeobject
}

\providecommand{\falmipsb}{
\toolsection{falmips64} is a compiler for the FALSE programming language targeting the MIPS64 hardware architecture.
It generates machine code for MIPS64 processors from programs written in FALSE and stores it in corresponding object files.
\flowgraph{\resource{FALSE\\source code} \ar[r] & \toolbox{falmips64} \ar[r] & \resource{object file}}
\seefalse\seemips\seeobject
}

\providecommand{\falmmix}{
\toolsection{falmmix} is a compiler for the FALSE programming language targeting the MMIX hardware architecture.
It generates machine code for MMIX processors from programs written in FALSE and stores it in corresponding object files.
\flowgraph{\resource{FALSE\\source code} \ar[r] & \toolbox{falmmix} \ar[r] & \resource{object file}}
\seefalse\seemmix\seeobject
}

\providecommand{\falorok}{
\toolsection{falor1k} is a compiler for the FALSE programming language targeting the OpenRISC 1000 hardware architecture.
It generates machine code for OpenRISC 1000 processors from programs written in FALSE and stores it in corresponding object files.
\flowgraph{\resource{FALSE\\source code} \ar[r] & \toolbox{falor1k} \ar[r] & \resource{object file}}
\seefalse\seeorok\seeobject
}

\providecommand{\falppca}{
\toolsection{falppc32} is a compiler for the FALSE programming language targeting the PowerPC hardware architecture.
It generates machine code for PowerPC processors from programs written in FALSE and stores it in corresponding object files.
The compiler generates machine code for the 32-bit operating mode defined by the PowerPC architecture.
\flowgraph{\resource{FALSE\\source code} \ar[r] & \toolbox{falppc32} \ar[r] & \resource{object file}}
\seefalse\seeppc\seeobject
}

\providecommand{\falppcb}{
\toolsection{falppc64} is a compiler for the FALSE programming language targeting the PowerPC hardware architecture.
It generates machine code for PowerPC processors from programs written in FALSE and stores it in corresponding object files.
The compiler generates machine code for the 64-bit operating mode defined by the PowerPC architecture.
\flowgraph{\resource{FALSE\\source code} \ar[r] & \toolbox{falppc64} \ar[r] & \resource{object file}}
\seefalse\seeppc\seeobject
}

\providecommand{\falrisc}{
\toolsection{falrisc} is a compiler for the FALSE programming language targeting the RISC hardware architecture.
It generates machine code for RISC processors from programs written in FALSE and stores it in corresponding object files.
\flowgraph{\resource{FALSE\\source code} \ar[r] & \toolbox{falrisc} \ar[r] & \resource{object file}}
\seefalse\seerisc\seeobject
}

\providecommand{\falwasm}{
\toolsection{falwasm} is a compiler for the FALSE programming language targeting the WebAssembly architecture.
It generates machine code for WebAssembly targets from programs written in FALSE and stores it in corresponding object files.
\flowgraph{\resource{FALSE\\source code} \ar[r] & \toolbox{falwasm} \ar[r] & \resource{object file}}
\seefalse\seewasm\seeobject
}

% Oberon tools

\providecommand{\obprint}{
\toolsection{obprint} is a pretty printer for the Oberon programming language.
It reformats the source code of Oberon modules and writes it to the standard output stream.
\flowgraph{\resource{Oberon\\source code} \ar[r] & \toolbox{obprint} \ar[r] & \resource{reformatted\\source code}}
\seeoberon
}

\providecommand{\obcheck}{
\toolsection{obcheck} is a syntactic and semantic checker for the Oberon programming language.
It just performs syntactic and semantic checks on Oberon modules and writes its diagnostic messages to the standard error stream.
In addition, it stores the interface of each module in a symbol file which is required when other modules import the module.
\flowgraph{\resource{Oberon\\source code} \ar[r] & \toolbox{obcheck} \ar[r] \ar@/l/[d] & \resource{diagnostic\\messages} \\ \variable{ECSIMPORT} \ar[ru] & \resource{symbol\\files} \ar@/r/[u]}
\seeoberon
}

\providecommand{\obdump}{
\toolsection{obdump} is a serializer for the Oberon programming language.
It dumps the complete internal representation of modules written in Oberon into an XML document.
\debuggingtool
\flowgraph{\resource{Oberon\\source code} \ar[r] & \toolbox{obdump} \ar[r] \ar@/l/[d] & \resource{internal\\representation} \\ \variable{ECSIMPORT} \ar[ru] & \resource{symbol\\files} \ar@/r/[u]}
\seeoberon
}

\providecommand{\obrun}{
\toolsection{obrun} is an interpreter for the Oberon programming language.
It processes and executes modules written in Oberon.
This tool does neither generate nor process symbol files while interpreting modules.
If a module is imported by another one, its filename has to be named before the other one in the list of command-line arguments.
\flowgraph{\resource{Oberon\\source code} \ar[r] & \toolbox{obrun} \ar@/u/[r] & \resource{input/\\output} \ar@/d/[l]}
\seeoberon
}

\providecommand{\obcpp}{
\toolsection{obcpp} is a transpiler for the Oberon programming language.
It translates programs written in Oberon into \cpp{} programs and stores them in corresponding source and header files.
In addition, it stores the interface of each module in a symbol file which is required when other modules import the module.
The same interface is provided by the generated header file which can be used in other parts of the \cpp{} program.
\flowgraph{\resource{Oberon\\source code} \ar[r] & \toolbox{obcpp} \ar[r] \ar@/l/[d] \ar[rd] & \resource{\cpp{}\\source file} \\ \variable{ECSIMPORT} \ar[ru] & \resource{symbol\\files} \ar@/r/[u] & \resource{\cpp{}\\header file}}
\seeoberon\seecpp
}

\providecommand{\obdoc}{
\toolsection{obdoc} is a generic documentation generator for the Oberon programming language.
It processes several Oberon modules and assembles all information therein into a generic documentation.
In addition, it stores the interface of each module in a symbol file which is required when other modules import the module.
\debuggingtool
\flowgraph{\resource{Oberon\\source code} \ar[r] & \toolbox{obdoc} \ar[r] \ar@/l/[d] & \resource{generic\\documentation} \\ \variable{ECSIMPORT} \ar[ru] & \resource{symbol\\files} \ar@/r/[u]}
\seeoberon\seedocumentation
}

\providecommand{\obhtml}{
\toolsection{obhtml} is an HTML documentation generator for the Oberon programming language.
It processes several Oberon modules and assembles all information therein into an HTML document.
In addition, it stores the interface of each module in a symbol file which is required when other modules import the module.
\flowgraph{\resource{Oberon\\source code} \ar[r] & \toolbox{obhtml} \ar[r] \ar@/l/[d] & \resource{HTML\\document} \\ \variable{ECSIMPORT} \ar[ru] & \resource{symbol\\files} \ar@/r/[u]}
\seeoberon\seedocumentation
}

\providecommand{\oblatex}{
\toolsection{oblatex} is a Latex documentation generator for the Oberon programming language.
It processes several Oberon modules and assembles all information therein into a Latex document.
In addition, it stores the interface of each module in a symbol file which is required when other modules import the module.
\flowgraph{\resource{Oberon\\source code} \ar[r] & \toolbox{oblatex} \ar[r] \ar@/l/[d] & \resource{Latex\\document} \\ \variable{ECSIMPORT} \ar[ru] & \resource{symbol\\files} \ar@/r/[u]}
\seeoberon\seedocumentation
}

\providecommand{\obcode}{
\toolsection{obcode} is an intermediate code generator for the Oberon programming language.
It generates intermediate code from modules written in Oberon and stores it in corresponding assembly files.
In addition, it stores the interface of each module in a symbol file which is required when other modules import the module.
Programs generated with this tool require additional runtime support that is stored in the \file{ob\-code\-run} library file.
\debuggingtool
\flowgraph{\resource{Oberon\\source code} \ar[r] & \toolbox{obcode} \ar[r] \ar@/l/[d] & \resource{intermediate\\code} \\ \variable{ECSIMPORT} \ar[ru] & \resource{symbol\\files} \ar@/r/[u]}
\seeoberon\seeassembly\seecode
}

\providecommand{\obamda}{
\toolsection{obamd16} is a compiler for the Oberon programming language targeting the AMD64 hardware architecture.
It generates machine code for AMD64 processors from modules written in Oberon and stores it in corresponding object files.
The compiler generates machine code for the 16-bit operating mode defined by the AMD64 architecture.
For debugging purposes, it also creates a debugging information file as well as an assembly file containing a listing of the generated machine code.
In addition, it stores the interface of each module in a symbol file which is required when other modules import the module.
Programs generated with this compiler require additional runtime support that is stored in the \file{ob\-amd16\-run} library file.
\flowgraph{\resource{Oberon\\source code} \ar[r] & \toolbox{obamd16} \ar[r] \ar@/l/[d] \ar[rd] & \resource{object file} \\ \variable{ECSIMPORT} \ar[ru] & \resource{symbol\\files} \ar@/r/[u] & \resource{debugging\\information}}
\seeoberon\seeassembly\seeamd\seeobject\seedebugging
}

\providecommand{\obamdb}{
\toolsection{obamd32} is a compiler for the Oberon programming language targeting the AMD64 hardware architecture.
It generates machine code for AMD64 processors from modules written in Oberon and stores it in corresponding object files.
The compiler generates machine code for the 32-bit operating mode defined by the AMD64 architecture.
For debugging purposes, it also creates a debugging information file as well as an assembly file containing a listing of the generated machine code.
In addition, it stores the interface of each module in a symbol file which is required when other modules import the module.
Programs generated with this compiler require additional runtime support that is stored in the \file{ob\-amd32\-run} library file.
\flowgraph{\resource{Oberon\\source code} \ar[r] & \toolbox{obamd32} \ar[r] \ar@/l/[d] \ar[rd] & \resource{object file} \\ \variable{ECSIMPORT} \ar[ru] & \resource{symbol\\files} \ar@/r/[u] & \resource{debugging\\information}}
\seeoberon\seeassembly\seeamd\seeobject\seedebugging
}

\providecommand{\obamdc}{
\toolsection{obamd64} is a compiler for the Oberon programming language targeting the AMD64 hardware architecture.
It generates machine code for AMD64 processors from modules written in Oberon and stores it in corresponding object files.
The compiler generates machine code for the 64-bit operating mode defined by the AMD64 architecture.
For debugging purposes, it also creates a debugging information file as well as an assembly file containing a listing of the generated machine code.
In addition, it stores the interface of each module in a symbol file which is required when other modules import the module.
Programs generated with this compiler require additional runtime support that is stored in the \file{ob\-amd64\-run} library file.
\flowgraph{\resource{Oberon\\source code} \ar[r] & \toolbox{obamd64} \ar[r] \ar@/l/[d] \ar[rd] & \resource{object file} \\ \variable{ECSIMPORT} \ar[ru] & \resource{symbol\\files} \ar@/r/[u] & \resource{debugging\\information}}
\seeoberon\seeassembly\seeamd\seeobject\seedebugging
}

\providecommand{\obarma}{
\toolsection{obarma32} is a compiler for the Oberon programming language targeting the ARM hardware architecture.
It generates machine code for ARM processors executing A32 instructions from modules written in Oberon and stores it in corresponding object files.
For debugging purposes, it also creates a debugging information file as well as an assembly file containing a listing of the generated machine code.
In addition, it stores the interface of each module in a symbol file which is required when other modules import the module.
Programs generated with this compiler require additional runtime support that is stored in the \file{ob\-arma32\-run} library file.
\flowgraph{\resource{Oberon\\source code} \ar[r] & \toolbox{obarma32} \ar[r] \ar@/l/[d] \ar[rd] & \resource{object file} \\ \variable{ECSIMPORT} \ar[ru] & \resource{symbol\\files} \ar@/r/[u] & \resource{debugging\\information}}
\seeoberon\seeassembly\seearm\seeobject\seedebugging
}

\providecommand{\obarmb}{
\toolsection{obarma64} is a compiler for the Oberon programming language targeting the ARM hardware architecture.
It generates machine code for ARM processors executing A64 instructions from modules written in Oberon and stores it in corresponding object files.
For debugging purposes, it also creates a debugging information file as well as an assembly file containing a listing of the generated machine code.
In addition, it stores the interface of each module in a symbol file which is required when other modules import the module.
Programs generated with this compiler require additional runtime support that is stored in the \file{ob\-arma64\-run} library file.
\flowgraph{\resource{Oberon\\source code} \ar[r] & \toolbox{obarma64} \ar[r] \ar@/l/[d] \ar[rd] & \resource{object file} \\ \variable{ECSIMPORT} \ar[ru] & \resource{symbol\\files} \ar@/r/[u] & \resource{debugging\\information}}
\seeoberon\seeassembly\seearm\seeobject\seedebugging
}

\providecommand{\obarmc}{
\toolsection{obarmt32} is a compiler for the Oberon programming language targeting the ARM hardware architecture.
It generates machine code for ARM processors without floating-point extension executing T32 instructions from modules written in Oberon and stores it in corresponding object files.
For debugging purposes, it also creates a debugging information file as well as an assembly file containing a listing of the generated machine code.
In addition, it stores the interface of each module in a symbol file which is required when other modules import the module.
Programs generated with this compiler require additional runtime support that is stored in the \file{ob\-armt32\-run} library file.
\flowgraph{\resource{Oberon\\source code} \ar[r] & \toolbox{obarmt32} \ar[r] \ar@/l/[d] \ar[rd] & \resource{object file} \\ \variable{ECSIMPORT} \ar[ru] & \resource{symbol\\files} \ar@/r/[u] & \resource{debugging\\information}}
\seeoberon\seeassembly\seearm\seeobject\seedebugging
}

\providecommand{\obarmcfpe}{
\toolsection{obarmt32fpe} is a compiler for the Oberon programming language targeting the ARM hardware architecture.
It generates machine code for ARM processors with floating-point extension executing T32 instructions from modules written in Oberon and stores it in corresponding object files.
For debugging purposes, it also creates a debugging information file as well as an assembly file containing a listing of the generated machine code.
In addition, it stores the interface of each module in a symbol file which is required when other modules import the module.
Programs generated with this compiler require additional runtime support that is stored in the \file{ob\-armt32\-fpe\-run} library file.
\flowgraph{\resource{Oberon\\source code} \ar[r] & \toolbox{obarmt32fpe} \ar[r] \ar@/l/[d] \ar[rd] & \resource{object file} \\ \variable{ECSIMPORT} \ar[ru] & \resource{symbol\\files} \ar@/r/[u] & \resource{debugging\\information}}
\seeoberon\seeassembly\seearm\seeobject\seedebugging
}

\providecommand{\obavr}{
\toolsection{obavr} is a compiler for the Oberon programming language targeting the AVR hardware architecture.
It generates machine code for AVR processors from modules written in Oberon and stores it in corresponding object files.
For debugging purposes, it also creates a debugging information file as well as an assembly file containing a listing of the generated machine code.
In addition, it stores the interface of each module in a symbol file which is required when other modules import the module.
Programs generated with this compiler require additional runtime support that is stored in the \file{ob\-avr\-run} library file.
\flowgraph{\resource{Oberon\\source code} \ar[r] & \toolbox{obavr} \ar[r] \ar@/l/[d] \ar[rd] & \resource{object file} \\ \variable{ECSIMPORT} \ar[ru] & \resource{symbol\\files} \ar@/r/[u] & \resource{debugging\\information}}
\seeoberon\seeassembly\seeavr\seeobject\seedebugging
}

\providecommand{\obavrtt}{
\toolsection{obavr32} is a compiler for the Oberon programming language targeting the AVR32 hardware architecture.
It generates machine code for AVR32 processors from modules written in Oberon and stores it in corresponding object files.
For debugging purposes, it also creates a debugging information file as well as an assembly file containing a listing of the generated machine code.
In addition, it stores the interface of each module in a symbol file which is required when other modules import the module.
Programs generated with this compiler require additional runtime support that is stored in the \file{ob\-avr32\-run} library file.
\flowgraph{\resource{Oberon\\source code} \ar[r] & \toolbox{obavr32} \ar[r] \ar@/l/[d] \ar[rd] & \resource{object file} \\ \variable{ECSIMPORT} \ar[ru] & \resource{symbol\\files} \ar@/r/[u] & \resource{debugging\\information}}
\seeoberon\seeassembly\seeavrtt\seeobject\seedebugging
}

\providecommand{\obmabk}{
\toolsection{obm68k} is a compiler for the Oberon programming language targeting the M68000 hardware architecture.
It generates machine code for M68000 processors from modules written in Oberon and stores it in corresponding object files.
For debugging purposes, it also creates a debugging information file as well as an assembly file containing a listing of the generated machine code.
In addition, it stores the interface of each module in a symbol file which is required when other modules import the module.
Programs generated with this compiler require additional runtime support that is stored in the \file{ob\-m68k\-run} library file.
\flowgraph{\resource{Oberon\\source code} \ar[r] & \toolbox{obm68k} \ar[r] \ar@/l/[d] \ar[rd] & \resource{object file} \\ \variable{ECSIMPORT} \ar[ru] & \resource{symbol\\files} \ar@/r/[u] & \resource{debugging\\information}}
\seeoberon\seeassembly\seemabk\seeobject\seedebugging
}

\providecommand{\obmibl}{
\toolsection{obmibl} is a compiler for the Oberon programming language targeting the MicroBlaze hardware architecture.
It generates machine code for MicroBlaze processors from modules written in Oberon and stores it in corresponding object files.
For debugging purposes, it also creates a debugging information file as well as an assembly file containing a listing of the generated machine code.
In addition, it stores the interface of each module in a symbol file which is required when other modules import the module.
Programs generated with this compiler require additional runtime support that is stored in the \file{ob\-mibl\-run} library file.
\flowgraph{\resource{Oberon\\source code} \ar[r] & \toolbox{obmibl} \ar[r] \ar@/l/[d] \ar[rd] & \resource{object file} \\ \variable{ECSIMPORT} \ar[ru] & \resource{symbol\\files} \ar@/r/[u] & \resource{debugging\\information}}
\seeoberon\seeassembly\seemibl\seeobject\seedebugging
}

\providecommand{\obmipsa}{
\toolsection{obmips32} is a compiler for the Oberon programming language targeting the MIPS32 hardware architecture.
It generates machine code for MIPS32 processors from modules written in Oberon and stores it in corresponding object files.
For debugging purposes, it also creates a debugging information file as well as an assembly file containing a listing of the generated machine code.
In addition, it stores the interface of each module in a symbol file which is required when other modules import the module.
Programs generated with this compiler require additional runtime support that is stored in the \file{ob\-mips32\-run} library file.
\flowgraph{\resource{Oberon\\source code} \ar[r] & \toolbox{obmips32} \ar[r] \ar@/l/[d] \ar[rd] & \resource{object file} \\ \variable{ECSIMPORT} \ar[ru] & \resource{symbol\\files} \ar@/r/[u] & \resource{debugging\\information}}
\seeoberon\seeassembly\seemips\seeobject\seedebugging
}

\providecommand{\obmipsb}{
\toolsection{obmips64} is a compiler for the Oberon programming language targeting the MIPS64 hardware architecture.
It generates machine code for MIPS64 processors from modules written in Oberon and stores it in corresponding object files.
For debugging purposes, it also creates a debugging information file as well as an assembly file containing a listing of the generated machine code.
In addition, it stores the interface of each module in a symbol file which is required when other modules import the module.
Programs generated with this compiler require additional runtime support that is stored in the \file{ob\-mips64\-run} library file.
\flowgraph{\resource{Oberon\\source code} \ar[r] & \toolbox{obmips64} \ar[r] \ar@/l/[d] \ar[rd] & \resource{object file} \\ \variable{ECSIMPORT} \ar[ru] & \resource{symbol\\files} \ar@/r/[u] & \resource{debugging\\information}}
\seeoberon\seeassembly\seemips\seeobject\seedebugging
}

\providecommand{\obmmix}{
\toolsection{obmmix} is a compiler for the Oberon programming language targeting the MMIX hardware architecture.
It generates machine code for MMIX processors from modules written in Oberon and stores it in corresponding object files.
For debugging purposes, it also creates a debugging information file as well as an assembly file containing a listing of the generated machine code.
In addition, it stores the interface of each module in a symbol file which is required when other modules import the module.
Programs generated with this compiler require additional runtime support that is stored in the \file{ob\-mmix\-run} library file.
\flowgraph{\resource{Oberon\\source code} \ar[r] & \toolbox{obmmix} \ar[r] \ar@/l/[d] \ar[rd] & \resource{object file} \\ \variable{ECSIMPORT} \ar[ru] & \resource{symbol\\files} \ar@/r/[u] & \resource{debugging\\information}}
\seeoberon\seeassembly\seemmix\seeobject\seedebugging
}

\providecommand{\oborok}{
\toolsection{obor1k} is a compiler for the Oberon programming language targeting the OpenRISC 1000 hardware architecture.
It generates machine code for OpenRISC 1000 processors from modules written in Oberon and stores it in corresponding object files.
For debugging purposes, it also creates a debugging information file as well as an assembly file containing a listing of the generated machine code.
In addition, it stores the interface of each module in a symbol file which is required when other modules import the module.
Programs generated with this compiler require additional runtime support that is stored in the \file{ob\-or1k\-run} library file.
\flowgraph{\resource{Oberon\\source code} \ar[r] & \toolbox{obor1k} \ar[r] \ar@/l/[d] \ar[rd] & \resource{object file} \\ \variable{ECSIMPORT} \ar[ru] & \resource{symbol\\files} \ar@/r/[u] & \resource{debugging\\information}}
\seeoberon\seeassembly\seeorok\seeobject\seedebugging
}

\providecommand{\obppca}{
\toolsection{obppc32} is a compiler for the Oberon programming language targeting the PowerPC hardware architecture.
It generates machine code for PowerPC processors from modules written in Oberon and stores it in corresponding object files.
The compiler generates machine code for the 32-bit operating mode defined by the PowerPC architecture.
For debugging purposes, it also creates a debugging information file as well as an assembly file containing a listing of the generated machine code.
In addition, it stores the interface of each module in a symbol file which is required when other modules import the module.
Programs generated with this compiler require additional runtime support that is stored in the \file{ob\-ppc32\-run} library file.
\flowgraph{\resource{Oberon\\source code} \ar[r] & \toolbox{obppc32} \ar[r] \ar@/l/[d] \ar[rd] & \resource{object file} \\ \variable{ECSIMPORT} \ar[ru] & \resource{symbol\\files} \ar@/r/[u] & \resource{debugging\\information}}
\seeoberon\seeassembly\seeppc\seeobject\seedebugging
}

\providecommand{\obppcb}{
\toolsection{obppc64} is a compiler for the Oberon programming language targeting the PowerPC hardware architecture.
It generates machine code for PowerPC processors from modules written in Oberon and stores it in corresponding object files.
The compiler generates machine code for the 64-bit operating mode defined by the PowerPC architecture.
For debugging purposes, it also creates a debugging information file as well as an assembly file containing a listing of the generated machine code.
In addition, it stores the interface of each module in a symbol file which is required when other modules import the module.
Programs generated with this compiler require additional runtime support that is stored in the \file{ob\-ppc64\-run} library file.
\flowgraph{\resource{Oberon\\source code} \ar[r] & \toolbox{obppc64} \ar[r] \ar@/l/[d] \ar[rd] & \resource{object file} \\ \variable{ECSIMPORT} \ar[ru] & \resource{symbol\\files} \ar@/r/[u] & \resource{debugging\\information}}
\seeoberon\seeassembly\seeppc\seeobject\seedebugging
}

\providecommand{\obrisc}{
\toolsection{obrisc} is a compiler for the Oberon programming language targeting the RISC hardware architecture.
It generates machine code for RISC processors from modules written in Oberon and stores it in corresponding object files.
For debugging purposes, it also creates a debugging information file as well as an assembly file containing a listing of the generated machine code.
In addition, it stores the interface of each module in a symbol file which is required when other modules import the module.
Programs generated with this compiler require additional runtime support that is stored in the \file{ob\-risc\-run} library file.
\flowgraph{\resource{Oberon\\source code} \ar[r] & \toolbox{obrisc} \ar[r] \ar@/l/[d] \ar[rd] & \resource{object file} \\ \variable{ECSIMPORT} \ar[ru] & \resource{symbol\\files} \ar@/r/[u] & \resource{debugging\\information}}
\seeoberon\seeassembly\seerisc\seeobject\seedebugging
}

\providecommand{\obwasm}{
\toolsection{obwasm} is a compiler for the Oberon programming language targeting the WebAssembly architecture.
It generates machine code for WebAssembly targets from modules written in Oberon and stores it in corresponding object files.
For debugging purposes, it also creates a debugging information file as well as an assembly file containing a listing of the generated machine code.
In addition, it stores the interface of each module in a symbol file which is required when other modules import the module.
Programs generated with this compiler require additional runtime support that is stored in the \file{ob\-wasm\-run} library file.
\flowgraph{\resource{Oberon\\source code} \ar[r] & \toolbox{obwasm} \ar[r] \ar@/l/[d] \ar[rd] & \resource{object file} \\ \variable{ECSIMPORT} \ar[ru] & \resource{symbol\\files} \ar@/r/[u] & \resource{debugging\\information}}
\seeoberon\seeassembly\seewasm\seeobject\seedebugging
}

% converter tools

\providecommand{\dbgdwarf}{
\toolsection{dbgdwarf} is a DWARF debugging information converter tool.
It converts debugging information into the DWARF debugging data format and stores it in corresponding object files~\cite{dwarffile}.
The resulting debugging object files can be combined with runtime support that creates Executable and Linking Format (ELF) files~\cite{elffile}.
\flowgraph{\resource{debugging\\information} \ar[r] & \toolbox{dbgdwarf} \ar[r] & \resource{debugging\\object file}}
\seeobject\seedebugging
}

% assembler tools

\providecommand{\asmprint}{
\toolsection{asmprint} is a pretty printer for generic assembly code.
It reformats generic assembly code and writes it to the standard output stream.
\flowgraph{\resource{generic assembly\\source code} \ar[r] & \toolbox{asmprint} \ar[r] & \resource{reformatted\\source code}}
\seeassembly
}

\providecommand{\amdaasm}{
\toolsection{amd16asm} is an assembler for the AMD64 hardware architecture.
It translates assembly code into machine code for AMD64 processors and stores it in corresponding object files.
By default, the assembler generates machine code for the 16-bit operating mode defined by the AMD64 architecture.
\flowgraph{\resource{AMD16 assembly\\source code} \ar[r] & \toolbox{amd16asm} \ar[r] & \resource{object file}}
\seeassembly\seeamd\seeobject
}

\providecommand{\amdadism}{
\toolsection{amd16dism} is a disassembler for the AMD64 hardware architecture.
It translates machine code from object files targeting AMD64 processors into assembly code and writes it to the standard output stream.
It assumes that the machine code was generated for the 16-bit operating mode defined by the AMD64 architecture.
\flowgraph{\resource{object file} \ar[r] & \toolbox{amd16dism} \ar[r] & \resource{disassembly\\listing}}
\seeassembly\seeamd\seeobject
}

\providecommand{\amdbasm}{
\toolsection{amd32asm} is an assembler for the AMD64 hardware architecture.
It translates assembly code into machine code for AMD64 processors and stores it in corresponding object files.
By default, the assembler generates machine code for the 32-bit operating mode defined by the AMD64 architecture.
\flowgraph{\resource{AMD32 assembly\\source code} \ar[r] & \toolbox{amd32asm} \ar[r] & \resource{object file}}
\seeassembly\seeamd\seeobject
}

\providecommand{\amdbdism}{
\toolsection{amd32dism} is a disassembler for the AMD64 hardware architecture.
It translates machine code from object files targeting AMD64 processors into assembly code and writes it to the standard output stream.
It assumes that the machine code was generated for the 32-bit operating mode defined by the AMD64 architecture.
\flowgraph{\resource{object file} \ar[r] & \toolbox{amd32dism} \ar[r] & \resource{disassembly\\listing}}
\seeassembly\seeamd\seeobject
}

\providecommand{\amdcasm}{
\toolsection{amd64asm} is an assembler for the AMD64 hardware architecture.
It translates assembly code into machine code for AMD64 processors and stores it in corresponding object files.
By default, the assembler generates machine code for the 64-bit operating mode defined by the AMD64 architecture.
\flowgraph{\resource{AMD64 assembly\\source code} \ar[r] & \toolbox{amd64asm} \ar[r] & \resource{object file}}
\seeassembly\seeamd\seeobject
}

\providecommand{\amdcdism}{
\toolsection{amd64dism} is a disassembler for the AMD64 hardware architecture.
It translates machine code from object files targeting AMD64 processors into assembly code and writes it to the standard output stream.
It assumes that the machine code was generated for the 64-bit operating mode defined by the AMD64 architecture.
\flowgraph{\resource{object file} \ar[r] & \toolbox{amd64dism} \ar[r] & \resource{disassembly\\listing}}
\seeassembly\seeamd\seeobject
}

\providecommand{\armaasm}{
\toolsection{arma32asm} is an assembler for the ARM hardware architecture.
It translates assembly code into machine code for ARM processors executing A32 instructions and stores it in corresponding object files.
\flowgraph{\resource{ARM A32 assembly\\source code} \ar[r] & \toolbox{arma32asm} \ar[r] & \resource{object file}}
\seeassembly\seearm\seeobject
}

\providecommand{\armadism}{
\toolsection{arma32dism} is a disassembler for the ARM hardware architecture.
It translates machine code from object files targeting ARM processors executing A32 instructions into assembly code and writes it to the standard output stream.
\flowgraph{\resource{object file} \ar[r] & \toolbox{arma32dism} \ar[r] & \resource{disassembly\\listing}}
\seeassembly\seearm\seeobject
}

\providecommand{\armbasm}{
\toolsection{arma64asm} is an assembler for the ARM hardware architecture.
It translates assembly code into machine code for ARM processors executing A64 instructions and stores it in corresponding object files.
\flowgraph{\resource{ARM A64 assembly\\source code} \ar[r] & \toolbox{arma64asm} \ar[r] & \resource{object file}}
\seeassembly\seearm\seeobject
}

\providecommand{\armbdism}{
\toolsection{arma64dism} is a disassembler for the ARM hardware architecture.
It translates machine code from object files targeting ARM processors executing A64 instructions into assembly code and writes it to the standard output stream.
\flowgraph{\resource{object file} \ar[r] & \toolbox{arma64dism} \ar[r] & \resource{disassembly\\listing}}
\seeassembly\seearm\seeobject
}

\providecommand{\armcasm}{
\toolsection{armt32asm} is an assembler for the ARM hardware architecture.
It translates assembly code into machine code for ARM processors executing T32 instructions and stores it in corresponding object files.
\flowgraph{\resource{ARM T32 assembly\\source code} \ar[r] & \toolbox{armt32asm} \ar[r] & \resource{object file}}
\seeassembly\seearm\seeobject
}

\providecommand{\armcdism}{
\toolsection{armt32dism} is a disassembler for the ARM hardware architecture.
It translates machine code from object files targeting ARM processors executing T32 instructions into assembly code and writes it to the standard output stream.
\flowgraph{\resource{object file} \ar[r] & \toolbox{armt32dism} \ar[r] & \resource{disassembly\\listing}}
\seeassembly\seearm\seeobject
}

\providecommand{\avrasm}{
\toolsection{avrasm} is an assembler for the AVR hardware architecture.
It translates assembly code into machine code for AVR processors and stores it in corresponding object files.
The identifiers \texttt{RXL}, \texttt{RXH}, \texttt{RYL}, \texttt{RYH}, \texttt{RZL}, and \texttt{RZH} are predefined and name the corresponding registers.
The identifiers \texttt{SPL} and \texttt{SPH} are also predefined and evaluate to the address of the corresponding registers.
\flowgraph{\resource{AVR assembly\\source code} \ar[r] & \toolbox{avrasm} \ar[r] & \resource{object file}}
\seeassembly\seeavr\seeobject
}

\providecommand{\avrdism}{
\toolsection{avrdism} is a disassembler for the AVR hardware architecture.
It translates machine code from object files targeting AVR processors into assembly code and writes it to the standard output stream.
\flowgraph{\resource{object file} \ar[r] & \toolbox{avrdism} \ar[r] & \resource{disassembly\\listing}}
\seeassembly\seeavr\seeobject
}

\providecommand{\avrttasm}{
\toolsection{avr32asm} is an assembler for the AVR32 hardware architecture.
It translates assembly code into machine code for AVR32 processors and stores it in corresponding object files.
\flowgraph{\resource{AVR32 assembly\\source code} \ar[r] & \toolbox{avr32asm} \ar[r] & \resource{object file}}
\seeassembly\seeavrtt\seeobject
}

\providecommand{\avrttdism}{
\toolsection{avr32dism} is a disassembler for the AVR32 hardware architecture.
It translates machine code from object files targeting AVR32 processors into assembly code and writes it to the standard output stream.
\flowgraph{\resource{object file} \ar[r] & \toolbox{avr32dism} \ar[r] & \resource{disassembly\\listing}}
\seeassembly\seeavrtt\seeobject
}

\providecommand{\mabkasm}{
\toolsection{m68kasm} is an assembler for the M68000 hardware architecture.
It translates assembly code into machine code for M68000 processors and stores it in corresponding object files.
\flowgraph{\resource{68000 assembly\\source code} \ar[r] & \toolbox{m68kasm} \ar[r] & \resource{object file}}
\seeassembly\seemabk\seeobject
}

\providecommand{\mabkdism}{
\toolsection{m68kdism} is a disassembler for the M68000 hardware architecture.
It translates machine code from object files targeting M68000 processors into assembly code and writes it to the standard output stream.
\flowgraph{\resource{object file} \ar[r] & \toolbox{m68kdism} \ar[r] & \resource{disassembly\\listing}}
\seeassembly\seemabk\seeobject
}

\providecommand{\miblasm}{
\toolsection{miblasm} is an assembler for the MicroBlaze hardware architecture.
It translates assembly code into machine code for MicroBlaze processors and stores it in corresponding object files.
\flowgraph{\resource{MicroBlaze assembly\\source code} \ar[r] & \toolbox{miblasm} \ar[r] & \resource{object file}}
\seeassembly\seemibl\seeobject
}

\providecommand{\mibldism}{
\toolsection{mibldism} is a disassembler for the MicroBlaze hardware architecture.
It translates machine code from object files targeting MicroBlaze processors into assembly code and writes it to the standard output stream.
\flowgraph{\resource{object file} \ar[r] & \toolbox{mibldism} \ar[r] & \resource{disassembly\\listing}}
\seeassembly\seemibl\seeobject
}

\providecommand{\mipsaasm}{
\toolsection{mips32asm} is an assembler for the MIPS32 hardware architecture.
It translates assembly code into machine code for MIPS32 processors and stores it in corresponding object files.
\flowgraph{\resource{MIPS32 assembly\\source code} \ar[r] & \toolbox{mips32asm} \ar[r] & \resource{object file}}
\seeassembly\seemips\seeobject
}

\providecommand{\mipsadism}{
\toolsection{mips32dism} is a disassembler for the MIPS32 hardware architecture.
It translates machine code from object files targeting MIPS32 processors into assembly code and writes it to the standard output stream.
\flowgraph{\resource{object file} \ar[r] & \toolbox{mips32dism} \ar[r] & \resource{disassembly\\listing}}
\seeassembly\seemips\seeobject
}

\providecommand{\mipsbasm}{
\toolsection{mips64asm} is an assembler for the MIPS64 hardware architecture.
It translates assembly code into machine code for MIPS64 processors and stores it in corresponding object files.
\flowgraph{\resource{MIPS64 assembly\\source code} \ar[r] & \toolbox{mips64asm} \ar[r] & \resource{object file}}
\seeassembly\seemips\seeobject
}

\providecommand{\mipsbdism}{
\toolsection{mips64dism} is a disassembler for the MIPS64 hardware architecture.
It translates machine code from object files targeting MIPS64 processors into assembly code and writes it to the standard output stream.
\flowgraph{\resource{object file} \ar[r] & \toolbox{mips64dism} \ar[r] & \resource{disassembly\\listing}}
\seeassembly\seemips\seeobject
}

\providecommand{\mmixasm}{
\toolsection{mmixasm} is an assembler for the MMIX hardware architecture.
It translates assembly code into machine code for MMIX processors and stores it in corresponding object files.
The names of all special registers are predefined and evaluate to the corresponding number.
\flowgraph{\resource{MMIX assembly\\source code} \ar[r] & \toolbox{mmixasm} \ar[r] & \resource{object file}}
\seeassembly\seemmix\seeobject
}

\providecommand{\mmixdism}{
\toolsection{mmixdism} is a disassembler for the MMIX hardware architecture.
It translates machine code from object files targeting MMIX processors into assembly code and writes it to the standard output stream.
\flowgraph{\resource{object file} \ar[r] & \toolbox{mmixdism} \ar[r] & \resource{disassembly\\listing}}
\seeassembly\seemmix\seeobject
}

\providecommand{\orokasm}{
\toolsection{or1kasm} is an assembler for the OpenRISC 1000 hardware architecture.
It translates assembly code into machine code for OpenRISC 1000 processors and stores it in corresponding object files.
\flowgraph{\resource{OpenRISC 1000 assembly\\source code} \ar[r] & \toolbox{or1kasm} \ar[r] & \resource{object file}}
\seeassembly\seeorok\seeobject
}

\providecommand{\orokdism}{
\toolsection{or1kdism} is a disassembler for the OpenRISC 1000 hardware architecture.
It translates machine code from object files targeting OpenRISC 1000 processors into assembly code and writes it to the standard output stream.
\flowgraph{\resource{object file} \ar[r] & \toolbox{or1kdism} \ar[r] & \resource{disassembly\\listing}}
\seeassembly\seeorok\seeobject
}

\providecommand{\ppcaasm}{
\toolsection{ppc32asm} is an assembler for the PowerPC hardware architecture.
It translates assembly code into machine code for PowerPC processors and stores it in corresponding object files.
By default, the assembler generates machine code for the 32-bit operating mode defined by the PowerPC architecture.
\flowgraph{\resource{PowerPC assembly\\source code} \ar[r] & \toolbox{ppc32asm} \ar[r] & \resource{object file}}
\seeassembly\seeppc\seeobject
}

\providecommand{\ppcadism}{
\toolsection{ppc32dism} is a disassembler for the PowerPC hardware architecture.
It translates machine code from object files targeting PowerPC processors into assembly code and writes it to the standard output stream.
It assumes that the machine code was generated for the 32-bit operating mode defined by the PowerPC architecture.
\flowgraph{\resource{object file} \ar[r] & \toolbox{ppc32dism} \ar[r] & \resource{disassembly\\listing}}
\seeassembly\seeppc\seeobject
}

\providecommand{\ppcbasm}{
\toolsection{ppc64asm} is an assembler for the PowerPC hardware architecture.
It translates assembly code into machine code for PowerPC processors and stores it in corresponding object files.
By default, the assembler generates machine code for the 64-bit operating mode defined by the PowerPC architecture.
\flowgraph{\resource{PowerPC assembly\\source code} \ar[r] & \toolbox{ppc64asm} \ar[r] & \resource{object file}}
\seeassembly\seeppc\seeobject
}

\providecommand{\ppcbdism}{
\toolsection{ppc64dism} is a disassembler for the PowerPC hardware architecture.
It translates machine code from object files targeting PowerPC processors into assembly code and writes it to the standard output stream.
It assumes that the machine code was generated for the 64-bit operating mode defined by the PowerPC architecture.
\flowgraph{\resource{object file} \ar[r] & \toolbox{ppc64dism} \ar[r] & \resource{disassembly\\listing}}
\seeassembly\seeppc\seeobject
}

\providecommand{\riscasm}{
\toolsection{riscasm} is an assembler for the RISC hardware architecture.
It translates assembly code into machine code for RISC processors and stores it in corresponding object files.
The names of all special registers are predefined and evaluate to the corresponding number.
\flowgraph{\resource{RISC assembly\\source code} \ar[r] & \toolbox{riscasm} \ar[r] & \resource{object file}}
\seeassembly\seerisc\seeobject
}

\providecommand{\riscdism}{
\toolsection{riscdism} is a disassembler for the RISC hardware architecture.
It translates machine code from object files targeting RISC processors into assembly code and writes it to the standard output stream.
\flowgraph{\resource{object file} \ar[r] & \toolbox{riscdism} \ar[r] & \resource{disassembly\\listing}}
\seeassembly\seerisc\seeobject
}

\providecommand{\wasmasm}{
\toolsection{wasmasm} is an assembler for the WebAssembly architecture.
It translates assembly code into machine code for WebAssembly targets and stores it in corresponding object files.
The names of all special registers are predefined and evaluate to the corresponding number.
\flowgraph{\resource{WebAssembly assembly\\source code} \ar[r] & \toolbox{wasmasm} \ar[r] & \resource{object file}}
\seeassembly\seewasm\seeobject
}

\providecommand{\wasmdism}{
\toolsection{wasmdism} is a disassembler for the WebAssembly architecture.
It translates machine code from object files targeting WebAssembly targets into assembly code and writes it to the standard output stream.
\flowgraph{\resource{object file} \ar[r] & \toolbox{wasmdism} \ar[r] & \resource{disassembly\\listing}}
\seeassembly\seewasm\seeobject
}

% linker tools

\providecommand{\linklib}{
\toolsection{linklib} is an object file combiner.
It creates a static library file by combining all object files given to it into a single one.
\flowgraph{\resource{object files} \ar[r] & \toolbox{linklib} \ar[r] & \resource{library file}}
\seeobject
}

\providecommand{\linkbin}{
\toolsection{linkbin} is a linker for plain binary files.
It links all object files given to it into a single image and stores it in a binary file that begins with the first linked section.
It also creates a map file that lists the address, type, name and size of all used sections.
The filename extension of the resulting binary file can be specified by putting it into a constant data section called \texttt{\_extension}.
\flowgraph{\resource{object files} \ar[r] & \toolbox{linkbin} \ar[r] \ar[d] & \resource{binary file} \\ & \resource{map file}}
\seeobject
}

\providecommand{\linkmem}{
\toolsection{linkmem} is a linker for plain binary files partitioned into random-access and read-only memory.
It links all object files given to it into two distinct images, one for data sections and one for code and constant data sections, and stores each image in a binary file that begins with the first linked section of the corresponding type.
It also creates a map file that lists the address, type, name and size of all used sections.
\flowgraph{\resource{object files} \ar[r] & \toolbox{linkmem} \ar[r] \ar[d] & \resource{RAM file/\\ROM file} \\ & \resource{map file}}
\seeobject
}

\providecommand{\linkprg}{
\toolsection{linkprg} is a linker for GEMDOS executable files.
It links all object files given to it into a single image and stores the image in an Atari GEMDOS executable file~\cite{gemdosfile}.
It also creates a map file that lists the address relative to the text segment, type, name and size of all used sections.
The filename extension of the resulting executable file can be specified by putting it into a constant data section called \texttt{\_extension}.
The GEMDOS executable file format requires all patch patterns of absolute link patches to consist of four full bitmasks with descending offsets.
\flowgraph{\resource{object files} \ar[r] & \toolbox{linkprg} \ar[r] \ar[d] & \resource{executable file} \\ & \resource{map file}}
\seeobject
}

\providecommand{\linkhex}{
\toolsection{linkhex} is a linker for Intel HEX files.
It links all code sections of the object files given to it into single image and stores the image in an Intel HEX file~\cite{hexfile} that begins with the first linked section.
It also creates a map file that lists the address, type, name and size of all used sections.
\flowgraph{\resource{object files} \ar[r] & \toolbox{linkhex} \ar[r] \ar[d] & \resource{HEX file} \\ & \resource{map file}}
\seeobject
}

\providecommand{\mapsearch}{
\toolsection{mapsearch} is a debugging tool.
It searches map files generated by linker tools for the name of a binary section that encompasses a memory address read from the standard input stream.
If additionally provided with one or more object files, it also stores an excerpt thereof in a separate object file called map search result which only contains the identified binary section for disassembling purposes.
\flowgraph{& \resource{map files/\\object files} \ar[d] \\ \resource{memory\\address} \ar[r] & \toolbox{mapsearch} \ar[r] \ar[d] & \resource{section name/\\relative offset} \\ & \resource{object file\\excerpt}}
\seeobject
}

\renewcommand{\seeavr}{}

\startchapter{AVR}{AVR Hardware Architecture Support}{avr}
{This \documentation{} describes how the \ecs{} supports the AVR hardware architecture.
This includes information about the assembler, disassembler, and the various compilers featured by the \ecs{} as well as the interoperability between these tools.}

\section{Introduction}

The \ecs{} features various compilers, an assembler, and a disassembler that target the AVR hardware architecture by Atmel.
Figure~\ref{fig:avrdataflow} shows the data flow in-between these tools.

\begin{figure}
\flowgraph{
\resource{intermediate\\code} \ar[d] & & \resource{assembly\\source code} \ar[d] \\
\converter{AVR\\Generator} \ar[r] \ar[rd] \ar[d] & \resource{assembly\\listing} \ar[r] & \converter{AVR\\Assembler} \ar[ld] \\
\resource{debugging\\information} & \resource{object file} \ar[d] \\
& \converter{AVR\\Disassembler} \ar[d] \\
& \resource{disassembly\\listing} \\
}\caption{Data flow within the tools targeting the AVR architecture}
\label{fig:avrdataflow}
\end{figure}

All compilers targeting the AVR architecture translate their programs using an intermediate code representation.
The AVR generator is able to translate the intermediate code representation of a program into machine code for AVR processors.
It stores the resulting binary code and data in so-called object files.
Additionally, the generator is able to create an assembly code listing of the machine code for debugging purposes.
This assembly code listing can also be processed by the assembler yielding exactly the same object file.
The disassembler is able to open object files and print a human-readable disassembly listing of their contents.
\seeobject\seecode

\section{Instruction Set}

Tools targeting the AVR architecture support the instruction set listed in Table~\ref{tab:avrset} and use the same assembly syntax as predefined by Atmel~\cite{avr:instructionset}.
\seeassembly

\instructionset{avr}{Supported AVR instruction set}{5}{6}

\section{Calling Convention}\index{Calling convention!of AVR}

The machine code generator and runtime support for the AVR architecture as provided by the \ecs{} use the following calling convention in order to enable interoperability.
In general, the order of memory accesses to values that consist of several octets is most significant byte first or big-endian respectively.

\subsection{Stack Operations}

Arguments for functions are in general passed using the stack according to the intermediate code specification.
See \Documentation{}~\documentationref{code}{Intermediate Code Representation} for more information about the role of the stack.
Function arguments are pushed on the stack in reverse order and cleaned by the caller.
Because the AVR instructions \texttt{push} and \texttt{pop} modify the stack register in a different order with respect to the one defined by the intermediate code,
all data on the stack is accessed using a one octet displacement.

\subsection{Floating-Point Support}

The AVR architecture does not support any floating-point operations natively.
The AVR runtime support has currently no software emulation for floating-point operations.

\subsection{Register Mapping}

The special-purpose registers defined by the intermediate code representation are mapped to their corresponding physical registers in the following way:

\begin{itemize}

\item Result Register\alignright\texttt{\$res}\nopagebreak

The intermediate code result register \texttt{\$res} is mapped to AVR registers \texttt{r0} up to \texttt{r7} depending on the size of the actual return type.

\item Stack Pointer Register\alignright\texttt{\$sp}\nopagebreak

The intermediate code stack pointer register \texttt{\$sp} is mapped to AVR registers \texttt{SPL} and \texttt{SPH} respectively.

\item Frame Pointer Register\alignright\texttt{\$fp}\nopagebreak

The intermediate code frame pointer register \texttt{\$fp} is mapped to AVR registers \texttt{r24} and \texttt{r25} respectively.

\item Link Register\alignright\texttt{\$lnk}\nopagebreak

The intermediate code link register \texttt{\$lnk} is not supported.

\end{itemize}

All other intermediate code registers are mapped as needed to the remaining physical registers.
Their contents and mapping are therefore considered volatile across function calls.

\section{Runtime Support}\index{Runtime support!for AVR}

The \ecs{} provides runtime support for the AVR architecture and runtime environments based on this hardware architecture in object files.
Users targeting a specific runtime environment have to use an appropriate linker together with these object files in order create an executable program.
This section gives information about all supported runtime environments based on the AVR hardware architecture as well as the required combination of linker and object files.

Basic architectural runtime support is provided by the object file \objfile{avr\-run}.
Users should always include this object file during linking regardless of the actual target runtime environment.
All other object files given to the linker should target the same hardware architecture.

The \ecs{} additionally supports remote execution of programs targeting AVR processors.
The runtime support is stored in the \objfile{avr\-re\-mote} object file.
It synchronizes the start of a program and its result with a host machine using the standard output stream.

Programs written in \cpp{} need additional runtime support stored in the \libfile{cpp\-avr\-run} library file.
Programs written in Oberon need additional runtime support stored in the \libfile{ob\-avr\-run} library file.
\seecpp\seeoberon

\subsection{ATmega32 Microcontrollers}

Programs targeting the ATmega32 microcontroller by Atmel~\cite{atmega32} are created using the \tool{link\-hex} linker tool.
It creates an Intel HEX file~\cite{hexfile} for programming the microcontroller if provided with the runtime support stored in \objfile{at\-mega32\-run} object file.
Calling the \tool{ecsd} utility tool using the \environment{at\-mega32} target environment achieves the same result.
The USART communication device is configured to use a baud rate of 115200~bps with eight data bits, no parity, and one stop bit.

\subsection{ATmega328 Microcontrollers}

Programs targeting the ATmega328 microcontroller by Atmel~\cite{atmega328} are created using the \tool{link\-hex} linker tool.
It creates an Intel HEX file~\cite{hexfile} for programming the microcontroller if provided with the runtime support stored in \objfile{at\-mega328\-run} object file.
Calling the \tool{ecsd} utility tool using the \environment{at\-mega328} target environment achieves the same result.
The USART communication device is configured to use a baud rate of 115200~bps with eight data bits, no parity, and one stop bit.

\subsection{ATmega8515 Microcontrollers}

Programs targeting the ATmega8515 microcontroller by Atmel~\cite{atmega8515} are created using the \tool{link\-hex} linker tool.
It creates an Intel HEX file~\cite{hexfile} for programming the microcontroller if provided with the runtime support stored in \objfile{at\-mega8515\-run} object file.
Calling the \tool{ecsd} utility tool using the \environment{at\-mega8515} target environment achieves the same result.
The USART communication device is configured to use a baud rate of 115200~bps with eight data bits, no parity, and one stop bit.

\section{AVR Tools}

The \ecs{} provides the following tools that are able to process object files targeting the AVR hardware architecture.
\interface

\cdavr
\cppavr
\falavr
\obavr
\avrasm
\avrdism
\linkhex

\concludechapter

// AVR32 instruction set definitions
// Copyright (C) Florian Negele

// This file is part of the Eigen Compiler Suite.

// The ECS is free software: you can redistribute it and/or modify
// it under the terms of the GNU General Public License as published by
// the Free Software Foundation, either version 3 of the License, or
// (at your option) any later version.

// The ECS is distributed in the hope that it will be useful,
// but WITHOUT ANY WARRANTY; without even the implied warranty of
// MERCHANTABILITY or FITNESS FOR A PARTICULAR PURPOSE.  See the
// GNU General Public License for more details.

// You should have received a copy of the GNU General Public License
// along with the ECS.  If not, see <https://www.gnu.org/licenses/>.

#ifndef INSTR
	#define INSTR(mnem, code, mask, type0, type1, type2, type3, type4)
#endif

#ifndef MNEM
	#define MNEM(name, mnem, ...)
#endif

#ifndef TYPE
	#define TYPE(type)
#endif

// mnemonics

MNEM (abs,            ABS,           Absolute Value)
MNEM (acall,          ACALL,         Application Call)
MNEM (acr,            ACR,           Add Carry to Register)
MNEM (adc,            ADC,           Add with Carry)
MNEM (add,            ADD,           Add without Carry)
MNEM (addabs,         ADDABS,        Add Absolute Value)
MNEM (addal,          ADDAL,         Conditional Add Always)
MNEM (addcc,          ADDCC,         Conditional Add if Carry Cleared)
MNEM (addcs,          ADDCS,         Conditional Add if Carry Set)
MNEM (addeq,          ADDEQ,         Conditional Add if Equal)
MNEM (addge,          ADDGE,         Conditional Add if Greater than or Equal)
MNEM (addgt,          ADDGT,         Conditional Add if Greater Than)
MNEM (addhh.w,        ADDHHW,        Add Halfwords into Word)
MNEM (addhi,          ADDHI,         Conditional Add if Higher)
MNEM (addhs,          ADDHS,         Conditional Add if Higher or Same)
MNEM (addle,          ADDLE,         Conditional Add if Less than or Equal)
MNEM (addlo,          ADDLO,         Conditional Add if Lower)
MNEM (addls,          ADDLS,         Conditional Add if Lower or Same)
MNEM (addlt,          ADDLT,         Conditional Add if Less Than)
MNEM (addmi,          ADDMI,         Conditional Add if Negative)
MNEM (addne,          ADDNE,         Conditional Add if Not Equal)
MNEM (addpl,          ADDPL,         Conditional Add if Positive)
MNEM (addqs,          ADDQS,         Conditional Add if Saturated)
MNEM (addvc,          ADDVC,         Conditional Add if Overflow Cleared)
MNEM (addvs,          ADDVS,         Conditional Add if Overflow Set)
MNEM (and,            AND,           Logical AND)
MNEM (andal,          ANDAL,         Conditional AND Always)
MNEM (andcc,          ANDCC,         Conditional AND if Carry Cleared)
MNEM (andcs,          ANDCS,         Conditional AND if Carry Set)
MNEM (andeq,          ANDEQ,         Conditional AND if Equal)
MNEM (andge,          ANDGE,         Conditional AND if Greater than or Equal)
MNEM (andgt,          ANDGT,         Conditional AND if Greater Than)
MNEM (andh,           ANDH,          Logical AND into High Half of Register)
MNEM (andhi,          ANDHI,         Conditional AND if Higher)
MNEM (andhs,          ANDHS,         Conditional AND if Higher or Same)
MNEM (andl,           ANDL,          Logical AND into Low Half of Register)
MNEM (andle,          ANDLE,         Conditional AND if Less than or Equal)
MNEM (andlo,          ANDLO,         Conditional AND if Lower)
MNEM (andls,          ANDLS,         Conditional AND if Lower or Same)
MNEM (andlt,          ANDLT,         Conditional AND if Less Than)
MNEM (andmi,          ANDMI,         Conditional AND if Negative)
MNEM (andn,           ANDN,          Logical AND NOT)
MNEM (andne,          ANDNE,         Conditional AND if Not Equal)
MNEM (andpl,          ANDPL,         Conditional AND if Positive)
MNEM (andqs,          ANDQS,         Conditional AND if Saturated)
MNEM (andvc,          ANDVC,         Conditional AND if Overflow Cleared)
MNEM (andvs,          ANDVS,         Conditional AND if Overflow Set)
MNEM (asr,            ASR,           Arithmetic Shift Right)
MNEM (bfexts,         BFEXTS,        Bitfield Extract and Sign-extend)
MNEM (bfextu,         BFEXTU,        Bitfield Extract and Zero-extend)
MNEM (bfins,          BFINS,         Bitfield Insert)
MNEM (bld,            BLD,           Bit Load from Register to C and Z)
MNEM (bral,           BRAL,          Branch Always)
MNEM (brcc,           BRCC,          Branch if Carry Cleared)
MNEM (brcs,           BRCS,          Branch if Carry Set)
MNEM (breakpoint,     BREAKPOINT,    Software Debug Breakpoint)
MNEM (breq,           BREQ,          Branch if Equal)
MNEM (brev,           BREV,          Bit Reverse)
MNEM (brge,           BRGE,          Branch if Greater than or Equal)
MNEM (brgt,           BRGT,          Branch if Greater Than)
MNEM (brhi,           BRHI,          Branch if Higher)
MNEM (brhs,           BRHS,          Branch if Higher or Same)
MNEM (brle,           BRLE,          Branch if Less than or Equal)
MNEM (brlo,           BRLO,          Branch if Lower)
MNEM (brls,           BRLS,          Branch if Lower or Same)
MNEM (brlt,           BRLT,          Branch if Less Than)
MNEM (brmi,           BRMI,          Branch if Less Negative)
MNEM (brne,           BRNE,          Branch if Not Equal)
MNEM (brpl,           BRPL,          Branch if Less Positive)
MNEM (brqs,           BRQS,          Branch if Saturated)
MNEM (brvc,           BRVC,          Branch if Overflow Cleared)
MNEM (brvs,           BRVS,          Branch if Overflow Set)
MNEM (bst,            BST,           Copy C to Register Bit)
MNEM (cache,          CACHE,         Perform Cache Control Operation)
MNEM (casts.b,        CASTSB,        Typecast Byte to Signed Word)
MNEM (casts.h,        CASTSH,        Typecast Halfword to Signed Word)
MNEM (castu.b,        CASTUB,        Typecast Byte to Unsigned Word)
MNEM (castu.h,        CASTUH,        Typecast Halfword to Unsigned Word)
MNEM (cbr,            CBR,           Clear Bit in Register)
MNEM (clz,            CLZ,           Count Leading Zeros)
MNEM (com,            COM,           One`s Compliment)
MNEM (cop,            COP,           Coprocessor Operation)
MNEM (cp.b,           CPB,           Compare Byte)
MNEM (cp.h,           CPH,           Compare Halfword)
MNEM (cp.w,           CPW,           Compare Word)
MNEM (cpc,            CPC,           Compare with Carry)
MNEM (csrf,           CSRF,          Clear Status Register Flag)
MNEM (csrfcz,         CSRFCZ,        Copy Status Register Flag to C and Z)
MNEM (divs,           DIVS,          Signed Divide)
MNEM (divu,           DIVU,          Unsigned Divide)
MNEM (eor,            EOR,           Logical EOR)
MNEM (eoral,          EORAL,         Conditional Logical EOR Always)
MNEM (eorcc,          EORCC,         Conditional Logical EOR if Carry Cleared)
MNEM (eorcs,          EORCS,         Conditional Logical EOR if Carry Set)
MNEM (eoreq,          EOREQ,         Conditional Logical EOR if Equal)
MNEM (eorge,          EORGE,         Conditional Logical EOR if Greater than or Equal)
MNEM (eorgt,          EORGT,         Conditional Logical EOR if Greater Than)
MNEM (eorh,           EORH,          Logical EOR into High Half of Register)
MNEM (eorhi,          EORHI,         Conditional Logical EOR if Higher)
MNEM (eorhs,          EORHS,         Conditional Logical EOR if Higher or Same)
MNEM (eorl,           EORL,          Logical EOR into Low Half of Register)
MNEM (eorle,          EORLE,         Conditional Logical EOR if Less than or Equal)
MNEM (eorlo,          EORLO,         Conditional Logical EOR if Lower)
MNEM (eorls,          EORLS,         Conditional Logical EOR if Lower or Same)
MNEM (eorlt,          EORLT,         Conditional Logical EOR if Less Than)
MNEM (eormi,          EORMI,         Conditional Logical EOR if Negative)
MNEM (eorne,          EORNE,         Conditional Logical EOR if Not Equal)
MNEM (eorpl,          EORPL,         Conditional Logical EOR if Positive)
MNEM (eorqs,          EORQS,         Conditional Logical EOR if Saturated)
MNEM (eorvc,          EORVC,         Conditional Logical EOR if Overflow Cleared)
MNEM (eorvs,          EORVS,         Conditional Logical EOR if Overflow Set)
MNEM (frs,            FRS,           Flush Return Stack)
MNEM (icall,          ICALL,         Indirect Call to Subroutine)
MNEM (incjosp,        INCJOSP,       Increment Java Operand Stack Pointer)
MNEM (ld.d,           LDD,           Load Doubleword)
MNEM (ld.sb,          LDSB,          Load Sign-extended Byte)
MNEM (ld.sbal,        LDSBAL,        Conditionally Load Sign-extended Byte Always)
MNEM (ld.sbcc,        LDSBCC,        Conditionally Load Sign-extended Byte if Carry Cleared)
MNEM (ld.sbcs,        LDSBCS,        Conditionally Load Sign-extended Byte if Carry Set)
MNEM (ld.sbeq,        LDSBEQ,        Conditionally Load Sign-extended Byte if Equal)
MNEM (ld.sbge,        LDSBGE,        Conditionally Load Sign-extended Byte if Greater than or Equal)
MNEM (ld.sbgt,        LDSBGT,        Conditionally Load Sign-extended Byte if Greater Than)
MNEM (ld.sbhi,        LDSBHI,        Conditionally Load Sign-extended Byte if Higher)
MNEM (ld.sbhs,        LDSBHS,        Conditionally Load Sign-extended Byte if Higher or Same)
MNEM (ld.sble,        LDSBLE,        Conditionally Load Sign-extended Byte if Less than or Equal)
MNEM (ld.sblo,        LDSBLO,        Conditionally Load Sign-extended Byte if Lower)
MNEM (ld.sbls,        LDSBLS,        Conditionally Load Sign-extended Byte if Lower or Same)
MNEM (ld.sblt,        LDSBLT,        Conditionally Load Sign-extended Byte if Less Than)
MNEM (ld.sbmi,        LDSBMI,        Conditionally Load Sign-extended Byte if Negative)
MNEM (ld.sbne,        LDSBNE,        Conditionally Load Sign-extended Byte if Not Equal)
MNEM (ld.sbpl,        LDSBPL,        Conditionally Load Sign-extended Byte if Positive)
MNEM (ld.sbqs,        LDSBQS,        Conditionally Load Sign-extended Byte if Saturated)
MNEM (ld.sbvc,        LDSBVC,        Conditionally Load Sign-extended Byte if Overflow Cleared)
MNEM (ld.sbvs,        LDSBVS,        Conditionally Load Sign-extended Byte if Overflow Set)
MNEM (ld.sh,          LDSH,          Load Sign-extended Halfword)
MNEM (ld.shal,        LDSHAL,        Conditionally Load Sign-extended Halfword Always)
MNEM (ld.shcc,        LDSHCC,        Conditionally Load Sign-extended Halfword if Carry Cleared)
MNEM (ld.shcs,        LDSHCS,        Conditionally Load Sign-extended Halfword if Carry Set)
MNEM (ld.sheq,        LDSHEQ,        Conditionally Load Sign-extended Halfword if Equal)
MNEM (ld.shge,        LDSHGE,        Conditionally Load Sign-extended Halfword if Greater than or Equal)
MNEM (ld.shgt,        LDSHGT,        Conditionally Load Sign-extended Halfword if Greater Than)
MNEM (ld.shhi,        LDSHHI,        Conditionally Load Sign-extended Halfword if Higher)
MNEM (ld.shhs,        LDSHHS,        Conditionally Load Sign-extended Halfword if Higher or Same)
MNEM (ld.shle,        LDSHLE,        Conditionally Load Sign-extended Halfword if Less than or Equal)
MNEM (ld.shlo,        LDSHLO,        Conditionally Load Sign-extended Halfword if Lower)
MNEM (ld.shls,        LDSHLS,        Conditionally Load Sign-extended Halfword if Lower or Same)
MNEM (ld.shlt,        LDSHLT,        Conditionally Load Sign-extended Halfword if Less Than)
MNEM (ld.shmi,        LDSHMI,        Conditionally Load Sign-extended Halfword if Negative)
MNEM (ld.shne,        LDSHNE,        Conditionally Load Sign-extended Halfword if Not Equal)
MNEM (ld.shpl,        LDSHPL,        Conditionally Load Sign-extended Halfword if Positive)
MNEM (ld.shqs,        LDSHQS,        Conditionally Load Sign-extended Halfword if Saturated)
MNEM (ld.shvc,        LDSHVC,        Conditionally Load Sign-extended Halfword if Overflow Cleared)
MNEM (ld.shvs,        LDSHVS,        Conditionally Load Sign-extended Halfword if Overflow Set)
MNEM (ld.ub,          LDUB,          Load Zero-extended Byte)
MNEM (ld.ubal,        LDUBAL,        Conditionally Load Zero-extended Byte Always)
MNEM (ld.ubcc,        LDUBCC,        Conditionally Load Zero-extended Byte if Carry Cleared)
MNEM (ld.ubcs,        LDUBCS,        Conditionally Load Zero-extended Byte if Carry Set)
MNEM (ld.ubeq,        LDUBEQ,        Conditionally Load Zero-extended Byte if Equal)
MNEM (ld.ubge,        LDUBGE,        Conditionally Load Zero-extended Byte if Greater than or Equal)
MNEM (ld.ubgt,        LDUBGT,        Conditionally Load Zero-extended Byte if Greater Than)
MNEM (ld.ubhi,        LDUBHI,        Conditionally Load Zero-extended Byte if Higher)
MNEM (ld.ubhs,        LDUBHS,        Conditionally Load Zero-extended Byte if Higher or Same)
MNEM (ld.uble,        LDUBLE,        Conditionally Load Zero-extended Byte if Less than or Equal)
MNEM (ld.ublo,        LDUBLO,        Conditionally Load Zero-extended Byte if Lower)
MNEM (ld.ubls,        LDUBLS,        Conditionally Load Zero-extended Byte if Lower or Same)
MNEM (ld.ublt,        LDUBLT,        Conditionally Load Zero-extended Byte if Less Than)
MNEM (ld.ubmi,        LDUBMI,        Conditionally Load Zero-extended Byte if Negative)
MNEM (ld.ubne,        LDUBNE,        Conditionally Load Zero-extended Byte if Not Equal)
MNEM (ld.ubpl,        LDUBPL,        Conditionally Load Zero-extended Byte if Positive)
MNEM (ld.ubqs,        LDUBQS,        Conditionally Load Zero-extended Byte if Saturated)
MNEM (ld.ubvc,        LDUBVC,        Conditionally Load Zero-extended Byte if Overflow Cleared)
MNEM (ld.ubvs,        LDUBVS,        Conditionally Load Zero-extended Byte if Overflow Set)
MNEM (ld.uh,          LDUH,          Load Zero-extended Halfword)
MNEM (ld.uhal,        LDUHAL,        Conditionally Load Zero-extended Halfword Always)
MNEM (ld.uhcc,        LDUHCC,        Conditionally Load Zero-extended Halfword if Carry Cleared)
MNEM (ld.uhcs,        LDUHCS,        Conditionally Load Zero-extended Halfword if Carry Set)
MNEM (ld.uheq,        LDUHEQ,        Conditionally Load Zero-extended Halfword if Equal)
MNEM (ld.uhge,        LDUHGE,        Conditionally Load Zero-extended Halfword if Greater than or Equal)
MNEM (ld.uhgt,        LDUHGT,        Conditionally Load Zero-extended Halfword if Greater Than)
MNEM (ld.uhhi,        LDUHHI,        Conditionally Load Zero-extended Halfword if Higher)
MNEM (ld.uhhs,        LDUHHS,        Conditionally Load Zero-extended Halfword if Higher or Same)
MNEM (ld.uhle,        LDUHLE,        Conditionally Load Zero-extended Halfword if Less than or Equal)
MNEM (ld.uhlo,        LDUHLO,        Conditionally Load Zero-extended Halfword if Lower)
MNEM (ld.uhls,        LDUHLS,        Conditionally Load Zero-extended Halfword if Lower or Same)
MNEM (ld.uhlt,        LDUHLT,        Conditionally Load Zero-extended Halfword if Less Than)
MNEM (ld.uhmi,        LDUHMI,        Conditionally Load Zero-extended Halfword if Negative)
MNEM (ld.uhne,        LDUHNE,        Conditionally Load Zero-extended Halfword if Not Equal)
MNEM (ld.uhpl,        LDUHPL,        Conditionally Load Zero-extended Halfword if Positive)
MNEM (ld.uhqs,        LDUHQS,        Conditionally Load Zero-extended Halfword if Saturated)
MNEM (ld.uhvc,        LDUHVC,        Conditionally Load Zero-extended Halfword if Overflow Cleared)
MNEM (ld.uhvs,        LDUHVS,        Conditionally Load Zero-extended Halfword if Overflow Set)
MNEM (ld.w,           LDW,           Load Word)
MNEM (ld.wal,         LDWAL,         Conditionally Word Always)
MNEM (ld.wcc,         LDWCC,         Conditionally Word if Carry Cleared)
MNEM (ld.wcs,         LDWCS,         Conditionally Word if Carry Set)
MNEM (ld.weq,         LDWEQ,         Conditionally Word if Equal)
MNEM (ld.wge,         LDWGE,         Conditionally Word if Greater than or Equal)
MNEM (ld.wgt,         LDWGT,         Conditionally Word if Greater Than)
MNEM (ld.whi,         LDWHI,         Conditionally Word if Higher)
MNEM (ld.whs,         LDWHS,         Conditionally Word if Higher or Same)
MNEM (ld.wle,         LDWLE,         Conditionally Word if Less than or Equal)
MNEM (ld.wlo,         LDWLO,         Conditionally Word if Lower)
MNEM (ld.wls,         LDWLS,         Conditionally Word if Lower or Same)
MNEM (ld.wlt,         LDWLT,         Conditionally Word if Less Than)
MNEM (ld.wmi,         LDWMI,         Conditionally Word if Negative)
MNEM (ld.wne,         LDWNE,         Conditionally Word if Not Equal)
MNEM (ld.wpl,         LDWPL,         Conditionally Word if Positive)
MNEM (ld.wqs,         LDWQS,         Conditionally Word if Saturated)
MNEM (ld.wvc,         LDWVC,         Conditionally Word if Overflow Cleared)
MNEM (ld.wvs,         LDWVS,         Conditionally Word if Overflow Set)
MNEM (ldc.d,          LDCD,          Load Coprocessor Doubleword)
MNEM (ldc.w,          LDCW,          Load Coprocessor Word)
MNEM (ldc0.d,         LDC0D,         Load Doubleword into Coprocessor 0)
MNEM (ldc0.w,         LDC0W,         Load Word into Coprocessor 0)
MNEM (lddpc,          LDDPC,         Load PC-relative with Displacement)
MNEM (lddsp,          LDDSP,         Load SP-relative with Displacement)
MNEM (ldins.b,        LDINSB,        Load and Insert Byte into register)
MNEM (ldins.h,        LDINSH,        Load and Insert Halfword into register)
MNEM (ldswp.sh,       LDSWPSH,       Load and Swap Sign-extended Halfword)
MNEM (ldswp.uh,       LDSWPUH,       Load and Swap Zero-extended Halfword)
MNEM (ldswp.w,        LDSWPW,        Load and Swap Word)
MNEM (lsl,            LSL,           Logical Shift Left)
MNEM (lsr,            LSR,           Logical Shift Right)
MNEM (mac,            MAC,           Multiply Accumulate)
MNEM (machh.d,        MACHHD,        Multiply Halfwords and Accumulate in Doubleword)
MNEM (machh.w,        MACHHW,        Multiply Halfwords and Accumulate in Word)
MNEM (macs.d,         MACSD,         Multiply Accumulate Signed)
MNEM (macsathh.w,     MACSATHHW,     Multiply-Accumulate Halfwords with Saturation into Word)
MNEM (macu.d,         MACUD,         Multiply Accumulate Unsigned)
MNEM (macwh.d,        MACWHD,        Multiply Word with Halfword and Accumulate in Doubleword)
MNEM (max,            MAX,           Return Maximum Value)
MNEM (mcall,          MCALL,         Subroutine Call)
MNEM (memc,           MEMC,          Clear Bit in Memory)
MNEM (mems,           MEMS,          Set Bit in Memory)
MNEM (memt,           MEMT,          Toggle Bit in Memory)
MNEM (mfdr,           MFDR,          Move from Debug Register)
MNEM (mfsr,           MFSR,          Move from System Register)
MNEM (min,            MIN,           Return Minimum Value)
MNEM (mov,            MOV,           Move Data into Register)
MNEM (moval,          MOVAL,         Conditional Move Register Always)
MNEM (movcc,          MOVCC,         Conditional Move Register if Carry Cleared)
MNEM (movcs,          MOVCS,         Conditional Move Register if Carry Set)
MNEM (moveq,          MOVEQ,         Conditional Move Register if Equal)
MNEM (movge,          MOVGE,         Conditional Move Register if Greater than or Equal)
MNEM (movgt,          MOVGT,         Conditional Move Register if Greater Than)
MNEM (movh,           MOVH,          Move Data into High Halfword of Register)
MNEM (movhi,          MOVHI,         Conditional Move Register if Higher)
MNEM (movhs,          MOVHS,         Conditional Move Register if Higher or Same)
MNEM (movl,           MOVL,          Move Data into Low Halfword of Register)
MNEM (movle,          MOVLE,         Conditional Move Register if Less than or Equal)
MNEM (movlo,          MOVLO,         Conditional Move Register if Lower)
MNEM (movls,          MOVLS,         Conditional Move Register if Lower or Same)
MNEM (movlt,          MOVLT,         Conditional Move Register if Less Than)
MNEM (movmi,          MOVMI,         Conditional Move Register if Negative)
MNEM (movne,          MOVNE,         Conditional Move Register if Not Equal)
MNEM (movpl,          MOVPL,         Conditional Move Register if Positive)
MNEM (movqs,          MOVQS,         Conditional Move Register if Saturated)
MNEM (movvc,          MOVVC,         Conditional Move Register if Overflow Cleared)
MNEM (movvs,          MOVVS,         Conditional Move Register if Overflow Set)
MNEM (mtdr,           MTDR,          Move to Debug Register)
MNEM (mtsr,           MTSR,          Move to System Register)
MNEM (mul,            MUL,           Multiply)
MNEM (mulhh.w,        MULHHW,        Multiply Halfword with Halfword)
MNEM (mulnhh.w,       MULNHHW,       Multiply Halfword with Negated Halfword)
MNEM (mulnwh.d,       MULNWHD,       Multiply Word with Negated Halfword)
MNEM (muls.d,         MULSD,         Multiply Signed)
MNEM (mulsathh.h,     MULSATHHH,     Multiply Halfwords with Saturation into Halfword)
MNEM (mulsathh.w,     MULSATHHW,     Multiply Halfwords with Saturation into Word)
MNEM (mulsatrndhh.h,  MULSATRNDHHH,  Multiply Halfwords with Saturation and Rounding into Halfword)
MNEM (mulsatrndhh.w,  MULSATRNDHHW,  Multiply Halfwords with Saturation and Rounding into Word)
MNEM (mulsatwh.w,     MULSATWHW,     Multiply Word and Halfword with Saturation into Word)
MNEM (mulu.d,         MULUD,         Multiply Unsigned)
MNEM (mulwh.d,        MULWHD,        Multiply Word with Halfword)
MNEM (musfr,          MUSFR,         Copy Register to Status Register)
MNEM (mustr,          MUSTR,         Copy Status Register to Register)
MNEM (mvcr.d,         MVCRD,         Move Doubleword Coprocessor Register to Register file)
MNEM (mvcr.w,         MVCRW,         Move Word Coprocessor Register to Register file)
MNEM (mvrc.d,         MVRCD,         Move Doubleword Register file Register to Coprocessor Register)
MNEM (mvrc.w,         MVRCW,         Move Word Coprocessor Word Register to Register file)
MNEM (neg,            NEG,           Two`s Complement)
MNEM (nop,            NOP,           No Operation)
MNEM (or,             OR,            Logical OR)
MNEM (oral,           ORAL,          Conditional Logical OR Always)
MNEM (orcc,           ORCC,          Conditional Logical OR if Carry Cleared)
MNEM (orcs,           ORCS,          Conditional Logical OR if Carry Set)
MNEM (oreq,           OREQ,          Conditional Logical OR if Equal)
MNEM (orge,           ORGE,          Conditional Logical OR if Greater than or Equal)
MNEM (orgt,           ORGT,          Conditional Logical OR if Greater Than)
MNEM (orh,            ORH,           Logical OR into High Half of Register)
MNEM (orhi,           ORHI,          Conditional Logical OR if Higher)
MNEM (orhs,           ORHS,          Conditional Logical OR if Higher or Same)
MNEM (orl,            ORL,           Logical OR into Low Half of Register)
MNEM (orle,           ORLE,          Conditional Logical OR if Less than or Equal)
MNEM (orlo,           ORLO,          Conditional Logical OR if Lower)
MNEM (orls,           ORLS,          Conditional Logical OR if Lower or Same)
MNEM (orlt,           ORLT,          Conditional Logical OR if Less Than)
MNEM (ormi,           ORMI,          Conditional Logical OR if Negative)
MNEM (orne,           ORNE,          Conditional Logical OR if Not Equal)
MNEM (orpl,           ORPL,          Conditional Logical OR if Positive)
MNEM (orqs,           ORQS,          Conditional Logical OR if Saturated)
MNEM (orvc,           ORVC,          Conditional Logical OR if Overflow Cleared)
MNEM (orvs,           ORVS,          Conditional Logical OR if Overflow Set)
MNEM (pabs.sb,        PABSSB,        Packed Absolute Value of Signed Bytes)
MNEM (pabs.sh,        PABSSH,        Packed Absolute Value of Signed Halfwords)
MNEM (packsh.sb,      PACKSHSB,      Pack Signed Halfwords to Signed Bytes)
MNEM (packsh.ub,      PACKSHUB,      Pack Signed Halfwords to Unsigned Bytes)
MNEM (packw.sh,       PACKWSHS,      Pack Words to Signed Halfwords)
MNEM (padd.b,         PADDB,         Packed Addition on Bytes)
MNEM (padd.h,         PADDH,         Packed Addition on Halfwords)
MNEM (paddh.sh,       PADDHSH,       Packed Addition with Halving on Signed Halfwords)
MNEM (paddh.ub,       PADDHUB,       Packed Addition with Halving on Unsigned Bytes)
MNEM (padds.sb,       PADDSSB,       Packed Addition with Saturation on Signed Bytes)
MNEM (padds.sh,       PADDSSH,       Packed Addition with Saturation on Signed Halfwords)
MNEM (padds.ub,       PADDSUB,       Packed Addition with Saturation on Unsigned Bytes)
MNEM (padds.uh,       PADDSUH,       Packed Addition with Saturation on Unsigned Halfwords)
MNEM (paddsub.h,      PADDSUBH,      Packed Halfword Addition and Subtraction)
MNEM (paddsubh.sh,    PADDSUBHSH,    Packed Halfword Addition and Subtraction with Halving)
MNEM (paddsubs.sh,    PADDSUBSSH,    Packed Signed Halfword Addition and Subtraction with Saturation)
MNEM (paddsubs.uh,    PADDSUBSUH,    Packed Unsigned Halfword Addition and Subtraction with Saturation)
MNEM (paddx.h,        PADDXH,        Packed Halfword Addition with Crossed Operand)
MNEM (paddxh.sh,      PADDXHSH,      Packed Signed Halfword Addition with Crossed Operand and Halving)
MNEM (paddxs.sh,      PADDXSSH,      Packed Signed Halfword Addition with Crossed Operand and Saturation)
MNEM (paddxs.uh,      PADDXSUH,      Packed Unsigned Halfword Addition with Crossed Operand and Saturation)
MNEM (pasr.b,         PASRB,         Packed Arithmetic Shift Right on Bytes)
MNEM (pasr.h,         PASRH,         Packed Arithmetic Shift Right on Halfwords)
MNEM (pavg.sh,        PAVGSH,        Packed Average of Signed Halfwords)
MNEM (pavg.ub,        PAVGUB,        Packed Average of Unsigned Bytes)
MNEM (plsl.b,         PLSLB,         Packed Logical Shift Left on Bytes)
MNEM (plsl.h,         PLSLH,         Packed Logical Shift Left on Halfwords)
MNEM (plsr.b,         PLSRB,         Packed Logical Shift Right on Bytes)
MNEM (plsr.h,         PLSRH,         Packed Logical Shift Right on Halfwords)
MNEM (pmax.sh,        PMAXSH,        Packed Maximum Value of Signed Halfwords)
MNEM (pmax.ub,        PMAXUB,        Packed Maximum Value of Unsigned Bytes)
MNEM (pmin.sh,        PMINSH,        Packed Minimum Value of Signed Halfwords)
MNEM (pmin.ub,        PMINUB,        Packed Minimum Value of Unsigned Bytes)
MNEM (popjc,          POPJC,         Pop Java Context from Frame)
MNEM (pref,           PREF,          Cache Prefetch)
MNEM (psad,           PSAD,          Packed Sum of Absolute Differences)
MNEM (psub.b,         PSUBB,         Packed Subtraction on Bytes)
MNEM (psub.h,         PSUBH,         Packed Subtraction on Halfwords)
MNEM (psubadd.h,      PSUBADDH,      Packed Halfword Subtraction and Addition)
MNEM (psubaddh.sh,    PSUBADDHSH,    Packed Signed Halfword Subtraction and Addition with Halving)
MNEM (psubadds.sh,    PSUBADDSSH,    Packed Halfword Subtraction and Addition with Saturation on Signed Halfwords)
MNEM (psubadds.uh,    PSUBADDSUH,    Packed Halfword Subtraction and Addition with Saturation on Unsigned Halfwords)
MNEM (psubh.sh,       PSUBHSH,       Packed Subtraction with Halving on Signed Halfwords)
MNEM (psubh.ub,       PSUBHUB,       Packed Subtraction with Halving on Unsigned Bytes)
MNEM (psubs.sb,       PSUBSSB,       Packed Subtraction with Saturation on Signed Bytes)
MNEM (psubs.sh,       PSUBSSH,       Packed Subtraction with Saturation on Signed Halfwords)
MNEM (psubs.ub,       PSUBSUB,       Packed Subtraction with Saturation on Unsigned Bytes)
MNEM (psubs.uh,       PSUBSUH,       Packed Subtraction with Saturation on Unsigned Halfwords)
MNEM (psubx.h,        PSUBXH,        Packed Halfword Subtraction with Crossed Operand)
MNEM (psubxh.sh,      PSUBXHSH,      Packed Signed Halfword Subtraction with Crossed Operand and Halving)
MNEM (psubxs.sh,      PSUBXSSH,      Packed Signed Halfword Subtraction with Crossed Operand and Saturation)
MNEM (psubxs.uh,      PSUBXSUH,      Packed Unsigned Halfword Subtraction with Crossed Operand and Saturation)
MNEM (punpcksb.h,     PUNPCKSBH,     Unpack Signed Bytes to Halfwords)
MNEM (punpckub.h,     PUNPCKUBH,     Unpack Unsigned Bytes to Halfwords)
MNEM (pushjc,         PUSHJC,        Push Java Context to Frame)
MNEM (rcall,          RCALL,         Relative Subroutine Call)
MNEM (retal,          RETAL,         Conditional Return from Subroutine Always)
MNEM (retcc,          RETCC,         Conditional Return from Subroutine if Carry Cleared)
MNEM (retcs,          RETCS,         Conditional Return from Subroutine if Carry Set)
MNEM (retd,           RETD,          Return from Debug Mode)
MNEM (rete,           RETE,          Return from Event Handler)
MNEM (reteq,          RETEQ,         Conditional Return from Subroutine if Equal)
MNEM (retge,          RETGE,         Conditional Return from Subroutine if Greater than or Equal)
MNEM (retgt,          RETGT,         Conditional Return from Subroutine if Greater Than)
MNEM (rethi,          RETHI,         Conditional Return from Subroutine if Higher)
MNEM (reths,          RETHS,         Conditional Return from Subroutine if Higher or Same)
MNEM (retj,           RETJ,          Return from Java Trap)
MNEM (retle,          RETLE,         Conditional Return from Subroutine if Less than or Equal)
MNEM (retlo,          RETLO,         Conditional Return from Subroutine if Lower)
MNEM (retls,          RETLS,         Conditional Return from Subroutine if Lower or Same)
MNEM (retlt,          RETLT,         Conditional Return from Subroutine if Less Than)
MNEM (retmi,          RETMI,         Conditional Return from Subroutine if Negative)
MNEM (retne,          RETNE,         Conditional Return from Subroutine if Not Equal)
MNEM (retpl,          RETPL,         Conditional Return from Subroutine if Positive)
MNEM (retqs,          RETQS,         Conditional Return from Subroutine if Saturated)
MNEM (rets,           RETS,          Return from Supervisor Call)
MNEM (retss,          RETSS,         Return from Secure State)
MNEM (retvc,          RETVC,         Conditional Return from Subroutine if Overflow Cleared)
MNEM (retvs,          RETVS,         Conditional Return from Subroutine if Overflow Set)
MNEM (rjmp,           RJMP,          Relative Jump)
MNEM (rol,            ROL,           Rotate Left through Carry)
MNEM (ror,            ROR,           Rotate Right through Carry)
MNEM (rsub,           RSUB,          Reverse Subtract)
MNEM (rsubal,         RSUBAL,        Conditional Reverse Subtract Always)
MNEM (rsubcc,         RSUBCC,        Conditional Reverse Subtract if Carry Cleared)
MNEM (rsubcs,         RSUBCS,        Conditional Reverse Subtract if Carry Set)
MNEM (rsubeq,         RSUBEQ,        Conditional Reverse Subtract if Equal)
MNEM (rsubge,         RSUBGE,        Conditional Reverse Subtract if Greater than or Equal)
MNEM (rsubgt,         RSUBGT,        Conditional Reverse Subtract if Greater Than)
MNEM (rsubhi,         RSUBHI,        Conditional Reverse Subtract if Higher)
MNEM (rsubhs,         RSUBHS,        Conditional Reverse Subtract if Higher or Same)
MNEM (rsuble,         RSUBLE,        Conditional Reverse Subtract if Less than or Equal)
MNEM (rsublo,         RSUBLO,        Conditional Reverse Subtract if Lower)
MNEM (rsubls,         RSUBLS,        Conditional Reverse Subtract if Lower or Same)
MNEM (rsublt,         RSUBLT,        Conditional Reverse Subtract if Less Than)
MNEM (rsubmi,         RSUBMI,        Conditional Reverse Subtract if Negative)
MNEM (rsubne,         RSUBNE,        Conditional Reverse Subtract if Not Equal)
MNEM (rsubpl,         RSUBPL,        Conditional Reverse Subtract if Positive)
MNEM (rsubqs,         RSUBQS,        Conditional Reverse Subtract if Saturated)
MNEM (rsubvc,         RSUBVC,        Conditional Reverse Subtract if Overflow Cleared)
MNEM (rsubvs,         RSUBVS,        Conditional Reverse Subtract if Overflow Set)
MNEM (satadd.h,       SATADDH,       Saturated Add of Halfwords)
MNEM (satadd.w,       SATADDW,       Saturated Add of Words)
MNEM (satrnds,        SATRNDS,       Saturate with Rounding Signed)
MNEM (satrndu,        SATRNDU,       Saturate with Rounding Unsigned)
MNEM (sats,           SATS,          Saturate Signed)
MNEM (satsub.h,       SATSUBH,       Saturated Subtract of Halfwords)
MNEM (satsub.w,       SATSUBW,       Saturated Subtract of Words)
MNEM (satu,           SATU,          Saturate Unsigned)
MNEM (sbc,            SBC,           Subtract with Carry)
MNEM (sbr,            SBR,           Set Bit in Register)
MNEM (scall,          SCALL,         Supervisor Call)
MNEM (scr,            SCR,           Subtract Carry from Register)
MNEM (sleep,          SLEEP,         Set CPU Activity Mode)
MNEM (sral,           SRAL,          Set Register Conditionally Always)
MNEM (srcc,           SRCC,          Set Register Conditionally if Carry Cleared)
MNEM (srcs,           SRCS,          Set Register Conditionally if Carry Set)
MNEM (sreq,           SREQ,          Set Register Conditionally if Equal)
MNEM (srge,           SRGE,          Set Register Conditionally if Greater than or Equal)
MNEM (srgt,           SRGT,          Set Register Conditionally if Greater Than)
MNEM (srhi,           SRHI,          Set Register Conditionally if Higher)
MNEM (srhs,           SRHS,          Set Register Conditionally if Higher or Same)
MNEM (srle,           SRLE,          Set Register Conditionally if Less than or Equal)
MNEM (srlo,           SRLO,          Set Register Conditionally if Lower)
MNEM (srls,           SRLS,          Set Register Conditionally if Lower or Same)
MNEM (srlt,           SRLT,          Set Register Conditionally if Less Than)
MNEM (srmi,           SRMI,          Set Register Conditionally if Negative)
MNEM (srne,           SRNE,          Set Register Conditionally if Not Equal)
MNEM (srpl,           SRPL,          Set Register Conditionally if Positive)
MNEM (srqs,           SRQS,          Set Register Conditionally if Saturated)
MNEM (srvc,           SRVC,          Set Register Conditionally if Overflow Cleared)
MNEM (srvs,           SRVS,          Set Register Conditionally if Overflow Set)
MNEM (sscall,         SSCALL,        Secure State Call)
MNEM (ssrf,           SSRF,          Set Status Register Flag)
MNEM (st.b,           STB,           Store Byte)
MNEM (st.bal,         STBAL,         Conditionally Store Byte Always)
MNEM (st.bcc,         STBCC,         Conditionally Store Byte if Carry Cleared)
MNEM (st.bcs,         STBCS,         Conditionally Store Byte if Carry Set)
MNEM (st.beq,         STBEQ,         Conditionally Store Byte if Equal)
MNEM (st.bge,         STBGE,         Conditionally Store Byte if Greater than or Equal)
MNEM (st.bgt,         STBGT,         Conditionally Store Byte if Greater Than)
MNEM (st.bhi,         STBHI,         Conditionally Store Byte if Higher)
MNEM (st.bhs,         STBHS,         Conditionally Store Byte if Higher or Same)
MNEM (st.ble,         STBLE,         Conditionally Store Byte if Less than or Equal)
MNEM (st.blo,         STBLO,         Conditionally Store Byte if Lower)
MNEM (st.bls,         STBLS,         Conditionally Store Byte if Lower or Same)
MNEM (st.blt,         STBLT,         Conditionally Store Byte if Less Than)
MNEM (st.bmi,         STBMI,         Conditionally Store Byte if Negative)
MNEM (st.bne,         STBNE,         Conditionally Store Byte if Not Equal)
MNEM (st.bpl,         STBPL,         Conditionally Store Byte if Positive)
MNEM (st.bqs,         STBQS,         Conditionally Store Byte if Saturated)
MNEM (st.bvc,         STBVC,         Conditionally Store Byte if Overflow Cleared)
MNEM (st.bvs,         STBVS,         Conditionally Store Byte if Overflow Set)
MNEM (st.d,           STD,           Store Doubleword)
MNEM (st.h,           STH,           Store Halfword)
MNEM (st.hal,         STHAL,         Conditionally Store Halfword Always)
MNEM (st.hcc,         STHCC,         Conditionally Store Halfword if Carry Cleared)
MNEM (st.hcs,         STHCS,         Conditionally Store Halfword if Carry Set)
MNEM (st.heq,         STHEQ,         Conditionally Store Halfword if Equal)
MNEM (st.hge,         STHGE,         Conditionally Store Halfword if Greater than or Equal)
MNEM (st.hgt,         STHGT,         Conditionally Store Halfword if Greater Than)
MNEM (st.hhi,         STHHI,         Conditionally Store Halfword if Higher)
MNEM (st.hhs,         STHHS,         Conditionally Store Halfword if Higher or Same)
MNEM (st.hle,         STHLE,         Conditionally Store Halfword if Less than or Equal)
MNEM (st.hlo,         STHLO,         Conditionally Store Halfword if Lower)
MNEM (st.hls,         STHLS,         Conditionally Store Halfword if Lower or Same)
MNEM (st.hlt,         STHLT,         Conditionally Store Halfword if Less Than)
MNEM (st.hmi,         STHMI,         Conditionally Store Halfword if Negative)
MNEM (st.hne,         STHNE,         Conditionally Store Halfword if Not Equal)
MNEM (st.hpl,         STHPL,         Conditionally Store Halfword if Positive)
MNEM (st.hqs,         STHQS,         Conditionally Store Halfword if Saturated)
MNEM (st.hvc,         STHVC,         Conditionally Store Halfword if Overflow Cleared)
MNEM (st.hvs,         STHVS,         Conditionally Store Halfword if Overflow Set)
MNEM (st.w,           STW,           Store Word)
MNEM (st.wal,         STWAL,         Conditionally Store Word Always)
MNEM (st.wcc,         STWCC,         Conditionally Store Word if Carry Cleared)
MNEM (st.wcs,         STWCS,         Conditionally Store Word if Carry Set)
MNEM (st.weq,         STWEQ,         Conditionally Store Word if Equal)
MNEM (st.wge,         STWGE,         Conditionally Store Word if Greater than or Equal)
MNEM (st.wgt,         STWGT,         Conditionally Store Word if Greater Than)
MNEM (st.whi,         STWHI,         Conditionally Store Word if Higher)
MNEM (st.whs,         STWHS,         Conditionally Store Word if Higher or Same)
MNEM (st.wle,         STWLE,         Conditionally Store Word if Less than or Equal)
MNEM (st.wlo,         STWLO,         Conditionally Store Word if Lower)
MNEM (st.wls,         STWLS,         Conditionally Store Word if Lower or Same)
MNEM (st.wlt,         STWLT,         Conditionally Store Word if Less Than)
MNEM (st.wmi,         STWMI,         Conditionally Store Word if Negative)
MNEM (st.wne,         STWNE,         Conditionally Store Word if Not Equal)
MNEM (st.wpl,         STWPL,         Conditionally Store Word if Positive)
MNEM (st.wqs,         STWQS,         Conditionally Store Word if Saturated)
MNEM (st.wvc,         STWVC,         Conditionally Store Word if Overflow Cleared)
MNEM (st.wvs,         STWVS,         Conditionally Store Word if Overflow Set)
MNEM (stc.d,          STCD,          Store Coprocessor Doubleword)
MNEM (stc.w,          STCW,          Store Coprocessor Word)
MNEM (stc0.d,         STC0D,         Store Coprocessor 0 Doubleword Register)
MNEM (stc0.w,         STC0W,         Store Coprocessor 0 Word Register)
MNEM (stcond,         STCOND,        Store Word Conditionally)
MNEM (stdsp,          STDSP,         Store SP-relative with Displacement)
MNEM (sthh.w,         STHHW,         Store Halfwords into Word)
MNEM (stswp.h,        STSWPH,        Swap and Store Halfword)
MNEM (stswp.w,        STSWPW,        Swap and Store Word)
MNEM (sub,            SUB,           Subtract (without Carry))
MNEM (subal,          SUBAL,         Conditionally Subtract Always)
MNEM (subcc,          SUBCC,         Conditionally Subtract if Carry Cleared)
MNEM (subcs,          SUBCS,         Conditionally Subtract if Carry Set)
MNEM (subeq,          SUBEQ,         Conditionally Subtract if Equal)
MNEM (subfal,         SUBFAL,        Conditionally Subtract and Update Flags Always)
MNEM (subfcc,         SUBFCC,        Conditionally Subtract and Update Flags if Carry Cleared)
MNEM (subfcs,         SUBFCS,        Conditionally Subtract and Update Flags if Carry Set)
MNEM (subfeq,         SUBFEQ,        Conditionally Subtract and Update Flags if Equal)
MNEM (subfge,         SUBFGE,        Conditionally Subtract and Update Flags if Greater than or Equal)
MNEM (subfgt,         SUBFGT,        Conditionally Subtract and Update Flags if Greater Than)
MNEM (subfhi,         SUBFHI,        Conditionally Subtract and Update Flags if Higher)
MNEM (subfhs,         SUBFHS,        Conditionally Subtract and Update Flags if Higher or Same)
MNEM (subfle,         SUBFLE,        Conditionally Subtract and Update Flags if Less than or Equal)
MNEM (subflo,         SUBFLO,        Conditionally Subtract and Update Flags if Lower)
MNEM (subfls,         SUBFLS,        Conditionally Subtract and Update Flags if Lower or Same)
MNEM (subflt,         SUBFLT,        Conditionally Subtract and Update Flags if Less Than)
MNEM (subfmi,         SUBFMI,        Conditionally Subtract and Update Flags if Negative)
MNEM (subfne,         SUBFNE,        Conditionally Subtract and Update Flags if Not Equal)
MNEM (subfpl,         SUBFPL,        Conditionally Subtract and Update Flags if Positive)
MNEM (subfqs,         SUBFQS,        Conditionally Subtract and Update Flags if Saturated)
MNEM (subfvc,         SUBFVC,        Conditionally Subtract and Update Flags if Overflow Cleared)
MNEM (subfvs,         SUBFVS,        Conditionally Subtract and Update Flags if Overflow Set)
MNEM (subge,          SUBGE,         Conditionally Subtract if Greater than or Equal)
MNEM (subgt,          SUBGT,         Conditionally Subtract if Greater Than)
MNEM (subhh.w,        SUBHHW,        Subtract Halfwords into Word)
MNEM (subhi,          SUBHI,         Conditionally Subtract if Higher)
MNEM (subhs,          SUBHS,         Conditionally Subtract if Higher or Same)
MNEM (suble,          SUBLE,         Conditionally Subtract if Less than or Equal)
MNEM (sublo,          SUBLO,         Conditionally Subtract if Lower)
MNEM (subls,          SUBLS,         Conditionally Subtract if Lower or Same)
MNEM (sublt,          SUBLT,         Conditionally Subtract if Less Than)
MNEM (submi,          SUBMI,         Conditionally Subtract if Negative)
MNEM (subne,          SUBNE,         Conditionally Subtract if Not Equal)
MNEM (subpl,          SUBPL,         Conditionally Subtract if Positive)
MNEM (subqs,          SUBQS,         Conditionally Subtract if Saturated)
MNEM (subvc,          SUBVC,         Conditionally Subtract if Overflow Cleared)
MNEM (subvs,          SUBVS,         Conditionally Subtract if Overflow Set)
MNEM (swap.b,         SWAPB,         Swap Bytes)
MNEM (swap.bh,        SWAPBH,        Swap Bytes in Halfword)
MNEM (swap.h,         SWAPH,         Swap Halfwords)
MNEM (sync,           SYNC,          Synchronize memory)
MNEM (tlbr,           TLBR,          Read TLB Entry)
MNEM (tlbs,           TLBS,          Search TLB For Entry)
MNEM (tlbw,           TLBW,          Write TLB Entry)
MNEM (tnbz,           TNBZ,          Test if No Byte is Equal to Zero)
MNEM (tst,            TST,           Test Register)
MNEM (xchg,           XCHG,          Exchange Register and Memory)

// instructions

INSTR (ABS,           0x00005c40,  0x0000fff0,  R0,              Void,         Void,      Void,  Void)

INSTR (ACALL,         0x0000d000,  0x0000f00f,  U48T4,           Void,         Void,      Void,  Void)

INSTR (ACR,           0x00005c00,  0x0000fff0,  R0,              Void,         Void,      Void,  Void)

INSTR (ADC,           0xe0000040,  0xe1f0fff0,  R0,              R25,          R16,       Void,  Void)

INSTR (ADD,           0x00000000,  0x0000e1f0,  R0,              R9,           Void,      Void,  Void)
INSTR (ADD,           0xe0000000,  0xe1f0ffc0,  R0,              R25,          R16L42,    Void,  Void)

INSTR (ADDEQ,         0xe1d0e000,  0xe1f0fff0,  R0,              R25,          R16,       Void,  Void)

INSTR (ADDNE,         0xe1d0e100,  0xe1f0fff0,  R0,              R25,          R16,       Void,  Void)

INSTR (ADDCC,         0xe1d0e200,  0xe1f0fff0,  R0,              R25,          R16,       Void,  Void)

INSTR (ADDHS,         0xe1d0e200,  0xe1f0fff0,  R0,              R25,          R16,       Void,  Void)

INSTR (ADDCS,         0xe1d0e300,  0xe1f0fff0,  R0,              R25,          R16,       Void,  Void)

INSTR (ADDLO,         0xe1d0e300,  0xe1f0fff0,  R0,              R25,          R16,       Void,  Void)

INSTR (ADDGE,         0xe1d0e400,  0xe1f0fff0,  R0,              R25,          R16,       Void,  Void)

INSTR (ADDLT,         0xe1d0e500,  0xe1f0fff0,  R0,              R25,          R16,       Void,  Void)

INSTR (ADDMI,         0xe1d0e600,  0xe1f0fff0,  R0,              R25,          R16,       Void,  Void)

INSTR (ADDPL,         0xe1d0e700,  0xe1f0fff0,  R0,              R25,          R16,       Void,  Void)

INSTR (ADDLS,         0xe1d0e800,  0xe1f0fff0,  R0,              R25,          R16,       Void,  Void)

INSTR (ADDGT,         0xe1d0e900,  0xe1f0fff0,  R0,              R25,          R16,       Void,  Void)

INSTR (ADDLE,         0xe1d0ea00,  0xe1f0fff0,  R0,              R25,          R16,       Void,  Void)

INSTR (ADDHI,         0xe1d0eb00,  0xe1f0fff0,  R0,              R25,          R16,       Void,  Void)

INSTR (ADDVS,         0xe1d0ec00,  0xe1f0fff0,  R0,              R25,          R16,       Void,  Void)

INSTR (ADDVC,         0xe1d0ed00,  0xe1f0fff0,  R0,              R25,          R16,       Void,  Void)

INSTR (ADDQS,         0xe1d0ee00,  0xe1f0fff0,  R0,              R25,          R16,       Void,  Void)

INSTR (ADDAL,         0xe1d0ef00,  0xe1f0fff0,  R0,              R25,          R16,       Void,  Void)

INSTR (ADDABS,        0xe0000e40,  0xe1f0fff0,  R0,              R25,          R16,       Void,  Void)

INSTR (ADDHHW,        0xe0000e00,  0xe1f0ffc0,  R0,              R25P51,       R16P41,    Void,  Void)

INSTR (AND,           0x00000060,  0x0000e1f0,  R0,              R9,           Void,      Void,  Void)
INSTR (AND,           0xe1e00000,  0xe1f0fe00,  R0,              R25,          R16L45,    Void,  Void)
INSTR (AND,           0xe1e00200,  0xe1f0fe00,  R0,              R25,          R16R45,    Void,  Void)

INSTR (ANDEQ,         0xe1d0e020,  0xe1f0fff0,  R0,              R25,          R16,       Void,  Void)

INSTR (ANDNE,         0xe1d0e120,  0xe1f0fff0,  R0,              R25,          R16,       Void,  Void)

INSTR (ANDCC,         0xe1d0e220,  0xe1f0fff0,  R0,              R25,          R16,       Void,  Void)

INSTR (ANDHS,         0xe1d0e220,  0xe1f0fff0,  R0,              R25,          R16,       Void,  Void)

INSTR (ANDCS,         0xe1d0e320,  0xe1f0fff0,  R0,              R25,          R16,       Void,  Void)

INSTR (ANDLO,         0xe1d0e320,  0xe1f0fff0,  R0,              R25,          R16,       Void,  Void)

INSTR (ANDGE,         0xe1d0e420,  0xe1f0fff0,  R0,              R25,          R16,       Void,  Void)

INSTR (ANDLT,         0xe1d0e520,  0xe1f0fff0,  R0,              R25,          R16,       Void,  Void)

INSTR (ANDMI,         0xe1d0e620,  0xe1f0fff0,  R0,              R25,          R16,       Void,  Void)

INSTR (ANDPL,         0xe1d0e720,  0xe1f0fff0,  R0,              R25,          R16,       Void,  Void)

INSTR (ANDLS,         0xe1d0e820,  0xe1f0fff0,  R0,              R25,          R16,       Void,  Void)

INSTR (ANDGT,         0xe1d0e920,  0xe1f0fff0,  R0,              R25,          R16,       Void,  Void)

INSTR (ANDLE,         0xe1d0ea20,  0xe1f0fff0,  R0,              R25,          R16,       Void,  Void)

INSTR (ANDHI,         0xe1d0eb20,  0xe1f0fff0,  R0,              R25,          R16,       Void,  Void)

INSTR (ANDVS,         0xe1d0ec20,  0xe1f0fff0,  R0,              R25,          R16,       Void,  Void)

INSTR (ANDVC,         0xe1d0ed20,  0xe1f0fff0,  R0,              R25,          R16,       Void,  Void)

INSTR (ANDQS,         0xe1d0ee20,  0xe1f0fff0,  R0,              R25,          R16,       Void,  Void)

INSTR (ANDAL,         0xe1d0ef20,  0xe1f0fff0,  R0,              R25,          R16,       Void,  Void)

INSTR (ANDH,          0xe4100000,  0xfff00000,  R16,             U016,         Void,      Void,  Void)
INSTR (ANDH,          0xe6100000,  0xfff00000,  R16,             U016,         COH,       Void,  Void)

INSTR (ANDL,          0xe0100000,  0xfff00000,  R16,             U016,         Void,      Void,  Void)
INSTR (ANDL,          0xe2100000,  0xfff00000,  R16,             U016,         COH,       Void,  Void)

INSTR (ANDN,          0x00000080,  0x0000e1f0,  R0,              R9,           Void,      Void,  Void)

INSTR (ASR,           0xe0000840,  0xe1f0fff0,  R0,              R25,          R16,       Void,  Void)
INSTR (ASR,           0x0000a140,  0x0000e1e0,  R0,              U4194,        Void,      Void,  Void)
INSTR (ASR,           0xe0001400,  0xe1f0ffe0,  R16,             R25,          U05,       Void,  Void)

INSTR (BFEXTS,        0xe1d0b000,  0xe1f0fc00,  R25,             R16,          U55,       U05,   Void)

INSTR (BFEXTU,        0xe1d0c000,  0xe1f0fc00,  R25,             R16,          U55,       U05,   Void)

INSTR (BFINS,         0xe1d0d000,  0xe1f0fc00,  R25,             R16,          U55,       U05,   Void)

INSTR (BLD,           0xedb00000,  0xfff0ffe0,  R16,             U05,          Void,      Void,  Void)

INSTR (BREQ,          0x0000c000,  0x0000f00f,  S48T2,           Void,         Void,      Void,  Void)
INSTR (BREQ,          0xe0800000,  0xe1ef0000,  S01620254T2,     Void,         Void,      Void,  Void)

INSTR (BRNE,          0x0000c001,  0x0000f00f,  S48T2,           Void,         Void,      Void,  Void)
INSTR (BRNE,          0xe0810000,  0xe1ef0000,  S01620254T2,     Void,         Void,      Void,  Void)

INSTR (BRCC,          0x0000c002,  0x0000f00f,  S48T2,           Void,         Void,      Void,  Void)
INSTR (BRCC,          0xe0820000,  0xe1ef0000,  S01620254T2,     Void,         Void,      Void,  Void)

INSTR (BRHS,          0x0000c002,  0x0000f00f,  S48T2,           Void,         Void,      Void,  Void)
INSTR (BRHS,          0xe0820000,  0xe1ef0000,  S01620254T2,     Void,         Void,      Void,  Void)

INSTR (BRCS,          0x0000c003,  0x0000f00f,  S48T2,           Void,         Void,      Void,  Void)
INSTR (BRCS,          0xe0830000,  0xe1ef0000,  S01620254T2,     Void,         Void,      Void,  Void)

INSTR (BRLO,          0x0000c003,  0x0000f00f,  S48T2,           Void,         Void,      Void,  Void)
INSTR (BRLO,          0xe0830000,  0xe1ef0000,  S01620254T2,     Void,         Void,      Void,  Void)

INSTR (BRGE,          0x0000c004,  0x0000f00f,  S48T2,           Void,         Void,      Void,  Void)
INSTR (BRGE,          0xe0840000,  0xe1ef0000,  S01620254T2,     Void,         Void,      Void,  Void)

INSTR (BRLT,          0x0000c005,  0x0000f00f,  S48T2,           Void,         Void,      Void,  Void)
INSTR (BRLT,          0xe0850000,  0xe1ef0000,  S01620254T2,     Void,         Void,      Void,  Void)

INSTR (BRMI,          0x0000c006,  0x0000f00f,  S48T2,           Void,         Void,      Void,  Void)
INSTR (BRMI,          0xe0860000,  0xe1ef0000,  S01620254T2,     Void,         Void,      Void,  Void)

INSTR (BRPL,          0x0000c007,  0x0000f00f,  S48T2,           Void,         Void,      Void,  Void)
INSTR (BRPL,          0xe0870000,  0xe1ef0000,  S01620254T2,     Void,         Void,      Void,  Void)

INSTR (BRLS,          0xe0880000,  0xe1ef0000,  S01620254T2,     Void,         Void,      Void,  Void)

INSTR (BRGT,          0xe0890000,  0xe1ef0000,  S01620254T2,     Void,         Void,      Void,  Void)

INSTR (BRLE,          0xe08a0000,  0xe1ef0000,  S01620254T2,     Void,         Void,      Void,  Void)

INSTR (BRHI,          0xe08b0000,  0xe1ef0000,  S01620254T2,     Void,         Void,      Void,  Void)

INSTR (BRVS,          0xe08c0000,  0xe1ef0000,  S01620254T2,     Void,         Void,      Void,  Void)

INSTR (BRVC,          0xe08d0000,  0xe1ef0000,  S01620254T2,     Void,         Void,      Void,  Void)

INSTR (BRQS,          0xe08e0000,  0xe1ef0000,  S01620254T2,     Void,         Void,      Void,  Void)

INSTR (BRAL,          0xe08f0000,  0xe1ef0000,  S01620254T2,     Void,         Void,      Void,  Void)

INSTR (BREAKPOINT,    0x0000d673,  0x0000ffff,  Void,            Void,         Void,      Void,  Void)

INSTR (BREV,          0x00005c90,  0x0000fff0,  R0,              Void,         Void,      Void,  Void)

INSTR (BST,           0xefb00000,  0xfff0ffe0,  R16,             U05,          Void,      Void,  Void)

INSTR (CACHE,         0xf4100000,  0xfff00000,  P16S011,         U115,         Void,      Void,  Void)

INSTR (CASTSH,        0x00005c80,  0x0000fff0,  R0,              Void,         Void,      Void,  Void)

INSTR (CASTSB,        0x00005c60,  0x0000fff0,  R0,              Void,         Void,      Void,  Void)

INSTR (CASTUH,        0x00005c70,  0x0000fff0,  R0,              Void,         Void,      Void,  Void)

INSTR (CASTUB,        0x00005c50,  0x0000fff0,  R0,              Void,         Void,      Void,  Void)

INSTR (CBR,           0x0000a1c0,  0x0000e1e0,  R0,              U4194,        Void,      Void,  Void)

INSTR (CLZ,           0xe0001200,  0xe1f0ffff,  R16,             R25,          Void,      Void,  Void)

INSTR (COM,           0x00005cd0,  0x0000fff0,  R0,              Void,         Void,      Void,  Void)

INSTR (COP,           0xe1a00000,  0xf9f00000,  CP13,            CR8,          CR4,       CR0,   COP)

INSTR (CPB,           0xe0001800,  0xe1f0ffff,  R16,             R25,          Void,      Void,  Void)

INSTR (CPH,           0xe0001900,  0xe1f0ffff,  R16,             R25,          Void,      Void,  Void)

INSTR (CPW,           0x00000030,  0x0000e1f0,  R0,              R9,           Void,      Void,  Void)
INSTR (CPW,           0x00005800,  0x0000fc00,  R0,              S46,          Void,      Void,  Void)
INSTR (CPW,           0xe0400000,  0xe1e00000,  R0,              S01620254,    Void,      Void,  Void)

INSTR (CPC,           0xe0001300,  0xe1f0ffff,  R16,             R25,          Void,      Void,  Void)
INSTR (CPC,           0x00005c20,  0x0000fff0,  R0,              Void,         Void,      Void,  Void)

INSTR (CSRF,          0x0000d403,  0x0000fe0f,  U45,             Void,         Void,      Void,  Void)

INSTR (CSRFCZ,        0x0000d003,  0x0000fe0f,  U45,             Void,         Void,      Void,  Void)

INSTR (DIVS,          0xe0000c00,  0xe1f0fff0,  R0D,             R25,          R16,       Void,  Void)

INSTR (DIVU,          0xe0000d00,  0xe1f0fff0,  R0D,             R25,          R16,       Void,  Void)

INSTR (EOR,           0x00000050,  0x0000e1f0,  R0,              R9,           Void,      Void,  Void)
INSTR (EOR,           0xe1e02000,  0xe1f0fe00,  R0,              R25,          R16L45,    Void,  Void)
INSTR (EOR,           0xe1e02200,  0xe1f0fe00,  R0,              R25,          R16R45,    Void,  Void)

INSTR (EOREQ,         0xe1d0e040,  0xe1f0fff0,  R0,              R25,          R16,       Void,  Void)

INSTR (EORNE,         0xe1d0e140,  0xe1f0fff0,  R0,              R25,          R16,       Void,  Void)

INSTR (EORCC,         0xe1d0e240,  0xe1f0fff0,  R0,              R25,          R16,       Void,  Void)

INSTR (EORHS,         0xe1d0e240,  0xe1f0fff0,  R0,              R25,          R16,       Void,  Void)

INSTR (EORCS,         0xe1d0e340,  0xe1f0fff0,  R0,              R25,          R16,       Void,  Void)

INSTR (EORLO,         0xe1d0e340,  0xe1f0fff0,  R0,              R25,          R16,       Void,  Void)

INSTR (EORGE,         0xe1d0e440,  0xe1f0fff0,  R0,              R25,          R16,       Void,  Void)

INSTR (EORLT,         0xe1d0e540,  0xe1f0fff0,  R0,              R25,          R16,       Void,  Void)

INSTR (EORMI,         0xe1d0e640,  0xe1f0fff0,  R0,              R25,          R16,       Void,  Void)

INSTR (EORPL,         0xe1d0e740,  0xe1f0fff0,  R0,              R25,          R16,       Void,  Void)

INSTR (EORLS,         0xe1d0e840,  0xe1f0fff0,  R0,              R25,          R16,       Void,  Void)

INSTR (EORGT,         0xe1d0e940,  0xe1f0fff0,  R0,              R25,          R16,       Void,  Void)

INSTR (EORLE,         0xe1d0ea40,  0xe1f0fff0,  R0,              R25,          R16,       Void,  Void)

INSTR (EORHI,         0xe1d0eb40,  0xe1f0fff0,  R0,              R25,          R16,       Void,  Void)

INSTR (EORVS,         0xe1d0ec40,  0xe1f0fff0,  R0,              R25,          R16,       Void,  Void)

INSTR (EORVC,         0xe1d0ed40,  0xe1f0fff0,  R0,              R25,          R16,       Void,  Void)

INSTR (EORQS,         0xe1d0ee40,  0xe1f0fff0,  R0,              R25,          R16,       Void,  Void)

INSTR (EORAL,         0xe1d0ef40,  0xe1f0fff0,  R0,              R25,          R16,       Void,  Void)

INSTR (EORH,          0xee100000,  0xfff00000,  R16,             U016,         Void,      Void,  Void)

INSTR (EORL,          0xec100000,  0xfff00000,  R16,             U016,         Void,      Void,  Void)

INSTR (FRS,           0x0000d743,  0x0000ffff,  Void,            Void,         Void,      Void,  Void)

INSTR (ICALL,         0x00005d10,  0x0000fff0,  R0,              Void,         Void,      Void,  Void)

INSTR (INCJOSP,       0x0000d683,  0x0000ff8f,  S43WZ,           Void,         Void,      Void,  Void)

INSTR (LDD,           0x0000a101,  0x0000e1f1,  R1D,             P9I,          Void,      Void,  Void)
INSTR (LDD,           0x0000a110,  0x0000e1f1,  R1D,             P9D,          Void,      Void,  Void)
INSTR (LDD,           0x0000a100,  0x0000e1f1,  R1D,             R9,           Void,      Void,  Void)
INSTR (LDD,           0xe0e00000,  0xe1f10000,  R16D,            P25S016,      Void,      Void,  Void)
INSTR (LDD,           0xe0000200,  0xe1f0ffc0,  R0D,             P25L16S42,    Void,      Void,  Void)

INSTR (LDSB,          0xe1200000,  0xe1f00000,  R16,             P25S016,      Void,      Void,  Void)
INSTR (LDSB,          0xe0000600,  0xe1f0ffc0,  R0,              P25L16S42,    Void,      Void,  Void)

INSTR (LDSBEQ,        0xe1f00600,  0xe1f0fe00,  R16,             P25U09,       Void,      Void,  Void)

INSTR (LDSBNE,        0xe1f01600,  0xe1f0fe00,  R16,             P25U09,       Void,      Void,  Void)

INSTR (LDSBCC,        0xe1f02600,  0xe1f0fe00,  R16,             P25U09,       Void,      Void,  Void)

INSTR (LDSBHS,        0xe1f02600,  0xe1f0fe00,  R16,             P25U09,       Void,      Void,  Void)

INSTR (LDSBCS,        0xe1f03600,  0xe1f0fe00,  R16,             P25U09,       Void,      Void,  Void)

INSTR (LDSBLO,        0xe1f03600,  0xe1f0fe00,  R16,             P25U09,       Void,      Void,  Void)

INSTR (LDSBGE,        0xe1f04600,  0xe1f0fe00,  R16,             P25U09,       Void,      Void,  Void)

INSTR (LDSBLT,        0xe1f05600,  0xe1f0fe00,  R16,             P25U09,       Void,      Void,  Void)

INSTR (LDSBMI,        0xe1f06600,  0xe1f0fe00,  R16,             P25U09,       Void,      Void,  Void)

INSTR (LDSBPL,        0xe1f07600,  0xe1f0fe00,  R16,             P25U09,       Void,      Void,  Void)

INSTR (LDSBLS,        0xe1f08600,  0xe1f0fe00,  R16,             P25U09,       Void,      Void,  Void)

INSTR (LDSBGT,        0xe1f09600,  0xe1f0fe00,  R16,             P25U09,       Void,      Void,  Void)

INSTR (LDSBLE,        0xe1f0a600,  0xe1f0fe00,  R16,             P25U09,       Void,      Void,  Void)

INSTR (LDSBHI,        0xe1f0b600,  0xe1f0fe00,  R16,             P25U09,       Void,      Void,  Void)

INSTR (LDSBVS,        0xe1f0c600,  0xe1f0fe00,  R16,             P25U09,       Void,      Void,  Void)

INSTR (LDSBVC,        0xe1f0d600,  0xe1f0fe00,  R16,             P25U09,       Void,      Void,  Void)

INSTR (LDSBQS,        0xe1f0e600,  0xe1f0fe00,  R16,             P25U09,       Void,      Void,  Void)

INSTR (LDSBAL,        0xe1f0f600,  0xe1f0fe00,  R16,             P25U09,       Void,      Void,  Void)

INSTR (LDUB,          0x00000130,  0x0000e1f0,  R0,              P9I,          Void,      Void,  Void)
INSTR (LDUB,          0x00000170,  0x0000e1f0,  R0,              P9D,          Void,      Void,  Void)
INSTR (LDUB,          0x00000180,  0x0000e180,  R0,              P9U43,        Void,      Void,  Void)
INSTR (LDUB,          0xe1300000,  0xe1f00000,  R16,             P25S016,      Void,      Void,  Void)
INSTR (LDUB,          0xe0000700,  0xe1f0ffc0,  R0,              P25L16S42,    Void,      Void,  Void)

INSTR (LDUBEQ,        0xe1f00800,  0xe1f0fe00,  R16,             P25U09,       Void,      Void,  Void)

INSTR (LDUBNE,        0xe1f01800,  0xe1f0fe00,  R16,             P25U09,       Void,      Void,  Void)

INSTR (LDUBCC,        0xe1f02800,  0xe1f0fe00,  R16,             P25U09,       Void,      Void,  Void)

INSTR (LDUBHS,        0xe1f02800,  0xe1f0fe00,  R16,             P25U09,       Void,      Void,  Void)

INSTR (LDUBCS,        0xe1f03800,  0xe1f0fe00,  R16,             P25U09,       Void,      Void,  Void)

INSTR (LDUBLO,        0xe1f03800,  0xe1f0fe00,  R16,             P25U09,       Void,      Void,  Void)

INSTR (LDUBGE,        0xe1f04800,  0xe1f0fe00,  R16,             P25U09,       Void,      Void,  Void)

INSTR (LDUBLT,        0xe1f05800,  0xe1f0fe00,  R16,             P25U09,       Void,      Void,  Void)

INSTR (LDUBMI,        0xe1f06800,  0xe1f0fe00,  R16,             P25U09,       Void,      Void,  Void)

INSTR (LDUBPL,        0xe1f07800,  0xe1f0fe00,  R16,             P25U09,       Void,      Void,  Void)

INSTR (LDUBLS,        0xe1f08800,  0xe1f0fe00,  R16,             P25U09,       Void,      Void,  Void)

INSTR (LDUBGT,        0xe1f09800,  0xe1f0fe00,  R16,             P25U09,       Void,      Void,  Void)

INSTR (LDUBLE,        0xe1f0a800,  0xe1f0fe00,  R16,             P25U09,       Void,      Void,  Void)

INSTR (LDUBHI,        0xe1f0b800,  0xe1f0fe00,  R16,             P25U09,       Void,      Void,  Void)

INSTR (LDUBVS,        0xe1f0c800,  0xe1f0fe00,  R16,             P25U09,       Void,      Void,  Void)

INSTR (LDUBVC,        0xe1f0d800,  0xe1f0fe00,  R16,             P25U09,       Void,      Void,  Void)

INSTR (LDUBQS,        0xe1f0e800,  0xe1f0fe00,  R16,             P25U09,       Void,      Void,  Void)

INSTR (LDUBAL,        0xe1f0f800,  0xe1f0fe00,  R16,             P25U09,       Void,      Void,  Void)

INSTR (LDSH,          0x00000110,  0x0000e1f0,  R0,              P9I,          Void,      Void,  Void)
INSTR (LDSH,          0x00000150,  0x0000e1f0,  R0,              P9D,          Void,      Void,  Void)
INSTR (LDSH,          0x00008000,  0x0000e180,  R0,              P9U43T2,      Void,      Void,  Void)
INSTR (LDSH,          0xe1000000,  0xe1f00000,  R16,             P25S016,      Void,      Void,  Void)
INSTR (LDSH,          0xe0000600,  0xe1f0ffc0,  R0,              P25L16S42,    Void,      Void,  Void)

INSTR (LDSHEQ,        0xe1f00200,  0xe1f0fe00,  R16,             P25U09T2,     Void,      Void,  Void)

INSTR (LDSHNE,        0xe1f01200,  0xe1f0fe00,  R16,             P25U09T2,     Void,      Void,  Void)

INSTR (LDSHCC,        0xe1f02200,  0xe1f0fe00,  R16,             P25U09T2,     Void,      Void,  Void)

INSTR (LDSHHS,        0xe1f02200,  0xe1f0fe00,  R16,             P25U09T2,     Void,      Void,  Void)

INSTR (LDSHCS,        0xe1f03200,  0xe1f0fe00,  R16,             P25U09T2,     Void,      Void,  Void)

INSTR (LDSHLO,        0xe1f03200,  0xe1f0fe00,  R16,             P25U09T2,     Void,      Void,  Void)

INSTR (LDSHGE,        0xe1f04200,  0xe1f0fe00,  R16,             P25U09T2,     Void,      Void,  Void)

INSTR (LDSHLT,        0xe1f05200,  0xe1f0fe00,  R16,             P25U09T2,     Void,      Void,  Void)

INSTR (LDSHMI,        0xe1f06200,  0xe1f0fe00,  R16,             P25U09T2,     Void,      Void,  Void)

INSTR (LDSHPL,        0xe1f07200,  0xe1f0fe00,  R16,             P25U09T2,     Void,      Void,  Void)

INSTR (LDSHLS,        0xe1f08200,  0xe1f0fe00,  R16,             P25U09T2,     Void,      Void,  Void)

INSTR (LDSHGT,        0xe1f09200,  0xe1f0fe00,  R16,             P25U09T2,     Void,      Void,  Void)

INSTR (LDSHLE,        0xe1f0a200,  0xe1f0fe00,  R16,             P25U09T2,     Void,      Void,  Void)

INSTR (LDSHHI,        0xe1f0b200,  0xe1f0fe00,  R16,             P25U09T2,     Void,      Void,  Void)

INSTR (LDSHVS,        0xe1f0c200,  0xe1f0fe00,  R16,             P25U09T2,     Void,      Void,  Void)

INSTR (LDSHVC,        0xe1f0d200,  0xe1f0fe00,  R16,             P25U09T2,     Void,      Void,  Void)

INSTR (LDSHQS,        0xe1f0e200,  0xe1f0fe00,  R16,             P25U09T2,     Void,      Void,  Void)

INSTR (LDSHAL,        0xe1f0f200,  0xe1f0fe00,  R16,             P25U09T2,     Void,      Void,  Void)

INSTR (LDUH,          0x00000120,  0x0000e1f0,  R0,              P9I,          Void,      Void,  Void)
INSTR (LDUH,          0x00000160,  0x0000e1f0,  R0,              P9D,          Void,      Void,  Void)
INSTR (LDUH,          0x00008080,  0x0000e180,  R0,              P9U43T2,      Void,      Void,  Void)
INSTR (LDUH,          0xe1100000,  0xe1f00000,  R16,             P25S016,      Void,      Void,  Void)
INSTR (LDUH,          0xe0000500,  0xe1f0ffc0,  R0,              P25L16S42,    Void,      Void,  Void)

INSTR (LDUHEQ,        0xe1f00400,  0xe1f0fe00,  R16,             P25U09T2,     Void,      Void,  Void)

INSTR (LDUHNE,        0xe1f01400,  0xe1f0fe00,  R16,             P25U09T2,     Void,      Void,  Void)

INSTR (LDUHCC,        0xe1f02400,  0xe1f0fe00,  R16,             P25U09T2,     Void,      Void,  Void)

INSTR (LDUHHS,        0xe1f02400,  0xe1f0fe00,  R16,             P25U09T2,     Void,      Void,  Void)

INSTR (LDUHCS,        0xe1f03400,  0xe1f0fe00,  R16,             P25U09T2,     Void,      Void,  Void)

INSTR (LDUHLO,        0xe1f03400,  0xe1f0fe00,  R16,             P25U09T2,     Void,      Void,  Void)

INSTR (LDUHGE,        0xe1f04400,  0xe1f0fe00,  R16,             P25U09T2,     Void,      Void,  Void)

INSTR (LDUHLT,        0xe1f05400,  0xe1f0fe00,  R16,             P25U09T2,     Void,      Void,  Void)

INSTR (LDUHMI,        0xe1f06400,  0xe1f0fe00,  R16,             P25U09T2,     Void,      Void,  Void)

INSTR (LDUHPL,        0xe1f07400,  0xe1f0fe00,  R16,             P25U09T2,     Void,      Void,  Void)

INSTR (LDUHLS,        0xe1f08400,  0xe1f0fe00,  R16,             P25U09T2,     Void,      Void,  Void)

INSTR (LDUHGT,        0xe1f09400,  0xe1f0fe00,  R16,             P25U09T2,     Void,      Void,  Void)

INSTR (LDUHLE,        0xe1f0a400,  0xe1f0fe00,  R16,             P25U09T2,     Void,      Void,  Void)

INSTR (LDUHHI,        0xe1f0b400,  0xe1f0fe00,  R16,             P25U09T2,     Void,      Void,  Void)

INSTR (LDUHVS,        0xe1f0c400,  0xe1f0fe00,  R16,             P25U09T2,     Void,      Void,  Void)

INSTR (LDUHVC,        0xe1f0d400,  0xe1f0fe00,  R16,             P25U09T2,     Void,      Void,  Void)

INSTR (LDUHQS,        0xe1f0e400,  0xe1f0fe00,  R16,             P25U09T2,     Void,      Void,  Void)

INSTR (LDUHAL,        0xe1f0f400,  0xe1f0fe00,  R16,             P25U09T2,     Void,      Void,  Void)

INSTR (LDW,           0x00000100,  0x0000e1f0,  R0,              P9I,          Void,      Void,  Void)
INSTR (LDW,           0x00000140,  0x0000e1f0,  R0,              P9D,          Void,      Void,  Void)
INSTR (LDW,           0x00006000,  0x0000e000,  R0,              P9U45T4,      Void,      Void,  Void)
INSTR (LDW,           0xe0f00000,  0xe1f00000,  R16,             P25S016,      Void,      Void,  Void)
INSTR (LDW,           0xe0000300,  0xe1f0ffc0,  R0,              P25L16S42,    Void,      Void,  Void)

INSTR (LDWEQ,         0xe1f00000,  0xe1f0fe00,  R16,             P25U09T4,     Void,      Void,  Void)

INSTR (LDWNE,         0xe1f01000,  0xe1f0fe00,  R16,             P25U09T4,     Void,      Void,  Void)

INSTR (LDWCC,         0xe1f02000,  0xe1f0fe00,  R16,             P25U09T4,     Void,      Void,  Void)

INSTR (LDWHS,         0xe1f02000,  0xe1f0fe00,  R16,             P25U09T4,     Void,      Void,  Void)

INSTR (LDWCS,         0xe1f03000,  0xe1f0fe00,  R16,             P25U09T4,     Void,      Void,  Void)

INSTR (LDWLO,         0xe1f03000,  0xe1f0fe00,  R16,             P25U09T4,     Void,      Void,  Void)

INSTR (LDWGE,         0xe1f04000,  0xe1f0fe00,  R16,             P25U09T4,     Void,      Void,  Void)

INSTR (LDWLT,         0xe1f05000,  0xe1f0fe00,  R16,             P25U09T4,     Void,      Void,  Void)

INSTR (LDWMI,         0xe1f06000,  0xe1f0fe00,  R16,             P25U09T4,     Void,      Void,  Void)

INSTR (LDWPL,         0xe1f07000,  0xe1f0fe00,  R16,             P25U09T4,     Void,      Void,  Void)

INSTR (LDWLS,         0xe1f08000,  0xe1f0fe00,  R16,             P25U09T4,     Void,      Void,  Void)

INSTR (LDWGT,         0xe1f09000,  0xe1f0fe00,  R16,             P25U09T4,     Void,      Void,  Void)

INSTR (LDWLE,         0xe1f0a000,  0xe1f0fe00,  R16,             P25U09T4,     Void,      Void,  Void)

INSTR (LDWHI,         0xe1f0b000,  0xe1f0fe00,  R16,             P25U09T4,     Void,      Void,  Void)

INSTR (LDWVS,         0xe1f0c000,  0xe1f0fe00,  R16,             P25U09T4,     Void,      Void,  Void)

INSTR (LDWVC,         0xe1f0d000,  0xe1f0fe00,  R16,             P25U09T4,     Void,      Void,  Void)

INSTR (LDWQS,         0xe1f0e000,  0xe1f0fe00,  R16,             P25U09T4,     Void,      Void,  Void)

INSTR (LDWAL,         0xe1f0f000,  0xe1f0fe00,  R16,             P25U09T4,     Void,      Void,  Void)

INSTR (LDCD,          0xe9a01000,  0xfff01100,  CP13,            CR9D,         P16U08T4,  Void,  Void)
INSTR (LDCD,          0xefa00050,  0xfff011ff,  CP13,            CR9D,         P16D,      Void,  Void)
INSTR (LDCD,          0xefa01040,  0xfff011c0,  CP13,            CR9D,         P16L0S42,  Void,  Void)

INSTR (LDCW,          0xe9a00000,  0xfff01000,  CP13,            CR8,          P16U08T4,  Void,  Void)
INSTR (LDCW,          0xefa00040,  0xfff010ff,  CP13,            CR8,          P16D,      Void,  Void)
INSTR (LDCW,          0xefa01000,  0xfff010c0,  CP13,            CR8,          P16L0S42,  Void,  Void)

INSTR (LDC0D,         0xf3a00000,  0xfff00100,  CR9D,            P16U08124T4,  Void,      Void,  Void)

INSTR (LDC0W,         0xf1a00000,  0xfff00000,  CR8,             P16U08124T4,  Void,      Void,  Void)

INSTR (LDDPC,         0x00004800,  0x0000f800,  R0,              PPCU47T4,     Void,      Void,  Void)

INSTR (LDDSP,         0x00004000,  0x0000f800,  R0,              PSPU47T4,     Void,      Void,  Void)

INSTR (LDINSB,        0xe1d04000,  0xe1f0c000,  R16P122,         P25S012,      Void,      Void,  Void)

INSTR (LDINSH,        0xe1d00000,  0xe1f0e000,  R16P121,         P25S012T2,    Void,      Void,  Void)

INSTR (LDSWPSH,       0xe1d02000,  0xe1f0f000,  R16,             P25S012T2,    Void,      Void,  Void)

INSTR (LDSWPUH,       0xe1d03000,  0xe1f0f000,  R16,             P25S012T2,    Void,      Void,  Void)

INSTR (LDSWPW,        0xe1d08000,  0xe1f0f000,  R16,             P25S012T4,    Void,      Void,  Void)

INSTR (LSL,           0xe0000940,  0xe1f0fff0,  R0,              R25,          R16,       Void,  Void)
INSTR (LSL,           0x0000a160,  0x0000e1e0,  R0,              U4194,        Void,      Void,  Void)
INSTR (LSL,           0xe0001500,  0xe1f0ffe0,  R16,             R25,          U05,       Void,  Void)

INSTR (LSR,           0xe0000a40,  0xe1f0fff0,  R0,              R25,          R16,       Void,  Void)
INSTR (LSR,           0x0000a180,  0x0000e1e0,  R0,              U4194,        Void,      Void,  Void)
INSTR (LSR,           0xe0001600,  0xe1f0ffe0,  R16,             R25,          U05,       Void,  Void)

INSTR (MAC,           0xe0000340,  0xe1f0fff0,  R0,              R25,          R16,       Void,  Void)

INSTR (MACHHD,        0xe0000580,  0xe1f0ffc0,  R0D,             R25P51,       R16P41,    Void,  Void)

INSTR (MACHHW,        0xe0000480,  0xe1f0ffc0,  R0,              R25P51,       R16P41,    Void,  Void)

INSTR (MACSD,         0xe0000540,  0xe1f0fff0,  R0D,             R25,          R16,       Void,  Void)

INSTR (MACSATHHW,     0xe0000680,  0xe1f0ffc0,  R0,              R25P51,       R16P41,    Void,  Void)

INSTR (MACUD,         0xe0000740,  0xe1f0fff0,  R0D,             R25,          R16,       Void,  Void)

INSTR (MACWHD,        0xe0000c80,  0xe1f0ffe0,  R0D,             R25,          R16P41,    Void,  Void)

INSTR (MAX,           0xe0000c40,  0xe1f0fff0,  R0,              R25,          R16,       Void,  Void)

INSTR (MCALL,         0xf0100000,  0xfff00000,  P16S016T4,       Void,         Void,      Void,  Void)

INSTR (MEMC,          0xf6100000,  0xfff00000,  S015T4,          U155,         Void,      Void,  Void)

INSTR (MEMS,          0xf8100000,  0xfff00000,  S015T4,          U155,         Void,      Void,  Void)

INSTR (MEMT,          0xfa100000,  0xfff00000,  S015T4,          U155,         Void,      Void,  Void)

INSTR (MFDR,          0xe5b00000,  0xfff0ff00,  R16,             U08T4,        Void,      Void,  Void)

INSTR (MFSR,          0xe1b00000,  0xfff0ff00,  R16,             U08T4,        Void,      Void,  Void)

INSTR (MIN,           0xe0000d40,  0xe1f0fff0,  R0,              R25,          R16,       Void,  Void)

INSTR (MOV,           0x00003000,  0x0000f000,  R0,              S48,          Void,      Void,  Void)
INSTR (MOV,           0xe0600000,  0xe1e00000,  R16,             S01620254,    Void,      Void,  Void)
INSTR (MOV,           0x00000090,  0x0000e1f0,  R0,              R9,           Void,      Void,  Void)

INSTR (MOVEQ,         0xe0001700,  0xe1f0ffff,  R16,             R25,          Void,      Void,  Void)
INSTR (MOVEQ,         0xf9b00000,  0xfff0ff00,  R16,             S08,          Void,      Void,  Void)

INSTR (MOVNE,         0xe0001710,  0xe1f0ffff,  R16,             R25,          Void,      Void,  Void)
INSTR (MOVNE,         0xf9b00100,  0xfff0ff00,  R16,             S08,          Void,      Void,  Void)

INSTR (MOVCC,         0xe0001720,  0xe1f0ffff,  R16,             R25,          Void,      Void,  Void)
INSTR (MOVCC,         0xf9b00200,  0xfff0ff00,  R16,             S08,          Void,      Void,  Void)

INSTR (MOVHS,         0xe0001720,  0xe1f0ffff,  R16,             R25,          Void,      Void,  Void)
INSTR (MOVHS,         0xf9b00200,  0xfff0ff00,  R16,             S08,          Void,      Void,  Void)

INSTR (MOVCS,         0xe0001730,  0xe1f0ffff,  R16,             R25,          Void,      Void,  Void)
INSTR (MOVCS,         0xf9b00300,  0xfff0ff00,  R16,             S08,          Void,      Void,  Void)

INSTR (MOVLO,         0xe0001730,  0xe1f0ffff,  R16,             R25,          Void,      Void,  Void)
INSTR (MOVLO,         0xf9b00300,  0xfff0ff00,  R16,             S08,          Void,      Void,  Void)

INSTR (MOVGE,         0xe0001740,  0xe1f0ffff,  R16,             R25,          Void,      Void,  Void)
INSTR (MOVGE,         0xf9b00400,  0xfff0ff00,  R16,             S08,          Void,      Void,  Void)

INSTR (MOVLT,         0xe0001750,  0xe1f0ffff,  R16,             R25,          Void,      Void,  Void)
INSTR (MOVLT,         0xf9b00500,  0xfff0ff00,  R16,             S08,          Void,      Void,  Void)

INSTR (MOVMI,         0xe0001760,  0xe1f0ffff,  R16,             R25,          Void,      Void,  Void)
INSTR (MOVMI,         0xf9b00600,  0xfff0ff00,  R16,             S08,          Void,      Void,  Void)

INSTR (MOVPL,         0xe0001770,  0xe1f0ffff,  R16,             R25,          Void,      Void,  Void)
INSTR (MOVPL,         0xf9b00700,  0xfff0ff00,  R16,             S08,          Void,      Void,  Void)

INSTR (MOVLS,         0xe0001780,  0xe1f0ffff,  R16,             R25,          Void,      Void,  Void)
INSTR (MOVLS,         0xf9b00800,  0xfff0ff00,  R16,             S08,          Void,      Void,  Void)

INSTR (MOVGT,         0xe0001790,  0xe1f0ffff,  R16,             R25,          Void,      Void,  Void)
INSTR (MOVGT,         0xf9b00900,  0xfff0ff00,  R16,             S08,          Void,      Void,  Void)

INSTR (MOVLE,         0xe00017a0,  0xe1f0ffff,  R16,             R25,          Void,      Void,  Void)
INSTR (MOVLE,         0xf9b00a00,  0xfff0ff00,  R16,             S08,          Void,      Void,  Void)

INSTR (MOVHI,         0xe00017b0,  0xe1f0ffff,  R16,             R25,          Void,      Void,  Void)
INSTR (MOVHI,         0xf9b00b00,  0xfff0ff00,  R16,             S08,          Void,      Void,  Void)

INSTR (MOVVS,         0xe00017c0,  0xe1f0ffff,  R16,             R25,          Void,      Void,  Void)
INSTR (MOVVS,         0xf9b00c00,  0xfff0ff00,  R16,             S08,          Void,      Void,  Void)

INSTR (MOVVC,         0xe00017d0,  0xe1f0ffff,  R16,             R25,          Void,      Void,  Void)
INSTR (MOVVC,         0xf9b00d00,  0xfff0ff00,  R16,             S08,          Void,      Void,  Void)

INSTR (MOVQS,         0xe00017e0,  0xe1f0ffff,  R16,             R25,          Void,      Void,  Void)
INSTR (MOVQS,         0xf9b00e00,  0xfff0ff00,  R16,             S08,          Void,      Void,  Void)

INSTR (MOVAL,         0xe00017f0,  0xe1f0ffff,  R16,             R25,          Void,      Void,  Void)
INSTR (MOVAL,         0xf9b00f00,  0xfff0ff00,  R16,             S08,          Void,      Void,  Void)

INSTR (MOVH,          0xfc100000,  0xfff00000,  R16,             U016,         Void,      Void,  Void)

INSTR (MOVL,          0xe0600000,  0xfff00000,  R16,             U016,         Void,      Void,  Void)

INSTR (MTDR,          0xe7b00000,  0xfff0ff00,  U08T4,           R16,          Void,      Void,  Void)

INSTR (MTSR,          0xe3b00000,  0xfff0ff00,  U08T4,           R16,          Void,      Void,  Void)

INSTR (MUL,           0x0000a130,  0x0000e1f0,  R0,              R9,           Void,      Void,  Void)
INSTR (MUL,           0xe0000240,  0xe1f0fff0,  R0,              R25,          R16,       Void,  Void)
INSTR (MUL,           0xe0001000,  0xe1f0ff00,  R16,             R25,          S08,       Void,  Void)

INSTR (MULHHW,        0xe0000780,  0xe1f0ffc0,  R0,              R25P51,       R16P41,    Void,  Void)

INSTR (MULNHHW,       0xe0000180,  0xe1f0ffc0,  R0,              R25P51,       R16P41,    Void,  Void)

INSTR (MULNWHD,       0xe0000280,  0xe1f0ffe0,  R0D,             R25,          R16P41,    Void,  Void)

INSTR (MULSD,         0xe0000440,  0xe1f0fff0,  R0D,             R25,          R16,       Void,  Void)

INSTR (MULSATHHH,     0xe0000880,  0xe1f0ffc0,  R0,              R25P51,       R16P41,    Void,  Void)

INSTR (MULSATHHW,     0xe0000980,  0xe1f0ffc0,  R0,              R25P51,       R16P41,    Void,  Void)

INSTR (MULSATRNDHHH,  0xe0000a80,  0xe1f0ffc0,  R0,              R25P51,       R16P41,    Void,  Void)

INSTR (MULSATRNDHHW,  0xe0000b80,  0xe1f0ffc0,  R0,              R25P51,       R16P41,    Void,  Void)

INSTR (MULSATWHW,     0xe0000e80,  0xe1f0ffe0,  R0,              R25,          R16P41,    Void,  Void)

INSTR (MULUD,         0xe0000640,  0xe1f0fff0,  R0D,             R25,          R16,       Void,  Void)

INSTR (MULWHD,        0xe0000d80,  0xe1f0ffe0,  R0D,             R25,          R16P41,    Void,  Void)

INSTR (MUSFR,         0x00005d30,  0x0000fff0,  R0,              Void,         Void,      Void,  Void)

INSTR (MUSTR,         0x00005d20,  0x0000fff0,  R0,              Void,         Void,      Void,  Void)

INSTR (MVCRD,         0xefa00010,  0xfff111ff,  CP13,            R16D,         CR9D,      Void,  Void)

INSTR (MVCRW,         0xefa00000,  0xfff010ff,  CP13,            R16,          CR8,       Void,  Void)

INSTR (MVRCD,         0xefa00030,  0xfff111ff,  CP13,            CR9D,         R16D,      Void,  Void)

INSTR (MVRCW,         0xefa00020,  0xfff010ff,  CP13,            CR8,          R16,       Void,  Void)

INSTR (NEG,           0x00005c30,  0x0000fff0,  R0,              Void,         Void,      Void,  Void)

INSTR (NOP,           0x0000d703,  0x0000ffff,  Void,            Void,         Void,      Void,  Void)

INSTR (OR,            0x00000040,  0x0000e1f0,  R0,              R9,           Void,      Void,  Void)
INSTR (OR,            0xe1e01000,  0xe1f0fe00,  R0,              R25,          R16L45,    Void,  Void)
INSTR (OR,            0xe1e01200,  0xe1f0fe00,  R0,              R25,          R16R45,    Void,  Void)

INSTR (OREQ,          0xe1d0e030,  0xe1f0fff0,  R0,              R25,          R16,       Void,  Void)

INSTR (ORNE,          0xe1d0e130,  0xe1f0fff0,  R0,              R25,          R16,       Void,  Void)

INSTR (ORCC,          0xe1d0e230,  0xe1f0fff0,  R0,              R25,          R16,       Void,  Void)

INSTR (ORHS,          0xe1d0e230,  0xe1f0fff0,  R0,              R25,          R16,       Void,  Void)

INSTR (ORCS,          0xe1d0e330,  0xe1f0fff0,  R0,              R25,          R16,       Void,  Void)

INSTR (ORLO,          0xe1d0e330,  0xe1f0fff0,  R0,              R25,          R16,       Void,  Void)

INSTR (ORGE,          0xe1d0e430,  0xe1f0fff0,  R0,              R25,          R16,       Void,  Void)

INSTR (ORLT,          0xe1d0e530,  0xe1f0fff0,  R0,              R25,          R16,       Void,  Void)

INSTR (ORMI,          0xe1d0e630,  0xe1f0fff0,  R0,              R25,          R16,       Void,  Void)

INSTR (ORPL,          0xe1d0e730,  0xe1f0fff0,  R0,              R25,          R16,       Void,  Void)

INSTR (ORLS,          0xe1d0e830,  0xe1f0fff0,  R0,              R25,          R16,       Void,  Void)

INSTR (ORGT,          0xe1d0e930,  0xe1f0fff0,  R0,              R25,          R16,       Void,  Void)

INSTR (ORLE,          0xe1d0ea30,  0xe1f0fff0,  R0,              R25,          R16,       Void,  Void)

INSTR (ORHI,          0xe1d0eb30,  0xe1f0fff0,  R0,              R25,          R16,       Void,  Void)

INSTR (ORVS,          0xe1d0ec30,  0xe1f0fff0,  R0,              R25,          R16,       Void,  Void)

INSTR (ORVC,          0xe1d0ed30,  0xe1f0fff0,  R0,              R25,          R16,       Void,  Void)

INSTR (ORQS,          0xe1d0ee30,  0xe1f0fff0,  R0,              R25,          R16,       Void,  Void)

INSTR (ORAL,          0xe1d0ef30,  0xe1f0fff0,  R0,              R25,          R16,       Void,  Void)

INSTR (ORH,           0xea100000,  0xfff00000,  R16,             U016,         Void,      Void,  Void)

INSTR (ORL,           0xe8100000,  0xfff00000,  R16,             U016,         Void,      Void,  Void)

INSTR (PABSSB,        0xe00023e0,  0xfff0fff0,  R0,              R16,          Void,      Void,  Void)

INSTR (PABSSH,        0xe00023f0,  0xfff0fff0,  R0,              R16,          Void,      Void,  Void)

INSTR (PACKSHUB,      0xe00024c0,  0xe1f0fff0,  R0,              R25,          R16,       Void,  Void)

INSTR (PACKSHSB,      0xe00024d0,  0xe1f0fff0,  R0,              R25,          R16,       Void,  Void)

INSTR (PACKWSHS,      0xe0002470,  0xe1f0fff0,  R0,              R25,          R16,       Void,  Void)

INSTR (PADDB,         0xe0002500,  0xe1f0fff0,  R0,              R25,          R16,       Void,  Void)

INSTR (PADDH,         0xe0002000,  0xe1f0fff0,  R0,              R25,          R16,       Void,  Void)

INSTR (PADDHUB,       0xe0002360,  0xe1f0fff0,  R0,              R25,          R16,       Void,  Void)

INSTR (PADDHSH,       0xe00020c0,  0xe1f0fff0,  R0,              R25,          R16,       Void,  Void)

INSTR (PADDSUB,       0xe0002340,  0xe1f0fff0,  R0,              R25,          R16,       Void,  Void)

INSTR (PADDSSB,       0xe0002320,  0xe1f0fff0,  R0,              R25,          R16,       Void,  Void)

INSTR (PADDSUH,       0xe0002080,  0xe1f0fff0,  R0,              R25,          R16,       Void,  Void)

INSTR (PADDSSH,       0xe0002040,  0xe1f0fff0,  R0,              R25,          R16,       Void,  Void)

INSTR (PADDSUBH,      0xe0002100,  0xe1f0ffc0,  R0,              R25P51,       R16P41,    Void,  Void)

INSTR (PADDSUBHSH,    0xe0002280,  0xe1f0ffc0,  R0,              R25P51,       R16P41,    Void,  Void)

INSTR (PADDSUBSUH,    0xe0002200,  0xe1f0ffc0,  R0,              R25P51,       R16P41,    Void,  Void)

INSTR (PADDSUBSSH,    0xe0002180,  0xe1f0ffc0,  R0,              R25P51,       R16P41,    Void,  Void)

INSTR (PADDXH,        0xe0002020,  0xe1f0fff0,  R0,              R25,          R16,       Void,  Void)

INSTR (PADDXHSH,      0xe00020e0,  0xe1f0fff0,  R0,              R25,          R16,       Void,  Void)

INSTR (PADDXSUH,      0xe00020a0,  0xe1f0fff0,  R0,              R25,          R16,       Void,  Void)

INSTR (PADDXSSH,      0xe0002060,  0xe1f0fff0,  R0,              R25,          R16,       Void,  Void)

INSTR (PASRB,         0xe0002410,  0xe1f8fff0,  R0,              R25,          U163,      Void,  Void)

INSTR (PASRH,         0xe0002440,  0xe1f0fff0,  R0,              R25,          U164,      Void,  Void)

INSTR (PAVGUB,        0xe00023c0,  0xe1f0fff0,  R0,              R25,          R16,       Void,  Void)

INSTR (PAVGSH,        0xe00023d0,  0xe1f0fff0,  R0,              R25,          R16,       Void,  Void)

INSTR (PLSLB,         0xe0002420,  0xe1f8fff0,  R0,              R25,          U163,      Void,  Void)

INSTR (PLSLH,         0xe0002450,  0xe1f0fff0,  R0,              R25,          U164,      Void,  Void)

INSTR (PLSRB,         0xe0002430,  0xe1f8fff0,  R0,              R25,          U163,      Void,  Void)

INSTR (PLSRH,         0xe0002460,  0xe1f0fff0,  R0,              R25,          U164,      Void,  Void)

INSTR (PMAXUB,        0xe0002380,  0xe1f0fff0,  R0,              R25,          R16,       Void,  Void)

INSTR (PMAXSH,        0xe0002390,  0xe1f0fff0,  R0,              R25,          R16,       Void,  Void)

INSTR (PMINUB,        0xe00023a0,  0xe1f0fff0,  R0,              R25,          R16,       Void,  Void)

INSTR (PMINSH,        0xe00023b0,  0xe1f0fff0,  R0,              R25,          R16,       Void,  Void)

INSTR (POPJC,         0x0000d713,  0x0000ffff,  Void,            Void,         Void,      Void,  Void)

INSTR (PREF,          0xf2100000,  0xfff00000,  P16S016,         Void,         Void,      Void,  Void)

INSTR (PSAD,          0xe0002400,  0xe1f0fff0,  R0,              R25,          R16,       Void,  Void)

INSTR (PSUBB,         0xe0002310,  0xe1f0fff0,  R0,              R25,          R16,       Void,  Void)

INSTR (PSUBH,         0xe0002010,  0xe1f0fff0,  R0,              R25,          R16,       Void,  Void)

INSTR (PSUBADDH,      0xe0002140,  0xe1f0ffc0,  R0,              R25P51,       R16P41,    Void,  Void)

INSTR (PSUBADDHSH,    0xe00022c0,  0xe1f0ffc0,  R0,              R25P51,       R16P41,    Void,  Void)

INSTR (PSUBADDSUH,    0xe0002240,  0xe1f0ffc0,  R0,              R25P51,       R16P41,    Void,  Void)

INSTR (PSUBADDSSH,    0xe00021c0,  0xe1f0ffc0,  R0,              R25P51,       R16P41,    Void,  Void)

INSTR (PSUBHUB,       0xe0002370,  0xe1f0fff0,  R0,              R25,          R16,       Void,  Void)

INSTR (PSUBHSH,       0xe00020d0,  0xe1f0fff0,  R0,              R25,          R16,       Void,  Void)

INSTR (PSUBSUB,       0xe0002350,  0xe1f0fff0,  R0,              R25,          R16,       Void,  Void)

INSTR (PSUBSSB,       0xe0002330,  0xe1f0fff0,  R0,              R25,          R16,       Void,  Void)

INSTR (PSUBSUH,       0xe0002090,  0xe1f0fff0,  R0,              R25,          R16,       Void,  Void)

INSTR (PSUBSSH,       0xe0002050,  0xe1f0fff0,  R0,              R25,          R16,       Void,  Void)

INSTR (PSUBXH,        0xe0002030,  0xe1f0fff0,  R0,              R25,          R16,       Void,  Void)

INSTR (PSUBXHSH,      0xe00020f0,  0xe1f0fff0,  R0,              R25,          R16,       Void,  Void)

INSTR (PSUBXSUH,      0xe00020b0,  0xe1f0fff0,  R0,              R25,          R16,       Void,  Void)

INSTR (PSUBXSSH,      0xe0002070,  0xe1f0fff0,  R0,              R25,          R16,       Void,  Void)

INSTR (PUNPCKSBH,     0xe00024a0,  0xe1f0ffe0,  R0,              R25P41,       Void,      Void,  Void)

INSTR (PUNPCKUBH,     0xe0002480,  0xe1f0ffe0,  R0,              R25P41,       Void,      Void,  Void)

INSTR (PUSHJC,        0x0000d723,  0x0000ffff,  Void,            Void,         Void,      Void,  Void)

INSTR (RCALL,         0x0000c00c,  0x0000f00c,  PPCS4802T2,      Void,         Void,      Void,  Void)
INSTR (RCALL,         0xe0a00000,  0xe1ef0000,  PPCS01620254T2,  Void,         Void,      Void,  Void)

INSTR (RETEQ,         0x00005e00,  0x0000fff0,  R0,              Void,         Void,      Void,  Void)

INSTR (RETNE,         0x00005e10,  0x0000fff0,  R0,              Void,         Void,      Void,  Void)

INSTR (RETCC,         0x00005e20,  0x0000fff0,  R0,              Void,         Void,      Void,  Void)

INSTR (RETHS,         0x00005e20,  0x0000fff0,  R0,              Void,         Void,      Void,  Void)

INSTR (RETCS,         0x00005e30,  0x0000fff0,  R0,              Void,         Void,      Void,  Void)

INSTR (RETLO,         0x00005e30,  0x0000fff0,  R0,              Void,         Void,      Void,  Void)

INSTR (RETGE,         0x00005e40,  0x0000fff0,  R0,              Void,         Void,      Void,  Void)

INSTR (RETLT,         0x00005e50,  0x0000fff0,  R0,              Void,         Void,      Void,  Void)

INSTR (RETMI,         0x00005e60,  0x0000fff0,  R0,              Void,         Void,      Void,  Void)

INSTR (RETPL,         0x00005e70,  0x0000fff0,  R0,              Void,         Void,      Void,  Void)

INSTR (RETLS,         0x00005e80,  0x0000fff0,  R0,              Void,         Void,      Void,  Void)

INSTR (RETGT,         0x00005e90,  0x0000fff0,  R0,              Void,         Void,      Void,  Void)

INSTR (RETLE,         0x00005ea0,  0x0000fff0,  R0,              Void,         Void,      Void,  Void)

INSTR (RETHI,         0x00005eb0,  0x0000fff0,  R0,              Void,         Void,      Void,  Void)

INSTR (RETVS,         0x00005ec0,  0x0000fff0,  R0,              Void,         Void,      Void,  Void)

INSTR (RETVC,         0x00005ed0,  0x0000fff0,  R0,              Void,         Void,      Void,  Void)

INSTR (RETQS,         0x00005ee0,  0x0000fff0,  R0,              Void,         Void,      Void,  Void)

INSTR (RETAL,         0x00005ef0,  0x0000fff0,  R0,              Void,         Void,      Void,  Void)

INSTR (RETD,          0x0000d623,  0x0000ffff,  Void,            Void,         Void,      Void,  Void)

INSTR (RETE,          0x0000d603,  0x0000ffff,  Void,            Void,         Void,      Void,  Void)

INSTR (RETJ,          0x0000d633,  0x0000ffff,  Void,            Void,         Void,      Void,  Void)

INSTR (RETS,          0x0000d613,  0x0000ffff,  Void,            Void,         Void,      Void,  Void)

INSTR (RETSS,         0x0000d763,  0x0000ffff,  Void,            Void,         Void,      Void,  Void)

INSTR (RJMP,          0x0000c008,  0x0000f00c,  PPCS4802T2,      Void,         Void,      Void,  Void)

INSTR (ROL,           0x00005cf0,  0x0000fff0,  R0,              Void,         Void,      Void,  Void)

INSTR (ROR,           0x00005d00,  0x0000fff0,  R0,              Void,         Void,      Void,  Void)

INSTR (RSUB,          0x00000020,  0x0000e1f0,  R0,              R9,           Void,      Void,  Void)
INSTR (RSUB,          0xe0001100,  0xe1f0ff00,  R16,             R25,          S08,       Void,  Void)

INSTR (RSUBEQ,        0xfbb00000,  0xfff0ff00,  R16,             S08,          Void,      Void,  Void)

INSTR (RSUBNE,        0xfbb00100,  0xfff0ff00,  R16,             S08,          Void,      Void,  Void)

INSTR (RSUBCC,        0xfbb00200,  0xfff0ff00,  R16,             S08,          Void,      Void,  Void)

INSTR (RSUBHS,        0xfbb00200,  0xfff0ff00,  R16,             S08,          Void,      Void,  Void)

INSTR (RSUBCS,        0xfbb00300,  0xfff0ff00,  R16,             S08,          Void,      Void,  Void)

INSTR (RSUBLO,        0xfbb00300,  0xfff0ff00,  R16,             S08,          Void,      Void,  Void)

INSTR (RSUBGE,        0xfbb00400,  0xfff0ff00,  R16,             S08,          Void,      Void,  Void)

INSTR (RSUBLT,        0xfbb00500,  0xfff0ff00,  R16,             S08,          Void,      Void,  Void)

INSTR (RSUBMI,        0xfbb00600,  0xfff0ff00,  R16,             S08,          Void,      Void,  Void)

INSTR (RSUBPL,        0xfbb00700,  0xfff0ff00,  R16,             S08,          Void,      Void,  Void)

INSTR (RSUBLS,        0xfbb00800,  0xfff0ff00,  R16,             S08,          Void,      Void,  Void)

INSTR (RSUBGT,        0xfbb00900,  0xfff0ff00,  R16,             S08,          Void,      Void,  Void)

INSTR (RSUBLE,        0xfbb00a00,  0xfff0ff00,  R16,             S08,          Void,      Void,  Void)

INSTR (RSUBHI,        0xfbb00b00,  0xfff0ff00,  R16,             S08,          Void,      Void,  Void)

INSTR (RSUBVS,        0xfbb00c00,  0xfff0ff00,  R16,             S08,          Void,      Void,  Void)

INSTR (RSUBVC,        0xfbb00d00,  0xfff0ff00,  R16,             S08,          Void,      Void,  Void)

INSTR (RSUBQS,        0xfbb00e00,  0xfff0ff00,  R16,             S08,          Void,      Void,  Void)

INSTR (RSUBAL,        0xfbb00f00,  0xfff0ff00,  R16,             S08,          Void,      Void,  Void)

INSTR (SATADDH,       0xe00002c0,  0xe1f0fff0,  R0,              R25,          R16,       Void,  Void)

INSTR (SATADDW,       0xe00000c0,  0xe1f0fff0,  R0,              R25,          R16,       Void,  Void)

INSTR (SATRNDS,       0xf3b00000,  0xfff0fc00,  R16R05,          U55,          Void,      Void,  Void)

INSTR (SATRNDU,       0xf3b00400,  0xfff0fc00,  R16R05,          U55,          Void,      Void,  Void)

INSTR (SATS,          0xf1b00000,  0xfff0fc00,  R16R05,          U55,          Void,      Void,  Void)

INSTR (SATSUBH,       0xe00003c0,  0xe1f0fff0,  R0,              R25,          R16,       Void,  Void)

INSTR (SATSUBW,       0xe00001c0,  0xe1f0fff0,  R0,              R25,          R16,       Void,  Void)
INSTR (SATSUBW,       0xe0d00000,  0xe1f00000,  R16,             R25,          S016,      Void,  Void)

INSTR (SATU,          0xf1b00400,  0xfff0fc00,  R16R05,          U55,          Void,      Void,  Void)

INSTR (SBC,           0xe0000140,  0xe1f0fff0,  R0,              R25,          R16,       Void,  Void)

INSTR (SBR,           0x0000a1a0,  0x0000e1e0,  R0,              U4194,        Void,      Void,  Void)

INSTR (SCALL,         0x0000d733,  0x0000ffff,  Void,            Void,         Void,      Void,  Void)

INSTR (SCR,           0x00005c10,  0x0000fff0,  R0,              Void,         Void,      Void,  Void)

INSTR (SLEEP,         0xe9b00000,  0xffffff00,  U08,             Void,         Void,      Void,  Void)

INSTR (SREQ,          0x00005f00,  0x0000fff0,  R0,              Void,         Void,      Void,  Void)

INSTR (SRNE,          0x00005f10,  0x0000fff0,  R0,              Void,         Void,      Void,  Void)

INSTR (SRCC,          0x00005f20,  0x0000fff0,  R0,              Void,         Void,      Void,  Void)

INSTR (SRHS,          0x00005f20,  0x0000fff0,  R0,              Void,         Void,      Void,  Void)

INSTR (SRCS,          0x00005f30,  0x0000fff0,  R0,              Void,         Void,      Void,  Void)

INSTR (SRLO,          0x00005f30,  0x0000fff0,  R0,              Void,         Void,      Void,  Void)

INSTR (SRGE,          0x00005f40,  0x0000fff0,  R0,              Void,         Void,      Void,  Void)

INSTR (SRLT,          0x00005f50,  0x0000fff0,  R0,              Void,         Void,      Void,  Void)

INSTR (SRMI,          0x00005f60,  0x0000fff0,  R0,              Void,         Void,      Void,  Void)

INSTR (SRPL,          0x00005f70,  0x0000fff0,  R0,              Void,         Void,      Void,  Void)

INSTR (SRLS,          0x00005f80,  0x0000fff0,  R0,              Void,         Void,      Void,  Void)

INSTR (SRGT,          0x00005f90,  0x0000fff0,  R0,              Void,         Void,      Void,  Void)

INSTR (SRLE,          0x00005fa0,  0x0000fff0,  R0,              Void,         Void,      Void,  Void)

INSTR (SRHI,          0x00005fb0,  0x0000fff0,  R0,              Void,         Void,      Void,  Void)

INSTR (SRVS,          0x00005fc0,  0x0000fff0,  R0,              Void,         Void,      Void,  Void)

INSTR (SRVC,          0x00005fd0,  0x0000fff0,  R0,              Void,         Void,      Void,  Void)

INSTR (SRQS,          0x00005fe0,  0x0000fff0,  R0,              Void,         Void,      Void,  Void)

INSTR (SRAL,          0x00005ff0,  0x0000fff0,  R0,              Void,         Void,      Void,  Void)

INSTR (SSCALL,        0x0000d753,  0x0000ffff,  Void,            Void,         Void,      Void,  Void)

INSTR (SSRF,          0x0000d203,  0x0000fe0f,  U45,             Void,         Void,      Void,  Void)

INSTR (STB,           0x000000c0,  0x0000e1f0,  P9I,             R0,           Void,      Void,  Void)
INSTR (STB,           0x000000f0,  0x0000e1f0,  P9D,             R0,           Void,      Void,  Void)
INSTR (STB,           0x0000a080,  0x0000e180,  P9U43,           R0,           Void,      Void,  Void)
INSTR (STB,           0xe1600000,  0xe1f00000,  P25S016,         R16,          Void,      Void,  Void)
INSTR (STB,           0xe0000b00,  0xe1f0ffc0,  P25L16S42,       R0,           Void,      Void,  Void)

INSTR (STBEQ,         0xe1f00e00,  0xe1f0fe00,  P25U09,          R16,          Void,      Void,  Void)

INSTR (STBNE,         0xe1f01e00,  0xe1f0fe00,  P25U09,          R16,          Void,      Void,  Void)

INSTR (STBCC,         0xe1f02e00,  0xe1f0fe00,  P25U09,          R16,          Void,      Void,  Void)

INSTR (STBHS,         0xe1f02e00,  0xe1f0fe00,  P25U09,          R16,          Void,      Void,  Void)

INSTR (STBCS,         0xe1f03e00,  0xe1f0fe00,  P25U09,          R16,          Void,      Void,  Void)

INSTR (STBLO,         0xe1f03e00,  0xe1f0fe00,  P25U09,          R16,          Void,      Void,  Void)

INSTR (STBGE,         0xe1f04e00,  0xe1f0fe00,  P25U09,          R16,          Void,      Void,  Void)

INSTR (STBLT,         0xe1f05e00,  0xe1f0fe00,  P25U09,          R16,          Void,      Void,  Void)

INSTR (STBMI,         0xe1f06e00,  0xe1f0fe00,  P25U09,          R16,          Void,      Void,  Void)

INSTR (STBPL,         0xe1f07e00,  0xe1f0fe00,  P25U09,          R16,          Void,      Void,  Void)

INSTR (STBLS,         0xe1f08e00,  0xe1f0fe00,  P25U09,          R16,          Void,      Void,  Void)

INSTR (STBGT,         0xe1f09e00,  0xe1f0fe00,  P25U09,          R16,          Void,      Void,  Void)

INSTR (STBLE,         0xe1f0ae00,  0xe1f0fe00,  P25U09,          R16,          Void,      Void,  Void)

INSTR (STBHI,         0xe1f0be00,  0xe1f0fe00,  P25U09,          R16,          Void,      Void,  Void)

INSTR (STBVS,         0xe1f0ce00,  0xe1f0fe00,  P25U09,          R16,          Void,      Void,  Void)

INSTR (STBVC,         0xe1f0de00,  0xe1f0fe00,  P25U09,          R16,          Void,      Void,  Void)

INSTR (STBQS,         0xe1f0ee00,  0xe1f0fe00,  P25U09,          R16,          Void,      Void,  Void)

INSTR (STBAL,         0xe1f0ee00,  0xe1f0fe00,  P25U09,          R16,          Void,      Void,  Void)

INSTR (STD,           0x0000a120,  0x0000e1f1,  P9I,             R1D,          Void,      Void,  Void)
INSTR (STD,           0x0000a121,  0x0000e1f1,  P9D,             R1D,          Void,      Void,  Void)
INSTR (STD,           0x0000a111,  0x0000e1f1,  R9,              R1D,          Void,      Void,  Void)
INSTR (STD,           0xe0e10000,  0xe1f10000,  P25S016,         R16D,         Void,      Void,  Void)
INSTR (STD,           0xe0000800,  0xe1f0ffc0,  P25L16S42,       R1D,          Void,      Void,  Void)

INSTR (STH,           0x000000b0,  0x0000e1f0,  P9I,             R0,           Void,      Void,  Void)
INSTR (STH,           0x000000e0,  0x0000e1f0,  P9D,             R0,           Void,      Void,  Void)
INSTR (STH,           0x0000a000,  0x0000e180,  P9U43,           R0,           Void,      Void,  Void)
INSTR (STH,           0xe1500000,  0xe1f00000,  P25S016,         R16,          Void,      Void,  Void)
INSTR (STH,           0xe0000a00,  0xe1f0ffc0,  P25L16S42,       R0,           Void,      Void,  Void)

INSTR (STHEQ,         0xe1f00c00,  0xe1f0fe00,  P25U09T2,        R16,          Void,      Void,  Void)

INSTR (STHNE,         0xe1f01c00,  0xe1f0fe00,  P25U09T2,        R16,          Void,      Void,  Void)

INSTR (STHCC,         0xe1f02c00,  0xe1f0fe00,  P25U09T2,        R16,          Void,      Void,  Void)

INSTR (STHHS,         0xe1f02c00,  0xe1f0fe00,  P25U09T2,        R16,          Void,      Void,  Void)

INSTR (STHCS,         0xe1f03c00,  0xe1f0fe00,  P25U09T2,        R16,          Void,      Void,  Void)

INSTR (STHLO,         0xe1f03c00,  0xe1f0fe00,  P25U09T2,        R16,          Void,      Void,  Void)

INSTR (STHGE,         0xe1f04c00,  0xe1f0fe00,  P25U09T2,        R16,          Void,      Void,  Void)

INSTR (STHLT,         0xe1f05c00,  0xe1f0fe00,  P25U09T2,        R16,          Void,      Void,  Void)

INSTR (STHMI,         0xe1f06c00,  0xe1f0fe00,  P25U09T2,        R16,          Void,      Void,  Void)

INSTR (STHPL,         0xe1f07c00,  0xe1f0fe00,  P25U09T2,        R16,          Void,      Void,  Void)

INSTR (STHLS,         0xe1f08c00,  0xe1f0fe00,  P25U09T2,        R16,          Void,      Void,  Void)

INSTR (STHGT,         0xe1f09c00,  0xe1f0fe00,  P25U09T2,        R16,          Void,      Void,  Void)

INSTR (STHLE,         0xe1f0ac00,  0xe1f0fe00,  P25U09T2,        R16,          Void,      Void,  Void)

INSTR (STHHI,         0xe1f0bc00,  0xe1f0fe00,  P25U09T2,        R16,          Void,      Void,  Void)

INSTR (STHVS,         0xe1f0cc00,  0xe1f0fe00,  P25U09T2,        R16,          Void,      Void,  Void)

INSTR (STHVC,         0xe1f0dc00,  0xe1f0fe00,  P25U09T2,        R16,          Void,      Void,  Void)

INSTR (STHQS,         0xe1f0ec00,  0xe1f0fe00,  P25U09T2,        R16,          Void,      Void,  Void)

INSTR (STHAL,         0xe1f0ec00,  0xe1f0fe00,  P25U09T2,        R16,          Void,      Void,  Void)

INSTR (STW,           0x000000a0,  0x0000e1f0,  P9I,             R0,           Void,      Void,  Void)
INSTR (STW,           0x000000d0,  0x0000e1f0,  P9D,             R0,           Void,      Void,  Void)
INSTR (STW,           0x00008100,  0x0000e100,  P9U44T4,         R0,           Void,      Void,  Void)
INSTR (STW,           0xe1400000,  0xe1f00000,  P25S016,         R16,          Void,      Void,  Void)
INSTR (STW,           0xe0000900,  0xe1f0ffc0,  P25L16S42,       R0,           Void,      Void,  Void)

INSTR (STWEQ,         0xe1f00a00,  0xe1f0fe00,  P25U09T4,        R16,          Void,      Void,  Void)

INSTR (STWNE,         0xe1f01a00,  0xe1f0fe00,  P25U09T4,        R16,          Void,      Void,  Void)

INSTR (STWCC,         0xe1f02a00,  0xe1f0fe00,  P25U09T4,        R16,          Void,      Void,  Void)

INSTR (STWHS,         0xe1f02a00,  0xe1f0fe00,  P25U09T4,        R16,          Void,      Void,  Void)

INSTR (STWCS,         0xe1f03a00,  0xe1f0fe00,  P25U09T4,        R16,          Void,      Void,  Void)

INSTR (STWLO,         0xe1f03a00,  0xe1f0fe00,  P25U09T4,        R16,          Void,      Void,  Void)

INSTR (STWGE,         0xe1f04a00,  0xe1f0fe00,  P25U09T4,        R16,          Void,      Void,  Void)

INSTR (STWLT,         0xe1f05a00,  0xe1f0fe00,  P25U09T4,        R16,          Void,      Void,  Void)

INSTR (STWMI,         0xe1f06a00,  0xe1f0fe00,  P25U09T4,        R16,          Void,      Void,  Void)

INSTR (STWPL,         0xe1f07a00,  0xe1f0fe00,  P25U09T4,        R16,          Void,      Void,  Void)

INSTR (STWLS,         0xe1f08a00,  0xe1f0fe00,  P25U09T4,        R16,          Void,      Void,  Void)

INSTR (STWGT,         0xe1f09a00,  0xe1f0fe00,  P25U09T4,        R16,          Void,      Void,  Void)

INSTR (STWLE,         0xe1f0aa00,  0xe1f0fe00,  P25U09T4,        R16,          Void,      Void,  Void)

INSTR (STWHI,         0xe1f0ba00,  0xe1f0fe00,  P25U09T4,        R16,          Void,      Void,  Void)

INSTR (STWVS,         0xe1f0ca00,  0xe1f0fe00,  P25U09T4,        R16,          Void,      Void,  Void)

INSTR (STWVC,         0xe1f0da00,  0xe1f0fe00,  P25U09T4,        R16,          Void,      Void,  Void)

INSTR (STWQS,         0xe1f0ea00,  0xe1f0fe00,  P25U09T4,        R16,          Void,      Void,  Void)

INSTR (STWAL,         0xe1f0ea00,  0xe1f0fe00,  P25U09T4,        R16,          Void,      Void,  Void)

INSTR (STCD,          0xeba01000,  0xfff01100,  CP13,            P16U08T4,     CR9D,      Void,  Void)
INSTR (STCD,          0xefa00070,  0xfff011ff,  CP13,            P16D,         CR9D,      Void,  Void)
INSTR (STCD,          0xefa010c0,  0xfff011c0,  CP13,            P16L0S42,     CR9D,      Void,  Void)

INSTR (STCW,          0xeba00000,  0xfff01000,  CP13,            P16U08T4,     CR8,       Void,  Void)
INSTR (STCW,          0xefa00060,  0xfff010ff,  CP13,            P16D,         CR8,       Void,  Void)
INSTR (STCW,          0xefa11000,  0xfff010c0,  CP13,            P16L0S42,     CR8,       Void,  Void)

INSTR (STC0D,         0xf7a00000,  0xfff00100,  P16U08124T4,     CR9D,         Void,      Void,  Void)

INSTR (STC0W,         0xf5a00000,  0xfff00000,  P16U08124T4,     CR8,          Void,      Void,  Void)

INSTR (STCOND,        0xe1700000,  0xe1f00000,  P25S016,         R16,          Void,      Void,  Void)

INSTR (STDSP,         0x00005000,  0x0000f800,  PSPU47T4,        R0,           Void,      Void,  Void)

INSTR (STHHW,         0xe1e0c000,  0xe1f0c000,  P0U48T4,         R25P131,      R16P121,   Void,  Void)
INSTR (STHHW,         0xe1e08000,  0xe1f0c0c0,  P0L8S42,         R25P131,      R16P121,   Void,  Void)

INSTR (STSWPH,        0xe1d09000,  0xe1f0f000,  P25S012T2,       R16,          Void,      Void,  Void)

INSTR (STSWPW,        0xe1d0a000,  0xe1f0f000,  P25S012T4,       R16,          Void,      Void,  Void)

INSTR (SUB,           0x00000010,  0x0000e1f0,  R0,              R9,           Void,      Void,  Void)
INSTR (SUB,           0xe0000100,  0xe1f0ffc0,  R0,              R25,          R16L42,    Void,  Void)
INSTR (SUB,           0x0000200d,  0x0000f00f,  RSP,             S48T4,        Void,      Void,  Void)
INSTR (SUB,           0x00002000,  0x0000f000,  R0,              S48,          Void,      Void,  Void)
INSTR (SUB,           0xe0200000,  0xe1e00000,  R16,             S01620254,    Void,      Void,  Void)
INSTR (SUB,           0xe0c00000,  0xe1f00000,  R16,             R25,          S016,      Void,  Void)

INSTR (SUBEQ,         0xf5b00000,  0xfff0ff00,  R16,             S08,          Void,      Void,  Void)
INSTR (SUBEQ,         0xe1d0e010,  0xe1f0fff0,  R0,              R25,          R16,       Void,  Void)

INSTR (SUBNE,         0xf5b00100,  0xfff0ff00,  R16,             S08,          Void,      Void,  Void)
INSTR (SUBNE,         0xe1d0e110,  0xe1f0fff0,  R0,              R25,          R16,       Void,  Void)

INSTR (SUBCC,         0xf5b00200,  0xfff0ff00,  R16,             S08,          Void,      Void,  Void)
INSTR (SUBCC,         0xe1d0e210,  0xe1f0fff0,  R0,              R25,          R16,       Void,  Void)

INSTR (SUBHS,         0xf5b00200,  0xfff0ff00,  R16,             S08,          Void,      Void,  Void)
INSTR (SUBHS,         0xe1d0e210,  0xe1f0fff0,  R0,              R25,          R16,       Void,  Void)

INSTR (SUBCS,         0xf5b00300,  0xfff0ff00,  R16,             S08,          Void,      Void,  Void)
INSTR (SUBCS,         0xe1d0e310,  0xe1f0fff0,  R0,              R25,          R16,       Void,  Void)

INSTR (SUBLO,         0xf5b00300,  0xfff0ff00,  R16,             S08,          Void,      Void,  Void)
INSTR (SUBLO,         0xe1d0e310,  0xe1f0fff0,  R0,              R25,          R16,       Void,  Void)

INSTR (SUBGE,         0xf5b00400,  0xfff0ff00,  R16,             S08,          Void,      Void,  Void)
INSTR (SUBGE,         0xe1d0e410,  0xe1f0fff0,  R0,              R25,          R16,       Void,  Void)

INSTR (SUBLT,         0xf5b00500,  0xfff0ff00,  R16,             S08,          Void,      Void,  Void)
INSTR (SUBLT,         0xe1d0e510,  0xe1f0fff0,  R0,              R25,          R16,       Void,  Void)

INSTR (SUBMI,         0xf5b00600,  0xfff0ff00,  R16,             S08,          Void,      Void,  Void)
INSTR (SUBMI,         0xe1d0e610,  0xe1f0fff0,  R0,              R25,          R16,       Void,  Void)

INSTR (SUBPL,         0xf5b00700,  0xfff0ff00,  R16,             S08,          Void,      Void,  Void)
INSTR (SUBPL,         0xe1d0e710,  0xe1f0fff0,  R0,              R25,          R16,       Void,  Void)

INSTR (SUBLS,         0xf5b00800,  0xfff0ff00,  R16,             S08,          Void,      Void,  Void)
INSTR (SUBLS,         0xe1d0e810,  0xe1f0fff0,  R0,              R25,          R16,       Void,  Void)

INSTR (SUBGT,         0xf5b00900,  0xfff0ff00,  R16,             S08,          Void,      Void,  Void)
INSTR (SUBGT,         0xe1d0e910,  0xe1f0fff0,  R0,              R25,          R16,       Void,  Void)

INSTR (SUBLE,         0xf5b00a00,  0xfff0ff00,  R16,             S08,          Void,      Void,  Void)
INSTR (SUBLE,         0xe1d0ea10,  0xe1f0fff0,  R0,              R25,          R16,       Void,  Void)

INSTR (SUBHI,         0xf5b00b00,  0xfff0ff00,  R16,             S08,          Void,      Void,  Void)
INSTR (SUBHI,         0xe1d0eb10,  0xe1f0fff0,  R0,              R25,          R16,       Void,  Void)

INSTR (SUBVS,         0xf5b00c00,  0xfff0ff00,  R16,             S08,          Void,      Void,  Void)
INSTR (SUBVS,         0xe1d0ec10,  0xe1f0fff0,  R0,              R25,          R16,       Void,  Void)

INSTR (SUBVC,         0xf5b00d00,  0xfff0ff00,  R16,             S08,          Void,      Void,  Void)
INSTR (SUBVC,         0xe1d0ed10,  0xe1f0fff0,  R0,              R25,          R16,       Void,  Void)

INSTR (SUBQS,         0xf5b00e00,  0xfff0ff00,  R16,             S08,          Void,      Void,  Void)
INSTR (SUBQS,         0xe1d0ee10,  0xe1f0fff0,  R0,              R25,          R16,       Void,  Void)

INSTR (SUBAL,         0xf5b00e00,  0xfff0ff00,  R16,             S08,          Void,      Void,  Void)
INSTR (SUBAL,         0xe1d0ee10,  0xe1f0fff0,  R0,              R25,          R16,       Void,  Void)

INSTR (SUBFEQ,        0xf7b00000,  0xfff0ff00,  R16,             S08,          Void,      Void,  Void)

INSTR (SUBFNE,        0xf7b00100,  0xfff0ff00,  R16,             S08,          Void,      Void,  Void)

INSTR (SUBFCC,        0xf7b00200,  0xfff0ff00,  R16,             S08,          Void,      Void,  Void)

INSTR (SUBFHS,        0xf7b00200,  0xfff0ff00,  R16,             S08,          Void,      Void,  Void)

INSTR (SUBFCS,        0xf7b00300,  0xfff0ff00,  R16,             S08,          Void,      Void,  Void)

INSTR (SUBFLO,        0xf7b00300,  0xfff0ff00,  R16,             S08,          Void,      Void,  Void)

INSTR (SUBFGE,        0xf7b00400,  0xfff0ff00,  R16,             S08,          Void,      Void,  Void)

INSTR (SUBFLT,        0xf7b00500,  0xfff0ff00,  R16,             S08,          Void,      Void,  Void)

INSTR (SUBFMI,        0xf7b00600,  0xfff0ff00,  R16,             S08,          Void,      Void,  Void)

INSTR (SUBFPL,        0xf7b00700,  0xfff0ff00,  R16,             S08,          Void,      Void,  Void)

INSTR (SUBFLS,        0xf7b00800,  0xfff0ff00,  R16,             S08,          Void,      Void,  Void)

INSTR (SUBFGT,        0xf7b00900,  0xfff0ff00,  R16,             S08,          Void,      Void,  Void)

INSTR (SUBFLE,        0xf7b00a00,  0xfff0ff00,  R16,             S08,          Void,      Void,  Void)

INSTR (SUBFHI,        0xf7b00b00,  0xfff0ff00,  R16,             S08,          Void,      Void,  Void)

INSTR (SUBFVS,        0xf7b00c00,  0xfff0ff00,  R16,             S08,          Void,      Void,  Void)

INSTR (SUBFVC,        0xf7b00d00,  0xfff0ff00,  R16,             S08,          Void,      Void,  Void)

INSTR (SUBFQS,        0xf7b00e00,  0xfff0ff00,  R16,             S08,          Void,      Void,  Void)

INSTR (SUBFAL,        0xf7b00e00,  0xfff0ff00,  R16,             S08,          Void,      Void,  Void)

INSTR (SUBHHW,        0xe0000f00,  0xe1f0ffc0,  R0,              R25P51,       R16P41,    Void,  Void)

INSTR (SWAPB,         0x00005cb0,  0x0000fff0,  R0,              Void,         Void,      Void,  Void)

INSTR (SWAPBH,        0x00005cc0,  0x0000fff0,  R0,              Void,         Void,      Void,  Void)

INSTR (SWAPH,         0x00005ca0,  0x0000fff0,  R0,              Void,         Void,      Void,  Void)

INSTR (SYNC,          0xebb00000,  0xffffff00,  U08,             Void,         Void,      Void,  Void)

INSTR (TLBR,          0x0000d643,  0x0000ffff,  Void,            Void,         Void,      Void,  Void)

INSTR (TLBS,          0x0000d653,  0x0000ffff,  Void,            Void,         Void,      Void,  Void)

INSTR (TLBW,          0x0000d663,  0x0000ffff,  Void,            Void,         Void,      Void,  Void)

INSTR (TNBZ,          0x00005ce0,  0x0000fff0,  R0,              Void,         Void,      Void,  Void)

INSTR (TST,           0x00000070,  0x0000e1f0,  R0,              R9,           Void,      Void,  Void)

INSTR (XCHG,          0xe0000b40,  0xe1f0fff0,  R0,              R25,          R16,       Void,  Void)

// operand types

TYPE (RSP)
TYPE (R0)
TYPE (R0D)
TYPE (R1D)
TYPE (R9)
TYPE (R16)
TYPE (R16D)
TYPE (R16L42)
TYPE (R16L45)
TYPE (R16R05)
TYPE (R16R45)
TYPE (R16P41)
TYPE (R16P121)
TYPE (R16P122)
TYPE (R25)
TYPE (R25P41)
TYPE (R25P51)
TYPE (R25P131)
TYPE (U05)
TYPE (U08)
TYPE (U08T4)
TYPE (U016)
TYPE (U4194)
TYPE (U45)
TYPE (U48T4)
TYPE (U55)
TYPE (U115)
TYPE (U155)
TYPE (U163)
TYPE (U164)
TYPE (S08)
TYPE (S015T4)
TYPE (S016)
TYPE (S01620254)
TYPE (S01620254T2)
TYPE (S43WZ)
TYPE (S46)
TYPE (S48)
TYPE (S48T2)
TYPE (S48T4)
TYPE (P0U48T4)
TYPE (P0L8S42)
TYPE (P9I)
TYPE (P9D)
TYPE (P9U43)
TYPE (P9U43T2)
TYPE (P9U44T4)
TYPE (P9U45T4)
TYPE (P16D)
TYPE (P16U08T4)
TYPE (P16U08124T4)
TYPE (P16S011)
TYPE (P16S016)
TYPE (P16S016T4)
TYPE (P16L0S42)
TYPE (P25U09)
TYPE (P25U09T2)
TYPE (P25U09T4)
TYPE (P25S012)
TYPE (P25S012T2)
TYPE (P25S012T4)
TYPE (P25S016)
TYPE (P25L16S42)
TYPE (PPCU47T4)
TYPE (PPCS01620254T2)
TYPE (PPCS4802T2)
TYPE (PSPU47T4)
TYPE (CP13)
TYPE (CR0)
TYPE (CR4)
TYPE (CR8)
TYPE (CR9D)
TYPE (COP)
TYPE (COH)

#undef INSTR
#undef MNEM
#undef TYPE

// M68000 instruction set definitions
// Copyright (C) Florian Negele

// This file is part of the Eigen Compiler Suite.

// The ECS is free software: you can redistribute it and/or modify
// it under the terms of the GNU General Public License as published by
// the Free Software Foundation, either version 3 of the License, or
// (at your option) any later version.

// The ECS is distributed in the hope that it will be useful,
// but WITHOUT ANY WARRANTY; without even the implied warranty of
// MERCHANTABILITY or FITNESS FOR A PARTICULAR PURPOSE.  See the
// GNU General Public License for more details.

// You should have received a copy of the GNU General Public License
// along with the ECS.  If not, see <https://www.gnu.org/licenses/>.

#ifndef INSTR
	#define INSTR(mnem, size, code, mask, type1, type2)
#endif

#ifndef MNEM
	#define MNEM(name, mnem, ...)
#endif

#ifndef TYPE
	#define TYPE(type)
#endif

// mnemonics

MNEM (abcd,     ABCD,     Add Decimal with Extend)
MNEM (add,      ADD,      Add)
MNEM (adda,     ADDA,     Add Address)
MNEM (addi,     ADDI,     Add Immediate)
MNEM (addq,     ADDQ,     Add Quick)
MNEM (addx,     ADDX,     Add Extended)
MNEM (and,      AND,      AND Logical)
MNEM (andi,     ANDI,     AND Immediate)
MNEM (asl,      ASL,      Arithmetic Shift Left)
MNEM (asr,      ASR,      Arithmetic Shift Right)
MNEM (bcc,      BCC,      Branch if Carry Clear)
MNEM (bchg,     BCHG,     Test a Bit and Change)
MNEM (bclr,     BCLR,     Test a Bit and Clear)
MNEM (bcs,      BCS,      Branch if Carry Set)
MNEM (beq,      BEQ,      Branch if Equal)
MNEM (bge,      BGE,      Branch if Greater or Equal)
MNEM (bgt,      BGT,      Branch if Greater Than)
MNEM (bhi,      BHI,      Branch if High)
MNEM (bhs,      BHS,      Branch if High or Same)
MNEM (ble,      BLE,      Branch if Less or Equal)
MNEM (blo,      BLO,      Branch if Low)
MNEM (bls,      BLS,      Branch if Low or Same)
MNEM (blt,      BLT,      Branch if Less Than)
MNEM (bmi,      BMI,      Branch if Minus)
MNEM (bne,      BNE,      Branch if Not Equal)
MNEM (bpl,      BPL,      Branch if Plus)
MNEM (bra,      BRA,      Branch Always)
MNEM (bset,     BSET,     Test a Bit and Set)
MNEM (bsr,      BSR,      Branch to Subroutine)
MNEM (btst,     BTST,     Test a Bit)
MNEM (bvc,      BVC,      Branch if Overflow Clear)
MNEM (bvs,      BVS,      Branch if Overflow Set)
MNEM (chk,      CHK,      Check Register Against Bounds)
MNEM (clr,      CLR,      Clear an Operand)
MNEM (cmp,      CMP,      Compare)
MNEM (cmpa,     CMPA,     Compare Address)
MNEM (cmpi,     CMPI,     Compare Immediate)
MNEM (cmpm,     CMPM,     Compare Memory)
MNEM (dbcc,     DBCC,     Test Condition, Decrement, and Branch if Carry Clear)
MNEM (dbcs,     DBCS,     Test Condition, Decrement, and Branch if Carry Set)
MNEM (dbeq,     DBEQ,     Test Condition, Decrement, and Branch if Equal)
MNEM (dbge,     DBGE,     Test Condition, Decrement, and Branch if Greater or Equal)
MNEM (dbgt,     DBGT,     Test Condition, Decrement, and Branch if Greater Than)
MNEM (dbhi,     DBHI,     Test Condition, Decrement, and Branch if High)
MNEM (dbhs,     DBHS,     Test Condition, Decrement, and Branch if High or Same)
MNEM (dble,     DBLE,     Test Condition, Decrement, and Branch if Less or Equal)
MNEM (dblo,     DBLO,     Test Condition, Decrement, and Branch if Low)
MNEM (dbls,     DBLS,     Test Condition, Decrement, and Branch if Low or Same)
MNEM (dblt,     DBLT,     Test Condition, Decrement, and Branch if Less Than)
MNEM (dbmi,     DBMI,     Test Condition, Decrement, and Branch if Minus)
MNEM (dbne,     DBNE,     Test Condition, Decrement, and Branch if Not Equal)
MNEM (dbpl,     DBPL,     Test Condition, Decrement, and Branch if Plus)
MNEM (dbvc,     DBVC,     Test Condition, Decrement, and Branch if Overflow Clear)
MNEM (dbvs,     DBVS,     Test Condition, Decrement, and Branch if Overflow Set)
MNEM (divs,     DIVS,     Signed Divide)
MNEM (divu,     DIVU,     Unsigned Divide)
MNEM (eor,      EOR,      Exclusive-OR Logical)
MNEM (eori,     EORI,     Exclusive-OR Immediate)
MNEM (exg,      EXG,      Exchange Registers)
MNEM (ext,      EXT,      Sign-Extend)
MNEM (illegal,  ILLEGAL,  Take Illegal Instruction Trap)
MNEM (jmp,      JMP,      Jump)
MNEM (jsr,      JSR,      Jump to Subroutine)
MNEM (lea,      LEA,      Load Effective Address)
MNEM (link,     LINK,     Link and Allocate)
MNEM (lsl,      LSL,      Logical Shift Left)
MNEM (lsr,      LSR,      Logical Shift Right)
MNEM (move,     MOVE,     Move Data from Source to Destination)
MNEM (movea,    MOVEA,    Move Address)
MNEM (movep,    MOVEP,    Move Peripheral Data)
MNEM (moveq,    MOVEQ,    Move Quick)
MNEM (muls,     MULS,     Signed Multiply)
MNEM (mulu,     MULU,     Unsigned Multiply)
MNEM (nbcd,     NBCD,     Negate Decimal with Extend)
MNEM (neg,      NEG,      Negate)
MNEM (negx,     NEGX,     Negate with Extend)
MNEM (nop,      NOP,      No Operation)
MNEM (not,      NOT,      Logical Complement)
MNEM (or,       OR,       Inclusive-OR Logical)
MNEM (ori,      ORI,      Inclusive-OR Immediate)
MNEM (pea,      PEA,      Push Effective Address)
MNEM (reset,    RESET,    Reset External Devices)
MNEM (rol,      ROL,      Rotate Left (Without Extend))
MNEM (ror,      ROR,      Rotate Right (Without Extend))
MNEM (roxl,     ROXL,     Rotate Left with Extend)
MNEM (roxr,     ROXR,     Rotate Right with Extend)
MNEM (rte,      RTE,      Return from Exception)
MNEM (rtr,      RTR,      Return and Restore Condition Codes)
MNEM (rts,      RTS,      Return from Subroutine)
MNEM (sbcd,     SBCD,     Subtract Decimal with Extend)
MNEM (scc,      SCC,      Set if Carry Clear)
MNEM (scs,      SCS,      Set if Carry Set)
MNEM (seq,      SEQ,      Set if Equal)
MNEM (sf,       SF,       Set if False)
MNEM (sge,      SGE,      Set if Greater or Equal)
MNEM (sgt,      SGT,      Set if Greater Than)
MNEM (shi,      SHI,      Set if High)
MNEM (shs,      SHS,      Set if High or Same)
MNEM (sle,      SLE,      Set if Less or Equal)
MNEM (slo,      SLO,      Set if Low)
MNEM (sls,      SLS,      Set if Low or Same)
MNEM (slt,      SLT,      Set if Less Than)
MNEM (smi,      SMI,      Set if Minus)
MNEM (sne,      SNE,      Set if Not Equal)
MNEM (spl,      SPL,      Set if Plus)
MNEM (st,       ST,       Set if True)
MNEM (stop,     STOP,     Load Status Register and Stop)
MNEM (sub,      SUB,      Subtract)
MNEM (suba,     SUBA,     Subtract Address)
MNEM (subi,     SUBI,     Subtract Immediate)
MNEM (subq,     SUBQ,     Subtract Quick)
MNEM (subx,     SUBX,     Subtract Extended)
MNEM (svc,      SVC,      Set if Overflow Clear)
MNEM (svs,      SVS,      Set if Overflow Set)
MNEM (swap,     SWAP,     Swap Register Halves)
MNEM (tas,      TAS,      Test and Set an Operand)
MNEM (trap,     TRAP,     Trap)
MNEM (trapv,    TRAPV,    Trap on Overflow)
MNEM (tst,      TST,      Test an Operand)
MNEM (unlk,     UNLK,     Unlink)

// integer instructions

INSTR (ABCD,     None,  0xc100,  0xf1f8,  RD0,     RD9)
INSTR (ABCD,     None,  0xc108,  0xf1f8,  RA0d,    RA9d)

INSTR (SBCD,     None,  0x8100,  0xf1f8,  RD0,     RD9)
INSTR (SBCD,     None,  0x8108,  0xf1f8,  RA0d,    RA9d)

INSTR (ADDX,     Byte,  0xd100,  0xf1f8,  RD0,     RD9)
INSTR (ADDX,     Byte,  0xd108,  0xf1f8,  RA0d,    RA9d)
INSTR (ADDX,     Word,  0xd140,  0xf1f8,  RD0,     RD9)
INSTR (ADDX,     Word,  0xd148,  0xf1f8,  RA0d,    RA9d)
INSTR (ADDX,     Long,  0xd180,  0xf1f8,  RD0,     RD9)
INSTR (ADDX,     Long,  0xd188,  0xf1f8,  RA0d,    RA9d)

INSTR (ADD,      Byte,  0xd000,  0xf1c0,  EA0B,    RD9)
INSTR (ADD,      Byte,  0xd100,  0xf1c0,  RD9,     EA0Bma)
INSTR (ADD,      Word,  0xd040,  0xf1c0,  EA0W,    RD9)
INSTR (ADD,      Word,  0xd140,  0xf1c0,  RD9,     EA0Wma)
INSTR (ADD,      Long,  0xd080,  0xf1c0,  EA0L,    RD9)
INSTR (ADD,      Long,  0xd180,  0xf1c0,  RD9,     EA0Lma)

INSTR (ADDA,     Word,  0xd0c0,  0xf1c0,  EA0W,    RA9)
INSTR (ADDA,     Long,  0xd1c0,  0xf1c0,  EA0L,    RA9)

INSTR (ADDI,     Byte,  0x0600,  0xffc0,  IB,      EA0Bda)
INSTR (ADDI,     Word,  0x0640,  0xffc0,  IW,      EA0Wda)
INSTR (ADDI,     Long,  0x0680,  0xffc0,  IL,      EA0Lda)

INSTR (ADDQ,     Byte,  0x5000,  0xf1c0,  I9,      EA0Ba)
INSTR (ADDQ,     Word,  0x5040,  0xf1c0,  I9,      EA0Wa)
INSTR (ADDQ,     Long,  0x5080,  0xf1c0,  I9,      EA0La)

INSTR (EXG,      None,  0xc140,  0xf1f8,  RD9,     RD0)
INSTR (EXG,      None,  0xc148,  0xf1f8,  RA9,     RA0)
INSTR (EXG,      None,  0xc188,  0xf1f8,  RD9,     RA0)

INSTR (AND,      Byte,  0xc000,  0xf1c0,  EA0Bd,   RD9)
INSTR (AND,      Byte,  0xc100,  0xf1c0,  RD9,     EA0Bma)
INSTR (AND,      Word,  0xc040,  0xf1c0,  EA0Wd,   RD9)
INSTR (AND,      Word,  0xc140,  0xf1c0,  RD9,     EA0Wma)
INSTR (AND,      Long,  0xc080,  0xf1c0,  EA0Ld,   RD9)
INSTR (AND,      Long,  0xc180,  0xf1c0,  RD9,     EA0Lma)

INSTR (ANDI,     None,  0x023c,  0xffff,  IB,      CCR)
INSTR (ANDI,     None,  0x027c,  0xffff,  IW,      SR)

INSTR (ANDI,     Byte,  0x0200,  0xffc0,  IB,      EA0Bda)
INSTR (ANDI,     Word,  0x0240,  0xffc0,  IW,      EA0Wda)
INSTR (ANDI,     Long,  0x0280,  0xffc0,  IL,      EA0Lda)

INSTR (ASL,      Byte,  0xe120,  0xf1f8,  RD9,     RD0)
INSTR (ASL,      Byte,  0xe100,  0xf1f8,  I9,      RD0)
INSTR (ASL,      Word,  0xe160,  0xf1f8,  RD9,     RD0)
INSTR (ASL,      Word,  0xe140,  0xf1f8,  I9,      RD0)
INSTR (ASL,      Word,  0xe1c0,  0xffc0,  EA0Wma,  Void)
INSTR (ASL,      Long,  0xe1a0,  0xf1f8,  RD9,     RD0)
INSTR (ASL,      Long,  0xe180,  0xf1f8,  I9,      RD0)

INSTR (ASR,      Byte,  0xe020,  0xf1f8,  RD9,     RD0)
INSTR (ASR,      Byte,  0xe000,  0xf1f8,  I9,      RD0)
INSTR (ASR,      Word,  0xe060,  0xf1f8,  RD9,     RD0)
INSTR (ASR,      Word,  0xe040,  0xf1f8,  I9,      RD0)
INSTR (ASR,      Word,  0xe0c0,  0xffc0,  EA0Wma,  Void)
INSTR (ASR,      Long,  0xe0a0,  0xf1f8,  RD9,     RD0)
INSTR (ASR,      Long,  0xe080,  0xf1f8,  I9,      RD0)

INSTR (BCC,      Word,  0x6400,  0xffff,  DW,      Void)
INSTR (BCC,      Byte,  0x6400,  0xff00,  DB,      Void)
INSTR (BCS,      Word,  0x6500,  0xffff,  DW,      Void)
INSTR (BCS,      Byte,  0x6500,  0xff00,  DB,      Void)
INSTR (BEQ,      Word,  0x6700,  0xffff,  DW,      Void)
INSTR (BEQ,      Byte,  0x6700,  0xff00,  DB,      Void)
INSTR (BGE,      Word,  0x6c00,  0xffff,  DW,      Void)
INSTR (BGE,      Byte,  0x6c00,  0xff00,  DB,      Void)
INSTR (BGT,      Word,  0x6e00,  0xffff,  DW,      Void)
INSTR (BGT,      Byte,  0x6e00,  0xff00,  DB,      Void)
INSTR (BHI,      Word,  0x6200,  0xffff,  DW,      Void)
INSTR (BHI,      Byte,  0x6200,  0xff00,  DB,      Void)
INSTR (BHS,      Word,  0x6400,  0xffff,  DW,      Void)
INSTR (BHS,      Byte,  0x6400,  0xff00,  DB,      Void)
INSTR (BLE,      Word,  0x6f00,  0xffff,  DW,      Void)
INSTR (BLE,      Byte,  0x6f00,  0xff00,  DB,      Void)
INSTR (BLO,      Word,  0x6500,  0xffff,  DW,      Void)
INSTR (BLO,      Byte,  0x6500,  0xff00,  DB,      Void)
INSTR (BLS,      Word,  0x6300,  0xffff,  DW,      Void)
INSTR (BLS,      Byte,  0x6300,  0xff00,  DB,      Void)
INSTR (BLT,      Word,  0x6d00,  0xffff,  DW,      Void)
INSTR (BLT,      Byte,  0x6d00,  0xff00,  DB,      Void)
INSTR (BMI,      Word,  0x6b00,  0xffff,  DW,      Void)
INSTR (BMI,      Byte,  0x6b00,  0xff00,  DB,      Void)
INSTR (BNE,      Word,  0x6600,  0xffff,  DW,      Void)
INSTR (BNE,      Byte,  0x6600,  0xff00,  DB,      Void)
INSTR (BPL,      Word,  0x6a00,  0xffff,  DW,      Void)
INSTR (BPL,      Byte,  0x6a00,  0xff00,  DB,      Void)
INSTR (BVC,      Word,  0x6800,  0xffff,  DW,      Void)
INSTR (BVC,      Byte,  0x6800,  0xff00,  DB,      Void)
INSTR (BVS,      Word,  0x6900,  0xffff,  DW,      Void)
INSTR (BVS,      Byte,  0x6900,  0xff00,  DB,      Void)

INSTR (MOVEP,    Word,  0x0188,  0xf1f8,  RD9,     RA0m)
INSTR (MOVEP,    Word,  0x0108,  0xf1f8,  RA0m,    RD9)

INSTR (MOVEP,    Long,  0x01c8,  0xf1f8,  RD9,     RA0m)
INSTR (MOVEP,    Long,  0x0148,  0xf1f8,  RA0m,    RD9)

INSTR (BCHG,     None,  0x0140,  0xf1c0,  RD9,     EA0Bda)
INSTR (BCHG,     None,  0x0840,  0xffc0,  IB,      EA0Bda)

INSTR (BCLR,     None,  0x0180,  0xf1c0,  RD9,     EA0Bda)
INSTR (BCLR,     None,  0x0880,  0xffc0,  IB,      EA0Bda)

INSTR (BRA,      Word,  0x6000,  0xffff,  DW,      Void)
INSTR (BRA,      Byte,  0x6000,  0xff00,  DB,      Void)

INSTR (BSET,     None,  0x01c0,  0xf1c0,  RD9,     EA0Bda)
INSTR (BSET,     None,  0x08c0,  0xffc0,  IB,      EA0Bda)

INSTR (BSR,      Word,  0x6100,  0xffff,  DW,      Void)
INSTR (BSR,      Byte,  0x6100,  0xff00,  DB,      Void)

INSTR (BTST,     None,  0x0100,  0xf1c0,  RD9,     EA0Bda)
INSTR (BTST,     None,  0x0800,  0xffc0,  IB,      EA0Bda)

INSTR (CHK,      None,  0x4180,  0xf1c0,  EA0Wd,   RD9)

INSTR (CLR,      Byte,  0x4200,  0xffc0,  EA0Bda,  Void)
INSTR (CLR,      Word,  0x4240,  0xffc0,  EA0Wda,  Void)
INSTR (CLR,      Long,  0x4280,  0xffc0,  EA0Lda,  Void)

INSTR (CMP,      Byte,  0xb000,  0xf1c0,  EA0B,    RD9)
INSTR (CMP,      Word,  0xb040,  0xf1c0,  EA0W,    RD9)
INSTR (CMP,      Long,  0xb080,  0xf1c0,  EA0L,    RD9)

INSTR (CMPA,     Word,  0xb0c0,  0xf1c0,  EA0W,    RA9)
INSTR (CMPA,     Long,  0xb1c0,  0xf1c0,  EA0L,    RA9)

INSTR (CMPI,     Byte,  0x0c00,  0xffc0,  IB,      EA0Bd)
INSTR (CMPI,     Word,  0x0c40,  0xffc0,  IW,      EA0Wd)
INSTR (CMPI,     Long,  0x0c80,  0xffc0,  IL,      EA0Ld)

INSTR (CMPM,     Byte,  0xb108,  0xf1f8,  RA0i,    RA9i)
INSTR (CMPM,     Word,  0xb148,  0xf1f8,  RA0i,    RA9i)
INSTR (CMPM,     Long,  0xb188,  0xf1f8,  RA0i,    RA9i)

INSTR (DBCC,     None,  0x54c8,  0xfff8,  RD0,     DW)
INSTR (DBCS,     None,  0x55c8,  0xfff8,  RD0,     DW)
INSTR (DBEQ,     None,  0x57c8,  0xfff8,  RD0,     DW)
INSTR (DBGE,     None,  0x5cc8,  0xfff8,  RD0,     DW)
INSTR (DBGT,     None,  0x5ec8,  0xfff8,  RD0,     DW)
INSTR (DBHI,     None,  0x52c8,  0xfff8,  RD0,     DW)
INSTR (DBHS,     None,  0x54c8,  0xfff8,  RD0,     DW)
INSTR (DBLE,     None,  0x5fc8,  0xfff8,  RD0,     DW)
INSTR (DBLO,     None,  0x55c8,  0xfff8,  RD0,     DW)
INSTR (DBLS,     None,  0x53c8,  0xfff8,  RD0,     DW)
INSTR (DBLT,     None,  0x5dc8,  0xfff8,  RD0,     DW)
INSTR (DBMI,     None,  0x5bc8,  0xfff8,  RD0,     DW)
INSTR (DBNE,     None,  0x56c8,  0xfff8,  RD0,     DW)
INSTR (DBPL,     None,  0x5ac8,  0xfff8,  RD0,     DW)
INSTR (DBVC,     None,  0x58c8,  0xfff8,  RD0,     DW)
INSTR (DBVS,     None,  0x59c8,  0xfff8,  RD0,     DW)

INSTR (DIVS,     None,  0x81c0,  0xf1c0,  EA0Wd,   RD9)

INSTR (DIVU,     None,  0x80c0,  0xf1c0,  EA0Wd,   RD9)

INSTR (EOR,      Byte,  0xb100,  0xf1c0,  RD9,     EA0Bda)
INSTR (EOR,      Word,  0xb140,  0xf1c0,  RD9,     EA0Wda)
INSTR (EOR,      Long,  0xb180,  0xf1c0,  RD9,     EA0Lda)

INSTR (EORI,     None,  0x0a3c,  0xffff,  IB,      CCR)
INSTR (EORI,     None,  0x0a7c,  0xffff,  IW,      SR)

INSTR (EORI,     Byte,  0x0a00,  0xffc0,  IB,      EA0Bda)
INSTR (EORI,     Word,  0x0a40,  0xffc0,  IW,      EA0Wda)
INSTR (EORI,     Long,  0x0a80,  0xffc0,  IL,      EA0Lda)

INSTR (EXT,      Word,  0x4880,  0xfff8,  RD0,     Void)
INSTR (EXT,      Long,  0x48c0,  0xfff8,  RD0,     Void)

INSTR (ILLEGAL,  None,  0x4afc,  0xffff,  Void,    Void)

INSTR (JMP,      None,  0x4ec0,  0xffc0,  EA0Lc,   Void)

INSTR (JSR,      None,  0x4e80,  0xffc0,  EA0Lc,   Void)

INSTR (LEA,      None,  0x41c0,  0xf1c0,  EA0Lc,   RA9)

INSTR (LINK,     None,  0x4e50,  0xfff8,  RA0,     IW)

INSTR (LSL,      Byte,  0xe128,  0xf1f8,  RD9,     RD0)
INSTR (LSL,      Byte,  0xe108,  0xf1f8,  I9,      RD0)
INSTR (LSL,      Word,  0xe168,  0xf1f8,  RD9,     RD0)
INSTR (LSL,      Word,  0xe148,  0xf1f8,  I9,      RD0)
INSTR (LSL,      Word,  0xe3c0,  0xffc0,  EA0Wma,  Void)
INSTR (LSL,      Long,  0xe1a8,  0xf1f8,  RD9,     RD0)
INSTR (LSL,      Long,  0xe188,  0xf1f8,  I9,      RD0)

INSTR (LSR,      Byte,  0xe028,  0xf1f8,  RD9,     RD0)
INSTR (LSR,      Byte,  0xe008,  0xf1f8,  I9,      RD0)
INSTR (LSR,      Word,  0xe068,  0xf1f8,  RD9,     RD0)
INSTR (LSR,      Word,  0xe048,  0xf1f8,  I9,      RD0)
INSTR (LSR,      Word,  0xe2c0,  0xffc0,  EA0Wma,  Void)
INSTR (LSR,      Long,  0xe0a8,  0xf1f8,  RD9,     RD0)
INSTR (LSR,      Long,  0xe088,  0xf1f8,  I9,      RD0)

INSTR (MOVEA,    Word,  0x3040,  0xf1c0,  EA0W,    RA9)
INSTR (MOVEA,    Long,  0x2040,  0xf1c0,  EA0L,    RA9)

INSTR (MOVE,     Byte,  0x1000,  0xf000,  EA0B,    EA6Bda)
INSTR (MOVE,     Word,  0x3000,  0xf000,  EA0W,    EA6Wda)
INSTR (MOVE,     Long,  0x2000,  0xf000,  EA0L,    EA6Lda)

INSTR (MOVE,     None,  0x42c0,  0xffc0,  CCR,     EA0Wda)
INSTR (MOVE,     None,  0x44c0,  0xffc0,  EA0Wd,   CCR)
INSTR (MOVE,     None,  0x40c0,  0xffc0,  SR,      EA0Wda)
INSTR (MOVE,     None,  0x46c0,  0xffc0,  EA0Wd,   SR)
INSTR (MOVE,     None,  0x4e68,  0xfff8,  USP,     RA0)
INSTR (MOVE,     None,  0x4e60,  0xfff8,  RA0,     USP)

INSTR (MOVEQ,    None,  0x7000,  0xf100,  I08,     RD9)

INSTR (MULS,     None,  0xc1c0,  0xf1c0,  EA0Wd,   RD9)

INSTR (MULU,     None,  0xc0c0,  0xf1c0,  EA0Wd,   RD9)

INSTR (NBCD,     None,  0x4800,  0xffc0,  EA0Bda,  Void)

INSTR (NEG,      Byte,  0x4400,  0xffc0,  EA0Bda,  Void)
INSTR (NEG,      Word,  0x4440,  0xffc0,  EA0Wda,  Void)
INSTR (NEG,      Long,  0x4480,  0xffc0,  EA0Lda,  Void)

INSTR (NEGX,     Byte,  0x4000,  0xffc0,  EA0Bda,  Void)
INSTR (NEGX,     Word,  0x4040,  0xffc0,  EA0Wda,  Void)
INSTR (NEGX,     Long,  0x4080,  0xffc0,  EA0Lda,  Void)

INSTR (NOP,      None,  0x4e71,  0xffff,  Void,    Void)

INSTR (NOT,      Byte,  0x4600,  0xffc0,  EA0Bda,  Void)
INSTR (NOT,      Word,  0x4640,  0xffc0,  EA0Wda,  Void)
INSTR (NOT,      Long,  0x4680,  0xffc0,  EA0Lda,  Void)

INSTR (OR,       Byte,  0x8000,  0xf1c0,  EA0Bd,   RD9)
INSTR (OR,       Byte,  0x8100,  0xf1c0,  RD9,     EA0Bma)
INSTR (OR,       Word,  0x8040,  0xf1c0,  EA0Wd,   RD9)
INSTR (OR,       Word,  0x8140,  0xf1c0,  RD9,     EA0Wma)
INSTR (OR,       Long,  0x8080,  0xf1c0,  EA0Ld,   RD9)
INSTR (OR,       Long,  0x8180,  0xf1c0,  RD9,     EA0Lma)

INSTR (ORI,      None,  0x003c,  0xffff,  IB,      CCR)
INSTR (ORI,      None,  0x007c,  0xffff,  IW,      SR)

INSTR (ORI,      Byte,  0x0000,  0xffc0,  IB,      EA0Bda)
INSTR (ORI,      Word,  0x0040,  0xffc0,  IW,      EA0Wda)
INSTR (ORI,      Long,  0x0080,  0xffc0,  IL,      EA0Lda)

INSTR (SWAP,     None,  0x4840,  0xfff8,  RD0,     Void)

INSTR (PEA,      None,  0x4840,  0xffc0,  EA0Lc,   Void)

INSTR (ROL,      Byte,  0xe138,  0xf1f8,  RD9,     RD0)
INSTR (ROL,      Byte,  0xe118,  0xf1f8,  I9,      RD0)
INSTR (ROL,      Word,  0xe178,  0xf1f8,  RD9,     RD0)
INSTR (ROL,      Word,  0xe158,  0xf1f8,  I9,      RD0)
INSTR (ROL,      Word,  0xe7c0,  0xffc0,  EA0Wma,  Void)
INSTR (ROL,      Long,  0xe1b8,  0xf1f8,  RD9,     RD0)
INSTR (ROL,      Long,  0xe198,  0xf1f8,  I9,      RD0)

INSTR (ROR,      Byte,  0xe038,  0xf1f8,  RD9,     RD0)
INSTR (ROR,      Byte,  0xe018,  0xf1f8,  I9,      RD0)
INSTR (ROR,      Word,  0xe078,  0xf1f8,  RD9,     RD0)
INSTR (ROR,      Word,  0xe058,  0xf1f8,  I9,      RD0)
INSTR (ROR,      Word,  0xe6c0,  0xffc0,  EA0Wma,  Void)
INSTR (ROR,      Long,  0xe0b8,  0xf1f8,  RD9,     RD0)
INSTR (ROR,      Long,  0xe098,  0xf1f8,  I9,      RD0)

INSTR (ROXL,     Byte,  0xe130,  0xf1f8,  RD9,     RD0)
INSTR (ROXL,     Byte,  0xe110,  0xf1f8,  I9,      RD0)
INSTR (ROXL,     Word,  0xe170,  0xf1f8,  RD9,     RD0)
INSTR (ROXL,     Word,  0xe150,  0xf1f8,  I9,      RD0)
INSTR (ROXL,     Word,  0xe5c0,  0xffc0,  EA0Wma,  Void)
INSTR (ROXL,     Long,  0xe1b8,  0xf1f8,  RD9,     RD0)
INSTR (ROXL,     Long,  0xe198,  0xf1f8,  I9,      RD0)

INSTR (ROXR,     Byte,  0xe030,  0xf1f8,  RD9,     RD0)
INSTR (ROXR,     Byte,  0xe010,  0xf1f8,  I9,      RD0)
INSTR (ROXR,     Word,  0xe070,  0xf1f8,  RD9,     RD0)
INSTR (ROXR,     Word,  0xe050,  0xf1f8,  I9,      RD0)
INSTR (ROXR,     Word,  0xe4c0,  0xffc0,  EA0Wma,  Void)
INSTR (ROXR,     Long,  0xe0b0,  0xf1f8,  RD9,     RD0)
INSTR (ROXR,     Long,  0xe090,  0xf1f8,  I9,      RD0)

INSTR (RTR,      None,  0x4e77,  0xffff,  Void,    Void)

INSTR (RTS,      None,  0x4e75,  0xffff,  Void,    Void)

INSTR (SCC,      None,  0x54c0,  0xffc0,  EA0Bda,  Void)
INSTR (SCS,      None,  0x55c0,  0xffc0,  EA0Bda,  Void)
INSTR (SEQ,      None,  0x57c0,  0xffc0,  EA0Bda,  Void)
INSTR (SF,       None,  0x51c0,  0xffc0,  EA0Bda,  Void)
INSTR (SGE,      None,  0x5cc0,  0xffc0,  EA0Bda,  Void)
INSTR (SGT,      None,  0x5ec0,  0xffc0,  EA0Bda,  Void)
INSTR (SHI,      None,  0x52c0,  0xffc0,  EA0Bda,  Void)
INSTR (SHS,      None,  0x54c0,  0xffc0,  EA0Bda,  Void)
INSTR (SLE,      None,  0x5fc0,  0xffc0,  EA0Bda,  Void)
INSTR (SLO,      None,  0x55c0,  0xffc0,  EA0Bda,  Void)
INSTR (SLS,      None,  0x53c0,  0xffc0,  EA0Bda,  Void)
INSTR (SLT,      None,  0x5dc0,  0xffc0,  EA0Bda,  Void)
INSTR (SMI,      None,  0x5bc0,  0xffc0,  EA0Bda,  Void)
INSTR (SNE,      None,  0x56c0,  0xffc0,  EA0Bda,  Void)
INSTR (SPL,      None,  0x5ac0,  0xffc0,  EA0Bda,  Void)
INSTR (ST,       None,  0x50c0,  0xffc0,  EA0Bda,  Void)
INSTR (SVC,      None,  0x58c0,  0xffc0,  EA0Bda,  Void)
INSTR (SVS,      None,  0x59c0,  0xffc0,  EA0Bda,  Void)

INSTR (SUBX,     Byte,  0x9100,  0xf1f8,  RD0,     RD9)
INSTR (SUBX,     Byte,  0x9108,  0xf1f8,  RA0d,    RA9d)
INSTR (SUBX,     Word,  0x9140,  0xf1f8,  RD0,     RD9)
INSTR (SUBX,     Word,  0x9148,  0xf1f8,  RA0d,    RA9d)
INSTR (SUBX,     Long,  0x9180,  0xf1f8,  RD0,     RD9)
INSTR (SUBX,     Long,  0x9188,  0xf1f8,  RA0d,    RA9d)

INSTR (SUB,      Byte,  0x9000,  0xf1c0,  EA0B,    RD9)
INSTR (SUB,      Byte,  0x9100,  0xf1c0,  RD9,     EA0Bma)
INSTR (SUB,      Word,  0x9040,  0xf1c0,  EA0W,    RD9)
INSTR (SUB,      Word,  0x9140,  0xf1c0,  RD9,     EA0Wma)
INSTR (SUB,      Long,  0x9080,  0xf1c0,  EA0L,    RD9)
INSTR (SUB,      Long,  0x9180,  0xf1c0,  RD9,     EA0Lma)

INSTR (SUBA,     Word,  0x90c0,  0xf1c0,  EA0W,    RA9)
INSTR (SUBA,     Long,  0x91c0,  0xf1c0,  EA0L,    RA9)

INSTR (SUBI,     Byte,  0x0400,  0xffc0,  IB,      EA0Bda)
INSTR (SUBI,     Word,  0x0440,  0xffc0,  IW,      EA0Wda)
INSTR (SUBI,     Long,  0x0480,  0xffc0,  IL,      EA0Lda)

INSTR (SUBQ,     Byte,  0x5100,  0xf1c0,  I9,      EA0Ba)
INSTR (SUBQ,     Word,  0x5140,  0xf1c0,  I9,      EA0Wa)
INSTR (SUBQ,     Long,  0x5180,  0xf1c0,  I9,      EA0La)

INSTR (TAS,      None,  0x4ac0,  0xffc0,  EA0Bda,  Void)

INSTR (TRAP,     None,  0x4e40,  0xfff0,  I04,     Void)

INSTR (TRAPV,    None,  0x4e76,  0xffff,  Void,    Void)

INSTR (TST,      Byte,  0x4a00,  0xffc0,  EA0B,    Void)
INSTR (TST,      Word,  0x4a40,  0xffc0,  EA0W,    Void)
INSTR (TST,      Long,  0x4a80,  0xffc0,  EA0L,    Void)

INSTR (UNLK,     None,  0x4e58,  0xfff8,  RA0,     Void)

// supervisor instructions

INSTR (RESET,    None,  0x4e70,  0xffff,  Void,    Void)

INSTR (RTE,      None,  0x4e73,  0xffff,  Void,    Void)

INSTR (STOP,     None,  0x4e72,  0xffff,  IW,      Void)

// operand types

TYPE (Void)
TYPE (IB)
TYPE (IW)
TYPE (IL)
TYPE (I04)
TYPE (I08)
TYPE (I9)
TYPE (DB)
TYPE (DW)
TYPE (RD0)
TYPE (RD9)
TYPE (RA0)
TYPE (RA9)
TYPE (RA0d)
TYPE (RA0i)
TYPE (RA0m)
TYPE (RA9d)
TYPE (RA9i)
TYPE (EA0B)
TYPE (EA0W)
TYPE (EA0L)
TYPE (EA0Ba)
TYPE (EA0Wa)
TYPE (EA0La)
TYPE (EA0Bd)
TYPE (EA0Wd)
TYPE (EA0Ld)
TYPE (EA0Lc)
TYPE (EA0Bda)
TYPE (EA0Wda)
TYPE (EA0Lda)
TYPE (EA0Bma)
TYPE (EA0Wma)
TYPE (EA0Lma)
TYPE (EA6Bda)
TYPE (EA6Wda)
TYPE (EA6Lda)
TYPE (USP)
TYPE (SR)
TYPE (CCR)

#undef INSTR
#undef MNEM
#undef TYPE

// MicroBlaze instruction set definitions
// Copyright (C) Florian Negele

// This file is part of the Eigen Compiler Suite.

// The ECS is free software: you can redistribute it and/or modify
// it under the terms of the GNU General Public License as published by
// the Free Software Foundation, either version 3 of the License, or
// (at your option) any later version.

// The ECS is distributed in the hope that it will be useful,
// but WITHOUT ANY WARRANTY; without even the implied warranty of
// MERCHANTABILITY or FITNESS FOR A PARTICULAR PURPOSE.  See the
// GNU General Public License for more details.

// You should have received a copy of the GNU General Public License
// along with the ECS.  If not, see <https://www.gnu.org/licenses/>.

#ifndef INSTR
	#define INSTR(mnem, code, mask, type1, type2, type3)
#endif

#ifndef MNEM
	#define MNEM(name, mnem, ...)
#endif

#ifndef SREG
	#define SREG(reg, name, number)
#endif

#ifndef TYPE
	#define TYPE(type)
#endif

// mnemonics

MNEM (add,        ADD,        Add)
MNEM (addc,       ADDC,       Add with Carry)
MNEM (addi,       ADDI,       Add Immediate)
MNEM (addic,      ADDIC,      Add Immediate with Carry)
MNEM (addik,      ADDIK,      Add Immediate and Keep Carry)
MNEM (addikc,     ADDIKC,     Add Immediate with Carry and Keep Carry)
MNEM (addk,       ADDK,       Add and Keep Carry)
MNEM (addkc,      ADDKC,      Add with Carry and Keep Carry)
MNEM (aget,       AGET,       Get Data from Stream Interface Atomic)
MNEM (agetd,      AGETD,      Get Data from Stream Interface Dynamic Atomic)
MNEM (and,        AND,        Logical AND)
MNEM (andi,       ANDI,       Logical AND with Immediate)
MNEM (andn,       ANDN,       Logical AND NOT)
MNEM (andni,      ANDNI,      Logical AND NOT with Immediate)
MNEM (aput,       APUT,       Put Data to Stream Interface Atomic)
MNEM (aputd,      APUTD,      Put Data to Stream Interface Dynamic Atomic)
MNEM (beq,        BEQ,        Branch if Equal)
MNEM (beqd,       BEQD,       Branch if Equal with Delay)
MNEM (beqi,       BEQI,       Branch Immediate if Equal)
MNEM (beqid,      BEQID,      Branch Immediate if Equal with Delay)
MNEM (bge,        BGE,        Branch if Greater or Equal)
MNEM (bged,       BGED,       Branch if Greater or Equal with Delay)
MNEM (bgei,       BGEI,       Branch Immediate if Greater or Equal)
MNEM (bgeid,      BGEID,      Branch Immediate if Greater or Equal with Delay)
MNEM (bgt,        BGT,        Branch if Greater Than)
MNEM (bgtd,       BGTD,       Branch if Greater Than with Delay)
MNEM (bgti,       BGTI,       Branch Immediate if Greater Than)
MNEM (bgtid,      BGTID,      Branch Immediate if Greater Than with Delay)
MNEM (ble,        BLE,        Branch if Less or Equal)
MNEM (bled,       BLED,       Branch if Less or Equal with Delay)
MNEM (blei,       BLEI,       Branch Immediate if Less or Equal)
MNEM (bleid,      BLEID,      Branch Immediate if Less or Equal with Delay)
MNEM (blt,        BLT,        Branch if Less Than)
MNEM (bltd,       BLTD,       Branch if Less Than with Delay)
MNEM (blti,       BLTI,       Branch Immediate if Less Than)
MNEM (bltid,      BLTID,      Branch Immediate if Less Than with Delay)
MNEM (bne,        BNE,        Branch if Not Equal)
MNEM (bned,       BNED,       Branch if Not Equal with Delay)
MNEM (bnei,       BNEI,       Branch Immediate if Not Equal)
MNEM (bneid,      BNEID,      Branch Immediate if Not Equal with Delay)
MNEM (br,         BR,         Branch)
MNEM (bra,        BRA,        Branch Absolute)
MNEM (brad,       BRAD,       Branch Absolute with Delay)
MNEM (brai,       BRAI,       Branch Absolute Immediate)
MNEM (braid,      BRAID,      Branch Absolute Immediate with Delay)
MNEM (brald,      BRALD,      Branch Absolute and Link with Delay)
MNEM (bralid,     BRALID,     Branch Absolute and Link Immediate with Delay)
MNEM (brd,        BRD,        Branch with Delay)
MNEM (bri,        BRI,        Branch Immediate)
MNEM (brid,       BRID,       Branch Immediate with Delay)
MNEM (brk,        BRK,        Break)
MNEM (brki,       BRKI,       Break Immediate)
MNEM (brld,       BRLD,       Branch and Link with Delay)
MNEM (brlid,      BRLID,      Branch and Link Immediate with Delay)
MNEM (bsll,       BSLL,       Barrel Shift Left Logical)
MNEM (bslli,      BSLLI,      Barrel Shift Left Logical Immediate)
MNEM (bsra,       BSRA,       Barrel Shift Right Arithmetical)
MNEM (bsrai,      BSRAI,      Barrel Shift Right Arithmetical Immediate)
MNEM (bsrl,       BSRL,       Barrel Shift Right Logical)
MNEM (bsrli,      BSRLI,      Barrel Shift Right Logical Immediate)
MNEM (caget,      CAGET,      Get Control from Stream Interface Atomic)
MNEM (cagetd,     CAGETD,     Get Control from Stream Interface Dynamic Atomic)
MNEM (caput,      CAPUT,      Put Control to Stream Interface Atomic)
MNEM (caputd,     CAPUTD,     Put Control to Stream Interface Dynamic Atomic)
MNEM (cget,       CGET,       Get Control from Stream Interface)
MNEM (cgetd,      CGETD,      Get Control from Stream Interface Dynamic)
MNEM (clz,        CLZ,        Count Leading Zeros)
MNEM (cmp,        CMP,        Signed Integer Compare)
MNEM (cmpu,       CMPU,       Unsigned Integer Compare)
MNEM (cput,       CPUT,       Put Control to Stream Interface)
MNEM (cputd,      CPUTD,      Put Control to Stream Interface Dynamic)
MNEM (eaget,      EAGET,      Get Data from Stream Interface Exceptional Atomic)
MNEM (eagetd,     EAGETD,     Get Data from Stream Interface Dynamic Exceptional Atomic)
MNEM (ecaget,     ECAGET,     Get Control from Stream Interface Exceptional Atomic)
MNEM (ecagetd,    ECAGETD,    Get Control from Stream Interface Dynamic Exceptional Atomic)
MNEM (ecget,      ECGET,      Get Control from Stream Interface Exceptional)
MNEM (ecgetd,     ECGETD,     Get Control from Stream Interface Dynamic Exceptional)
MNEM (eget,       EGET,       Get Data from Stream Interface Exceptional)
MNEM (egetd,      EGETD,      Get Data from Stream Interface Dynamic Exceptional)
MNEM (fadd,       FADD,       Floating Point Arithmetic Add)
MNEM (fcmp.eq,    FCMPEQ,     Equal floating point comparison)
MNEM (fcmp.ge,    FCMPGE,     Greater-or-Equal floating point comparison)
MNEM (fcmp.gt,    FCMPGT,     Greater-than floating point comparison)
MNEM (fcmp.le,    FCMPLE,     Less-or-Equal floating point comparison)
MNEM (fcmp.lt,    FCMPLT,     Less-than floating point comparison)
MNEM (fcmp.ne,    FCMPNE,     Not-Equal floating point comparison)
MNEM (fcmp.un,    FCMPUN,     Unordered floating point comparison)
MNEM (fdiv,       FDIV,       Floating Point Arithmetic Division)
MNEM (fint,       FINT,       Floating Point Convert Float to Integer)
MNEM (flt,        FLT,        Floating Point Convert Integer to Float)
MNEM (fmul,       FMUL,       Floating Point Arithmetic Multiplication)
MNEM (frsub,      FRSUB,      Reverse Floating Point Arithmetic Subtraction)
MNEM (fsqrt,      FSQRT,      Floating Point Arithmetic Square Root)
MNEM (get,        GET,        Get Data from Stream Interface)
MNEM (getd,       GETD,       Get Data from Stream Interface Dynamic)
MNEM (idiv,       IDIV,       Signed Integer Divide)
MNEM (idivu,      IDIVU,      Unsigned Integer Divide)
MNEM (imm,        IMM,        Immediate)
MNEM (lbu,        LBU,        Load Byte Unsigned)
MNEM (lbui,       LBUI,       Load Byte Unsigned Immediate)
MNEM (lbur,       LBUR,       Load Byte Unsigned Reserved)
MNEM (lhu,        LHU,        Load Halfword Unsigned)
MNEM (lhui,       LHUI,       Load Halfword Unsigned Immediate)
MNEM (lhur,       LHUR,       Load Halfword Unsigned Reserved)
MNEM (lw,         LW,         Load Word)
MNEM (lwi,        LWI,        Load Word Immediate)
MNEM (lwr,        LWR,        Load Word Reserved)
MNEM (lwx,        LWX,        Load Word Exclusive)
MNEM (mbar,       MBAR,       Memory Barrier)
MNEM (mfs,        MFS,        Move From Special Purpose Register)
MNEM (msrclr,     MSRCLR,     Read MSR and clear bits in MSR)
MNEM (msrset,     MSRSET,     Read MSR and set bits in MSR)
MNEM (mts,        MTS,        Move To Special Purpose Register)
MNEM (mul,        MUL,        Multiply)
MNEM (mulh,       MULH,       Multiply High)
MNEM (mulhsu,     MULHSU,     Multiply High Signed Unsigned)
MNEM (mulhu,      MULHU,      Multiply High Unsigned)
MNEM (muli,       MULI,       Multiply Immediate)
MNEM (naget,      NAGET,      Get Data from Stream Interface Non-Blocking Atomic)
MNEM (nagetd,     NAGETD,     Get Data from Stream Interface Dynamic Non-Blocking Atomic)
MNEM (naput,      NAPUT,      Put Data to Stream Interface Non-Blocking Atomic)
MNEM (naputd,     NAPUTD,     Put Data to Stream Interface Dynamic Non-Blocking Atomic)
MNEM (ncaget,     NCAGET,     Get Control from Stream Interface Non-Blocking Atomic)
MNEM (ncagetd,    NCAGETD,    Get Control from Stream Interface Dynamic Non-Blocking Atomic)
MNEM (ncaput,     NCAPUT,     Put Control to Stream Interface Non-Blocking Atomic)
MNEM (ncaputd,    NCAPUTD,    Put Control to Stream Interface Dynamic Non-Blocking Atomic)
MNEM (ncget,      NCGET,      Get Control from Stream Interface Non-Blocking)
MNEM (ncgetd,     NCGETD,     Get Control from Stream Interface Dynamic Non-Blocking)
MNEM (ncput,      NCPUT,      Put Control to Stream Interface Non-Blocking)
MNEM (ncputd,     NCPUTD,     Put Control to Stream Interface Dynamic Non-Blocking)
MNEM (neaget,     NEAGET,     Get Data from Stream Interface Non-Blocking Exceptional Atomic)
MNEM (neagetd,    NEAGETD,    Get Data from Stream Interface Dynamic Non-Blocking Exceptional Atomic)
MNEM (necaget,    NECAGET,    Get Control from Stream Interface Non-Blocking Exceptional Atomic)
MNEM (necagetd,   NECAGETD,   Get Control from Stream Interface Dynamic Non-Blocking Exceptional Atomic)
MNEM (necget,     NECGET,     Get Control from Stream Interface Non-Blocking Exceptional)
MNEM (necgetd,    NECGETD,    Get Control from Stream Interface Dynamic Non-Blocking Exceptional)
MNEM (neget,      NEGET,      Get Data from Stream Interface Non-Blocking Exceptional)
MNEM (negetd,     NEGETD,     Get Data from Stream Interface Dynamic Non-Blocking Exceptional)
MNEM (nget,       NGET,       Get Data from Stream Interface Non-Blocking)
MNEM (ngetd,      NGETD,      Get Data from Stream Interface Dynamic Non-Blocking)
MNEM (nop,        NOP,        No Operation)
MNEM (nput,       NPUT,       Put Data to Stream Interface Non-Blocking)
MNEM (nputd,      NPUTD,      Put Data to Stream Interface Dynamic Non-Blocking)
MNEM (or,         OR,         Logical OR)
MNEM (ori,        ORI,        Logical OR with Immediate)
MNEM (pcmpbf,     PCMPBF,     Pattern Compare Byte Find)
MNEM (pcmpeq,     PCMPEQ,     Pattern Compare Equal)
MNEM (pcmpne,     PCMPNE,     Pattern Compare Not Equal)
MNEM (put,        PUT,        Put Data to Stream Interface)
MNEM (putd,       PUTD,       Put Data to Stream Interface Dynamic)
MNEM (rsub,       RSUB,       Arithmetic Reverse Subtract)
MNEM (rsubc,      RSUBC,      Arithmetic Reverse Subtract with Carry)
MNEM (rsubi,      RSUBI,      Arithmetic Reverse Subtract Immediate)
MNEM (rsubic,     RSUBIC,     Arithmetic Reverse Subtract Immediate with Carry)
MNEM (rsubik,     RSUBIK,     Arithmetic Reverse Subtract Immediate and Keep Carry)
MNEM (rsubikc,    RSUBIKC,    Arithmetic Reverse Subtract Immediate with Carry and Keep Carry)
MNEM (rsubk,      RSUBK,      Arithmetic Reverse Subtract and Keep Carry)
MNEM (rsubkc,     RSUBKC,     Arithmetic Reverse Subtract with Carry and Keep Carry)
MNEM (rtbd,       RTBD,       Return from Break)
MNEM (rted,       RTED,       Return from Exception)
MNEM (rtid,       RTID,       Return from Interrupt)
MNEM (rtsd,       RTSD,       Return from Subroutine)
MNEM (sb,         SB,         Store Byte)
MNEM (sbi,        SBI,        Store Byte Immediate)
MNEM (sbr,        SBR,        Store Byte Reserved)
MNEM (sext16,     SEXT16,     Sign Extend Halfword)
MNEM (sext8,      SEXT8,      Sign Extend Byte)
MNEM (sh,         SH,         Store Halfword)
MNEM (shi,        SHI,        Store Halfword Immediate)
MNEM (shr,        SHR,        Store Halfword Reserved)
MNEM (sra,        SRA,        Shift Right Arithmetic)
MNEM (src,        SRC,        Shift Right with Carry)
MNEM (srl,        SRL,        Shift Right Logical)
MNEM (sw,         SW,         Store Word)
MNEM (swapb,      SWAPB,      Swap Bytes)
MNEM (swaph,      SWAPH,      Swap Halfwords)
MNEM (swi,        SWI,        Store Word Immediate)
MNEM (swr,        SWR,        Store Word Reserved)
MNEM (swx,        SWX,        Store Word Exclusive)
MNEM (taget,      TAGET,      Get Data from Stream Interface Test-Only Atomic)
MNEM (tagetd,     TAGETD,     Get Data from Stream Interface Dynamic Test-Only Atomic)
MNEM (taput,      TAPUT,      Put Data to Stream Interface Test-Only Atomic)
MNEM (taputd,     TAPUTD,     Put Data to Stream Interface Dynamic Test-Only Atomic)
MNEM (tcaget,     TCAGET,     Get Control from Stream Interface Test-Only Atomic)
MNEM (tcagetd,    TCAGETD,    Get Control from Stream Interface Dynamic Test-Only Atomic)
MNEM (tcaput,     TCAPUT,     Put Control to Stream Interface Test-Only Atomic)
MNEM (tcaputd,    TCAPUTD,    Put Control to Stream Interface Dynamic Test-Only Atomic)
MNEM (tcget,      TCGET,      Get Control from Stream Interface Test-Only)
MNEM (tcgetd,     TCGETD,     Get Control from Stream Interface Dynamic Test-Only)
MNEM (tcput,      TCPUT,      Put Control to Stream Interface Test-Only)
MNEM (tcputd,     TCPUTD,     Put Control to Stream Interface Dynamic Test-Only)
MNEM (teaget,     TEAGET,     Get Data from Stream Interface Test-Only Exceptional Atomic)
MNEM (teagetd,    TEAGETD,    Get Data from Stream Interface Dynamic Test-Only Exceptional Atomic)
MNEM (tecaget,    TECAGET,    Get Control from Stream Interface Test-Only Exceptional Atomic)
MNEM (tecagetd,   TECAGETD,   Get Control from Stream Interface Dynamic Test-Only Exceptional Atomic)
MNEM (tecget,     TECGET,     Get Control from Stream Interface Test-Only Exceptional)
MNEM (tecgetd,    TECGETD,    Get Control from Stream Interface Dynamic Test-Only Exceptional)
MNEM (teget,      TEGET,      Get Data from Stream Interface Test-Only Exceptional)
MNEM (tegetd,     TEGETD,     Get Data from Stream Interface Dynamic Test-Only Exceptional)
MNEM (tget,       TGET,       Get Data from Stream Interface Test-Only)
MNEM (tgetd,      TGETD,      Get Data from Stream Interface Dynamic Test-Only)
MNEM (tnaget,     TNAGET,     Get Data from Stream Interface Test-Only Non-Blocking Atomic)
MNEM (tnagetd,    TNAGETD,    Get Data from Stream Interface Dynamic Test-Only Non-Blocking Atomic)
MNEM (tnaput,     TNAPUT,     Put Data to Stream Interface Test-Only Non-Blocking Atomic)
MNEM (tnaputd,    TNAPUTD,    Put Data to Stream Interface Dynamic Test-Only Non-Blocking Atomic)
MNEM (tncaget,    TNCAGET,    Get Control from Stream Interface Test-Only Non-Blocking Atomic)
MNEM (tncagetd,   TNCAGETD,   Get Control from Stream Interface Dynamic Test-Only Non-Blocking Atomic)
MNEM (tncaput,    TNCAPUT,    Put Control to Stream Interface Test-Only Non-Blocking Atomic)
MNEM (tncaputd,   TNCAPUTD,   Put Control to Stream Interface Dynamic Test-Only Non-Blocking Atomic)
MNEM (tncget,     TNCGET,     Get Control from Stream Interface Test-Only Non-Blocking)
MNEM (tncgetd,    TNCGETD,    Get Control from Stream Interface Dynamic Test-Only Non-Blocking)
MNEM (tncput,     TNCPUT,     Put Control to Stream Interface Test-Only Non-Blocking)
MNEM (tncputd,    TNCPUTD,    Put Control to Stream Interface Dynamic Test-Only Non-Blocking)
MNEM (tneaget,    TNEAGET,    Get Data from Stream Interface Test-Only Non-Blocking Exceptional Atomic)
MNEM (tneagetd,   TNEAGETD,   Get Data from Stream Interface Dynamic Test-Only Non-Blocking Exceptional Atomic)
MNEM (tnecaget,   TNECAGET,   Get Control from Stream Interface Test-Only Non-Blocking Exceptional Atomic)
MNEM (tnecagetd,  TNECAGETD,  Get Control from Stream Interface Dynamic Test-Only Non-Blocking Exceptional Atomic)
MNEM (tnecget,    TNECGET,    Get Control from Stream Interface Test-Only Non-Blocking Exceptional)
MNEM (tnecgetd,   TNECGETD,   Get Control from Stream Interface Dynamic Test-Only Non-Blocking Exceptional)
MNEM (tneget,     TNEGET,     Get Data from Stream Interface Test-Only Non-Blocking Exceptional)
MNEM (tnegetd,    TNEGETD,    Get Data from Stream Interface Dynamic Test-Only Non-Blocking Exceptional)
MNEM (tnget,      TNGET,      Get Data from Stream Interface Test-Only Non-Blocking)
MNEM (tngetd,     TNGETD,     Get Data from Stream Interface Dynamic Test-Only Non-Blocking)
MNEM (tnput,      TNPUT,      Put Data to Stream Interface Test-Only Non-Blocking)
MNEM (tnputd,     TNPUTD,     Put Data to Stream Interface Dynamic Test-Only Non-Blocking)
MNEM (tput,       TPUT,       Put Data to Stream Interface Test-Only)
MNEM (tputd,      TPUTD,      Put Data to Stream Interface Dynamic Test-Only)
MNEM (wdc,        WDC,        Write to Data Cache)
MNEM (wdc.clear,  WDCCLEAR,   Write to Data Cache Clear)
MNEM (wdc.flush,  WDCFLUSH,   Write to Data Cache Flush)
MNEM (wic,        WIC,        Write to Instruction Cache)
MNEM (xor,        XOR,        Logical Exclusive OR)
MNEM (xori,       XORI,       Logical Exclusive OR with Immediate)

// instructions

INSTR (NOP,        0x00000000,  0xffffffff,  Void,  Void,  Void)
INSTR (ADD,        0x00000000,  0xfc0007ff,  RD,    RA,    RB)
INSTR (ADDC,       0x08000000,  0xfc0007ff,  RD,    RA,    RB)
INSTR (ADDK,       0x10000000,  0xfc0007ff,  RD,    RA,    RB)
INSTR (ADDKC,      0x18000000,  0xfc0007ff,  RD,    RA,    RB)
INSTR (ADDI,       0x20000000,  0xfc000000,  RD,    RA,    S16)
INSTR (ADDIC,      0x28000000,  0xfc000000,  RD,    RA,    S16)
INSTR (ADDIK,      0x30000000,  0xfc000000,  RD,    RA,    S16)
INSTR (ADDIKC,     0x38000000,  0xfc000000,  RD,    RA,    S16)
INSTR (AND,        0x84000000,  0xfc0007ff,  RD,    RA,    RB)
INSTR (ANDI,       0xa4000000,  0xfc000000,  RD,    RA,    S16)
INSTR (ANDN,       0x8c000000,  0xfc0007ff,  RD,    RA,    RB)
INSTR (ANDNI,      0xac000000,  0xfc000000,  RD,    RA,    S16)
INSTR (BEQ,        0x9c000000,  0xffe007ff,  RA,    RB,    Void)
INSTR (BEQD,       0x9e000000,  0xffe007ff,  RA,    RB,    Void)
INSTR (BEQI,       0xbc000000,  0xffe00000,  RA,    O16,   Void)
INSTR (BEQID,      0xbe000000,  0xffe00000,  RA,    O16,   Void)
INSTR (BGE,        0x9ca00000,  0xffe007ff,  RA,    RB,    Void)
INSTR (BGED,       0x9ea00000,  0xffe007ff,  RA,    RB,    Void)
INSTR (BGEI,       0xbca00000,  0xffe00000,  RA,    O16,   Void)
INSTR (BGEID,      0xbea00000,  0xffe00000,  RA,    O16,   Void)
INSTR (BGT,        0x9c800000,  0xffe007ff,  RA,    RB,    Void)
INSTR (BGTD,       0x9e800000,  0xffe007ff,  RA,    RB,    Void)
INSTR (BGTI,       0xbc800000,  0xffe00000,  RA,    O16,   Void)
INSTR (BGTID,      0xbe800000,  0xffe00000,  RA,    O16,   Void)
INSTR (BLE,        0x9c600000,  0xffe007ff,  RA,    RB,    Void)
INSTR (BLED,       0x9e600000,  0xffe007ff,  RA,    RB,    Void)
INSTR (BLEI,       0xbc600000,  0xffe00000,  RA,    O16,   Void)
INSTR (BLEID,      0xbe600000,  0xffe00000,  RA,    O16,   Void)
INSTR (BLT,        0x9c400000,  0xffe007ff,  RA,    RB,    Void)
INSTR (BLTD,       0x9e400000,  0xffe007ff,  RA,    RB,    Void)
INSTR (BLTI,       0xbc400000,  0xffe00000,  RA,    O16,   Void)
INSTR (BLTID,      0xbe400000,  0xffe00000,  RA,    O16,   Void)
INSTR (BNE,        0x9c200000,  0xffe007ff,  RA,    RB,    Void)
INSTR (BNED,       0x9e200000,  0xffe007ff,  RA,    RB,    Void)
INSTR (BNEI,       0xbc200000,  0xffe00000,  RA,    O16,   Void)
INSTR (BNEID,      0xbe200000,  0xffe00000,  RA,    O16,   Void)
INSTR (BR,         0x98000000,  0xfc1f07ff,  RB,    Void,  Void)
INSTR (BRA,        0x98080000,  0xfc1f07ff,  RB,    Void,  Void)
INSTR (BRD,        0x98100000,  0xfc1f07ff,  RB,    Void,  Void)
INSTR (BRAD,       0x98180000,  0xfc1f07ff,  RB,    Void,  Void)
INSTR (BRLD,       0x98140000,  0xfc1f07ff,  RD,    RB,    Void)
INSTR (BRALD,      0x981c0000,  0xfc1f07ff,  RD,    RB,    Void)
INSTR (BRI,        0xb8000000,  0xfc1f0000,  O16,   Void,  Void)
INSTR (BRAI,       0xb8080000,  0xfc1f0000,  S16,   Void,  Void)
INSTR (BRID,       0xb8100000,  0xfc1f0000,  O16,   Void,  Void)
INSTR (BRAID,      0xb8180000,  0xfc1f0000,  O16,   Void,  Void)
INSTR (BRLID,      0xb8140000,  0xfc1f0000,  RD,    O16,   Void)
INSTR (BRALID,     0xb81c0000,  0xfc1f0000,  RD,    S16,   Void)
INSTR (BRK,        0x980c0000,  0xfc1f07ff,  RD,    RB,    Void)
INSTR (BRKI,       0xb80c0000,  0xfc1f0000,  RD,    S16,   Void)
INSTR (BSRL,       0x44000000,  0xfc0007ff,  RD,    RA,    RB)
INSTR (BSRA,       0x44000200,  0xfc0007ff,  RD,    RA,    RB)
INSTR (BSLL,       0x44000400,  0xfc0007ff,  RD,    RA,    RB)
INSTR (BSRLI,      0x64000000,  0xfc00ffe0,  RD,    RA,    U5)
INSTR (BSRAI,      0x64000200,  0xfc00ffe0,  RD,    RA,    U5)
INSTR (BSLLI,      0x64000400,  0xfc00ffe0,  RD,    RA,    U5)
INSTR (CLZ,        0x900000e0,  0xfc00ffff,  RD,    RA,    Void)
INSTR (CMP,        0x14000001,  0xfc0007ff,  RD,    RA,    RB)
INSTR (CMPU,       0x14000003,  0xfc0007ff,  RD,    RA,    RB)
INSTR (FADD,       0x58000000,  0xfc0007ff,  RD,    RA,    RB)
INSTR (FRSUB,      0x58000080,  0xfc0007ff,  RD,    RA,    RB)
INSTR (FMUL,       0x58000100,  0xfc0007ff,  RD,    RA,    RB)
INSTR (FDIV,       0x58000180,  0xfc0007ff,  RD,    RA,    RB)
INSTR (FCMPUN,     0x58000200,  0xfc0007ff,  RD,    RA,    RB)
INSTR (FCMPLT,     0x58000210,  0xfc0007ff,  RD,    RA,    RB)
INSTR (FCMPEQ,     0x58000220,  0xfc0007ff,  RD,    RA,    RB)
INSTR (FCMPLE,     0x58000230,  0xfc0007ff,  RD,    RA,    RB)
INSTR (FCMPGT,     0x58000240,  0xfc0007ff,  RD,    RA,    RB)
INSTR (FCMPNE,     0x58000250,  0xfc0007ff,  RD,    RA,    RB)
INSTR (FCMPGE,     0x58000260,  0xfc0007ff,  RD,    RA,    RB)
INSTR (FLT,        0x58000280,  0xfc00ffff,  RD,    RA,    Void)
INSTR (FINT,       0x58000300,  0xfc00ffff,  RD,    RA,    Void)
INSTR (FSQRT,      0x58000380,  0xfc00ffff,  RD,    RA,    Void)
INSTR (GET,        0x6c000000,  0xfc1ffff0,  RD,    FSL,   Void)
INSTR (TGET,       0x6c001000,  0xfc1ffff0,  RD,    FSL,   Void)
INSTR (NGET,       0x6c004000,  0xfc1ffff0,  RD,    FSL,   Void)
INSTR (TNGET,      0x6c005000,  0xfc1ffff0,  RD,    FSL,   Void)
INSTR (EGET,       0x6c000400,  0xfc1ffff0,  RD,    FSL,   Void)
INSTR (TEGET,      0x6c001400,  0xfc1ffff0,  RD,    FSL,   Void)
INSTR (NEGET,      0x6c004400,  0xfc1ffff0,  RD,    FSL,   Void)
INSTR (TNEGET,     0x6c005400,  0xfc1ffff0,  RD,    FSL,   Void)
INSTR (AGET,       0x6c000800,  0xfc1ffff0,  RD,    FSL,   Void)
INSTR (TAGET,      0x6c001800,  0xfc1ffff0,  RD,    FSL,   Void)
INSTR (NAGET,      0x6c004800,  0xfc1ffff0,  RD,    FSL,   Void)
INSTR (TNAGET,     0x6c005800,  0xfc1ffff0,  RD,    FSL,   Void)
INSTR (EAGET,      0x6c000c00,  0xfc1ffff0,  RD,    FSL,   Void)
INSTR (TEAGET,     0x6c001c00,  0xfc1ffff0,  RD,    FSL,   Void)
INSTR (NEAGET,     0x6c004c00,  0xfc1ffff0,  RD,    FSL,   Void)
INSTR (TNEAGET,    0x6c005c00,  0xfc1ffff0,  RD,    FSL,   Void)
INSTR (CGET,       0x6c002000,  0xfc1ffff0,  RD,    FSL,   Void)
INSTR (TCGET,      0x6c003000,  0xfc1ffff0,  RD,    FSL,   Void)
INSTR (NCGET,      0x6c006000,  0xfc1ffff0,  RD,    FSL,   Void)
INSTR (TNCGET,     0x6c007000,  0xfc1ffff0,  RD,    FSL,   Void)
INSTR (ECGET,      0x6c002400,  0xfc1ffff0,  RD,    FSL,   Void)
INSTR (TECGET,     0x6c003400,  0xfc1ffff0,  RD,    FSL,   Void)
INSTR (NECGET,     0x6c006400,  0xfc1ffff0,  RD,    FSL,   Void)
INSTR (TNECGET,    0x6c007400,  0xfc1ffff0,  RD,    FSL,   Void)
INSTR (CAGET,      0x6c002800,  0xfc1ffff0,  RD,    FSL,   Void)
INSTR (TCAGET,     0x6c003800,  0xfc1ffff0,  RD,    FSL,   Void)
INSTR (NCAGET,     0x6c006800,  0xfc1ffff0,  RD,    FSL,   Void)
INSTR (TNCAGET,    0x6c007800,  0xfc1ffff0,  RD,    FSL,   Void)
INSTR (ECAGET,     0x6c002c00,  0xfc1ffff0,  RD,    FSL,   Void)
INSTR (TECAGET,    0x6c003c00,  0xfc1ffff0,  RD,    FSL,   Void)
INSTR (NECAGET,    0x6c006c00,  0xfc1ffff0,  RD,    FSL,   Void)
INSTR (TNECAGET,   0x6c007c00,  0xfc1ffff0,  RD,    FSL,   Void)
INSTR (GETD,       0x4c000000,  0xfc1f07ff,  RD,    RB,    Void)
INSTR (TGETD,      0x4c000080,  0xfc1f07ff,  RD,    RB,    Void)
INSTR (NGETD,      0x4c000200,  0xfc1f07ff,  RD,    RB,    Void)
INSTR (TNGETD,     0x4c000280,  0xfc1f07ff,  RD,    RB,    Void)
INSTR (EGETD,      0x4c000020,  0xfc1f07ff,  RD,    RB,    Void)
INSTR (TEGETD,     0x4c0000a0,  0xfc1f07ff,  RD,    RB,    Void)
INSTR (NEGETD,     0x4c000220,  0xfc1f07ff,  RD,    RB,    Void)
INSTR (TNEGETD,    0x4c0002a0,  0xfc1f07ff,  RD,    RB,    Void)
INSTR (AGETD,      0x4c000040,  0xfc1f07ff,  RD,    RB,    Void)
INSTR (TAGETD,     0x4c0002c0,  0xfc1f07ff,  RD,    RB,    Void)
INSTR (NAGETD,     0x4c000040,  0xfc1f07ff,  RD,    RB,    Void)
INSTR (TNAGETD,    0x4c0002c0,  0xfc1f07ff,  RD,    RB,    Void)
INSTR (EAGETD,     0x4c000060,  0xfc1f07ff,  RD,    RB,    Void)
INSTR (TEAGETD,    0x4c0000e0,  0xfc1f07ff,  RD,    RB,    Void)
INSTR (NEAGETD,    0x4c000260,  0xfc1f07ff,  RD,    RB,    Void)
INSTR (TNEAGETD,   0x4c0002e0,  0xfc1f07ff,  RD,    RB,    Void)
INSTR (CGETD,      0x4c000100,  0xfc1f07ff,  RD,    RB,    Void)
INSTR (TCGETD,     0x4c000180,  0xfc1f07ff,  RD,    RB,    Void)
INSTR (NCGETD,     0x4c000300,  0xfc1f07ff,  RD,    RB,    Void)
INSTR (TNCGETD,    0x4c000380,  0xfc1f07ff,  RD,    RB,    Void)
INSTR (ECGETD,     0x4c000120,  0xfc1f07ff,  RD,    RB,    Void)
INSTR (TECGETD,    0x4c0001a0,  0xfc1f07ff,  RD,    RB,    Void)
INSTR (NECGETD,    0x4c000320,  0xfc1f07ff,  RD,    RB,    Void)
INSTR (TNECGETD,   0x4c0003a0,  0xfc1f07ff,  RD,    RB,    Void)
INSTR (CAGETD,     0x4c000140,  0xfc1f07ff,  RD,    RB,    Void)
INSTR (TCAGETD,    0x4c0001c0,  0xfc1f07ff,  RD,    RB,    Void)
INSTR (NCAGETD,    0x4c000340,  0xfc1f07ff,  RD,    RB,    Void)
INSTR (TNCAGETD,   0x4c0003c0,  0xfc1f07ff,  RD,    RB,    Void)
INSTR (ECAGETD,    0x4c000160,  0xfc1f07ff,  RD,    RB,    Void)
INSTR (TECAGETD,   0x4c0001e0,  0xfc1f07ff,  RD,    RB,    Void)
INSTR (NECAGETD,   0x4c000360,  0xfc1f07ff,  RD,    RB,    Void)
INSTR (TNECAGETD,  0x4c0003e0,  0xfc1f07ff,  RD,    RB,    Void)
INSTR (IDIV,       0x48000000,  0xfc0007ff,  RD,    RA,    RB)
INSTR (IDIVU,      0x48000010,  0xfc0007ff,  RD,    RA,    RB)
INSTR (IMM,        0xd0000000,  0xffff0000,  S16,   Void,  Void)
INSTR (LBU,        0xc0000000,  0xfc0007ff,  RD,    RA,    RB)
INSTR (LBUR,       0xc0000200,  0xfc0007ff,  RD,    RA,    RB)
INSTR (LBUI,       0xe0000000,  0xfc000000,  RD,    RA,    S16)
INSTR (LHU,        0xc4000000,  0xfc0007ff,  RD,    RA,    RB)
INSTR (LHUR,       0xc4000200,  0xfc0007ff,  RD,    RA,    RB)
INSTR (LHUI,       0xe4000000,  0xfc000000,  RD,    RA,    S16)
INSTR (LW,         0xc8000000,  0xfc0007ff,  RD,    RA,    RB)
INSTR (LWR,        0xc8000200,  0xfc0007ff,  RD,    RA,    RB)
INSTR (LWI,        0xe8000000,  0xfc000000,  RD,    RA,    S16)
INSTR (LWX,        0xc8000400,  0xfc0007ff,  RD,    RA,    RB)
INSTR (MBAR,       0xd8020004,  0xfc1fffff,  U521,  Void,  Void)
INSTR (MFS,        0x94008000,  0xfc1fc000,  RD,    RS,    Void)
INSTR (MSRCLR,     0x94110000,  0xfc1f8000,  RD,    U15,   Void)
INSTR (MSRSET,     0x94100000,  0xfc1f8000,  RD,    U15,   Void)
INSTR (MTS,        0x9400c000,  0xffe0c000,  RT,    RA,    Void)
INSTR (MUL,        0x40000000,  0xfc0007ff,  RD,    RA,    RB)
INSTR (MULH,       0x40000001,  0xfc0007ff,  RD,    RA,    RB)
INSTR (MULHU,      0x40000003,  0xfc0007ff,  RD,    RA,    RB)
INSTR (MULHSU,     0x40000002,  0xfc0007ff,  RD,    RA,    RB)
INSTR (MULI,       0x60000000,  0xfc000000,  RD,    RA,    S16)
INSTR (OR,         0x80000000,  0xfc0007ff,  RD,    RA,    RB)
INSTR (ORI,        0xa0000000,  0xfc000000,  RD,    RA,    S16)
INSTR (PCMPBF,     0x80000400,  0xfc0007ff,  RD,    RA,    RB)
INSTR (PCMPEQ,     0x88000400,  0xfc0007ff,  RD,    RA,    RB)
INSTR (PCMPNE,     0x8c000400,  0xfc0007ff,  RD,    RA,    RB)
INSTR (PUT,        0x6c008000,  0xffe0fff0,  RA,    FSL,   Void)
INSTR (NPUT,       0x6c00c000,  0xffe0fff0,  RA,    FSL,   Void)
INSTR (APUT,       0x6c008800,  0xffe0fff0,  RA,    FSL,   Void)
INSTR (NAPUT,      0x6c00c800,  0xffe0fff0,  RA,    FSL,   Void)
INSTR (TPUT,       0x6c009000,  0xffe0fff0,  FSL,   Void,  Void)
INSTR (TNPUT,      0x6c00d000,  0xffe0fff0,  FSL,   Void,  Void)
INSTR (TAPUT,      0x6c009800,  0xffe0fff0,  FSL,   Void,  Void)
INSTR (TNAPUT,     0x6c00d800,  0xffe0fff0,  FSL,   Void,  Void)
INSTR (CPUT,       0x6c00a000,  0xffe0fff0,  RA,    FSL,   Void)
INSTR (NCPUT,      0x6c00e000,  0xffe0fff0,  RA,    FSL,   Void)
INSTR (CAPUT,      0x6c00a800,  0xffe0fff0,  RA,    FSL,   Void)
INSTR (NCAPUT,     0x6c00e800,  0xffe0fff0,  RA,    FSL,   Void)
INSTR (TCPUT,      0x6c00b000,  0xffe0fff0,  FSL,   Void,  Void)
INSTR (TNCPUT,     0x6c00f000,  0xffe0fff0,  FSL,   Void,  Void)
INSTR (TCAPUT,     0x6c00b800,  0xffe0fff0,  FSL,   Void,  Void)
INSTR (TNCAPUT,    0x6c00f800,  0xffe0fff0,  FSL,   Void,  Void)
INSTR (PUTD,       0x4c000400,  0xffe007ff,  RA,    RB,    Void)
INSTR (NPUTD,      0x4c000600,  0xffe007ff,  RA,    RB,    Void)
INSTR (APUTD,      0x4c000440,  0xffe007ff,  RA,    RB,    Void)
INSTR (NAPUTD,     0x4c000640,  0xffe007ff,  RA,    RB,    Void)
INSTR (TPUTD,      0x4c000480,  0xffe007ff,  RB,    Void,  Void)
INSTR (TNPUTD,     0x4c000680,  0xffe007ff,  RB,    Void,  Void)
INSTR (TAPUTD,     0x4c0004c0,  0xffe007ff,  RB,    Void,  Void)
INSTR (TNAPUTD,    0x4c0006c0,  0xffe007ff,  RB,    Void,  Void)
INSTR (CPUTD,      0x4c000500,  0xffe007ff,  RA,    RB,    Void)
INSTR (NCPUTD,     0x4c000700,  0xffe007ff,  RA,    RB,    Void)
INSTR (CAPUTD,     0x4c000540,  0xffe007ff,  RA,    RB,    Void)
INSTR (NCAPUTD,    0x4c000740,  0xffe007ff,  RA,    RB,    Void)
INSTR (TCPUTD,     0x4c000580,  0xffe007ff,  RB,    Void,  Void)
INSTR (TNCPUTD,    0x4c000780,  0xffe007ff,  RB,    Void,  Void)
INSTR (TCAPUTD,    0x4c0005c0,  0xffe007ff,  RB,    Void,  Void)
INSTR (TNCAPUTD,   0x4c0007c0,  0xffe007ff,  RB,    Void,  Void)
INSTR (RSUB,       0x04000000,  0xfc0007ff,  RD,    RA,    RB)
INSTR (RSUBC,      0x0c000000,  0xfc0007ff,  RD,    RA,    RB)
INSTR (RSUBK,      0x14000000,  0xfc0007ff,  RD,    RA,    RB)
INSTR (RSUBKC,     0x1c000000,  0xfc0007ff,  RD,    RA,    RB)
INSTR (RSUBI,      0x24000000,  0xfc000000,  RD,    RA,    S16)
INSTR (RSUBIC,     0x2c000000,  0xfc000000,  RD,    RA,    S16)
INSTR (RSUBIK,     0x34000000,  0xfc000000,  RD,    RA,    S16)
INSTR (RSUBIKC,    0x3c000000,  0xfc000000,  RD,    RA,    S16)
INSTR (RTBD,       0xb6400000,  0xffe00000,  RA,    S16,   Void)
INSTR (RTID,       0xb6200000,  0xffe00000,  RA,    S16,   Void)
INSTR (RTED,       0xb6800000,  0xffe00000,  RA,    S16,   Void)
INSTR (RTSD,       0xb6000000,  0xffe00000,  RA,    S16,   Void)
INSTR (SB,         0xd0000000,  0xfc0007ff,  RD,    RA,    RB)
INSTR (SBR,        0xd0000200,  0xfc0007ff,  RD,    RA,    RB)
INSTR (SBI,        0xf0000000,  0xfc000000,  RD,    RA,    S16)
INSTR (SEXT8,      0x90000060,  0xfc00ffff,  RD,    RA,    Void)
INSTR (SEXT16,     0x90000061,  0xfc00ffff,  RD,    RA,    Void)
INSTR (SH,         0xd4000000,  0xfc0007ff,  RD,    RA,    RB)
INSTR (SHR,        0xd4000200,  0xfc0007ff,  RD,    RA,    RB)
INSTR (SHI,        0xf4000000,  0xfc000000,  RD,    RA,    S16)
INSTR (SRA,        0x90000001,  0xfc00ffff,  RD,    RA,    Void)
INSTR (SRC,        0x90000021,  0xfc00ffff,  RD,    RA,    Void)
INSTR (SRL,        0x90000041,  0xfc00ffff,  RD,    RA,    Void)
INSTR (SW,         0xd8000000,  0xfc0007ff,  RD,    RA,    RB)
INSTR (SWR,        0xd8000200,  0xfc0007ff,  RD,    RA,    RB)
INSTR (SWAPB,      0x900001e0,  0xfc00ffff,  RD,    RA,    Void)
INSTR (SWAPH,      0x900001e2,  0xfc00ffff,  RD,    RA,    Void)
INSTR (SWI,        0xf8000000,  0xfc000000,  RD,    RA,    S16)
INSTR (SWX,        0xd8000400,  0xfc0007ff,  RD,    RA,    RB)
INSTR (WDC,        0x90000064,  0xffe007ff,  RA,    RB,    Void)
INSTR (WDCFLUSH,   0x90000074,  0xffe007ff,  RA,    RB,    Void)
INSTR (WDCCLEAR,   0x90000066,  0xffe007ff,  RA,    RB,    Void)
INSTR (WIC,        0x90000068,  0xffe007ff,  RA,    RB,    Void)
INSTR (XOR,        0x88000000,  0xfc0007ff,  RD,    RA,    RB)
INSTR (XORI,       0xa8000000,  0xfc000000,  RD,    RA,    S16)

// operand types

TYPE (RA)
TYPE (RB)
TYPE (RD)
TYPE (RS)
TYPE (RT)
TYPE (U5)
TYPE (U521)
TYPE (U15)
TYPE (S16)
TYPE (O16)
TYPE (FSL)

// special registers

SREG (RPC,     rpc,     0x0000)
SREG (RMSR,    rmsr,    0x0001)
SREG (REAR,    rear,    0x0003)
SREG (RESR,    resr,    0x0005)
SREG (RFSR,    rfsr,    0x0007)
SREG (RBTR,    rbtr,    0x000b)
SREG (REDR,    redr,    0x000d)
SREG (RSLR,    rslr,    0x0800)
SREG (RSHR,    rshr,    0x0802)
SREG (RPID,    rpid,    0x1000)
SREG (RZPR,    rzpr,    0x1001)
SREG (RTLBX,   rtlbx,   0x1002)
SREG (RTLBLO,  rtlblo,  0x1003)
SREG (RTLBHI,  rtlbhi,  0x1004)
SREG (RPVR0,   rpvr0,   0x2000)
SREG (RPVR1,   rpvr1,   0x2001)
SREG (RPVR2,   rpvr2,   0x2002)
SREG (RPVR3,   rpvr3,   0x2003)
SREG (RPVR4,   rpvr4,   0x2004)
SREG (RPVR5,   rpvr5,   0x2005)
SREG (RPVR6,   rpvr6,   0x2006)
SREG (RPVR7,   rpvr7,   0x2007)
SREG (RPVR8,   rpvr8,   0x2008)
SREG (RPVR9,   rpvr9,   0x2009)
SREG (RPVR10,  rpvr10,  0x200a)
SREG (RPVR11,  rpvr11,  0x200b)

#undef INSTR
#undef MNEM
#undef SREG
#undef TYPE

// MIPS instruction set definitions
// Copyright (C) Florian Negele

// This file is part of the Eigen Compiler Suite.

// The ECS is free software: you can redistribute it and/or modify
// it under the terms of the GNU General Public License as published by
// the Free Software Foundation, either version 3 of the License, or
// (at your option) any later version.

// The ECS is distributed in the hope that it will be useful,
// but WITHOUT ANY WARRANTY; without even the implied warranty of
// MERCHANTABILITY or FITNESS FOR A PARTICULAR PURPOSE.  See the
// GNU General Public License for more details.

// You should have received a copy of the GNU General Public License
// along with the ECS.  If not, see <https://www.gnu.org/licenses/>.

#ifndef INSTR
	#define INSTR(mnem, code, mask, type1, type2, type3, type4, architecture)
#endif

#ifndef MNEM
	#define MNEM(name, mnem, ...)
#endif

#ifndef TYPE
	#define TYPE(type)
#endif

// mnemonics

MNEM (abs.d,      ABSD,     Double Floating Point Absolute Value)
MNEM (abs.ps,     ABSPS,    Paired Single Floating Point Absolute Value)
MNEM (abs.s,      ABSS,     Single Floating Point Absolute Value)
MNEM (add,        ADD,      Add Word)
MNEM (add.d,      ADDD,     Double Floating Point Add)
MNEM (add.ps,     ADDPS,    Paired Single Floating Point Add)
MNEM (add.s,      ADDS,     Single Floating Point Add)
MNEM (addi,       ADDI,     Add Immediate Word)
MNEM (addiu,      ADDIU,    Add Immediate Unsigned Word)
MNEM (addu,       ADDU,     Add Unsigned Word)
MNEM (alnv.ps,    ALNVPS,   Floating Point Align Variable)
MNEM (and,        AND,      And)
MNEM (andi,       ANDI,     And Immediate)
MNEM (b,          B,        Unconditional Branch)
MNEM (bal,        BAL,      Branch and Link)
MNEM (bc1f,       BC1F,     Branch on FP False)
MNEM (bc1fl,      BC1FL,    Branch on FP False Likely)
MNEM (bc1t,       BC1T,     Branch on FP True)
MNEM (bc1tl,      BC1TL,    Branch on FP True Likely)
MNEM (bc2f,       BC2F,     Branch on FP False)
MNEM (bc2fl,      BC2FL,    Branch on FP False Likely)
MNEM (bc2t,       BC2T,     Branch on FP True)
MNEM (bc2tl,      BC2TL,    Branch on FP True Likely)
MNEM (beq,        BEQ,      Branch on Equal)
MNEM (beql,       BEQL,     Branch on Equal Likely)
MNEM (bgez,       BGEZ,     Branch on Greater Than or Equal to Zero)
MNEM (bgezal,     BGEZAL,   Branch on Greater Than or Equal to Zero and Link)
MNEM (bgezall,    BGEZALL,  Branch on Greater Than or Equal to Zero and Link Likely)
MNEM (bgezl,      BGEZL,    Branch on Greater Than or Equal to Zero Likely)
MNEM (bgtz,       BGTZ,     Branch on Greater Than Zero)
MNEM (bgtzl,      BGTZL,    Branch on Greater Than Zero Likely)
MNEM (blez,       BLEZ,     Branch on Less Than or Equal to Zero)
MNEM (blezl,      BLEZL,    Branch on Less Than or Equal to Zero Likely)
MNEM (bltz,       BLTZ,     Branch on Less Than Zero)
MNEM (bltzal,     BLTZAL,   Branch on Less Than Zero and Link)
MNEM (bltzall,    BLTZALL,  Branch on Less Than Zero and Link Likely)
MNEM (bltzl,      BLTZL,    Branch on Less Than Zero Likely)
MNEM (bne,        BNE,      Branch on Not Equal)
MNEM (bnel,       BNEL,     Branch on Not Equal Likely)
MNEM (break,      BREAK,    Breakpoint)
MNEM (c.eq.d,     CEQD,     Double Floating Point Compare Equal)
MNEM (c.eq.ps,    CEQPS,    Paired Single Floating Point Compare Equal)
MNEM (c.eq.s,     CEQS,     Single Floating Point Compare Equal)
MNEM (c.f.d,      CFD,      Double Floating Point Compare False)
MNEM (c.f.ps,     CFPS,     Paired Single Floating Point Compare False)
MNEM (c.f.s,      CFS,      Single Floating Point Compare False)
MNEM (c.ole.d,    COLED,    Double Floating Point Compare Ordered or Less Than or Equal)
MNEM (c.ole.ps,   COLEPS,   Paired Single Floating Point Compare Ordered or Less Than or Equal)
MNEM (c.ole.s,    COLES,    Single Floating Point Compare Ordered or Less Than or Equal)
MNEM (c.olt.d,    COLTD,    Double Floating Point Compare Ordered or Less Than)
MNEM (c.olt.ps,   COLTPS,   Paired Single Floating Point Compare Ordered or Less Than)
MNEM (c.olt.s,    COLTS,    Single Floating Point Compare Ordered or Less Than)
MNEM (c.ueq.d,    CUEQD,    Double Floating Point Compare Unordered or Equal)
MNEM (c.ueq.ps,   CUEQPS,   Paired Single Floating Point Compare Unordered or Equal)
MNEM (c.ueq.s,    CUEQS,    Single Floating Point Compare Unordered or Equal)
MNEM (c.ule.d,    CULED,    Double Floating Point Compare Unordered or Less Than or Equal)
MNEM (c.ule.ps,   CULEPS,   Paired Single Floating Point Compare Unordered or Less Than or Equal)
MNEM (c.ule.s,    CULES,    Single Floating Point Compare Unordered or Less Than or Equal)
MNEM (c.ult.d,    CULTD,    Double Floating Point Compare Unordered or Less Than)
MNEM (c.ult.ps,   CULTPS,   Paired Single Floating Point Compare Unordered or Less Than)
MNEM (c.ult.s,    CULTS,    Single Floating Point Compare Unordered or Less Than)
MNEM (c.un.d,     CUND,     Double Floating Point Compare Unordered)
MNEM (c.un.ps,    CUNPS,    Paired Single Floating Point Compare Unordered)
MNEM (c.un.s,     CUNS,     Single Floating Point Compare Unordered)
MNEM (cache,      CACHE,    Perform Cache Operation)
MNEM (ceil.l.d,   CEILLD,   Double Fixed Point Ceiling Convert to Long Fixed Point)
MNEM (ceil.l.s,   CEILLS,   Single Fixed Point Ceiling Convert to Long Fixed Point)
MNEM (ceil.w.d,   CEILWD,   Double Floating Point Ceiling Convert to Word Fixed Point)
MNEM (ceil.w.s,   CEILWS,   Single Floating Point Ceiling Convert to Word Fixed Point)
MNEM (cfc1,       CFC1,     Move Control Word From Floating Point)
MNEM (cfc2,       CFC2,     Move Control Word From Coprocessor 2)
MNEM (clo,        CLO,      Count Leading Ones in Word)
MNEM (clz,        CLZ,      Count Leading Zeros in Word)
MNEM (ctc1,       CTC1,     Move Control Word To Floating Point)
MNEM (ctc2,       CTC2,     Move Control Word To Coprocessor 2)
MNEM (cvt.d.l,    CVTDL,    Long Fixed Point Convert to Double Floating Point)
MNEM (cvt.d.s,    CVTDS,    Single Floating Point Convert to Double Floating Point)
MNEM (cvt.d.w,    CVTDW,    Word Fixed Point Convert to Double Floating Point)
MNEM (cvt.l.d,    CVTLD,    Long Fixed Point Convert to Long Fixed Point)
MNEM (cvt.l.s,    CVTLS,    Single Floating Point Convert to Long Fixed Point)
MNEM (cvt.ps.s,   CVTPSS,   Single Floating Point Convert to Paired Single)
MNEM (cvt.s.d,    CVTSD,    Double Floating Point Convert to Single Floating Point)
MNEM (cvt.s.l,    CVTSL,    Long Fixed Point Convert to Single Floating Point)
MNEM (cvt.s.pl,   CVTSPL,   Floating Point Convert Pair Lower to Single Floating Point)
MNEM (cvt.s.pu,   CVTSPU,   Floating Point Convert Pair Upper to Single Floating Point)
MNEM (cvt.s.w,    CVTSW,    Word Fixed Point Convert to Single Floating Point)
MNEM (cvt.w.d,    CVTWD,    Long Fixed Point Convert to Word Fixed Point)
MNEM (cvt.w.s,    CVTWS,    Single Floating Point Convert to Word Fixed Point)
MNEM (dadd,       DADD,     Doubleword Add)
MNEM (daddi,      DADDI,    Doubleword Add Immediate)
MNEM (daddiu,     DADDIU,   Doubleword Add Immediate Unsigned)
MNEM (daddu,      DADDU,    Doubleword Add Unsigned)
MNEM (dclo,       DCLO,     Count Leading Ones in Doubleword)
MNEM (dclz,       DCLZ,     Count Leading Zeros in Doubleword)
MNEM (ddiv,       DDIV,     Doubleword Divide)
MNEM (ddivu,      DDIVU,    Doubleword Divide Unsigned)
MNEM (deret,      DERET,    Debug Exception Return)
MNEM (dext,       DEXT,     Doubleword Extract Bit Field)
MNEM (dextm,      DEXTM,    Doubleword Extract Bit Field Middle)
MNEM (dextu,      DEXTU,    Doubleword Extract Bit Field Upper)
MNEM (di,         DI,       Disable Interrupts)
MNEM (dins,       DINS,     Doubleword Insert Bit Field)
MNEM (dinsm,      DINSM,    Doubleword Insert Bit Field Middle)
MNEM (dinsu,      DINSU,    Doubleword Insert Bit Field Upper)
MNEM (div,        DIV,      Divide Word)
MNEM (div.d,      DIVD,     Double Floating Point Divide)
MNEM (div.s,      DIVS,     Single Floating Point Divide)
MNEM (divu,       DIVU,     Divide Unsigned Word)
MNEM (dmfc0,      DMFC0,    Doubleword Move From Coprocessor 0)
MNEM (dmfc1,      DMFC1,    Doubleword Move From Floating Point)
MNEM (dmfc2,      DMFC2,    Doubleword Move From Coprocessor 2)
MNEM (dmtc0,      DMTC0,    Doubleword Move To Coprocessor 0)
MNEM (dmtc1,      DMTC1,    Doubleword Move To Floating Point)
MNEM (dmtc2,      DMTC2,    Doubleword Move To Coprocessor 2)
MNEM (dmult,      DMULT,    Doubleword Multiply)
MNEM (dmultu,     DMULTU,   Doubleword Multiply Unsigned)
MNEM (drotr,      DROTR,    Doubleword Rotate Right)
MNEM (drotr32,    DROTR32,  Doubleword Rotate Right Plus 32)
MNEM (drotrv,     DROTRV,   Doubleword Rotate Right Variable)
MNEM (dsbh,       DSBH,     Doubleword Swap Bytes Within Halfwords)
MNEM (dshd,       DSHD,     Doubleword Swap Bytes Within Doublewords)
MNEM (dsll,       DSLL,     Doubleword Shift Left Logical)
MNEM (dsll32,     DSLL32,   Doubleword Shift Left Logical Plus 32)
MNEM (dsllv,      DSLLV,    Doubleword Shift Left Logical Variable)
MNEM (dsra,       DSRA,     Doubleword Shift Right Arithmetic)
MNEM (dsra32,     DSRA32,   Doubleword Shift Right Arithmetic Plus 32)
MNEM (dsrav,      DSRAV,    Doubleword Shift Right Arithmetic Variable)
MNEM (dsrl,       DSRL,     Doubleword Shift Right Logical)
MNEM (dsrl32,     DSRL32,   Doubleword Shift Right Logical Plus 32)
MNEM (dsrlv,      DSRLV,    Doubleword Shift Right Logical Variable)
MNEM (dsub,       DSUB,     Doubleword Subtract)
MNEM (dsubu,      DSUBU,    Doubleword Subtract Unsigned)
MNEM (ehb,        EHB,      Execution Hazard Barrier)
MNEM (ei,         EI,       Enable Interrupts)
MNEM (eret,       ERET,     Exception Return)
MNEM (ext,        EXT,      Extract Bit Field)
MNEM (floor.l.d,  FLOORLD,  Double Floating Point Floor Convert to Long Fixed Point)
MNEM (floor.l.s,  FLOORLS,  Single Floating Point Floor Convert to Long Fixed Point)
MNEM (floor.w.d,  FLOORWD,  Double Floating Point Floor Convert to Word Fixed Point)
MNEM (floor.w.s,  FLOORWS,  Single Floating Point Floor Convert to Word Fixed Point)
MNEM (ins,        INS,      Insert Bit Field)
MNEM (j,          J,        Jump)
MNEM (jal,        JAL,      Jump and Link)
MNEM (jalr,       JALR,     Jump and Link Register)
MNEM (jalr.hb,    JALRHB,   Jump and Link Register with Hazard Barrier)
MNEM (jalx,       JALX,     Jump and Link Exchange)
MNEM (jr,         JR,       Jump Register)
MNEM (jr.hb,      JRHB,     Jump Register with Hazard Barrier)
MNEM (lb,         LB,       Load Byte)
MNEM (lbu,        LBU,      Load Byte Unsigned)
MNEM (ld,         LD,       Load Doubleword)
MNEM (ldc1,       LDC1,     Load Doubleword to Floating Point)
MNEM (ldc2,       LDC2,     Load Doubleword to Coprocessor 2)
MNEM (ldl,        LDL,      Load Doubleword Left)
MNEM (ldr,        LDR,      Load Doubleword Right)
MNEM (ldxc1,      LDXC1,    Load Doubleword Indexed to Floating Point)
MNEM (lh,         LH,       Load Halfword)
MNEM (lhu,        LHU,      Load Halfword Unsigned)
MNEM (li,         LI,       Load Immediate)
MNEM (ll,         LL,       Load Linked Word)
MNEM (lld,        LLD,      Load Linked Doubleword)
MNEM (lui,        LUI,      Load Upper Immediate)
MNEM (luxc1,      LUXC1,    Load Doubleword Indexed Unaligned to Floating Point)
MNEM (lw,         LW,       Load Word)
MNEM (lwc1,       LWC1,     Load Word to Floating Point)
MNEM (lwc2,       LWC2,     Load Word to Coprocessor 2)
MNEM (lwl,        LWL,      Load Word Left)
MNEM (lwr,        LWR,      Load Word Right)
MNEM (lwu,        LWU,      Load Word Unsigned)
MNEM (lwxc1,      LWXC1,    Load Word Indexed to Floating Point)
MNEM (madd,       MADD,     Multiply and Add Word to Hi/Lo)
MNEM (madd.d,     MADDD,    Double Floating Point Multiply Add)
MNEM (madd.ps,    MADDPS,   Paired Single Floating Point Multiply Add)
MNEM (madd.s,     MADDS,    Single Floating Point Multiply Add)
MNEM (maddu,      MADDU,    Multiply and Add Unsigned Word to Hi/Lo)
MNEM (mfc0,       MFC0,     Move From Coprocessor 0)
MNEM (mfc1,       MFC1,     Move From Floating Point)
MNEM (mfc2,       MFC2,     Move From Coprocessor 2)
MNEM (mfhc1,      MFHC1,    Move Word From High Half of Floating Point)
MNEM (mfhc2,      MFHC2,    Move Word From High Half of Move From Coprocessor 2 Register)
MNEM (mfhi,       MFHI,     Move From HI Register)
MNEM (mflo,       MFLO,     Move From LO Register)
MNEM (mov.d,      MOVD,     Double Floating Point Move)
MNEM (mov.ps,     MOVPS,    Paired Single Floating Point Move)
MNEM (mov.s,      MOVS,     Single Floating Point Move)
MNEM (movf,       MOVF,     Move Conditional on Floating Point False)
MNEM (movf.d,     MOVFD,    Double Floating Point Move Conditional on Floating Point False)
MNEM (movf.ps,    MOVFPS,   Paired Single Floating Point Move Conditional on Floating Point False)
MNEM (movf.s,     MOVFS,    Single Floating Point Move Conditional on Floating Point False)
MNEM (movn,       MOVN,     Move Conditional on Not Zero)
MNEM (movn.d,     MOVND,    Double Floating Point Move Conditional on Not Zero)
MNEM (movn.ps,    MOVNPS,   Paired Single Floating Point Move Conditional on Not Zero)
MNEM (movn.s,     MOVNS,    Single Floating Point Move Conditional on Not Zero)
MNEM (movt,       MOVT,     Move Conditional on Floating Point True)
MNEM (movt.d,     MOVTD,    Double Floating Point Move Conditional on Floating Point True)
MNEM (movt.ps,    MOVTPS,   Paired Single Floating Point Move Conditional on Floating Point True)
MNEM (movt.s,     MOVTS,    Single Floating Point Move Conditional on Floating Point True)
MNEM (movz,       MOVZ,     Move Conditional on Zero)
MNEM (movz.d,     MOVZD,    Double Floating Point Move Conditional on Zero)
MNEM (movz.ps,    MOVZPS,   Paired Single Floating Point Move Conditional on Zero)
MNEM (movz.s,     MOVZS,    Single Floating Point Move Conditional on Zero)
MNEM (msub,       MSUB,     Multiply and Subtract Word to Hi/Lo)
MNEM (msub.d,     MSUBD,    Double Floating Point Multiply Subtract)
MNEM (msub.ps,    MSUBPS,   Paired Single Floating Point Multiply Subtract)
MNEM (msub.s,     MSUBS,    Single Floating Point Multiply Subtract)
MNEM (msubu,      MSUBU,    Multiply and Subtract Unsigned Word to Hi/Lo)
MNEM (mtc0,       MTC0,     Move To Coprocessor 0)
MNEM (mtc1,       MTC1,     Move To Floating Point)
MNEM (mtc2,       MTC2,     Move To Coprocessor 2)
MNEM (mthc1,      MTHC1,    Move Word To High Half of Floating Point)
MNEM (mthc2,      MTHC2,    Move Word To High Half of Move From Coprocessor 2 Register)
MNEM (mthi,       MTHI,     Move To HI Register)
MNEM (mtlo,       MTLO,     Move To LO Register)
MNEM (mul,        MUL,      Multiply Word to GPR)
MNEM (mul.d,      MULD,     Double Floating Point Multiply)
MNEM (mul.ps,     MULPS,    Double Floating Point Multiply)
MNEM (mul.s,      MULS,     Single Floating Point Multiply)
MNEM (mult,       MULT,     Multiply Word)
MNEM (multu,      MULTU,    Multiply Unsigned Word)
MNEM (neg.d,      NEGD,     Double Floating Point Negate)
MNEM (neg.ps,     NEGPS,    Paired Single Floating Point Negate)
MNEM (neg.s,      NEGS,     Single Floating Point Negate)
MNEM (nmadd.d,    NMADDD,   Double Floating Point Negative Multiply Add)
MNEM (nmadd.ps,   NMADDPS,  Paired Single Floating Point Negative Multiply Add)
MNEM (nmadd.s,    NMADDS,   Single Floating Point Negative Multiply Add)
MNEM (nmsub.d,    NMSUBD,   Double Floating Point Negative Multiply Subtract)
MNEM (nmsub.ps,   NMSUBPS,  Paired Single Floating Point Negative Multiply Subtract)
MNEM (nmsub.s,    NMSUBS,   Single Floating Point Negative Multiply Subtract)
MNEM (nop,        NOP,      No Operation)
MNEM (nor,        NOR,      Not Or)
MNEM (or,         OR,       Or)
MNEM (ori,        ORI,      Or Immediate)
MNEM (pause,      PAUSE,    Wait for the LLBit to clear)
MNEM (pll.ps,     PLLPS,    Pair Lower Lower)
MNEM (plu.ps,     PLUPS,    Pair Lower Upper)
MNEM (pref,       PREF,     Prefetch)
MNEM (prefx,      PREFX,    Prefetch)
MNEM (pul.ps,     PULPS,    Pair Upper Lower)
MNEM (puu.ps,     PUUPS,    Pair Upper Upper)
MNEM (rdhdwr,     RDHDWR,   Read Hardware Register)
MNEM (rdpgpr,     RDPGPR,   Read GPR from Previous Shadow Set)
MNEM (recip.d,    RECIPD,   Double Reciprocal Approximation)
MNEM (recip.s,    RECIPS,   Single Reciprocal Approximation)
MNEM (rotr,       ROTR,     Rotate Word Right)
MNEM (rotrv,      ROTRV,    Rotate Word Right Variable)
MNEM (round.l.d,  ROUNDLD,  Double Floating Point Round To Long Fixed Point)
MNEM (round.l.s,  ROUNDLS,  Single Floating Point Round To Long Fixed Point)
MNEM (round.w.d,  ROUNDWD,  Double Floating Point Round To Word Fixed Point)
MNEM (round.w.s,  ROUNDWS,  Single Floating Point Round To Word Fixed Point)
MNEM (rsqrt.d,    RSQRTD,   Double Reciprocal Square Root Approximation)
MNEM (rsqrt.s,    RSQRTS,   Single Reciprocal Square Root Approximation)
MNEM (sb,         SB,       Store Byte)
MNEM (sc,         SC,       Store Conditional Word)
MNEM (scd,        SCD,      Store Conditional Doubleword)
MNEM (sd,         SD,       Store Doubleword)
MNEM (sdbbp,      SDBBP,    Software Debug Breakpoint)
MNEM (sdc1,       SDC1,     Store Doubleword to Floating Point)
MNEM (sdc2,       SDC2,     Store Doubleword to Coprocessor 2)
MNEM (sdl,        SDL,      Store Doubleword Left)
MNEM (sdr,        SDR,      Store Doubleword Right)
MNEM (sdxc1,      SDXC1,    Store Doubleword Indexed to Floating Point)
MNEM (seb,        SEB,      Sign-Extend Byte)
MNEM (seh,        SEH,      Sign-Extend Halfword)
MNEM (sh,         SH,       Store Halfword)
MNEM (sll,        SLL,      Shift Word Left Logical)
MNEM (sllv,       SLLV,     Shift Word Left Logical Variable)
MNEM (slt,        SLT,      Set on Less Than)
MNEM (slti,       SLTI,     Set on Less Than Immediate)
MNEM (sltiu,      SLTIU,    Set on Less Than Immediate Unsigned)
MNEM (sltu,       SLTU,     Set on Less Than Unsigned)
MNEM (sqrt.d,     SQRTD,    Double Floating Point Square Root)
MNEM (sqrt.s,     SQRTS,    Single Floating Point Square Root)
MNEM (sra,        SRA,      Shift Word Right Arithmetic)
MNEM (srav,       SRAV,     Shift Word Right Arithmetic Variable)
MNEM (srl,        SRL,      Shift Word Right Logical)
MNEM (srlv,       SRLV,     Shift Word Right Logical Variable)
MNEM (ssnop,      SSNOP,    Superscalar No Operation)
MNEM (sub,        SUB,      Subtract Word)
MNEM (sub.d,      SUBD,     Double Floating Point Subtract)
MNEM (sub.ps,     SUBPS,    Paired Single Floating Point Subtract)
MNEM (sub.s,      SUBS,     Single Floating Point Subtract)
MNEM (subu,       SUBU,     Subtract Unsigned Word)
MNEM (suxc1,      SUXC1,    Store Doubleword Indexed Unaligned from Floating Point)
MNEM (sw,         SW,       Store Word)
MNEM (swc1,       SWC1,     Store Word from Floating Point)
MNEM (swc2,       SWC2,     Store Word from Coprocessor 2)
MNEM (swl,        SWL,      Store Word Left)
MNEM (swr,        SWR,      Store Word Right)
MNEM (swxc1,      SWXC1,    Store Word Indexed from Floating Point)
MNEM (sync,       SYNC,     To order loads and stores for shared memory)
MNEM (syscall,    SYSCALL,  System Call)
MNEM (teq,        TEQ,      Trap if Equal)
MNEM (teqi,       TEQI,     Trap if Equal Immediate)
MNEM (tge,        TGE,      Trap if Greater Or Equal)
MNEM (tgei,       TGEI,     Trap if Greater Or Equal Immediate)
MNEM (tgeiu,      TGEIU,    Trap if Greater Or Equal Immediate Unsigned)
MNEM (tgeu,       TGEU,     Trap if Greater Or Equal Unsigned)
MNEM (tlbp,       TLBP,     Probe TLB for Matching Entry)
MNEM (tlbr,       TLBR,     Read Indexed TLB Entry)
MNEM (tlbwi,      TLBWI,    Write Indexed TLB Entry)
MNEM (tlbwr,      TLBWR,    Write Random TLB Entry)
MNEM (tlt,        TLT,      Trap if Less Than)
MNEM (tlti,       TLTI,     Trap if Less Than Immediate)
MNEM (tltiu,      TLTIU,    Trap if Less Than Immediate Unsigned)
MNEM (tltu,       TLTU,     Trap if Less Than Unsigned)
MNEM (tne,        TNE,      Trap if Not Equal)
MNEM (tnei,       TNEI,     Trap if Not Equal Immediate)
MNEM (trunc.l.d,  TRUNCLD,  Double Floating Point Truncate to Long Fixed Point)
MNEM (trunc.l.s,  TRUNCLS,  Single Floating Point Truncate to Long Fixed Point)
MNEM (trunc.w.d,  TRUNCWD,  Double Floating Point Truncate to Word Fixed Point)
MNEM (trunc.w.s,  TRUNCWS,  Single Floating Point Truncate to Word Fixed Point)
MNEM (wait,       WAIT,     Enter Standby Mode)
MNEM (wrpgpr,     WRPGPR,   Write to GPR in Previous Shadow Set)
MNEM (wsbh,       WSBH,     Word Swap Bytes Within Halfwords)
MNEM (xor,        XOR,      Exclusive Or)
MNEM (xori,       XORI,     Exclusive Or Immediate)

// instructions

INSTR (NOP,      0x00000000,  0xffffffff,  Void,    Void,    Void,    Void,     MIPS32)
INSTR (BREAK,    0x0000000d,  0xffffffff,  Void,    Void,    Void,    Void,     MIPS32)
INSTR (DERET,    0x4200001f,  0xffffffff,  Void,    Void,    Void,    Void,     MIPS32)
INSTR (EHB,      0x000000c0,  0xffffffff,  Void,    Void,    Void,    Void,     MIPS32)
INSTR (ERET,     0x42000018,  0xffffffff,  Void,    Void,    Void,    Void,     MIPS32)
INSTR (PAUSE,    0x00000140,  0xffffffff,  Void,    Void,    Void,    Void,     MIPS32)
INSTR (SDBBP,    0x7000003f,  0xffffffff,  Void,    Void,    Void,    Void,     MIPS32)
INSTR (SSNOP,    0x00000040,  0xffffffff,  Void,    Void,    Void,    Void,     MIPS32)
INSTR (SYSCALL,  0x0000000c,  0xffffffff,  Void,    Void,    Void,    Void,     MIPS32)
INSTR (TLBP,     0x42000008,  0xffffffff,  Void,    Void,    Void,    Void,     MIPS32)
INSTR (TLBR,     0x42000001,  0xffffffff,  Void,    Void,    Void,    Void,     MIPS32)
INSTR (TLBWI,    0x42000002,  0xffffffff,  Void,    Void,    Void,    Void,     MIPS32)
INSTR (TLBWR,    0x42000006,  0xffffffff,  Void,    Void,    Void,    Void,     MIPS32)
INSTR (WAIT,     0x42000020,  0xffffffff,  Void,    Void,    Void,    Void,     MIPS32)
INSTR (ABSS,     0x46000005,  0xffff003f,  F6,      F11,     Void,    Void,     MIPS32)
INSTR (ABSD,     0x46200005,  0xffff003f,  F6,      F11,     Void,    Void,     MIPS32)
INSTR (ABSPS,    0x46c00005,  0xffff003f,  F6,      F11,     Void,    Void,     MIPS32)
INSTR (ADD,      0x00000020,  0xfc0007ff,  R11,     R21,     R16,     Void,     MIPS32)
INSTR (ADDS,     0x46000000,  0xffe0003f,  F6,      F11,     F16,     Void,     MIPS32)
INSTR (ADDD,     0x46200000,  0xffe0003f,  F6,      F11,     F16,     Void,     MIPS32)
INSTR (ADDPS,    0x46c00000,  0xffe0003f,  F6,      F11,     F16,     Void,     MIPS32)
INSTR (ADDI,     0x20000000,  0xfc000000,  R16,     R21,     S016,    Void,     MIPS32)
INSTR (ADDIU,    0x24000000,  0xfc000000,  R16,     R21,     S016,    Void,     MIPS32)
INSTR (LI,       0x24000000,  0xffe00000,  R16,     S016,    Void,    Void,     MIPS32)
INSTR (ADDU,     0x00000021,  0xfc0007ff,  R11,     R21,     R16,     Void,     MIPS32)
INSTR (ALNVPS,   0x4c00001e,  0xfc00003f,  F6,      F11,     F16,     R21,      MIPS32)
INSTR (AND,      0x00000024,  0xfc0007ff,  R11,     R21,     R16,     Void,     MIPS32)
INSTR (ANDI,     0x30000000,  0xfc000000,  R16,     R21,     U016,    Void,     MIPS32)
INSTR (B,        0x10000000,  0xffff0000,  S016T4,  Void,    Void,    Void,     MIPS32)
INSTR (BAL,      0x04110000,  0xffff0000,  S016T4,  Void,    Void,    Void,     MIPS32)
INSTR (BC1F,     0x45000000,  0xffe30000,  U183,    S016T4,  Void,    Void,     MIPS32)
INSTR (BC1FL,    0x45020000,  0xffe30000,  U183,    S016T4,  Void,    Void,     MIPS32)
INSTR (BC1T,     0x45010000,  0xffe30000,  U183,    S016T4,  Void,    Void,     MIPS32)
INSTR (BC1TL,    0x45030000,  0xffe30000,  U183,    S016T4,  Void,    Void,     MIPS32)
INSTR (BC2F,     0x49000000,  0xffe30000,  U183,    S016T4,  Void,    Void,     MIPS32)
INSTR (BC2FL,    0x49020000,  0xffe30000,  U183,    S016T4,  Void,    Void,     MIPS32)
INSTR (BC2T,     0x49010000,  0xffe30000,  U183,    S016T4,  Void,    Void,     MIPS32)
INSTR (BC2TL,    0x49030000,  0xffe30000,  U183,    S016T4,  Void,    Void,     MIPS32)
INSTR (BEQ,      0x10000000,  0xfc000000,  R21,     R16,     S016T4,  Void,     MIPS32)
INSTR (BEQL,     0x50000000,  0xfc000000,  R21,     R16,     S016T4,  Void,     MIPS32)
INSTR (BGEZ,     0x04010000,  0xfc1f0000,  R21,     S016T4,  Void,    Void,     MIPS32)
INSTR (BGEZAL,   0x04110000,  0xfc1f0000,  R21,     S016T4,  Void,    Void,     MIPS32)
INSTR (BGEZALL,  0x04130000,  0xfc1f0000,  R21,     S016T4,  Void,    Void,     MIPS32)
INSTR (BGEZL,    0x04030000,  0xfc1f0000,  R21,     S016T4,  Void,    Void,     MIPS32)
INSTR (BGTZ,     0x1c000000,  0xfc1f0000,  R21,     S016T4,  Void,    Void,     MIPS32)
INSTR (BGTZL,    0x5c000000,  0xfc1f0000,  R21,     S016T4,  Void,    Void,     MIPS32)
INSTR (BLEZ,     0x18000000,  0xfc1f0000,  R21,     S016T4,  Void,    Void,     MIPS32)
INSTR (BLEZL,    0x58000000,  0xfc1f0000,  R21,     S016T4,  Void,    Void,     MIPS32)
INSTR (BLTZ,     0x04000000,  0xfc1f0000,  R21,     S016T4,  Void,    Void,     MIPS32)
INSTR (BLTZAL,   0x04100000,  0xfc1f0000,  R21,     S016T4,  Void,    Void,     MIPS32)
INSTR (BLTZALL,  0x04120000,  0xfc1f0000,  R21,     S016T4,  Void,    Void,     MIPS32)
INSTR (BLTZL,    0x04020000,  0xfc1f0000,  R21,     S016T4,  Void,    Void,     MIPS32)
INSTR (BNE,      0x14000000,  0xfc000000,  R21,     R16,     S016T4,  Void,     MIPS32)
INSTR (BNEL,     0x54000000,  0xfc000000,  R21,     R16,     S016T4,  Void,     MIPS32)
INSTR (CFS,      0x46000030,  0xffe000ff,  U83,     F11,     F16,     Void,     MIPS32)
INSTR (CFD,      0x46200030,  0xffe000ff,  U83,     F11,     F16,     Void,     MIPS32)
INSTR (CFPS,     0x46c00030,  0xffe000ff,  U83,     F11,     F16,     Void,     MIPS32)
INSTR (CUNS,     0x46000031,  0xffe000ff,  U83,     F11,     F16,     Void,     MIPS32)
INSTR (CUND,     0x46200031,  0xffe000ff,  U83,     F11,     F16,     Void,     MIPS32)
INSTR (CUNPS,    0x46c00031,  0xffe000ff,  U83,     F11,     F16,     Void,     MIPS32)
INSTR (CEQS,     0x46000032,  0xffe000ff,  U83,     F11,     F16,     Void,     MIPS32)
INSTR (CEQD,     0x46200032,  0xffe000ff,  U83,     F11,     F16,     Void,     MIPS32)
INSTR (CEQPS,    0x46c00032,  0xffe000ff,  U83,     F11,     F16,     Void,     MIPS32)
INSTR (CUEQS,    0x46000033,  0xffe000ff,  U83,     F11,     F16,     Void,     MIPS32)
INSTR (CUEQD,    0x46200033,  0xffe000ff,  U83,     F11,     F16,     Void,     MIPS32)
INSTR (CUEQPS,   0x46c00033,  0xffe000ff,  U83,     F11,     F16,     Void,     MIPS32)
INSTR (COLTS,    0x46000034,  0xffe000ff,  U83,     F11,     F16,     Void,     MIPS32)
INSTR (COLTD,    0x46200034,  0xffe000ff,  U83,     F11,     F16,     Void,     MIPS32)
INSTR (COLTPS,   0x46c00034,  0xffe000ff,  U83,     F11,     F16,     Void,     MIPS32)
INSTR (CULTS,    0x46000035,  0xffe000ff,  U83,     F11,     F16,     Void,     MIPS32)
INSTR (CULTD,    0x46200035,  0xffe000ff,  U83,     F11,     F16,     Void,     MIPS32)
INSTR (CULTPS,   0x46c00035,  0xffe000ff,  U83,     F11,     F16,     Void,     MIPS32)
INSTR (COLES,    0x46000036,  0xffe000ff,  U83,     F11,     F16,     Void,     MIPS32)
INSTR (COLED,    0x46200036,  0xffe000ff,  U83,     F11,     F16,     Void,     MIPS32)
INSTR (COLEPS,   0x46c00036,  0xffe000ff,  U83,     F11,     F16,     Void,     MIPS32)
INSTR (CULES,    0x46000037,  0xffe000ff,  U83,     F11,     F16,     Void,     MIPS32)
INSTR (CULED,    0x46200037,  0xffe000ff,  U83,     F11,     F16,     Void,     MIPS32)
INSTR (CULEPS,   0x46c00037,  0xffe000ff,  U83,     F11,     F16,     Void,     MIPS32)
INSTR (CACHE,    0xbc000000,  0xfc000000,  U165,    S016,    R21LS,   Void,     MIPS32)
INSTR (CEILLS,   0x4600000a,  0xfc1f003f,  F6,      F11,     Void,    Void,     MIPS32)
INSTR (CEILLD,   0x4620000a,  0xfc1f003f,  F6,      F11,     Void,    Void,     MIPS32)
INSTR (CEILWS,   0x4600000e,  0xfc1f003f,  F6,      F11,     Void,    Void,     MIPS32)
INSTR (CEILWD,   0x4620000e,  0xfc1f003f,  F6,      F11,     Void,    Void,     MIPS32)
INSTR (CFC1,     0x42400000,  0xffe007ff,  R16,     F11,     Void,    Void,     MIPS32)
INSTR (CFC2,     0x48400000,  0xffe00000,  R16,     U016,    Void,    Void,     MIPS32)
INSTR (CLO,      0x70000021,  0xfc0007ff,  R11,     R21,     Void,    Void,     MIPS32)
INSTR (CLZ,      0x70000020,  0xfc0007ff,  R11,     R21,     Void,    Void,     MIPS32)
INSTR (CTC1,     0x42c00000,  0xffe007ff,  R16,     F11,     Void,    Void,     MIPS32)
INSTR (CTC2,     0x48c00000,  0xffe00000,  R16,     U016,    Void,    Void,     MIPS32)
INSTR (CVTDS,    0x46000021,  0xffff003f,  F6,      F11,     Void,    Void,     MIPS32)
INSTR (CVTDW,    0x46800021,  0xffff003f,  F6,      F11,     Void,    Void,     MIPS32)
INSTR (CVTDL,    0x46400021,  0xffff003f,  F6,      F11,     Void,    Void,     MIPS32)
INSTR (CVTLS,    0x46000025,  0xffff003f,  F6,      F11,     Void,    Void,     MIPS32)
INSTR (CVTLD,    0x46400025,  0xffff003f,  F6,      F11,     Void,    Void,     MIPS32)
INSTR (CVTPSS,   0x46000026,  0xffe0003f,  F6,      F11,     F16,     Void,     MIPS32)
INSTR (CVTSD,    0x46000020,  0xffff003f,  F6,      F11,     Void,    Void,     MIPS32)
INSTR (CVTSW,    0x46800020,  0xffff003f,  F6,      F11,     Void,    Void,     MIPS32)
INSTR (CVTSL,    0x46400020,  0xffff003f,  F6,      F11,     Void,    Void,     MIPS32)
INSTR (CVTSPL,   0x46c00028,  0xffff003f,  F6,      F11,     Void,    Void,     MIPS32)
INSTR (CVTSPU,   0x46c00020,  0xffff003f,  F6,      F11,     Void,    Void,     MIPS32)
INSTR (CVTWS,    0x46000024,  0xffff003f,  F6,      F11,     Void,    Void,     MIPS32)
INSTR (CVTWD,    0x46400024,  0xffff003f,  F6,      F11,     Void,    Void,     MIPS32)
INSTR (DADD,     0x0000002c,  0xfc0007ff,  R11,     R21,     R16,     Void,     MIPS64)
INSTR (DADDI,    0x60000000,  0xfc000000,  R16,     R21,     S016,    Void,     MIPS64)
INSTR (DADDIU,   0x64000000,  0xfc000000,  R16,     R21,     S016,    Void,     MIPS64)
INSTR (DADDU,    0x0000002d,  0xfc0007ff,  R11,     R21,     R16,     Void,     MIPS64)
INSTR (DCLO,     0x70000025,  0xfc0007ff,  R11,     R21,     Void,    Void,     MIPS64)
INSTR (DCLZ,     0x70000024,  0xfc0007ff,  R11,     R21,     Void,    Void,     MIPS64)
INSTR (DDIV,     0x0000001e,  0xfc00ffff,  R21,     R16,     Void,    Void,     MIPS64)
INSTR (DDIVU,    0x0000001f,  0xfc00ffff,  R21,     R16,     Void,    Void,     MIPS64)
INSTR (DEXT,     0x7c000003,  0xfc00003f,  R16,     R21,     U65,     U115M1,   MIPS64)
INSTR (DEXTM,    0x7c000001,  0xfc00003f,  R16,     R21,     U65,     U115M33,  MIPS64)
INSTR (DEXTU,    0x7c000002,  0xfc00003f,  R16,     R21,     U65M32,  U115M1,   MIPS64)
INSTR (DI,       0x41606000,  0xffe0ffff,  R16,     Void,    Void,    Void,     MIPS32)
INSTR (DINS,     0x7c000007,  0xfc00003f,  R16,     R21,     U65,     U115M1,   MIPS64)
INSTR (DINSM,    0x7c000005,  0xfc00003f,  R16,     R21,     U65,     U115M33,  MIPS64)
INSTR (DINSU,    0x7c000006,  0xfc00003f,  R16,     R21,     U65M32,  U115M1,   MIPS64)
INSTR (DIV,      0x0000001a,  0xfc00ffff,  R21,     R16,     Void,    Void,     MIPS32)
INSTR (DIVS,     0x46000003,  0xffe0003f,  F6,      F11,     F16,     Void,     MIPS32)
INSTR (DIVD,     0x46200003,  0xffe0003f,  F6,      F11,     F16,     Void,     MIPS32)
INSTR (DIVU,     0x0000001b,  0xfc00ffff,  R21,     R16,     Void,    Void,     MIPS32)
INSTR (DMFC0,    0x40200000,  0xffe007f8,  R16,     R11,     U03,     Void,     MIPS64)
INSTR (DMFC1,    0x44200000,  0xffe007ff,  R16,     F11,     Void,    Void,     MIPS64)
INSTR (DMFC2,    0x48200000,  0xffe007f8,  R16,     R11,     U03,     Void,     MIPS64)
INSTR (DMTC0,    0x40a00000,  0xffe007f8,  R16,     R11,     U03,     Void,     MIPS64)
INSTR (DMTC1,    0x44a00000,  0xffe007ff,  R16,     F11,     Void,    Void,     MIPS64)
INSTR (DMTC2,    0x48a00000,  0xffe007f8,  R16,     R11,     U03,     Void,     MIPS64)
INSTR (DMULT,    0x0000001c,  0xfc00ffff,  R21,     R16,     Void,    Void,     MIPS64)
INSTR (DMULTU,   0x0000001d,  0xfc00ffff,  R21,     R16,     Void,    Void,     MIPS64)
INSTR (DROTR,    0x0020003a,  0xffe0003f,  R11,     R16,     U65,     Void,     MIPS64)
INSTR (DROTR32,  0x0020003e,  0xffe0003f,  R11,     R16,     U65M32,  Void,     MIPS64)
INSTR (DROTRV,   0x00000056,  0xfc00007f,  R11,     R16,     R21,     Void,     MIPS64)
INSTR (DSBH,     0x7c0000a4,  0xffe007ff,  R11,     R16,     Void,    Void,     MIPS64)
INSTR (DSHD,     0x7c000164,  0xffe007ff,  R11,     R16,     Void,    Void,     MIPS64)
INSTR (DSLL,     0x00000038,  0xffe0003f,  R11,     R16,     U65,     Void,     MIPS64)
INSTR (DSLL32,   0x0000003c,  0xffe0003f,  R11,     R16,     U65M32,  Void,     MIPS64)
INSTR (DSLLV,    0x00000014,  0xfc00007f,  R11,     R16,     R21,     Void,     MIPS64)
INSTR (DSRA,     0x0000003d,  0xffe0003f,  R11,     R16,     U65,     Void,     MIPS64)
INSTR (DSRA32,   0x0000003f,  0xffe0003f,  R11,     R16,     U65M32,  Void,     MIPS64)
INSTR (DSRAV,    0x00000017,  0xfc00007f,  R11,     R16,     R21,     Void,     MIPS64)
INSTR (DSRL,     0x00000036,  0xffe0003f,  R11,     R16,     U65,     Void,     MIPS64)
INSTR (DSRL32,   0x0000003e,  0xffe0003f,  R11,     R16,     U65M32,  Void,     MIPS64)
INSTR (DSRLV,    0x00000016,  0xfc00007f,  R11,     R16,     R21,     Void,     MIPS64)
INSTR (DSUB,     0x0000002e,  0xfc0007ff,  R11,     R21,     R16,     Void,     MIPS64)
INSTR (DSUBU,    0x0000002f,  0xfc0007ff,  R11,     R21,     R16,     Void,     MIPS64)
INSTR (EI,       0x41606020,  0xffe0ffff,  R16,     Void,    Void,    Void,     MIPS32)
INSTR (EXT,      0x7c000000,  0xfc00003f,  R16,     R21,     U65,     U115M1,   MIPS32)
INSTR (FLOORLS,  0x4600000d,  0xffff003f,  F6,      F11,     Void,    Void,     MIPS32)
INSTR (FLOORLD,  0x4620000d,  0xffff003f,  F6,      F11,     Void,    Void,     MIPS32)
INSTR (FLOORWS,  0x4600000f,  0xffff003f,  F6,      F11,     Void,    Void,     MIPS32)
INSTR (FLOORWD,  0x4620000f,  0xffff003f,  F6,      F11,     Void,    Void,     MIPS32)
INSTR (INS,      0x7c000004,  0xfc00003f,  R16,     R21,     U65,     U115M1,   MIPS32)
INSTR (J,        0x08000000,  0xfc000000,  U026T4,  Void,    Void,    Void,     MIPS32)
INSTR (JAL,      0x0c000000,  0xfc000000,  U026T4,  Void,    Void,    Void,     MIPS32)
INSTR (JALR,     0x00000009,  0xfc1f07ff,  R11,     R21,     Void,    Void,     MIPS32)
INSTR (JALRHB,   0x00000409,  0xfc1f043f,  R11,     R21,     Void,    Void,     MIPS32)
INSTR (JALX,     0x74000000,  0xfc000000,  U026T4,  Void,    Void,    Void,     MIPS64)
INSTR (JR,       0x00000008,  0xfc1fffff,  R21,     Void,    Void,    Void,     MIPS32)
INSTR (JRHB,     0x00000408,  0xfc1ffc3f,  R21,     Void,    Void,    Void,     MIPS32)
INSTR (LB,       0x80000000,  0xfc000000,  R16,     S016,    R21LS,   Void,     MIPS32)
INSTR (LBU,      0x90000000,  0xfc000000,  R16,     S016,    R21LS,   Void,     MIPS32)
INSTR (LD,       0xdc000000,  0xfc000000,  R16,     S016,    R21LS,   Void,     MIPS64)
INSTR (LDC1,     0xd4000000,  0xfc000000,  F16,     S016,    R21LS,   Void,     MIPS32)
INSTR (LDC2,     0xd8000000,  0xfc000000,  R16,     S016,    R21LS,   Void,     MIPS32)
INSTR (LDL,      0x68000000,  0xfc000000,  R16,     S016,    R21LS,   Void,     MIPS64)
INSTR (LDR,      0x6c000000,  0xfc000000,  R16,     S016,    R21LS,   Void,     MIPS64)
INSTR (LDXC1,    0x4c000001,  0xfc00f83f,  F6,      R16,     R21LS,   Void,     MIPS32)
INSTR (LH,       0x84000000,  0xfc000000,  R16,     S016,    R21LS,   Void,     MIPS32)
INSTR (LHU,      0x94000000,  0xfc000000,  R16,     S016,    R21LS,   Void,     MIPS32)
INSTR (LL,       0xc0000000,  0xfc000000,  R16,     S016,    R21LS,   Void,     MIPS32)
INSTR (LLD,      0xd0000000,  0xfc000000,  R16,     S016,    R21LS,   Void,     MIPS64)
INSTR (LUI,      0x3c000000,  0xffe00000,  R16,     S016,    Void,    Void,     MIPS32)
INSTR (LUXC1,    0x4c000005,  0xfc00f83f,  F6,      R16,     R21LS,   Void,     MIPS32)
INSTR (LW,       0x8c000000,  0xfc000000,  R16,     S016,    R21LS,   Void,     MIPS32)
INSTR (LWC1,     0xc4000000,  0xfc000000,  F16,     S016,    R21LS,   Void,     MIPS32)
INSTR (LWC2,     0xc8000000,  0xfc000000,  R16,     S016,    R21LS,   Void,     MIPS32)
INSTR (LWL,      0x88000000,  0xfc000000,  R16,     S016,    R21LS,   Void,     MIPS32)
INSTR (LWR,      0x98000000,  0xfc000000,  R16,     S016,    R21LS,   Void,     MIPS32)
INSTR (LWU,      0x9c000000,  0xfc000000,  R16,     S016,    R21LS,   Void,     MIPS64)
INSTR (LWXC1,    0x4c000000,  0xfc00f83f,  F6,      R16,     R21LS,   Void,     MIPS32)
INSTR (MADD,     0x70000000,  0xfc00ffff,  R21,     R16,     Void,    Void,     MIPS32)
INSTR (MADDS,    0x4c000020,  0xfc00003f,  F6,      F21,     F11,     F16,      MIPS32)
INSTR (MADDD,    0x4c000021,  0xfc00003f,  F6,      F21,     F11,     F16,      MIPS32)
INSTR (MADDPS,   0x4c000026,  0xfc00003f,  F6,      F21,     F11,     F16,      MIPS32)
INSTR (MADDU,    0x70000001,  0xfc00ffff,  R21,     R16,     Void,    Void,     MIPS32)
INSTR (MFC0,     0x40000000,  0xffe007f8,  R16,     R11,     U03,     Void,     MIPS32)
INSTR (MFC1,     0x44000000,  0xffe007ff,  R16,     F11,     Void,    Void,     MIPS32)
INSTR (MFC2,     0x48000000,  0xffe007f8,  R16,     R11,     U03,     Void,     MIPS32)
INSTR (MFHC1,    0x44600000,  0xffe007ff,  R16,     F11,     Void,    Void,     MIPS32)
INSTR (MFHC2,    0x48600000,  0xffe007f8,  R16,     R11,     U03,     Void,     MIPS32)
INSTR (MFHI,     0x00000010,  0xffff07ff,  R11,     Void,    Void,    Void,     MIPS32)
INSTR (MFLO,     0x00000012,  0xffff07ff,  R11,     Void,    Void,    Void,     MIPS32)
INSTR (MOVS,     0x46000006,  0xffff003f,  F6,      F11,     Void,    Void,     MIPS32)
INSTR (MOVD,     0x46200006,  0xffff003f,  F6,      F11,     Void,    Void,     MIPS32)
INSTR (MOVPS,    0x46c00006,  0xffff003f,  F6,      F11,     Void,    Void,     MIPS32)
INSTR (MOVF,     0x00000001,  0xfc0307ff,  R11,     R21,     U183,    Void,     MIPS32)
INSTR (MOVFS,    0x46000011,  0xffe3003f,  F6,      F11,     U183,    Void,     MIPS32)
INSTR (MOVFD,    0x46200011,  0xffe3003f,  F6,      F11,     U183,    Void,     MIPS32)
INSTR (MOVFPS,   0x46c00011,  0xffe3003f,  F6,      F11,     U183,    Void,     MIPS32)
INSTR (MOVN,     0x0000000b,  0xfc0007ff,  R11,     R21,     R16,     Void,     MIPS32)
INSTR (MOVNS,    0x46000013,  0xffe0003f,  F6,      F11,     R16,     Void,     MIPS32)
INSTR (MOVND,    0x46200013,  0xffe0003f,  F6,      F11,     R16,     Void,     MIPS32)
INSTR (MOVNPS,   0x46c00013,  0xffe0003f,  F6,      F11,     R16,     Void,     MIPS32)
INSTR (MOVT,     0x00010001,  0xfc0307ff,  R11,     R21,     U183,    Void,     MIPS32)
INSTR (MOVTS,    0x46010011,  0xffe3003f,  F6,      F11,     U183,    Void,     MIPS32)
INSTR (MOVTD,    0x46210011,  0xffe3003f,  F6,      F11,     U183,    Void,     MIPS32)
INSTR (MOVTPS,   0x46c10011,  0xffe3003f,  F6,      F11,     U183,    Void,     MIPS32)
INSTR (MOVZ,     0x0000000a,  0xfc0007ff,  R11,     R21,     R16,     Void,     MIPS32)
INSTR (MOVZS,    0x46000012,  0xffe0003f,  F6,      F11,     R16,     Void,     MIPS32)
INSTR (MOVZD,    0x46200012,  0xffe0003f,  F6,      F11,     R16,     Void,     MIPS32)
INSTR (MOVZPS,   0x46c00012,  0xffe0003f,  F6,      F11,     R16,     Void,     MIPS32)
INSTR (MSUB,     0x70000004,  0xfc00ffff,  R21,     R16,     Void,    Void,     MIPS32)
INSTR (MSUBS,    0x4c000028,  0xfc00003f,  F6,      F21,     F11,     F16,      MIPS32)
INSTR (MSUBD,    0x4c000029,  0xfc00003f,  F6,      F21,     F11,     F16,      MIPS32)
INSTR (MSUBPS,   0x4c00002e,  0xfc00003f,  F6,      F21,     F11,     F16,      MIPS32)
INSTR (MSUBU,    0x70000005,  0xfc00ffff,  R21,     R16,     Void,    Void,     MIPS32)
INSTR (MTC0,     0x40800000,  0xffe007f8,  R16,     R11,     U03,     Void,     MIPS32)
INSTR (MTC1,     0x44800000,  0xffe007ff,  R16,     F11,     Void,    Void,     MIPS32)
INSTR (MTC2,     0x48800000,  0xffe007f8,  R16,     R11,     U03,     Void,     MIPS32)
INSTR (MTHC1,    0x44e00000,  0xffe007ff,  R16,     F11,     Void,    Void,     MIPS32)
INSTR (MTHC2,    0x48e00000,  0xffe007f8,  R16,     R11,     U03,     Void,     MIPS32)
INSTR (MTHI,     0x00000011,  0xffff07ff,  R11,     Void,    Void,    Void,     MIPS32)
INSTR (MTLO,     0x00000013,  0xffff07ff,  R11,     Void,    Void,    Void,     MIPS32)
INSTR (MUL,      0x70000002,  0xfc0007ff,  R11,     R21,     R16,     Void,     MIPS32)
INSTR (MULS,     0x46000002,  0xffe0003f,  F6,      F11,     F16,     Void,     MIPS32)
INSTR (MULD,     0x46200002,  0xffe0003f,  F6,      F11,     F16,     Void,     MIPS32)
INSTR (MULPS,    0x46c00002,  0xffe0003f,  F6,      F11,     F16,     Void,     MIPS32)
INSTR (MULT,     0x00000018,  0xfc00ffff,  R21,     R16,     Void,    Void,     MIPS32)
INSTR (MULTU,    0x00000019,  0xfc00ffff,  R21,     R16,     Void,    Void,     MIPS32)
INSTR (NEGS,     0x46000007,  0xffff003f,  F6,      F11,     Void,    Void,     MIPS32)
INSTR (NEGD,     0x46200007,  0xffff003f,  F6,      F11,     Void,    Void,     MIPS32)
INSTR (NEGPS,    0x46c00007,  0xffff003f,  F6,      F11,     Void,    Void,     MIPS32)
INSTR (NMADDS,   0x4c000030,  0xfc00003f,  F6,      F21,     F11,     F16,      MIPS32)
INSTR (NMADDD,   0x4c000031,  0xfc00003f,  F6,      F21,     F11,     F16,      MIPS32)
INSTR (NMADDPS,  0x4c000036,  0xfc00003f,  F6,      F21,     F11,     F16,      MIPS32)
INSTR (NMSUBS,   0x4c000038,  0xfc00003f,  F6,      F21,     F11,     F16,      MIPS32)
INSTR (NMSUBD,   0x4c000039,  0xfc00003f,  F6,      F21,     F11,     F16,      MIPS32)
INSTR (NMSUBPS,  0x4c00003e,  0xfc00003f,  F6,      F21,     F11,     F16,      MIPS32)
INSTR (NOR,      0x00000027,  0xfc0007ff,  R11,     R21,     R16,     Void,     MIPS32)
INSTR (OR,       0x00000025,  0xfc0007ff,  R11,     R21,     R16,     Void,     MIPS32)
INSTR (ORI,      0x34000000,  0xfc000000,  R16,     R21,     U016,    Void,     MIPS32)
INSTR (PLLPS,    0x46c0002c,  0xffe0003f,  F6,      F11,     F16,     Void,     MIPS32)
INSTR (PLUPS,    0x46c0002d,  0xffe0003f,  F6,      F11,     F16,     Void,     MIPS32)
INSTR (PREF,     0xcc000000,  0xfc000000,  U165,    S016,    R21LS,   Void,     MIPS32)
INSTR (PREFX,    0x4c00000f,  0xfc0007ff,  U115,    R16,     R21LS,   Void,     MIPS32)
INSTR (PULPS,    0x46c0002e,  0xffe0003f,  F6,      F11,     F16,     Void,     MIPS32)
INSTR (PUUPS,    0x46c0002f,  0xffe0003f,  F6,      F11,     F16,     Void,     MIPS32)
INSTR (RDHDWR,   0x7c00003b,  0xffe007ff,  R16,     R11,     Void,    Void,     MIPS32)
INSTR (RDPGPR,   0x41400000,  0xffe007ff,  R11,     R16,     Void,    Void,     MIPS32)
INSTR (RECIPS,   0x46000015,  0xffff003f,  F6,      F11,     Void,    Void,     MIPS32)
INSTR (RECIPD,   0x46200015,  0xffff003f,  F6,      F11,     Void,    Void,     MIPS32)
INSTR (ROTR,     0x00200002,  0xffe0003f,  R11,     R16,     U65,     Void,     MIPS32)
INSTR (ROTRV,    0x00000046,  0xfc00007f,  R11,     R16,     R21,     Void,     MIPS32)
INSTR (ROUNDLS,  0x46000008,  0xffff003f,  F6,      F11,     Void,    Void,     MIPS32)
INSTR (ROUNDLD,  0x46200008,  0xffff003f,  F6,      F11,     Void,    Void,     MIPS32)
INSTR (ROUNDWS,  0x4600000c,  0xffff003f,  F6,      F11,     Void,    Void,     MIPS32)
INSTR (ROUNDWD,  0x4620000c,  0xffff003f,  F6,      F11,     Void,    Void,     MIPS32)
INSTR (RSQRTS,   0x46000016,  0xffff003f,  F6,      F11,     Void,    Void,     MIPS32)
INSTR (RSQRTD,   0x46200016,  0xffff003f,  F6,      F11,     Void,    Void,     MIPS32)
INSTR (SB,       0xa0000000,  0xfc000000,  R16,     S016,    R21LS,   Void,     MIPS32)
INSTR (SC,       0xe0000000,  0xfc000000,  R16,     S016,    R21LS,   Void,     MIPS32)
INSTR (SCD,      0xf0000000,  0xfc000000,  R16,     S016,    R21LS,   Void,     MIPS64)
INSTR (SD,       0xfc000000,  0xfc000000,  R16,     S016,    R21LS,   Void,     MIPS64)
INSTR (SDC1,     0xf4000000,  0xfc000000,  F16,     S016,    R21LS,   Void,     MIPS32)
INSTR (SDC2,     0xf8000000,  0xfc000000,  R16,     S016,    R21LS,   Void,     MIPS32)
INSTR (SDL,      0xb0000000,  0xfc000000,  R16,     S016,    R21LS,   Void,     MIPS64)
INSTR (SDR,      0xb4000000,  0xfc000000,  R16,     S016,    R21LS,   Void,     MIPS64)
INSTR (SDXC1,    0x4c000009,  0xfc0007ff,  F11,     R16,     R21LS,   Void,     MIPS32)
INSTR (SEB,      0x7c000420,  0xffe007ff,  R11,     R16,     Void,    Void,     MIPS32)
INSTR (SEH,      0x7c000620,  0xffe007ff,  R11,     R16,     Void,    Void,     MIPS32)
INSTR (SH,       0xa4000000,  0xfc000000,  R16,     S016,    R21LS,   Void,     MIPS32)
INSTR (SLL,      0x00000000,  0xffe0003f,  R11,     R16,     U65,     Void,     MIPS32)
INSTR (SLLV,     0x00000004,  0xfc0007ff,  R11,     R16,     R21,     Void,     MIPS32)
INSTR (SLT,      0x0000002a,  0xfc0007ff,  R11,     R21,     R16,     Void,     MIPS32)
INSTR (SLTI,     0x28000000,  0xfc000000,  R16,     R21,     S016,    Void,     MIPS32)
INSTR (SLTIU,    0x2c000000,  0xfc000000,  R16,     R21,     S016,    Void,     MIPS32)
INSTR (SLTU,     0x0000002b,  0xfc0007ff,  R11,     R21,     R16,     Void,     MIPS32)
INSTR (SQRTS,    0x46000004,  0xffff003f,  F6,      F11,     Void,    Void,     MIPS32)
INSTR (SQRTD,    0x46200004,  0xffff003f,  F6,      F11,     Void,    Void,     MIPS32)
INSTR (SRA,      0x00000003,  0xffe0003f,  R11,     R16,     U65,     Void,     MIPS32)
INSTR (SRAV,     0x00000007,  0xfc0007ff,  R11,     R16,     R21,     Void,     MIPS32)
INSTR (SRL,      0x00000002,  0xffe0003f,  R11,     R16,     U65,     Void,     MIPS32)
INSTR (SRLV,     0x00000006,  0xfc0007ff,  R11,     R16,     R21,     Void,     MIPS32)
INSTR (SUB,      0x00000022,  0xfc0007ff,  R11,     R21,     R16,     Void,     MIPS32)
INSTR (SUBS,     0x46000001,  0xffe0003f,  F6,      F11,     F16,     Void,     MIPS32)
INSTR (SUBD,     0x46200001,  0xffe0003f,  F6,      F11,     F16,     Void,     MIPS32)
INSTR (SUBPS,    0x46c00001,  0xffe0003f,  F6,      F11,     F16,     Void,     MIPS32)
INSTR (SUBU,     0x00000023,  0xfc0007ff,  R11,     R21,     R16,     Void,     MIPS32)
INSTR (SUXC1,    0x4c00000d,  0xfc0007ff,  F11,     R16,     R21LS,   Void,     MIPS32)
INSTR (SW,       0xac000000,  0xfc000000,  R16,     S016,    R21LS,   Void,     MIPS32)
INSTR (SWC1,     0xe4000000,  0xfc000000,  F16,     S016,    R21LS,   Void,     MIPS32)
INSTR (SWC2,     0xe8000000,  0xfc000000,  R16,     S016,    R21LS,   Void,     MIPS32)
INSTR (SWL,      0xa8000000,  0xfc000000,  R16,     S016,    R21LS,   Void,     MIPS32)
INSTR (SWR,      0xb8000000,  0xfc000000,  R16,     S016,    R21LS,   Void,     MIPS32)
INSTR (SWXC1,    0x4c000008,  0xfc0007ff,  F11,     R16,     R21LS,   Void,     MIPS32)
INSTR (SYNC,     0x0000000f,  0xfffff83f,  U65,     Void,    Void,    Void,     MIPS32)
INSTR (TEQ,      0x00000034,  0xfc00ffff,  R21,     R16,     Void,    Void,     MIPS32)
INSTR (TEQI,     0x040c0000,  0xfc1f0000,  R21,     S016,    Void,    Void,     MIPS32)
INSTR (TGE,      0x00000030,  0xfc00ffff,  R21,     R16,     Void,    Void,     MIPS32)
INSTR (TGEI,     0x04040000,  0xfc1f0000,  R21,     S016,    Void,    Void,     MIPS32)
INSTR (TGEIU,    0x04090000,  0xfc1f0000,  R21,     S016,    Void,    Void,     MIPS32)
INSTR (TGEU,     0x00000031,  0xfc00ffff,  R21,     R16,     Void,    Void,     MIPS32)
INSTR (TLT,      0x00000032,  0xfc00ffff,  R21,     R16,     Void,    Void,     MIPS32)
INSTR (TLTI,     0x040a0000,  0xfc1f0000,  R21,     S016,    Void,    Void,     MIPS32)
INSTR (TLTIU,    0x040b0000,  0xfc1f0000,  R21,     S016,    Void,    Void,     MIPS32)
INSTR (TLTU,     0x00000033,  0xfc00ffff,  R21,     R16,     Void,    Void,     MIPS32)
INSTR (TNE,      0x00000036,  0xfc00ffff,  R21,     R16,     Void,    Void,     MIPS32)
INSTR (TNEI,     0x040e0000,  0xfc1f0000,  R21,     S016,    Void,    Void,     MIPS32)
INSTR (TRUNCLS,  0x46000009,  0xffff003f,  F6,      F11,     Void,    Void,     MIPS32)
INSTR (TRUNCLD,  0x46200009,  0xffff003f,  F6,      F11,     Void,    Void,     MIPS32)
INSTR (TRUNCWS,  0x4600000d,  0xffff003f,  F6,      F11,     Void,    Void,     MIPS32)
INSTR (TRUNCWD,  0x4620000d,  0xffff003f,  F6,      F11,     Void,    Void,     MIPS32)
INSTR (WRPGPR,   0x41c00000,  0xffe007ff,  R11,     R16,     Void,    Void,     MIPS32)
INSTR (WSBH,     0x7c0000a0,  0xffe007ff,  R11,     R16,     Void,    Void,     MIPS32)
INSTR (XOR,      0x00000026,  0xfc0007ff,  R11,     R21,     R16,     Void,     MIPS32)
INSTR (XORI,     0x38000000,  0xfc000000,  R16,     R21,     U016,    Void,     MIPS32)

// operand types

TYPE (R11)
TYPE (R16)
TYPE (R21)
TYPE (R21LS)
TYPE (F6)
TYPE (F11)
TYPE (F16)
TYPE (F21)
TYPE (S016)
TYPE (S016T4)
TYPE (U03)
TYPE (U016)
TYPE (U026T4)
TYPE (U65)
TYPE (U65M32)
TYPE (U83)
TYPE (U115)
TYPE (U115M1)
TYPE (U115M33)
TYPE (U165)
TYPE (U183)

#undef INSTR
#undef MNEM
#undef TYPE

// MMIX instruction set definitions
// Copyright (C) Florian Negele

// This file is part of the Eigen Compiler Suite.

// The ECS is free software: you can redistribute it and/or modify
// it under the terms of the GNU General Public License as published by
// the Free Software Foundation, either version 3 of the License, or
// (at your option) any later version.

// The ECS is distributed in the hope that it will be useful,
// but WITHOUT ANY WARRANTY; without even the implied warranty of
// MERCHANTABILITY or FITNESS FOR A PARTICULAR PURPOSE.  See the
// GNU General Public License for more details.

// You should have received a copy of the GNU General Public License
// along with the ECS.  If not, see <https://www.gnu.org/licenses/>.

#ifndef FORMAT
	#define FORMAT(format)
#endif

#ifndef INSTR
	#define INSTR(mnem, opcode, format)
#endif

#ifndef MNEM
	#define MNEM(name, mnem, ...)
#endif

#ifndef SREG
	#define SREG(reg, name, number)
#endif

// mnemonics

MNEM (16addu,   T16ADDU,   Times 16 and Add Unsigned)
MNEM (16addui,  T16ADDUI,  Times 16 and Add Unsigned Immediate)
MNEM (2addu,    T2ADDU,    Times 2 and Add Unsigned)
MNEM (2addui,   T2ADDUI,   Times 2 and Add Unsigned Immediate)
MNEM (4addu,    T4ADDU,    Times 4 and Add Unsigned)
MNEM (4addui,   T4ADDUI,   Times 4 and Add Unsigned Immediate)
MNEM (8addu,    T8ADDU,    Times 8 and Add Unsigned)
MNEM (8addui,   T8ADDUI,   Times 8 and Add Unsigned Immediate)
MNEM (add,      ADD,       Add)
MNEM (addi,     ADDI,      Add Immediate)
MNEM (addu,     ADDU,      Add Unsigned)
MNEM (addui,    ADDUI,     Add Unsigned Immediate)
MNEM (and,      AND,       Bitwise AND)
MNEM (andi,     ANDI,      Bitwise AND Immediate)
MNEM (andn,     ANDN,      Bitwise AND-Not)
MNEM (andnh,    ANDNH,     Bitwise AND-Not with High Wyde)
MNEM (andni,    ANDNI,     Bitwise AND-Not Immediate)
MNEM (andnl,    ANDNL,     Bitwise AND-Not with Low Wyde)
MNEM (andnmh,   ANDNMH,    Bitwise AND-Not with Medium High Wyde)
MNEM (andnml,   ANDNML,    Bitwise AND-Not with Medium Low Wyde)
MNEM (bdif,     BDIF,      Byte Difference)
MNEM (bdifi,    BDIFI,     Byte Difference Immediate)
MNEM (bev,      BEV,       Branch if Even)
MNEM (bevb,     BEVB,      Branch if Even Backward)
MNEM (bn,       BN,        Branch if Negative)
MNEM (bnb,      BNB,       Branch if Negative Backward)
MNEM (bnn,      BNN,       Branch if Nonnegative)
MNEM (bnnb,     BNNB,      Branch if Nonnegative Backward)
MNEM (bnp,      BNP,       Branch if Nonpositive)
MNEM (bnpb,     BNPB,      Branch if Nonpositive Backward)
MNEM (bnz,      BNZ,       Branch if Nonzero)
MNEM (bnzb,     BNZB,      Branch if Nonzero Backward)
MNEM (bod,      BOD,       Branch if Odd)
MNEM (bodb,     BODB,      Branch if Odd Backward)
MNEM (bp,       BP,        Branch if Positive)
MNEM (bpb,      BPB,       Branch if Positive Backward)
MNEM (bz,       BZ,        Branch if Zero)
MNEM (bzb,      BZB,       Branch if Zero Backward)
MNEM (cmp,      CMP,       Compare)
MNEM (cmpi,     CMPI,      Compare Immediate)
MNEM (cmpu,     CMPU,      Compare Unsigned)
MNEM (cmpui,    CMPUI,     Compare Unsigned Immediate)
MNEM (csev,     CSEV,      Conditionally Set if Even)
MNEM (csevi,    CSEVI,     Conditionally Set if Even Immediate)
MNEM (csn,      CSN,       Conditionally Set if Negative)
MNEM (csni,     CSNI,      Conditionally Set if Negative Immediate)
MNEM (csnn,     CSNN,      Conditionally Set if Nonnegative)
MNEM (csnni,    CSNNI,     Conditionally Set if Nonnegative Immediate)
MNEM (csnp,     CSNP,      Conditionally Set if Nonpositive)
MNEM (csnpi,    CSNPI,     Conditionally Set if Nonpositive Immediate)
MNEM (csnz,     CSNZ,      Conditionally Set if Nonzero)
MNEM (csnzi,    CSNZI,     Conditionally Set if Nonzero Immediate)
MNEM (csod,     CSOD,      Conditionally Set if Odd)
MNEM (csodi,    CSODI,     Conditionally Set if Odd Immediate)
MNEM (csp,      CSP,       Conditionally Set if Positive)
MNEM (cspi,     CSPI,      Conditionally Set if Positive Immediate)
MNEM (cswap,    CSWAP,     Compare and Swap Octabytes)
MNEM (cswapi,   CSWAPI,    Compare and Swap Octabytes Immediate)
MNEM (csz,      CSZ,       Conditionally Set if Zero)
MNEM (cszi,     CSZI,      Conditionally Set if Zero Immediate)
MNEM (div,      DIV,       Divide)
MNEM (divi,     DIVI,      Divide Immediate)
MNEM (divu,     DIVU,      Divide Unsigned)
MNEM (divui,    DIVUI,     Divide Unsigned Immediate)
MNEM (fadd,     FADD,      Floating Add)
MNEM (fcmp,     FCMP,      Floating Compare)
MNEM (fcmpe,    FCMPE,     Floating Compare (with Respect to Epsilon))
MNEM (fdiv,     FDIV,      Floating Divide)
MNEM (feql,     FEQL,      Floating equal to)
MNEM (feqle,    FEQLE,     Floating equivalent (with Respect to Epsilon))
MNEM (fint,     FINT,      Floating integer)
MNEM (fix,      FIX,       Convert Floating to Fixed)
MNEM (fixu,     FIXU,      Convert Floating to Fixed Unsigned)
MNEM (flot,     FLOT,      Convert Fixed to Floating)
MNEM (floti,    FLOTI,     Convert Fixed to Floating Immediate)
MNEM (flotu,    FLOTU,     Convert Fixed to Floating Unsigned)
MNEM (flotui,   FLOTUI,    Convert Fixed to Floating Unsigned Immediate)
MNEM (fmul,     FMUL,      Floating Multiply)
MNEM (frem,     FREM,      Floating Remainder)
MNEM (fsqrt,    FSQRT,     Floating Square Root)
MNEM (fsub,     FSUB,      Floating Subtract)
MNEM (fun,      FUN,       Floating Unordered)
MNEM (fune,     FUNE,      Floating Unordered (with Respect to Epsilon))
MNEM (get,      GET,       Get from Special Register)
MNEM (geta,     GETA,      Get Address)
MNEM (getab,    GETAB,     Get Address Backward)
MNEM (go,       GO,        Go to Location)
MNEM (goi,      GOI,       Go to Location Immediate)
MNEM (inch,     INCH,      Increase by High Wyde)
MNEM (incl,     INCL,      Increase by Low Wyde)
MNEM (incmh,    INCMH,     Increase by Medium High Wyde)
MNEM (incml,    INCML,     Increase by Medium Low Wyde)
MNEM (jmp,      JMP,       Jump)
MNEM (jmpb,     JMPB,      Jump Backward)
MNEM (ldb,      LDB,       Load Byte)
MNEM (ldbi,     LDBI,      Load Byte Immediate)
MNEM (ldbu,     LDBU,      Load Byte Unsigned)
MNEM (ldbui,    LDBUI,     Load Byte Unsigned Immediate)
MNEM (ldht,     LDHT,      Load High Tetra)
MNEM (ldhti,    LDHTI,     Load High Tetra Immediate)
MNEM (ldo,      LDO,       Load Octa)
MNEM (ldoi,     LDOI,      Load Octa Immediate)
MNEM (ldou,     LDOU,      Load Octa Unsigned)
MNEM (ldoui,    LDOUI,     Load Octa Unsigned Immediate)
MNEM (ldsf,     LDSF,      Load Short Float)
MNEM (ldsfi,    LDSFI,     Load Short Float Immediate)
MNEM (ldt,      LDT,       Load Tetra)
MNEM (ldti,     LDTI,      Load Tetra Immediate)
MNEM (ldtu,     LDTU,      Load Tetra Unsigned)
MNEM (ldtui,    LDTUI,     Load Tetra Unsigned Immediate)
MNEM (ldunc,    LDUNC,     Load Octa Uncached)
MNEM (ldunci,   LDUNCI,    Load Octa Uncached Immediate)
MNEM (ldvts,    LDVTS,     Load virtual Translation Status)
MNEM (ldvtsi,   LDVTSI,    Load virtual Translation Status Immediate)
MNEM (ldw,      LDW,       Load Wyde)
MNEM (ldwi,     LDWI,      Load Wyde Immediate)
MNEM (ldwu,     LDWU,      Load Wyde Unsigned)
MNEM (ldwui,    LDWUI,     Load Wyde Unsigned Immediate)
MNEM (mor,      MOR,       Multiple OR)
MNEM (mori,     MORI,      Multiple OR Immediate)
MNEM (mul,      MUL,       Multiply)
MNEM (muli,     MULI,      Multiply Immediate)
MNEM (mulu,     MULU,      Multiply Unsigned)
MNEM (mului,    MULUI,     Multiply Unsigned Immediate)
MNEM (mux,      MUX,       Bitwise Multiplex)
MNEM (muxi,     MUXI,      Bitwise Multiplex Immediate)
MNEM (mxor,     MXOR,      Multiple Exclusive-OR)
MNEM (mxori,    MXORI,     Multiple Exclusive-OR Immediate)
MNEM (nand,     NAND,      Bitwise Not-AND)
MNEM (nandi,    NANDI,     Bitwise Not-AND Immediate)
MNEM (neg,      NEG,       Negate)
MNEM (negi,     NEGI,      Negate Immediate)
MNEM (negu,     NEGU,      Negate Unsigned)
MNEM (negui,    NEGUI,     Negate Unsigned Immediate)
MNEM (nor,      NOR,       Bitwise Not-OR)
MNEM (nori,     NORI,      Bitwise Not-OR Immediate)
MNEM (nxor,     NXOR,      Bitwise not-Exclusive-OR)
MNEM (nxori,    NXORI,     Bitwise not-Exclusive-OR Immediate)
MNEM (odif,     ODIF,      Octa Difference)
MNEM (odifi,    ODIFI,     Octa Difference Immediate)
MNEM (or,       OR,        Bitwise OR)
MNEM (orh,      ORH,       Bitwise OR with High Wyde)
MNEM (ori,      ORI,       Bitwise OR Immediate)
MNEM (orl,      ORL,       Bitwise OR with Low Wyde)
MNEM (ormh,     ORMH,      Bitwise OR with Medium High Wyde)
MNEM (orml,     ORML,      Bitwise OR with Medium Low Wyde)
MNEM (orn,      ORN,       Bitwise OR-Not)
MNEM (orni,     ORNI,      Bitwise OR-Not Immediate)
MNEM (pbev,     PBEV,      Probable Branch if Even)
MNEM (pbevb,    PBEVB,     Probable Branch if Even Backward)
MNEM (pbn,      PBN,       Probable Branch if Negative)
MNEM (pbnb,     PBNB,      Probable Branch if Negative Backward)
MNEM (pbnn,     PBNN,      Probable Branch if Nonnegative)
MNEM (pbnnb,    PBNNB,     Probable Branch if Nonnegative Backward)
MNEM (pbnp,     PBNP,      Probable Branch if Nonpositive)
MNEM (pbnpb,    PBNPB,     Probable Branch if Nonpositive Backward)
MNEM (pbnz,     PBNZ,      Probable Branch if Nonzero)
MNEM (pbnzb,    PBNZB,     Probable Branch if Nonzero Backward)
MNEM (pbod,     PBOD,      Probable Branch if Odd)
MNEM (pbodb,    PBODB,     Probable Branch if Odd Backward)
MNEM (pbp,      PBP,       Probable Branch if Positive)
MNEM (pbpb,     PBPB,      Probable Branch if Positive Backward)
MNEM (pbz,      PBZ,       Probable Branch if Zero)
MNEM (pbzb,     PBZB,      Probable Branch if Zero Backward)
MNEM (pop,      POP,       Pop Registers and Return from Subroutine)
MNEM (prego,    PREGO,     Prefetch to Go)
MNEM (pregoi,   PREGOI,    Prefetch to Go Immediate)
MNEM (preld,    PRELD,     Preload Data)
MNEM (preldi,   PRELDI,    Preload Data Immediate)
MNEM (prest,    PREST,     Prestore Data)
MNEM (presti,   PRESTI,    Prestore Data Immediate)
MNEM (pushgo,   PUSHGO,    Push Registers and Go)
MNEM (pushgoi,  PUSHGOI,   Push Registers and Go Immediate)
MNEM (pushj,    PUSHJ,     Push Registers and Jump)
MNEM (pushjb,   PUSHJB,    Push Registers and Jump Backward)
MNEM (put,      PUT,       Put into Special Register)
MNEM (puti,     PUTI,      Put into Special Register Immediate)
MNEM (resume,   RESUME,    Resume after Interrupt)
MNEM (sadd,     SADD,      Sideways Add)
MNEM (saddi,    SADDI,     Sideways Add Immediate)
MNEM (save,     SAVE,      Save Process State)
MNEM (seth,     SETH,      Set to High Wyde)
MNEM (setl,     SETL,      Set to Low Wyde)
MNEM (setmh,    SETMH,     Set to Medium High Wyde)
MNEM (setml,    SETML,     Set to Medium Low Wyde)
MNEM (sflot,    SFLOT,     Convert Fixed to Short Float)
MNEM (sfloti,   SFLOTI,    Convert Fixed to Short Float Immediate)
MNEM (sflotu,   SFLOTU,    Convert Fixed to Short Float Unsigned)
MNEM (sflotui,  SFLOTUI,   Convert Fixed to Short Float Unsigned Immediate)
MNEM (sl,       SL,        Shift Left)
MNEM (sli,      SLI,       Shift Left Immediate)
MNEM (slu,      SLU,       Shift Left Unsigned)
MNEM (slui,     SLUI,      Shift Left Unsigned Immediate)
MNEM (sr,       SR,        Shift Right)
MNEM (sri,      SRI,       Shift Right Immediate)
MNEM (sru,      SRU,       Shift Right Unsigned)
MNEM (srui,     SRUI,      Shift Right Unsigned Immediate)
MNEM (stb,      STB,       Store Byte)
MNEM (stbi,     STBI,      Store Byte Immediate)
MNEM (stbu,     STBU,      Store Byte Unsigned)
MNEM (stbui,    STBUI,     Store Byte Unsigned Immediate)
MNEM (stco,     STCO,      Store Constant Octa)
MNEM (stcoi,    STCOI,     Store Constant Octa Immediate)
MNEM (stht,     STHT,      Store High Tetra)
MNEM (sthti,    STHTI,     Store High Tetra Immediate)
MNEM (sto,      STO,       Store Octa)
MNEM (stoi,     STOI,      Store Octa Immediate)
MNEM (stou,     STOU,      Store Octa Unsigned)
MNEM (stoui,    STOUI,     Store Octa Unsigned Immediate)
MNEM (stsf,     STSF,      Store Short Float)
MNEM (stsfi,    STSFI,     Store Short Float Immediate)
MNEM (stt,      STT,       Store Tetra)
MNEM (stti,     STTI,      Store Tetra Immediate)
MNEM (sttu,     STTU,      Store Tetra Unsigned)
MNEM (sttui,    STTUI,     Store Tetra Unsigned Immediate)
MNEM (stunc,    STUNC,     Store Octa Uncached)
MNEM (stunci,   STUNCI,    Store Octa Uncached Immediate)
MNEM (stw,      STW,       Store Wyde)
MNEM (stwi,     STWI,      Store Wyde Immediate)
MNEM (stwu,     STWU,      Store Wyde Unsigned)
MNEM (stwui,    STWUI,     Store Wyde Unsigned Immediate)
MNEM (sub,      SUB,       Subtract)
MNEM (subi,     SUBI,      Subtract Immediate)
MNEM (subu,     SUBU,      Subtract Unsigned)
MNEM (subui,    SUBUI,     Subtract Unsigned Immediate)
MNEM (swym,     SWYM,      Sympathize with Your Machinery)
MNEM (sync,     SYNC,      Synchronize)
MNEM (syncd,    SYNCD,     Synchronize Data)
MNEM (syncdi,   SYNCDI,    Synchronize Data Immediate)
MNEM (syncid,   SYNCID,    Synchronize Instructions and Data)
MNEM (syncidi,  SYNCIDI,   Synchronize Instructions and Data Immediate)
MNEM (tdif,     TDIF,      Tetra Difference)
MNEM (tdifi,    TDIFI,     Tetra Difference Immediate)
MNEM (trap,     TRAP,      Trap)
MNEM (trip,     TRIP,      Trip)
MNEM (unsave,   UNSAVE,    Restore Process State)
MNEM (wdif,     WDIF,      Wyde Difference)
MNEM (wdifi,    WDIFI,     Wyde Difference Immediate)
MNEM (xor,      XOR,       Bitwise Exclusive-OR)
MNEM (xori,     XORI,      Bitwise Exclusive-OR Immediate)
MNEM (zsev,     ZSEV,      Zero or Set if Even)
MNEM (zsevi,    ZSEVI,     Zero or Set if Even Immediate)
MNEM (zsn,      ZSN,       Zero or Set if Negative)
MNEM (zsni,     ZSNI,      Zero or Set if Negative Immediate)
MNEM (zsnn,     ZSNN,      Zero or Set if Nonnegative)
MNEM (zsnni,    ZSNNI,     Zero or Set if Nonnegative Immediate)
MNEM (zsnp,     ZSNP,      Zero or Set if Nonpositive)
MNEM (zsnpi,    ZSNPI,     Zero or Set if Nonpositive Immediate)
MNEM (zsnz,     ZSNZ,      Zero or Set if Nonzero)
MNEM (zsnzi,    ZSNZI,     Zero or Set if Nonzero Immediate)
MNEM (zsod,     ZSOD,      Zero or Set if Odd)
MNEM (zsodi,    ZSODI,     Zero or Set if Odd Immediate)
MNEM (zsp,      ZSP,       Zero or Set if Positive)
MNEM (zspi,     ZSPI,      Zero or Set if Positive Immediate)
MNEM (zsz,      ZSZ,       Zero or Set if Zero)
MNEM (zszi,     ZSZI,      Zero or Set if Zero Immediate)

// instructions

INSTR (LDB,       0x80,  RXRYRIZ)
INSTR (LDBI,      0x81,  RXRYIZ)
INSTR (LDBU,      0x82,  RXRYRIZ)
INSTR (LDBUI,     0x83,  RXRYIZ)
INSTR (LDW,       0x84,  RXRYRIZ)
INSTR (LDWI,      0x85,  RXRYIZ)
INSTR (LDWU,      0x86,  RXRYRIZ)
INSTR (LDWUI,     0x87,  RXRYIZ)
INSTR (LDT,       0x88,  RXRYRIZ)
INSTR (LDTI,      0x89,  RXRYIZ)
INSTR (LDTU,      0x8a,  RXRYRIZ)
INSTR (LDTUI,     0x8b,  RXRYIZ)
INSTR (LDO,       0x8c,  RXRYRIZ)
INSTR (LDOI,      0x8d,  RXRYIZ)
INSTR (LDOU,      0x8e,  RXRYRIZ)
INSTR (LDOUI,     0x8f,  RXRYIZ)
INSTR (LDHT,      0x92,  RXRYRIZ)
INSTR (LDHTI,     0x93,  RXRYIZ)

INSTR (STB,       0xa0,  RXRYRIZ)
INSTR (STBI,      0xa1,  RXRYIZ)
INSTR (STBU,      0xa2,  RXRYRIZ)
INSTR (STBUI,     0xa3,  RXRYIZ)
INSTR (STW,       0xa4,  RXRYRIZ)
INSTR (STWI,      0xa5,  RXRYIZ)
INSTR (STWU,      0xa6,  RXRYRIZ)
INSTR (STWUI,     0xa7,  RXRYIZ)
INSTR (STT,       0xa8,  RXRYRIZ)
INSTR (STTI,      0xa9,  RXRYIZ)
INSTR (STTU,      0xaa,  RXRYRIZ)
INSTR (STTUI,     0xab,  RXRYIZ)
INSTR (STO,       0xac,  RXRYRIZ)
INSTR (STOI,      0xad,  RXRYIZ)
INSTR (STOU,      0xae,  RXRYRIZ)
INSTR (STOUI,     0xaf,  RXRYIZ)
INSTR (STHT,      0xb2,  RXRYRIZ)
INSTR (STHTI,     0xb3,  RXRYIZ)
INSTR (STCO,      0xb4,  IXRYRIZ)
INSTR (STCOI,     0xb5,  IXRYIZ)

INSTR (ADD,       0x20,  RXRYRIZ)
INSTR (ADDI,      0x21,  RXRYIZ)
INSTR (ADDU,      0x22,  RXRYRIZ)
INSTR (ADDUI,     0x23,  RXRYIZ)
INSTR (T2ADDU,    0x28,  RXRYRIZ)
INSTR (T2ADDUI,   0x29,  RXRYIZ)
INSTR (T4ADDU,    0x2a,  RXRYRIZ)
INSTR (T4ADDUI,   0x2b,  RXRYIZ)
INSTR (T8ADDU,    0x2c,  RXRYRIZ)
INSTR (T8ADDUI,   0x2d,  RXRYIZ)
INSTR (T16ADDU,   0x2e,  RXRYRIZ)
INSTR (T16ADDUI,  0x2f,  RXRYIZ)
INSTR (SUB,       0x24,  RXRYRIZ)
INSTR (SUBI,      0x25,  RXRYIZ)
INSTR (SUBU,      0x26,  RXRYRIZ)
INSTR (SUBUI,     0x27,  RXRYIZ)
INSTR (NEG,       0x34,  RXIYRIZ)
INSTR (NEGI,      0x35,  RXIYIZ)
INSTR (NEGU,      0x36,  RXIYRIZ)
INSTR (NEGUI,     0x37,  RXIYIZ)

INSTR (AND,       0xc8,  RXRYRIZ)
INSTR (ANDI,      0xc9,  RXRYIZ)
INSTR (OR,        0xc0,  RXRYRIZ)
INSTR (ORI,       0xc1,  RXRYIZ)
INSTR (XOR,       0xc6,  RXRYRIZ)
INSTR (XORI,      0xc7,  RXRYIZ)
INSTR (ANDN,      0xca,  RXRYRIZ)
INSTR (ANDNI,     0xcb,  RXRYIZ)
INSTR (ORN,       0xc2,  RXRYRIZ)
INSTR (ORNI,      0xc3,  RXRYIZ)
INSTR (NAND,      0xcc,  RXRYRIZ)
INSTR (NANDI,     0xcd,  RXRYIZ)
INSTR (NOR,       0xc4,  RXRYRIZ)
INSTR (NORI,      0xc5,  RXRYIZ)
INSTR (NXOR,      0xce,  RXRYRIZ)
INSTR (NXORI,     0xcf,  RXRYIZ)
INSTR (MUX,       0xd8,  RXRYRIZ)
INSTR (MUXI,      0xd9,  RXRYIZ)

INSTR (BDIF,      0xd0,  RXRYRIZ)
INSTR (BDIFI,     0xd1,  RXRYIZ)
INSTR (WDIF,      0xd2,  RXRYRIZ)
INSTR (WDIFI,     0xd3,  RXRYIZ)
INSTR (TDIF,      0xd4,  RXRYRIZ)
INSTR (TDIFI,     0xd5,  RXRYIZ)
INSTR (ODIF,      0xd6,  RXRYRIZ)
INSTR (ODIFI,     0xd7,  RXRYIZ)

INSTR (SADD,      0xda,  RXRYRIZ)
INSTR (SADDI,     0xdb,  RXRYIZ)
INSTR (MOR,       0xdc,  RXRYRIZ)
INSTR (MORI,      0xdd,  RXRYIZ)
INSTR (MXOR,      0xde,  RXRYRIZ)
INSTR (MXORI,     0xdf,  RXRYIZ)

INSTR (SETH,      0xe0,  RXIYZ)
INSTR (SETMH,     0xe1,  RXIYZ)
INSTR (SETML,     0xe2,  RXIYZ)
INSTR (SETL,      0xe3,  RXIYZ)
INSTR (INCH,      0xe4,  RXIYZ)
INSTR (INCMH,     0xe5,  RXIYZ)
INSTR (INCML,     0xe6,  RXIYZ)
INSTR (INCL,      0xe7,  RXIYZ)
INSTR (ORH,       0xe8,  RXIYZ)
INSTR (ORMH,      0xe9,  RXIYZ)
INSTR (ORML,      0xea,  RXIYZ)
INSTR (ORL,       0xeb,  RXIYZ)
INSTR (ANDNH,     0xec,  RXIYZ)
INSTR (ANDNMH,    0xed,  RXIYZ)
INSTR (ANDNML,    0xee,  RXIYZ)
INSTR (ANDNL,     0xef,  RXIYZ)

INSTR (SL,        0x38,  RXRYRIZ)
INSTR (SLI,       0x39,  RXRYIZ)
INSTR (SLU,       0x3a,  RXRYRIZ)
INSTR (SLUI,      0x3b,  RXRYIZ)
INSTR (SR,        0x3c,  RXRYRIZ)
INSTR (SRI,       0x3d,  RXRYIZ)
INSTR (SRU,       0x3e,  RXRYRIZ)
INSTR (SRUI,      0x3f,  RXRYIZ)

INSTR (CMP,       0x30,  RXRYRIZ)
INSTR (CMPI,      0x31,  RXRYIZ)
INSTR (CMPU,      0x32,  RXRYRIZ)
INSTR (CMPUI,     0x33,  RXRYIZ)

INSTR (CSN,       0x60,  RXRYRIZ)
INSTR (CSNI,      0x61,  RXRYIZ)
INSTR (CSZ,       0x62,  RXRYRIZ)
INSTR (CSZI,      0x63,  RXRYIZ)
INSTR (CSP,       0x64,  RXRYRIZ)
INSTR (CSPI,      0x65,  RXRYIZ)
INSTR (CSOD,      0x66,  RXRYRIZ)
INSTR (CSODI,     0x67,  RXRYIZ)
INSTR (CSNN,      0x68,  RXRYRIZ)
INSTR (CSNNI,     0x69,  RXRYIZ)
INSTR (CSNZ,      0x6a,  RXRYRIZ)
INSTR (CSNZI,     0x6b,  RXRYIZ)
INSTR (CSNP,      0x6c,  RXRYRIZ)
INSTR (CSNPI,     0x6d,  RXRYIZ)
INSTR (CSEV,      0x6e,  RXRYRIZ)
INSTR (CSEVI,     0x6f,  RXRYIZ)
INSTR (ZSN,       0x70,  RXRYRIZ)
INSTR (ZSNI,      0x71,  RXRYIZ)
INSTR (ZSZ,       0x72,  RXRYRIZ)
INSTR (ZSZI,      0x73,  RXRYIZ)
INSTR (ZSP,       0x74,  RXRYRIZ)
INSTR (ZSPI,      0x75,  RXRYIZ)
INSTR (ZSOD,      0x76,  RXRYRIZ)
INSTR (ZSODI,     0x77,  RXRYIZ)
INSTR (ZSNN,      0x78,  RXRYRIZ)
INSTR (ZSNNI,     0x79,  RXRYIZ)
INSTR (ZSNZ,      0x7a,  RXRYRIZ)
INSTR (ZSNZI,     0x7b,  RXRYIZ)
INSTR (ZSNP,      0x7c,  RXRYRIZ)
INSTR (ZSNPI,     0x7d,  RXRYIZ)
INSTR (ZSEV,      0x7e,  RXRYRIZ)
INSTR (ZSEVI,     0x7f,  RXRYIZ)

INSTR (BN,        0x40,  RXDYZ)
INSTR (BNB,       0x41,  RXDYZ)
INSTR (BZ,        0x42,  RXDYZ)
INSTR (BZB,       0x43,  RXDYZ)
INSTR (BP,        0x44,  RXDYZ)
INSTR (BPB,       0x45,  RXDYZ)
INSTR (BOD,       0x46,  RXDYZ)
INSTR (BODB,      0x47,  RXDYZ)
INSTR (BNN,       0x48,  RXDYZ)
INSTR (BNNB,      0x49,  RXDYZ)
INSTR (BNZ,       0x4a,  RXDYZ)
INSTR (BNZB,      0x4b,  RXDYZ)
INSTR (BNP,       0x4c,  RXDYZ)
INSTR (BNPB,      0x4d,  RXDYZ)
INSTR (BEV,       0x4e,  RXDYZ)
INSTR (BEVB,      0x4f,  RXDYZ)
INSTR (PBN,       0x50,  RXDYZ)
INSTR (PBNB,      0x51,  RXDYZ)
INSTR (PBZ,       0x52,  RXDYZ)
INSTR (PBZB,      0x53,  RXDYZ)
INSTR (PBP,       0x54,  RXDYZ)
INSTR (PBPB,      0x55,  RXDYZ)
INSTR (PBOD,      0x56,  RXDYZ)
INSTR (PBODB,     0x57,  RXDYZ)
INSTR (PBNN,      0x58,  RXDYZ)
INSTR (PBNNB,     0x59,  RXDYZ)
INSTR (PBNZ,      0x5a,  RXDYZ)
INSTR (PBNZB,     0x5b,  RXDYZ)
INSTR (PBNP,      0x5c,  RXDYZ)
INSTR (PBNPB,     0x5d,  RXDYZ)
INSTR (PBEV,      0x5e,  RXDYZ)
INSTR (PBEVB,     0x5f,  RXDYZ)

INSTR (GETA,      0xf4,  RXDYZ)
INSTR (GETAB,     0xf5,  RXDYZ)

INSTR (JMP,       0xf0,  DXYZ)
INSTR (JMPB,      0xf1,  DXYZ)
INSTR (GO,        0x9e,  RXRYRIZ)
INSTR (GOI,       0x9f,  RXRYIZ)

INSTR (MUL,       0x18,  RXRYRIZ)
INSTR (MULI,      0x19,  RXRYIZ)
INSTR (MULU,      0x1a,  RXRYRIZ)
INSTR (MULUI,     0x1b,  RXRYIZ)
INSTR (DIV,       0x1c,  RXRYRIZ)
INSTR (DIVI,      0x1d,  RXRYIZ)
INSTR (DIVU,      0x1e,  RXRYRIZ)
INSTR (DIVUI,     0x1f,  RXRYIZ)

INSTR (FADD,      0x04,  RXRYRZ)
INSTR (FSUB,      0x06,  RXRYRZ)
INSTR (FMUL,      0x10,  RXRYRZ)
INSTR (FDIV,      0x14,  RXRYRZ)
INSTR (FREM,      0x16,  RXRYRZ)
INSTR (FSQRT,     0x15,  RXIYRZ)
INSTR (FINT,      0x17,  RXIYRZ)

INSTR (FCMP,      0x01,  RXRYRZ)
INSTR (FEQL,      0x03,  RXRYRZ)
INSTR (FUN,       0x02,  RXRYRZ)

INSTR (FCMPE,     0x11,  RXRYRZ)
INSTR (FEQLE,     0x13,  RXRYRZ)
INSTR (FUNE,      0x12,  RXRYRZ)

INSTR (LDSF,      0x90,  RXRYRIZ)
INSTR (LDSFI,     0x91,  RXRYIZ)
INSTR (STSF,      0xb0,  RXRYRIZ)
INSTR (STSFI,     0xb1,  RXRYIZ)

INSTR (FIX,       0x05,  RXIYRZ)
INSTR (FIXU,      0x07,  RXIYRZ)
INSTR (FLOT,      0x08,  RXIYRIZ)
INSTR (FLOTI,     0x09,  RXIYIZ)
INSTR (FLOTU,     0x0a,  RXIYRIZ)
INSTR (FLOTUI,    0x0b,  RXIYIZ)
INSTR (SFLOT,     0x0c,  RXIYRIZ)
INSTR (SFLOTI,    0x0d,  RXIYIZ)
INSTR (SFLOTU,    0x0e,  RXIYRIZ)
INSTR (SFLOTUI,   0x0f,  RXIYIZ)

INSTR (PUSHJ,     0xf2,  RXDYZ)
INSTR (PUSHJB,    0xf3,  RXDYZ)
INSTR (PUSHGO,    0xbe,  RXRYRIZ)
INSTR (PUSHGOI,   0xbf,  RXRYIZ)
INSTR (POP,       0xf8,  RXDYZ)

INSTR (LDUNC,     0x96,  RXRYRIZ)
INSTR (LDUNCI,    0x97,  RXRYIZ)
INSTR (STUNC,     0xb6,  RXRYRIZ)
INSTR (STUNCI,    0xb7,  RXRYIZ)
INSTR (PRELD,     0x9a,  RXRYRIZ)
INSTR (PRELDI,    0x9b,  RXRYIZ)
INSTR (PREGO,     0x9c,  RXRYRIZ)
INSTR (PREGOI,    0x9d,  RXRYIZ)
INSTR (PREST,     0xba,  RXRYRIZ)
INSTR (PRESTI,    0xbb,  RXRYIZ)
INSTR (SYNCD,     0xb8,  RXRYRIZ)
INSTR (SYNCDI,    0xb9,  RXRYIZ)
INSTR (SYNCID,    0xbc,  RXRYRIZ)
INSTR (SYNCIDI,   0xbd,  RXRYIZ)

INSTR (CSWAP,     0x94,  RXRYRIZ)
INSTR (CSWAPI,    0x95,  RXRYIZ)
INSTR (SYNC,      0xfc,  IXIYIZ)

INSTR (TRAP,      0x00,  IXIYIZ)
INSTR (TRIP,      0xff,  IXIYIZ)

INSTR (RESUME,    0xf9,  IZ)

INSTR (GET,       0xfe,  RXIZ)
INSTR (PUT,       0xf6,  IXRIZ)
INSTR (PUTI,      0xf7,  IXIZ)

INSTR (SAVE,      0xfa,  RX)
INSTR (UNSAVE,    0xfb,  RX)

INSTR (LDVTS,     0x98,  RXRYRIZ)
INSTR (LDVTSI,    0x99,  RXRYIZ)

INSTR (SWYM,      0xfd,  IXIYIZ)

// formats

FORMAT (RXRYRZ)
FORMAT (RXRYRIZ)
FORMAT (RXRYIZ)
FORMAT (RXIYRZ)
FORMAT (RXIYRIZ)
FORMAT (RXIYIZ)
FORMAT (RXIYZ)
FORMAT (RXIZ)
FORMAT (RXDYZ)
FORMAT (RX)
FORMAT (IXRYRIZ)
FORMAT (IXRYIZ)
FORMAT (IXIYIZ)
FORMAT (IXRIZ)
FORMAT (IXIZ)
FORMAT (IZ)
FORMAT (DXYZ)

// special registers

SREG (RA,   rA,   21)
SREG (RB,   rB,   0)
SREG (RC,   rC,   8)
SREG (RD,   rD,   1)
SREG (RE,   rE,   2)
SREG (RF,   rF,   22)
SREG (RG,   rG,   19)
SREG (RH,   rH,   3)
SREG (RI,   rI,   12)
SREG (RJ,   rJ,   4)
SREG (RK,   rK,   15)
SREG (RL,   rL,   20)
SREG (RM,   rM,   5)
SREG (RN,   rN,   9)
SREG (RO,   rO,   10)
SREG (RP,   rp,   23)
SREG (RQ,   rQ,   16)
SREG (RR,   rR,   6)
SREG (RT,   rT,   13)
SREG (RU,   rU,   17)
SREG (RV,   rV,   18)
SREG (RW,   rW,   24)
SREG (RX,   rX,   25)
SREG (RY,   rY,   26)
SREG (RZ,   rZ,   27)
SREG (RBB,  rBB,  7)
SREG (RTT,  rTT,  14)
SREG (RWW,  rWW,  28)
SREG (RXX,  rXX,  29)
SREG (RYY,  rYY,  30)
SREG (rZZ,  rZZ,  31)

#undef FORMAT
#undef INSTR
#undef MNEM
#undef SREG

% OpenRISC 1000 architecture documentation
% Copyright (C) Florian Negele

% This file is part of the Eigen Compiler Suite.

% Permission is granted to copy, distribute and/or modify this document
% under the terms of the GNU Free Documentation License, Version 1.3
% or any later version published by the Free Software Foundation.

% You should have received a copy of the GNU Free Documentation License
% along with the ECS.  If not, see <https://www.gnu.org/licenses/>.

% Generic documentation utilities
% Copyright (C) Florian Negele

% This file is part of the Eigen Compiler Suite.

% Permission is granted to copy, distribute and/or modify this document
% under the terms of the GNU Free Documentation License, Version 1.3
% or any later version published by the Free Software Foundation.

% You should have received a copy of the GNU Free Documentation License
% along with the ECS.  If not, see <https://www.gnu.org/licenses/>.

\providecommand{\cpp}{C\texttt{++}}
\providecommand{\opt}{_\mathit{opt}}
\providecommand{\tool}[1]{\texttt{#1}}
\providecommand{\version}{Version 0.0.40}
\providecommand{\resource}[1]{*++\txt{#1}}
\providecommand{\ecs}{Eigen Compiler Suite}
\providecommand{\changed}[1]{\underline{#1}}
\providecommand{\toolbox}[1]{\converter{#1}}
\providecommand{\file}{}\renewcommand{\file}[1]{\texttt{#1}}
\providecommand{\alignright}{\hfill\linebreak[0]\hspace*{\fill}}
\providecommand{\converter}[1]{*++[F][F*:white][F,:gray]\txt{#1}}
\providecommand{\documentation}{\ifbook chapter\else document\fi}
\providecommand{\Documentation}{\ifbook Chapter\else Document\fi}
\providecommand{\variable}[1]{\resource{\texttt{\small#1}\\variable}}
\providecommand{\documentationref}[2]{\ifbook\ref{#1}\else``\href{#1}{#2}''~\cite{#1}\fi}
\providecommand{\objfile}[1]{\texttt{#1}\index[runtime]{#1 object file@\texttt{#1} object file}}
\providecommand{\libfile}[1]{\texttt{#1}\index[runtime]{#1 library file@\texttt{#1} library file}}
\providecommand{\epigraph}[2]{\ifbook\begin{quote}\flushright\textit{#1}\par--- #2\end{quote}\fi}
\providecommand{\environmentvariable}[1]{\texttt{#1}\index{Environment variables!#1@\texttt{#1}}}
\providecommand{\environment}[1]{\texttt{#1}\index[environment]{#1 environment@\texttt{#1} environment}}
\providecommand{\toolsection}{}\renewcommand{\toolsection}[1]{\subsection{#1}\label{\prefix:#1}\tool{#1}}
\providecommand{\instruction}{}\renewcommand{\instruction}[2]{\noindent\qquad\pdftooltip{\texttt{#1}}{#2}\refstepcounter{instruction}\par}
\providecommand{\flowgraph}{}\renewcommand{\flowgraph}[1]{\par\sffamily\begin{displaymath}\xymatrix@=4ex{#1}\end{displaymath}\normalfont\par}
\providecommand{\instructionset}{}\renewcommand{\instructionset}[4]{\setcounter{instruction}{0}\begin{multicols}{\ifbook#3\else#4\fi}[{\captionof{table}[#2]{#2 (\ref*{#1:instructions}~instructions)}\label{tab:#1set}\vspace{-2ex}}]\footnotesize\raggedcolumns\input{#1.set}\label{#1:instructions}\end{multicols}}

\providecommand{\gpl}{GNU General Public License}
\providecommand{\rse}{ECS Runtime Support Exception}
\providecommand{\fdl}{\href{https://www.gnu.org/licenses/fdl.html}{GNU Free Documentation License}}

\providecommand{\docbegin}{}
\providecommand{\docend}{}
\providecommand{\doclabel}[1]{\hypertarget{#1}}
\providecommand{\doclink}[2]{\hyperlink{#1}{#2}}
\providecommand{\docsection}[3]{\hypertarget{#1}{\subsection{#2}}\label{sec:#1}\index[library]{#2@#3}}
\providecommand{\docsectionstar}[1]{}
\providecommand{\docsubbegin}{\begin{description}}
\providecommand{\docsubend}{\end{description}}
\providecommand{\docsubsection}[3]{\item[\hypertarget{#1}{#2}]\index[library]{#2@#3}}
\providecommand{\docsubsectionstar}[1]{\smallskip}
\providecommand{\docsubsubsection}[3]{\docsubsection{#1}{#2}{#3}}
\providecommand{\docsubsubsectionstar}[1]{}
\providecommand{\docsubsubsubsection}[3]{}
\providecommand{\docsubsubsubsectionstar}[1]{}
\providecommand{\doctable}{}

\providecommand{\debuggingtool}{}\renewcommand{\debuggingtool}{This tool is provided for debugging purposes.
It allows exposing and modifying an internal data structure that is usually not accessible.
}

\providecommand{\interface}{All tools accept command-line arguments which are taken as names of plain text files containing the source code.
If no arguments are provided, the standard input stream is used instead.
Output files are generated in the current working directory and have the same name as the input file being processed whereas the filename extension gets replaced by an appropriate suffix.
\seeinterface
}

\providecommand{\license}{\noindent Copyright \copyright{} Florian Negele\par\medskip\noindent
Permission is granted to copy, distribute and/or modify this document under the terms of the
\fdl{}, Version 1.3 or any later version published by the \href{https://fsf.org/}{Free Software Foundation}.
}

\providecommand{\ecslogosurface}{
\fill[darkgray] (0,0,0) -- (0,0,3) -- (0,3,3) -- (0,3,1) -- (0,4,1) -- (0,4,3) -- (0,5,3) -- (0,5,0) -- (0,2,0) -- (0,2,2) -- (0,1,2) -- (0,1,0) -- cycle;
\fill[gray] (0,5,0) -- (0,5,3) -- (1,5,3) -- (1,5,1) -- (2,5,1) -- (2,5,3) -- (3,5,3) -- (3,5,0) -- cycle;
\fill[lightgray] (0,0,0) -- (0,1,0) -- (2,1,0) -- (2,4,0) -- (1,4,0) -- (1,3,0) -- (2,3,0) -- (2,2,0) -- (0,2,0) -- (0,5,0) -- (3,5,0) -- (3,0,0) -- cycle;
\begin{scope}[line width=0.5]
\begin{scope}[gray]
\draw (0,0,0) -- (0,1,0);
\draw (2,1,0) -- (2,2,0);
\draw (0,1,2) -- (0,2,2);
\draw (0,2,0) -- (0,5,0);
\draw (2,3,0) -- (2,4,0);
\end{scope}
\begin{scope}[lightgray]
\draw (0,1,0) -- (0,1,2);
\draw (0,3,1) -- (0,3,3);
\draw (0,5,0) -- (0,5,3);
\draw (2,5,1) -- (2,5,3);
\end{scope}
\begin{scope}[white]
\draw (0,1,0) -- (2,1,0);
\draw (1,3,0) -- (2,3,0);
\draw (0,5,0) -- (3,5,0);
\end{scope}
\end{scope}
}

\providecommand{\ecslogo}[1]{
\begin{tikzpicture}[scale={(#1)/((sin(45)+cos(45))*3cm)},x={({-cos(45)*1cm},{sin(45)*sin(30)*1cm})},y={({0cm},{(cos(30)*1cm})},z={({sin(45)*1cm},{cos(45)*sin(30)*1cm})}]
\begin{scope}[darkgray,line width=1]
\draw (0,0,0) -- (0,0,3) -- (0,3,3) -- (2,3,3) -- (2,5,3) -- (3,5,3) -- (3,5,0) -- (3,0,0) -- cycle;
\draw (0,3,1) -- (0,4,1) -- (0,4,3) -- (0,5,3) -- (1,5,3) -- (1,5,1) -- (2,5,1);
\draw (1,3,0) -- (1,4,0) -- (2,4,0);
\end{scope}
\fill[darkgray] (2,0,0) -- (2,0,3) -- (2,5,3) -- (2,5,1) -- (2,4,1) -- (2,4,0) -- cycle;
\fill[lightgray] (2,0,2) -- (0,0,2) -- (0,2,2) -- (2,2,2) -- cycle;
\fill[gray] (0,1,0) -- (2,1,0) -- (2,1,2) -- (0,1,2) -- cycle;
\fill[gray] (0,3,1) -- (0,3,3) -- (2,3,3) -- (2,3,0) -- (1,3,0) -- (1,3,1) -- cycle;
\ecslogosurface
\end{tikzpicture}
}

\providecommand{\shadowedecslogo}[3]{
\begin{tikzpicture}[scale={(#1)/((sin(#2)+cos(#2))*3cm)},x={({-cos(#2)*1cm},{sin(#2)*sin(#3)*1cm})},y={({0cm},{(cos(#3)*1cm})},z={({sin(#2)*1cm},{cos(#2)*sin(#3)*1cm})}]
\shade[top color=lightgray!50!white,bottom color=white,middle color=lightgray!50!white] (0,0,0) -- (3,0,0) -- (3,{-0.5-3*sin(#2)*sin(#3)/cos(#3)},0) -- (0,-0.5,0) -- cycle;
\shade[top color=darkgray!50!gray,bottom color=white,middle color=darkgray!50!white] (0,0,0) -- (0,0,3) -- (0,{-0.5-3*cos(#2)*sin(#3)/cos(#3)},3) -- (0,-0.5,0) -- cycle;
\begin{scope}[y={({(cos(#2)+sin(#2))*0.5cm},{(cos(#2)*sin(#3)-sin(#2)*sin(#3))*0.5cm})}]
\useasboundingbox (3,0,0) -- (0,0,0) -- (0,0,3);
\shade[left color=darkgray!80!black,right color=lightgray,middle color=gray] (0,0,0) -- (0,1,0) -- (0,1,0.5) -- (0,2,0) -- (0,5,0) -- (0,5,3) -- (1,5,3) -- (1,4,3) -- (1,4,2.5) -- (1,3,3) -- (2,5,3) -- (3,5,3) -- (3,0,3) -- cycle;
\clip (0,0,0) -- (0,0,3) -- ({-3*sin(#2)/cos(#2)},0,0) -- cycle;
\shade[left color=darkgray,right color=lightgray!50!gray] (0,0,0) -- (0,1,0) -- (0,1,0.5) -- (0,2,0) -- (0,5,0) -- (0,5,3) -- (1,5,3) -- (1,4,3) -- (1,4,2.5) -- (1,3,3) -- (2,5,3) -- (3,5,3) -- (3,0,3) -- cycle;
\end{scope}
\shade[left color=darkgray,right color=darkgray!80!black] (2,0,0) -- (2,0,3) -- (2,5,3) -- (2,5,1) -- (2,4,1) -- (2,4,0) -- cycle;
\shade[left color=darkgray!90!black,right color=gray!80!darkgray] (2,0,2) -- (0,0,2) -- (0,2,2) -- (2,2,2) -- cycle;
\shade[top color=darkgray!90!black,bottom color=gray!80!darkgray] (0,1,0) -- (2,1,0) -- (2,1,2) -- (0,1,2) -- cycle;
\shade[top color=darkgray!90!black,bottom color=gray!80!darkgray] (0,3,1) -- (0,3,3) -- (2,3,3) -- (2,3,0) -- (1,3,0) -- (1,3,1) -- cycle;
\fill[gray] (2,1,0) -- (1.5,1,0.5) -- (0,1,0.5) -- (0,1,0) -- cycle;
\fill[gray] (1,3,2) -- (0.5,3,2) -- (0.5,3,3) -- (1,3,3) -- cycle;
\fill[gray] (2,3,0) -- (1.5,3,0.5) -- (1,3,0.5) -- (1,3,0) -- cycle;
\ecslogosurface
\end{tikzpicture}
}

\providecommand{\cpplogo}[1]{
\begin{tikzpicture}[scale=(#1)/512em]
\fill[gray] (435.2794,398.7159) -- (247.1911,507.3075) .. controls (236.3563,513.5642) and (218.6240,513.5642) .. (207.7892,507.3075) -- (19.7009,398.7159) .. controls (8.8646,392.4606) and (0.0000,377.1043) .. (0.0000,364.5924) -- (0.0000,147.4076) .. controls (0.8430,132.8363) and (8.2856,120.7683) .. (19.7009,113.2842) -- (207.7892,4.6926) .. controls (218.6240,-1.5642) and (236.3564,-1.5642) .. (247.1911,4.6926) -- (435.2794,113.2842) .. controls (447.5273,121.4304) and (454.4987,133.6918) .. (454.9803,147.4076) -- (454.9803,364.5924) .. controls (454.5404,377.7571) and (446.6566,391.0351) .. (435.2794,398.7159) -- cycle(75.8301,255.9993) .. controls (74.9389,404.0881) and (273.2892,469.4783) .. (358.8263,331.8769) -- (293.1917,293.8965) .. controls (253.5702,359.4301) and (155.1909,335.9977) .. (151.6601,255.9993) .. controls (152.7204,182.2703) and (249.4137,148.0211) .. (293.1961,218.1065) -- (358.8308,180.1276) .. controls (283.4477,49.2645) and (79.6318,96.3470) .. (75.8301,255.9993) -- cycle(379.1503,247.5747) -- (362.2982,247.5747) -- (362.2982,230.7226) -- (345.4490,230.7226) -- (345.4490,247.5747) -- (328.5969,247.5747) -- (328.5969,264.4254) -- (345.4490,264.4254) -- (345.4490,281.2759) -- (362.2982,281.2759) -- (362.2982,264.4254) -- (379.1503,264.4254) -- cycle(442.3420,247.5747) -- (425.4899,247.5747) -- (425.4899,230.7226) -- (408.6408,230.7226) -- (408.6408,247.5747) -- (391.7886,247.5747) -- (391.7886,264.4254) -- (408.6408,264.4254) -- (408.6408,281.2759) -- (425.4899,281.2759) -- (425.4899,264.4254) -- (442.3420,264.4254) -- cycle;
\end{tikzpicture}
}

\providecommand{\fallogo}[1]{
\begin{tikzpicture}[scale=(#1)/512em]
\fill[gray] (185.7774,0.0000) .. controls (200.4486,15.9798) and (226.8966,8.7148) .. (235.0426,31.5836) .. controls (249.5297,58.0598) and (247.9581,97.9161) .. (280.3335,110.9762) .. controls (309.1690,120.3496) and (337.8406,104.2727) .. (366.5753,103.9379) .. controls (373.4449,111.5171) and (379.2885,128.2574) .. (383.9755,108.9744) .. controls (396.6979,102.5615) and (437.2808,107.6681) .. (426.9652,124.3252) .. controls (408.9822,121.0785) and (412.4742,146.0729) .. (426.5192,131.4996) .. controls (433.8413,120.8489) and (465.1541,126.5522) .. (441.9067,135.7950) .. controls (396.1879,157.7478) and (344.1112,161.5079) .. (298.5528,183.5702) .. controls (277.7471,193.5198) and (284.6941,218.7163) .. (285.2127,236.9640) .. controls (292.3599,316.2826) and (307.3929,394.6311) .. (317.1198,473.6154) .. controls (329.0637,505.4736) and (292.1195,528.5004) .. (265.9183,511.2761) .. controls (237.9284,499.2462) and (237.3684,465.2681) .. (230.9102,439.9421) .. controls (218.6692,374.3397) and (215.6307,306.9662) .. (198.1732,242.3977) .. controls (183.1379,232.7444) and (164.4245,256.0298) .. (149.0430,261.4799) .. controls (116.9328,279.2585) and (87.1822,308.5851) .. (48.2293,307.8914) .. controls (21.3220,306.9037) and (-15.9107,281.8761) .. (7.2921,252.7908) .. controls (29.7799,220.6177) and (67.5177,204.3028) .. (100.9287,185.9449) .. controls (130.8217,170.8906) and (161.1548,156.5903) .. (191.0278,141.5847) .. controls (196.1738,120.0520) and (186.6049,95.2409) .. (186.8382,72.4353) .. controls (185.5234,48.4204) and (183.1700,23.9341) .. (185.7774,0.0000) -- cycle;
\end{tikzpicture}
}

\providecommand{\oblogo}[1]{
\begin{tikzpicture}[scale=(#1)/512em]
\fill[gray] (160.3865,208.9117) .. controls (154.0879,214.6478) and (149.0735,221.2409) .. (145.4125,228.5384) .. controls (184.8790,248.4273) and (234.7122,269.8787) .. (297.5493,291.8782) .. controls (300.3943,281.4769) and (300.9552,268.7619) .. (300.4023,255.2389) .. controls (248.9909,244.7891) and (200.0310,225.9279) .. (160.3865,208.9117) -- cycle(225.7398,392.6996) .. controls (308.0209,392.1716) and (359.3326,345.9277) .. (368.7203,285.2098) .. controls (376.6742,197.1784) and (311.7194,141.3342) .. (205.4287,142.1456) .. controls (139.9485,141.4804) and (88.7155,166.1957) .. (73.5775,228.0086) .. controls (52.0297,320.3408) and (123.4078,391.0103) .. (225.7398,392.6996) -- cycle(216.0739,176.4733) .. controls (268.9183,179.2424) and (315.8292,206.5488) .. (312.7454,265.1139) .. controls (313.2769,315.6384) and (286.5993,353.4946) .. (216.6040,355.7934) .. controls (162.4657,355.7934) and (126.0914,317.5023) .. (126.0914,260.5103) .. controls (126.1733,214.2900) and (163.3363,176.2849) .. (216.0739,176.4733) -- cycle(76.4897,189.1754) .. controls (13.1586,147.5631) and (0.0000,119.4207) .. (0.0000,119.4207) -- (90.6499,170.1632) .. controls (85.3004,175.8497) and (80.5994,182.1633) .. (76.4897,189.1754) -- cycle(353.9486,119.3004) -- (402.9482,119.3004) .. controls (427.0025,137.0797) and (450.9893,162.7034) .. (474.9529,191.0213) .. controls (509.3540,228.5339) and (531.3391,294.2091) .. (487.8149,312.1206) .. controls (462.8165,324.7652) and (394.3874,316.8943) .. (373.8912,313.6651) .. controls (379.9291,297.7449) and (383.2899,278.4204) .. (381.4989,257.7214) .. controls (420.3069,248.0321) and (421.9610,218.3461) .. (407.7867,192.6417) .. controls (391.1113,162.4018) and (370.1114,132.9097) .. (353.9486,119.3004) -- cycle;
\end{tikzpicture}
}

\providecommand{\markuptable}{
\begin{table}
\sffamily\centering
\begin{tabular}{@{}lcl@{}}
\toprule
\texttt{//italics//} & $\rightarrow$ & \textit{italics} \\
\midrule
\texttt{**bold**} & $\rightarrow$ & \textbf{bold} \\
\midrule
\texttt{\# ordered list} & & 1 ordered list \\
\texttt{\# second item} & $\rightarrow$ & 2 second item \\
\texttt{\#\# sub item} & & \hspace{1em} 1 sub item \\
\midrule
\texttt{* unordered list} & & $\bullet$ unordered list \\
\texttt{* second item} & $\rightarrow$ & $\bullet$ second item \\
\texttt{** sub item} & & \hspace{1em} $\bullet$ sub item \\
\midrule
\texttt{link to [[label]]} & $\rightarrow$ & link to \underline{label} \\
\midrule
\texttt{<{}<label>{}> definition } & $\rightarrow$ & definition \\
\midrule
\texttt{[[url|link name]]} & $\rightarrow$ & \underline{link name} \\
\midrule\addlinespace
\texttt{= large heading} & & {\Large large heading} \smallskip \\
\texttt{== medium heading} & $\rightarrow$ & {\large medium heading} \\
\texttt{=== small heading} & & small heading \\
\midrule
\texttt{no line break} & & no line break for paragraphs \\
\texttt{for paragraphs} & $\rightarrow$ \\
& & use empty line \\
\texttt{use empty line} \\
\midrule
\texttt{force\textbackslash\textbackslash line break} & $\rightarrow$ & force \\
& & line break \\
\midrule
\texttt{horizontal line} & $\rightarrow$ & horizontal line \\
\texttt{----} & & \hrulefill \\
\midrule
\texttt{|=a|=table|=header} & & \underline{a \enspace table \enspace header} \\
\texttt{|a|table|row} & $\rightarrow$ & a \enspace table \enspace row \\
\texttt{|b|table|row} & & b \enspace table \enspace row \\
\midrule
\texttt{\{\{\{} \\
\texttt{unformatted} & $\rightarrow$ & \texttt{unformatted} \\
\texttt{code} & & \texttt{code} \\
\texttt{\}\}\}} \\
\midrule\addlinespace
\texttt{@ new article} & & {\Large 1.\ new article} \smallskip \\
\texttt{@ second article} & $\rightarrow$ & {\Large 2.\ second article} \smallskip \\
\texttt{@@ sub article} & & {\large 2.1.\ sub article} \\
\bottomrule
\end{tabular}
\normalfont\caption{Elements of the generic documentation markup language}
\label{tab:docmarkup}
\end{table}
}

\providecommand{\startchapter}[4]{
\documentclass[11pt,a4paper]{article}
\usepackage{booktabs}
\usepackage[format=hang,labelfont=bf]{caption}
\usepackage{changepage}
\usepackage[T1]{fontenc}
\usepackage[margin=2cm]{geometry}
\usepackage{hyperref}
\usepackage[american]{isodate}
\usepackage{lmodern}
\usepackage{longtable}
\usepackage{mathptmx}
\usepackage{microtype}
\usepackage[toc]{multitoc}
\usepackage{multirow}
\usepackage[all]{nowidow}
\usepackage{pdfcomment}
\usepackage{syntax}
\usepackage{tikz}
\usepackage[all]{xy}
\hypersetup{pdfborder={0 0 0},bookmarksnumbered=true,pdftitle={\ecs{}: #2},pdfauthor={Florian Negele},pdfsubject={\ecs{}},pdfkeywords={#1}}
\setlength{\grammarindent}{8em}\setlength{\grammarparsep}{0.2ex}
\setlength{\columnsep}{2em}
\newcommand{\prefix}{}
\newcounter{instruction}
\bibliographystyle{unsrt}
\renewcommand{\index}[2][]{}
\renewcommand{\arraystretch}{1.05}
\renewcommand{\floatpagefraction}{0.7}
\renewcommand{\syntleft}{\itshape}\renewcommand{\syntright}{}
\title{\vspace{-5ex}\Huge{\ecs{}}\medskip\hrule}
\author{\huge{#2}}
\date{\medskip\version}
\newif\ifbook\bookfalse
\pagestyle{headings}
\frenchspacing
\begin{document}
\maketitle\thispagestyle{empty}\noindent#4\setlength{\columnseprule}{0.4pt}\tableofcontents\setlength{\columnseprule}{0pt}\vfill\pagebreak[3]\null\vfill\bigskip\noindent
\parbox{\textwidth-4em}{\license The contents of this \documentation{} are part of the \href{manual}{\ecs{} User Manual}~\cite{manual} and correspond to Chapter ``\href{manual\##3}{#1}''.\alignright\mbox{\today}}
\parbox{4em}{\flushright\ecslogo{3em}}
\clearpage
}

\providecommand{\concludechapter}{
\vfill\pagebreak[3]\null\vfill
\thispagestyle{myheadings}\markright{REFERENCES}
\noindent\begin{minipage}{\textwidth}\begin{multicols}{2}[\section*{References}]
\renewcommand{\section}[2]{}\small\bibliography{references}
\end{multicols}\end{minipage}\end{document}
}

\providecommand{\startpresentation}[2]{
\documentclass[14pt,aspectratio=43,usepdftitle=false]{beamer}
\usepackage{booktabs}
\usepackage{etex}
\usepackage{multicol}
\usepackage{tikz}
\usepackage[all]{xy}
\bibliographystyle{unsrt}
\setlength{\columnsep}{1em}
\setlength{\leftmargini}{1em}
\setbeamercolor{title}{fg=black}
\setbeamercolor{structure}{fg=darkgray}
\setbeamercolor{bibliography item}{fg=darkgray}
\setbeamerfont{title}{series=\bfseries}
\setbeamerfont{subtitle}{series=\normalfont}
\setbeamerfont*{frametitle}{parent=title}
\setbeamerfont{block title}{series=\bfseries}
\setbeamerfont*{framesubtitle}{parent=subtitle}
\setbeamersize{text margin left=1em,text margin right=1em}
\setbeamertemplate{navigation symbols}{}
\setbeamertemplate{itemize item}[circle]{}
\setbeamertemplate{bibliography item}[triangle]{}
\setbeamertemplate{bibliography entry author}{\usebeamercolor[fg]{bibliography item}}
\setbeamertemplate{frametitle}{\medskip\usebeamerfont{frametitle}\color{gray}\raisebox{-2.5ex}[0ex][0ex]{\rule{0.1em}{4.5ex}}}
\addtobeamertemplate{frametitle}{}{\hspace{0.4em}\usebeamercolor[fg]{title}\insertframetitle\par\vspace{0.2ex}\hspace{0.5em}\usebeamerfont{framesubtitle}\insertframesubtitle}
\hypersetup{pdfborder={0 0 0},bookmarksnumbered=true,bookmarksopen=true,bookmarksopenlevel=0,pdftitle={\ecs{}: #1},pdfauthor={Florian Negele},pdfsubject={\ecs{}},pdfkeywords={#1}}
\renewcommand{\flowgraph}[1]{\resizebox{\textwidth}{!}{$$\xymatrix{##1}$$}}
\title{\ecs{}\medskip\hrule\medskip}
\institute{\shadowedecslogo{5em}{30}{15}}
\date{\version}
\subtitle{#1}
\begin{document}
\begin{frame}[plain]\titlepage\nocite{manual}\end{frame}
\begin{frame}{Contents}{#1}\begin{center}\tableofcontents\end{center}\end{frame}
}

\providecommand{\concludepresentation}{
\begin{frame}{References}\begin{footnotesize}\setlength{\columnseprule}{0.4pt}\begin{multicols}{2}\bibliography{references}\end{multicols}\end{footnotesize}\end{frame}
\end{document}
}

\providecommand{\startbook}[1]{
\documentclass[10pt,paper=17cm:24cm,DIV=13,twoside=semi,headings=normal,numbers=noendperiod,cleardoublepage=plain]{scrbook}
\usepackage{atveryend}
\usepackage{booktabs}
\usepackage{caption}
\usepackage{changepage}
\usepackage[T1]{fontenc}
\usepackage{imakeidx}
\usepackage{hyperref}
\usepackage[american]{isodate}
\usepackage{lmodern}
\usepackage{longtable}
\usepackage{mathptmx}
\usepackage[final]{microtype}
\usepackage{multicol}
\usepackage{multirow}
\usepackage[all]{nowidow}
\usepackage{pdfcomment}
\usepackage{scrlayer-scrpage}
\usepackage{setspace}
\usepackage{syntax}
\usepackage[eventxtindent=4pt,oddtxtexdent=4pt]{thumbs}
\usepackage{tikz}
\usepackage[all]{xy}
\hyphenation{Micro-Blaze Open-Cores Open-RISC Power-PC}
\hypersetup{pdfborder={0 0 0},bookmarksnumbered=true,bookmarksopen=true,bookmarksopenlevel=0,pdftitle={\ecs{}: #1},pdfauthor={Florian Negele},pdfsubject={\ecs{}},pdfkeywords={#1}}
\setlength{\grammarindent}{8em}\setlength{\grammarparsep}{0.7ex}
\setkomafont{captionlabel}{\usekomafont{descriptionlabel}}
\renewcommand{\arraystretch}{1.05}\setstretch{1.1}
\renewcommand{\chapterformat}{\thechapter\autodot\enskip\raisebox{-1ex}[0ex][0ex]{\color{gray}\rule{0.1em}{3.5ex}}\enskip}
\renewcommand{\startchapter}[4]{\hypertarget{##3}{\chapter{##1}}\label{##3}##4\addthumb{##1}{\LARGE\sffamily\bfseries\thechapter}{white}{gray}\renewcommand{\prefix}{##3}}
\renewcommand{\concludechapter}{\clearpage{\stopthumb\cleardoublepage}}
\renewcommand{\syntleft}{\itshape}\renewcommand{\syntright}{}
\renewcommand{\floatpagefraction}{0.7}
\renewcommand{\partheademptypage}{}
\DeclareMicrotypeAlias{lmss}{cmr}
\newcommand{\prefix}{}
\newcounter{instruction}
\bibliographystyle{unsrt}
\newif\ifbook\booktrue
\makeindex[intoc,title=Index]
\makeindex[intoc,name=tools,title=Index of Tools,columns=3]
\makeindex[intoc,name=library,title=Index of Library Names]
\makeindex[intoc,name=runtime,title=Index of Runtime Support]
\makeindex[intoc,name=environment,title=Index of Target Environments]
\indexsetup{toclevel=chapter,headers={\indexname}{\indexname}}
\frenchspacing
\begin{document}
\pagenumbering{alph}
\begin{titlepage}\centering
\huge\sffamily\null\vfill\textbf{\ecs{}}\bigskip\hrule\bigskip#1
\normalsize\normalfont\vfill\vfill\shadowedecslogo{10em}{30}{15}
\large\vfill\vfill\version
\end{titlepage}
\null\vfill
\thispagestyle{empty}
\noindent\today\par\medskip
\license A copy of this license is included in Appendix~\ref{fdl} on page~\pageref{fdl}.
All product names used herein are for identification purposes only and may be trademarks of their respective companies.
\concludechapter
\frontmatter
\setcounter{tocdepth}{1}
\tableofcontents
\setcounter{tocdepth}{2}
\concludechapter
\listoffigures
\concludechapter
\listoftables
\concludechapter
}

\providecommand{\concludebook}{
\backmatter
\addtocontents{toc}{\protect\setcounter{tocdepth}{-1}}
\phantomsection\addcontentsline{toc}{part}{Bibliography}
\bibliography{references}
\concludechapter
\phantomsection\addcontentsline{toc}{part}{Indexes}
\printindex
\concludechapter
\indexprologue{\label{idx:tools}}
\printindex[tools]
\concludechapter
\printindex[library]
\concludechapter
\indexprologue{\label{idx:runtime}}
\printindex[runtime]
\concludechapter
\indexprologue{\label{idx:environment}}
\printindex[environment]
\concludechapter
\pagestyle{empty}\pagenumbering{Alph}\null\clearpage
\null\vfill\centering\ecslogo{4em}\par\medskip\license
\end{document}
}

% chapter references

\providecommand{\seedocumentationref}{}\renewcommand{\seedocumentationref}[3]{#1, see \Documentation{}~\documentationref{#2}{#3}. }
\providecommand{\seeinterface}{}\renewcommand{\seeinterface}{\ifbook See \Documentation{}~\documentationref{interface}{User Interface} for more information about the common user interface of all of these tools. \fi}
\providecommand{\seeguide}{}\renewcommand{\seeguide}{\seedocumentationref{For basic examples of using some of these tools in practice}{guide}{User Guide}}
\providecommand{\seecpp}{}\renewcommand{\seecpp}{\seedocumentationref{For more information about the \cpp{} programming language and its implementation by the \ecs{}}{cpp}{User Manual for \cpp{}}}
\providecommand{\seefalse}{}\renewcommand{\seefalse}{\seedocumentationref{For more information about the FALSE programming language and its implementation by the \ecs{}}{false}{User Manual for FALSE}}
\providecommand{\seeoberon}{}\renewcommand{\seeoberon}{\seedocumentationref{For more information about the Oberon programming language and its implementation by the \ecs{}}{oberon}{User Manual for Oberon}}
\providecommand{\seeassembly}{}\renewcommand{\seeassembly}{\seedocumentationref{For more information about the generic assembly language and how to use it}{assembly}{Generic Assembly Language Specification}}
\providecommand{\seeamd}{}\renewcommand{\seeamd}{\seedocumentationref{For more information about how the \ecs{} supports the AMD64 hardware architecture}{amd64}{AMD64 Hardware Architecture Support}}
\providecommand{\seearm}{}\renewcommand{\seearm}{\seedocumentationref{For more information about how the \ecs{} supports the ARM hardware architecture}{arm}{ARM Hardware Architecture Support}}
\providecommand{\seeavr}{}\renewcommand{\seeavr}{\seedocumentationref{For more information about how the \ecs{} supports the AVR hardware architecture}{avr}{AVR Hardware Architecture Support}}
\providecommand{\seeavrtt}{}\renewcommand{\seeavrtt}{\seedocumentationref{For more information about how the \ecs{} supports the AVR32 hardware architecture}{avr32}{AVR32 Hardware Architecture Support}}
\providecommand{\seemabk}{}\renewcommand{\seemabk}{\seedocumentationref{For more information about how the \ecs{} supports the M68000 hardware architecture}{m68k}{M68000 Hardware Architecture Support}}
\providecommand{\seemibl}{}\renewcommand{\seemibl}{\seedocumentationref{For more information about how the \ecs{} supports the MicroBlaze hardware architecture}{mibl}{MicroBlaze Hardware Architecture Support}}
\providecommand{\seemips}{}\renewcommand{\seemips}{\seedocumentationref{For more information about how the \ecs{} supports the MIPS32 and MIPS64 hardware architectures}{mips}{MIPS Hardware Architecture Support}}
\providecommand{\seemmix}{}\renewcommand{\seemmix}{\seedocumentationref{For more information about how the \ecs{} supports the MMIX hardware architecture}{mmix}{MMIX Hardware Architecture Support}}
\providecommand{\seeorok}{}\renewcommand{\seeorok}{\seedocumentationref{For more information about how the \ecs{} supports the OpenRISC 1000 hardware architecture}{or1k}{OpenRISC 1000 Hardware Architecture Support}}
\providecommand{\seeppc}{}\renewcommand{\seeppc}{\seedocumentationref{For more information about how the \ecs{} supports the PowerPC hardware architecture}{ppc}{PowerPC Hardware Architecture Support}}
\providecommand{\seerisc}{}\renewcommand{\seerisc}{\seedocumentationref{For more information about how the \ecs{} supports the RISC hardware architecture}{risc}{RISC Hardware Architecture Support}}
\providecommand{\seewasm}{}\renewcommand{\seewasm}{\seedocumentationref{For more information about how the \ecs{} supports the WebAssembly architecture}{wasm}{WebAssembly Architecture Support}}
\providecommand{\seedocumentation}{}\renewcommand{\seedocumentation}{\seedocumentationref{For more information about generic documentations and their generation by the \ecs{}}{documentation}{Generic Documentation Generation}}
\providecommand{\seedebugging}{}\renewcommand{\seedebugging}{\seedocumentationref{For more information about debugging information and its representation}{debugging}{Debugging Information Representation}}
\providecommand{\seecode}{}\renewcommand{\seecode}{\seedocumentationref{For more information about intermediate code and its purpose}{code}{Intermediate Code Representation}}
\providecommand{\seeobject}{}\renewcommand{\seeobject}{\seedocumentationref{For more information about object files and their purpose}{object}{Object File Representation}}

% generic documentation tools

\providecommand{\docprint}{
\toolsection{docprint} is a pretty printer for generic documentations.
It reformats generic documentations and writes it to the standard output stream.
\debuggingtool
\flowgraph{\resource{generic\\documentation} \ar[r] & \toolbox{docprint} \ar[r] & \resource{generic\\documentation}}
\seedocumentation
}

\providecommand{\doccheck}{
\toolsection{doccheck} is a syntactic and semantic checker for generic documentations.
It just performs syntactic and semantic checks on generic documentations and writes its diagnostic messages to the standard error stream.
\debuggingtool
\flowgraph{\resource{generic\\documentation} \ar[r] & \toolbox{doccheck} \ar[r] & \resource{diagnostic\\messages}}
\seedocumentation
}

\providecommand{\dochtml}{
\toolsection{dochtml} is an HTML documentation generator for generic documentations.
It processes several generic documentations and assembles all information therein into an HTML document.
\debuggingtool
\flowgraph{\resource{generic\\documentation} \ar[r] & \toolbox{dochtml} \ar[r] & \resource{HTML\\document}}
\seedocumentation
}

\providecommand{\doclatex}{
\toolsection{doclatex} is a Latex documentation generator for generic documentations.
It processes several generic documentations and assembles all information therein into a Latex document.
\debuggingtool
\flowgraph{\resource{generic\\documentation} \ar[r] & \toolbox{doclatex} \ar[r] & \resource{Latex\\document}}
\seedocumentation
}

% intermediate code tools

\providecommand{\cdcheck}{
\toolsection{cdcheck} is a syntactic and semantic checker for intermediate code.
It just performs syntactic and semantic checks on programs written in intermediate code and writes its diagnostic messages to the standard error stream.
\debuggingtool
\flowgraph{\resource{intermediate\\code} \ar[r] & \toolbox{cdcheck} \ar[r] & \resource{diagnostic\\messages}}
\seeassembly\seecode
}

\providecommand{\cdopt}{
\toolsection{cdopt} is an optimizer for intermediate code.
It performs various optimizations on programs written in intermediate code and writes the result to the standard output stream.
\debuggingtool
\flowgraph{\resource{intermediate\\code} \ar[r] & \toolbox{cdopt} \ar[r] & \resource{optimized\\code}}
\seeassembly\seecode
}

\providecommand{\cdrun}{
\toolsection{cdrun} is an interpreter for intermediate code.
It processes and executes programs written in intermediate code.
The following code sections are predefined and have the usual semantics:
\texttt{abort}, \texttt{\_Exit}, \texttt{fflush}, \texttt{floor}, \texttt{fputc}, \texttt{free}, \texttt{getchar}, \texttt{malloc}, and \texttt{putchar}.
Diagnostic messages about invalid operations include the name of the executed code section and the index of the erroneous instruction.
\debuggingtool
\flowgraph{\resource{intermediate\\code} \ar[r] & \toolbox{cdrun} \ar@/u/[r] & \resource{input/\\output} \ar@/d/[l]}
\seeassembly\seecode
}

\providecommand{\cdamda}{
\toolsection{cdamd16} is a compiler for intermediate code targeting the AMD64 hardware architecture.
It generates machine code for AMD64 processors from programs written in intermediate code and stores it in corresponding object files.
The compiler generates machine code for the 16-bit operating mode defined by the AMD64 architecture.
It also creates a debugging information file as well as an assembly file containing a listing of the generated machine code.
\debuggingtool
\flowgraph{\resource{intermediate\\code} \ar[r] & \toolbox{cdamd16} \ar[r] \ar[d] \ar[rd] & \resource{object file} \\ & \resource{assembly\\listing} & \resource{debugging\\information}}
\seeassembly\seeamd\seeobject\seecode\seedebugging
}

\providecommand{\cdamdb}{
\toolsection{cdamd32} is a compiler for intermediate code targeting the AMD64 hardware architecture.
It generates machine code for AMD64 processors from programs written in intermediate code and stores it in corresponding object files.
The compiler generates machine code for the 32-bit operating mode defined by the AMD64 architecture.
It also creates a debugging information file as well as an assembly file containing a listing of the generated machine code.
\debuggingtool
\flowgraph{\resource{intermediate\\code} \ar[r] & \toolbox{cdamd32} \ar[r] \ar[d] \ar[rd] & \resource{object file} \\ & \resource{assembly\\listing} & \resource{debugging\\information}}
\seeassembly\seeamd\seeobject\seecode\seedebugging
}

\providecommand{\cdamdc}{
\toolsection{cdamd64} is a compiler for intermediate code targeting the AMD64 hardware architecture.
It generates machine code for AMD64 processors from programs written in intermediate code and stores it in corresponding object files.
The compiler generates machine code for the 64-bit operating mode defined by the AMD64 architecture.
It also creates a debugging information file as well as an assembly file containing a listing of the generated machine code.
\debuggingtool
\flowgraph{\resource{intermediate\\code} \ar[r] & \toolbox{cdamd64} \ar[r] \ar[d] \ar[rd] & \resource{object file} \\ & \resource{assembly\\listing} & \resource{debugging\\information}}
\seeassembly\seeamd\seeobject\seecode\seedebugging
}

\providecommand{\cdarma}{
\toolsection{cdarma32} is a compiler for intermediate code targeting the ARM hardware architecture.
It generates machine code for ARM processors executing A32 instructions from programs written in intermediate code and stores it in corresponding object files.
It also creates a debugging information file as well as an assembly file containing a listing of the generated machine code.
\debuggingtool
\flowgraph{\resource{intermediate\\code} \ar[r] & \toolbox{cdarma32} \ar[r] \ar[d] \ar[rd] & \resource{object file} \\ & \resource{assembly\\listing} & \resource{debugging\\information}}
\seeassembly\seearm\seeobject\seecode\seedebugging
}

\providecommand{\cdarmb}{
\toolsection{cdarma64} is a compiler for intermediate code targeting the ARM hardware architecture.
It generates machine code for ARM processors executing A64 instructions from programs written in intermediate code and stores it in corresponding object files.
It also creates a debugging information file as well as an assembly file containing a listing of the generated machine code.
\debuggingtool
\flowgraph{\resource{intermediate\\code} \ar[r] & \toolbox{cdarma64} \ar[r] \ar[d] \ar[rd] & \resource{object file} \\ & \resource{assembly\\listing} & \resource{debugging\\information}}
\seeassembly\seearm\seeobject\seecode\seedebugging
}

\providecommand{\cdarmc}{
\toolsection{cdarmt32} is a compiler for intermediate code targeting the ARM hardware architecture.
It generates machine code for ARM processors without floating-point extension executing T32 instructions from programs written in intermediate code and stores it in corresponding object files.
It also creates a debugging information file as well as an assembly file containing a listing of the generated machine code.
\debuggingtool
\flowgraph{\resource{intermediate\\code} \ar[r] & \toolbox{cdarmt32} \ar[r] \ar[d] \ar[rd] & \resource{object file} \\ & \resource{assembly\\listing} & \resource{debugging\\information}}
\seeassembly\seearm\seeobject\seecode\seedebugging
}

\providecommand{\cdarmcfpe}{
\toolsection{cdarmt32fpe} is a compiler for intermediate code targeting the ARM hardware architecture.
It generates machine code for ARM processors with floating-point extension executing T32 instructions from programs written in intermediate code and stores it in corresponding object files.
It also creates a debugging information file as well as an assembly file containing a listing of the generated machine code.
\debuggingtool
\flowgraph{\resource{intermediate\\code} \ar[r] & \toolbox{cdarmt32fpe} \ar[r] \ar[d] \ar[rd] & \resource{object file} \\ & \resource{assembly\\listing} & \resource{debugging\\information}}
\seeassembly\seearm\seeobject\seecode\seedebugging
}

\providecommand{\cdavr}{
\toolsection{cdavr} is a compiler for intermediate code targeting the AVR hardware architecture.
It generates machine code for AVR processors from programs written in intermediate code and stores it in corresponding object files.
It also creates a debugging information file as well as an assembly file containing a listing of the generated machine code.
\debuggingtool
\flowgraph{\resource{intermediate\\code} \ar[r] & \toolbox{cdavr} \ar[r] \ar[d] \ar[rd] & \resource{object file} \\ & \resource{assembly\\listing} & \resource{debugging\\information}}
\seeassembly\seeavr\seeobject\seecode\seedebugging
}

\providecommand{\cdavrtt}{
\toolsection{cdavr32} is a compiler for intermediate code targeting the AVR32 hardware architecture.
It generates machine code for AVR32 processors from programs written in intermediate code and stores it in corresponding object files.
It also creates a debugging information file as well as an assembly file containing a listing of the generated machine code.
\debuggingtool
\flowgraph{\resource{intermediate\\code} \ar[r] & \toolbox{cdavr32} \ar[r] \ar[d] \ar[rd] & \resource{object file} \\ & \resource{assembly\\listing} & \resource{debugging\\information}}
\seeassembly\seeavrtt\seeobject\seecode\seedebugging
}

\providecommand{\cdmabk}{
\toolsection{cdm68k} is a compiler for intermediate code targeting the M68000 hardware architecture.
It generates machine code for M68000 processors from programs written in intermediate code and stores it in corresponding object files.
It also creates a debugging information file as well as an assembly file containing a listing of the generated machine code.
\debuggingtool
\flowgraph{\resource{intermediate\\code} \ar[r] & \toolbox{cdm68k} \ar[r] \ar[d] \ar[rd] & \resource{object file} \\ & \resource{assembly\\listing} & \resource{debugging\\information}}
\seeassembly\seemabk\seeobject\seecode\seedebugging
}

\providecommand{\cdmibl}{
\toolsection{cdmibl} is a compiler for intermediate code targeting the MicroBlaze hardware architecture.
It generates machine code for MicroBlaze processors from programs written in intermediate code and stores it in corresponding object files.
It also creates a debugging information file as well as an assembly file containing a listing of the generated machine code.
\debuggingtool
\flowgraph{\resource{intermediate\\code} \ar[r] & \toolbox{cdmibl} \ar[r] \ar[d] \ar[rd] & \resource{object file} \\ & \resource{assembly\\listing} & \resource{debugging\\information}}
\seeassembly\seemibl\seeobject\seecode\seedebugging
}

\providecommand{\cdmipsa}{
\toolsection{cdmips32} is a compiler for intermediate code targeting the MIPS32 hardware architecture.
It generates machine code for MIPS32 processors from programs written in intermediate code and stores it in corresponding object files.
It also creates a debugging information file as well as an assembly file containing a listing of the generated machine code.
\debuggingtool
\flowgraph{\resource{intermediate\\code} \ar[r] & \toolbox{cdmips32} \ar[r] \ar[d] \ar[rd] & \resource{object file} \\ & \resource{assembly\\listing} & \resource{debugging\\information}}
\seeassembly\seemips\seeobject\seecode\seedebugging
}

\providecommand{\cdmipsb}{
\toolsection{cdmips64} is a compiler for intermediate code targeting the MIPS64 hardware architecture.
It generates machine code for MIPS64 processors from programs written in intermediate code and stores it in corresponding object files.
It also creates a debugging information file as well as an assembly file containing a listing of the generated machine code.
\debuggingtool
\flowgraph{\resource{intermediate\\code} \ar[r] & \toolbox{cdmips64} \ar[r] \ar[d] \ar[rd] & \resource{object file} \\ & \resource{assembly\\listing} & \resource{debugging\\information}}
\seeassembly\seemips\seeobject\seecode\seedebugging
}

\providecommand{\cdmmix}{
\toolsection{cdmmix} is a compiler for intermediate code targeting the MMIX hardware architecture.
It generates machine code for MMIX processors from programs written in intermediate code and stores it in corresponding object files.
It also creates a debugging information file as well as an assembly file containing a listing of the generated machine code.
\debuggingtool
\flowgraph{\resource{intermediate\\code} \ar[r] & \toolbox{cdmmix} \ar[r] \ar[d] \ar[rd] & \resource{object file} \\ & \resource{assembly\\listing} & \resource{debugging\\information}}
\seeassembly\seemmix\seeobject\seecode\seedebugging
}

\providecommand{\cdorok}{
\toolsection{cdor1k} is a compiler for intermediate code targeting the OpenRISC 1000 hardware architecture.
It generates machine code for OpenRISC 1000 processors from programs written in intermediate code and stores it in corresponding object files.
It also creates a debugging information file as well as an assembly file containing a listing of the generated machine code.
\debuggingtool
\flowgraph{\resource{intermediate\\code} \ar[r] & \toolbox{cdor1k} \ar[r] \ar[d] \ar[rd] & \resource{object file} \\ & \resource{assembly\\listing} & \resource{debugging\\information}}
\seeassembly\seeorok\seeobject\seecode\seedebugging
}

\providecommand{\cdppca}{
\toolsection{cdppc32} is a compiler for intermediate code targeting the PowerPC hardware architecture.
It generates machine code for PowerPC processors from programs written in intermediate code and stores it in corresponding object files.
The compiler generates machine code for the 32-bit operating mode defined by the PowerPC architecture.
It also creates a debugging information file as well as an assembly file containing a listing of the generated machine code.
\debuggingtool
\flowgraph{\resource{intermediate\\code} \ar[r] & \toolbox{cdppc32} \ar[r] \ar[d] \ar[rd] & \resource{object file} \\ & \resource{assembly\\listing} & \resource{debugging\\information}}
\seeassembly\seeppc\seeobject\seecode\seedebugging
}

\providecommand{\cdppcb}{
\toolsection{cdppc64} is a compiler for intermediate code targeting the PowerPC hardware architecture.
It generates machine code for PowerPC processors from programs written in intermediate code and stores it in corresponding object files.
The compiler generates machine code for the 64-bit operating mode defined by the PowerPC architecture.
It also creates a debugging information file as well as an assembly file containing a listing of the generated machine code.
\debuggingtool
\flowgraph{\resource{intermediate\\code} \ar[r] & \toolbox{cdppc64} \ar[r] \ar[d] \ar[rd] & \resource{object file} \\ & \resource{assembly\\listing} & \resource{debugging\\information}}
\seeassembly\seeppc\seeobject\seecode\seedebugging
}

\providecommand{\cdrisc}{
\toolsection{cdrisc} is a compiler for intermediate code targeting the RISC hardware architecture.
It generates machine code for RISC processors from programs written in intermediate code and stores it in corresponding object files.
It also creates a debugging information file as well as an assembly file containing a listing of the generated machine code.
\debuggingtool
\flowgraph{\resource{intermediate\\code} \ar[r] & \toolbox{cdrisc} \ar[r] \ar[d] \ar[rd] & \resource{object file} \\ & \resource{assembly\\listing} & \resource{debugging\\information}}
\seeassembly\seerisc\seeobject\seecode\seedebugging
}

\providecommand{\cdwasm}{
\toolsection{cdwasm} is a compiler for intermediate code targeting the WebAssembly architecture.
It generates machine code for WebAssembly targets from programs written in intermediate code and stores it in corresponding object files.
It also creates a debugging information file as well as an assembly file containing a listing of the generated machine code.
\debuggingtool
\flowgraph{\resource{intermediate\\code} \ar[r] & \toolbox{cdwasm} \ar[r] \ar[d] \ar[rd] & \resource{object file} \\ & \resource{assembly\\listing} & \resource{debugging\\information}}
\seeassembly\seewasm\seeobject\seecode\seedebugging
}

% C++ tools

\providecommand{\cppprep}{
\toolsection{cppprep} is a preprocessor for the \cpp{} programming language.
It preprocesses source code according to the rules of \cpp{} and writes it to the standard output stream.
Only the macro names \texttt{\_\_DATE\_\_}, \texttt{\_\_FILE\_\_}, \texttt{\_\_LINE\_\_}, and \texttt{\_\_TIME\_\_} are predefined.
\flowgraph{\resource{\cpp{} or other\\source code} \ar[r] & \toolbox{cppprep} \ar[r] & \resource{preprocessed\\source code} \\ & \variable{ECSINCLUDE} \ar[u]}
\seecpp
}

\providecommand{\cppprint}{
\toolsection{cppprint} is a pretty printer for the \cpp{} programming language.
It reformats the source code of \cpp{} programs and writes it to the standard output stream.
\flowgraph{\resource{\cpp{}\\source code} \ar[r] & \toolbox{cppprint} \ar[r] & \resource{reformatted\\source code} \\ & \variable{ECSINCLUDE} \ar[u]}
\seecpp
}

\providecommand{\cppcheck}{
\toolsection{cppcheck} is a syntactic and semantic checker for the \cpp{} programming language.
It just performs syntactic and semantic checks on \cpp{} programs and writes its diagnostic messages to the standard error stream.
\flowgraph{\resource{\cpp{}\\source code} \ar[r] & \toolbox{cppcheck} \ar[r] & \resource{diagnostic\\messages} \\ & \variable{ECSINCLUDE} \ar[u]}
\seecpp
}

\providecommand{\cppdump}{
\toolsection{cppdump} is a serializer for the \cpp{} programming language.
It dumps the complete internal representation of programs written in \cpp{} into an XML document.
\debuggingtool
\flowgraph{\resource{\cpp{}\\source code} \ar[r] & \toolbox{cppdump} \ar[r] & \resource{internal\\representation} \\ & \variable{ECSINCLUDE} \ar[u]}
\seecpp
}

\providecommand{\cpprun}{
\toolsection{cpprun} is an interpreter for the \cpp{} programming language.
It processes and executes programs written in \cpp{}.
The macro \texttt{\_\_run\_\_} is predefined in order to enable programmers to identify this tool while interpreting.
\flowgraph{\resource{\cpp{}\\source code} \ar[r] & \toolbox{cpprun} \ar@/u/[r] & \resource{input/\\output} \ar@/d/[l] \\ & \variable{ECSINCLUDE} \ar[u]}
\seecpp
}

\providecommand{\cppdoc}{
\toolsection{cppdoc} is a generic documentation generator for the \cpp{} programming language.
It processes several \cpp{} source files and assembles all information therein into a generic documentation.
\debuggingtool
\flowgraph{\resource{\cpp{}\\source code} \ar[r] & \toolbox{cppdoc} \ar[r] & \resource{generic\\documentation} \\ & \variable{ECSINCLUDE} \ar[u]}
\seecpp\seedocumentation
}

\providecommand{\cpphtml}{
\toolsection{cpphtml} is an HTML documentation generator for the \cpp{} programming language.
It processes several \cpp{} source files and assembles all information therein into an HTML document.
\flowgraph{\resource{\cpp{}\\source code} \ar[r] & \toolbox{cpphtml} \ar[r] & \resource{HTML\\document} \\ & \variable{ECSINCLUDE} \ar[u]}
\seecpp\seedocumentation
}

\providecommand{\cpplatex}{
\toolsection{cpplatex} is a Latex documentation generator for the \cpp{} programming language.
It processes several \cpp{} source files and assembles all information therein into a Latex document.
\flowgraph{\resource{\cpp{}\\source code} \ar[r] & \toolbox{cpplatex} \ar[r] & \resource{Latex\\document} \\ & \variable{ECSINCLUDE} \ar[u]}
\seecpp\seedocumentation
}

\providecommand{\cppcode}{
\toolsection{cppcode} is an intermediate code generator for the \cpp{} programming language.
It generates intermediate code from programs written in \cpp{} and stores it in corresponding assembly files.
The macro \texttt{\_\_code\_\_} is predefined in order to enable programmers to identify this tool while generating intermediate code.
Programs generated with this tool require additional runtime support that is stored in the \file{cpp\-code\-run} library file.
\debuggingtool
\flowgraph{\resource{\cpp{}\\source code} \ar[r] & \toolbox{cppcode} \ar[r] & \resource{intermediate\\code} \\ & \variable{ECSINCLUDE} \ar[u]}
\seecpp\seeassembly\seecode
}

\providecommand{\cppamda}{
\toolsection{cppamd16} is a compiler for the \cpp{} programming language targeting the AMD64 hardware architecture.
It generates machine code for AMD64 processors from programs written in \cpp{} and stores it in corresponding object files.
The compiler generates machine code for the 16-bit operating mode defined by the AMD64 architecture.
For debugging purposes, it also creates a debugging information file as well as an assembly file containing a listing of the generated machine code.
The macro \texttt{\_\_amd16\_\_} is predefined in order to enable programmers to identify this tool and its target architecture while compiling.
Programs generated with this compiler require additional runtime support that is stored in the \file{cpp\-amd16\-run} library file.
\flowgraph{\resource{\cpp{}\\source code} \ar[r] & \toolbox{cppamd16} \ar[r] \ar[d] \ar[rd] & \resource{object file} \\ \variable{ECSINCLUDE} \ar[ru] & \resource{debugging\\information} & \resource{assembly\\listing}}
\seecpp\seeassembly\seeamd\seeobject\seedebugging
}

\providecommand{\cppamdb}{
\toolsection{cppamd32} is a compiler for the \cpp{} programming language targeting the AMD64 hardware architecture.
It generates machine code for AMD64 processors from programs written in \cpp{} and stores it in corresponding object files.
The compiler generates machine code for the 32-bit operating mode defined by the AMD64 architecture.
For debugging purposes, it also creates a debugging information file as well as an assembly file containing a listing of the generated machine code.
The macro \texttt{\_\_amd32\_\_} is predefined in order to enable programmers to identify this tool and its target architecture while compiling.
Programs generated with this compiler require additional runtime support that is stored in the \file{cpp\-amd32\-run} library file.
\flowgraph{\resource{\cpp{}\\source code} \ar[r] & \toolbox{cppamd32} \ar[r] \ar[d] \ar[rd] & \resource{object file} \\ \variable{ECSINCLUDE} \ar[ru] & \resource{debugging\\information} & \resource{assembly\\listing}}
\seecpp\seeassembly\seeamd\seeobject\seedebugging
}

\providecommand{\cppamdc}{
\toolsection{cppamd64} is a compiler for the \cpp{} programming language targeting the AMD64 hardware architecture.
It generates machine code for AMD64 processors from programs written in \cpp{} and stores it in corresponding object files.
The compiler generates machine code for the 64-bit operating mode defined by the AMD64 architecture.
For debugging purposes, it also creates a debugging information file as well as an assembly file containing a listing of the generated machine code.
The macro \texttt{\_\_amd64\_\_} is predefined in order to enable programmers to identify this tool and its target architecture while compiling.
Programs generated with this compiler require additional runtime support that is stored in the \file{cpp\-amd64\-run} library file.
\flowgraph{\resource{\cpp{}\\source code} \ar[r] & \toolbox{cppamd64} \ar[r] \ar[d] \ar[rd] & \resource{object file} \\ \variable{ECSINCLUDE} \ar[ru] & \resource{debugging\\information} & \resource{assembly\\listing}}
\seecpp\seeassembly\seeamd\seeobject\seedebugging
}

\providecommand{\cpparma}{
\toolsection{cpparma32} is a compiler for the \cpp{} programming language targeting the ARM hardware architecture.
It generates machine code for ARM processors executing A32 instructions from programs written in \cpp{} and stores it in corresponding object files.
For debugging purposes, it also creates a debugging information file as well as an assembly file containing a listing of the generated machine code.
The macro \texttt{\_\_arma32\_\_} is predefined in order to enable programmers to identify this tool and its target architecture while compiling.
Programs generated with this compiler require additional runtime support that is stored in the \file{cpp\-arma32\-run} library file.
\flowgraph{\resource{\cpp{}\\source code} \ar[r] & \toolbox{cpparma32} \ar[r] \ar[d] \ar[rd] & \resource{object file} \\ \variable{ECSINCLUDE} \ar[ru] & \resource{debugging\\information} & \resource{assembly\\listing}}
\seecpp\seeassembly\seearm\seeobject\seedebugging
}

\providecommand{\cpparmb}{
\toolsection{cpparma64} is a compiler for the \cpp{} programming language targeting the ARM hardware architecture.
It generates machine code for ARM processors executing A64 instructions from programs written in \cpp{} and stores it in corresponding object files.
For debugging purposes, it also creates a debugging information file as well as an assembly file containing a listing of the generated machine code.
The macro \texttt{\_\_arma64\_\_} is predefined in order to enable programmers to identify this tool and its target architecture while compiling.
Programs generated with this compiler require additional runtime support that is stored in the \file{cpp\-arma64\-run} library file.
\flowgraph{\resource{\cpp{}\\source code} \ar[r] & \toolbox{cpparma64} \ar[r] \ar[d] \ar[rd] & \resource{object file} \\ \variable{ECSINCLUDE} \ar[ru] & \resource{debugging\\information} & \resource{assembly\\listing}}
\seecpp\seeassembly\seearm\seeobject\seedebugging
}

\providecommand{\cpparmc}{
\toolsection{cpparmt32} is a compiler for the \cpp{} programming language targeting the ARM hardware architecture.
It generates machine code for ARM processors without floating-point extension executing T32 instructions from programs written in \cpp{} and stores it in corresponding object files.
For debugging purposes, it also creates a debugging information file as well as an assembly file containing a listing of the generated machine code.
The macro \texttt{\_\_armt32\_\_} is predefined in order to enable programmers to identify this tool and its target architecture while compiling.
Programs generated with this compiler require additional runtime support that is stored in the \file{cpp\-armt32\-run} library file.
\flowgraph{\resource{\cpp{}\\source code} \ar[r] & \toolbox{cpparmt32} \ar[r] \ar[d] \ar[rd] & \resource{object file} \\ \variable{ECSINCLUDE} \ar[ru] & \resource{debugging\\information} & \resource{assembly\\listing}}
\seecpp\seeassembly\seearm\seeobject\seedebugging
}

\providecommand{\cpparmcfpe}{
\toolsection{cpparmt32fpe} is a compiler for the \cpp{} programming language targeting the ARM hardware architecture.
It generates machine code for ARM processors with floating-point extension executing T32 instructions from programs written in \cpp{} and stores it in corresponding object files.
For debugging purposes, it also creates a debugging information file as well as an assembly file containing a listing of the generated machine code.
The macro \texttt{\_\_armt32fpe\_\_} is predefined in order to enable programmers to identify this tool and its target architecture while compiling.
Programs generated with this compiler require additional runtime support that is stored in the \file{cpp\-armt32\-fpe\-run} library file.
\flowgraph{\resource{\cpp{}\\source code} \ar[r] & \toolbox{cpparmt32fpe} \ar[r] \ar[d] \ar[rd] & \resource{object file} \\ \variable{ECSINCLUDE} \ar[ru] & \resource{debugging\\information} & \resource{assembly\\listing}}
\seecpp\seeassembly\seearm\seeobject\seedebugging
}

\providecommand{\cppavr}{
\toolsection{cppavr} is a compiler for the \cpp{} programming language targeting the AVR hardware architecture.
It generates machine code for AVR processors from programs written in \cpp{} and stores it in corresponding object files.
For debugging purposes, it also creates a debugging information file as well as an assembly file containing a listing of the generated machine code.
The macro \texttt{\_\_avr\_\_} is predefined in order to enable programmers to identify this tool and its target architecture while compiling.
Programs generated with this compiler require additional runtime support that is stored in the \file{cpp\-avr\-run} library file.
\flowgraph{\resource{\cpp{}\\source code} \ar[r] & \toolbox{cppavr} \ar[r] \ar[d] \ar[rd] & \resource{object file} \\ \variable{ECSINCLUDE} \ar[ru] & \resource{debugging\\information} & \resource{assembly\\listing}}
\seecpp\seeassembly\seeavr\seeobject\seedebugging
}

\providecommand{\cppavrtt}{
\toolsection{cppavr32} is a compiler for the \cpp{} programming language targeting the AVR32 hardware architecture.
It generates machine code for AVR32 processors from programs written in \cpp{} and stores it in corresponding object files.
For debugging purposes, it also creates a debugging information file as well as an assembly file containing a listing of the generated machine code.
The macro \texttt{\_\_avr32\_\_} is predefined in order to enable programmers to identify this tool and its target architecture while compiling.
Programs generated with this compiler require additional runtime support that is stored in the \file{cpp\-avr32\-run} library file.
\flowgraph{\resource{\cpp{}\\source code} \ar[r] & \toolbox{cppavr32} \ar[r] \ar[d] \ar[rd] & \resource{object file} \\ \variable{ECSINCLUDE} \ar[ru] & \resource{debugging\\information} & \resource{assembly\\listing}}
\seecpp\seeassembly\seeavrtt\seeobject\seedebugging
}

\providecommand{\cppmabk}{
\toolsection{cppm68k} is a compiler for the \cpp{} programming language targeting the M68000 hardware architecture.
It generates machine code for M68000 processors from programs written in \cpp{} and stores it in corresponding object files.
For debugging purposes, it also creates a debugging information file as well as an assembly file containing a listing of the generated machine code.
The macro \texttt{\_\_m68k\_\_} is predefined in order to enable programmers to identify this tool and its target architecture while compiling.
Programs generated with this compiler require additional runtime support that is stored in the \file{cpp\-m68k\-run} library file.
\flowgraph{\resource{\cpp{}\\source code} \ar[r] & \toolbox{cppm68k} \ar[r] \ar[d] \ar[rd] & \resource{object file} \\ \variable{ECSINCLUDE} \ar[ru] & \resource{debugging\\information} & \resource{assembly\\listing}}
\seecpp\seeassembly\seemabk\seeobject\seedebugging
}

\providecommand{\cppmibl}{
\toolsection{cppmibl} is a compiler for the \cpp{} programming language targeting the MicroBlaze hardware architecture.
It generates machine code for MicroBlaze processors from programs written in \cpp{} and stores it in corresponding object files.
For debugging purposes, it also creates a debugging information file as well as an assembly file containing a listing of the generated machine code.
The macro \texttt{\_\_mibl\_\_} is predefined in order to enable programmers to identify this tool and its target architecture while compiling.
Programs generated with this compiler require additional runtime support that is stored in the \file{cpp\-mibl\-run} library file.
\flowgraph{\resource{\cpp{}\\source code} \ar[r] & \toolbox{cppmibl} \ar[r] \ar[d] \ar[rd] & \resource{object file} \\ \variable{ECSINCLUDE} \ar[ru] & \resource{debugging\\information} & \resource{assembly\\listing}}
\seecpp\seeassembly\seemibl\seeobject\seedebugging
}

\providecommand{\cppmipsa}{
\toolsection{cppmips32} is a compiler for the \cpp{} programming language targeting the MIPS32 hardware architecture.
It generates machine code for MIPS32 processors from programs written in \cpp{} and stores it in corresponding object files.
For debugging purposes, it also creates a debugging information file as well as an assembly file containing a listing of the generated machine code.
The macro \texttt{\_\_mips32\_\_} is predefined in order to enable programmers to identify this tool and its target architecture while compiling.
Programs generated with this compiler require additional runtime support that is stored in the \file{cpp\-mips32\-run} library file.
\flowgraph{\resource{\cpp{}\\source code} \ar[r] & \toolbox{cppmips32} \ar[r] \ar[d] \ar[rd] & \resource{object file} \\ \variable{ECSINCLUDE} \ar[ru] & \resource{debugging\\information} & \resource{assembly\\listing}}
\seecpp\seeassembly\seemips\seeobject\seedebugging
}

\providecommand{\cppmipsb}{
\toolsection{cppmips64} is a compiler for the \cpp{} programming language targeting the MIPS64 hardware architecture.
It generates machine code for MIPS64 processors from programs written in \cpp{} and stores it in corresponding object files.
For debugging purposes, it also creates a debugging information file as well as an assembly file containing a listing of the generated machine code.
The macro \texttt{\_\_mips64\_\_} is predefined in order to enable programmers to identify this tool and its target architecture while compiling.
Programs generated with this compiler require additional runtime support that is stored in the \file{cpp\-mips64\-run} library file.
\flowgraph{\resource{\cpp{}\\source code} \ar[r] & \toolbox{cppmips64} \ar[r] \ar[d] \ar[rd] & \resource{object file} \\ \variable{ECSINCLUDE} \ar[ru] & \resource{debugging\\information} & \resource{assembly\\listing}}
\seecpp\seeassembly\seemips\seeobject\seedebugging
}

\providecommand{\cppmmix}{
\toolsection{cppmmix} is a compiler for the \cpp{} programming language targeting the MMIX hardware architecture.
It generates machine code for MMIX processors from programs written in \cpp{} and stores it in corresponding object files.
For debugging purposes, it also creates a debugging information file as well as an assembly file containing a listing of the generated machine code.
The macro \texttt{\_\_mmix\_\_} is predefined in order to enable programmers to identify this tool and its target architecture while compiling.
Programs generated with this compiler require additional runtime support that is stored in the \file{cpp\-mmix\-run} library file.
\flowgraph{\resource{\cpp{}\\source code} \ar[r] & \toolbox{cppmmix} \ar[r] \ar[d] \ar[rd] & \resource{object file} \\ \variable{ECSINCLUDE} \ar[ru] & \resource{debugging\\information} & \resource{assembly\\listing}}
\seecpp\seeassembly\seemmix\seeobject\seedebugging
}

\providecommand{\cpporok}{
\toolsection{cppor1k} is a compiler for the \cpp{} programming language targeting the OpenRISC 1000 hardware architecture.
It generates machine code for OpenRISC 1000 processors from programs written in \cpp{} and stores it in corresponding object files.
For debugging purposes, it also creates a debugging information file as well as an assembly file containing a listing of the generated machine code.
The macro \texttt{\_\_or1k\_\_} is predefined in order to enable programmers to identify this tool and its target architecture while compiling.
Programs generated with this compiler require additional runtime support that is stored in the \file{cpp\-or1k\-run} library file.
\flowgraph{\resource{\cpp{}\\source code} \ar[r] & \toolbox{cppor1k} \ar[r] \ar[d] \ar[rd] & \resource{object file} \\ \variable{ECSINCLUDE} \ar[ru] & \resource{debugging\\information} & \resource{assembly\\listing}}
\seecpp\seeassembly\seeorok\seeobject\seedebugging
}

\providecommand{\cppppca}{
\toolsection{cppppc32} is a compiler for the \cpp{} programming language targeting the PowerPC hardware architecture.
It generates machine code for PowerPC processors from programs written in \cpp{} and stores it in corresponding object files.
The compiler generates machine code for the 32-bit operating mode defined by the PowerPC architecture.
For debugging purposes, it also creates a debugging information file as well as an assembly file containing a listing of the generated machine code.
The macro \texttt{\_\_ppc32\_\_} is predefined in order to enable programmers to identify this tool and its target architecture while compiling.
Programs generated with this compiler require additional runtime support that is stored in the \file{cpp\-ppc32\-run} library file.
\flowgraph{\resource{\cpp{}\\source code} \ar[r] & \toolbox{cppppc32} \ar[r] \ar[d] \ar[rd] & \resource{object file} \\ \variable{ECSINCLUDE} \ar[ru] & \resource{debugging\\information} & \resource{assembly\\listing}}
\seecpp\seeassembly\seeppc\seeobject\seedebugging
}

\providecommand{\cppppcb}{
\toolsection{cppppc64} is a compiler for the \cpp{} programming language targeting the PowerPC hardware architecture.
It generates machine code for PowerPC processors from programs written in \cpp{} and stores it in corresponding object files.
The compiler generates machine code for the 64-bit operating mode defined by the PowerPC architecture.
For debugging purposes, it also creates a debugging information file as well as an assembly file containing a listing of the generated machine code.
The macro \texttt{\_\_ppc64\_\_} is predefined in order to enable programmers to identify this tool and its target architecture while compiling.
Programs generated with this compiler require additional runtime support that is stored in the \file{cpp\-ppc64\-run} library file.
\flowgraph{\resource{\cpp{}\\source code} \ar[r] & \toolbox{cppppc64} \ar[r] \ar[d] \ar[rd] & \resource{object file} \\ \variable{ECSINCLUDE} \ar[ru] & \resource{debugging\\information} & \resource{assembly\\listing}}
\seecpp\seeassembly\seeppc\seeobject\seedebugging
}

\providecommand{\cpprisc}{
\toolsection{cpprisc} is a compiler for the \cpp{} programming language targeting the RISC hardware architecture.
It generates machine code for RISC processors from programs written in \cpp{} and stores it in corresponding object files.
For debugging purposes, it also creates a debugging information file as well as an assembly file containing a listing of the generated machine code.
The macro \texttt{\_\_risc\_\_} is predefined in order to enable programmers to identify this tool and its target architecture while compiling.
Programs generated with this compiler require additional runtime support that is stored in the \file{cpp\-risc\-run} library file.
\flowgraph{\resource{\cpp{}\\source code} \ar[r] & \toolbox{cpprisc} \ar[r] \ar[d] \ar[rd] & \resource{object file} \\ \variable{ECSINCLUDE} \ar[ru] & \resource{debugging\\information} & \resource{assembly\\listing}}
\seecpp\seeassembly\seerisc\seeobject\seedebugging
}

\providecommand{\cppwasm}{
\toolsection{cppwasm} is a compiler for the \cpp{} programming language targeting the WebAssembly architecture.
It generates machine code for WebAssembly targets from programs written in \cpp{} and stores it in corresponding object files.
For debugging purposes, it also creates a debugging information file as well as an assembly file containing a listing of the generated machine code.
The macro \texttt{\_\_wasm\_\_} is predefined in order to enable programmers to identify this tool and its target architecture while compiling.
Programs generated with this compiler require additional runtime support that is stored in the \file{cpp\-wasm\-run} library file.
\flowgraph{\resource{\cpp{}\\source code} \ar[r] & \toolbox{cppwasm} \ar[r] \ar[d] \ar[rd] & \resource{object file} \\ \variable{ECSINCLUDE} \ar[ru] & \resource{debugging\\information} & \resource{assembly\\listing}}
\seecpp\seeassembly\seewasm\seeobject\seedebugging
}

% FALSE tools

\providecommand{\falprint}{
\toolsection{falprint} is a pretty printer for the FALSE programming language.
It reformats the source code of FALSE programs and writes it to the standard output stream.
\flowgraph{\resource{FALSE\\source code} \ar[r] & \toolbox{falprint} \ar[r] & \resource{reformatted\\source code}}
\seefalse
}

\providecommand{\falcheck}{
\toolsection{falcheck} is a syntactic and semantic checker for the FALSE programming language.
It just performs syntactic and semantic checks on FALSE programs and writes its diagnostic messages to the standard error stream.
\flowgraph{\resource{FALSE\\source code} \ar[r] & \toolbox{falcheck} \ar[r] & \resource{diagnostic\\messages}}
\seefalse
}

\providecommand{\faldump}{
\toolsection{faldump} is a serializer for the FALSE programming language.
It dumps the complete internal representation of programs written in FALSE into an XML document.
\debuggingtool
\flowgraph{\resource{FALSE\\source code} \ar[r] & \toolbox{faldump} \ar[r] & \resource{internal\\representation}}
\seefalse
}

\providecommand{\falrun}{
\toolsection{falrun} is an interpreter for the FALSE programming language.
It processes and executes programs written in FALSE\@.
\flowgraph{\resource{FALSE\\source code} \ar[r] & \toolbox{falrun} \ar@/u/[r] & \resource{input/\\output} \ar@/d/[l]}
\seefalse
}

\providecommand{\falcpp}{
\toolsection{falcpp} is a transpiler for the FALSE programming language.
It translates programs written in FALSE into \cpp{} programs and stores them in corresponding source files.
\flowgraph{\resource{FALSE\\source code} \ar[r] & \toolbox{falcpp} \ar[r] & \resource{\cpp{}\\source file}}
\seefalse\seecpp
}

\providecommand{\falcode}{
\toolsection{falcode} is an intermediate code generator for the FALSE programming language.
It generates intermediate code from programs written in FALSE and stores it in corresponding assembly files.
\debuggingtool
\flowgraph{\resource{FALSE\\source code} \ar[r] & \toolbox{falcode} \ar[r] & \resource{intermediate\\code}}
\seefalse\seeassembly\seecode
}

\providecommand{\falamda}{
\toolsection{falamd16} is a compiler for the FALSE programming language targeting the AMD64 hardware architecture.
It generates machine code for AMD64 processors from programs written in FALSE and stores it in corresponding object files.
The compiler generates machine code for the 16-bit operating mode defined by the AMD64 architecture.
\flowgraph{\resource{FALSE\\source code} \ar[r] & \toolbox{falamd16} \ar[r] & \resource{object file}}
\seefalse\seeamd\seeobject
}

\providecommand{\falamdb}{
\toolsection{falamd32} is a compiler for the FALSE programming language targeting the AMD64 hardware architecture.
It generates machine code for AMD64 processors from programs written in FALSE and stores it in corresponding object files.
The compiler generates machine code for the 32-bit operating mode defined by the AMD64 architecture.
\flowgraph{\resource{FALSE\\source code} \ar[r] & \toolbox{falamd32} \ar[r] & \resource{object file}}
\seefalse\seeamd\seeobject
}

\providecommand{\falamdc}{
\toolsection{falamd64} is a compiler for the FALSE programming language targeting the AMD64 hardware architecture.
It generates machine code for AMD64 processors from programs written in FALSE and stores it in corresponding object files.
The compiler generates machine code for the 64-bit operating mode defined by the AMD64 architecture.
\flowgraph{\resource{FALSE\\source code} \ar[r] & \toolbox{falamd64} \ar[r] & \resource{object file}}
\seefalse\seeamd\seeobject
}

\providecommand{\falarma}{
\toolsection{falarma32} is a compiler for the FALSE programming language targeting the ARM hardware architecture.
It generates machine code for ARM processors executing A32 instructions from programs written in FALSE and stores it in corresponding object files.
\flowgraph{\resource{FALSE\\source code} \ar[r] & \toolbox{falarma32} \ar[r] & \resource{object file}}
\seefalse\seearm\seeobject
}

\providecommand{\falarmb}{
\toolsection{falarma64} is a compiler for the FALSE programming language targeting the ARM hardware architecture.
It generates machine code for ARM processors executing A64 instructions from programs written in FALSE and stores it in corresponding object files.
\flowgraph{\resource{FALSE\\source code} \ar[r] & \toolbox{falarma64} \ar[r] & \resource{object file}}
\seefalse\seearm\seeobject
}

\providecommand{\falarmc}{
\toolsection{falarmt32} is a compiler for the FALSE programming language targeting the ARM hardware architecture.
It generates machine code for ARM processors without floating-point extension executing T32 instructions from programs written in FALSE and stores it in corresponding object files.
\flowgraph{\resource{FALSE\\source code} \ar[r] & \toolbox{falarmt32} \ar[r] & \resource{object file}}
\seefalse\seearm\seeobject
}

\providecommand{\falarmcfpe}{
\toolsection{falarmt32fpe} is a compiler for the FALSE programming language targeting the ARM hardware architecture.
It generates machine code for ARM processors with floating-point extension executing T32 instructions from programs written in FALSE and stores it in corresponding object files.
\flowgraph{\resource{FALSE\\source code} \ar[r] & \toolbox{falarmt32fpe} \ar[r] & \resource{object file}}
\seefalse\seearm\seeobject
}

\providecommand{\falavr}{
\toolsection{falavr} is a compiler for the FALSE programming language targeting the AVR hardware architecture.
It generates machine code for AVR processors from programs written in FALSE and stores it in corresponding object files.
\flowgraph{\resource{FALSE\\source code} \ar[r] & \toolbox{falavr} \ar[r] & \resource{object file}}
\seefalse\seeavr\seeobject
}

\providecommand{\falavrtt}{
\toolsection{falavr32} is a compiler for the FALSE programming language targeting the AVR32 hardware architecture.
It generates machine code for AVR32 processors from programs written in FALSE and stores it in corresponding object files.
\flowgraph{\resource{FALSE\\source code} \ar[r] & \toolbox{falavr32} \ar[r] & \resource{object file}}
\seefalse\seeavrtt\seeobject
}

\providecommand{\falmabk}{
\toolsection{falm68k} is a compiler for the FALSE programming language targeting the M68000 hardware architecture.
It generates machine code for M68000 processors from programs written in FALSE and stores it in corresponding object files.
\flowgraph{\resource{FALSE\\source code} \ar[r] & \toolbox{falm68k} \ar[r] & \resource{object file}}
\seefalse\seemabk\seeobject
}

\providecommand{\falmibl}{
\toolsection{falmibl} is a compiler for the FALSE programming language targeting the MicroBlaze hardware architecture.
It generates machine code for MicroBlaze processors from programs written in FALSE and stores it in corresponding object files.
\flowgraph{\resource{FALSE\\source code} \ar[r] & \toolbox{falmibl} \ar[r] & \resource{object file}}
\seefalse\seemibl\seeobject
}

\providecommand{\falmipsa}{
\toolsection{falmips32} is a compiler for the FALSE programming language targeting the MIPS32 hardware architecture.
It generates machine code for MIPS32 processors from programs written in FALSE and stores it in corresponding object files.
\flowgraph{\resource{FALSE\\source code} \ar[r] & \toolbox{falmips32} \ar[r] & \resource{object file}}
\seefalse\seemips\seeobject
}

\providecommand{\falmipsb}{
\toolsection{falmips64} is a compiler for the FALSE programming language targeting the MIPS64 hardware architecture.
It generates machine code for MIPS64 processors from programs written in FALSE and stores it in corresponding object files.
\flowgraph{\resource{FALSE\\source code} \ar[r] & \toolbox{falmips64} \ar[r] & \resource{object file}}
\seefalse\seemips\seeobject
}

\providecommand{\falmmix}{
\toolsection{falmmix} is a compiler for the FALSE programming language targeting the MMIX hardware architecture.
It generates machine code for MMIX processors from programs written in FALSE and stores it in corresponding object files.
\flowgraph{\resource{FALSE\\source code} \ar[r] & \toolbox{falmmix} \ar[r] & \resource{object file}}
\seefalse\seemmix\seeobject
}

\providecommand{\falorok}{
\toolsection{falor1k} is a compiler for the FALSE programming language targeting the OpenRISC 1000 hardware architecture.
It generates machine code for OpenRISC 1000 processors from programs written in FALSE and stores it in corresponding object files.
\flowgraph{\resource{FALSE\\source code} \ar[r] & \toolbox{falor1k} \ar[r] & \resource{object file}}
\seefalse\seeorok\seeobject
}

\providecommand{\falppca}{
\toolsection{falppc32} is a compiler for the FALSE programming language targeting the PowerPC hardware architecture.
It generates machine code for PowerPC processors from programs written in FALSE and stores it in corresponding object files.
The compiler generates machine code for the 32-bit operating mode defined by the PowerPC architecture.
\flowgraph{\resource{FALSE\\source code} \ar[r] & \toolbox{falppc32} \ar[r] & \resource{object file}}
\seefalse\seeppc\seeobject
}

\providecommand{\falppcb}{
\toolsection{falppc64} is a compiler for the FALSE programming language targeting the PowerPC hardware architecture.
It generates machine code for PowerPC processors from programs written in FALSE and stores it in corresponding object files.
The compiler generates machine code for the 64-bit operating mode defined by the PowerPC architecture.
\flowgraph{\resource{FALSE\\source code} \ar[r] & \toolbox{falppc64} \ar[r] & \resource{object file}}
\seefalse\seeppc\seeobject
}

\providecommand{\falrisc}{
\toolsection{falrisc} is a compiler for the FALSE programming language targeting the RISC hardware architecture.
It generates machine code for RISC processors from programs written in FALSE and stores it in corresponding object files.
\flowgraph{\resource{FALSE\\source code} \ar[r] & \toolbox{falrisc} \ar[r] & \resource{object file}}
\seefalse\seerisc\seeobject
}

\providecommand{\falwasm}{
\toolsection{falwasm} is a compiler for the FALSE programming language targeting the WebAssembly architecture.
It generates machine code for WebAssembly targets from programs written in FALSE and stores it in corresponding object files.
\flowgraph{\resource{FALSE\\source code} \ar[r] & \toolbox{falwasm} \ar[r] & \resource{object file}}
\seefalse\seewasm\seeobject
}

% Oberon tools

\providecommand{\obprint}{
\toolsection{obprint} is a pretty printer for the Oberon programming language.
It reformats the source code of Oberon modules and writes it to the standard output stream.
\flowgraph{\resource{Oberon\\source code} \ar[r] & \toolbox{obprint} \ar[r] & \resource{reformatted\\source code}}
\seeoberon
}

\providecommand{\obcheck}{
\toolsection{obcheck} is a syntactic and semantic checker for the Oberon programming language.
It just performs syntactic and semantic checks on Oberon modules and writes its diagnostic messages to the standard error stream.
In addition, it stores the interface of each module in a symbol file which is required when other modules import the module.
\flowgraph{\resource{Oberon\\source code} \ar[r] & \toolbox{obcheck} \ar[r] \ar@/l/[d] & \resource{diagnostic\\messages} \\ \variable{ECSIMPORT} \ar[ru] & \resource{symbol\\files} \ar@/r/[u]}
\seeoberon
}

\providecommand{\obdump}{
\toolsection{obdump} is a serializer for the Oberon programming language.
It dumps the complete internal representation of modules written in Oberon into an XML document.
\debuggingtool
\flowgraph{\resource{Oberon\\source code} \ar[r] & \toolbox{obdump} \ar[r] \ar@/l/[d] & \resource{internal\\representation} \\ \variable{ECSIMPORT} \ar[ru] & \resource{symbol\\files} \ar@/r/[u]}
\seeoberon
}

\providecommand{\obrun}{
\toolsection{obrun} is an interpreter for the Oberon programming language.
It processes and executes modules written in Oberon.
This tool does neither generate nor process symbol files while interpreting modules.
If a module is imported by another one, its filename has to be named before the other one in the list of command-line arguments.
\flowgraph{\resource{Oberon\\source code} \ar[r] & \toolbox{obrun} \ar@/u/[r] & \resource{input/\\output} \ar@/d/[l]}
\seeoberon
}

\providecommand{\obcpp}{
\toolsection{obcpp} is a transpiler for the Oberon programming language.
It translates programs written in Oberon into \cpp{} programs and stores them in corresponding source and header files.
In addition, it stores the interface of each module in a symbol file which is required when other modules import the module.
The same interface is provided by the generated header file which can be used in other parts of the \cpp{} program.
\flowgraph{\resource{Oberon\\source code} \ar[r] & \toolbox{obcpp} \ar[r] \ar@/l/[d] \ar[rd] & \resource{\cpp{}\\source file} \\ \variable{ECSIMPORT} \ar[ru] & \resource{symbol\\files} \ar@/r/[u] & \resource{\cpp{}\\header file}}
\seeoberon\seecpp
}

\providecommand{\obdoc}{
\toolsection{obdoc} is a generic documentation generator for the Oberon programming language.
It processes several Oberon modules and assembles all information therein into a generic documentation.
In addition, it stores the interface of each module in a symbol file which is required when other modules import the module.
\debuggingtool
\flowgraph{\resource{Oberon\\source code} \ar[r] & \toolbox{obdoc} \ar[r] \ar@/l/[d] & \resource{generic\\documentation} \\ \variable{ECSIMPORT} \ar[ru] & \resource{symbol\\files} \ar@/r/[u]}
\seeoberon\seedocumentation
}

\providecommand{\obhtml}{
\toolsection{obhtml} is an HTML documentation generator for the Oberon programming language.
It processes several Oberon modules and assembles all information therein into an HTML document.
In addition, it stores the interface of each module in a symbol file which is required when other modules import the module.
\flowgraph{\resource{Oberon\\source code} \ar[r] & \toolbox{obhtml} \ar[r] \ar@/l/[d] & \resource{HTML\\document} \\ \variable{ECSIMPORT} \ar[ru] & \resource{symbol\\files} \ar@/r/[u]}
\seeoberon\seedocumentation
}

\providecommand{\oblatex}{
\toolsection{oblatex} is a Latex documentation generator for the Oberon programming language.
It processes several Oberon modules and assembles all information therein into a Latex document.
In addition, it stores the interface of each module in a symbol file which is required when other modules import the module.
\flowgraph{\resource{Oberon\\source code} \ar[r] & \toolbox{oblatex} \ar[r] \ar@/l/[d] & \resource{Latex\\document} \\ \variable{ECSIMPORT} \ar[ru] & \resource{symbol\\files} \ar@/r/[u]}
\seeoberon\seedocumentation
}

\providecommand{\obcode}{
\toolsection{obcode} is an intermediate code generator for the Oberon programming language.
It generates intermediate code from modules written in Oberon and stores it in corresponding assembly files.
In addition, it stores the interface of each module in a symbol file which is required when other modules import the module.
Programs generated with this tool require additional runtime support that is stored in the \file{ob\-code\-run} library file.
\debuggingtool
\flowgraph{\resource{Oberon\\source code} \ar[r] & \toolbox{obcode} \ar[r] \ar@/l/[d] & \resource{intermediate\\code} \\ \variable{ECSIMPORT} \ar[ru] & \resource{symbol\\files} \ar@/r/[u]}
\seeoberon\seeassembly\seecode
}

\providecommand{\obamda}{
\toolsection{obamd16} is a compiler for the Oberon programming language targeting the AMD64 hardware architecture.
It generates machine code for AMD64 processors from modules written in Oberon and stores it in corresponding object files.
The compiler generates machine code for the 16-bit operating mode defined by the AMD64 architecture.
For debugging purposes, it also creates a debugging information file as well as an assembly file containing a listing of the generated machine code.
In addition, it stores the interface of each module in a symbol file which is required when other modules import the module.
Programs generated with this compiler require additional runtime support that is stored in the \file{ob\-amd16\-run} library file.
\flowgraph{\resource{Oberon\\source code} \ar[r] & \toolbox{obamd16} \ar[r] \ar@/l/[d] \ar[rd] & \resource{object file} \\ \variable{ECSIMPORT} \ar[ru] & \resource{symbol\\files} \ar@/r/[u] & \resource{debugging\\information}}
\seeoberon\seeassembly\seeamd\seeobject\seedebugging
}

\providecommand{\obamdb}{
\toolsection{obamd32} is a compiler for the Oberon programming language targeting the AMD64 hardware architecture.
It generates machine code for AMD64 processors from modules written in Oberon and stores it in corresponding object files.
The compiler generates machine code for the 32-bit operating mode defined by the AMD64 architecture.
For debugging purposes, it also creates a debugging information file as well as an assembly file containing a listing of the generated machine code.
In addition, it stores the interface of each module in a symbol file which is required when other modules import the module.
Programs generated with this compiler require additional runtime support that is stored in the \file{ob\-amd32\-run} library file.
\flowgraph{\resource{Oberon\\source code} \ar[r] & \toolbox{obamd32} \ar[r] \ar@/l/[d] \ar[rd] & \resource{object file} \\ \variable{ECSIMPORT} \ar[ru] & \resource{symbol\\files} \ar@/r/[u] & \resource{debugging\\information}}
\seeoberon\seeassembly\seeamd\seeobject\seedebugging
}

\providecommand{\obamdc}{
\toolsection{obamd64} is a compiler for the Oberon programming language targeting the AMD64 hardware architecture.
It generates machine code for AMD64 processors from modules written in Oberon and stores it in corresponding object files.
The compiler generates machine code for the 64-bit operating mode defined by the AMD64 architecture.
For debugging purposes, it also creates a debugging information file as well as an assembly file containing a listing of the generated machine code.
In addition, it stores the interface of each module in a symbol file which is required when other modules import the module.
Programs generated with this compiler require additional runtime support that is stored in the \file{ob\-amd64\-run} library file.
\flowgraph{\resource{Oberon\\source code} \ar[r] & \toolbox{obamd64} \ar[r] \ar@/l/[d] \ar[rd] & \resource{object file} \\ \variable{ECSIMPORT} \ar[ru] & \resource{symbol\\files} \ar@/r/[u] & \resource{debugging\\information}}
\seeoberon\seeassembly\seeamd\seeobject\seedebugging
}

\providecommand{\obarma}{
\toolsection{obarma32} is a compiler for the Oberon programming language targeting the ARM hardware architecture.
It generates machine code for ARM processors executing A32 instructions from modules written in Oberon and stores it in corresponding object files.
For debugging purposes, it also creates a debugging information file as well as an assembly file containing a listing of the generated machine code.
In addition, it stores the interface of each module in a symbol file which is required when other modules import the module.
Programs generated with this compiler require additional runtime support that is stored in the \file{ob\-arma32\-run} library file.
\flowgraph{\resource{Oberon\\source code} \ar[r] & \toolbox{obarma32} \ar[r] \ar@/l/[d] \ar[rd] & \resource{object file} \\ \variable{ECSIMPORT} \ar[ru] & \resource{symbol\\files} \ar@/r/[u] & \resource{debugging\\information}}
\seeoberon\seeassembly\seearm\seeobject\seedebugging
}

\providecommand{\obarmb}{
\toolsection{obarma64} is a compiler for the Oberon programming language targeting the ARM hardware architecture.
It generates machine code for ARM processors executing A64 instructions from modules written in Oberon and stores it in corresponding object files.
For debugging purposes, it also creates a debugging information file as well as an assembly file containing a listing of the generated machine code.
In addition, it stores the interface of each module in a symbol file which is required when other modules import the module.
Programs generated with this compiler require additional runtime support that is stored in the \file{ob\-arma64\-run} library file.
\flowgraph{\resource{Oberon\\source code} \ar[r] & \toolbox{obarma64} \ar[r] \ar@/l/[d] \ar[rd] & \resource{object file} \\ \variable{ECSIMPORT} \ar[ru] & \resource{symbol\\files} \ar@/r/[u] & \resource{debugging\\information}}
\seeoberon\seeassembly\seearm\seeobject\seedebugging
}

\providecommand{\obarmc}{
\toolsection{obarmt32} is a compiler for the Oberon programming language targeting the ARM hardware architecture.
It generates machine code for ARM processors without floating-point extension executing T32 instructions from modules written in Oberon and stores it in corresponding object files.
For debugging purposes, it also creates a debugging information file as well as an assembly file containing a listing of the generated machine code.
In addition, it stores the interface of each module in a symbol file which is required when other modules import the module.
Programs generated with this compiler require additional runtime support that is stored in the \file{ob\-armt32\-run} library file.
\flowgraph{\resource{Oberon\\source code} \ar[r] & \toolbox{obarmt32} \ar[r] \ar@/l/[d] \ar[rd] & \resource{object file} \\ \variable{ECSIMPORT} \ar[ru] & \resource{symbol\\files} \ar@/r/[u] & \resource{debugging\\information}}
\seeoberon\seeassembly\seearm\seeobject\seedebugging
}

\providecommand{\obarmcfpe}{
\toolsection{obarmt32fpe} is a compiler for the Oberon programming language targeting the ARM hardware architecture.
It generates machine code for ARM processors with floating-point extension executing T32 instructions from modules written in Oberon and stores it in corresponding object files.
For debugging purposes, it also creates a debugging information file as well as an assembly file containing a listing of the generated machine code.
In addition, it stores the interface of each module in a symbol file which is required when other modules import the module.
Programs generated with this compiler require additional runtime support that is stored in the \file{ob\-armt32\-fpe\-run} library file.
\flowgraph{\resource{Oberon\\source code} \ar[r] & \toolbox{obarmt32fpe} \ar[r] \ar@/l/[d] \ar[rd] & \resource{object file} \\ \variable{ECSIMPORT} \ar[ru] & \resource{symbol\\files} \ar@/r/[u] & \resource{debugging\\information}}
\seeoberon\seeassembly\seearm\seeobject\seedebugging
}

\providecommand{\obavr}{
\toolsection{obavr} is a compiler for the Oberon programming language targeting the AVR hardware architecture.
It generates machine code for AVR processors from modules written in Oberon and stores it in corresponding object files.
For debugging purposes, it also creates a debugging information file as well as an assembly file containing a listing of the generated machine code.
In addition, it stores the interface of each module in a symbol file which is required when other modules import the module.
Programs generated with this compiler require additional runtime support that is stored in the \file{ob\-avr\-run} library file.
\flowgraph{\resource{Oberon\\source code} \ar[r] & \toolbox{obavr} \ar[r] \ar@/l/[d] \ar[rd] & \resource{object file} \\ \variable{ECSIMPORT} \ar[ru] & \resource{symbol\\files} \ar@/r/[u] & \resource{debugging\\information}}
\seeoberon\seeassembly\seeavr\seeobject\seedebugging
}

\providecommand{\obavrtt}{
\toolsection{obavr32} is a compiler for the Oberon programming language targeting the AVR32 hardware architecture.
It generates machine code for AVR32 processors from modules written in Oberon and stores it in corresponding object files.
For debugging purposes, it also creates a debugging information file as well as an assembly file containing a listing of the generated machine code.
In addition, it stores the interface of each module in a symbol file which is required when other modules import the module.
Programs generated with this compiler require additional runtime support that is stored in the \file{ob\-avr32\-run} library file.
\flowgraph{\resource{Oberon\\source code} \ar[r] & \toolbox{obavr32} \ar[r] \ar@/l/[d] \ar[rd] & \resource{object file} \\ \variable{ECSIMPORT} \ar[ru] & \resource{symbol\\files} \ar@/r/[u] & \resource{debugging\\information}}
\seeoberon\seeassembly\seeavrtt\seeobject\seedebugging
}

\providecommand{\obmabk}{
\toolsection{obm68k} is a compiler for the Oberon programming language targeting the M68000 hardware architecture.
It generates machine code for M68000 processors from modules written in Oberon and stores it in corresponding object files.
For debugging purposes, it also creates a debugging information file as well as an assembly file containing a listing of the generated machine code.
In addition, it stores the interface of each module in a symbol file which is required when other modules import the module.
Programs generated with this compiler require additional runtime support that is stored in the \file{ob\-m68k\-run} library file.
\flowgraph{\resource{Oberon\\source code} \ar[r] & \toolbox{obm68k} \ar[r] \ar@/l/[d] \ar[rd] & \resource{object file} \\ \variable{ECSIMPORT} \ar[ru] & \resource{symbol\\files} \ar@/r/[u] & \resource{debugging\\information}}
\seeoberon\seeassembly\seemabk\seeobject\seedebugging
}

\providecommand{\obmibl}{
\toolsection{obmibl} is a compiler for the Oberon programming language targeting the MicroBlaze hardware architecture.
It generates machine code for MicroBlaze processors from modules written in Oberon and stores it in corresponding object files.
For debugging purposes, it also creates a debugging information file as well as an assembly file containing a listing of the generated machine code.
In addition, it stores the interface of each module in a symbol file which is required when other modules import the module.
Programs generated with this compiler require additional runtime support that is stored in the \file{ob\-mibl\-run} library file.
\flowgraph{\resource{Oberon\\source code} \ar[r] & \toolbox{obmibl} \ar[r] \ar@/l/[d] \ar[rd] & \resource{object file} \\ \variable{ECSIMPORT} \ar[ru] & \resource{symbol\\files} \ar@/r/[u] & \resource{debugging\\information}}
\seeoberon\seeassembly\seemibl\seeobject\seedebugging
}

\providecommand{\obmipsa}{
\toolsection{obmips32} is a compiler for the Oberon programming language targeting the MIPS32 hardware architecture.
It generates machine code for MIPS32 processors from modules written in Oberon and stores it in corresponding object files.
For debugging purposes, it also creates a debugging information file as well as an assembly file containing a listing of the generated machine code.
In addition, it stores the interface of each module in a symbol file which is required when other modules import the module.
Programs generated with this compiler require additional runtime support that is stored in the \file{ob\-mips32\-run} library file.
\flowgraph{\resource{Oberon\\source code} \ar[r] & \toolbox{obmips32} \ar[r] \ar@/l/[d] \ar[rd] & \resource{object file} \\ \variable{ECSIMPORT} \ar[ru] & \resource{symbol\\files} \ar@/r/[u] & \resource{debugging\\information}}
\seeoberon\seeassembly\seemips\seeobject\seedebugging
}

\providecommand{\obmipsb}{
\toolsection{obmips64} is a compiler for the Oberon programming language targeting the MIPS64 hardware architecture.
It generates machine code for MIPS64 processors from modules written in Oberon and stores it in corresponding object files.
For debugging purposes, it also creates a debugging information file as well as an assembly file containing a listing of the generated machine code.
In addition, it stores the interface of each module in a symbol file which is required when other modules import the module.
Programs generated with this compiler require additional runtime support that is stored in the \file{ob\-mips64\-run} library file.
\flowgraph{\resource{Oberon\\source code} \ar[r] & \toolbox{obmips64} \ar[r] \ar@/l/[d] \ar[rd] & \resource{object file} \\ \variable{ECSIMPORT} \ar[ru] & \resource{symbol\\files} \ar@/r/[u] & \resource{debugging\\information}}
\seeoberon\seeassembly\seemips\seeobject\seedebugging
}

\providecommand{\obmmix}{
\toolsection{obmmix} is a compiler for the Oberon programming language targeting the MMIX hardware architecture.
It generates machine code for MMIX processors from modules written in Oberon and stores it in corresponding object files.
For debugging purposes, it also creates a debugging information file as well as an assembly file containing a listing of the generated machine code.
In addition, it stores the interface of each module in a symbol file which is required when other modules import the module.
Programs generated with this compiler require additional runtime support that is stored in the \file{ob\-mmix\-run} library file.
\flowgraph{\resource{Oberon\\source code} \ar[r] & \toolbox{obmmix} \ar[r] \ar@/l/[d] \ar[rd] & \resource{object file} \\ \variable{ECSIMPORT} \ar[ru] & \resource{symbol\\files} \ar@/r/[u] & \resource{debugging\\information}}
\seeoberon\seeassembly\seemmix\seeobject\seedebugging
}

\providecommand{\oborok}{
\toolsection{obor1k} is a compiler for the Oberon programming language targeting the OpenRISC 1000 hardware architecture.
It generates machine code for OpenRISC 1000 processors from modules written in Oberon and stores it in corresponding object files.
For debugging purposes, it also creates a debugging information file as well as an assembly file containing a listing of the generated machine code.
In addition, it stores the interface of each module in a symbol file which is required when other modules import the module.
Programs generated with this compiler require additional runtime support that is stored in the \file{ob\-or1k\-run} library file.
\flowgraph{\resource{Oberon\\source code} \ar[r] & \toolbox{obor1k} \ar[r] \ar@/l/[d] \ar[rd] & \resource{object file} \\ \variable{ECSIMPORT} \ar[ru] & \resource{symbol\\files} \ar@/r/[u] & \resource{debugging\\information}}
\seeoberon\seeassembly\seeorok\seeobject\seedebugging
}

\providecommand{\obppca}{
\toolsection{obppc32} is a compiler for the Oberon programming language targeting the PowerPC hardware architecture.
It generates machine code for PowerPC processors from modules written in Oberon and stores it in corresponding object files.
The compiler generates machine code for the 32-bit operating mode defined by the PowerPC architecture.
For debugging purposes, it also creates a debugging information file as well as an assembly file containing a listing of the generated machine code.
In addition, it stores the interface of each module in a symbol file which is required when other modules import the module.
Programs generated with this compiler require additional runtime support that is stored in the \file{ob\-ppc32\-run} library file.
\flowgraph{\resource{Oberon\\source code} \ar[r] & \toolbox{obppc32} \ar[r] \ar@/l/[d] \ar[rd] & \resource{object file} \\ \variable{ECSIMPORT} \ar[ru] & \resource{symbol\\files} \ar@/r/[u] & \resource{debugging\\information}}
\seeoberon\seeassembly\seeppc\seeobject\seedebugging
}

\providecommand{\obppcb}{
\toolsection{obppc64} is a compiler for the Oberon programming language targeting the PowerPC hardware architecture.
It generates machine code for PowerPC processors from modules written in Oberon and stores it in corresponding object files.
The compiler generates machine code for the 64-bit operating mode defined by the PowerPC architecture.
For debugging purposes, it also creates a debugging information file as well as an assembly file containing a listing of the generated machine code.
In addition, it stores the interface of each module in a symbol file which is required when other modules import the module.
Programs generated with this compiler require additional runtime support that is stored in the \file{ob\-ppc64\-run} library file.
\flowgraph{\resource{Oberon\\source code} \ar[r] & \toolbox{obppc64} \ar[r] \ar@/l/[d] \ar[rd] & \resource{object file} \\ \variable{ECSIMPORT} \ar[ru] & \resource{symbol\\files} \ar@/r/[u] & \resource{debugging\\information}}
\seeoberon\seeassembly\seeppc\seeobject\seedebugging
}

\providecommand{\obrisc}{
\toolsection{obrisc} is a compiler for the Oberon programming language targeting the RISC hardware architecture.
It generates machine code for RISC processors from modules written in Oberon and stores it in corresponding object files.
For debugging purposes, it also creates a debugging information file as well as an assembly file containing a listing of the generated machine code.
In addition, it stores the interface of each module in a symbol file which is required when other modules import the module.
Programs generated with this compiler require additional runtime support that is stored in the \file{ob\-risc\-run} library file.
\flowgraph{\resource{Oberon\\source code} \ar[r] & \toolbox{obrisc} \ar[r] \ar@/l/[d] \ar[rd] & \resource{object file} \\ \variable{ECSIMPORT} \ar[ru] & \resource{symbol\\files} \ar@/r/[u] & \resource{debugging\\information}}
\seeoberon\seeassembly\seerisc\seeobject\seedebugging
}

\providecommand{\obwasm}{
\toolsection{obwasm} is a compiler for the Oberon programming language targeting the WebAssembly architecture.
It generates machine code for WebAssembly targets from modules written in Oberon and stores it in corresponding object files.
For debugging purposes, it also creates a debugging information file as well as an assembly file containing a listing of the generated machine code.
In addition, it stores the interface of each module in a symbol file which is required when other modules import the module.
Programs generated with this compiler require additional runtime support that is stored in the \file{ob\-wasm\-run} library file.
\flowgraph{\resource{Oberon\\source code} \ar[r] & \toolbox{obwasm} \ar[r] \ar@/l/[d] \ar[rd] & \resource{object file} \\ \variable{ECSIMPORT} \ar[ru] & \resource{symbol\\files} \ar@/r/[u] & \resource{debugging\\information}}
\seeoberon\seeassembly\seewasm\seeobject\seedebugging
}

% converter tools

\providecommand{\dbgdwarf}{
\toolsection{dbgdwarf} is a DWARF debugging information converter tool.
It converts debugging information into the DWARF debugging data format and stores it in corresponding object files~\cite{dwarffile}.
The resulting debugging object files can be combined with runtime support that creates Executable and Linking Format (ELF) files~\cite{elffile}.
\flowgraph{\resource{debugging\\information} \ar[r] & \toolbox{dbgdwarf} \ar[r] & \resource{debugging\\object file}}
\seeobject\seedebugging
}

% assembler tools

\providecommand{\asmprint}{
\toolsection{asmprint} is a pretty printer for generic assembly code.
It reformats generic assembly code and writes it to the standard output stream.
\flowgraph{\resource{generic assembly\\source code} \ar[r] & \toolbox{asmprint} \ar[r] & \resource{reformatted\\source code}}
\seeassembly
}

\providecommand{\amdaasm}{
\toolsection{amd16asm} is an assembler for the AMD64 hardware architecture.
It translates assembly code into machine code for AMD64 processors and stores it in corresponding object files.
By default, the assembler generates machine code for the 16-bit operating mode defined by the AMD64 architecture.
\flowgraph{\resource{AMD16 assembly\\source code} \ar[r] & \toolbox{amd16asm} \ar[r] & \resource{object file}}
\seeassembly\seeamd\seeobject
}

\providecommand{\amdadism}{
\toolsection{amd16dism} is a disassembler for the AMD64 hardware architecture.
It translates machine code from object files targeting AMD64 processors into assembly code and writes it to the standard output stream.
It assumes that the machine code was generated for the 16-bit operating mode defined by the AMD64 architecture.
\flowgraph{\resource{object file} \ar[r] & \toolbox{amd16dism} \ar[r] & \resource{disassembly\\listing}}
\seeassembly\seeamd\seeobject
}

\providecommand{\amdbasm}{
\toolsection{amd32asm} is an assembler for the AMD64 hardware architecture.
It translates assembly code into machine code for AMD64 processors and stores it in corresponding object files.
By default, the assembler generates machine code for the 32-bit operating mode defined by the AMD64 architecture.
\flowgraph{\resource{AMD32 assembly\\source code} \ar[r] & \toolbox{amd32asm} \ar[r] & \resource{object file}}
\seeassembly\seeamd\seeobject
}

\providecommand{\amdbdism}{
\toolsection{amd32dism} is a disassembler for the AMD64 hardware architecture.
It translates machine code from object files targeting AMD64 processors into assembly code and writes it to the standard output stream.
It assumes that the machine code was generated for the 32-bit operating mode defined by the AMD64 architecture.
\flowgraph{\resource{object file} \ar[r] & \toolbox{amd32dism} \ar[r] & \resource{disassembly\\listing}}
\seeassembly\seeamd\seeobject
}

\providecommand{\amdcasm}{
\toolsection{amd64asm} is an assembler for the AMD64 hardware architecture.
It translates assembly code into machine code for AMD64 processors and stores it in corresponding object files.
By default, the assembler generates machine code for the 64-bit operating mode defined by the AMD64 architecture.
\flowgraph{\resource{AMD64 assembly\\source code} \ar[r] & \toolbox{amd64asm} \ar[r] & \resource{object file}}
\seeassembly\seeamd\seeobject
}

\providecommand{\amdcdism}{
\toolsection{amd64dism} is a disassembler for the AMD64 hardware architecture.
It translates machine code from object files targeting AMD64 processors into assembly code and writes it to the standard output stream.
It assumes that the machine code was generated for the 64-bit operating mode defined by the AMD64 architecture.
\flowgraph{\resource{object file} \ar[r] & \toolbox{amd64dism} \ar[r] & \resource{disassembly\\listing}}
\seeassembly\seeamd\seeobject
}

\providecommand{\armaasm}{
\toolsection{arma32asm} is an assembler for the ARM hardware architecture.
It translates assembly code into machine code for ARM processors executing A32 instructions and stores it in corresponding object files.
\flowgraph{\resource{ARM A32 assembly\\source code} \ar[r] & \toolbox{arma32asm} \ar[r] & \resource{object file}}
\seeassembly\seearm\seeobject
}

\providecommand{\armadism}{
\toolsection{arma32dism} is a disassembler for the ARM hardware architecture.
It translates machine code from object files targeting ARM processors executing A32 instructions into assembly code and writes it to the standard output stream.
\flowgraph{\resource{object file} \ar[r] & \toolbox{arma32dism} \ar[r] & \resource{disassembly\\listing}}
\seeassembly\seearm\seeobject
}

\providecommand{\armbasm}{
\toolsection{arma64asm} is an assembler for the ARM hardware architecture.
It translates assembly code into machine code for ARM processors executing A64 instructions and stores it in corresponding object files.
\flowgraph{\resource{ARM A64 assembly\\source code} \ar[r] & \toolbox{arma64asm} \ar[r] & \resource{object file}}
\seeassembly\seearm\seeobject
}

\providecommand{\armbdism}{
\toolsection{arma64dism} is a disassembler for the ARM hardware architecture.
It translates machine code from object files targeting ARM processors executing A64 instructions into assembly code and writes it to the standard output stream.
\flowgraph{\resource{object file} \ar[r] & \toolbox{arma64dism} \ar[r] & \resource{disassembly\\listing}}
\seeassembly\seearm\seeobject
}

\providecommand{\armcasm}{
\toolsection{armt32asm} is an assembler for the ARM hardware architecture.
It translates assembly code into machine code for ARM processors executing T32 instructions and stores it in corresponding object files.
\flowgraph{\resource{ARM T32 assembly\\source code} \ar[r] & \toolbox{armt32asm} \ar[r] & \resource{object file}}
\seeassembly\seearm\seeobject
}

\providecommand{\armcdism}{
\toolsection{armt32dism} is a disassembler for the ARM hardware architecture.
It translates machine code from object files targeting ARM processors executing T32 instructions into assembly code and writes it to the standard output stream.
\flowgraph{\resource{object file} \ar[r] & \toolbox{armt32dism} \ar[r] & \resource{disassembly\\listing}}
\seeassembly\seearm\seeobject
}

\providecommand{\avrasm}{
\toolsection{avrasm} is an assembler for the AVR hardware architecture.
It translates assembly code into machine code for AVR processors and stores it in corresponding object files.
The identifiers \texttt{RXL}, \texttt{RXH}, \texttt{RYL}, \texttt{RYH}, \texttt{RZL}, and \texttt{RZH} are predefined and name the corresponding registers.
The identifiers \texttt{SPL} and \texttt{SPH} are also predefined and evaluate to the address of the corresponding registers.
\flowgraph{\resource{AVR assembly\\source code} \ar[r] & \toolbox{avrasm} \ar[r] & \resource{object file}}
\seeassembly\seeavr\seeobject
}

\providecommand{\avrdism}{
\toolsection{avrdism} is a disassembler for the AVR hardware architecture.
It translates machine code from object files targeting AVR processors into assembly code and writes it to the standard output stream.
\flowgraph{\resource{object file} \ar[r] & \toolbox{avrdism} \ar[r] & \resource{disassembly\\listing}}
\seeassembly\seeavr\seeobject
}

\providecommand{\avrttasm}{
\toolsection{avr32asm} is an assembler for the AVR32 hardware architecture.
It translates assembly code into machine code for AVR32 processors and stores it in corresponding object files.
\flowgraph{\resource{AVR32 assembly\\source code} \ar[r] & \toolbox{avr32asm} \ar[r] & \resource{object file}}
\seeassembly\seeavrtt\seeobject
}

\providecommand{\avrttdism}{
\toolsection{avr32dism} is a disassembler for the AVR32 hardware architecture.
It translates machine code from object files targeting AVR32 processors into assembly code and writes it to the standard output stream.
\flowgraph{\resource{object file} \ar[r] & \toolbox{avr32dism} \ar[r] & \resource{disassembly\\listing}}
\seeassembly\seeavrtt\seeobject
}

\providecommand{\mabkasm}{
\toolsection{m68kasm} is an assembler for the M68000 hardware architecture.
It translates assembly code into machine code for M68000 processors and stores it in corresponding object files.
\flowgraph{\resource{68000 assembly\\source code} \ar[r] & \toolbox{m68kasm} \ar[r] & \resource{object file}}
\seeassembly\seemabk\seeobject
}

\providecommand{\mabkdism}{
\toolsection{m68kdism} is a disassembler for the M68000 hardware architecture.
It translates machine code from object files targeting M68000 processors into assembly code and writes it to the standard output stream.
\flowgraph{\resource{object file} \ar[r] & \toolbox{m68kdism} \ar[r] & \resource{disassembly\\listing}}
\seeassembly\seemabk\seeobject
}

\providecommand{\miblasm}{
\toolsection{miblasm} is an assembler for the MicroBlaze hardware architecture.
It translates assembly code into machine code for MicroBlaze processors and stores it in corresponding object files.
\flowgraph{\resource{MicroBlaze assembly\\source code} \ar[r] & \toolbox{miblasm} \ar[r] & \resource{object file}}
\seeassembly\seemibl\seeobject
}

\providecommand{\mibldism}{
\toolsection{mibldism} is a disassembler for the MicroBlaze hardware architecture.
It translates machine code from object files targeting MicroBlaze processors into assembly code and writes it to the standard output stream.
\flowgraph{\resource{object file} \ar[r] & \toolbox{mibldism} \ar[r] & \resource{disassembly\\listing}}
\seeassembly\seemibl\seeobject
}

\providecommand{\mipsaasm}{
\toolsection{mips32asm} is an assembler for the MIPS32 hardware architecture.
It translates assembly code into machine code for MIPS32 processors and stores it in corresponding object files.
\flowgraph{\resource{MIPS32 assembly\\source code} \ar[r] & \toolbox{mips32asm} \ar[r] & \resource{object file}}
\seeassembly\seemips\seeobject
}

\providecommand{\mipsadism}{
\toolsection{mips32dism} is a disassembler for the MIPS32 hardware architecture.
It translates machine code from object files targeting MIPS32 processors into assembly code and writes it to the standard output stream.
\flowgraph{\resource{object file} \ar[r] & \toolbox{mips32dism} \ar[r] & \resource{disassembly\\listing}}
\seeassembly\seemips\seeobject
}

\providecommand{\mipsbasm}{
\toolsection{mips64asm} is an assembler for the MIPS64 hardware architecture.
It translates assembly code into machine code for MIPS64 processors and stores it in corresponding object files.
\flowgraph{\resource{MIPS64 assembly\\source code} \ar[r] & \toolbox{mips64asm} \ar[r] & \resource{object file}}
\seeassembly\seemips\seeobject
}

\providecommand{\mipsbdism}{
\toolsection{mips64dism} is a disassembler for the MIPS64 hardware architecture.
It translates machine code from object files targeting MIPS64 processors into assembly code and writes it to the standard output stream.
\flowgraph{\resource{object file} \ar[r] & \toolbox{mips64dism} \ar[r] & \resource{disassembly\\listing}}
\seeassembly\seemips\seeobject
}

\providecommand{\mmixasm}{
\toolsection{mmixasm} is an assembler for the MMIX hardware architecture.
It translates assembly code into machine code for MMIX processors and stores it in corresponding object files.
The names of all special registers are predefined and evaluate to the corresponding number.
\flowgraph{\resource{MMIX assembly\\source code} \ar[r] & \toolbox{mmixasm} \ar[r] & \resource{object file}}
\seeassembly\seemmix\seeobject
}

\providecommand{\mmixdism}{
\toolsection{mmixdism} is a disassembler for the MMIX hardware architecture.
It translates machine code from object files targeting MMIX processors into assembly code and writes it to the standard output stream.
\flowgraph{\resource{object file} \ar[r] & \toolbox{mmixdism} \ar[r] & \resource{disassembly\\listing}}
\seeassembly\seemmix\seeobject
}

\providecommand{\orokasm}{
\toolsection{or1kasm} is an assembler for the OpenRISC 1000 hardware architecture.
It translates assembly code into machine code for OpenRISC 1000 processors and stores it in corresponding object files.
\flowgraph{\resource{OpenRISC 1000 assembly\\source code} \ar[r] & \toolbox{or1kasm} \ar[r] & \resource{object file}}
\seeassembly\seeorok\seeobject
}

\providecommand{\orokdism}{
\toolsection{or1kdism} is a disassembler for the OpenRISC 1000 hardware architecture.
It translates machine code from object files targeting OpenRISC 1000 processors into assembly code and writes it to the standard output stream.
\flowgraph{\resource{object file} \ar[r] & \toolbox{or1kdism} \ar[r] & \resource{disassembly\\listing}}
\seeassembly\seeorok\seeobject
}

\providecommand{\ppcaasm}{
\toolsection{ppc32asm} is an assembler for the PowerPC hardware architecture.
It translates assembly code into machine code for PowerPC processors and stores it in corresponding object files.
By default, the assembler generates machine code for the 32-bit operating mode defined by the PowerPC architecture.
\flowgraph{\resource{PowerPC assembly\\source code} \ar[r] & \toolbox{ppc32asm} \ar[r] & \resource{object file}}
\seeassembly\seeppc\seeobject
}

\providecommand{\ppcadism}{
\toolsection{ppc32dism} is a disassembler for the PowerPC hardware architecture.
It translates machine code from object files targeting PowerPC processors into assembly code and writes it to the standard output stream.
It assumes that the machine code was generated for the 32-bit operating mode defined by the PowerPC architecture.
\flowgraph{\resource{object file} \ar[r] & \toolbox{ppc32dism} \ar[r] & \resource{disassembly\\listing}}
\seeassembly\seeppc\seeobject
}

\providecommand{\ppcbasm}{
\toolsection{ppc64asm} is an assembler for the PowerPC hardware architecture.
It translates assembly code into machine code for PowerPC processors and stores it in corresponding object files.
By default, the assembler generates machine code for the 64-bit operating mode defined by the PowerPC architecture.
\flowgraph{\resource{PowerPC assembly\\source code} \ar[r] & \toolbox{ppc64asm} \ar[r] & \resource{object file}}
\seeassembly\seeppc\seeobject
}

\providecommand{\ppcbdism}{
\toolsection{ppc64dism} is a disassembler for the PowerPC hardware architecture.
It translates machine code from object files targeting PowerPC processors into assembly code and writes it to the standard output stream.
It assumes that the machine code was generated for the 64-bit operating mode defined by the PowerPC architecture.
\flowgraph{\resource{object file} \ar[r] & \toolbox{ppc64dism} \ar[r] & \resource{disassembly\\listing}}
\seeassembly\seeppc\seeobject
}

\providecommand{\riscasm}{
\toolsection{riscasm} is an assembler for the RISC hardware architecture.
It translates assembly code into machine code for RISC processors and stores it in corresponding object files.
The names of all special registers are predefined and evaluate to the corresponding number.
\flowgraph{\resource{RISC assembly\\source code} \ar[r] & \toolbox{riscasm} \ar[r] & \resource{object file}}
\seeassembly\seerisc\seeobject
}

\providecommand{\riscdism}{
\toolsection{riscdism} is a disassembler for the RISC hardware architecture.
It translates machine code from object files targeting RISC processors into assembly code and writes it to the standard output stream.
\flowgraph{\resource{object file} \ar[r] & \toolbox{riscdism} \ar[r] & \resource{disassembly\\listing}}
\seeassembly\seerisc\seeobject
}

\providecommand{\wasmasm}{
\toolsection{wasmasm} is an assembler for the WebAssembly architecture.
It translates assembly code into machine code for WebAssembly targets and stores it in corresponding object files.
The names of all special registers are predefined and evaluate to the corresponding number.
\flowgraph{\resource{WebAssembly assembly\\source code} \ar[r] & \toolbox{wasmasm} \ar[r] & \resource{object file}}
\seeassembly\seewasm\seeobject
}

\providecommand{\wasmdism}{
\toolsection{wasmdism} is a disassembler for the WebAssembly architecture.
It translates machine code from object files targeting WebAssembly targets into assembly code and writes it to the standard output stream.
\flowgraph{\resource{object file} \ar[r] & \toolbox{wasmdism} \ar[r] & \resource{disassembly\\listing}}
\seeassembly\seewasm\seeobject
}

% linker tools

\providecommand{\linklib}{
\toolsection{linklib} is an object file combiner.
It creates a static library file by combining all object files given to it into a single one.
\flowgraph{\resource{object files} \ar[r] & \toolbox{linklib} \ar[r] & \resource{library file}}
\seeobject
}

\providecommand{\linkbin}{
\toolsection{linkbin} is a linker for plain binary files.
It links all object files given to it into a single image and stores it in a binary file that begins with the first linked section.
It also creates a map file that lists the address, type, name and size of all used sections.
The filename extension of the resulting binary file can be specified by putting it into a constant data section called \texttt{\_extension}.
\flowgraph{\resource{object files} \ar[r] & \toolbox{linkbin} \ar[r] \ar[d] & \resource{binary file} \\ & \resource{map file}}
\seeobject
}

\providecommand{\linkmem}{
\toolsection{linkmem} is a linker for plain binary files partitioned into random-access and read-only memory.
It links all object files given to it into two distinct images, one for data sections and one for code and constant data sections, and stores each image in a binary file that begins with the first linked section of the corresponding type.
It also creates a map file that lists the address, type, name and size of all used sections.
\flowgraph{\resource{object files} \ar[r] & \toolbox{linkmem} \ar[r] \ar[d] & \resource{RAM file/\\ROM file} \\ & \resource{map file}}
\seeobject
}

\providecommand{\linkprg}{
\toolsection{linkprg} is a linker for GEMDOS executable files.
It links all object files given to it into a single image and stores the image in an Atari GEMDOS executable file~\cite{gemdosfile}.
It also creates a map file that lists the address relative to the text segment, type, name and size of all used sections.
The filename extension of the resulting executable file can be specified by putting it into a constant data section called \texttt{\_extension}.
The GEMDOS executable file format requires all patch patterns of absolute link patches to consist of four full bitmasks with descending offsets.
\flowgraph{\resource{object files} \ar[r] & \toolbox{linkprg} \ar[r] \ar[d] & \resource{executable file} \\ & \resource{map file}}
\seeobject
}

\providecommand{\linkhex}{
\toolsection{linkhex} is a linker for Intel HEX files.
It links all code sections of the object files given to it into single image and stores the image in an Intel HEX file~\cite{hexfile} that begins with the first linked section.
It also creates a map file that lists the address, type, name and size of all used sections.
\flowgraph{\resource{object files} \ar[r] & \toolbox{linkhex} \ar[r] \ar[d] & \resource{HEX file} \\ & \resource{map file}}
\seeobject
}

\providecommand{\mapsearch}{
\toolsection{mapsearch} is a debugging tool.
It searches map files generated by linker tools for the name of a binary section that encompasses a memory address read from the standard input stream.
If additionally provided with one or more object files, it also stores an excerpt thereof in a separate object file called map search result which only contains the identified binary section for disassembling purposes.
\flowgraph{& \resource{map files/\\object files} \ar[d] \\ \resource{memory\\address} \ar[r] & \toolbox{mapsearch} \ar[r] \ar[d] & \resource{section name/\\relative offset} \\ & \resource{object file\\excerpt}}
\seeobject
}

\renewcommand{\seeorok}{}

\startchapter{OpenRISC 1000}{OpenRISC 1000 Hardware Architecture Support}{or1k}
{This \documentation{} describes how the \ecs{} supports the OpenRISC 1000 hardware architecture.
This includes information about the assembler, disassembler, and the various compilers featured by the \ecs{} as well as the interoperability between these tools.}

\section{Introduction}

The \ecs{} features various compilers, an assembler, and a disassembler that target the OpenRISC 1000 hardware architecture by OpenCores.
Figure~\ref{fig:or1kdataflow} shows the data flow in-between these tools.

\begin{figure}
\flowgraph{
\resource{intermediate\\code} \ar[d] & & \resource{assembly\\source code} \ar[d] \\
\converter{OpenRISC 1000\\Generator} \ar[r] \ar[rd] \ar[d] & \resource{assembly\\listing} \ar[r] & \converter{OpenRISC 1000\\Assembler} \ar[ld] \\
\resource{debugging\\information} & \resource{object file} \ar[d] \\
& \converter{OpenRISC 1000\\Disassembler} \ar[d] \\
& \resource{disassembly\\listing} \\
}\caption{Data flow within the tools targeting the OpenRISC 1000 architecture}
\label{fig:or1kdataflow}
\end{figure}

All compilers targeting the OpenRISC 1000 architecture translate their programs using an intermediate code representation.
The OpenRISC 1000 generator is able to translate the intermediate code representation of a program into machine code for OpenRISC 1000 processors.
It stores the resulting binary code and data in so-called object files.
Additionally, the generator is able to create an assembly code listing of the machine code for debugging purposes.
This assembly code listing can also be processed by the assembler yielding exactly the same object file.
The disassembler is able to open object files and print a human-readable disassembly listing of their contents.
\seeobject\seecode

\section{Instruction Set}

Tools targeting the OpenRISC 1000 architecture support the instruction set listed in Table~\ref{tab:or1kset} and use the same assembly syntax as predefined by OpenCores~\cite{or1k:instructionset}.
\seeassembly

\instructionset{or1k}{Supported OpenRISC 1000 instruction set}{5}{6}

\section{Calling Convention}\index{Calling convention!of OpenRISC 1000}

The machine code generator and runtime support for the OpenRISC 1000 architecture as provided by the \ecs{} use the following calling convention in order to enable interoperability.

\subsection{Stack Operations}

Arguments for functions are in general passed using the stack according to the intermediate code specification.
See \Documentation{}~\documentationref{code}{Intermediate Code Representation} for more information about the role of the stack.
Function arguments are pushed on the stack in reverse order and cleaned by the caller.

\subsection{Register Mapping}

The special-purpose registers defined by the intermediate code representation are mapped to their corresponding physical registers in the following way:

\begin{itemize}

\item Result Register\alignright\texttt{\$res}\nopagebreak

The intermediate code result register \texttt{\$res} is mapped to the OpenRISC 1000 registers \texttt{r11} and \texttt{r12} depending on the size of the actual return type.

\item Stack Pointer Register\alignright\texttt{\$sp}\nopagebreak

The intermediate code stack pointer register \texttt{\$sp} is mapped to the OpenRISC 1000 register \texttt{r1}.

\item Frame Pointer Register\alignright\texttt{\$fp}\nopagebreak

The intermediate code frame pointer register \texttt{\$fp} is mapped to the OpenRISC 1000 register \texttt{r2}.

\item Link Register\alignright\texttt{\$lnk}\nopagebreak

The intermediate code link register \texttt{\$lnk} is supported and mapped to the OpenRISC 1000 register \texttt{r9}.

\end{itemize}

All other intermediate code registers are mapped as needed to the remaining physical registers.
Their contents and mapping are therefore considered volatile across function calls.

\section{Runtime Support}\index{Runtime support!for OpenRISC 1000}

The \ecs{} provides runtime support for the OpenRISC 1000 architecture and runtime environments based on this hardware architecture in object files.
Users targeting a specific runtime environment have to use an appropriate linker together with these object files in order create an executable program.
This section gives information about all supported runtime environments based on the OpenRISC 1000 hardware architecture as well as the required combination of linker and object files.

Basic architectural runtime support is provided by the object file \objfile{or1k\-run}.
Users should always include this object file during linking regardless of the actual target runtime environment.
All other object files given to the linker should target the same hardware architecture.

Programs written in \cpp{} need additional runtime support stored in the \libfile{cpp\-or1k\-run} library file.
Programs written in Oberon need additional runtime support stored in the \libfile{ob\-or1k\-run} library file.
\seecpp\seeoberon

Programs targeting the OpenRISC 1000 simulator are created using the \tool{link\-bin} linker tool.
It creates a Common Object File Format (COFF) file~\cite{cofffile} if provided with the runtime support stored in the \objfile{or1k\-sim\-run} object file.
Calling the \tool{ecsd} utility tool using the \environment{or1k\-sim} target environment achieves the same result.

\section{OpenRISC 1000 Tools}

The \ecs{} provides the following tools that are able to process object files targeting the OpenRISC 1000 hardware architecture.
\interface

\cdorok
\cpporok
\falorok
\oborok
\orokasm
\orokdism
\linkbin

\concludechapter

% PowerPC architecture documentation
% Copyright (C) Florian Negele

% This file is part of the Eigen Compiler Suite.

% Permission is granted to copy, distribute and/or modify this document
% under the terms of the GNU Free Documentation License, Version 1.3
% or any later version published by the Free Software Foundation.

% You should have received a copy of the GNU Free Documentation License
% along with the ECS.  If not, see <https://www.gnu.org/licenses/>.

% Generic documentation utilities
% Copyright (C) Florian Negele

% This file is part of the Eigen Compiler Suite.

% Permission is granted to copy, distribute and/or modify this document
% under the terms of the GNU Free Documentation License, Version 1.3
% or any later version published by the Free Software Foundation.

% You should have received a copy of the GNU Free Documentation License
% along with the ECS.  If not, see <https://www.gnu.org/licenses/>.

\providecommand{\cpp}{C\texttt{++}}
\providecommand{\opt}{_\mathit{opt}}
\providecommand{\tool}[1]{\texttt{#1}}
\providecommand{\version}{Version 0.0.40}
\providecommand{\resource}[1]{*++\txt{#1}}
\providecommand{\ecs}{Eigen Compiler Suite}
\providecommand{\changed}[1]{\underline{#1}}
\providecommand{\toolbox}[1]{\converter{#1}}
\providecommand{\file}{}\renewcommand{\file}[1]{\texttt{#1}}
\providecommand{\alignright}{\hfill\linebreak[0]\hspace*{\fill}}
\providecommand{\converter}[1]{*++[F][F*:white][F,:gray]\txt{#1}}
\providecommand{\documentation}{\ifbook chapter\else document\fi}
\providecommand{\Documentation}{\ifbook Chapter\else Document\fi}
\providecommand{\variable}[1]{\resource{\texttt{\small#1}\\variable}}
\providecommand{\documentationref}[2]{\ifbook\ref{#1}\else``\href{#1}{#2}''~\cite{#1}\fi}
\providecommand{\objfile}[1]{\texttt{#1}\index[runtime]{#1 object file@\texttt{#1} object file}}
\providecommand{\libfile}[1]{\texttt{#1}\index[runtime]{#1 library file@\texttt{#1} library file}}
\providecommand{\epigraph}[2]{\ifbook\begin{quote}\flushright\textit{#1}\par--- #2\end{quote}\fi}
\providecommand{\environmentvariable}[1]{\texttt{#1}\index{Environment variables!#1@\texttt{#1}}}
\providecommand{\environment}[1]{\texttt{#1}\index[environment]{#1 environment@\texttt{#1} environment}}
\providecommand{\toolsection}{}\renewcommand{\toolsection}[1]{\subsection{#1}\label{\prefix:#1}\tool{#1}}
\providecommand{\instruction}{}\renewcommand{\instruction}[2]{\noindent\qquad\pdftooltip{\texttt{#1}}{#2}\refstepcounter{instruction}\par}
\providecommand{\flowgraph}{}\renewcommand{\flowgraph}[1]{\par\sffamily\begin{displaymath}\xymatrix@=4ex{#1}\end{displaymath}\normalfont\par}
\providecommand{\instructionset}{}\renewcommand{\instructionset}[4]{\setcounter{instruction}{0}\begin{multicols}{\ifbook#3\else#4\fi}[{\captionof{table}[#2]{#2 (\ref*{#1:instructions}~instructions)}\label{tab:#1set}\vspace{-2ex}}]\footnotesize\raggedcolumns\input{#1.set}\label{#1:instructions}\end{multicols}}

\providecommand{\gpl}{GNU General Public License}
\providecommand{\rse}{ECS Runtime Support Exception}
\providecommand{\fdl}{\href{https://www.gnu.org/licenses/fdl.html}{GNU Free Documentation License}}

\providecommand{\docbegin}{}
\providecommand{\docend}{}
\providecommand{\doclabel}[1]{\hypertarget{#1}}
\providecommand{\doclink}[2]{\hyperlink{#1}{#2}}
\providecommand{\docsection}[3]{\hypertarget{#1}{\subsection{#2}}\label{sec:#1}\index[library]{#2@#3}}
\providecommand{\docsectionstar}[1]{}
\providecommand{\docsubbegin}{\begin{description}}
\providecommand{\docsubend}{\end{description}}
\providecommand{\docsubsection}[3]{\item[\hypertarget{#1}{#2}]\index[library]{#2@#3}}
\providecommand{\docsubsectionstar}[1]{\smallskip}
\providecommand{\docsubsubsection}[3]{\docsubsection{#1}{#2}{#3}}
\providecommand{\docsubsubsectionstar}[1]{}
\providecommand{\docsubsubsubsection}[3]{}
\providecommand{\docsubsubsubsectionstar}[1]{}
\providecommand{\doctable}{}

\providecommand{\debuggingtool}{}\renewcommand{\debuggingtool}{This tool is provided for debugging purposes.
It allows exposing and modifying an internal data structure that is usually not accessible.
}

\providecommand{\interface}{All tools accept command-line arguments which are taken as names of plain text files containing the source code.
If no arguments are provided, the standard input stream is used instead.
Output files are generated in the current working directory and have the same name as the input file being processed whereas the filename extension gets replaced by an appropriate suffix.
\seeinterface
}

\providecommand{\license}{\noindent Copyright \copyright{} Florian Negele\par\medskip\noindent
Permission is granted to copy, distribute and/or modify this document under the terms of the
\fdl{}, Version 1.3 or any later version published by the \href{https://fsf.org/}{Free Software Foundation}.
}

\providecommand{\ecslogosurface}{
\fill[darkgray] (0,0,0) -- (0,0,3) -- (0,3,3) -- (0,3,1) -- (0,4,1) -- (0,4,3) -- (0,5,3) -- (0,5,0) -- (0,2,0) -- (0,2,2) -- (0,1,2) -- (0,1,0) -- cycle;
\fill[gray] (0,5,0) -- (0,5,3) -- (1,5,3) -- (1,5,1) -- (2,5,1) -- (2,5,3) -- (3,5,3) -- (3,5,0) -- cycle;
\fill[lightgray] (0,0,0) -- (0,1,0) -- (2,1,0) -- (2,4,0) -- (1,4,0) -- (1,3,0) -- (2,3,0) -- (2,2,0) -- (0,2,0) -- (0,5,0) -- (3,5,0) -- (3,0,0) -- cycle;
\begin{scope}[line width=0.5]
\begin{scope}[gray]
\draw (0,0,0) -- (0,1,0);
\draw (2,1,0) -- (2,2,0);
\draw (0,1,2) -- (0,2,2);
\draw (0,2,0) -- (0,5,0);
\draw (2,3,0) -- (2,4,0);
\end{scope}
\begin{scope}[lightgray]
\draw (0,1,0) -- (0,1,2);
\draw (0,3,1) -- (0,3,3);
\draw (0,5,0) -- (0,5,3);
\draw (2,5,1) -- (2,5,3);
\end{scope}
\begin{scope}[white]
\draw (0,1,0) -- (2,1,0);
\draw (1,3,0) -- (2,3,0);
\draw (0,5,0) -- (3,5,0);
\end{scope}
\end{scope}
}

\providecommand{\ecslogo}[1]{
\begin{tikzpicture}[scale={(#1)/((sin(45)+cos(45))*3cm)},x={({-cos(45)*1cm},{sin(45)*sin(30)*1cm})},y={({0cm},{(cos(30)*1cm})},z={({sin(45)*1cm},{cos(45)*sin(30)*1cm})}]
\begin{scope}[darkgray,line width=1]
\draw (0,0,0) -- (0,0,3) -- (0,3,3) -- (2,3,3) -- (2,5,3) -- (3,5,3) -- (3,5,0) -- (3,0,0) -- cycle;
\draw (0,3,1) -- (0,4,1) -- (0,4,3) -- (0,5,3) -- (1,5,3) -- (1,5,1) -- (2,5,1);
\draw (1,3,0) -- (1,4,0) -- (2,4,0);
\end{scope}
\fill[darkgray] (2,0,0) -- (2,0,3) -- (2,5,3) -- (2,5,1) -- (2,4,1) -- (2,4,0) -- cycle;
\fill[lightgray] (2,0,2) -- (0,0,2) -- (0,2,2) -- (2,2,2) -- cycle;
\fill[gray] (0,1,0) -- (2,1,0) -- (2,1,2) -- (0,1,2) -- cycle;
\fill[gray] (0,3,1) -- (0,3,3) -- (2,3,3) -- (2,3,0) -- (1,3,0) -- (1,3,1) -- cycle;
\ecslogosurface
\end{tikzpicture}
}

\providecommand{\shadowedecslogo}[3]{
\begin{tikzpicture}[scale={(#1)/((sin(#2)+cos(#2))*3cm)},x={({-cos(#2)*1cm},{sin(#2)*sin(#3)*1cm})},y={({0cm},{(cos(#3)*1cm})},z={({sin(#2)*1cm},{cos(#2)*sin(#3)*1cm})}]
\shade[top color=lightgray!50!white,bottom color=white,middle color=lightgray!50!white] (0,0,0) -- (3,0,0) -- (3,{-0.5-3*sin(#2)*sin(#3)/cos(#3)},0) -- (0,-0.5,0) -- cycle;
\shade[top color=darkgray!50!gray,bottom color=white,middle color=darkgray!50!white] (0,0,0) -- (0,0,3) -- (0,{-0.5-3*cos(#2)*sin(#3)/cos(#3)},3) -- (0,-0.5,0) -- cycle;
\begin{scope}[y={({(cos(#2)+sin(#2))*0.5cm},{(cos(#2)*sin(#3)-sin(#2)*sin(#3))*0.5cm})}]
\useasboundingbox (3,0,0) -- (0,0,0) -- (0,0,3);
\shade[left color=darkgray!80!black,right color=lightgray,middle color=gray] (0,0,0) -- (0,1,0) -- (0,1,0.5) -- (0,2,0) -- (0,5,0) -- (0,5,3) -- (1,5,3) -- (1,4,3) -- (1,4,2.5) -- (1,3,3) -- (2,5,3) -- (3,5,3) -- (3,0,3) -- cycle;
\clip (0,0,0) -- (0,0,3) -- ({-3*sin(#2)/cos(#2)},0,0) -- cycle;
\shade[left color=darkgray,right color=lightgray!50!gray] (0,0,0) -- (0,1,0) -- (0,1,0.5) -- (0,2,0) -- (0,5,0) -- (0,5,3) -- (1,5,3) -- (1,4,3) -- (1,4,2.5) -- (1,3,3) -- (2,5,3) -- (3,5,3) -- (3,0,3) -- cycle;
\end{scope}
\shade[left color=darkgray,right color=darkgray!80!black] (2,0,0) -- (2,0,3) -- (2,5,3) -- (2,5,1) -- (2,4,1) -- (2,4,0) -- cycle;
\shade[left color=darkgray!90!black,right color=gray!80!darkgray] (2,0,2) -- (0,0,2) -- (0,2,2) -- (2,2,2) -- cycle;
\shade[top color=darkgray!90!black,bottom color=gray!80!darkgray] (0,1,0) -- (2,1,0) -- (2,1,2) -- (0,1,2) -- cycle;
\shade[top color=darkgray!90!black,bottom color=gray!80!darkgray] (0,3,1) -- (0,3,3) -- (2,3,3) -- (2,3,0) -- (1,3,0) -- (1,3,1) -- cycle;
\fill[gray] (2,1,0) -- (1.5,1,0.5) -- (0,1,0.5) -- (0,1,0) -- cycle;
\fill[gray] (1,3,2) -- (0.5,3,2) -- (0.5,3,3) -- (1,3,3) -- cycle;
\fill[gray] (2,3,0) -- (1.5,3,0.5) -- (1,3,0.5) -- (1,3,0) -- cycle;
\ecslogosurface
\end{tikzpicture}
}

\providecommand{\cpplogo}[1]{
\begin{tikzpicture}[scale=(#1)/512em]
\fill[gray] (435.2794,398.7159) -- (247.1911,507.3075) .. controls (236.3563,513.5642) and (218.6240,513.5642) .. (207.7892,507.3075) -- (19.7009,398.7159) .. controls (8.8646,392.4606) and (0.0000,377.1043) .. (0.0000,364.5924) -- (0.0000,147.4076) .. controls (0.8430,132.8363) and (8.2856,120.7683) .. (19.7009,113.2842) -- (207.7892,4.6926) .. controls (218.6240,-1.5642) and (236.3564,-1.5642) .. (247.1911,4.6926) -- (435.2794,113.2842) .. controls (447.5273,121.4304) and (454.4987,133.6918) .. (454.9803,147.4076) -- (454.9803,364.5924) .. controls (454.5404,377.7571) and (446.6566,391.0351) .. (435.2794,398.7159) -- cycle(75.8301,255.9993) .. controls (74.9389,404.0881) and (273.2892,469.4783) .. (358.8263,331.8769) -- (293.1917,293.8965) .. controls (253.5702,359.4301) and (155.1909,335.9977) .. (151.6601,255.9993) .. controls (152.7204,182.2703) and (249.4137,148.0211) .. (293.1961,218.1065) -- (358.8308,180.1276) .. controls (283.4477,49.2645) and (79.6318,96.3470) .. (75.8301,255.9993) -- cycle(379.1503,247.5747) -- (362.2982,247.5747) -- (362.2982,230.7226) -- (345.4490,230.7226) -- (345.4490,247.5747) -- (328.5969,247.5747) -- (328.5969,264.4254) -- (345.4490,264.4254) -- (345.4490,281.2759) -- (362.2982,281.2759) -- (362.2982,264.4254) -- (379.1503,264.4254) -- cycle(442.3420,247.5747) -- (425.4899,247.5747) -- (425.4899,230.7226) -- (408.6408,230.7226) -- (408.6408,247.5747) -- (391.7886,247.5747) -- (391.7886,264.4254) -- (408.6408,264.4254) -- (408.6408,281.2759) -- (425.4899,281.2759) -- (425.4899,264.4254) -- (442.3420,264.4254) -- cycle;
\end{tikzpicture}
}

\providecommand{\fallogo}[1]{
\begin{tikzpicture}[scale=(#1)/512em]
\fill[gray] (185.7774,0.0000) .. controls (200.4486,15.9798) and (226.8966,8.7148) .. (235.0426,31.5836) .. controls (249.5297,58.0598) and (247.9581,97.9161) .. (280.3335,110.9762) .. controls (309.1690,120.3496) and (337.8406,104.2727) .. (366.5753,103.9379) .. controls (373.4449,111.5171) and (379.2885,128.2574) .. (383.9755,108.9744) .. controls (396.6979,102.5615) and (437.2808,107.6681) .. (426.9652,124.3252) .. controls (408.9822,121.0785) and (412.4742,146.0729) .. (426.5192,131.4996) .. controls (433.8413,120.8489) and (465.1541,126.5522) .. (441.9067,135.7950) .. controls (396.1879,157.7478) and (344.1112,161.5079) .. (298.5528,183.5702) .. controls (277.7471,193.5198) and (284.6941,218.7163) .. (285.2127,236.9640) .. controls (292.3599,316.2826) and (307.3929,394.6311) .. (317.1198,473.6154) .. controls (329.0637,505.4736) and (292.1195,528.5004) .. (265.9183,511.2761) .. controls (237.9284,499.2462) and (237.3684,465.2681) .. (230.9102,439.9421) .. controls (218.6692,374.3397) and (215.6307,306.9662) .. (198.1732,242.3977) .. controls (183.1379,232.7444) and (164.4245,256.0298) .. (149.0430,261.4799) .. controls (116.9328,279.2585) and (87.1822,308.5851) .. (48.2293,307.8914) .. controls (21.3220,306.9037) and (-15.9107,281.8761) .. (7.2921,252.7908) .. controls (29.7799,220.6177) and (67.5177,204.3028) .. (100.9287,185.9449) .. controls (130.8217,170.8906) and (161.1548,156.5903) .. (191.0278,141.5847) .. controls (196.1738,120.0520) and (186.6049,95.2409) .. (186.8382,72.4353) .. controls (185.5234,48.4204) and (183.1700,23.9341) .. (185.7774,0.0000) -- cycle;
\end{tikzpicture}
}

\providecommand{\oblogo}[1]{
\begin{tikzpicture}[scale=(#1)/512em]
\fill[gray] (160.3865,208.9117) .. controls (154.0879,214.6478) and (149.0735,221.2409) .. (145.4125,228.5384) .. controls (184.8790,248.4273) and (234.7122,269.8787) .. (297.5493,291.8782) .. controls (300.3943,281.4769) and (300.9552,268.7619) .. (300.4023,255.2389) .. controls (248.9909,244.7891) and (200.0310,225.9279) .. (160.3865,208.9117) -- cycle(225.7398,392.6996) .. controls (308.0209,392.1716) and (359.3326,345.9277) .. (368.7203,285.2098) .. controls (376.6742,197.1784) and (311.7194,141.3342) .. (205.4287,142.1456) .. controls (139.9485,141.4804) and (88.7155,166.1957) .. (73.5775,228.0086) .. controls (52.0297,320.3408) and (123.4078,391.0103) .. (225.7398,392.6996) -- cycle(216.0739,176.4733) .. controls (268.9183,179.2424) and (315.8292,206.5488) .. (312.7454,265.1139) .. controls (313.2769,315.6384) and (286.5993,353.4946) .. (216.6040,355.7934) .. controls (162.4657,355.7934) and (126.0914,317.5023) .. (126.0914,260.5103) .. controls (126.1733,214.2900) and (163.3363,176.2849) .. (216.0739,176.4733) -- cycle(76.4897,189.1754) .. controls (13.1586,147.5631) and (0.0000,119.4207) .. (0.0000,119.4207) -- (90.6499,170.1632) .. controls (85.3004,175.8497) and (80.5994,182.1633) .. (76.4897,189.1754) -- cycle(353.9486,119.3004) -- (402.9482,119.3004) .. controls (427.0025,137.0797) and (450.9893,162.7034) .. (474.9529,191.0213) .. controls (509.3540,228.5339) and (531.3391,294.2091) .. (487.8149,312.1206) .. controls (462.8165,324.7652) and (394.3874,316.8943) .. (373.8912,313.6651) .. controls (379.9291,297.7449) and (383.2899,278.4204) .. (381.4989,257.7214) .. controls (420.3069,248.0321) and (421.9610,218.3461) .. (407.7867,192.6417) .. controls (391.1113,162.4018) and (370.1114,132.9097) .. (353.9486,119.3004) -- cycle;
\end{tikzpicture}
}

\providecommand{\markuptable}{
\begin{table}
\sffamily\centering
\begin{tabular}{@{}lcl@{}}
\toprule
\texttt{//italics//} & $\rightarrow$ & \textit{italics} \\
\midrule
\texttt{**bold**} & $\rightarrow$ & \textbf{bold} \\
\midrule
\texttt{\# ordered list} & & 1 ordered list \\
\texttt{\# second item} & $\rightarrow$ & 2 second item \\
\texttt{\#\# sub item} & & \hspace{1em} 1 sub item \\
\midrule
\texttt{* unordered list} & & $\bullet$ unordered list \\
\texttt{* second item} & $\rightarrow$ & $\bullet$ second item \\
\texttt{** sub item} & & \hspace{1em} $\bullet$ sub item \\
\midrule
\texttt{link to [[label]]} & $\rightarrow$ & link to \underline{label} \\
\midrule
\texttt{<{}<label>{}> definition } & $\rightarrow$ & definition \\
\midrule
\texttt{[[url|link name]]} & $\rightarrow$ & \underline{link name} \\
\midrule\addlinespace
\texttt{= large heading} & & {\Large large heading} \smallskip \\
\texttt{== medium heading} & $\rightarrow$ & {\large medium heading} \\
\texttt{=== small heading} & & small heading \\
\midrule
\texttt{no line break} & & no line break for paragraphs \\
\texttt{for paragraphs} & $\rightarrow$ \\
& & use empty line \\
\texttt{use empty line} \\
\midrule
\texttt{force\textbackslash\textbackslash line break} & $\rightarrow$ & force \\
& & line break \\
\midrule
\texttt{horizontal line} & $\rightarrow$ & horizontal line \\
\texttt{----} & & \hrulefill \\
\midrule
\texttt{|=a|=table|=header} & & \underline{a \enspace table \enspace header} \\
\texttt{|a|table|row} & $\rightarrow$ & a \enspace table \enspace row \\
\texttt{|b|table|row} & & b \enspace table \enspace row \\
\midrule
\texttt{\{\{\{} \\
\texttt{unformatted} & $\rightarrow$ & \texttt{unformatted} \\
\texttt{code} & & \texttt{code} \\
\texttt{\}\}\}} \\
\midrule\addlinespace
\texttt{@ new article} & & {\Large 1.\ new article} \smallskip \\
\texttt{@ second article} & $\rightarrow$ & {\Large 2.\ second article} \smallskip \\
\texttt{@@ sub article} & & {\large 2.1.\ sub article} \\
\bottomrule
\end{tabular}
\normalfont\caption{Elements of the generic documentation markup language}
\label{tab:docmarkup}
\end{table}
}

\providecommand{\startchapter}[4]{
\documentclass[11pt,a4paper]{article}
\usepackage{booktabs}
\usepackage[format=hang,labelfont=bf]{caption}
\usepackage{changepage}
\usepackage[T1]{fontenc}
\usepackage[margin=2cm]{geometry}
\usepackage{hyperref}
\usepackage[american]{isodate}
\usepackage{lmodern}
\usepackage{longtable}
\usepackage{mathptmx}
\usepackage{microtype}
\usepackage[toc]{multitoc}
\usepackage{multirow}
\usepackage[all]{nowidow}
\usepackage{pdfcomment}
\usepackage{syntax}
\usepackage{tikz}
\usepackage[all]{xy}
\hypersetup{pdfborder={0 0 0},bookmarksnumbered=true,pdftitle={\ecs{}: #2},pdfauthor={Florian Negele},pdfsubject={\ecs{}},pdfkeywords={#1}}
\setlength{\grammarindent}{8em}\setlength{\grammarparsep}{0.2ex}
\setlength{\columnsep}{2em}
\newcommand{\prefix}{}
\newcounter{instruction}
\bibliographystyle{unsrt}
\renewcommand{\index}[2][]{}
\renewcommand{\arraystretch}{1.05}
\renewcommand{\floatpagefraction}{0.7}
\renewcommand{\syntleft}{\itshape}\renewcommand{\syntright}{}
\title{\vspace{-5ex}\Huge{\ecs{}}\medskip\hrule}
\author{\huge{#2}}
\date{\medskip\version}
\newif\ifbook\bookfalse
\pagestyle{headings}
\frenchspacing
\begin{document}
\maketitle\thispagestyle{empty}\noindent#4\setlength{\columnseprule}{0.4pt}\tableofcontents\setlength{\columnseprule}{0pt}\vfill\pagebreak[3]\null\vfill\bigskip\noindent
\parbox{\textwidth-4em}{\license The contents of this \documentation{} are part of the \href{manual}{\ecs{} User Manual}~\cite{manual} and correspond to Chapter ``\href{manual\##3}{#1}''.\alignright\mbox{\today}}
\parbox{4em}{\flushright\ecslogo{3em}}
\clearpage
}

\providecommand{\concludechapter}{
\vfill\pagebreak[3]\null\vfill
\thispagestyle{myheadings}\markright{REFERENCES}
\noindent\begin{minipage}{\textwidth}\begin{multicols}{2}[\section*{References}]
\renewcommand{\section}[2]{}\small\bibliography{references}
\end{multicols}\end{minipage}\end{document}
}

\providecommand{\startpresentation}[2]{
\documentclass[14pt,aspectratio=43,usepdftitle=false]{beamer}
\usepackage{booktabs}
\usepackage{etex}
\usepackage{multicol}
\usepackage{tikz}
\usepackage[all]{xy}
\bibliographystyle{unsrt}
\setlength{\columnsep}{1em}
\setlength{\leftmargini}{1em}
\setbeamercolor{title}{fg=black}
\setbeamercolor{structure}{fg=darkgray}
\setbeamercolor{bibliography item}{fg=darkgray}
\setbeamerfont{title}{series=\bfseries}
\setbeamerfont{subtitle}{series=\normalfont}
\setbeamerfont*{frametitle}{parent=title}
\setbeamerfont{block title}{series=\bfseries}
\setbeamerfont*{framesubtitle}{parent=subtitle}
\setbeamersize{text margin left=1em,text margin right=1em}
\setbeamertemplate{navigation symbols}{}
\setbeamertemplate{itemize item}[circle]{}
\setbeamertemplate{bibliography item}[triangle]{}
\setbeamertemplate{bibliography entry author}{\usebeamercolor[fg]{bibliography item}}
\setbeamertemplate{frametitle}{\medskip\usebeamerfont{frametitle}\color{gray}\raisebox{-2.5ex}[0ex][0ex]{\rule{0.1em}{4.5ex}}}
\addtobeamertemplate{frametitle}{}{\hspace{0.4em}\usebeamercolor[fg]{title}\insertframetitle\par\vspace{0.2ex}\hspace{0.5em}\usebeamerfont{framesubtitle}\insertframesubtitle}
\hypersetup{pdfborder={0 0 0},bookmarksnumbered=true,bookmarksopen=true,bookmarksopenlevel=0,pdftitle={\ecs{}: #1},pdfauthor={Florian Negele},pdfsubject={\ecs{}},pdfkeywords={#1}}
\renewcommand{\flowgraph}[1]{\resizebox{\textwidth}{!}{$$\xymatrix{##1}$$}}
\title{\ecs{}\medskip\hrule\medskip}
\institute{\shadowedecslogo{5em}{30}{15}}
\date{\version}
\subtitle{#1}
\begin{document}
\begin{frame}[plain]\titlepage\nocite{manual}\end{frame}
\begin{frame}{Contents}{#1}\begin{center}\tableofcontents\end{center}\end{frame}
}

\providecommand{\concludepresentation}{
\begin{frame}{References}\begin{footnotesize}\setlength{\columnseprule}{0.4pt}\begin{multicols}{2}\bibliography{references}\end{multicols}\end{footnotesize}\end{frame}
\end{document}
}

\providecommand{\startbook}[1]{
\documentclass[10pt,paper=17cm:24cm,DIV=13,twoside=semi,headings=normal,numbers=noendperiod,cleardoublepage=plain]{scrbook}
\usepackage{atveryend}
\usepackage{booktabs}
\usepackage{caption}
\usepackage{changepage}
\usepackage[T1]{fontenc}
\usepackage{imakeidx}
\usepackage{hyperref}
\usepackage[american]{isodate}
\usepackage{lmodern}
\usepackage{longtable}
\usepackage{mathptmx}
\usepackage[final]{microtype}
\usepackage{multicol}
\usepackage{multirow}
\usepackage[all]{nowidow}
\usepackage{pdfcomment}
\usepackage{scrlayer-scrpage}
\usepackage{setspace}
\usepackage{syntax}
\usepackage[eventxtindent=4pt,oddtxtexdent=4pt]{thumbs}
\usepackage{tikz}
\usepackage[all]{xy}
\hyphenation{Micro-Blaze Open-Cores Open-RISC Power-PC}
\hypersetup{pdfborder={0 0 0},bookmarksnumbered=true,bookmarksopen=true,bookmarksopenlevel=0,pdftitle={\ecs{}: #1},pdfauthor={Florian Negele},pdfsubject={\ecs{}},pdfkeywords={#1}}
\setlength{\grammarindent}{8em}\setlength{\grammarparsep}{0.7ex}
\setkomafont{captionlabel}{\usekomafont{descriptionlabel}}
\renewcommand{\arraystretch}{1.05}\setstretch{1.1}
\renewcommand{\chapterformat}{\thechapter\autodot\enskip\raisebox{-1ex}[0ex][0ex]{\color{gray}\rule{0.1em}{3.5ex}}\enskip}
\renewcommand{\startchapter}[4]{\hypertarget{##3}{\chapter{##1}}\label{##3}##4\addthumb{##1}{\LARGE\sffamily\bfseries\thechapter}{white}{gray}\renewcommand{\prefix}{##3}}
\renewcommand{\concludechapter}{\clearpage{\stopthumb\cleardoublepage}}
\renewcommand{\syntleft}{\itshape}\renewcommand{\syntright}{}
\renewcommand{\floatpagefraction}{0.7}
\renewcommand{\partheademptypage}{}
\DeclareMicrotypeAlias{lmss}{cmr}
\newcommand{\prefix}{}
\newcounter{instruction}
\bibliographystyle{unsrt}
\newif\ifbook\booktrue
\makeindex[intoc,title=Index]
\makeindex[intoc,name=tools,title=Index of Tools,columns=3]
\makeindex[intoc,name=library,title=Index of Library Names]
\makeindex[intoc,name=runtime,title=Index of Runtime Support]
\makeindex[intoc,name=environment,title=Index of Target Environments]
\indexsetup{toclevel=chapter,headers={\indexname}{\indexname}}
\frenchspacing
\begin{document}
\pagenumbering{alph}
\begin{titlepage}\centering
\huge\sffamily\null\vfill\textbf{\ecs{}}\bigskip\hrule\bigskip#1
\normalsize\normalfont\vfill\vfill\shadowedecslogo{10em}{30}{15}
\large\vfill\vfill\version
\end{titlepage}
\null\vfill
\thispagestyle{empty}
\noindent\today\par\medskip
\license A copy of this license is included in Appendix~\ref{fdl} on page~\pageref{fdl}.
All product names used herein are for identification purposes only and may be trademarks of their respective companies.
\concludechapter
\frontmatter
\setcounter{tocdepth}{1}
\tableofcontents
\setcounter{tocdepth}{2}
\concludechapter
\listoffigures
\concludechapter
\listoftables
\concludechapter
}

\providecommand{\concludebook}{
\backmatter
\addtocontents{toc}{\protect\setcounter{tocdepth}{-1}}
\phantomsection\addcontentsline{toc}{part}{Bibliography}
\bibliography{references}
\concludechapter
\phantomsection\addcontentsline{toc}{part}{Indexes}
\printindex
\concludechapter
\indexprologue{\label{idx:tools}}
\printindex[tools]
\concludechapter
\printindex[library]
\concludechapter
\indexprologue{\label{idx:runtime}}
\printindex[runtime]
\concludechapter
\indexprologue{\label{idx:environment}}
\printindex[environment]
\concludechapter
\pagestyle{empty}\pagenumbering{Alph}\null\clearpage
\null\vfill\centering\ecslogo{4em}\par\medskip\license
\end{document}
}

% chapter references

\providecommand{\seedocumentationref}{}\renewcommand{\seedocumentationref}[3]{#1, see \Documentation{}~\documentationref{#2}{#3}. }
\providecommand{\seeinterface}{}\renewcommand{\seeinterface}{\ifbook See \Documentation{}~\documentationref{interface}{User Interface} for more information about the common user interface of all of these tools. \fi}
\providecommand{\seeguide}{}\renewcommand{\seeguide}{\seedocumentationref{For basic examples of using some of these tools in practice}{guide}{User Guide}}
\providecommand{\seecpp}{}\renewcommand{\seecpp}{\seedocumentationref{For more information about the \cpp{} programming language and its implementation by the \ecs{}}{cpp}{User Manual for \cpp{}}}
\providecommand{\seefalse}{}\renewcommand{\seefalse}{\seedocumentationref{For more information about the FALSE programming language and its implementation by the \ecs{}}{false}{User Manual for FALSE}}
\providecommand{\seeoberon}{}\renewcommand{\seeoberon}{\seedocumentationref{For more information about the Oberon programming language and its implementation by the \ecs{}}{oberon}{User Manual for Oberon}}
\providecommand{\seeassembly}{}\renewcommand{\seeassembly}{\seedocumentationref{For more information about the generic assembly language and how to use it}{assembly}{Generic Assembly Language Specification}}
\providecommand{\seeamd}{}\renewcommand{\seeamd}{\seedocumentationref{For more information about how the \ecs{} supports the AMD64 hardware architecture}{amd64}{AMD64 Hardware Architecture Support}}
\providecommand{\seearm}{}\renewcommand{\seearm}{\seedocumentationref{For more information about how the \ecs{} supports the ARM hardware architecture}{arm}{ARM Hardware Architecture Support}}
\providecommand{\seeavr}{}\renewcommand{\seeavr}{\seedocumentationref{For more information about how the \ecs{} supports the AVR hardware architecture}{avr}{AVR Hardware Architecture Support}}
\providecommand{\seeavrtt}{}\renewcommand{\seeavrtt}{\seedocumentationref{For more information about how the \ecs{} supports the AVR32 hardware architecture}{avr32}{AVR32 Hardware Architecture Support}}
\providecommand{\seemabk}{}\renewcommand{\seemabk}{\seedocumentationref{For more information about how the \ecs{} supports the M68000 hardware architecture}{m68k}{M68000 Hardware Architecture Support}}
\providecommand{\seemibl}{}\renewcommand{\seemibl}{\seedocumentationref{For more information about how the \ecs{} supports the MicroBlaze hardware architecture}{mibl}{MicroBlaze Hardware Architecture Support}}
\providecommand{\seemips}{}\renewcommand{\seemips}{\seedocumentationref{For more information about how the \ecs{} supports the MIPS32 and MIPS64 hardware architectures}{mips}{MIPS Hardware Architecture Support}}
\providecommand{\seemmix}{}\renewcommand{\seemmix}{\seedocumentationref{For more information about how the \ecs{} supports the MMIX hardware architecture}{mmix}{MMIX Hardware Architecture Support}}
\providecommand{\seeorok}{}\renewcommand{\seeorok}{\seedocumentationref{For more information about how the \ecs{} supports the OpenRISC 1000 hardware architecture}{or1k}{OpenRISC 1000 Hardware Architecture Support}}
\providecommand{\seeppc}{}\renewcommand{\seeppc}{\seedocumentationref{For more information about how the \ecs{} supports the PowerPC hardware architecture}{ppc}{PowerPC Hardware Architecture Support}}
\providecommand{\seerisc}{}\renewcommand{\seerisc}{\seedocumentationref{For more information about how the \ecs{} supports the RISC hardware architecture}{risc}{RISC Hardware Architecture Support}}
\providecommand{\seewasm}{}\renewcommand{\seewasm}{\seedocumentationref{For more information about how the \ecs{} supports the WebAssembly architecture}{wasm}{WebAssembly Architecture Support}}
\providecommand{\seedocumentation}{}\renewcommand{\seedocumentation}{\seedocumentationref{For more information about generic documentations and their generation by the \ecs{}}{documentation}{Generic Documentation Generation}}
\providecommand{\seedebugging}{}\renewcommand{\seedebugging}{\seedocumentationref{For more information about debugging information and its representation}{debugging}{Debugging Information Representation}}
\providecommand{\seecode}{}\renewcommand{\seecode}{\seedocumentationref{For more information about intermediate code and its purpose}{code}{Intermediate Code Representation}}
\providecommand{\seeobject}{}\renewcommand{\seeobject}{\seedocumentationref{For more information about object files and their purpose}{object}{Object File Representation}}

% generic documentation tools

\providecommand{\docprint}{
\toolsection{docprint} is a pretty printer for generic documentations.
It reformats generic documentations and writes it to the standard output stream.
\debuggingtool
\flowgraph{\resource{generic\\documentation} \ar[r] & \toolbox{docprint} \ar[r] & \resource{generic\\documentation}}
\seedocumentation
}

\providecommand{\doccheck}{
\toolsection{doccheck} is a syntactic and semantic checker for generic documentations.
It just performs syntactic and semantic checks on generic documentations and writes its diagnostic messages to the standard error stream.
\debuggingtool
\flowgraph{\resource{generic\\documentation} \ar[r] & \toolbox{doccheck} \ar[r] & \resource{diagnostic\\messages}}
\seedocumentation
}

\providecommand{\dochtml}{
\toolsection{dochtml} is an HTML documentation generator for generic documentations.
It processes several generic documentations and assembles all information therein into an HTML document.
\debuggingtool
\flowgraph{\resource{generic\\documentation} \ar[r] & \toolbox{dochtml} \ar[r] & \resource{HTML\\document}}
\seedocumentation
}

\providecommand{\doclatex}{
\toolsection{doclatex} is a Latex documentation generator for generic documentations.
It processes several generic documentations and assembles all information therein into a Latex document.
\debuggingtool
\flowgraph{\resource{generic\\documentation} \ar[r] & \toolbox{doclatex} \ar[r] & \resource{Latex\\document}}
\seedocumentation
}

% intermediate code tools

\providecommand{\cdcheck}{
\toolsection{cdcheck} is a syntactic and semantic checker for intermediate code.
It just performs syntactic and semantic checks on programs written in intermediate code and writes its diagnostic messages to the standard error stream.
\debuggingtool
\flowgraph{\resource{intermediate\\code} \ar[r] & \toolbox{cdcheck} \ar[r] & \resource{diagnostic\\messages}}
\seeassembly\seecode
}

\providecommand{\cdopt}{
\toolsection{cdopt} is an optimizer for intermediate code.
It performs various optimizations on programs written in intermediate code and writes the result to the standard output stream.
\debuggingtool
\flowgraph{\resource{intermediate\\code} \ar[r] & \toolbox{cdopt} \ar[r] & \resource{optimized\\code}}
\seeassembly\seecode
}

\providecommand{\cdrun}{
\toolsection{cdrun} is an interpreter for intermediate code.
It processes and executes programs written in intermediate code.
The following code sections are predefined and have the usual semantics:
\texttt{abort}, \texttt{\_Exit}, \texttt{fflush}, \texttt{floor}, \texttt{fputc}, \texttt{free}, \texttt{getchar}, \texttt{malloc}, and \texttt{putchar}.
Diagnostic messages about invalid operations include the name of the executed code section and the index of the erroneous instruction.
\debuggingtool
\flowgraph{\resource{intermediate\\code} \ar[r] & \toolbox{cdrun} \ar@/u/[r] & \resource{input/\\output} \ar@/d/[l]}
\seeassembly\seecode
}

\providecommand{\cdamda}{
\toolsection{cdamd16} is a compiler for intermediate code targeting the AMD64 hardware architecture.
It generates machine code for AMD64 processors from programs written in intermediate code and stores it in corresponding object files.
The compiler generates machine code for the 16-bit operating mode defined by the AMD64 architecture.
It also creates a debugging information file as well as an assembly file containing a listing of the generated machine code.
\debuggingtool
\flowgraph{\resource{intermediate\\code} \ar[r] & \toolbox{cdamd16} \ar[r] \ar[d] \ar[rd] & \resource{object file} \\ & \resource{assembly\\listing} & \resource{debugging\\information}}
\seeassembly\seeamd\seeobject\seecode\seedebugging
}

\providecommand{\cdamdb}{
\toolsection{cdamd32} is a compiler for intermediate code targeting the AMD64 hardware architecture.
It generates machine code for AMD64 processors from programs written in intermediate code and stores it in corresponding object files.
The compiler generates machine code for the 32-bit operating mode defined by the AMD64 architecture.
It also creates a debugging information file as well as an assembly file containing a listing of the generated machine code.
\debuggingtool
\flowgraph{\resource{intermediate\\code} \ar[r] & \toolbox{cdamd32} \ar[r] \ar[d] \ar[rd] & \resource{object file} \\ & \resource{assembly\\listing} & \resource{debugging\\information}}
\seeassembly\seeamd\seeobject\seecode\seedebugging
}

\providecommand{\cdamdc}{
\toolsection{cdamd64} is a compiler for intermediate code targeting the AMD64 hardware architecture.
It generates machine code for AMD64 processors from programs written in intermediate code and stores it in corresponding object files.
The compiler generates machine code for the 64-bit operating mode defined by the AMD64 architecture.
It also creates a debugging information file as well as an assembly file containing a listing of the generated machine code.
\debuggingtool
\flowgraph{\resource{intermediate\\code} \ar[r] & \toolbox{cdamd64} \ar[r] \ar[d] \ar[rd] & \resource{object file} \\ & \resource{assembly\\listing} & \resource{debugging\\information}}
\seeassembly\seeamd\seeobject\seecode\seedebugging
}

\providecommand{\cdarma}{
\toolsection{cdarma32} is a compiler for intermediate code targeting the ARM hardware architecture.
It generates machine code for ARM processors executing A32 instructions from programs written in intermediate code and stores it in corresponding object files.
It also creates a debugging information file as well as an assembly file containing a listing of the generated machine code.
\debuggingtool
\flowgraph{\resource{intermediate\\code} \ar[r] & \toolbox{cdarma32} \ar[r] \ar[d] \ar[rd] & \resource{object file} \\ & \resource{assembly\\listing} & \resource{debugging\\information}}
\seeassembly\seearm\seeobject\seecode\seedebugging
}

\providecommand{\cdarmb}{
\toolsection{cdarma64} is a compiler for intermediate code targeting the ARM hardware architecture.
It generates machine code for ARM processors executing A64 instructions from programs written in intermediate code and stores it in corresponding object files.
It also creates a debugging information file as well as an assembly file containing a listing of the generated machine code.
\debuggingtool
\flowgraph{\resource{intermediate\\code} \ar[r] & \toolbox{cdarma64} \ar[r] \ar[d] \ar[rd] & \resource{object file} \\ & \resource{assembly\\listing} & \resource{debugging\\information}}
\seeassembly\seearm\seeobject\seecode\seedebugging
}

\providecommand{\cdarmc}{
\toolsection{cdarmt32} is a compiler for intermediate code targeting the ARM hardware architecture.
It generates machine code for ARM processors without floating-point extension executing T32 instructions from programs written in intermediate code and stores it in corresponding object files.
It also creates a debugging information file as well as an assembly file containing a listing of the generated machine code.
\debuggingtool
\flowgraph{\resource{intermediate\\code} \ar[r] & \toolbox{cdarmt32} \ar[r] \ar[d] \ar[rd] & \resource{object file} \\ & \resource{assembly\\listing} & \resource{debugging\\information}}
\seeassembly\seearm\seeobject\seecode\seedebugging
}

\providecommand{\cdarmcfpe}{
\toolsection{cdarmt32fpe} is a compiler for intermediate code targeting the ARM hardware architecture.
It generates machine code for ARM processors with floating-point extension executing T32 instructions from programs written in intermediate code and stores it in corresponding object files.
It also creates a debugging information file as well as an assembly file containing a listing of the generated machine code.
\debuggingtool
\flowgraph{\resource{intermediate\\code} \ar[r] & \toolbox{cdarmt32fpe} \ar[r] \ar[d] \ar[rd] & \resource{object file} \\ & \resource{assembly\\listing} & \resource{debugging\\information}}
\seeassembly\seearm\seeobject\seecode\seedebugging
}

\providecommand{\cdavr}{
\toolsection{cdavr} is a compiler for intermediate code targeting the AVR hardware architecture.
It generates machine code for AVR processors from programs written in intermediate code and stores it in corresponding object files.
It also creates a debugging information file as well as an assembly file containing a listing of the generated machine code.
\debuggingtool
\flowgraph{\resource{intermediate\\code} \ar[r] & \toolbox{cdavr} \ar[r] \ar[d] \ar[rd] & \resource{object file} \\ & \resource{assembly\\listing} & \resource{debugging\\information}}
\seeassembly\seeavr\seeobject\seecode\seedebugging
}

\providecommand{\cdavrtt}{
\toolsection{cdavr32} is a compiler for intermediate code targeting the AVR32 hardware architecture.
It generates machine code for AVR32 processors from programs written in intermediate code and stores it in corresponding object files.
It also creates a debugging information file as well as an assembly file containing a listing of the generated machine code.
\debuggingtool
\flowgraph{\resource{intermediate\\code} \ar[r] & \toolbox{cdavr32} \ar[r] \ar[d] \ar[rd] & \resource{object file} \\ & \resource{assembly\\listing} & \resource{debugging\\information}}
\seeassembly\seeavrtt\seeobject\seecode\seedebugging
}

\providecommand{\cdmabk}{
\toolsection{cdm68k} is a compiler for intermediate code targeting the M68000 hardware architecture.
It generates machine code for M68000 processors from programs written in intermediate code and stores it in corresponding object files.
It also creates a debugging information file as well as an assembly file containing a listing of the generated machine code.
\debuggingtool
\flowgraph{\resource{intermediate\\code} \ar[r] & \toolbox{cdm68k} \ar[r] \ar[d] \ar[rd] & \resource{object file} \\ & \resource{assembly\\listing} & \resource{debugging\\information}}
\seeassembly\seemabk\seeobject\seecode\seedebugging
}

\providecommand{\cdmibl}{
\toolsection{cdmibl} is a compiler for intermediate code targeting the MicroBlaze hardware architecture.
It generates machine code for MicroBlaze processors from programs written in intermediate code and stores it in corresponding object files.
It also creates a debugging information file as well as an assembly file containing a listing of the generated machine code.
\debuggingtool
\flowgraph{\resource{intermediate\\code} \ar[r] & \toolbox{cdmibl} \ar[r] \ar[d] \ar[rd] & \resource{object file} \\ & \resource{assembly\\listing} & \resource{debugging\\information}}
\seeassembly\seemibl\seeobject\seecode\seedebugging
}

\providecommand{\cdmipsa}{
\toolsection{cdmips32} is a compiler for intermediate code targeting the MIPS32 hardware architecture.
It generates machine code for MIPS32 processors from programs written in intermediate code and stores it in corresponding object files.
It also creates a debugging information file as well as an assembly file containing a listing of the generated machine code.
\debuggingtool
\flowgraph{\resource{intermediate\\code} \ar[r] & \toolbox{cdmips32} \ar[r] \ar[d] \ar[rd] & \resource{object file} \\ & \resource{assembly\\listing} & \resource{debugging\\information}}
\seeassembly\seemips\seeobject\seecode\seedebugging
}

\providecommand{\cdmipsb}{
\toolsection{cdmips64} is a compiler for intermediate code targeting the MIPS64 hardware architecture.
It generates machine code for MIPS64 processors from programs written in intermediate code and stores it in corresponding object files.
It also creates a debugging information file as well as an assembly file containing a listing of the generated machine code.
\debuggingtool
\flowgraph{\resource{intermediate\\code} \ar[r] & \toolbox{cdmips64} \ar[r] \ar[d] \ar[rd] & \resource{object file} \\ & \resource{assembly\\listing} & \resource{debugging\\information}}
\seeassembly\seemips\seeobject\seecode\seedebugging
}

\providecommand{\cdmmix}{
\toolsection{cdmmix} is a compiler for intermediate code targeting the MMIX hardware architecture.
It generates machine code for MMIX processors from programs written in intermediate code and stores it in corresponding object files.
It also creates a debugging information file as well as an assembly file containing a listing of the generated machine code.
\debuggingtool
\flowgraph{\resource{intermediate\\code} \ar[r] & \toolbox{cdmmix} \ar[r] \ar[d] \ar[rd] & \resource{object file} \\ & \resource{assembly\\listing} & \resource{debugging\\information}}
\seeassembly\seemmix\seeobject\seecode\seedebugging
}

\providecommand{\cdorok}{
\toolsection{cdor1k} is a compiler for intermediate code targeting the OpenRISC 1000 hardware architecture.
It generates machine code for OpenRISC 1000 processors from programs written in intermediate code and stores it in corresponding object files.
It also creates a debugging information file as well as an assembly file containing a listing of the generated machine code.
\debuggingtool
\flowgraph{\resource{intermediate\\code} \ar[r] & \toolbox{cdor1k} \ar[r] \ar[d] \ar[rd] & \resource{object file} \\ & \resource{assembly\\listing} & \resource{debugging\\information}}
\seeassembly\seeorok\seeobject\seecode\seedebugging
}

\providecommand{\cdppca}{
\toolsection{cdppc32} is a compiler for intermediate code targeting the PowerPC hardware architecture.
It generates machine code for PowerPC processors from programs written in intermediate code and stores it in corresponding object files.
The compiler generates machine code for the 32-bit operating mode defined by the PowerPC architecture.
It also creates a debugging information file as well as an assembly file containing a listing of the generated machine code.
\debuggingtool
\flowgraph{\resource{intermediate\\code} \ar[r] & \toolbox{cdppc32} \ar[r] \ar[d] \ar[rd] & \resource{object file} \\ & \resource{assembly\\listing} & \resource{debugging\\information}}
\seeassembly\seeppc\seeobject\seecode\seedebugging
}

\providecommand{\cdppcb}{
\toolsection{cdppc64} is a compiler for intermediate code targeting the PowerPC hardware architecture.
It generates machine code for PowerPC processors from programs written in intermediate code and stores it in corresponding object files.
The compiler generates machine code for the 64-bit operating mode defined by the PowerPC architecture.
It also creates a debugging information file as well as an assembly file containing a listing of the generated machine code.
\debuggingtool
\flowgraph{\resource{intermediate\\code} \ar[r] & \toolbox{cdppc64} \ar[r] \ar[d] \ar[rd] & \resource{object file} \\ & \resource{assembly\\listing} & \resource{debugging\\information}}
\seeassembly\seeppc\seeobject\seecode\seedebugging
}

\providecommand{\cdrisc}{
\toolsection{cdrisc} is a compiler for intermediate code targeting the RISC hardware architecture.
It generates machine code for RISC processors from programs written in intermediate code and stores it in corresponding object files.
It also creates a debugging information file as well as an assembly file containing a listing of the generated machine code.
\debuggingtool
\flowgraph{\resource{intermediate\\code} \ar[r] & \toolbox{cdrisc} \ar[r] \ar[d] \ar[rd] & \resource{object file} \\ & \resource{assembly\\listing} & \resource{debugging\\information}}
\seeassembly\seerisc\seeobject\seecode\seedebugging
}

\providecommand{\cdwasm}{
\toolsection{cdwasm} is a compiler for intermediate code targeting the WebAssembly architecture.
It generates machine code for WebAssembly targets from programs written in intermediate code and stores it in corresponding object files.
It also creates a debugging information file as well as an assembly file containing a listing of the generated machine code.
\debuggingtool
\flowgraph{\resource{intermediate\\code} \ar[r] & \toolbox{cdwasm} \ar[r] \ar[d] \ar[rd] & \resource{object file} \\ & \resource{assembly\\listing} & \resource{debugging\\information}}
\seeassembly\seewasm\seeobject\seecode\seedebugging
}

% C++ tools

\providecommand{\cppprep}{
\toolsection{cppprep} is a preprocessor for the \cpp{} programming language.
It preprocesses source code according to the rules of \cpp{} and writes it to the standard output stream.
Only the macro names \texttt{\_\_DATE\_\_}, \texttt{\_\_FILE\_\_}, \texttt{\_\_LINE\_\_}, and \texttt{\_\_TIME\_\_} are predefined.
\flowgraph{\resource{\cpp{} or other\\source code} \ar[r] & \toolbox{cppprep} \ar[r] & \resource{preprocessed\\source code} \\ & \variable{ECSINCLUDE} \ar[u]}
\seecpp
}

\providecommand{\cppprint}{
\toolsection{cppprint} is a pretty printer for the \cpp{} programming language.
It reformats the source code of \cpp{} programs and writes it to the standard output stream.
\flowgraph{\resource{\cpp{}\\source code} \ar[r] & \toolbox{cppprint} \ar[r] & \resource{reformatted\\source code} \\ & \variable{ECSINCLUDE} \ar[u]}
\seecpp
}

\providecommand{\cppcheck}{
\toolsection{cppcheck} is a syntactic and semantic checker for the \cpp{} programming language.
It just performs syntactic and semantic checks on \cpp{} programs and writes its diagnostic messages to the standard error stream.
\flowgraph{\resource{\cpp{}\\source code} \ar[r] & \toolbox{cppcheck} \ar[r] & \resource{diagnostic\\messages} \\ & \variable{ECSINCLUDE} \ar[u]}
\seecpp
}

\providecommand{\cppdump}{
\toolsection{cppdump} is a serializer for the \cpp{} programming language.
It dumps the complete internal representation of programs written in \cpp{} into an XML document.
\debuggingtool
\flowgraph{\resource{\cpp{}\\source code} \ar[r] & \toolbox{cppdump} \ar[r] & \resource{internal\\representation} \\ & \variable{ECSINCLUDE} \ar[u]}
\seecpp
}

\providecommand{\cpprun}{
\toolsection{cpprun} is an interpreter for the \cpp{} programming language.
It processes and executes programs written in \cpp{}.
The macro \texttt{\_\_run\_\_} is predefined in order to enable programmers to identify this tool while interpreting.
\flowgraph{\resource{\cpp{}\\source code} \ar[r] & \toolbox{cpprun} \ar@/u/[r] & \resource{input/\\output} \ar@/d/[l] \\ & \variable{ECSINCLUDE} \ar[u]}
\seecpp
}

\providecommand{\cppdoc}{
\toolsection{cppdoc} is a generic documentation generator for the \cpp{} programming language.
It processes several \cpp{} source files and assembles all information therein into a generic documentation.
\debuggingtool
\flowgraph{\resource{\cpp{}\\source code} \ar[r] & \toolbox{cppdoc} \ar[r] & \resource{generic\\documentation} \\ & \variable{ECSINCLUDE} \ar[u]}
\seecpp\seedocumentation
}

\providecommand{\cpphtml}{
\toolsection{cpphtml} is an HTML documentation generator for the \cpp{} programming language.
It processes several \cpp{} source files and assembles all information therein into an HTML document.
\flowgraph{\resource{\cpp{}\\source code} \ar[r] & \toolbox{cpphtml} \ar[r] & \resource{HTML\\document} \\ & \variable{ECSINCLUDE} \ar[u]}
\seecpp\seedocumentation
}

\providecommand{\cpplatex}{
\toolsection{cpplatex} is a Latex documentation generator for the \cpp{} programming language.
It processes several \cpp{} source files and assembles all information therein into a Latex document.
\flowgraph{\resource{\cpp{}\\source code} \ar[r] & \toolbox{cpplatex} \ar[r] & \resource{Latex\\document} \\ & \variable{ECSINCLUDE} \ar[u]}
\seecpp\seedocumentation
}

\providecommand{\cppcode}{
\toolsection{cppcode} is an intermediate code generator for the \cpp{} programming language.
It generates intermediate code from programs written in \cpp{} and stores it in corresponding assembly files.
The macro \texttt{\_\_code\_\_} is predefined in order to enable programmers to identify this tool while generating intermediate code.
Programs generated with this tool require additional runtime support that is stored in the \file{cpp\-code\-run} library file.
\debuggingtool
\flowgraph{\resource{\cpp{}\\source code} \ar[r] & \toolbox{cppcode} \ar[r] & \resource{intermediate\\code} \\ & \variable{ECSINCLUDE} \ar[u]}
\seecpp\seeassembly\seecode
}

\providecommand{\cppamda}{
\toolsection{cppamd16} is a compiler for the \cpp{} programming language targeting the AMD64 hardware architecture.
It generates machine code for AMD64 processors from programs written in \cpp{} and stores it in corresponding object files.
The compiler generates machine code for the 16-bit operating mode defined by the AMD64 architecture.
For debugging purposes, it also creates a debugging information file as well as an assembly file containing a listing of the generated machine code.
The macro \texttt{\_\_amd16\_\_} is predefined in order to enable programmers to identify this tool and its target architecture while compiling.
Programs generated with this compiler require additional runtime support that is stored in the \file{cpp\-amd16\-run} library file.
\flowgraph{\resource{\cpp{}\\source code} \ar[r] & \toolbox{cppamd16} \ar[r] \ar[d] \ar[rd] & \resource{object file} \\ \variable{ECSINCLUDE} \ar[ru] & \resource{debugging\\information} & \resource{assembly\\listing}}
\seecpp\seeassembly\seeamd\seeobject\seedebugging
}

\providecommand{\cppamdb}{
\toolsection{cppamd32} is a compiler for the \cpp{} programming language targeting the AMD64 hardware architecture.
It generates machine code for AMD64 processors from programs written in \cpp{} and stores it in corresponding object files.
The compiler generates machine code for the 32-bit operating mode defined by the AMD64 architecture.
For debugging purposes, it also creates a debugging information file as well as an assembly file containing a listing of the generated machine code.
The macro \texttt{\_\_amd32\_\_} is predefined in order to enable programmers to identify this tool and its target architecture while compiling.
Programs generated with this compiler require additional runtime support that is stored in the \file{cpp\-amd32\-run} library file.
\flowgraph{\resource{\cpp{}\\source code} \ar[r] & \toolbox{cppamd32} \ar[r] \ar[d] \ar[rd] & \resource{object file} \\ \variable{ECSINCLUDE} \ar[ru] & \resource{debugging\\information} & \resource{assembly\\listing}}
\seecpp\seeassembly\seeamd\seeobject\seedebugging
}

\providecommand{\cppamdc}{
\toolsection{cppamd64} is a compiler for the \cpp{} programming language targeting the AMD64 hardware architecture.
It generates machine code for AMD64 processors from programs written in \cpp{} and stores it in corresponding object files.
The compiler generates machine code for the 64-bit operating mode defined by the AMD64 architecture.
For debugging purposes, it also creates a debugging information file as well as an assembly file containing a listing of the generated machine code.
The macro \texttt{\_\_amd64\_\_} is predefined in order to enable programmers to identify this tool and its target architecture while compiling.
Programs generated with this compiler require additional runtime support that is stored in the \file{cpp\-amd64\-run} library file.
\flowgraph{\resource{\cpp{}\\source code} \ar[r] & \toolbox{cppamd64} \ar[r] \ar[d] \ar[rd] & \resource{object file} \\ \variable{ECSINCLUDE} \ar[ru] & \resource{debugging\\information} & \resource{assembly\\listing}}
\seecpp\seeassembly\seeamd\seeobject\seedebugging
}

\providecommand{\cpparma}{
\toolsection{cpparma32} is a compiler for the \cpp{} programming language targeting the ARM hardware architecture.
It generates machine code for ARM processors executing A32 instructions from programs written in \cpp{} and stores it in corresponding object files.
For debugging purposes, it also creates a debugging information file as well as an assembly file containing a listing of the generated machine code.
The macro \texttt{\_\_arma32\_\_} is predefined in order to enable programmers to identify this tool and its target architecture while compiling.
Programs generated with this compiler require additional runtime support that is stored in the \file{cpp\-arma32\-run} library file.
\flowgraph{\resource{\cpp{}\\source code} \ar[r] & \toolbox{cpparma32} \ar[r] \ar[d] \ar[rd] & \resource{object file} \\ \variable{ECSINCLUDE} \ar[ru] & \resource{debugging\\information} & \resource{assembly\\listing}}
\seecpp\seeassembly\seearm\seeobject\seedebugging
}

\providecommand{\cpparmb}{
\toolsection{cpparma64} is a compiler for the \cpp{} programming language targeting the ARM hardware architecture.
It generates machine code for ARM processors executing A64 instructions from programs written in \cpp{} and stores it in corresponding object files.
For debugging purposes, it also creates a debugging information file as well as an assembly file containing a listing of the generated machine code.
The macro \texttt{\_\_arma64\_\_} is predefined in order to enable programmers to identify this tool and its target architecture while compiling.
Programs generated with this compiler require additional runtime support that is stored in the \file{cpp\-arma64\-run} library file.
\flowgraph{\resource{\cpp{}\\source code} \ar[r] & \toolbox{cpparma64} \ar[r] \ar[d] \ar[rd] & \resource{object file} \\ \variable{ECSINCLUDE} \ar[ru] & \resource{debugging\\information} & \resource{assembly\\listing}}
\seecpp\seeassembly\seearm\seeobject\seedebugging
}

\providecommand{\cpparmc}{
\toolsection{cpparmt32} is a compiler for the \cpp{} programming language targeting the ARM hardware architecture.
It generates machine code for ARM processors without floating-point extension executing T32 instructions from programs written in \cpp{} and stores it in corresponding object files.
For debugging purposes, it also creates a debugging information file as well as an assembly file containing a listing of the generated machine code.
The macro \texttt{\_\_armt32\_\_} is predefined in order to enable programmers to identify this tool and its target architecture while compiling.
Programs generated with this compiler require additional runtime support that is stored in the \file{cpp\-armt32\-run} library file.
\flowgraph{\resource{\cpp{}\\source code} \ar[r] & \toolbox{cpparmt32} \ar[r] \ar[d] \ar[rd] & \resource{object file} \\ \variable{ECSINCLUDE} \ar[ru] & \resource{debugging\\information} & \resource{assembly\\listing}}
\seecpp\seeassembly\seearm\seeobject\seedebugging
}

\providecommand{\cpparmcfpe}{
\toolsection{cpparmt32fpe} is a compiler for the \cpp{} programming language targeting the ARM hardware architecture.
It generates machine code for ARM processors with floating-point extension executing T32 instructions from programs written in \cpp{} and stores it in corresponding object files.
For debugging purposes, it also creates a debugging information file as well as an assembly file containing a listing of the generated machine code.
The macro \texttt{\_\_armt32fpe\_\_} is predefined in order to enable programmers to identify this tool and its target architecture while compiling.
Programs generated with this compiler require additional runtime support that is stored in the \file{cpp\-armt32\-fpe\-run} library file.
\flowgraph{\resource{\cpp{}\\source code} \ar[r] & \toolbox{cpparmt32fpe} \ar[r] \ar[d] \ar[rd] & \resource{object file} \\ \variable{ECSINCLUDE} \ar[ru] & \resource{debugging\\information} & \resource{assembly\\listing}}
\seecpp\seeassembly\seearm\seeobject\seedebugging
}

\providecommand{\cppavr}{
\toolsection{cppavr} is a compiler for the \cpp{} programming language targeting the AVR hardware architecture.
It generates machine code for AVR processors from programs written in \cpp{} and stores it in corresponding object files.
For debugging purposes, it also creates a debugging information file as well as an assembly file containing a listing of the generated machine code.
The macro \texttt{\_\_avr\_\_} is predefined in order to enable programmers to identify this tool and its target architecture while compiling.
Programs generated with this compiler require additional runtime support that is stored in the \file{cpp\-avr\-run} library file.
\flowgraph{\resource{\cpp{}\\source code} \ar[r] & \toolbox{cppavr} \ar[r] \ar[d] \ar[rd] & \resource{object file} \\ \variable{ECSINCLUDE} \ar[ru] & \resource{debugging\\information} & \resource{assembly\\listing}}
\seecpp\seeassembly\seeavr\seeobject\seedebugging
}

\providecommand{\cppavrtt}{
\toolsection{cppavr32} is a compiler for the \cpp{} programming language targeting the AVR32 hardware architecture.
It generates machine code for AVR32 processors from programs written in \cpp{} and stores it in corresponding object files.
For debugging purposes, it also creates a debugging information file as well as an assembly file containing a listing of the generated machine code.
The macro \texttt{\_\_avr32\_\_} is predefined in order to enable programmers to identify this tool and its target architecture while compiling.
Programs generated with this compiler require additional runtime support that is stored in the \file{cpp\-avr32\-run} library file.
\flowgraph{\resource{\cpp{}\\source code} \ar[r] & \toolbox{cppavr32} \ar[r] \ar[d] \ar[rd] & \resource{object file} \\ \variable{ECSINCLUDE} \ar[ru] & \resource{debugging\\information} & \resource{assembly\\listing}}
\seecpp\seeassembly\seeavrtt\seeobject\seedebugging
}

\providecommand{\cppmabk}{
\toolsection{cppm68k} is a compiler for the \cpp{} programming language targeting the M68000 hardware architecture.
It generates machine code for M68000 processors from programs written in \cpp{} and stores it in corresponding object files.
For debugging purposes, it also creates a debugging information file as well as an assembly file containing a listing of the generated machine code.
The macro \texttt{\_\_m68k\_\_} is predefined in order to enable programmers to identify this tool and its target architecture while compiling.
Programs generated with this compiler require additional runtime support that is stored in the \file{cpp\-m68k\-run} library file.
\flowgraph{\resource{\cpp{}\\source code} \ar[r] & \toolbox{cppm68k} \ar[r] \ar[d] \ar[rd] & \resource{object file} \\ \variable{ECSINCLUDE} \ar[ru] & \resource{debugging\\information} & \resource{assembly\\listing}}
\seecpp\seeassembly\seemabk\seeobject\seedebugging
}

\providecommand{\cppmibl}{
\toolsection{cppmibl} is a compiler for the \cpp{} programming language targeting the MicroBlaze hardware architecture.
It generates machine code for MicroBlaze processors from programs written in \cpp{} and stores it in corresponding object files.
For debugging purposes, it also creates a debugging information file as well as an assembly file containing a listing of the generated machine code.
The macro \texttt{\_\_mibl\_\_} is predefined in order to enable programmers to identify this tool and its target architecture while compiling.
Programs generated with this compiler require additional runtime support that is stored in the \file{cpp\-mibl\-run} library file.
\flowgraph{\resource{\cpp{}\\source code} \ar[r] & \toolbox{cppmibl} \ar[r] \ar[d] \ar[rd] & \resource{object file} \\ \variable{ECSINCLUDE} \ar[ru] & \resource{debugging\\information} & \resource{assembly\\listing}}
\seecpp\seeassembly\seemibl\seeobject\seedebugging
}

\providecommand{\cppmipsa}{
\toolsection{cppmips32} is a compiler for the \cpp{} programming language targeting the MIPS32 hardware architecture.
It generates machine code for MIPS32 processors from programs written in \cpp{} and stores it in corresponding object files.
For debugging purposes, it also creates a debugging information file as well as an assembly file containing a listing of the generated machine code.
The macro \texttt{\_\_mips32\_\_} is predefined in order to enable programmers to identify this tool and its target architecture while compiling.
Programs generated with this compiler require additional runtime support that is stored in the \file{cpp\-mips32\-run} library file.
\flowgraph{\resource{\cpp{}\\source code} \ar[r] & \toolbox{cppmips32} \ar[r] \ar[d] \ar[rd] & \resource{object file} \\ \variable{ECSINCLUDE} \ar[ru] & \resource{debugging\\information} & \resource{assembly\\listing}}
\seecpp\seeassembly\seemips\seeobject\seedebugging
}

\providecommand{\cppmipsb}{
\toolsection{cppmips64} is a compiler for the \cpp{} programming language targeting the MIPS64 hardware architecture.
It generates machine code for MIPS64 processors from programs written in \cpp{} and stores it in corresponding object files.
For debugging purposes, it also creates a debugging information file as well as an assembly file containing a listing of the generated machine code.
The macro \texttt{\_\_mips64\_\_} is predefined in order to enable programmers to identify this tool and its target architecture while compiling.
Programs generated with this compiler require additional runtime support that is stored in the \file{cpp\-mips64\-run} library file.
\flowgraph{\resource{\cpp{}\\source code} \ar[r] & \toolbox{cppmips64} \ar[r] \ar[d] \ar[rd] & \resource{object file} \\ \variable{ECSINCLUDE} \ar[ru] & \resource{debugging\\information} & \resource{assembly\\listing}}
\seecpp\seeassembly\seemips\seeobject\seedebugging
}

\providecommand{\cppmmix}{
\toolsection{cppmmix} is a compiler for the \cpp{} programming language targeting the MMIX hardware architecture.
It generates machine code for MMIX processors from programs written in \cpp{} and stores it in corresponding object files.
For debugging purposes, it also creates a debugging information file as well as an assembly file containing a listing of the generated machine code.
The macro \texttt{\_\_mmix\_\_} is predefined in order to enable programmers to identify this tool and its target architecture while compiling.
Programs generated with this compiler require additional runtime support that is stored in the \file{cpp\-mmix\-run} library file.
\flowgraph{\resource{\cpp{}\\source code} \ar[r] & \toolbox{cppmmix} \ar[r] \ar[d] \ar[rd] & \resource{object file} \\ \variable{ECSINCLUDE} \ar[ru] & \resource{debugging\\information} & \resource{assembly\\listing}}
\seecpp\seeassembly\seemmix\seeobject\seedebugging
}

\providecommand{\cpporok}{
\toolsection{cppor1k} is a compiler for the \cpp{} programming language targeting the OpenRISC 1000 hardware architecture.
It generates machine code for OpenRISC 1000 processors from programs written in \cpp{} and stores it in corresponding object files.
For debugging purposes, it also creates a debugging information file as well as an assembly file containing a listing of the generated machine code.
The macro \texttt{\_\_or1k\_\_} is predefined in order to enable programmers to identify this tool and its target architecture while compiling.
Programs generated with this compiler require additional runtime support that is stored in the \file{cpp\-or1k\-run} library file.
\flowgraph{\resource{\cpp{}\\source code} \ar[r] & \toolbox{cppor1k} \ar[r] \ar[d] \ar[rd] & \resource{object file} \\ \variable{ECSINCLUDE} \ar[ru] & \resource{debugging\\information} & \resource{assembly\\listing}}
\seecpp\seeassembly\seeorok\seeobject\seedebugging
}

\providecommand{\cppppca}{
\toolsection{cppppc32} is a compiler for the \cpp{} programming language targeting the PowerPC hardware architecture.
It generates machine code for PowerPC processors from programs written in \cpp{} and stores it in corresponding object files.
The compiler generates machine code for the 32-bit operating mode defined by the PowerPC architecture.
For debugging purposes, it also creates a debugging information file as well as an assembly file containing a listing of the generated machine code.
The macro \texttt{\_\_ppc32\_\_} is predefined in order to enable programmers to identify this tool and its target architecture while compiling.
Programs generated with this compiler require additional runtime support that is stored in the \file{cpp\-ppc32\-run} library file.
\flowgraph{\resource{\cpp{}\\source code} \ar[r] & \toolbox{cppppc32} \ar[r] \ar[d] \ar[rd] & \resource{object file} \\ \variable{ECSINCLUDE} \ar[ru] & \resource{debugging\\information} & \resource{assembly\\listing}}
\seecpp\seeassembly\seeppc\seeobject\seedebugging
}

\providecommand{\cppppcb}{
\toolsection{cppppc64} is a compiler for the \cpp{} programming language targeting the PowerPC hardware architecture.
It generates machine code for PowerPC processors from programs written in \cpp{} and stores it in corresponding object files.
The compiler generates machine code for the 64-bit operating mode defined by the PowerPC architecture.
For debugging purposes, it also creates a debugging information file as well as an assembly file containing a listing of the generated machine code.
The macro \texttt{\_\_ppc64\_\_} is predefined in order to enable programmers to identify this tool and its target architecture while compiling.
Programs generated with this compiler require additional runtime support that is stored in the \file{cpp\-ppc64\-run} library file.
\flowgraph{\resource{\cpp{}\\source code} \ar[r] & \toolbox{cppppc64} \ar[r] \ar[d] \ar[rd] & \resource{object file} \\ \variable{ECSINCLUDE} \ar[ru] & \resource{debugging\\information} & \resource{assembly\\listing}}
\seecpp\seeassembly\seeppc\seeobject\seedebugging
}

\providecommand{\cpprisc}{
\toolsection{cpprisc} is a compiler for the \cpp{} programming language targeting the RISC hardware architecture.
It generates machine code for RISC processors from programs written in \cpp{} and stores it in corresponding object files.
For debugging purposes, it also creates a debugging information file as well as an assembly file containing a listing of the generated machine code.
The macro \texttt{\_\_risc\_\_} is predefined in order to enable programmers to identify this tool and its target architecture while compiling.
Programs generated with this compiler require additional runtime support that is stored in the \file{cpp\-risc\-run} library file.
\flowgraph{\resource{\cpp{}\\source code} \ar[r] & \toolbox{cpprisc} \ar[r] \ar[d] \ar[rd] & \resource{object file} \\ \variable{ECSINCLUDE} \ar[ru] & \resource{debugging\\information} & \resource{assembly\\listing}}
\seecpp\seeassembly\seerisc\seeobject\seedebugging
}

\providecommand{\cppwasm}{
\toolsection{cppwasm} is a compiler for the \cpp{} programming language targeting the WebAssembly architecture.
It generates machine code for WebAssembly targets from programs written in \cpp{} and stores it in corresponding object files.
For debugging purposes, it also creates a debugging information file as well as an assembly file containing a listing of the generated machine code.
The macro \texttt{\_\_wasm\_\_} is predefined in order to enable programmers to identify this tool and its target architecture while compiling.
Programs generated with this compiler require additional runtime support that is stored in the \file{cpp\-wasm\-run} library file.
\flowgraph{\resource{\cpp{}\\source code} \ar[r] & \toolbox{cppwasm} \ar[r] \ar[d] \ar[rd] & \resource{object file} \\ \variable{ECSINCLUDE} \ar[ru] & \resource{debugging\\information} & \resource{assembly\\listing}}
\seecpp\seeassembly\seewasm\seeobject\seedebugging
}

% FALSE tools

\providecommand{\falprint}{
\toolsection{falprint} is a pretty printer for the FALSE programming language.
It reformats the source code of FALSE programs and writes it to the standard output stream.
\flowgraph{\resource{FALSE\\source code} \ar[r] & \toolbox{falprint} \ar[r] & \resource{reformatted\\source code}}
\seefalse
}

\providecommand{\falcheck}{
\toolsection{falcheck} is a syntactic and semantic checker for the FALSE programming language.
It just performs syntactic and semantic checks on FALSE programs and writes its diagnostic messages to the standard error stream.
\flowgraph{\resource{FALSE\\source code} \ar[r] & \toolbox{falcheck} \ar[r] & \resource{diagnostic\\messages}}
\seefalse
}

\providecommand{\faldump}{
\toolsection{faldump} is a serializer for the FALSE programming language.
It dumps the complete internal representation of programs written in FALSE into an XML document.
\debuggingtool
\flowgraph{\resource{FALSE\\source code} \ar[r] & \toolbox{faldump} \ar[r] & \resource{internal\\representation}}
\seefalse
}

\providecommand{\falrun}{
\toolsection{falrun} is an interpreter for the FALSE programming language.
It processes and executes programs written in FALSE\@.
\flowgraph{\resource{FALSE\\source code} \ar[r] & \toolbox{falrun} \ar@/u/[r] & \resource{input/\\output} \ar@/d/[l]}
\seefalse
}

\providecommand{\falcpp}{
\toolsection{falcpp} is a transpiler for the FALSE programming language.
It translates programs written in FALSE into \cpp{} programs and stores them in corresponding source files.
\flowgraph{\resource{FALSE\\source code} \ar[r] & \toolbox{falcpp} \ar[r] & \resource{\cpp{}\\source file}}
\seefalse\seecpp
}

\providecommand{\falcode}{
\toolsection{falcode} is an intermediate code generator for the FALSE programming language.
It generates intermediate code from programs written in FALSE and stores it in corresponding assembly files.
\debuggingtool
\flowgraph{\resource{FALSE\\source code} \ar[r] & \toolbox{falcode} \ar[r] & \resource{intermediate\\code}}
\seefalse\seeassembly\seecode
}

\providecommand{\falamda}{
\toolsection{falamd16} is a compiler for the FALSE programming language targeting the AMD64 hardware architecture.
It generates machine code for AMD64 processors from programs written in FALSE and stores it in corresponding object files.
The compiler generates machine code for the 16-bit operating mode defined by the AMD64 architecture.
\flowgraph{\resource{FALSE\\source code} \ar[r] & \toolbox{falamd16} \ar[r] & \resource{object file}}
\seefalse\seeamd\seeobject
}

\providecommand{\falamdb}{
\toolsection{falamd32} is a compiler for the FALSE programming language targeting the AMD64 hardware architecture.
It generates machine code for AMD64 processors from programs written in FALSE and stores it in corresponding object files.
The compiler generates machine code for the 32-bit operating mode defined by the AMD64 architecture.
\flowgraph{\resource{FALSE\\source code} \ar[r] & \toolbox{falamd32} \ar[r] & \resource{object file}}
\seefalse\seeamd\seeobject
}

\providecommand{\falamdc}{
\toolsection{falamd64} is a compiler for the FALSE programming language targeting the AMD64 hardware architecture.
It generates machine code for AMD64 processors from programs written in FALSE and stores it in corresponding object files.
The compiler generates machine code for the 64-bit operating mode defined by the AMD64 architecture.
\flowgraph{\resource{FALSE\\source code} \ar[r] & \toolbox{falamd64} \ar[r] & \resource{object file}}
\seefalse\seeamd\seeobject
}

\providecommand{\falarma}{
\toolsection{falarma32} is a compiler for the FALSE programming language targeting the ARM hardware architecture.
It generates machine code for ARM processors executing A32 instructions from programs written in FALSE and stores it in corresponding object files.
\flowgraph{\resource{FALSE\\source code} \ar[r] & \toolbox{falarma32} \ar[r] & \resource{object file}}
\seefalse\seearm\seeobject
}

\providecommand{\falarmb}{
\toolsection{falarma64} is a compiler for the FALSE programming language targeting the ARM hardware architecture.
It generates machine code for ARM processors executing A64 instructions from programs written in FALSE and stores it in corresponding object files.
\flowgraph{\resource{FALSE\\source code} \ar[r] & \toolbox{falarma64} \ar[r] & \resource{object file}}
\seefalse\seearm\seeobject
}

\providecommand{\falarmc}{
\toolsection{falarmt32} is a compiler for the FALSE programming language targeting the ARM hardware architecture.
It generates machine code for ARM processors without floating-point extension executing T32 instructions from programs written in FALSE and stores it in corresponding object files.
\flowgraph{\resource{FALSE\\source code} \ar[r] & \toolbox{falarmt32} \ar[r] & \resource{object file}}
\seefalse\seearm\seeobject
}

\providecommand{\falarmcfpe}{
\toolsection{falarmt32fpe} is a compiler for the FALSE programming language targeting the ARM hardware architecture.
It generates machine code for ARM processors with floating-point extension executing T32 instructions from programs written in FALSE and stores it in corresponding object files.
\flowgraph{\resource{FALSE\\source code} \ar[r] & \toolbox{falarmt32fpe} \ar[r] & \resource{object file}}
\seefalse\seearm\seeobject
}

\providecommand{\falavr}{
\toolsection{falavr} is a compiler for the FALSE programming language targeting the AVR hardware architecture.
It generates machine code for AVR processors from programs written in FALSE and stores it in corresponding object files.
\flowgraph{\resource{FALSE\\source code} \ar[r] & \toolbox{falavr} \ar[r] & \resource{object file}}
\seefalse\seeavr\seeobject
}

\providecommand{\falavrtt}{
\toolsection{falavr32} is a compiler for the FALSE programming language targeting the AVR32 hardware architecture.
It generates machine code for AVR32 processors from programs written in FALSE and stores it in corresponding object files.
\flowgraph{\resource{FALSE\\source code} \ar[r] & \toolbox{falavr32} \ar[r] & \resource{object file}}
\seefalse\seeavrtt\seeobject
}

\providecommand{\falmabk}{
\toolsection{falm68k} is a compiler for the FALSE programming language targeting the M68000 hardware architecture.
It generates machine code for M68000 processors from programs written in FALSE and stores it in corresponding object files.
\flowgraph{\resource{FALSE\\source code} \ar[r] & \toolbox{falm68k} \ar[r] & \resource{object file}}
\seefalse\seemabk\seeobject
}

\providecommand{\falmibl}{
\toolsection{falmibl} is a compiler for the FALSE programming language targeting the MicroBlaze hardware architecture.
It generates machine code for MicroBlaze processors from programs written in FALSE and stores it in corresponding object files.
\flowgraph{\resource{FALSE\\source code} \ar[r] & \toolbox{falmibl} \ar[r] & \resource{object file}}
\seefalse\seemibl\seeobject
}

\providecommand{\falmipsa}{
\toolsection{falmips32} is a compiler for the FALSE programming language targeting the MIPS32 hardware architecture.
It generates machine code for MIPS32 processors from programs written in FALSE and stores it in corresponding object files.
\flowgraph{\resource{FALSE\\source code} \ar[r] & \toolbox{falmips32} \ar[r] & \resource{object file}}
\seefalse\seemips\seeobject
}

\providecommand{\falmipsb}{
\toolsection{falmips64} is a compiler for the FALSE programming language targeting the MIPS64 hardware architecture.
It generates machine code for MIPS64 processors from programs written in FALSE and stores it in corresponding object files.
\flowgraph{\resource{FALSE\\source code} \ar[r] & \toolbox{falmips64} \ar[r] & \resource{object file}}
\seefalse\seemips\seeobject
}

\providecommand{\falmmix}{
\toolsection{falmmix} is a compiler for the FALSE programming language targeting the MMIX hardware architecture.
It generates machine code for MMIX processors from programs written in FALSE and stores it in corresponding object files.
\flowgraph{\resource{FALSE\\source code} \ar[r] & \toolbox{falmmix} \ar[r] & \resource{object file}}
\seefalse\seemmix\seeobject
}

\providecommand{\falorok}{
\toolsection{falor1k} is a compiler for the FALSE programming language targeting the OpenRISC 1000 hardware architecture.
It generates machine code for OpenRISC 1000 processors from programs written in FALSE and stores it in corresponding object files.
\flowgraph{\resource{FALSE\\source code} \ar[r] & \toolbox{falor1k} \ar[r] & \resource{object file}}
\seefalse\seeorok\seeobject
}

\providecommand{\falppca}{
\toolsection{falppc32} is a compiler for the FALSE programming language targeting the PowerPC hardware architecture.
It generates machine code for PowerPC processors from programs written in FALSE and stores it in corresponding object files.
The compiler generates machine code for the 32-bit operating mode defined by the PowerPC architecture.
\flowgraph{\resource{FALSE\\source code} \ar[r] & \toolbox{falppc32} \ar[r] & \resource{object file}}
\seefalse\seeppc\seeobject
}

\providecommand{\falppcb}{
\toolsection{falppc64} is a compiler for the FALSE programming language targeting the PowerPC hardware architecture.
It generates machine code for PowerPC processors from programs written in FALSE and stores it in corresponding object files.
The compiler generates machine code for the 64-bit operating mode defined by the PowerPC architecture.
\flowgraph{\resource{FALSE\\source code} \ar[r] & \toolbox{falppc64} \ar[r] & \resource{object file}}
\seefalse\seeppc\seeobject
}

\providecommand{\falrisc}{
\toolsection{falrisc} is a compiler for the FALSE programming language targeting the RISC hardware architecture.
It generates machine code for RISC processors from programs written in FALSE and stores it in corresponding object files.
\flowgraph{\resource{FALSE\\source code} \ar[r] & \toolbox{falrisc} \ar[r] & \resource{object file}}
\seefalse\seerisc\seeobject
}

\providecommand{\falwasm}{
\toolsection{falwasm} is a compiler for the FALSE programming language targeting the WebAssembly architecture.
It generates machine code for WebAssembly targets from programs written in FALSE and stores it in corresponding object files.
\flowgraph{\resource{FALSE\\source code} \ar[r] & \toolbox{falwasm} \ar[r] & \resource{object file}}
\seefalse\seewasm\seeobject
}

% Oberon tools

\providecommand{\obprint}{
\toolsection{obprint} is a pretty printer for the Oberon programming language.
It reformats the source code of Oberon modules and writes it to the standard output stream.
\flowgraph{\resource{Oberon\\source code} \ar[r] & \toolbox{obprint} \ar[r] & \resource{reformatted\\source code}}
\seeoberon
}

\providecommand{\obcheck}{
\toolsection{obcheck} is a syntactic and semantic checker for the Oberon programming language.
It just performs syntactic and semantic checks on Oberon modules and writes its diagnostic messages to the standard error stream.
In addition, it stores the interface of each module in a symbol file which is required when other modules import the module.
\flowgraph{\resource{Oberon\\source code} \ar[r] & \toolbox{obcheck} \ar[r] \ar@/l/[d] & \resource{diagnostic\\messages} \\ \variable{ECSIMPORT} \ar[ru] & \resource{symbol\\files} \ar@/r/[u]}
\seeoberon
}

\providecommand{\obdump}{
\toolsection{obdump} is a serializer for the Oberon programming language.
It dumps the complete internal representation of modules written in Oberon into an XML document.
\debuggingtool
\flowgraph{\resource{Oberon\\source code} \ar[r] & \toolbox{obdump} \ar[r] \ar@/l/[d] & \resource{internal\\representation} \\ \variable{ECSIMPORT} \ar[ru] & \resource{symbol\\files} \ar@/r/[u]}
\seeoberon
}

\providecommand{\obrun}{
\toolsection{obrun} is an interpreter for the Oberon programming language.
It processes and executes modules written in Oberon.
This tool does neither generate nor process symbol files while interpreting modules.
If a module is imported by another one, its filename has to be named before the other one in the list of command-line arguments.
\flowgraph{\resource{Oberon\\source code} \ar[r] & \toolbox{obrun} \ar@/u/[r] & \resource{input/\\output} \ar@/d/[l]}
\seeoberon
}

\providecommand{\obcpp}{
\toolsection{obcpp} is a transpiler for the Oberon programming language.
It translates programs written in Oberon into \cpp{} programs and stores them in corresponding source and header files.
In addition, it stores the interface of each module in a symbol file which is required when other modules import the module.
The same interface is provided by the generated header file which can be used in other parts of the \cpp{} program.
\flowgraph{\resource{Oberon\\source code} \ar[r] & \toolbox{obcpp} \ar[r] \ar@/l/[d] \ar[rd] & \resource{\cpp{}\\source file} \\ \variable{ECSIMPORT} \ar[ru] & \resource{symbol\\files} \ar@/r/[u] & \resource{\cpp{}\\header file}}
\seeoberon\seecpp
}

\providecommand{\obdoc}{
\toolsection{obdoc} is a generic documentation generator for the Oberon programming language.
It processes several Oberon modules and assembles all information therein into a generic documentation.
In addition, it stores the interface of each module in a symbol file which is required when other modules import the module.
\debuggingtool
\flowgraph{\resource{Oberon\\source code} \ar[r] & \toolbox{obdoc} \ar[r] \ar@/l/[d] & \resource{generic\\documentation} \\ \variable{ECSIMPORT} \ar[ru] & \resource{symbol\\files} \ar@/r/[u]}
\seeoberon\seedocumentation
}

\providecommand{\obhtml}{
\toolsection{obhtml} is an HTML documentation generator for the Oberon programming language.
It processes several Oberon modules and assembles all information therein into an HTML document.
In addition, it stores the interface of each module in a symbol file which is required when other modules import the module.
\flowgraph{\resource{Oberon\\source code} \ar[r] & \toolbox{obhtml} \ar[r] \ar@/l/[d] & \resource{HTML\\document} \\ \variable{ECSIMPORT} \ar[ru] & \resource{symbol\\files} \ar@/r/[u]}
\seeoberon\seedocumentation
}

\providecommand{\oblatex}{
\toolsection{oblatex} is a Latex documentation generator for the Oberon programming language.
It processes several Oberon modules and assembles all information therein into a Latex document.
In addition, it stores the interface of each module in a symbol file which is required when other modules import the module.
\flowgraph{\resource{Oberon\\source code} \ar[r] & \toolbox{oblatex} \ar[r] \ar@/l/[d] & \resource{Latex\\document} \\ \variable{ECSIMPORT} \ar[ru] & \resource{symbol\\files} \ar@/r/[u]}
\seeoberon\seedocumentation
}

\providecommand{\obcode}{
\toolsection{obcode} is an intermediate code generator for the Oberon programming language.
It generates intermediate code from modules written in Oberon and stores it in corresponding assembly files.
In addition, it stores the interface of each module in a symbol file which is required when other modules import the module.
Programs generated with this tool require additional runtime support that is stored in the \file{ob\-code\-run} library file.
\debuggingtool
\flowgraph{\resource{Oberon\\source code} \ar[r] & \toolbox{obcode} \ar[r] \ar@/l/[d] & \resource{intermediate\\code} \\ \variable{ECSIMPORT} \ar[ru] & \resource{symbol\\files} \ar@/r/[u]}
\seeoberon\seeassembly\seecode
}

\providecommand{\obamda}{
\toolsection{obamd16} is a compiler for the Oberon programming language targeting the AMD64 hardware architecture.
It generates machine code for AMD64 processors from modules written in Oberon and stores it in corresponding object files.
The compiler generates machine code for the 16-bit operating mode defined by the AMD64 architecture.
For debugging purposes, it also creates a debugging information file as well as an assembly file containing a listing of the generated machine code.
In addition, it stores the interface of each module in a symbol file which is required when other modules import the module.
Programs generated with this compiler require additional runtime support that is stored in the \file{ob\-amd16\-run} library file.
\flowgraph{\resource{Oberon\\source code} \ar[r] & \toolbox{obamd16} \ar[r] \ar@/l/[d] \ar[rd] & \resource{object file} \\ \variable{ECSIMPORT} \ar[ru] & \resource{symbol\\files} \ar@/r/[u] & \resource{debugging\\information}}
\seeoberon\seeassembly\seeamd\seeobject\seedebugging
}

\providecommand{\obamdb}{
\toolsection{obamd32} is a compiler for the Oberon programming language targeting the AMD64 hardware architecture.
It generates machine code for AMD64 processors from modules written in Oberon and stores it in corresponding object files.
The compiler generates machine code for the 32-bit operating mode defined by the AMD64 architecture.
For debugging purposes, it also creates a debugging information file as well as an assembly file containing a listing of the generated machine code.
In addition, it stores the interface of each module in a symbol file which is required when other modules import the module.
Programs generated with this compiler require additional runtime support that is stored in the \file{ob\-amd32\-run} library file.
\flowgraph{\resource{Oberon\\source code} \ar[r] & \toolbox{obamd32} \ar[r] \ar@/l/[d] \ar[rd] & \resource{object file} \\ \variable{ECSIMPORT} \ar[ru] & \resource{symbol\\files} \ar@/r/[u] & \resource{debugging\\information}}
\seeoberon\seeassembly\seeamd\seeobject\seedebugging
}

\providecommand{\obamdc}{
\toolsection{obamd64} is a compiler for the Oberon programming language targeting the AMD64 hardware architecture.
It generates machine code for AMD64 processors from modules written in Oberon and stores it in corresponding object files.
The compiler generates machine code for the 64-bit operating mode defined by the AMD64 architecture.
For debugging purposes, it also creates a debugging information file as well as an assembly file containing a listing of the generated machine code.
In addition, it stores the interface of each module in a symbol file which is required when other modules import the module.
Programs generated with this compiler require additional runtime support that is stored in the \file{ob\-amd64\-run} library file.
\flowgraph{\resource{Oberon\\source code} \ar[r] & \toolbox{obamd64} \ar[r] \ar@/l/[d] \ar[rd] & \resource{object file} \\ \variable{ECSIMPORT} \ar[ru] & \resource{symbol\\files} \ar@/r/[u] & \resource{debugging\\information}}
\seeoberon\seeassembly\seeamd\seeobject\seedebugging
}

\providecommand{\obarma}{
\toolsection{obarma32} is a compiler for the Oberon programming language targeting the ARM hardware architecture.
It generates machine code for ARM processors executing A32 instructions from modules written in Oberon and stores it in corresponding object files.
For debugging purposes, it also creates a debugging information file as well as an assembly file containing a listing of the generated machine code.
In addition, it stores the interface of each module in a symbol file which is required when other modules import the module.
Programs generated with this compiler require additional runtime support that is stored in the \file{ob\-arma32\-run} library file.
\flowgraph{\resource{Oberon\\source code} \ar[r] & \toolbox{obarma32} \ar[r] \ar@/l/[d] \ar[rd] & \resource{object file} \\ \variable{ECSIMPORT} \ar[ru] & \resource{symbol\\files} \ar@/r/[u] & \resource{debugging\\information}}
\seeoberon\seeassembly\seearm\seeobject\seedebugging
}

\providecommand{\obarmb}{
\toolsection{obarma64} is a compiler for the Oberon programming language targeting the ARM hardware architecture.
It generates machine code for ARM processors executing A64 instructions from modules written in Oberon and stores it in corresponding object files.
For debugging purposes, it also creates a debugging information file as well as an assembly file containing a listing of the generated machine code.
In addition, it stores the interface of each module in a symbol file which is required when other modules import the module.
Programs generated with this compiler require additional runtime support that is stored in the \file{ob\-arma64\-run} library file.
\flowgraph{\resource{Oberon\\source code} \ar[r] & \toolbox{obarma64} \ar[r] \ar@/l/[d] \ar[rd] & \resource{object file} \\ \variable{ECSIMPORT} \ar[ru] & \resource{symbol\\files} \ar@/r/[u] & \resource{debugging\\information}}
\seeoberon\seeassembly\seearm\seeobject\seedebugging
}

\providecommand{\obarmc}{
\toolsection{obarmt32} is a compiler for the Oberon programming language targeting the ARM hardware architecture.
It generates machine code for ARM processors without floating-point extension executing T32 instructions from modules written in Oberon and stores it in corresponding object files.
For debugging purposes, it also creates a debugging information file as well as an assembly file containing a listing of the generated machine code.
In addition, it stores the interface of each module in a symbol file which is required when other modules import the module.
Programs generated with this compiler require additional runtime support that is stored in the \file{ob\-armt32\-run} library file.
\flowgraph{\resource{Oberon\\source code} \ar[r] & \toolbox{obarmt32} \ar[r] \ar@/l/[d] \ar[rd] & \resource{object file} \\ \variable{ECSIMPORT} \ar[ru] & \resource{symbol\\files} \ar@/r/[u] & \resource{debugging\\information}}
\seeoberon\seeassembly\seearm\seeobject\seedebugging
}

\providecommand{\obarmcfpe}{
\toolsection{obarmt32fpe} is a compiler for the Oberon programming language targeting the ARM hardware architecture.
It generates machine code for ARM processors with floating-point extension executing T32 instructions from modules written in Oberon and stores it in corresponding object files.
For debugging purposes, it also creates a debugging information file as well as an assembly file containing a listing of the generated machine code.
In addition, it stores the interface of each module in a symbol file which is required when other modules import the module.
Programs generated with this compiler require additional runtime support that is stored in the \file{ob\-armt32\-fpe\-run} library file.
\flowgraph{\resource{Oberon\\source code} \ar[r] & \toolbox{obarmt32fpe} \ar[r] \ar@/l/[d] \ar[rd] & \resource{object file} \\ \variable{ECSIMPORT} \ar[ru] & \resource{symbol\\files} \ar@/r/[u] & \resource{debugging\\information}}
\seeoberon\seeassembly\seearm\seeobject\seedebugging
}

\providecommand{\obavr}{
\toolsection{obavr} is a compiler for the Oberon programming language targeting the AVR hardware architecture.
It generates machine code for AVR processors from modules written in Oberon and stores it in corresponding object files.
For debugging purposes, it also creates a debugging information file as well as an assembly file containing a listing of the generated machine code.
In addition, it stores the interface of each module in a symbol file which is required when other modules import the module.
Programs generated with this compiler require additional runtime support that is stored in the \file{ob\-avr\-run} library file.
\flowgraph{\resource{Oberon\\source code} \ar[r] & \toolbox{obavr} \ar[r] \ar@/l/[d] \ar[rd] & \resource{object file} \\ \variable{ECSIMPORT} \ar[ru] & \resource{symbol\\files} \ar@/r/[u] & \resource{debugging\\information}}
\seeoberon\seeassembly\seeavr\seeobject\seedebugging
}

\providecommand{\obavrtt}{
\toolsection{obavr32} is a compiler for the Oberon programming language targeting the AVR32 hardware architecture.
It generates machine code for AVR32 processors from modules written in Oberon and stores it in corresponding object files.
For debugging purposes, it also creates a debugging information file as well as an assembly file containing a listing of the generated machine code.
In addition, it stores the interface of each module in a symbol file which is required when other modules import the module.
Programs generated with this compiler require additional runtime support that is stored in the \file{ob\-avr32\-run} library file.
\flowgraph{\resource{Oberon\\source code} \ar[r] & \toolbox{obavr32} \ar[r] \ar@/l/[d] \ar[rd] & \resource{object file} \\ \variable{ECSIMPORT} \ar[ru] & \resource{symbol\\files} \ar@/r/[u] & \resource{debugging\\information}}
\seeoberon\seeassembly\seeavrtt\seeobject\seedebugging
}

\providecommand{\obmabk}{
\toolsection{obm68k} is a compiler for the Oberon programming language targeting the M68000 hardware architecture.
It generates machine code for M68000 processors from modules written in Oberon and stores it in corresponding object files.
For debugging purposes, it also creates a debugging information file as well as an assembly file containing a listing of the generated machine code.
In addition, it stores the interface of each module in a symbol file which is required when other modules import the module.
Programs generated with this compiler require additional runtime support that is stored in the \file{ob\-m68k\-run} library file.
\flowgraph{\resource{Oberon\\source code} \ar[r] & \toolbox{obm68k} \ar[r] \ar@/l/[d] \ar[rd] & \resource{object file} \\ \variable{ECSIMPORT} \ar[ru] & \resource{symbol\\files} \ar@/r/[u] & \resource{debugging\\information}}
\seeoberon\seeassembly\seemabk\seeobject\seedebugging
}

\providecommand{\obmibl}{
\toolsection{obmibl} is a compiler for the Oberon programming language targeting the MicroBlaze hardware architecture.
It generates machine code for MicroBlaze processors from modules written in Oberon and stores it in corresponding object files.
For debugging purposes, it also creates a debugging information file as well as an assembly file containing a listing of the generated machine code.
In addition, it stores the interface of each module in a symbol file which is required when other modules import the module.
Programs generated with this compiler require additional runtime support that is stored in the \file{ob\-mibl\-run} library file.
\flowgraph{\resource{Oberon\\source code} \ar[r] & \toolbox{obmibl} \ar[r] \ar@/l/[d] \ar[rd] & \resource{object file} \\ \variable{ECSIMPORT} \ar[ru] & \resource{symbol\\files} \ar@/r/[u] & \resource{debugging\\information}}
\seeoberon\seeassembly\seemibl\seeobject\seedebugging
}

\providecommand{\obmipsa}{
\toolsection{obmips32} is a compiler for the Oberon programming language targeting the MIPS32 hardware architecture.
It generates machine code for MIPS32 processors from modules written in Oberon and stores it in corresponding object files.
For debugging purposes, it also creates a debugging information file as well as an assembly file containing a listing of the generated machine code.
In addition, it stores the interface of each module in a symbol file which is required when other modules import the module.
Programs generated with this compiler require additional runtime support that is stored in the \file{ob\-mips32\-run} library file.
\flowgraph{\resource{Oberon\\source code} \ar[r] & \toolbox{obmips32} \ar[r] \ar@/l/[d] \ar[rd] & \resource{object file} \\ \variable{ECSIMPORT} \ar[ru] & \resource{symbol\\files} \ar@/r/[u] & \resource{debugging\\information}}
\seeoberon\seeassembly\seemips\seeobject\seedebugging
}

\providecommand{\obmipsb}{
\toolsection{obmips64} is a compiler for the Oberon programming language targeting the MIPS64 hardware architecture.
It generates machine code for MIPS64 processors from modules written in Oberon and stores it in corresponding object files.
For debugging purposes, it also creates a debugging information file as well as an assembly file containing a listing of the generated machine code.
In addition, it stores the interface of each module in a symbol file which is required when other modules import the module.
Programs generated with this compiler require additional runtime support that is stored in the \file{ob\-mips64\-run} library file.
\flowgraph{\resource{Oberon\\source code} \ar[r] & \toolbox{obmips64} \ar[r] \ar@/l/[d] \ar[rd] & \resource{object file} \\ \variable{ECSIMPORT} \ar[ru] & \resource{symbol\\files} \ar@/r/[u] & \resource{debugging\\information}}
\seeoberon\seeassembly\seemips\seeobject\seedebugging
}

\providecommand{\obmmix}{
\toolsection{obmmix} is a compiler for the Oberon programming language targeting the MMIX hardware architecture.
It generates machine code for MMIX processors from modules written in Oberon and stores it in corresponding object files.
For debugging purposes, it also creates a debugging information file as well as an assembly file containing a listing of the generated machine code.
In addition, it stores the interface of each module in a symbol file which is required when other modules import the module.
Programs generated with this compiler require additional runtime support that is stored in the \file{ob\-mmix\-run} library file.
\flowgraph{\resource{Oberon\\source code} \ar[r] & \toolbox{obmmix} \ar[r] \ar@/l/[d] \ar[rd] & \resource{object file} \\ \variable{ECSIMPORT} \ar[ru] & \resource{symbol\\files} \ar@/r/[u] & \resource{debugging\\information}}
\seeoberon\seeassembly\seemmix\seeobject\seedebugging
}

\providecommand{\oborok}{
\toolsection{obor1k} is a compiler for the Oberon programming language targeting the OpenRISC 1000 hardware architecture.
It generates machine code for OpenRISC 1000 processors from modules written in Oberon and stores it in corresponding object files.
For debugging purposes, it also creates a debugging information file as well as an assembly file containing a listing of the generated machine code.
In addition, it stores the interface of each module in a symbol file which is required when other modules import the module.
Programs generated with this compiler require additional runtime support that is stored in the \file{ob\-or1k\-run} library file.
\flowgraph{\resource{Oberon\\source code} \ar[r] & \toolbox{obor1k} \ar[r] \ar@/l/[d] \ar[rd] & \resource{object file} \\ \variable{ECSIMPORT} \ar[ru] & \resource{symbol\\files} \ar@/r/[u] & \resource{debugging\\information}}
\seeoberon\seeassembly\seeorok\seeobject\seedebugging
}

\providecommand{\obppca}{
\toolsection{obppc32} is a compiler for the Oberon programming language targeting the PowerPC hardware architecture.
It generates machine code for PowerPC processors from modules written in Oberon and stores it in corresponding object files.
The compiler generates machine code for the 32-bit operating mode defined by the PowerPC architecture.
For debugging purposes, it also creates a debugging information file as well as an assembly file containing a listing of the generated machine code.
In addition, it stores the interface of each module in a symbol file which is required when other modules import the module.
Programs generated with this compiler require additional runtime support that is stored in the \file{ob\-ppc32\-run} library file.
\flowgraph{\resource{Oberon\\source code} \ar[r] & \toolbox{obppc32} \ar[r] \ar@/l/[d] \ar[rd] & \resource{object file} \\ \variable{ECSIMPORT} \ar[ru] & \resource{symbol\\files} \ar@/r/[u] & \resource{debugging\\information}}
\seeoberon\seeassembly\seeppc\seeobject\seedebugging
}

\providecommand{\obppcb}{
\toolsection{obppc64} is a compiler for the Oberon programming language targeting the PowerPC hardware architecture.
It generates machine code for PowerPC processors from modules written in Oberon and stores it in corresponding object files.
The compiler generates machine code for the 64-bit operating mode defined by the PowerPC architecture.
For debugging purposes, it also creates a debugging information file as well as an assembly file containing a listing of the generated machine code.
In addition, it stores the interface of each module in a symbol file which is required when other modules import the module.
Programs generated with this compiler require additional runtime support that is stored in the \file{ob\-ppc64\-run} library file.
\flowgraph{\resource{Oberon\\source code} \ar[r] & \toolbox{obppc64} \ar[r] \ar@/l/[d] \ar[rd] & \resource{object file} \\ \variable{ECSIMPORT} \ar[ru] & \resource{symbol\\files} \ar@/r/[u] & \resource{debugging\\information}}
\seeoberon\seeassembly\seeppc\seeobject\seedebugging
}

\providecommand{\obrisc}{
\toolsection{obrisc} is a compiler for the Oberon programming language targeting the RISC hardware architecture.
It generates machine code for RISC processors from modules written in Oberon and stores it in corresponding object files.
For debugging purposes, it also creates a debugging information file as well as an assembly file containing a listing of the generated machine code.
In addition, it stores the interface of each module in a symbol file which is required when other modules import the module.
Programs generated with this compiler require additional runtime support that is stored in the \file{ob\-risc\-run} library file.
\flowgraph{\resource{Oberon\\source code} \ar[r] & \toolbox{obrisc} \ar[r] \ar@/l/[d] \ar[rd] & \resource{object file} \\ \variable{ECSIMPORT} \ar[ru] & \resource{symbol\\files} \ar@/r/[u] & \resource{debugging\\information}}
\seeoberon\seeassembly\seerisc\seeobject\seedebugging
}

\providecommand{\obwasm}{
\toolsection{obwasm} is a compiler for the Oberon programming language targeting the WebAssembly architecture.
It generates machine code for WebAssembly targets from modules written in Oberon and stores it in corresponding object files.
For debugging purposes, it also creates a debugging information file as well as an assembly file containing a listing of the generated machine code.
In addition, it stores the interface of each module in a symbol file which is required when other modules import the module.
Programs generated with this compiler require additional runtime support that is stored in the \file{ob\-wasm\-run} library file.
\flowgraph{\resource{Oberon\\source code} \ar[r] & \toolbox{obwasm} \ar[r] \ar@/l/[d] \ar[rd] & \resource{object file} \\ \variable{ECSIMPORT} \ar[ru] & \resource{symbol\\files} \ar@/r/[u] & \resource{debugging\\information}}
\seeoberon\seeassembly\seewasm\seeobject\seedebugging
}

% converter tools

\providecommand{\dbgdwarf}{
\toolsection{dbgdwarf} is a DWARF debugging information converter tool.
It converts debugging information into the DWARF debugging data format and stores it in corresponding object files~\cite{dwarffile}.
The resulting debugging object files can be combined with runtime support that creates Executable and Linking Format (ELF) files~\cite{elffile}.
\flowgraph{\resource{debugging\\information} \ar[r] & \toolbox{dbgdwarf} \ar[r] & \resource{debugging\\object file}}
\seeobject\seedebugging
}

% assembler tools

\providecommand{\asmprint}{
\toolsection{asmprint} is a pretty printer for generic assembly code.
It reformats generic assembly code and writes it to the standard output stream.
\flowgraph{\resource{generic assembly\\source code} \ar[r] & \toolbox{asmprint} \ar[r] & \resource{reformatted\\source code}}
\seeassembly
}

\providecommand{\amdaasm}{
\toolsection{amd16asm} is an assembler for the AMD64 hardware architecture.
It translates assembly code into machine code for AMD64 processors and stores it in corresponding object files.
By default, the assembler generates machine code for the 16-bit operating mode defined by the AMD64 architecture.
\flowgraph{\resource{AMD16 assembly\\source code} \ar[r] & \toolbox{amd16asm} \ar[r] & \resource{object file}}
\seeassembly\seeamd\seeobject
}

\providecommand{\amdadism}{
\toolsection{amd16dism} is a disassembler for the AMD64 hardware architecture.
It translates machine code from object files targeting AMD64 processors into assembly code and writes it to the standard output stream.
It assumes that the machine code was generated for the 16-bit operating mode defined by the AMD64 architecture.
\flowgraph{\resource{object file} \ar[r] & \toolbox{amd16dism} \ar[r] & \resource{disassembly\\listing}}
\seeassembly\seeamd\seeobject
}

\providecommand{\amdbasm}{
\toolsection{amd32asm} is an assembler for the AMD64 hardware architecture.
It translates assembly code into machine code for AMD64 processors and stores it in corresponding object files.
By default, the assembler generates machine code for the 32-bit operating mode defined by the AMD64 architecture.
\flowgraph{\resource{AMD32 assembly\\source code} \ar[r] & \toolbox{amd32asm} \ar[r] & \resource{object file}}
\seeassembly\seeamd\seeobject
}

\providecommand{\amdbdism}{
\toolsection{amd32dism} is a disassembler for the AMD64 hardware architecture.
It translates machine code from object files targeting AMD64 processors into assembly code and writes it to the standard output stream.
It assumes that the machine code was generated for the 32-bit operating mode defined by the AMD64 architecture.
\flowgraph{\resource{object file} \ar[r] & \toolbox{amd32dism} \ar[r] & \resource{disassembly\\listing}}
\seeassembly\seeamd\seeobject
}

\providecommand{\amdcasm}{
\toolsection{amd64asm} is an assembler for the AMD64 hardware architecture.
It translates assembly code into machine code for AMD64 processors and stores it in corresponding object files.
By default, the assembler generates machine code for the 64-bit operating mode defined by the AMD64 architecture.
\flowgraph{\resource{AMD64 assembly\\source code} \ar[r] & \toolbox{amd64asm} \ar[r] & \resource{object file}}
\seeassembly\seeamd\seeobject
}

\providecommand{\amdcdism}{
\toolsection{amd64dism} is a disassembler for the AMD64 hardware architecture.
It translates machine code from object files targeting AMD64 processors into assembly code and writes it to the standard output stream.
It assumes that the machine code was generated for the 64-bit operating mode defined by the AMD64 architecture.
\flowgraph{\resource{object file} \ar[r] & \toolbox{amd64dism} \ar[r] & \resource{disassembly\\listing}}
\seeassembly\seeamd\seeobject
}

\providecommand{\armaasm}{
\toolsection{arma32asm} is an assembler for the ARM hardware architecture.
It translates assembly code into machine code for ARM processors executing A32 instructions and stores it in corresponding object files.
\flowgraph{\resource{ARM A32 assembly\\source code} \ar[r] & \toolbox{arma32asm} \ar[r] & \resource{object file}}
\seeassembly\seearm\seeobject
}

\providecommand{\armadism}{
\toolsection{arma32dism} is a disassembler for the ARM hardware architecture.
It translates machine code from object files targeting ARM processors executing A32 instructions into assembly code and writes it to the standard output stream.
\flowgraph{\resource{object file} \ar[r] & \toolbox{arma32dism} \ar[r] & \resource{disassembly\\listing}}
\seeassembly\seearm\seeobject
}

\providecommand{\armbasm}{
\toolsection{arma64asm} is an assembler for the ARM hardware architecture.
It translates assembly code into machine code for ARM processors executing A64 instructions and stores it in corresponding object files.
\flowgraph{\resource{ARM A64 assembly\\source code} \ar[r] & \toolbox{arma64asm} \ar[r] & \resource{object file}}
\seeassembly\seearm\seeobject
}

\providecommand{\armbdism}{
\toolsection{arma64dism} is a disassembler for the ARM hardware architecture.
It translates machine code from object files targeting ARM processors executing A64 instructions into assembly code and writes it to the standard output stream.
\flowgraph{\resource{object file} \ar[r] & \toolbox{arma64dism} \ar[r] & \resource{disassembly\\listing}}
\seeassembly\seearm\seeobject
}

\providecommand{\armcasm}{
\toolsection{armt32asm} is an assembler for the ARM hardware architecture.
It translates assembly code into machine code for ARM processors executing T32 instructions and stores it in corresponding object files.
\flowgraph{\resource{ARM T32 assembly\\source code} \ar[r] & \toolbox{armt32asm} \ar[r] & \resource{object file}}
\seeassembly\seearm\seeobject
}

\providecommand{\armcdism}{
\toolsection{armt32dism} is a disassembler for the ARM hardware architecture.
It translates machine code from object files targeting ARM processors executing T32 instructions into assembly code and writes it to the standard output stream.
\flowgraph{\resource{object file} \ar[r] & \toolbox{armt32dism} \ar[r] & \resource{disassembly\\listing}}
\seeassembly\seearm\seeobject
}

\providecommand{\avrasm}{
\toolsection{avrasm} is an assembler for the AVR hardware architecture.
It translates assembly code into machine code for AVR processors and stores it in corresponding object files.
The identifiers \texttt{RXL}, \texttt{RXH}, \texttt{RYL}, \texttt{RYH}, \texttt{RZL}, and \texttt{RZH} are predefined and name the corresponding registers.
The identifiers \texttt{SPL} and \texttt{SPH} are also predefined and evaluate to the address of the corresponding registers.
\flowgraph{\resource{AVR assembly\\source code} \ar[r] & \toolbox{avrasm} \ar[r] & \resource{object file}}
\seeassembly\seeavr\seeobject
}

\providecommand{\avrdism}{
\toolsection{avrdism} is a disassembler for the AVR hardware architecture.
It translates machine code from object files targeting AVR processors into assembly code and writes it to the standard output stream.
\flowgraph{\resource{object file} \ar[r] & \toolbox{avrdism} \ar[r] & \resource{disassembly\\listing}}
\seeassembly\seeavr\seeobject
}

\providecommand{\avrttasm}{
\toolsection{avr32asm} is an assembler for the AVR32 hardware architecture.
It translates assembly code into machine code for AVR32 processors and stores it in corresponding object files.
\flowgraph{\resource{AVR32 assembly\\source code} \ar[r] & \toolbox{avr32asm} \ar[r] & \resource{object file}}
\seeassembly\seeavrtt\seeobject
}

\providecommand{\avrttdism}{
\toolsection{avr32dism} is a disassembler for the AVR32 hardware architecture.
It translates machine code from object files targeting AVR32 processors into assembly code and writes it to the standard output stream.
\flowgraph{\resource{object file} \ar[r] & \toolbox{avr32dism} \ar[r] & \resource{disassembly\\listing}}
\seeassembly\seeavrtt\seeobject
}

\providecommand{\mabkasm}{
\toolsection{m68kasm} is an assembler for the M68000 hardware architecture.
It translates assembly code into machine code for M68000 processors and stores it in corresponding object files.
\flowgraph{\resource{68000 assembly\\source code} \ar[r] & \toolbox{m68kasm} \ar[r] & \resource{object file}}
\seeassembly\seemabk\seeobject
}

\providecommand{\mabkdism}{
\toolsection{m68kdism} is a disassembler for the M68000 hardware architecture.
It translates machine code from object files targeting M68000 processors into assembly code and writes it to the standard output stream.
\flowgraph{\resource{object file} \ar[r] & \toolbox{m68kdism} \ar[r] & \resource{disassembly\\listing}}
\seeassembly\seemabk\seeobject
}

\providecommand{\miblasm}{
\toolsection{miblasm} is an assembler for the MicroBlaze hardware architecture.
It translates assembly code into machine code for MicroBlaze processors and stores it in corresponding object files.
\flowgraph{\resource{MicroBlaze assembly\\source code} \ar[r] & \toolbox{miblasm} \ar[r] & \resource{object file}}
\seeassembly\seemibl\seeobject
}

\providecommand{\mibldism}{
\toolsection{mibldism} is a disassembler for the MicroBlaze hardware architecture.
It translates machine code from object files targeting MicroBlaze processors into assembly code and writes it to the standard output stream.
\flowgraph{\resource{object file} \ar[r] & \toolbox{mibldism} \ar[r] & \resource{disassembly\\listing}}
\seeassembly\seemibl\seeobject
}

\providecommand{\mipsaasm}{
\toolsection{mips32asm} is an assembler for the MIPS32 hardware architecture.
It translates assembly code into machine code for MIPS32 processors and stores it in corresponding object files.
\flowgraph{\resource{MIPS32 assembly\\source code} \ar[r] & \toolbox{mips32asm} \ar[r] & \resource{object file}}
\seeassembly\seemips\seeobject
}

\providecommand{\mipsadism}{
\toolsection{mips32dism} is a disassembler for the MIPS32 hardware architecture.
It translates machine code from object files targeting MIPS32 processors into assembly code and writes it to the standard output stream.
\flowgraph{\resource{object file} \ar[r] & \toolbox{mips32dism} \ar[r] & \resource{disassembly\\listing}}
\seeassembly\seemips\seeobject
}

\providecommand{\mipsbasm}{
\toolsection{mips64asm} is an assembler for the MIPS64 hardware architecture.
It translates assembly code into machine code for MIPS64 processors and stores it in corresponding object files.
\flowgraph{\resource{MIPS64 assembly\\source code} \ar[r] & \toolbox{mips64asm} \ar[r] & \resource{object file}}
\seeassembly\seemips\seeobject
}

\providecommand{\mipsbdism}{
\toolsection{mips64dism} is a disassembler for the MIPS64 hardware architecture.
It translates machine code from object files targeting MIPS64 processors into assembly code and writes it to the standard output stream.
\flowgraph{\resource{object file} \ar[r] & \toolbox{mips64dism} \ar[r] & \resource{disassembly\\listing}}
\seeassembly\seemips\seeobject
}

\providecommand{\mmixasm}{
\toolsection{mmixasm} is an assembler for the MMIX hardware architecture.
It translates assembly code into machine code for MMIX processors and stores it in corresponding object files.
The names of all special registers are predefined and evaluate to the corresponding number.
\flowgraph{\resource{MMIX assembly\\source code} \ar[r] & \toolbox{mmixasm} \ar[r] & \resource{object file}}
\seeassembly\seemmix\seeobject
}

\providecommand{\mmixdism}{
\toolsection{mmixdism} is a disassembler for the MMIX hardware architecture.
It translates machine code from object files targeting MMIX processors into assembly code and writes it to the standard output stream.
\flowgraph{\resource{object file} \ar[r] & \toolbox{mmixdism} \ar[r] & \resource{disassembly\\listing}}
\seeassembly\seemmix\seeobject
}

\providecommand{\orokasm}{
\toolsection{or1kasm} is an assembler for the OpenRISC 1000 hardware architecture.
It translates assembly code into machine code for OpenRISC 1000 processors and stores it in corresponding object files.
\flowgraph{\resource{OpenRISC 1000 assembly\\source code} \ar[r] & \toolbox{or1kasm} \ar[r] & \resource{object file}}
\seeassembly\seeorok\seeobject
}

\providecommand{\orokdism}{
\toolsection{or1kdism} is a disassembler for the OpenRISC 1000 hardware architecture.
It translates machine code from object files targeting OpenRISC 1000 processors into assembly code and writes it to the standard output stream.
\flowgraph{\resource{object file} \ar[r] & \toolbox{or1kdism} \ar[r] & \resource{disassembly\\listing}}
\seeassembly\seeorok\seeobject
}

\providecommand{\ppcaasm}{
\toolsection{ppc32asm} is an assembler for the PowerPC hardware architecture.
It translates assembly code into machine code for PowerPC processors and stores it in corresponding object files.
By default, the assembler generates machine code for the 32-bit operating mode defined by the PowerPC architecture.
\flowgraph{\resource{PowerPC assembly\\source code} \ar[r] & \toolbox{ppc32asm} \ar[r] & \resource{object file}}
\seeassembly\seeppc\seeobject
}

\providecommand{\ppcadism}{
\toolsection{ppc32dism} is a disassembler for the PowerPC hardware architecture.
It translates machine code from object files targeting PowerPC processors into assembly code and writes it to the standard output stream.
It assumes that the machine code was generated for the 32-bit operating mode defined by the PowerPC architecture.
\flowgraph{\resource{object file} \ar[r] & \toolbox{ppc32dism} \ar[r] & \resource{disassembly\\listing}}
\seeassembly\seeppc\seeobject
}

\providecommand{\ppcbasm}{
\toolsection{ppc64asm} is an assembler for the PowerPC hardware architecture.
It translates assembly code into machine code for PowerPC processors and stores it in corresponding object files.
By default, the assembler generates machine code for the 64-bit operating mode defined by the PowerPC architecture.
\flowgraph{\resource{PowerPC assembly\\source code} \ar[r] & \toolbox{ppc64asm} \ar[r] & \resource{object file}}
\seeassembly\seeppc\seeobject
}

\providecommand{\ppcbdism}{
\toolsection{ppc64dism} is a disassembler for the PowerPC hardware architecture.
It translates machine code from object files targeting PowerPC processors into assembly code and writes it to the standard output stream.
It assumes that the machine code was generated for the 64-bit operating mode defined by the PowerPC architecture.
\flowgraph{\resource{object file} \ar[r] & \toolbox{ppc64dism} \ar[r] & \resource{disassembly\\listing}}
\seeassembly\seeppc\seeobject
}

\providecommand{\riscasm}{
\toolsection{riscasm} is an assembler for the RISC hardware architecture.
It translates assembly code into machine code for RISC processors and stores it in corresponding object files.
The names of all special registers are predefined and evaluate to the corresponding number.
\flowgraph{\resource{RISC assembly\\source code} \ar[r] & \toolbox{riscasm} \ar[r] & \resource{object file}}
\seeassembly\seerisc\seeobject
}

\providecommand{\riscdism}{
\toolsection{riscdism} is a disassembler for the RISC hardware architecture.
It translates machine code from object files targeting RISC processors into assembly code and writes it to the standard output stream.
\flowgraph{\resource{object file} \ar[r] & \toolbox{riscdism} \ar[r] & \resource{disassembly\\listing}}
\seeassembly\seerisc\seeobject
}

\providecommand{\wasmasm}{
\toolsection{wasmasm} is an assembler for the WebAssembly architecture.
It translates assembly code into machine code for WebAssembly targets and stores it in corresponding object files.
The names of all special registers are predefined and evaluate to the corresponding number.
\flowgraph{\resource{WebAssembly assembly\\source code} \ar[r] & \toolbox{wasmasm} \ar[r] & \resource{object file}}
\seeassembly\seewasm\seeobject
}

\providecommand{\wasmdism}{
\toolsection{wasmdism} is a disassembler for the WebAssembly architecture.
It translates machine code from object files targeting WebAssembly targets into assembly code and writes it to the standard output stream.
\flowgraph{\resource{object file} \ar[r] & \toolbox{wasmdism} \ar[r] & \resource{disassembly\\listing}}
\seeassembly\seewasm\seeobject
}

% linker tools

\providecommand{\linklib}{
\toolsection{linklib} is an object file combiner.
It creates a static library file by combining all object files given to it into a single one.
\flowgraph{\resource{object files} \ar[r] & \toolbox{linklib} \ar[r] & \resource{library file}}
\seeobject
}

\providecommand{\linkbin}{
\toolsection{linkbin} is a linker for plain binary files.
It links all object files given to it into a single image and stores it in a binary file that begins with the first linked section.
It also creates a map file that lists the address, type, name and size of all used sections.
The filename extension of the resulting binary file can be specified by putting it into a constant data section called \texttt{\_extension}.
\flowgraph{\resource{object files} \ar[r] & \toolbox{linkbin} \ar[r] \ar[d] & \resource{binary file} \\ & \resource{map file}}
\seeobject
}

\providecommand{\linkmem}{
\toolsection{linkmem} is a linker for plain binary files partitioned into random-access and read-only memory.
It links all object files given to it into two distinct images, one for data sections and one for code and constant data sections, and stores each image in a binary file that begins with the first linked section of the corresponding type.
It also creates a map file that lists the address, type, name and size of all used sections.
\flowgraph{\resource{object files} \ar[r] & \toolbox{linkmem} \ar[r] \ar[d] & \resource{RAM file/\\ROM file} \\ & \resource{map file}}
\seeobject
}

\providecommand{\linkprg}{
\toolsection{linkprg} is a linker for GEMDOS executable files.
It links all object files given to it into a single image and stores the image in an Atari GEMDOS executable file~\cite{gemdosfile}.
It also creates a map file that lists the address relative to the text segment, type, name and size of all used sections.
The filename extension of the resulting executable file can be specified by putting it into a constant data section called \texttt{\_extension}.
The GEMDOS executable file format requires all patch patterns of absolute link patches to consist of four full bitmasks with descending offsets.
\flowgraph{\resource{object files} \ar[r] & \toolbox{linkprg} \ar[r] \ar[d] & \resource{executable file} \\ & \resource{map file}}
\seeobject
}

\providecommand{\linkhex}{
\toolsection{linkhex} is a linker for Intel HEX files.
It links all code sections of the object files given to it into single image and stores the image in an Intel HEX file~\cite{hexfile} that begins with the first linked section.
It also creates a map file that lists the address, type, name and size of all used sections.
\flowgraph{\resource{object files} \ar[r] & \toolbox{linkhex} \ar[r] \ar[d] & \resource{HEX file} \\ & \resource{map file}}
\seeobject
}

\providecommand{\mapsearch}{
\toolsection{mapsearch} is a debugging tool.
It searches map files generated by linker tools for the name of a binary section that encompasses a memory address read from the standard input stream.
If additionally provided with one or more object files, it also stores an excerpt thereof in a separate object file called map search result which only contains the identified binary section for disassembling purposes.
\flowgraph{& \resource{map files/\\object files} \ar[d] \\ \resource{memory\\address} \ar[r] & \toolbox{mapsearch} \ar[r] \ar[d] & \resource{section name/\\relative offset} \\ & \resource{object file\\excerpt}}
\seeobject
}

\renewcommand{\seeppc}{}

\startchapter{PowerPC}{PowerPC Hardware Architecture Support}{ppc}
{This \documentation{} describes how the \ecs{} supports the PowerPC hardware architecture.
This includes information about the assembler, disassembler, and the various compilers featured by the \ecs{} as well as the interoperability between these tools.}

\section{Introduction}

The \ecs{} features various compilers, assemblers, and disassemblers that target the PowerPC hardware architecture by AIM\@.
Figure~\ref{fig:ppcdataflow} shows the data flow in-between these tools.

\begin{figure}
\flowgraph{
\resource{intermediate\\code} \ar[d] & & \resource{assembly\\source code} \ar[d] \\
\converter{PowerPC\\Generator} \ar[r] \ar[rd] \ar[d] & \resource{assembly\\listing} \ar[r] & \converter{PowerPC\\Assembler} \ar[ld] \\
\resource{debugging\\information} & \resource{object file} \ar[d] \\
& \converter{PowerPC\\Disassembler} \ar[d] \\
& \resource{disassembly\\listing} \\
}\caption{Data flow within the tools targeting the PowerPC architecture}
\label{fig:ppcdataflow}
\end{figure}

All compilers targeting the PowerPC architecture translate their programs using an intermediate code representation.
The PowerPC generator is able to translate the intermediate code representation of a program into machine code for PowerPC processors.
It stores the resulting binary code and data in so-called object files.
Additionally, the generator is able to create an assembly code listing of the machine code for debugging purposes.
This assembly code listing can also be processed by the assemblers yielding exactly the same object file.
The disassemblers are able to open object files and print a human-readable disassembly listing of their contents.
\seeobject\seecode

\section{Instruction Set}

Tools targeting the PowerPC architecture support the instruction set listed in Table~\ref{tab:ppcset} and use the same assembly syntax as predefined by IBM~\cite{ppc:instructionset}.
\seeassembly

\instructionset{ppc}{Supported PowerPC instruction set}{5}{6}

The actual operating mode the compilers and assemblers generate machine code for by default, is indicated by the number in the suffix of their names.
The assemblers allow users to temporarily switch between the supported operating modes by passing 32 or 64 as operand to the bit mode directive.

\section{Calling Convention}\index{Calling convention!of PowerPC}

The machine code generator and runtime support for the PowerPC architecture as provided by the \ecs{} use the following calling convention in order to enable interoperability.

\subsection{Stack Operations}

Arguments for functions are in general passed using the stack according to the intermediate code specification.
See \Documentation{}~\documentationref{code}{Intermediate Code Representation} for more information about the role of the stack.
Function arguments are pushed on the stack in reverse order and cleaned by the caller.

\subsection{Register Mapping}

The special-purpose registers defined by the intermediate code representation are mapped to their corresponding physical registers in the following way:

\begin{itemize}

\item Result Register\alignright\texttt{\$res}\nopagebreak

The intermediate code result register \texttt{\$res} is mapped to PowerPC registers \texttt{r2} and \texttt{r3} depending on the size of the actual result value.
Floating-point results are stored in the register \texttt{f0}.

\item Stack Pointer Register\alignright\texttt{\$sp}\nopagebreak

The intermediate code stack pointer register \texttt{\$sp} is mapped to the PowerPC register \texttt{r29}.

\item Frame Pointer Register\alignright\texttt{\$fp}\nopagebreak

The intermediate code frame pointer register \texttt{\$fp} is mapped to the PowerPC register \texttt{r30}.

\item Link Register\alignright\texttt{\$lnk}\nopagebreak

The intermediate code link register \texttt{\$lnk} is not supported.

\end{itemize}

All other intermediate code registers are mapped as needed to the remaining physical registers.
Their contents and mapping are therefore considered volatile across function calls.
Registers holding integer and address values are mapped to the general-purpose registers,
while registers holding floating-point values are mapped to the floating-point registers.

\section{Runtime Support}\index{Runtime support!for PowerPC}

The \ecs{} provides runtime support for the PowerPC architecture and runtime environments based on this hardware architecture in object files.
Users targeting a specific runtime environment have to use an appropriate linker together with these object files in order create an executable program.
This section gives information about all supported runtime environments based on the PowerPC hardware architecture as well as the required combination of linker and object files.

Basic architectural runtime support is provided by the object files \objfile{ppc32\-run} and \objfile{ppc64\-run}.
Users should always include one of these object files during linking regardless of the actual target runtime environment.
All other object files given to the linker should target the same hardware architecture.

Programs written in \cpp{} need additional runtime support stored in the \libfile{cpp\-ppc32\-run} and \libfile{cpp\-ppc64\-run} library files respectively.
Programs written in Oberon need additional runtime support stored in the \libfile{ob\-ppc32\-run} and \libfile{ob\-ppc64\-run} library files respectively.
\seecpp\seeoberon

\section{PowerPC Tools}

The \ecs{} provides the following tools that are able to process object files targeting the PowerPC hardware architecture.
\interface

\cdppca
\cdppcb
\cppppca
\cppppcb
\falppca
\falppcb
\obppca
\obppcb
\ppcaasm
\ppcbasm
\ppcadism
\ppcbdism
\linkbin

\concludechapter

% RISC architecture documentation
% Copyright (C) Florian Negele

% This file is part of the Eigen Compiler Suite.

% Permission is granted to copy, distribute and/or modify this document
% under the terms of the GNU Free Documentation License, Version 1.3
% or any later version published by the Free Software Foundation.

% You should have received a copy of the GNU Free Documentation License
% along with the ECS.  If not, see <https://www.gnu.org/licenses/>.

% Generic documentation utilities
% Copyright (C) Florian Negele

% This file is part of the Eigen Compiler Suite.

% Permission is granted to copy, distribute and/or modify this document
% under the terms of the GNU Free Documentation License, Version 1.3
% or any later version published by the Free Software Foundation.

% You should have received a copy of the GNU Free Documentation License
% along with the ECS.  If not, see <https://www.gnu.org/licenses/>.

\providecommand{\cpp}{C\texttt{++}}
\providecommand{\opt}{_\mathit{opt}}
\providecommand{\tool}[1]{\texttt{#1}}
\providecommand{\version}{Version 0.0.40}
\providecommand{\resource}[1]{*++\txt{#1}}
\providecommand{\ecs}{Eigen Compiler Suite}
\providecommand{\changed}[1]{\underline{#1}}
\providecommand{\toolbox}[1]{\converter{#1}}
\providecommand{\file}{}\renewcommand{\file}[1]{\texttt{#1}}
\providecommand{\alignright}{\hfill\linebreak[0]\hspace*{\fill}}
\providecommand{\converter}[1]{*++[F][F*:white][F,:gray]\txt{#1}}
\providecommand{\documentation}{\ifbook chapter\else document\fi}
\providecommand{\Documentation}{\ifbook Chapter\else Document\fi}
\providecommand{\variable}[1]{\resource{\texttt{\small#1}\\variable}}
\providecommand{\documentationref}[2]{\ifbook\ref{#1}\else``\href{#1}{#2}''~\cite{#1}\fi}
\providecommand{\objfile}[1]{\texttt{#1}\index[runtime]{#1 object file@\texttt{#1} object file}}
\providecommand{\libfile}[1]{\texttt{#1}\index[runtime]{#1 library file@\texttt{#1} library file}}
\providecommand{\epigraph}[2]{\ifbook\begin{quote}\flushright\textit{#1}\par--- #2\end{quote}\fi}
\providecommand{\environmentvariable}[1]{\texttt{#1}\index{Environment variables!#1@\texttt{#1}}}
\providecommand{\environment}[1]{\texttt{#1}\index[environment]{#1 environment@\texttt{#1} environment}}
\providecommand{\toolsection}{}\renewcommand{\toolsection}[1]{\subsection{#1}\label{\prefix:#1}\tool{#1}}
\providecommand{\instruction}{}\renewcommand{\instruction}[2]{\noindent\qquad\pdftooltip{\texttt{#1}}{#2}\refstepcounter{instruction}\par}
\providecommand{\flowgraph}{}\renewcommand{\flowgraph}[1]{\par\sffamily\begin{displaymath}\xymatrix@=4ex{#1}\end{displaymath}\normalfont\par}
\providecommand{\instructionset}{}\renewcommand{\instructionset}[4]{\setcounter{instruction}{0}\begin{multicols}{\ifbook#3\else#4\fi}[{\captionof{table}[#2]{#2 (\ref*{#1:instructions}~instructions)}\label{tab:#1set}\vspace{-2ex}}]\footnotesize\raggedcolumns\input{#1.set}\label{#1:instructions}\end{multicols}}

\providecommand{\gpl}{GNU General Public License}
\providecommand{\rse}{ECS Runtime Support Exception}
\providecommand{\fdl}{\href{https://www.gnu.org/licenses/fdl.html}{GNU Free Documentation License}}

\providecommand{\docbegin}{}
\providecommand{\docend}{}
\providecommand{\doclabel}[1]{\hypertarget{#1}}
\providecommand{\doclink}[2]{\hyperlink{#1}{#2}}
\providecommand{\docsection}[3]{\hypertarget{#1}{\subsection{#2}}\label{sec:#1}\index[library]{#2@#3}}
\providecommand{\docsectionstar}[1]{}
\providecommand{\docsubbegin}{\begin{description}}
\providecommand{\docsubend}{\end{description}}
\providecommand{\docsubsection}[3]{\item[\hypertarget{#1}{#2}]\index[library]{#2@#3}}
\providecommand{\docsubsectionstar}[1]{\smallskip}
\providecommand{\docsubsubsection}[3]{\docsubsection{#1}{#2}{#3}}
\providecommand{\docsubsubsectionstar}[1]{}
\providecommand{\docsubsubsubsection}[3]{}
\providecommand{\docsubsubsubsectionstar}[1]{}
\providecommand{\doctable}{}

\providecommand{\debuggingtool}{}\renewcommand{\debuggingtool}{This tool is provided for debugging purposes.
It allows exposing and modifying an internal data structure that is usually not accessible.
}

\providecommand{\interface}{All tools accept command-line arguments which are taken as names of plain text files containing the source code.
If no arguments are provided, the standard input stream is used instead.
Output files are generated in the current working directory and have the same name as the input file being processed whereas the filename extension gets replaced by an appropriate suffix.
\seeinterface
}

\providecommand{\license}{\noindent Copyright \copyright{} Florian Negele\par\medskip\noindent
Permission is granted to copy, distribute and/or modify this document under the terms of the
\fdl{}, Version 1.3 or any later version published by the \href{https://fsf.org/}{Free Software Foundation}.
}

\providecommand{\ecslogosurface}{
\fill[darkgray] (0,0,0) -- (0,0,3) -- (0,3,3) -- (0,3,1) -- (0,4,1) -- (0,4,3) -- (0,5,3) -- (0,5,0) -- (0,2,0) -- (0,2,2) -- (0,1,2) -- (0,1,0) -- cycle;
\fill[gray] (0,5,0) -- (0,5,3) -- (1,5,3) -- (1,5,1) -- (2,5,1) -- (2,5,3) -- (3,5,3) -- (3,5,0) -- cycle;
\fill[lightgray] (0,0,0) -- (0,1,0) -- (2,1,0) -- (2,4,0) -- (1,4,0) -- (1,3,0) -- (2,3,0) -- (2,2,0) -- (0,2,0) -- (0,5,0) -- (3,5,0) -- (3,0,0) -- cycle;
\begin{scope}[line width=0.5]
\begin{scope}[gray]
\draw (0,0,0) -- (0,1,0);
\draw (2,1,0) -- (2,2,0);
\draw (0,1,2) -- (0,2,2);
\draw (0,2,0) -- (0,5,0);
\draw (2,3,0) -- (2,4,0);
\end{scope}
\begin{scope}[lightgray]
\draw (0,1,0) -- (0,1,2);
\draw (0,3,1) -- (0,3,3);
\draw (0,5,0) -- (0,5,3);
\draw (2,5,1) -- (2,5,3);
\end{scope}
\begin{scope}[white]
\draw (0,1,0) -- (2,1,0);
\draw (1,3,0) -- (2,3,0);
\draw (0,5,0) -- (3,5,0);
\end{scope}
\end{scope}
}

\providecommand{\ecslogo}[1]{
\begin{tikzpicture}[scale={(#1)/((sin(45)+cos(45))*3cm)},x={({-cos(45)*1cm},{sin(45)*sin(30)*1cm})},y={({0cm},{(cos(30)*1cm})},z={({sin(45)*1cm},{cos(45)*sin(30)*1cm})}]
\begin{scope}[darkgray,line width=1]
\draw (0,0,0) -- (0,0,3) -- (0,3,3) -- (2,3,3) -- (2,5,3) -- (3,5,3) -- (3,5,0) -- (3,0,0) -- cycle;
\draw (0,3,1) -- (0,4,1) -- (0,4,3) -- (0,5,3) -- (1,5,3) -- (1,5,1) -- (2,5,1);
\draw (1,3,0) -- (1,4,0) -- (2,4,0);
\end{scope}
\fill[darkgray] (2,0,0) -- (2,0,3) -- (2,5,3) -- (2,5,1) -- (2,4,1) -- (2,4,0) -- cycle;
\fill[lightgray] (2,0,2) -- (0,0,2) -- (0,2,2) -- (2,2,2) -- cycle;
\fill[gray] (0,1,0) -- (2,1,0) -- (2,1,2) -- (0,1,2) -- cycle;
\fill[gray] (0,3,1) -- (0,3,3) -- (2,3,3) -- (2,3,0) -- (1,3,0) -- (1,3,1) -- cycle;
\ecslogosurface
\end{tikzpicture}
}

\providecommand{\shadowedecslogo}[3]{
\begin{tikzpicture}[scale={(#1)/((sin(#2)+cos(#2))*3cm)},x={({-cos(#2)*1cm},{sin(#2)*sin(#3)*1cm})},y={({0cm},{(cos(#3)*1cm})},z={({sin(#2)*1cm},{cos(#2)*sin(#3)*1cm})}]
\shade[top color=lightgray!50!white,bottom color=white,middle color=lightgray!50!white] (0,0,0) -- (3,0,0) -- (3,{-0.5-3*sin(#2)*sin(#3)/cos(#3)},0) -- (0,-0.5,0) -- cycle;
\shade[top color=darkgray!50!gray,bottom color=white,middle color=darkgray!50!white] (0,0,0) -- (0,0,3) -- (0,{-0.5-3*cos(#2)*sin(#3)/cos(#3)},3) -- (0,-0.5,0) -- cycle;
\begin{scope}[y={({(cos(#2)+sin(#2))*0.5cm},{(cos(#2)*sin(#3)-sin(#2)*sin(#3))*0.5cm})}]
\useasboundingbox (3,0,0) -- (0,0,0) -- (0,0,3);
\shade[left color=darkgray!80!black,right color=lightgray,middle color=gray] (0,0,0) -- (0,1,0) -- (0,1,0.5) -- (0,2,0) -- (0,5,0) -- (0,5,3) -- (1,5,3) -- (1,4,3) -- (1,4,2.5) -- (1,3,3) -- (2,5,3) -- (3,5,3) -- (3,0,3) -- cycle;
\clip (0,0,0) -- (0,0,3) -- ({-3*sin(#2)/cos(#2)},0,0) -- cycle;
\shade[left color=darkgray,right color=lightgray!50!gray] (0,0,0) -- (0,1,0) -- (0,1,0.5) -- (0,2,0) -- (0,5,0) -- (0,5,3) -- (1,5,3) -- (1,4,3) -- (1,4,2.5) -- (1,3,3) -- (2,5,3) -- (3,5,3) -- (3,0,3) -- cycle;
\end{scope}
\shade[left color=darkgray,right color=darkgray!80!black] (2,0,0) -- (2,0,3) -- (2,5,3) -- (2,5,1) -- (2,4,1) -- (2,4,0) -- cycle;
\shade[left color=darkgray!90!black,right color=gray!80!darkgray] (2,0,2) -- (0,0,2) -- (0,2,2) -- (2,2,2) -- cycle;
\shade[top color=darkgray!90!black,bottom color=gray!80!darkgray] (0,1,0) -- (2,1,0) -- (2,1,2) -- (0,1,2) -- cycle;
\shade[top color=darkgray!90!black,bottom color=gray!80!darkgray] (0,3,1) -- (0,3,3) -- (2,3,3) -- (2,3,0) -- (1,3,0) -- (1,3,1) -- cycle;
\fill[gray] (2,1,0) -- (1.5,1,0.5) -- (0,1,0.5) -- (0,1,0) -- cycle;
\fill[gray] (1,3,2) -- (0.5,3,2) -- (0.5,3,3) -- (1,3,3) -- cycle;
\fill[gray] (2,3,0) -- (1.5,3,0.5) -- (1,3,0.5) -- (1,3,0) -- cycle;
\ecslogosurface
\end{tikzpicture}
}

\providecommand{\cpplogo}[1]{
\begin{tikzpicture}[scale=(#1)/512em]
\fill[gray] (435.2794,398.7159) -- (247.1911,507.3075) .. controls (236.3563,513.5642) and (218.6240,513.5642) .. (207.7892,507.3075) -- (19.7009,398.7159) .. controls (8.8646,392.4606) and (0.0000,377.1043) .. (0.0000,364.5924) -- (0.0000,147.4076) .. controls (0.8430,132.8363) and (8.2856,120.7683) .. (19.7009,113.2842) -- (207.7892,4.6926) .. controls (218.6240,-1.5642) and (236.3564,-1.5642) .. (247.1911,4.6926) -- (435.2794,113.2842) .. controls (447.5273,121.4304) and (454.4987,133.6918) .. (454.9803,147.4076) -- (454.9803,364.5924) .. controls (454.5404,377.7571) and (446.6566,391.0351) .. (435.2794,398.7159) -- cycle(75.8301,255.9993) .. controls (74.9389,404.0881) and (273.2892,469.4783) .. (358.8263,331.8769) -- (293.1917,293.8965) .. controls (253.5702,359.4301) and (155.1909,335.9977) .. (151.6601,255.9993) .. controls (152.7204,182.2703) and (249.4137,148.0211) .. (293.1961,218.1065) -- (358.8308,180.1276) .. controls (283.4477,49.2645) and (79.6318,96.3470) .. (75.8301,255.9993) -- cycle(379.1503,247.5747) -- (362.2982,247.5747) -- (362.2982,230.7226) -- (345.4490,230.7226) -- (345.4490,247.5747) -- (328.5969,247.5747) -- (328.5969,264.4254) -- (345.4490,264.4254) -- (345.4490,281.2759) -- (362.2982,281.2759) -- (362.2982,264.4254) -- (379.1503,264.4254) -- cycle(442.3420,247.5747) -- (425.4899,247.5747) -- (425.4899,230.7226) -- (408.6408,230.7226) -- (408.6408,247.5747) -- (391.7886,247.5747) -- (391.7886,264.4254) -- (408.6408,264.4254) -- (408.6408,281.2759) -- (425.4899,281.2759) -- (425.4899,264.4254) -- (442.3420,264.4254) -- cycle;
\end{tikzpicture}
}

\providecommand{\fallogo}[1]{
\begin{tikzpicture}[scale=(#1)/512em]
\fill[gray] (185.7774,0.0000) .. controls (200.4486,15.9798) and (226.8966,8.7148) .. (235.0426,31.5836) .. controls (249.5297,58.0598) and (247.9581,97.9161) .. (280.3335,110.9762) .. controls (309.1690,120.3496) and (337.8406,104.2727) .. (366.5753,103.9379) .. controls (373.4449,111.5171) and (379.2885,128.2574) .. (383.9755,108.9744) .. controls (396.6979,102.5615) and (437.2808,107.6681) .. (426.9652,124.3252) .. controls (408.9822,121.0785) and (412.4742,146.0729) .. (426.5192,131.4996) .. controls (433.8413,120.8489) and (465.1541,126.5522) .. (441.9067,135.7950) .. controls (396.1879,157.7478) and (344.1112,161.5079) .. (298.5528,183.5702) .. controls (277.7471,193.5198) and (284.6941,218.7163) .. (285.2127,236.9640) .. controls (292.3599,316.2826) and (307.3929,394.6311) .. (317.1198,473.6154) .. controls (329.0637,505.4736) and (292.1195,528.5004) .. (265.9183,511.2761) .. controls (237.9284,499.2462) and (237.3684,465.2681) .. (230.9102,439.9421) .. controls (218.6692,374.3397) and (215.6307,306.9662) .. (198.1732,242.3977) .. controls (183.1379,232.7444) and (164.4245,256.0298) .. (149.0430,261.4799) .. controls (116.9328,279.2585) and (87.1822,308.5851) .. (48.2293,307.8914) .. controls (21.3220,306.9037) and (-15.9107,281.8761) .. (7.2921,252.7908) .. controls (29.7799,220.6177) and (67.5177,204.3028) .. (100.9287,185.9449) .. controls (130.8217,170.8906) and (161.1548,156.5903) .. (191.0278,141.5847) .. controls (196.1738,120.0520) and (186.6049,95.2409) .. (186.8382,72.4353) .. controls (185.5234,48.4204) and (183.1700,23.9341) .. (185.7774,0.0000) -- cycle;
\end{tikzpicture}
}

\providecommand{\oblogo}[1]{
\begin{tikzpicture}[scale=(#1)/512em]
\fill[gray] (160.3865,208.9117) .. controls (154.0879,214.6478) and (149.0735,221.2409) .. (145.4125,228.5384) .. controls (184.8790,248.4273) and (234.7122,269.8787) .. (297.5493,291.8782) .. controls (300.3943,281.4769) and (300.9552,268.7619) .. (300.4023,255.2389) .. controls (248.9909,244.7891) and (200.0310,225.9279) .. (160.3865,208.9117) -- cycle(225.7398,392.6996) .. controls (308.0209,392.1716) and (359.3326,345.9277) .. (368.7203,285.2098) .. controls (376.6742,197.1784) and (311.7194,141.3342) .. (205.4287,142.1456) .. controls (139.9485,141.4804) and (88.7155,166.1957) .. (73.5775,228.0086) .. controls (52.0297,320.3408) and (123.4078,391.0103) .. (225.7398,392.6996) -- cycle(216.0739,176.4733) .. controls (268.9183,179.2424) and (315.8292,206.5488) .. (312.7454,265.1139) .. controls (313.2769,315.6384) and (286.5993,353.4946) .. (216.6040,355.7934) .. controls (162.4657,355.7934) and (126.0914,317.5023) .. (126.0914,260.5103) .. controls (126.1733,214.2900) and (163.3363,176.2849) .. (216.0739,176.4733) -- cycle(76.4897,189.1754) .. controls (13.1586,147.5631) and (0.0000,119.4207) .. (0.0000,119.4207) -- (90.6499,170.1632) .. controls (85.3004,175.8497) and (80.5994,182.1633) .. (76.4897,189.1754) -- cycle(353.9486,119.3004) -- (402.9482,119.3004) .. controls (427.0025,137.0797) and (450.9893,162.7034) .. (474.9529,191.0213) .. controls (509.3540,228.5339) and (531.3391,294.2091) .. (487.8149,312.1206) .. controls (462.8165,324.7652) and (394.3874,316.8943) .. (373.8912,313.6651) .. controls (379.9291,297.7449) and (383.2899,278.4204) .. (381.4989,257.7214) .. controls (420.3069,248.0321) and (421.9610,218.3461) .. (407.7867,192.6417) .. controls (391.1113,162.4018) and (370.1114,132.9097) .. (353.9486,119.3004) -- cycle;
\end{tikzpicture}
}

\providecommand{\markuptable}{
\begin{table}
\sffamily\centering
\begin{tabular}{@{}lcl@{}}
\toprule
\texttt{//italics//} & $\rightarrow$ & \textit{italics} \\
\midrule
\texttt{**bold**} & $\rightarrow$ & \textbf{bold} \\
\midrule
\texttt{\# ordered list} & & 1 ordered list \\
\texttt{\# second item} & $\rightarrow$ & 2 second item \\
\texttt{\#\# sub item} & & \hspace{1em} 1 sub item \\
\midrule
\texttt{* unordered list} & & $\bullet$ unordered list \\
\texttt{* second item} & $\rightarrow$ & $\bullet$ second item \\
\texttt{** sub item} & & \hspace{1em} $\bullet$ sub item \\
\midrule
\texttt{link to [[label]]} & $\rightarrow$ & link to \underline{label} \\
\midrule
\texttt{<{}<label>{}> definition } & $\rightarrow$ & definition \\
\midrule
\texttt{[[url|link name]]} & $\rightarrow$ & \underline{link name} \\
\midrule\addlinespace
\texttt{= large heading} & & {\Large large heading} \smallskip \\
\texttt{== medium heading} & $\rightarrow$ & {\large medium heading} \\
\texttt{=== small heading} & & small heading \\
\midrule
\texttt{no line break} & & no line break for paragraphs \\
\texttt{for paragraphs} & $\rightarrow$ \\
& & use empty line \\
\texttt{use empty line} \\
\midrule
\texttt{force\textbackslash\textbackslash line break} & $\rightarrow$ & force \\
& & line break \\
\midrule
\texttt{horizontal line} & $\rightarrow$ & horizontal line \\
\texttt{----} & & \hrulefill \\
\midrule
\texttt{|=a|=table|=header} & & \underline{a \enspace table \enspace header} \\
\texttt{|a|table|row} & $\rightarrow$ & a \enspace table \enspace row \\
\texttt{|b|table|row} & & b \enspace table \enspace row \\
\midrule
\texttt{\{\{\{} \\
\texttt{unformatted} & $\rightarrow$ & \texttt{unformatted} \\
\texttt{code} & & \texttt{code} \\
\texttt{\}\}\}} \\
\midrule\addlinespace
\texttt{@ new article} & & {\Large 1.\ new article} \smallskip \\
\texttt{@ second article} & $\rightarrow$ & {\Large 2.\ second article} \smallskip \\
\texttt{@@ sub article} & & {\large 2.1.\ sub article} \\
\bottomrule
\end{tabular}
\normalfont\caption{Elements of the generic documentation markup language}
\label{tab:docmarkup}
\end{table}
}

\providecommand{\startchapter}[4]{
\documentclass[11pt,a4paper]{article}
\usepackage{booktabs}
\usepackage[format=hang,labelfont=bf]{caption}
\usepackage{changepage}
\usepackage[T1]{fontenc}
\usepackage[margin=2cm]{geometry}
\usepackage{hyperref}
\usepackage[american]{isodate}
\usepackage{lmodern}
\usepackage{longtable}
\usepackage{mathptmx}
\usepackage{microtype}
\usepackage[toc]{multitoc}
\usepackage{multirow}
\usepackage[all]{nowidow}
\usepackage{pdfcomment}
\usepackage{syntax}
\usepackage{tikz}
\usepackage[all]{xy}
\hypersetup{pdfborder={0 0 0},bookmarksnumbered=true,pdftitle={\ecs{}: #2},pdfauthor={Florian Negele},pdfsubject={\ecs{}},pdfkeywords={#1}}
\setlength{\grammarindent}{8em}\setlength{\grammarparsep}{0.2ex}
\setlength{\columnsep}{2em}
\newcommand{\prefix}{}
\newcounter{instruction}
\bibliographystyle{unsrt}
\renewcommand{\index}[2][]{}
\renewcommand{\arraystretch}{1.05}
\renewcommand{\floatpagefraction}{0.7}
\renewcommand{\syntleft}{\itshape}\renewcommand{\syntright}{}
\title{\vspace{-5ex}\Huge{\ecs{}}\medskip\hrule}
\author{\huge{#2}}
\date{\medskip\version}
\newif\ifbook\bookfalse
\pagestyle{headings}
\frenchspacing
\begin{document}
\maketitle\thispagestyle{empty}\noindent#4\setlength{\columnseprule}{0.4pt}\tableofcontents\setlength{\columnseprule}{0pt}\vfill\pagebreak[3]\null\vfill\bigskip\noindent
\parbox{\textwidth-4em}{\license The contents of this \documentation{} are part of the \href{manual}{\ecs{} User Manual}~\cite{manual} and correspond to Chapter ``\href{manual\##3}{#1}''.\alignright\mbox{\today}}
\parbox{4em}{\flushright\ecslogo{3em}}
\clearpage
}

\providecommand{\concludechapter}{
\vfill\pagebreak[3]\null\vfill
\thispagestyle{myheadings}\markright{REFERENCES}
\noindent\begin{minipage}{\textwidth}\begin{multicols}{2}[\section*{References}]
\renewcommand{\section}[2]{}\small\bibliography{references}
\end{multicols}\end{minipage}\end{document}
}

\providecommand{\startpresentation}[2]{
\documentclass[14pt,aspectratio=43,usepdftitle=false]{beamer}
\usepackage{booktabs}
\usepackage{etex}
\usepackage{multicol}
\usepackage{tikz}
\usepackage[all]{xy}
\bibliographystyle{unsrt}
\setlength{\columnsep}{1em}
\setlength{\leftmargini}{1em}
\setbeamercolor{title}{fg=black}
\setbeamercolor{structure}{fg=darkgray}
\setbeamercolor{bibliography item}{fg=darkgray}
\setbeamerfont{title}{series=\bfseries}
\setbeamerfont{subtitle}{series=\normalfont}
\setbeamerfont*{frametitle}{parent=title}
\setbeamerfont{block title}{series=\bfseries}
\setbeamerfont*{framesubtitle}{parent=subtitle}
\setbeamersize{text margin left=1em,text margin right=1em}
\setbeamertemplate{navigation symbols}{}
\setbeamertemplate{itemize item}[circle]{}
\setbeamertemplate{bibliography item}[triangle]{}
\setbeamertemplate{bibliography entry author}{\usebeamercolor[fg]{bibliography item}}
\setbeamertemplate{frametitle}{\medskip\usebeamerfont{frametitle}\color{gray}\raisebox{-2.5ex}[0ex][0ex]{\rule{0.1em}{4.5ex}}}
\addtobeamertemplate{frametitle}{}{\hspace{0.4em}\usebeamercolor[fg]{title}\insertframetitle\par\vspace{0.2ex}\hspace{0.5em}\usebeamerfont{framesubtitle}\insertframesubtitle}
\hypersetup{pdfborder={0 0 0},bookmarksnumbered=true,bookmarksopen=true,bookmarksopenlevel=0,pdftitle={\ecs{}: #1},pdfauthor={Florian Negele},pdfsubject={\ecs{}},pdfkeywords={#1}}
\renewcommand{\flowgraph}[1]{\resizebox{\textwidth}{!}{$$\xymatrix{##1}$$}}
\title{\ecs{}\medskip\hrule\medskip}
\institute{\shadowedecslogo{5em}{30}{15}}
\date{\version}
\subtitle{#1}
\begin{document}
\begin{frame}[plain]\titlepage\nocite{manual}\end{frame}
\begin{frame}{Contents}{#1}\begin{center}\tableofcontents\end{center}\end{frame}
}

\providecommand{\concludepresentation}{
\begin{frame}{References}\begin{footnotesize}\setlength{\columnseprule}{0.4pt}\begin{multicols}{2}\bibliography{references}\end{multicols}\end{footnotesize}\end{frame}
\end{document}
}

\providecommand{\startbook}[1]{
\documentclass[10pt,paper=17cm:24cm,DIV=13,twoside=semi,headings=normal,numbers=noendperiod,cleardoublepage=plain]{scrbook}
\usepackage{atveryend}
\usepackage{booktabs}
\usepackage{caption}
\usepackage{changepage}
\usepackage[T1]{fontenc}
\usepackage{imakeidx}
\usepackage{hyperref}
\usepackage[american]{isodate}
\usepackage{lmodern}
\usepackage{longtable}
\usepackage{mathptmx}
\usepackage[final]{microtype}
\usepackage{multicol}
\usepackage{multirow}
\usepackage[all]{nowidow}
\usepackage{pdfcomment}
\usepackage{scrlayer-scrpage}
\usepackage{setspace}
\usepackage{syntax}
\usepackage[eventxtindent=4pt,oddtxtexdent=4pt]{thumbs}
\usepackage{tikz}
\usepackage[all]{xy}
\hyphenation{Micro-Blaze Open-Cores Open-RISC Power-PC}
\hypersetup{pdfborder={0 0 0},bookmarksnumbered=true,bookmarksopen=true,bookmarksopenlevel=0,pdftitle={\ecs{}: #1},pdfauthor={Florian Negele},pdfsubject={\ecs{}},pdfkeywords={#1}}
\setlength{\grammarindent}{8em}\setlength{\grammarparsep}{0.7ex}
\setkomafont{captionlabel}{\usekomafont{descriptionlabel}}
\renewcommand{\arraystretch}{1.05}\setstretch{1.1}
\renewcommand{\chapterformat}{\thechapter\autodot\enskip\raisebox{-1ex}[0ex][0ex]{\color{gray}\rule{0.1em}{3.5ex}}\enskip}
\renewcommand{\startchapter}[4]{\hypertarget{##3}{\chapter{##1}}\label{##3}##4\addthumb{##1}{\LARGE\sffamily\bfseries\thechapter}{white}{gray}\renewcommand{\prefix}{##3}}
\renewcommand{\concludechapter}{\clearpage{\stopthumb\cleardoublepage}}
\renewcommand{\syntleft}{\itshape}\renewcommand{\syntright}{}
\renewcommand{\floatpagefraction}{0.7}
\renewcommand{\partheademptypage}{}
\DeclareMicrotypeAlias{lmss}{cmr}
\newcommand{\prefix}{}
\newcounter{instruction}
\bibliographystyle{unsrt}
\newif\ifbook\booktrue
\makeindex[intoc,title=Index]
\makeindex[intoc,name=tools,title=Index of Tools,columns=3]
\makeindex[intoc,name=library,title=Index of Library Names]
\makeindex[intoc,name=runtime,title=Index of Runtime Support]
\makeindex[intoc,name=environment,title=Index of Target Environments]
\indexsetup{toclevel=chapter,headers={\indexname}{\indexname}}
\frenchspacing
\begin{document}
\pagenumbering{alph}
\begin{titlepage}\centering
\huge\sffamily\null\vfill\textbf{\ecs{}}\bigskip\hrule\bigskip#1
\normalsize\normalfont\vfill\vfill\shadowedecslogo{10em}{30}{15}
\large\vfill\vfill\version
\end{titlepage}
\null\vfill
\thispagestyle{empty}
\noindent\today\par\medskip
\license A copy of this license is included in Appendix~\ref{fdl} on page~\pageref{fdl}.
All product names used herein are for identification purposes only and may be trademarks of their respective companies.
\concludechapter
\frontmatter
\setcounter{tocdepth}{1}
\tableofcontents
\setcounter{tocdepth}{2}
\concludechapter
\listoffigures
\concludechapter
\listoftables
\concludechapter
}

\providecommand{\concludebook}{
\backmatter
\addtocontents{toc}{\protect\setcounter{tocdepth}{-1}}
\phantomsection\addcontentsline{toc}{part}{Bibliography}
\bibliography{references}
\concludechapter
\phantomsection\addcontentsline{toc}{part}{Indexes}
\printindex
\concludechapter
\indexprologue{\label{idx:tools}}
\printindex[tools]
\concludechapter
\printindex[library]
\concludechapter
\indexprologue{\label{idx:runtime}}
\printindex[runtime]
\concludechapter
\indexprologue{\label{idx:environment}}
\printindex[environment]
\concludechapter
\pagestyle{empty}\pagenumbering{Alph}\null\clearpage
\null\vfill\centering\ecslogo{4em}\par\medskip\license
\end{document}
}

% chapter references

\providecommand{\seedocumentationref}{}\renewcommand{\seedocumentationref}[3]{#1, see \Documentation{}~\documentationref{#2}{#3}. }
\providecommand{\seeinterface}{}\renewcommand{\seeinterface}{\ifbook See \Documentation{}~\documentationref{interface}{User Interface} for more information about the common user interface of all of these tools. \fi}
\providecommand{\seeguide}{}\renewcommand{\seeguide}{\seedocumentationref{For basic examples of using some of these tools in practice}{guide}{User Guide}}
\providecommand{\seecpp}{}\renewcommand{\seecpp}{\seedocumentationref{For more information about the \cpp{} programming language and its implementation by the \ecs{}}{cpp}{User Manual for \cpp{}}}
\providecommand{\seefalse}{}\renewcommand{\seefalse}{\seedocumentationref{For more information about the FALSE programming language and its implementation by the \ecs{}}{false}{User Manual for FALSE}}
\providecommand{\seeoberon}{}\renewcommand{\seeoberon}{\seedocumentationref{For more information about the Oberon programming language and its implementation by the \ecs{}}{oberon}{User Manual for Oberon}}
\providecommand{\seeassembly}{}\renewcommand{\seeassembly}{\seedocumentationref{For more information about the generic assembly language and how to use it}{assembly}{Generic Assembly Language Specification}}
\providecommand{\seeamd}{}\renewcommand{\seeamd}{\seedocumentationref{For more information about how the \ecs{} supports the AMD64 hardware architecture}{amd64}{AMD64 Hardware Architecture Support}}
\providecommand{\seearm}{}\renewcommand{\seearm}{\seedocumentationref{For more information about how the \ecs{} supports the ARM hardware architecture}{arm}{ARM Hardware Architecture Support}}
\providecommand{\seeavr}{}\renewcommand{\seeavr}{\seedocumentationref{For more information about how the \ecs{} supports the AVR hardware architecture}{avr}{AVR Hardware Architecture Support}}
\providecommand{\seeavrtt}{}\renewcommand{\seeavrtt}{\seedocumentationref{For more information about how the \ecs{} supports the AVR32 hardware architecture}{avr32}{AVR32 Hardware Architecture Support}}
\providecommand{\seemabk}{}\renewcommand{\seemabk}{\seedocumentationref{For more information about how the \ecs{} supports the M68000 hardware architecture}{m68k}{M68000 Hardware Architecture Support}}
\providecommand{\seemibl}{}\renewcommand{\seemibl}{\seedocumentationref{For more information about how the \ecs{} supports the MicroBlaze hardware architecture}{mibl}{MicroBlaze Hardware Architecture Support}}
\providecommand{\seemips}{}\renewcommand{\seemips}{\seedocumentationref{For more information about how the \ecs{} supports the MIPS32 and MIPS64 hardware architectures}{mips}{MIPS Hardware Architecture Support}}
\providecommand{\seemmix}{}\renewcommand{\seemmix}{\seedocumentationref{For more information about how the \ecs{} supports the MMIX hardware architecture}{mmix}{MMIX Hardware Architecture Support}}
\providecommand{\seeorok}{}\renewcommand{\seeorok}{\seedocumentationref{For more information about how the \ecs{} supports the OpenRISC 1000 hardware architecture}{or1k}{OpenRISC 1000 Hardware Architecture Support}}
\providecommand{\seeppc}{}\renewcommand{\seeppc}{\seedocumentationref{For more information about how the \ecs{} supports the PowerPC hardware architecture}{ppc}{PowerPC Hardware Architecture Support}}
\providecommand{\seerisc}{}\renewcommand{\seerisc}{\seedocumentationref{For more information about how the \ecs{} supports the RISC hardware architecture}{risc}{RISC Hardware Architecture Support}}
\providecommand{\seewasm}{}\renewcommand{\seewasm}{\seedocumentationref{For more information about how the \ecs{} supports the WebAssembly architecture}{wasm}{WebAssembly Architecture Support}}
\providecommand{\seedocumentation}{}\renewcommand{\seedocumentation}{\seedocumentationref{For more information about generic documentations and their generation by the \ecs{}}{documentation}{Generic Documentation Generation}}
\providecommand{\seedebugging}{}\renewcommand{\seedebugging}{\seedocumentationref{For more information about debugging information and its representation}{debugging}{Debugging Information Representation}}
\providecommand{\seecode}{}\renewcommand{\seecode}{\seedocumentationref{For more information about intermediate code and its purpose}{code}{Intermediate Code Representation}}
\providecommand{\seeobject}{}\renewcommand{\seeobject}{\seedocumentationref{For more information about object files and their purpose}{object}{Object File Representation}}

% generic documentation tools

\providecommand{\docprint}{
\toolsection{docprint} is a pretty printer for generic documentations.
It reformats generic documentations and writes it to the standard output stream.
\debuggingtool
\flowgraph{\resource{generic\\documentation} \ar[r] & \toolbox{docprint} \ar[r] & \resource{generic\\documentation}}
\seedocumentation
}

\providecommand{\doccheck}{
\toolsection{doccheck} is a syntactic and semantic checker for generic documentations.
It just performs syntactic and semantic checks on generic documentations and writes its diagnostic messages to the standard error stream.
\debuggingtool
\flowgraph{\resource{generic\\documentation} \ar[r] & \toolbox{doccheck} \ar[r] & \resource{diagnostic\\messages}}
\seedocumentation
}

\providecommand{\dochtml}{
\toolsection{dochtml} is an HTML documentation generator for generic documentations.
It processes several generic documentations and assembles all information therein into an HTML document.
\debuggingtool
\flowgraph{\resource{generic\\documentation} \ar[r] & \toolbox{dochtml} \ar[r] & \resource{HTML\\document}}
\seedocumentation
}

\providecommand{\doclatex}{
\toolsection{doclatex} is a Latex documentation generator for generic documentations.
It processes several generic documentations and assembles all information therein into a Latex document.
\debuggingtool
\flowgraph{\resource{generic\\documentation} \ar[r] & \toolbox{doclatex} \ar[r] & \resource{Latex\\document}}
\seedocumentation
}

% intermediate code tools

\providecommand{\cdcheck}{
\toolsection{cdcheck} is a syntactic and semantic checker for intermediate code.
It just performs syntactic and semantic checks on programs written in intermediate code and writes its diagnostic messages to the standard error stream.
\debuggingtool
\flowgraph{\resource{intermediate\\code} \ar[r] & \toolbox{cdcheck} \ar[r] & \resource{diagnostic\\messages}}
\seeassembly\seecode
}

\providecommand{\cdopt}{
\toolsection{cdopt} is an optimizer for intermediate code.
It performs various optimizations on programs written in intermediate code and writes the result to the standard output stream.
\debuggingtool
\flowgraph{\resource{intermediate\\code} \ar[r] & \toolbox{cdopt} \ar[r] & \resource{optimized\\code}}
\seeassembly\seecode
}

\providecommand{\cdrun}{
\toolsection{cdrun} is an interpreter for intermediate code.
It processes and executes programs written in intermediate code.
The following code sections are predefined and have the usual semantics:
\texttt{abort}, \texttt{\_Exit}, \texttt{fflush}, \texttt{floor}, \texttt{fputc}, \texttt{free}, \texttt{getchar}, \texttt{malloc}, and \texttt{putchar}.
Diagnostic messages about invalid operations include the name of the executed code section and the index of the erroneous instruction.
\debuggingtool
\flowgraph{\resource{intermediate\\code} \ar[r] & \toolbox{cdrun} \ar@/u/[r] & \resource{input/\\output} \ar@/d/[l]}
\seeassembly\seecode
}

\providecommand{\cdamda}{
\toolsection{cdamd16} is a compiler for intermediate code targeting the AMD64 hardware architecture.
It generates machine code for AMD64 processors from programs written in intermediate code and stores it in corresponding object files.
The compiler generates machine code for the 16-bit operating mode defined by the AMD64 architecture.
It also creates a debugging information file as well as an assembly file containing a listing of the generated machine code.
\debuggingtool
\flowgraph{\resource{intermediate\\code} \ar[r] & \toolbox{cdamd16} \ar[r] \ar[d] \ar[rd] & \resource{object file} \\ & \resource{assembly\\listing} & \resource{debugging\\information}}
\seeassembly\seeamd\seeobject\seecode\seedebugging
}

\providecommand{\cdamdb}{
\toolsection{cdamd32} is a compiler for intermediate code targeting the AMD64 hardware architecture.
It generates machine code for AMD64 processors from programs written in intermediate code and stores it in corresponding object files.
The compiler generates machine code for the 32-bit operating mode defined by the AMD64 architecture.
It also creates a debugging information file as well as an assembly file containing a listing of the generated machine code.
\debuggingtool
\flowgraph{\resource{intermediate\\code} \ar[r] & \toolbox{cdamd32} \ar[r] \ar[d] \ar[rd] & \resource{object file} \\ & \resource{assembly\\listing} & \resource{debugging\\information}}
\seeassembly\seeamd\seeobject\seecode\seedebugging
}

\providecommand{\cdamdc}{
\toolsection{cdamd64} is a compiler for intermediate code targeting the AMD64 hardware architecture.
It generates machine code for AMD64 processors from programs written in intermediate code and stores it in corresponding object files.
The compiler generates machine code for the 64-bit operating mode defined by the AMD64 architecture.
It also creates a debugging information file as well as an assembly file containing a listing of the generated machine code.
\debuggingtool
\flowgraph{\resource{intermediate\\code} \ar[r] & \toolbox{cdamd64} \ar[r] \ar[d] \ar[rd] & \resource{object file} \\ & \resource{assembly\\listing} & \resource{debugging\\information}}
\seeassembly\seeamd\seeobject\seecode\seedebugging
}

\providecommand{\cdarma}{
\toolsection{cdarma32} is a compiler for intermediate code targeting the ARM hardware architecture.
It generates machine code for ARM processors executing A32 instructions from programs written in intermediate code and stores it in corresponding object files.
It also creates a debugging information file as well as an assembly file containing a listing of the generated machine code.
\debuggingtool
\flowgraph{\resource{intermediate\\code} \ar[r] & \toolbox{cdarma32} \ar[r] \ar[d] \ar[rd] & \resource{object file} \\ & \resource{assembly\\listing} & \resource{debugging\\information}}
\seeassembly\seearm\seeobject\seecode\seedebugging
}

\providecommand{\cdarmb}{
\toolsection{cdarma64} is a compiler for intermediate code targeting the ARM hardware architecture.
It generates machine code for ARM processors executing A64 instructions from programs written in intermediate code and stores it in corresponding object files.
It also creates a debugging information file as well as an assembly file containing a listing of the generated machine code.
\debuggingtool
\flowgraph{\resource{intermediate\\code} \ar[r] & \toolbox{cdarma64} \ar[r] \ar[d] \ar[rd] & \resource{object file} \\ & \resource{assembly\\listing} & \resource{debugging\\information}}
\seeassembly\seearm\seeobject\seecode\seedebugging
}

\providecommand{\cdarmc}{
\toolsection{cdarmt32} is a compiler for intermediate code targeting the ARM hardware architecture.
It generates machine code for ARM processors without floating-point extension executing T32 instructions from programs written in intermediate code and stores it in corresponding object files.
It also creates a debugging information file as well as an assembly file containing a listing of the generated machine code.
\debuggingtool
\flowgraph{\resource{intermediate\\code} \ar[r] & \toolbox{cdarmt32} \ar[r] \ar[d] \ar[rd] & \resource{object file} \\ & \resource{assembly\\listing} & \resource{debugging\\information}}
\seeassembly\seearm\seeobject\seecode\seedebugging
}

\providecommand{\cdarmcfpe}{
\toolsection{cdarmt32fpe} is a compiler for intermediate code targeting the ARM hardware architecture.
It generates machine code for ARM processors with floating-point extension executing T32 instructions from programs written in intermediate code and stores it in corresponding object files.
It also creates a debugging information file as well as an assembly file containing a listing of the generated machine code.
\debuggingtool
\flowgraph{\resource{intermediate\\code} \ar[r] & \toolbox{cdarmt32fpe} \ar[r] \ar[d] \ar[rd] & \resource{object file} \\ & \resource{assembly\\listing} & \resource{debugging\\information}}
\seeassembly\seearm\seeobject\seecode\seedebugging
}

\providecommand{\cdavr}{
\toolsection{cdavr} is a compiler for intermediate code targeting the AVR hardware architecture.
It generates machine code for AVR processors from programs written in intermediate code and stores it in corresponding object files.
It also creates a debugging information file as well as an assembly file containing a listing of the generated machine code.
\debuggingtool
\flowgraph{\resource{intermediate\\code} \ar[r] & \toolbox{cdavr} \ar[r] \ar[d] \ar[rd] & \resource{object file} \\ & \resource{assembly\\listing} & \resource{debugging\\information}}
\seeassembly\seeavr\seeobject\seecode\seedebugging
}

\providecommand{\cdavrtt}{
\toolsection{cdavr32} is a compiler for intermediate code targeting the AVR32 hardware architecture.
It generates machine code for AVR32 processors from programs written in intermediate code and stores it in corresponding object files.
It also creates a debugging information file as well as an assembly file containing a listing of the generated machine code.
\debuggingtool
\flowgraph{\resource{intermediate\\code} \ar[r] & \toolbox{cdavr32} \ar[r] \ar[d] \ar[rd] & \resource{object file} \\ & \resource{assembly\\listing} & \resource{debugging\\information}}
\seeassembly\seeavrtt\seeobject\seecode\seedebugging
}

\providecommand{\cdmabk}{
\toolsection{cdm68k} is a compiler for intermediate code targeting the M68000 hardware architecture.
It generates machine code for M68000 processors from programs written in intermediate code and stores it in corresponding object files.
It also creates a debugging information file as well as an assembly file containing a listing of the generated machine code.
\debuggingtool
\flowgraph{\resource{intermediate\\code} \ar[r] & \toolbox{cdm68k} \ar[r] \ar[d] \ar[rd] & \resource{object file} \\ & \resource{assembly\\listing} & \resource{debugging\\information}}
\seeassembly\seemabk\seeobject\seecode\seedebugging
}

\providecommand{\cdmibl}{
\toolsection{cdmibl} is a compiler for intermediate code targeting the MicroBlaze hardware architecture.
It generates machine code for MicroBlaze processors from programs written in intermediate code and stores it in corresponding object files.
It also creates a debugging information file as well as an assembly file containing a listing of the generated machine code.
\debuggingtool
\flowgraph{\resource{intermediate\\code} \ar[r] & \toolbox{cdmibl} \ar[r] \ar[d] \ar[rd] & \resource{object file} \\ & \resource{assembly\\listing} & \resource{debugging\\information}}
\seeassembly\seemibl\seeobject\seecode\seedebugging
}

\providecommand{\cdmipsa}{
\toolsection{cdmips32} is a compiler for intermediate code targeting the MIPS32 hardware architecture.
It generates machine code for MIPS32 processors from programs written in intermediate code and stores it in corresponding object files.
It also creates a debugging information file as well as an assembly file containing a listing of the generated machine code.
\debuggingtool
\flowgraph{\resource{intermediate\\code} \ar[r] & \toolbox{cdmips32} \ar[r] \ar[d] \ar[rd] & \resource{object file} \\ & \resource{assembly\\listing} & \resource{debugging\\information}}
\seeassembly\seemips\seeobject\seecode\seedebugging
}

\providecommand{\cdmipsb}{
\toolsection{cdmips64} is a compiler for intermediate code targeting the MIPS64 hardware architecture.
It generates machine code for MIPS64 processors from programs written in intermediate code and stores it in corresponding object files.
It also creates a debugging information file as well as an assembly file containing a listing of the generated machine code.
\debuggingtool
\flowgraph{\resource{intermediate\\code} \ar[r] & \toolbox{cdmips64} \ar[r] \ar[d] \ar[rd] & \resource{object file} \\ & \resource{assembly\\listing} & \resource{debugging\\information}}
\seeassembly\seemips\seeobject\seecode\seedebugging
}

\providecommand{\cdmmix}{
\toolsection{cdmmix} is a compiler for intermediate code targeting the MMIX hardware architecture.
It generates machine code for MMIX processors from programs written in intermediate code and stores it in corresponding object files.
It also creates a debugging information file as well as an assembly file containing a listing of the generated machine code.
\debuggingtool
\flowgraph{\resource{intermediate\\code} \ar[r] & \toolbox{cdmmix} \ar[r] \ar[d] \ar[rd] & \resource{object file} \\ & \resource{assembly\\listing} & \resource{debugging\\information}}
\seeassembly\seemmix\seeobject\seecode\seedebugging
}

\providecommand{\cdorok}{
\toolsection{cdor1k} is a compiler for intermediate code targeting the OpenRISC 1000 hardware architecture.
It generates machine code for OpenRISC 1000 processors from programs written in intermediate code and stores it in corresponding object files.
It also creates a debugging information file as well as an assembly file containing a listing of the generated machine code.
\debuggingtool
\flowgraph{\resource{intermediate\\code} \ar[r] & \toolbox{cdor1k} \ar[r] \ar[d] \ar[rd] & \resource{object file} \\ & \resource{assembly\\listing} & \resource{debugging\\information}}
\seeassembly\seeorok\seeobject\seecode\seedebugging
}

\providecommand{\cdppca}{
\toolsection{cdppc32} is a compiler for intermediate code targeting the PowerPC hardware architecture.
It generates machine code for PowerPC processors from programs written in intermediate code and stores it in corresponding object files.
The compiler generates machine code for the 32-bit operating mode defined by the PowerPC architecture.
It also creates a debugging information file as well as an assembly file containing a listing of the generated machine code.
\debuggingtool
\flowgraph{\resource{intermediate\\code} \ar[r] & \toolbox{cdppc32} \ar[r] \ar[d] \ar[rd] & \resource{object file} \\ & \resource{assembly\\listing} & \resource{debugging\\information}}
\seeassembly\seeppc\seeobject\seecode\seedebugging
}

\providecommand{\cdppcb}{
\toolsection{cdppc64} is a compiler for intermediate code targeting the PowerPC hardware architecture.
It generates machine code for PowerPC processors from programs written in intermediate code and stores it in corresponding object files.
The compiler generates machine code for the 64-bit operating mode defined by the PowerPC architecture.
It also creates a debugging information file as well as an assembly file containing a listing of the generated machine code.
\debuggingtool
\flowgraph{\resource{intermediate\\code} \ar[r] & \toolbox{cdppc64} \ar[r] \ar[d] \ar[rd] & \resource{object file} \\ & \resource{assembly\\listing} & \resource{debugging\\information}}
\seeassembly\seeppc\seeobject\seecode\seedebugging
}

\providecommand{\cdrisc}{
\toolsection{cdrisc} is a compiler for intermediate code targeting the RISC hardware architecture.
It generates machine code for RISC processors from programs written in intermediate code and stores it in corresponding object files.
It also creates a debugging information file as well as an assembly file containing a listing of the generated machine code.
\debuggingtool
\flowgraph{\resource{intermediate\\code} \ar[r] & \toolbox{cdrisc} \ar[r] \ar[d] \ar[rd] & \resource{object file} \\ & \resource{assembly\\listing} & \resource{debugging\\information}}
\seeassembly\seerisc\seeobject\seecode\seedebugging
}

\providecommand{\cdwasm}{
\toolsection{cdwasm} is a compiler for intermediate code targeting the WebAssembly architecture.
It generates machine code for WebAssembly targets from programs written in intermediate code and stores it in corresponding object files.
It also creates a debugging information file as well as an assembly file containing a listing of the generated machine code.
\debuggingtool
\flowgraph{\resource{intermediate\\code} \ar[r] & \toolbox{cdwasm} \ar[r] \ar[d] \ar[rd] & \resource{object file} \\ & \resource{assembly\\listing} & \resource{debugging\\information}}
\seeassembly\seewasm\seeobject\seecode\seedebugging
}

% C++ tools

\providecommand{\cppprep}{
\toolsection{cppprep} is a preprocessor for the \cpp{} programming language.
It preprocesses source code according to the rules of \cpp{} and writes it to the standard output stream.
Only the macro names \texttt{\_\_DATE\_\_}, \texttt{\_\_FILE\_\_}, \texttt{\_\_LINE\_\_}, and \texttt{\_\_TIME\_\_} are predefined.
\flowgraph{\resource{\cpp{} or other\\source code} \ar[r] & \toolbox{cppprep} \ar[r] & \resource{preprocessed\\source code} \\ & \variable{ECSINCLUDE} \ar[u]}
\seecpp
}

\providecommand{\cppprint}{
\toolsection{cppprint} is a pretty printer for the \cpp{} programming language.
It reformats the source code of \cpp{} programs and writes it to the standard output stream.
\flowgraph{\resource{\cpp{}\\source code} \ar[r] & \toolbox{cppprint} \ar[r] & \resource{reformatted\\source code} \\ & \variable{ECSINCLUDE} \ar[u]}
\seecpp
}

\providecommand{\cppcheck}{
\toolsection{cppcheck} is a syntactic and semantic checker for the \cpp{} programming language.
It just performs syntactic and semantic checks on \cpp{} programs and writes its diagnostic messages to the standard error stream.
\flowgraph{\resource{\cpp{}\\source code} \ar[r] & \toolbox{cppcheck} \ar[r] & \resource{diagnostic\\messages} \\ & \variable{ECSINCLUDE} \ar[u]}
\seecpp
}

\providecommand{\cppdump}{
\toolsection{cppdump} is a serializer for the \cpp{} programming language.
It dumps the complete internal representation of programs written in \cpp{} into an XML document.
\debuggingtool
\flowgraph{\resource{\cpp{}\\source code} \ar[r] & \toolbox{cppdump} \ar[r] & \resource{internal\\representation} \\ & \variable{ECSINCLUDE} \ar[u]}
\seecpp
}

\providecommand{\cpprun}{
\toolsection{cpprun} is an interpreter for the \cpp{} programming language.
It processes and executes programs written in \cpp{}.
The macro \texttt{\_\_run\_\_} is predefined in order to enable programmers to identify this tool while interpreting.
\flowgraph{\resource{\cpp{}\\source code} \ar[r] & \toolbox{cpprun} \ar@/u/[r] & \resource{input/\\output} \ar@/d/[l] \\ & \variable{ECSINCLUDE} \ar[u]}
\seecpp
}

\providecommand{\cppdoc}{
\toolsection{cppdoc} is a generic documentation generator for the \cpp{} programming language.
It processes several \cpp{} source files and assembles all information therein into a generic documentation.
\debuggingtool
\flowgraph{\resource{\cpp{}\\source code} \ar[r] & \toolbox{cppdoc} \ar[r] & \resource{generic\\documentation} \\ & \variable{ECSINCLUDE} \ar[u]}
\seecpp\seedocumentation
}

\providecommand{\cpphtml}{
\toolsection{cpphtml} is an HTML documentation generator for the \cpp{} programming language.
It processes several \cpp{} source files and assembles all information therein into an HTML document.
\flowgraph{\resource{\cpp{}\\source code} \ar[r] & \toolbox{cpphtml} \ar[r] & \resource{HTML\\document} \\ & \variable{ECSINCLUDE} \ar[u]}
\seecpp\seedocumentation
}

\providecommand{\cpplatex}{
\toolsection{cpplatex} is a Latex documentation generator for the \cpp{} programming language.
It processes several \cpp{} source files and assembles all information therein into a Latex document.
\flowgraph{\resource{\cpp{}\\source code} \ar[r] & \toolbox{cpplatex} \ar[r] & \resource{Latex\\document} \\ & \variable{ECSINCLUDE} \ar[u]}
\seecpp\seedocumentation
}

\providecommand{\cppcode}{
\toolsection{cppcode} is an intermediate code generator for the \cpp{} programming language.
It generates intermediate code from programs written in \cpp{} and stores it in corresponding assembly files.
The macro \texttt{\_\_code\_\_} is predefined in order to enable programmers to identify this tool while generating intermediate code.
Programs generated with this tool require additional runtime support that is stored in the \file{cpp\-code\-run} library file.
\debuggingtool
\flowgraph{\resource{\cpp{}\\source code} \ar[r] & \toolbox{cppcode} \ar[r] & \resource{intermediate\\code} \\ & \variable{ECSINCLUDE} \ar[u]}
\seecpp\seeassembly\seecode
}

\providecommand{\cppamda}{
\toolsection{cppamd16} is a compiler for the \cpp{} programming language targeting the AMD64 hardware architecture.
It generates machine code for AMD64 processors from programs written in \cpp{} and stores it in corresponding object files.
The compiler generates machine code for the 16-bit operating mode defined by the AMD64 architecture.
For debugging purposes, it also creates a debugging information file as well as an assembly file containing a listing of the generated machine code.
The macro \texttt{\_\_amd16\_\_} is predefined in order to enable programmers to identify this tool and its target architecture while compiling.
Programs generated with this compiler require additional runtime support that is stored in the \file{cpp\-amd16\-run} library file.
\flowgraph{\resource{\cpp{}\\source code} \ar[r] & \toolbox{cppamd16} \ar[r] \ar[d] \ar[rd] & \resource{object file} \\ \variable{ECSINCLUDE} \ar[ru] & \resource{debugging\\information} & \resource{assembly\\listing}}
\seecpp\seeassembly\seeamd\seeobject\seedebugging
}

\providecommand{\cppamdb}{
\toolsection{cppamd32} is a compiler for the \cpp{} programming language targeting the AMD64 hardware architecture.
It generates machine code for AMD64 processors from programs written in \cpp{} and stores it in corresponding object files.
The compiler generates machine code for the 32-bit operating mode defined by the AMD64 architecture.
For debugging purposes, it also creates a debugging information file as well as an assembly file containing a listing of the generated machine code.
The macro \texttt{\_\_amd32\_\_} is predefined in order to enable programmers to identify this tool and its target architecture while compiling.
Programs generated with this compiler require additional runtime support that is stored in the \file{cpp\-amd32\-run} library file.
\flowgraph{\resource{\cpp{}\\source code} \ar[r] & \toolbox{cppamd32} \ar[r] \ar[d] \ar[rd] & \resource{object file} \\ \variable{ECSINCLUDE} \ar[ru] & \resource{debugging\\information} & \resource{assembly\\listing}}
\seecpp\seeassembly\seeamd\seeobject\seedebugging
}

\providecommand{\cppamdc}{
\toolsection{cppamd64} is a compiler for the \cpp{} programming language targeting the AMD64 hardware architecture.
It generates machine code for AMD64 processors from programs written in \cpp{} and stores it in corresponding object files.
The compiler generates machine code for the 64-bit operating mode defined by the AMD64 architecture.
For debugging purposes, it also creates a debugging information file as well as an assembly file containing a listing of the generated machine code.
The macro \texttt{\_\_amd64\_\_} is predefined in order to enable programmers to identify this tool and its target architecture while compiling.
Programs generated with this compiler require additional runtime support that is stored in the \file{cpp\-amd64\-run} library file.
\flowgraph{\resource{\cpp{}\\source code} \ar[r] & \toolbox{cppamd64} \ar[r] \ar[d] \ar[rd] & \resource{object file} \\ \variable{ECSINCLUDE} \ar[ru] & \resource{debugging\\information} & \resource{assembly\\listing}}
\seecpp\seeassembly\seeamd\seeobject\seedebugging
}

\providecommand{\cpparma}{
\toolsection{cpparma32} is a compiler for the \cpp{} programming language targeting the ARM hardware architecture.
It generates machine code for ARM processors executing A32 instructions from programs written in \cpp{} and stores it in corresponding object files.
For debugging purposes, it also creates a debugging information file as well as an assembly file containing a listing of the generated machine code.
The macro \texttt{\_\_arma32\_\_} is predefined in order to enable programmers to identify this tool and its target architecture while compiling.
Programs generated with this compiler require additional runtime support that is stored in the \file{cpp\-arma32\-run} library file.
\flowgraph{\resource{\cpp{}\\source code} \ar[r] & \toolbox{cpparma32} \ar[r] \ar[d] \ar[rd] & \resource{object file} \\ \variable{ECSINCLUDE} \ar[ru] & \resource{debugging\\information} & \resource{assembly\\listing}}
\seecpp\seeassembly\seearm\seeobject\seedebugging
}

\providecommand{\cpparmb}{
\toolsection{cpparma64} is a compiler for the \cpp{} programming language targeting the ARM hardware architecture.
It generates machine code for ARM processors executing A64 instructions from programs written in \cpp{} and stores it in corresponding object files.
For debugging purposes, it also creates a debugging information file as well as an assembly file containing a listing of the generated machine code.
The macro \texttt{\_\_arma64\_\_} is predefined in order to enable programmers to identify this tool and its target architecture while compiling.
Programs generated with this compiler require additional runtime support that is stored in the \file{cpp\-arma64\-run} library file.
\flowgraph{\resource{\cpp{}\\source code} \ar[r] & \toolbox{cpparma64} \ar[r] \ar[d] \ar[rd] & \resource{object file} \\ \variable{ECSINCLUDE} \ar[ru] & \resource{debugging\\information} & \resource{assembly\\listing}}
\seecpp\seeassembly\seearm\seeobject\seedebugging
}

\providecommand{\cpparmc}{
\toolsection{cpparmt32} is a compiler for the \cpp{} programming language targeting the ARM hardware architecture.
It generates machine code for ARM processors without floating-point extension executing T32 instructions from programs written in \cpp{} and stores it in corresponding object files.
For debugging purposes, it also creates a debugging information file as well as an assembly file containing a listing of the generated machine code.
The macro \texttt{\_\_armt32\_\_} is predefined in order to enable programmers to identify this tool and its target architecture while compiling.
Programs generated with this compiler require additional runtime support that is stored in the \file{cpp\-armt32\-run} library file.
\flowgraph{\resource{\cpp{}\\source code} \ar[r] & \toolbox{cpparmt32} \ar[r] \ar[d] \ar[rd] & \resource{object file} \\ \variable{ECSINCLUDE} \ar[ru] & \resource{debugging\\information} & \resource{assembly\\listing}}
\seecpp\seeassembly\seearm\seeobject\seedebugging
}

\providecommand{\cpparmcfpe}{
\toolsection{cpparmt32fpe} is a compiler for the \cpp{} programming language targeting the ARM hardware architecture.
It generates machine code for ARM processors with floating-point extension executing T32 instructions from programs written in \cpp{} and stores it in corresponding object files.
For debugging purposes, it also creates a debugging information file as well as an assembly file containing a listing of the generated machine code.
The macro \texttt{\_\_armt32fpe\_\_} is predefined in order to enable programmers to identify this tool and its target architecture while compiling.
Programs generated with this compiler require additional runtime support that is stored in the \file{cpp\-armt32\-fpe\-run} library file.
\flowgraph{\resource{\cpp{}\\source code} \ar[r] & \toolbox{cpparmt32fpe} \ar[r] \ar[d] \ar[rd] & \resource{object file} \\ \variable{ECSINCLUDE} \ar[ru] & \resource{debugging\\information} & \resource{assembly\\listing}}
\seecpp\seeassembly\seearm\seeobject\seedebugging
}

\providecommand{\cppavr}{
\toolsection{cppavr} is a compiler for the \cpp{} programming language targeting the AVR hardware architecture.
It generates machine code for AVR processors from programs written in \cpp{} and stores it in corresponding object files.
For debugging purposes, it also creates a debugging information file as well as an assembly file containing a listing of the generated machine code.
The macro \texttt{\_\_avr\_\_} is predefined in order to enable programmers to identify this tool and its target architecture while compiling.
Programs generated with this compiler require additional runtime support that is stored in the \file{cpp\-avr\-run} library file.
\flowgraph{\resource{\cpp{}\\source code} \ar[r] & \toolbox{cppavr} \ar[r] \ar[d] \ar[rd] & \resource{object file} \\ \variable{ECSINCLUDE} \ar[ru] & \resource{debugging\\information} & \resource{assembly\\listing}}
\seecpp\seeassembly\seeavr\seeobject\seedebugging
}

\providecommand{\cppavrtt}{
\toolsection{cppavr32} is a compiler for the \cpp{} programming language targeting the AVR32 hardware architecture.
It generates machine code for AVR32 processors from programs written in \cpp{} and stores it in corresponding object files.
For debugging purposes, it also creates a debugging information file as well as an assembly file containing a listing of the generated machine code.
The macro \texttt{\_\_avr32\_\_} is predefined in order to enable programmers to identify this tool and its target architecture while compiling.
Programs generated with this compiler require additional runtime support that is stored in the \file{cpp\-avr32\-run} library file.
\flowgraph{\resource{\cpp{}\\source code} \ar[r] & \toolbox{cppavr32} \ar[r] \ar[d] \ar[rd] & \resource{object file} \\ \variable{ECSINCLUDE} \ar[ru] & \resource{debugging\\information} & \resource{assembly\\listing}}
\seecpp\seeassembly\seeavrtt\seeobject\seedebugging
}

\providecommand{\cppmabk}{
\toolsection{cppm68k} is a compiler for the \cpp{} programming language targeting the M68000 hardware architecture.
It generates machine code for M68000 processors from programs written in \cpp{} and stores it in corresponding object files.
For debugging purposes, it also creates a debugging information file as well as an assembly file containing a listing of the generated machine code.
The macro \texttt{\_\_m68k\_\_} is predefined in order to enable programmers to identify this tool and its target architecture while compiling.
Programs generated with this compiler require additional runtime support that is stored in the \file{cpp\-m68k\-run} library file.
\flowgraph{\resource{\cpp{}\\source code} \ar[r] & \toolbox{cppm68k} \ar[r] \ar[d] \ar[rd] & \resource{object file} \\ \variable{ECSINCLUDE} \ar[ru] & \resource{debugging\\information} & \resource{assembly\\listing}}
\seecpp\seeassembly\seemabk\seeobject\seedebugging
}

\providecommand{\cppmibl}{
\toolsection{cppmibl} is a compiler for the \cpp{} programming language targeting the MicroBlaze hardware architecture.
It generates machine code for MicroBlaze processors from programs written in \cpp{} and stores it in corresponding object files.
For debugging purposes, it also creates a debugging information file as well as an assembly file containing a listing of the generated machine code.
The macro \texttt{\_\_mibl\_\_} is predefined in order to enable programmers to identify this tool and its target architecture while compiling.
Programs generated with this compiler require additional runtime support that is stored in the \file{cpp\-mibl\-run} library file.
\flowgraph{\resource{\cpp{}\\source code} \ar[r] & \toolbox{cppmibl} \ar[r] \ar[d] \ar[rd] & \resource{object file} \\ \variable{ECSINCLUDE} \ar[ru] & \resource{debugging\\information} & \resource{assembly\\listing}}
\seecpp\seeassembly\seemibl\seeobject\seedebugging
}

\providecommand{\cppmipsa}{
\toolsection{cppmips32} is a compiler for the \cpp{} programming language targeting the MIPS32 hardware architecture.
It generates machine code for MIPS32 processors from programs written in \cpp{} and stores it in corresponding object files.
For debugging purposes, it also creates a debugging information file as well as an assembly file containing a listing of the generated machine code.
The macro \texttt{\_\_mips32\_\_} is predefined in order to enable programmers to identify this tool and its target architecture while compiling.
Programs generated with this compiler require additional runtime support that is stored in the \file{cpp\-mips32\-run} library file.
\flowgraph{\resource{\cpp{}\\source code} \ar[r] & \toolbox{cppmips32} \ar[r] \ar[d] \ar[rd] & \resource{object file} \\ \variable{ECSINCLUDE} \ar[ru] & \resource{debugging\\information} & \resource{assembly\\listing}}
\seecpp\seeassembly\seemips\seeobject\seedebugging
}

\providecommand{\cppmipsb}{
\toolsection{cppmips64} is a compiler for the \cpp{} programming language targeting the MIPS64 hardware architecture.
It generates machine code for MIPS64 processors from programs written in \cpp{} and stores it in corresponding object files.
For debugging purposes, it also creates a debugging information file as well as an assembly file containing a listing of the generated machine code.
The macro \texttt{\_\_mips64\_\_} is predefined in order to enable programmers to identify this tool and its target architecture while compiling.
Programs generated with this compiler require additional runtime support that is stored in the \file{cpp\-mips64\-run} library file.
\flowgraph{\resource{\cpp{}\\source code} \ar[r] & \toolbox{cppmips64} \ar[r] \ar[d] \ar[rd] & \resource{object file} \\ \variable{ECSINCLUDE} \ar[ru] & \resource{debugging\\information} & \resource{assembly\\listing}}
\seecpp\seeassembly\seemips\seeobject\seedebugging
}

\providecommand{\cppmmix}{
\toolsection{cppmmix} is a compiler for the \cpp{} programming language targeting the MMIX hardware architecture.
It generates machine code for MMIX processors from programs written in \cpp{} and stores it in corresponding object files.
For debugging purposes, it also creates a debugging information file as well as an assembly file containing a listing of the generated machine code.
The macro \texttt{\_\_mmix\_\_} is predefined in order to enable programmers to identify this tool and its target architecture while compiling.
Programs generated with this compiler require additional runtime support that is stored in the \file{cpp\-mmix\-run} library file.
\flowgraph{\resource{\cpp{}\\source code} \ar[r] & \toolbox{cppmmix} \ar[r] \ar[d] \ar[rd] & \resource{object file} \\ \variable{ECSINCLUDE} \ar[ru] & \resource{debugging\\information} & \resource{assembly\\listing}}
\seecpp\seeassembly\seemmix\seeobject\seedebugging
}

\providecommand{\cpporok}{
\toolsection{cppor1k} is a compiler for the \cpp{} programming language targeting the OpenRISC 1000 hardware architecture.
It generates machine code for OpenRISC 1000 processors from programs written in \cpp{} and stores it in corresponding object files.
For debugging purposes, it also creates a debugging information file as well as an assembly file containing a listing of the generated machine code.
The macro \texttt{\_\_or1k\_\_} is predefined in order to enable programmers to identify this tool and its target architecture while compiling.
Programs generated with this compiler require additional runtime support that is stored in the \file{cpp\-or1k\-run} library file.
\flowgraph{\resource{\cpp{}\\source code} \ar[r] & \toolbox{cppor1k} \ar[r] \ar[d] \ar[rd] & \resource{object file} \\ \variable{ECSINCLUDE} \ar[ru] & \resource{debugging\\information} & \resource{assembly\\listing}}
\seecpp\seeassembly\seeorok\seeobject\seedebugging
}

\providecommand{\cppppca}{
\toolsection{cppppc32} is a compiler for the \cpp{} programming language targeting the PowerPC hardware architecture.
It generates machine code for PowerPC processors from programs written in \cpp{} and stores it in corresponding object files.
The compiler generates machine code for the 32-bit operating mode defined by the PowerPC architecture.
For debugging purposes, it also creates a debugging information file as well as an assembly file containing a listing of the generated machine code.
The macro \texttt{\_\_ppc32\_\_} is predefined in order to enable programmers to identify this tool and its target architecture while compiling.
Programs generated with this compiler require additional runtime support that is stored in the \file{cpp\-ppc32\-run} library file.
\flowgraph{\resource{\cpp{}\\source code} \ar[r] & \toolbox{cppppc32} \ar[r] \ar[d] \ar[rd] & \resource{object file} \\ \variable{ECSINCLUDE} \ar[ru] & \resource{debugging\\information} & \resource{assembly\\listing}}
\seecpp\seeassembly\seeppc\seeobject\seedebugging
}

\providecommand{\cppppcb}{
\toolsection{cppppc64} is a compiler for the \cpp{} programming language targeting the PowerPC hardware architecture.
It generates machine code for PowerPC processors from programs written in \cpp{} and stores it in corresponding object files.
The compiler generates machine code for the 64-bit operating mode defined by the PowerPC architecture.
For debugging purposes, it also creates a debugging information file as well as an assembly file containing a listing of the generated machine code.
The macro \texttt{\_\_ppc64\_\_} is predefined in order to enable programmers to identify this tool and its target architecture while compiling.
Programs generated with this compiler require additional runtime support that is stored in the \file{cpp\-ppc64\-run} library file.
\flowgraph{\resource{\cpp{}\\source code} \ar[r] & \toolbox{cppppc64} \ar[r] \ar[d] \ar[rd] & \resource{object file} \\ \variable{ECSINCLUDE} \ar[ru] & \resource{debugging\\information} & \resource{assembly\\listing}}
\seecpp\seeassembly\seeppc\seeobject\seedebugging
}

\providecommand{\cpprisc}{
\toolsection{cpprisc} is a compiler for the \cpp{} programming language targeting the RISC hardware architecture.
It generates machine code for RISC processors from programs written in \cpp{} and stores it in corresponding object files.
For debugging purposes, it also creates a debugging information file as well as an assembly file containing a listing of the generated machine code.
The macro \texttt{\_\_risc\_\_} is predefined in order to enable programmers to identify this tool and its target architecture while compiling.
Programs generated with this compiler require additional runtime support that is stored in the \file{cpp\-risc\-run} library file.
\flowgraph{\resource{\cpp{}\\source code} \ar[r] & \toolbox{cpprisc} \ar[r] \ar[d] \ar[rd] & \resource{object file} \\ \variable{ECSINCLUDE} \ar[ru] & \resource{debugging\\information} & \resource{assembly\\listing}}
\seecpp\seeassembly\seerisc\seeobject\seedebugging
}

\providecommand{\cppwasm}{
\toolsection{cppwasm} is a compiler for the \cpp{} programming language targeting the WebAssembly architecture.
It generates machine code for WebAssembly targets from programs written in \cpp{} and stores it in corresponding object files.
For debugging purposes, it also creates a debugging information file as well as an assembly file containing a listing of the generated machine code.
The macro \texttt{\_\_wasm\_\_} is predefined in order to enable programmers to identify this tool and its target architecture while compiling.
Programs generated with this compiler require additional runtime support that is stored in the \file{cpp\-wasm\-run} library file.
\flowgraph{\resource{\cpp{}\\source code} \ar[r] & \toolbox{cppwasm} \ar[r] \ar[d] \ar[rd] & \resource{object file} \\ \variable{ECSINCLUDE} \ar[ru] & \resource{debugging\\information} & \resource{assembly\\listing}}
\seecpp\seeassembly\seewasm\seeobject\seedebugging
}

% FALSE tools

\providecommand{\falprint}{
\toolsection{falprint} is a pretty printer for the FALSE programming language.
It reformats the source code of FALSE programs and writes it to the standard output stream.
\flowgraph{\resource{FALSE\\source code} \ar[r] & \toolbox{falprint} \ar[r] & \resource{reformatted\\source code}}
\seefalse
}

\providecommand{\falcheck}{
\toolsection{falcheck} is a syntactic and semantic checker for the FALSE programming language.
It just performs syntactic and semantic checks on FALSE programs and writes its diagnostic messages to the standard error stream.
\flowgraph{\resource{FALSE\\source code} \ar[r] & \toolbox{falcheck} \ar[r] & \resource{diagnostic\\messages}}
\seefalse
}

\providecommand{\faldump}{
\toolsection{faldump} is a serializer for the FALSE programming language.
It dumps the complete internal representation of programs written in FALSE into an XML document.
\debuggingtool
\flowgraph{\resource{FALSE\\source code} \ar[r] & \toolbox{faldump} \ar[r] & \resource{internal\\representation}}
\seefalse
}

\providecommand{\falrun}{
\toolsection{falrun} is an interpreter for the FALSE programming language.
It processes and executes programs written in FALSE\@.
\flowgraph{\resource{FALSE\\source code} \ar[r] & \toolbox{falrun} \ar@/u/[r] & \resource{input/\\output} \ar@/d/[l]}
\seefalse
}

\providecommand{\falcpp}{
\toolsection{falcpp} is a transpiler for the FALSE programming language.
It translates programs written in FALSE into \cpp{} programs and stores them in corresponding source files.
\flowgraph{\resource{FALSE\\source code} \ar[r] & \toolbox{falcpp} \ar[r] & \resource{\cpp{}\\source file}}
\seefalse\seecpp
}

\providecommand{\falcode}{
\toolsection{falcode} is an intermediate code generator for the FALSE programming language.
It generates intermediate code from programs written in FALSE and stores it in corresponding assembly files.
\debuggingtool
\flowgraph{\resource{FALSE\\source code} \ar[r] & \toolbox{falcode} \ar[r] & \resource{intermediate\\code}}
\seefalse\seeassembly\seecode
}

\providecommand{\falamda}{
\toolsection{falamd16} is a compiler for the FALSE programming language targeting the AMD64 hardware architecture.
It generates machine code for AMD64 processors from programs written in FALSE and stores it in corresponding object files.
The compiler generates machine code for the 16-bit operating mode defined by the AMD64 architecture.
\flowgraph{\resource{FALSE\\source code} \ar[r] & \toolbox{falamd16} \ar[r] & \resource{object file}}
\seefalse\seeamd\seeobject
}

\providecommand{\falamdb}{
\toolsection{falamd32} is a compiler for the FALSE programming language targeting the AMD64 hardware architecture.
It generates machine code for AMD64 processors from programs written in FALSE and stores it in corresponding object files.
The compiler generates machine code for the 32-bit operating mode defined by the AMD64 architecture.
\flowgraph{\resource{FALSE\\source code} \ar[r] & \toolbox{falamd32} \ar[r] & \resource{object file}}
\seefalse\seeamd\seeobject
}

\providecommand{\falamdc}{
\toolsection{falamd64} is a compiler for the FALSE programming language targeting the AMD64 hardware architecture.
It generates machine code for AMD64 processors from programs written in FALSE and stores it in corresponding object files.
The compiler generates machine code for the 64-bit operating mode defined by the AMD64 architecture.
\flowgraph{\resource{FALSE\\source code} \ar[r] & \toolbox{falamd64} \ar[r] & \resource{object file}}
\seefalse\seeamd\seeobject
}

\providecommand{\falarma}{
\toolsection{falarma32} is a compiler for the FALSE programming language targeting the ARM hardware architecture.
It generates machine code for ARM processors executing A32 instructions from programs written in FALSE and stores it in corresponding object files.
\flowgraph{\resource{FALSE\\source code} \ar[r] & \toolbox{falarma32} \ar[r] & \resource{object file}}
\seefalse\seearm\seeobject
}

\providecommand{\falarmb}{
\toolsection{falarma64} is a compiler for the FALSE programming language targeting the ARM hardware architecture.
It generates machine code for ARM processors executing A64 instructions from programs written in FALSE and stores it in corresponding object files.
\flowgraph{\resource{FALSE\\source code} \ar[r] & \toolbox{falarma64} \ar[r] & \resource{object file}}
\seefalse\seearm\seeobject
}

\providecommand{\falarmc}{
\toolsection{falarmt32} is a compiler for the FALSE programming language targeting the ARM hardware architecture.
It generates machine code for ARM processors without floating-point extension executing T32 instructions from programs written in FALSE and stores it in corresponding object files.
\flowgraph{\resource{FALSE\\source code} \ar[r] & \toolbox{falarmt32} \ar[r] & \resource{object file}}
\seefalse\seearm\seeobject
}

\providecommand{\falarmcfpe}{
\toolsection{falarmt32fpe} is a compiler for the FALSE programming language targeting the ARM hardware architecture.
It generates machine code for ARM processors with floating-point extension executing T32 instructions from programs written in FALSE and stores it in corresponding object files.
\flowgraph{\resource{FALSE\\source code} \ar[r] & \toolbox{falarmt32fpe} \ar[r] & \resource{object file}}
\seefalse\seearm\seeobject
}

\providecommand{\falavr}{
\toolsection{falavr} is a compiler for the FALSE programming language targeting the AVR hardware architecture.
It generates machine code for AVR processors from programs written in FALSE and stores it in corresponding object files.
\flowgraph{\resource{FALSE\\source code} \ar[r] & \toolbox{falavr} \ar[r] & \resource{object file}}
\seefalse\seeavr\seeobject
}

\providecommand{\falavrtt}{
\toolsection{falavr32} is a compiler for the FALSE programming language targeting the AVR32 hardware architecture.
It generates machine code for AVR32 processors from programs written in FALSE and stores it in corresponding object files.
\flowgraph{\resource{FALSE\\source code} \ar[r] & \toolbox{falavr32} \ar[r] & \resource{object file}}
\seefalse\seeavrtt\seeobject
}

\providecommand{\falmabk}{
\toolsection{falm68k} is a compiler for the FALSE programming language targeting the M68000 hardware architecture.
It generates machine code for M68000 processors from programs written in FALSE and stores it in corresponding object files.
\flowgraph{\resource{FALSE\\source code} \ar[r] & \toolbox{falm68k} \ar[r] & \resource{object file}}
\seefalse\seemabk\seeobject
}

\providecommand{\falmibl}{
\toolsection{falmibl} is a compiler for the FALSE programming language targeting the MicroBlaze hardware architecture.
It generates machine code for MicroBlaze processors from programs written in FALSE and stores it in corresponding object files.
\flowgraph{\resource{FALSE\\source code} \ar[r] & \toolbox{falmibl} \ar[r] & \resource{object file}}
\seefalse\seemibl\seeobject
}

\providecommand{\falmipsa}{
\toolsection{falmips32} is a compiler for the FALSE programming language targeting the MIPS32 hardware architecture.
It generates machine code for MIPS32 processors from programs written in FALSE and stores it in corresponding object files.
\flowgraph{\resource{FALSE\\source code} \ar[r] & \toolbox{falmips32} \ar[r] & \resource{object file}}
\seefalse\seemips\seeobject
}

\providecommand{\falmipsb}{
\toolsection{falmips64} is a compiler for the FALSE programming language targeting the MIPS64 hardware architecture.
It generates machine code for MIPS64 processors from programs written in FALSE and stores it in corresponding object files.
\flowgraph{\resource{FALSE\\source code} \ar[r] & \toolbox{falmips64} \ar[r] & \resource{object file}}
\seefalse\seemips\seeobject
}

\providecommand{\falmmix}{
\toolsection{falmmix} is a compiler for the FALSE programming language targeting the MMIX hardware architecture.
It generates machine code for MMIX processors from programs written in FALSE and stores it in corresponding object files.
\flowgraph{\resource{FALSE\\source code} \ar[r] & \toolbox{falmmix} \ar[r] & \resource{object file}}
\seefalse\seemmix\seeobject
}

\providecommand{\falorok}{
\toolsection{falor1k} is a compiler for the FALSE programming language targeting the OpenRISC 1000 hardware architecture.
It generates machine code for OpenRISC 1000 processors from programs written in FALSE and stores it in corresponding object files.
\flowgraph{\resource{FALSE\\source code} \ar[r] & \toolbox{falor1k} \ar[r] & \resource{object file}}
\seefalse\seeorok\seeobject
}

\providecommand{\falppca}{
\toolsection{falppc32} is a compiler for the FALSE programming language targeting the PowerPC hardware architecture.
It generates machine code for PowerPC processors from programs written in FALSE and stores it in corresponding object files.
The compiler generates machine code for the 32-bit operating mode defined by the PowerPC architecture.
\flowgraph{\resource{FALSE\\source code} \ar[r] & \toolbox{falppc32} \ar[r] & \resource{object file}}
\seefalse\seeppc\seeobject
}

\providecommand{\falppcb}{
\toolsection{falppc64} is a compiler for the FALSE programming language targeting the PowerPC hardware architecture.
It generates machine code for PowerPC processors from programs written in FALSE and stores it in corresponding object files.
The compiler generates machine code for the 64-bit operating mode defined by the PowerPC architecture.
\flowgraph{\resource{FALSE\\source code} \ar[r] & \toolbox{falppc64} \ar[r] & \resource{object file}}
\seefalse\seeppc\seeobject
}

\providecommand{\falrisc}{
\toolsection{falrisc} is a compiler for the FALSE programming language targeting the RISC hardware architecture.
It generates machine code for RISC processors from programs written in FALSE and stores it in corresponding object files.
\flowgraph{\resource{FALSE\\source code} \ar[r] & \toolbox{falrisc} \ar[r] & \resource{object file}}
\seefalse\seerisc\seeobject
}

\providecommand{\falwasm}{
\toolsection{falwasm} is a compiler for the FALSE programming language targeting the WebAssembly architecture.
It generates machine code for WebAssembly targets from programs written in FALSE and stores it in corresponding object files.
\flowgraph{\resource{FALSE\\source code} \ar[r] & \toolbox{falwasm} \ar[r] & \resource{object file}}
\seefalse\seewasm\seeobject
}

% Oberon tools

\providecommand{\obprint}{
\toolsection{obprint} is a pretty printer for the Oberon programming language.
It reformats the source code of Oberon modules and writes it to the standard output stream.
\flowgraph{\resource{Oberon\\source code} \ar[r] & \toolbox{obprint} \ar[r] & \resource{reformatted\\source code}}
\seeoberon
}

\providecommand{\obcheck}{
\toolsection{obcheck} is a syntactic and semantic checker for the Oberon programming language.
It just performs syntactic and semantic checks on Oberon modules and writes its diagnostic messages to the standard error stream.
In addition, it stores the interface of each module in a symbol file which is required when other modules import the module.
\flowgraph{\resource{Oberon\\source code} \ar[r] & \toolbox{obcheck} \ar[r] \ar@/l/[d] & \resource{diagnostic\\messages} \\ \variable{ECSIMPORT} \ar[ru] & \resource{symbol\\files} \ar@/r/[u]}
\seeoberon
}

\providecommand{\obdump}{
\toolsection{obdump} is a serializer for the Oberon programming language.
It dumps the complete internal representation of modules written in Oberon into an XML document.
\debuggingtool
\flowgraph{\resource{Oberon\\source code} \ar[r] & \toolbox{obdump} \ar[r] \ar@/l/[d] & \resource{internal\\representation} \\ \variable{ECSIMPORT} \ar[ru] & \resource{symbol\\files} \ar@/r/[u]}
\seeoberon
}

\providecommand{\obrun}{
\toolsection{obrun} is an interpreter for the Oberon programming language.
It processes and executes modules written in Oberon.
This tool does neither generate nor process symbol files while interpreting modules.
If a module is imported by another one, its filename has to be named before the other one in the list of command-line arguments.
\flowgraph{\resource{Oberon\\source code} \ar[r] & \toolbox{obrun} \ar@/u/[r] & \resource{input/\\output} \ar@/d/[l]}
\seeoberon
}

\providecommand{\obcpp}{
\toolsection{obcpp} is a transpiler for the Oberon programming language.
It translates programs written in Oberon into \cpp{} programs and stores them in corresponding source and header files.
In addition, it stores the interface of each module in a symbol file which is required when other modules import the module.
The same interface is provided by the generated header file which can be used in other parts of the \cpp{} program.
\flowgraph{\resource{Oberon\\source code} \ar[r] & \toolbox{obcpp} \ar[r] \ar@/l/[d] \ar[rd] & \resource{\cpp{}\\source file} \\ \variable{ECSIMPORT} \ar[ru] & \resource{symbol\\files} \ar@/r/[u] & \resource{\cpp{}\\header file}}
\seeoberon\seecpp
}

\providecommand{\obdoc}{
\toolsection{obdoc} is a generic documentation generator for the Oberon programming language.
It processes several Oberon modules and assembles all information therein into a generic documentation.
In addition, it stores the interface of each module in a symbol file which is required when other modules import the module.
\debuggingtool
\flowgraph{\resource{Oberon\\source code} \ar[r] & \toolbox{obdoc} \ar[r] \ar@/l/[d] & \resource{generic\\documentation} \\ \variable{ECSIMPORT} \ar[ru] & \resource{symbol\\files} \ar@/r/[u]}
\seeoberon\seedocumentation
}

\providecommand{\obhtml}{
\toolsection{obhtml} is an HTML documentation generator for the Oberon programming language.
It processes several Oberon modules and assembles all information therein into an HTML document.
In addition, it stores the interface of each module in a symbol file which is required when other modules import the module.
\flowgraph{\resource{Oberon\\source code} \ar[r] & \toolbox{obhtml} \ar[r] \ar@/l/[d] & \resource{HTML\\document} \\ \variable{ECSIMPORT} \ar[ru] & \resource{symbol\\files} \ar@/r/[u]}
\seeoberon\seedocumentation
}

\providecommand{\oblatex}{
\toolsection{oblatex} is a Latex documentation generator for the Oberon programming language.
It processes several Oberon modules and assembles all information therein into a Latex document.
In addition, it stores the interface of each module in a symbol file which is required when other modules import the module.
\flowgraph{\resource{Oberon\\source code} \ar[r] & \toolbox{oblatex} \ar[r] \ar@/l/[d] & \resource{Latex\\document} \\ \variable{ECSIMPORT} \ar[ru] & \resource{symbol\\files} \ar@/r/[u]}
\seeoberon\seedocumentation
}

\providecommand{\obcode}{
\toolsection{obcode} is an intermediate code generator for the Oberon programming language.
It generates intermediate code from modules written in Oberon and stores it in corresponding assembly files.
In addition, it stores the interface of each module in a symbol file which is required when other modules import the module.
Programs generated with this tool require additional runtime support that is stored in the \file{ob\-code\-run} library file.
\debuggingtool
\flowgraph{\resource{Oberon\\source code} \ar[r] & \toolbox{obcode} \ar[r] \ar@/l/[d] & \resource{intermediate\\code} \\ \variable{ECSIMPORT} \ar[ru] & \resource{symbol\\files} \ar@/r/[u]}
\seeoberon\seeassembly\seecode
}

\providecommand{\obamda}{
\toolsection{obamd16} is a compiler for the Oberon programming language targeting the AMD64 hardware architecture.
It generates machine code for AMD64 processors from modules written in Oberon and stores it in corresponding object files.
The compiler generates machine code for the 16-bit operating mode defined by the AMD64 architecture.
For debugging purposes, it also creates a debugging information file as well as an assembly file containing a listing of the generated machine code.
In addition, it stores the interface of each module in a symbol file which is required when other modules import the module.
Programs generated with this compiler require additional runtime support that is stored in the \file{ob\-amd16\-run} library file.
\flowgraph{\resource{Oberon\\source code} \ar[r] & \toolbox{obamd16} \ar[r] \ar@/l/[d] \ar[rd] & \resource{object file} \\ \variable{ECSIMPORT} \ar[ru] & \resource{symbol\\files} \ar@/r/[u] & \resource{debugging\\information}}
\seeoberon\seeassembly\seeamd\seeobject\seedebugging
}

\providecommand{\obamdb}{
\toolsection{obamd32} is a compiler for the Oberon programming language targeting the AMD64 hardware architecture.
It generates machine code for AMD64 processors from modules written in Oberon and stores it in corresponding object files.
The compiler generates machine code for the 32-bit operating mode defined by the AMD64 architecture.
For debugging purposes, it also creates a debugging information file as well as an assembly file containing a listing of the generated machine code.
In addition, it stores the interface of each module in a symbol file which is required when other modules import the module.
Programs generated with this compiler require additional runtime support that is stored in the \file{ob\-amd32\-run} library file.
\flowgraph{\resource{Oberon\\source code} \ar[r] & \toolbox{obamd32} \ar[r] \ar@/l/[d] \ar[rd] & \resource{object file} \\ \variable{ECSIMPORT} \ar[ru] & \resource{symbol\\files} \ar@/r/[u] & \resource{debugging\\information}}
\seeoberon\seeassembly\seeamd\seeobject\seedebugging
}

\providecommand{\obamdc}{
\toolsection{obamd64} is a compiler for the Oberon programming language targeting the AMD64 hardware architecture.
It generates machine code for AMD64 processors from modules written in Oberon and stores it in corresponding object files.
The compiler generates machine code for the 64-bit operating mode defined by the AMD64 architecture.
For debugging purposes, it also creates a debugging information file as well as an assembly file containing a listing of the generated machine code.
In addition, it stores the interface of each module in a symbol file which is required when other modules import the module.
Programs generated with this compiler require additional runtime support that is stored in the \file{ob\-amd64\-run} library file.
\flowgraph{\resource{Oberon\\source code} \ar[r] & \toolbox{obamd64} \ar[r] \ar@/l/[d] \ar[rd] & \resource{object file} \\ \variable{ECSIMPORT} \ar[ru] & \resource{symbol\\files} \ar@/r/[u] & \resource{debugging\\information}}
\seeoberon\seeassembly\seeamd\seeobject\seedebugging
}

\providecommand{\obarma}{
\toolsection{obarma32} is a compiler for the Oberon programming language targeting the ARM hardware architecture.
It generates machine code for ARM processors executing A32 instructions from modules written in Oberon and stores it in corresponding object files.
For debugging purposes, it also creates a debugging information file as well as an assembly file containing a listing of the generated machine code.
In addition, it stores the interface of each module in a symbol file which is required when other modules import the module.
Programs generated with this compiler require additional runtime support that is stored in the \file{ob\-arma32\-run} library file.
\flowgraph{\resource{Oberon\\source code} \ar[r] & \toolbox{obarma32} \ar[r] \ar@/l/[d] \ar[rd] & \resource{object file} \\ \variable{ECSIMPORT} \ar[ru] & \resource{symbol\\files} \ar@/r/[u] & \resource{debugging\\information}}
\seeoberon\seeassembly\seearm\seeobject\seedebugging
}

\providecommand{\obarmb}{
\toolsection{obarma64} is a compiler for the Oberon programming language targeting the ARM hardware architecture.
It generates machine code for ARM processors executing A64 instructions from modules written in Oberon and stores it in corresponding object files.
For debugging purposes, it also creates a debugging information file as well as an assembly file containing a listing of the generated machine code.
In addition, it stores the interface of each module in a symbol file which is required when other modules import the module.
Programs generated with this compiler require additional runtime support that is stored in the \file{ob\-arma64\-run} library file.
\flowgraph{\resource{Oberon\\source code} \ar[r] & \toolbox{obarma64} \ar[r] \ar@/l/[d] \ar[rd] & \resource{object file} \\ \variable{ECSIMPORT} \ar[ru] & \resource{symbol\\files} \ar@/r/[u] & \resource{debugging\\information}}
\seeoberon\seeassembly\seearm\seeobject\seedebugging
}

\providecommand{\obarmc}{
\toolsection{obarmt32} is a compiler for the Oberon programming language targeting the ARM hardware architecture.
It generates machine code for ARM processors without floating-point extension executing T32 instructions from modules written in Oberon and stores it in corresponding object files.
For debugging purposes, it also creates a debugging information file as well as an assembly file containing a listing of the generated machine code.
In addition, it stores the interface of each module in a symbol file which is required when other modules import the module.
Programs generated with this compiler require additional runtime support that is stored in the \file{ob\-armt32\-run} library file.
\flowgraph{\resource{Oberon\\source code} \ar[r] & \toolbox{obarmt32} \ar[r] \ar@/l/[d] \ar[rd] & \resource{object file} \\ \variable{ECSIMPORT} \ar[ru] & \resource{symbol\\files} \ar@/r/[u] & \resource{debugging\\information}}
\seeoberon\seeassembly\seearm\seeobject\seedebugging
}

\providecommand{\obarmcfpe}{
\toolsection{obarmt32fpe} is a compiler for the Oberon programming language targeting the ARM hardware architecture.
It generates machine code for ARM processors with floating-point extension executing T32 instructions from modules written in Oberon and stores it in corresponding object files.
For debugging purposes, it also creates a debugging information file as well as an assembly file containing a listing of the generated machine code.
In addition, it stores the interface of each module in a symbol file which is required when other modules import the module.
Programs generated with this compiler require additional runtime support that is stored in the \file{ob\-armt32\-fpe\-run} library file.
\flowgraph{\resource{Oberon\\source code} \ar[r] & \toolbox{obarmt32fpe} \ar[r] \ar@/l/[d] \ar[rd] & \resource{object file} \\ \variable{ECSIMPORT} \ar[ru] & \resource{symbol\\files} \ar@/r/[u] & \resource{debugging\\information}}
\seeoberon\seeassembly\seearm\seeobject\seedebugging
}

\providecommand{\obavr}{
\toolsection{obavr} is a compiler for the Oberon programming language targeting the AVR hardware architecture.
It generates machine code for AVR processors from modules written in Oberon and stores it in corresponding object files.
For debugging purposes, it also creates a debugging information file as well as an assembly file containing a listing of the generated machine code.
In addition, it stores the interface of each module in a symbol file which is required when other modules import the module.
Programs generated with this compiler require additional runtime support that is stored in the \file{ob\-avr\-run} library file.
\flowgraph{\resource{Oberon\\source code} \ar[r] & \toolbox{obavr} \ar[r] \ar@/l/[d] \ar[rd] & \resource{object file} \\ \variable{ECSIMPORT} \ar[ru] & \resource{symbol\\files} \ar@/r/[u] & \resource{debugging\\information}}
\seeoberon\seeassembly\seeavr\seeobject\seedebugging
}

\providecommand{\obavrtt}{
\toolsection{obavr32} is a compiler for the Oberon programming language targeting the AVR32 hardware architecture.
It generates machine code for AVR32 processors from modules written in Oberon and stores it in corresponding object files.
For debugging purposes, it also creates a debugging information file as well as an assembly file containing a listing of the generated machine code.
In addition, it stores the interface of each module in a symbol file which is required when other modules import the module.
Programs generated with this compiler require additional runtime support that is stored in the \file{ob\-avr32\-run} library file.
\flowgraph{\resource{Oberon\\source code} \ar[r] & \toolbox{obavr32} \ar[r] \ar@/l/[d] \ar[rd] & \resource{object file} \\ \variable{ECSIMPORT} \ar[ru] & \resource{symbol\\files} \ar@/r/[u] & \resource{debugging\\information}}
\seeoberon\seeassembly\seeavrtt\seeobject\seedebugging
}

\providecommand{\obmabk}{
\toolsection{obm68k} is a compiler for the Oberon programming language targeting the M68000 hardware architecture.
It generates machine code for M68000 processors from modules written in Oberon and stores it in corresponding object files.
For debugging purposes, it also creates a debugging information file as well as an assembly file containing a listing of the generated machine code.
In addition, it stores the interface of each module in a symbol file which is required when other modules import the module.
Programs generated with this compiler require additional runtime support that is stored in the \file{ob\-m68k\-run} library file.
\flowgraph{\resource{Oberon\\source code} \ar[r] & \toolbox{obm68k} \ar[r] \ar@/l/[d] \ar[rd] & \resource{object file} \\ \variable{ECSIMPORT} \ar[ru] & \resource{symbol\\files} \ar@/r/[u] & \resource{debugging\\information}}
\seeoberon\seeassembly\seemabk\seeobject\seedebugging
}

\providecommand{\obmibl}{
\toolsection{obmibl} is a compiler for the Oberon programming language targeting the MicroBlaze hardware architecture.
It generates machine code for MicroBlaze processors from modules written in Oberon and stores it in corresponding object files.
For debugging purposes, it also creates a debugging information file as well as an assembly file containing a listing of the generated machine code.
In addition, it stores the interface of each module in a symbol file which is required when other modules import the module.
Programs generated with this compiler require additional runtime support that is stored in the \file{ob\-mibl\-run} library file.
\flowgraph{\resource{Oberon\\source code} \ar[r] & \toolbox{obmibl} \ar[r] \ar@/l/[d] \ar[rd] & \resource{object file} \\ \variable{ECSIMPORT} \ar[ru] & \resource{symbol\\files} \ar@/r/[u] & \resource{debugging\\information}}
\seeoberon\seeassembly\seemibl\seeobject\seedebugging
}

\providecommand{\obmipsa}{
\toolsection{obmips32} is a compiler for the Oberon programming language targeting the MIPS32 hardware architecture.
It generates machine code for MIPS32 processors from modules written in Oberon and stores it in corresponding object files.
For debugging purposes, it also creates a debugging information file as well as an assembly file containing a listing of the generated machine code.
In addition, it stores the interface of each module in a symbol file which is required when other modules import the module.
Programs generated with this compiler require additional runtime support that is stored in the \file{ob\-mips32\-run} library file.
\flowgraph{\resource{Oberon\\source code} \ar[r] & \toolbox{obmips32} \ar[r] \ar@/l/[d] \ar[rd] & \resource{object file} \\ \variable{ECSIMPORT} \ar[ru] & \resource{symbol\\files} \ar@/r/[u] & \resource{debugging\\information}}
\seeoberon\seeassembly\seemips\seeobject\seedebugging
}

\providecommand{\obmipsb}{
\toolsection{obmips64} is a compiler for the Oberon programming language targeting the MIPS64 hardware architecture.
It generates machine code for MIPS64 processors from modules written in Oberon and stores it in corresponding object files.
For debugging purposes, it also creates a debugging information file as well as an assembly file containing a listing of the generated machine code.
In addition, it stores the interface of each module in a symbol file which is required when other modules import the module.
Programs generated with this compiler require additional runtime support that is stored in the \file{ob\-mips64\-run} library file.
\flowgraph{\resource{Oberon\\source code} \ar[r] & \toolbox{obmips64} \ar[r] \ar@/l/[d] \ar[rd] & \resource{object file} \\ \variable{ECSIMPORT} \ar[ru] & \resource{symbol\\files} \ar@/r/[u] & \resource{debugging\\information}}
\seeoberon\seeassembly\seemips\seeobject\seedebugging
}

\providecommand{\obmmix}{
\toolsection{obmmix} is a compiler for the Oberon programming language targeting the MMIX hardware architecture.
It generates machine code for MMIX processors from modules written in Oberon and stores it in corresponding object files.
For debugging purposes, it also creates a debugging information file as well as an assembly file containing a listing of the generated machine code.
In addition, it stores the interface of each module in a symbol file which is required when other modules import the module.
Programs generated with this compiler require additional runtime support that is stored in the \file{ob\-mmix\-run} library file.
\flowgraph{\resource{Oberon\\source code} \ar[r] & \toolbox{obmmix} \ar[r] \ar@/l/[d] \ar[rd] & \resource{object file} \\ \variable{ECSIMPORT} \ar[ru] & \resource{symbol\\files} \ar@/r/[u] & \resource{debugging\\information}}
\seeoberon\seeassembly\seemmix\seeobject\seedebugging
}

\providecommand{\oborok}{
\toolsection{obor1k} is a compiler for the Oberon programming language targeting the OpenRISC 1000 hardware architecture.
It generates machine code for OpenRISC 1000 processors from modules written in Oberon and stores it in corresponding object files.
For debugging purposes, it also creates a debugging information file as well as an assembly file containing a listing of the generated machine code.
In addition, it stores the interface of each module in a symbol file which is required when other modules import the module.
Programs generated with this compiler require additional runtime support that is stored in the \file{ob\-or1k\-run} library file.
\flowgraph{\resource{Oberon\\source code} \ar[r] & \toolbox{obor1k} \ar[r] \ar@/l/[d] \ar[rd] & \resource{object file} \\ \variable{ECSIMPORT} \ar[ru] & \resource{symbol\\files} \ar@/r/[u] & \resource{debugging\\information}}
\seeoberon\seeassembly\seeorok\seeobject\seedebugging
}

\providecommand{\obppca}{
\toolsection{obppc32} is a compiler for the Oberon programming language targeting the PowerPC hardware architecture.
It generates machine code for PowerPC processors from modules written in Oberon and stores it in corresponding object files.
The compiler generates machine code for the 32-bit operating mode defined by the PowerPC architecture.
For debugging purposes, it also creates a debugging information file as well as an assembly file containing a listing of the generated machine code.
In addition, it stores the interface of each module in a symbol file which is required when other modules import the module.
Programs generated with this compiler require additional runtime support that is stored in the \file{ob\-ppc32\-run} library file.
\flowgraph{\resource{Oberon\\source code} \ar[r] & \toolbox{obppc32} \ar[r] \ar@/l/[d] \ar[rd] & \resource{object file} \\ \variable{ECSIMPORT} \ar[ru] & \resource{symbol\\files} \ar@/r/[u] & \resource{debugging\\information}}
\seeoberon\seeassembly\seeppc\seeobject\seedebugging
}

\providecommand{\obppcb}{
\toolsection{obppc64} is a compiler for the Oberon programming language targeting the PowerPC hardware architecture.
It generates machine code for PowerPC processors from modules written in Oberon and stores it in corresponding object files.
The compiler generates machine code for the 64-bit operating mode defined by the PowerPC architecture.
For debugging purposes, it also creates a debugging information file as well as an assembly file containing a listing of the generated machine code.
In addition, it stores the interface of each module in a symbol file which is required when other modules import the module.
Programs generated with this compiler require additional runtime support that is stored in the \file{ob\-ppc64\-run} library file.
\flowgraph{\resource{Oberon\\source code} \ar[r] & \toolbox{obppc64} \ar[r] \ar@/l/[d] \ar[rd] & \resource{object file} \\ \variable{ECSIMPORT} \ar[ru] & \resource{symbol\\files} \ar@/r/[u] & \resource{debugging\\information}}
\seeoberon\seeassembly\seeppc\seeobject\seedebugging
}

\providecommand{\obrisc}{
\toolsection{obrisc} is a compiler for the Oberon programming language targeting the RISC hardware architecture.
It generates machine code for RISC processors from modules written in Oberon and stores it in corresponding object files.
For debugging purposes, it also creates a debugging information file as well as an assembly file containing a listing of the generated machine code.
In addition, it stores the interface of each module in a symbol file which is required when other modules import the module.
Programs generated with this compiler require additional runtime support that is stored in the \file{ob\-risc\-run} library file.
\flowgraph{\resource{Oberon\\source code} \ar[r] & \toolbox{obrisc} \ar[r] \ar@/l/[d] \ar[rd] & \resource{object file} \\ \variable{ECSIMPORT} \ar[ru] & \resource{symbol\\files} \ar@/r/[u] & \resource{debugging\\information}}
\seeoberon\seeassembly\seerisc\seeobject\seedebugging
}

\providecommand{\obwasm}{
\toolsection{obwasm} is a compiler for the Oberon programming language targeting the WebAssembly architecture.
It generates machine code for WebAssembly targets from modules written in Oberon and stores it in corresponding object files.
For debugging purposes, it also creates a debugging information file as well as an assembly file containing a listing of the generated machine code.
In addition, it stores the interface of each module in a symbol file which is required when other modules import the module.
Programs generated with this compiler require additional runtime support that is stored in the \file{ob\-wasm\-run} library file.
\flowgraph{\resource{Oberon\\source code} \ar[r] & \toolbox{obwasm} \ar[r] \ar@/l/[d] \ar[rd] & \resource{object file} \\ \variable{ECSIMPORT} \ar[ru] & \resource{symbol\\files} \ar@/r/[u] & \resource{debugging\\information}}
\seeoberon\seeassembly\seewasm\seeobject\seedebugging
}

% converter tools

\providecommand{\dbgdwarf}{
\toolsection{dbgdwarf} is a DWARF debugging information converter tool.
It converts debugging information into the DWARF debugging data format and stores it in corresponding object files~\cite{dwarffile}.
The resulting debugging object files can be combined with runtime support that creates Executable and Linking Format (ELF) files~\cite{elffile}.
\flowgraph{\resource{debugging\\information} \ar[r] & \toolbox{dbgdwarf} \ar[r] & \resource{debugging\\object file}}
\seeobject\seedebugging
}

% assembler tools

\providecommand{\asmprint}{
\toolsection{asmprint} is a pretty printer for generic assembly code.
It reformats generic assembly code and writes it to the standard output stream.
\flowgraph{\resource{generic assembly\\source code} \ar[r] & \toolbox{asmprint} \ar[r] & \resource{reformatted\\source code}}
\seeassembly
}

\providecommand{\amdaasm}{
\toolsection{amd16asm} is an assembler for the AMD64 hardware architecture.
It translates assembly code into machine code for AMD64 processors and stores it in corresponding object files.
By default, the assembler generates machine code for the 16-bit operating mode defined by the AMD64 architecture.
\flowgraph{\resource{AMD16 assembly\\source code} \ar[r] & \toolbox{amd16asm} \ar[r] & \resource{object file}}
\seeassembly\seeamd\seeobject
}

\providecommand{\amdadism}{
\toolsection{amd16dism} is a disassembler for the AMD64 hardware architecture.
It translates machine code from object files targeting AMD64 processors into assembly code and writes it to the standard output stream.
It assumes that the machine code was generated for the 16-bit operating mode defined by the AMD64 architecture.
\flowgraph{\resource{object file} \ar[r] & \toolbox{amd16dism} \ar[r] & \resource{disassembly\\listing}}
\seeassembly\seeamd\seeobject
}

\providecommand{\amdbasm}{
\toolsection{amd32asm} is an assembler for the AMD64 hardware architecture.
It translates assembly code into machine code for AMD64 processors and stores it in corresponding object files.
By default, the assembler generates machine code for the 32-bit operating mode defined by the AMD64 architecture.
\flowgraph{\resource{AMD32 assembly\\source code} \ar[r] & \toolbox{amd32asm} \ar[r] & \resource{object file}}
\seeassembly\seeamd\seeobject
}

\providecommand{\amdbdism}{
\toolsection{amd32dism} is a disassembler for the AMD64 hardware architecture.
It translates machine code from object files targeting AMD64 processors into assembly code and writes it to the standard output stream.
It assumes that the machine code was generated for the 32-bit operating mode defined by the AMD64 architecture.
\flowgraph{\resource{object file} \ar[r] & \toolbox{amd32dism} \ar[r] & \resource{disassembly\\listing}}
\seeassembly\seeamd\seeobject
}

\providecommand{\amdcasm}{
\toolsection{amd64asm} is an assembler for the AMD64 hardware architecture.
It translates assembly code into machine code for AMD64 processors and stores it in corresponding object files.
By default, the assembler generates machine code for the 64-bit operating mode defined by the AMD64 architecture.
\flowgraph{\resource{AMD64 assembly\\source code} \ar[r] & \toolbox{amd64asm} \ar[r] & \resource{object file}}
\seeassembly\seeamd\seeobject
}

\providecommand{\amdcdism}{
\toolsection{amd64dism} is a disassembler for the AMD64 hardware architecture.
It translates machine code from object files targeting AMD64 processors into assembly code and writes it to the standard output stream.
It assumes that the machine code was generated for the 64-bit operating mode defined by the AMD64 architecture.
\flowgraph{\resource{object file} \ar[r] & \toolbox{amd64dism} \ar[r] & \resource{disassembly\\listing}}
\seeassembly\seeamd\seeobject
}

\providecommand{\armaasm}{
\toolsection{arma32asm} is an assembler for the ARM hardware architecture.
It translates assembly code into machine code for ARM processors executing A32 instructions and stores it in corresponding object files.
\flowgraph{\resource{ARM A32 assembly\\source code} \ar[r] & \toolbox{arma32asm} \ar[r] & \resource{object file}}
\seeassembly\seearm\seeobject
}

\providecommand{\armadism}{
\toolsection{arma32dism} is a disassembler for the ARM hardware architecture.
It translates machine code from object files targeting ARM processors executing A32 instructions into assembly code and writes it to the standard output stream.
\flowgraph{\resource{object file} \ar[r] & \toolbox{arma32dism} \ar[r] & \resource{disassembly\\listing}}
\seeassembly\seearm\seeobject
}

\providecommand{\armbasm}{
\toolsection{arma64asm} is an assembler for the ARM hardware architecture.
It translates assembly code into machine code for ARM processors executing A64 instructions and stores it in corresponding object files.
\flowgraph{\resource{ARM A64 assembly\\source code} \ar[r] & \toolbox{arma64asm} \ar[r] & \resource{object file}}
\seeassembly\seearm\seeobject
}

\providecommand{\armbdism}{
\toolsection{arma64dism} is a disassembler for the ARM hardware architecture.
It translates machine code from object files targeting ARM processors executing A64 instructions into assembly code and writes it to the standard output stream.
\flowgraph{\resource{object file} \ar[r] & \toolbox{arma64dism} \ar[r] & \resource{disassembly\\listing}}
\seeassembly\seearm\seeobject
}

\providecommand{\armcasm}{
\toolsection{armt32asm} is an assembler for the ARM hardware architecture.
It translates assembly code into machine code for ARM processors executing T32 instructions and stores it in corresponding object files.
\flowgraph{\resource{ARM T32 assembly\\source code} \ar[r] & \toolbox{armt32asm} \ar[r] & \resource{object file}}
\seeassembly\seearm\seeobject
}

\providecommand{\armcdism}{
\toolsection{armt32dism} is a disassembler for the ARM hardware architecture.
It translates machine code from object files targeting ARM processors executing T32 instructions into assembly code and writes it to the standard output stream.
\flowgraph{\resource{object file} \ar[r] & \toolbox{armt32dism} \ar[r] & \resource{disassembly\\listing}}
\seeassembly\seearm\seeobject
}

\providecommand{\avrasm}{
\toolsection{avrasm} is an assembler for the AVR hardware architecture.
It translates assembly code into machine code for AVR processors and stores it in corresponding object files.
The identifiers \texttt{RXL}, \texttt{RXH}, \texttt{RYL}, \texttt{RYH}, \texttt{RZL}, and \texttt{RZH} are predefined and name the corresponding registers.
The identifiers \texttt{SPL} and \texttt{SPH} are also predefined and evaluate to the address of the corresponding registers.
\flowgraph{\resource{AVR assembly\\source code} \ar[r] & \toolbox{avrasm} \ar[r] & \resource{object file}}
\seeassembly\seeavr\seeobject
}

\providecommand{\avrdism}{
\toolsection{avrdism} is a disassembler for the AVR hardware architecture.
It translates machine code from object files targeting AVR processors into assembly code and writes it to the standard output stream.
\flowgraph{\resource{object file} \ar[r] & \toolbox{avrdism} \ar[r] & \resource{disassembly\\listing}}
\seeassembly\seeavr\seeobject
}

\providecommand{\avrttasm}{
\toolsection{avr32asm} is an assembler for the AVR32 hardware architecture.
It translates assembly code into machine code for AVR32 processors and stores it in corresponding object files.
\flowgraph{\resource{AVR32 assembly\\source code} \ar[r] & \toolbox{avr32asm} \ar[r] & \resource{object file}}
\seeassembly\seeavrtt\seeobject
}

\providecommand{\avrttdism}{
\toolsection{avr32dism} is a disassembler for the AVR32 hardware architecture.
It translates machine code from object files targeting AVR32 processors into assembly code and writes it to the standard output stream.
\flowgraph{\resource{object file} \ar[r] & \toolbox{avr32dism} \ar[r] & \resource{disassembly\\listing}}
\seeassembly\seeavrtt\seeobject
}

\providecommand{\mabkasm}{
\toolsection{m68kasm} is an assembler for the M68000 hardware architecture.
It translates assembly code into machine code for M68000 processors and stores it in corresponding object files.
\flowgraph{\resource{68000 assembly\\source code} \ar[r] & \toolbox{m68kasm} \ar[r] & \resource{object file}}
\seeassembly\seemabk\seeobject
}

\providecommand{\mabkdism}{
\toolsection{m68kdism} is a disassembler for the M68000 hardware architecture.
It translates machine code from object files targeting M68000 processors into assembly code and writes it to the standard output stream.
\flowgraph{\resource{object file} \ar[r] & \toolbox{m68kdism} \ar[r] & \resource{disassembly\\listing}}
\seeassembly\seemabk\seeobject
}

\providecommand{\miblasm}{
\toolsection{miblasm} is an assembler for the MicroBlaze hardware architecture.
It translates assembly code into machine code for MicroBlaze processors and stores it in corresponding object files.
\flowgraph{\resource{MicroBlaze assembly\\source code} \ar[r] & \toolbox{miblasm} \ar[r] & \resource{object file}}
\seeassembly\seemibl\seeobject
}

\providecommand{\mibldism}{
\toolsection{mibldism} is a disassembler for the MicroBlaze hardware architecture.
It translates machine code from object files targeting MicroBlaze processors into assembly code and writes it to the standard output stream.
\flowgraph{\resource{object file} \ar[r] & \toolbox{mibldism} \ar[r] & \resource{disassembly\\listing}}
\seeassembly\seemibl\seeobject
}

\providecommand{\mipsaasm}{
\toolsection{mips32asm} is an assembler for the MIPS32 hardware architecture.
It translates assembly code into machine code for MIPS32 processors and stores it in corresponding object files.
\flowgraph{\resource{MIPS32 assembly\\source code} \ar[r] & \toolbox{mips32asm} \ar[r] & \resource{object file}}
\seeassembly\seemips\seeobject
}

\providecommand{\mipsadism}{
\toolsection{mips32dism} is a disassembler for the MIPS32 hardware architecture.
It translates machine code from object files targeting MIPS32 processors into assembly code and writes it to the standard output stream.
\flowgraph{\resource{object file} \ar[r] & \toolbox{mips32dism} \ar[r] & \resource{disassembly\\listing}}
\seeassembly\seemips\seeobject
}

\providecommand{\mipsbasm}{
\toolsection{mips64asm} is an assembler for the MIPS64 hardware architecture.
It translates assembly code into machine code for MIPS64 processors and stores it in corresponding object files.
\flowgraph{\resource{MIPS64 assembly\\source code} \ar[r] & \toolbox{mips64asm} \ar[r] & \resource{object file}}
\seeassembly\seemips\seeobject
}

\providecommand{\mipsbdism}{
\toolsection{mips64dism} is a disassembler for the MIPS64 hardware architecture.
It translates machine code from object files targeting MIPS64 processors into assembly code and writes it to the standard output stream.
\flowgraph{\resource{object file} \ar[r] & \toolbox{mips64dism} \ar[r] & \resource{disassembly\\listing}}
\seeassembly\seemips\seeobject
}

\providecommand{\mmixasm}{
\toolsection{mmixasm} is an assembler for the MMIX hardware architecture.
It translates assembly code into machine code for MMIX processors and stores it in corresponding object files.
The names of all special registers are predefined and evaluate to the corresponding number.
\flowgraph{\resource{MMIX assembly\\source code} \ar[r] & \toolbox{mmixasm} \ar[r] & \resource{object file}}
\seeassembly\seemmix\seeobject
}

\providecommand{\mmixdism}{
\toolsection{mmixdism} is a disassembler for the MMIX hardware architecture.
It translates machine code from object files targeting MMIX processors into assembly code and writes it to the standard output stream.
\flowgraph{\resource{object file} \ar[r] & \toolbox{mmixdism} \ar[r] & \resource{disassembly\\listing}}
\seeassembly\seemmix\seeobject
}

\providecommand{\orokasm}{
\toolsection{or1kasm} is an assembler for the OpenRISC 1000 hardware architecture.
It translates assembly code into machine code for OpenRISC 1000 processors and stores it in corresponding object files.
\flowgraph{\resource{OpenRISC 1000 assembly\\source code} \ar[r] & \toolbox{or1kasm} \ar[r] & \resource{object file}}
\seeassembly\seeorok\seeobject
}

\providecommand{\orokdism}{
\toolsection{or1kdism} is a disassembler for the OpenRISC 1000 hardware architecture.
It translates machine code from object files targeting OpenRISC 1000 processors into assembly code and writes it to the standard output stream.
\flowgraph{\resource{object file} \ar[r] & \toolbox{or1kdism} \ar[r] & \resource{disassembly\\listing}}
\seeassembly\seeorok\seeobject
}

\providecommand{\ppcaasm}{
\toolsection{ppc32asm} is an assembler for the PowerPC hardware architecture.
It translates assembly code into machine code for PowerPC processors and stores it in corresponding object files.
By default, the assembler generates machine code for the 32-bit operating mode defined by the PowerPC architecture.
\flowgraph{\resource{PowerPC assembly\\source code} \ar[r] & \toolbox{ppc32asm} \ar[r] & \resource{object file}}
\seeassembly\seeppc\seeobject
}

\providecommand{\ppcadism}{
\toolsection{ppc32dism} is a disassembler for the PowerPC hardware architecture.
It translates machine code from object files targeting PowerPC processors into assembly code and writes it to the standard output stream.
It assumes that the machine code was generated for the 32-bit operating mode defined by the PowerPC architecture.
\flowgraph{\resource{object file} \ar[r] & \toolbox{ppc32dism} \ar[r] & \resource{disassembly\\listing}}
\seeassembly\seeppc\seeobject
}

\providecommand{\ppcbasm}{
\toolsection{ppc64asm} is an assembler for the PowerPC hardware architecture.
It translates assembly code into machine code for PowerPC processors and stores it in corresponding object files.
By default, the assembler generates machine code for the 64-bit operating mode defined by the PowerPC architecture.
\flowgraph{\resource{PowerPC assembly\\source code} \ar[r] & \toolbox{ppc64asm} \ar[r] & \resource{object file}}
\seeassembly\seeppc\seeobject
}

\providecommand{\ppcbdism}{
\toolsection{ppc64dism} is a disassembler for the PowerPC hardware architecture.
It translates machine code from object files targeting PowerPC processors into assembly code and writes it to the standard output stream.
It assumes that the machine code was generated for the 64-bit operating mode defined by the PowerPC architecture.
\flowgraph{\resource{object file} \ar[r] & \toolbox{ppc64dism} \ar[r] & \resource{disassembly\\listing}}
\seeassembly\seeppc\seeobject
}

\providecommand{\riscasm}{
\toolsection{riscasm} is an assembler for the RISC hardware architecture.
It translates assembly code into machine code for RISC processors and stores it in corresponding object files.
The names of all special registers are predefined and evaluate to the corresponding number.
\flowgraph{\resource{RISC assembly\\source code} \ar[r] & \toolbox{riscasm} \ar[r] & \resource{object file}}
\seeassembly\seerisc\seeobject
}

\providecommand{\riscdism}{
\toolsection{riscdism} is a disassembler for the RISC hardware architecture.
It translates machine code from object files targeting RISC processors into assembly code and writes it to the standard output stream.
\flowgraph{\resource{object file} \ar[r] & \toolbox{riscdism} \ar[r] & \resource{disassembly\\listing}}
\seeassembly\seerisc\seeobject
}

\providecommand{\wasmasm}{
\toolsection{wasmasm} is an assembler for the WebAssembly architecture.
It translates assembly code into machine code for WebAssembly targets and stores it in corresponding object files.
The names of all special registers are predefined and evaluate to the corresponding number.
\flowgraph{\resource{WebAssembly assembly\\source code} \ar[r] & \toolbox{wasmasm} \ar[r] & \resource{object file}}
\seeassembly\seewasm\seeobject
}

\providecommand{\wasmdism}{
\toolsection{wasmdism} is a disassembler for the WebAssembly architecture.
It translates machine code from object files targeting WebAssembly targets into assembly code and writes it to the standard output stream.
\flowgraph{\resource{object file} \ar[r] & \toolbox{wasmdism} \ar[r] & \resource{disassembly\\listing}}
\seeassembly\seewasm\seeobject
}

% linker tools

\providecommand{\linklib}{
\toolsection{linklib} is an object file combiner.
It creates a static library file by combining all object files given to it into a single one.
\flowgraph{\resource{object files} \ar[r] & \toolbox{linklib} \ar[r] & \resource{library file}}
\seeobject
}

\providecommand{\linkbin}{
\toolsection{linkbin} is a linker for plain binary files.
It links all object files given to it into a single image and stores it in a binary file that begins with the first linked section.
It also creates a map file that lists the address, type, name and size of all used sections.
The filename extension of the resulting binary file can be specified by putting it into a constant data section called \texttt{\_extension}.
\flowgraph{\resource{object files} \ar[r] & \toolbox{linkbin} \ar[r] \ar[d] & \resource{binary file} \\ & \resource{map file}}
\seeobject
}

\providecommand{\linkmem}{
\toolsection{linkmem} is a linker for plain binary files partitioned into random-access and read-only memory.
It links all object files given to it into two distinct images, one for data sections and one for code and constant data sections, and stores each image in a binary file that begins with the first linked section of the corresponding type.
It also creates a map file that lists the address, type, name and size of all used sections.
\flowgraph{\resource{object files} \ar[r] & \toolbox{linkmem} \ar[r] \ar[d] & \resource{RAM file/\\ROM file} \\ & \resource{map file}}
\seeobject
}

\providecommand{\linkprg}{
\toolsection{linkprg} is a linker for GEMDOS executable files.
It links all object files given to it into a single image and stores the image in an Atari GEMDOS executable file~\cite{gemdosfile}.
It also creates a map file that lists the address relative to the text segment, type, name and size of all used sections.
The filename extension of the resulting executable file can be specified by putting it into a constant data section called \texttt{\_extension}.
The GEMDOS executable file format requires all patch patterns of absolute link patches to consist of four full bitmasks with descending offsets.
\flowgraph{\resource{object files} \ar[r] & \toolbox{linkprg} \ar[r] \ar[d] & \resource{executable file} \\ & \resource{map file}}
\seeobject
}

\providecommand{\linkhex}{
\toolsection{linkhex} is a linker for Intel HEX files.
It links all code sections of the object files given to it into single image and stores the image in an Intel HEX file~\cite{hexfile} that begins with the first linked section.
It also creates a map file that lists the address, type, name and size of all used sections.
\flowgraph{\resource{object files} \ar[r] & \toolbox{linkhex} \ar[r] \ar[d] & \resource{HEX file} \\ & \resource{map file}}
\seeobject
}

\providecommand{\mapsearch}{
\toolsection{mapsearch} is a debugging tool.
It searches map files generated by linker tools for the name of a binary section that encompasses a memory address read from the standard input stream.
If additionally provided with one or more object files, it also stores an excerpt thereof in a separate object file called map search result which only contains the identified binary section for disassembling purposes.
\flowgraph{& \resource{map files/\\object files} \ar[d] \\ \resource{memory\\address} \ar[r] & \toolbox{mapsearch} \ar[r] \ar[d] & \resource{section name/\\relative offset} \\ & \resource{object file\\excerpt}}
\seeobject
}

\renewcommand{\seerisc}{}

\startchapter{RISC}{RISC Hardware Architecture Support}{risc}
{This \documentation{} describes how the \ecs{} supports the RISC hardware architecture.
This includes information about the assembler, disassembler, and the various compilers featured by the \ecs{} as well as the interoperability between these tools.}

\section{Introduction}

The \ecs{} features various compilers, an assembler, and a disassembler that target the RISC hardware architecture.
Figure~\ref{fig:riscdataflow} shows the data flow in-between these tools.

\begin{figure}
\flowgraph{
\resource{intermediate\\code} \ar[d] & & \resource{assembly\\source code} \ar[d] \\
\converter{RISC\\Generator} \ar[r] \ar[rd] \ar[d] & \resource{assembly\\listing} \ar[r] & \converter{RISC\\Assembler} \ar[ld] \\
\resource{debugging\\information} & \resource{object file} \ar[d] \\
& \converter{RISC\\Disassembler} \ar[d] \\
& \resource{disassembly\\listing} \\
}\caption{Data flow within the tools targeting the RISC architecture}
\label{fig:riscdataflow}
\end{figure}

All compilers targeting the RISC architecture translate their programs using an intermediate code representation.
The RISC generator is able to translate the intermediate code representation of a program into machine code for RISC processors.
It stores the resulting binary code and data in so-called object files.
Additionally, the generator is able to create an assembly code listing of the machine code for debugging purposes.
This assembly code listing can also be processed by the assembler yielding exactly the same object file.
The disassembler is able to open object files and print a human-readable disassembly listing of their contents.
\seeobject\seecode

\section{Instruction Set}

Tools targeting the RISC architecture support the instruction set listed in Table~\ref{tab:riscset} and use the same assembly syntax as predefined by Niklaus Wirth~\cite{risc:instructionset}.
\seeassembly

\instructionset{risc}{Supported RISC instruction set}{5}{6}

\section{Calling Convention}\index{Calling convention!of RISC}

The machine code generator and runtime support for the RISC architecture as provided by the \ecs{} use the following calling convention in order to enable interoperability.

\subsection{Stack Operations}

Arguments for functions as well as the return address are in general passed using the stack according to the intermediate code specification.
See \Documentation{}~\documentationref{code}{Intermediate Code Representation} for more information about the role of the stack.
Function arguments are pushed on the stack in reverse order and cleaned by the caller.

\subsection{Register Mapping}

The special-purpose registers defined by the intermediate code representation are mapped to their corresponding physical registers in the following way:

\begin{itemize}

\item Result Register\alignright\texttt{\$res}\nopagebreak

The intermediate code result register \texttt{\$res} is mapped to RISC registers \texttt{r0} and \texttt{r1} depending on the size of the actual return type.

\item Stack Pointer Register\alignright\texttt{\$sp}\nopagebreak

The intermediate code stack pointer register \texttt{\$sp} is mapped to RISC register \texttt{r14}.

\item Frame Pointer Register\alignright\texttt{\$fp}\nopagebreak

The intermediate code frame pointer register \texttt{\$fp} is mapped to RISC register \texttt{r13}.

\item Link Register\alignright\texttt{\$lnk}\nopagebreak

The intermediate code link register \texttt{\$lnk} is mapped to RISC register \texttt{r15}.

\end{itemize}

All other intermediate code registers are mapped as needed to the remaining physical registers.
Their contents and mapping are therefore considered volatile across function calls.

\section{Runtime Support}\index{Runtime support!for RISC}

The \ecs{} provides runtime support for the RISC architecture and runtime environments based on this hardware architecture in object files.
Users targeting a specific runtime environment have to use an appropriate linker together with these object files in order create an executable program.
This section gives information about all supported runtime environments based on the RISC hardware architecture as well as the required combination of linker and object files.

Basic architectural runtime support is provided by the object file \objfile{risc\-run}.
Users should always include this object file during linking regardless of the actual target runtime environment.
All other object files given to the linker should target the same hardware architecture.

Programs written in \cpp{} need additional runtime support stored in the \libfile{cpp\-risc\-run} library file.
Programs written in Oberon need additional runtime support stored in the \libfile{ob\-risc\-run} library file.
\seecpp\seeoberon

Programs targeting RISC microcontrollers are created using the \tool{link\-bin} linker tool.
It creates a bootloader that can be stored in RISC disk images at block \texttt{0x80004} if provided with the runtime support stored in the \objfile{risc\-disk\-run} object file.
Calling the \tool{ecsd} utility tool using the \environment{risc\-disk} target environment achieves the same result.

\section{RISC Tools}

The \ecs{} provides the following tools that are able to process object files targeting the RISC hardware architecture.
\interface

\cdrisc
\cpprisc
\falrisc
\obrisc
\riscasm
\riscdism
\linkbin

\concludechapter

// WebAssembly instruction set definitions
// Copyright (C) Florian Negele

// This file is part of the Eigen Compiler Suite.

// The ECS is free software: you can redistribute it and/or modify
// it under the terms of the GNU General Public License as published by
// the Free Software Foundation, either version 3 of the License, or
// (at your option) any later version.

// The ECS is distributed in the hope that it will be useful,
// but WITHOUT ANY WARRANTY; without even the implied warranty of
// MERCHANTABILITY or FITNESS FOR A PARTICULAR PURPOSE.  See the
// GNU General Public License for more details.

// You should have received a copy of the GNU General Public License
// along with the ECS.  If not, see <https://www.gnu.org/licenses/>.

#ifndef INSTR
	#define INSTR(mnem, code, integer, type1, type2, type3)
#endif

#ifndef MNEM
	#define MNEM(name, mnem)
#endif

#ifndef TYPE
	#define TYPE(type)
#endif

#ifndef TYPECODE
	#define TYPECODE(name, code, value)
#endif

// mnemonics

MNEM (block,                          BLOCK)
MNEM (br,                             BR)
MNEM (br_if,                          BR_IF)
MNEM (br_table,                       BR_TABLE)
MNEM (call,                           CALL)
MNEM (call_indirect,                  CALL_INDIRECT)
MNEM (data.drop,                      DATADROP)
MNEM (drop,                           DROP)
MNEM (elem.drop,                      ELEMDROP)
MNEM (else,                           ELSE)
MNEM (end,                            END)
MNEM (f32.abs,                        F32ABS)
MNEM (f32.add,                        F32ADD)
MNEM (f32.ceil,                       F32CEIL)
MNEM (f32.const,                      F32CONST)
MNEM (f32.convert_i32_s,              F32CONVERT_I32_S)
MNEM (f32.convert_i32_u,              F32CONVERT_I32_U)
MNEM (f32.convert_i64_s,              F32CONVERT_I64_S)
MNEM (f32.convert_i64_u,              F32CONVERT_I64_U)
MNEM (f32.copysign,                   F32COPYSIGN)
MNEM (f32.demote_f64,                 F32DEMOTE_F64)
MNEM (f32.div,                        F32DIV)
MNEM (f32.eq,                         F32EQ)
MNEM (f32.floor,                      F32FLOOR)
MNEM (f32.ge,                         F32GE)
MNEM (f32.gt,                         F32GT)
MNEM (f32.le,                         F32LE)
MNEM (f32.load,                       F32LOAD)
MNEM (f32.lt,                         F32LT)
MNEM (f32.max,                        F32MAX)
MNEM (f32.min,                        F32MIN)
MNEM (f32.mul,                        F32MUL)
MNEM (f32.ne,                         F32NE)
MNEM (f32.nearest,                    F32NEAREST)
MNEM (f32.neg,                        F32NEG)
MNEM (f32.reinterpret_i32,            F32REINTERPRET_I32)
MNEM (f32.sqrt,                       F32SQRT)
MNEM (f32.store,                      F32STORE)
MNEM (f32.sub,                        F32SUB)
MNEM (f32.trunc,                      F32TRUNC)
MNEM (f32x4.abs,                      F32X4ABS)
MNEM (f32x4.add,                      F32X4ADD)
MNEM (f32x4.ceil,                     F32X4CEIL)
MNEM (f32x4.convert_i32x4_s,          F32X4CONVERT_I32X4_S)
MNEM (f32x4.convert_i32x4_u,          F32X4CONVERT_I32X4_U)
MNEM (f32x4.demote_f64x2_zero,        F32X4DEMOTE_F64X2_ZERO)
MNEM (f32x4.div,                      F32X4DIV)
MNEM (f32x4.eq,                       F32X4EQ)
MNEM (f32x4.extract_lane,             F32X4EXTRACT_LANE)
MNEM (f32x4.floor,                    F32X4FLOOR)
MNEM (f32x4.ge,                       F32X4GE)
MNEM (f32x4.gt,                       F32X4GT)
MNEM (f32x4.le,                       F32X4LE)
MNEM (f32x4.lt,                       F32X4LT)
MNEM (f32x4.max,                      F32X4MAX)
MNEM (f32x4.min,                      F32X4MIN)
MNEM (f32x4.mul,                      F32X4MUL)
MNEM (f32x4.ne,                       F32X4NE)
MNEM (f32x4.nearest,                  F32X4NEAREST)
MNEM (f32x4.neg,                      F32X4NEG)
MNEM (f32x4.pmax,                     F32X4PMAX)
MNEM (f32x4.pmin,                     F32X4PMIN)
MNEM (f32x4.replace_lane,             F32X4REPLACE_LANE)
MNEM (f32x4.splat,                    F32X4SPLAT)
MNEM (f32x4.sqrt,                     F32X4SQRT)
MNEM (f32x4.sub,                      F32X4SUB)
MNEM (f32x4.trunc,                    F32X4TRUNC)
MNEM (f64.abs,                        F64ABS)
MNEM (f64.add,                        F64ADD)
MNEM (f64.ceil,                       F64CEIL)
MNEM (f64.const,                      F64CONST)
MNEM (f64.convert_i32_s,              F64CONVERT_I32_S)
MNEM (f64.convert_i32_u,              F64CONVERT_I32_U)
MNEM (f64.convert_i64_s,              F64CONVERT_I64_S)
MNEM (f64.convert_i64_u,              F64CONVERT_I64_U)
MNEM (f64.copysign,                   F64COPYSIGN)
MNEM (f64.div,                        F64DIV)
MNEM (f64.eq,                         F64EQ)
MNEM (f64.floor,                      F64FLOOR)
MNEM (f64.ge,                         F64GE)
MNEM (f64.gt,                         F64GT)
MNEM (f64.le,                         F64LE)
MNEM (f64.load,                       F64LOAD)
MNEM (f64.lt,                         F64LT)
MNEM (f64.max,                        F64MAX)
MNEM (f64.min,                        F64MIN)
MNEM (f64.mul,                        F64MUL)
MNEM (f64.ne,                         F64NE)
MNEM (f64.nearest,                    F64NEAREST)
MNEM (f64.neg,                        F64NEG)
MNEM (f64.promote_f32,                F64PROMOTE_F32)
MNEM (f64.reinterpret_i64,            F64REINTERPRET_I64)
MNEM (f64.sqrt,                       F64SQRT)
MNEM (f64.store,                      F64STORE)
MNEM (f64.sub,                        F64SUB)
MNEM (f64.trunc,                      F64TRUNC)
MNEM (f64x2.abs,                      F64X2ABS)
MNEM (f64x2.add,                      F64X2ADD)
MNEM (f64x2.ceil,                     F64X2CEIL)
MNEM (f64x2.convert_low_i32x4_s,      F64X2CONVERT_LOW_I32X4_S)
MNEM (f64x2.convert_low_i32x4_u,      F64X2CONVERT_LOW_I32X4_U)
MNEM (f64x2.div,                      F64X2DIV)
MNEM (f64x2.eq,                       F64X2EQ)
MNEM (f64x2.extract_lane,             F64X2EXTRACT_LANE)
MNEM (f64x2.floor,                    F64X2FLOOR)
MNEM (f64x2.ge,                       F64X2GE)
MNEM (f64x2.gt,                       F64X2GT)
MNEM (f64x2.le,                       F64X2LE)
MNEM (f64x2.lt,                       F64X2LT)
MNEM (f64x2.max,                      F64X2MAX)
MNEM (f64x2.min,                      F64X2MIN)
MNEM (f64x2.mul,                      F64X2MUL)
MNEM (f64x2.ne,                       F64X2NE)
MNEM (f64x2.nearest,                  F64X2NEAREST)
MNEM (f64x2.neg,                      F64X2NEG)
MNEM (f64x2.pmax,                     F64X2PMAX)
MNEM (f64x2.pmin,                     F64X2PMIN)
MNEM (f64x2.promote_low_f32x4,        F64X2PROMOTE_LOW_F32X4)
MNEM (f64x2.replace_lane,             F64X2REPLACE_LANE)
MNEM (f64x2.splat,                    F64X2SPLAT)
MNEM (f64x2.sqrt,                     F64X2SQRT)
MNEM (f64x2.sub,                      F64X2SUB)
MNEM (f64x2.trunc,                    F64X2TRUNC)
MNEM (global.get,                     GLOBALGET)
MNEM (global.set,                     GLOBALSET)
MNEM (i16x8.abs,                      I16X8ABS)
MNEM (i16x8.add,                      I16X8ADD)
MNEM (i16x8.add_sat_s,                I16X8ADD_SAT_S)
MNEM (i16x8.add_sat_u,                I16X8ADD_SAT_U)
MNEM (i16x8.all_true,                 I16X8ALL_TRUE)
MNEM (i16x8.avgr_u,                   I16X8AVGR_U)
MNEM (i16x8.bitmask,                  I16X8BITMASK)
MNEM (i16x8.eq,                       I16X8EQ)
MNEM (i16x8.extadd_pairwise_i8x16_s,  I16X8EXTADD_PAIRWISE_I8X16_S)
MNEM (i16x8.extadd_pairwise_i8x16_u,  I16X8EXTADD_PAIRWISE_I8X16_U)
MNEM (i16x8.extend_high_i8x16_s,      I16X8EXTEND_HIGH_I8X16_S)
MNEM (i16x8.extend_high_i8x16_u,      I16X8EXTEND_HIGH_I8X16_U)
MNEM (i16x8.extend_low_i8x16_s,       I16X8EXTEND_LOW_I8X16_S)
MNEM (i16x8.extend_low_i8x16_u,       I16X8EXTEND_LOW_I8X16_U)
MNEM (i16x8.extmul_high_i8x16_s,      I16X8EXTMUL_HIGH_I8X16_S)
MNEM (i16x8.extmul_high_i8x16_u,      I16X8EXTMUL_HIGH_I8X16_U)
MNEM (i16x8.extmul_low_i8x16_s,       I16X8EXTMUL_LOW_I8X16_S)
MNEM (i16x8.extmul_low_i8x16_u,       I16X8EXTMUL_LOW_I8X16_U)
MNEM (i16x8.extract_lane_s,           I16X8EXTRACT_LANE_S)
MNEM (i16x8.extract_lane_u,           I16X8EXTRACT_LANE_U)
MNEM (i16x8.ge_s,                     I16X8GE_S)
MNEM (i16x8.ge_u,                     I16X8GE_U)
MNEM (i16x8.gt_s,                     I16X8GT_S)
MNEM (i16x8.gt_u,                     I16X8GT_U)
MNEM (i16x8.le_s,                     I16X8LE_S)
MNEM (i16x8.le_u,                     I16X8LE_U)
MNEM (i16x8.lt_s,                     I16X8LT_S)
MNEM (i16x8.lt_u,                     I16X8LT_U)
MNEM (i16x8.max_s,                    I16X8MAX_S)
MNEM (i16x8.max_u,                    I16X8MAX_U)
MNEM (i16x8.min_s,                    I16X8MIN_S)
MNEM (i16x8.min_u,                    I16X8MIN_U)
MNEM (i16x8.mul,                      I16X8MUL)
MNEM (i16x8.narrow_i32x4_s,           I16X8NARROW_I32X4_S)
MNEM (i16x8.narrow_i32x4_u,           I16X8NARROW_I32X4_U)
MNEM (i16x8.ne,                       I16X8NE)
MNEM (i16x8.neg,                      I16X8NEG)
MNEM (i16x8.q15mulr_sat_s,            I16X8Q15MULR_SAT_S)
MNEM (i16x8.replace_lane,             I16X8REPLACE_LANE)
MNEM (i16x8.shl,                      I16X8SHL)
MNEM (i16x8.shr_s,                    I16X8SHR_S)
MNEM (i16x8.shr_u,                    I16X8SHR_U)
MNEM (i16x8.splat,                    I16X8SPLAT)
MNEM (i16x8.sub,                      I16X8SUB)
MNEM (i16x8.sub_sat_s,                I16X8SUB_SAT_S)
MNEM (i16x8.sub_sat_u,                I16X8SUB_SAT_U)
MNEM (i32,                            I32)
MNEM (i32.add,                        I32ADD)
MNEM (i32.and,                        I32AND)
MNEM (i32.clz,                        I32CLZ)
MNEM (i32.const,                      I32CONST)
MNEM (i32.ctz,                        I32CTZ)
MNEM (i32.div_s,                      I32DIV_S)
MNEM (i32.div_u,                      I32DIV_U)
MNEM (i32.eq,                         I32EQ)
MNEM (i32.eqz,                        I32EQZ)
MNEM (i32.extend16_s,                 I32EXTEND16_S)
MNEM (i32.extend8_s,                  I32EXTEND8_S)
MNEM (i32.ge_s,                       I32GE_S)
MNEM (i32.ge_u,                       I32GE_U)
MNEM (i32.gt_s,                       I32GT_S)
MNEM (i32.gt_u,                       I32GT_U)
MNEM (i32.le_s,                       I32LE_S)
MNEM (i32.le_u,                       I32LE_U)
MNEM (i32.load,                       I32LOAD)
MNEM (i32.load16_s,                   I32LOAD16_S)
MNEM (i32.load16_u,                   I32LOAD16_U)
MNEM (i32.load8_s,                    I32LOAD8_S)
MNEM (i32.load8_u,                    I32LOAD8_U)
MNEM (i32.lt_s,                       I32LT_S)
MNEM (i32.lt_u,                       I32LT_U)
MNEM (i32.mul,                        I32MUL)
MNEM (i32.ne,                         I32NE)
MNEM (i32.or,                         I32OR)
MNEM (i32.popcnt,                     I32POPCNT)
MNEM (i32.reinterpret_f32,            I32REINTERPRET_F32)
MNEM (i32.rem_s,                      I32REM_S)
MNEM (i32.rem_u,                      I32REM_U)
MNEM (i32.rotl,                       I32ROTL)
MNEM (i32.rotr,                       I32ROTR)
MNEM (i32.shl,                        I32SHL)
MNEM (i32.shr_s,                      I32SHR_S)
MNEM (i32.shr_u,                      I32SHR_U)
MNEM (i32.store,                      I32STORE)
MNEM (i32.store16,                    I32STORE16)
MNEM (i32.store8,                     I32STORE8)
MNEM (i32.sub,                        I32SUB)
MNEM (i32.trunc_f32_s,                I32TRUNC_F32_S)
MNEM (i32.trunc_f32_u,                I32TRUNC_F32_U)
MNEM (i32.trunc_f64_s,                I32TRUNC_F64_S)
MNEM (i32.trunc_f64_u,                I32TRUNC_F64_U)
MNEM (i32.trunc_sat_f32_s,            I32TRUNC_SAT_F32_S)
MNEM (i32.trunc_sat_f32_u,            I32TRUNC_SAT_F32_U)
MNEM (i32.trunc_sat_f64_s,            I32TRUNC_SAT_F64_S)
MNEM (i32.trunc_sat_f64_u,            I32TRUNC_SAT_F64_U)
MNEM (i32.wrap_i64,                   I32WRAP_I64)
MNEM (i32.xor,                        I32XOR)
MNEM (i32x4.abs,                      I32X4ABS)
MNEM (i32x4.add,                      I32X4ADD)
MNEM (i32x4.all_true,                 I32X4ALL_TRUE)
MNEM (i32x4.bitmask,                  I32X4BITMASK)
MNEM (i32x4.dot_i16x8_s,              I32X4DOT_I16X8_S)
MNEM (i32x4.eq,                       I32X4EQ)
MNEM (i32x4.extadd_pairwise_i16x8_s,  I32X4EXTADD_PAIRWISE_I16X8_S)
MNEM (i32x4.extadd_pairwise_i16x8_u,  I32X4EXTADD_PAIRWISE_I16X8_U)
MNEM (i32x4.extend_high_i16x8_s,      I32X4EXTEND_HIGH_I16X8_S)
MNEM (i32x4.extend_high_i16x8_u,      I32X4EXTEND_HIGH_I16X8_U)
MNEM (i32x4.extend_low_i16x8_s,       I32X4EXTEND_LOW_I16X8_S)
MNEM (i32x4.extend_low_i16x8_u,       I32X4EXTEND_LOW_I16X8_U)
MNEM (i32x4.extmul_high_i16x8_s,      I32X4EXTMUL_HIGH_I16X8_S)
MNEM (i32x4.extmul_high_i16x8_u,      I32X4EXTMUL_HIGH_I16X8_U)
MNEM (i32x4.extmul_low_i16x8_s,       I32X4EXTMUL_LOW_I16X8_S)
MNEM (i32x4.extmul_low_i16x8_u,       I32X4EXTMUL_LOW_I16X8_U)
MNEM (i32x4.extract_lane,             I32X4EXTRACT_LANE)
MNEM (i32x4.ge_s,                     I32X4GE_S)
MNEM (i32x4.ge_u,                     I32X4GE_U)
MNEM (i32x4.gt_s,                     I32X4GT_S)
MNEM (i32x4.gt_u,                     I32X4GT_U)
MNEM (i32x4.le_s,                     I32X4LE_S)
MNEM (i32x4.le_u,                     I32X4LE_U)
MNEM (i32x4.lt_s,                     I32X4LT_S)
MNEM (i32x4.lt_u,                     I32X4LT_U)
MNEM (i32x4.max_s,                    I32X4MAX_S)
MNEM (i32x4.max_u,                    I32X4MAX_U)
MNEM (i32x4.min_s,                    I32X4MIN_S)
MNEM (i32x4.min_u,                    I32X4MIN_U)
MNEM (i32x4.mul,                      I32X4MUL)
MNEM (i32x4.ne,                       I32X4NE)
MNEM (i32x4.neg,                      I32X4NEG)
MNEM (i32x4.replace_lane,             I32X4REPLACE_LANE)
MNEM (i32x4.shl,                      I32X4SHL)
MNEM (i32x4.shr_s,                    I32X4SHR_S)
MNEM (i32x4.shr_u,                    I32X4SHR_U)
MNEM (i32x4.splat,                    I32X4SPLAT)
MNEM (i32x4.sub,                      I32X4SUB)
MNEM (i32x4.trunc_sat_f32x4_s,        I32X4TRUNC_SAT_F32X4_S)
MNEM (i32x4.trunc_sat_f32x4_u,        I32X4TRUNC_SAT_F32X4_U)
MNEM (i32x4.trunc_sat_f64x2_s_zero,   I32X4TRUNC_SAT_F64X2_S_ZERO)
MNEM (i32x4.trunc_sat_f64x2_u_zero,   I32X4TRUNC_SAT_F64X2_U_ZERO)
MNEM (i64.add,                        I64ADD)
MNEM (i64.and,                        I64AND)
MNEM (i64.clz,                        I64CLZ)
MNEM (i64.const,                      I64CONST)
MNEM (i64.ctz,                        I64CTZ)
MNEM (i64.div_s,                      I64DIV_S)
MNEM (i64.div_u,                      I64DIV_U)
MNEM (i64.eq,                         I64EQ)
MNEM (i64.eqz,                        I64EQZ)
MNEM (i64.extend16_s,                 I64EXTEND16_S)
MNEM (i64.extend32_s,                 I64EXTEND32_S)
MNEM (i64.extend8_s,                  I64EXTEND8_S)
MNEM (i64.extend_i32_s,               I64EXTEND_I32_S)
MNEM (i64.extend_i32_u,               I64EXTEND_I32_U)
MNEM (i64.ge_s,                       I64GE_S)
MNEM (i64.ge_u,                       I64GE_U)
MNEM (i64.gt_s,                       I64GT_S)
MNEM (i64.gt_u,                       I64GT_U)
MNEM (i64.le_s,                       I64LE_S)
MNEM (i64.le_u,                       I64LE_U)
MNEM (i64.load,                       I64LOAD)
MNEM (i64.load16_s,                   I64LOAD16_S)
MNEM (i64.load16_u,                   I64LOAD16_U)
MNEM (i64.load32_s,                   I64LOAD32_S)
MNEM (i64.load32_u,                   I64LOAD32_U)
MNEM (i64.load8_s,                    I64LOAD8_S)
MNEM (i64.load8_u,                    I64LOAD8_U)
MNEM (i64.lt_s,                       I64LT_S)
MNEM (i64.lt_u,                       I64LT_U)
MNEM (i64.mul,                        I64MUL)
MNEM (i64.ne,                         I64NE)
MNEM (i64.or,                         I64OR)
MNEM (i64.popcnt,                     I64POPCNT)
MNEM (i64.reinterpret_f64,            I64REINTERPRET_F64)
MNEM (i64.rem_s,                      I64REM_S)
MNEM (i64.rem_u,                      I64REM_U)
MNEM (i64.rotl,                       I64ROTL)
MNEM (i64.rotr,                       I64ROTR)
MNEM (i64.shl,                        I64SHL)
MNEM (i64.shr_s,                      I64SHR_S)
MNEM (i64.shr_u,                      I64SHR_U)
MNEM (i64.store,                      I64STORE)
MNEM (i64.store16,                    I64STORE16)
MNEM (i64.store32,                    I64STORE32)
MNEM (i64.store8,                     I64STORE8)
MNEM (i64.sub,                        I64SUB)
MNEM (i64.trunc_f32_s,                I64TRUNC_F32_S)
MNEM (i64.trunc_f32_u,                I64TRUNC_F32_U)
MNEM (i64.trunc_f64_s,                I64TRUNC_F64_S)
MNEM (i64.trunc_f64_u,                I64TRUNC_F64_U)
MNEM (i64.trunc_sat_f32_s,            I64TRUNC_SAT_F32_S)
MNEM (i64.trunc_sat_f32_u,            I64TRUNC_SAT_F32_U)
MNEM (i64.trunc_sat_f64_s,            I64TRUNC_SAT_F64_S)
MNEM (i64.trunc_sat_f64_u,            I64TRUNC_SAT_F64_U)
MNEM (i64.xor,                        I64XOR)
MNEM (i64x2.abs,                      I64X2ABS)
MNEM (i64x2.add,                      I64X2ADD)
MNEM (i64x2.all_true,                 I64X2ALL_TRUE)
MNEM (i64x2.bitmask,                  I64X2BITMASK)
MNEM (i64x2.eq,                       I64X2EQ)
MNEM (i64x2.extend_high_i32x4_s,      I64X2EXTEND_HIGH_I32X4_S)
MNEM (i64x2.extend_high_i32x4_u,      I64X2EXTEND_HIGH_I32X4_U)
MNEM (i64x2.extend_low_i32x4_s,       I64X2EXTEND_LOW_I32X4_S)
MNEM (i64x2.extend_low_i32x4_u,       I64X2EXTEND_LOW_I32X4_U)
MNEM (i64x2.extmul_high_i32x4_s,      I64X2EXTMUL_HIGH_I32X4_S)
MNEM (i64x2.extmul_high_i32x4_u,      I64X2EXTMUL_HIGH_I32X4_U)
MNEM (i64x2.extmul_low_i32x4_s,       I64X2EXTMUL_LOW_I32X4_S)
MNEM (i64x2.extmul_low_i32x4_u,       I64X2EXTMUL_LOW_I32X4_U)
MNEM (i64x2.extract_lane,             I64X2EXTRACT_LANE)
MNEM (i64x2.ge_s,                     I64X2GE_S)
MNEM (i64x2.gt_s,                     I64X2GT_S)
MNEM (i64x2.le_s,                     I64X2LE_S)
MNEM (i64x2.lt_s,                     I64X2LT_S)
MNEM (i64x2.mul,                      I64X2MUL)
MNEM (i64x2.ne,                       I64X2NE)
MNEM (i64x2.neg,                      I64X2NEG)
MNEM (i64x2.replace_lane,             I64X2REPLACE_LANE)
MNEM (i64x2.shl,                      I64X2SHL)
MNEM (i64x2.shr_s,                    I64X2SHR_S)
MNEM (i64x2.shr_u,                    I64X2SHR_U)
MNEM (i64x2.splat,                    I64X2SPLAT)
MNEM (i64x2.sub,                      I64X2SUB)
MNEM (i8x16.abs,                      I8X16ABS)
MNEM (i8x16.add,                      I8X16ADD)
MNEM (i8x16.add_sat_s,                I8X16ADD_SAT_S)
MNEM (i8x16.add_sat_u,                I8X16ADD_SAT_U)
MNEM (i8x16.all_true,                 I8X16ALL_TRUE)
MNEM (i8x16.avgr_u,                   I8X16AVGR_U)
MNEM (i8x16.bitmask,                  I8X16BITMASK)
MNEM (i8x16.eq,                       I8X16EQ)
MNEM (i8x16.extract_lane_s,           I8X16EXTRACT_LANE_S)
MNEM (i8x16.extract_lane_u,           I8X16EXTRACT_LANE_U)
MNEM (i8x16.ge_s,                     I8X16GE_S)
MNEM (i8x16.ge_u,                     I8X16GE_U)
MNEM (i8x16.gt_s,                     I8X16GT_S)
MNEM (i8x16.gt_u,                     I8X16GT_U)
MNEM (i8x16.le_s,                     I8X16LE_S)
MNEM (i8x16.le_u,                     I8X16LE_U)
MNEM (i8x16.lt_s,                     I8X16LT_S)
MNEM (i8x16.lt_u,                     I8X16LT_U)
MNEM (i8x16.max_s,                    I8X16MAX_S)
MNEM (i8x16.max_u,                    I8X16MAX_U)
MNEM (i8x16.min_s,                    I8X16MIN_S)
MNEM (i8x16.min_u,                    I8X16MIN_U)
MNEM (i8x16.narrow_i16x8_s,           I8X16NARROW_I16X8_S)
MNEM (i8x16.narrow_i16x8_u,           I8X16NARROW_I16X8_U)
MNEM (i8x16.ne,                       I8X16NE)
MNEM (i8x16.neg,                      I8X16NEG)
MNEM (i8x16.popcnt,                   I8X16POPCNT)
MNEM (i8x16.replace_lane,             I8X16REPLACE_LANE)
MNEM (i8x16.shl,                      I8X16SHL)
MNEM (i8x16.shr_s,                    I8X16SHR_S)
MNEM (i8x16.shr_u,                    I8X16SHR_U)
MNEM (i8x16.shuffle,                  I8X16SHUFFLE)
MNEM (i8x16.splat,                    I8X16SPLAT)
MNEM (i8x16.sub,                      I8X16SUB)
MNEM (i8x16.sub_sat_s,                I8X16SUB_SAT_S)
MNEM (i8x16.sub_sat_u,                I8X16SUB_SAT_U)
MNEM (i8x16.swizzle,                  I8X16SWIZZLE)
MNEM (if,                             IF)
MNEM (label,                          LABEL)
MNEM (lane,                           LANE)
MNEM (local.get,                      LOCALGET)
MNEM (local.set,                      LOCALSET)
MNEM (local.tee,                      LOCALTEE)
MNEM (loop,                           LOOP)
MNEM (memory.copy,                    MEMORYCOPY)
MNEM (memory.fill,                    MEMORYFILL)
MNEM (memory.grow,                    MEMORYGROW)
MNEM (memory.init,                    MEMORYINIT)
MNEM (memory.size,                    MEMORYSIZE)
MNEM (nop,                            NOP)
MNEM (ref.func,                       REFFUNC)
MNEM (ref.is_null,                    REFIS_NULL)
MNEM (ref.null,                       REFNULL)
MNEM (return,                         RETURN)
MNEM (s32,                            S32)
MNEM (select,                         SELECT)
MNEM (table.copy,                     TABLECOPY)
MNEM (table.fill,                     TABLEFILL)
MNEM (table.get,                      TABLEGET)
MNEM (table.grow,                     TABLEGROW)
MNEM (table.init,                     TABLEINIT)
MNEM (table.set,                      TABLESET)
MNEM (table.size,                     TABLESIZE)
MNEM (u32,                            U32)
MNEM (unreachable,                    UNREACHABLE)
MNEM (v128.and,                       V128AND)
MNEM (v128.andnot,                    V128ANDNOT)
MNEM (v128.any_true,                  V128ANY_TRUE)
MNEM (v128.bitselect,                 V128BITSELECT)
MNEM (v128.const,                     V128CONST)
MNEM (v128.load,                      V128LOAD)
MNEM (v128.load16_lane,               V128LOAD16_LANE)
MNEM (v128.load16_splat,              V128LOAD16_SPLAT)
MNEM (v128.load16x4_s,                V128LOAD16X4_S)
MNEM (v128.load16x4_u,                V128LOAD16X4_U)
MNEM (v128.load32_lane,               V128LOAD32_LANE)
MNEM (v128.load32_splat,              V128LOAD32_SPLAT)
MNEM (v128.load32_zero,               V128LOAD32_ZERO)
MNEM (v128.load32x2_s,                V128LOAD32X2_S)
MNEM (v128.load32x2_u,                V128LOAD32X2_U)
MNEM (v128.load64_lane,               V128LOAD64_LANE)
MNEM (v128.load64_splat,              V128LOAD64_SPLAT)
MNEM (v128.load64_zero,               V128LOAD64_ZERO)
MNEM (v128.load8_lane,                V128LOAD8_LANE)
MNEM (v128.load8_splat,               V128LOAD8_SPLAT)
MNEM (v128.load8x8_s,                 V128LOAD8X8_S)
MNEM (v128.load8x8_u,                 V128LOAD8X8_U)
MNEM (v128.not,                       V128NOT)
MNEM (v128.or,                        V128OR)
MNEM (v128.store,                     V128STORE)
MNEM (v128.store16_lane,              V128STORE16_LANE)
MNEM (v128.store32_lane,              V128STORE32_LANE)
MNEM (v128.store64_lane,              V128STORE64_LANE)
MNEM (v128.store8_lane,               V128STORE8_LANE)
MNEM (v128.xor,                       V128XOR)
MNEM (valtype,                        VALTYPE)
MNEM (vec,                            VEC)

// control instructions

INSTR (UNREACHABLE,                   0x00,  0,    Void,           Void,        Void)
INSTR (NOP,                           0x01,  0,    Void,           Void,        Void)
INSTR (BLOCK,                         0x02,  0,    BlockType,      Void,        Void)
INSTR (LOOP,                          0x03,  0,    BlockType,      Void,        Void)
INSTR (IF,                            0x04,  0,    BlockType,      Void,        Void)
INSTR (ELSE,                          0x05,  0,    Void,           Void,        Void)
INSTR (BR,                            0x0C,  0,    LabelIndex,     Void,        Void)
INSTR (END,                           0x0B,  0,    Void,           Void,        Void)
INSTR (BR_IF,                         0x0D,  0,    LabelIndex,     Void,        Void)
INSTR (BR_TABLE,                      0x0E,  0,    Void,           Void,        Void)
INSTR (RETURN,                        0x0F,  0,    Void,           Void,        Void)
INSTR (CALL,                          0x10,  0,    FunctionIndex,  Void,        Void)
INSTR (CALL,                          0x10,  0,    Void,           Void,        Void)
INSTR (CALL_INDIRECT,                 0x11,  0,    TypeIndex,      TableIndex,  Void)
INSTR (CALL_INDIRECT,                 0x11,  0,    Void,           Void,        Void)

// reference instructions

INSTR (REFNULL,                       0xD0,  0,    ReferenceType,  Void,        Void)
INSTR (REFIS_NULL,                    0xD1,  0,    Void,           Void,        Void)
INSTR (REFFUNC,                       0xD2,  0,    FunctionIndex,  Void,        Void)
INSTR (REFFUNC,                       0xD2,  0,    Void,           Void,        Void)

// parametric instructions

INSTR (DROP,                          0x1A,  0,    Void,           Void,        Void)
INSTR (SELECT,                        0x1B,  0,    Void,           Void,        Void)
INSTR (SELECT,                        0x1C,  0,    ValueType,      Void,        Void)

// variable instructions

INSTR (LOCALGET,                      0x20,  0,    LocalIndex,     Void,        Void)
INSTR (LOCALGET,                      0x20,  0,    Void,           Void,        Void)
INSTR (LOCALSET,                      0x21,  0,    LocalIndex,     Void,        Void)
INSTR (LOCALSET,                      0x21,  0,    Void,           Void,        Void)
INSTR (LOCALTEE,                      0x22,  0,    LocalIndex,     Void,        Void)
INSTR (LOCALTEE,                      0x22,  0,    Void,           Void,        Void)
INSTR (GLOBALGET,                     0x23,  0,    GlobalIndex,    Void,        Void)
INSTR (GLOBALGET,                     0x23,  0,    Void,           Void,        Void)
INSTR (GLOBALSET,                     0x24,  0,    GlobalIndex,    Void,        Void)
INSTR (GLOBALSET,                     0x24,  0,    Void,           Void,        Void)

// table instructions

INSTR (TABLEGET,                      0x25,  0,    TableIndex,     Void,        Void)
INSTR (TABLEGET,                      0x25,  0,    Void,           Void,        Void)
INSTR (TABLESET,                      0x26,  0,    TableIndex,     Void,        Void)
INSTR (TABLESET,                      0x26,  0,    Void,           Void,        Void)
INSTR (TABLEINIT,                     0xFC,  12,   ElementIndex,   TableIndex,  Void)
INSTR (TABLEINIT,                     0xFC,  12,   Void,           Void,        Void)
INSTR (ELEMDROP,                      0xFC,  13,   ElementIndex,   Void,        Void)
INSTR (ELEMDROP,                      0xFC,  13,   Void,           Void,        Void)
INSTR (TABLECOPY,                     0xFC,  14,   TableIndex,     TableIndex,  Void)
INSTR (TABLECOPY,                     0xFC,  14,   Void,           Void,        Void)
INSTR (TABLEGROW,                     0xFC,  15,   TableIndex,     Void,        Void)
INSTR (TABLEGROW,                     0xFC,  15,   Void,           Void,        Void)
INSTR (TABLESIZE,                     0xFC,  16,   TableIndex,     Void,        Void)
INSTR (TABLESIZE,                     0xFC,  16,   Void,           Void,        Void)
INSTR (TABLEFILL,                     0xFC,  17,   TableIndex,     Void,        Void)
INSTR (TABLEFILL,                     0xFC,  17,   Void,           Void,        Void)

// memory instructions

INSTR (I32LOAD,                       0x28,  0,    Alignment,      Offset,      Void)
INSTR (I64LOAD,                       0x29,  0,    Alignment,      Offset,      Void)
INSTR (F32LOAD,                       0x2A,  0,    Alignment,      Offset,      Void)
INSTR (F64LOAD,                       0x2B,  0,    Alignment,      Offset,      Void)
INSTR (I32LOAD8_S,                    0x2C,  0,    Alignment,      Offset,      Void)
INSTR (I32LOAD8_U,                    0x2D,  0,    Alignment,      Offset,      Void)
INSTR (I32LOAD16_S,                   0x2E,  0,    Alignment,      Offset,      Void)
INSTR (I32LOAD16_U,                   0x2F,  0,    Alignment,      Offset,      Void)
INSTR (I64LOAD8_S,                    0x30,  0,    Alignment,      Offset,      Void)
INSTR (I64LOAD8_U,                    0x31,  0,    Alignment,      Offset,      Void)
INSTR (I64LOAD16_S,                   0x32,  0,    Alignment,      Offset,      Void)
INSTR (I64LOAD16_U,                   0x33,  0,    Alignment,      Offset,      Void)
INSTR (I64LOAD32_S,                   0x34,  0,    Alignment,      Offset,      Void)
INSTR (I64LOAD32_U,                   0x35,  0,    Alignment,      Offset,      Void)
INSTR (I32STORE,                      0x36,  0,    Alignment,      Offset,      Void)
INSTR (I64STORE,                      0x37,  0,    Alignment,      Offset,      Void)
INSTR (F32STORE,                      0x38,  0,    Alignment,      Offset,      Void)
INSTR (F64STORE,                      0x39,  0,    Alignment,      Offset,      Void)
INSTR (I32STORE8,                     0x3A,  0,    Alignment,      Offset,      Void)
INSTR (I32STORE16,                    0x3B,  0,    Alignment,      Offset,      Void)
INSTR (I64STORE8,                     0x3C,  0,    Alignment,      Offset,      Void)
INSTR (I64STORE16,                    0x3D,  0,    Alignment,      Offset,      Void)
INSTR (I64STORE32,                    0x3E,  0,    Alignment,      Offset,      Void)
INSTR (MEMORYSIZE,                    0x3F,  0,    ZeroIndex,      Void,        Void)
INSTR (MEMORYGROW,                    0x40,  0,    ZeroIndex,      Void,        Void)
INSTR (MEMORYINIT,                    0xFC,  8,    DataIndex,      ZeroIndex,   Void)
INSTR (MEMORYINIT,                    0xFC,  8,    Void,           Void,        Void)
INSTR (DATADROP,                      0xFC,  9,    DataIndex,      Void,        Void)
INSTR (DATADROP,                      0xFC,  9,    Void,           Void,        Void)
INSTR (MEMORYCOPY,                    0xFC,  10,   ZeroIndex,      ZeroIndex,   Void)
INSTR (MEMORYFILL,                    0xFC,  11,   ZeroIndex,      Void,        Void)

// numeric instructions

INSTR (I32CONST,                      0x41,  0,    Signed32,       Void,        Void)
INSTR (I32CONST,                      0x41,  0,    Void,           Void,        Void)
INSTR (I64CONST,                      0x42,  0,    Signed64,       Void,        Void)
INSTR (I64CONST,                      0x42,  0,    Void,           Void,        Void)
INSTR (F32CONST,                      0x43,  0,    Float32,        Void,        Void)
INSTR (F64CONST,                      0x44,  0,    Float64,        Void,        Void)
INSTR (I32EQZ,                        0x45,  0,    Void,           Void,        Void)
INSTR (I32EQ,                         0x46,  0,    Void,           Void,        Void)
INSTR (I32NE,                         0x47,  0,    Void,           Void,        Void)
INSTR (I32LT_S,                       0x48,  0,    Void,           Void,        Void)
INSTR (I32LT_U,                       0x49,  0,    Void,           Void,        Void)
INSTR (I32GT_S,                       0x4A,  0,    Void,           Void,        Void)
INSTR (I32GT_U,                       0x4B,  0,    Void,           Void,        Void)
INSTR (I32LE_S,                       0x4C,  0,    Void,           Void,        Void)
INSTR (I32LE_U,                       0x4D,  0,    Void,           Void,        Void)
INSTR (I32GE_S,                       0x4E,  0,    Void,           Void,        Void)
INSTR (I32GE_U,                       0x4F,  0,    Void,           Void,        Void)
INSTR (I64EQZ,                        0x50,  0,    Void,           Void,        Void)
INSTR (I64EQ,                         0x51,  0,    Void,           Void,        Void)
INSTR (I64NE,                         0x52,  0,    Void,           Void,        Void)
INSTR (I64LT_S,                       0x53,  0,    Void,           Void,        Void)
INSTR (I64LT_U,                       0x54,  0,    Void,           Void,        Void)
INSTR (I64GT_S,                       0x55,  0,    Void,           Void,        Void)
INSTR (I64GT_U,                       0x56,  0,    Void,           Void,        Void)
INSTR (I64LE_S,                       0x57,  0,    Void,           Void,        Void)
INSTR (I64LE_U,                       0x58,  0,    Void,           Void,        Void)
INSTR (I64GE_S,                       0x59,  0,    Void,           Void,        Void)
INSTR (I64GE_U,                       0x5A,  0,    Void,           Void,        Void)
INSTR (F32EQ,                         0x5B,  0,    Void,           Void,        Void)
INSTR (F32NE,                         0x5C,  0,    Void,           Void,        Void)
INSTR (F32LT,                         0x5D,  0,    Void,           Void,        Void)
INSTR (F32GT,                         0x5E,  0,    Void,           Void,        Void)
INSTR (F32LE,                         0x5F,  0,    Void,           Void,        Void)
INSTR (F32GE,                         0x60,  0,    Void,           Void,        Void)
INSTR (F64EQ,                         0x61,  0,    Void,           Void,        Void)
INSTR (F64NE,                         0x62,  0,    Void,           Void,        Void)
INSTR (F64LT,                         0x63,  0,    Void,           Void,        Void)
INSTR (F64GT,                         0x64,  0,    Void,           Void,        Void)
INSTR (F64LE,                         0x65,  0,    Void,           Void,        Void)
INSTR (F64GE,                         0x66,  0,    Void,           Void,        Void)
INSTR (I32CLZ,                        0x67,  0,    Void,           Void,        Void)
INSTR (I32CTZ,                        0x68,  0,    Void,           Void,        Void)
INSTR (I32POPCNT,                     0x69,  0,    Void,           Void,        Void)
INSTR (I32ADD,                        0x6A,  0,    Void,           Void,        Void)
INSTR (I32SUB,                        0x6B,  0,    Void,           Void,        Void)
INSTR (I32MUL,                        0x6C,  0,    Void,           Void,        Void)
INSTR (I32DIV_S,                      0x6D,  0,    Void,           Void,        Void)
INSTR (I32DIV_U,                      0x6E,  0,    Void,           Void,        Void)
INSTR (I32REM_S,                      0x6F,  0,    Void,           Void,        Void)
INSTR (I32REM_U,                      0x70,  0,    Void,           Void,        Void)
INSTR (I32AND,                        0x71,  0,    Void,           Void,        Void)
INSTR (I32OR,                         0x72,  0,    Void,           Void,        Void)
INSTR (I32XOR,                        0x73,  0,    Void,           Void,        Void)
INSTR (I32SHL,                        0x74,  0,    Void,           Void,        Void)
INSTR (I32SHR_S,                      0x75,  0,    Void,           Void,        Void)
INSTR (I32SHR_U,                      0x76,  0,    Void,           Void,        Void)
INSTR (I32ROTL,                       0x77,  0,    Void,           Void,        Void)
INSTR (I32ROTR,                       0x78,  0,    Void,           Void,        Void)
INSTR (I64CLZ,                        0x79,  0,    Void,           Void,        Void)
INSTR (I64CTZ,                        0x7A,  0,    Void,           Void,        Void)
INSTR (I64POPCNT,                     0x7B,  0,    Void,           Void,        Void)
INSTR (I64ADD,                        0x7C,  0,    Void,           Void,        Void)
INSTR (I64SUB,                        0x7D,  0,    Void,           Void,        Void)
INSTR (I64MUL,                        0x7E,  0,    Void,           Void,        Void)
INSTR (I64DIV_S,                      0x7F,  0,    Void,           Void,        Void)
INSTR (I64DIV_U,                      0x80,  0,    Void,           Void,        Void)
INSTR (I64REM_S,                      0x81,  0,    Void,           Void,        Void)
INSTR (I64REM_U,                      0x82,  0,    Void,           Void,        Void)
INSTR (I64AND,                        0x83,  0,    Void,           Void,        Void)
INSTR (I64OR,                         0x84,  0,    Void,           Void,        Void)
INSTR (I64XOR,                        0x85,  0,    Void,           Void,        Void)
INSTR (I64SHL,                        0x86,  0,    Void,           Void,        Void)
INSTR (I64SHR_S,                      0x87,  0,    Void,           Void,        Void)
INSTR (I64SHR_U,                      0x88,  0,    Void,           Void,        Void)
INSTR (I64ROTL,                       0x89,  0,    Void,           Void,        Void)
INSTR (I64ROTR,                       0x8A,  0,    Void,           Void,        Void)
INSTR (F32ABS,                        0x8B,  0,    Void,           Void,        Void)
INSTR (F32NEG,                        0x8C,  0,    Void,           Void,        Void)
INSTR (F32CEIL,                       0x8D,  0,    Void,           Void,        Void)
INSTR (F32FLOOR,                      0x8E,  0,    Void,           Void,        Void)
INSTR (F32TRUNC,                      0x8F,  0,    Void,           Void,        Void)
INSTR (F32NEAREST,                    0x90,  0,    Void,           Void,        Void)
INSTR (F32SQRT,                       0x91,  0,    Void,           Void,        Void)
INSTR (F32ADD,                        0x92,  0,    Void,           Void,        Void)
INSTR (F32SUB,                        0x93,  0,    Void,           Void,        Void)
INSTR (F32MUL,                        0x94,  0,    Void,           Void,        Void)
INSTR (F32DIV,                        0x95,  0,    Void,           Void,        Void)
INSTR (F32MIN,                        0x96,  0,    Void,           Void,        Void)
INSTR (F32MAX,                        0x97,  0,    Void,           Void,        Void)
INSTR (F32COPYSIGN,                   0x98,  0,    Void,           Void,        Void)
INSTR (F64ABS,                        0x99,  0,    Void,           Void,        Void)
INSTR (F64NEG,                        0x9A,  0,    Void,           Void,        Void)
INSTR (F64CEIL,                       0x9B,  0,    Void,           Void,        Void)
INSTR (F64FLOOR,                      0x9C,  0,    Void,           Void,        Void)
INSTR (F64TRUNC,                      0x9D,  0,    Void,           Void,        Void)
INSTR (F64NEAREST,                    0x9E,  0,    Void,           Void,        Void)
INSTR (F64SQRT,                       0x9F,  0,    Void,           Void,        Void)
INSTR (F64ADD,                        0xA0,  0,    Void,           Void,        Void)
INSTR (F64SUB,                        0xA1,  0,    Void,           Void,        Void)
INSTR (F64MUL,                        0xA2,  0,    Void,           Void,        Void)
INSTR (F64DIV,                        0xA3,  0,    Void,           Void,        Void)
INSTR (F64MIN,                        0xA4,  0,    Void,           Void,        Void)
INSTR (F64MAX,                        0xA5,  0,    Void,           Void,        Void)
INSTR (F64COPYSIGN,                   0xA6,  0,    Void,           Void,        Void)
INSTR (I32WRAP_I64,                   0xA7,  0,    Void,           Void,        Void)
INSTR (I32TRUNC_F32_S,                0xA8,  0,    Void,           Void,        Void)
INSTR (I32TRUNC_F32_U,                0xA9,  0,    Void,           Void,        Void)
INSTR (I32TRUNC_F64_S,                0xAA,  0,    Void,           Void,        Void)
INSTR (I32TRUNC_F64_U,                0xAB,  0,    Void,           Void,        Void)
INSTR (I64EXTEND_I32_S,               0xAC,  0,    Void,           Void,        Void)
INSTR (I64EXTEND_I32_U,               0xAD,  0,    Void,           Void,        Void)
INSTR (I64TRUNC_F32_S,                0xAE,  0,    Void,           Void,        Void)
INSTR (I64TRUNC_F32_U,                0xAF,  0,    Void,           Void,        Void)
INSTR (I64TRUNC_F64_S,                0xB0,  0,    Void,           Void,        Void)
INSTR (I64TRUNC_F64_U,                0xB1,  0,    Void,           Void,        Void)
INSTR (F32CONVERT_I32_S,              0xB2,  0,    Void,           Void,        Void)
INSTR (F32CONVERT_I32_U,              0xB3,  0,    Void,           Void,        Void)
INSTR (F32CONVERT_I64_S,              0xB4,  0,    Void,           Void,        Void)
INSTR (F32CONVERT_I64_U,              0xB5,  0,    Void,           Void,        Void)
INSTR (F32DEMOTE_F64,                 0xB6,  0,    Void,           Void,        Void)
INSTR (F64CONVERT_I32_S,              0xB7,  0,    Void,           Void,        Void)
INSTR (F64CONVERT_I32_U,              0xB8,  0,    Void,           Void,        Void)
INSTR (F64CONVERT_I64_S,              0xB9,  0,    Void,           Void,        Void)
INSTR (F64CONVERT_I64_U,              0xBA,  0,    Void,           Void,        Void)
INSTR (F64PROMOTE_F32,                0xBB,  0,    Void,           Void,        Void)
INSTR (I32REINTERPRET_F32,            0xBC,  0,    Void,           Void,        Void)
INSTR (I64REINTERPRET_F64,            0xBD,  0,    Void,           Void,        Void)
INSTR (F32REINTERPRET_I32,            0xBE,  0,    Void,           Void,        Void)
INSTR (F64REINTERPRET_I64,            0xBF,  0,    Void,           Void,        Void)
INSTR (I32EXTEND8_S,                  0xC0,  0,    Void,           Void,        Void)
INSTR (I32EXTEND16_S,                 0xC1,  0,    Void,           Void,        Void)
INSTR (I64EXTEND8_S,                  0xC2,  0,    Void,           Void,        Void)
INSTR (I64EXTEND16_S,                 0xC3,  0,    Void,           Void,        Void)
INSTR (I64EXTEND32_S,                 0xC4,  0,    Void,           Void,        Void)
INSTR (I32TRUNC_SAT_F32_S,            0xFC,  0,    Void,           Void,        Void)
INSTR (I32TRUNC_SAT_F32_U,            0xFC,  1,    Void,           Void,        Void)
INSTR (I32TRUNC_SAT_F64_S,            0xFC,  2,    Void,           Void,        Void)
INSTR (I32TRUNC_SAT_F64_U,            0xFC,  3,    Void,           Void,        Void)
INSTR (I64TRUNC_SAT_F32_S,            0xFC,  4,    Void,           Void,        Void)
INSTR (I64TRUNC_SAT_F32_U,            0xFC,  5,    Void,           Void,        Void)
INSTR (I64TRUNC_SAT_F64_S,            0xFC,  6,    Void,           Void,        Void)
INSTR (I64TRUNC_SAT_F64_U,            0xFC,  7,    Void,           Void,        Void)

// vector instructions

INSTR (V128LOAD,                      0xFD,  0,    Alignment,      Offset,      Void)
INSTR (V128LOAD8X8_S,                 0xFD,  1,    Alignment,      Offset,      Void)
INSTR (V128LOAD8X8_U,                 0xFD,  2,    Alignment,      Offset,      Void)
INSTR (V128LOAD16X4_S,                0xFD,  3,    Alignment,      Offset,      Void)
INSTR (V128LOAD16X4_U,                0xFD,  4,    Alignment,      Offset,      Void)
INSTR (V128LOAD32X2_S,                0xFD,  5,    Alignment,      Offset,      Void)
INSTR (V128LOAD32X2_U,                0xFD,  6,    Alignment,      Offset,      Void)
INSTR (V128LOAD8_SPLAT,               0xFD,  7,    Alignment,      Offset,      Void)
INSTR (V128LOAD16_SPLAT,              0xFD,  8,    Alignment,      Offset,      Void)
INSTR (V128LOAD32_SPLAT,              0xFD,  9,    Alignment,      Offset,      Void)
INSTR (V128LOAD64_SPLAT,              0xFD,  10,   Alignment,      Offset,      Void)
INSTR (V128LOAD32_ZERO,               0xFD,  92,   Alignment,      Offset,      Void)
INSTR (V128LOAD64_ZERO,               0xFD,  93,   Alignment,      Offset,      Void)
INSTR (V128STORE,                     0xFD,  11,   Alignment,      Offset,      Void)
INSTR (V128LOAD8_LANE,                0xFD,  84,   Alignment,      Offset,      LaneIndex)
INSTR (V128LOAD16_LANE,               0xFD,  85,   Alignment,      Offset,      LaneIndex)
INSTR (V128LOAD32_LANE,               0xFD,  86,   Alignment,      Offset,      LaneIndex)
INSTR (V128LOAD64_LANE,               0xFD,  87,   Alignment,      Offset,      LaneIndex)
INSTR (V128STORE8_LANE,               0xFD,  88,   Alignment,      Offset,      LaneIndex)
INSTR (V128STORE16_LANE,              0xFD,  89,   Alignment,      Offset,      LaneIndex)
INSTR (V128STORE32_LANE,              0xFD,  90,   Alignment,      Offset,      LaneIndex)
INSTR (V128STORE64_LANE,              0xFD,  91,   Alignment,      Offset,      LaneIndex)
INSTR (V128CONST,                     0xFD,  12,   Void,           Void,        Void)
INSTR (I8X16SHUFFLE,                  0xFD,  13,   Void,           Void,        Void)
INSTR (I8X16EXTRACT_LANE_S,           0xFD,  21,   LaneIndex,      Void,        Void)
INSTR (I8X16EXTRACT_LANE_U,           0xFD,  22,   LaneIndex,      Void,        Void)
INSTR (I8X16REPLACE_LANE,             0xFD,  23,   LaneIndex,      Void,        Void)
INSTR (I16X8EXTRACT_LANE_S,           0xFD,  24,   LaneIndex,      Void,        Void)
INSTR (I16X8EXTRACT_LANE_U,           0xFD,  25,   LaneIndex,      Void,        Void)
INSTR (I16X8REPLACE_LANE,             0xFD,  26,   LaneIndex,      Void,        Void)
INSTR (I32X4EXTRACT_LANE,             0xFD,  27,   LaneIndex,      Void,        Void)
INSTR (I32X4REPLACE_LANE,             0xFD,  28,   LaneIndex,      Void,        Void)
INSTR (I64X2EXTRACT_LANE,             0xFD,  29,   LaneIndex,      Void,        Void)
INSTR (I64X2REPLACE_LANE,             0xFD,  30,   LaneIndex,      Void,        Void)
INSTR (F32X4EXTRACT_LANE,             0xFD,  31,   LaneIndex,      Void,        Void)
INSTR (F32X4REPLACE_LANE,             0xFD,  32,   LaneIndex,      Void,        Void)
INSTR (F64X2EXTRACT_LANE,             0xFD,  33,   LaneIndex,      Void,        Void)
INSTR (F64X2REPLACE_LANE,             0xFD,  34,   LaneIndex,      Void,        Void)
INSTR (I8X16SWIZZLE,                  0xFD,  14,   Void,           Void,        Void)
INSTR (I8X16SPLAT,                    0xFD,  15,   Void,           Void,        Void)
INSTR (I16X8SPLAT,                    0xFD,  16,   Void,           Void,        Void)
INSTR (I32X4SPLAT,                    0xFD,  17,   Void,           Void,        Void)
INSTR (I64X2SPLAT,                    0xFD,  18,   Void,           Void,        Void)
INSTR (F32X4SPLAT,                    0xFD,  19,   Void,           Void,        Void)
INSTR (F64X2SPLAT,                    0xFD,  20,   Void,           Void,        Void)
INSTR (I8X16EQ,                       0xFD,  35,   Void,           Void,        Void)
INSTR (I8X16NE,                       0xFD,  36,   Void,           Void,        Void)
INSTR (I8X16LT_S,                     0xFD,  37,   Void,           Void,        Void)
INSTR (I8X16LT_U,                     0xFD,  38,   Void,           Void,        Void)
INSTR (I8X16GT_S,                     0xFD,  39,   Void,           Void,        Void)
INSTR (I8X16GT_U,                     0xFD,  40,   Void,           Void,        Void)
INSTR (I8X16LE_S,                     0xFD,  41,   Void,           Void,        Void)
INSTR (I8X16LE_U,                     0xFD,  42,   Void,           Void,        Void)
INSTR (I8X16GE_S,                     0xFD,  43,   Void,           Void,        Void)
INSTR (I8X16GE_U,                     0xFD,  44,   Void,           Void,        Void)
INSTR (I16X8EQ,                       0xFD,  45,   Void,           Void,        Void)
INSTR (I16X8NE,                       0xFD,  46,   Void,           Void,        Void)
INSTR (I16X8LT_S,                     0xFD,  47,   Void,           Void,        Void)
INSTR (I16X8LT_U,                     0xFD,  48,   Void,           Void,        Void)
INSTR (I16X8GT_S,                     0xFD,  49,   Void,           Void,        Void)
INSTR (I16X8GT_U,                     0xFD,  50,   Void,           Void,        Void)
INSTR (I16X8LE_S,                     0xFD,  51,   Void,           Void,        Void)
INSTR (I16X8LE_U,                     0xFD,  52,   Void,           Void,        Void)
INSTR (I16X8GE_S,                     0xFD,  53,   Void,           Void,        Void)
INSTR (I16X8GE_U,                     0xFD,  54,   Void,           Void,        Void)
INSTR (I32X4EQ,                       0xFD,  55,   Void,           Void,        Void)
INSTR (I32X4NE,                       0xFD,  56,   Void,           Void,        Void)
INSTR (I32X4LT_S,                     0xFD,  57,   Void,           Void,        Void)
INSTR (I32X4LT_U,                     0xFD,  58,   Void,           Void,        Void)
INSTR (I32X4GT_S,                     0xFD,  59,   Void,           Void,        Void)
INSTR (I32X4GT_U,                     0xFD,  60,   Void,           Void,        Void)
INSTR (I32X4LE_S,                     0xFD,  61,   Void,           Void,        Void)
INSTR (I32X4LE_U,                     0xFD,  62,   Void,           Void,        Void)
INSTR (I32X4GE_S,                     0xFD,  63,   Void,           Void,        Void)
INSTR (I32X4GE_U,                     0xFD,  64,   Void,           Void,        Void)
INSTR (I64X2EQ,                       0xFD,  214,  Void,           Void,        Void)
INSTR (I64X2NE,                       0xFD,  215,  Void,           Void,        Void)
INSTR (I64X2LT_S,                     0xFD,  216,  Void,           Void,        Void)
INSTR (I64X2GT_S,                     0xFD,  217,  Void,           Void,        Void)
INSTR (I64X2LE_S,                     0xFD,  218,  Void,           Void,        Void)
INSTR (I64X2GE_S,                     0xFD,  219,  Void,           Void,        Void)
INSTR (F32X4EQ,                       0xFD,  65,   Void,           Void,        Void)
INSTR (F32X4NE,                       0xFD,  66,   Void,           Void,        Void)
INSTR (F32X4LT,                       0xFD,  67,   Void,           Void,        Void)
INSTR (F32X4GT,                       0xFD,  68,   Void,           Void,        Void)
INSTR (F32X4LE,                       0xFD,  69,   Void,           Void,        Void)
INSTR (F32X4GE,                       0xFD,  70,   Void,           Void,        Void)
INSTR (F64X2EQ,                       0xFD,  71,   Void,           Void,        Void)
INSTR (F64X2NE,                       0xFD,  72,   Void,           Void,        Void)
INSTR (F64X2LT,                       0xFD,  73,   Void,           Void,        Void)
INSTR (F64X2GT,                       0xFD,  74,   Void,           Void,        Void)
INSTR (F64X2LE,                       0xFD,  75,   Void,           Void,        Void)
INSTR (F64X2GE,                       0xFD,  76,   Void,           Void,        Void)
INSTR (V128NOT,                       0xFD,  77,   Void,           Void,        Void)
INSTR (V128AND,                       0xFD,  78,   Void,           Void,        Void)
INSTR (V128ANDNOT,                    0xFD,  79,   Void,           Void,        Void)
INSTR (V128OR,                        0xFD,  80,   Void,           Void,        Void)
INSTR (V128XOR,                       0xFD,  81,   Void,           Void,        Void)
INSTR (V128BITSELECT,                 0xFD,  82,   Void,           Void,        Void)
INSTR (V128ANY_TRUE,                  0xFD,  83,   Void,           Void,        Void)
INSTR (I8X16ABS,                      0xFD,  96,   Void,           Void,        Void)
INSTR (I8X16NEG,                      0xFD,  97,   Void,           Void,        Void)
INSTR (I8X16POPCNT,                   0xFD,  98,   Void,           Void,        Void)
INSTR (I8X16ALL_TRUE,                 0xFD,  99,   Void,           Void,        Void)
INSTR (I8X16BITMASK,                  0xFD,  100,  Void,           Void,        Void)
INSTR (I8X16NARROW_I16X8_S,           0xFD,  101,  Void,           Void,        Void)
INSTR (I8X16NARROW_I16X8_U,           0xFD,  102,  Void,           Void,        Void)
INSTR (I8X16SHL,                      0xFD,  107,  Void,           Void,        Void)
INSTR (I8X16SHR_S,                    0xFD,  108,  Void,           Void,        Void)
INSTR (I8X16SHR_U,                    0xFD,  109,  Void,           Void,        Void)
INSTR (I8X16ADD,                      0xFD,  110,  Void,           Void,        Void)
INSTR (I8X16ADD_SAT_S,                0xFD,  111,  Void,           Void,        Void)
INSTR (I8X16ADD_SAT_U,                0xFD,  112,  Void,           Void,        Void)
INSTR (I8X16SUB,                      0xFD,  113,  Void,           Void,        Void)
INSTR (I8X16SUB_SAT_S,                0xFD,  114,  Void,           Void,        Void)
INSTR (I8X16SUB_SAT_U,                0xFD,  115,  Void,           Void,        Void)
INSTR (I8X16MIN_S,                    0xFD,  118,  Void,           Void,        Void)
INSTR (I8X16MIN_U,                    0xFD,  119,  Void,           Void,        Void)
INSTR (I8X16MAX_S,                    0xFD,  120,  Void,           Void,        Void)
INSTR (I8X16MAX_U,                    0xFD,  121,  Void,           Void,        Void)
INSTR (I8X16AVGR_U,                   0xFD,  123,  Void,           Void,        Void)
INSTR (I16X8EXTADD_PAIRWISE_I8X16_S,  0xFD,  124,  Void,           Void,        Void)
INSTR (I16X8EXTADD_PAIRWISE_I8X16_U,  0xFD,  125,  Void,           Void,        Void)
INSTR (I16X8ABS,                      0xFD,  128,  Void,           Void,        Void)
INSTR (I16X8NEG,                      0xFD,  129,  Void,           Void,        Void)
INSTR (I16X8Q15MULR_SAT_S,            0xFD,  130,  Void,           Void,        Void)
INSTR (I16X8ALL_TRUE,                 0xFD,  131,  Void,           Void,        Void)
INSTR (I16X8BITMASK,                  0xFD,  132,  Void,           Void,        Void)
INSTR (I16X8NARROW_I32X4_S,           0xFD,  133,  Void,           Void,        Void)
INSTR (I16X8NARROW_I32X4_U,           0xFD,  134,  Void,           Void,        Void)
INSTR (I16X8EXTEND_LOW_I8X16_S,       0xFD,  135,  Void,           Void,        Void)
INSTR (I16X8EXTEND_HIGH_I8X16_S,      0xFD,  136,  Void,           Void,        Void)
INSTR (I16X8EXTEND_LOW_I8X16_U,       0xFD,  137,  Void,           Void,        Void)
INSTR (I16X8EXTEND_HIGH_I8X16_U,      0xFD,  138,  Void,           Void,        Void)
INSTR (I16X8SHL,                      0xFD,  139,  Void,           Void,        Void)
INSTR (I16X8SHR_S,                    0xFD,  140,  Void,           Void,        Void)
INSTR (I16X8SHR_U,                    0xFD,  141,  Void,           Void,        Void)
INSTR (I16X8ADD,                      0xFD,  142,  Void,           Void,        Void)
INSTR (I16X8ADD_SAT_S,                0xFD,  143,  Void,           Void,        Void)
INSTR (I16X8ADD_SAT_U,                0xFD,  144,  Void,           Void,        Void)
INSTR (I16X8SUB,                      0xFD,  145,  Void,           Void,        Void)
INSTR (I16X8SUB_SAT_S,                0xFD,  146,  Void,           Void,        Void)
INSTR (I16X8SUB_SAT_U,                0xFD,  147,  Void,           Void,        Void)
INSTR (I16X8MUL,                      0xFD,  149,  Void,           Void,        Void)
INSTR (I16X8MIN_S,                    0xFD,  150,  Void,           Void,        Void)
INSTR (I16X8MIN_U,                    0xFD,  151,  Void,           Void,        Void)
INSTR (I16X8MAX_S,                    0xFD,  152,  Void,           Void,        Void)
INSTR (I16X8MAX_U,                    0xFD,  153,  Void,           Void,        Void)
INSTR (I16X8AVGR_U,                   0xFD,  155,  Void,           Void,        Void)
INSTR (I16X8EXTMUL_LOW_I8X16_S,       0xFD,  156,  Void,           Void,        Void)
INSTR (I16X8EXTMUL_HIGH_I8X16_S,      0xFD,  157,  Void,           Void,        Void)
INSTR (I16X8EXTMUL_LOW_I8X16_U,       0xFD,  158,  Void,           Void,        Void)
INSTR (I16X8EXTMUL_HIGH_I8X16_U,      0xFD,  159,  Void,           Void,        Void)
INSTR (I32X4EXTADD_PAIRWISE_I16X8_S,  0xFD,  126,  Void,           Void,        Void)
INSTR (I32X4EXTADD_PAIRWISE_I16X8_U,  0xFD,  127,  Void,           Void,        Void)
INSTR (I32X4ABS,                      0xFD,  160,  Void,           Void,        Void)
INSTR (I32X4NEG,                      0xFD,  161,  Void,           Void,        Void)
INSTR (I32X4ALL_TRUE,                 0xFD,  163,  Void,           Void,        Void)
INSTR (I32X4BITMASK,                  0xFD,  164,  Void,           Void,        Void)
INSTR (I32X4EXTEND_LOW_I16X8_S,       0xFD,  167,  Void,           Void,        Void)
INSTR (I32X4EXTEND_HIGH_I16X8_S,      0xFD,  168,  Void,           Void,        Void)
INSTR (I32X4EXTEND_LOW_I16X8_U,       0xFD,  169,  Void,           Void,        Void)
INSTR (I32X4EXTEND_HIGH_I16X8_U,      0xFD,  170,  Void,           Void,        Void)
INSTR (I32X4SHL,                      0xFD,  171,  Void,           Void,        Void)
INSTR (I32X4SHR_S,                    0xFD,  172,  Void,           Void,        Void)
INSTR (I32X4SHR_U,                    0xFD,  173,  Void,           Void,        Void)
INSTR (I32X4ADD,                      0xFD,  174,  Void,           Void,        Void)
INSTR (I32X4SUB,                      0xFD,  177,  Void,           Void,        Void)
INSTR (I32X4MUL,                      0xFD,  181,  Void,           Void,        Void)
INSTR (I32X4MIN_S,                    0xFD,  182,  Void,           Void,        Void)
INSTR (I32X4MIN_U,                    0xFD,  183,  Void,           Void,        Void)
INSTR (I32X4MAX_S,                    0xFD,  184,  Void,           Void,        Void)
INSTR (I32X4MAX_U,                    0xFD,  185,  Void,           Void,        Void)
INSTR (I32X4DOT_I16X8_S,              0xFD,  186,  Void,           Void,        Void)
INSTR (I32X4EXTMUL_LOW_I16X8_S,       0xFD,  188,  Void,           Void,        Void)
INSTR (I32X4EXTMUL_HIGH_I16X8_S,      0xFD,  189,  Void,           Void,        Void)
INSTR (I32X4EXTMUL_LOW_I16X8_U,       0xFD,  190,  Void,           Void,        Void)
INSTR (I32X4EXTMUL_HIGH_I16X8_U,      0xFD,  191,  Void,           Void,        Void)
INSTR (I64X2ABS,                      0xFD,  192,  Void,           Void,        Void)
INSTR (I64X2NEG,                      0xFD,  193,  Void,           Void,        Void)
INSTR (I64X2ALL_TRUE,                 0xFD,  195,  Void,           Void,        Void)
INSTR (I64X2BITMASK,                  0xFD,  196,  Void,           Void,        Void)
INSTR (I64X2EXTEND_LOW_I32X4_S,       0xFD,  199,  Void,           Void,        Void)
INSTR (I64X2EXTEND_HIGH_I32X4_S,      0xFD,  200,  Void,           Void,        Void)
INSTR (I64X2EXTEND_LOW_I32X4_U,       0xFD,  201,  Void,           Void,        Void)
INSTR (I64X2EXTEND_HIGH_I32X4_U,      0xFD,  202,  Void,           Void,        Void)
INSTR (I64X2SHL,                      0xFD,  203,  Void,           Void,        Void)
INSTR (I64X2SHR_S,                    0xFD,  204,  Void,           Void,        Void)
INSTR (I64X2SHR_U,                    0xFD,  205,  Void,           Void,        Void)
INSTR (I64X2ADD,                      0xFD,  206,  Void,           Void,        Void)
INSTR (I64X2SUB,                      0xFD,  209,  Void,           Void,        Void)
INSTR (I64X2MUL,                      0xFD,  213,  Void,           Void,        Void)
INSTR (I64X2EXTMUL_LOW_I32X4_S,       0xFD,  220,  Void,           Void,        Void)
INSTR (I64X2EXTMUL_HIGH_I32X4_S,      0xFD,  221,  Void,           Void,        Void)
INSTR (I64X2EXTMUL_LOW_I32X4_U,       0xFD,  222,  Void,           Void,        Void)
INSTR (I64X2EXTMUL_HIGH_I32X4_U,      0xFD,  223,  Void,           Void,        Void)
INSTR (F32X4CEIL,                     0xFD,  103,  Void,           Void,        Void)
INSTR (F32X4FLOOR,                    0xFD,  104,  Void,           Void,        Void)
INSTR (F32X4TRUNC,                    0xFD,  105,  Void,           Void,        Void)
INSTR (F32X4NEAREST,                  0xFD,  106,  Void,           Void,        Void)
INSTR (F32X4ABS,                      0xFD,  224,  Void,           Void,        Void)
INSTR (F32X4NEG,                      0xFD,  225,  Void,           Void,        Void)
INSTR (F32X4SQRT,                     0xFD,  227,  Void,           Void,        Void)
INSTR (F32X4ADD,                      0xFD,  228,  Void,           Void,        Void)
INSTR (F32X4SUB,                      0xFD,  229,  Void,           Void,        Void)
INSTR (F32X4MUL,                      0xFD,  230,  Void,           Void,        Void)
INSTR (F32X4DIV,                      0xFD,  231,  Void,           Void,        Void)
INSTR (F32X4MIN,                      0xFD,  232,  Void,           Void,        Void)
INSTR (F32X4MAX,                      0xFD,  233,  Void,           Void,        Void)
INSTR (F32X4PMIN,                     0xFD,  234,  Void,           Void,        Void)
INSTR (F32X4PMAX,                     0xFD,  235,  Void,           Void,        Void)
INSTR (F64X2CEIL,                     0xFD,  116,  Void,           Void,        Void)
INSTR (F64X2FLOOR,                    0xFD,  117,  Void,           Void,        Void)
INSTR (F64X2TRUNC,                    0xFD,  122,  Void,           Void,        Void)
INSTR (F64X2NEAREST,                  0xFD,  148,  Void,           Void,        Void)
INSTR (F64X2ABS,                      0xFD,  236,  Void,           Void,        Void)
INSTR (F64X2NEG,                      0xFD,  237,  Void,           Void,        Void)
INSTR (F64X2SQRT,                     0xFD,  239,  Void,           Void,        Void)
INSTR (F64X2ADD,                      0xFD,  240,  Void,           Void,        Void)
INSTR (F64X2SUB,                      0xFD,  241,  Void,           Void,        Void)
INSTR (F64X2MUL,                      0xFD,  242,  Void,           Void,        Void)
INSTR (F64X2DIV,                      0xFD,  243,  Void,           Void,        Void)
INSTR (F64X2MIN,                      0xFD,  244,  Void,           Void,        Void)
INSTR (F64X2MAX,                      0xFD,  245,  Void,           Void,        Void)
INSTR (F64X2PMIN,                     0xFD,  246,  Void,           Void,        Void)
INSTR (F64X2PMAX,                     0xFD,  247,  Void,           Void,        Void)
INSTR (I32X4TRUNC_SAT_F32X4_S,        0xFD,  248,  Void,           Void,        Void)
INSTR (I32X4TRUNC_SAT_F32X4_U,        0xFD,  249,  Void,           Void,        Void)
INSTR (F32X4CONVERT_I32X4_S,          0xFD,  250,  Void,           Void,        Void)
INSTR (F32X4CONVERT_I32X4_U,          0xFD,  251,  Void,           Void,        Void)
INSTR (I32X4TRUNC_SAT_F64X2_S_ZERO,   0xFD,  252,  Void,           Void,        Void)
INSTR (I32X4TRUNC_SAT_F64X2_U_ZERO,   0xFD,  253,  Void,           Void,        Void)
INSTR (F64X2CONVERT_LOW_I32X4_S,      0xFD,  254,  Void,           Void,        Void)
INSTR (F64X2CONVERT_LOW_I32X4_U,      0xFD,  255,  Void,           Void,        Void)
INSTR (F32X4DEMOTE_F64X2_ZERO,        0xFD,  94,   Void,           Void,        Void)
INSTR (F64X2PROMOTE_LOW_F32X4,        0xFD,  95,   Void,           Void,        Void)

// pseudo instructions

INSTR (I32,                           0xFE,  0,    Unsigned32,     Void,        Void)
INSTR (LABEL,                         0xFE,  0,    LabelIndex,     Void,        Void)
INSTR (LANE,                          0xFE,  0,    LaneIndex,      Void,        Void)
INSTR (S32,                           0xFE,  0,    Signed32,       Void,        Void)
INSTR (U32,                           0xFE,  0,    Unsigned32,     Void,        Void)
INSTR (VALTYPE,                       0xFE,  0,    ValueType,      Void,        Void)
INSTR (VEC,                           0xFE,  0,    Unsigned32,     Void,        Void)

// operand types

TYPE (Alignment)
TYPE (BlockType)
TYPE (DataIndex)
TYPE (ElementIndex)
TYPE (EmptyType)
TYPE (Float32)
TYPE (Float64)
TYPE (FunctionIndex)
TYPE (GlobalIndex)
TYPE (LabelIndex)
TYPE (LaneIndex)
TYPE (LocalIndex)
TYPE (NumberType)
TYPE (Offset)
TYPE (ReferenceType)
TYPE (Signed32)
TYPE (Signed64)
TYPE (TableIndex)
TYPE (TypeIndex)
TYPE (Unsigned32)
TYPE (Unsigned64)
TYPE (ValueType)
TYPE (VectorType)
TYPE (ZeroIndex)

// type codes

TYPECODE (externref,  ExternRef,  0x6F)
TYPECODE (f32,        F32,        0x7D)
TYPECODE (f64,        F64,        0x7C)
TYPECODE (funcref,    FuncRef,    0x70)
TYPECODE (i32,        I32,        0x7F)
TYPECODE (i64,        I64,        0x7E)
TYPECODE (v128,       V128,       0x7B)

#undef INSTR
#undef MNEM
#undef TYPE
#undef TYPECODE

% Xtensa architecture documentation
% Copyright (C) Florian Negele

% This file is part of the Eigen Compiler Suite.

% Permission is granted to copy, distribute and/or modify this document
% under the terms of the GNU Free Documentation License, Version 1.3
% or any later version published by the Free Software Foundation.

% You should have received a copy of the GNU Free Documentation License
% along with the ECS.  If not, see <https://www.gnu.org/licenses/>.

% Generic documentation utilities
% Copyright (C) Florian Negele

% This file is part of the Eigen Compiler Suite.

% Permission is granted to copy, distribute and/or modify this document
% under the terms of the GNU Free Documentation License, Version 1.3
% or any later version published by the Free Software Foundation.

% You should have received a copy of the GNU Free Documentation License
% along with the ECS.  If not, see <https://www.gnu.org/licenses/>.

\providecommand{\cpp}{C\texttt{++}}
\providecommand{\opt}{_\mathit{opt}}
\providecommand{\tool}[1]{\texttt{#1}}
\providecommand{\version}{Version 0.0.40}
\providecommand{\resource}[1]{*++\txt{#1}}
\providecommand{\ecs}{Eigen Compiler Suite}
\providecommand{\changed}[1]{\underline{#1}}
\providecommand{\toolbox}[1]{\converter{#1}}
\providecommand{\file}{}\renewcommand{\file}[1]{\texttt{#1}}
\providecommand{\alignright}{\hfill\linebreak[0]\hspace*{\fill}}
\providecommand{\converter}[1]{*++[F][F*:white][F,:gray]\txt{#1}}
\providecommand{\documentation}{\ifbook chapter\else document\fi}
\providecommand{\Documentation}{\ifbook Chapter\else Document\fi}
\providecommand{\variable}[1]{\resource{\texttt{\small#1}\\variable}}
\providecommand{\documentationref}[2]{\ifbook\ref{#1}\else``\href{#1}{#2}''~\cite{#1}\fi}
\providecommand{\objfile}[1]{\texttt{#1}\index[runtime]{#1 object file@\texttt{#1} object file}}
\providecommand{\libfile}[1]{\texttt{#1}\index[runtime]{#1 library file@\texttt{#1} library file}}
\providecommand{\epigraph}[2]{\ifbook\begin{quote}\flushright\textit{#1}\par--- #2\end{quote}\fi}
\providecommand{\environmentvariable}[1]{\texttt{#1}\index{Environment variables!#1@\texttt{#1}}}
\providecommand{\environment}[1]{\texttt{#1}\index[environment]{#1 environment@\texttt{#1} environment}}
\providecommand{\toolsection}{}\renewcommand{\toolsection}[1]{\subsection{#1}\label{\prefix:#1}\tool{#1}}
\providecommand{\instruction}{}\renewcommand{\instruction}[2]{\noindent\qquad\pdftooltip{\texttt{#1}}{#2}\refstepcounter{instruction}\par}
\providecommand{\flowgraph}{}\renewcommand{\flowgraph}[1]{\par\sffamily\begin{displaymath}\xymatrix@=4ex{#1}\end{displaymath}\normalfont\par}
\providecommand{\instructionset}{}\renewcommand{\instructionset}[4]{\setcounter{instruction}{0}\begin{multicols}{\ifbook#3\else#4\fi}[{\captionof{table}[#2]{#2 (\ref*{#1:instructions}~instructions)}\label{tab:#1set}\vspace{-2ex}}]\footnotesize\raggedcolumns\input{#1.set}\label{#1:instructions}\end{multicols}}

\providecommand{\gpl}{GNU General Public License}
\providecommand{\rse}{ECS Runtime Support Exception}
\providecommand{\fdl}{\href{https://www.gnu.org/licenses/fdl.html}{GNU Free Documentation License}}

\providecommand{\docbegin}{}
\providecommand{\docend}{}
\providecommand{\doclabel}[1]{\hypertarget{#1}}
\providecommand{\doclink}[2]{\hyperlink{#1}{#2}}
\providecommand{\docsection}[3]{\hypertarget{#1}{\subsection{#2}}\label{sec:#1}\index[library]{#2@#3}}
\providecommand{\docsectionstar}[1]{}
\providecommand{\docsubbegin}{\begin{description}}
\providecommand{\docsubend}{\end{description}}
\providecommand{\docsubsection}[3]{\item[\hypertarget{#1}{#2}]\index[library]{#2@#3}}
\providecommand{\docsubsectionstar}[1]{\smallskip}
\providecommand{\docsubsubsection}[3]{\docsubsection{#1}{#2}{#3}}
\providecommand{\docsubsubsectionstar}[1]{}
\providecommand{\docsubsubsubsection}[3]{}
\providecommand{\docsubsubsubsectionstar}[1]{}
\providecommand{\doctable}{}

\providecommand{\debuggingtool}{}\renewcommand{\debuggingtool}{This tool is provided for debugging purposes.
It allows exposing and modifying an internal data structure that is usually not accessible.
}

\providecommand{\interface}{All tools accept command-line arguments which are taken as names of plain text files containing the source code.
If no arguments are provided, the standard input stream is used instead.
Output files are generated in the current working directory and have the same name as the input file being processed whereas the filename extension gets replaced by an appropriate suffix.
\seeinterface
}

\providecommand{\license}{\noindent Copyright \copyright{} Florian Negele\par\medskip\noindent
Permission is granted to copy, distribute and/or modify this document under the terms of the
\fdl{}, Version 1.3 or any later version published by the \href{https://fsf.org/}{Free Software Foundation}.
}

\providecommand{\ecslogosurface}{
\fill[darkgray] (0,0,0) -- (0,0,3) -- (0,3,3) -- (0,3,1) -- (0,4,1) -- (0,4,3) -- (0,5,3) -- (0,5,0) -- (0,2,0) -- (0,2,2) -- (0,1,2) -- (0,1,0) -- cycle;
\fill[gray] (0,5,0) -- (0,5,3) -- (1,5,3) -- (1,5,1) -- (2,5,1) -- (2,5,3) -- (3,5,3) -- (3,5,0) -- cycle;
\fill[lightgray] (0,0,0) -- (0,1,0) -- (2,1,0) -- (2,4,0) -- (1,4,0) -- (1,3,0) -- (2,3,0) -- (2,2,0) -- (0,2,0) -- (0,5,0) -- (3,5,0) -- (3,0,0) -- cycle;
\begin{scope}[line width=0.5]
\begin{scope}[gray]
\draw (0,0,0) -- (0,1,0);
\draw (2,1,0) -- (2,2,0);
\draw (0,1,2) -- (0,2,2);
\draw (0,2,0) -- (0,5,0);
\draw (2,3,0) -- (2,4,0);
\end{scope}
\begin{scope}[lightgray]
\draw (0,1,0) -- (0,1,2);
\draw (0,3,1) -- (0,3,3);
\draw (0,5,0) -- (0,5,3);
\draw (2,5,1) -- (2,5,3);
\end{scope}
\begin{scope}[white]
\draw (0,1,0) -- (2,1,0);
\draw (1,3,0) -- (2,3,0);
\draw (0,5,0) -- (3,5,0);
\end{scope}
\end{scope}
}

\providecommand{\ecslogo}[1]{
\begin{tikzpicture}[scale={(#1)/((sin(45)+cos(45))*3cm)},x={({-cos(45)*1cm},{sin(45)*sin(30)*1cm})},y={({0cm},{(cos(30)*1cm})},z={({sin(45)*1cm},{cos(45)*sin(30)*1cm})}]
\begin{scope}[darkgray,line width=1]
\draw (0,0,0) -- (0,0,3) -- (0,3,3) -- (2,3,3) -- (2,5,3) -- (3,5,3) -- (3,5,0) -- (3,0,0) -- cycle;
\draw (0,3,1) -- (0,4,1) -- (0,4,3) -- (0,5,3) -- (1,5,3) -- (1,5,1) -- (2,5,1);
\draw (1,3,0) -- (1,4,0) -- (2,4,0);
\end{scope}
\fill[darkgray] (2,0,0) -- (2,0,3) -- (2,5,3) -- (2,5,1) -- (2,4,1) -- (2,4,0) -- cycle;
\fill[lightgray] (2,0,2) -- (0,0,2) -- (0,2,2) -- (2,2,2) -- cycle;
\fill[gray] (0,1,0) -- (2,1,0) -- (2,1,2) -- (0,1,2) -- cycle;
\fill[gray] (0,3,1) -- (0,3,3) -- (2,3,3) -- (2,3,0) -- (1,3,0) -- (1,3,1) -- cycle;
\ecslogosurface
\end{tikzpicture}
}

\providecommand{\shadowedecslogo}[3]{
\begin{tikzpicture}[scale={(#1)/((sin(#2)+cos(#2))*3cm)},x={({-cos(#2)*1cm},{sin(#2)*sin(#3)*1cm})},y={({0cm},{(cos(#3)*1cm})},z={({sin(#2)*1cm},{cos(#2)*sin(#3)*1cm})}]
\shade[top color=lightgray!50!white,bottom color=white,middle color=lightgray!50!white] (0,0,0) -- (3,0,0) -- (3,{-0.5-3*sin(#2)*sin(#3)/cos(#3)},0) -- (0,-0.5,0) -- cycle;
\shade[top color=darkgray!50!gray,bottom color=white,middle color=darkgray!50!white] (0,0,0) -- (0,0,3) -- (0,{-0.5-3*cos(#2)*sin(#3)/cos(#3)},3) -- (0,-0.5,0) -- cycle;
\begin{scope}[y={({(cos(#2)+sin(#2))*0.5cm},{(cos(#2)*sin(#3)-sin(#2)*sin(#3))*0.5cm})}]
\useasboundingbox (3,0,0) -- (0,0,0) -- (0,0,3);
\shade[left color=darkgray!80!black,right color=lightgray,middle color=gray] (0,0,0) -- (0,1,0) -- (0,1,0.5) -- (0,2,0) -- (0,5,0) -- (0,5,3) -- (1,5,3) -- (1,4,3) -- (1,4,2.5) -- (1,3,3) -- (2,5,3) -- (3,5,3) -- (3,0,3) -- cycle;
\clip (0,0,0) -- (0,0,3) -- ({-3*sin(#2)/cos(#2)},0,0) -- cycle;
\shade[left color=darkgray,right color=lightgray!50!gray] (0,0,0) -- (0,1,0) -- (0,1,0.5) -- (0,2,0) -- (0,5,0) -- (0,5,3) -- (1,5,3) -- (1,4,3) -- (1,4,2.5) -- (1,3,3) -- (2,5,3) -- (3,5,3) -- (3,0,3) -- cycle;
\end{scope}
\shade[left color=darkgray,right color=darkgray!80!black] (2,0,0) -- (2,0,3) -- (2,5,3) -- (2,5,1) -- (2,4,1) -- (2,4,0) -- cycle;
\shade[left color=darkgray!90!black,right color=gray!80!darkgray] (2,0,2) -- (0,0,2) -- (0,2,2) -- (2,2,2) -- cycle;
\shade[top color=darkgray!90!black,bottom color=gray!80!darkgray] (0,1,0) -- (2,1,0) -- (2,1,2) -- (0,1,2) -- cycle;
\shade[top color=darkgray!90!black,bottom color=gray!80!darkgray] (0,3,1) -- (0,3,3) -- (2,3,3) -- (2,3,0) -- (1,3,0) -- (1,3,1) -- cycle;
\fill[gray] (2,1,0) -- (1.5,1,0.5) -- (0,1,0.5) -- (0,1,0) -- cycle;
\fill[gray] (1,3,2) -- (0.5,3,2) -- (0.5,3,3) -- (1,3,3) -- cycle;
\fill[gray] (2,3,0) -- (1.5,3,0.5) -- (1,3,0.5) -- (1,3,0) -- cycle;
\ecslogosurface
\end{tikzpicture}
}

\providecommand{\cpplogo}[1]{
\begin{tikzpicture}[scale=(#1)/512em]
\fill[gray] (435.2794,398.7159) -- (247.1911,507.3075) .. controls (236.3563,513.5642) and (218.6240,513.5642) .. (207.7892,507.3075) -- (19.7009,398.7159) .. controls (8.8646,392.4606) and (0.0000,377.1043) .. (0.0000,364.5924) -- (0.0000,147.4076) .. controls (0.8430,132.8363) and (8.2856,120.7683) .. (19.7009,113.2842) -- (207.7892,4.6926) .. controls (218.6240,-1.5642) and (236.3564,-1.5642) .. (247.1911,4.6926) -- (435.2794,113.2842) .. controls (447.5273,121.4304) and (454.4987,133.6918) .. (454.9803,147.4076) -- (454.9803,364.5924) .. controls (454.5404,377.7571) and (446.6566,391.0351) .. (435.2794,398.7159) -- cycle(75.8301,255.9993) .. controls (74.9389,404.0881) and (273.2892,469.4783) .. (358.8263,331.8769) -- (293.1917,293.8965) .. controls (253.5702,359.4301) and (155.1909,335.9977) .. (151.6601,255.9993) .. controls (152.7204,182.2703) and (249.4137,148.0211) .. (293.1961,218.1065) -- (358.8308,180.1276) .. controls (283.4477,49.2645) and (79.6318,96.3470) .. (75.8301,255.9993) -- cycle(379.1503,247.5747) -- (362.2982,247.5747) -- (362.2982,230.7226) -- (345.4490,230.7226) -- (345.4490,247.5747) -- (328.5969,247.5747) -- (328.5969,264.4254) -- (345.4490,264.4254) -- (345.4490,281.2759) -- (362.2982,281.2759) -- (362.2982,264.4254) -- (379.1503,264.4254) -- cycle(442.3420,247.5747) -- (425.4899,247.5747) -- (425.4899,230.7226) -- (408.6408,230.7226) -- (408.6408,247.5747) -- (391.7886,247.5747) -- (391.7886,264.4254) -- (408.6408,264.4254) -- (408.6408,281.2759) -- (425.4899,281.2759) -- (425.4899,264.4254) -- (442.3420,264.4254) -- cycle;
\end{tikzpicture}
}

\providecommand{\fallogo}[1]{
\begin{tikzpicture}[scale=(#1)/512em]
\fill[gray] (185.7774,0.0000) .. controls (200.4486,15.9798) and (226.8966,8.7148) .. (235.0426,31.5836) .. controls (249.5297,58.0598) and (247.9581,97.9161) .. (280.3335,110.9762) .. controls (309.1690,120.3496) and (337.8406,104.2727) .. (366.5753,103.9379) .. controls (373.4449,111.5171) and (379.2885,128.2574) .. (383.9755,108.9744) .. controls (396.6979,102.5615) and (437.2808,107.6681) .. (426.9652,124.3252) .. controls (408.9822,121.0785) and (412.4742,146.0729) .. (426.5192,131.4996) .. controls (433.8413,120.8489) and (465.1541,126.5522) .. (441.9067,135.7950) .. controls (396.1879,157.7478) and (344.1112,161.5079) .. (298.5528,183.5702) .. controls (277.7471,193.5198) and (284.6941,218.7163) .. (285.2127,236.9640) .. controls (292.3599,316.2826) and (307.3929,394.6311) .. (317.1198,473.6154) .. controls (329.0637,505.4736) and (292.1195,528.5004) .. (265.9183,511.2761) .. controls (237.9284,499.2462) and (237.3684,465.2681) .. (230.9102,439.9421) .. controls (218.6692,374.3397) and (215.6307,306.9662) .. (198.1732,242.3977) .. controls (183.1379,232.7444) and (164.4245,256.0298) .. (149.0430,261.4799) .. controls (116.9328,279.2585) and (87.1822,308.5851) .. (48.2293,307.8914) .. controls (21.3220,306.9037) and (-15.9107,281.8761) .. (7.2921,252.7908) .. controls (29.7799,220.6177) and (67.5177,204.3028) .. (100.9287,185.9449) .. controls (130.8217,170.8906) and (161.1548,156.5903) .. (191.0278,141.5847) .. controls (196.1738,120.0520) and (186.6049,95.2409) .. (186.8382,72.4353) .. controls (185.5234,48.4204) and (183.1700,23.9341) .. (185.7774,0.0000) -- cycle;
\end{tikzpicture}
}

\providecommand{\oblogo}[1]{
\begin{tikzpicture}[scale=(#1)/512em]
\fill[gray] (160.3865,208.9117) .. controls (154.0879,214.6478) and (149.0735,221.2409) .. (145.4125,228.5384) .. controls (184.8790,248.4273) and (234.7122,269.8787) .. (297.5493,291.8782) .. controls (300.3943,281.4769) and (300.9552,268.7619) .. (300.4023,255.2389) .. controls (248.9909,244.7891) and (200.0310,225.9279) .. (160.3865,208.9117) -- cycle(225.7398,392.6996) .. controls (308.0209,392.1716) and (359.3326,345.9277) .. (368.7203,285.2098) .. controls (376.6742,197.1784) and (311.7194,141.3342) .. (205.4287,142.1456) .. controls (139.9485,141.4804) and (88.7155,166.1957) .. (73.5775,228.0086) .. controls (52.0297,320.3408) and (123.4078,391.0103) .. (225.7398,392.6996) -- cycle(216.0739,176.4733) .. controls (268.9183,179.2424) and (315.8292,206.5488) .. (312.7454,265.1139) .. controls (313.2769,315.6384) and (286.5993,353.4946) .. (216.6040,355.7934) .. controls (162.4657,355.7934) and (126.0914,317.5023) .. (126.0914,260.5103) .. controls (126.1733,214.2900) and (163.3363,176.2849) .. (216.0739,176.4733) -- cycle(76.4897,189.1754) .. controls (13.1586,147.5631) and (0.0000,119.4207) .. (0.0000,119.4207) -- (90.6499,170.1632) .. controls (85.3004,175.8497) and (80.5994,182.1633) .. (76.4897,189.1754) -- cycle(353.9486,119.3004) -- (402.9482,119.3004) .. controls (427.0025,137.0797) and (450.9893,162.7034) .. (474.9529,191.0213) .. controls (509.3540,228.5339) and (531.3391,294.2091) .. (487.8149,312.1206) .. controls (462.8165,324.7652) and (394.3874,316.8943) .. (373.8912,313.6651) .. controls (379.9291,297.7449) and (383.2899,278.4204) .. (381.4989,257.7214) .. controls (420.3069,248.0321) and (421.9610,218.3461) .. (407.7867,192.6417) .. controls (391.1113,162.4018) and (370.1114,132.9097) .. (353.9486,119.3004) -- cycle;
\end{tikzpicture}
}

\providecommand{\markuptable}{
\begin{table}
\sffamily\centering
\begin{tabular}{@{}lcl@{}}
\toprule
\texttt{//italics//} & $\rightarrow$ & \textit{italics} \\
\midrule
\texttt{**bold**} & $\rightarrow$ & \textbf{bold} \\
\midrule
\texttt{\# ordered list} & & 1 ordered list \\
\texttt{\# second item} & $\rightarrow$ & 2 second item \\
\texttt{\#\# sub item} & & \hspace{1em} 1 sub item \\
\midrule
\texttt{* unordered list} & & $\bullet$ unordered list \\
\texttt{* second item} & $\rightarrow$ & $\bullet$ second item \\
\texttt{** sub item} & & \hspace{1em} $\bullet$ sub item \\
\midrule
\texttt{link to [[label]]} & $\rightarrow$ & link to \underline{label} \\
\midrule
\texttt{<{}<label>{}> definition } & $\rightarrow$ & definition \\
\midrule
\texttt{[[url|link name]]} & $\rightarrow$ & \underline{link name} \\
\midrule\addlinespace
\texttt{= large heading} & & {\Large large heading} \smallskip \\
\texttt{== medium heading} & $\rightarrow$ & {\large medium heading} \\
\texttt{=== small heading} & & small heading \\
\midrule
\texttt{no line break} & & no line break for paragraphs \\
\texttt{for paragraphs} & $\rightarrow$ \\
& & use empty line \\
\texttt{use empty line} \\
\midrule
\texttt{force\textbackslash\textbackslash line break} & $\rightarrow$ & force \\
& & line break \\
\midrule
\texttt{horizontal line} & $\rightarrow$ & horizontal line \\
\texttt{----} & & \hrulefill \\
\midrule
\texttt{|=a|=table|=header} & & \underline{a \enspace table \enspace header} \\
\texttt{|a|table|row} & $\rightarrow$ & a \enspace table \enspace row \\
\texttt{|b|table|row} & & b \enspace table \enspace row \\
\midrule
\texttt{\{\{\{} \\
\texttt{unformatted} & $\rightarrow$ & \texttt{unformatted} \\
\texttt{code} & & \texttt{code} \\
\texttt{\}\}\}} \\
\midrule\addlinespace
\texttt{@ new article} & & {\Large 1.\ new article} \smallskip \\
\texttt{@ second article} & $\rightarrow$ & {\Large 2.\ second article} \smallskip \\
\texttt{@@ sub article} & & {\large 2.1.\ sub article} \\
\bottomrule
\end{tabular}
\normalfont\caption{Elements of the generic documentation markup language}
\label{tab:docmarkup}
\end{table}
}

\providecommand{\startchapter}[4]{
\documentclass[11pt,a4paper]{article}
\usepackage{booktabs}
\usepackage[format=hang,labelfont=bf]{caption}
\usepackage{changepage}
\usepackage[T1]{fontenc}
\usepackage[margin=2cm]{geometry}
\usepackage{hyperref}
\usepackage[american]{isodate}
\usepackage{lmodern}
\usepackage{longtable}
\usepackage{mathptmx}
\usepackage{microtype}
\usepackage[toc]{multitoc}
\usepackage{multirow}
\usepackage[all]{nowidow}
\usepackage{pdfcomment}
\usepackage{syntax}
\usepackage{tikz}
\usepackage[all]{xy}
\hypersetup{pdfborder={0 0 0},bookmarksnumbered=true,pdftitle={\ecs{}: #2},pdfauthor={Florian Negele},pdfsubject={\ecs{}},pdfkeywords={#1}}
\setlength{\grammarindent}{8em}\setlength{\grammarparsep}{0.2ex}
\setlength{\columnsep}{2em}
\newcommand{\prefix}{}
\newcounter{instruction}
\bibliographystyle{unsrt}
\renewcommand{\index}[2][]{}
\renewcommand{\arraystretch}{1.05}
\renewcommand{\floatpagefraction}{0.7}
\renewcommand{\syntleft}{\itshape}\renewcommand{\syntright}{}
\title{\vspace{-5ex}\Huge{\ecs{}}\medskip\hrule}
\author{\huge{#2}}
\date{\medskip\version}
\newif\ifbook\bookfalse
\pagestyle{headings}
\frenchspacing
\begin{document}
\maketitle\thispagestyle{empty}\noindent#4\setlength{\columnseprule}{0.4pt}\tableofcontents\setlength{\columnseprule}{0pt}\vfill\pagebreak[3]\null\vfill\bigskip\noindent
\parbox{\textwidth-4em}{\license The contents of this \documentation{} are part of the \href{manual}{\ecs{} User Manual}~\cite{manual} and correspond to Chapter ``\href{manual\##3}{#1}''.\alignright\mbox{\today}}
\parbox{4em}{\flushright\ecslogo{3em}}
\clearpage
}

\providecommand{\concludechapter}{
\vfill\pagebreak[3]\null\vfill
\thispagestyle{myheadings}\markright{REFERENCES}
\noindent\begin{minipage}{\textwidth}\begin{multicols}{2}[\section*{References}]
\renewcommand{\section}[2]{}\small\bibliography{references}
\end{multicols}\end{minipage}\end{document}
}

\providecommand{\startpresentation}[2]{
\documentclass[14pt,aspectratio=43,usepdftitle=false]{beamer}
\usepackage{booktabs}
\usepackage{etex}
\usepackage{multicol}
\usepackage{tikz}
\usepackage[all]{xy}
\bibliographystyle{unsrt}
\setlength{\columnsep}{1em}
\setlength{\leftmargini}{1em}
\setbeamercolor{title}{fg=black}
\setbeamercolor{structure}{fg=darkgray}
\setbeamercolor{bibliography item}{fg=darkgray}
\setbeamerfont{title}{series=\bfseries}
\setbeamerfont{subtitle}{series=\normalfont}
\setbeamerfont*{frametitle}{parent=title}
\setbeamerfont{block title}{series=\bfseries}
\setbeamerfont*{framesubtitle}{parent=subtitle}
\setbeamersize{text margin left=1em,text margin right=1em}
\setbeamertemplate{navigation symbols}{}
\setbeamertemplate{itemize item}[circle]{}
\setbeamertemplate{bibliography item}[triangle]{}
\setbeamertemplate{bibliography entry author}{\usebeamercolor[fg]{bibliography item}}
\setbeamertemplate{frametitle}{\medskip\usebeamerfont{frametitle}\color{gray}\raisebox{-2.5ex}[0ex][0ex]{\rule{0.1em}{4.5ex}}}
\addtobeamertemplate{frametitle}{}{\hspace{0.4em}\usebeamercolor[fg]{title}\insertframetitle\par\vspace{0.2ex}\hspace{0.5em}\usebeamerfont{framesubtitle}\insertframesubtitle}
\hypersetup{pdfborder={0 0 0},bookmarksnumbered=true,bookmarksopen=true,bookmarksopenlevel=0,pdftitle={\ecs{}: #1},pdfauthor={Florian Negele},pdfsubject={\ecs{}},pdfkeywords={#1}}
\renewcommand{\flowgraph}[1]{\resizebox{\textwidth}{!}{$$\xymatrix{##1}$$}}
\title{\ecs{}\medskip\hrule\medskip}
\institute{\shadowedecslogo{5em}{30}{15}}
\date{\version}
\subtitle{#1}
\begin{document}
\begin{frame}[plain]\titlepage\nocite{manual}\end{frame}
\begin{frame}{Contents}{#1}\begin{center}\tableofcontents\end{center}\end{frame}
}

\providecommand{\concludepresentation}{
\begin{frame}{References}\begin{footnotesize}\setlength{\columnseprule}{0.4pt}\begin{multicols}{2}\bibliography{references}\end{multicols}\end{footnotesize}\end{frame}
\end{document}
}

\providecommand{\startbook}[1]{
\documentclass[10pt,paper=17cm:24cm,DIV=13,twoside=semi,headings=normal,numbers=noendperiod,cleardoublepage=plain]{scrbook}
\usepackage{atveryend}
\usepackage{booktabs}
\usepackage{caption}
\usepackage{changepage}
\usepackage[T1]{fontenc}
\usepackage{imakeidx}
\usepackage{hyperref}
\usepackage[american]{isodate}
\usepackage{lmodern}
\usepackage{longtable}
\usepackage{mathptmx}
\usepackage[final]{microtype}
\usepackage{multicol}
\usepackage{multirow}
\usepackage[all]{nowidow}
\usepackage{pdfcomment}
\usepackage{scrlayer-scrpage}
\usepackage{setspace}
\usepackage{syntax}
\usepackage[eventxtindent=4pt,oddtxtexdent=4pt]{thumbs}
\usepackage{tikz}
\usepackage[all]{xy}
\hyphenation{Micro-Blaze Open-Cores Open-RISC Power-PC}
\hypersetup{pdfborder={0 0 0},bookmarksnumbered=true,bookmarksopen=true,bookmarksopenlevel=0,pdftitle={\ecs{}: #1},pdfauthor={Florian Negele},pdfsubject={\ecs{}},pdfkeywords={#1}}
\setlength{\grammarindent}{8em}\setlength{\grammarparsep}{0.7ex}
\setkomafont{captionlabel}{\usekomafont{descriptionlabel}}
\renewcommand{\arraystretch}{1.05}\setstretch{1.1}
\renewcommand{\chapterformat}{\thechapter\autodot\enskip\raisebox{-1ex}[0ex][0ex]{\color{gray}\rule{0.1em}{3.5ex}}\enskip}
\renewcommand{\startchapter}[4]{\hypertarget{##3}{\chapter{##1}}\label{##3}##4\addthumb{##1}{\LARGE\sffamily\bfseries\thechapter}{white}{gray}\renewcommand{\prefix}{##3}}
\renewcommand{\concludechapter}{\clearpage{\stopthumb\cleardoublepage}}
\renewcommand{\syntleft}{\itshape}\renewcommand{\syntright}{}
\renewcommand{\floatpagefraction}{0.7}
\renewcommand{\partheademptypage}{}
\DeclareMicrotypeAlias{lmss}{cmr}
\newcommand{\prefix}{}
\newcounter{instruction}
\bibliographystyle{unsrt}
\newif\ifbook\booktrue
\makeindex[intoc,title=Index]
\makeindex[intoc,name=tools,title=Index of Tools,columns=3]
\makeindex[intoc,name=library,title=Index of Library Names]
\makeindex[intoc,name=runtime,title=Index of Runtime Support]
\makeindex[intoc,name=environment,title=Index of Target Environments]
\indexsetup{toclevel=chapter,headers={\indexname}{\indexname}}
\frenchspacing
\begin{document}
\pagenumbering{alph}
\begin{titlepage}\centering
\huge\sffamily\null\vfill\textbf{\ecs{}}\bigskip\hrule\bigskip#1
\normalsize\normalfont\vfill\vfill\shadowedecslogo{10em}{30}{15}
\large\vfill\vfill\version
\end{titlepage}
\null\vfill
\thispagestyle{empty}
\noindent\today\par\medskip
\license A copy of this license is included in Appendix~\ref{fdl} on page~\pageref{fdl}.
All product names used herein are for identification purposes only and may be trademarks of their respective companies.
\concludechapter
\frontmatter
\setcounter{tocdepth}{1}
\tableofcontents
\setcounter{tocdepth}{2}
\concludechapter
\listoffigures
\concludechapter
\listoftables
\concludechapter
}

\providecommand{\concludebook}{
\backmatter
\addtocontents{toc}{\protect\setcounter{tocdepth}{-1}}
\phantomsection\addcontentsline{toc}{part}{Bibliography}
\bibliography{references}
\concludechapter
\phantomsection\addcontentsline{toc}{part}{Indexes}
\printindex
\concludechapter
\indexprologue{\label{idx:tools}}
\printindex[tools]
\concludechapter
\printindex[library]
\concludechapter
\indexprologue{\label{idx:runtime}}
\printindex[runtime]
\concludechapter
\indexprologue{\label{idx:environment}}
\printindex[environment]
\concludechapter
\pagestyle{empty}\pagenumbering{Alph}\null\clearpage
\null\vfill\centering\ecslogo{4em}\par\medskip\license
\end{document}
}

% chapter references

\providecommand{\seedocumentationref}{}\renewcommand{\seedocumentationref}[3]{#1, see \Documentation{}~\documentationref{#2}{#3}. }
\providecommand{\seeinterface}{}\renewcommand{\seeinterface}{\ifbook See \Documentation{}~\documentationref{interface}{User Interface} for more information about the common user interface of all of these tools. \fi}
\providecommand{\seeguide}{}\renewcommand{\seeguide}{\seedocumentationref{For basic examples of using some of these tools in practice}{guide}{User Guide}}
\providecommand{\seecpp}{}\renewcommand{\seecpp}{\seedocumentationref{For more information about the \cpp{} programming language and its implementation by the \ecs{}}{cpp}{User Manual for \cpp{}}}
\providecommand{\seefalse}{}\renewcommand{\seefalse}{\seedocumentationref{For more information about the FALSE programming language and its implementation by the \ecs{}}{false}{User Manual for FALSE}}
\providecommand{\seeoberon}{}\renewcommand{\seeoberon}{\seedocumentationref{For more information about the Oberon programming language and its implementation by the \ecs{}}{oberon}{User Manual for Oberon}}
\providecommand{\seeassembly}{}\renewcommand{\seeassembly}{\seedocumentationref{For more information about the generic assembly language and how to use it}{assembly}{Generic Assembly Language Specification}}
\providecommand{\seeamd}{}\renewcommand{\seeamd}{\seedocumentationref{For more information about how the \ecs{} supports the AMD64 hardware architecture}{amd64}{AMD64 Hardware Architecture Support}}
\providecommand{\seearm}{}\renewcommand{\seearm}{\seedocumentationref{For more information about how the \ecs{} supports the ARM hardware architecture}{arm}{ARM Hardware Architecture Support}}
\providecommand{\seeavr}{}\renewcommand{\seeavr}{\seedocumentationref{For more information about how the \ecs{} supports the AVR hardware architecture}{avr}{AVR Hardware Architecture Support}}
\providecommand{\seeavrtt}{}\renewcommand{\seeavrtt}{\seedocumentationref{For more information about how the \ecs{} supports the AVR32 hardware architecture}{avr32}{AVR32 Hardware Architecture Support}}
\providecommand{\seemabk}{}\renewcommand{\seemabk}{\seedocumentationref{For more information about how the \ecs{} supports the M68000 hardware architecture}{m68k}{M68000 Hardware Architecture Support}}
\providecommand{\seemibl}{}\renewcommand{\seemibl}{\seedocumentationref{For more information about how the \ecs{} supports the MicroBlaze hardware architecture}{mibl}{MicroBlaze Hardware Architecture Support}}
\providecommand{\seemips}{}\renewcommand{\seemips}{\seedocumentationref{For more information about how the \ecs{} supports the MIPS32 and MIPS64 hardware architectures}{mips}{MIPS Hardware Architecture Support}}
\providecommand{\seemmix}{}\renewcommand{\seemmix}{\seedocumentationref{For more information about how the \ecs{} supports the MMIX hardware architecture}{mmix}{MMIX Hardware Architecture Support}}
\providecommand{\seeorok}{}\renewcommand{\seeorok}{\seedocumentationref{For more information about how the \ecs{} supports the OpenRISC 1000 hardware architecture}{or1k}{OpenRISC 1000 Hardware Architecture Support}}
\providecommand{\seeppc}{}\renewcommand{\seeppc}{\seedocumentationref{For more information about how the \ecs{} supports the PowerPC hardware architecture}{ppc}{PowerPC Hardware Architecture Support}}
\providecommand{\seerisc}{}\renewcommand{\seerisc}{\seedocumentationref{For more information about how the \ecs{} supports the RISC hardware architecture}{risc}{RISC Hardware Architecture Support}}
\providecommand{\seewasm}{}\renewcommand{\seewasm}{\seedocumentationref{For more information about how the \ecs{} supports the WebAssembly architecture}{wasm}{WebAssembly Architecture Support}}
\providecommand{\seedocumentation}{}\renewcommand{\seedocumentation}{\seedocumentationref{For more information about generic documentations and their generation by the \ecs{}}{documentation}{Generic Documentation Generation}}
\providecommand{\seedebugging}{}\renewcommand{\seedebugging}{\seedocumentationref{For more information about debugging information and its representation}{debugging}{Debugging Information Representation}}
\providecommand{\seecode}{}\renewcommand{\seecode}{\seedocumentationref{For more information about intermediate code and its purpose}{code}{Intermediate Code Representation}}
\providecommand{\seeobject}{}\renewcommand{\seeobject}{\seedocumentationref{For more information about object files and their purpose}{object}{Object File Representation}}

% generic documentation tools

\providecommand{\docprint}{
\toolsection{docprint} is a pretty printer for generic documentations.
It reformats generic documentations and writes it to the standard output stream.
\debuggingtool
\flowgraph{\resource{generic\\documentation} \ar[r] & \toolbox{docprint} \ar[r] & \resource{generic\\documentation}}
\seedocumentation
}

\providecommand{\doccheck}{
\toolsection{doccheck} is a syntactic and semantic checker for generic documentations.
It just performs syntactic and semantic checks on generic documentations and writes its diagnostic messages to the standard error stream.
\debuggingtool
\flowgraph{\resource{generic\\documentation} \ar[r] & \toolbox{doccheck} \ar[r] & \resource{diagnostic\\messages}}
\seedocumentation
}

\providecommand{\dochtml}{
\toolsection{dochtml} is an HTML documentation generator for generic documentations.
It processes several generic documentations and assembles all information therein into an HTML document.
\debuggingtool
\flowgraph{\resource{generic\\documentation} \ar[r] & \toolbox{dochtml} \ar[r] & \resource{HTML\\document}}
\seedocumentation
}

\providecommand{\doclatex}{
\toolsection{doclatex} is a Latex documentation generator for generic documentations.
It processes several generic documentations and assembles all information therein into a Latex document.
\debuggingtool
\flowgraph{\resource{generic\\documentation} \ar[r] & \toolbox{doclatex} \ar[r] & \resource{Latex\\document}}
\seedocumentation
}

% intermediate code tools

\providecommand{\cdcheck}{
\toolsection{cdcheck} is a syntactic and semantic checker for intermediate code.
It just performs syntactic and semantic checks on programs written in intermediate code and writes its diagnostic messages to the standard error stream.
\debuggingtool
\flowgraph{\resource{intermediate\\code} \ar[r] & \toolbox{cdcheck} \ar[r] & \resource{diagnostic\\messages}}
\seeassembly\seecode
}

\providecommand{\cdopt}{
\toolsection{cdopt} is an optimizer for intermediate code.
It performs various optimizations on programs written in intermediate code and writes the result to the standard output stream.
\debuggingtool
\flowgraph{\resource{intermediate\\code} \ar[r] & \toolbox{cdopt} \ar[r] & \resource{optimized\\code}}
\seeassembly\seecode
}

\providecommand{\cdrun}{
\toolsection{cdrun} is an interpreter for intermediate code.
It processes and executes programs written in intermediate code.
The following code sections are predefined and have the usual semantics:
\texttt{abort}, \texttt{\_Exit}, \texttt{fflush}, \texttt{floor}, \texttt{fputc}, \texttt{free}, \texttt{getchar}, \texttt{malloc}, and \texttt{putchar}.
Diagnostic messages about invalid operations include the name of the executed code section and the index of the erroneous instruction.
\debuggingtool
\flowgraph{\resource{intermediate\\code} \ar[r] & \toolbox{cdrun} \ar@/u/[r] & \resource{input/\\output} \ar@/d/[l]}
\seeassembly\seecode
}

\providecommand{\cdamda}{
\toolsection{cdamd16} is a compiler for intermediate code targeting the AMD64 hardware architecture.
It generates machine code for AMD64 processors from programs written in intermediate code and stores it in corresponding object files.
The compiler generates machine code for the 16-bit operating mode defined by the AMD64 architecture.
It also creates a debugging information file as well as an assembly file containing a listing of the generated machine code.
\debuggingtool
\flowgraph{\resource{intermediate\\code} \ar[r] & \toolbox{cdamd16} \ar[r] \ar[d] \ar[rd] & \resource{object file} \\ & \resource{assembly\\listing} & \resource{debugging\\information}}
\seeassembly\seeamd\seeobject\seecode\seedebugging
}

\providecommand{\cdamdb}{
\toolsection{cdamd32} is a compiler for intermediate code targeting the AMD64 hardware architecture.
It generates machine code for AMD64 processors from programs written in intermediate code and stores it in corresponding object files.
The compiler generates machine code for the 32-bit operating mode defined by the AMD64 architecture.
It also creates a debugging information file as well as an assembly file containing a listing of the generated machine code.
\debuggingtool
\flowgraph{\resource{intermediate\\code} \ar[r] & \toolbox{cdamd32} \ar[r] \ar[d] \ar[rd] & \resource{object file} \\ & \resource{assembly\\listing} & \resource{debugging\\information}}
\seeassembly\seeamd\seeobject\seecode\seedebugging
}

\providecommand{\cdamdc}{
\toolsection{cdamd64} is a compiler for intermediate code targeting the AMD64 hardware architecture.
It generates machine code for AMD64 processors from programs written in intermediate code and stores it in corresponding object files.
The compiler generates machine code for the 64-bit operating mode defined by the AMD64 architecture.
It also creates a debugging information file as well as an assembly file containing a listing of the generated machine code.
\debuggingtool
\flowgraph{\resource{intermediate\\code} \ar[r] & \toolbox{cdamd64} \ar[r] \ar[d] \ar[rd] & \resource{object file} \\ & \resource{assembly\\listing} & \resource{debugging\\information}}
\seeassembly\seeamd\seeobject\seecode\seedebugging
}

\providecommand{\cdarma}{
\toolsection{cdarma32} is a compiler for intermediate code targeting the ARM hardware architecture.
It generates machine code for ARM processors executing A32 instructions from programs written in intermediate code and stores it in corresponding object files.
It also creates a debugging information file as well as an assembly file containing a listing of the generated machine code.
\debuggingtool
\flowgraph{\resource{intermediate\\code} \ar[r] & \toolbox{cdarma32} \ar[r] \ar[d] \ar[rd] & \resource{object file} \\ & \resource{assembly\\listing} & \resource{debugging\\information}}
\seeassembly\seearm\seeobject\seecode\seedebugging
}

\providecommand{\cdarmb}{
\toolsection{cdarma64} is a compiler for intermediate code targeting the ARM hardware architecture.
It generates machine code for ARM processors executing A64 instructions from programs written in intermediate code and stores it in corresponding object files.
It also creates a debugging information file as well as an assembly file containing a listing of the generated machine code.
\debuggingtool
\flowgraph{\resource{intermediate\\code} \ar[r] & \toolbox{cdarma64} \ar[r] \ar[d] \ar[rd] & \resource{object file} \\ & \resource{assembly\\listing} & \resource{debugging\\information}}
\seeassembly\seearm\seeobject\seecode\seedebugging
}

\providecommand{\cdarmc}{
\toolsection{cdarmt32} is a compiler for intermediate code targeting the ARM hardware architecture.
It generates machine code for ARM processors without floating-point extension executing T32 instructions from programs written in intermediate code and stores it in corresponding object files.
It also creates a debugging information file as well as an assembly file containing a listing of the generated machine code.
\debuggingtool
\flowgraph{\resource{intermediate\\code} \ar[r] & \toolbox{cdarmt32} \ar[r] \ar[d] \ar[rd] & \resource{object file} \\ & \resource{assembly\\listing} & \resource{debugging\\information}}
\seeassembly\seearm\seeobject\seecode\seedebugging
}

\providecommand{\cdarmcfpe}{
\toolsection{cdarmt32fpe} is a compiler for intermediate code targeting the ARM hardware architecture.
It generates machine code for ARM processors with floating-point extension executing T32 instructions from programs written in intermediate code and stores it in corresponding object files.
It also creates a debugging information file as well as an assembly file containing a listing of the generated machine code.
\debuggingtool
\flowgraph{\resource{intermediate\\code} \ar[r] & \toolbox{cdarmt32fpe} \ar[r] \ar[d] \ar[rd] & \resource{object file} \\ & \resource{assembly\\listing} & \resource{debugging\\information}}
\seeassembly\seearm\seeobject\seecode\seedebugging
}

\providecommand{\cdavr}{
\toolsection{cdavr} is a compiler for intermediate code targeting the AVR hardware architecture.
It generates machine code for AVR processors from programs written in intermediate code and stores it in corresponding object files.
It also creates a debugging information file as well as an assembly file containing a listing of the generated machine code.
\debuggingtool
\flowgraph{\resource{intermediate\\code} \ar[r] & \toolbox{cdavr} \ar[r] \ar[d] \ar[rd] & \resource{object file} \\ & \resource{assembly\\listing} & \resource{debugging\\information}}
\seeassembly\seeavr\seeobject\seecode\seedebugging
}

\providecommand{\cdavrtt}{
\toolsection{cdavr32} is a compiler for intermediate code targeting the AVR32 hardware architecture.
It generates machine code for AVR32 processors from programs written in intermediate code and stores it in corresponding object files.
It also creates a debugging information file as well as an assembly file containing a listing of the generated machine code.
\debuggingtool
\flowgraph{\resource{intermediate\\code} \ar[r] & \toolbox{cdavr32} \ar[r] \ar[d] \ar[rd] & \resource{object file} \\ & \resource{assembly\\listing} & \resource{debugging\\information}}
\seeassembly\seeavrtt\seeobject\seecode\seedebugging
}

\providecommand{\cdmabk}{
\toolsection{cdm68k} is a compiler for intermediate code targeting the M68000 hardware architecture.
It generates machine code for M68000 processors from programs written in intermediate code and stores it in corresponding object files.
It also creates a debugging information file as well as an assembly file containing a listing of the generated machine code.
\debuggingtool
\flowgraph{\resource{intermediate\\code} \ar[r] & \toolbox{cdm68k} \ar[r] \ar[d] \ar[rd] & \resource{object file} \\ & \resource{assembly\\listing} & \resource{debugging\\information}}
\seeassembly\seemabk\seeobject\seecode\seedebugging
}

\providecommand{\cdmibl}{
\toolsection{cdmibl} is a compiler for intermediate code targeting the MicroBlaze hardware architecture.
It generates machine code for MicroBlaze processors from programs written in intermediate code and stores it in corresponding object files.
It also creates a debugging information file as well as an assembly file containing a listing of the generated machine code.
\debuggingtool
\flowgraph{\resource{intermediate\\code} \ar[r] & \toolbox{cdmibl} \ar[r] \ar[d] \ar[rd] & \resource{object file} \\ & \resource{assembly\\listing} & \resource{debugging\\information}}
\seeassembly\seemibl\seeobject\seecode\seedebugging
}

\providecommand{\cdmipsa}{
\toolsection{cdmips32} is a compiler for intermediate code targeting the MIPS32 hardware architecture.
It generates machine code for MIPS32 processors from programs written in intermediate code and stores it in corresponding object files.
It also creates a debugging information file as well as an assembly file containing a listing of the generated machine code.
\debuggingtool
\flowgraph{\resource{intermediate\\code} \ar[r] & \toolbox{cdmips32} \ar[r] \ar[d] \ar[rd] & \resource{object file} \\ & \resource{assembly\\listing} & \resource{debugging\\information}}
\seeassembly\seemips\seeobject\seecode\seedebugging
}

\providecommand{\cdmipsb}{
\toolsection{cdmips64} is a compiler for intermediate code targeting the MIPS64 hardware architecture.
It generates machine code for MIPS64 processors from programs written in intermediate code and stores it in corresponding object files.
It also creates a debugging information file as well as an assembly file containing a listing of the generated machine code.
\debuggingtool
\flowgraph{\resource{intermediate\\code} \ar[r] & \toolbox{cdmips64} \ar[r] \ar[d] \ar[rd] & \resource{object file} \\ & \resource{assembly\\listing} & \resource{debugging\\information}}
\seeassembly\seemips\seeobject\seecode\seedebugging
}

\providecommand{\cdmmix}{
\toolsection{cdmmix} is a compiler for intermediate code targeting the MMIX hardware architecture.
It generates machine code for MMIX processors from programs written in intermediate code and stores it in corresponding object files.
It also creates a debugging information file as well as an assembly file containing a listing of the generated machine code.
\debuggingtool
\flowgraph{\resource{intermediate\\code} \ar[r] & \toolbox{cdmmix} \ar[r] \ar[d] \ar[rd] & \resource{object file} \\ & \resource{assembly\\listing} & \resource{debugging\\information}}
\seeassembly\seemmix\seeobject\seecode\seedebugging
}

\providecommand{\cdorok}{
\toolsection{cdor1k} is a compiler for intermediate code targeting the OpenRISC 1000 hardware architecture.
It generates machine code for OpenRISC 1000 processors from programs written in intermediate code and stores it in corresponding object files.
It also creates a debugging information file as well as an assembly file containing a listing of the generated machine code.
\debuggingtool
\flowgraph{\resource{intermediate\\code} \ar[r] & \toolbox{cdor1k} \ar[r] \ar[d] \ar[rd] & \resource{object file} \\ & \resource{assembly\\listing} & \resource{debugging\\information}}
\seeassembly\seeorok\seeobject\seecode\seedebugging
}

\providecommand{\cdppca}{
\toolsection{cdppc32} is a compiler for intermediate code targeting the PowerPC hardware architecture.
It generates machine code for PowerPC processors from programs written in intermediate code and stores it in corresponding object files.
The compiler generates machine code for the 32-bit operating mode defined by the PowerPC architecture.
It also creates a debugging information file as well as an assembly file containing a listing of the generated machine code.
\debuggingtool
\flowgraph{\resource{intermediate\\code} \ar[r] & \toolbox{cdppc32} \ar[r] \ar[d] \ar[rd] & \resource{object file} \\ & \resource{assembly\\listing} & \resource{debugging\\information}}
\seeassembly\seeppc\seeobject\seecode\seedebugging
}

\providecommand{\cdppcb}{
\toolsection{cdppc64} is a compiler for intermediate code targeting the PowerPC hardware architecture.
It generates machine code for PowerPC processors from programs written in intermediate code and stores it in corresponding object files.
The compiler generates machine code for the 64-bit operating mode defined by the PowerPC architecture.
It also creates a debugging information file as well as an assembly file containing a listing of the generated machine code.
\debuggingtool
\flowgraph{\resource{intermediate\\code} \ar[r] & \toolbox{cdppc64} \ar[r] \ar[d] \ar[rd] & \resource{object file} \\ & \resource{assembly\\listing} & \resource{debugging\\information}}
\seeassembly\seeppc\seeobject\seecode\seedebugging
}

\providecommand{\cdrisc}{
\toolsection{cdrisc} is a compiler for intermediate code targeting the RISC hardware architecture.
It generates machine code for RISC processors from programs written in intermediate code and stores it in corresponding object files.
It also creates a debugging information file as well as an assembly file containing a listing of the generated machine code.
\debuggingtool
\flowgraph{\resource{intermediate\\code} \ar[r] & \toolbox{cdrisc} \ar[r] \ar[d] \ar[rd] & \resource{object file} \\ & \resource{assembly\\listing} & \resource{debugging\\information}}
\seeassembly\seerisc\seeobject\seecode\seedebugging
}

\providecommand{\cdwasm}{
\toolsection{cdwasm} is a compiler for intermediate code targeting the WebAssembly architecture.
It generates machine code for WebAssembly targets from programs written in intermediate code and stores it in corresponding object files.
It also creates a debugging information file as well as an assembly file containing a listing of the generated machine code.
\debuggingtool
\flowgraph{\resource{intermediate\\code} \ar[r] & \toolbox{cdwasm} \ar[r] \ar[d] \ar[rd] & \resource{object file} \\ & \resource{assembly\\listing} & \resource{debugging\\information}}
\seeassembly\seewasm\seeobject\seecode\seedebugging
}

% C++ tools

\providecommand{\cppprep}{
\toolsection{cppprep} is a preprocessor for the \cpp{} programming language.
It preprocesses source code according to the rules of \cpp{} and writes it to the standard output stream.
Only the macro names \texttt{\_\_DATE\_\_}, \texttt{\_\_FILE\_\_}, \texttt{\_\_LINE\_\_}, and \texttt{\_\_TIME\_\_} are predefined.
\flowgraph{\resource{\cpp{} or other\\source code} \ar[r] & \toolbox{cppprep} \ar[r] & \resource{preprocessed\\source code} \\ & \variable{ECSINCLUDE} \ar[u]}
\seecpp
}

\providecommand{\cppprint}{
\toolsection{cppprint} is a pretty printer for the \cpp{} programming language.
It reformats the source code of \cpp{} programs and writes it to the standard output stream.
\flowgraph{\resource{\cpp{}\\source code} \ar[r] & \toolbox{cppprint} \ar[r] & \resource{reformatted\\source code} \\ & \variable{ECSINCLUDE} \ar[u]}
\seecpp
}

\providecommand{\cppcheck}{
\toolsection{cppcheck} is a syntactic and semantic checker for the \cpp{} programming language.
It just performs syntactic and semantic checks on \cpp{} programs and writes its diagnostic messages to the standard error stream.
\flowgraph{\resource{\cpp{}\\source code} \ar[r] & \toolbox{cppcheck} \ar[r] & \resource{diagnostic\\messages} \\ & \variable{ECSINCLUDE} \ar[u]}
\seecpp
}

\providecommand{\cppdump}{
\toolsection{cppdump} is a serializer for the \cpp{} programming language.
It dumps the complete internal representation of programs written in \cpp{} into an XML document.
\debuggingtool
\flowgraph{\resource{\cpp{}\\source code} \ar[r] & \toolbox{cppdump} \ar[r] & \resource{internal\\representation} \\ & \variable{ECSINCLUDE} \ar[u]}
\seecpp
}

\providecommand{\cpprun}{
\toolsection{cpprun} is an interpreter for the \cpp{} programming language.
It processes and executes programs written in \cpp{}.
The macro \texttt{\_\_run\_\_} is predefined in order to enable programmers to identify this tool while interpreting.
\flowgraph{\resource{\cpp{}\\source code} \ar[r] & \toolbox{cpprun} \ar@/u/[r] & \resource{input/\\output} \ar@/d/[l] \\ & \variable{ECSINCLUDE} \ar[u]}
\seecpp
}

\providecommand{\cppdoc}{
\toolsection{cppdoc} is a generic documentation generator for the \cpp{} programming language.
It processes several \cpp{} source files and assembles all information therein into a generic documentation.
\debuggingtool
\flowgraph{\resource{\cpp{}\\source code} \ar[r] & \toolbox{cppdoc} \ar[r] & \resource{generic\\documentation} \\ & \variable{ECSINCLUDE} \ar[u]}
\seecpp\seedocumentation
}

\providecommand{\cpphtml}{
\toolsection{cpphtml} is an HTML documentation generator for the \cpp{} programming language.
It processes several \cpp{} source files and assembles all information therein into an HTML document.
\flowgraph{\resource{\cpp{}\\source code} \ar[r] & \toolbox{cpphtml} \ar[r] & \resource{HTML\\document} \\ & \variable{ECSINCLUDE} \ar[u]}
\seecpp\seedocumentation
}

\providecommand{\cpplatex}{
\toolsection{cpplatex} is a Latex documentation generator for the \cpp{} programming language.
It processes several \cpp{} source files and assembles all information therein into a Latex document.
\flowgraph{\resource{\cpp{}\\source code} \ar[r] & \toolbox{cpplatex} \ar[r] & \resource{Latex\\document} \\ & \variable{ECSINCLUDE} \ar[u]}
\seecpp\seedocumentation
}

\providecommand{\cppcode}{
\toolsection{cppcode} is an intermediate code generator for the \cpp{} programming language.
It generates intermediate code from programs written in \cpp{} and stores it in corresponding assembly files.
The macro \texttt{\_\_code\_\_} is predefined in order to enable programmers to identify this tool while generating intermediate code.
Programs generated with this tool require additional runtime support that is stored in the \file{cpp\-code\-run} library file.
\debuggingtool
\flowgraph{\resource{\cpp{}\\source code} \ar[r] & \toolbox{cppcode} \ar[r] & \resource{intermediate\\code} \\ & \variable{ECSINCLUDE} \ar[u]}
\seecpp\seeassembly\seecode
}

\providecommand{\cppamda}{
\toolsection{cppamd16} is a compiler for the \cpp{} programming language targeting the AMD64 hardware architecture.
It generates machine code for AMD64 processors from programs written in \cpp{} and stores it in corresponding object files.
The compiler generates machine code for the 16-bit operating mode defined by the AMD64 architecture.
For debugging purposes, it also creates a debugging information file as well as an assembly file containing a listing of the generated machine code.
The macro \texttt{\_\_amd16\_\_} is predefined in order to enable programmers to identify this tool and its target architecture while compiling.
Programs generated with this compiler require additional runtime support that is stored in the \file{cpp\-amd16\-run} library file.
\flowgraph{\resource{\cpp{}\\source code} \ar[r] & \toolbox{cppamd16} \ar[r] \ar[d] \ar[rd] & \resource{object file} \\ \variable{ECSINCLUDE} \ar[ru] & \resource{debugging\\information} & \resource{assembly\\listing}}
\seecpp\seeassembly\seeamd\seeobject\seedebugging
}

\providecommand{\cppamdb}{
\toolsection{cppamd32} is a compiler for the \cpp{} programming language targeting the AMD64 hardware architecture.
It generates machine code for AMD64 processors from programs written in \cpp{} and stores it in corresponding object files.
The compiler generates machine code for the 32-bit operating mode defined by the AMD64 architecture.
For debugging purposes, it also creates a debugging information file as well as an assembly file containing a listing of the generated machine code.
The macro \texttt{\_\_amd32\_\_} is predefined in order to enable programmers to identify this tool and its target architecture while compiling.
Programs generated with this compiler require additional runtime support that is stored in the \file{cpp\-amd32\-run} library file.
\flowgraph{\resource{\cpp{}\\source code} \ar[r] & \toolbox{cppamd32} \ar[r] \ar[d] \ar[rd] & \resource{object file} \\ \variable{ECSINCLUDE} \ar[ru] & \resource{debugging\\information} & \resource{assembly\\listing}}
\seecpp\seeassembly\seeamd\seeobject\seedebugging
}

\providecommand{\cppamdc}{
\toolsection{cppamd64} is a compiler for the \cpp{} programming language targeting the AMD64 hardware architecture.
It generates machine code for AMD64 processors from programs written in \cpp{} and stores it in corresponding object files.
The compiler generates machine code for the 64-bit operating mode defined by the AMD64 architecture.
For debugging purposes, it also creates a debugging information file as well as an assembly file containing a listing of the generated machine code.
The macro \texttt{\_\_amd64\_\_} is predefined in order to enable programmers to identify this tool and its target architecture while compiling.
Programs generated with this compiler require additional runtime support that is stored in the \file{cpp\-amd64\-run} library file.
\flowgraph{\resource{\cpp{}\\source code} \ar[r] & \toolbox{cppamd64} \ar[r] \ar[d] \ar[rd] & \resource{object file} \\ \variable{ECSINCLUDE} \ar[ru] & \resource{debugging\\information} & \resource{assembly\\listing}}
\seecpp\seeassembly\seeamd\seeobject\seedebugging
}

\providecommand{\cpparma}{
\toolsection{cpparma32} is a compiler for the \cpp{} programming language targeting the ARM hardware architecture.
It generates machine code for ARM processors executing A32 instructions from programs written in \cpp{} and stores it in corresponding object files.
For debugging purposes, it also creates a debugging information file as well as an assembly file containing a listing of the generated machine code.
The macro \texttt{\_\_arma32\_\_} is predefined in order to enable programmers to identify this tool and its target architecture while compiling.
Programs generated with this compiler require additional runtime support that is stored in the \file{cpp\-arma32\-run} library file.
\flowgraph{\resource{\cpp{}\\source code} \ar[r] & \toolbox{cpparma32} \ar[r] \ar[d] \ar[rd] & \resource{object file} \\ \variable{ECSINCLUDE} \ar[ru] & \resource{debugging\\information} & \resource{assembly\\listing}}
\seecpp\seeassembly\seearm\seeobject\seedebugging
}

\providecommand{\cpparmb}{
\toolsection{cpparma64} is a compiler for the \cpp{} programming language targeting the ARM hardware architecture.
It generates machine code for ARM processors executing A64 instructions from programs written in \cpp{} and stores it in corresponding object files.
For debugging purposes, it also creates a debugging information file as well as an assembly file containing a listing of the generated machine code.
The macro \texttt{\_\_arma64\_\_} is predefined in order to enable programmers to identify this tool and its target architecture while compiling.
Programs generated with this compiler require additional runtime support that is stored in the \file{cpp\-arma64\-run} library file.
\flowgraph{\resource{\cpp{}\\source code} \ar[r] & \toolbox{cpparma64} \ar[r] \ar[d] \ar[rd] & \resource{object file} \\ \variable{ECSINCLUDE} \ar[ru] & \resource{debugging\\information} & \resource{assembly\\listing}}
\seecpp\seeassembly\seearm\seeobject\seedebugging
}

\providecommand{\cpparmc}{
\toolsection{cpparmt32} is a compiler for the \cpp{} programming language targeting the ARM hardware architecture.
It generates machine code for ARM processors without floating-point extension executing T32 instructions from programs written in \cpp{} and stores it in corresponding object files.
For debugging purposes, it also creates a debugging information file as well as an assembly file containing a listing of the generated machine code.
The macro \texttt{\_\_armt32\_\_} is predefined in order to enable programmers to identify this tool and its target architecture while compiling.
Programs generated with this compiler require additional runtime support that is stored in the \file{cpp\-armt32\-run} library file.
\flowgraph{\resource{\cpp{}\\source code} \ar[r] & \toolbox{cpparmt32} \ar[r] \ar[d] \ar[rd] & \resource{object file} \\ \variable{ECSINCLUDE} \ar[ru] & \resource{debugging\\information} & \resource{assembly\\listing}}
\seecpp\seeassembly\seearm\seeobject\seedebugging
}

\providecommand{\cpparmcfpe}{
\toolsection{cpparmt32fpe} is a compiler for the \cpp{} programming language targeting the ARM hardware architecture.
It generates machine code for ARM processors with floating-point extension executing T32 instructions from programs written in \cpp{} and stores it in corresponding object files.
For debugging purposes, it also creates a debugging information file as well as an assembly file containing a listing of the generated machine code.
The macro \texttt{\_\_armt32fpe\_\_} is predefined in order to enable programmers to identify this tool and its target architecture while compiling.
Programs generated with this compiler require additional runtime support that is stored in the \file{cpp\-armt32\-fpe\-run} library file.
\flowgraph{\resource{\cpp{}\\source code} \ar[r] & \toolbox{cpparmt32fpe} \ar[r] \ar[d] \ar[rd] & \resource{object file} \\ \variable{ECSINCLUDE} \ar[ru] & \resource{debugging\\information} & \resource{assembly\\listing}}
\seecpp\seeassembly\seearm\seeobject\seedebugging
}

\providecommand{\cppavr}{
\toolsection{cppavr} is a compiler for the \cpp{} programming language targeting the AVR hardware architecture.
It generates machine code for AVR processors from programs written in \cpp{} and stores it in corresponding object files.
For debugging purposes, it also creates a debugging information file as well as an assembly file containing a listing of the generated machine code.
The macro \texttt{\_\_avr\_\_} is predefined in order to enable programmers to identify this tool and its target architecture while compiling.
Programs generated with this compiler require additional runtime support that is stored in the \file{cpp\-avr\-run} library file.
\flowgraph{\resource{\cpp{}\\source code} \ar[r] & \toolbox{cppavr} \ar[r] \ar[d] \ar[rd] & \resource{object file} \\ \variable{ECSINCLUDE} \ar[ru] & \resource{debugging\\information} & \resource{assembly\\listing}}
\seecpp\seeassembly\seeavr\seeobject\seedebugging
}

\providecommand{\cppavrtt}{
\toolsection{cppavr32} is a compiler for the \cpp{} programming language targeting the AVR32 hardware architecture.
It generates machine code for AVR32 processors from programs written in \cpp{} and stores it in corresponding object files.
For debugging purposes, it also creates a debugging information file as well as an assembly file containing a listing of the generated machine code.
The macro \texttt{\_\_avr32\_\_} is predefined in order to enable programmers to identify this tool and its target architecture while compiling.
Programs generated with this compiler require additional runtime support that is stored in the \file{cpp\-avr32\-run} library file.
\flowgraph{\resource{\cpp{}\\source code} \ar[r] & \toolbox{cppavr32} \ar[r] \ar[d] \ar[rd] & \resource{object file} \\ \variable{ECSINCLUDE} \ar[ru] & \resource{debugging\\information} & \resource{assembly\\listing}}
\seecpp\seeassembly\seeavrtt\seeobject\seedebugging
}

\providecommand{\cppmabk}{
\toolsection{cppm68k} is a compiler for the \cpp{} programming language targeting the M68000 hardware architecture.
It generates machine code for M68000 processors from programs written in \cpp{} and stores it in corresponding object files.
For debugging purposes, it also creates a debugging information file as well as an assembly file containing a listing of the generated machine code.
The macro \texttt{\_\_m68k\_\_} is predefined in order to enable programmers to identify this tool and its target architecture while compiling.
Programs generated with this compiler require additional runtime support that is stored in the \file{cpp\-m68k\-run} library file.
\flowgraph{\resource{\cpp{}\\source code} \ar[r] & \toolbox{cppm68k} \ar[r] \ar[d] \ar[rd] & \resource{object file} \\ \variable{ECSINCLUDE} \ar[ru] & \resource{debugging\\information} & \resource{assembly\\listing}}
\seecpp\seeassembly\seemabk\seeobject\seedebugging
}

\providecommand{\cppmibl}{
\toolsection{cppmibl} is a compiler for the \cpp{} programming language targeting the MicroBlaze hardware architecture.
It generates machine code for MicroBlaze processors from programs written in \cpp{} and stores it in corresponding object files.
For debugging purposes, it also creates a debugging information file as well as an assembly file containing a listing of the generated machine code.
The macro \texttt{\_\_mibl\_\_} is predefined in order to enable programmers to identify this tool and its target architecture while compiling.
Programs generated with this compiler require additional runtime support that is stored in the \file{cpp\-mibl\-run} library file.
\flowgraph{\resource{\cpp{}\\source code} \ar[r] & \toolbox{cppmibl} \ar[r] \ar[d] \ar[rd] & \resource{object file} \\ \variable{ECSINCLUDE} \ar[ru] & \resource{debugging\\information} & \resource{assembly\\listing}}
\seecpp\seeassembly\seemibl\seeobject\seedebugging
}

\providecommand{\cppmipsa}{
\toolsection{cppmips32} is a compiler for the \cpp{} programming language targeting the MIPS32 hardware architecture.
It generates machine code for MIPS32 processors from programs written in \cpp{} and stores it in corresponding object files.
For debugging purposes, it also creates a debugging information file as well as an assembly file containing a listing of the generated machine code.
The macro \texttt{\_\_mips32\_\_} is predefined in order to enable programmers to identify this tool and its target architecture while compiling.
Programs generated with this compiler require additional runtime support that is stored in the \file{cpp\-mips32\-run} library file.
\flowgraph{\resource{\cpp{}\\source code} \ar[r] & \toolbox{cppmips32} \ar[r] \ar[d] \ar[rd] & \resource{object file} \\ \variable{ECSINCLUDE} \ar[ru] & \resource{debugging\\information} & \resource{assembly\\listing}}
\seecpp\seeassembly\seemips\seeobject\seedebugging
}

\providecommand{\cppmipsb}{
\toolsection{cppmips64} is a compiler for the \cpp{} programming language targeting the MIPS64 hardware architecture.
It generates machine code for MIPS64 processors from programs written in \cpp{} and stores it in corresponding object files.
For debugging purposes, it also creates a debugging information file as well as an assembly file containing a listing of the generated machine code.
The macro \texttt{\_\_mips64\_\_} is predefined in order to enable programmers to identify this tool and its target architecture while compiling.
Programs generated with this compiler require additional runtime support that is stored in the \file{cpp\-mips64\-run} library file.
\flowgraph{\resource{\cpp{}\\source code} \ar[r] & \toolbox{cppmips64} \ar[r] \ar[d] \ar[rd] & \resource{object file} \\ \variable{ECSINCLUDE} \ar[ru] & \resource{debugging\\information} & \resource{assembly\\listing}}
\seecpp\seeassembly\seemips\seeobject\seedebugging
}

\providecommand{\cppmmix}{
\toolsection{cppmmix} is a compiler for the \cpp{} programming language targeting the MMIX hardware architecture.
It generates machine code for MMIX processors from programs written in \cpp{} and stores it in corresponding object files.
For debugging purposes, it also creates a debugging information file as well as an assembly file containing a listing of the generated machine code.
The macro \texttt{\_\_mmix\_\_} is predefined in order to enable programmers to identify this tool and its target architecture while compiling.
Programs generated with this compiler require additional runtime support that is stored in the \file{cpp\-mmix\-run} library file.
\flowgraph{\resource{\cpp{}\\source code} \ar[r] & \toolbox{cppmmix} \ar[r] \ar[d] \ar[rd] & \resource{object file} \\ \variable{ECSINCLUDE} \ar[ru] & \resource{debugging\\information} & \resource{assembly\\listing}}
\seecpp\seeassembly\seemmix\seeobject\seedebugging
}

\providecommand{\cpporok}{
\toolsection{cppor1k} is a compiler for the \cpp{} programming language targeting the OpenRISC 1000 hardware architecture.
It generates machine code for OpenRISC 1000 processors from programs written in \cpp{} and stores it in corresponding object files.
For debugging purposes, it also creates a debugging information file as well as an assembly file containing a listing of the generated machine code.
The macro \texttt{\_\_or1k\_\_} is predefined in order to enable programmers to identify this tool and its target architecture while compiling.
Programs generated with this compiler require additional runtime support that is stored in the \file{cpp\-or1k\-run} library file.
\flowgraph{\resource{\cpp{}\\source code} \ar[r] & \toolbox{cppor1k} \ar[r] \ar[d] \ar[rd] & \resource{object file} \\ \variable{ECSINCLUDE} \ar[ru] & \resource{debugging\\information} & \resource{assembly\\listing}}
\seecpp\seeassembly\seeorok\seeobject\seedebugging
}

\providecommand{\cppppca}{
\toolsection{cppppc32} is a compiler for the \cpp{} programming language targeting the PowerPC hardware architecture.
It generates machine code for PowerPC processors from programs written in \cpp{} and stores it in corresponding object files.
The compiler generates machine code for the 32-bit operating mode defined by the PowerPC architecture.
For debugging purposes, it also creates a debugging information file as well as an assembly file containing a listing of the generated machine code.
The macro \texttt{\_\_ppc32\_\_} is predefined in order to enable programmers to identify this tool and its target architecture while compiling.
Programs generated with this compiler require additional runtime support that is stored in the \file{cpp\-ppc32\-run} library file.
\flowgraph{\resource{\cpp{}\\source code} \ar[r] & \toolbox{cppppc32} \ar[r] \ar[d] \ar[rd] & \resource{object file} \\ \variable{ECSINCLUDE} \ar[ru] & \resource{debugging\\information} & \resource{assembly\\listing}}
\seecpp\seeassembly\seeppc\seeobject\seedebugging
}

\providecommand{\cppppcb}{
\toolsection{cppppc64} is a compiler for the \cpp{} programming language targeting the PowerPC hardware architecture.
It generates machine code for PowerPC processors from programs written in \cpp{} and stores it in corresponding object files.
The compiler generates machine code for the 64-bit operating mode defined by the PowerPC architecture.
For debugging purposes, it also creates a debugging information file as well as an assembly file containing a listing of the generated machine code.
The macro \texttt{\_\_ppc64\_\_} is predefined in order to enable programmers to identify this tool and its target architecture while compiling.
Programs generated with this compiler require additional runtime support that is stored in the \file{cpp\-ppc64\-run} library file.
\flowgraph{\resource{\cpp{}\\source code} \ar[r] & \toolbox{cppppc64} \ar[r] \ar[d] \ar[rd] & \resource{object file} \\ \variable{ECSINCLUDE} \ar[ru] & \resource{debugging\\information} & \resource{assembly\\listing}}
\seecpp\seeassembly\seeppc\seeobject\seedebugging
}

\providecommand{\cpprisc}{
\toolsection{cpprisc} is a compiler for the \cpp{} programming language targeting the RISC hardware architecture.
It generates machine code for RISC processors from programs written in \cpp{} and stores it in corresponding object files.
For debugging purposes, it also creates a debugging information file as well as an assembly file containing a listing of the generated machine code.
The macro \texttt{\_\_risc\_\_} is predefined in order to enable programmers to identify this tool and its target architecture while compiling.
Programs generated with this compiler require additional runtime support that is stored in the \file{cpp\-risc\-run} library file.
\flowgraph{\resource{\cpp{}\\source code} \ar[r] & \toolbox{cpprisc} \ar[r] \ar[d] \ar[rd] & \resource{object file} \\ \variable{ECSINCLUDE} \ar[ru] & \resource{debugging\\information} & \resource{assembly\\listing}}
\seecpp\seeassembly\seerisc\seeobject\seedebugging
}

\providecommand{\cppwasm}{
\toolsection{cppwasm} is a compiler for the \cpp{} programming language targeting the WebAssembly architecture.
It generates machine code for WebAssembly targets from programs written in \cpp{} and stores it in corresponding object files.
For debugging purposes, it also creates a debugging information file as well as an assembly file containing a listing of the generated machine code.
The macro \texttt{\_\_wasm\_\_} is predefined in order to enable programmers to identify this tool and its target architecture while compiling.
Programs generated with this compiler require additional runtime support that is stored in the \file{cpp\-wasm\-run} library file.
\flowgraph{\resource{\cpp{}\\source code} \ar[r] & \toolbox{cppwasm} \ar[r] \ar[d] \ar[rd] & \resource{object file} \\ \variable{ECSINCLUDE} \ar[ru] & \resource{debugging\\information} & \resource{assembly\\listing}}
\seecpp\seeassembly\seewasm\seeobject\seedebugging
}

% FALSE tools

\providecommand{\falprint}{
\toolsection{falprint} is a pretty printer for the FALSE programming language.
It reformats the source code of FALSE programs and writes it to the standard output stream.
\flowgraph{\resource{FALSE\\source code} \ar[r] & \toolbox{falprint} \ar[r] & \resource{reformatted\\source code}}
\seefalse
}

\providecommand{\falcheck}{
\toolsection{falcheck} is a syntactic and semantic checker for the FALSE programming language.
It just performs syntactic and semantic checks on FALSE programs and writes its diagnostic messages to the standard error stream.
\flowgraph{\resource{FALSE\\source code} \ar[r] & \toolbox{falcheck} \ar[r] & \resource{diagnostic\\messages}}
\seefalse
}

\providecommand{\faldump}{
\toolsection{faldump} is a serializer for the FALSE programming language.
It dumps the complete internal representation of programs written in FALSE into an XML document.
\debuggingtool
\flowgraph{\resource{FALSE\\source code} \ar[r] & \toolbox{faldump} \ar[r] & \resource{internal\\representation}}
\seefalse
}

\providecommand{\falrun}{
\toolsection{falrun} is an interpreter for the FALSE programming language.
It processes and executes programs written in FALSE\@.
\flowgraph{\resource{FALSE\\source code} \ar[r] & \toolbox{falrun} \ar@/u/[r] & \resource{input/\\output} \ar@/d/[l]}
\seefalse
}

\providecommand{\falcpp}{
\toolsection{falcpp} is a transpiler for the FALSE programming language.
It translates programs written in FALSE into \cpp{} programs and stores them in corresponding source files.
\flowgraph{\resource{FALSE\\source code} \ar[r] & \toolbox{falcpp} \ar[r] & \resource{\cpp{}\\source file}}
\seefalse\seecpp
}

\providecommand{\falcode}{
\toolsection{falcode} is an intermediate code generator for the FALSE programming language.
It generates intermediate code from programs written in FALSE and stores it in corresponding assembly files.
\debuggingtool
\flowgraph{\resource{FALSE\\source code} \ar[r] & \toolbox{falcode} \ar[r] & \resource{intermediate\\code}}
\seefalse\seeassembly\seecode
}

\providecommand{\falamda}{
\toolsection{falamd16} is a compiler for the FALSE programming language targeting the AMD64 hardware architecture.
It generates machine code for AMD64 processors from programs written in FALSE and stores it in corresponding object files.
The compiler generates machine code for the 16-bit operating mode defined by the AMD64 architecture.
\flowgraph{\resource{FALSE\\source code} \ar[r] & \toolbox{falamd16} \ar[r] & \resource{object file}}
\seefalse\seeamd\seeobject
}

\providecommand{\falamdb}{
\toolsection{falamd32} is a compiler for the FALSE programming language targeting the AMD64 hardware architecture.
It generates machine code for AMD64 processors from programs written in FALSE and stores it in corresponding object files.
The compiler generates machine code for the 32-bit operating mode defined by the AMD64 architecture.
\flowgraph{\resource{FALSE\\source code} \ar[r] & \toolbox{falamd32} \ar[r] & \resource{object file}}
\seefalse\seeamd\seeobject
}

\providecommand{\falamdc}{
\toolsection{falamd64} is a compiler for the FALSE programming language targeting the AMD64 hardware architecture.
It generates machine code for AMD64 processors from programs written in FALSE and stores it in corresponding object files.
The compiler generates machine code for the 64-bit operating mode defined by the AMD64 architecture.
\flowgraph{\resource{FALSE\\source code} \ar[r] & \toolbox{falamd64} \ar[r] & \resource{object file}}
\seefalse\seeamd\seeobject
}

\providecommand{\falarma}{
\toolsection{falarma32} is a compiler for the FALSE programming language targeting the ARM hardware architecture.
It generates machine code for ARM processors executing A32 instructions from programs written in FALSE and stores it in corresponding object files.
\flowgraph{\resource{FALSE\\source code} \ar[r] & \toolbox{falarma32} \ar[r] & \resource{object file}}
\seefalse\seearm\seeobject
}

\providecommand{\falarmb}{
\toolsection{falarma64} is a compiler for the FALSE programming language targeting the ARM hardware architecture.
It generates machine code for ARM processors executing A64 instructions from programs written in FALSE and stores it in corresponding object files.
\flowgraph{\resource{FALSE\\source code} \ar[r] & \toolbox{falarma64} \ar[r] & \resource{object file}}
\seefalse\seearm\seeobject
}

\providecommand{\falarmc}{
\toolsection{falarmt32} is a compiler for the FALSE programming language targeting the ARM hardware architecture.
It generates machine code for ARM processors without floating-point extension executing T32 instructions from programs written in FALSE and stores it in corresponding object files.
\flowgraph{\resource{FALSE\\source code} \ar[r] & \toolbox{falarmt32} \ar[r] & \resource{object file}}
\seefalse\seearm\seeobject
}

\providecommand{\falarmcfpe}{
\toolsection{falarmt32fpe} is a compiler for the FALSE programming language targeting the ARM hardware architecture.
It generates machine code for ARM processors with floating-point extension executing T32 instructions from programs written in FALSE and stores it in corresponding object files.
\flowgraph{\resource{FALSE\\source code} \ar[r] & \toolbox{falarmt32fpe} \ar[r] & \resource{object file}}
\seefalse\seearm\seeobject
}

\providecommand{\falavr}{
\toolsection{falavr} is a compiler for the FALSE programming language targeting the AVR hardware architecture.
It generates machine code for AVR processors from programs written in FALSE and stores it in corresponding object files.
\flowgraph{\resource{FALSE\\source code} \ar[r] & \toolbox{falavr} \ar[r] & \resource{object file}}
\seefalse\seeavr\seeobject
}

\providecommand{\falavrtt}{
\toolsection{falavr32} is a compiler for the FALSE programming language targeting the AVR32 hardware architecture.
It generates machine code for AVR32 processors from programs written in FALSE and stores it in corresponding object files.
\flowgraph{\resource{FALSE\\source code} \ar[r] & \toolbox{falavr32} \ar[r] & \resource{object file}}
\seefalse\seeavrtt\seeobject
}

\providecommand{\falmabk}{
\toolsection{falm68k} is a compiler for the FALSE programming language targeting the M68000 hardware architecture.
It generates machine code for M68000 processors from programs written in FALSE and stores it in corresponding object files.
\flowgraph{\resource{FALSE\\source code} \ar[r] & \toolbox{falm68k} \ar[r] & \resource{object file}}
\seefalse\seemabk\seeobject
}

\providecommand{\falmibl}{
\toolsection{falmibl} is a compiler for the FALSE programming language targeting the MicroBlaze hardware architecture.
It generates machine code for MicroBlaze processors from programs written in FALSE and stores it in corresponding object files.
\flowgraph{\resource{FALSE\\source code} \ar[r] & \toolbox{falmibl} \ar[r] & \resource{object file}}
\seefalse\seemibl\seeobject
}

\providecommand{\falmipsa}{
\toolsection{falmips32} is a compiler for the FALSE programming language targeting the MIPS32 hardware architecture.
It generates machine code for MIPS32 processors from programs written in FALSE and stores it in corresponding object files.
\flowgraph{\resource{FALSE\\source code} \ar[r] & \toolbox{falmips32} \ar[r] & \resource{object file}}
\seefalse\seemips\seeobject
}

\providecommand{\falmipsb}{
\toolsection{falmips64} is a compiler for the FALSE programming language targeting the MIPS64 hardware architecture.
It generates machine code for MIPS64 processors from programs written in FALSE and stores it in corresponding object files.
\flowgraph{\resource{FALSE\\source code} \ar[r] & \toolbox{falmips64} \ar[r] & \resource{object file}}
\seefalse\seemips\seeobject
}

\providecommand{\falmmix}{
\toolsection{falmmix} is a compiler for the FALSE programming language targeting the MMIX hardware architecture.
It generates machine code for MMIX processors from programs written in FALSE and stores it in corresponding object files.
\flowgraph{\resource{FALSE\\source code} \ar[r] & \toolbox{falmmix} \ar[r] & \resource{object file}}
\seefalse\seemmix\seeobject
}

\providecommand{\falorok}{
\toolsection{falor1k} is a compiler for the FALSE programming language targeting the OpenRISC 1000 hardware architecture.
It generates machine code for OpenRISC 1000 processors from programs written in FALSE and stores it in corresponding object files.
\flowgraph{\resource{FALSE\\source code} \ar[r] & \toolbox{falor1k} \ar[r] & \resource{object file}}
\seefalse\seeorok\seeobject
}

\providecommand{\falppca}{
\toolsection{falppc32} is a compiler for the FALSE programming language targeting the PowerPC hardware architecture.
It generates machine code for PowerPC processors from programs written in FALSE and stores it in corresponding object files.
The compiler generates machine code for the 32-bit operating mode defined by the PowerPC architecture.
\flowgraph{\resource{FALSE\\source code} \ar[r] & \toolbox{falppc32} \ar[r] & \resource{object file}}
\seefalse\seeppc\seeobject
}

\providecommand{\falppcb}{
\toolsection{falppc64} is a compiler for the FALSE programming language targeting the PowerPC hardware architecture.
It generates machine code for PowerPC processors from programs written in FALSE and stores it in corresponding object files.
The compiler generates machine code for the 64-bit operating mode defined by the PowerPC architecture.
\flowgraph{\resource{FALSE\\source code} \ar[r] & \toolbox{falppc64} \ar[r] & \resource{object file}}
\seefalse\seeppc\seeobject
}

\providecommand{\falrisc}{
\toolsection{falrisc} is a compiler for the FALSE programming language targeting the RISC hardware architecture.
It generates machine code for RISC processors from programs written in FALSE and stores it in corresponding object files.
\flowgraph{\resource{FALSE\\source code} \ar[r] & \toolbox{falrisc} \ar[r] & \resource{object file}}
\seefalse\seerisc\seeobject
}

\providecommand{\falwasm}{
\toolsection{falwasm} is a compiler for the FALSE programming language targeting the WebAssembly architecture.
It generates machine code for WebAssembly targets from programs written in FALSE and stores it in corresponding object files.
\flowgraph{\resource{FALSE\\source code} \ar[r] & \toolbox{falwasm} \ar[r] & \resource{object file}}
\seefalse\seewasm\seeobject
}

% Oberon tools

\providecommand{\obprint}{
\toolsection{obprint} is a pretty printer for the Oberon programming language.
It reformats the source code of Oberon modules and writes it to the standard output stream.
\flowgraph{\resource{Oberon\\source code} \ar[r] & \toolbox{obprint} \ar[r] & \resource{reformatted\\source code}}
\seeoberon
}

\providecommand{\obcheck}{
\toolsection{obcheck} is a syntactic and semantic checker for the Oberon programming language.
It just performs syntactic and semantic checks on Oberon modules and writes its diagnostic messages to the standard error stream.
In addition, it stores the interface of each module in a symbol file which is required when other modules import the module.
\flowgraph{\resource{Oberon\\source code} \ar[r] & \toolbox{obcheck} \ar[r] \ar@/l/[d] & \resource{diagnostic\\messages} \\ \variable{ECSIMPORT} \ar[ru] & \resource{symbol\\files} \ar@/r/[u]}
\seeoberon
}

\providecommand{\obdump}{
\toolsection{obdump} is a serializer for the Oberon programming language.
It dumps the complete internal representation of modules written in Oberon into an XML document.
\debuggingtool
\flowgraph{\resource{Oberon\\source code} \ar[r] & \toolbox{obdump} \ar[r] \ar@/l/[d] & \resource{internal\\representation} \\ \variable{ECSIMPORT} \ar[ru] & \resource{symbol\\files} \ar@/r/[u]}
\seeoberon
}

\providecommand{\obrun}{
\toolsection{obrun} is an interpreter for the Oberon programming language.
It processes and executes modules written in Oberon.
This tool does neither generate nor process symbol files while interpreting modules.
If a module is imported by another one, its filename has to be named before the other one in the list of command-line arguments.
\flowgraph{\resource{Oberon\\source code} \ar[r] & \toolbox{obrun} \ar@/u/[r] & \resource{input/\\output} \ar@/d/[l]}
\seeoberon
}

\providecommand{\obcpp}{
\toolsection{obcpp} is a transpiler for the Oberon programming language.
It translates programs written in Oberon into \cpp{} programs and stores them in corresponding source and header files.
In addition, it stores the interface of each module in a symbol file which is required when other modules import the module.
The same interface is provided by the generated header file which can be used in other parts of the \cpp{} program.
\flowgraph{\resource{Oberon\\source code} \ar[r] & \toolbox{obcpp} \ar[r] \ar@/l/[d] \ar[rd] & \resource{\cpp{}\\source file} \\ \variable{ECSIMPORT} \ar[ru] & \resource{symbol\\files} \ar@/r/[u] & \resource{\cpp{}\\header file}}
\seeoberon\seecpp
}

\providecommand{\obdoc}{
\toolsection{obdoc} is a generic documentation generator for the Oberon programming language.
It processes several Oberon modules and assembles all information therein into a generic documentation.
In addition, it stores the interface of each module in a symbol file which is required when other modules import the module.
\debuggingtool
\flowgraph{\resource{Oberon\\source code} \ar[r] & \toolbox{obdoc} \ar[r] \ar@/l/[d] & \resource{generic\\documentation} \\ \variable{ECSIMPORT} \ar[ru] & \resource{symbol\\files} \ar@/r/[u]}
\seeoberon\seedocumentation
}

\providecommand{\obhtml}{
\toolsection{obhtml} is an HTML documentation generator for the Oberon programming language.
It processes several Oberon modules and assembles all information therein into an HTML document.
In addition, it stores the interface of each module in a symbol file which is required when other modules import the module.
\flowgraph{\resource{Oberon\\source code} \ar[r] & \toolbox{obhtml} \ar[r] \ar@/l/[d] & \resource{HTML\\document} \\ \variable{ECSIMPORT} \ar[ru] & \resource{symbol\\files} \ar@/r/[u]}
\seeoberon\seedocumentation
}

\providecommand{\oblatex}{
\toolsection{oblatex} is a Latex documentation generator for the Oberon programming language.
It processes several Oberon modules and assembles all information therein into a Latex document.
In addition, it stores the interface of each module in a symbol file which is required when other modules import the module.
\flowgraph{\resource{Oberon\\source code} \ar[r] & \toolbox{oblatex} \ar[r] \ar@/l/[d] & \resource{Latex\\document} \\ \variable{ECSIMPORT} \ar[ru] & \resource{symbol\\files} \ar@/r/[u]}
\seeoberon\seedocumentation
}

\providecommand{\obcode}{
\toolsection{obcode} is an intermediate code generator for the Oberon programming language.
It generates intermediate code from modules written in Oberon and stores it in corresponding assembly files.
In addition, it stores the interface of each module in a symbol file which is required when other modules import the module.
Programs generated with this tool require additional runtime support that is stored in the \file{ob\-code\-run} library file.
\debuggingtool
\flowgraph{\resource{Oberon\\source code} \ar[r] & \toolbox{obcode} \ar[r] \ar@/l/[d] & \resource{intermediate\\code} \\ \variable{ECSIMPORT} \ar[ru] & \resource{symbol\\files} \ar@/r/[u]}
\seeoberon\seeassembly\seecode
}

\providecommand{\obamda}{
\toolsection{obamd16} is a compiler for the Oberon programming language targeting the AMD64 hardware architecture.
It generates machine code for AMD64 processors from modules written in Oberon and stores it in corresponding object files.
The compiler generates machine code for the 16-bit operating mode defined by the AMD64 architecture.
For debugging purposes, it also creates a debugging information file as well as an assembly file containing a listing of the generated machine code.
In addition, it stores the interface of each module in a symbol file which is required when other modules import the module.
Programs generated with this compiler require additional runtime support that is stored in the \file{ob\-amd16\-run} library file.
\flowgraph{\resource{Oberon\\source code} \ar[r] & \toolbox{obamd16} \ar[r] \ar@/l/[d] \ar[rd] & \resource{object file} \\ \variable{ECSIMPORT} \ar[ru] & \resource{symbol\\files} \ar@/r/[u] & \resource{debugging\\information}}
\seeoberon\seeassembly\seeamd\seeobject\seedebugging
}

\providecommand{\obamdb}{
\toolsection{obamd32} is a compiler for the Oberon programming language targeting the AMD64 hardware architecture.
It generates machine code for AMD64 processors from modules written in Oberon and stores it in corresponding object files.
The compiler generates machine code for the 32-bit operating mode defined by the AMD64 architecture.
For debugging purposes, it also creates a debugging information file as well as an assembly file containing a listing of the generated machine code.
In addition, it stores the interface of each module in a symbol file which is required when other modules import the module.
Programs generated with this compiler require additional runtime support that is stored in the \file{ob\-amd32\-run} library file.
\flowgraph{\resource{Oberon\\source code} \ar[r] & \toolbox{obamd32} \ar[r] \ar@/l/[d] \ar[rd] & \resource{object file} \\ \variable{ECSIMPORT} \ar[ru] & \resource{symbol\\files} \ar@/r/[u] & \resource{debugging\\information}}
\seeoberon\seeassembly\seeamd\seeobject\seedebugging
}

\providecommand{\obamdc}{
\toolsection{obamd64} is a compiler for the Oberon programming language targeting the AMD64 hardware architecture.
It generates machine code for AMD64 processors from modules written in Oberon and stores it in corresponding object files.
The compiler generates machine code for the 64-bit operating mode defined by the AMD64 architecture.
For debugging purposes, it also creates a debugging information file as well as an assembly file containing a listing of the generated machine code.
In addition, it stores the interface of each module in a symbol file which is required when other modules import the module.
Programs generated with this compiler require additional runtime support that is stored in the \file{ob\-amd64\-run} library file.
\flowgraph{\resource{Oberon\\source code} \ar[r] & \toolbox{obamd64} \ar[r] \ar@/l/[d] \ar[rd] & \resource{object file} \\ \variable{ECSIMPORT} \ar[ru] & \resource{symbol\\files} \ar@/r/[u] & \resource{debugging\\information}}
\seeoberon\seeassembly\seeamd\seeobject\seedebugging
}

\providecommand{\obarma}{
\toolsection{obarma32} is a compiler for the Oberon programming language targeting the ARM hardware architecture.
It generates machine code for ARM processors executing A32 instructions from modules written in Oberon and stores it in corresponding object files.
For debugging purposes, it also creates a debugging information file as well as an assembly file containing a listing of the generated machine code.
In addition, it stores the interface of each module in a symbol file which is required when other modules import the module.
Programs generated with this compiler require additional runtime support that is stored in the \file{ob\-arma32\-run} library file.
\flowgraph{\resource{Oberon\\source code} \ar[r] & \toolbox{obarma32} \ar[r] \ar@/l/[d] \ar[rd] & \resource{object file} \\ \variable{ECSIMPORT} \ar[ru] & \resource{symbol\\files} \ar@/r/[u] & \resource{debugging\\information}}
\seeoberon\seeassembly\seearm\seeobject\seedebugging
}

\providecommand{\obarmb}{
\toolsection{obarma64} is a compiler for the Oberon programming language targeting the ARM hardware architecture.
It generates machine code for ARM processors executing A64 instructions from modules written in Oberon and stores it in corresponding object files.
For debugging purposes, it also creates a debugging information file as well as an assembly file containing a listing of the generated machine code.
In addition, it stores the interface of each module in a symbol file which is required when other modules import the module.
Programs generated with this compiler require additional runtime support that is stored in the \file{ob\-arma64\-run} library file.
\flowgraph{\resource{Oberon\\source code} \ar[r] & \toolbox{obarma64} \ar[r] \ar@/l/[d] \ar[rd] & \resource{object file} \\ \variable{ECSIMPORT} \ar[ru] & \resource{symbol\\files} \ar@/r/[u] & \resource{debugging\\information}}
\seeoberon\seeassembly\seearm\seeobject\seedebugging
}

\providecommand{\obarmc}{
\toolsection{obarmt32} is a compiler for the Oberon programming language targeting the ARM hardware architecture.
It generates machine code for ARM processors without floating-point extension executing T32 instructions from modules written in Oberon and stores it in corresponding object files.
For debugging purposes, it also creates a debugging information file as well as an assembly file containing a listing of the generated machine code.
In addition, it stores the interface of each module in a symbol file which is required when other modules import the module.
Programs generated with this compiler require additional runtime support that is stored in the \file{ob\-armt32\-run} library file.
\flowgraph{\resource{Oberon\\source code} \ar[r] & \toolbox{obarmt32} \ar[r] \ar@/l/[d] \ar[rd] & \resource{object file} \\ \variable{ECSIMPORT} \ar[ru] & \resource{symbol\\files} \ar@/r/[u] & \resource{debugging\\information}}
\seeoberon\seeassembly\seearm\seeobject\seedebugging
}

\providecommand{\obarmcfpe}{
\toolsection{obarmt32fpe} is a compiler for the Oberon programming language targeting the ARM hardware architecture.
It generates machine code for ARM processors with floating-point extension executing T32 instructions from modules written in Oberon and stores it in corresponding object files.
For debugging purposes, it also creates a debugging information file as well as an assembly file containing a listing of the generated machine code.
In addition, it stores the interface of each module in a symbol file which is required when other modules import the module.
Programs generated with this compiler require additional runtime support that is stored in the \file{ob\-armt32\-fpe\-run} library file.
\flowgraph{\resource{Oberon\\source code} \ar[r] & \toolbox{obarmt32fpe} \ar[r] \ar@/l/[d] \ar[rd] & \resource{object file} \\ \variable{ECSIMPORT} \ar[ru] & \resource{symbol\\files} \ar@/r/[u] & \resource{debugging\\information}}
\seeoberon\seeassembly\seearm\seeobject\seedebugging
}

\providecommand{\obavr}{
\toolsection{obavr} is a compiler for the Oberon programming language targeting the AVR hardware architecture.
It generates machine code for AVR processors from modules written in Oberon and stores it in corresponding object files.
For debugging purposes, it also creates a debugging information file as well as an assembly file containing a listing of the generated machine code.
In addition, it stores the interface of each module in a symbol file which is required when other modules import the module.
Programs generated with this compiler require additional runtime support that is stored in the \file{ob\-avr\-run} library file.
\flowgraph{\resource{Oberon\\source code} \ar[r] & \toolbox{obavr} \ar[r] \ar@/l/[d] \ar[rd] & \resource{object file} \\ \variable{ECSIMPORT} \ar[ru] & \resource{symbol\\files} \ar@/r/[u] & \resource{debugging\\information}}
\seeoberon\seeassembly\seeavr\seeobject\seedebugging
}

\providecommand{\obavrtt}{
\toolsection{obavr32} is a compiler for the Oberon programming language targeting the AVR32 hardware architecture.
It generates machine code for AVR32 processors from modules written in Oberon and stores it in corresponding object files.
For debugging purposes, it also creates a debugging information file as well as an assembly file containing a listing of the generated machine code.
In addition, it stores the interface of each module in a symbol file which is required when other modules import the module.
Programs generated with this compiler require additional runtime support that is stored in the \file{ob\-avr32\-run} library file.
\flowgraph{\resource{Oberon\\source code} \ar[r] & \toolbox{obavr32} \ar[r] \ar@/l/[d] \ar[rd] & \resource{object file} \\ \variable{ECSIMPORT} \ar[ru] & \resource{symbol\\files} \ar@/r/[u] & \resource{debugging\\information}}
\seeoberon\seeassembly\seeavrtt\seeobject\seedebugging
}

\providecommand{\obmabk}{
\toolsection{obm68k} is a compiler for the Oberon programming language targeting the M68000 hardware architecture.
It generates machine code for M68000 processors from modules written in Oberon and stores it in corresponding object files.
For debugging purposes, it also creates a debugging information file as well as an assembly file containing a listing of the generated machine code.
In addition, it stores the interface of each module in a symbol file which is required when other modules import the module.
Programs generated with this compiler require additional runtime support that is stored in the \file{ob\-m68k\-run} library file.
\flowgraph{\resource{Oberon\\source code} \ar[r] & \toolbox{obm68k} \ar[r] \ar@/l/[d] \ar[rd] & \resource{object file} \\ \variable{ECSIMPORT} \ar[ru] & \resource{symbol\\files} \ar@/r/[u] & \resource{debugging\\information}}
\seeoberon\seeassembly\seemabk\seeobject\seedebugging
}

\providecommand{\obmibl}{
\toolsection{obmibl} is a compiler for the Oberon programming language targeting the MicroBlaze hardware architecture.
It generates machine code for MicroBlaze processors from modules written in Oberon and stores it in corresponding object files.
For debugging purposes, it also creates a debugging information file as well as an assembly file containing a listing of the generated machine code.
In addition, it stores the interface of each module in a symbol file which is required when other modules import the module.
Programs generated with this compiler require additional runtime support that is stored in the \file{ob\-mibl\-run} library file.
\flowgraph{\resource{Oberon\\source code} \ar[r] & \toolbox{obmibl} \ar[r] \ar@/l/[d] \ar[rd] & \resource{object file} \\ \variable{ECSIMPORT} \ar[ru] & \resource{symbol\\files} \ar@/r/[u] & \resource{debugging\\information}}
\seeoberon\seeassembly\seemibl\seeobject\seedebugging
}

\providecommand{\obmipsa}{
\toolsection{obmips32} is a compiler for the Oberon programming language targeting the MIPS32 hardware architecture.
It generates machine code for MIPS32 processors from modules written in Oberon and stores it in corresponding object files.
For debugging purposes, it also creates a debugging information file as well as an assembly file containing a listing of the generated machine code.
In addition, it stores the interface of each module in a symbol file which is required when other modules import the module.
Programs generated with this compiler require additional runtime support that is stored in the \file{ob\-mips32\-run} library file.
\flowgraph{\resource{Oberon\\source code} \ar[r] & \toolbox{obmips32} \ar[r] \ar@/l/[d] \ar[rd] & \resource{object file} \\ \variable{ECSIMPORT} \ar[ru] & \resource{symbol\\files} \ar@/r/[u] & \resource{debugging\\information}}
\seeoberon\seeassembly\seemips\seeobject\seedebugging
}

\providecommand{\obmipsb}{
\toolsection{obmips64} is a compiler for the Oberon programming language targeting the MIPS64 hardware architecture.
It generates machine code for MIPS64 processors from modules written in Oberon and stores it in corresponding object files.
For debugging purposes, it also creates a debugging information file as well as an assembly file containing a listing of the generated machine code.
In addition, it stores the interface of each module in a symbol file which is required when other modules import the module.
Programs generated with this compiler require additional runtime support that is stored in the \file{ob\-mips64\-run} library file.
\flowgraph{\resource{Oberon\\source code} \ar[r] & \toolbox{obmips64} \ar[r] \ar@/l/[d] \ar[rd] & \resource{object file} \\ \variable{ECSIMPORT} \ar[ru] & \resource{symbol\\files} \ar@/r/[u] & \resource{debugging\\information}}
\seeoberon\seeassembly\seemips\seeobject\seedebugging
}

\providecommand{\obmmix}{
\toolsection{obmmix} is a compiler for the Oberon programming language targeting the MMIX hardware architecture.
It generates machine code for MMIX processors from modules written in Oberon and stores it in corresponding object files.
For debugging purposes, it also creates a debugging information file as well as an assembly file containing a listing of the generated machine code.
In addition, it stores the interface of each module in a symbol file which is required when other modules import the module.
Programs generated with this compiler require additional runtime support that is stored in the \file{ob\-mmix\-run} library file.
\flowgraph{\resource{Oberon\\source code} \ar[r] & \toolbox{obmmix} \ar[r] \ar@/l/[d] \ar[rd] & \resource{object file} \\ \variable{ECSIMPORT} \ar[ru] & \resource{symbol\\files} \ar@/r/[u] & \resource{debugging\\information}}
\seeoberon\seeassembly\seemmix\seeobject\seedebugging
}

\providecommand{\oborok}{
\toolsection{obor1k} is a compiler for the Oberon programming language targeting the OpenRISC 1000 hardware architecture.
It generates machine code for OpenRISC 1000 processors from modules written in Oberon and stores it in corresponding object files.
For debugging purposes, it also creates a debugging information file as well as an assembly file containing a listing of the generated machine code.
In addition, it stores the interface of each module in a symbol file which is required when other modules import the module.
Programs generated with this compiler require additional runtime support that is stored in the \file{ob\-or1k\-run} library file.
\flowgraph{\resource{Oberon\\source code} \ar[r] & \toolbox{obor1k} \ar[r] \ar@/l/[d] \ar[rd] & \resource{object file} \\ \variable{ECSIMPORT} \ar[ru] & \resource{symbol\\files} \ar@/r/[u] & \resource{debugging\\information}}
\seeoberon\seeassembly\seeorok\seeobject\seedebugging
}

\providecommand{\obppca}{
\toolsection{obppc32} is a compiler for the Oberon programming language targeting the PowerPC hardware architecture.
It generates machine code for PowerPC processors from modules written in Oberon and stores it in corresponding object files.
The compiler generates machine code for the 32-bit operating mode defined by the PowerPC architecture.
For debugging purposes, it also creates a debugging information file as well as an assembly file containing a listing of the generated machine code.
In addition, it stores the interface of each module in a symbol file which is required when other modules import the module.
Programs generated with this compiler require additional runtime support that is stored in the \file{ob\-ppc32\-run} library file.
\flowgraph{\resource{Oberon\\source code} \ar[r] & \toolbox{obppc32} \ar[r] \ar@/l/[d] \ar[rd] & \resource{object file} \\ \variable{ECSIMPORT} \ar[ru] & \resource{symbol\\files} \ar@/r/[u] & \resource{debugging\\information}}
\seeoberon\seeassembly\seeppc\seeobject\seedebugging
}

\providecommand{\obppcb}{
\toolsection{obppc64} is a compiler for the Oberon programming language targeting the PowerPC hardware architecture.
It generates machine code for PowerPC processors from modules written in Oberon and stores it in corresponding object files.
The compiler generates machine code for the 64-bit operating mode defined by the PowerPC architecture.
For debugging purposes, it also creates a debugging information file as well as an assembly file containing a listing of the generated machine code.
In addition, it stores the interface of each module in a symbol file which is required when other modules import the module.
Programs generated with this compiler require additional runtime support that is stored in the \file{ob\-ppc64\-run} library file.
\flowgraph{\resource{Oberon\\source code} \ar[r] & \toolbox{obppc64} \ar[r] \ar@/l/[d] \ar[rd] & \resource{object file} \\ \variable{ECSIMPORT} \ar[ru] & \resource{symbol\\files} \ar@/r/[u] & \resource{debugging\\information}}
\seeoberon\seeassembly\seeppc\seeobject\seedebugging
}

\providecommand{\obrisc}{
\toolsection{obrisc} is a compiler for the Oberon programming language targeting the RISC hardware architecture.
It generates machine code for RISC processors from modules written in Oberon and stores it in corresponding object files.
For debugging purposes, it also creates a debugging information file as well as an assembly file containing a listing of the generated machine code.
In addition, it stores the interface of each module in a symbol file which is required when other modules import the module.
Programs generated with this compiler require additional runtime support that is stored in the \file{ob\-risc\-run} library file.
\flowgraph{\resource{Oberon\\source code} \ar[r] & \toolbox{obrisc} \ar[r] \ar@/l/[d] \ar[rd] & \resource{object file} \\ \variable{ECSIMPORT} \ar[ru] & \resource{symbol\\files} \ar@/r/[u] & \resource{debugging\\information}}
\seeoberon\seeassembly\seerisc\seeobject\seedebugging
}

\providecommand{\obwasm}{
\toolsection{obwasm} is a compiler for the Oberon programming language targeting the WebAssembly architecture.
It generates machine code for WebAssembly targets from modules written in Oberon and stores it in corresponding object files.
For debugging purposes, it also creates a debugging information file as well as an assembly file containing a listing of the generated machine code.
In addition, it stores the interface of each module in a symbol file which is required when other modules import the module.
Programs generated with this compiler require additional runtime support that is stored in the \file{ob\-wasm\-run} library file.
\flowgraph{\resource{Oberon\\source code} \ar[r] & \toolbox{obwasm} \ar[r] \ar@/l/[d] \ar[rd] & \resource{object file} \\ \variable{ECSIMPORT} \ar[ru] & \resource{symbol\\files} \ar@/r/[u] & \resource{debugging\\information}}
\seeoberon\seeassembly\seewasm\seeobject\seedebugging
}

% converter tools

\providecommand{\dbgdwarf}{
\toolsection{dbgdwarf} is a DWARF debugging information converter tool.
It converts debugging information into the DWARF debugging data format and stores it in corresponding object files~\cite{dwarffile}.
The resulting debugging object files can be combined with runtime support that creates Executable and Linking Format (ELF) files~\cite{elffile}.
\flowgraph{\resource{debugging\\information} \ar[r] & \toolbox{dbgdwarf} \ar[r] & \resource{debugging\\object file}}
\seeobject\seedebugging
}

% assembler tools

\providecommand{\asmprint}{
\toolsection{asmprint} is a pretty printer for generic assembly code.
It reformats generic assembly code and writes it to the standard output stream.
\flowgraph{\resource{generic assembly\\source code} \ar[r] & \toolbox{asmprint} \ar[r] & \resource{reformatted\\source code}}
\seeassembly
}

\providecommand{\amdaasm}{
\toolsection{amd16asm} is an assembler for the AMD64 hardware architecture.
It translates assembly code into machine code for AMD64 processors and stores it in corresponding object files.
By default, the assembler generates machine code for the 16-bit operating mode defined by the AMD64 architecture.
\flowgraph{\resource{AMD16 assembly\\source code} \ar[r] & \toolbox{amd16asm} \ar[r] & \resource{object file}}
\seeassembly\seeamd\seeobject
}

\providecommand{\amdadism}{
\toolsection{amd16dism} is a disassembler for the AMD64 hardware architecture.
It translates machine code from object files targeting AMD64 processors into assembly code and writes it to the standard output stream.
It assumes that the machine code was generated for the 16-bit operating mode defined by the AMD64 architecture.
\flowgraph{\resource{object file} \ar[r] & \toolbox{amd16dism} \ar[r] & \resource{disassembly\\listing}}
\seeassembly\seeamd\seeobject
}

\providecommand{\amdbasm}{
\toolsection{amd32asm} is an assembler for the AMD64 hardware architecture.
It translates assembly code into machine code for AMD64 processors and stores it in corresponding object files.
By default, the assembler generates machine code for the 32-bit operating mode defined by the AMD64 architecture.
\flowgraph{\resource{AMD32 assembly\\source code} \ar[r] & \toolbox{amd32asm} \ar[r] & \resource{object file}}
\seeassembly\seeamd\seeobject
}

\providecommand{\amdbdism}{
\toolsection{amd32dism} is a disassembler for the AMD64 hardware architecture.
It translates machine code from object files targeting AMD64 processors into assembly code and writes it to the standard output stream.
It assumes that the machine code was generated for the 32-bit operating mode defined by the AMD64 architecture.
\flowgraph{\resource{object file} \ar[r] & \toolbox{amd32dism} \ar[r] & \resource{disassembly\\listing}}
\seeassembly\seeamd\seeobject
}

\providecommand{\amdcasm}{
\toolsection{amd64asm} is an assembler for the AMD64 hardware architecture.
It translates assembly code into machine code for AMD64 processors and stores it in corresponding object files.
By default, the assembler generates machine code for the 64-bit operating mode defined by the AMD64 architecture.
\flowgraph{\resource{AMD64 assembly\\source code} \ar[r] & \toolbox{amd64asm} \ar[r] & \resource{object file}}
\seeassembly\seeamd\seeobject
}

\providecommand{\amdcdism}{
\toolsection{amd64dism} is a disassembler for the AMD64 hardware architecture.
It translates machine code from object files targeting AMD64 processors into assembly code and writes it to the standard output stream.
It assumes that the machine code was generated for the 64-bit operating mode defined by the AMD64 architecture.
\flowgraph{\resource{object file} \ar[r] & \toolbox{amd64dism} \ar[r] & \resource{disassembly\\listing}}
\seeassembly\seeamd\seeobject
}

\providecommand{\armaasm}{
\toolsection{arma32asm} is an assembler for the ARM hardware architecture.
It translates assembly code into machine code for ARM processors executing A32 instructions and stores it in corresponding object files.
\flowgraph{\resource{ARM A32 assembly\\source code} \ar[r] & \toolbox{arma32asm} \ar[r] & \resource{object file}}
\seeassembly\seearm\seeobject
}

\providecommand{\armadism}{
\toolsection{arma32dism} is a disassembler for the ARM hardware architecture.
It translates machine code from object files targeting ARM processors executing A32 instructions into assembly code and writes it to the standard output stream.
\flowgraph{\resource{object file} \ar[r] & \toolbox{arma32dism} \ar[r] & \resource{disassembly\\listing}}
\seeassembly\seearm\seeobject
}

\providecommand{\armbasm}{
\toolsection{arma64asm} is an assembler for the ARM hardware architecture.
It translates assembly code into machine code for ARM processors executing A64 instructions and stores it in corresponding object files.
\flowgraph{\resource{ARM A64 assembly\\source code} \ar[r] & \toolbox{arma64asm} \ar[r] & \resource{object file}}
\seeassembly\seearm\seeobject
}

\providecommand{\armbdism}{
\toolsection{arma64dism} is a disassembler for the ARM hardware architecture.
It translates machine code from object files targeting ARM processors executing A64 instructions into assembly code and writes it to the standard output stream.
\flowgraph{\resource{object file} \ar[r] & \toolbox{arma64dism} \ar[r] & \resource{disassembly\\listing}}
\seeassembly\seearm\seeobject
}

\providecommand{\armcasm}{
\toolsection{armt32asm} is an assembler for the ARM hardware architecture.
It translates assembly code into machine code for ARM processors executing T32 instructions and stores it in corresponding object files.
\flowgraph{\resource{ARM T32 assembly\\source code} \ar[r] & \toolbox{armt32asm} \ar[r] & \resource{object file}}
\seeassembly\seearm\seeobject
}

\providecommand{\armcdism}{
\toolsection{armt32dism} is a disassembler for the ARM hardware architecture.
It translates machine code from object files targeting ARM processors executing T32 instructions into assembly code and writes it to the standard output stream.
\flowgraph{\resource{object file} \ar[r] & \toolbox{armt32dism} \ar[r] & \resource{disassembly\\listing}}
\seeassembly\seearm\seeobject
}

\providecommand{\avrasm}{
\toolsection{avrasm} is an assembler for the AVR hardware architecture.
It translates assembly code into machine code for AVR processors and stores it in corresponding object files.
The identifiers \texttt{RXL}, \texttt{RXH}, \texttt{RYL}, \texttt{RYH}, \texttt{RZL}, and \texttt{RZH} are predefined and name the corresponding registers.
The identifiers \texttt{SPL} and \texttt{SPH} are also predefined and evaluate to the address of the corresponding registers.
\flowgraph{\resource{AVR assembly\\source code} \ar[r] & \toolbox{avrasm} \ar[r] & \resource{object file}}
\seeassembly\seeavr\seeobject
}

\providecommand{\avrdism}{
\toolsection{avrdism} is a disassembler for the AVR hardware architecture.
It translates machine code from object files targeting AVR processors into assembly code and writes it to the standard output stream.
\flowgraph{\resource{object file} \ar[r] & \toolbox{avrdism} \ar[r] & \resource{disassembly\\listing}}
\seeassembly\seeavr\seeobject
}

\providecommand{\avrttasm}{
\toolsection{avr32asm} is an assembler for the AVR32 hardware architecture.
It translates assembly code into machine code for AVR32 processors and stores it in corresponding object files.
\flowgraph{\resource{AVR32 assembly\\source code} \ar[r] & \toolbox{avr32asm} \ar[r] & \resource{object file}}
\seeassembly\seeavrtt\seeobject
}

\providecommand{\avrttdism}{
\toolsection{avr32dism} is a disassembler for the AVR32 hardware architecture.
It translates machine code from object files targeting AVR32 processors into assembly code and writes it to the standard output stream.
\flowgraph{\resource{object file} \ar[r] & \toolbox{avr32dism} \ar[r] & \resource{disassembly\\listing}}
\seeassembly\seeavrtt\seeobject
}

\providecommand{\mabkasm}{
\toolsection{m68kasm} is an assembler for the M68000 hardware architecture.
It translates assembly code into machine code for M68000 processors and stores it in corresponding object files.
\flowgraph{\resource{68000 assembly\\source code} \ar[r] & \toolbox{m68kasm} \ar[r] & \resource{object file}}
\seeassembly\seemabk\seeobject
}

\providecommand{\mabkdism}{
\toolsection{m68kdism} is a disassembler for the M68000 hardware architecture.
It translates machine code from object files targeting M68000 processors into assembly code and writes it to the standard output stream.
\flowgraph{\resource{object file} \ar[r] & \toolbox{m68kdism} \ar[r] & \resource{disassembly\\listing}}
\seeassembly\seemabk\seeobject
}

\providecommand{\miblasm}{
\toolsection{miblasm} is an assembler for the MicroBlaze hardware architecture.
It translates assembly code into machine code for MicroBlaze processors and stores it in corresponding object files.
\flowgraph{\resource{MicroBlaze assembly\\source code} \ar[r] & \toolbox{miblasm} \ar[r] & \resource{object file}}
\seeassembly\seemibl\seeobject
}

\providecommand{\mibldism}{
\toolsection{mibldism} is a disassembler for the MicroBlaze hardware architecture.
It translates machine code from object files targeting MicroBlaze processors into assembly code and writes it to the standard output stream.
\flowgraph{\resource{object file} \ar[r] & \toolbox{mibldism} \ar[r] & \resource{disassembly\\listing}}
\seeassembly\seemibl\seeobject
}

\providecommand{\mipsaasm}{
\toolsection{mips32asm} is an assembler for the MIPS32 hardware architecture.
It translates assembly code into machine code for MIPS32 processors and stores it in corresponding object files.
\flowgraph{\resource{MIPS32 assembly\\source code} \ar[r] & \toolbox{mips32asm} \ar[r] & \resource{object file}}
\seeassembly\seemips\seeobject
}

\providecommand{\mipsadism}{
\toolsection{mips32dism} is a disassembler for the MIPS32 hardware architecture.
It translates machine code from object files targeting MIPS32 processors into assembly code and writes it to the standard output stream.
\flowgraph{\resource{object file} \ar[r] & \toolbox{mips32dism} \ar[r] & \resource{disassembly\\listing}}
\seeassembly\seemips\seeobject
}

\providecommand{\mipsbasm}{
\toolsection{mips64asm} is an assembler for the MIPS64 hardware architecture.
It translates assembly code into machine code for MIPS64 processors and stores it in corresponding object files.
\flowgraph{\resource{MIPS64 assembly\\source code} \ar[r] & \toolbox{mips64asm} \ar[r] & \resource{object file}}
\seeassembly\seemips\seeobject
}

\providecommand{\mipsbdism}{
\toolsection{mips64dism} is a disassembler for the MIPS64 hardware architecture.
It translates machine code from object files targeting MIPS64 processors into assembly code and writes it to the standard output stream.
\flowgraph{\resource{object file} \ar[r] & \toolbox{mips64dism} \ar[r] & \resource{disassembly\\listing}}
\seeassembly\seemips\seeobject
}

\providecommand{\mmixasm}{
\toolsection{mmixasm} is an assembler for the MMIX hardware architecture.
It translates assembly code into machine code for MMIX processors and stores it in corresponding object files.
The names of all special registers are predefined and evaluate to the corresponding number.
\flowgraph{\resource{MMIX assembly\\source code} \ar[r] & \toolbox{mmixasm} \ar[r] & \resource{object file}}
\seeassembly\seemmix\seeobject
}

\providecommand{\mmixdism}{
\toolsection{mmixdism} is a disassembler for the MMIX hardware architecture.
It translates machine code from object files targeting MMIX processors into assembly code and writes it to the standard output stream.
\flowgraph{\resource{object file} \ar[r] & \toolbox{mmixdism} \ar[r] & \resource{disassembly\\listing}}
\seeassembly\seemmix\seeobject
}

\providecommand{\orokasm}{
\toolsection{or1kasm} is an assembler for the OpenRISC 1000 hardware architecture.
It translates assembly code into machine code for OpenRISC 1000 processors and stores it in corresponding object files.
\flowgraph{\resource{OpenRISC 1000 assembly\\source code} \ar[r] & \toolbox{or1kasm} \ar[r] & \resource{object file}}
\seeassembly\seeorok\seeobject
}

\providecommand{\orokdism}{
\toolsection{or1kdism} is a disassembler for the OpenRISC 1000 hardware architecture.
It translates machine code from object files targeting OpenRISC 1000 processors into assembly code and writes it to the standard output stream.
\flowgraph{\resource{object file} \ar[r] & \toolbox{or1kdism} \ar[r] & \resource{disassembly\\listing}}
\seeassembly\seeorok\seeobject
}

\providecommand{\ppcaasm}{
\toolsection{ppc32asm} is an assembler for the PowerPC hardware architecture.
It translates assembly code into machine code for PowerPC processors and stores it in corresponding object files.
By default, the assembler generates machine code for the 32-bit operating mode defined by the PowerPC architecture.
\flowgraph{\resource{PowerPC assembly\\source code} \ar[r] & \toolbox{ppc32asm} \ar[r] & \resource{object file}}
\seeassembly\seeppc\seeobject
}

\providecommand{\ppcadism}{
\toolsection{ppc32dism} is a disassembler for the PowerPC hardware architecture.
It translates machine code from object files targeting PowerPC processors into assembly code and writes it to the standard output stream.
It assumes that the machine code was generated for the 32-bit operating mode defined by the PowerPC architecture.
\flowgraph{\resource{object file} \ar[r] & \toolbox{ppc32dism} \ar[r] & \resource{disassembly\\listing}}
\seeassembly\seeppc\seeobject
}

\providecommand{\ppcbasm}{
\toolsection{ppc64asm} is an assembler for the PowerPC hardware architecture.
It translates assembly code into machine code for PowerPC processors and stores it in corresponding object files.
By default, the assembler generates machine code for the 64-bit operating mode defined by the PowerPC architecture.
\flowgraph{\resource{PowerPC assembly\\source code} \ar[r] & \toolbox{ppc64asm} \ar[r] & \resource{object file}}
\seeassembly\seeppc\seeobject
}

\providecommand{\ppcbdism}{
\toolsection{ppc64dism} is a disassembler for the PowerPC hardware architecture.
It translates machine code from object files targeting PowerPC processors into assembly code and writes it to the standard output stream.
It assumes that the machine code was generated for the 64-bit operating mode defined by the PowerPC architecture.
\flowgraph{\resource{object file} \ar[r] & \toolbox{ppc64dism} \ar[r] & \resource{disassembly\\listing}}
\seeassembly\seeppc\seeobject
}

\providecommand{\riscasm}{
\toolsection{riscasm} is an assembler for the RISC hardware architecture.
It translates assembly code into machine code for RISC processors and stores it in corresponding object files.
The names of all special registers are predefined and evaluate to the corresponding number.
\flowgraph{\resource{RISC assembly\\source code} \ar[r] & \toolbox{riscasm} \ar[r] & \resource{object file}}
\seeassembly\seerisc\seeobject
}

\providecommand{\riscdism}{
\toolsection{riscdism} is a disassembler for the RISC hardware architecture.
It translates machine code from object files targeting RISC processors into assembly code and writes it to the standard output stream.
\flowgraph{\resource{object file} \ar[r] & \toolbox{riscdism} \ar[r] & \resource{disassembly\\listing}}
\seeassembly\seerisc\seeobject
}

\providecommand{\wasmasm}{
\toolsection{wasmasm} is an assembler for the WebAssembly architecture.
It translates assembly code into machine code for WebAssembly targets and stores it in corresponding object files.
The names of all special registers are predefined and evaluate to the corresponding number.
\flowgraph{\resource{WebAssembly assembly\\source code} \ar[r] & \toolbox{wasmasm} \ar[r] & \resource{object file}}
\seeassembly\seewasm\seeobject
}

\providecommand{\wasmdism}{
\toolsection{wasmdism} is a disassembler for the WebAssembly architecture.
It translates machine code from object files targeting WebAssembly targets into assembly code and writes it to the standard output stream.
\flowgraph{\resource{object file} \ar[r] & \toolbox{wasmdism} \ar[r] & \resource{disassembly\\listing}}
\seeassembly\seewasm\seeobject
}

% linker tools

\providecommand{\linklib}{
\toolsection{linklib} is an object file combiner.
It creates a static library file by combining all object files given to it into a single one.
\flowgraph{\resource{object files} \ar[r] & \toolbox{linklib} \ar[r] & \resource{library file}}
\seeobject
}

\providecommand{\linkbin}{
\toolsection{linkbin} is a linker for plain binary files.
It links all object files given to it into a single image and stores it in a binary file that begins with the first linked section.
It also creates a map file that lists the address, type, name and size of all used sections.
The filename extension of the resulting binary file can be specified by putting it into a constant data section called \texttt{\_extension}.
\flowgraph{\resource{object files} \ar[r] & \toolbox{linkbin} \ar[r] \ar[d] & \resource{binary file} \\ & \resource{map file}}
\seeobject
}

\providecommand{\linkmem}{
\toolsection{linkmem} is a linker for plain binary files partitioned into random-access and read-only memory.
It links all object files given to it into two distinct images, one for data sections and one for code and constant data sections, and stores each image in a binary file that begins with the first linked section of the corresponding type.
It also creates a map file that lists the address, type, name and size of all used sections.
\flowgraph{\resource{object files} \ar[r] & \toolbox{linkmem} \ar[r] \ar[d] & \resource{RAM file/\\ROM file} \\ & \resource{map file}}
\seeobject
}

\providecommand{\linkprg}{
\toolsection{linkprg} is a linker for GEMDOS executable files.
It links all object files given to it into a single image and stores the image in an Atari GEMDOS executable file~\cite{gemdosfile}.
It also creates a map file that lists the address relative to the text segment, type, name and size of all used sections.
The filename extension of the resulting executable file can be specified by putting it into a constant data section called \texttt{\_extension}.
The GEMDOS executable file format requires all patch patterns of absolute link patches to consist of four full bitmasks with descending offsets.
\flowgraph{\resource{object files} \ar[r] & \toolbox{linkprg} \ar[r] \ar[d] & \resource{executable file} \\ & \resource{map file}}
\seeobject
}

\providecommand{\linkhex}{
\toolsection{linkhex} is a linker for Intel HEX files.
It links all code sections of the object files given to it into single image and stores the image in an Intel HEX file~\cite{hexfile} that begins with the first linked section.
It also creates a map file that lists the address, type, name and size of all used sections.
\flowgraph{\resource{object files} \ar[r] & \toolbox{linkhex} \ar[r] \ar[d] & \resource{HEX file} \\ & \resource{map file}}
\seeobject
}

\providecommand{\mapsearch}{
\toolsection{mapsearch} is a debugging tool.
It searches map files generated by linker tools for the name of a binary section that encompasses a memory address read from the standard input stream.
If additionally provided with one or more object files, it also stores an excerpt thereof in a separate object file called map search result which only contains the identified binary section for disassembling purposes.
\flowgraph{& \resource{map files/\\object files} \ar[d] \\ \resource{memory\\address} \ar[r] & \toolbox{mapsearch} \ar[r] \ar[d] & \resource{section name/\\relative offset} \\ & \resource{object file\\excerpt}}
\seeobject
}

\renewcommand{\seextensa}{}

\startchapter{Xtensa}{Xtensa Hardware Architecture Support}{xtensa}
{This \documentation{} describes how the \ecs{} supports the Xtensa hardware architecture.
This includes information about the assembler, disassembler, and the various compilers featured by the \ecs{} as well as the interoperability between these tools.}

\section{Introduction}

The \ecs{} features various compilers, an assembler, and a disassembler that target the Xtensa hardware architecture by Cadence Design Systems.
Figure~\ref{fig:xtensadataflow} shows the data flow in-between these tools.

\begin{figure}
\flowgraph{
\resource{intermediate\\code} \ar[d] & & \resource{assembly\\source code} \ar[d] \\
\converter{Xtensa\\Generator} \ar[r] \ar[rd] \ar[d] & \resource{assembly\\listing} \ar[r] & \converter{Xtensa\\Assembler} \ar[ld] \\
\resource{debugging\\information} & \resource{object file} \ar[d] \\
& \converter{Xtensa\\Disassembler} \ar[d] \\
& \resource{disassembly\\listing} \\
}\caption{Data flow within the tools targeting the Xtensa architecture}
\label{fig:xtensadataflow}
\end{figure}

All compilers targeting the Xtensa architecture translate their programs using an intermediate code representation.
The Xtensa generator is able to translate the intermediate code representation of a program into machine code for Xtensa processors.
It stores the resulting binary code and data in so-called object files.
Additionally, the generator is able to create an assembly code listing of the machine code for debugging purposes.
This assembly code listing can also be processed by the assembler yielding exactly the same object file.
The disassembler is able to open object files and print a human-readable disassembly listing of their contents.
\seeobject\seecode

\section{Instruction Set}

Tools targeting the Xtensa architecture support the instruction set listed in Table~\ref{tab:xtensaset} and use the same assembly syntax as predefined by Cadence Design Systems~\cite{xtensa:instructionset}.
\seeassembly

\instructionset{xtensa}{Supported Xtensa instruction set}{4}{5}

\section{Calling Convention}\index{Calling convention!of Xtensa}

The machine code generator and runtime support for the Xtensa architecture as provided by the \ecs{} use the following calling convention in order to enable interoperability.

\subsection{Stack Operations}

Arguments for functions as well as the return address are in general passed using the stack according to the intermediate code specification.
See \Documentation{}~\documentationref{code}{Intermediate Code Representation} for more information about the role of the stack.
Function arguments are pushed on the stack in reverse order and cleaned by the caller.

\subsection{Floating-Point Support}

The Xtensa architecture optionally supports floating-point operations.
The generator is able to generate native floating-point operations for processors that do support them.

\subsection{Register Mapping}

The special-purpose registers defined by the intermediate code representation are mapped to their corresponding physical registers in the following way:

\begin{itemize}

\item Result Register\alignright\texttt{\$res}\nopagebreak

The intermediate code result register \texttt{\$res} is mapped to Xtensa registers \texttt{a2}.
64-bit wide results are stored in registers \texttt{a2} and \texttt{a3}.
If supported natively, floating-point results are stored in register \texttt{f0}.

\item Stack Pointer Register\alignright\texttt{\$sp}\nopagebreak

The intermediate code stack pointer register \texttt{\$sp} is mapped to Xtensa register \texttt{a1}.

\item Frame Pointer Register\alignright\texttt{\$fp}\nopagebreak

The intermediate code frame pointer register \texttt{\$fp} is mapped to Xtensa register \texttt{a15}.

\item Link Register\alignright\texttt{\$lnk}\nopagebreak

The intermediate code link register \texttt{\$lnk} is supported and mapped to Xtensa register \texttt{a0}.

\end{itemize}

All other intermediate code registers are mapped as needed to the remaining physical registers.
Their contents and mapping are therefore considered volatile across function calls.

\section{Runtime Support}\index{Runtime support!for Xtensa}

The \ecs{} provides runtime support for the Xtensa architecture and runtime environments based on this hardware architecture in object files.
Users targeting a specific runtime environment have to use an appropriate linker together with these object files in order create an executable program.
This section gives information about all supported runtime environments based on the Xtensa hardware architecture as well as the required combination of linker and object files.

Basic architectural runtime support is provided by the object file \objfile{xtensa\-run}.
Users should always include this object file during linking regardless of the actual target runtime environment.
All other object files given to the linker should target the same hardware architecture.

Programs written in \cpp{} need additional runtime support stored in the \libfile{cpp\-xtensa\-run} library file.
Programs written in Oberon need additional runtime support stored in the \libfile{ob\-xtensa\-run} library file.
\seecpp\seeoberon

Programs targeting Linux-based operating systems are created using the \tool{link\-bin} linker tool.
It creates Executable and Linking Format (ELF) files~\cite{elffile} if provided with the runtime support stored in the \objfile{xtensa\-linux\-run} object file.
Calling the \tool{ecsd} utility tool using the \environment{xtensa\-linux} target environment achieves the same result.

\section{Xtensa Tools}

The \ecs{} provides the following tools that are able to process object files targeting the Xtensa hardware architecture.
\interface

\cdxtensa
\cppxtensa
\falxtensa
\obxtensa
\xtensaasm
\xtensadism
\linkbin

\concludechapter


\part{\ecs{} Internals}
% Object file representation
% Copyright (C) Florian Negele

% This file is part of the Eigen Compiler Suite.

% Permission is granted to copy, distribute and/or modify this document
% under the terms of the GNU Free Documentation License, Version 1.3
% or any later version published by the Free Software Foundation.

% You should have received a copy of the GNU Free Documentation License
% along with the ECS.  If not, see <https://www.gnu.org/licenses/>.

% Generic documentation utilities
% Copyright (C) Florian Negele

% This file is part of the Eigen Compiler Suite.

% Permission is granted to copy, distribute and/or modify this document
% under the terms of the GNU Free Documentation License, Version 1.3
% or any later version published by the Free Software Foundation.

% You should have received a copy of the GNU Free Documentation License
% along with the ECS.  If not, see <https://www.gnu.org/licenses/>.

\providecommand{\cpp}{C\texttt{++}}
\providecommand{\opt}{_\mathit{opt}}
\providecommand{\tool}[1]{\texttt{#1}}
\providecommand{\version}{Version 0.0.40}
\providecommand{\resource}[1]{*++\txt{#1}}
\providecommand{\ecs}{Eigen Compiler Suite}
\providecommand{\changed}[1]{\underline{#1}}
\providecommand{\toolbox}[1]{\converter{#1}}
\providecommand{\file}{}\renewcommand{\file}[1]{\texttt{#1}}
\providecommand{\alignright}{\hfill\linebreak[0]\hspace*{\fill}}
\providecommand{\converter}[1]{*++[F][F*:white][F,:gray]\txt{#1}}
\providecommand{\documentation}{\ifbook chapter\else document\fi}
\providecommand{\Documentation}{\ifbook Chapter\else Document\fi}
\providecommand{\variable}[1]{\resource{\texttt{\small#1}\\variable}}
\providecommand{\documentationref}[2]{\ifbook\ref{#1}\else``\href{#1}{#2}''~\cite{#1}\fi}
\providecommand{\objfile}[1]{\texttt{#1}\index[runtime]{#1 object file@\texttt{#1} object file}}
\providecommand{\libfile}[1]{\texttt{#1}\index[runtime]{#1 library file@\texttt{#1} library file}}
\providecommand{\epigraph}[2]{\ifbook\begin{quote}\flushright\textit{#1}\par--- #2\end{quote}\fi}
\providecommand{\environmentvariable}[1]{\texttt{#1}\index{Environment variables!#1@\texttt{#1}}}
\providecommand{\environment}[1]{\texttt{#1}\index[environment]{#1 environment@\texttt{#1} environment}}
\providecommand{\toolsection}{}\renewcommand{\toolsection}[1]{\subsection{#1}\label{\prefix:#1}\tool{#1}}
\providecommand{\instruction}{}\renewcommand{\instruction}[2]{\noindent\qquad\pdftooltip{\texttt{#1}}{#2}\refstepcounter{instruction}\par}
\providecommand{\flowgraph}{}\renewcommand{\flowgraph}[1]{\par\sffamily\begin{displaymath}\xymatrix@=4ex{#1}\end{displaymath}\normalfont\par}
\providecommand{\instructionset}{}\renewcommand{\instructionset}[4]{\setcounter{instruction}{0}\begin{multicols}{\ifbook#3\else#4\fi}[{\captionof{table}[#2]{#2 (\ref*{#1:instructions}~instructions)}\label{tab:#1set}\vspace{-2ex}}]\footnotesize\raggedcolumns\input{#1.set}\label{#1:instructions}\end{multicols}}

\providecommand{\gpl}{GNU General Public License}
\providecommand{\rse}{ECS Runtime Support Exception}
\providecommand{\fdl}{\href{https://www.gnu.org/licenses/fdl.html}{GNU Free Documentation License}}

\providecommand{\docbegin}{}
\providecommand{\docend}{}
\providecommand{\doclabel}[1]{\hypertarget{#1}}
\providecommand{\doclink}[2]{\hyperlink{#1}{#2}}
\providecommand{\docsection}[3]{\hypertarget{#1}{\subsection{#2}}\label{sec:#1}\index[library]{#2@#3}}
\providecommand{\docsectionstar}[1]{}
\providecommand{\docsubbegin}{\begin{description}}
\providecommand{\docsubend}{\end{description}}
\providecommand{\docsubsection}[3]{\item[\hypertarget{#1}{#2}]\index[library]{#2@#3}}
\providecommand{\docsubsectionstar}[1]{\smallskip}
\providecommand{\docsubsubsection}[3]{\docsubsection{#1}{#2}{#3}}
\providecommand{\docsubsubsectionstar}[1]{}
\providecommand{\docsubsubsubsection}[3]{}
\providecommand{\docsubsubsubsectionstar}[1]{}
\providecommand{\doctable}{}

\providecommand{\debuggingtool}{}\renewcommand{\debuggingtool}{This tool is provided for debugging purposes.
It allows exposing and modifying an internal data structure that is usually not accessible.
}

\providecommand{\interface}{All tools accept command-line arguments which are taken as names of plain text files containing the source code.
If no arguments are provided, the standard input stream is used instead.
Output files are generated in the current working directory and have the same name as the input file being processed whereas the filename extension gets replaced by an appropriate suffix.
\seeinterface
}

\providecommand{\license}{\noindent Copyright \copyright{} Florian Negele\par\medskip\noindent
Permission is granted to copy, distribute and/or modify this document under the terms of the
\fdl{}, Version 1.3 or any later version published by the \href{https://fsf.org/}{Free Software Foundation}.
}

\providecommand{\ecslogosurface}{
\fill[darkgray] (0,0,0) -- (0,0,3) -- (0,3,3) -- (0,3,1) -- (0,4,1) -- (0,4,3) -- (0,5,3) -- (0,5,0) -- (0,2,0) -- (0,2,2) -- (0,1,2) -- (0,1,0) -- cycle;
\fill[gray] (0,5,0) -- (0,5,3) -- (1,5,3) -- (1,5,1) -- (2,5,1) -- (2,5,3) -- (3,5,3) -- (3,5,0) -- cycle;
\fill[lightgray] (0,0,0) -- (0,1,0) -- (2,1,0) -- (2,4,0) -- (1,4,0) -- (1,3,0) -- (2,3,0) -- (2,2,0) -- (0,2,0) -- (0,5,0) -- (3,5,0) -- (3,0,0) -- cycle;
\begin{scope}[line width=0.5]
\begin{scope}[gray]
\draw (0,0,0) -- (0,1,0);
\draw (2,1,0) -- (2,2,0);
\draw (0,1,2) -- (0,2,2);
\draw (0,2,0) -- (0,5,0);
\draw (2,3,0) -- (2,4,0);
\end{scope}
\begin{scope}[lightgray]
\draw (0,1,0) -- (0,1,2);
\draw (0,3,1) -- (0,3,3);
\draw (0,5,0) -- (0,5,3);
\draw (2,5,1) -- (2,5,3);
\end{scope}
\begin{scope}[white]
\draw (0,1,0) -- (2,1,0);
\draw (1,3,0) -- (2,3,0);
\draw (0,5,0) -- (3,5,0);
\end{scope}
\end{scope}
}

\providecommand{\ecslogo}[1]{
\begin{tikzpicture}[scale={(#1)/((sin(45)+cos(45))*3cm)},x={({-cos(45)*1cm},{sin(45)*sin(30)*1cm})},y={({0cm},{(cos(30)*1cm})},z={({sin(45)*1cm},{cos(45)*sin(30)*1cm})}]
\begin{scope}[darkgray,line width=1]
\draw (0,0,0) -- (0,0,3) -- (0,3,3) -- (2,3,3) -- (2,5,3) -- (3,5,3) -- (3,5,0) -- (3,0,0) -- cycle;
\draw (0,3,1) -- (0,4,1) -- (0,4,3) -- (0,5,3) -- (1,5,3) -- (1,5,1) -- (2,5,1);
\draw (1,3,0) -- (1,4,0) -- (2,4,0);
\end{scope}
\fill[darkgray] (2,0,0) -- (2,0,3) -- (2,5,3) -- (2,5,1) -- (2,4,1) -- (2,4,0) -- cycle;
\fill[lightgray] (2,0,2) -- (0,0,2) -- (0,2,2) -- (2,2,2) -- cycle;
\fill[gray] (0,1,0) -- (2,1,0) -- (2,1,2) -- (0,1,2) -- cycle;
\fill[gray] (0,3,1) -- (0,3,3) -- (2,3,3) -- (2,3,0) -- (1,3,0) -- (1,3,1) -- cycle;
\ecslogosurface
\end{tikzpicture}
}

\providecommand{\shadowedecslogo}[3]{
\begin{tikzpicture}[scale={(#1)/((sin(#2)+cos(#2))*3cm)},x={({-cos(#2)*1cm},{sin(#2)*sin(#3)*1cm})},y={({0cm},{(cos(#3)*1cm})},z={({sin(#2)*1cm},{cos(#2)*sin(#3)*1cm})}]
\shade[top color=lightgray!50!white,bottom color=white,middle color=lightgray!50!white] (0,0,0) -- (3,0,0) -- (3,{-0.5-3*sin(#2)*sin(#3)/cos(#3)},0) -- (0,-0.5,0) -- cycle;
\shade[top color=darkgray!50!gray,bottom color=white,middle color=darkgray!50!white] (0,0,0) -- (0,0,3) -- (0,{-0.5-3*cos(#2)*sin(#3)/cos(#3)},3) -- (0,-0.5,0) -- cycle;
\begin{scope}[y={({(cos(#2)+sin(#2))*0.5cm},{(cos(#2)*sin(#3)-sin(#2)*sin(#3))*0.5cm})}]
\useasboundingbox (3,0,0) -- (0,0,0) -- (0,0,3);
\shade[left color=darkgray!80!black,right color=lightgray,middle color=gray] (0,0,0) -- (0,1,0) -- (0,1,0.5) -- (0,2,0) -- (0,5,0) -- (0,5,3) -- (1,5,3) -- (1,4,3) -- (1,4,2.5) -- (1,3,3) -- (2,5,3) -- (3,5,3) -- (3,0,3) -- cycle;
\clip (0,0,0) -- (0,0,3) -- ({-3*sin(#2)/cos(#2)},0,0) -- cycle;
\shade[left color=darkgray,right color=lightgray!50!gray] (0,0,0) -- (0,1,0) -- (0,1,0.5) -- (0,2,0) -- (0,5,0) -- (0,5,3) -- (1,5,3) -- (1,4,3) -- (1,4,2.5) -- (1,3,3) -- (2,5,3) -- (3,5,3) -- (3,0,3) -- cycle;
\end{scope}
\shade[left color=darkgray,right color=darkgray!80!black] (2,0,0) -- (2,0,3) -- (2,5,3) -- (2,5,1) -- (2,4,1) -- (2,4,0) -- cycle;
\shade[left color=darkgray!90!black,right color=gray!80!darkgray] (2,0,2) -- (0,0,2) -- (0,2,2) -- (2,2,2) -- cycle;
\shade[top color=darkgray!90!black,bottom color=gray!80!darkgray] (0,1,0) -- (2,1,0) -- (2,1,2) -- (0,1,2) -- cycle;
\shade[top color=darkgray!90!black,bottom color=gray!80!darkgray] (0,3,1) -- (0,3,3) -- (2,3,3) -- (2,3,0) -- (1,3,0) -- (1,3,1) -- cycle;
\fill[gray] (2,1,0) -- (1.5,1,0.5) -- (0,1,0.5) -- (0,1,0) -- cycle;
\fill[gray] (1,3,2) -- (0.5,3,2) -- (0.5,3,3) -- (1,3,3) -- cycle;
\fill[gray] (2,3,0) -- (1.5,3,0.5) -- (1,3,0.5) -- (1,3,0) -- cycle;
\ecslogosurface
\end{tikzpicture}
}

\providecommand{\cpplogo}[1]{
\begin{tikzpicture}[scale=(#1)/512em]
\fill[gray] (435.2794,398.7159) -- (247.1911,507.3075) .. controls (236.3563,513.5642) and (218.6240,513.5642) .. (207.7892,507.3075) -- (19.7009,398.7159) .. controls (8.8646,392.4606) and (0.0000,377.1043) .. (0.0000,364.5924) -- (0.0000,147.4076) .. controls (0.8430,132.8363) and (8.2856,120.7683) .. (19.7009,113.2842) -- (207.7892,4.6926) .. controls (218.6240,-1.5642) and (236.3564,-1.5642) .. (247.1911,4.6926) -- (435.2794,113.2842) .. controls (447.5273,121.4304) and (454.4987,133.6918) .. (454.9803,147.4076) -- (454.9803,364.5924) .. controls (454.5404,377.7571) and (446.6566,391.0351) .. (435.2794,398.7159) -- cycle(75.8301,255.9993) .. controls (74.9389,404.0881) and (273.2892,469.4783) .. (358.8263,331.8769) -- (293.1917,293.8965) .. controls (253.5702,359.4301) and (155.1909,335.9977) .. (151.6601,255.9993) .. controls (152.7204,182.2703) and (249.4137,148.0211) .. (293.1961,218.1065) -- (358.8308,180.1276) .. controls (283.4477,49.2645) and (79.6318,96.3470) .. (75.8301,255.9993) -- cycle(379.1503,247.5747) -- (362.2982,247.5747) -- (362.2982,230.7226) -- (345.4490,230.7226) -- (345.4490,247.5747) -- (328.5969,247.5747) -- (328.5969,264.4254) -- (345.4490,264.4254) -- (345.4490,281.2759) -- (362.2982,281.2759) -- (362.2982,264.4254) -- (379.1503,264.4254) -- cycle(442.3420,247.5747) -- (425.4899,247.5747) -- (425.4899,230.7226) -- (408.6408,230.7226) -- (408.6408,247.5747) -- (391.7886,247.5747) -- (391.7886,264.4254) -- (408.6408,264.4254) -- (408.6408,281.2759) -- (425.4899,281.2759) -- (425.4899,264.4254) -- (442.3420,264.4254) -- cycle;
\end{tikzpicture}
}

\providecommand{\fallogo}[1]{
\begin{tikzpicture}[scale=(#1)/512em]
\fill[gray] (185.7774,0.0000) .. controls (200.4486,15.9798) and (226.8966,8.7148) .. (235.0426,31.5836) .. controls (249.5297,58.0598) and (247.9581,97.9161) .. (280.3335,110.9762) .. controls (309.1690,120.3496) and (337.8406,104.2727) .. (366.5753,103.9379) .. controls (373.4449,111.5171) and (379.2885,128.2574) .. (383.9755,108.9744) .. controls (396.6979,102.5615) and (437.2808,107.6681) .. (426.9652,124.3252) .. controls (408.9822,121.0785) and (412.4742,146.0729) .. (426.5192,131.4996) .. controls (433.8413,120.8489) and (465.1541,126.5522) .. (441.9067,135.7950) .. controls (396.1879,157.7478) and (344.1112,161.5079) .. (298.5528,183.5702) .. controls (277.7471,193.5198) and (284.6941,218.7163) .. (285.2127,236.9640) .. controls (292.3599,316.2826) and (307.3929,394.6311) .. (317.1198,473.6154) .. controls (329.0637,505.4736) and (292.1195,528.5004) .. (265.9183,511.2761) .. controls (237.9284,499.2462) and (237.3684,465.2681) .. (230.9102,439.9421) .. controls (218.6692,374.3397) and (215.6307,306.9662) .. (198.1732,242.3977) .. controls (183.1379,232.7444) and (164.4245,256.0298) .. (149.0430,261.4799) .. controls (116.9328,279.2585) and (87.1822,308.5851) .. (48.2293,307.8914) .. controls (21.3220,306.9037) and (-15.9107,281.8761) .. (7.2921,252.7908) .. controls (29.7799,220.6177) and (67.5177,204.3028) .. (100.9287,185.9449) .. controls (130.8217,170.8906) and (161.1548,156.5903) .. (191.0278,141.5847) .. controls (196.1738,120.0520) and (186.6049,95.2409) .. (186.8382,72.4353) .. controls (185.5234,48.4204) and (183.1700,23.9341) .. (185.7774,0.0000) -- cycle;
\end{tikzpicture}
}

\providecommand{\oblogo}[1]{
\begin{tikzpicture}[scale=(#1)/512em]
\fill[gray] (160.3865,208.9117) .. controls (154.0879,214.6478) and (149.0735,221.2409) .. (145.4125,228.5384) .. controls (184.8790,248.4273) and (234.7122,269.8787) .. (297.5493,291.8782) .. controls (300.3943,281.4769) and (300.9552,268.7619) .. (300.4023,255.2389) .. controls (248.9909,244.7891) and (200.0310,225.9279) .. (160.3865,208.9117) -- cycle(225.7398,392.6996) .. controls (308.0209,392.1716) and (359.3326,345.9277) .. (368.7203,285.2098) .. controls (376.6742,197.1784) and (311.7194,141.3342) .. (205.4287,142.1456) .. controls (139.9485,141.4804) and (88.7155,166.1957) .. (73.5775,228.0086) .. controls (52.0297,320.3408) and (123.4078,391.0103) .. (225.7398,392.6996) -- cycle(216.0739,176.4733) .. controls (268.9183,179.2424) and (315.8292,206.5488) .. (312.7454,265.1139) .. controls (313.2769,315.6384) and (286.5993,353.4946) .. (216.6040,355.7934) .. controls (162.4657,355.7934) and (126.0914,317.5023) .. (126.0914,260.5103) .. controls (126.1733,214.2900) and (163.3363,176.2849) .. (216.0739,176.4733) -- cycle(76.4897,189.1754) .. controls (13.1586,147.5631) and (0.0000,119.4207) .. (0.0000,119.4207) -- (90.6499,170.1632) .. controls (85.3004,175.8497) and (80.5994,182.1633) .. (76.4897,189.1754) -- cycle(353.9486,119.3004) -- (402.9482,119.3004) .. controls (427.0025,137.0797) and (450.9893,162.7034) .. (474.9529,191.0213) .. controls (509.3540,228.5339) and (531.3391,294.2091) .. (487.8149,312.1206) .. controls (462.8165,324.7652) and (394.3874,316.8943) .. (373.8912,313.6651) .. controls (379.9291,297.7449) and (383.2899,278.4204) .. (381.4989,257.7214) .. controls (420.3069,248.0321) and (421.9610,218.3461) .. (407.7867,192.6417) .. controls (391.1113,162.4018) and (370.1114,132.9097) .. (353.9486,119.3004) -- cycle;
\end{tikzpicture}
}

\providecommand{\markuptable}{
\begin{table}
\sffamily\centering
\begin{tabular}{@{}lcl@{}}
\toprule
\texttt{//italics//} & $\rightarrow$ & \textit{italics} \\
\midrule
\texttt{**bold**} & $\rightarrow$ & \textbf{bold} \\
\midrule
\texttt{\# ordered list} & & 1 ordered list \\
\texttt{\# second item} & $\rightarrow$ & 2 second item \\
\texttt{\#\# sub item} & & \hspace{1em} 1 sub item \\
\midrule
\texttt{* unordered list} & & $\bullet$ unordered list \\
\texttt{* second item} & $\rightarrow$ & $\bullet$ second item \\
\texttt{** sub item} & & \hspace{1em} $\bullet$ sub item \\
\midrule
\texttt{link to [[label]]} & $\rightarrow$ & link to \underline{label} \\
\midrule
\texttt{<{}<label>{}> definition } & $\rightarrow$ & definition \\
\midrule
\texttt{[[url|link name]]} & $\rightarrow$ & \underline{link name} \\
\midrule\addlinespace
\texttt{= large heading} & & {\Large large heading} \smallskip \\
\texttt{== medium heading} & $\rightarrow$ & {\large medium heading} \\
\texttt{=== small heading} & & small heading \\
\midrule
\texttt{no line break} & & no line break for paragraphs \\
\texttt{for paragraphs} & $\rightarrow$ \\
& & use empty line \\
\texttt{use empty line} \\
\midrule
\texttt{force\textbackslash\textbackslash line break} & $\rightarrow$ & force \\
& & line break \\
\midrule
\texttt{horizontal line} & $\rightarrow$ & horizontal line \\
\texttt{----} & & \hrulefill \\
\midrule
\texttt{|=a|=table|=header} & & \underline{a \enspace table \enspace header} \\
\texttt{|a|table|row} & $\rightarrow$ & a \enspace table \enspace row \\
\texttt{|b|table|row} & & b \enspace table \enspace row \\
\midrule
\texttt{\{\{\{} \\
\texttt{unformatted} & $\rightarrow$ & \texttt{unformatted} \\
\texttt{code} & & \texttt{code} \\
\texttt{\}\}\}} \\
\midrule\addlinespace
\texttt{@ new article} & & {\Large 1.\ new article} \smallskip \\
\texttt{@ second article} & $\rightarrow$ & {\Large 2.\ second article} \smallskip \\
\texttt{@@ sub article} & & {\large 2.1.\ sub article} \\
\bottomrule
\end{tabular}
\normalfont\caption{Elements of the generic documentation markup language}
\label{tab:docmarkup}
\end{table}
}

\providecommand{\startchapter}[4]{
\documentclass[11pt,a4paper]{article}
\usepackage{booktabs}
\usepackage[format=hang,labelfont=bf]{caption}
\usepackage{changepage}
\usepackage[T1]{fontenc}
\usepackage[margin=2cm]{geometry}
\usepackage{hyperref}
\usepackage[american]{isodate}
\usepackage{lmodern}
\usepackage{longtable}
\usepackage{mathptmx}
\usepackage{microtype}
\usepackage[toc]{multitoc}
\usepackage{multirow}
\usepackage[all]{nowidow}
\usepackage{pdfcomment}
\usepackage{syntax}
\usepackage{tikz}
\usepackage[all]{xy}
\hypersetup{pdfborder={0 0 0},bookmarksnumbered=true,pdftitle={\ecs{}: #2},pdfauthor={Florian Negele},pdfsubject={\ecs{}},pdfkeywords={#1}}
\setlength{\grammarindent}{8em}\setlength{\grammarparsep}{0.2ex}
\setlength{\columnsep}{2em}
\newcommand{\prefix}{}
\newcounter{instruction}
\bibliographystyle{unsrt}
\renewcommand{\index}[2][]{}
\renewcommand{\arraystretch}{1.05}
\renewcommand{\floatpagefraction}{0.7}
\renewcommand{\syntleft}{\itshape}\renewcommand{\syntright}{}
\title{\vspace{-5ex}\Huge{\ecs{}}\medskip\hrule}
\author{\huge{#2}}
\date{\medskip\version}
\newif\ifbook\bookfalse
\pagestyle{headings}
\frenchspacing
\begin{document}
\maketitle\thispagestyle{empty}\noindent#4\setlength{\columnseprule}{0.4pt}\tableofcontents\setlength{\columnseprule}{0pt}\vfill\pagebreak[3]\null\vfill\bigskip\noindent
\parbox{\textwidth-4em}{\license The contents of this \documentation{} are part of the \href{manual}{\ecs{} User Manual}~\cite{manual} and correspond to Chapter ``\href{manual\##3}{#1}''.\alignright\mbox{\today}}
\parbox{4em}{\flushright\ecslogo{3em}}
\clearpage
}

\providecommand{\concludechapter}{
\vfill\pagebreak[3]\null\vfill
\thispagestyle{myheadings}\markright{REFERENCES}
\noindent\begin{minipage}{\textwidth}\begin{multicols}{2}[\section*{References}]
\renewcommand{\section}[2]{}\small\bibliography{references}
\end{multicols}\end{minipage}\end{document}
}

\providecommand{\startpresentation}[2]{
\documentclass[14pt,aspectratio=43,usepdftitle=false]{beamer}
\usepackage{booktabs}
\usepackage{etex}
\usepackage{multicol}
\usepackage{tikz}
\usepackage[all]{xy}
\bibliographystyle{unsrt}
\setlength{\columnsep}{1em}
\setlength{\leftmargini}{1em}
\setbeamercolor{title}{fg=black}
\setbeamercolor{structure}{fg=darkgray}
\setbeamercolor{bibliography item}{fg=darkgray}
\setbeamerfont{title}{series=\bfseries}
\setbeamerfont{subtitle}{series=\normalfont}
\setbeamerfont*{frametitle}{parent=title}
\setbeamerfont{block title}{series=\bfseries}
\setbeamerfont*{framesubtitle}{parent=subtitle}
\setbeamersize{text margin left=1em,text margin right=1em}
\setbeamertemplate{navigation symbols}{}
\setbeamertemplate{itemize item}[circle]{}
\setbeamertemplate{bibliography item}[triangle]{}
\setbeamertemplate{bibliography entry author}{\usebeamercolor[fg]{bibliography item}}
\setbeamertemplate{frametitle}{\medskip\usebeamerfont{frametitle}\color{gray}\raisebox{-2.5ex}[0ex][0ex]{\rule{0.1em}{4.5ex}}}
\addtobeamertemplate{frametitle}{}{\hspace{0.4em}\usebeamercolor[fg]{title}\insertframetitle\par\vspace{0.2ex}\hspace{0.5em}\usebeamerfont{framesubtitle}\insertframesubtitle}
\hypersetup{pdfborder={0 0 0},bookmarksnumbered=true,bookmarksopen=true,bookmarksopenlevel=0,pdftitle={\ecs{}: #1},pdfauthor={Florian Negele},pdfsubject={\ecs{}},pdfkeywords={#1}}
\renewcommand{\flowgraph}[1]{\resizebox{\textwidth}{!}{$$\xymatrix{##1}$$}}
\title{\ecs{}\medskip\hrule\medskip}
\institute{\shadowedecslogo{5em}{30}{15}}
\date{\version}
\subtitle{#1}
\begin{document}
\begin{frame}[plain]\titlepage\nocite{manual}\end{frame}
\begin{frame}{Contents}{#1}\begin{center}\tableofcontents\end{center}\end{frame}
}

\providecommand{\concludepresentation}{
\begin{frame}{References}\begin{footnotesize}\setlength{\columnseprule}{0.4pt}\begin{multicols}{2}\bibliography{references}\end{multicols}\end{footnotesize}\end{frame}
\end{document}
}

\providecommand{\startbook}[1]{
\documentclass[10pt,paper=17cm:24cm,DIV=13,twoside=semi,headings=normal,numbers=noendperiod,cleardoublepage=plain]{scrbook}
\usepackage{atveryend}
\usepackage{booktabs}
\usepackage{caption}
\usepackage{changepage}
\usepackage[T1]{fontenc}
\usepackage{imakeidx}
\usepackage{hyperref}
\usepackage[american]{isodate}
\usepackage{lmodern}
\usepackage{longtable}
\usepackage{mathptmx}
\usepackage[final]{microtype}
\usepackage{multicol}
\usepackage{multirow}
\usepackage[all]{nowidow}
\usepackage{pdfcomment}
\usepackage{scrlayer-scrpage}
\usepackage{setspace}
\usepackage{syntax}
\usepackage[eventxtindent=4pt,oddtxtexdent=4pt]{thumbs}
\usepackage{tikz}
\usepackage[all]{xy}
\hyphenation{Micro-Blaze Open-Cores Open-RISC Power-PC}
\hypersetup{pdfborder={0 0 0},bookmarksnumbered=true,bookmarksopen=true,bookmarksopenlevel=0,pdftitle={\ecs{}: #1},pdfauthor={Florian Negele},pdfsubject={\ecs{}},pdfkeywords={#1}}
\setlength{\grammarindent}{8em}\setlength{\grammarparsep}{0.7ex}
\setkomafont{captionlabel}{\usekomafont{descriptionlabel}}
\renewcommand{\arraystretch}{1.05}\setstretch{1.1}
\renewcommand{\chapterformat}{\thechapter\autodot\enskip\raisebox{-1ex}[0ex][0ex]{\color{gray}\rule{0.1em}{3.5ex}}\enskip}
\renewcommand{\startchapter}[4]{\hypertarget{##3}{\chapter{##1}}\label{##3}##4\addthumb{##1}{\LARGE\sffamily\bfseries\thechapter}{white}{gray}\renewcommand{\prefix}{##3}}
\renewcommand{\concludechapter}{\clearpage{\stopthumb\cleardoublepage}}
\renewcommand{\syntleft}{\itshape}\renewcommand{\syntright}{}
\renewcommand{\floatpagefraction}{0.7}
\renewcommand{\partheademptypage}{}
\DeclareMicrotypeAlias{lmss}{cmr}
\newcommand{\prefix}{}
\newcounter{instruction}
\bibliographystyle{unsrt}
\newif\ifbook\booktrue
\makeindex[intoc,title=Index]
\makeindex[intoc,name=tools,title=Index of Tools,columns=3]
\makeindex[intoc,name=library,title=Index of Library Names]
\makeindex[intoc,name=runtime,title=Index of Runtime Support]
\makeindex[intoc,name=environment,title=Index of Target Environments]
\indexsetup{toclevel=chapter,headers={\indexname}{\indexname}}
\frenchspacing
\begin{document}
\pagenumbering{alph}
\begin{titlepage}\centering
\huge\sffamily\null\vfill\textbf{\ecs{}}\bigskip\hrule\bigskip#1
\normalsize\normalfont\vfill\vfill\shadowedecslogo{10em}{30}{15}
\large\vfill\vfill\version
\end{titlepage}
\null\vfill
\thispagestyle{empty}
\noindent\today\par\medskip
\license A copy of this license is included in Appendix~\ref{fdl} on page~\pageref{fdl}.
All product names used herein are for identification purposes only and may be trademarks of their respective companies.
\concludechapter
\frontmatter
\setcounter{tocdepth}{1}
\tableofcontents
\setcounter{tocdepth}{2}
\concludechapter
\listoffigures
\concludechapter
\listoftables
\concludechapter
}

\providecommand{\concludebook}{
\backmatter
\addtocontents{toc}{\protect\setcounter{tocdepth}{-1}}
\phantomsection\addcontentsline{toc}{part}{Bibliography}
\bibliography{references}
\concludechapter
\phantomsection\addcontentsline{toc}{part}{Indexes}
\printindex
\concludechapter
\indexprologue{\label{idx:tools}}
\printindex[tools]
\concludechapter
\printindex[library]
\concludechapter
\indexprologue{\label{idx:runtime}}
\printindex[runtime]
\concludechapter
\indexprologue{\label{idx:environment}}
\printindex[environment]
\concludechapter
\pagestyle{empty}\pagenumbering{Alph}\null\clearpage
\null\vfill\centering\ecslogo{4em}\par\medskip\license
\end{document}
}

% chapter references

\providecommand{\seedocumentationref}{}\renewcommand{\seedocumentationref}[3]{#1, see \Documentation{}~\documentationref{#2}{#3}. }
\providecommand{\seeinterface}{}\renewcommand{\seeinterface}{\ifbook See \Documentation{}~\documentationref{interface}{User Interface} for more information about the common user interface of all of these tools. \fi}
\providecommand{\seeguide}{}\renewcommand{\seeguide}{\seedocumentationref{For basic examples of using some of these tools in practice}{guide}{User Guide}}
\providecommand{\seecpp}{}\renewcommand{\seecpp}{\seedocumentationref{For more information about the \cpp{} programming language and its implementation by the \ecs{}}{cpp}{User Manual for \cpp{}}}
\providecommand{\seefalse}{}\renewcommand{\seefalse}{\seedocumentationref{For more information about the FALSE programming language and its implementation by the \ecs{}}{false}{User Manual for FALSE}}
\providecommand{\seeoberon}{}\renewcommand{\seeoberon}{\seedocumentationref{For more information about the Oberon programming language and its implementation by the \ecs{}}{oberon}{User Manual for Oberon}}
\providecommand{\seeassembly}{}\renewcommand{\seeassembly}{\seedocumentationref{For more information about the generic assembly language and how to use it}{assembly}{Generic Assembly Language Specification}}
\providecommand{\seeamd}{}\renewcommand{\seeamd}{\seedocumentationref{For more information about how the \ecs{} supports the AMD64 hardware architecture}{amd64}{AMD64 Hardware Architecture Support}}
\providecommand{\seearm}{}\renewcommand{\seearm}{\seedocumentationref{For more information about how the \ecs{} supports the ARM hardware architecture}{arm}{ARM Hardware Architecture Support}}
\providecommand{\seeavr}{}\renewcommand{\seeavr}{\seedocumentationref{For more information about how the \ecs{} supports the AVR hardware architecture}{avr}{AVR Hardware Architecture Support}}
\providecommand{\seeavrtt}{}\renewcommand{\seeavrtt}{\seedocumentationref{For more information about how the \ecs{} supports the AVR32 hardware architecture}{avr32}{AVR32 Hardware Architecture Support}}
\providecommand{\seemabk}{}\renewcommand{\seemabk}{\seedocumentationref{For more information about how the \ecs{} supports the M68000 hardware architecture}{m68k}{M68000 Hardware Architecture Support}}
\providecommand{\seemibl}{}\renewcommand{\seemibl}{\seedocumentationref{For more information about how the \ecs{} supports the MicroBlaze hardware architecture}{mibl}{MicroBlaze Hardware Architecture Support}}
\providecommand{\seemips}{}\renewcommand{\seemips}{\seedocumentationref{For more information about how the \ecs{} supports the MIPS32 and MIPS64 hardware architectures}{mips}{MIPS Hardware Architecture Support}}
\providecommand{\seemmix}{}\renewcommand{\seemmix}{\seedocumentationref{For more information about how the \ecs{} supports the MMIX hardware architecture}{mmix}{MMIX Hardware Architecture Support}}
\providecommand{\seeorok}{}\renewcommand{\seeorok}{\seedocumentationref{For more information about how the \ecs{} supports the OpenRISC 1000 hardware architecture}{or1k}{OpenRISC 1000 Hardware Architecture Support}}
\providecommand{\seeppc}{}\renewcommand{\seeppc}{\seedocumentationref{For more information about how the \ecs{} supports the PowerPC hardware architecture}{ppc}{PowerPC Hardware Architecture Support}}
\providecommand{\seerisc}{}\renewcommand{\seerisc}{\seedocumentationref{For more information about how the \ecs{} supports the RISC hardware architecture}{risc}{RISC Hardware Architecture Support}}
\providecommand{\seewasm}{}\renewcommand{\seewasm}{\seedocumentationref{For more information about how the \ecs{} supports the WebAssembly architecture}{wasm}{WebAssembly Architecture Support}}
\providecommand{\seedocumentation}{}\renewcommand{\seedocumentation}{\seedocumentationref{For more information about generic documentations and their generation by the \ecs{}}{documentation}{Generic Documentation Generation}}
\providecommand{\seedebugging}{}\renewcommand{\seedebugging}{\seedocumentationref{For more information about debugging information and its representation}{debugging}{Debugging Information Representation}}
\providecommand{\seecode}{}\renewcommand{\seecode}{\seedocumentationref{For more information about intermediate code and its purpose}{code}{Intermediate Code Representation}}
\providecommand{\seeobject}{}\renewcommand{\seeobject}{\seedocumentationref{For more information about object files and their purpose}{object}{Object File Representation}}

% generic documentation tools

\providecommand{\docprint}{
\toolsection{docprint} is a pretty printer for generic documentations.
It reformats generic documentations and writes it to the standard output stream.
\debuggingtool
\flowgraph{\resource{generic\\documentation} \ar[r] & \toolbox{docprint} \ar[r] & \resource{generic\\documentation}}
\seedocumentation
}

\providecommand{\doccheck}{
\toolsection{doccheck} is a syntactic and semantic checker for generic documentations.
It just performs syntactic and semantic checks on generic documentations and writes its diagnostic messages to the standard error stream.
\debuggingtool
\flowgraph{\resource{generic\\documentation} \ar[r] & \toolbox{doccheck} \ar[r] & \resource{diagnostic\\messages}}
\seedocumentation
}

\providecommand{\dochtml}{
\toolsection{dochtml} is an HTML documentation generator for generic documentations.
It processes several generic documentations and assembles all information therein into an HTML document.
\debuggingtool
\flowgraph{\resource{generic\\documentation} \ar[r] & \toolbox{dochtml} \ar[r] & \resource{HTML\\document}}
\seedocumentation
}

\providecommand{\doclatex}{
\toolsection{doclatex} is a Latex documentation generator for generic documentations.
It processes several generic documentations and assembles all information therein into a Latex document.
\debuggingtool
\flowgraph{\resource{generic\\documentation} \ar[r] & \toolbox{doclatex} \ar[r] & \resource{Latex\\document}}
\seedocumentation
}

% intermediate code tools

\providecommand{\cdcheck}{
\toolsection{cdcheck} is a syntactic and semantic checker for intermediate code.
It just performs syntactic and semantic checks on programs written in intermediate code and writes its diagnostic messages to the standard error stream.
\debuggingtool
\flowgraph{\resource{intermediate\\code} \ar[r] & \toolbox{cdcheck} \ar[r] & \resource{diagnostic\\messages}}
\seeassembly\seecode
}

\providecommand{\cdopt}{
\toolsection{cdopt} is an optimizer for intermediate code.
It performs various optimizations on programs written in intermediate code and writes the result to the standard output stream.
\debuggingtool
\flowgraph{\resource{intermediate\\code} \ar[r] & \toolbox{cdopt} \ar[r] & \resource{optimized\\code}}
\seeassembly\seecode
}

\providecommand{\cdrun}{
\toolsection{cdrun} is an interpreter for intermediate code.
It processes and executes programs written in intermediate code.
The following code sections are predefined and have the usual semantics:
\texttt{abort}, \texttt{\_Exit}, \texttt{fflush}, \texttt{floor}, \texttt{fputc}, \texttt{free}, \texttt{getchar}, \texttt{malloc}, and \texttt{putchar}.
Diagnostic messages about invalid operations include the name of the executed code section and the index of the erroneous instruction.
\debuggingtool
\flowgraph{\resource{intermediate\\code} \ar[r] & \toolbox{cdrun} \ar@/u/[r] & \resource{input/\\output} \ar@/d/[l]}
\seeassembly\seecode
}

\providecommand{\cdamda}{
\toolsection{cdamd16} is a compiler for intermediate code targeting the AMD64 hardware architecture.
It generates machine code for AMD64 processors from programs written in intermediate code and stores it in corresponding object files.
The compiler generates machine code for the 16-bit operating mode defined by the AMD64 architecture.
It also creates a debugging information file as well as an assembly file containing a listing of the generated machine code.
\debuggingtool
\flowgraph{\resource{intermediate\\code} \ar[r] & \toolbox{cdamd16} \ar[r] \ar[d] \ar[rd] & \resource{object file} \\ & \resource{assembly\\listing} & \resource{debugging\\information}}
\seeassembly\seeamd\seeobject\seecode\seedebugging
}

\providecommand{\cdamdb}{
\toolsection{cdamd32} is a compiler for intermediate code targeting the AMD64 hardware architecture.
It generates machine code for AMD64 processors from programs written in intermediate code and stores it in corresponding object files.
The compiler generates machine code for the 32-bit operating mode defined by the AMD64 architecture.
It also creates a debugging information file as well as an assembly file containing a listing of the generated machine code.
\debuggingtool
\flowgraph{\resource{intermediate\\code} \ar[r] & \toolbox{cdamd32} \ar[r] \ar[d] \ar[rd] & \resource{object file} \\ & \resource{assembly\\listing} & \resource{debugging\\information}}
\seeassembly\seeamd\seeobject\seecode\seedebugging
}

\providecommand{\cdamdc}{
\toolsection{cdamd64} is a compiler for intermediate code targeting the AMD64 hardware architecture.
It generates machine code for AMD64 processors from programs written in intermediate code and stores it in corresponding object files.
The compiler generates machine code for the 64-bit operating mode defined by the AMD64 architecture.
It also creates a debugging information file as well as an assembly file containing a listing of the generated machine code.
\debuggingtool
\flowgraph{\resource{intermediate\\code} \ar[r] & \toolbox{cdamd64} \ar[r] \ar[d] \ar[rd] & \resource{object file} \\ & \resource{assembly\\listing} & \resource{debugging\\information}}
\seeassembly\seeamd\seeobject\seecode\seedebugging
}

\providecommand{\cdarma}{
\toolsection{cdarma32} is a compiler for intermediate code targeting the ARM hardware architecture.
It generates machine code for ARM processors executing A32 instructions from programs written in intermediate code and stores it in corresponding object files.
It also creates a debugging information file as well as an assembly file containing a listing of the generated machine code.
\debuggingtool
\flowgraph{\resource{intermediate\\code} \ar[r] & \toolbox{cdarma32} \ar[r] \ar[d] \ar[rd] & \resource{object file} \\ & \resource{assembly\\listing} & \resource{debugging\\information}}
\seeassembly\seearm\seeobject\seecode\seedebugging
}

\providecommand{\cdarmb}{
\toolsection{cdarma64} is a compiler for intermediate code targeting the ARM hardware architecture.
It generates machine code for ARM processors executing A64 instructions from programs written in intermediate code and stores it in corresponding object files.
It also creates a debugging information file as well as an assembly file containing a listing of the generated machine code.
\debuggingtool
\flowgraph{\resource{intermediate\\code} \ar[r] & \toolbox{cdarma64} \ar[r] \ar[d] \ar[rd] & \resource{object file} \\ & \resource{assembly\\listing} & \resource{debugging\\information}}
\seeassembly\seearm\seeobject\seecode\seedebugging
}

\providecommand{\cdarmc}{
\toolsection{cdarmt32} is a compiler for intermediate code targeting the ARM hardware architecture.
It generates machine code for ARM processors without floating-point extension executing T32 instructions from programs written in intermediate code and stores it in corresponding object files.
It also creates a debugging information file as well as an assembly file containing a listing of the generated machine code.
\debuggingtool
\flowgraph{\resource{intermediate\\code} \ar[r] & \toolbox{cdarmt32} \ar[r] \ar[d] \ar[rd] & \resource{object file} \\ & \resource{assembly\\listing} & \resource{debugging\\information}}
\seeassembly\seearm\seeobject\seecode\seedebugging
}

\providecommand{\cdarmcfpe}{
\toolsection{cdarmt32fpe} is a compiler for intermediate code targeting the ARM hardware architecture.
It generates machine code for ARM processors with floating-point extension executing T32 instructions from programs written in intermediate code and stores it in corresponding object files.
It also creates a debugging information file as well as an assembly file containing a listing of the generated machine code.
\debuggingtool
\flowgraph{\resource{intermediate\\code} \ar[r] & \toolbox{cdarmt32fpe} \ar[r] \ar[d] \ar[rd] & \resource{object file} \\ & \resource{assembly\\listing} & \resource{debugging\\information}}
\seeassembly\seearm\seeobject\seecode\seedebugging
}

\providecommand{\cdavr}{
\toolsection{cdavr} is a compiler for intermediate code targeting the AVR hardware architecture.
It generates machine code for AVR processors from programs written in intermediate code and stores it in corresponding object files.
It also creates a debugging information file as well as an assembly file containing a listing of the generated machine code.
\debuggingtool
\flowgraph{\resource{intermediate\\code} \ar[r] & \toolbox{cdavr} \ar[r] \ar[d] \ar[rd] & \resource{object file} \\ & \resource{assembly\\listing} & \resource{debugging\\information}}
\seeassembly\seeavr\seeobject\seecode\seedebugging
}

\providecommand{\cdavrtt}{
\toolsection{cdavr32} is a compiler for intermediate code targeting the AVR32 hardware architecture.
It generates machine code for AVR32 processors from programs written in intermediate code and stores it in corresponding object files.
It also creates a debugging information file as well as an assembly file containing a listing of the generated machine code.
\debuggingtool
\flowgraph{\resource{intermediate\\code} \ar[r] & \toolbox{cdavr32} \ar[r] \ar[d] \ar[rd] & \resource{object file} \\ & \resource{assembly\\listing} & \resource{debugging\\information}}
\seeassembly\seeavrtt\seeobject\seecode\seedebugging
}

\providecommand{\cdmabk}{
\toolsection{cdm68k} is a compiler for intermediate code targeting the M68000 hardware architecture.
It generates machine code for M68000 processors from programs written in intermediate code and stores it in corresponding object files.
It also creates a debugging information file as well as an assembly file containing a listing of the generated machine code.
\debuggingtool
\flowgraph{\resource{intermediate\\code} \ar[r] & \toolbox{cdm68k} \ar[r] \ar[d] \ar[rd] & \resource{object file} \\ & \resource{assembly\\listing} & \resource{debugging\\information}}
\seeassembly\seemabk\seeobject\seecode\seedebugging
}

\providecommand{\cdmibl}{
\toolsection{cdmibl} is a compiler for intermediate code targeting the MicroBlaze hardware architecture.
It generates machine code for MicroBlaze processors from programs written in intermediate code and stores it in corresponding object files.
It also creates a debugging information file as well as an assembly file containing a listing of the generated machine code.
\debuggingtool
\flowgraph{\resource{intermediate\\code} \ar[r] & \toolbox{cdmibl} \ar[r] \ar[d] \ar[rd] & \resource{object file} \\ & \resource{assembly\\listing} & \resource{debugging\\information}}
\seeassembly\seemibl\seeobject\seecode\seedebugging
}

\providecommand{\cdmipsa}{
\toolsection{cdmips32} is a compiler for intermediate code targeting the MIPS32 hardware architecture.
It generates machine code for MIPS32 processors from programs written in intermediate code and stores it in corresponding object files.
It also creates a debugging information file as well as an assembly file containing a listing of the generated machine code.
\debuggingtool
\flowgraph{\resource{intermediate\\code} \ar[r] & \toolbox{cdmips32} \ar[r] \ar[d] \ar[rd] & \resource{object file} \\ & \resource{assembly\\listing} & \resource{debugging\\information}}
\seeassembly\seemips\seeobject\seecode\seedebugging
}

\providecommand{\cdmipsb}{
\toolsection{cdmips64} is a compiler for intermediate code targeting the MIPS64 hardware architecture.
It generates machine code for MIPS64 processors from programs written in intermediate code and stores it in corresponding object files.
It also creates a debugging information file as well as an assembly file containing a listing of the generated machine code.
\debuggingtool
\flowgraph{\resource{intermediate\\code} \ar[r] & \toolbox{cdmips64} \ar[r] \ar[d] \ar[rd] & \resource{object file} \\ & \resource{assembly\\listing} & \resource{debugging\\information}}
\seeassembly\seemips\seeobject\seecode\seedebugging
}

\providecommand{\cdmmix}{
\toolsection{cdmmix} is a compiler for intermediate code targeting the MMIX hardware architecture.
It generates machine code for MMIX processors from programs written in intermediate code and stores it in corresponding object files.
It also creates a debugging information file as well as an assembly file containing a listing of the generated machine code.
\debuggingtool
\flowgraph{\resource{intermediate\\code} \ar[r] & \toolbox{cdmmix} \ar[r] \ar[d] \ar[rd] & \resource{object file} \\ & \resource{assembly\\listing} & \resource{debugging\\information}}
\seeassembly\seemmix\seeobject\seecode\seedebugging
}

\providecommand{\cdorok}{
\toolsection{cdor1k} is a compiler for intermediate code targeting the OpenRISC 1000 hardware architecture.
It generates machine code for OpenRISC 1000 processors from programs written in intermediate code and stores it in corresponding object files.
It also creates a debugging information file as well as an assembly file containing a listing of the generated machine code.
\debuggingtool
\flowgraph{\resource{intermediate\\code} \ar[r] & \toolbox{cdor1k} \ar[r] \ar[d] \ar[rd] & \resource{object file} \\ & \resource{assembly\\listing} & \resource{debugging\\information}}
\seeassembly\seeorok\seeobject\seecode\seedebugging
}

\providecommand{\cdppca}{
\toolsection{cdppc32} is a compiler for intermediate code targeting the PowerPC hardware architecture.
It generates machine code for PowerPC processors from programs written in intermediate code and stores it in corresponding object files.
The compiler generates machine code for the 32-bit operating mode defined by the PowerPC architecture.
It also creates a debugging information file as well as an assembly file containing a listing of the generated machine code.
\debuggingtool
\flowgraph{\resource{intermediate\\code} \ar[r] & \toolbox{cdppc32} \ar[r] \ar[d] \ar[rd] & \resource{object file} \\ & \resource{assembly\\listing} & \resource{debugging\\information}}
\seeassembly\seeppc\seeobject\seecode\seedebugging
}

\providecommand{\cdppcb}{
\toolsection{cdppc64} is a compiler for intermediate code targeting the PowerPC hardware architecture.
It generates machine code for PowerPC processors from programs written in intermediate code and stores it in corresponding object files.
The compiler generates machine code for the 64-bit operating mode defined by the PowerPC architecture.
It also creates a debugging information file as well as an assembly file containing a listing of the generated machine code.
\debuggingtool
\flowgraph{\resource{intermediate\\code} \ar[r] & \toolbox{cdppc64} \ar[r] \ar[d] \ar[rd] & \resource{object file} \\ & \resource{assembly\\listing} & \resource{debugging\\information}}
\seeassembly\seeppc\seeobject\seecode\seedebugging
}

\providecommand{\cdrisc}{
\toolsection{cdrisc} is a compiler for intermediate code targeting the RISC hardware architecture.
It generates machine code for RISC processors from programs written in intermediate code and stores it in corresponding object files.
It also creates a debugging information file as well as an assembly file containing a listing of the generated machine code.
\debuggingtool
\flowgraph{\resource{intermediate\\code} \ar[r] & \toolbox{cdrisc} \ar[r] \ar[d] \ar[rd] & \resource{object file} \\ & \resource{assembly\\listing} & \resource{debugging\\information}}
\seeassembly\seerisc\seeobject\seecode\seedebugging
}

\providecommand{\cdwasm}{
\toolsection{cdwasm} is a compiler for intermediate code targeting the WebAssembly architecture.
It generates machine code for WebAssembly targets from programs written in intermediate code and stores it in corresponding object files.
It also creates a debugging information file as well as an assembly file containing a listing of the generated machine code.
\debuggingtool
\flowgraph{\resource{intermediate\\code} \ar[r] & \toolbox{cdwasm} \ar[r] \ar[d] \ar[rd] & \resource{object file} \\ & \resource{assembly\\listing} & \resource{debugging\\information}}
\seeassembly\seewasm\seeobject\seecode\seedebugging
}

% C++ tools

\providecommand{\cppprep}{
\toolsection{cppprep} is a preprocessor for the \cpp{} programming language.
It preprocesses source code according to the rules of \cpp{} and writes it to the standard output stream.
Only the macro names \texttt{\_\_DATE\_\_}, \texttt{\_\_FILE\_\_}, \texttt{\_\_LINE\_\_}, and \texttt{\_\_TIME\_\_} are predefined.
\flowgraph{\resource{\cpp{} or other\\source code} \ar[r] & \toolbox{cppprep} \ar[r] & \resource{preprocessed\\source code} \\ & \variable{ECSINCLUDE} \ar[u]}
\seecpp
}

\providecommand{\cppprint}{
\toolsection{cppprint} is a pretty printer for the \cpp{} programming language.
It reformats the source code of \cpp{} programs and writes it to the standard output stream.
\flowgraph{\resource{\cpp{}\\source code} \ar[r] & \toolbox{cppprint} \ar[r] & \resource{reformatted\\source code} \\ & \variable{ECSINCLUDE} \ar[u]}
\seecpp
}

\providecommand{\cppcheck}{
\toolsection{cppcheck} is a syntactic and semantic checker for the \cpp{} programming language.
It just performs syntactic and semantic checks on \cpp{} programs and writes its diagnostic messages to the standard error stream.
\flowgraph{\resource{\cpp{}\\source code} \ar[r] & \toolbox{cppcheck} \ar[r] & \resource{diagnostic\\messages} \\ & \variable{ECSINCLUDE} \ar[u]}
\seecpp
}

\providecommand{\cppdump}{
\toolsection{cppdump} is a serializer for the \cpp{} programming language.
It dumps the complete internal representation of programs written in \cpp{} into an XML document.
\debuggingtool
\flowgraph{\resource{\cpp{}\\source code} \ar[r] & \toolbox{cppdump} \ar[r] & \resource{internal\\representation} \\ & \variable{ECSINCLUDE} \ar[u]}
\seecpp
}

\providecommand{\cpprun}{
\toolsection{cpprun} is an interpreter for the \cpp{} programming language.
It processes and executes programs written in \cpp{}.
The macro \texttt{\_\_run\_\_} is predefined in order to enable programmers to identify this tool while interpreting.
\flowgraph{\resource{\cpp{}\\source code} \ar[r] & \toolbox{cpprun} \ar@/u/[r] & \resource{input/\\output} \ar@/d/[l] \\ & \variable{ECSINCLUDE} \ar[u]}
\seecpp
}

\providecommand{\cppdoc}{
\toolsection{cppdoc} is a generic documentation generator for the \cpp{} programming language.
It processes several \cpp{} source files and assembles all information therein into a generic documentation.
\debuggingtool
\flowgraph{\resource{\cpp{}\\source code} \ar[r] & \toolbox{cppdoc} \ar[r] & \resource{generic\\documentation} \\ & \variable{ECSINCLUDE} \ar[u]}
\seecpp\seedocumentation
}

\providecommand{\cpphtml}{
\toolsection{cpphtml} is an HTML documentation generator for the \cpp{} programming language.
It processes several \cpp{} source files and assembles all information therein into an HTML document.
\flowgraph{\resource{\cpp{}\\source code} \ar[r] & \toolbox{cpphtml} \ar[r] & \resource{HTML\\document} \\ & \variable{ECSINCLUDE} \ar[u]}
\seecpp\seedocumentation
}

\providecommand{\cpplatex}{
\toolsection{cpplatex} is a Latex documentation generator for the \cpp{} programming language.
It processes several \cpp{} source files and assembles all information therein into a Latex document.
\flowgraph{\resource{\cpp{}\\source code} \ar[r] & \toolbox{cpplatex} \ar[r] & \resource{Latex\\document} \\ & \variable{ECSINCLUDE} \ar[u]}
\seecpp\seedocumentation
}

\providecommand{\cppcode}{
\toolsection{cppcode} is an intermediate code generator for the \cpp{} programming language.
It generates intermediate code from programs written in \cpp{} and stores it in corresponding assembly files.
The macro \texttt{\_\_code\_\_} is predefined in order to enable programmers to identify this tool while generating intermediate code.
Programs generated with this tool require additional runtime support that is stored in the \file{cpp\-code\-run} library file.
\debuggingtool
\flowgraph{\resource{\cpp{}\\source code} \ar[r] & \toolbox{cppcode} \ar[r] & \resource{intermediate\\code} \\ & \variable{ECSINCLUDE} \ar[u]}
\seecpp\seeassembly\seecode
}

\providecommand{\cppamda}{
\toolsection{cppamd16} is a compiler for the \cpp{} programming language targeting the AMD64 hardware architecture.
It generates machine code for AMD64 processors from programs written in \cpp{} and stores it in corresponding object files.
The compiler generates machine code for the 16-bit operating mode defined by the AMD64 architecture.
For debugging purposes, it also creates a debugging information file as well as an assembly file containing a listing of the generated machine code.
The macro \texttt{\_\_amd16\_\_} is predefined in order to enable programmers to identify this tool and its target architecture while compiling.
Programs generated with this compiler require additional runtime support that is stored in the \file{cpp\-amd16\-run} library file.
\flowgraph{\resource{\cpp{}\\source code} \ar[r] & \toolbox{cppamd16} \ar[r] \ar[d] \ar[rd] & \resource{object file} \\ \variable{ECSINCLUDE} \ar[ru] & \resource{debugging\\information} & \resource{assembly\\listing}}
\seecpp\seeassembly\seeamd\seeobject\seedebugging
}

\providecommand{\cppamdb}{
\toolsection{cppamd32} is a compiler for the \cpp{} programming language targeting the AMD64 hardware architecture.
It generates machine code for AMD64 processors from programs written in \cpp{} and stores it in corresponding object files.
The compiler generates machine code for the 32-bit operating mode defined by the AMD64 architecture.
For debugging purposes, it also creates a debugging information file as well as an assembly file containing a listing of the generated machine code.
The macro \texttt{\_\_amd32\_\_} is predefined in order to enable programmers to identify this tool and its target architecture while compiling.
Programs generated with this compiler require additional runtime support that is stored in the \file{cpp\-amd32\-run} library file.
\flowgraph{\resource{\cpp{}\\source code} \ar[r] & \toolbox{cppamd32} \ar[r] \ar[d] \ar[rd] & \resource{object file} \\ \variable{ECSINCLUDE} \ar[ru] & \resource{debugging\\information} & \resource{assembly\\listing}}
\seecpp\seeassembly\seeamd\seeobject\seedebugging
}

\providecommand{\cppamdc}{
\toolsection{cppamd64} is a compiler for the \cpp{} programming language targeting the AMD64 hardware architecture.
It generates machine code for AMD64 processors from programs written in \cpp{} and stores it in corresponding object files.
The compiler generates machine code for the 64-bit operating mode defined by the AMD64 architecture.
For debugging purposes, it also creates a debugging information file as well as an assembly file containing a listing of the generated machine code.
The macro \texttt{\_\_amd64\_\_} is predefined in order to enable programmers to identify this tool and its target architecture while compiling.
Programs generated with this compiler require additional runtime support that is stored in the \file{cpp\-amd64\-run} library file.
\flowgraph{\resource{\cpp{}\\source code} \ar[r] & \toolbox{cppamd64} \ar[r] \ar[d] \ar[rd] & \resource{object file} \\ \variable{ECSINCLUDE} \ar[ru] & \resource{debugging\\information} & \resource{assembly\\listing}}
\seecpp\seeassembly\seeamd\seeobject\seedebugging
}

\providecommand{\cpparma}{
\toolsection{cpparma32} is a compiler for the \cpp{} programming language targeting the ARM hardware architecture.
It generates machine code for ARM processors executing A32 instructions from programs written in \cpp{} and stores it in corresponding object files.
For debugging purposes, it also creates a debugging information file as well as an assembly file containing a listing of the generated machine code.
The macro \texttt{\_\_arma32\_\_} is predefined in order to enable programmers to identify this tool and its target architecture while compiling.
Programs generated with this compiler require additional runtime support that is stored in the \file{cpp\-arma32\-run} library file.
\flowgraph{\resource{\cpp{}\\source code} \ar[r] & \toolbox{cpparma32} \ar[r] \ar[d] \ar[rd] & \resource{object file} \\ \variable{ECSINCLUDE} \ar[ru] & \resource{debugging\\information} & \resource{assembly\\listing}}
\seecpp\seeassembly\seearm\seeobject\seedebugging
}

\providecommand{\cpparmb}{
\toolsection{cpparma64} is a compiler for the \cpp{} programming language targeting the ARM hardware architecture.
It generates machine code for ARM processors executing A64 instructions from programs written in \cpp{} and stores it in corresponding object files.
For debugging purposes, it also creates a debugging information file as well as an assembly file containing a listing of the generated machine code.
The macro \texttt{\_\_arma64\_\_} is predefined in order to enable programmers to identify this tool and its target architecture while compiling.
Programs generated with this compiler require additional runtime support that is stored in the \file{cpp\-arma64\-run} library file.
\flowgraph{\resource{\cpp{}\\source code} \ar[r] & \toolbox{cpparma64} \ar[r] \ar[d] \ar[rd] & \resource{object file} \\ \variable{ECSINCLUDE} \ar[ru] & \resource{debugging\\information} & \resource{assembly\\listing}}
\seecpp\seeassembly\seearm\seeobject\seedebugging
}

\providecommand{\cpparmc}{
\toolsection{cpparmt32} is a compiler for the \cpp{} programming language targeting the ARM hardware architecture.
It generates machine code for ARM processors without floating-point extension executing T32 instructions from programs written in \cpp{} and stores it in corresponding object files.
For debugging purposes, it also creates a debugging information file as well as an assembly file containing a listing of the generated machine code.
The macro \texttt{\_\_armt32\_\_} is predefined in order to enable programmers to identify this tool and its target architecture while compiling.
Programs generated with this compiler require additional runtime support that is stored in the \file{cpp\-armt32\-run} library file.
\flowgraph{\resource{\cpp{}\\source code} \ar[r] & \toolbox{cpparmt32} \ar[r] \ar[d] \ar[rd] & \resource{object file} \\ \variable{ECSINCLUDE} \ar[ru] & \resource{debugging\\information} & \resource{assembly\\listing}}
\seecpp\seeassembly\seearm\seeobject\seedebugging
}

\providecommand{\cpparmcfpe}{
\toolsection{cpparmt32fpe} is a compiler for the \cpp{} programming language targeting the ARM hardware architecture.
It generates machine code for ARM processors with floating-point extension executing T32 instructions from programs written in \cpp{} and stores it in corresponding object files.
For debugging purposes, it also creates a debugging information file as well as an assembly file containing a listing of the generated machine code.
The macro \texttt{\_\_armt32fpe\_\_} is predefined in order to enable programmers to identify this tool and its target architecture while compiling.
Programs generated with this compiler require additional runtime support that is stored in the \file{cpp\-armt32\-fpe\-run} library file.
\flowgraph{\resource{\cpp{}\\source code} \ar[r] & \toolbox{cpparmt32fpe} \ar[r] \ar[d] \ar[rd] & \resource{object file} \\ \variable{ECSINCLUDE} \ar[ru] & \resource{debugging\\information} & \resource{assembly\\listing}}
\seecpp\seeassembly\seearm\seeobject\seedebugging
}

\providecommand{\cppavr}{
\toolsection{cppavr} is a compiler for the \cpp{} programming language targeting the AVR hardware architecture.
It generates machine code for AVR processors from programs written in \cpp{} and stores it in corresponding object files.
For debugging purposes, it also creates a debugging information file as well as an assembly file containing a listing of the generated machine code.
The macro \texttt{\_\_avr\_\_} is predefined in order to enable programmers to identify this tool and its target architecture while compiling.
Programs generated with this compiler require additional runtime support that is stored in the \file{cpp\-avr\-run} library file.
\flowgraph{\resource{\cpp{}\\source code} \ar[r] & \toolbox{cppavr} \ar[r] \ar[d] \ar[rd] & \resource{object file} \\ \variable{ECSINCLUDE} \ar[ru] & \resource{debugging\\information} & \resource{assembly\\listing}}
\seecpp\seeassembly\seeavr\seeobject\seedebugging
}

\providecommand{\cppavrtt}{
\toolsection{cppavr32} is a compiler for the \cpp{} programming language targeting the AVR32 hardware architecture.
It generates machine code for AVR32 processors from programs written in \cpp{} and stores it in corresponding object files.
For debugging purposes, it also creates a debugging information file as well as an assembly file containing a listing of the generated machine code.
The macro \texttt{\_\_avr32\_\_} is predefined in order to enable programmers to identify this tool and its target architecture while compiling.
Programs generated with this compiler require additional runtime support that is stored in the \file{cpp\-avr32\-run} library file.
\flowgraph{\resource{\cpp{}\\source code} \ar[r] & \toolbox{cppavr32} \ar[r] \ar[d] \ar[rd] & \resource{object file} \\ \variable{ECSINCLUDE} \ar[ru] & \resource{debugging\\information} & \resource{assembly\\listing}}
\seecpp\seeassembly\seeavrtt\seeobject\seedebugging
}

\providecommand{\cppmabk}{
\toolsection{cppm68k} is a compiler for the \cpp{} programming language targeting the M68000 hardware architecture.
It generates machine code for M68000 processors from programs written in \cpp{} and stores it in corresponding object files.
For debugging purposes, it also creates a debugging information file as well as an assembly file containing a listing of the generated machine code.
The macro \texttt{\_\_m68k\_\_} is predefined in order to enable programmers to identify this tool and its target architecture while compiling.
Programs generated with this compiler require additional runtime support that is stored in the \file{cpp\-m68k\-run} library file.
\flowgraph{\resource{\cpp{}\\source code} \ar[r] & \toolbox{cppm68k} \ar[r] \ar[d] \ar[rd] & \resource{object file} \\ \variable{ECSINCLUDE} \ar[ru] & \resource{debugging\\information} & \resource{assembly\\listing}}
\seecpp\seeassembly\seemabk\seeobject\seedebugging
}

\providecommand{\cppmibl}{
\toolsection{cppmibl} is a compiler for the \cpp{} programming language targeting the MicroBlaze hardware architecture.
It generates machine code for MicroBlaze processors from programs written in \cpp{} and stores it in corresponding object files.
For debugging purposes, it also creates a debugging information file as well as an assembly file containing a listing of the generated machine code.
The macro \texttt{\_\_mibl\_\_} is predefined in order to enable programmers to identify this tool and its target architecture while compiling.
Programs generated with this compiler require additional runtime support that is stored in the \file{cpp\-mibl\-run} library file.
\flowgraph{\resource{\cpp{}\\source code} \ar[r] & \toolbox{cppmibl} \ar[r] \ar[d] \ar[rd] & \resource{object file} \\ \variable{ECSINCLUDE} \ar[ru] & \resource{debugging\\information} & \resource{assembly\\listing}}
\seecpp\seeassembly\seemibl\seeobject\seedebugging
}

\providecommand{\cppmipsa}{
\toolsection{cppmips32} is a compiler for the \cpp{} programming language targeting the MIPS32 hardware architecture.
It generates machine code for MIPS32 processors from programs written in \cpp{} and stores it in corresponding object files.
For debugging purposes, it also creates a debugging information file as well as an assembly file containing a listing of the generated machine code.
The macro \texttt{\_\_mips32\_\_} is predefined in order to enable programmers to identify this tool and its target architecture while compiling.
Programs generated with this compiler require additional runtime support that is stored in the \file{cpp\-mips32\-run} library file.
\flowgraph{\resource{\cpp{}\\source code} \ar[r] & \toolbox{cppmips32} \ar[r] \ar[d] \ar[rd] & \resource{object file} \\ \variable{ECSINCLUDE} \ar[ru] & \resource{debugging\\information} & \resource{assembly\\listing}}
\seecpp\seeassembly\seemips\seeobject\seedebugging
}

\providecommand{\cppmipsb}{
\toolsection{cppmips64} is a compiler for the \cpp{} programming language targeting the MIPS64 hardware architecture.
It generates machine code for MIPS64 processors from programs written in \cpp{} and stores it in corresponding object files.
For debugging purposes, it also creates a debugging information file as well as an assembly file containing a listing of the generated machine code.
The macro \texttt{\_\_mips64\_\_} is predefined in order to enable programmers to identify this tool and its target architecture while compiling.
Programs generated with this compiler require additional runtime support that is stored in the \file{cpp\-mips64\-run} library file.
\flowgraph{\resource{\cpp{}\\source code} \ar[r] & \toolbox{cppmips64} \ar[r] \ar[d] \ar[rd] & \resource{object file} \\ \variable{ECSINCLUDE} \ar[ru] & \resource{debugging\\information} & \resource{assembly\\listing}}
\seecpp\seeassembly\seemips\seeobject\seedebugging
}

\providecommand{\cppmmix}{
\toolsection{cppmmix} is a compiler for the \cpp{} programming language targeting the MMIX hardware architecture.
It generates machine code for MMIX processors from programs written in \cpp{} and stores it in corresponding object files.
For debugging purposes, it also creates a debugging information file as well as an assembly file containing a listing of the generated machine code.
The macro \texttt{\_\_mmix\_\_} is predefined in order to enable programmers to identify this tool and its target architecture while compiling.
Programs generated with this compiler require additional runtime support that is stored in the \file{cpp\-mmix\-run} library file.
\flowgraph{\resource{\cpp{}\\source code} \ar[r] & \toolbox{cppmmix} \ar[r] \ar[d] \ar[rd] & \resource{object file} \\ \variable{ECSINCLUDE} \ar[ru] & \resource{debugging\\information} & \resource{assembly\\listing}}
\seecpp\seeassembly\seemmix\seeobject\seedebugging
}

\providecommand{\cpporok}{
\toolsection{cppor1k} is a compiler for the \cpp{} programming language targeting the OpenRISC 1000 hardware architecture.
It generates machine code for OpenRISC 1000 processors from programs written in \cpp{} and stores it in corresponding object files.
For debugging purposes, it also creates a debugging information file as well as an assembly file containing a listing of the generated machine code.
The macro \texttt{\_\_or1k\_\_} is predefined in order to enable programmers to identify this tool and its target architecture while compiling.
Programs generated with this compiler require additional runtime support that is stored in the \file{cpp\-or1k\-run} library file.
\flowgraph{\resource{\cpp{}\\source code} \ar[r] & \toolbox{cppor1k} \ar[r] \ar[d] \ar[rd] & \resource{object file} \\ \variable{ECSINCLUDE} \ar[ru] & \resource{debugging\\information} & \resource{assembly\\listing}}
\seecpp\seeassembly\seeorok\seeobject\seedebugging
}

\providecommand{\cppppca}{
\toolsection{cppppc32} is a compiler for the \cpp{} programming language targeting the PowerPC hardware architecture.
It generates machine code for PowerPC processors from programs written in \cpp{} and stores it in corresponding object files.
The compiler generates machine code for the 32-bit operating mode defined by the PowerPC architecture.
For debugging purposes, it also creates a debugging information file as well as an assembly file containing a listing of the generated machine code.
The macro \texttt{\_\_ppc32\_\_} is predefined in order to enable programmers to identify this tool and its target architecture while compiling.
Programs generated with this compiler require additional runtime support that is stored in the \file{cpp\-ppc32\-run} library file.
\flowgraph{\resource{\cpp{}\\source code} \ar[r] & \toolbox{cppppc32} \ar[r] \ar[d] \ar[rd] & \resource{object file} \\ \variable{ECSINCLUDE} \ar[ru] & \resource{debugging\\information} & \resource{assembly\\listing}}
\seecpp\seeassembly\seeppc\seeobject\seedebugging
}

\providecommand{\cppppcb}{
\toolsection{cppppc64} is a compiler for the \cpp{} programming language targeting the PowerPC hardware architecture.
It generates machine code for PowerPC processors from programs written in \cpp{} and stores it in corresponding object files.
The compiler generates machine code for the 64-bit operating mode defined by the PowerPC architecture.
For debugging purposes, it also creates a debugging information file as well as an assembly file containing a listing of the generated machine code.
The macro \texttt{\_\_ppc64\_\_} is predefined in order to enable programmers to identify this tool and its target architecture while compiling.
Programs generated with this compiler require additional runtime support that is stored in the \file{cpp\-ppc64\-run} library file.
\flowgraph{\resource{\cpp{}\\source code} \ar[r] & \toolbox{cppppc64} \ar[r] \ar[d] \ar[rd] & \resource{object file} \\ \variable{ECSINCLUDE} \ar[ru] & \resource{debugging\\information} & \resource{assembly\\listing}}
\seecpp\seeassembly\seeppc\seeobject\seedebugging
}

\providecommand{\cpprisc}{
\toolsection{cpprisc} is a compiler for the \cpp{} programming language targeting the RISC hardware architecture.
It generates machine code for RISC processors from programs written in \cpp{} and stores it in corresponding object files.
For debugging purposes, it also creates a debugging information file as well as an assembly file containing a listing of the generated machine code.
The macro \texttt{\_\_risc\_\_} is predefined in order to enable programmers to identify this tool and its target architecture while compiling.
Programs generated with this compiler require additional runtime support that is stored in the \file{cpp\-risc\-run} library file.
\flowgraph{\resource{\cpp{}\\source code} \ar[r] & \toolbox{cpprisc} \ar[r] \ar[d] \ar[rd] & \resource{object file} \\ \variable{ECSINCLUDE} \ar[ru] & \resource{debugging\\information} & \resource{assembly\\listing}}
\seecpp\seeassembly\seerisc\seeobject\seedebugging
}

\providecommand{\cppwasm}{
\toolsection{cppwasm} is a compiler for the \cpp{} programming language targeting the WebAssembly architecture.
It generates machine code for WebAssembly targets from programs written in \cpp{} and stores it in corresponding object files.
For debugging purposes, it also creates a debugging information file as well as an assembly file containing a listing of the generated machine code.
The macro \texttt{\_\_wasm\_\_} is predefined in order to enable programmers to identify this tool and its target architecture while compiling.
Programs generated with this compiler require additional runtime support that is stored in the \file{cpp\-wasm\-run} library file.
\flowgraph{\resource{\cpp{}\\source code} \ar[r] & \toolbox{cppwasm} \ar[r] \ar[d] \ar[rd] & \resource{object file} \\ \variable{ECSINCLUDE} \ar[ru] & \resource{debugging\\information} & \resource{assembly\\listing}}
\seecpp\seeassembly\seewasm\seeobject\seedebugging
}

% FALSE tools

\providecommand{\falprint}{
\toolsection{falprint} is a pretty printer for the FALSE programming language.
It reformats the source code of FALSE programs and writes it to the standard output stream.
\flowgraph{\resource{FALSE\\source code} \ar[r] & \toolbox{falprint} \ar[r] & \resource{reformatted\\source code}}
\seefalse
}

\providecommand{\falcheck}{
\toolsection{falcheck} is a syntactic and semantic checker for the FALSE programming language.
It just performs syntactic and semantic checks on FALSE programs and writes its diagnostic messages to the standard error stream.
\flowgraph{\resource{FALSE\\source code} \ar[r] & \toolbox{falcheck} \ar[r] & \resource{diagnostic\\messages}}
\seefalse
}

\providecommand{\faldump}{
\toolsection{faldump} is a serializer for the FALSE programming language.
It dumps the complete internal representation of programs written in FALSE into an XML document.
\debuggingtool
\flowgraph{\resource{FALSE\\source code} \ar[r] & \toolbox{faldump} \ar[r] & \resource{internal\\representation}}
\seefalse
}

\providecommand{\falrun}{
\toolsection{falrun} is an interpreter for the FALSE programming language.
It processes and executes programs written in FALSE\@.
\flowgraph{\resource{FALSE\\source code} \ar[r] & \toolbox{falrun} \ar@/u/[r] & \resource{input/\\output} \ar@/d/[l]}
\seefalse
}

\providecommand{\falcpp}{
\toolsection{falcpp} is a transpiler for the FALSE programming language.
It translates programs written in FALSE into \cpp{} programs and stores them in corresponding source files.
\flowgraph{\resource{FALSE\\source code} \ar[r] & \toolbox{falcpp} \ar[r] & \resource{\cpp{}\\source file}}
\seefalse\seecpp
}

\providecommand{\falcode}{
\toolsection{falcode} is an intermediate code generator for the FALSE programming language.
It generates intermediate code from programs written in FALSE and stores it in corresponding assembly files.
\debuggingtool
\flowgraph{\resource{FALSE\\source code} \ar[r] & \toolbox{falcode} \ar[r] & \resource{intermediate\\code}}
\seefalse\seeassembly\seecode
}

\providecommand{\falamda}{
\toolsection{falamd16} is a compiler for the FALSE programming language targeting the AMD64 hardware architecture.
It generates machine code for AMD64 processors from programs written in FALSE and stores it in corresponding object files.
The compiler generates machine code for the 16-bit operating mode defined by the AMD64 architecture.
\flowgraph{\resource{FALSE\\source code} \ar[r] & \toolbox{falamd16} \ar[r] & \resource{object file}}
\seefalse\seeamd\seeobject
}

\providecommand{\falamdb}{
\toolsection{falamd32} is a compiler for the FALSE programming language targeting the AMD64 hardware architecture.
It generates machine code for AMD64 processors from programs written in FALSE and stores it in corresponding object files.
The compiler generates machine code for the 32-bit operating mode defined by the AMD64 architecture.
\flowgraph{\resource{FALSE\\source code} \ar[r] & \toolbox{falamd32} \ar[r] & \resource{object file}}
\seefalse\seeamd\seeobject
}

\providecommand{\falamdc}{
\toolsection{falamd64} is a compiler for the FALSE programming language targeting the AMD64 hardware architecture.
It generates machine code for AMD64 processors from programs written in FALSE and stores it in corresponding object files.
The compiler generates machine code for the 64-bit operating mode defined by the AMD64 architecture.
\flowgraph{\resource{FALSE\\source code} \ar[r] & \toolbox{falamd64} \ar[r] & \resource{object file}}
\seefalse\seeamd\seeobject
}

\providecommand{\falarma}{
\toolsection{falarma32} is a compiler for the FALSE programming language targeting the ARM hardware architecture.
It generates machine code for ARM processors executing A32 instructions from programs written in FALSE and stores it in corresponding object files.
\flowgraph{\resource{FALSE\\source code} \ar[r] & \toolbox{falarma32} \ar[r] & \resource{object file}}
\seefalse\seearm\seeobject
}

\providecommand{\falarmb}{
\toolsection{falarma64} is a compiler for the FALSE programming language targeting the ARM hardware architecture.
It generates machine code for ARM processors executing A64 instructions from programs written in FALSE and stores it in corresponding object files.
\flowgraph{\resource{FALSE\\source code} \ar[r] & \toolbox{falarma64} \ar[r] & \resource{object file}}
\seefalse\seearm\seeobject
}

\providecommand{\falarmc}{
\toolsection{falarmt32} is a compiler for the FALSE programming language targeting the ARM hardware architecture.
It generates machine code for ARM processors without floating-point extension executing T32 instructions from programs written in FALSE and stores it in corresponding object files.
\flowgraph{\resource{FALSE\\source code} \ar[r] & \toolbox{falarmt32} \ar[r] & \resource{object file}}
\seefalse\seearm\seeobject
}

\providecommand{\falarmcfpe}{
\toolsection{falarmt32fpe} is a compiler for the FALSE programming language targeting the ARM hardware architecture.
It generates machine code for ARM processors with floating-point extension executing T32 instructions from programs written in FALSE and stores it in corresponding object files.
\flowgraph{\resource{FALSE\\source code} \ar[r] & \toolbox{falarmt32fpe} \ar[r] & \resource{object file}}
\seefalse\seearm\seeobject
}

\providecommand{\falavr}{
\toolsection{falavr} is a compiler for the FALSE programming language targeting the AVR hardware architecture.
It generates machine code for AVR processors from programs written in FALSE and stores it in corresponding object files.
\flowgraph{\resource{FALSE\\source code} \ar[r] & \toolbox{falavr} \ar[r] & \resource{object file}}
\seefalse\seeavr\seeobject
}

\providecommand{\falavrtt}{
\toolsection{falavr32} is a compiler for the FALSE programming language targeting the AVR32 hardware architecture.
It generates machine code for AVR32 processors from programs written in FALSE and stores it in corresponding object files.
\flowgraph{\resource{FALSE\\source code} \ar[r] & \toolbox{falavr32} \ar[r] & \resource{object file}}
\seefalse\seeavrtt\seeobject
}

\providecommand{\falmabk}{
\toolsection{falm68k} is a compiler for the FALSE programming language targeting the M68000 hardware architecture.
It generates machine code for M68000 processors from programs written in FALSE and stores it in corresponding object files.
\flowgraph{\resource{FALSE\\source code} \ar[r] & \toolbox{falm68k} \ar[r] & \resource{object file}}
\seefalse\seemabk\seeobject
}

\providecommand{\falmibl}{
\toolsection{falmibl} is a compiler for the FALSE programming language targeting the MicroBlaze hardware architecture.
It generates machine code for MicroBlaze processors from programs written in FALSE and stores it in corresponding object files.
\flowgraph{\resource{FALSE\\source code} \ar[r] & \toolbox{falmibl} \ar[r] & \resource{object file}}
\seefalse\seemibl\seeobject
}

\providecommand{\falmipsa}{
\toolsection{falmips32} is a compiler for the FALSE programming language targeting the MIPS32 hardware architecture.
It generates machine code for MIPS32 processors from programs written in FALSE and stores it in corresponding object files.
\flowgraph{\resource{FALSE\\source code} \ar[r] & \toolbox{falmips32} \ar[r] & \resource{object file}}
\seefalse\seemips\seeobject
}

\providecommand{\falmipsb}{
\toolsection{falmips64} is a compiler for the FALSE programming language targeting the MIPS64 hardware architecture.
It generates machine code for MIPS64 processors from programs written in FALSE and stores it in corresponding object files.
\flowgraph{\resource{FALSE\\source code} \ar[r] & \toolbox{falmips64} \ar[r] & \resource{object file}}
\seefalse\seemips\seeobject
}

\providecommand{\falmmix}{
\toolsection{falmmix} is a compiler for the FALSE programming language targeting the MMIX hardware architecture.
It generates machine code for MMIX processors from programs written in FALSE and stores it in corresponding object files.
\flowgraph{\resource{FALSE\\source code} \ar[r] & \toolbox{falmmix} \ar[r] & \resource{object file}}
\seefalse\seemmix\seeobject
}

\providecommand{\falorok}{
\toolsection{falor1k} is a compiler for the FALSE programming language targeting the OpenRISC 1000 hardware architecture.
It generates machine code for OpenRISC 1000 processors from programs written in FALSE and stores it in corresponding object files.
\flowgraph{\resource{FALSE\\source code} \ar[r] & \toolbox{falor1k} \ar[r] & \resource{object file}}
\seefalse\seeorok\seeobject
}

\providecommand{\falppca}{
\toolsection{falppc32} is a compiler for the FALSE programming language targeting the PowerPC hardware architecture.
It generates machine code for PowerPC processors from programs written in FALSE and stores it in corresponding object files.
The compiler generates machine code for the 32-bit operating mode defined by the PowerPC architecture.
\flowgraph{\resource{FALSE\\source code} \ar[r] & \toolbox{falppc32} \ar[r] & \resource{object file}}
\seefalse\seeppc\seeobject
}

\providecommand{\falppcb}{
\toolsection{falppc64} is a compiler for the FALSE programming language targeting the PowerPC hardware architecture.
It generates machine code for PowerPC processors from programs written in FALSE and stores it in corresponding object files.
The compiler generates machine code for the 64-bit operating mode defined by the PowerPC architecture.
\flowgraph{\resource{FALSE\\source code} \ar[r] & \toolbox{falppc64} \ar[r] & \resource{object file}}
\seefalse\seeppc\seeobject
}

\providecommand{\falrisc}{
\toolsection{falrisc} is a compiler for the FALSE programming language targeting the RISC hardware architecture.
It generates machine code for RISC processors from programs written in FALSE and stores it in corresponding object files.
\flowgraph{\resource{FALSE\\source code} \ar[r] & \toolbox{falrisc} \ar[r] & \resource{object file}}
\seefalse\seerisc\seeobject
}

\providecommand{\falwasm}{
\toolsection{falwasm} is a compiler for the FALSE programming language targeting the WebAssembly architecture.
It generates machine code for WebAssembly targets from programs written in FALSE and stores it in corresponding object files.
\flowgraph{\resource{FALSE\\source code} \ar[r] & \toolbox{falwasm} \ar[r] & \resource{object file}}
\seefalse\seewasm\seeobject
}

% Oberon tools

\providecommand{\obprint}{
\toolsection{obprint} is a pretty printer for the Oberon programming language.
It reformats the source code of Oberon modules and writes it to the standard output stream.
\flowgraph{\resource{Oberon\\source code} \ar[r] & \toolbox{obprint} \ar[r] & \resource{reformatted\\source code}}
\seeoberon
}

\providecommand{\obcheck}{
\toolsection{obcheck} is a syntactic and semantic checker for the Oberon programming language.
It just performs syntactic and semantic checks on Oberon modules and writes its diagnostic messages to the standard error stream.
In addition, it stores the interface of each module in a symbol file which is required when other modules import the module.
\flowgraph{\resource{Oberon\\source code} \ar[r] & \toolbox{obcheck} \ar[r] \ar@/l/[d] & \resource{diagnostic\\messages} \\ \variable{ECSIMPORT} \ar[ru] & \resource{symbol\\files} \ar@/r/[u]}
\seeoberon
}

\providecommand{\obdump}{
\toolsection{obdump} is a serializer for the Oberon programming language.
It dumps the complete internal representation of modules written in Oberon into an XML document.
\debuggingtool
\flowgraph{\resource{Oberon\\source code} \ar[r] & \toolbox{obdump} \ar[r] \ar@/l/[d] & \resource{internal\\representation} \\ \variable{ECSIMPORT} \ar[ru] & \resource{symbol\\files} \ar@/r/[u]}
\seeoberon
}

\providecommand{\obrun}{
\toolsection{obrun} is an interpreter for the Oberon programming language.
It processes and executes modules written in Oberon.
This tool does neither generate nor process symbol files while interpreting modules.
If a module is imported by another one, its filename has to be named before the other one in the list of command-line arguments.
\flowgraph{\resource{Oberon\\source code} \ar[r] & \toolbox{obrun} \ar@/u/[r] & \resource{input/\\output} \ar@/d/[l]}
\seeoberon
}

\providecommand{\obcpp}{
\toolsection{obcpp} is a transpiler for the Oberon programming language.
It translates programs written in Oberon into \cpp{} programs and stores them in corresponding source and header files.
In addition, it stores the interface of each module in a symbol file which is required when other modules import the module.
The same interface is provided by the generated header file which can be used in other parts of the \cpp{} program.
\flowgraph{\resource{Oberon\\source code} \ar[r] & \toolbox{obcpp} \ar[r] \ar@/l/[d] \ar[rd] & \resource{\cpp{}\\source file} \\ \variable{ECSIMPORT} \ar[ru] & \resource{symbol\\files} \ar@/r/[u] & \resource{\cpp{}\\header file}}
\seeoberon\seecpp
}

\providecommand{\obdoc}{
\toolsection{obdoc} is a generic documentation generator for the Oberon programming language.
It processes several Oberon modules and assembles all information therein into a generic documentation.
In addition, it stores the interface of each module in a symbol file which is required when other modules import the module.
\debuggingtool
\flowgraph{\resource{Oberon\\source code} \ar[r] & \toolbox{obdoc} \ar[r] \ar@/l/[d] & \resource{generic\\documentation} \\ \variable{ECSIMPORT} \ar[ru] & \resource{symbol\\files} \ar@/r/[u]}
\seeoberon\seedocumentation
}

\providecommand{\obhtml}{
\toolsection{obhtml} is an HTML documentation generator for the Oberon programming language.
It processes several Oberon modules and assembles all information therein into an HTML document.
In addition, it stores the interface of each module in a symbol file which is required when other modules import the module.
\flowgraph{\resource{Oberon\\source code} \ar[r] & \toolbox{obhtml} \ar[r] \ar@/l/[d] & \resource{HTML\\document} \\ \variable{ECSIMPORT} \ar[ru] & \resource{symbol\\files} \ar@/r/[u]}
\seeoberon\seedocumentation
}

\providecommand{\oblatex}{
\toolsection{oblatex} is a Latex documentation generator for the Oberon programming language.
It processes several Oberon modules and assembles all information therein into a Latex document.
In addition, it stores the interface of each module in a symbol file which is required when other modules import the module.
\flowgraph{\resource{Oberon\\source code} \ar[r] & \toolbox{oblatex} \ar[r] \ar@/l/[d] & \resource{Latex\\document} \\ \variable{ECSIMPORT} \ar[ru] & \resource{symbol\\files} \ar@/r/[u]}
\seeoberon\seedocumentation
}

\providecommand{\obcode}{
\toolsection{obcode} is an intermediate code generator for the Oberon programming language.
It generates intermediate code from modules written in Oberon and stores it in corresponding assembly files.
In addition, it stores the interface of each module in a symbol file which is required when other modules import the module.
Programs generated with this tool require additional runtime support that is stored in the \file{ob\-code\-run} library file.
\debuggingtool
\flowgraph{\resource{Oberon\\source code} \ar[r] & \toolbox{obcode} \ar[r] \ar@/l/[d] & \resource{intermediate\\code} \\ \variable{ECSIMPORT} \ar[ru] & \resource{symbol\\files} \ar@/r/[u]}
\seeoberon\seeassembly\seecode
}

\providecommand{\obamda}{
\toolsection{obamd16} is a compiler for the Oberon programming language targeting the AMD64 hardware architecture.
It generates machine code for AMD64 processors from modules written in Oberon and stores it in corresponding object files.
The compiler generates machine code for the 16-bit operating mode defined by the AMD64 architecture.
For debugging purposes, it also creates a debugging information file as well as an assembly file containing a listing of the generated machine code.
In addition, it stores the interface of each module in a symbol file which is required when other modules import the module.
Programs generated with this compiler require additional runtime support that is stored in the \file{ob\-amd16\-run} library file.
\flowgraph{\resource{Oberon\\source code} \ar[r] & \toolbox{obamd16} \ar[r] \ar@/l/[d] \ar[rd] & \resource{object file} \\ \variable{ECSIMPORT} \ar[ru] & \resource{symbol\\files} \ar@/r/[u] & \resource{debugging\\information}}
\seeoberon\seeassembly\seeamd\seeobject\seedebugging
}

\providecommand{\obamdb}{
\toolsection{obamd32} is a compiler for the Oberon programming language targeting the AMD64 hardware architecture.
It generates machine code for AMD64 processors from modules written in Oberon and stores it in corresponding object files.
The compiler generates machine code for the 32-bit operating mode defined by the AMD64 architecture.
For debugging purposes, it also creates a debugging information file as well as an assembly file containing a listing of the generated machine code.
In addition, it stores the interface of each module in a symbol file which is required when other modules import the module.
Programs generated with this compiler require additional runtime support that is stored in the \file{ob\-amd32\-run} library file.
\flowgraph{\resource{Oberon\\source code} \ar[r] & \toolbox{obamd32} \ar[r] \ar@/l/[d] \ar[rd] & \resource{object file} \\ \variable{ECSIMPORT} \ar[ru] & \resource{symbol\\files} \ar@/r/[u] & \resource{debugging\\information}}
\seeoberon\seeassembly\seeamd\seeobject\seedebugging
}

\providecommand{\obamdc}{
\toolsection{obamd64} is a compiler for the Oberon programming language targeting the AMD64 hardware architecture.
It generates machine code for AMD64 processors from modules written in Oberon and stores it in corresponding object files.
The compiler generates machine code for the 64-bit operating mode defined by the AMD64 architecture.
For debugging purposes, it also creates a debugging information file as well as an assembly file containing a listing of the generated machine code.
In addition, it stores the interface of each module in a symbol file which is required when other modules import the module.
Programs generated with this compiler require additional runtime support that is stored in the \file{ob\-amd64\-run} library file.
\flowgraph{\resource{Oberon\\source code} \ar[r] & \toolbox{obamd64} \ar[r] \ar@/l/[d] \ar[rd] & \resource{object file} \\ \variable{ECSIMPORT} \ar[ru] & \resource{symbol\\files} \ar@/r/[u] & \resource{debugging\\information}}
\seeoberon\seeassembly\seeamd\seeobject\seedebugging
}

\providecommand{\obarma}{
\toolsection{obarma32} is a compiler for the Oberon programming language targeting the ARM hardware architecture.
It generates machine code for ARM processors executing A32 instructions from modules written in Oberon and stores it in corresponding object files.
For debugging purposes, it also creates a debugging information file as well as an assembly file containing a listing of the generated machine code.
In addition, it stores the interface of each module in a symbol file which is required when other modules import the module.
Programs generated with this compiler require additional runtime support that is stored in the \file{ob\-arma32\-run} library file.
\flowgraph{\resource{Oberon\\source code} \ar[r] & \toolbox{obarma32} \ar[r] \ar@/l/[d] \ar[rd] & \resource{object file} \\ \variable{ECSIMPORT} \ar[ru] & \resource{symbol\\files} \ar@/r/[u] & \resource{debugging\\information}}
\seeoberon\seeassembly\seearm\seeobject\seedebugging
}

\providecommand{\obarmb}{
\toolsection{obarma64} is a compiler for the Oberon programming language targeting the ARM hardware architecture.
It generates machine code for ARM processors executing A64 instructions from modules written in Oberon and stores it in corresponding object files.
For debugging purposes, it also creates a debugging information file as well as an assembly file containing a listing of the generated machine code.
In addition, it stores the interface of each module in a symbol file which is required when other modules import the module.
Programs generated with this compiler require additional runtime support that is stored in the \file{ob\-arma64\-run} library file.
\flowgraph{\resource{Oberon\\source code} \ar[r] & \toolbox{obarma64} \ar[r] \ar@/l/[d] \ar[rd] & \resource{object file} \\ \variable{ECSIMPORT} \ar[ru] & \resource{symbol\\files} \ar@/r/[u] & \resource{debugging\\information}}
\seeoberon\seeassembly\seearm\seeobject\seedebugging
}

\providecommand{\obarmc}{
\toolsection{obarmt32} is a compiler for the Oberon programming language targeting the ARM hardware architecture.
It generates machine code for ARM processors without floating-point extension executing T32 instructions from modules written in Oberon and stores it in corresponding object files.
For debugging purposes, it also creates a debugging information file as well as an assembly file containing a listing of the generated machine code.
In addition, it stores the interface of each module in a symbol file which is required when other modules import the module.
Programs generated with this compiler require additional runtime support that is stored in the \file{ob\-armt32\-run} library file.
\flowgraph{\resource{Oberon\\source code} \ar[r] & \toolbox{obarmt32} \ar[r] \ar@/l/[d] \ar[rd] & \resource{object file} \\ \variable{ECSIMPORT} \ar[ru] & \resource{symbol\\files} \ar@/r/[u] & \resource{debugging\\information}}
\seeoberon\seeassembly\seearm\seeobject\seedebugging
}

\providecommand{\obarmcfpe}{
\toolsection{obarmt32fpe} is a compiler for the Oberon programming language targeting the ARM hardware architecture.
It generates machine code for ARM processors with floating-point extension executing T32 instructions from modules written in Oberon and stores it in corresponding object files.
For debugging purposes, it also creates a debugging information file as well as an assembly file containing a listing of the generated machine code.
In addition, it stores the interface of each module in a symbol file which is required when other modules import the module.
Programs generated with this compiler require additional runtime support that is stored in the \file{ob\-armt32\-fpe\-run} library file.
\flowgraph{\resource{Oberon\\source code} \ar[r] & \toolbox{obarmt32fpe} \ar[r] \ar@/l/[d] \ar[rd] & \resource{object file} \\ \variable{ECSIMPORT} \ar[ru] & \resource{symbol\\files} \ar@/r/[u] & \resource{debugging\\information}}
\seeoberon\seeassembly\seearm\seeobject\seedebugging
}

\providecommand{\obavr}{
\toolsection{obavr} is a compiler for the Oberon programming language targeting the AVR hardware architecture.
It generates machine code for AVR processors from modules written in Oberon and stores it in corresponding object files.
For debugging purposes, it also creates a debugging information file as well as an assembly file containing a listing of the generated machine code.
In addition, it stores the interface of each module in a symbol file which is required when other modules import the module.
Programs generated with this compiler require additional runtime support that is stored in the \file{ob\-avr\-run} library file.
\flowgraph{\resource{Oberon\\source code} \ar[r] & \toolbox{obavr} \ar[r] \ar@/l/[d] \ar[rd] & \resource{object file} \\ \variable{ECSIMPORT} \ar[ru] & \resource{symbol\\files} \ar@/r/[u] & \resource{debugging\\information}}
\seeoberon\seeassembly\seeavr\seeobject\seedebugging
}

\providecommand{\obavrtt}{
\toolsection{obavr32} is a compiler for the Oberon programming language targeting the AVR32 hardware architecture.
It generates machine code for AVR32 processors from modules written in Oberon and stores it in corresponding object files.
For debugging purposes, it also creates a debugging information file as well as an assembly file containing a listing of the generated machine code.
In addition, it stores the interface of each module in a symbol file which is required when other modules import the module.
Programs generated with this compiler require additional runtime support that is stored in the \file{ob\-avr32\-run} library file.
\flowgraph{\resource{Oberon\\source code} \ar[r] & \toolbox{obavr32} \ar[r] \ar@/l/[d] \ar[rd] & \resource{object file} \\ \variable{ECSIMPORT} \ar[ru] & \resource{symbol\\files} \ar@/r/[u] & \resource{debugging\\information}}
\seeoberon\seeassembly\seeavrtt\seeobject\seedebugging
}

\providecommand{\obmabk}{
\toolsection{obm68k} is a compiler for the Oberon programming language targeting the M68000 hardware architecture.
It generates machine code for M68000 processors from modules written in Oberon and stores it in corresponding object files.
For debugging purposes, it also creates a debugging information file as well as an assembly file containing a listing of the generated machine code.
In addition, it stores the interface of each module in a symbol file which is required when other modules import the module.
Programs generated with this compiler require additional runtime support that is stored in the \file{ob\-m68k\-run} library file.
\flowgraph{\resource{Oberon\\source code} \ar[r] & \toolbox{obm68k} \ar[r] \ar@/l/[d] \ar[rd] & \resource{object file} \\ \variable{ECSIMPORT} \ar[ru] & \resource{symbol\\files} \ar@/r/[u] & \resource{debugging\\information}}
\seeoberon\seeassembly\seemabk\seeobject\seedebugging
}

\providecommand{\obmibl}{
\toolsection{obmibl} is a compiler for the Oberon programming language targeting the MicroBlaze hardware architecture.
It generates machine code for MicroBlaze processors from modules written in Oberon and stores it in corresponding object files.
For debugging purposes, it also creates a debugging information file as well as an assembly file containing a listing of the generated machine code.
In addition, it stores the interface of each module in a symbol file which is required when other modules import the module.
Programs generated with this compiler require additional runtime support that is stored in the \file{ob\-mibl\-run} library file.
\flowgraph{\resource{Oberon\\source code} \ar[r] & \toolbox{obmibl} \ar[r] \ar@/l/[d] \ar[rd] & \resource{object file} \\ \variable{ECSIMPORT} \ar[ru] & \resource{symbol\\files} \ar@/r/[u] & \resource{debugging\\information}}
\seeoberon\seeassembly\seemibl\seeobject\seedebugging
}

\providecommand{\obmipsa}{
\toolsection{obmips32} is a compiler for the Oberon programming language targeting the MIPS32 hardware architecture.
It generates machine code for MIPS32 processors from modules written in Oberon and stores it in corresponding object files.
For debugging purposes, it also creates a debugging information file as well as an assembly file containing a listing of the generated machine code.
In addition, it stores the interface of each module in a symbol file which is required when other modules import the module.
Programs generated with this compiler require additional runtime support that is stored in the \file{ob\-mips32\-run} library file.
\flowgraph{\resource{Oberon\\source code} \ar[r] & \toolbox{obmips32} \ar[r] \ar@/l/[d] \ar[rd] & \resource{object file} \\ \variable{ECSIMPORT} \ar[ru] & \resource{symbol\\files} \ar@/r/[u] & \resource{debugging\\information}}
\seeoberon\seeassembly\seemips\seeobject\seedebugging
}

\providecommand{\obmipsb}{
\toolsection{obmips64} is a compiler for the Oberon programming language targeting the MIPS64 hardware architecture.
It generates machine code for MIPS64 processors from modules written in Oberon and stores it in corresponding object files.
For debugging purposes, it also creates a debugging information file as well as an assembly file containing a listing of the generated machine code.
In addition, it stores the interface of each module in a symbol file which is required when other modules import the module.
Programs generated with this compiler require additional runtime support that is stored in the \file{ob\-mips64\-run} library file.
\flowgraph{\resource{Oberon\\source code} \ar[r] & \toolbox{obmips64} \ar[r] \ar@/l/[d] \ar[rd] & \resource{object file} \\ \variable{ECSIMPORT} \ar[ru] & \resource{symbol\\files} \ar@/r/[u] & \resource{debugging\\information}}
\seeoberon\seeassembly\seemips\seeobject\seedebugging
}

\providecommand{\obmmix}{
\toolsection{obmmix} is a compiler for the Oberon programming language targeting the MMIX hardware architecture.
It generates machine code for MMIX processors from modules written in Oberon and stores it in corresponding object files.
For debugging purposes, it also creates a debugging information file as well as an assembly file containing a listing of the generated machine code.
In addition, it stores the interface of each module in a symbol file which is required when other modules import the module.
Programs generated with this compiler require additional runtime support that is stored in the \file{ob\-mmix\-run} library file.
\flowgraph{\resource{Oberon\\source code} \ar[r] & \toolbox{obmmix} \ar[r] \ar@/l/[d] \ar[rd] & \resource{object file} \\ \variable{ECSIMPORT} \ar[ru] & \resource{symbol\\files} \ar@/r/[u] & \resource{debugging\\information}}
\seeoberon\seeassembly\seemmix\seeobject\seedebugging
}

\providecommand{\oborok}{
\toolsection{obor1k} is a compiler for the Oberon programming language targeting the OpenRISC 1000 hardware architecture.
It generates machine code for OpenRISC 1000 processors from modules written in Oberon and stores it in corresponding object files.
For debugging purposes, it also creates a debugging information file as well as an assembly file containing a listing of the generated machine code.
In addition, it stores the interface of each module in a symbol file which is required when other modules import the module.
Programs generated with this compiler require additional runtime support that is stored in the \file{ob\-or1k\-run} library file.
\flowgraph{\resource{Oberon\\source code} \ar[r] & \toolbox{obor1k} \ar[r] \ar@/l/[d] \ar[rd] & \resource{object file} \\ \variable{ECSIMPORT} \ar[ru] & \resource{symbol\\files} \ar@/r/[u] & \resource{debugging\\information}}
\seeoberon\seeassembly\seeorok\seeobject\seedebugging
}

\providecommand{\obppca}{
\toolsection{obppc32} is a compiler for the Oberon programming language targeting the PowerPC hardware architecture.
It generates machine code for PowerPC processors from modules written in Oberon and stores it in corresponding object files.
The compiler generates machine code for the 32-bit operating mode defined by the PowerPC architecture.
For debugging purposes, it also creates a debugging information file as well as an assembly file containing a listing of the generated machine code.
In addition, it stores the interface of each module in a symbol file which is required when other modules import the module.
Programs generated with this compiler require additional runtime support that is stored in the \file{ob\-ppc32\-run} library file.
\flowgraph{\resource{Oberon\\source code} \ar[r] & \toolbox{obppc32} \ar[r] \ar@/l/[d] \ar[rd] & \resource{object file} \\ \variable{ECSIMPORT} \ar[ru] & \resource{symbol\\files} \ar@/r/[u] & \resource{debugging\\information}}
\seeoberon\seeassembly\seeppc\seeobject\seedebugging
}

\providecommand{\obppcb}{
\toolsection{obppc64} is a compiler for the Oberon programming language targeting the PowerPC hardware architecture.
It generates machine code for PowerPC processors from modules written in Oberon and stores it in corresponding object files.
The compiler generates machine code for the 64-bit operating mode defined by the PowerPC architecture.
For debugging purposes, it also creates a debugging information file as well as an assembly file containing a listing of the generated machine code.
In addition, it stores the interface of each module in a symbol file which is required when other modules import the module.
Programs generated with this compiler require additional runtime support that is stored in the \file{ob\-ppc64\-run} library file.
\flowgraph{\resource{Oberon\\source code} \ar[r] & \toolbox{obppc64} \ar[r] \ar@/l/[d] \ar[rd] & \resource{object file} \\ \variable{ECSIMPORT} \ar[ru] & \resource{symbol\\files} \ar@/r/[u] & \resource{debugging\\information}}
\seeoberon\seeassembly\seeppc\seeobject\seedebugging
}

\providecommand{\obrisc}{
\toolsection{obrisc} is a compiler for the Oberon programming language targeting the RISC hardware architecture.
It generates machine code for RISC processors from modules written in Oberon and stores it in corresponding object files.
For debugging purposes, it also creates a debugging information file as well as an assembly file containing a listing of the generated machine code.
In addition, it stores the interface of each module in a symbol file which is required when other modules import the module.
Programs generated with this compiler require additional runtime support that is stored in the \file{ob\-risc\-run} library file.
\flowgraph{\resource{Oberon\\source code} \ar[r] & \toolbox{obrisc} \ar[r] \ar@/l/[d] \ar[rd] & \resource{object file} \\ \variable{ECSIMPORT} \ar[ru] & \resource{symbol\\files} \ar@/r/[u] & \resource{debugging\\information}}
\seeoberon\seeassembly\seerisc\seeobject\seedebugging
}

\providecommand{\obwasm}{
\toolsection{obwasm} is a compiler for the Oberon programming language targeting the WebAssembly architecture.
It generates machine code for WebAssembly targets from modules written in Oberon and stores it in corresponding object files.
For debugging purposes, it also creates a debugging information file as well as an assembly file containing a listing of the generated machine code.
In addition, it stores the interface of each module in a symbol file which is required when other modules import the module.
Programs generated with this compiler require additional runtime support that is stored in the \file{ob\-wasm\-run} library file.
\flowgraph{\resource{Oberon\\source code} \ar[r] & \toolbox{obwasm} \ar[r] \ar@/l/[d] \ar[rd] & \resource{object file} \\ \variable{ECSIMPORT} \ar[ru] & \resource{symbol\\files} \ar@/r/[u] & \resource{debugging\\information}}
\seeoberon\seeassembly\seewasm\seeobject\seedebugging
}

% converter tools

\providecommand{\dbgdwarf}{
\toolsection{dbgdwarf} is a DWARF debugging information converter tool.
It converts debugging information into the DWARF debugging data format and stores it in corresponding object files~\cite{dwarffile}.
The resulting debugging object files can be combined with runtime support that creates Executable and Linking Format (ELF) files~\cite{elffile}.
\flowgraph{\resource{debugging\\information} \ar[r] & \toolbox{dbgdwarf} \ar[r] & \resource{debugging\\object file}}
\seeobject\seedebugging
}

% assembler tools

\providecommand{\asmprint}{
\toolsection{asmprint} is a pretty printer for generic assembly code.
It reformats generic assembly code and writes it to the standard output stream.
\flowgraph{\resource{generic assembly\\source code} \ar[r] & \toolbox{asmprint} \ar[r] & \resource{reformatted\\source code}}
\seeassembly
}

\providecommand{\amdaasm}{
\toolsection{amd16asm} is an assembler for the AMD64 hardware architecture.
It translates assembly code into machine code for AMD64 processors and stores it in corresponding object files.
By default, the assembler generates machine code for the 16-bit operating mode defined by the AMD64 architecture.
\flowgraph{\resource{AMD16 assembly\\source code} \ar[r] & \toolbox{amd16asm} \ar[r] & \resource{object file}}
\seeassembly\seeamd\seeobject
}

\providecommand{\amdadism}{
\toolsection{amd16dism} is a disassembler for the AMD64 hardware architecture.
It translates machine code from object files targeting AMD64 processors into assembly code and writes it to the standard output stream.
It assumes that the machine code was generated for the 16-bit operating mode defined by the AMD64 architecture.
\flowgraph{\resource{object file} \ar[r] & \toolbox{amd16dism} \ar[r] & \resource{disassembly\\listing}}
\seeassembly\seeamd\seeobject
}

\providecommand{\amdbasm}{
\toolsection{amd32asm} is an assembler for the AMD64 hardware architecture.
It translates assembly code into machine code for AMD64 processors and stores it in corresponding object files.
By default, the assembler generates machine code for the 32-bit operating mode defined by the AMD64 architecture.
\flowgraph{\resource{AMD32 assembly\\source code} \ar[r] & \toolbox{amd32asm} \ar[r] & \resource{object file}}
\seeassembly\seeamd\seeobject
}

\providecommand{\amdbdism}{
\toolsection{amd32dism} is a disassembler for the AMD64 hardware architecture.
It translates machine code from object files targeting AMD64 processors into assembly code and writes it to the standard output stream.
It assumes that the machine code was generated for the 32-bit operating mode defined by the AMD64 architecture.
\flowgraph{\resource{object file} \ar[r] & \toolbox{amd32dism} \ar[r] & \resource{disassembly\\listing}}
\seeassembly\seeamd\seeobject
}

\providecommand{\amdcasm}{
\toolsection{amd64asm} is an assembler for the AMD64 hardware architecture.
It translates assembly code into machine code for AMD64 processors and stores it in corresponding object files.
By default, the assembler generates machine code for the 64-bit operating mode defined by the AMD64 architecture.
\flowgraph{\resource{AMD64 assembly\\source code} \ar[r] & \toolbox{amd64asm} \ar[r] & \resource{object file}}
\seeassembly\seeamd\seeobject
}

\providecommand{\amdcdism}{
\toolsection{amd64dism} is a disassembler for the AMD64 hardware architecture.
It translates machine code from object files targeting AMD64 processors into assembly code and writes it to the standard output stream.
It assumes that the machine code was generated for the 64-bit operating mode defined by the AMD64 architecture.
\flowgraph{\resource{object file} \ar[r] & \toolbox{amd64dism} \ar[r] & \resource{disassembly\\listing}}
\seeassembly\seeamd\seeobject
}

\providecommand{\armaasm}{
\toolsection{arma32asm} is an assembler for the ARM hardware architecture.
It translates assembly code into machine code for ARM processors executing A32 instructions and stores it in corresponding object files.
\flowgraph{\resource{ARM A32 assembly\\source code} \ar[r] & \toolbox{arma32asm} \ar[r] & \resource{object file}}
\seeassembly\seearm\seeobject
}

\providecommand{\armadism}{
\toolsection{arma32dism} is a disassembler for the ARM hardware architecture.
It translates machine code from object files targeting ARM processors executing A32 instructions into assembly code and writes it to the standard output stream.
\flowgraph{\resource{object file} \ar[r] & \toolbox{arma32dism} \ar[r] & \resource{disassembly\\listing}}
\seeassembly\seearm\seeobject
}

\providecommand{\armbasm}{
\toolsection{arma64asm} is an assembler for the ARM hardware architecture.
It translates assembly code into machine code for ARM processors executing A64 instructions and stores it in corresponding object files.
\flowgraph{\resource{ARM A64 assembly\\source code} \ar[r] & \toolbox{arma64asm} \ar[r] & \resource{object file}}
\seeassembly\seearm\seeobject
}

\providecommand{\armbdism}{
\toolsection{arma64dism} is a disassembler for the ARM hardware architecture.
It translates machine code from object files targeting ARM processors executing A64 instructions into assembly code and writes it to the standard output stream.
\flowgraph{\resource{object file} \ar[r] & \toolbox{arma64dism} \ar[r] & \resource{disassembly\\listing}}
\seeassembly\seearm\seeobject
}

\providecommand{\armcasm}{
\toolsection{armt32asm} is an assembler for the ARM hardware architecture.
It translates assembly code into machine code for ARM processors executing T32 instructions and stores it in corresponding object files.
\flowgraph{\resource{ARM T32 assembly\\source code} \ar[r] & \toolbox{armt32asm} \ar[r] & \resource{object file}}
\seeassembly\seearm\seeobject
}

\providecommand{\armcdism}{
\toolsection{armt32dism} is a disassembler for the ARM hardware architecture.
It translates machine code from object files targeting ARM processors executing T32 instructions into assembly code and writes it to the standard output stream.
\flowgraph{\resource{object file} \ar[r] & \toolbox{armt32dism} \ar[r] & \resource{disassembly\\listing}}
\seeassembly\seearm\seeobject
}

\providecommand{\avrasm}{
\toolsection{avrasm} is an assembler for the AVR hardware architecture.
It translates assembly code into machine code for AVR processors and stores it in corresponding object files.
The identifiers \texttt{RXL}, \texttt{RXH}, \texttt{RYL}, \texttt{RYH}, \texttt{RZL}, and \texttt{RZH} are predefined and name the corresponding registers.
The identifiers \texttt{SPL} and \texttt{SPH} are also predefined and evaluate to the address of the corresponding registers.
\flowgraph{\resource{AVR assembly\\source code} \ar[r] & \toolbox{avrasm} \ar[r] & \resource{object file}}
\seeassembly\seeavr\seeobject
}

\providecommand{\avrdism}{
\toolsection{avrdism} is a disassembler for the AVR hardware architecture.
It translates machine code from object files targeting AVR processors into assembly code and writes it to the standard output stream.
\flowgraph{\resource{object file} \ar[r] & \toolbox{avrdism} \ar[r] & \resource{disassembly\\listing}}
\seeassembly\seeavr\seeobject
}

\providecommand{\avrttasm}{
\toolsection{avr32asm} is an assembler for the AVR32 hardware architecture.
It translates assembly code into machine code for AVR32 processors and stores it in corresponding object files.
\flowgraph{\resource{AVR32 assembly\\source code} \ar[r] & \toolbox{avr32asm} \ar[r] & \resource{object file}}
\seeassembly\seeavrtt\seeobject
}

\providecommand{\avrttdism}{
\toolsection{avr32dism} is a disassembler for the AVR32 hardware architecture.
It translates machine code from object files targeting AVR32 processors into assembly code and writes it to the standard output stream.
\flowgraph{\resource{object file} \ar[r] & \toolbox{avr32dism} \ar[r] & \resource{disassembly\\listing}}
\seeassembly\seeavrtt\seeobject
}

\providecommand{\mabkasm}{
\toolsection{m68kasm} is an assembler for the M68000 hardware architecture.
It translates assembly code into machine code for M68000 processors and stores it in corresponding object files.
\flowgraph{\resource{68000 assembly\\source code} \ar[r] & \toolbox{m68kasm} \ar[r] & \resource{object file}}
\seeassembly\seemabk\seeobject
}

\providecommand{\mabkdism}{
\toolsection{m68kdism} is a disassembler for the M68000 hardware architecture.
It translates machine code from object files targeting M68000 processors into assembly code and writes it to the standard output stream.
\flowgraph{\resource{object file} \ar[r] & \toolbox{m68kdism} \ar[r] & \resource{disassembly\\listing}}
\seeassembly\seemabk\seeobject
}

\providecommand{\miblasm}{
\toolsection{miblasm} is an assembler for the MicroBlaze hardware architecture.
It translates assembly code into machine code for MicroBlaze processors and stores it in corresponding object files.
\flowgraph{\resource{MicroBlaze assembly\\source code} \ar[r] & \toolbox{miblasm} \ar[r] & \resource{object file}}
\seeassembly\seemibl\seeobject
}

\providecommand{\mibldism}{
\toolsection{mibldism} is a disassembler for the MicroBlaze hardware architecture.
It translates machine code from object files targeting MicroBlaze processors into assembly code and writes it to the standard output stream.
\flowgraph{\resource{object file} \ar[r] & \toolbox{mibldism} \ar[r] & \resource{disassembly\\listing}}
\seeassembly\seemibl\seeobject
}

\providecommand{\mipsaasm}{
\toolsection{mips32asm} is an assembler for the MIPS32 hardware architecture.
It translates assembly code into machine code for MIPS32 processors and stores it in corresponding object files.
\flowgraph{\resource{MIPS32 assembly\\source code} \ar[r] & \toolbox{mips32asm} \ar[r] & \resource{object file}}
\seeassembly\seemips\seeobject
}

\providecommand{\mipsadism}{
\toolsection{mips32dism} is a disassembler for the MIPS32 hardware architecture.
It translates machine code from object files targeting MIPS32 processors into assembly code and writes it to the standard output stream.
\flowgraph{\resource{object file} \ar[r] & \toolbox{mips32dism} \ar[r] & \resource{disassembly\\listing}}
\seeassembly\seemips\seeobject
}

\providecommand{\mipsbasm}{
\toolsection{mips64asm} is an assembler for the MIPS64 hardware architecture.
It translates assembly code into machine code for MIPS64 processors and stores it in corresponding object files.
\flowgraph{\resource{MIPS64 assembly\\source code} \ar[r] & \toolbox{mips64asm} \ar[r] & \resource{object file}}
\seeassembly\seemips\seeobject
}

\providecommand{\mipsbdism}{
\toolsection{mips64dism} is a disassembler for the MIPS64 hardware architecture.
It translates machine code from object files targeting MIPS64 processors into assembly code and writes it to the standard output stream.
\flowgraph{\resource{object file} \ar[r] & \toolbox{mips64dism} \ar[r] & \resource{disassembly\\listing}}
\seeassembly\seemips\seeobject
}

\providecommand{\mmixasm}{
\toolsection{mmixasm} is an assembler for the MMIX hardware architecture.
It translates assembly code into machine code for MMIX processors and stores it in corresponding object files.
The names of all special registers are predefined and evaluate to the corresponding number.
\flowgraph{\resource{MMIX assembly\\source code} \ar[r] & \toolbox{mmixasm} \ar[r] & \resource{object file}}
\seeassembly\seemmix\seeobject
}

\providecommand{\mmixdism}{
\toolsection{mmixdism} is a disassembler for the MMIX hardware architecture.
It translates machine code from object files targeting MMIX processors into assembly code and writes it to the standard output stream.
\flowgraph{\resource{object file} \ar[r] & \toolbox{mmixdism} \ar[r] & \resource{disassembly\\listing}}
\seeassembly\seemmix\seeobject
}

\providecommand{\orokasm}{
\toolsection{or1kasm} is an assembler for the OpenRISC 1000 hardware architecture.
It translates assembly code into machine code for OpenRISC 1000 processors and stores it in corresponding object files.
\flowgraph{\resource{OpenRISC 1000 assembly\\source code} \ar[r] & \toolbox{or1kasm} \ar[r] & \resource{object file}}
\seeassembly\seeorok\seeobject
}

\providecommand{\orokdism}{
\toolsection{or1kdism} is a disassembler for the OpenRISC 1000 hardware architecture.
It translates machine code from object files targeting OpenRISC 1000 processors into assembly code and writes it to the standard output stream.
\flowgraph{\resource{object file} \ar[r] & \toolbox{or1kdism} \ar[r] & \resource{disassembly\\listing}}
\seeassembly\seeorok\seeobject
}

\providecommand{\ppcaasm}{
\toolsection{ppc32asm} is an assembler for the PowerPC hardware architecture.
It translates assembly code into machine code for PowerPC processors and stores it in corresponding object files.
By default, the assembler generates machine code for the 32-bit operating mode defined by the PowerPC architecture.
\flowgraph{\resource{PowerPC assembly\\source code} \ar[r] & \toolbox{ppc32asm} \ar[r] & \resource{object file}}
\seeassembly\seeppc\seeobject
}

\providecommand{\ppcadism}{
\toolsection{ppc32dism} is a disassembler for the PowerPC hardware architecture.
It translates machine code from object files targeting PowerPC processors into assembly code and writes it to the standard output stream.
It assumes that the machine code was generated for the 32-bit operating mode defined by the PowerPC architecture.
\flowgraph{\resource{object file} \ar[r] & \toolbox{ppc32dism} \ar[r] & \resource{disassembly\\listing}}
\seeassembly\seeppc\seeobject
}

\providecommand{\ppcbasm}{
\toolsection{ppc64asm} is an assembler for the PowerPC hardware architecture.
It translates assembly code into machine code for PowerPC processors and stores it in corresponding object files.
By default, the assembler generates machine code for the 64-bit operating mode defined by the PowerPC architecture.
\flowgraph{\resource{PowerPC assembly\\source code} \ar[r] & \toolbox{ppc64asm} \ar[r] & \resource{object file}}
\seeassembly\seeppc\seeobject
}

\providecommand{\ppcbdism}{
\toolsection{ppc64dism} is a disassembler for the PowerPC hardware architecture.
It translates machine code from object files targeting PowerPC processors into assembly code and writes it to the standard output stream.
It assumes that the machine code was generated for the 64-bit operating mode defined by the PowerPC architecture.
\flowgraph{\resource{object file} \ar[r] & \toolbox{ppc64dism} \ar[r] & \resource{disassembly\\listing}}
\seeassembly\seeppc\seeobject
}

\providecommand{\riscasm}{
\toolsection{riscasm} is an assembler for the RISC hardware architecture.
It translates assembly code into machine code for RISC processors and stores it in corresponding object files.
The names of all special registers are predefined and evaluate to the corresponding number.
\flowgraph{\resource{RISC assembly\\source code} \ar[r] & \toolbox{riscasm} \ar[r] & \resource{object file}}
\seeassembly\seerisc\seeobject
}

\providecommand{\riscdism}{
\toolsection{riscdism} is a disassembler for the RISC hardware architecture.
It translates machine code from object files targeting RISC processors into assembly code and writes it to the standard output stream.
\flowgraph{\resource{object file} \ar[r] & \toolbox{riscdism} \ar[r] & \resource{disassembly\\listing}}
\seeassembly\seerisc\seeobject
}

\providecommand{\wasmasm}{
\toolsection{wasmasm} is an assembler for the WebAssembly architecture.
It translates assembly code into machine code for WebAssembly targets and stores it in corresponding object files.
The names of all special registers are predefined and evaluate to the corresponding number.
\flowgraph{\resource{WebAssembly assembly\\source code} \ar[r] & \toolbox{wasmasm} \ar[r] & \resource{object file}}
\seeassembly\seewasm\seeobject
}

\providecommand{\wasmdism}{
\toolsection{wasmdism} is a disassembler for the WebAssembly architecture.
It translates machine code from object files targeting WebAssembly targets into assembly code and writes it to the standard output stream.
\flowgraph{\resource{object file} \ar[r] & \toolbox{wasmdism} \ar[r] & \resource{disassembly\\listing}}
\seeassembly\seewasm\seeobject
}

% linker tools

\providecommand{\linklib}{
\toolsection{linklib} is an object file combiner.
It creates a static library file by combining all object files given to it into a single one.
\flowgraph{\resource{object files} \ar[r] & \toolbox{linklib} \ar[r] & \resource{library file}}
\seeobject
}

\providecommand{\linkbin}{
\toolsection{linkbin} is a linker for plain binary files.
It links all object files given to it into a single image and stores it in a binary file that begins with the first linked section.
It also creates a map file that lists the address, type, name and size of all used sections.
The filename extension of the resulting binary file can be specified by putting it into a constant data section called \texttt{\_extension}.
\flowgraph{\resource{object files} \ar[r] & \toolbox{linkbin} \ar[r] \ar[d] & \resource{binary file} \\ & \resource{map file}}
\seeobject
}

\providecommand{\linkmem}{
\toolsection{linkmem} is a linker for plain binary files partitioned into random-access and read-only memory.
It links all object files given to it into two distinct images, one for data sections and one for code and constant data sections, and stores each image in a binary file that begins with the first linked section of the corresponding type.
It also creates a map file that lists the address, type, name and size of all used sections.
\flowgraph{\resource{object files} \ar[r] & \toolbox{linkmem} \ar[r] \ar[d] & \resource{RAM file/\\ROM file} \\ & \resource{map file}}
\seeobject
}

\providecommand{\linkprg}{
\toolsection{linkprg} is a linker for GEMDOS executable files.
It links all object files given to it into a single image and stores the image in an Atari GEMDOS executable file~\cite{gemdosfile}.
It also creates a map file that lists the address relative to the text segment, type, name and size of all used sections.
The filename extension of the resulting executable file can be specified by putting it into a constant data section called \texttt{\_extension}.
The GEMDOS executable file format requires all patch patterns of absolute link patches to consist of four full bitmasks with descending offsets.
\flowgraph{\resource{object files} \ar[r] & \toolbox{linkprg} \ar[r] \ar[d] & \resource{executable file} \\ & \resource{map file}}
\seeobject
}

\providecommand{\linkhex}{
\toolsection{linkhex} is a linker for Intel HEX files.
It links all code sections of the object files given to it into single image and stores the image in an Intel HEX file~\cite{hexfile} that begins with the first linked section.
It also creates a map file that lists the address, type, name and size of all used sections.
\flowgraph{\resource{object files} \ar[r] & \toolbox{linkhex} \ar[r] \ar[d] & \resource{HEX file} \\ & \resource{map file}}
\seeobject
}

\providecommand{\mapsearch}{
\toolsection{mapsearch} is a debugging tool.
It searches map files generated by linker tools for the name of a binary section that encompasses a memory address read from the standard input stream.
If additionally provided with one or more object files, it also stores an excerpt thereof in a separate object file called map search result which only contains the identified binary section for disassembling purposes.
\flowgraph{& \resource{map files/\\object files} \ar[d] \\ \resource{memory\\address} \ar[r] & \toolbox{mapsearch} \ar[r] \ar[d] & \resource{section name/\\relative offset} \\ & \resource{object file\\excerpt}}
\seeobject
}

\renewcommand{\seeobject}{}

\startchapter{Object Files}{Object File Representation}{object}
{This \documentation{} describes the purpose and the open format of object files which are used by the \ecs{} to represent generic binary code and data.
Additionally, it describes the functionality and interface of the linker tools provided by the \ecs{}.}

\epigraph{The finest eloquence is that which gets things done.}{David Lloyd George}

\section{Introduction}

\emph{Object files} are containers for all binary information that is required to build executable programs.
This includes information about so-called \emph{sections} that describe the contents and location of the code and the global data of a program.
Additionally, object files contain information about how these sections are interconnected.

The \ecs{} features several assemblers for different hardware architectures as well as compilers for different programming languages.
Although all of these tools process different types of input files, they all generate the same type of output file, namely one object file per translation unit.
Other tools like disassemblers and linkers on the other hand are able to continue processing the generated object files.
Linkers for example allocate space for all sections and establish the necessary interconnections in order to create an executable binary program.
Which tools actually generated the object files in the first place is irrelevant during this linking process.
Object files therefore clearly separate the source code from its binary representation by forming a generic abstraction of code and data as shown in Figure~\ref{fig:objdataflow}.

\begin{figure}
\flowgraph{
\resource{language\\source code} \ar[d] & \resource{assembly\\source code} \ar[d] & \resource{debugging\\information} \ar[d] \\
\converter{Compiler} \ar[rd] & \converter{Assembler\vphantom{Compiler}} \ar[d] & \converter{Converter\vphantom{Compiler}} \ar[ld] \\
& \resource{object files} \ar[ld] \ar[d] \ar[rd] \\
\converter{Linker} \ar[d] & \converter{Disassembler} \ar[d] & \converter{Combiner} \ar[d] \\
\resource{executable\\program} & \resource{disassembly\\listing} & \resource{static\\library} \\
}\caption[Object files as a generic abstraction of code and data]{Object files as a generic abstraction of code and data in-between the various kinds of tools of the \ecs{}}
\label{fig:objdataflow}
\end{figure}

Combining object files to build programs offers several advantages.
First, it enables \emph{interoperability}\index{Interoperability} between different compilers and assemblers which allows writing code and data in one programming language and using it from another.
Furthermore, using object files as a binary intermediate representation of a program effectively decouples the compilation from the linking stage and allows compilers and assemblers to become completely independent from the runtime environments targeted by linkers and their actual executable file formats.
Additionally, the use of object files enables \emph{separate compilation}\index{Separate compilation} which allows saving development time by compiling only those parts of a program that have actually changed instead of all of them.
This also allows \emph{statically linking}\index{Statically linking} against precompiled object and library files which typically provide the necessary runtime support for programming languages, the hardware architecture and runtime environment, as well as additional external libraries.

\section{Object File Structure}

Object files consist of a list of sections called \emph{binary sections}\index{Binary sections}.
Each binary section describes a contiguous sequence of relocatable binary data like machine code and contains information about how the section interconnects with other binary sections.
Sections have a unique name and carry additional information about their type and the way they should be allocated in memory by the linker.
This section describes all of this information and how it is used during linking.

\subsection{Section Types}\label{sec:objsectiontypes}

Each binary section is characterized by its \emph{section type}\index{Section types}.
Section types distinguish between code and data as required by Harvard architectures and specify how the corresponding memory shall be allocated by the linker.
The \ecs{} defines the following section types:

\begin{itemize}

\item Standard Code Sections\alignright\syntax{"code"}\nopagebreak

Standard code sections are typically used to model functions which are usually called by other code sections.
Standard code sections have no special requirements for their placement in memory other than an optional alignment.

\item Initializing Code Sections\alignright\syntax{"initcode"}\nopagebreak

Initializing code sections are placed by linkers in front of standard code sections.
This guarantees that initializing code sections are executed automatically before any other code section.
Unlike standard code sections, initializing code sections do not mimic functions as they are executed one after the other rather than being called.

\item Data Initializing Code Sections\alignright\syntax{"initdata"}\nopagebreak

This type of section is the same as initializing code sections except that they are executed at the very beginning of the program.
They are placed by linkers in front of other initializing code sections.
They therefore allow global data to be initialized upon the execution of other initializing code sections.
Data initializing code sections are typically used to initialize global data in environments that do not allow automatic memory initializations of data sections.
This explicitly includes writing data to constant data sections.

\item Standard Data Sections\alignright\syntax{"data"}\nopagebreak

Standard data sections provide the space and contents of global data.
This data is usually modified during the execution of a program by the code within code sections.
Data sections may contain predefined data that is initialized automatically by the linker.
If there are environments that do not allow global data to be initialized this way, data initializing code sections have to be used instead.

\item Constant Data Sections\alignright\syntax{"const"}\nopagebreak

Constant data sections are data sections that are not supposed to change their contents during the execution of the program.
Linkers may therefore place them in a special read-only memory if available.

\item Metadata Sections\alignright\syntax{"header" $\mid$ "trailer"}\nopagebreak

Metadata sections are constant data sections that contain metadata about a program.
Linkers place heading metadata sections at the beginning and trailing metadata sections at the end of the resulting binary file.
This allows mimicking the layout of some specific binary file format while linking several object files into a single binary file.

\end{itemize}

\subsection{Section Options}\label{sec:objsectionoptions}

The \ecs{} defines three freely combinable \emph{section options}\index{Section options}\index{Options, of sections}.
They describe how the linker shall treat sections that are never used or have the same name:

\begin{itemize}

\item Required Sections\alignright\syntax{"required"}\nopagebreak

By default, the linker does not allocate space for binary sections that are never used.
The required section option ensures that a section is always part of the binary program.
All other sections are only used if referenced directly or indirectly by required sections.
Standard code sections called \texttt{main} represent the \emph{entry point}\index{Entry points} of programs and are always implicitly required.
They are placed right after initializing code sections and in front of all other standard code sections such that their code gets executed automatically.

\item Duplicable Sections\alignright\syntax{"duplicable"}\nopagebreak

By default, the linker diagnoses an error whenever it encounters two sections that have the same name.
The duplicable section option specifies that two sections with the same name may be merged together if they are also equal otherwise.
Binary sections are equal if not only their name is the same but also all other information they carry including the binary data.
Whenever the linker encounters two sections that are equal and marked as duplicable, it discards one of them and redirects all references to the other one.
Duplicable sections are typically used for global constants like strings that may be defined in several object files but do not actually need to be duplicated in memory.

\item Replaceable Sections\alignright\syntax{"replaceable"}\nopagebreak

The replaceable section option specifies that a section is only used as long as there is no other section with the same name.
Whenever the linker encounters two sections that have the same name and one of them is marked as replaceable, it discards the replaceable one and redirects all references to the other one.

\end{itemize}

In addition to its name, a section may also have one or more so-called \emph{alias names} which allow referring to individual parts of its binary data by name.
The semantics of duplicable and replaceable sections as described above also applies to all of their alias names.

\subsection{Section Placement}\label{sec:objsectionplacement}

Usually, code and data sections are placed by linkers anywhere in memory\index{Section placement}\index{Placement, of sections}.
During the linking stage, linkers assign a unique memory address to every non-empty section that is actually used in the program.
References of a section to other sections are later resolved by their name and yield the actual address of the named section.
Referencing sections by name has the advantage that a programmer does not need to know or specify anything about the actual address of a section.

Sometimes however, the target runtime environment or hardware architecture requires programmers to assign a predefined address to a section.
The \ecs{} therefore allows programmers to define the address where a section has to be placed by the linker in the following two ways:

\begin{itemize}

\item Aligned Sections\alignright\syntax{"aligned" <Alignment>}\nopagebreak

Aligned sections have an arbitrary address that is a multiple of a positive power of two called their \emph{alignment}.
Usually, the underlying hardware imposes requirements on the alignment of data and code when they are referenced or called.
The \ecs{} aligns the address of aligned sections according to their alignment expressed in multiples of octets.

\item Fixed Sections\alignright\syntax{"fixed" <Origin>}\nopagebreak

Fixed sections have a predefined address called their \emph{origin}.
Before linkers layout any other section in memory, they place fixed sections at their origin expressed in octets.
All others sections are then placed according to the semantics of their section type and alignment.

\end{itemize}

Sections that have the same section type and alignment may additionally be assigned to a so-called \emph{group}.
This allows grouping otherwise unrelated sections together by forcing linkers to place them adjacent to each other.
Each group can be referred to by name which yields a virtual section that contains all sections within that group.
The groups themselves are placed consecutively in lexicographic order followed by all remaining sections of the same type.

Apart from the configurable placement of sections as described above and in Section~\ref{sec:objsectiontypes}, sections are in general placed in order of occurrence.
This allows initializing code sections to depend on the results of already executed code.
The only exception are sections without binary data which always precede all other sections of the same type and group.

\subsection{Section References}

Code in code sections usually needs to call functions defined in other code sections or to refer to data stored in data sections.
Data sections on the other hand are sometimes used to store the address of other data and code sections.
In order to establish this interconnection, the \ecs{} defines \emph{section links}\index{Section links}.
Section links allow a code or data section to refer to another section by name.
Besides placing sections in memory, the main task of the linker is to resolve these references.

Each binary section maintains a list of links where each link refers to a different section.
In addition to the name of the referenced section, each link contains a list of so-called \emph{link patches}.
A link patch defines where in the data of the binary section and how exactly linkers have to write the actual address of the referenced section.
This information includes the offset within the data of the section the address has to be patched as well as a displacement that is added to the address beforehand.
Additionally, the patch specifies whether the address is absolute or has to be relative to the actual memory address of the patch.
The address can also be scaled according to potential alignment constraints.
The \emph{patch pattern} finally specifies how the resulting address has to be written to memory in order to comply with predefined instruction encodings and endianness constraints.

\section{Object File Format}

Object files are stored as plain text files according to the complete syntax specification given in Figure~\ref{fig:objfileformat}.
They consist of the textual representation of an arbitrary number of binary sections according to the following syntax:

\begin{figure}
\centering\ifbook\small\fi\setlength{\grammarparsep}{0ex}
\begin{minipage}{32em}\begin{grammar}
<Object-File> = <Sections>$\opt$ \par
<Sections> = <Section> $\mid$ <Sections> <Section> \par
<Section> = <Type> <Size> <Options>$\opt$ <Name> <Aliases>$\opt$ <Group>$\opt$ <Placement> \\ <Segments>$\opt$ <Links>$\opt$ \par
<Type> = "code" $\mid$ "initcode" $\mid$ "initdata" $\mid$ \\ "data" $\mid$ "const" $\mid$ "header" $\mid$ "trailer" \par
<Size> = decimal-integer \par
<Options> = <Option> $\mid$ <Options> <Option> \par
<Option> = "required" $\mid$ "duplicable" $\mid$ "replaceable" \par
<Name> = double-quoted-string \par
<Aliases> = <Alias> $\mid$ <Aliases> <Alias> \par
<Alias> = <Offset> <Name> \par
<Offset> = decimal-integer \par
<Group> = <Name> \par
<Placement> = "aligned" <Alignment> $\mid$ "fixed" <Origin> \par
<Alignment> = decimal-integer \par
<Origin> = decimal-integer \par
<Segments> = <Segment> $\mid$ <Segments> <Segment> \par
<Segment> = <Offset> <Octets> \par
<Octets> = <Octet> $\mid$ <Octets><\ ><Octet> \par
<Octet> = <High-Quartet><\ ><Low-Quartet> \par
<High-Quartet> = hexadecimal-digit \par
<Low-Quartet> = hexadecimal-digit \par
<Links> = <Link> $\mid$ <Links> <Link> \par
<Link> = <Name> <Patches>$\opt$ \par
<Patches> = <Patch> $\mid$ <Patches> <Patch> \par
<Patch> = <Offset> <Mode> <Displacement> <Scale> <Pattern> \par
<Mode> = "abs" $\mid$ "rel" $\mid$ "siz" $\mid$ "ext" $\mid$ "pos" $\mid$ "idx" $\mid$ "cnt" \par
<Displacement> = signed-decimal-integer \par
<Scale> = decimal-integer \par
<Pattern> = <Masks> $\mid$ <Flag><\ ><Size> \par
<Masks> = <Mask> $\mid$ <Masks><\ ><Mask> \par
<Mask> = <Offset><\ ><Bitmask> \par
<Bitmask> = <Octet> \par
<Flag> = "+" $\mid$ "-" \par
\end{grammar}\end{minipage}
\caption{Syntax of the object file format}
\label{fig:objfileformat}
\end{figure}

\begin{quote}\begin{grammar}
<Object-File> = <Sections>$\opt$ \par
\end{grammar}\end{quote}

\subsection{Binary Sections}

Binary sections are represented in the object file as text according to the following syntax:

\begin{quote}\begin{grammar}
<Sections> = <Section> $\mid$ <Sections> <Section> \par
<Section> = <Type> <Size> <Options>$\opt$ <Name> <Aliases>$\opt$ <Group>$\opt$ <Placement> \\ <Segments>$\opt$ <Links>$\opt$ \par
<Type> = "code" $\mid$ "initcode" $\mid$ "initdata" $\mid$ \ifbook\\\fi "data" $\mid$ "const" $\mid$ "header" $\mid$ "trailer" \par
<Size> = decimal-integer \par
<Options> = <Option> $\mid$ <Options> <Option> \par
<Option> = "required" $\mid$ "duplicable" $\mid$ "replaceable" \par
<Name> = double-quoted-string \par
<Group> = <Name> \par
<Placement> = "aligned" <Alignment> $\mid$ "fixed" <Origin> \par
<Alignment> = decimal-integer \par
<Origin> = decimal-integer \par
\end{grammar}\end{quote}

The valid identifiers for the type of the binary section correspond to the types mentioned in Section~\ref{sec:objsectiontypes}.
The section size specifies the total number of octets occupied by the binary data of the section.
The valid identifiers for the options of the binary section correspond to the options mentioned in Section~\ref{sec:objsectionoptions}.
The optional group name and the alignment or origin of an aligned or fixed section influence its placement as described in Section~\ref{sec:objsectionplacement}.

\subsection{Alias Names}

The alias names of binary sections are represented in the object file as text according to the following syntax:

\begin{quote}\begin{grammar}
<Aliases> = <Alias> $\mid$ <Aliases> <Alias> \par
<Alias> = <Offset> <Name> \par
<Offset> = decimal-integer \par
<Name> = double-quoted-string \par
\end{grammar}\end{quote}

Alias names may not be duplicated and should differ from the name of the binary section.
The offset of an alias allows referring to a specific octet within the data of the binary section using a different name.
All names are represented using double-quoted strings and may contain standard escape sequences.

\subsection{Binary Data}

The data of a binary section is partitioned into segments which are represented in the object file as text according to the following syntax.
There may not be any white-space character in-between the hexadecimal digits of octets:

\begin{quote}\begin{grammar}
<Segments> = <Segment> $\mid$ <Segments> <Segment> \par
<Segment> = <Offset> <Octets> \par
<Offset> = decimal-integer \par
<Octets> = <Octet> $\mid$ <Octets><\ ><Octet> \par
<Octet> = <High-Quartet><\ ><Low-Quartet> \par
<High-Quartet> = hexadecimal-digit \par
<Low-Quartet> = hexadecimal-digit \par
\end{grammar}\end{quote}

Each segment contains one or more octets that represent the binary data of the section starting at the corresponding offset.
This offset plus the number of octets may not exceed the overall section size.
Overlapping segments define the binary data of a section in order of occurrence.
Binary data not covered by any segment is initialized to zero.

\subsection{Section Links}

The links of the binary sections are represented in the object file as text according to the following syntax:

\begin{quote}\begin{grammar}
<Links> = <Link> $\mid$ <Links> <Link> \par
<Link> = <Name> <Patches>$\opt$ \par
<Name> = double-quoted-string \par
\end{grammar}\end{quote}

The name of a link identifies the section that has to be referenced.
If it contains question marks, the actual name of the referenced section begins behind the last question mark.
If none of the sections named in front of a question mark are actually used, the referenced section evaluates either to zero or to the optional section named behind a colon following the referenced section.

\subsection{Link Patches}

The patches of section links are represented in the object file as text according to the following syntax:

\begin{quote}\begin{grammar}
<Patches> = <Patch> $\mid$ <Patches> <Patch> \par
<Patch> = <Offset> <Mode> <Displacement> <Scale> <Pattern> \par
<Offset> = decimal-integer \par
<Mode> = "abs" $\mid$ "rel" $\mid$ "siz" $\mid$ "ext" $\mid$ "pos" $\mid$ "idx" $\mid$ "cnt" \par
<Displacement> = signed-decimal-integer \par
<Scale> = decimal-integer \par
\end{grammar}\end{quote}

The offset specifies the position of the octet within the current binary section where the referenced section has to be patched by linkers.
The remainder of the patch specifies how the referenced section shall actually be evaluated.
The patch mode \texttt{abs} specifies that the referenced section evaluates to its absolute address expressed in octets.
The patch mode \texttt{rel} specifies that the absolute address of the referenced section shall be decremented by the target address of the patch in order to yield a relative address.
If the patch mode \texttt{siz} is used, the referenced section evaluates to its binary size expressed in octets.
The patch mode \texttt{ext} yields the absolute address of the referenced section plus its binary size.
Using the patch mode \texttt{pos}, the referenced section evaluates to its position relative to the beginning of its associated group expressed in octets.
The patch mode \texttt{idx} specifies that the referenced section evaluates to the index in the sequence of sections of its group.
The patch mode \texttt{cnt} can be used to reference groups and evaluates to the number of sections contained therein.
The evaluated value is then incremented or decremented by the optionally signed displacement and shifted to the right by the specified scale.
Overlapping link patches are applied in order of occurrence.

\subsection{Patch Patterns}

The patterns of link patches are represented in the object file as text according to the following syntax.
There may not be any white-space character in-between the elements of a pattern:

\begin{quote}\begin{grammar}
<Pattern> = <Masks> $\mid$ <Flag><\ ><Size> \par
<Masks> = <Mask> $\mid$ <Masks><\ ><Mask> \par
<Mask> = <Offset><\ ><Bitmask> \par
<Offset> = decimal-integer \par
<Bitmask> = <Octet> \par
<Octet> = <High-Quartet><\ ><Low-Quartet> \par
<High-Quartet> = hexadecimal-digit \par
<Low-Quartet> = hexadecimal-digit \par
<Flag> = "+" $\mid$ "-" \par
<Size> = decimal-integer \par
\end{grammar}\end{quote}

A pattern consists of a list of one to eight masks which define how the actual address has to be written to binary data.
Each mask contains an offset and the corresponding bitmask.
The single-digit offset specifies the relative displacement of the mask with respect to the enclosing link patch which may not exceed the overall section size.
The bitmask is given as the value of a single octet in hexadecimal form.
It specifies the number of bits consecutively taken from the address, as well as the mask which is used to write that part of the address value to binary data.
The pattern \texttt{5ff403} for example consists of two masks.
The first mask \texttt{5ff} tells the linker to write the value of the first eight bits of the address to all bits of the octet at the target address displaced by five.
The second mask \texttt{403} tells the linker to write the value of the ninth and tenth bit of the address to the lowest two bits of the octet at the target address displaced by four.
Patterns which consist exclusively of full bitmasks with consecutive offsets can also be represented by a flag indicating ascending or descending offsets followed by the number of masks contained therein.
The pattern \texttt{0ff1ff2ff} for example can be abbreviated by \texttt{+3}.

\section{Linker Tools}

Linkers process object files that were previously generated by the various compilers and assemblers of the \ecs{}.
The task of a linker is to merge all object files given to it into an executable program and to store the resulting binary image using a specific file format.
All of these linker tools accept command-line arguments which are taken as names of the actual object files.
If no arguments are provided, object files are read from the standard input stream.
Linkers create output files in the current working directory by reusing the name of the first object file whereas the filename extension gets replaced by an appropriate suffix.
\seeinterface\seeguide

\linklib
\linkbin
\linkmem
\linkhex
\linkprg
\mapsearch

\concludechapter

// Intermediate code instruction set definitions
// Copyright (C) Florian Negele

// This file is part of the Eigen Compiler Suite.

// The ECS is free software: you can redistribute it and/or modify
// it under the terms of the GNU General Public License as published by
// the Free Software Foundation, either version 3 of the License, or
// (at your option) any later version.

// The ECS is distributed in the hope that it will be useful,
// but WITHOUT ANY WARRANTY; without even the implied warranty of
// MERCHANTABILITY or FITNESS FOR A PARTICULAR PURPOSE.  See the
// GNU General Public License for more details.

// You should have received a copy of the GNU General Public License
// along with the ECS.  If not, see <https://www.gnu.org/licenses/>.

#ifndef CLASS
	#define CLASS(class, model1, model2, model3, model4)
#endif

#ifndef INSTR
	#define INSTR(name, mnem, class1, class2, class3, ...)
#endif

// instructions

INSTR (add,    ADD,    RegMem,        ImmRegAdrMem,  ImmRegAdrMem,  Addition)
INSTR (alias,  ALIAS,  Str,           None,          None,          Alias Name Definition)
INSTR (and,    AND,    RegMem,        ImmRegMem,     ImmRegMem,     Logical AND)
INSTR (array,  ARRAY,  Siz,           Siz,           None,          Array Type Declaration)
INSTR (asm,    ASM,    Str,           Siz,           Str,           Inline Assembly)
INSTR (br,     BR,     Off,           None,          None,          Unconditional Branch)
INSTR (break,  BREAK,  None,          None,          None,          Breakpoint)
INSTR (breq,   BREQ,   Off,           ImmRegAdrMem,  ImmRegAdrMem,  Branch if Equal)
INSTR (brge,   BRGE,   Off,           ImmRegAdrMem,  ImmRegAdrMem,  Branch if Greater Than or Equal)
INSTR (brlt,   BRLT,   Off,           ImmRegAdrMem,  ImmRegAdrMem,  Branch if Less Than)
INSTR (brne,   BRNE,   Off,           ImmRegAdrMem,  ImmRegAdrMem,  Branch if Not Equal)
INSTR (call,   CALL,   Function,      Siz,           None,          Function Call)
INSTR (conv,   CONV,   RegMem,        ImmRegAdrMem,  None,          Datum Conversion)
INSTR (copy,   COPY,   Pointer,       Pointer,       ImmRegMem,     Data Copy)
INSTR (def,    DEF,    ImmAdr,        None,          None,          Datum Definition)
INSTR (div,    DIV,    RegMem,        ImmRegMem,     ImmRegMem,     Division)
INSTR (enter,  ENTER,  Siz,           None,          None,          Stack Frame Creation)
INSTR (enum,   ENUM,   Off,           None,          None,          Enumeration Type Declaration)
INSTR (field,  FIELD,  Str,           Siz,           Imm,           Field Declaration)
INSTR (fill,   FILL,   Pointer,       ImmRegMem,     ImmRegAdrMem,  Data Initialization)
INSTR (fix,    FIX,    Reg,           None,          None,          Fix Register Mapping)
INSTR (func,   FUNC,   Off,           None,          None,          Function Type Declaration)
INSTR (jump,   JUMP,   Function,      None,          None,          Indirect Branch)
INSTR (leave,  LEAVE,  None,          None,          None,          Stack Frame Deletion)
INSTR (loc,    LOC,    Str,           Siz,           Siz,           Source Code Location)
INSTR (lsh,    LSH,    RegMem,        ImmRegMem,     ImmRegMem,     Left Shift)
INSTR (mod,    MOD,    RegMem,        ImmRegMem,     ImmRegMem,     Modulo)
INSTR (mov,    MOV,    RegMem,        ImmRegAdrMem,  None,          Datum Copy)
INSTR (mul,    MUL,    RegMem,        ImmRegMem,     ImmRegMem,     Multiplication)
INSTR (neg,    NEG,    RegMem,        ImmRegMem,     None,          Negation)
INSTR (nop,    NOP,    None,          None,          None,          No Operation)
INSTR (not,    NOT,    RegMem,        ImmRegMem,     None,          Logical NOT)
INSTR (or,     OR,     RegMem,        ImmRegMem,     ImmRegMem,     Logical OR)
INSTR (pop,    POP,    RegMem,        None,          None,          Pop from Stack)
INSTR (ptr,    PTR,    None,          None,          None,          Pointer Type Declaration)
INSTR (push,   PUSH,   ImmRegAdrMem,  None,          None,          Push onto Stack)
INSTR (rec,    REC,    Off,           Siz,           None,          Record Type Declaration)
INSTR (ref,    REF,    None,          None,          None,          Reference Type Declaration)
INSTR (req,    REQ,    Str,           None,          None,          Name Requirement)
INSTR (res,    RES,    Siz,           None,          None,          Space Reservation)
INSTR (ret,    RET,    None,          None,          None,          Return from Function)
INSTR (rsh,    RSH,    RegMem,        ImmRegMem,     ImmRegMem,     Right Shift)
INSTR (sub,    SUB,    RegMem,        ImmRegAdrMem,  ImmRegAdrMem,  Subtraction)
INSTR (sym,    SYM,    Off,           Str,           ImmRegMem,     Symbol Declaration)
INSTR (trap,   TRAP,   Siz,           None,          None,          Abnormal Program Termination)
INSTR (type,   TYPE,   StrTyp,        None,          None,          Basic Type Declaration)
INSTR (unfix,  UNFIX,  Reg,           None,          None,          Unfix Register Mapping)
INSTR (value,  VALUE,  Str,           Imm,           None,          Enumerator Declaration)
INSTR (void,   VOID,   None,          None,          None,          Void Type Declaration)
INSTR (xor,    XOR,    RegMem,        ImmRegMem,     ImmRegMem,     Logical Exclusive OR)

// operand classes

CLASS (Reg,           Register,   Void,      Void,     Void)
CLASS (Siz,           Size,       Void,      Void,     Void)
CLASS (Off,           Offset,     Void,      Void,     Void)
CLASS (Str,           String,     Void,      Void,     Void)
CLASS (StrTyp,        String,     Type,      Void,     Void)
CLASS (Imm,           Immediate,  Void,      Void,     Void)
CLASS (ImmAdr,        Immediate,  Address,   Void,     Void)
CLASS (RegMem,        Register,   Memory,    Void,     Void)
CLASS (ImmRegMem,     Immediate,  Register,  Memory,   Void)
CLASS (ImmRegAdrMem,  Immediate,  Register,  Address,  Memory)

#undef CLASS
#undef INSTR

% Debugging information representation
% Copyright (C) Florian Negele

% This file is part of the Eigen Compiler Suite.

% Permission is granted to copy, distribute and/or modify this document
% under the terms of the GNU Free Documentation License, Version 1.3
% or any later version published by the Free Software Foundation.

% You should have received a copy of the GNU Free Documentation License
% along with the ECS.  If not, see <https://www.gnu.org/licenses/>.

% Generic documentation utilities
% Copyright (C) Florian Negele

% This file is part of the Eigen Compiler Suite.

% Permission is granted to copy, distribute and/or modify this document
% under the terms of the GNU Free Documentation License, Version 1.3
% or any later version published by the Free Software Foundation.

% You should have received a copy of the GNU Free Documentation License
% along with the ECS.  If not, see <https://www.gnu.org/licenses/>.

\providecommand{\cpp}{C\texttt{++}}
\providecommand{\opt}{_\mathit{opt}}
\providecommand{\tool}[1]{\texttt{#1}}
\providecommand{\version}{Version 0.0.40}
\providecommand{\resource}[1]{*++\txt{#1}}
\providecommand{\ecs}{Eigen Compiler Suite}
\providecommand{\changed}[1]{\underline{#1}}
\providecommand{\toolbox}[1]{\converter{#1}}
\providecommand{\file}{}\renewcommand{\file}[1]{\texttt{#1}}
\providecommand{\alignright}{\hfill\linebreak[0]\hspace*{\fill}}
\providecommand{\converter}[1]{*++[F][F*:white][F,:gray]\txt{#1}}
\providecommand{\documentation}{\ifbook chapter\else document\fi}
\providecommand{\Documentation}{\ifbook Chapter\else Document\fi}
\providecommand{\variable}[1]{\resource{\texttt{\small#1}\\variable}}
\providecommand{\documentationref}[2]{\ifbook\ref{#1}\else``\href{#1}{#2}''~\cite{#1}\fi}
\providecommand{\objfile}[1]{\texttt{#1}\index[runtime]{#1 object file@\texttt{#1} object file}}
\providecommand{\libfile}[1]{\texttt{#1}\index[runtime]{#1 library file@\texttt{#1} library file}}
\providecommand{\epigraph}[2]{\ifbook\begin{quote}\flushright\textit{#1}\par--- #2\end{quote}\fi}
\providecommand{\environmentvariable}[1]{\texttt{#1}\index{Environment variables!#1@\texttt{#1}}}
\providecommand{\environment}[1]{\texttt{#1}\index[environment]{#1 environment@\texttt{#1} environment}}
\providecommand{\toolsection}{}\renewcommand{\toolsection}[1]{\subsection{#1}\label{\prefix:#1}\tool{#1}}
\providecommand{\instruction}{}\renewcommand{\instruction}[2]{\noindent\qquad\pdftooltip{\texttt{#1}}{#2}\refstepcounter{instruction}\par}
\providecommand{\flowgraph}{}\renewcommand{\flowgraph}[1]{\par\sffamily\begin{displaymath}\xymatrix@=4ex{#1}\end{displaymath}\normalfont\par}
\providecommand{\instructionset}{}\renewcommand{\instructionset}[4]{\setcounter{instruction}{0}\begin{multicols}{\ifbook#3\else#4\fi}[{\captionof{table}[#2]{#2 (\ref*{#1:instructions}~instructions)}\label{tab:#1set}\vspace{-2ex}}]\footnotesize\raggedcolumns\input{#1.set}\label{#1:instructions}\end{multicols}}

\providecommand{\gpl}{GNU General Public License}
\providecommand{\rse}{ECS Runtime Support Exception}
\providecommand{\fdl}{\href{https://www.gnu.org/licenses/fdl.html}{GNU Free Documentation License}}

\providecommand{\docbegin}{}
\providecommand{\docend}{}
\providecommand{\doclabel}[1]{\hypertarget{#1}}
\providecommand{\doclink}[2]{\hyperlink{#1}{#2}}
\providecommand{\docsection}[3]{\hypertarget{#1}{\subsection{#2}}\label{sec:#1}\index[library]{#2@#3}}
\providecommand{\docsectionstar}[1]{}
\providecommand{\docsubbegin}{\begin{description}}
\providecommand{\docsubend}{\end{description}}
\providecommand{\docsubsection}[3]{\item[\hypertarget{#1}{#2}]\index[library]{#2@#3}}
\providecommand{\docsubsectionstar}[1]{\smallskip}
\providecommand{\docsubsubsection}[3]{\docsubsection{#1}{#2}{#3}}
\providecommand{\docsubsubsectionstar}[1]{}
\providecommand{\docsubsubsubsection}[3]{}
\providecommand{\docsubsubsubsectionstar}[1]{}
\providecommand{\doctable}{}

\providecommand{\debuggingtool}{}\renewcommand{\debuggingtool}{This tool is provided for debugging purposes.
It allows exposing and modifying an internal data structure that is usually not accessible.
}

\providecommand{\interface}{All tools accept command-line arguments which are taken as names of plain text files containing the source code.
If no arguments are provided, the standard input stream is used instead.
Output files are generated in the current working directory and have the same name as the input file being processed whereas the filename extension gets replaced by an appropriate suffix.
\seeinterface
}

\providecommand{\license}{\noindent Copyright \copyright{} Florian Negele\par\medskip\noindent
Permission is granted to copy, distribute and/or modify this document under the terms of the
\fdl{}, Version 1.3 or any later version published by the \href{https://fsf.org/}{Free Software Foundation}.
}

\providecommand{\ecslogosurface}{
\fill[darkgray] (0,0,0) -- (0,0,3) -- (0,3,3) -- (0,3,1) -- (0,4,1) -- (0,4,3) -- (0,5,3) -- (0,5,0) -- (0,2,0) -- (0,2,2) -- (0,1,2) -- (0,1,0) -- cycle;
\fill[gray] (0,5,0) -- (0,5,3) -- (1,5,3) -- (1,5,1) -- (2,5,1) -- (2,5,3) -- (3,5,3) -- (3,5,0) -- cycle;
\fill[lightgray] (0,0,0) -- (0,1,0) -- (2,1,0) -- (2,4,0) -- (1,4,0) -- (1,3,0) -- (2,3,0) -- (2,2,0) -- (0,2,0) -- (0,5,0) -- (3,5,0) -- (3,0,0) -- cycle;
\begin{scope}[line width=0.5]
\begin{scope}[gray]
\draw (0,0,0) -- (0,1,0);
\draw (2,1,0) -- (2,2,0);
\draw (0,1,2) -- (0,2,2);
\draw (0,2,0) -- (0,5,0);
\draw (2,3,0) -- (2,4,0);
\end{scope}
\begin{scope}[lightgray]
\draw (0,1,0) -- (0,1,2);
\draw (0,3,1) -- (0,3,3);
\draw (0,5,0) -- (0,5,3);
\draw (2,5,1) -- (2,5,3);
\end{scope}
\begin{scope}[white]
\draw (0,1,0) -- (2,1,0);
\draw (1,3,0) -- (2,3,0);
\draw (0,5,0) -- (3,5,0);
\end{scope}
\end{scope}
}

\providecommand{\ecslogo}[1]{
\begin{tikzpicture}[scale={(#1)/((sin(45)+cos(45))*3cm)},x={({-cos(45)*1cm},{sin(45)*sin(30)*1cm})},y={({0cm},{(cos(30)*1cm})},z={({sin(45)*1cm},{cos(45)*sin(30)*1cm})}]
\begin{scope}[darkgray,line width=1]
\draw (0,0,0) -- (0,0,3) -- (0,3,3) -- (2,3,3) -- (2,5,3) -- (3,5,3) -- (3,5,0) -- (3,0,0) -- cycle;
\draw (0,3,1) -- (0,4,1) -- (0,4,3) -- (0,5,3) -- (1,5,3) -- (1,5,1) -- (2,5,1);
\draw (1,3,0) -- (1,4,0) -- (2,4,0);
\end{scope}
\fill[darkgray] (2,0,0) -- (2,0,3) -- (2,5,3) -- (2,5,1) -- (2,4,1) -- (2,4,0) -- cycle;
\fill[lightgray] (2,0,2) -- (0,0,2) -- (0,2,2) -- (2,2,2) -- cycle;
\fill[gray] (0,1,0) -- (2,1,0) -- (2,1,2) -- (0,1,2) -- cycle;
\fill[gray] (0,3,1) -- (0,3,3) -- (2,3,3) -- (2,3,0) -- (1,3,0) -- (1,3,1) -- cycle;
\ecslogosurface
\end{tikzpicture}
}

\providecommand{\shadowedecslogo}[3]{
\begin{tikzpicture}[scale={(#1)/((sin(#2)+cos(#2))*3cm)},x={({-cos(#2)*1cm},{sin(#2)*sin(#3)*1cm})},y={({0cm},{(cos(#3)*1cm})},z={({sin(#2)*1cm},{cos(#2)*sin(#3)*1cm})}]
\shade[top color=lightgray!50!white,bottom color=white,middle color=lightgray!50!white] (0,0,0) -- (3,0,0) -- (3,{-0.5-3*sin(#2)*sin(#3)/cos(#3)},0) -- (0,-0.5,0) -- cycle;
\shade[top color=darkgray!50!gray,bottom color=white,middle color=darkgray!50!white] (0,0,0) -- (0,0,3) -- (0,{-0.5-3*cos(#2)*sin(#3)/cos(#3)},3) -- (0,-0.5,0) -- cycle;
\begin{scope}[y={({(cos(#2)+sin(#2))*0.5cm},{(cos(#2)*sin(#3)-sin(#2)*sin(#3))*0.5cm})}]
\useasboundingbox (3,0,0) -- (0,0,0) -- (0,0,3);
\shade[left color=darkgray!80!black,right color=lightgray,middle color=gray] (0,0,0) -- (0,1,0) -- (0,1,0.5) -- (0,2,0) -- (0,5,0) -- (0,5,3) -- (1,5,3) -- (1,4,3) -- (1,4,2.5) -- (1,3,3) -- (2,5,3) -- (3,5,3) -- (3,0,3) -- cycle;
\clip (0,0,0) -- (0,0,3) -- ({-3*sin(#2)/cos(#2)},0,0) -- cycle;
\shade[left color=darkgray,right color=lightgray!50!gray] (0,0,0) -- (0,1,0) -- (0,1,0.5) -- (0,2,0) -- (0,5,0) -- (0,5,3) -- (1,5,3) -- (1,4,3) -- (1,4,2.5) -- (1,3,3) -- (2,5,3) -- (3,5,3) -- (3,0,3) -- cycle;
\end{scope}
\shade[left color=darkgray,right color=darkgray!80!black] (2,0,0) -- (2,0,3) -- (2,5,3) -- (2,5,1) -- (2,4,1) -- (2,4,0) -- cycle;
\shade[left color=darkgray!90!black,right color=gray!80!darkgray] (2,0,2) -- (0,0,2) -- (0,2,2) -- (2,2,2) -- cycle;
\shade[top color=darkgray!90!black,bottom color=gray!80!darkgray] (0,1,0) -- (2,1,0) -- (2,1,2) -- (0,1,2) -- cycle;
\shade[top color=darkgray!90!black,bottom color=gray!80!darkgray] (0,3,1) -- (0,3,3) -- (2,3,3) -- (2,3,0) -- (1,3,0) -- (1,3,1) -- cycle;
\fill[gray] (2,1,0) -- (1.5,1,0.5) -- (0,1,0.5) -- (0,1,0) -- cycle;
\fill[gray] (1,3,2) -- (0.5,3,2) -- (0.5,3,3) -- (1,3,3) -- cycle;
\fill[gray] (2,3,0) -- (1.5,3,0.5) -- (1,3,0.5) -- (1,3,0) -- cycle;
\ecslogosurface
\end{tikzpicture}
}

\providecommand{\cpplogo}[1]{
\begin{tikzpicture}[scale=(#1)/512em]
\fill[gray] (435.2794,398.7159) -- (247.1911,507.3075) .. controls (236.3563,513.5642) and (218.6240,513.5642) .. (207.7892,507.3075) -- (19.7009,398.7159) .. controls (8.8646,392.4606) and (0.0000,377.1043) .. (0.0000,364.5924) -- (0.0000,147.4076) .. controls (0.8430,132.8363) and (8.2856,120.7683) .. (19.7009,113.2842) -- (207.7892,4.6926) .. controls (218.6240,-1.5642) and (236.3564,-1.5642) .. (247.1911,4.6926) -- (435.2794,113.2842) .. controls (447.5273,121.4304) and (454.4987,133.6918) .. (454.9803,147.4076) -- (454.9803,364.5924) .. controls (454.5404,377.7571) and (446.6566,391.0351) .. (435.2794,398.7159) -- cycle(75.8301,255.9993) .. controls (74.9389,404.0881) and (273.2892,469.4783) .. (358.8263,331.8769) -- (293.1917,293.8965) .. controls (253.5702,359.4301) and (155.1909,335.9977) .. (151.6601,255.9993) .. controls (152.7204,182.2703) and (249.4137,148.0211) .. (293.1961,218.1065) -- (358.8308,180.1276) .. controls (283.4477,49.2645) and (79.6318,96.3470) .. (75.8301,255.9993) -- cycle(379.1503,247.5747) -- (362.2982,247.5747) -- (362.2982,230.7226) -- (345.4490,230.7226) -- (345.4490,247.5747) -- (328.5969,247.5747) -- (328.5969,264.4254) -- (345.4490,264.4254) -- (345.4490,281.2759) -- (362.2982,281.2759) -- (362.2982,264.4254) -- (379.1503,264.4254) -- cycle(442.3420,247.5747) -- (425.4899,247.5747) -- (425.4899,230.7226) -- (408.6408,230.7226) -- (408.6408,247.5747) -- (391.7886,247.5747) -- (391.7886,264.4254) -- (408.6408,264.4254) -- (408.6408,281.2759) -- (425.4899,281.2759) -- (425.4899,264.4254) -- (442.3420,264.4254) -- cycle;
\end{tikzpicture}
}

\providecommand{\fallogo}[1]{
\begin{tikzpicture}[scale=(#1)/512em]
\fill[gray] (185.7774,0.0000) .. controls (200.4486,15.9798) and (226.8966,8.7148) .. (235.0426,31.5836) .. controls (249.5297,58.0598) and (247.9581,97.9161) .. (280.3335,110.9762) .. controls (309.1690,120.3496) and (337.8406,104.2727) .. (366.5753,103.9379) .. controls (373.4449,111.5171) and (379.2885,128.2574) .. (383.9755,108.9744) .. controls (396.6979,102.5615) and (437.2808,107.6681) .. (426.9652,124.3252) .. controls (408.9822,121.0785) and (412.4742,146.0729) .. (426.5192,131.4996) .. controls (433.8413,120.8489) and (465.1541,126.5522) .. (441.9067,135.7950) .. controls (396.1879,157.7478) and (344.1112,161.5079) .. (298.5528,183.5702) .. controls (277.7471,193.5198) and (284.6941,218.7163) .. (285.2127,236.9640) .. controls (292.3599,316.2826) and (307.3929,394.6311) .. (317.1198,473.6154) .. controls (329.0637,505.4736) and (292.1195,528.5004) .. (265.9183,511.2761) .. controls (237.9284,499.2462) and (237.3684,465.2681) .. (230.9102,439.9421) .. controls (218.6692,374.3397) and (215.6307,306.9662) .. (198.1732,242.3977) .. controls (183.1379,232.7444) and (164.4245,256.0298) .. (149.0430,261.4799) .. controls (116.9328,279.2585) and (87.1822,308.5851) .. (48.2293,307.8914) .. controls (21.3220,306.9037) and (-15.9107,281.8761) .. (7.2921,252.7908) .. controls (29.7799,220.6177) and (67.5177,204.3028) .. (100.9287,185.9449) .. controls (130.8217,170.8906) and (161.1548,156.5903) .. (191.0278,141.5847) .. controls (196.1738,120.0520) and (186.6049,95.2409) .. (186.8382,72.4353) .. controls (185.5234,48.4204) and (183.1700,23.9341) .. (185.7774,0.0000) -- cycle;
\end{tikzpicture}
}

\providecommand{\oblogo}[1]{
\begin{tikzpicture}[scale=(#1)/512em]
\fill[gray] (160.3865,208.9117) .. controls (154.0879,214.6478) and (149.0735,221.2409) .. (145.4125,228.5384) .. controls (184.8790,248.4273) and (234.7122,269.8787) .. (297.5493,291.8782) .. controls (300.3943,281.4769) and (300.9552,268.7619) .. (300.4023,255.2389) .. controls (248.9909,244.7891) and (200.0310,225.9279) .. (160.3865,208.9117) -- cycle(225.7398,392.6996) .. controls (308.0209,392.1716) and (359.3326,345.9277) .. (368.7203,285.2098) .. controls (376.6742,197.1784) and (311.7194,141.3342) .. (205.4287,142.1456) .. controls (139.9485,141.4804) and (88.7155,166.1957) .. (73.5775,228.0086) .. controls (52.0297,320.3408) and (123.4078,391.0103) .. (225.7398,392.6996) -- cycle(216.0739,176.4733) .. controls (268.9183,179.2424) and (315.8292,206.5488) .. (312.7454,265.1139) .. controls (313.2769,315.6384) and (286.5993,353.4946) .. (216.6040,355.7934) .. controls (162.4657,355.7934) and (126.0914,317.5023) .. (126.0914,260.5103) .. controls (126.1733,214.2900) and (163.3363,176.2849) .. (216.0739,176.4733) -- cycle(76.4897,189.1754) .. controls (13.1586,147.5631) and (0.0000,119.4207) .. (0.0000,119.4207) -- (90.6499,170.1632) .. controls (85.3004,175.8497) and (80.5994,182.1633) .. (76.4897,189.1754) -- cycle(353.9486,119.3004) -- (402.9482,119.3004) .. controls (427.0025,137.0797) and (450.9893,162.7034) .. (474.9529,191.0213) .. controls (509.3540,228.5339) and (531.3391,294.2091) .. (487.8149,312.1206) .. controls (462.8165,324.7652) and (394.3874,316.8943) .. (373.8912,313.6651) .. controls (379.9291,297.7449) and (383.2899,278.4204) .. (381.4989,257.7214) .. controls (420.3069,248.0321) and (421.9610,218.3461) .. (407.7867,192.6417) .. controls (391.1113,162.4018) and (370.1114,132.9097) .. (353.9486,119.3004) -- cycle;
\end{tikzpicture}
}

\providecommand{\markuptable}{
\begin{table}
\sffamily\centering
\begin{tabular}{@{}lcl@{}}
\toprule
\texttt{//italics//} & $\rightarrow$ & \textit{italics} \\
\midrule
\texttt{**bold**} & $\rightarrow$ & \textbf{bold} \\
\midrule
\texttt{\# ordered list} & & 1 ordered list \\
\texttt{\# second item} & $\rightarrow$ & 2 second item \\
\texttt{\#\# sub item} & & \hspace{1em} 1 sub item \\
\midrule
\texttt{* unordered list} & & $\bullet$ unordered list \\
\texttt{* second item} & $\rightarrow$ & $\bullet$ second item \\
\texttt{** sub item} & & \hspace{1em} $\bullet$ sub item \\
\midrule
\texttt{link to [[label]]} & $\rightarrow$ & link to \underline{label} \\
\midrule
\texttt{<{}<label>{}> definition } & $\rightarrow$ & definition \\
\midrule
\texttt{[[url|link name]]} & $\rightarrow$ & \underline{link name} \\
\midrule\addlinespace
\texttt{= large heading} & & {\Large large heading} \smallskip \\
\texttt{== medium heading} & $\rightarrow$ & {\large medium heading} \\
\texttt{=== small heading} & & small heading \\
\midrule
\texttt{no line break} & & no line break for paragraphs \\
\texttt{for paragraphs} & $\rightarrow$ \\
& & use empty line \\
\texttt{use empty line} \\
\midrule
\texttt{force\textbackslash\textbackslash line break} & $\rightarrow$ & force \\
& & line break \\
\midrule
\texttt{horizontal line} & $\rightarrow$ & horizontal line \\
\texttt{----} & & \hrulefill \\
\midrule
\texttt{|=a|=table|=header} & & \underline{a \enspace table \enspace header} \\
\texttt{|a|table|row} & $\rightarrow$ & a \enspace table \enspace row \\
\texttt{|b|table|row} & & b \enspace table \enspace row \\
\midrule
\texttt{\{\{\{} \\
\texttt{unformatted} & $\rightarrow$ & \texttt{unformatted} \\
\texttt{code} & & \texttt{code} \\
\texttt{\}\}\}} \\
\midrule\addlinespace
\texttt{@ new article} & & {\Large 1.\ new article} \smallskip \\
\texttt{@ second article} & $\rightarrow$ & {\Large 2.\ second article} \smallskip \\
\texttt{@@ sub article} & & {\large 2.1.\ sub article} \\
\bottomrule
\end{tabular}
\normalfont\caption{Elements of the generic documentation markup language}
\label{tab:docmarkup}
\end{table}
}

\providecommand{\startchapter}[4]{
\documentclass[11pt,a4paper]{article}
\usepackage{booktabs}
\usepackage[format=hang,labelfont=bf]{caption}
\usepackage{changepage}
\usepackage[T1]{fontenc}
\usepackage[margin=2cm]{geometry}
\usepackage{hyperref}
\usepackage[american]{isodate}
\usepackage{lmodern}
\usepackage{longtable}
\usepackage{mathptmx}
\usepackage{microtype}
\usepackage[toc]{multitoc}
\usepackage{multirow}
\usepackage[all]{nowidow}
\usepackage{pdfcomment}
\usepackage{syntax}
\usepackage{tikz}
\usepackage[all]{xy}
\hypersetup{pdfborder={0 0 0},bookmarksnumbered=true,pdftitle={\ecs{}: #2},pdfauthor={Florian Negele},pdfsubject={\ecs{}},pdfkeywords={#1}}
\setlength{\grammarindent}{8em}\setlength{\grammarparsep}{0.2ex}
\setlength{\columnsep}{2em}
\newcommand{\prefix}{}
\newcounter{instruction}
\bibliographystyle{unsrt}
\renewcommand{\index}[2][]{}
\renewcommand{\arraystretch}{1.05}
\renewcommand{\floatpagefraction}{0.7}
\renewcommand{\syntleft}{\itshape}\renewcommand{\syntright}{}
\title{\vspace{-5ex}\Huge{\ecs{}}\medskip\hrule}
\author{\huge{#2}}
\date{\medskip\version}
\newif\ifbook\bookfalse
\pagestyle{headings}
\frenchspacing
\begin{document}
\maketitle\thispagestyle{empty}\noindent#4\setlength{\columnseprule}{0.4pt}\tableofcontents\setlength{\columnseprule}{0pt}\vfill\pagebreak[3]\null\vfill\bigskip\noindent
\parbox{\textwidth-4em}{\license The contents of this \documentation{} are part of the \href{manual}{\ecs{} User Manual}~\cite{manual} and correspond to Chapter ``\href{manual\##3}{#1}''.\alignright\mbox{\today}}
\parbox{4em}{\flushright\ecslogo{3em}}
\clearpage
}

\providecommand{\concludechapter}{
\vfill\pagebreak[3]\null\vfill
\thispagestyle{myheadings}\markright{REFERENCES}
\noindent\begin{minipage}{\textwidth}\begin{multicols}{2}[\section*{References}]
\renewcommand{\section}[2]{}\small\bibliography{references}
\end{multicols}\end{minipage}\end{document}
}

\providecommand{\startpresentation}[2]{
\documentclass[14pt,aspectratio=43,usepdftitle=false]{beamer}
\usepackage{booktabs}
\usepackage{etex}
\usepackage{multicol}
\usepackage{tikz}
\usepackage[all]{xy}
\bibliographystyle{unsrt}
\setlength{\columnsep}{1em}
\setlength{\leftmargini}{1em}
\setbeamercolor{title}{fg=black}
\setbeamercolor{structure}{fg=darkgray}
\setbeamercolor{bibliography item}{fg=darkgray}
\setbeamerfont{title}{series=\bfseries}
\setbeamerfont{subtitle}{series=\normalfont}
\setbeamerfont*{frametitle}{parent=title}
\setbeamerfont{block title}{series=\bfseries}
\setbeamerfont*{framesubtitle}{parent=subtitle}
\setbeamersize{text margin left=1em,text margin right=1em}
\setbeamertemplate{navigation symbols}{}
\setbeamertemplate{itemize item}[circle]{}
\setbeamertemplate{bibliography item}[triangle]{}
\setbeamertemplate{bibliography entry author}{\usebeamercolor[fg]{bibliography item}}
\setbeamertemplate{frametitle}{\medskip\usebeamerfont{frametitle}\color{gray}\raisebox{-2.5ex}[0ex][0ex]{\rule{0.1em}{4.5ex}}}
\addtobeamertemplate{frametitle}{}{\hspace{0.4em}\usebeamercolor[fg]{title}\insertframetitle\par\vspace{0.2ex}\hspace{0.5em}\usebeamerfont{framesubtitle}\insertframesubtitle}
\hypersetup{pdfborder={0 0 0},bookmarksnumbered=true,bookmarksopen=true,bookmarksopenlevel=0,pdftitle={\ecs{}: #1},pdfauthor={Florian Negele},pdfsubject={\ecs{}},pdfkeywords={#1}}
\renewcommand{\flowgraph}[1]{\resizebox{\textwidth}{!}{$$\xymatrix{##1}$$}}
\title{\ecs{}\medskip\hrule\medskip}
\institute{\shadowedecslogo{5em}{30}{15}}
\date{\version}
\subtitle{#1}
\begin{document}
\begin{frame}[plain]\titlepage\nocite{manual}\end{frame}
\begin{frame}{Contents}{#1}\begin{center}\tableofcontents\end{center}\end{frame}
}

\providecommand{\concludepresentation}{
\begin{frame}{References}\begin{footnotesize}\setlength{\columnseprule}{0.4pt}\begin{multicols}{2}\bibliography{references}\end{multicols}\end{footnotesize}\end{frame}
\end{document}
}

\providecommand{\startbook}[1]{
\documentclass[10pt,paper=17cm:24cm,DIV=13,twoside=semi,headings=normal,numbers=noendperiod,cleardoublepage=plain]{scrbook}
\usepackage{atveryend}
\usepackage{booktabs}
\usepackage{caption}
\usepackage{changepage}
\usepackage[T1]{fontenc}
\usepackage{imakeidx}
\usepackage{hyperref}
\usepackage[american]{isodate}
\usepackage{lmodern}
\usepackage{longtable}
\usepackage{mathptmx}
\usepackage[final]{microtype}
\usepackage{multicol}
\usepackage{multirow}
\usepackage[all]{nowidow}
\usepackage{pdfcomment}
\usepackage{scrlayer-scrpage}
\usepackage{setspace}
\usepackage{syntax}
\usepackage[eventxtindent=4pt,oddtxtexdent=4pt]{thumbs}
\usepackage{tikz}
\usepackage[all]{xy}
\hyphenation{Micro-Blaze Open-Cores Open-RISC Power-PC}
\hypersetup{pdfborder={0 0 0},bookmarksnumbered=true,bookmarksopen=true,bookmarksopenlevel=0,pdftitle={\ecs{}: #1},pdfauthor={Florian Negele},pdfsubject={\ecs{}},pdfkeywords={#1}}
\setlength{\grammarindent}{8em}\setlength{\grammarparsep}{0.7ex}
\setkomafont{captionlabel}{\usekomafont{descriptionlabel}}
\renewcommand{\arraystretch}{1.05}\setstretch{1.1}
\renewcommand{\chapterformat}{\thechapter\autodot\enskip\raisebox{-1ex}[0ex][0ex]{\color{gray}\rule{0.1em}{3.5ex}}\enskip}
\renewcommand{\startchapter}[4]{\hypertarget{##3}{\chapter{##1}}\label{##3}##4\addthumb{##1}{\LARGE\sffamily\bfseries\thechapter}{white}{gray}\renewcommand{\prefix}{##3}}
\renewcommand{\concludechapter}{\clearpage{\stopthumb\cleardoublepage}}
\renewcommand{\syntleft}{\itshape}\renewcommand{\syntright}{}
\renewcommand{\floatpagefraction}{0.7}
\renewcommand{\partheademptypage}{}
\DeclareMicrotypeAlias{lmss}{cmr}
\newcommand{\prefix}{}
\newcounter{instruction}
\bibliographystyle{unsrt}
\newif\ifbook\booktrue
\makeindex[intoc,title=Index]
\makeindex[intoc,name=tools,title=Index of Tools,columns=3]
\makeindex[intoc,name=library,title=Index of Library Names]
\makeindex[intoc,name=runtime,title=Index of Runtime Support]
\makeindex[intoc,name=environment,title=Index of Target Environments]
\indexsetup{toclevel=chapter,headers={\indexname}{\indexname}}
\frenchspacing
\begin{document}
\pagenumbering{alph}
\begin{titlepage}\centering
\huge\sffamily\null\vfill\textbf{\ecs{}}\bigskip\hrule\bigskip#1
\normalsize\normalfont\vfill\vfill\shadowedecslogo{10em}{30}{15}
\large\vfill\vfill\version
\end{titlepage}
\null\vfill
\thispagestyle{empty}
\noindent\today\par\medskip
\license A copy of this license is included in Appendix~\ref{fdl} on page~\pageref{fdl}.
All product names used herein are for identification purposes only and may be trademarks of their respective companies.
\concludechapter
\frontmatter
\setcounter{tocdepth}{1}
\tableofcontents
\setcounter{tocdepth}{2}
\concludechapter
\listoffigures
\concludechapter
\listoftables
\concludechapter
}

\providecommand{\concludebook}{
\backmatter
\addtocontents{toc}{\protect\setcounter{tocdepth}{-1}}
\phantomsection\addcontentsline{toc}{part}{Bibliography}
\bibliography{references}
\concludechapter
\phantomsection\addcontentsline{toc}{part}{Indexes}
\printindex
\concludechapter
\indexprologue{\label{idx:tools}}
\printindex[tools]
\concludechapter
\printindex[library]
\concludechapter
\indexprologue{\label{idx:runtime}}
\printindex[runtime]
\concludechapter
\indexprologue{\label{idx:environment}}
\printindex[environment]
\concludechapter
\pagestyle{empty}\pagenumbering{Alph}\null\clearpage
\null\vfill\centering\ecslogo{4em}\par\medskip\license
\end{document}
}

% chapter references

\providecommand{\seedocumentationref}{}\renewcommand{\seedocumentationref}[3]{#1, see \Documentation{}~\documentationref{#2}{#3}. }
\providecommand{\seeinterface}{}\renewcommand{\seeinterface}{\ifbook See \Documentation{}~\documentationref{interface}{User Interface} for more information about the common user interface of all of these tools. \fi}
\providecommand{\seeguide}{}\renewcommand{\seeguide}{\seedocumentationref{For basic examples of using some of these tools in practice}{guide}{User Guide}}
\providecommand{\seecpp}{}\renewcommand{\seecpp}{\seedocumentationref{For more information about the \cpp{} programming language and its implementation by the \ecs{}}{cpp}{User Manual for \cpp{}}}
\providecommand{\seefalse}{}\renewcommand{\seefalse}{\seedocumentationref{For more information about the FALSE programming language and its implementation by the \ecs{}}{false}{User Manual for FALSE}}
\providecommand{\seeoberon}{}\renewcommand{\seeoberon}{\seedocumentationref{For more information about the Oberon programming language and its implementation by the \ecs{}}{oberon}{User Manual for Oberon}}
\providecommand{\seeassembly}{}\renewcommand{\seeassembly}{\seedocumentationref{For more information about the generic assembly language and how to use it}{assembly}{Generic Assembly Language Specification}}
\providecommand{\seeamd}{}\renewcommand{\seeamd}{\seedocumentationref{For more information about how the \ecs{} supports the AMD64 hardware architecture}{amd64}{AMD64 Hardware Architecture Support}}
\providecommand{\seearm}{}\renewcommand{\seearm}{\seedocumentationref{For more information about how the \ecs{} supports the ARM hardware architecture}{arm}{ARM Hardware Architecture Support}}
\providecommand{\seeavr}{}\renewcommand{\seeavr}{\seedocumentationref{For more information about how the \ecs{} supports the AVR hardware architecture}{avr}{AVR Hardware Architecture Support}}
\providecommand{\seeavrtt}{}\renewcommand{\seeavrtt}{\seedocumentationref{For more information about how the \ecs{} supports the AVR32 hardware architecture}{avr32}{AVR32 Hardware Architecture Support}}
\providecommand{\seemabk}{}\renewcommand{\seemabk}{\seedocumentationref{For more information about how the \ecs{} supports the M68000 hardware architecture}{m68k}{M68000 Hardware Architecture Support}}
\providecommand{\seemibl}{}\renewcommand{\seemibl}{\seedocumentationref{For more information about how the \ecs{} supports the MicroBlaze hardware architecture}{mibl}{MicroBlaze Hardware Architecture Support}}
\providecommand{\seemips}{}\renewcommand{\seemips}{\seedocumentationref{For more information about how the \ecs{} supports the MIPS32 and MIPS64 hardware architectures}{mips}{MIPS Hardware Architecture Support}}
\providecommand{\seemmix}{}\renewcommand{\seemmix}{\seedocumentationref{For more information about how the \ecs{} supports the MMIX hardware architecture}{mmix}{MMIX Hardware Architecture Support}}
\providecommand{\seeorok}{}\renewcommand{\seeorok}{\seedocumentationref{For more information about how the \ecs{} supports the OpenRISC 1000 hardware architecture}{or1k}{OpenRISC 1000 Hardware Architecture Support}}
\providecommand{\seeppc}{}\renewcommand{\seeppc}{\seedocumentationref{For more information about how the \ecs{} supports the PowerPC hardware architecture}{ppc}{PowerPC Hardware Architecture Support}}
\providecommand{\seerisc}{}\renewcommand{\seerisc}{\seedocumentationref{For more information about how the \ecs{} supports the RISC hardware architecture}{risc}{RISC Hardware Architecture Support}}
\providecommand{\seewasm}{}\renewcommand{\seewasm}{\seedocumentationref{For more information about how the \ecs{} supports the WebAssembly architecture}{wasm}{WebAssembly Architecture Support}}
\providecommand{\seedocumentation}{}\renewcommand{\seedocumentation}{\seedocumentationref{For more information about generic documentations and their generation by the \ecs{}}{documentation}{Generic Documentation Generation}}
\providecommand{\seedebugging}{}\renewcommand{\seedebugging}{\seedocumentationref{For more information about debugging information and its representation}{debugging}{Debugging Information Representation}}
\providecommand{\seecode}{}\renewcommand{\seecode}{\seedocumentationref{For more information about intermediate code and its purpose}{code}{Intermediate Code Representation}}
\providecommand{\seeobject}{}\renewcommand{\seeobject}{\seedocumentationref{For more information about object files and their purpose}{object}{Object File Representation}}

% generic documentation tools

\providecommand{\docprint}{
\toolsection{docprint} is a pretty printer for generic documentations.
It reformats generic documentations and writes it to the standard output stream.
\debuggingtool
\flowgraph{\resource{generic\\documentation} \ar[r] & \toolbox{docprint} \ar[r] & \resource{generic\\documentation}}
\seedocumentation
}

\providecommand{\doccheck}{
\toolsection{doccheck} is a syntactic and semantic checker for generic documentations.
It just performs syntactic and semantic checks on generic documentations and writes its diagnostic messages to the standard error stream.
\debuggingtool
\flowgraph{\resource{generic\\documentation} \ar[r] & \toolbox{doccheck} \ar[r] & \resource{diagnostic\\messages}}
\seedocumentation
}

\providecommand{\dochtml}{
\toolsection{dochtml} is an HTML documentation generator for generic documentations.
It processes several generic documentations and assembles all information therein into an HTML document.
\debuggingtool
\flowgraph{\resource{generic\\documentation} \ar[r] & \toolbox{dochtml} \ar[r] & \resource{HTML\\document}}
\seedocumentation
}

\providecommand{\doclatex}{
\toolsection{doclatex} is a Latex documentation generator for generic documentations.
It processes several generic documentations and assembles all information therein into a Latex document.
\debuggingtool
\flowgraph{\resource{generic\\documentation} \ar[r] & \toolbox{doclatex} \ar[r] & \resource{Latex\\document}}
\seedocumentation
}

% intermediate code tools

\providecommand{\cdcheck}{
\toolsection{cdcheck} is a syntactic and semantic checker for intermediate code.
It just performs syntactic and semantic checks on programs written in intermediate code and writes its diagnostic messages to the standard error stream.
\debuggingtool
\flowgraph{\resource{intermediate\\code} \ar[r] & \toolbox{cdcheck} \ar[r] & \resource{diagnostic\\messages}}
\seeassembly\seecode
}

\providecommand{\cdopt}{
\toolsection{cdopt} is an optimizer for intermediate code.
It performs various optimizations on programs written in intermediate code and writes the result to the standard output stream.
\debuggingtool
\flowgraph{\resource{intermediate\\code} \ar[r] & \toolbox{cdopt} \ar[r] & \resource{optimized\\code}}
\seeassembly\seecode
}

\providecommand{\cdrun}{
\toolsection{cdrun} is an interpreter for intermediate code.
It processes and executes programs written in intermediate code.
The following code sections are predefined and have the usual semantics:
\texttt{abort}, \texttt{\_Exit}, \texttt{fflush}, \texttt{floor}, \texttt{fputc}, \texttt{free}, \texttt{getchar}, \texttt{malloc}, and \texttt{putchar}.
Diagnostic messages about invalid operations include the name of the executed code section and the index of the erroneous instruction.
\debuggingtool
\flowgraph{\resource{intermediate\\code} \ar[r] & \toolbox{cdrun} \ar@/u/[r] & \resource{input/\\output} \ar@/d/[l]}
\seeassembly\seecode
}

\providecommand{\cdamda}{
\toolsection{cdamd16} is a compiler for intermediate code targeting the AMD64 hardware architecture.
It generates machine code for AMD64 processors from programs written in intermediate code and stores it in corresponding object files.
The compiler generates machine code for the 16-bit operating mode defined by the AMD64 architecture.
It also creates a debugging information file as well as an assembly file containing a listing of the generated machine code.
\debuggingtool
\flowgraph{\resource{intermediate\\code} \ar[r] & \toolbox{cdamd16} \ar[r] \ar[d] \ar[rd] & \resource{object file} \\ & \resource{assembly\\listing} & \resource{debugging\\information}}
\seeassembly\seeamd\seeobject\seecode\seedebugging
}

\providecommand{\cdamdb}{
\toolsection{cdamd32} is a compiler for intermediate code targeting the AMD64 hardware architecture.
It generates machine code for AMD64 processors from programs written in intermediate code and stores it in corresponding object files.
The compiler generates machine code for the 32-bit operating mode defined by the AMD64 architecture.
It also creates a debugging information file as well as an assembly file containing a listing of the generated machine code.
\debuggingtool
\flowgraph{\resource{intermediate\\code} \ar[r] & \toolbox{cdamd32} \ar[r] \ar[d] \ar[rd] & \resource{object file} \\ & \resource{assembly\\listing} & \resource{debugging\\information}}
\seeassembly\seeamd\seeobject\seecode\seedebugging
}

\providecommand{\cdamdc}{
\toolsection{cdamd64} is a compiler for intermediate code targeting the AMD64 hardware architecture.
It generates machine code for AMD64 processors from programs written in intermediate code and stores it in corresponding object files.
The compiler generates machine code for the 64-bit operating mode defined by the AMD64 architecture.
It also creates a debugging information file as well as an assembly file containing a listing of the generated machine code.
\debuggingtool
\flowgraph{\resource{intermediate\\code} \ar[r] & \toolbox{cdamd64} \ar[r] \ar[d] \ar[rd] & \resource{object file} \\ & \resource{assembly\\listing} & \resource{debugging\\information}}
\seeassembly\seeamd\seeobject\seecode\seedebugging
}

\providecommand{\cdarma}{
\toolsection{cdarma32} is a compiler for intermediate code targeting the ARM hardware architecture.
It generates machine code for ARM processors executing A32 instructions from programs written in intermediate code and stores it in corresponding object files.
It also creates a debugging information file as well as an assembly file containing a listing of the generated machine code.
\debuggingtool
\flowgraph{\resource{intermediate\\code} \ar[r] & \toolbox{cdarma32} \ar[r] \ar[d] \ar[rd] & \resource{object file} \\ & \resource{assembly\\listing} & \resource{debugging\\information}}
\seeassembly\seearm\seeobject\seecode\seedebugging
}

\providecommand{\cdarmb}{
\toolsection{cdarma64} is a compiler for intermediate code targeting the ARM hardware architecture.
It generates machine code for ARM processors executing A64 instructions from programs written in intermediate code and stores it in corresponding object files.
It also creates a debugging information file as well as an assembly file containing a listing of the generated machine code.
\debuggingtool
\flowgraph{\resource{intermediate\\code} \ar[r] & \toolbox{cdarma64} \ar[r] \ar[d] \ar[rd] & \resource{object file} \\ & \resource{assembly\\listing} & \resource{debugging\\information}}
\seeassembly\seearm\seeobject\seecode\seedebugging
}

\providecommand{\cdarmc}{
\toolsection{cdarmt32} is a compiler for intermediate code targeting the ARM hardware architecture.
It generates machine code for ARM processors without floating-point extension executing T32 instructions from programs written in intermediate code and stores it in corresponding object files.
It also creates a debugging information file as well as an assembly file containing a listing of the generated machine code.
\debuggingtool
\flowgraph{\resource{intermediate\\code} \ar[r] & \toolbox{cdarmt32} \ar[r] \ar[d] \ar[rd] & \resource{object file} \\ & \resource{assembly\\listing} & \resource{debugging\\information}}
\seeassembly\seearm\seeobject\seecode\seedebugging
}

\providecommand{\cdarmcfpe}{
\toolsection{cdarmt32fpe} is a compiler for intermediate code targeting the ARM hardware architecture.
It generates machine code for ARM processors with floating-point extension executing T32 instructions from programs written in intermediate code and stores it in corresponding object files.
It also creates a debugging information file as well as an assembly file containing a listing of the generated machine code.
\debuggingtool
\flowgraph{\resource{intermediate\\code} \ar[r] & \toolbox{cdarmt32fpe} \ar[r] \ar[d] \ar[rd] & \resource{object file} \\ & \resource{assembly\\listing} & \resource{debugging\\information}}
\seeassembly\seearm\seeobject\seecode\seedebugging
}

\providecommand{\cdavr}{
\toolsection{cdavr} is a compiler for intermediate code targeting the AVR hardware architecture.
It generates machine code for AVR processors from programs written in intermediate code and stores it in corresponding object files.
It also creates a debugging information file as well as an assembly file containing a listing of the generated machine code.
\debuggingtool
\flowgraph{\resource{intermediate\\code} \ar[r] & \toolbox{cdavr} \ar[r] \ar[d] \ar[rd] & \resource{object file} \\ & \resource{assembly\\listing} & \resource{debugging\\information}}
\seeassembly\seeavr\seeobject\seecode\seedebugging
}

\providecommand{\cdavrtt}{
\toolsection{cdavr32} is a compiler for intermediate code targeting the AVR32 hardware architecture.
It generates machine code for AVR32 processors from programs written in intermediate code and stores it in corresponding object files.
It also creates a debugging information file as well as an assembly file containing a listing of the generated machine code.
\debuggingtool
\flowgraph{\resource{intermediate\\code} \ar[r] & \toolbox{cdavr32} \ar[r] \ar[d] \ar[rd] & \resource{object file} \\ & \resource{assembly\\listing} & \resource{debugging\\information}}
\seeassembly\seeavrtt\seeobject\seecode\seedebugging
}

\providecommand{\cdmabk}{
\toolsection{cdm68k} is a compiler for intermediate code targeting the M68000 hardware architecture.
It generates machine code for M68000 processors from programs written in intermediate code and stores it in corresponding object files.
It also creates a debugging information file as well as an assembly file containing a listing of the generated machine code.
\debuggingtool
\flowgraph{\resource{intermediate\\code} \ar[r] & \toolbox{cdm68k} \ar[r] \ar[d] \ar[rd] & \resource{object file} \\ & \resource{assembly\\listing} & \resource{debugging\\information}}
\seeassembly\seemabk\seeobject\seecode\seedebugging
}

\providecommand{\cdmibl}{
\toolsection{cdmibl} is a compiler for intermediate code targeting the MicroBlaze hardware architecture.
It generates machine code for MicroBlaze processors from programs written in intermediate code and stores it in corresponding object files.
It also creates a debugging information file as well as an assembly file containing a listing of the generated machine code.
\debuggingtool
\flowgraph{\resource{intermediate\\code} \ar[r] & \toolbox{cdmibl} \ar[r] \ar[d] \ar[rd] & \resource{object file} \\ & \resource{assembly\\listing} & \resource{debugging\\information}}
\seeassembly\seemibl\seeobject\seecode\seedebugging
}

\providecommand{\cdmipsa}{
\toolsection{cdmips32} is a compiler for intermediate code targeting the MIPS32 hardware architecture.
It generates machine code for MIPS32 processors from programs written in intermediate code and stores it in corresponding object files.
It also creates a debugging information file as well as an assembly file containing a listing of the generated machine code.
\debuggingtool
\flowgraph{\resource{intermediate\\code} \ar[r] & \toolbox{cdmips32} \ar[r] \ar[d] \ar[rd] & \resource{object file} \\ & \resource{assembly\\listing} & \resource{debugging\\information}}
\seeassembly\seemips\seeobject\seecode\seedebugging
}

\providecommand{\cdmipsb}{
\toolsection{cdmips64} is a compiler for intermediate code targeting the MIPS64 hardware architecture.
It generates machine code for MIPS64 processors from programs written in intermediate code and stores it in corresponding object files.
It also creates a debugging information file as well as an assembly file containing a listing of the generated machine code.
\debuggingtool
\flowgraph{\resource{intermediate\\code} \ar[r] & \toolbox{cdmips64} \ar[r] \ar[d] \ar[rd] & \resource{object file} \\ & \resource{assembly\\listing} & \resource{debugging\\information}}
\seeassembly\seemips\seeobject\seecode\seedebugging
}

\providecommand{\cdmmix}{
\toolsection{cdmmix} is a compiler for intermediate code targeting the MMIX hardware architecture.
It generates machine code for MMIX processors from programs written in intermediate code and stores it in corresponding object files.
It also creates a debugging information file as well as an assembly file containing a listing of the generated machine code.
\debuggingtool
\flowgraph{\resource{intermediate\\code} \ar[r] & \toolbox{cdmmix} \ar[r] \ar[d] \ar[rd] & \resource{object file} \\ & \resource{assembly\\listing} & \resource{debugging\\information}}
\seeassembly\seemmix\seeobject\seecode\seedebugging
}

\providecommand{\cdorok}{
\toolsection{cdor1k} is a compiler for intermediate code targeting the OpenRISC 1000 hardware architecture.
It generates machine code for OpenRISC 1000 processors from programs written in intermediate code and stores it in corresponding object files.
It also creates a debugging information file as well as an assembly file containing a listing of the generated machine code.
\debuggingtool
\flowgraph{\resource{intermediate\\code} \ar[r] & \toolbox{cdor1k} \ar[r] \ar[d] \ar[rd] & \resource{object file} \\ & \resource{assembly\\listing} & \resource{debugging\\information}}
\seeassembly\seeorok\seeobject\seecode\seedebugging
}

\providecommand{\cdppca}{
\toolsection{cdppc32} is a compiler for intermediate code targeting the PowerPC hardware architecture.
It generates machine code for PowerPC processors from programs written in intermediate code and stores it in corresponding object files.
The compiler generates machine code for the 32-bit operating mode defined by the PowerPC architecture.
It also creates a debugging information file as well as an assembly file containing a listing of the generated machine code.
\debuggingtool
\flowgraph{\resource{intermediate\\code} \ar[r] & \toolbox{cdppc32} \ar[r] \ar[d] \ar[rd] & \resource{object file} \\ & \resource{assembly\\listing} & \resource{debugging\\information}}
\seeassembly\seeppc\seeobject\seecode\seedebugging
}

\providecommand{\cdppcb}{
\toolsection{cdppc64} is a compiler for intermediate code targeting the PowerPC hardware architecture.
It generates machine code for PowerPC processors from programs written in intermediate code and stores it in corresponding object files.
The compiler generates machine code for the 64-bit operating mode defined by the PowerPC architecture.
It also creates a debugging information file as well as an assembly file containing a listing of the generated machine code.
\debuggingtool
\flowgraph{\resource{intermediate\\code} \ar[r] & \toolbox{cdppc64} \ar[r] \ar[d] \ar[rd] & \resource{object file} \\ & \resource{assembly\\listing} & \resource{debugging\\information}}
\seeassembly\seeppc\seeobject\seecode\seedebugging
}

\providecommand{\cdrisc}{
\toolsection{cdrisc} is a compiler for intermediate code targeting the RISC hardware architecture.
It generates machine code for RISC processors from programs written in intermediate code and stores it in corresponding object files.
It also creates a debugging information file as well as an assembly file containing a listing of the generated machine code.
\debuggingtool
\flowgraph{\resource{intermediate\\code} \ar[r] & \toolbox{cdrisc} \ar[r] \ar[d] \ar[rd] & \resource{object file} \\ & \resource{assembly\\listing} & \resource{debugging\\information}}
\seeassembly\seerisc\seeobject\seecode\seedebugging
}

\providecommand{\cdwasm}{
\toolsection{cdwasm} is a compiler for intermediate code targeting the WebAssembly architecture.
It generates machine code for WebAssembly targets from programs written in intermediate code and stores it in corresponding object files.
It also creates a debugging information file as well as an assembly file containing a listing of the generated machine code.
\debuggingtool
\flowgraph{\resource{intermediate\\code} \ar[r] & \toolbox{cdwasm} \ar[r] \ar[d] \ar[rd] & \resource{object file} \\ & \resource{assembly\\listing} & \resource{debugging\\information}}
\seeassembly\seewasm\seeobject\seecode\seedebugging
}

% C++ tools

\providecommand{\cppprep}{
\toolsection{cppprep} is a preprocessor for the \cpp{} programming language.
It preprocesses source code according to the rules of \cpp{} and writes it to the standard output stream.
Only the macro names \texttt{\_\_DATE\_\_}, \texttt{\_\_FILE\_\_}, \texttt{\_\_LINE\_\_}, and \texttt{\_\_TIME\_\_} are predefined.
\flowgraph{\resource{\cpp{} or other\\source code} \ar[r] & \toolbox{cppprep} \ar[r] & \resource{preprocessed\\source code} \\ & \variable{ECSINCLUDE} \ar[u]}
\seecpp
}

\providecommand{\cppprint}{
\toolsection{cppprint} is a pretty printer for the \cpp{} programming language.
It reformats the source code of \cpp{} programs and writes it to the standard output stream.
\flowgraph{\resource{\cpp{}\\source code} \ar[r] & \toolbox{cppprint} \ar[r] & \resource{reformatted\\source code} \\ & \variable{ECSINCLUDE} \ar[u]}
\seecpp
}

\providecommand{\cppcheck}{
\toolsection{cppcheck} is a syntactic and semantic checker for the \cpp{} programming language.
It just performs syntactic and semantic checks on \cpp{} programs and writes its diagnostic messages to the standard error stream.
\flowgraph{\resource{\cpp{}\\source code} \ar[r] & \toolbox{cppcheck} \ar[r] & \resource{diagnostic\\messages} \\ & \variable{ECSINCLUDE} \ar[u]}
\seecpp
}

\providecommand{\cppdump}{
\toolsection{cppdump} is a serializer for the \cpp{} programming language.
It dumps the complete internal representation of programs written in \cpp{} into an XML document.
\debuggingtool
\flowgraph{\resource{\cpp{}\\source code} \ar[r] & \toolbox{cppdump} \ar[r] & \resource{internal\\representation} \\ & \variable{ECSINCLUDE} \ar[u]}
\seecpp
}

\providecommand{\cpprun}{
\toolsection{cpprun} is an interpreter for the \cpp{} programming language.
It processes and executes programs written in \cpp{}.
The macro \texttt{\_\_run\_\_} is predefined in order to enable programmers to identify this tool while interpreting.
\flowgraph{\resource{\cpp{}\\source code} \ar[r] & \toolbox{cpprun} \ar@/u/[r] & \resource{input/\\output} \ar@/d/[l] \\ & \variable{ECSINCLUDE} \ar[u]}
\seecpp
}

\providecommand{\cppdoc}{
\toolsection{cppdoc} is a generic documentation generator for the \cpp{} programming language.
It processes several \cpp{} source files and assembles all information therein into a generic documentation.
\debuggingtool
\flowgraph{\resource{\cpp{}\\source code} \ar[r] & \toolbox{cppdoc} \ar[r] & \resource{generic\\documentation} \\ & \variable{ECSINCLUDE} \ar[u]}
\seecpp\seedocumentation
}

\providecommand{\cpphtml}{
\toolsection{cpphtml} is an HTML documentation generator for the \cpp{} programming language.
It processes several \cpp{} source files and assembles all information therein into an HTML document.
\flowgraph{\resource{\cpp{}\\source code} \ar[r] & \toolbox{cpphtml} \ar[r] & \resource{HTML\\document} \\ & \variable{ECSINCLUDE} \ar[u]}
\seecpp\seedocumentation
}

\providecommand{\cpplatex}{
\toolsection{cpplatex} is a Latex documentation generator for the \cpp{} programming language.
It processes several \cpp{} source files and assembles all information therein into a Latex document.
\flowgraph{\resource{\cpp{}\\source code} \ar[r] & \toolbox{cpplatex} \ar[r] & \resource{Latex\\document} \\ & \variable{ECSINCLUDE} \ar[u]}
\seecpp\seedocumentation
}

\providecommand{\cppcode}{
\toolsection{cppcode} is an intermediate code generator for the \cpp{} programming language.
It generates intermediate code from programs written in \cpp{} and stores it in corresponding assembly files.
The macro \texttt{\_\_code\_\_} is predefined in order to enable programmers to identify this tool while generating intermediate code.
Programs generated with this tool require additional runtime support that is stored in the \file{cpp\-code\-run} library file.
\debuggingtool
\flowgraph{\resource{\cpp{}\\source code} \ar[r] & \toolbox{cppcode} \ar[r] & \resource{intermediate\\code} \\ & \variable{ECSINCLUDE} \ar[u]}
\seecpp\seeassembly\seecode
}

\providecommand{\cppamda}{
\toolsection{cppamd16} is a compiler for the \cpp{} programming language targeting the AMD64 hardware architecture.
It generates machine code for AMD64 processors from programs written in \cpp{} and stores it in corresponding object files.
The compiler generates machine code for the 16-bit operating mode defined by the AMD64 architecture.
For debugging purposes, it also creates a debugging information file as well as an assembly file containing a listing of the generated machine code.
The macro \texttt{\_\_amd16\_\_} is predefined in order to enable programmers to identify this tool and its target architecture while compiling.
Programs generated with this compiler require additional runtime support that is stored in the \file{cpp\-amd16\-run} library file.
\flowgraph{\resource{\cpp{}\\source code} \ar[r] & \toolbox{cppamd16} \ar[r] \ar[d] \ar[rd] & \resource{object file} \\ \variable{ECSINCLUDE} \ar[ru] & \resource{debugging\\information} & \resource{assembly\\listing}}
\seecpp\seeassembly\seeamd\seeobject\seedebugging
}

\providecommand{\cppamdb}{
\toolsection{cppamd32} is a compiler for the \cpp{} programming language targeting the AMD64 hardware architecture.
It generates machine code for AMD64 processors from programs written in \cpp{} and stores it in corresponding object files.
The compiler generates machine code for the 32-bit operating mode defined by the AMD64 architecture.
For debugging purposes, it also creates a debugging information file as well as an assembly file containing a listing of the generated machine code.
The macro \texttt{\_\_amd32\_\_} is predefined in order to enable programmers to identify this tool and its target architecture while compiling.
Programs generated with this compiler require additional runtime support that is stored in the \file{cpp\-amd32\-run} library file.
\flowgraph{\resource{\cpp{}\\source code} \ar[r] & \toolbox{cppamd32} \ar[r] \ar[d] \ar[rd] & \resource{object file} \\ \variable{ECSINCLUDE} \ar[ru] & \resource{debugging\\information} & \resource{assembly\\listing}}
\seecpp\seeassembly\seeamd\seeobject\seedebugging
}

\providecommand{\cppamdc}{
\toolsection{cppamd64} is a compiler for the \cpp{} programming language targeting the AMD64 hardware architecture.
It generates machine code for AMD64 processors from programs written in \cpp{} and stores it in corresponding object files.
The compiler generates machine code for the 64-bit operating mode defined by the AMD64 architecture.
For debugging purposes, it also creates a debugging information file as well as an assembly file containing a listing of the generated machine code.
The macro \texttt{\_\_amd64\_\_} is predefined in order to enable programmers to identify this tool and its target architecture while compiling.
Programs generated with this compiler require additional runtime support that is stored in the \file{cpp\-amd64\-run} library file.
\flowgraph{\resource{\cpp{}\\source code} \ar[r] & \toolbox{cppamd64} \ar[r] \ar[d] \ar[rd] & \resource{object file} \\ \variable{ECSINCLUDE} \ar[ru] & \resource{debugging\\information} & \resource{assembly\\listing}}
\seecpp\seeassembly\seeamd\seeobject\seedebugging
}

\providecommand{\cpparma}{
\toolsection{cpparma32} is a compiler for the \cpp{} programming language targeting the ARM hardware architecture.
It generates machine code for ARM processors executing A32 instructions from programs written in \cpp{} and stores it in corresponding object files.
For debugging purposes, it also creates a debugging information file as well as an assembly file containing a listing of the generated machine code.
The macro \texttt{\_\_arma32\_\_} is predefined in order to enable programmers to identify this tool and its target architecture while compiling.
Programs generated with this compiler require additional runtime support that is stored in the \file{cpp\-arma32\-run} library file.
\flowgraph{\resource{\cpp{}\\source code} \ar[r] & \toolbox{cpparma32} \ar[r] \ar[d] \ar[rd] & \resource{object file} \\ \variable{ECSINCLUDE} \ar[ru] & \resource{debugging\\information} & \resource{assembly\\listing}}
\seecpp\seeassembly\seearm\seeobject\seedebugging
}

\providecommand{\cpparmb}{
\toolsection{cpparma64} is a compiler for the \cpp{} programming language targeting the ARM hardware architecture.
It generates machine code for ARM processors executing A64 instructions from programs written in \cpp{} and stores it in corresponding object files.
For debugging purposes, it also creates a debugging information file as well as an assembly file containing a listing of the generated machine code.
The macro \texttt{\_\_arma64\_\_} is predefined in order to enable programmers to identify this tool and its target architecture while compiling.
Programs generated with this compiler require additional runtime support that is stored in the \file{cpp\-arma64\-run} library file.
\flowgraph{\resource{\cpp{}\\source code} \ar[r] & \toolbox{cpparma64} \ar[r] \ar[d] \ar[rd] & \resource{object file} \\ \variable{ECSINCLUDE} \ar[ru] & \resource{debugging\\information} & \resource{assembly\\listing}}
\seecpp\seeassembly\seearm\seeobject\seedebugging
}

\providecommand{\cpparmc}{
\toolsection{cpparmt32} is a compiler for the \cpp{} programming language targeting the ARM hardware architecture.
It generates machine code for ARM processors without floating-point extension executing T32 instructions from programs written in \cpp{} and stores it in corresponding object files.
For debugging purposes, it also creates a debugging information file as well as an assembly file containing a listing of the generated machine code.
The macro \texttt{\_\_armt32\_\_} is predefined in order to enable programmers to identify this tool and its target architecture while compiling.
Programs generated with this compiler require additional runtime support that is stored in the \file{cpp\-armt32\-run} library file.
\flowgraph{\resource{\cpp{}\\source code} \ar[r] & \toolbox{cpparmt32} \ar[r] \ar[d] \ar[rd] & \resource{object file} \\ \variable{ECSINCLUDE} \ar[ru] & \resource{debugging\\information} & \resource{assembly\\listing}}
\seecpp\seeassembly\seearm\seeobject\seedebugging
}

\providecommand{\cpparmcfpe}{
\toolsection{cpparmt32fpe} is a compiler for the \cpp{} programming language targeting the ARM hardware architecture.
It generates machine code for ARM processors with floating-point extension executing T32 instructions from programs written in \cpp{} and stores it in corresponding object files.
For debugging purposes, it also creates a debugging information file as well as an assembly file containing a listing of the generated machine code.
The macro \texttt{\_\_armt32fpe\_\_} is predefined in order to enable programmers to identify this tool and its target architecture while compiling.
Programs generated with this compiler require additional runtime support that is stored in the \file{cpp\-armt32\-fpe\-run} library file.
\flowgraph{\resource{\cpp{}\\source code} \ar[r] & \toolbox{cpparmt32fpe} \ar[r] \ar[d] \ar[rd] & \resource{object file} \\ \variable{ECSINCLUDE} \ar[ru] & \resource{debugging\\information} & \resource{assembly\\listing}}
\seecpp\seeassembly\seearm\seeobject\seedebugging
}

\providecommand{\cppavr}{
\toolsection{cppavr} is a compiler for the \cpp{} programming language targeting the AVR hardware architecture.
It generates machine code for AVR processors from programs written in \cpp{} and stores it in corresponding object files.
For debugging purposes, it also creates a debugging information file as well as an assembly file containing a listing of the generated machine code.
The macro \texttt{\_\_avr\_\_} is predefined in order to enable programmers to identify this tool and its target architecture while compiling.
Programs generated with this compiler require additional runtime support that is stored in the \file{cpp\-avr\-run} library file.
\flowgraph{\resource{\cpp{}\\source code} \ar[r] & \toolbox{cppavr} \ar[r] \ar[d] \ar[rd] & \resource{object file} \\ \variable{ECSINCLUDE} \ar[ru] & \resource{debugging\\information} & \resource{assembly\\listing}}
\seecpp\seeassembly\seeavr\seeobject\seedebugging
}

\providecommand{\cppavrtt}{
\toolsection{cppavr32} is a compiler for the \cpp{} programming language targeting the AVR32 hardware architecture.
It generates machine code for AVR32 processors from programs written in \cpp{} and stores it in corresponding object files.
For debugging purposes, it also creates a debugging information file as well as an assembly file containing a listing of the generated machine code.
The macro \texttt{\_\_avr32\_\_} is predefined in order to enable programmers to identify this tool and its target architecture while compiling.
Programs generated with this compiler require additional runtime support that is stored in the \file{cpp\-avr32\-run} library file.
\flowgraph{\resource{\cpp{}\\source code} \ar[r] & \toolbox{cppavr32} \ar[r] \ar[d] \ar[rd] & \resource{object file} \\ \variable{ECSINCLUDE} \ar[ru] & \resource{debugging\\information} & \resource{assembly\\listing}}
\seecpp\seeassembly\seeavrtt\seeobject\seedebugging
}

\providecommand{\cppmabk}{
\toolsection{cppm68k} is a compiler for the \cpp{} programming language targeting the M68000 hardware architecture.
It generates machine code for M68000 processors from programs written in \cpp{} and stores it in corresponding object files.
For debugging purposes, it also creates a debugging information file as well as an assembly file containing a listing of the generated machine code.
The macro \texttt{\_\_m68k\_\_} is predefined in order to enable programmers to identify this tool and its target architecture while compiling.
Programs generated with this compiler require additional runtime support that is stored in the \file{cpp\-m68k\-run} library file.
\flowgraph{\resource{\cpp{}\\source code} \ar[r] & \toolbox{cppm68k} \ar[r] \ar[d] \ar[rd] & \resource{object file} \\ \variable{ECSINCLUDE} \ar[ru] & \resource{debugging\\information} & \resource{assembly\\listing}}
\seecpp\seeassembly\seemabk\seeobject\seedebugging
}

\providecommand{\cppmibl}{
\toolsection{cppmibl} is a compiler for the \cpp{} programming language targeting the MicroBlaze hardware architecture.
It generates machine code for MicroBlaze processors from programs written in \cpp{} and stores it in corresponding object files.
For debugging purposes, it also creates a debugging information file as well as an assembly file containing a listing of the generated machine code.
The macro \texttt{\_\_mibl\_\_} is predefined in order to enable programmers to identify this tool and its target architecture while compiling.
Programs generated with this compiler require additional runtime support that is stored in the \file{cpp\-mibl\-run} library file.
\flowgraph{\resource{\cpp{}\\source code} \ar[r] & \toolbox{cppmibl} \ar[r] \ar[d] \ar[rd] & \resource{object file} \\ \variable{ECSINCLUDE} \ar[ru] & \resource{debugging\\information} & \resource{assembly\\listing}}
\seecpp\seeassembly\seemibl\seeobject\seedebugging
}

\providecommand{\cppmipsa}{
\toolsection{cppmips32} is a compiler for the \cpp{} programming language targeting the MIPS32 hardware architecture.
It generates machine code for MIPS32 processors from programs written in \cpp{} and stores it in corresponding object files.
For debugging purposes, it also creates a debugging information file as well as an assembly file containing a listing of the generated machine code.
The macro \texttt{\_\_mips32\_\_} is predefined in order to enable programmers to identify this tool and its target architecture while compiling.
Programs generated with this compiler require additional runtime support that is stored in the \file{cpp\-mips32\-run} library file.
\flowgraph{\resource{\cpp{}\\source code} \ar[r] & \toolbox{cppmips32} \ar[r] \ar[d] \ar[rd] & \resource{object file} \\ \variable{ECSINCLUDE} \ar[ru] & \resource{debugging\\information} & \resource{assembly\\listing}}
\seecpp\seeassembly\seemips\seeobject\seedebugging
}

\providecommand{\cppmipsb}{
\toolsection{cppmips64} is a compiler for the \cpp{} programming language targeting the MIPS64 hardware architecture.
It generates machine code for MIPS64 processors from programs written in \cpp{} and stores it in corresponding object files.
For debugging purposes, it also creates a debugging information file as well as an assembly file containing a listing of the generated machine code.
The macro \texttt{\_\_mips64\_\_} is predefined in order to enable programmers to identify this tool and its target architecture while compiling.
Programs generated with this compiler require additional runtime support that is stored in the \file{cpp\-mips64\-run} library file.
\flowgraph{\resource{\cpp{}\\source code} \ar[r] & \toolbox{cppmips64} \ar[r] \ar[d] \ar[rd] & \resource{object file} \\ \variable{ECSINCLUDE} \ar[ru] & \resource{debugging\\information} & \resource{assembly\\listing}}
\seecpp\seeassembly\seemips\seeobject\seedebugging
}

\providecommand{\cppmmix}{
\toolsection{cppmmix} is a compiler for the \cpp{} programming language targeting the MMIX hardware architecture.
It generates machine code for MMIX processors from programs written in \cpp{} and stores it in corresponding object files.
For debugging purposes, it also creates a debugging information file as well as an assembly file containing a listing of the generated machine code.
The macro \texttt{\_\_mmix\_\_} is predefined in order to enable programmers to identify this tool and its target architecture while compiling.
Programs generated with this compiler require additional runtime support that is stored in the \file{cpp\-mmix\-run} library file.
\flowgraph{\resource{\cpp{}\\source code} \ar[r] & \toolbox{cppmmix} \ar[r] \ar[d] \ar[rd] & \resource{object file} \\ \variable{ECSINCLUDE} \ar[ru] & \resource{debugging\\information} & \resource{assembly\\listing}}
\seecpp\seeassembly\seemmix\seeobject\seedebugging
}

\providecommand{\cpporok}{
\toolsection{cppor1k} is a compiler for the \cpp{} programming language targeting the OpenRISC 1000 hardware architecture.
It generates machine code for OpenRISC 1000 processors from programs written in \cpp{} and stores it in corresponding object files.
For debugging purposes, it also creates a debugging information file as well as an assembly file containing a listing of the generated machine code.
The macro \texttt{\_\_or1k\_\_} is predefined in order to enable programmers to identify this tool and its target architecture while compiling.
Programs generated with this compiler require additional runtime support that is stored in the \file{cpp\-or1k\-run} library file.
\flowgraph{\resource{\cpp{}\\source code} \ar[r] & \toolbox{cppor1k} \ar[r] \ar[d] \ar[rd] & \resource{object file} \\ \variable{ECSINCLUDE} \ar[ru] & \resource{debugging\\information} & \resource{assembly\\listing}}
\seecpp\seeassembly\seeorok\seeobject\seedebugging
}

\providecommand{\cppppca}{
\toolsection{cppppc32} is a compiler for the \cpp{} programming language targeting the PowerPC hardware architecture.
It generates machine code for PowerPC processors from programs written in \cpp{} and stores it in corresponding object files.
The compiler generates machine code for the 32-bit operating mode defined by the PowerPC architecture.
For debugging purposes, it also creates a debugging information file as well as an assembly file containing a listing of the generated machine code.
The macro \texttt{\_\_ppc32\_\_} is predefined in order to enable programmers to identify this tool and its target architecture while compiling.
Programs generated with this compiler require additional runtime support that is stored in the \file{cpp\-ppc32\-run} library file.
\flowgraph{\resource{\cpp{}\\source code} \ar[r] & \toolbox{cppppc32} \ar[r] \ar[d] \ar[rd] & \resource{object file} \\ \variable{ECSINCLUDE} \ar[ru] & \resource{debugging\\information} & \resource{assembly\\listing}}
\seecpp\seeassembly\seeppc\seeobject\seedebugging
}

\providecommand{\cppppcb}{
\toolsection{cppppc64} is a compiler for the \cpp{} programming language targeting the PowerPC hardware architecture.
It generates machine code for PowerPC processors from programs written in \cpp{} and stores it in corresponding object files.
The compiler generates machine code for the 64-bit operating mode defined by the PowerPC architecture.
For debugging purposes, it also creates a debugging information file as well as an assembly file containing a listing of the generated machine code.
The macro \texttt{\_\_ppc64\_\_} is predefined in order to enable programmers to identify this tool and its target architecture while compiling.
Programs generated with this compiler require additional runtime support that is stored in the \file{cpp\-ppc64\-run} library file.
\flowgraph{\resource{\cpp{}\\source code} \ar[r] & \toolbox{cppppc64} \ar[r] \ar[d] \ar[rd] & \resource{object file} \\ \variable{ECSINCLUDE} \ar[ru] & \resource{debugging\\information} & \resource{assembly\\listing}}
\seecpp\seeassembly\seeppc\seeobject\seedebugging
}

\providecommand{\cpprisc}{
\toolsection{cpprisc} is a compiler for the \cpp{} programming language targeting the RISC hardware architecture.
It generates machine code for RISC processors from programs written in \cpp{} and stores it in corresponding object files.
For debugging purposes, it also creates a debugging information file as well as an assembly file containing a listing of the generated machine code.
The macro \texttt{\_\_risc\_\_} is predefined in order to enable programmers to identify this tool and its target architecture while compiling.
Programs generated with this compiler require additional runtime support that is stored in the \file{cpp\-risc\-run} library file.
\flowgraph{\resource{\cpp{}\\source code} \ar[r] & \toolbox{cpprisc} \ar[r] \ar[d] \ar[rd] & \resource{object file} \\ \variable{ECSINCLUDE} \ar[ru] & \resource{debugging\\information} & \resource{assembly\\listing}}
\seecpp\seeassembly\seerisc\seeobject\seedebugging
}

\providecommand{\cppwasm}{
\toolsection{cppwasm} is a compiler for the \cpp{} programming language targeting the WebAssembly architecture.
It generates machine code for WebAssembly targets from programs written in \cpp{} and stores it in corresponding object files.
For debugging purposes, it also creates a debugging information file as well as an assembly file containing a listing of the generated machine code.
The macro \texttt{\_\_wasm\_\_} is predefined in order to enable programmers to identify this tool and its target architecture while compiling.
Programs generated with this compiler require additional runtime support that is stored in the \file{cpp\-wasm\-run} library file.
\flowgraph{\resource{\cpp{}\\source code} \ar[r] & \toolbox{cppwasm} \ar[r] \ar[d] \ar[rd] & \resource{object file} \\ \variable{ECSINCLUDE} \ar[ru] & \resource{debugging\\information} & \resource{assembly\\listing}}
\seecpp\seeassembly\seewasm\seeobject\seedebugging
}

% FALSE tools

\providecommand{\falprint}{
\toolsection{falprint} is a pretty printer for the FALSE programming language.
It reformats the source code of FALSE programs and writes it to the standard output stream.
\flowgraph{\resource{FALSE\\source code} \ar[r] & \toolbox{falprint} \ar[r] & \resource{reformatted\\source code}}
\seefalse
}

\providecommand{\falcheck}{
\toolsection{falcheck} is a syntactic and semantic checker for the FALSE programming language.
It just performs syntactic and semantic checks on FALSE programs and writes its diagnostic messages to the standard error stream.
\flowgraph{\resource{FALSE\\source code} \ar[r] & \toolbox{falcheck} \ar[r] & \resource{diagnostic\\messages}}
\seefalse
}

\providecommand{\faldump}{
\toolsection{faldump} is a serializer for the FALSE programming language.
It dumps the complete internal representation of programs written in FALSE into an XML document.
\debuggingtool
\flowgraph{\resource{FALSE\\source code} \ar[r] & \toolbox{faldump} \ar[r] & \resource{internal\\representation}}
\seefalse
}

\providecommand{\falrun}{
\toolsection{falrun} is an interpreter for the FALSE programming language.
It processes and executes programs written in FALSE\@.
\flowgraph{\resource{FALSE\\source code} \ar[r] & \toolbox{falrun} \ar@/u/[r] & \resource{input/\\output} \ar@/d/[l]}
\seefalse
}

\providecommand{\falcpp}{
\toolsection{falcpp} is a transpiler for the FALSE programming language.
It translates programs written in FALSE into \cpp{} programs and stores them in corresponding source files.
\flowgraph{\resource{FALSE\\source code} \ar[r] & \toolbox{falcpp} \ar[r] & \resource{\cpp{}\\source file}}
\seefalse\seecpp
}

\providecommand{\falcode}{
\toolsection{falcode} is an intermediate code generator for the FALSE programming language.
It generates intermediate code from programs written in FALSE and stores it in corresponding assembly files.
\debuggingtool
\flowgraph{\resource{FALSE\\source code} \ar[r] & \toolbox{falcode} \ar[r] & \resource{intermediate\\code}}
\seefalse\seeassembly\seecode
}

\providecommand{\falamda}{
\toolsection{falamd16} is a compiler for the FALSE programming language targeting the AMD64 hardware architecture.
It generates machine code for AMD64 processors from programs written in FALSE and stores it in corresponding object files.
The compiler generates machine code for the 16-bit operating mode defined by the AMD64 architecture.
\flowgraph{\resource{FALSE\\source code} \ar[r] & \toolbox{falamd16} \ar[r] & \resource{object file}}
\seefalse\seeamd\seeobject
}

\providecommand{\falamdb}{
\toolsection{falamd32} is a compiler for the FALSE programming language targeting the AMD64 hardware architecture.
It generates machine code for AMD64 processors from programs written in FALSE and stores it in corresponding object files.
The compiler generates machine code for the 32-bit operating mode defined by the AMD64 architecture.
\flowgraph{\resource{FALSE\\source code} \ar[r] & \toolbox{falamd32} \ar[r] & \resource{object file}}
\seefalse\seeamd\seeobject
}

\providecommand{\falamdc}{
\toolsection{falamd64} is a compiler for the FALSE programming language targeting the AMD64 hardware architecture.
It generates machine code for AMD64 processors from programs written in FALSE and stores it in corresponding object files.
The compiler generates machine code for the 64-bit operating mode defined by the AMD64 architecture.
\flowgraph{\resource{FALSE\\source code} \ar[r] & \toolbox{falamd64} \ar[r] & \resource{object file}}
\seefalse\seeamd\seeobject
}

\providecommand{\falarma}{
\toolsection{falarma32} is a compiler for the FALSE programming language targeting the ARM hardware architecture.
It generates machine code for ARM processors executing A32 instructions from programs written in FALSE and stores it in corresponding object files.
\flowgraph{\resource{FALSE\\source code} \ar[r] & \toolbox{falarma32} \ar[r] & \resource{object file}}
\seefalse\seearm\seeobject
}

\providecommand{\falarmb}{
\toolsection{falarma64} is a compiler for the FALSE programming language targeting the ARM hardware architecture.
It generates machine code for ARM processors executing A64 instructions from programs written in FALSE and stores it in corresponding object files.
\flowgraph{\resource{FALSE\\source code} \ar[r] & \toolbox{falarma64} \ar[r] & \resource{object file}}
\seefalse\seearm\seeobject
}

\providecommand{\falarmc}{
\toolsection{falarmt32} is a compiler for the FALSE programming language targeting the ARM hardware architecture.
It generates machine code for ARM processors without floating-point extension executing T32 instructions from programs written in FALSE and stores it in corresponding object files.
\flowgraph{\resource{FALSE\\source code} \ar[r] & \toolbox{falarmt32} \ar[r] & \resource{object file}}
\seefalse\seearm\seeobject
}

\providecommand{\falarmcfpe}{
\toolsection{falarmt32fpe} is a compiler for the FALSE programming language targeting the ARM hardware architecture.
It generates machine code for ARM processors with floating-point extension executing T32 instructions from programs written in FALSE and stores it in corresponding object files.
\flowgraph{\resource{FALSE\\source code} \ar[r] & \toolbox{falarmt32fpe} \ar[r] & \resource{object file}}
\seefalse\seearm\seeobject
}

\providecommand{\falavr}{
\toolsection{falavr} is a compiler for the FALSE programming language targeting the AVR hardware architecture.
It generates machine code for AVR processors from programs written in FALSE and stores it in corresponding object files.
\flowgraph{\resource{FALSE\\source code} \ar[r] & \toolbox{falavr} \ar[r] & \resource{object file}}
\seefalse\seeavr\seeobject
}

\providecommand{\falavrtt}{
\toolsection{falavr32} is a compiler for the FALSE programming language targeting the AVR32 hardware architecture.
It generates machine code for AVR32 processors from programs written in FALSE and stores it in corresponding object files.
\flowgraph{\resource{FALSE\\source code} \ar[r] & \toolbox{falavr32} \ar[r] & \resource{object file}}
\seefalse\seeavrtt\seeobject
}

\providecommand{\falmabk}{
\toolsection{falm68k} is a compiler for the FALSE programming language targeting the M68000 hardware architecture.
It generates machine code for M68000 processors from programs written in FALSE and stores it in corresponding object files.
\flowgraph{\resource{FALSE\\source code} \ar[r] & \toolbox{falm68k} \ar[r] & \resource{object file}}
\seefalse\seemabk\seeobject
}

\providecommand{\falmibl}{
\toolsection{falmibl} is a compiler for the FALSE programming language targeting the MicroBlaze hardware architecture.
It generates machine code for MicroBlaze processors from programs written in FALSE and stores it in corresponding object files.
\flowgraph{\resource{FALSE\\source code} \ar[r] & \toolbox{falmibl} \ar[r] & \resource{object file}}
\seefalse\seemibl\seeobject
}

\providecommand{\falmipsa}{
\toolsection{falmips32} is a compiler for the FALSE programming language targeting the MIPS32 hardware architecture.
It generates machine code for MIPS32 processors from programs written in FALSE and stores it in corresponding object files.
\flowgraph{\resource{FALSE\\source code} \ar[r] & \toolbox{falmips32} \ar[r] & \resource{object file}}
\seefalse\seemips\seeobject
}

\providecommand{\falmipsb}{
\toolsection{falmips64} is a compiler for the FALSE programming language targeting the MIPS64 hardware architecture.
It generates machine code for MIPS64 processors from programs written in FALSE and stores it in corresponding object files.
\flowgraph{\resource{FALSE\\source code} \ar[r] & \toolbox{falmips64} \ar[r] & \resource{object file}}
\seefalse\seemips\seeobject
}

\providecommand{\falmmix}{
\toolsection{falmmix} is a compiler for the FALSE programming language targeting the MMIX hardware architecture.
It generates machine code for MMIX processors from programs written in FALSE and stores it in corresponding object files.
\flowgraph{\resource{FALSE\\source code} \ar[r] & \toolbox{falmmix} \ar[r] & \resource{object file}}
\seefalse\seemmix\seeobject
}

\providecommand{\falorok}{
\toolsection{falor1k} is a compiler for the FALSE programming language targeting the OpenRISC 1000 hardware architecture.
It generates machine code for OpenRISC 1000 processors from programs written in FALSE and stores it in corresponding object files.
\flowgraph{\resource{FALSE\\source code} \ar[r] & \toolbox{falor1k} \ar[r] & \resource{object file}}
\seefalse\seeorok\seeobject
}

\providecommand{\falppca}{
\toolsection{falppc32} is a compiler for the FALSE programming language targeting the PowerPC hardware architecture.
It generates machine code for PowerPC processors from programs written in FALSE and stores it in corresponding object files.
The compiler generates machine code for the 32-bit operating mode defined by the PowerPC architecture.
\flowgraph{\resource{FALSE\\source code} \ar[r] & \toolbox{falppc32} \ar[r] & \resource{object file}}
\seefalse\seeppc\seeobject
}

\providecommand{\falppcb}{
\toolsection{falppc64} is a compiler for the FALSE programming language targeting the PowerPC hardware architecture.
It generates machine code for PowerPC processors from programs written in FALSE and stores it in corresponding object files.
The compiler generates machine code for the 64-bit operating mode defined by the PowerPC architecture.
\flowgraph{\resource{FALSE\\source code} \ar[r] & \toolbox{falppc64} \ar[r] & \resource{object file}}
\seefalse\seeppc\seeobject
}

\providecommand{\falrisc}{
\toolsection{falrisc} is a compiler for the FALSE programming language targeting the RISC hardware architecture.
It generates machine code for RISC processors from programs written in FALSE and stores it in corresponding object files.
\flowgraph{\resource{FALSE\\source code} \ar[r] & \toolbox{falrisc} \ar[r] & \resource{object file}}
\seefalse\seerisc\seeobject
}

\providecommand{\falwasm}{
\toolsection{falwasm} is a compiler for the FALSE programming language targeting the WebAssembly architecture.
It generates machine code for WebAssembly targets from programs written in FALSE and stores it in corresponding object files.
\flowgraph{\resource{FALSE\\source code} \ar[r] & \toolbox{falwasm} \ar[r] & \resource{object file}}
\seefalse\seewasm\seeobject
}

% Oberon tools

\providecommand{\obprint}{
\toolsection{obprint} is a pretty printer for the Oberon programming language.
It reformats the source code of Oberon modules and writes it to the standard output stream.
\flowgraph{\resource{Oberon\\source code} \ar[r] & \toolbox{obprint} \ar[r] & \resource{reformatted\\source code}}
\seeoberon
}

\providecommand{\obcheck}{
\toolsection{obcheck} is a syntactic and semantic checker for the Oberon programming language.
It just performs syntactic and semantic checks on Oberon modules and writes its diagnostic messages to the standard error stream.
In addition, it stores the interface of each module in a symbol file which is required when other modules import the module.
\flowgraph{\resource{Oberon\\source code} \ar[r] & \toolbox{obcheck} \ar[r] \ar@/l/[d] & \resource{diagnostic\\messages} \\ \variable{ECSIMPORT} \ar[ru] & \resource{symbol\\files} \ar@/r/[u]}
\seeoberon
}

\providecommand{\obdump}{
\toolsection{obdump} is a serializer for the Oberon programming language.
It dumps the complete internal representation of modules written in Oberon into an XML document.
\debuggingtool
\flowgraph{\resource{Oberon\\source code} \ar[r] & \toolbox{obdump} \ar[r] \ar@/l/[d] & \resource{internal\\representation} \\ \variable{ECSIMPORT} \ar[ru] & \resource{symbol\\files} \ar@/r/[u]}
\seeoberon
}

\providecommand{\obrun}{
\toolsection{obrun} is an interpreter for the Oberon programming language.
It processes and executes modules written in Oberon.
This tool does neither generate nor process symbol files while interpreting modules.
If a module is imported by another one, its filename has to be named before the other one in the list of command-line arguments.
\flowgraph{\resource{Oberon\\source code} \ar[r] & \toolbox{obrun} \ar@/u/[r] & \resource{input/\\output} \ar@/d/[l]}
\seeoberon
}

\providecommand{\obcpp}{
\toolsection{obcpp} is a transpiler for the Oberon programming language.
It translates programs written in Oberon into \cpp{} programs and stores them in corresponding source and header files.
In addition, it stores the interface of each module in a symbol file which is required when other modules import the module.
The same interface is provided by the generated header file which can be used in other parts of the \cpp{} program.
\flowgraph{\resource{Oberon\\source code} \ar[r] & \toolbox{obcpp} \ar[r] \ar@/l/[d] \ar[rd] & \resource{\cpp{}\\source file} \\ \variable{ECSIMPORT} \ar[ru] & \resource{symbol\\files} \ar@/r/[u] & \resource{\cpp{}\\header file}}
\seeoberon\seecpp
}

\providecommand{\obdoc}{
\toolsection{obdoc} is a generic documentation generator for the Oberon programming language.
It processes several Oberon modules and assembles all information therein into a generic documentation.
In addition, it stores the interface of each module in a symbol file which is required when other modules import the module.
\debuggingtool
\flowgraph{\resource{Oberon\\source code} \ar[r] & \toolbox{obdoc} \ar[r] \ar@/l/[d] & \resource{generic\\documentation} \\ \variable{ECSIMPORT} \ar[ru] & \resource{symbol\\files} \ar@/r/[u]}
\seeoberon\seedocumentation
}

\providecommand{\obhtml}{
\toolsection{obhtml} is an HTML documentation generator for the Oberon programming language.
It processes several Oberon modules and assembles all information therein into an HTML document.
In addition, it stores the interface of each module in a symbol file which is required when other modules import the module.
\flowgraph{\resource{Oberon\\source code} \ar[r] & \toolbox{obhtml} \ar[r] \ar@/l/[d] & \resource{HTML\\document} \\ \variable{ECSIMPORT} \ar[ru] & \resource{symbol\\files} \ar@/r/[u]}
\seeoberon\seedocumentation
}

\providecommand{\oblatex}{
\toolsection{oblatex} is a Latex documentation generator for the Oberon programming language.
It processes several Oberon modules and assembles all information therein into a Latex document.
In addition, it stores the interface of each module in a symbol file which is required when other modules import the module.
\flowgraph{\resource{Oberon\\source code} \ar[r] & \toolbox{oblatex} \ar[r] \ar@/l/[d] & \resource{Latex\\document} \\ \variable{ECSIMPORT} \ar[ru] & \resource{symbol\\files} \ar@/r/[u]}
\seeoberon\seedocumentation
}

\providecommand{\obcode}{
\toolsection{obcode} is an intermediate code generator for the Oberon programming language.
It generates intermediate code from modules written in Oberon and stores it in corresponding assembly files.
In addition, it stores the interface of each module in a symbol file which is required when other modules import the module.
Programs generated with this tool require additional runtime support that is stored in the \file{ob\-code\-run} library file.
\debuggingtool
\flowgraph{\resource{Oberon\\source code} \ar[r] & \toolbox{obcode} \ar[r] \ar@/l/[d] & \resource{intermediate\\code} \\ \variable{ECSIMPORT} \ar[ru] & \resource{symbol\\files} \ar@/r/[u]}
\seeoberon\seeassembly\seecode
}

\providecommand{\obamda}{
\toolsection{obamd16} is a compiler for the Oberon programming language targeting the AMD64 hardware architecture.
It generates machine code for AMD64 processors from modules written in Oberon and stores it in corresponding object files.
The compiler generates machine code for the 16-bit operating mode defined by the AMD64 architecture.
For debugging purposes, it also creates a debugging information file as well as an assembly file containing a listing of the generated machine code.
In addition, it stores the interface of each module in a symbol file which is required when other modules import the module.
Programs generated with this compiler require additional runtime support that is stored in the \file{ob\-amd16\-run} library file.
\flowgraph{\resource{Oberon\\source code} \ar[r] & \toolbox{obamd16} \ar[r] \ar@/l/[d] \ar[rd] & \resource{object file} \\ \variable{ECSIMPORT} \ar[ru] & \resource{symbol\\files} \ar@/r/[u] & \resource{debugging\\information}}
\seeoberon\seeassembly\seeamd\seeobject\seedebugging
}

\providecommand{\obamdb}{
\toolsection{obamd32} is a compiler for the Oberon programming language targeting the AMD64 hardware architecture.
It generates machine code for AMD64 processors from modules written in Oberon and stores it in corresponding object files.
The compiler generates machine code for the 32-bit operating mode defined by the AMD64 architecture.
For debugging purposes, it also creates a debugging information file as well as an assembly file containing a listing of the generated machine code.
In addition, it stores the interface of each module in a symbol file which is required when other modules import the module.
Programs generated with this compiler require additional runtime support that is stored in the \file{ob\-amd32\-run} library file.
\flowgraph{\resource{Oberon\\source code} \ar[r] & \toolbox{obamd32} \ar[r] \ar@/l/[d] \ar[rd] & \resource{object file} \\ \variable{ECSIMPORT} \ar[ru] & \resource{symbol\\files} \ar@/r/[u] & \resource{debugging\\information}}
\seeoberon\seeassembly\seeamd\seeobject\seedebugging
}

\providecommand{\obamdc}{
\toolsection{obamd64} is a compiler for the Oberon programming language targeting the AMD64 hardware architecture.
It generates machine code for AMD64 processors from modules written in Oberon and stores it in corresponding object files.
The compiler generates machine code for the 64-bit operating mode defined by the AMD64 architecture.
For debugging purposes, it also creates a debugging information file as well as an assembly file containing a listing of the generated machine code.
In addition, it stores the interface of each module in a symbol file which is required when other modules import the module.
Programs generated with this compiler require additional runtime support that is stored in the \file{ob\-amd64\-run} library file.
\flowgraph{\resource{Oberon\\source code} \ar[r] & \toolbox{obamd64} \ar[r] \ar@/l/[d] \ar[rd] & \resource{object file} \\ \variable{ECSIMPORT} \ar[ru] & \resource{symbol\\files} \ar@/r/[u] & \resource{debugging\\information}}
\seeoberon\seeassembly\seeamd\seeobject\seedebugging
}

\providecommand{\obarma}{
\toolsection{obarma32} is a compiler for the Oberon programming language targeting the ARM hardware architecture.
It generates machine code for ARM processors executing A32 instructions from modules written in Oberon and stores it in corresponding object files.
For debugging purposes, it also creates a debugging information file as well as an assembly file containing a listing of the generated machine code.
In addition, it stores the interface of each module in a symbol file which is required when other modules import the module.
Programs generated with this compiler require additional runtime support that is stored in the \file{ob\-arma32\-run} library file.
\flowgraph{\resource{Oberon\\source code} \ar[r] & \toolbox{obarma32} \ar[r] \ar@/l/[d] \ar[rd] & \resource{object file} \\ \variable{ECSIMPORT} \ar[ru] & \resource{symbol\\files} \ar@/r/[u] & \resource{debugging\\information}}
\seeoberon\seeassembly\seearm\seeobject\seedebugging
}

\providecommand{\obarmb}{
\toolsection{obarma64} is a compiler for the Oberon programming language targeting the ARM hardware architecture.
It generates machine code for ARM processors executing A64 instructions from modules written in Oberon and stores it in corresponding object files.
For debugging purposes, it also creates a debugging information file as well as an assembly file containing a listing of the generated machine code.
In addition, it stores the interface of each module in a symbol file which is required when other modules import the module.
Programs generated with this compiler require additional runtime support that is stored in the \file{ob\-arma64\-run} library file.
\flowgraph{\resource{Oberon\\source code} \ar[r] & \toolbox{obarma64} \ar[r] \ar@/l/[d] \ar[rd] & \resource{object file} \\ \variable{ECSIMPORT} \ar[ru] & \resource{symbol\\files} \ar@/r/[u] & \resource{debugging\\information}}
\seeoberon\seeassembly\seearm\seeobject\seedebugging
}

\providecommand{\obarmc}{
\toolsection{obarmt32} is a compiler for the Oberon programming language targeting the ARM hardware architecture.
It generates machine code for ARM processors without floating-point extension executing T32 instructions from modules written in Oberon and stores it in corresponding object files.
For debugging purposes, it also creates a debugging information file as well as an assembly file containing a listing of the generated machine code.
In addition, it stores the interface of each module in a symbol file which is required when other modules import the module.
Programs generated with this compiler require additional runtime support that is stored in the \file{ob\-armt32\-run} library file.
\flowgraph{\resource{Oberon\\source code} \ar[r] & \toolbox{obarmt32} \ar[r] \ar@/l/[d] \ar[rd] & \resource{object file} \\ \variable{ECSIMPORT} \ar[ru] & \resource{symbol\\files} \ar@/r/[u] & \resource{debugging\\information}}
\seeoberon\seeassembly\seearm\seeobject\seedebugging
}

\providecommand{\obarmcfpe}{
\toolsection{obarmt32fpe} is a compiler for the Oberon programming language targeting the ARM hardware architecture.
It generates machine code for ARM processors with floating-point extension executing T32 instructions from modules written in Oberon and stores it in corresponding object files.
For debugging purposes, it also creates a debugging information file as well as an assembly file containing a listing of the generated machine code.
In addition, it stores the interface of each module in a symbol file which is required when other modules import the module.
Programs generated with this compiler require additional runtime support that is stored in the \file{ob\-armt32\-fpe\-run} library file.
\flowgraph{\resource{Oberon\\source code} \ar[r] & \toolbox{obarmt32fpe} \ar[r] \ar@/l/[d] \ar[rd] & \resource{object file} \\ \variable{ECSIMPORT} \ar[ru] & \resource{symbol\\files} \ar@/r/[u] & \resource{debugging\\information}}
\seeoberon\seeassembly\seearm\seeobject\seedebugging
}

\providecommand{\obavr}{
\toolsection{obavr} is a compiler for the Oberon programming language targeting the AVR hardware architecture.
It generates machine code for AVR processors from modules written in Oberon and stores it in corresponding object files.
For debugging purposes, it also creates a debugging information file as well as an assembly file containing a listing of the generated machine code.
In addition, it stores the interface of each module in a symbol file which is required when other modules import the module.
Programs generated with this compiler require additional runtime support that is stored in the \file{ob\-avr\-run} library file.
\flowgraph{\resource{Oberon\\source code} \ar[r] & \toolbox{obavr} \ar[r] \ar@/l/[d] \ar[rd] & \resource{object file} \\ \variable{ECSIMPORT} \ar[ru] & \resource{symbol\\files} \ar@/r/[u] & \resource{debugging\\information}}
\seeoberon\seeassembly\seeavr\seeobject\seedebugging
}

\providecommand{\obavrtt}{
\toolsection{obavr32} is a compiler for the Oberon programming language targeting the AVR32 hardware architecture.
It generates machine code for AVR32 processors from modules written in Oberon and stores it in corresponding object files.
For debugging purposes, it also creates a debugging information file as well as an assembly file containing a listing of the generated machine code.
In addition, it stores the interface of each module in a symbol file which is required when other modules import the module.
Programs generated with this compiler require additional runtime support that is stored in the \file{ob\-avr32\-run} library file.
\flowgraph{\resource{Oberon\\source code} \ar[r] & \toolbox{obavr32} \ar[r] \ar@/l/[d] \ar[rd] & \resource{object file} \\ \variable{ECSIMPORT} \ar[ru] & \resource{symbol\\files} \ar@/r/[u] & \resource{debugging\\information}}
\seeoberon\seeassembly\seeavrtt\seeobject\seedebugging
}

\providecommand{\obmabk}{
\toolsection{obm68k} is a compiler for the Oberon programming language targeting the M68000 hardware architecture.
It generates machine code for M68000 processors from modules written in Oberon and stores it in corresponding object files.
For debugging purposes, it also creates a debugging information file as well as an assembly file containing a listing of the generated machine code.
In addition, it stores the interface of each module in a symbol file which is required when other modules import the module.
Programs generated with this compiler require additional runtime support that is stored in the \file{ob\-m68k\-run} library file.
\flowgraph{\resource{Oberon\\source code} \ar[r] & \toolbox{obm68k} \ar[r] \ar@/l/[d] \ar[rd] & \resource{object file} \\ \variable{ECSIMPORT} \ar[ru] & \resource{symbol\\files} \ar@/r/[u] & \resource{debugging\\information}}
\seeoberon\seeassembly\seemabk\seeobject\seedebugging
}

\providecommand{\obmibl}{
\toolsection{obmibl} is a compiler for the Oberon programming language targeting the MicroBlaze hardware architecture.
It generates machine code for MicroBlaze processors from modules written in Oberon and stores it in corresponding object files.
For debugging purposes, it also creates a debugging information file as well as an assembly file containing a listing of the generated machine code.
In addition, it stores the interface of each module in a symbol file which is required when other modules import the module.
Programs generated with this compiler require additional runtime support that is stored in the \file{ob\-mibl\-run} library file.
\flowgraph{\resource{Oberon\\source code} \ar[r] & \toolbox{obmibl} \ar[r] \ar@/l/[d] \ar[rd] & \resource{object file} \\ \variable{ECSIMPORT} \ar[ru] & \resource{symbol\\files} \ar@/r/[u] & \resource{debugging\\information}}
\seeoberon\seeassembly\seemibl\seeobject\seedebugging
}

\providecommand{\obmipsa}{
\toolsection{obmips32} is a compiler for the Oberon programming language targeting the MIPS32 hardware architecture.
It generates machine code for MIPS32 processors from modules written in Oberon and stores it in corresponding object files.
For debugging purposes, it also creates a debugging information file as well as an assembly file containing a listing of the generated machine code.
In addition, it stores the interface of each module in a symbol file which is required when other modules import the module.
Programs generated with this compiler require additional runtime support that is stored in the \file{ob\-mips32\-run} library file.
\flowgraph{\resource{Oberon\\source code} \ar[r] & \toolbox{obmips32} \ar[r] \ar@/l/[d] \ar[rd] & \resource{object file} \\ \variable{ECSIMPORT} \ar[ru] & \resource{symbol\\files} \ar@/r/[u] & \resource{debugging\\information}}
\seeoberon\seeassembly\seemips\seeobject\seedebugging
}

\providecommand{\obmipsb}{
\toolsection{obmips64} is a compiler for the Oberon programming language targeting the MIPS64 hardware architecture.
It generates machine code for MIPS64 processors from modules written in Oberon and stores it in corresponding object files.
For debugging purposes, it also creates a debugging information file as well as an assembly file containing a listing of the generated machine code.
In addition, it stores the interface of each module in a symbol file which is required when other modules import the module.
Programs generated with this compiler require additional runtime support that is stored in the \file{ob\-mips64\-run} library file.
\flowgraph{\resource{Oberon\\source code} \ar[r] & \toolbox{obmips64} \ar[r] \ar@/l/[d] \ar[rd] & \resource{object file} \\ \variable{ECSIMPORT} \ar[ru] & \resource{symbol\\files} \ar@/r/[u] & \resource{debugging\\information}}
\seeoberon\seeassembly\seemips\seeobject\seedebugging
}

\providecommand{\obmmix}{
\toolsection{obmmix} is a compiler for the Oberon programming language targeting the MMIX hardware architecture.
It generates machine code for MMIX processors from modules written in Oberon and stores it in corresponding object files.
For debugging purposes, it also creates a debugging information file as well as an assembly file containing a listing of the generated machine code.
In addition, it stores the interface of each module in a symbol file which is required when other modules import the module.
Programs generated with this compiler require additional runtime support that is stored in the \file{ob\-mmix\-run} library file.
\flowgraph{\resource{Oberon\\source code} \ar[r] & \toolbox{obmmix} \ar[r] \ar@/l/[d] \ar[rd] & \resource{object file} \\ \variable{ECSIMPORT} \ar[ru] & \resource{symbol\\files} \ar@/r/[u] & \resource{debugging\\information}}
\seeoberon\seeassembly\seemmix\seeobject\seedebugging
}

\providecommand{\oborok}{
\toolsection{obor1k} is a compiler for the Oberon programming language targeting the OpenRISC 1000 hardware architecture.
It generates machine code for OpenRISC 1000 processors from modules written in Oberon and stores it in corresponding object files.
For debugging purposes, it also creates a debugging information file as well as an assembly file containing a listing of the generated machine code.
In addition, it stores the interface of each module in a symbol file which is required when other modules import the module.
Programs generated with this compiler require additional runtime support that is stored in the \file{ob\-or1k\-run} library file.
\flowgraph{\resource{Oberon\\source code} \ar[r] & \toolbox{obor1k} \ar[r] \ar@/l/[d] \ar[rd] & \resource{object file} \\ \variable{ECSIMPORT} \ar[ru] & \resource{symbol\\files} \ar@/r/[u] & \resource{debugging\\information}}
\seeoberon\seeassembly\seeorok\seeobject\seedebugging
}

\providecommand{\obppca}{
\toolsection{obppc32} is a compiler for the Oberon programming language targeting the PowerPC hardware architecture.
It generates machine code for PowerPC processors from modules written in Oberon and stores it in corresponding object files.
The compiler generates machine code for the 32-bit operating mode defined by the PowerPC architecture.
For debugging purposes, it also creates a debugging information file as well as an assembly file containing a listing of the generated machine code.
In addition, it stores the interface of each module in a symbol file which is required when other modules import the module.
Programs generated with this compiler require additional runtime support that is stored in the \file{ob\-ppc32\-run} library file.
\flowgraph{\resource{Oberon\\source code} \ar[r] & \toolbox{obppc32} \ar[r] \ar@/l/[d] \ar[rd] & \resource{object file} \\ \variable{ECSIMPORT} \ar[ru] & \resource{symbol\\files} \ar@/r/[u] & \resource{debugging\\information}}
\seeoberon\seeassembly\seeppc\seeobject\seedebugging
}

\providecommand{\obppcb}{
\toolsection{obppc64} is a compiler for the Oberon programming language targeting the PowerPC hardware architecture.
It generates machine code for PowerPC processors from modules written in Oberon and stores it in corresponding object files.
The compiler generates machine code for the 64-bit operating mode defined by the PowerPC architecture.
For debugging purposes, it also creates a debugging information file as well as an assembly file containing a listing of the generated machine code.
In addition, it stores the interface of each module in a symbol file which is required when other modules import the module.
Programs generated with this compiler require additional runtime support that is stored in the \file{ob\-ppc64\-run} library file.
\flowgraph{\resource{Oberon\\source code} \ar[r] & \toolbox{obppc64} \ar[r] \ar@/l/[d] \ar[rd] & \resource{object file} \\ \variable{ECSIMPORT} \ar[ru] & \resource{symbol\\files} \ar@/r/[u] & \resource{debugging\\information}}
\seeoberon\seeassembly\seeppc\seeobject\seedebugging
}

\providecommand{\obrisc}{
\toolsection{obrisc} is a compiler for the Oberon programming language targeting the RISC hardware architecture.
It generates machine code for RISC processors from modules written in Oberon and stores it in corresponding object files.
For debugging purposes, it also creates a debugging information file as well as an assembly file containing a listing of the generated machine code.
In addition, it stores the interface of each module in a symbol file which is required when other modules import the module.
Programs generated with this compiler require additional runtime support that is stored in the \file{ob\-risc\-run} library file.
\flowgraph{\resource{Oberon\\source code} \ar[r] & \toolbox{obrisc} \ar[r] \ar@/l/[d] \ar[rd] & \resource{object file} \\ \variable{ECSIMPORT} \ar[ru] & \resource{symbol\\files} \ar@/r/[u] & \resource{debugging\\information}}
\seeoberon\seeassembly\seerisc\seeobject\seedebugging
}

\providecommand{\obwasm}{
\toolsection{obwasm} is a compiler for the Oberon programming language targeting the WebAssembly architecture.
It generates machine code for WebAssembly targets from modules written in Oberon and stores it in corresponding object files.
For debugging purposes, it also creates a debugging information file as well as an assembly file containing a listing of the generated machine code.
In addition, it stores the interface of each module in a symbol file which is required when other modules import the module.
Programs generated with this compiler require additional runtime support that is stored in the \file{ob\-wasm\-run} library file.
\flowgraph{\resource{Oberon\\source code} \ar[r] & \toolbox{obwasm} \ar[r] \ar@/l/[d] \ar[rd] & \resource{object file} \\ \variable{ECSIMPORT} \ar[ru] & \resource{symbol\\files} \ar@/r/[u] & \resource{debugging\\information}}
\seeoberon\seeassembly\seewasm\seeobject\seedebugging
}

% converter tools

\providecommand{\dbgdwarf}{
\toolsection{dbgdwarf} is a DWARF debugging information converter tool.
It converts debugging information into the DWARF debugging data format and stores it in corresponding object files~\cite{dwarffile}.
The resulting debugging object files can be combined with runtime support that creates Executable and Linking Format (ELF) files~\cite{elffile}.
\flowgraph{\resource{debugging\\information} \ar[r] & \toolbox{dbgdwarf} \ar[r] & \resource{debugging\\object file}}
\seeobject\seedebugging
}

% assembler tools

\providecommand{\asmprint}{
\toolsection{asmprint} is a pretty printer for generic assembly code.
It reformats generic assembly code and writes it to the standard output stream.
\flowgraph{\resource{generic assembly\\source code} \ar[r] & \toolbox{asmprint} \ar[r] & \resource{reformatted\\source code}}
\seeassembly
}

\providecommand{\amdaasm}{
\toolsection{amd16asm} is an assembler for the AMD64 hardware architecture.
It translates assembly code into machine code for AMD64 processors and stores it in corresponding object files.
By default, the assembler generates machine code for the 16-bit operating mode defined by the AMD64 architecture.
\flowgraph{\resource{AMD16 assembly\\source code} \ar[r] & \toolbox{amd16asm} \ar[r] & \resource{object file}}
\seeassembly\seeamd\seeobject
}

\providecommand{\amdadism}{
\toolsection{amd16dism} is a disassembler for the AMD64 hardware architecture.
It translates machine code from object files targeting AMD64 processors into assembly code and writes it to the standard output stream.
It assumes that the machine code was generated for the 16-bit operating mode defined by the AMD64 architecture.
\flowgraph{\resource{object file} \ar[r] & \toolbox{amd16dism} \ar[r] & \resource{disassembly\\listing}}
\seeassembly\seeamd\seeobject
}

\providecommand{\amdbasm}{
\toolsection{amd32asm} is an assembler for the AMD64 hardware architecture.
It translates assembly code into machine code for AMD64 processors and stores it in corresponding object files.
By default, the assembler generates machine code for the 32-bit operating mode defined by the AMD64 architecture.
\flowgraph{\resource{AMD32 assembly\\source code} \ar[r] & \toolbox{amd32asm} \ar[r] & \resource{object file}}
\seeassembly\seeamd\seeobject
}

\providecommand{\amdbdism}{
\toolsection{amd32dism} is a disassembler for the AMD64 hardware architecture.
It translates machine code from object files targeting AMD64 processors into assembly code and writes it to the standard output stream.
It assumes that the machine code was generated for the 32-bit operating mode defined by the AMD64 architecture.
\flowgraph{\resource{object file} \ar[r] & \toolbox{amd32dism} \ar[r] & \resource{disassembly\\listing}}
\seeassembly\seeamd\seeobject
}

\providecommand{\amdcasm}{
\toolsection{amd64asm} is an assembler for the AMD64 hardware architecture.
It translates assembly code into machine code for AMD64 processors and stores it in corresponding object files.
By default, the assembler generates machine code for the 64-bit operating mode defined by the AMD64 architecture.
\flowgraph{\resource{AMD64 assembly\\source code} \ar[r] & \toolbox{amd64asm} \ar[r] & \resource{object file}}
\seeassembly\seeamd\seeobject
}

\providecommand{\amdcdism}{
\toolsection{amd64dism} is a disassembler for the AMD64 hardware architecture.
It translates machine code from object files targeting AMD64 processors into assembly code and writes it to the standard output stream.
It assumes that the machine code was generated for the 64-bit operating mode defined by the AMD64 architecture.
\flowgraph{\resource{object file} \ar[r] & \toolbox{amd64dism} \ar[r] & \resource{disassembly\\listing}}
\seeassembly\seeamd\seeobject
}

\providecommand{\armaasm}{
\toolsection{arma32asm} is an assembler for the ARM hardware architecture.
It translates assembly code into machine code for ARM processors executing A32 instructions and stores it in corresponding object files.
\flowgraph{\resource{ARM A32 assembly\\source code} \ar[r] & \toolbox{arma32asm} \ar[r] & \resource{object file}}
\seeassembly\seearm\seeobject
}

\providecommand{\armadism}{
\toolsection{arma32dism} is a disassembler for the ARM hardware architecture.
It translates machine code from object files targeting ARM processors executing A32 instructions into assembly code and writes it to the standard output stream.
\flowgraph{\resource{object file} \ar[r] & \toolbox{arma32dism} \ar[r] & \resource{disassembly\\listing}}
\seeassembly\seearm\seeobject
}

\providecommand{\armbasm}{
\toolsection{arma64asm} is an assembler for the ARM hardware architecture.
It translates assembly code into machine code for ARM processors executing A64 instructions and stores it in corresponding object files.
\flowgraph{\resource{ARM A64 assembly\\source code} \ar[r] & \toolbox{arma64asm} \ar[r] & \resource{object file}}
\seeassembly\seearm\seeobject
}

\providecommand{\armbdism}{
\toolsection{arma64dism} is a disassembler for the ARM hardware architecture.
It translates machine code from object files targeting ARM processors executing A64 instructions into assembly code and writes it to the standard output stream.
\flowgraph{\resource{object file} \ar[r] & \toolbox{arma64dism} \ar[r] & \resource{disassembly\\listing}}
\seeassembly\seearm\seeobject
}

\providecommand{\armcasm}{
\toolsection{armt32asm} is an assembler for the ARM hardware architecture.
It translates assembly code into machine code for ARM processors executing T32 instructions and stores it in corresponding object files.
\flowgraph{\resource{ARM T32 assembly\\source code} \ar[r] & \toolbox{armt32asm} \ar[r] & \resource{object file}}
\seeassembly\seearm\seeobject
}

\providecommand{\armcdism}{
\toolsection{armt32dism} is a disassembler for the ARM hardware architecture.
It translates machine code from object files targeting ARM processors executing T32 instructions into assembly code and writes it to the standard output stream.
\flowgraph{\resource{object file} \ar[r] & \toolbox{armt32dism} \ar[r] & \resource{disassembly\\listing}}
\seeassembly\seearm\seeobject
}

\providecommand{\avrasm}{
\toolsection{avrasm} is an assembler for the AVR hardware architecture.
It translates assembly code into machine code for AVR processors and stores it in corresponding object files.
The identifiers \texttt{RXL}, \texttt{RXH}, \texttt{RYL}, \texttt{RYH}, \texttt{RZL}, and \texttt{RZH} are predefined and name the corresponding registers.
The identifiers \texttt{SPL} and \texttt{SPH} are also predefined and evaluate to the address of the corresponding registers.
\flowgraph{\resource{AVR assembly\\source code} \ar[r] & \toolbox{avrasm} \ar[r] & \resource{object file}}
\seeassembly\seeavr\seeobject
}

\providecommand{\avrdism}{
\toolsection{avrdism} is a disassembler for the AVR hardware architecture.
It translates machine code from object files targeting AVR processors into assembly code and writes it to the standard output stream.
\flowgraph{\resource{object file} \ar[r] & \toolbox{avrdism} \ar[r] & \resource{disassembly\\listing}}
\seeassembly\seeavr\seeobject
}

\providecommand{\avrttasm}{
\toolsection{avr32asm} is an assembler for the AVR32 hardware architecture.
It translates assembly code into machine code for AVR32 processors and stores it in corresponding object files.
\flowgraph{\resource{AVR32 assembly\\source code} \ar[r] & \toolbox{avr32asm} \ar[r] & \resource{object file}}
\seeassembly\seeavrtt\seeobject
}

\providecommand{\avrttdism}{
\toolsection{avr32dism} is a disassembler for the AVR32 hardware architecture.
It translates machine code from object files targeting AVR32 processors into assembly code and writes it to the standard output stream.
\flowgraph{\resource{object file} \ar[r] & \toolbox{avr32dism} \ar[r] & \resource{disassembly\\listing}}
\seeassembly\seeavrtt\seeobject
}

\providecommand{\mabkasm}{
\toolsection{m68kasm} is an assembler for the M68000 hardware architecture.
It translates assembly code into machine code for M68000 processors and stores it in corresponding object files.
\flowgraph{\resource{68000 assembly\\source code} \ar[r] & \toolbox{m68kasm} \ar[r] & \resource{object file}}
\seeassembly\seemabk\seeobject
}

\providecommand{\mabkdism}{
\toolsection{m68kdism} is a disassembler for the M68000 hardware architecture.
It translates machine code from object files targeting M68000 processors into assembly code and writes it to the standard output stream.
\flowgraph{\resource{object file} \ar[r] & \toolbox{m68kdism} \ar[r] & \resource{disassembly\\listing}}
\seeassembly\seemabk\seeobject
}

\providecommand{\miblasm}{
\toolsection{miblasm} is an assembler for the MicroBlaze hardware architecture.
It translates assembly code into machine code for MicroBlaze processors and stores it in corresponding object files.
\flowgraph{\resource{MicroBlaze assembly\\source code} \ar[r] & \toolbox{miblasm} \ar[r] & \resource{object file}}
\seeassembly\seemibl\seeobject
}

\providecommand{\mibldism}{
\toolsection{mibldism} is a disassembler for the MicroBlaze hardware architecture.
It translates machine code from object files targeting MicroBlaze processors into assembly code and writes it to the standard output stream.
\flowgraph{\resource{object file} \ar[r] & \toolbox{mibldism} \ar[r] & \resource{disassembly\\listing}}
\seeassembly\seemibl\seeobject
}

\providecommand{\mipsaasm}{
\toolsection{mips32asm} is an assembler for the MIPS32 hardware architecture.
It translates assembly code into machine code for MIPS32 processors and stores it in corresponding object files.
\flowgraph{\resource{MIPS32 assembly\\source code} \ar[r] & \toolbox{mips32asm} \ar[r] & \resource{object file}}
\seeassembly\seemips\seeobject
}

\providecommand{\mipsadism}{
\toolsection{mips32dism} is a disassembler for the MIPS32 hardware architecture.
It translates machine code from object files targeting MIPS32 processors into assembly code and writes it to the standard output stream.
\flowgraph{\resource{object file} \ar[r] & \toolbox{mips32dism} \ar[r] & \resource{disassembly\\listing}}
\seeassembly\seemips\seeobject
}

\providecommand{\mipsbasm}{
\toolsection{mips64asm} is an assembler for the MIPS64 hardware architecture.
It translates assembly code into machine code for MIPS64 processors and stores it in corresponding object files.
\flowgraph{\resource{MIPS64 assembly\\source code} \ar[r] & \toolbox{mips64asm} \ar[r] & \resource{object file}}
\seeassembly\seemips\seeobject
}

\providecommand{\mipsbdism}{
\toolsection{mips64dism} is a disassembler for the MIPS64 hardware architecture.
It translates machine code from object files targeting MIPS64 processors into assembly code and writes it to the standard output stream.
\flowgraph{\resource{object file} \ar[r] & \toolbox{mips64dism} \ar[r] & \resource{disassembly\\listing}}
\seeassembly\seemips\seeobject
}

\providecommand{\mmixasm}{
\toolsection{mmixasm} is an assembler for the MMIX hardware architecture.
It translates assembly code into machine code for MMIX processors and stores it in corresponding object files.
The names of all special registers are predefined and evaluate to the corresponding number.
\flowgraph{\resource{MMIX assembly\\source code} \ar[r] & \toolbox{mmixasm} \ar[r] & \resource{object file}}
\seeassembly\seemmix\seeobject
}

\providecommand{\mmixdism}{
\toolsection{mmixdism} is a disassembler for the MMIX hardware architecture.
It translates machine code from object files targeting MMIX processors into assembly code and writes it to the standard output stream.
\flowgraph{\resource{object file} \ar[r] & \toolbox{mmixdism} \ar[r] & \resource{disassembly\\listing}}
\seeassembly\seemmix\seeobject
}

\providecommand{\orokasm}{
\toolsection{or1kasm} is an assembler for the OpenRISC 1000 hardware architecture.
It translates assembly code into machine code for OpenRISC 1000 processors and stores it in corresponding object files.
\flowgraph{\resource{OpenRISC 1000 assembly\\source code} \ar[r] & \toolbox{or1kasm} \ar[r] & \resource{object file}}
\seeassembly\seeorok\seeobject
}

\providecommand{\orokdism}{
\toolsection{or1kdism} is a disassembler for the OpenRISC 1000 hardware architecture.
It translates machine code from object files targeting OpenRISC 1000 processors into assembly code and writes it to the standard output stream.
\flowgraph{\resource{object file} \ar[r] & \toolbox{or1kdism} \ar[r] & \resource{disassembly\\listing}}
\seeassembly\seeorok\seeobject
}

\providecommand{\ppcaasm}{
\toolsection{ppc32asm} is an assembler for the PowerPC hardware architecture.
It translates assembly code into machine code for PowerPC processors and stores it in corresponding object files.
By default, the assembler generates machine code for the 32-bit operating mode defined by the PowerPC architecture.
\flowgraph{\resource{PowerPC assembly\\source code} \ar[r] & \toolbox{ppc32asm} \ar[r] & \resource{object file}}
\seeassembly\seeppc\seeobject
}

\providecommand{\ppcadism}{
\toolsection{ppc32dism} is a disassembler for the PowerPC hardware architecture.
It translates machine code from object files targeting PowerPC processors into assembly code and writes it to the standard output stream.
It assumes that the machine code was generated for the 32-bit operating mode defined by the PowerPC architecture.
\flowgraph{\resource{object file} \ar[r] & \toolbox{ppc32dism} \ar[r] & \resource{disassembly\\listing}}
\seeassembly\seeppc\seeobject
}

\providecommand{\ppcbasm}{
\toolsection{ppc64asm} is an assembler for the PowerPC hardware architecture.
It translates assembly code into machine code for PowerPC processors and stores it in corresponding object files.
By default, the assembler generates machine code for the 64-bit operating mode defined by the PowerPC architecture.
\flowgraph{\resource{PowerPC assembly\\source code} \ar[r] & \toolbox{ppc64asm} \ar[r] & \resource{object file}}
\seeassembly\seeppc\seeobject
}

\providecommand{\ppcbdism}{
\toolsection{ppc64dism} is a disassembler for the PowerPC hardware architecture.
It translates machine code from object files targeting PowerPC processors into assembly code and writes it to the standard output stream.
It assumes that the machine code was generated for the 64-bit operating mode defined by the PowerPC architecture.
\flowgraph{\resource{object file} \ar[r] & \toolbox{ppc64dism} \ar[r] & \resource{disassembly\\listing}}
\seeassembly\seeppc\seeobject
}

\providecommand{\riscasm}{
\toolsection{riscasm} is an assembler for the RISC hardware architecture.
It translates assembly code into machine code for RISC processors and stores it in corresponding object files.
The names of all special registers are predefined and evaluate to the corresponding number.
\flowgraph{\resource{RISC assembly\\source code} \ar[r] & \toolbox{riscasm} \ar[r] & \resource{object file}}
\seeassembly\seerisc\seeobject
}

\providecommand{\riscdism}{
\toolsection{riscdism} is a disassembler for the RISC hardware architecture.
It translates machine code from object files targeting RISC processors into assembly code and writes it to the standard output stream.
\flowgraph{\resource{object file} \ar[r] & \toolbox{riscdism} \ar[r] & \resource{disassembly\\listing}}
\seeassembly\seerisc\seeobject
}

\providecommand{\wasmasm}{
\toolsection{wasmasm} is an assembler for the WebAssembly architecture.
It translates assembly code into machine code for WebAssembly targets and stores it in corresponding object files.
The names of all special registers are predefined and evaluate to the corresponding number.
\flowgraph{\resource{WebAssembly assembly\\source code} \ar[r] & \toolbox{wasmasm} \ar[r] & \resource{object file}}
\seeassembly\seewasm\seeobject
}

\providecommand{\wasmdism}{
\toolsection{wasmdism} is a disassembler for the WebAssembly architecture.
It translates machine code from object files targeting WebAssembly targets into assembly code and writes it to the standard output stream.
\flowgraph{\resource{object file} \ar[r] & \toolbox{wasmdism} \ar[r] & \resource{disassembly\\listing}}
\seeassembly\seewasm\seeobject
}

% linker tools

\providecommand{\linklib}{
\toolsection{linklib} is an object file combiner.
It creates a static library file by combining all object files given to it into a single one.
\flowgraph{\resource{object files} \ar[r] & \toolbox{linklib} \ar[r] & \resource{library file}}
\seeobject
}

\providecommand{\linkbin}{
\toolsection{linkbin} is a linker for plain binary files.
It links all object files given to it into a single image and stores it in a binary file that begins with the first linked section.
It also creates a map file that lists the address, type, name and size of all used sections.
The filename extension of the resulting binary file can be specified by putting it into a constant data section called \texttt{\_extension}.
\flowgraph{\resource{object files} \ar[r] & \toolbox{linkbin} \ar[r] \ar[d] & \resource{binary file} \\ & \resource{map file}}
\seeobject
}

\providecommand{\linkmem}{
\toolsection{linkmem} is a linker for plain binary files partitioned into random-access and read-only memory.
It links all object files given to it into two distinct images, one for data sections and one for code and constant data sections, and stores each image in a binary file that begins with the first linked section of the corresponding type.
It also creates a map file that lists the address, type, name and size of all used sections.
\flowgraph{\resource{object files} \ar[r] & \toolbox{linkmem} \ar[r] \ar[d] & \resource{RAM file/\\ROM file} \\ & \resource{map file}}
\seeobject
}

\providecommand{\linkprg}{
\toolsection{linkprg} is a linker for GEMDOS executable files.
It links all object files given to it into a single image and stores the image in an Atari GEMDOS executable file~\cite{gemdosfile}.
It also creates a map file that lists the address relative to the text segment, type, name and size of all used sections.
The filename extension of the resulting executable file can be specified by putting it into a constant data section called \texttt{\_extension}.
The GEMDOS executable file format requires all patch patterns of absolute link patches to consist of four full bitmasks with descending offsets.
\flowgraph{\resource{object files} \ar[r] & \toolbox{linkprg} \ar[r] \ar[d] & \resource{executable file} \\ & \resource{map file}}
\seeobject
}

\providecommand{\linkhex}{
\toolsection{linkhex} is a linker for Intel HEX files.
It links all code sections of the object files given to it into single image and stores the image in an Intel HEX file~\cite{hexfile} that begins with the first linked section.
It also creates a map file that lists the address, type, name and size of all used sections.
\flowgraph{\resource{object files} \ar[r] & \toolbox{linkhex} \ar[r] \ar[d] & \resource{HEX file} \\ & \resource{map file}}
\seeobject
}

\providecommand{\mapsearch}{
\toolsection{mapsearch} is a debugging tool.
It searches map files generated by linker tools for the name of a binary section that encompasses a memory address read from the standard input stream.
If additionally provided with one or more object files, it also stores an excerpt thereof in a separate object file called map search result which only contains the identified binary section for disassembling purposes.
\flowgraph{& \resource{map files/\\object files} \ar[d] \\ \resource{memory\\address} \ar[r] & \toolbox{mapsearch} \ar[r] \ar[d] & \resource{section name/\\relative offset} \\ & \resource{object file\\excerpt}}
\seeobject
}

\renewcommand{\seedebugging}{}

\startchapter{Debugging Information}{Debugging Information Representation}{debugging}
{This \documentation{} describes the debugging information generated by the compilers of the \ecs{} and its open file format.
Additionally, it describes the functionality and interface of the debugging information converter tools provided by the \ecs{}.}

\epigraph{Every failure is a step to success.}{William Whewell}

\section{Introduction}

Although the \ecs{} does not provide its own debugger tool, its compiler tools do collect and store \emph{debugging information} for external debuggers.
The debugging information generated alongside an object file consists of an abstract representation of all programming language constructs like functions, variables, data types, and statements compiled into the object file.
It is designed to enable debuggers to support \emph{program animation} and \emph{memory inspection}.
For this purpose, the \ecs{} provides converter tools which generate a binary representation of the debugging information and store it in specific debugging data formats using additional object files as shown in Figure~\ref{fig:dbgdataflow}.
The resulting debugging object files can later be optionally linked together with the original object file.
\seeobject

\begin{figure}
\flowgraph{
\resource{source code} \ar[d] & \resource{source code} \ar[d] & \resource{source code} \ar[d] \\
\converter{Compiler \textit{A}} \ar[d] \ar[rd] & \converter{Compiler \textit{B}} \ar[d] & \converter{Compiler \textit{C}} \ar[d] \ar[ld] \\
\resource{object file} & \resource{debugging\\information} \ar[ld] \ar[d] \ar[rd] & \resource{object file} \\
\converter{Converter \textit{X}} \ar[d] & \converter{Converter \textit{Y}} \ar[d] & \converter{Converter \textit{Z}} \ar[d] \\
\resource{debugging\\object file} & \resource{debugging\\object file} & \resource{debugging\\object file} \\
}\caption[Debugging information representation of programs]{Debugging information representation of programs in-between compilers and converters}
\label{fig:dbgdataflow}
\end{figure}

Converting debugging information into separate object files effectively decouples a compiler from the debugger of the target runtime environment and its required debugging data format.
It also ensures that compilers always generate the same output regardless of whether the resulting binary executable is subject to debugging or not.
The following sections describe the semantics of the debugging information representation alongside the syntax of its textual file format.

\section{Debugging Information Structure}

The debugging information generated by the various compiler tools of the \ecs{} always consists of a short description of target hardware architecture.
It also contains a list of \emph{information entries} which are generic representations of programming language constructs like functions, variables, or data types.

\subsection{Information Entries}\label{sec:dbginformationentries}

Each information entry has a unique name and a \emph{source code location} referring to its textual declaration in the source code.
The \ecs{} supports the following kinds of information entries:

\begin{itemize}

\item Code Entry\alignright\syntax{"code"}\nopagebreak

A code entry represents a functional unit of the programming language like a function or procedure and corresponds to the code section in the object file with the same name.
It contains a \emph{type declaration} which describes the memory layout and data representation of the returned result.
It additionally holds of a list of \emph{symbol declarations} which describe the storage objects declared by this information entry.
It also contains a list of \emph{breakpoints} which associate source code locations of statements with the corresponding machine code offsets in the code section.

\item Data Entry\alignright\syntax{"data"}\nopagebreak

A data entry represents a data unit of the programming language like a global variable and corresponds to the data section in the object file with the same name.
It contains a type declaration for the storage object represented by the data section.

\item Type Entry\alignright\syntax{"type"}\nopagebreak

A type entry represents a named type definition of the programming language.
It contains a type declaration and may omit the source code location for predefined types.

\end{itemize}

\subsection{Symbol Declarations}\label{sec:dbgsymboldeclarations}

A symbol declaration refers to a single storage object managed by a code section like a local variable or parameter.
It has a unique name and a source code location referring to its corresponding declaration in the source code.
It also contains a type declaration for the data type of its storage object and a description of its \emph{lifetime} in terms of a range of offsets in the code section wherein the symbol is considered alive.
The \ecs{} supports the following kinds of symbol declarations:

\begin{itemize}

\item Constant Declaration\nopagebreak

A constant declaration refers to a constant or storage object that is not supposed to change its value.
Since a constant declaration must not necessarily refer to an actually existing storage object, it is represented using its constant value.

\item Register Declaration\nopagebreak

A register declaration refers to a storage object which is stored in a register.
The register of the target hardware architecture is represented using its name.

\item Variable Declaration\nopagebreak

A variable declaration refers to a storage object that has an actual memory address.
The address refers either to a data section or to a displaced register that typically names the frame pointer.

\end{itemize}

\subsection{Type Declarations}\label{sec:dbgtypedeclarations}

A type declaration describes the layout and data representation of memory regions occupied by storage objects.
The \ecs{} supports the following kinds of type declarations:

\begin{itemize}

\item Void Type\alignright\syntax{"void"}\nopagebreak

A void type represents an unspecified, ambiguous, or nonexistent type of the programming language.

\item Type Name\alignright\syntax{<Name>}\nopagebreak

A type name refers to the specified type entry, see Section~\ref{sec:dbginformationentries}.

\item Signed Integer Type\alignright\syntax{"signed"}\nopagebreak

A signed integer type represents a storage object with a given size that stores a single signed integer value using two's complement as signed magnitude representation.

\item Unsigned Integer Type\alignright\syntax{"unsigned"}\nopagebreak

An unsigned integer type represents a storage object with a given size that stores a single unsigned integer value.

\item Floating-point Number Type\alignright\syntax{"float"}\nopagebreak

A floating-point number type represents a storage object with a given size that stores a single floating-point number value according to formats defined in the IEEE standard for floating-point arithmetic~\cite{ieee1985}.

\item Enumeration Type\alignright\syntax{"enumeration"}\nopagebreak

An enumeration type represents a storage object that stores a single integer value from a predefined set of named constants called \emph{enumerators}.
It contains the type of the underlying integer type and declarations for each enumerator.

\item Array Type\alignright\syntax{"array"}\nopagebreak

An array type represents a collection of consecutive storage objects with the same type called its elements.
It contains the index of its first element, the number of elements in case the array has a static size, and a declaration for the element type.

\item Record Type\alignright\syntax{"record"}\nopagebreak

A record type describes a user-defined data structure that contains an arbitrary number of storage objects called fields.
It contains an overall size of the data structure and declarations for each field.

\item Pointer Type\alignright\syntax{"pointer"}\nopagebreak

A pointer type represents a dereferencable storage object that stores a potentially invalid address of another storage object.
It contains the type of the referenced storage object.

\item Reference Type\alignright\syntax{"reference"}\nopagebreak

A reference type represents a dereferencable storage object that stores a valid address of another storage object.
It contains the type of the referenced storage object.

\item Function Type\alignright\syntax{"function"}\nopagebreak

A function type represents the type of a functional unit of the programming language.
It contains type declarations for the returned result and each function parameter.

\end{itemize}

\section{Debugging Information Format}

The debugging information generated by compiler tools is stored in plain text files according to the complete syntax specification given in Figure~\ref{fig:dbgfileformat}.
A debugging information file consists of a description of its target as well as an arbitrary number of sources and information entries according to the following syntax:

\begin{figure}
\centering\ifbook\small\fi\setlength{\grammarparsep}{0ex}
\begin{minipage}{34em}\begin{grammar}
<Information> = <Target> <Sources> <Entries>$\opt$ \par
<Target> = <Name> <Endianness> <Pointer> \par
<Name> = double-quoted-string \par
<Endianness> = "little" $\mid$ "big" \par
<Pointer> = <Size> \par
<Size> = decimal-integer \par
<Sources> = <Source> $\mid$ <Sources> <Source> \par
<Source> = double-quoted-string \par
<Entries> = <Entry> $\mid$ <Entries> <Entry> \par
<Entry> = "code" <Name> <Location> <Type> <Size> <Symbols>$\opt$ <Breakpoints>$\opt$ $\mid$ \\ "data" <Name> <Location> <Type> <Size> $\mid$ \\ "type" <Name> <Location>$\opt$ <Type> \par
<Location> = <Index> <Line> <Column> \par
<Index> = decimal-integer \par
<Line> = decimal-integer \par
<Column> = decimal-integer \par
<Type> = "void" $\mid$ <Name> $\mid$ "signed" <Size> $\mid$ "unsigned" <Size> $\mid$ "float" <Size> $\mid$ \\ "array" <Index> <Size> <Type> $\mid$ "record" <Size> <Fields>$\opt$ $\mid$ \\ "pointer" <Type> $\mid$ "reference" <Type> $\mid$ \\ "function" <Size> <Type> <Parameters>$\opt$ $\mid$ \\ "enumeration" <Type> <Enumerators> \par
<Fields> = <Field> $\mid$ <Fields> <Field> \par
<Field> = <Name> <Location> <Type> <Offset> <Bitmask> \par
<Offset> = decimal-integer \par
<Bitmask> = decimal-integer \par
<Parameters> = <Parameter> $\mid$ <Parameters> <Parameter> \par
<Parameter> = <Type> \par
<Enumerators> = <Enumerator> $\mid$ <Enumerators> <Enumerator> \par
<Enumerator> = <Name> <Location> <Value> \par
<Value> = signed-decimal-integer $\mid$ decimal-integer \par
<Symbols> = <Symbol> $\mid$ <Symbols> <Symbol> \par
<Symbol> = <Name> <Location> <Kind> <Type> <Lifetime> \par
<Kind> = <Constant> $\mid$ <Register> $\mid$ <Variable> \par
<Constant> = "signed" signed-decimal-integer $\mid$ \\ "unsigned" decimal-integer $\mid$ \\ "float" decimal-floating-point \par
<Register> = <Name> \par
<Variable> = <Register> <Displacement> \par
<Displacement> = signed-decimal-integer \par
<Lifetime> = <Begin> <End> \par
<Begin> = <Offset> \par
<End> = <Offset> \par
<Breakpoints> = <Breakpoint> $\mid$ <Breakpoints> <Breakpoint> \par
<Breakpoint> = <Offset> <Location> \par
\end{grammar}\end{minipage}
\caption{Syntax of the debugging information file format}
\label{fig:dbgfileformat}
\end{figure}

\begin{quote}\begin{grammar}
<Information> = <Target> <Sources> <Entries>$\opt$ \par
<Sources> = <Source> $\mid$ <Sources> <Source> \par
<Source> = double-quoted-string \par
\end{grammar}\end{quote}

The sequence of sources lists all filenames used during compilation and is referenced by index in source code locations.

\subsection{Target Description}

The description of the target architecture is represented in the debugging information file as text according to the following syntax:

\begin{quote}\begin{grammar}
<Target> = <Name> <Endianness> <Pointer> \par
<Name> = double-quoted-string \par
<Endianness> = "little" $\mid$ "big" \par
<Pointer> = <Size> \par
<Size> = decimal-integer \par
\end{grammar}\end{quote}

The description specifies the unique name of the target architecture as well as its endianness and pointer size expressed in octets.

\subsection{Information Entries}

Information entries are represented in the debugging information file as text according to the following syntax:

\begin{quote}\begin{grammar}
<Entries> = <Entry> $\mid$ <Entries> <Entry> \par
<Entry> = "code" <Name> <Location> <Type> <Size> <Symbols>$\opt$ <Breakpoints>$\opt$ $\mid$ \\ "data" <Name> <Location> <Type> <Size> $\mid$ \\ "type" <Name> <Location>$\opt$ <Type> \par
<Name> = double-quoted-string \par
<Size> = decimal-integer \par
\end{grammar}\end{quote}

The valid identifiers for the kind of the information entry correspond to the information entries described in Section~\ref{sec:dbginformationentries}.
The size of an entry is expressed in octets.

\subsection{Source Code Locations}

Source code locations associated with information entries, symbol declarations, field declarations, enumerator declarations, and breakpoints are represented in the debugging information file as text according to the following syntax:

\begin{quote}\begin{grammar}
<Location> = <Index> <Line> <Column> \par
<Index> = decimal-integer \par
<Line> = decimal-integer \par
<Column> = decimal-integer \par
\end{grammar}\end{quote}

The zero-based index refers to one of the sources listed at the beginning of the debugging information in order of occurrence.
Line and column numbering starts with one where all white-space characters are counted as single characters.

\subsection{Symbol Declarations}

Symbol declarations are represented in the debugging information file as text according to the following syntax:

\begin{quote}\begin{grammar}
<Symbols> = <Symbol> $\mid$ <Symbols> <Symbol> \par
<Symbol> = <Name> <Location> <Kind> <Type> <Lifetime> \par
<Name> = double-quoted-string \par
<Kind> = <Constant> $\mid$ <Register> $\mid$ <Variable> \par
<Constant> = "signed" signed-decimal-integer $\mid$ \\ "unsigned" decimal-integer $\mid$ \\ "float" decimal-floating-point \par
<Register> = <Name> \par
<Variable> = <Register> <Displacement> \par
<Displacement> = signed-decimal-integer \par
<Lifetime> = <Begin> <End> \par
<Begin> = <Offset> \par
<End> = <Offset> \par
<Offset> = decimal-integer \par
\end{grammar}\end{quote}

The valid kinds of symbols correspond to the symbol declarations described in Section~\ref{sec:dbgsymboldeclarations}.
The offsets of symbol lifetimes refer to machine code instructions relative to the beginning of the code section corresponding to the enclosing code entry and are expressed in octets.
A symbol declaration with an empty name conventionally denotes the result of the code section rather than a local variable or parameter.
A variable with a displacement of zero names a data section rather than a register.

\subsection{Type Declarations}

Type declarations are represented in the debugging information file as text according to the following syntax:

\begin{quote}\begin{grammar}
<Type> = "void" $\mid$ <Name> $\mid$ "signed" <Size> $\mid$ "unsigned" <Size> $\mid$ "float" <Size> $\mid$ \\ "array" <Index> <Size> <Type> $\mid$ "record" <Size> <Fields>$\opt$ $\mid$ \\ "pointer" <Type> $\mid$ "reference" <Type> $\mid$ \\ "function" <Size> <Type> <Parameters>$\opt$ $\mid$ \\ "enumeration" <Type> <Enumerators> \par
<Name> = double-quoted-string \par
<Size> = decimal-integer \par
<Index> = decimal-integer \par
<Parameters> = <Parameter> $\mid$ <Parameters> <Parameter> \par
<Parameter> = <Type> \par
\end{grammar}\end{quote}

The valid kinds of types correspond to the type declarations described in Section~\ref{sec:dbgtypedeclarations}.
The size of a type is expressed in octets except for array types where it denotes the number of array elements, and for function types where it denotes the number of parameters.

\subsection{Field Declarations}

Field declarations are represented in the debugging information file as text according to the following syntax:

\begin{quote}\begin{grammar}
<Fields> = <Field> $\mid$ <Fields> <Field> \par
<Field> = <Name> <Location> <Type> <Offset> <Bitmask> \par
<Name> = double-quoted-string \par
<Offset> = decimal-integer \par
<Bitmask> = decimal-integer \par
\end{grammar}\end{quote}

The offset specifies the address of the storage object relative to the beginning of the enclosing data structure and is expressed in octets.
The binary value of the non-zero bitmask specifies the number and consecutive sequence of bits occupied by the storage object in case a bit field is declared.
A field declaration with an empty name conventionally denotes inheritance rather than composition.

\subsection{Enumerator Declarations}

Enumerator declarations are represented in the debugging information file as text according to the following syntax:

\begin{quote}\begin{grammar}
<Enumerators> = <Enumerator> $\mid$ <Enumerators> <Enumerator> \par
<Enumerator> = <Name> <Location> <Value> \par
<Name> = double-quoted-string \par
<Value> = signed-decimal-integer $\mid$ decimal-integer \par
\end{grammar}\end{quote}

The constant value of an enumerator is an optionally signed integer that belongs to the underlying integer type of the corresponding enumeration.

\subsection{Breakpoints}

Breakpoints are represented in the debugging information file as text according to the following syntax:

\begin{quote}\begin{grammar}
<Breakpoints> = <Breakpoint> $\mid$ <Breakpoints> <Breakpoint> \par
<Breakpoint> = <Offset> <Location> \par
<Offset> = decimal-integer \par
\end{grammar}\end{quote}

The offset refers to a machine code instruction relative to the beginning of the code section corresponding to the enclosing code entry and is expressed in octets.

\section{Converter Tools}

Converters process debugging information files that were previously generated by the various compiler tools of the \ecs{}.
They store debugging information in object files that represent the same information in a binary debugging data format suitable for the debugger of the target runtime environment.
\interface

\dbgdwarf

\concludechapter

// Generic documentation definitions
// Copyright (C) Florian Negele

// This file is part of the Eigen Compiler Suite.

// The ECS is free software: you can redistribute it and/or modify
// it under the terms of the GNU General Public License as published by
// the Free Software Foundation, either version 3 of the License, or
// (at your option) any later version.

// The ECS is distributed in the hope that it will be useful,
// but WITHOUT ANY WARRANTY; without even the implied warranty of
// MERCHANTABILITY or FITNESS FOR A PARTICULAR PURPOSE.  See the
// GNU General Public License for more details.

// You should have received a copy of the GNU General Public License
// along with the ECS.  If not, see <https://www.gnu.org/licenses/>.

#ifndef SYMBOL
	#define SYMBOL(symbol, name)
#endif

// markers
SYMBOL (Article,  "@")
SYMBOL (Heading,  "=")
SYMBOL (Number,   "#")
SYMBOL (Bullet,   "*")

// separators
SYMBOL (Line,       "----")
SYMBOL (Header,     "|=")
SYMBOL (CodeBegin,  "{{{")
SYMBOL (CodeEnd,    "}}}")
SYMBOL (LineBreak,  "\\\\")
SYMBOL (Space,      "space")
SYMBOL (Tab,        "tabulator")
SYMBOL (Newline,    "new-line character")

// formats
SYMBOL (Italic,      "//")
SYMBOL (Bold,        "**")
SYMBOL (LinkBegin,   "[[")
SYMBOL (LinkEnd,     "]]")
SYMBOL (LabelBegin,  "<<")
SYMBOL (LabelEnd,    ">>")
SYMBOL (Pipe,        "|")

// literals
SYMBOL (String,  "string")
SYMBOL (Code,    "code")
SYMBOL (URL,     "URL")

#undef SYMBOL

% Extensions to the Eigen Compiler Suite
% Copyright (C) Florian Negele

% This file is part of the Eigen Compiler Suite.

% Permission is granted to copy, distribute and/or modify this document
% under the terms of the GNU Free Documentation License, Version 1.3
% or any later version published by the Free Software Foundation.

% You should have received a copy of the GNU Free Documentation License
% along with the ECS.  If not, see <https://www.gnu.org/licenses/>.

% Generic documentation utilities
% Copyright (C) Florian Negele

% This file is part of the Eigen Compiler Suite.

% Permission is granted to copy, distribute and/or modify this document
% under the terms of the GNU Free Documentation License, Version 1.3
% or any later version published by the Free Software Foundation.

% You should have received a copy of the GNU Free Documentation License
% along with the ECS.  If not, see <https://www.gnu.org/licenses/>.

\providecommand{\cpp}{C\texttt{++}}
\providecommand{\opt}{_\mathit{opt}}
\providecommand{\tool}[1]{\texttt{#1}}
\providecommand{\version}{Version 0.0.40}
\providecommand{\resource}[1]{*++\txt{#1}}
\providecommand{\ecs}{Eigen Compiler Suite}
\providecommand{\changed}[1]{\underline{#1}}
\providecommand{\toolbox}[1]{\converter{#1}}
\providecommand{\file}{}\renewcommand{\file}[1]{\texttt{#1}}
\providecommand{\alignright}{\hfill\linebreak[0]\hspace*{\fill}}
\providecommand{\converter}[1]{*++[F][F*:white][F,:gray]\txt{#1}}
\providecommand{\documentation}{\ifbook chapter\else document\fi}
\providecommand{\Documentation}{\ifbook Chapter\else Document\fi}
\providecommand{\variable}[1]{\resource{\texttt{\small#1}\\variable}}
\providecommand{\documentationref}[2]{\ifbook\ref{#1}\else``\href{#1}{#2}''~\cite{#1}\fi}
\providecommand{\objfile}[1]{\texttt{#1}\index[runtime]{#1 object file@\texttt{#1} object file}}
\providecommand{\libfile}[1]{\texttt{#1}\index[runtime]{#1 library file@\texttt{#1} library file}}
\providecommand{\epigraph}[2]{\ifbook\begin{quote}\flushright\textit{#1}\par--- #2\end{quote}\fi}
\providecommand{\environmentvariable}[1]{\texttt{#1}\index{Environment variables!#1@\texttt{#1}}}
\providecommand{\environment}[1]{\texttt{#1}\index[environment]{#1 environment@\texttt{#1} environment}}
\providecommand{\toolsection}{}\renewcommand{\toolsection}[1]{\subsection{#1}\label{\prefix:#1}\tool{#1}}
\providecommand{\instruction}{}\renewcommand{\instruction}[2]{\noindent\qquad\pdftooltip{\texttt{#1}}{#2}\refstepcounter{instruction}\par}
\providecommand{\flowgraph}{}\renewcommand{\flowgraph}[1]{\par\sffamily\begin{displaymath}\xymatrix@=4ex{#1}\end{displaymath}\normalfont\par}
\providecommand{\instructionset}{}\renewcommand{\instructionset}[4]{\setcounter{instruction}{0}\begin{multicols}{\ifbook#3\else#4\fi}[{\captionof{table}[#2]{#2 (\ref*{#1:instructions}~instructions)}\label{tab:#1set}\vspace{-2ex}}]\footnotesize\raggedcolumns\input{#1.set}\label{#1:instructions}\end{multicols}}

\providecommand{\gpl}{GNU General Public License}
\providecommand{\rse}{ECS Runtime Support Exception}
\providecommand{\fdl}{\href{https://www.gnu.org/licenses/fdl.html}{GNU Free Documentation License}}

\providecommand{\docbegin}{}
\providecommand{\docend}{}
\providecommand{\doclabel}[1]{\hypertarget{#1}}
\providecommand{\doclink}[2]{\hyperlink{#1}{#2}}
\providecommand{\docsection}[3]{\hypertarget{#1}{\subsection{#2}}\label{sec:#1}\index[library]{#2@#3}}
\providecommand{\docsectionstar}[1]{}
\providecommand{\docsubbegin}{\begin{description}}
\providecommand{\docsubend}{\end{description}}
\providecommand{\docsubsection}[3]{\item[\hypertarget{#1}{#2}]\index[library]{#2@#3}}
\providecommand{\docsubsectionstar}[1]{\smallskip}
\providecommand{\docsubsubsection}[3]{\docsubsection{#1}{#2}{#3}}
\providecommand{\docsubsubsectionstar}[1]{}
\providecommand{\docsubsubsubsection}[3]{}
\providecommand{\docsubsubsubsectionstar}[1]{}
\providecommand{\doctable}{}

\providecommand{\debuggingtool}{}\renewcommand{\debuggingtool}{This tool is provided for debugging purposes.
It allows exposing and modifying an internal data structure that is usually not accessible.
}

\providecommand{\interface}{All tools accept command-line arguments which are taken as names of plain text files containing the source code.
If no arguments are provided, the standard input stream is used instead.
Output files are generated in the current working directory and have the same name as the input file being processed whereas the filename extension gets replaced by an appropriate suffix.
\seeinterface
}

\providecommand{\license}{\noindent Copyright \copyright{} Florian Negele\par\medskip\noindent
Permission is granted to copy, distribute and/or modify this document under the terms of the
\fdl{}, Version 1.3 or any later version published by the \href{https://fsf.org/}{Free Software Foundation}.
}

\providecommand{\ecslogosurface}{
\fill[darkgray] (0,0,0) -- (0,0,3) -- (0,3,3) -- (0,3,1) -- (0,4,1) -- (0,4,3) -- (0,5,3) -- (0,5,0) -- (0,2,0) -- (0,2,2) -- (0,1,2) -- (0,1,0) -- cycle;
\fill[gray] (0,5,0) -- (0,5,3) -- (1,5,3) -- (1,5,1) -- (2,5,1) -- (2,5,3) -- (3,5,3) -- (3,5,0) -- cycle;
\fill[lightgray] (0,0,0) -- (0,1,0) -- (2,1,0) -- (2,4,0) -- (1,4,0) -- (1,3,0) -- (2,3,0) -- (2,2,0) -- (0,2,0) -- (0,5,0) -- (3,5,0) -- (3,0,0) -- cycle;
\begin{scope}[line width=0.5]
\begin{scope}[gray]
\draw (0,0,0) -- (0,1,0);
\draw (2,1,0) -- (2,2,0);
\draw (0,1,2) -- (0,2,2);
\draw (0,2,0) -- (0,5,0);
\draw (2,3,0) -- (2,4,0);
\end{scope}
\begin{scope}[lightgray]
\draw (0,1,0) -- (0,1,2);
\draw (0,3,1) -- (0,3,3);
\draw (0,5,0) -- (0,5,3);
\draw (2,5,1) -- (2,5,3);
\end{scope}
\begin{scope}[white]
\draw (0,1,0) -- (2,1,0);
\draw (1,3,0) -- (2,3,0);
\draw (0,5,0) -- (3,5,0);
\end{scope}
\end{scope}
}

\providecommand{\ecslogo}[1]{
\begin{tikzpicture}[scale={(#1)/((sin(45)+cos(45))*3cm)},x={({-cos(45)*1cm},{sin(45)*sin(30)*1cm})},y={({0cm},{(cos(30)*1cm})},z={({sin(45)*1cm},{cos(45)*sin(30)*1cm})}]
\begin{scope}[darkgray,line width=1]
\draw (0,0,0) -- (0,0,3) -- (0,3,3) -- (2,3,3) -- (2,5,3) -- (3,5,3) -- (3,5,0) -- (3,0,0) -- cycle;
\draw (0,3,1) -- (0,4,1) -- (0,4,3) -- (0,5,3) -- (1,5,3) -- (1,5,1) -- (2,5,1);
\draw (1,3,0) -- (1,4,0) -- (2,4,0);
\end{scope}
\fill[darkgray] (2,0,0) -- (2,0,3) -- (2,5,3) -- (2,5,1) -- (2,4,1) -- (2,4,0) -- cycle;
\fill[lightgray] (2,0,2) -- (0,0,2) -- (0,2,2) -- (2,2,2) -- cycle;
\fill[gray] (0,1,0) -- (2,1,0) -- (2,1,2) -- (0,1,2) -- cycle;
\fill[gray] (0,3,1) -- (0,3,3) -- (2,3,3) -- (2,3,0) -- (1,3,0) -- (1,3,1) -- cycle;
\ecslogosurface
\end{tikzpicture}
}

\providecommand{\shadowedecslogo}[3]{
\begin{tikzpicture}[scale={(#1)/((sin(#2)+cos(#2))*3cm)},x={({-cos(#2)*1cm},{sin(#2)*sin(#3)*1cm})},y={({0cm},{(cos(#3)*1cm})},z={({sin(#2)*1cm},{cos(#2)*sin(#3)*1cm})}]
\shade[top color=lightgray!50!white,bottom color=white,middle color=lightgray!50!white] (0,0,0) -- (3,0,0) -- (3,{-0.5-3*sin(#2)*sin(#3)/cos(#3)},0) -- (0,-0.5,0) -- cycle;
\shade[top color=darkgray!50!gray,bottom color=white,middle color=darkgray!50!white] (0,0,0) -- (0,0,3) -- (0,{-0.5-3*cos(#2)*sin(#3)/cos(#3)},3) -- (0,-0.5,0) -- cycle;
\begin{scope}[y={({(cos(#2)+sin(#2))*0.5cm},{(cos(#2)*sin(#3)-sin(#2)*sin(#3))*0.5cm})}]
\useasboundingbox (3,0,0) -- (0,0,0) -- (0,0,3);
\shade[left color=darkgray!80!black,right color=lightgray,middle color=gray] (0,0,0) -- (0,1,0) -- (0,1,0.5) -- (0,2,0) -- (0,5,0) -- (0,5,3) -- (1,5,3) -- (1,4,3) -- (1,4,2.5) -- (1,3,3) -- (2,5,3) -- (3,5,3) -- (3,0,3) -- cycle;
\clip (0,0,0) -- (0,0,3) -- ({-3*sin(#2)/cos(#2)},0,0) -- cycle;
\shade[left color=darkgray,right color=lightgray!50!gray] (0,0,0) -- (0,1,0) -- (0,1,0.5) -- (0,2,0) -- (0,5,0) -- (0,5,3) -- (1,5,3) -- (1,4,3) -- (1,4,2.5) -- (1,3,3) -- (2,5,3) -- (3,5,3) -- (3,0,3) -- cycle;
\end{scope}
\shade[left color=darkgray,right color=darkgray!80!black] (2,0,0) -- (2,0,3) -- (2,5,3) -- (2,5,1) -- (2,4,1) -- (2,4,0) -- cycle;
\shade[left color=darkgray!90!black,right color=gray!80!darkgray] (2,0,2) -- (0,0,2) -- (0,2,2) -- (2,2,2) -- cycle;
\shade[top color=darkgray!90!black,bottom color=gray!80!darkgray] (0,1,0) -- (2,1,0) -- (2,1,2) -- (0,1,2) -- cycle;
\shade[top color=darkgray!90!black,bottom color=gray!80!darkgray] (0,3,1) -- (0,3,3) -- (2,3,3) -- (2,3,0) -- (1,3,0) -- (1,3,1) -- cycle;
\fill[gray] (2,1,0) -- (1.5,1,0.5) -- (0,1,0.5) -- (0,1,0) -- cycle;
\fill[gray] (1,3,2) -- (0.5,3,2) -- (0.5,3,3) -- (1,3,3) -- cycle;
\fill[gray] (2,3,0) -- (1.5,3,0.5) -- (1,3,0.5) -- (1,3,0) -- cycle;
\ecslogosurface
\end{tikzpicture}
}

\providecommand{\cpplogo}[1]{
\begin{tikzpicture}[scale=(#1)/512em]
\fill[gray] (435.2794,398.7159) -- (247.1911,507.3075) .. controls (236.3563,513.5642) and (218.6240,513.5642) .. (207.7892,507.3075) -- (19.7009,398.7159) .. controls (8.8646,392.4606) and (0.0000,377.1043) .. (0.0000,364.5924) -- (0.0000,147.4076) .. controls (0.8430,132.8363) and (8.2856,120.7683) .. (19.7009,113.2842) -- (207.7892,4.6926) .. controls (218.6240,-1.5642) and (236.3564,-1.5642) .. (247.1911,4.6926) -- (435.2794,113.2842) .. controls (447.5273,121.4304) and (454.4987,133.6918) .. (454.9803,147.4076) -- (454.9803,364.5924) .. controls (454.5404,377.7571) and (446.6566,391.0351) .. (435.2794,398.7159) -- cycle(75.8301,255.9993) .. controls (74.9389,404.0881) and (273.2892,469.4783) .. (358.8263,331.8769) -- (293.1917,293.8965) .. controls (253.5702,359.4301) and (155.1909,335.9977) .. (151.6601,255.9993) .. controls (152.7204,182.2703) and (249.4137,148.0211) .. (293.1961,218.1065) -- (358.8308,180.1276) .. controls (283.4477,49.2645) and (79.6318,96.3470) .. (75.8301,255.9993) -- cycle(379.1503,247.5747) -- (362.2982,247.5747) -- (362.2982,230.7226) -- (345.4490,230.7226) -- (345.4490,247.5747) -- (328.5969,247.5747) -- (328.5969,264.4254) -- (345.4490,264.4254) -- (345.4490,281.2759) -- (362.2982,281.2759) -- (362.2982,264.4254) -- (379.1503,264.4254) -- cycle(442.3420,247.5747) -- (425.4899,247.5747) -- (425.4899,230.7226) -- (408.6408,230.7226) -- (408.6408,247.5747) -- (391.7886,247.5747) -- (391.7886,264.4254) -- (408.6408,264.4254) -- (408.6408,281.2759) -- (425.4899,281.2759) -- (425.4899,264.4254) -- (442.3420,264.4254) -- cycle;
\end{tikzpicture}
}

\providecommand{\fallogo}[1]{
\begin{tikzpicture}[scale=(#1)/512em]
\fill[gray] (185.7774,0.0000) .. controls (200.4486,15.9798) and (226.8966,8.7148) .. (235.0426,31.5836) .. controls (249.5297,58.0598) and (247.9581,97.9161) .. (280.3335,110.9762) .. controls (309.1690,120.3496) and (337.8406,104.2727) .. (366.5753,103.9379) .. controls (373.4449,111.5171) and (379.2885,128.2574) .. (383.9755,108.9744) .. controls (396.6979,102.5615) and (437.2808,107.6681) .. (426.9652,124.3252) .. controls (408.9822,121.0785) and (412.4742,146.0729) .. (426.5192,131.4996) .. controls (433.8413,120.8489) and (465.1541,126.5522) .. (441.9067,135.7950) .. controls (396.1879,157.7478) and (344.1112,161.5079) .. (298.5528,183.5702) .. controls (277.7471,193.5198) and (284.6941,218.7163) .. (285.2127,236.9640) .. controls (292.3599,316.2826) and (307.3929,394.6311) .. (317.1198,473.6154) .. controls (329.0637,505.4736) and (292.1195,528.5004) .. (265.9183,511.2761) .. controls (237.9284,499.2462) and (237.3684,465.2681) .. (230.9102,439.9421) .. controls (218.6692,374.3397) and (215.6307,306.9662) .. (198.1732,242.3977) .. controls (183.1379,232.7444) and (164.4245,256.0298) .. (149.0430,261.4799) .. controls (116.9328,279.2585) and (87.1822,308.5851) .. (48.2293,307.8914) .. controls (21.3220,306.9037) and (-15.9107,281.8761) .. (7.2921,252.7908) .. controls (29.7799,220.6177) and (67.5177,204.3028) .. (100.9287,185.9449) .. controls (130.8217,170.8906) and (161.1548,156.5903) .. (191.0278,141.5847) .. controls (196.1738,120.0520) and (186.6049,95.2409) .. (186.8382,72.4353) .. controls (185.5234,48.4204) and (183.1700,23.9341) .. (185.7774,0.0000) -- cycle;
\end{tikzpicture}
}

\providecommand{\oblogo}[1]{
\begin{tikzpicture}[scale=(#1)/512em]
\fill[gray] (160.3865,208.9117) .. controls (154.0879,214.6478) and (149.0735,221.2409) .. (145.4125,228.5384) .. controls (184.8790,248.4273) and (234.7122,269.8787) .. (297.5493,291.8782) .. controls (300.3943,281.4769) and (300.9552,268.7619) .. (300.4023,255.2389) .. controls (248.9909,244.7891) and (200.0310,225.9279) .. (160.3865,208.9117) -- cycle(225.7398,392.6996) .. controls (308.0209,392.1716) and (359.3326,345.9277) .. (368.7203,285.2098) .. controls (376.6742,197.1784) and (311.7194,141.3342) .. (205.4287,142.1456) .. controls (139.9485,141.4804) and (88.7155,166.1957) .. (73.5775,228.0086) .. controls (52.0297,320.3408) and (123.4078,391.0103) .. (225.7398,392.6996) -- cycle(216.0739,176.4733) .. controls (268.9183,179.2424) and (315.8292,206.5488) .. (312.7454,265.1139) .. controls (313.2769,315.6384) and (286.5993,353.4946) .. (216.6040,355.7934) .. controls (162.4657,355.7934) and (126.0914,317.5023) .. (126.0914,260.5103) .. controls (126.1733,214.2900) and (163.3363,176.2849) .. (216.0739,176.4733) -- cycle(76.4897,189.1754) .. controls (13.1586,147.5631) and (0.0000,119.4207) .. (0.0000,119.4207) -- (90.6499,170.1632) .. controls (85.3004,175.8497) and (80.5994,182.1633) .. (76.4897,189.1754) -- cycle(353.9486,119.3004) -- (402.9482,119.3004) .. controls (427.0025,137.0797) and (450.9893,162.7034) .. (474.9529,191.0213) .. controls (509.3540,228.5339) and (531.3391,294.2091) .. (487.8149,312.1206) .. controls (462.8165,324.7652) and (394.3874,316.8943) .. (373.8912,313.6651) .. controls (379.9291,297.7449) and (383.2899,278.4204) .. (381.4989,257.7214) .. controls (420.3069,248.0321) and (421.9610,218.3461) .. (407.7867,192.6417) .. controls (391.1113,162.4018) and (370.1114,132.9097) .. (353.9486,119.3004) -- cycle;
\end{tikzpicture}
}

\providecommand{\markuptable}{
\begin{table}
\sffamily\centering
\begin{tabular}{@{}lcl@{}}
\toprule
\texttt{//italics//} & $\rightarrow$ & \textit{italics} \\
\midrule
\texttt{**bold**} & $\rightarrow$ & \textbf{bold} \\
\midrule
\texttt{\# ordered list} & & 1 ordered list \\
\texttt{\# second item} & $\rightarrow$ & 2 second item \\
\texttt{\#\# sub item} & & \hspace{1em} 1 sub item \\
\midrule
\texttt{* unordered list} & & $\bullet$ unordered list \\
\texttt{* second item} & $\rightarrow$ & $\bullet$ second item \\
\texttt{** sub item} & & \hspace{1em} $\bullet$ sub item \\
\midrule
\texttt{link to [[label]]} & $\rightarrow$ & link to \underline{label} \\
\midrule
\texttt{<{}<label>{}> definition } & $\rightarrow$ & definition \\
\midrule
\texttt{[[url|link name]]} & $\rightarrow$ & \underline{link name} \\
\midrule\addlinespace
\texttt{= large heading} & & {\Large large heading} \smallskip \\
\texttt{== medium heading} & $\rightarrow$ & {\large medium heading} \\
\texttt{=== small heading} & & small heading \\
\midrule
\texttt{no line break} & & no line break for paragraphs \\
\texttt{for paragraphs} & $\rightarrow$ \\
& & use empty line \\
\texttt{use empty line} \\
\midrule
\texttt{force\textbackslash\textbackslash line break} & $\rightarrow$ & force \\
& & line break \\
\midrule
\texttt{horizontal line} & $\rightarrow$ & horizontal line \\
\texttt{----} & & \hrulefill \\
\midrule
\texttt{|=a|=table|=header} & & \underline{a \enspace table \enspace header} \\
\texttt{|a|table|row} & $\rightarrow$ & a \enspace table \enspace row \\
\texttt{|b|table|row} & & b \enspace table \enspace row \\
\midrule
\texttt{\{\{\{} \\
\texttt{unformatted} & $\rightarrow$ & \texttt{unformatted} \\
\texttt{code} & & \texttt{code} \\
\texttt{\}\}\}} \\
\midrule\addlinespace
\texttt{@ new article} & & {\Large 1.\ new article} \smallskip \\
\texttt{@ second article} & $\rightarrow$ & {\Large 2.\ second article} \smallskip \\
\texttt{@@ sub article} & & {\large 2.1.\ sub article} \\
\bottomrule
\end{tabular}
\normalfont\caption{Elements of the generic documentation markup language}
\label{tab:docmarkup}
\end{table}
}

\providecommand{\startchapter}[4]{
\documentclass[11pt,a4paper]{article}
\usepackage{booktabs}
\usepackage[format=hang,labelfont=bf]{caption}
\usepackage{changepage}
\usepackage[T1]{fontenc}
\usepackage[margin=2cm]{geometry}
\usepackage{hyperref}
\usepackage[american]{isodate}
\usepackage{lmodern}
\usepackage{longtable}
\usepackage{mathptmx}
\usepackage{microtype}
\usepackage[toc]{multitoc}
\usepackage{multirow}
\usepackage[all]{nowidow}
\usepackage{pdfcomment}
\usepackage{syntax}
\usepackage{tikz}
\usepackage[all]{xy}
\hypersetup{pdfborder={0 0 0},bookmarksnumbered=true,pdftitle={\ecs{}: #2},pdfauthor={Florian Negele},pdfsubject={\ecs{}},pdfkeywords={#1}}
\setlength{\grammarindent}{8em}\setlength{\grammarparsep}{0.2ex}
\setlength{\columnsep}{2em}
\newcommand{\prefix}{}
\newcounter{instruction}
\bibliographystyle{unsrt}
\renewcommand{\index}[2][]{}
\renewcommand{\arraystretch}{1.05}
\renewcommand{\floatpagefraction}{0.7}
\renewcommand{\syntleft}{\itshape}\renewcommand{\syntright}{}
\title{\vspace{-5ex}\Huge{\ecs{}}\medskip\hrule}
\author{\huge{#2}}
\date{\medskip\version}
\newif\ifbook\bookfalse
\pagestyle{headings}
\frenchspacing
\begin{document}
\maketitle\thispagestyle{empty}\noindent#4\setlength{\columnseprule}{0.4pt}\tableofcontents\setlength{\columnseprule}{0pt}\vfill\pagebreak[3]\null\vfill\bigskip\noindent
\parbox{\textwidth-4em}{\license The contents of this \documentation{} are part of the \href{manual}{\ecs{} User Manual}~\cite{manual} and correspond to Chapter ``\href{manual\##3}{#1}''.\alignright\mbox{\today}}
\parbox{4em}{\flushright\ecslogo{3em}}
\clearpage
}

\providecommand{\concludechapter}{
\vfill\pagebreak[3]\null\vfill
\thispagestyle{myheadings}\markright{REFERENCES}
\noindent\begin{minipage}{\textwidth}\begin{multicols}{2}[\section*{References}]
\renewcommand{\section}[2]{}\small\bibliography{references}
\end{multicols}\end{minipage}\end{document}
}

\providecommand{\startpresentation}[2]{
\documentclass[14pt,aspectratio=43,usepdftitle=false]{beamer}
\usepackage{booktabs}
\usepackage{etex}
\usepackage{multicol}
\usepackage{tikz}
\usepackage[all]{xy}
\bibliographystyle{unsrt}
\setlength{\columnsep}{1em}
\setlength{\leftmargini}{1em}
\setbeamercolor{title}{fg=black}
\setbeamercolor{structure}{fg=darkgray}
\setbeamercolor{bibliography item}{fg=darkgray}
\setbeamerfont{title}{series=\bfseries}
\setbeamerfont{subtitle}{series=\normalfont}
\setbeamerfont*{frametitle}{parent=title}
\setbeamerfont{block title}{series=\bfseries}
\setbeamerfont*{framesubtitle}{parent=subtitle}
\setbeamersize{text margin left=1em,text margin right=1em}
\setbeamertemplate{navigation symbols}{}
\setbeamertemplate{itemize item}[circle]{}
\setbeamertemplate{bibliography item}[triangle]{}
\setbeamertemplate{bibliography entry author}{\usebeamercolor[fg]{bibliography item}}
\setbeamertemplate{frametitle}{\medskip\usebeamerfont{frametitle}\color{gray}\raisebox{-2.5ex}[0ex][0ex]{\rule{0.1em}{4.5ex}}}
\addtobeamertemplate{frametitle}{}{\hspace{0.4em}\usebeamercolor[fg]{title}\insertframetitle\par\vspace{0.2ex}\hspace{0.5em}\usebeamerfont{framesubtitle}\insertframesubtitle}
\hypersetup{pdfborder={0 0 0},bookmarksnumbered=true,bookmarksopen=true,bookmarksopenlevel=0,pdftitle={\ecs{}: #1},pdfauthor={Florian Negele},pdfsubject={\ecs{}},pdfkeywords={#1}}
\renewcommand{\flowgraph}[1]{\resizebox{\textwidth}{!}{$$\xymatrix{##1}$$}}
\title{\ecs{}\medskip\hrule\medskip}
\institute{\shadowedecslogo{5em}{30}{15}}
\date{\version}
\subtitle{#1}
\begin{document}
\begin{frame}[plain]\titlepage\nocite{manual}\end{frame}
\begin{frame}{Contents}{#1}\begin{center}\tableofcontents\end{center}\end{frame}
}

\providecommand{\concludepresentation}{
\begin{frame}{References}\begin{footnotesize}\setlength{\columnseprule}{0.4pt}\begin{multicols}{2}\bibliography{references}\end{multicols}\end{footnotesize}\end{frame}
\end{document}
}

\providecommand{\startbook}[1]{
\documentclass[10pt,paper=17cm:24cm,DIV=13,twoside=semi,headings=normal,numbers=noendperiod,cleardoublepage=plain]{scrbook}
\usepackage{atveryend}
\usepackage{booktabs}
\usepackage{caption}
\usepackage{changepage}
\usepackage[T1]{fontenc}
\usepackage{imakeidx}
\usepackage{hyperref}
\usepackage[american]{isodate}
\usepackage{lmodern}
\usepackage{longtable}
\usepackage{mathptmx}
\usepackage[final]{microtype}
\usepackage{multicol}
\usepackage{multirow}
\usepackage[all]{nowidow}
\usepackage{pdfcomment}
\usepackage{scrlayer-scrpage}
\usepackage{setspace}
\usepackage{syntax}
\usepackage[eventxtindent=4pt,oddtxtexdent=4pt]{thumbs}
\usepackage{tikz}
\usepackage[all]{xy}
\hyphenation{Micro-Blaze Open-Cores Open-RISC Power-PC}
\hypersetup{pdfborder={0 0 0},bookmarksnumbered=true,bookmarksopen=true,bookmarksopenlevel=0,pdftitle={\ecs{}: #1},pdfauthor={Florian Negele},pdfsubject={\ecs{}},pdfkeywords={#1}}
\setlength{\grammarindent}{8em}\setlength{\grammarparsep}{0.7ex}
\setkomafont{captionlabel}{\usekomafont{descriptionlabel}}
\renewcommand{\arraystretch}{1.05}\setstretch{1.1}
\renewcommand{\chapterformat}{\thechapter\autodot\enskip\raisebox{-1ex}[0ex][0ex]{\color{gray}\rule{0.1em}{3.5ex}}\enskip}
\renewcommand{\startchapter}[4]{\hypertarget{##3}{\chapter{##1}}\label{##3}##4\addthumb{##1}{\LARGE\sffamily\bfseries\thechapter}{white}{gray}\renewcommand{\prefix}{##3}}
\renewcommand{\concludechapter}{\clearpage{\stopthumb\cleardoublepage}}
\renewcommand{\syntleft}{\itshape}\renewcommand{\syntright}{}
\renewcommand{\floatpagefraction}{0.7}
\renewcommand{\partheademptypage}{}
\DeclareMicrotypeAlias{lmss}{cmr}
\newcommand{\prefix}{}
\newcounter{instruction}
\bibliographystyle{unsrt}
\newif\ifbook\booktrue
\makeindex[intoc,title=Index]
\makeindex[intoc,name=tools,title=Index of Tools,columns=3]
\makeindex[intoc,name=library,title=Index of Library Names]
\makeindex[intoc,name=runtime,title=Index of Runtime Support]
\makeindex[intoc,name=environment,title=Index of Target Environments]
\indexsetup{toclevel=chapter,headers={\indexname}{\indexname}}
\frenchspacing
\begin{document}
\pagenumbering{alph}
\begin{titlepage}\centering
\huge\sffamily\null\vfill\textbf{\ecs{}}\bigskip\hrule\bigskip#1
\normalsize\normalfont\vfill\vfill\shadowedecslogo{10em}{30}{15}
\large\vfill\vfill\version
\end{titlepage}
\null\vfill
\thispagestyle{empty}
\noindent\today\par\medskip
\license A copy of this license is included in Appendix~\ref{fdl} on page~\pageref{fdl}.
All product names used herein are for identification purposes only and may be trademarks of their respective companies.
\concludechapter
\frontmatter
\setcounter{tocdepth}{1}
\tableofcontents
\setcounter{tocdepth}{2}
\concludechapter
\listoffigures
\concludechapter
\listoftables
\concludechapter
}

\providecommand{\concludebook}{
\backmatter
\addtocontents{toc}{\protect\setcounter{tocdepth}{-1}}
\phantomsection\addcontentsline{toc}{part}{Bibliography}
\bibliography{references}
\concludechapter
\phantomsection\addcontentsline{toc}{part}{Indexes}
\printindex
\concludechapter
\indexprologue{\label{idx:tools}}
\printindex[tools]
\concludechapter
\printindex[library]
\concludechapter
\indexprologue{\label{idx:runtime}}
\printindex[runtime]
\concludechapter
\indexprologue{\label{idx:environment}}
\printindex[environment]
\concludechapter
\pagestyle{empty}\pagenumbering{Alph}\null\clearpage
\null\vfill\centering\ecslogo{4em}\par\medskip\license
\end{document}
}

% chapter references

\providecommand{\seedocumentationref}{}\renewcommand{\seedocumentationref}[3]{#1, see \Documentation{}~\documentationref{#2}{#3}. }
\providecommand{\seeinterface}{}\renewcommand{\seeinterface}{\ifbook See \Documentation{}~\documentationref{interface}{User Interface} for more information about the common user interface of all of these tools. \fi}
\providecommand{\seeguide}{}\renewcommand{\seeguide}{\seedocumentationref{For basic examples of using some of these tools in practice}{guide}{User Guide}}
\providecommand{\seecpp}{}\renewcommand{\seecpp}{\seedocumentationref{For more information about the \cpp{} programming language and its implementation by the \ecs{}}{cpp}{User Manual for \cpp{}}}
\providecommand{\seefalse}{}\renewcommand{\seefalse}{\seedocumentationref{For more information about the FALSE programming language and its implementation by the \ecs{}}{false}{User Manual for FALSE}}
\providecommand{\seeoberon}{}\renewcommand{\seeoberon}{\seedocumentationref{For more information about the Oberon programming language and its implementation by the \ecs{}}{oberon}{User Manual for Oberon}}
\providecommand{\seeassembly}{}\renewcommand{\seeassembly}{\seedocumentationref{For more information about the generic assembly language and how to use it}{assembly}{Generic Assembly Language Specification}}
\providecommand{\seeamd}{}\renewcommand{\seeamd}{\seedocumentationref{For more information about how the \ecs{} supports the AMD64 hardware architecture}{amd64}{AMD64 Hardware Architecture Support}}
\providecommand{\seearm}{}\renewcommand{\seearm}{\seedocumentationref{For more information about how the \ecs{} supports the ARM hardware architecture}{arm}{ARM Hardware Architecture Support}}
\providecommand{\seeavr}{}\renewcommand{\seeavr}{\seedocumentationref{For more information about how the \ecs{} supports the AVR hardware architecture}{avr}{AVR Hardware Architecture Support}}
\providecommand{\seeavrtt}{}\renewcommand{\seeavrtt}{\seedocumentationref{For more information about how the \ecs{} supports the AVR32 hardware architecture}{avr32}{AVR32 Hardware Architecture Support}}
\providecommand{\seemabk}{}\renewcommand{\seemabk}{\seedocumentationref{For more information about how the \ecs{} supports the M68000 hardware architecture}{m68k}{M68000 Hardware Architecture Support}}
\providecommand{\seemibl}{}\renewcommand{\seemibl}{\seedocumentationref{For more information about how the \ecs{} supports the MicroBlaze hardware architecture}{mibl}{MicroBlaze Hardware Architecture Support}}
\providecommand{\seemips}{}\renewcommand{\seemips}{\seedocumentationref{For more information about how the \ecs{} supports the MIPS32 and MIPS64 hardware architectures}{mips}{MIPS Hardware Architecture Support}}
\providecommand{\seemmix}{}\renewcommand{\seemmix}{\seedocumentationref{For more information about how the \ecs{} supports the MMIX hardware architecture}{mmix}{MMIX Hardware Architecture Support}}
\providecommand{\seeorok}{}\renewcommand{\seeorok}{\seedocumentationref{For more information about how the \ecs{} supports the OpenRISC 1000 hardware architecture}{or1k}{OpenRISC 1000 Hardware Architecture Support}}
\providecommand{\seeppc}{}\renewcommand{\seeppc}{\seedocumentationref{For more information about how the \ecs{} supports the PowerPC hardware architecture}{ppc}{PowerPC Hardware Architecture Support}}
\providecommand{\seerisc}{}\renewcommand{\seerisc}{\seedocumentationref{For more information about how the \ecs{} supports the RISC hardware architecture}{risc}{RISC Hardware Architecture Support}}
\providecommand{\seewasm}{}\renewcommand{\seewasm}{\seedocumentationref{For more information about how the \ecs{} supports the WebAssembly architecture}{wasm}{WebAssembly Architecture Support}}
\providecommand{\seedocumentation}{}\renewcommand{\seedocumentation}{\seedocumentationref{For more information about generic documentations and their generation by the \ecs{}}{documentation}{Generic Documentation Generation}}
\providecommand{\seedebugging}{}\renewcommand{\seedebugging}{\seedocumentationref{For more information about debugging information and its representation}{debugging}{Debugging Information Representation}}
\providecommand{\seecode}{}\renewcommand{\seecode}{\seedocumentationref{For more information about intermediate code and its purpose}{code}{Intermediate Code Representation}}
\providecommand{\seeobject}{}\renewcommand{\seeobject}{\seedocumentationref{For more information about object files and their purpose}{object}{Object File Representation}}

% generic documentation tools

\providecommand{\docprint}{
\toolsection{docprint} is a pretty printer for generic documentations.
It reformats generic documentations and writes it to the standard output stream.
\debuggingtool
\flowgraph{\resource{generic\\documentation} \ar[r] & \toolbox{docprint} \ar[r] & \resource{generic\\documentation}}
\seedocumentation
}

\providecommand{\doccheck}{
\toolsection{doccheck} is a syntactic and semantic checker for generic documentations.
It just performs syntactic and semantic checks on generic documentations and writes its diagnostic messages to the standard error stream.
\debuggingtool
\flowgraph{\resource{generic\\documentation} \ar[r] & \toolbox{doccheck} \ar[r] & \resource{diagnostic\\messages}}
\seedocumentation
}

\providecommand{\dochtml}{
\toolsection{dochtml} is an HTML documentation generator for generic documentations.
It processes several generic documentations and assembles all information therein into an HTML document.
\debuggingtool
\flowgraph{\resource{generic\\documentation} \ar[r] & \toolbox{dochtml} \ar[r] & \resource{HTML\\document}}
\seedocumentation
}

\providecommand{\doclatex}{
\toolsection{doclatex} is a Latex documentation generator for generic documentations.
It processes several generic documentations and assembles all information therein into a Latex document.
\debuggingtool
\flowgraph{\resource{generic\\documentation} \ar[r] & \toolbox{doclatex} \ar[r] & \resource{Latex\\document}}
\seedocumentation
}

% intermediate code tools

\providecommand{\cdcheck}{
\toolsection{cdcheck} is a syntactic and semantic checker for intermediate code.
It just performs syntactic and semantic checks on programs written in intermediate code and writes its diagnostic messages to the standard error stream.
\debuggingtool
\flowgraph{\resource{intermediate\\code} \ar[r] & \toolbox{cdcheck} \ar[r] & \resource{diagnostic\\messages}}
\seeassembly\seecode
}

\providecommand{\cdopt}{
\toolsection{cdopt} is an optimizer for intermediate code.
It performs various optimizations on programs written in intermediate code and writes the result to the standard output stream.
\debuggingtool
\flowgraph{\resource{intermediate\\code} \ar[r] & \toolbox{cdopt} \ar[r] & \resource{optimized\\code}}
\seeassembly\seecode
}

\providecommand{\cdrun}{
\toolsection{cdrun} is an interpreter for intermediate code.
It processes and executes programs written in intermediate code.
The following code sections are predefined and have the usual semantics:
\texttt{abort}, \texttt{\_Exit}, \texttt{fflush}, \texttt{floor}, \texttt{fputc}, \texttt{free}, \texttt{getchar}, \texttt{malloc}, and \texttt{putchar}.
Diagnostic messages about invalid operations include the name of the executed code section and the index of the erroneous instruction.
\debuggingtool
\flowgraph{\resource{intermediate\\code} \ar[r] & \toolbox{cdrun} \ar@/u/[r] & \resource{input/\\output} \ar@/d/[l]}
\seeassembly\seecode
}

\providecommand{\cdamda}{
\toolsection{cdamd16} is a compiler for intermediate code targeting the AMD64 hardware architecture.
It generates machine code for AMD64 processors from programs written in intermediate code and stores it in corresponding object files.
The compiler generates machine code for the 16-bit operating mode defined by the AMD64 architecture.
It also creates a debugging information file as well as an assembly file containing a listing of the generated machine code.
\debuggingtool
\flowgraph{\resource{intermediate\\code} \ar[r] & \toolbox{cdamd16} \ar[r] \ar[d] \ar[rd] & \resource{object file} \\ & \resource{assembly\\listing} & \resource{debugging\\information}}
\seeassembly\seeamd\seeobject\seecode\seedebugging
}

\providecommand{\cdamdb}{
\toolsection{cdamd32} is a compiler for intermediate code targeting the AMD64 hardware architecture.
It generates machine code for AMD64 processors from programs written in intermediate code and stores it in corresponding object files.
The compiler generates machine code for the 32-bit operating mode defined by the AMD64 architecture.
It also creates a debugging information file as well as an assembly file containing a listing of the generated machine code.
\debuggingtool
\flowgraph{\resource{intermediate\\code} \ar[r] & \toolbox{cdamd32} \ar[r] \ar[d] \ar[rd] & \resource{object file} \\ & \resource{assembly\\listing} & \resource{debugging\\information}}
\seeassembly\seeamd\seeobject\seecode\seedebugging
}

\providecommand{\cdamdc}{
\toolsection{cdamd64} is a compiler for intermediate code targeting the AMD64 hardware architecture.
It generates machine code for AMD64 processors from programs written in intermediate code and stores it in corresponding object files.
The compiler generates machine code for the 64-bit operating mode defined by the AMD64 architecture.
It also creates a debugging information file as well as an assembly file containing a listing of the generated machine code.
\debuggingtool
\flowgraph{\resource{intermediate\\code} \ar[r] & \toolbox{cdamd64} \ar[r] \ar[d] \ar[rd] & \resource{object file} \\ & \resource{assembly\\listing} & \resource{debugging\\information}}
\seeassembly\seeamd\seeobject\seecode\seedebugging
}

\providecommand{\cdarma}{
\toolsection{cdarma32} is a compiler for intermediate code targeting the ARM hardware architecture.
It generates machine code for ARM processors executing A32 instructions from programs written in intermediate code and stores it in corresponding object files.
It also creates a debugging information file as well as an assembly file containing a listing of the generated machine code.
\debuggingtool
\flowgraph{\resource{intermediate\\code} \ar[r] & \toolbox{cdarma32} \ar[r] \ar[d] \ar[rd] & \resource{object file} \\ & \resource{assembly\\listing} & \resource{debugging\\information}}
\seeassembly\seearm\seeobject\seecode\seedebugging
}

\providecommand{\cdarmb}{
\toolsection{cdarma64} is a compiler for intermediate code targeting the ARM hardware architecture.
It generates machine code for ARM processors executing A64 instructions from programs written in intermediate code and stores it in corresponding object files.
It also creates a debugging information file as well as an assembly file containing a listing of the generated machine code.
\debuggingtool
\flowgraph{\resource{intermediate\\code} \ar[r] & \toolbox{cdarma64} \ar[r] \ar[d] \ar[rd] & \resource{object file} \\ & \resource{assembly\\listing} & \resource{debugging\\information}}
\seeassembly\seearm\seeobject\seecode\seedebugging
}

\providecommand{\cdarmc}{
\toolsection{cdarmt32} is a compiler for intermediate code targeting the ARM hardware architecture.
It generates machine code for ARM processors without floating-point extension executing T32 instructions from programs written in intermediate code and stores it in corresponding object files.
It also creates a debugging information file as well as an assembly file containing a listing of the generated machine code.
\debuggingtool
\flowgraph{\resource{intermediate\\code} \ar[r] & \toolbox{cdarmt32} \ar[r] \ar[d] \ar[rd] & \resource{object file} \\ & \resource{assembly\\listing} & \resource{debugging\\information}}
\seeassembly\seearm\seeobject\seecode\seedebugging
}

\providecommand{\cdarmcfpe}{
\toolsection{cdarmt32fpe} is a compiler for intermediate code targeting the ARM hardware architecture.
It generates machine code for ARM processors with floating-point extension executing T32 instructions from programs written in intermediate code and stores it in corresponding object files.
It also creates a debugging information file as well as an assembly file containing a listing of the generated machine code.
\debuggingtool
\flowgraph{\resource{intermediate\\code} \ar[r] & \toolbox{cdarmt32fpe} \ar[r] \ar[d] \ar[rd] & \resource{object file} \\ & \resource{assembly\\listing} & \resource{debugging\\information}}
\seeassembly\seearm\seeobject\seecode\seedebugging
}

\providecommand{\cdavr}{
\toolsection{cdavr} is a compiler for intermediate code targeting the AVR hardware architecture.
It generates machine code for AVR processors from programs written in intermediate code and stores it in corresponding object files.
It also creates a debugging information file as well as an assembly file containing a listing of the generated machine code.
\debuggingtool
\flowgraph{\resource{intermediate\\code} \ar[r] & \toolbox{cdavr} \ar[r] \ar[d] \ar[rd] & \resource{object file} \\ & \resource{assembly\\listing} & \resource{debugging\\information}}
\seeassembly\seeavr\seeobject\seecode\seedebugging
}

\providecommand{\cdavrtt}{
\toolsection{cdavr32} is a compiler for intermediate code targeting the AVR32 hardware architecture.
It generates machine code for AVR32 processors from programs written in intermediate code and stores it in corresponding object files.
It also creates a debugging information file as well as an assembly file containing a listing of the generated machine code.
\debuggingtool
\flowgraph{\resource{intermediate\\code} \ar[r] & \toolbox{cdavr32} \ar[r] \ar[d] \ar[rd] & \resource{object file} \\ & \resource{assembly\\listing} & \resource{debugging\\information}}
\seeassembly\seeavrtt\seeobject\seecode\seedebugging
}

\providecommand{\cdmabk}{
\toolsection{cdm68k} is a compiler for intermediate code targeting the M68000 hardware architecture.
It generates machine code for M68000 processors from programs written in intermediate code and stores it in corresponding object files.
It also creates a debugging information file as well as an assembly file containing a listing of the generated machine code.
\debuggingtool
\flowgraph{\resource{intermediate\\code} \ar[r] & \toolbox{cdm68k} \ar[r] \ar[d] \ar[rd] & \resource{object file} \\ & \resource{assembly\\listing} & \resource{debugging\\information}}
\seeassembly\seemabk\seeobject\seecode\seedebugging
}

\providecommand{\cdmibl}{
\toolsection{cdmibl} is a compiler for intermediate code targeting the MicroBlaze hardware architecture.
It generates machine code for MicroBlaze processors from programs written in intermediate code and stores it in corresponding object files.
It also creates a debugging information file as well as an assembly file containing a listing of the generated machine code.
\debuggingtool
\flowgraph{\resource{intermediate\\code} \ar[r] & \toolbox{cdmibl} \ar[r] \ar[d] \ar[rd] & \resource{object file} \\ & \resource{assembly\\listing} & \resource{debugging\\information}}
\seeassembly\seemibl\seeobject\seecode\seedebugging
}

\providecommand{\cdmipsa}{
\toolsection{cdmips32} is a compiler for intermediate code targeting the MIPS32 hardware architecture.
It generates machine code for MIPS32 processors from programs written in intermediate code and stores it in corresponding object files.
It also creates a debugging information file as well as an assembly file containing a listing of the generated machine code.
\debuggingtool
\flowgraph{\resource{intermediate\\code} \ar[r] & \toolbox{cdmips32} \ar[r] \ar[d] \ar[rd] & \resource{object file} \\ & \resource{assembly\\listing} & \resource{debugging\\information}}
\seeassembly\seemips\seeobject\seecode\seedebugging
}

\providecommand{\cdmipsb}{
\toolsection{cdmips64} is a compiler for intermediate code targeting the MIPS64 hardware architecture.
It generates machine code for MIPS64 processors from programs written in intermediate code and stores it in corresponding object files.
It also creates a debugging information file as well as an assembly file containing a listing of the generated machine code.
\debuggingtool
\flowgraph{\resource{intermediate\\code} \ar[r] & \toolbox{cdmips64} \ar[r] \ar[d] \ar[rd] & \resource{object file} \\ & \resource{assembly\\listing} & \resource{debugging\\information}}
\seeassembly\seemips\seeobject\seecode\seedebugging
}

\providecommand{\cdmmix}{
\toolsection{cdmmix} is a compiler for intermediate code targeting the MMIX hardware architecture.
It generates machine code for MMIX processors from programs written in intermediate code and stores it in corresponding object files.
It also creates a debugging information file as well as an assembly file containing a listing of the generated machine code.
\debuggingtool
\flowgraph{\resource{intermediate\\code} \ar[r] & \toolbox{cdmmix} \ar[r] \ar[d] \ar[rd] & \resource{object file} \\ & \resource{assembly\\listing} & \resource{debugging\\information}}
\seeassembly\seemmix\seeobject\seecode\seedebugging
}

\providecommand{\cdorok}{
\toolsection{cdor1k} is a compiler for intermediate code targeting the OpenRISC 1000 hardware architecture.
It generates machine code for OpenRISC 1000 processors from programs written in intermediate code and stores it in corresponding object files.
It also creates a debugging information file as well as an assembly file containing a listing of the generated machine code.
\debuggingtool
\flowgraph{\resource{intermediate\\code} \ar[r] & \toolbox{cdor1k} \ar[r] \ar[d] \ar[rd] & \resource{object file} \\ & \resource{assembly\\listing} & \resource{debugging\\information}}
\seeassembly\seeorok\seeobject\seecode\seedebugging
}

\providecommand{\cdppca}{
\toolsection{cdppc32} is a compiler for intermediate code targeting the PowerPC hardware architecture.
It generates machine code for PowerPC processors from programs written in intermediate code and stores it in corresponding object files.
The compiler generates machine code for the 32-bit operating mode defined by the PowerPC architecture.
It also creates a debugging information file as well as an assembly file containing a listing of the generated machine code.
\debuggingtool
\flowgraph{\resource{intermediate\\code} \ar[r] & \toolbox{cdppc32} \ar[r] \ar[d] \ar[rd] & \resource{object file} \\ & \resource{assembly\\listing} & \resource{debugging\\information}}
\seeassembly\seeppc\seeobject\seecode\seedebugging
}

\providecommand{\cdppcb}{
\toolsection{cdppc64} is a compiler for intermediate code targeting the PowerPC hardware architecture.
It generates machine code for PowerPC processors from programs written in intermediate code and stores it in corresponding object files.
The compiler generates machine code for the 64-bit operating mode defined by the PowerPC architecture.
It also creates a debugging information file as well as an assembly file containing a listing of the generated machine code.
\debuggingtool
\flowgraph{\resource{intermediate\\code} \ar[r] & \toolbox{cdppc64} \ar[r] \ar[d] \ar[rd] & \resource{object file} \\ & \resource{assembly\\listing} & \resource{debugging\\information}}
\seeassembly\seeppc\seeobject\seecode\seedebugging
}

\providecommand{\cdrisc}{
\toolsection{cdrisc} is a compiler for intermediate code targeting the RISC hardware architecture.
It generates machine code for RISC processors from programs written in intermediate code and stores it in corresponding object files.
It also creates a debugging information file as well as an assembly file containing a listing of the generated machine code.
\debuggingtool
\flowgraph{\resource{intermediate\\code} \ar[r] & \toolbox{cdrisc} \ar[r] \ar[d] \ar[rd] & \resource{object file} \\ & \resource{assembly\\listing} & \resource{debugging\\information}}
\seeassembly\seerisc\seeobject\seecode\seedebugging
}

\providecommand{\cdwasm}{
\toolsection{cdwasm} is a compiler for intermediate code targeting the WebAssembly architecture.
It generates machine code for WebAssembly targets from programs written in intermediate code and stores it in corresponding object files.
It also creates a debugging information file as well as an assembly file containing a listing of the generated machine code.
\debuggingtool
\flowgraph{\resource{intermediate\\code} \ar[r] & \toolbox{cdwasm} \ar[r] \ar[d] \ar[rd] & \resource{object file} \\ & \resource{assembly\\listing} & \resource{debugging\\information}}
\seeassembly\seewasm\seeobject\seecode\seedebugging
}

% C++ tools

\providecommand{\cppprep}{
\toolsection{cppprep} is a preprocessor for the \cpp{} programming language.
It preprocesses source code according to the rules of \cpp{} and writes it to the standard output stream.
Only the macro names \texttt{\_\_DATE\_\_}, \texttt{\_\_FILE\_\_}, \texttt{\_\_LINE\_\_}, and \texttt{\_\_TIME\_\_} are predefined.
\flowgraph{\resource{\cpp{} or other\\source code} \ar[r] & \toolbox{cppprep} \ar[r] & \resource{preprocessed\\source code} \\ & \variable{ECSINCLUDE} \ar[u]}
\seecpp
}

\providecommand{\cppprint}{
\toolsection{cppprint} is a pretty printer for the \cpp{} programming language.
It reformats the source code of \cpp{} programs and writes it to the standard output stream.
\flowgraph{\resource{\cpp{}\\source code} \ar[r] & \toolbox{cppprint} \ar[r] & \resource{reformatted\\source code} \\ & \variable{ECSINCLUDE} \ar[u]}
\seecpp
}

\providecommand{\cppcheck}{
\toolsection{cppcheck} is a syntactic and semantic checker for the \cpp{} programming language.
It just performs syntactic and semantic checks on \cpp{} programs and writes its diagnostic messages to the standard error stream.
\flowgraph{\resource{\cpp{}\\source code} \ar[r] & \toolbox{cppcheck} \ar[r] & \resource{diagnostic\\messages} \\ & \variable{ECSINCLUDE} \ar[u]}
\seecpp
}

\providecommand{\cppdump}{
\toolsection{cppdump} is a serializer for the \cpp{} programming language.
It dumps the complete internal representation of programs written in \cpp{} into an XML document.
\debuggingtool
\flowgraph{\resource{\cpp{}\\source code} \ar[r] & \toolbox{cppdump} \ar[r] & \resource{internal\\representation} \\ & \variable{ECSINCLUDE} \ar[u]}
\seecpp
}

\providecommand{\cpprun}{
\toolsection{cpprun} is an interpreter for the \cpp{} programming language.
It processes and executes programs written in \cpp{}.
The macro \texttt{\_\_run\_\_} is predefined in order to enable programmers to identify this tool while interpreting.
\flowgraph{\resource{\cpp{}\\source code} \ar[r] & \toolbox{cpprun} \ar@/u/[r] & \resource{input/\\output} \ar@/d/[l] \\ & \variable{ECSINCLUDE} \ar[u]}
\seecpp
}

\providecommand{\cppdoc}{
\toolsection{cppdoc} is a generic documentation generator for the \cpp{} programming language.
It processes several \cpp{} source files and assembles all information therein into a generic documentation.
\debuggingtool
\flowgraph{\resource{\cpp{}\\source code} \ar[r] & \toolbox{cppdoc} \ar[r] & \resource{generic\\documentation} \\ & \variable{ECSINCLUDE} \ar[u]}
\seecpp\seedocumentation
}

\providecommand{\cpphtml}{
\toolsection{cpphtml} is an HTML documentation generator for the \cpp{} programming language.
It processes several \cpp{} source files and assembles all information therein into an HTML document.
\flowgraph{\resource{\cpp{}\\source code} \ar[r] & \toolbox{cpphtml} \ar[r] & \resource{HTML\\document} \\ & \variable{ECSINCLUDE} \ar[u]}
\seecpp\seedocumentation
}

\providecommand{\cpplatex}{
\toolsection{cpplatex} is a Latex documentation generator for the \cpp{} programming language.
It processes several \cpp{} source files and assembles all information therein into a Latex document.
\flowgraph{\resource{\cpp{}\\source code} \ar[r] & \toolbox{cpplatex} \ar[r] & \resource{Latex\\document} \\ & \variable{ECSINCLUDE} \ar[u]}
\seecpp\seedocumentation
}

\providecommand{\cppcode}{
\toolsection{cppcode} is an intermediate code generator for the \cpp{} programming language.
It generates intermediate code from programs written in \cpp{} and stores it in corresponding assembly files.
The macro \texttt{\_\_code\_\_} is predefined in order to enable programmers to identify this tool while generating intermediate code.
Programs generated with this tool require additional runtime support that is stored in the \file{cpp\-code\-run} library file.
\debuggingtool
\flowgraph{\resource{\cpp{}\\source code} \ar[r] & \toolbox{cppcode} \ar[r] & \resource{intermediate\\code} \\ & \variable{ECSINCLUDE} \ar[u]}
\seecpp\seeassembly\seecode
}

\providecommand{\cppamda}{
\toolsection{cppamd16} is a compiler for the \cpp{} programming language targeting the AMD64 hardware architecture.
It generates machine code for AMD64 processors from programs written in \cpp{} and stores it in corresponding object files.
The compiler generates machine code for the 16-bit operating mode defined by the AMD64 architecture.
For debugging purposes, it also creates a debugging information file as well as an assembly file containing a listing of the generated machine code.
The macro \texttt{\_\_amd16\_\_} is predefined in order to enable programmers to identify this tool and its target architecture while compiling.
Programs generated with this compiler require additional runtime support that is stored in the \file{cpp\-amd16\-run} library file.
\flowgraph{\resource{\cpp{}\\source code} \ar[r] & \toolbox{cppamd16} \ar[r] \ar[d] \ar[rd] & \resource{object file} \\ \variable{ECSINCLUDE} \ar[ru] & \resource{debugging\\information} & \resource{assembly\\listing}}
\seecpp\seeassembly\seeamd\seeobject\seedebugging
}

\providecommand{\cppamdb}{
\toolsection{cppamd32} is a compiler for the \cpp{} programming language targeting the AMD64 hardware architecture.
It generates machine code for AMD64 processors from programs written in \cpp{} and stores it in corresponding object files.
The compiler generates machine code for the 32-bit operating mode defined by the AMD64 architecture.
For debugging purposes, it also creates a debugging information file as well as an assembly file containing a listing of the generated machine code.
The macro \texttt{\_\_amd32\_\_} is predefined in order to enable programmers to identify this tool and its target architecture while compiling.
Programs generated with this compiler require additional runtime support that is stored in the \file{cpp\-amd32\-run} library file.
\flowgraph{\resource{\cpp{}\\source code} \ar[r] & \toolbox{cppamd32} \ar[r] \ar[d] \ar[rd] & \resource{object file} \\ \variable{ECSINCLUDE} \ar[ru] & \resource{debugging\\information} & \resource{assembly\\listing}}
\seecpp\seeassembly\seeamd\seeobject\seedebugging
}

\providecommand{\cppamdc}{
\toolsection{cppamd64} is a compiler for the \cpp{} programming language targeting the AMD64 hardware architecture.
It generates machine code for AMD64 processors from programs written in \cpp{} and stores it in corresponding object files.
The compiler generates machine code for the 64-bit operating mode defined by the AMD64 architecture.
For debugging purposes, it also creates a debugging information file as well as an assembly file containing a listing of the generated machine code.
The macro \texttt{\_\_amd64\_\_} is predefined in order to enable programmers to identify this tool and its target architecture while compiling.
Programs generated with this compiler require additional runtime support that is stored in the \file{cpp\-amd64\-run} library file.
\flowgraph{\resource{\cpp{}\\source code} \ar[r] & \toolbox{cppamd64} \ar[r] \ar[d] \ar[rd] & \resource{object file} \\ \variable{ECSINCLUDE} \ar[ru] & \resource{debugging\\information} & \resource{assembly\\listing}}
\seecpp\seeassembly\seeamd\seeobject\seedebugging
}

\providecommand{\cpparma}{
\toolsection{cpparma32} is a compiler for the \cpp{} programming language targeting the ARM hardware architecture.
It generates machine code for ARM processors executing A32 instructions from programs written in \cpp{} and stores it in corresponding object files.
For debugging purposes, it also creates a debugging information file as well as an assembly file containing a listing of the generated machine code.
The macro \texttt{\_\_arma32\_\_} is predefined in order to enable programmers to identify this tool and its target architecture while compiling.
Programs generated with this compiler require additional runtime support that is stored in the \file{cpp\-arma32\-run} library file.
\flowgraph{\resource{\cpp{}\\source code} \ar[r] & \toolbox{cpparma32} \ar[r] \ar[d] \ar[rd] & \resource{object file} \\ \variable{ECSINCLUDE} \ar[ru] & \resource{debugging\\information} & \resource{assembly\\listing}}
\seecpp\seeassembly\seearm\seeobject\seedebugging
}

\providecommand{\cpparmb}{
\toolsection{cpparma64} is a compiler for the \cpp{} programming language targeting the ARM hardware architecture.
It generates machine code for ARM processors executing A64 instructions from programs written in \cpp{} and stores it in corresponding object files.
For debugging purposes, it also creates a debugging information file as well as an assembly file containing a listing of the generated machine code.
The macro \texttt{\_\_arma64\_\_} is predefined in order to enable programmers to identify this tool and its target architecture while compiling.
Programs generated with this compiler require additional runtime support that is stored in the \file{cpp\-arma64\-run} library file.
\flowgraph{\resource{\cpp{}\\source code} \ar[r] & \toolbox{cpparma64} \ar[r] \ar[d] \ar[rd] & \resource{object file} \\ \variable{ECSINCLUDE} \ar[ru] & \resource{debugging\\information} & \resource{assembly\\listing}}
\seecpp\seeassembly\seearm\seeobject\seedebugging
}

\providecommand{\cpparmc}{
\toolsection{cpparmt32} is a compiler for the \cpp{} programming language targeting the ARM hardware architecture.
It generates machine code for ARM processors without floating-point extension executing T32 instructions from programs written in \cpp{} and stores it in corresponding object files.
For debugging purposes, it also creates a debugging information file as well as an assembly file containing a listing of the generated machine code.
The macro \texttt{\_\_armt32\_\_} is predefined in order to enable programmers to identify this tool and its target architecture while compiling.
Programs generated with this compiler require additional runtime support that is stored in the \file{cpp\-armt32\-run} library file.
\flowgraph{\resource{\cpp{}\\source code} \ar[r] & \toolbox{cpparmt32} \ar[r] \ar[d] \ar[rd] & \resource{object file} \\ \variable{ECSINCLUDE} \ar[ru] & \resource{debugging\\information} & \resource{assembly\\listing}}
\seecpp\seeassembly\seearm\seeobject\seedebugging
}

\providecommand{\cpparmcfpe}{
\toolsection{cpparmt32fpe} is a compiler for the \cpp{} programming language targeting the ARM hardware architecture.
It generates machine code for ARM processors with floating-point extension executing T32 instructions from programs written in \cpp{} and stores it in corresponding object files.
For debugging purposes, it also creates a debugging information file as well as an assembly file containing a listing of the generated machine code.
The macro \texttt{\_\_armt32fpe\_\_} is predefined in order to enable programmers to identify this tool and its target architecture while compiling.
Programs generated with this compiler require additional runtime support that is stored in the \file{cpp\-armt32\-fpe\-run} library file.
\flowgraph{\resource{\cpp{}\\source code} \ar[r] & \toolbox{cpparmt32fpe} \ar[r] \ar[d] \ar[rd] & \resource{object file} \\ \variable{ECSINCLUDE} \ar[ru] & \resource{debugging\\information} & \resource{assembly\\listing}}
\seecpp\seeassembly\seearm\seeobject\seedebugging
}

\providecommand{\cppavr}{
\toolsection{cppavr} is a compiler for the \cpp{} programming language targeting the AVR hardware architecture.
It generates machine code for AVR processors from programs written in \cpp{} and stores it in corresponding object files.
For debugging purposes, it also creates a debugging information file as well as an assembly file containing a listing of the generated machine code.
The macro \texttt{\_\_avr\_\_} is predefined in order to enable programmers to identify this tool and its target architecture while compiling.
Programs generated with this compiler require additional runtime support that is stored in the \file{cpp\-avr\-run} library file.
\flowgraph{\resource{\cpp{}\\source code} \ar[r] & \toolbox{cppavr} \ar[r] \ar[d] \ar[rd] & \resource{object file} \\ \variable{ECSINCLUDE} \ar[ru] & \resource{debugging\\information} & \resource{assembly\\listing}}
\seecpp\seeassembly\seeavr\seeobject\seedebugging
}

\providecommand{\cppavrtt}{
\toolsection{cppavr32} is a compiler for the \cpp{} programming language targeting the AVR32 hardware architecture.
It generates machine code for AVR32 processors from programs written in \cpp{} and stores it in corresponding object files.
For debugging purposes, it also creates a debugging information file as well as an assembly file containing a listing of the generated machine code.
The macro \texttt{\_\_avr32\_\_} is predefined in order to enable programmers to identify this tool and its target architecture while compiling.
Programs generated with this compiler require additional runtime support that is stored in the \file{cpp\-avr32\-run} library file.
\flowgraph{\resource{\cpp{}\\source code} \ar[r] & \toolbox{cppavr32} \ar[r] \ar[d] \ar[rd] & \resource{object file} \\ \variable{ECSINCLUDE} \ar[ru] & \resource{debugging\\information} & \resource{assembly\\listing}}
\seecpp\seeassembly\seeavrtt\seeobject\seedebugging
}

\providecommand{\cppmabk}{
\toolsection{cppm68k} is a compiler for the \cpp{} programming language targeting the M68000 hardware architecture.
It generates machine code for M68000 processors from programs written in \cpp{} and stores it in corresponding object files.
For debugging purposes, it also creates a debugging information file as well as an assembly file containing a listing of the generated machine code.
The macro \texttt{\_\_m68k\_\_} is predefined in order to enable programmers to identify this tool and its target architecture while compiling.
Programs generated with this compiler require additional runtime support that is stored in the \file{cpp\-m68k\-run} library file.
\flowgraph{\resource{\cpp{}\\source code} \ar[r] & \toolbox{cppm68k} \ar[r] \ar[d] \ar[rd] & \resource{object file} \\ \variable{ECSINCLUDE} \ar[ru] & \resource{debugging\\information} & \resource{assembly\\listing}}
\seecpp\seeassembly\seemabk\seeobject\seedebugging
}

\providecommand{\cppmibl}{
\toolsection{cppmibl} is a compiler for the \cpp{} programming language targeting the MicroBlaze hardware architecture.
It generates machine code for MicroBlaze processors from programs written in \cpp{} and stores it in corresponding object files.
For debugging purposes, it also creates a debugging information file as well as an assembly file containing a listing of the generated machine code.
The macro \texttt{\_\_mibl\_\_} is predefined in order to enable programmers to identify this tool and its target architecture while compiling.
Programs generated with this compiler require additional runtime support that is stored in the \file{cpp\-mibl\-run} library file.
\flowgraph{\resource{\cpp{}\\source code} \ar[r] & \toolbox{cppmibl} \ar[r] \ar[d] \ar[rd] & \resource{object file} \\ \variable{ECSINCLUDE} \ar[ru] & \resource{debugging\\information} & \resource{assembly\\listing}}
\seecpp\seeassembly\seemibl\seeobject\seedebugging
}

\providecommand{\cppmipsa}{
\toolsection{cppmips32} is a compiler for the \cpp{} programming language targeting the MIPS32 hardware architecture.
It generates machine code for MIPS32 processors from programs written in \cpp{} and stores it in corresponding object files.
For debugging purposes, it also creates a debugging information file as well as an assembly file containing a listing of the generated machine code.
The macro \texttt{\_\_mips32\_\_} is predefined in order to enable programmers to identify this tool and its target architecture while compiling.
Programs generated with this compiler require additional runtime support that is stored in the \file{cpp\-mips32\-run} library file.
\flowgraph{\resource{\cpp{}\\source code} \ar[r] & \toolbox{cppmips32} \ar[r] \ar[d] \ar[rd] & \resource{object file} \\ \variable{ECSINCLUDE} \ar[ru] & \resource{debugging\\information} & \resource{assembly\\listing}}
\seecpp\seeassembly\seemips\seeobject\seedebugging
}

\providecommand{\cppmipsb}{
\toolsection{cppmips64} is a compiler for the \cpp{} programming language targeting the MIPS64 hardware architecture.
It generates machine code for MIPS64 processors from programs written in \cpp{} and stores it in corresponding object files.
For debugging purposes, it also creates a debugging information file as well as an assembly file containing a listing of the generated machine code.
The macro \texttt{\_\_mips64\_\_} is predefined in order to enable programmers to identify this tool and its target architecture while compiling.
Programs generated with this compiler require additional runtime support that is stored in the \file{cpp\-mips64\-run} library file.
\flowgraph{\resource{\cpp{}\\source code} \ar[r] & \toolbox{cppmips64} \ar[r] \ar[d] \ar[rd] & \resource{object file} \\ \variable{ECSINCLUDE} \ar[ru] & \resource{debugging\\information} & \resource{assembly\\listing}}
\seecpp\seeassembly\seemips\seeobject\seedebugging
}

\providecommand{\cppmmix}{
\toolsection{cppmmix} is a compiler for the \cpp{} programming language targeting the MMIX hardware architecture.
It generates machine code for MMIX processors from programs written in \cpp{} and stores it in corresponding object files.
For debugging purposes, it also creates a debugging information file as well as an assembly file containing a listing of the generated machine code.
The macro \texttt{\_\_mmix\_\_} is predefined in order to enable programmers to identify this tool and its target architecture while compiling.
Programs generated with this compiler require additional runtime support that is stored in the \file{cpp\-mmix\-run} library file.
\flowgraph{\resource{\cpp{}\\source code} \ar[r] & \toolbox{cppmmix} \ar[r] \ar[d] \ar[rd] & \resource{object file} \\ \variable{ECSINCLUDE} \ar[ru] & \resource{debugging\\information} & \resource{assembly\\listing}}
\seecpp\seeassembly\seemmix\seeobject\seedebugging
}

\providecommand{\cpporok}{
\toolsection{cppor1k} is a compiler for the \cpp{} programming language targeting the OpenRISC 1000 hardware architecture.
It generates machine code for OpenRISC 1000 processors from programs written in \cpp{} and stores it in corresponding object files.
For debugging purposes, it also creates a debugging information file as well as an assembly file containing a listing of the generated machine code.
The macro \texttt{\_\_or1k\_\_} is predefined in order to enable programmers to identify this tool and its target architecture while compiling.
Programs generated with this compiler require additional runtime support that is stored in the \file{cpp\-or1k\-run} library file.
\flowgraph{\resource{\cpp{}\\source code} \ar[r] & \toolbox{cppor1k} \ar[r] \ar[d] \ar[rd] & \resource{object file} \\ \variable{ECSINCLUDE} \ar[ru] & \resource{debugging\\information} & \resource{assembly\\listing}}
\seecpp\seeassembly\seeorok\seeobject\seedebugging
}

\providecommand{\cppppca}{
\toolsection{cppppc32} is a compiler for the \cpp{} programming language targeting the PowerPC hardware architecture.
It generates machine code for PowerPC processors from programs written in \cpp{} and stores it in corresponding object files.
The compiler generates machine code for the 32-bit operating mode defined by the PowerPC architecture.
For debugging purposes, it also creates a debugging information file as well as an assembly file containing a listing of the generated machine code.
The macro \texttt{\_\_ppc32\_\_} is predefined in order to enable programmers to identify this tool and its target architecture while compiling.
Programs generated with this compiler require additional runtime support that is stored in the \file{cpp\-ppc32\-run} library file.
\flowgraph{\resource{\cpp{}\\source code} \ar[r] & \toolbox{cppppc32} \ar[r] \ar[d] \ar[rd] & \resource{object file} \\ \variable{ECSINCLUDE} \ar[ru] & \resource{debugging\\information} & \resource{assembly\\listing}}
\seecpp\seeassembly\seeppc\seeobject\seedebugging
}

\providecommand{\cppppcb}{
\toolsection{cppppc64} is a compiler for the \cpp{} programming language targeting the PowerPC hardware architecture.
It generates machine code for PowerPC processors from programs written in \cpp{} and stores it in corresponding object files.
The compiler generates machine code for the 64-bit operating mode defined by the PowerPC architecture.
For debugging purposes, it also creates a debugging information file as well as an assembly file containing a listing of the generated machine code.
The macro \texttt{\_\_ppc64\_\_} is predefined in order to enable programmers to identify this tool and its target architecture while compiling.
Programs generated with this compiler require additional runtime support that is stored in the \file{cpp\-ppc64\-run} library file.
\flowgraph{\resource{\cpp{}\\source code} \ar[r] & \toolbox{cppppc64} \ar[r] \ar[d] \ar[rd] & \resource{object file} \\ \variable{ECSINCLUDE} \ar[ru] & \resource{debugging\\information} & \resource{assembly\\listing}}
\seecpp\seeassembly\seeppc\seeobject\seedebugging
}

\providecommand{\cpprisc}{
\toolsection{cpprisc} is a compiler for the \cpp{} programming language targeting the RISC hardware architecture.
It generates machine code for RISC processors from programs written in \cpp{} and stores it in corresponding object files.
For debugging purposes, it also creates a debugging information file as well as an assembly file containing a listing of the generated machine code.
The macro \texttt{\_\_risc\_\_} is predefined in order to enable programmers to identify this tool and its target architecture while compiling.
Programs generated with this compiler require additional runtime support that is stored in the \file{cpp\-risc\-run} library file.
\flowgraph{\resource{\cpp{}\\source code} \ar[r] & \toolbox{cpprisc} \ar[r] \ar[d] \ar[rd] & \resource{object file} \\ \variable{ECSINCLUDE} \ar[ru] & \resource{debugging\\information} & \resource{assembly\\listing}}
\seecpp\seeassembly\seerisc\seeobject\seedebugging
}

\providecommand{\cppwasm}{
\toolsection{cppwasm} is a compiler for the \cpp{} programming language targeting the WebAssembly architecture.
It generates machine code for WebAssembly targets from programs written in \cpp{} and stores it in corresponding object files.
For debugging purposes, it also creates a debugging information file as well as an assembly file containing a listing of the generated machine code.
The macro \texttt{\_\_wasm\_\_} is predefined in order to enable programmers to identify this tool and its target architecture while compiling.
Programs generated with this compiler require additional runtime support that is stored in the \file{cpp\-wasm\-run} library file.
\flowgraph{\resource{\cpp{}\\source code} \ar[r] & \toolbox{cppwasm} \ar[r] \ar[d] \ar[rd] & \resource{object file} \\ \variable{ECSINCLUDE} \ar[ru] & \resource{debugging\\information} & \resource{assembly\\listing}}
\seecpp\seeassembly\seewasm\seeobject\seedebugging
}

% FALSE tools

\providecommand{\falprint}{
\toolsection{falprint} is a pretty printer for the FALSE programming language.
It reformats the source code of FALSE programs and writes it to the standard output stream.
\flowgraph{\resource{FALSE\\source code} \ar[r] & \toolbox{falprint} \ar[r] & \resource{reformatted\\source code}}
\seefalse
}

\providecommand{\falcheck}{
\toolsection{falcheck} is a syntactic and semantic checker for the FALSE programming language.
It just performs syntactic and semantic checks on FALSE programs and writes its diagnostic messages to the standard error stream.
\flowgraph{\resource{FALSE\\source code} \ar[r] & \toolbox{falcheck} \ar[r] & \resource{diagnostic\\messages}}
\seefalse
}

\providecommand{\faldump}{
\toolsection{faldump} is a serializer for the FALSE programming language.
It dumps the complete internal representation of programs written in FALSE into an XML document.
\debuggingtool
\flowgraph{\resource{FALSE\\source code} \ar[r] & \toolbox{faldump} \ar[r] & \resource{internal\\representation}}
\seefalse
}

\providecommand{\falrun}{
\toolsection{falrun} is an interpreter for the FALSE programming language.
It processes and executes programs written in FALSE\@.
\flowgraph{\resource{FALSE\\source code} \ar[r] & \toolbox{falrun} \ar@/u/[r] & \resource{input/\\output} \ar@/d/[l]}
\seefalse
}

\providecommand{\falcpp}{
\toolsection{falcpp} is a transpiler for the FALSE programming language.
It translates programs written in FALSE into \cpp{} programs and stores them in corresponding source files.
\flowgraph{\resource{FALSE\\source code} \ar[r] & \toolbox{falcpp} \ar[r] & \resource{\cpp{}\\source file}}
\seefalse\seecpp
}

\providecommand{\falcode}{
\toolsection{falcode} is an intermediate code generator for the FALSE programming language.
It generates intermediate code from programs written in FALSE and stores it in corresponding assembly files.
\debuggingtool
\flowgraph{\resource{FALSE\\source code} \ar[r] & \toolbox{falcode} \ar[r] & \resource{intermediate\\code}}
\seefalse\seeassembly\seecode
}

\providecommand{\falamda}{
\toolsection{falamd16} is a compiler for the FALSE programming language targeting the AMD64 hardware architecture.
It generates machine code for AMD64 processors from programs written in FALSE and stores it in corresponding object files.
The compiler generates machine code for the 16-bit operating mode defined by the AMD64 architecture.
\flowgraph{\resource{FALSE\\source code} \ar[r] & \toolbox{falamd16} \ar[r] & \resource{object file}}
\seefalse\seeamd\seeobject
}

\providecommand{\falamdb}{
\toolsection{falamd32} is a compiler for the FALSE programming language targeting the AMD64 hardware architecture.
It generates machine code for AMD64 processors from programs written in FALSE and stores it in corresponding object files.
The compiler generates machine code for the 32-bit operating mode defined by the AMD64 architecture.
\flowgraph{\resource{FALSE\\source code} \ar[r] & \toolbox{falamd32} \ar[r] & \resource{object file}}
\seefalse\seeamd\seeobject
}

\providecommand{\falamdc}{
\toolsection{falamd64} is a compiler for the FALSE programming language targeting the AMD64 hardware architecture.
It generates machine code for AMD64 processors from programs written in FALSE and stores it in corresponding object files.
The compiler generates machine code for the 64-bit operating mode defined by the AMD64 architecture.
\flowgraph{\resource{FALSE\\source code} \ar[r] & \toolbox{falamd64} \ar[r] & \resource{object file}}
\seefalse\seeamd\seeobject
}

\providecommand{\falarma}{
\toolsection{falarma32} is a compiler for the FALSE programming language targeting the ARM hardware architecture.
It generates machine code for ARM processors executing A32 instructions from programs written in FALSE and stores it in corresponding object files.
\flowgraph{\resource{FALSE\\source code} \ar[r] & \toolbox{falarma32} \ar[r] & \resource{object file}}
\seefalse\seearm\seeobject
}

\providecommand{\falarmb}{
\toolsection{falarma64} is a compiler for the FALSE programming language targeting the ARM hardware architecture.
It generates machine code for ARM processors executing A64 instructions from programs written in FALSE and stores it in corresponding object files.
\flowgraph{\resource{FALSE\\source code} \ar[r] & \toolbox{falarma64} \ar[r] & \resource{object file}}
\seefalse\seearm\seeobject
}

\providecommand{\falarmc}{
\toolsection{falarmt32} is a compiler for the FALSE programming language targeting the ARM hardware architecture.
It generates machine code for ARM processors without floating-point extension executing T32 instructions from programs written in FALSE and stores it in corresponding object files.
\flowgraph{\resource{FALSE\\source code} \ar[r] & \toolbox{falarmt32} \ar[r] & \resource{object file}}
\seefalse\seearm\seeobject
}

\providecommand{\falarmcfpe}{
\toolsection{falarmt32fpe} is a compiler for the FALSE programming language targeting the ARM hardware architecture.
It generates machine code for ARM processors with floating-point extension executing T32 instructions from programs written in FALSE and stores it in corresponding object files.
\flowgraph{\resource{FALSE\\source code} \ar[r] & \toolbox{falarmt32fpe} \ar[r] & \resource{object file}}
\seefalse\seearm\seeobject
}

\providecommand{\falavr}{
\toolsection{falavr} is a compiler for the FALSE programming language targeting the AVR hardware architecture.
It generates machine code for AVR processors from programs written in FALSE and stores it in corresponding object files.
\flowgraph{\resource{FALSE\\source code} \ar[r] & \toolbox{falavr} \ar[r] & \resource{object file}}
\seefalse\seeavr\seeobject
}

\providecommand{\falavrtt}{
\toolsection{falavr32} is a compiler for the FALSE programming language targeting the AVR32 hardware architecture.
It generates machine code for AVR32 processors from programs written in FALSE and stores it in corresponding object files.
\flowgraph{\resource{FALSE\\source code} \ar[r] & \toolbox{falavr32} \ar[r] & \resource{object file}}
\seefalse\seeavrtt\seeobject
}

\providecommand{\falmabk}{
\toolsection{falm68k} is a compiler for the FALSE programming language targeting the M68000 hardware architecture.
It generates machine code for M68000 processors from programs written in FALSE and stores it in corresponding object files.
\flowgraph{\resource{FALSE\\source code} \ar[r] & \toolbox{falm68k} \ar[r] & \resource{object file}}
\seefalse\seemabk\seeobject
}

\providecommand{\falmibl}{
\toolsection{falmibl} is a compiler for the FALSE programming language targeting the MicroBlaze hardware architecture.
It generates machine code for MicroBlaze processors from programs written in FALSE and stores it in corresponding object files.
\flowgraph{\resource{FALSE\\source code} \ar[r] & \toolbox{falmibl} \ar[r] & \resource{object file}}
\seefalse\seemibl\seeobject
}

\providecommand{\falmipsa}{
\toolsection{falmips32} is a compiler for the FALSE programming language targeting the MIPS32 hardware architecture.
It generates machine code for MIPS32 processors from programs written in FALSE and stores it in corresponding object files.
\flowgraph{\resource{FALSE\\source code} \ar[r] & \toolbox{falmips32} \ar[r] & \resource{object file}}
\seefalse\seemips\seeobject
}

\providecommand{\falmipsb}{
\toolsection{falmips64} is a compiler for the FALSE programming language targeting the MIPS64 hardware architecture.
It generates machine code for MIPS64 processors from programs written in FALSE and stores it in corresponding object files.
\flowgraph{\resource{FALSE\\source code} \ar[r] & \toolbox{falmips64} \ar[r] & \resource{object file}}
\seefalse\seemips\seeobject
}

\providecommand{\falmmix}{
\toolsection{falmmix} is a compiler for the FALSE programming language targeting the MMIX hardware architecture.
It generates machine code for MMIX processors from programs written in FALSE and stores it in corresponding object files.
\flowgraph{\resource{FALSE\\source code} \ar[r] & \toolbox{falmmix} \ar[r] & \resource{object file}}
\seefalse\seemmix\seeobject
}

\providecommand{\falorok}{
\toolsection{falor1k} is a compiler for the FALSE programming language targeting the OpenRISC 1000 hardware architecture.
It generates machine code for OpenRISC 1000 processors from programs written in FALSE and stores it in corresponding object files.
\flowgraph{\resource{FALSE\\source code} \ar[r] & \toolbox{falor1k} \ar[r] & \resource{object file}}
\seefalse\seeorok\seeobject
}

\providecommand{\falppca}{
\toolsection{falppc32} is a compiler for the FALSE programming language targeting the PowerPC hardware architecture.
It generates machine code for PowerPC processors from programs written in FALSE and stores it in corresponding object files.
The compiler generates machine code for the 32-bit operating mode defined by the PowerPC architecture.
\flowgraph{\resource{FALSE\\source code} \ar[r] & \toolbox{falppc32} \ar[r] & \resource{object file}}
\seefalse\seeppc\seeobject
}

\providecommand{\falppcb}{
\toolsection{falppc64} is a compiler for the FALSE programming language targeting the PowerPC hardware architecture.
It generates machine code for PowerPC processors from programs written in FALSE and stores it in corresponding object files.
The compiler generates machine code for the 64-bit operating mode defined by the PowerPC architecture.
\flowgraph{\resource{FALSE\\source code} \ar[r] & \toolbox{falppc64} \ar[r] & \resource{object file}}
\seefalse\seeppc\seeobject
}

\providecommand{\falrisc}{
\toolsection{falrisc} is a compiler for the FALSE programming language targeting the RISC hardware architecture.
It generates machine code for RISC processors from programs written in FALSE and stores it in corresponding object files.
\flowgraph{\resource{FALSE\\source code} \ar[r] & \toolbox{falrisc} \ar[r] & \resource{object file}}
\seefalse\seerisc\seeobject
}

\providecommand{\falwasm}{
\toolsection{falwasm} is a compiler for the FALSE programming language targeting the WebAssembly architecture.
It generates machine code for WebAssembly targets from programs written in FALSE and stores it in corresponding object files.
\flowgraph{\resource{FALSE\\source code} \ar[r] & \toolbox{falwasm} \ar[r] & \resource{object file}}
\seefalse\seewasm\seeobject
}

% Oberon tools

\providecommand{\obprint}{
\toolsection{obprint} is a pretty printer for the Oberon programming language.
It reformats the source code of Oberon modules and writes it to the standard output stream.
\flowgraph{\resource{Oberon\\source code} \ar[r] & \toolbox{obprint} \ar[r] & \resource{reformatted\\source code}}
\seeoberon
}

\providecommand{\obcheck}{
\toolsection{obcheck} is a syntactic and semantic checker for the Oberon programming language.
It just performs syntactic and semantic checks on Oberon modules and writes its diagnostic messages to the standard error stream.
In addition, it stores the interface of each module in a symbol file which is required when other modules import the module.
\flowgraph{\resource{Oberon\\source code} \ar[r] & \toolbox{obcheck} \ar[r] \ar@/l/[d] & \resource{diagnostic\\messages} \\ \variable{ECSIMPORT} \ar[ru] & \resource{symbol\\files} \ar@/r/[u]}
\seeoberon
}

\providecommand{\obdump}{
\toolsection{obdump} is a serializer for the Oberon programming language.
It dumps the complete internal representation of modules written in Oberon into an XML document.
\debuggingtool
\flowgraph{\resource{Oberon\\source code} \ar[r] & \toolbox{obdump} \ar[r] \ar@/l/[d] & \resource{internal\\representation} \\ \variable{ECSIMPORT} \ar[ru] & \resource{symbol\\files} \ar@/r/[u]}
\seeoberon
}

\providecommand{\obrun}{
\toolsection{obrun} is an interpreter for the Oberon programming language.
It processes and executes modules written in Oberon.
This tool does neither generate nor process symbol files while interpreting modules.
If a module is imported by another one, its filename has to be named before the other one in the list of command-line arguments.
\flowgraph{\resource{Oberon\\source code} \ar[r] & \toolbox{obrun} \ar@/u/[r] & \resource{input/\\output} \ar@/d/[l]}
\seeoberon
}

\providecommand{\obcpp}{
\toolsection{obcpp} is a transpiler for the Oberon programming language.
It translates programs written in Oberon into \cpp{} programs and stores them in corresponding source and header files.
In addition, it stores the interface of each module in a symbol file which is required when other modules import the module.
The same interface is provided by the generated header file which can be used in other parts of the \cpp{} program.
\flowgraph{\resource{Oberon\\source code} \ar[r] & \toolbox{obcpp} \ar[r] \ar@/l/[d] \ar[rd] & \resource{\cpp{}\\source file} \\ \variable{ECSIMPORT} \ar[ru] & \resource{symbol\\files} \ar@/r/[u] & \resource{\cpp{}\\header file}}
\seeoberon\seecpp
}

\providecommand{\obdoc}{
\toolsection{obdoc} is a generic documentation generator for the Oberon programming language.
It processes several Oberon modules and assembles all information therein into a generic documentation.
In addition, it stores the interface of each module in a symbol file which is required when other modules import the module.
\debuggingtool
\flowgraph{\resource{Oberon\\source code} \ar[r] & \toolbox{obdoc} \ar[r] \ar@/l/[d] & \resource{generic\\documentation} \\ \variable{ECSIMPORT} \ar[ru] & \resource{symbol\\files} \ar@/r/[u]}
\seeoberon\seedocumentation
}

\providecommand{\obhtml}{
\toolsection{obhtml} is an HTML documentation generator for the Oberon programming language.
It processes several Oberon modules and assembles all information therein into an HTML document.
In addition, it stores the interface of each module in a symbol file which is required when other modules import the module.
\flowgraph{\resource{Oberon\\source code} \ar[r] & \toolbox{obhtml} \ar[r] \ar@/l/[d] & \resource{HTML\\document} \\ \variable{ECSIMPORT} \ar[ru] & \resource{symbol\\files} \ar@/r/[u]}
\seeoberon\seedocumentation
}

\providecommand{\oblatex}{
\toolsection{oblatex} is a Latex documentation generator for the Oberon programming language.
It processes several Oberon modules and assembles all information therein into a Latex document.
In addition, it stores the interface of each module in a symbol file which is required when other modules import the module.
\flowgraph{\resource{Oberon\\source code} \ar[r] & \toolbox{oblatex} \ar[r] \ar@/l/[d] & \resource{Latex\\document} \\ \variable{ECSIMPORT} \ar[ru] & \resource{symbol\\files} \ar@/r/[u]}
\seeoberon\seedocumentation
}

\providecommand{\obcode}{
\toolsection{obcode} is an intermediate code generator for the Oberon programming language.
It generates intermediate code from modules written in Oberon and stores it in corresponding assembly files.
In addition, it stores the interface of each module in a symbol file which is required when other modules import the module.
Programs generated with this tool require additional runtime support that is stored in the \file{ob\-code\-run} library file.
\debuggingtool
\flowgraph{\resource{Oberon\\source code} \ar[r] & \toolbox{obcode} \ar[r] \ar@/l/[d] & \resource{intermediate\\code} \\ \variable{ECSIMPORT} \ar[ru] & \resource{symbol\\files} \ar@/r/[u]}
\seeoberon\seeassembly\seecode
}

\providecommand{\obamda}{
\toolsection{obamd16} is a compiler for the Oberon programming language targeting the AMD64 hardware architecture.
It generates machine code for AMD64 processors from modules written in Oberon and stores it in corresponding object files.
The compiler generates machine code for the 16-bit operating mode defined by the AMD64 architecture.
For debugging purposes, it also creates a debugging information file as well as an assembly file containing a listing of the generated machine code.
In addition, it stores the interface of each module in a symbol file which is required when other modules import the module.
Programs generated with this compiler require additional runtime support that is stored in the \file{ob\-amd16\-run} library file.
\flowgraph{\resource{Oberon\\source code} \ar[r] & \toolbox{obamd16} \ar[r] \ar@/l/[d] \ar[rd] & \resource{object file} \\ \variable{ECSIMPORT} \ar[ru] & \resource{symbol\\files} \ar@/r/[u] & \resource{debugging\\information}}
\seeoberon\seeassembly\seeamd\seeobject\seedebugging
}

\providecommand{\obamdb}{
\toolsection{obamd32} is a compiler for the Oberon programming language targeting the AMD64 hardware architecture.
It generates machine code for AMD64 processors from modules written in Oberon and stores it in corresponding object files.
The compiler generates machine code for the 32-bit operating mode defined by the AMD64 architecture.
For debugging purposes, it also creates a debugging information file as well as an assembly file containing a listing of the generated machine code.
In addition, it stores the interface of each module in a symbol file which is required when other modules import the module.
Programs generated with this compiler require additional runtime support that is stored in the \file{ob\-amd32\-run} library file.
\flowgraph{\resource{Oberon\\source code} \ar[r] & \toolbox{obamd32} \ar[r] \ar@/l/[d] \ar[rd] & \resource{object file} \\ \variable{ECSIMPORT} \ar[ru] & \resource{symbol\\files} \ar@/r/[u] & \resource{debugging\\information}}
\seeoberon\seeassembly\seeamd\seeobject\seedebugging
}

\providecommand{\obamdc}{
\toolsection{obamd64} is a compiler for the Oberon programming language targeting the AMD64 hardware architecture.
It generates machine code for AMD64 processors from modules written in Oberon and stores it in corresponding object files.
The compiler generates machine code for the 64-bit operating mode defined by the AMD64 architecture.
For debugging purposes, it also creates a debugging information file as well as an assembly file containing a listing of the generated machine code.
In addition, it stores the interface of each module in a symbol file which is required when other modules import the module.
Programs generated with this compiler require additional runtime support that is stored in the \file{ob\-amd64\-run} library file.
\flowgraph{\resource{Oberon\\source code} \ar[r] & \toolbox{obamd64} \ar[r] \ar@/l/[d] \ar[rd] & \resource{object file} \\ \variable{ECSIMPORT} \ar[ru] & \resource{symbol\\files} \ar@/r/[u] & \resource{debugging\\information}}
\seeoberon\seeassembly\seeamd\seeobject\seedebugging
}

\providecommand{\obarma}{
\toolsection{obarma32} is a compiler for the Oberon programming language targeting the ARM hardware architecture.
It generates machine code for ARM processors executing A32 instructions from modules written in Oberon and stores it in corresponding object files.
For debugging purposes, it also creates a debugging information file as well as an assembly file containing a listing of the generated machine code.
In addition, it stores the interface of each module in a symbol file which is required when other modules import the module.
Programs generated with this compiler require additional runtime support that is stored in the \file{ob\-arma32\-run} library file.
\flowgraph{\resource{Oberon\\source code} \ar[r] & \toolbox{obarma32} \ar[r] \ar@/l/[d] \ar[rd] & \resource{object file} \\ \variable{ECSIMPORT} \ar[ru] & \resource{symbol\\files} \ar@/r/[u] & \resource{debugging\\information}}
\seeoberon\seeassembly\seearm\seeobject\seedebugging
}

\providecommand{\obarmb}{
\toolsection{obarma64} is a compiler for the Oberon programming language targeting the ARM hardware architecture.
It generates machine code for ARM processors executing A64 instructions from modules written in Oberon and stores it in corresponding object files.
For debugging purposes, it also creates a debugging information file as well as an assembly file containing a listing of the generated machine code.
In addition, it stores the interface of each module in a symbol file which is required when other modules import the module.
Programs generated with this compiler require additional runtime support that is stored in the \file{ob\-arma64\-run} library file.
\flowgraph{\resource{Oberon\\source code} \ar[r] & \toolbox{obarma64} \ar[r] \ar@/l/[d] \ar[rd] & \resource{object file} \\ \variable{ECSIMPORT} \ar[ru] & \resource{symbol\\files} \ar@/r/[u] & \resource{debugging\\information}}
\seeoberon\seeassembly\seearm\seeobject\seedebugging
}

\providecommand{\obarmc}{
\toolsection{obarmt32} is a compiler for the Oberon programming language targeting the ARM hardware architecture.
It generates machine code for ARM processors without floating-point extension executing T32 instructions from modules written in Oberon and stores it in corresponding object files.
For debugging purposes, it also creates a debugging information file as well as an assembly file containing a listing of the generated machine code.
In addition, it stores the interface of each module in a symbol file which is required when other modules import the module.
Programs generated with this compiler require additional runtime support that is stored in the \file{ob\-armt32\-run} library file.
\flowgraph{\resource{Oberon\\source code} \ar[r] & \toolbox{obarmt32} \ar[r] \ar@/l/[d] \ar[rd] & \resource{object file} \\ \variable{ECSIMPORT} \ar[ru] & \resource{symbol\\files} \ar@/r/[u] & \resource{debugging\\information}}
\seeoberon\seeassembly\seearm\seeobject\seedebugging
}

\providecommand{\obarmcfpe}{
\toolsection{obarmt32fpe} is a compiler for the Oberon programming language targeting the ARM hardware architecture.
It generates machine code for ARM processors with floating-point extension executing T32 instructions from modules written in Oberon and stores it in corresponding object files.
For debugging purposes, it also creates a debugging information file as well as an assembly file containing a listing of the generated machine code.
In addition, it stores the interface of each module in a symbol file which is required when other modules import the module.
Programs generated with this compiler require additional runtime support that is stored in the \file{ob\-armt32\-fpe\-run} library file.
\flowgraph{\resource{Oberon\\source code} \ar[r] & \toolbox{obarmt32fpe} \ar[r] \ar@/l/[d] \ar[rd] & \resource{object file} \\ \variable{ECSIMPORT} \ar[ru] & \resource{symbol\\files} \ar@/r/[u] & \resource{debugging\\information}}
\seeoberon\seeassembly\seearm\seeobject\seedebugging
}

\providecommand{\obavr}{
\toolsection{obavr} is a compiler for the Oberon programming language targeting the AVR hardware architecture.
It generates machine code for AVR processors from modules written in Oberon and stores it in corresponding object files.
For debugging purposes, it also creates a debugging information file as well as an assembly file containing a listing of the generated machine code.
In addition, it stores the interface of each module in a symbol file which is required when other modules import the module.
Programs generated with this compiler require additional runtime support that is stored in the \file{ob\-avr\-run} library file.
\flowgraph{\resource{Oberon\\source code} \ar[r] & \toolbox{obavr} \ar[r] \ar@/l/[d] \ar[rd] & \resource{object file} \\ \variable{ECSIMPORT} \ar[ru] & \resource{symbol\\files} \ar@/r/[u] & \resource{debugging\\information}}
\seeoberon\seeassembly\seeavr\seeobject\seedebugging
}

\providecommand{\obavrtt}{
\toolsection{obavr32} is a compiler for the Oberon programming language targeting the AVR32 hardware architecture.
It generates machine code for AVR32 processors from modules written in Oberon and stores it in corresponding object files.
For debugging purposes, it also creates a debugging information file as well as an assembly file containing a listing of the generated machine code.
In addition, it stores the interface of each module in a symbol file which is required when other modules import the module.
Programs generated with this compiler require additional runtime support that is stored in the \file{ob\-avr32\-run} library file.
\flowgraph{\resource{Oberon\\source code} \ar[r] & \toolbox{obavr32} \ar[r] \ar@/l/[d] \ar[rd] & \resource{object file} \\ \variable{ECSIMPORT} \ar[ru] & \resource{symbol\\files} \ar@/r/[u] & \resource{debugging\\information}}
\seeoberon\seeassembly\seeavrtt\seeobject\seedebugging
}

\providecommand{\obmabk}{
\toolsection{obm68k} is a compiler for the Oberon programming language targeting the M68000 hardware architecture.
It generates machine code for M68000 processors from modules written in Oberon and stores it in corresponding object files.
For debugging purposes, it also creates a debugging information file as well as an assembly file containing a listing of the generated machine code.
In addition, it stores the interface of each module in a symbol file which is required when other modules import the module.
Programs generated with this compiler require additional runtime support that is stored in the \file{ob\-m68k\-run} library file.
\flowgraph{\resource{Oberon\\source code} \ar[r] & \toolbox{obm68k} \ar[r] \ar@/l/[d] \ar[rd] & \resource{object file} \\ \variable{ECSIMPORT} \ar[ru] & \resource{symbol\\files} \ar@/r/[u] & \resource{debugging\\information}}
\seeoberon\seeassembly\seemabk\seeobject\seedebugging
}

\providecommand{\obmibl}{
\toolsection{obmibl} is a compiler for the Oberon programming language targeting the MicroBlaze hardware architecture.
It generates machine code for MicroBlaze processors from modules written in Oberon and stores it in corresponding object files.
For debugging purposes, it also creates a debugging information file as well as an assembly file containing a listing of the generated machine code.
In addition, it stores the interface of each module in a symbol file which is required when other modules import the module.
Programs generated with this compiler require additional runtime support that is stored in the \file{ob\-mibl\-run} library file.
\flowgraph{\resource{Oberon\\source code} \ar[r] & \toolbox{obmibl} \ar[r] \ar@/l/[d] \ar[rd] & \resource{object file} \\ \variable{ECSIMPORT} \ar[ru] & \resource{symbol\\files} \ar@/r/[u] & \resource{debugging\\information}}
\seeoberon\seeassembly\seemibl\seeobject\seedebugging
}

\providecommand{\obmipsa}{
\toolsection{obmips32} is a compiler for the Oberon programming language targeting the MIPS32 hardware architecture.
It generates machine code for MIPS32 processors from modules written in Oberon and stores it in corresponding object files.
For debugging purposes, it also creates a debugging information file as well as an assembly file containing a listing of the generated machine code.
In addition, it stores the interface of each module in a symbol file which is required when other modules import the module.
Programs generated with this compiler require additional runtime support that is stored in the \file{ob\-mips32\-run} library file.
\flowgraph{\resource{Oberon\\source code} \ar[r] & \toolbox{obmips32} \ar[r] \ar@/l/[d] \ar[rd] & \resource{object file} \\ \variable{ECSIMPORT} \ar[ru] & \resource{symbol\\files} \ar@/r/[u] & \resource{debugging\\information}}
\seeoberon\seeassembly\seemips\seeobject\seedebugging
}

\providecommand{\obmipsb}{
\toolsection{obmips64} is a compiler for the Oberon programming language targeting the MIPS64 hardware architecture.
It generates machine code for MIPS64 processors from modules written in Oberon and stores it in corresponding object files.
For debugging purposes, it also creates a debugging information file as well as an assembly file containing a listing of the generated machine code.
In addition, it stores the interface of each module in a symbol file which is required when other modules import the module.
Programs generated with this compiler require additional runtime support that is stored in the \file{ob\-mips64\-run} library file.
\flowgraph{\resource{Oberon\\source code} \ar[r] & \toolbox{obmips64} \ar[r] \ar@/l/[d] \ar[rd] & \resource{object file} \\ \variable{ECSIMPORT} \ar[ru] & \resource{symbol\\files} \ar@/r/[u] & \resource{debugging\\information}}
\seeoberon\seeassembly\seemips\seeobject\seedebugging
}

\providecommand{\obmmix}{
\toolsection{obmmix} is a compiler for the Oberon programming language targeting the MMIX hardware architecture.
It generates machine code for MMIX processors from modules written in Oberon and stores it in corresponding object files.
For debugging purposes, it also creates a debugging information file as well as an assembly file containing a listing of the generated machine code.
In addition, it stores the interface of each module in a symbol file which is required when other modules import the module.
Programs generated with this compiler require additional runtime support that is stored in the \file{ob\-mmix\-run} library file.
\flowgraph{\resource{Oberon\\source code} \ar[r] & \toolbox{obmmix} \ar[r] \ar@/l/[d] \ar[rd] & \resource{object file} \\ \variable{ECSIMPORT} \ar[ru] & \resource{symbol\\files} \ar@/r/[u] & \resource{debugging\\information}}
\seeoberon\seeassembly\seemmix\seeobject\seedebugging
}

\providecommand{\oborok}{
\toolsection{obor1k} is a compiler for the Oberon programming language targeting the OpenRISC 1000 hardware architecture.
It generates machine code for OpenRISC 1000 processors from modules written in Oberon and stores it in corresponding object files.
For debugging purposes, it also creates a debugging information file as well as an assembly file containing a listing of the generated machine code.
In addition, it stores the interface of each module in a symbol file which is required when other modules import the module.
Programs generated with this compiler require additional runtime support that is stored in the \file{ob\-or1k\-run} library file.
\flowgraph{\resource{Oberon\\source code} \ar[r] & \toolbox{obor1k} \ar[r] \ar@/l/[d] \ar[rd] & \resource{object file} \\ \variable{ECSIMPORT} \ar[ru] & \resource{symbol\\files} \ar@/r/[u] & \resource{debugging\\information}}
\seeoberon\seeassembly\seeorok\seeobject\seedebugging
}

\providecommand{\obppca}{
\toolsection{obppc32} is a compiler for the Oberon programming language targeting the PowerPC hardware architecture.
It generates machine code for PowerPC processors from modules written in Oberon and stores it in corresponding object files.
The compiler generates machine code for the 32-bit operating mode defined by the PowerPC architecture.
For debugging purposes, it also creates a debugging information file as well as an assembly file containing a listing of the generated machine code.
In addition, it stores the interface of each module in a symbol file which is required when other modules import the module.
Programs generated with this compiler require additional runtime support that is stored in the \file{ob\-ppc32\-run} library file.
\flowgraph{\resource{Oberon\\source code} \ar[r] & \toolbox{obppc32} \ar[r] \ar@/l/[d] \ar[rd] & \resource{object file} \\ \variable{ECSIMPORT} \ar[ru] & \resource{symbol\\files} \ar@/r/[u] & \resource{debugging\\information}}
\seeoberon\seeassembly\seeppc\seeobject\seedebugging
}

\providecommand{\obppcb}{
\toolsection{obppc64} is a compiler for the Oberon programming language targeting the PowerPC hardware architecture.
It generates machine code for PowerPC processors from modules written in Oberon and stores it in corresponding object files.
The compiler generates machine code for the 64-bit operating mode defined by the PowerPC architecture.
For debugging purposes, it also creates a debugging information file as well as an assembly file containing a listing of the generated machine code.
In addition, it stores the interface of each module in a symbol file which is required when other modules import the module.
Programs generated with this compiler require additional runtime support that is stored in the \file{ob\-ppc64\-run} library file.
\flowgraph{\resource{Oberon\\source code} \ar[r] & \toolbox{obppc64} \ar[r] \ar@/l/[d] \ar[rd] & \resource{object file} \\ \variable{ECSIMPORT} \ar[ru] & \resource{symbol\\files} \ar@/r/[u] & \resource{debugging\\information}}
\seeoberon\seeassembly\seeppc\seeobject\seedebugging
}

\providecommand{\obrisc}{
\toolsection{obrisc} is a compiler for the Oberon programming language targeting the RISC hardware architecture.
It generates machine code for RISC processors from modules written in Oberon and stores it in corresponding object files.
For debugging purposes, it also creates a debugging information file as well as an assembly file containing a listing of the generated machine code.
In addition, it stores the interface of each module in a symbol file which is required when other modules import the module.
Programs generated with this compiler require additional runtime support that is stored in the \file{ob\-risc\-run} library file.
\flowgraph{\resource{Oberon\\source code} \ar[r] & \toolbox{obrisc} \ar[r] \ar@/l/[d] \ar[rd] & \resource{object file} \\ \variable{ECSIMPORT} \ar[ru] & \resource{symbol\\files} \ar@/r/[u] & \resource{debugging\\information}}
\seeoberon\seeassembly\seerisc\seeobject\seedebugging
}

\providecommand{\obwasm}{
\toolsection{obwasm} is a compiler for the Oberon programming language targeting the WebAssembly architecture.
It generates machine code for WebAssembly targets from modules written in Oberon and stores it in corresponding object files.
For debugging purposes, it also creates a debugging information file as well as an assembly file containing a listing of the generated machine code.
In addition, it stores the interface of each module in a symbol file which is required when other modules import the module.
Programs generated with this compiler require additional runtime support that is stored in the \file{ob\-wasm\-run} library file.
\flowgraph{\resource{Oberon\\source code} \ar[r] & \toolbox{obwasm} \ar[r] \ar@/l/[d] \ar[rd] & \resource{object file} \\ \variable{ECSIMPORT} \ar[ru] & \resource{symbol\\files} \ar@/r/[u] & \resource{debugging\\information}}
\seeoberon\seeassembly\seewasm\seeobject\seedebugging
}

% converter tools

\providecommand{\dbgdwarf}{
\toolsection{dbgdwarf} is a DWARF debugging information converter tool.
It converts debugging information into the DWARF debugging data format and stores it in corresponding object files~\cite{dwarffile}.
The resulting debugging object files can be combined with runtime support that creates Executable and Linking Format (ELF) files~\cite{elffile}.
\flowgraph{\resource{debugging\\information} \ar[r] & \toolbox{dbgdwarf} \ar[r] & \resource{debugging\\object file}}
\seeobject\seedebugging
}

% assembler tools

\providecommand{\asmprint}{
\toolsection{asmprint} is a pretty printer for generic assembly code.
It reformats generic assembly code and writes it to the standard output stream.
\flowgraph{\resource{generic assembly\\source code} \ar[r] & \toolbox{asmprint} \ar[r] & \resource{reformatted\\source code}}
\seeassembly
}

\providecommand{\amdaasm}{
\toolsection{amd16asm} is an assembler for the AMD64 hardware architecture.
It translates assembly code into machine code for AMD64 processors and stores it in corresponding object files.
By default, the assembler generates machine code for the 16-bit operating mode defined by the AMD64 architecture.
\flowgraph{\resource{AMD16 assembly\\source code} \ar[r] & \toolbox{amd16asm} \ar[r] & \resource{object file}}
\seeassembly\seeamd\seeobject
}

\providecommand{\amdadism}{
\toolsection{amd16dism} is a disassembler for the AMD64 hardware architecture.
It translates machine code from object files targeting AMD64 processors into assembly code and writes it to the standard output stream.
It assumes that the machine code was generated for the 16-bit operating mode defined by the AMD64 architecture.
\flowgraph{\resource{object file} \ar[r] & \toolbox{amd16dism} \ar[r] & \resource{disassembly\\listing}}
\seeassembly\seeamd\seeobject
}

\providecommand{\amdbasm}{
\toolsection{amd32asm} is an assembler for the AMD64 hardware architecture.
It translates assembly code into machine code for AMD64 processors and stores it in corresponding object files.
By default, the assembler generates machine code for the 32-bit operating mode defined by the AMD64 architecture.
\flowgraph{\resource{AMD32 assembly\\source code} \ar[r] & \toolbox{amd32asm} \ar[r] & \resource{object file}}
\seeassembly\seeamd\seeobject
}

\providecommand{\amdbdism}{
\toolsection{amd32dism} is a disassembler for the AMD64 hardware architecture.
It translates machine code from object files targeting AMD64 processors into assembly code and writes it to the standard output stream.
It assumes that the machine code was generated for the 32-bit operating mode defined by the AMD64 architecture.
\flowgraph{\resource{object file} \ar[r] & \toolbox{amd32dism} \ar[r] & \resource{disassembly\\listing}}
\seeassembly\seeamd\seeobject
}

\providecommand{\amdcasm}{
\toolsection{amd64asm} is an assembler for the AMD64 hardware architecture.
It translates assembly code into machine code for AMD64 processors and stores it in corresponding object files.
By default, the assembler generates machine code for the 64-bit operating mode defined by the AMD64 architecture.
\flowgraph{\resource{AMD64 assembly\\source code} \ar[r] & \toolbox{amd64asm} \ar[r] & \resource{object file}}
\seeassembly\seeamd\seeobject
}

\providecommand{\amdcdism}{
\toolsection{amd64dism} is a disassembler for the AMD64 hardware architecture.
It translates machine code from object files targeting AMD64 processors into assembly code and writes it to the standard output stream.
It assumes that the machine code was generated for the 64-bit operating mode defined by the AMD64 architecture.
\flowgraph{\resource{object file} \ar[r] & \toolbox{amd64dism} \ar[r] & \resource{disassembly\\listing}}
\seeassembly\seeamd\seeobject
}

\providecommand{\armaasm}{
\toolsection{arma32asm} is an assembler for the ARM hardware architecture.
It translates assembly code into machine code for ARM processors executing A32 instructions and stores it in corresponding object files.
\flowgraph{\resource{ARM A32 assembly\\source code} \ar[r] & \toolbox{arma32asm} \ar[r] & \resource{object file}}
\seeassembly\seearm\seeobject
}

\providecommand{\armadism}{
\toolsection{arma32dism} is a disassembler for the ARM hardware architecture.
It translates machine code from object files targeting ARM processors executing A32 instructions into assembly code and writes it to the standard output stream.
\flowgraph{\resource{object file} \ar[r] & \toolbox{arma32dism} \ar[r] & \resource{disassembly\\listing}}
\seeassembly\seearm\seeobject
}

\providecommand{\armbasm}{
\toolsection{arma64asm} is an assembler for the ARM hardware architecture.
It translates assembly code into machine code for ARM processors executing A64 instructions and stores it in corresponding object files.
\flowgraph{\resource{ARM A64 assembly\\source code} \ar[r] & \toolbox{arma64asm} \ar[r] & \resource{object file}}
\seeassembly\seearm\seeobject
}

\providecommand{\armbdism}{
\toolsection{arma64dism} is a disassembler for the ARM hardware architecture.
It translates machine code from object files targeting ARM processors executing A64 instructions into assembly code and writes it to the standard output stream.
\flowgraph{\resource{object file} \ar[r] & \toolbox{arma64dism} \ar[r] & \resource{disassembly\\listing}}
\seeassembly\seearm\seeobject
}

\providecommand{\armcasm}{
\toolsection{armt32asm} is an assembler for the ARM hardware architecture.
It translates assembly code into machine code for ARM processors executing T32 instructions and stores it in corresponding object files.
\flowgraph{\resource{ARM T32 assembly\\source code} \ar[r] & \toolbox{armt32asm} \ar[r] & \resource{object file}}
\seeassembly\seearm\seeobject
}

\providecommand{\armcdism}{
\toolsection{armt32dism} is a disassembler for the ARM hardware architecture.
It translates machine code from object files targeting ARM processors executing T32 instructions into assembly code and writes it to the standard output stream.
\flowgraph{\resource{object file} \ar[r] & \toolbox{armt32dism} \ar[r] & \resource{disassembly\\listing}}
\seeassembly\seearm\seeobject
}

\providecommand{\avrasm}{
\toolsection{avrasm} is an assembler for the AVR hardware architecture.
It translates assembly code into machine code for AVR processors and stores it in corresponding object files.
The identifiers \texttt{RXL}, \texttt{RXH}, \texttt{RYL}, \texttt{RYH}, \texttt{RZL}, and \texttt{RZH} are predefined and name the corresponding registers.
The identifiers \texttt{SPL} and \texttt{SPH} are also predefined and evaluate to the address of the corresponding registers.
\flowgraph{\resource{AVR assembly\\source code} \ar[r] & \toolbox{avrasm} \ar[r] & \resource{object file}}
\seeassembly\seeavr\seeobject
}

\providecommand{\avrdism}{
\toolsection{avrdism} is a disassembler for the AVR hardware architecture.
It translates machine code from object files targeting AVR processors into assembly code and writes it to the standard output stream.
\flowgraph{\resource{object file} \ar[r] & \toolbox{avrdism} \ar[r] & \resource{disassembly\\listing}}
\seeassembly\seeavr\seeobject
}

\providecommand{\avrttasm}{
\toolsection{avr32asm} is an assembler for the AVR32 hardware architecture.
It translates assembly code into machine code for AVR32 processors and stores it in corresponding object files.
\flowgraph{\resource{AVR32 assembly\\source code} \ar[r] & \toolbox{avr32asm} \ar[r] & \resource{object file}}
\seeassembly\seeavrtt\seeobject
}

\providecommand{\avrttdism}{
\toolsection{avr32dism} is a disassembler for the AVR32 hardware architecture.
It translates machine code from object files targeting AVR32 processors into assembly code and writes it to the standard output stream.
\flowgraph{\resource{object file} \ar[r] & \toolbox{avr32dism} \ar[r] & \resource{disassembly\\listing}}
\seeassembly\seeavrtt\seeobject
}

\providecommand{\mabkasm}{
\toolsection{m68kasm} is an assembler for the M68000 hardware architecture.
It translates assembly code into machine code for M68000 processors and stores it in corresponding object files.
\flowgraph{\resource{68000 assembly\\source code} \ar[r] & \toolbox{m68kasm} \ar[r] & \resource{object file}}
\seeassembly\seemabk\seeobject
}

\providecommand{\mabkdism}{
\toolsection{m68kdism} is a disassembler for the M68000 hardware architecture.
It translates machine code from object files targeting M68000 processors into assembly code and writes it to the standard output stream.
\flowgraph{\resource{object file} \ar[r] & \toolbox{m68kdism} \ar[r] & \resource{disassembly\\listing}}
\seeassembly\seemabk\seeobject
}

\providecommand{\miblasm}{
\toolsection{miblasm} is an assembler for the MicroBlaze hardware architecture.
It translates assembly code into machine code for MicroBlaze processors and stores it in corresponding object files.
\flowgraph{\resource{MicroBlaze assembly\\source code} \ar[r] & \toolbox{miblasm} \ar[r] & \resource{object file}}
\seeassembly\seemibl\seeobject
}

\providecommand{\mibldism}{
\toolsection{mibldism} is a disassembler for the MicroBlaze hardware architecture.
It translates machine code from object files targeting MicroBlaze processors into assembly code and writes it to the standard output stream.
\flowgraph{\resource{object file} \ar[r] & \toolbox{mibldism} \ar[r] & \resource{disassembly\\listing}}
\seeassembly\seemibl\seeobject
}

\providecommand{\mipsaasm}{
\toolsection{mips32asm} is an assembler for the MIPS32 hardware architecture.
It translates assembly code into machine code for MIPS32 processors and stores it in corresponding object files.
\flowgraph{\resource{MIPS32 assembly\\source code} \ar[r] & \toolbox{mips32asm} \ar[r] & \resource{object file}}
\seeassembly\seemips\seeobject
}

\providecommand{\mipsadism}{
\toolsection{mips32dism} is a disassembler for the MIPS32 hardware architecture.
It translates machine code from object files targeting MIPS32 processors into assembly code and writes it to the standard output stream.
\flowgraph{\resource{object file} \ar[r] & \toolbox{mips32dism} \ar[r] & \resource{disassembly\\listing}}
\seeassembly\seemips\seeobject
}

\providecommand{\mipsbasm}{
\toolsection{mips64asm} is an assembler for the MIPS64 hardware architecture.
It translates assembly code into machine code for MIPS64 processors and stores it in corresponding object files.
\flowgraph{\resource{MIPS64 assembly\\source code} \ar[r] & \toolbox{mips64asm} \ar[r] & \resource{object file}}
\seeassembly\seemips\seeobject
}

\providecommand{\mipsbdism}{
\toolsection{mips64dism} is a disassembler for the MIPS64 hardware architecture.
It translates machine code from object files targeting MIPS64 processors into assembly code and writes it to the standard output stream.
\flowgraph{\resource{object file} \ar[r] & \toolbox{mips64dism} \ar[r] & \resource{disassembly\\listing}}
\seeassembly\seemips\seeobject
}

\providecommand{\mmixasm}{
\toolsection{mmixasm} is an assembler for the MMIX hardware architecture.
It translates assembly code into machine code for MMIX processors and stores it in corresponding object files.
The names of all special registers are predefined and evaluate to the corresponding number.
\flowgraph{\resource{MMIX assembly\\source code} \ar[r] & \toolbox{mmixasm} \ar[r] & \resource{object file}}
\seeassembly\seemmix\seeobject
}

\providecommand{\mmixdism}{
\toolsection{mmixdism} is a disassembler for the MMIX hardware architecture.
It translates machine code from object files targeting MMIX processors into assembly code and writes it to the standard output stream.
\flowgraph{\resource{object file} \ar[r] & \toolbox{mmixdism} \ar[r] & \resource{disassembly\\listing}}
\seeassembly\seemmix\seeobject
}

\providecommand{\orokasm}{
\toolsection{or1kasm} is an assembler for the OpenRISC 1000 hardware architecture.
It translates assembly code into machine code for OpenRISC 1000 processors and stores it in corresponding object files.
\flowgraph{\resource{OpenRISC 1000 assembly\\source code} \ar[r] & \toolbox{or1kasm} \ar[r] & \resource{object file}}
\seeassembly\seeorok\seeobject
}

\providecommand{\orokdism}{
\toolsection{or1kdism} is a disassembler for the OpenRISC 1000 hardware architecture.
It translates machine code from object files targeting OpenRISC 1000 processors into assembly code and writes it to the standard output stream.
\flowgraph{\resource{object file} \ar[r] & \toolbox{or1kdism} \ar[r] & \resource{disassembly\\listing}}
\seeassembly\seeorok\seeobject
}

\providecommand{\ppcaasm}{
\toolsection{ppc32asm} is an assembler for the PowerPC hardware architecture.
It translates assembly code into machine code for PowerPC processors and stores it in corresponding object files.
By default, the assembler generates machine code for the 32-bit operating mode defined by the PowerPC architecture.
\flowgraph{\resource{PowerPC assembly\\source code} \ar[r] & \toolbox{ppc32asm} \ar[r] & \resource{object file}}
\seeassembly\seeppc\seeobject
}

\providecommand{\ppcadism}{
\toolsection{ppc32dism} is a disassembler for the PowerPC hardware architecture.
It translates machine code from object files targeting PowerPC processors into assembly code and writes it to the standard output stream.
It assumes that the machine code was generated for the 32-bit operating mode defined by the PowerPC architecture.
\flowgraph{\resource{object file} \ar[r] & \toolbox{ppc32dism} \ar[r] & \resource{disassembly\\listing}}
\seeassembly\seeppc\seeobject
}

\providecommand{\ppcbasm}{
\toolsection{ppc64asm} is an assembler for the PowerPC hardware architecture.
It translates assembly code into machine code for PowerPC processors and stores it in corresponding object files.
By default, the assembler generates machine code for the 64-bit operating mode defined by the PowerPC architecture.
\flowgraph{\resource{PowerPC assembly\\source code} \ar[r] & \toolbox{ppc64asm} \ar[r] & \resource{object file}}
\seeassembly\seeppc\seeobject
}

\providecommand{\ppcbdism}{
\toolsection{ppc64dism} is a disassembler for the PowerPC hardware architecture.
It translates machine code from object files targeting PowerPC processors into assembly code and writes it to the standard output stream.
It assumes that the machine code was generated for the 64-bit operating mode defined by the PowerPC architecture.
\flowgraph{\resource{object file} \ar[r] & \toolbox{ppc64dism} \ar[r] & \resource{disassembly\\listing}}
\seeassembly\seeppc\seeobject
}

\providecommand{\riscasm}{
\toolsection{riscasm} is an assembler for the RISC hardware architecture.
It translates assembly code into machine code for RISC processors and stores it in corresponding object files.
The names of all special registers are predefined and evaluate to the corresponding number.
\flowgraph{\resource{RISC assembly\\source code} \ar[r] & \toolbox{riscasm} \ar[r] & \resource{object file}}
\seeassembly\seerisc\seeobject
}

\providecommand{\riscdism}{
\toolsection{riscdism} is a disassembler for the RISC hardware architecture.
It translates machine code from object files targeting RISC processors into assembly code and writes it to the standard output stream.
\flowgraph{\resource{object file} \ar[r] & \toolbox{riscdism} \ar[r] & \resource{disassembly\\listing}}
\seeassembly\seerisc\seeobject
}

\providecommand{\wasmasm}{
\toolsection{wasmasm} is an assembler for the WebAssembly architecture.
It translates assembly code into machine code for WebAssembly targets and stores it in corresponding object files.
The names of all special registers are predefined and evaluate to the corresponding number.
\flowgraph{\resource{WebAssembly assembly\\source code} \ar[r] & \toolbox{wasmasm} \ar[r] & \resource{object file}}
\seeassembly\seewasm\seeobject
}

\providecommand{\wasmdism}{
\toolsection{wasmdism} is a disassembler for the WebAssembly architecture.
It translates machine code from object files targeting WebAssembly targets into assembly code and writes it to the standard output stream.
\flowgraph{\resource{object file} \ar[r] & \toolbox{wasmdism} \ar[r] & \resource{disassembly\\listing}}
\seeassembly\seewasm\seeobject
}

% linker tools

\providecommand{\linklib}{
\toolsection{linklib} is an object file combiner.
It creates a static library file by combining all object files given to it into a single one.
\flowgraph{\resource{object files} \ar[r] & \toolbox{linklib} \ar[r] & \resource{library file}}
\seeobject
}

\providecommand{\linkbin}{
\toolsection{linkbin} is a linker for plain binary files.
It links all object files given to it into a single image and stores it in a binary file that begins with the first linked section.
It also creates a map file that lists the address, type, name and size of all used sections.
The filename extension of the resulting binary file can be specified by putting it into a constant data section called \texttt{\_extension}.
\flowgraph{\resource{object files} \ar[r] & \toolbox{linkbin} \ar[r] \ar[d] & \resource{binary file} \\ & \resource{map file}}
\seeobject
}

\providecommand{\linkmem}{
\toolsection{linkmem} is a linker for plain binary files partitioned into random-access and read-only memory.
It links all object files given to it into two distinct images, one for data sections and one for code and constant data sections, and stores each image in a binary file that begins with the first linked section of the corresponding type.
It also creates a map file that lists the address, type, name and size of all used sections.
\flowgraph{\resource{object files} \ar[r] & \toolbox{linkmem} \ar[r] \ar[d] & \resource{RAM file/\\ROM file} \\ & \resource{map file}}
\seeobject
}

\providecommand{\linkprg}{
\toolsection{linkprg} is a linker for GEMDOS executable files.
It links all object files given to it into a single image and stores the image in an Atari GEMDOS executable file~\cite{gemdosfile}.
It also creates a map file that lists the address relative to the text segment, type, name and size of all used sections.
The filename extension of the resulting executable file can be specified by putting it into a constant data section called \texttt{\_extension}.
The GEMDOS executable file format requires all patch patterns of absolute link patches to consist of four full bitmasks with descending offsets.
\flowgraph{\resource{object files} \ar[r] & \toolbox{linkprg} \ar[r] \ar[d] & \resource{executable file} \\ & \resource{map file}}
\seeobject
}

\providecommand{\linkhex}{
\toolsection{linkhex} is a linker for Intel HEX files.
It links all code sections of the object files given to it into single image and stores the image in an Intel HEX file~\cite{hexfile} that begins with the first linked section.
It also creates a map file that lists the address, type, name and size of all used sections.
\flowgraph{\resource{object files} \ar[r] & \toolbox{linkhex} \ar[r] \ar[d] & \resource{HEX file} \\ & \resource{map file}}
\seeobject
}

\providecommand{\mapsearch}{
\toolsection{mapsearch} is a debugging tool.
It searches map files generated by linker tools for the name of a binary section that encompasses a memory address read from the standard input stream.
If additionally provided with one or more object files, it also stores an excerpt thereof in a separate object file called map search result which only contains the identified binary section for disassembling purposes.
\flowgraph{& \resource{map files/\\object files} \ar[d] \\ \resource{memory\\address} \ar[r] & \toolbox{mapsearch} \ar[r] \ar[d] & \resource{section name/\\relative offset} \\ & \resource{object file\\excerpt}}
\seeobject
}


\startchapter{Extensions}{Extensions to the \ecs{}}{extensions}
{The \ecs{} features a variety of tools like compilers, assemblers, and linkers.
All of these tools implement different programming languages, target different hardware architectures, or support different runtime environments respectively.
This \documentation{} describes the abstractions and utilities provided by the \ecs{} that facilitate the development of additional tools and runtime environments.}

\epigraph{Nought may endure but mutability.}{Percy Bysshe Shelley}

\section{Introduction}

The \ecs{} is a complete tool chain targeting a variety of programming languages, hardware architectures, and runtime environments.
Internally, it features several abstractions that are helpful for programmers implementing these tools.
Figure~\ref{fig:extabstractions} shows all abstractions and visualizes possible combinations of the different tools.

\begin{figure}
\flowgraph{
\resource{source code} \ar[d] & \resource{source code} \ar[d] & \resource{source code} \ar[d] \\
\converter{Front-End\\for programming\\language \textit{A}} \ar[rd] & \converter{Front-End\\for programming\\language \textit{B}} \ar[d] & \converter{Front-End\\for programming\\language \textit{C}} \ar[ld] \\
& \resource{intermediate\\code} \ar[ld] \ar[d] & \resource{assembly\\source code} \ar[d] \\
\converter{Back-End\\for hardware\\architecture \textit{X}} \ar[d] \ar[rd] & \converter{Back-End\\for hardware\\architecture \textit{Y}} \ar[d] & \converter{Assembler\\for hardware\\architecture \textit{Z}} \ar[ld] \\
\resource {debugging\\information} & \resource{object\\file} \ar[ld] \ar[d] \ar[rd] \\
\converter{Linker\\for runtime\\environment \textit{1}} \ar[d] & \converter{Linker\\for runtime\\environment \textit{2}} \ar[d] & \converter{Disassembler\\for hardware\\architecture \textit{3}} \ar[d] \\
\resource{executable\\binary image} & \resource{executable\\binary image} & \resource{disassembly\\listing} \\
}\caption{The main abstractions of the \ecs{}}
\label{fig:extabstractions}
\end{figure}

The goal of the abstractions is to enable and simplify all possible combinations of the different aspects of the tool chain.
This is achieved by reducing the required programming interfaces to a minimum.
As a result, adding support for another programming language, hardware architecture, or runtime environment is also simplified.
The following sections describe the steps that are required to extend the \ecs{} into this direction.

\section{Diagnostics}

Compilers and assemblers as provided by the \ecs{} translate source code written by programmers.
Therefore, source code may contain syntax or semantic errors which have to be diagnosed by these tools.
The \ecs{} provides a generic diagnostics facility for programmers that enables a consistent reporting of errors, warnings, and additional information.
Besides the actual text, diagnostic messages also include the name of the diagnosed source code and the position therein.
Unless otherwise specified, the source code position is based on text lines.

\section{Application Drivers}

The \ecs{} provides a generic driver facility for programmers that allows all applications written with it to share the same user interface.
This framework covers the overall error handling as well as the processing scheme for all kinds of input given to the applications.
\interface

\section{Programming Languages}

This section describes the tools and representations typically provided by the \ecs{} for a specific programming language and how they are implemented.
Figure~\ref{fig:extdataflow} shows the tools and representations of typical implementations of programming languages.

\begin{figure}
\flowgraph{
& & \resource{source code} \ar[d] & & \\
& & \converter{Lexer} \ar[d] \\
& & \resource{tokens} \ar[d] \\
& & \converter{Parser} \ar[d] \\
*=<2em,0em>\txt{\rotatebox{90}{Front-End}} \ar@{-}`r[uuuu]`[rruuuu]+D \ar@{-}`r[dddd]`[rrdddd]+U & \converter{Serializer} \ar[d] & \resource{abstract\\syntax tree} \ar[l] \ar[d] \ar[r] & \converter{Pretty Printer} \ar[d] \\
& \resource{internal\\representation} & \converter{Semantic\\Checker} \ar[d] & \resource{reformatted\\source code} & *=<2em,0em>\txt{\rotatebox{270}{Compiler}} \ar@{-}`l[uuuuu]`[lluuuuu]+D \ar@{-}`l[ddddd]`[llddddd]+U \\
& \converter{Interpreter} \ar@/l/[d] & \resource{attributed\\syntax tree} \ar[l] \ar[d] \ar[r] & \converter{Transpiler} \ar[d] \\
& \resource{input/\\output} \ar@/r/[u] & \converter{Intermediate\\Code Emitter} \ar[d] & \resource{translated\\source code} \\
& & \resource{intermediate\\code} \ar[d] \ar@/u/[r] & \converter{Optimizer} \ar@/d/[l] \\
*=<2em,0em>\txt{\rotatebox{90}{Back-End}} \ar@{-}`r[u]`[rru]+D \ar@{-}`r[d]`[rrd]+U & \resource{assembly\\listing} & \converter{Machine Code\\Generator} \ar[l] \ar[d] \ar[r] & \resource{debugging\\information} \\
& & \resource{object file} & & \\
}\caption{Data flow within a typical implementation of a programming language}
\label{fig:extdataflow}
\end{figure}

The source code of a program written in a programming language is most often represented in an abstract syntax tree.
This tree is created by a \emph{parser}\index{Parsers} that recognizes the syntax of the programming language.
Parsers typically use a so-called \emph{lexer}\index{Lexers} that is able to tokenize the source code and extract symbols like keywords and operators.
The abstract syntax tree is the base for all tools described in the following sections.
Therefore, all of these tools contain a parser and a lexer in order to create the syntax tree.
Although these tools use the information represented in the syntax tree for different purposes, they all traverse it in one or several consecutive stages.

\subsection{Pretty Printers}

Pretty printers just traverse the complete abstract syntax tree by reconverting its nodes into tokens again.
These tokens are then printed using a consistent and well-arranged layout.
Pretty printers usually do not alter the abstract syntax tree and are often able to reformat source code that contains semantic errors.

\subsection{Semantic Checkers}

Semantic checkers traverse the abstract syntax tree and attribute its nodes with semantic information.
They are able to diagnose violations of the semantic rules of the programming language.
For that purpose, semantic checkers sometimes use additional semantic information that is stored separately.
The functionality provided by semantic checkers is in many cases reused by the remaining tools of this section.
Standalone semantic checkers are useful for automated testing and verification of the implementation of a programming language.

\subsection{Serializers}

Serializers dump all information stored in the internal representation of a program in a human readable format for debugging purposes.
The \ecs{} provides a generic serialization facility for representing attributed syntax trees as XML documents.
This serialization format has the advantage of being standardized and easy to parse for other development tools potentially making use of the same internal program representation.

\subsection{Interpreters}

Interpreters are able to execute the program given in the syntax tree and to emulate a complete runtime environment for it.
They do not translate the syntax tree into another form, but traverse its nodes by simulating their runtime behavior as defined by the programming language.
Interpreters are useful for automated testing and verification of the implementation of a programming language or executing scripts inside a program.
For this particular case the \ecs{} provides a generic framework that allows embedded interpreters to interface with their environment.

\subsection{Transpilers}

Transpilers translate programs written in one programming language into programs written in other programming languages.
They therefore behave like pretty printers, except that the target programming language differs from the source programming language.
Additionally, transpilers usually need the semantic information provided by semantic checkers in order to translate the source code correctly.

\subsection{Documentation Generators}

Documentation generators extract the structure of the source code and combine it with annotations provided by the programmer into generic documentations.
This generic representation of the extracted information is used afterward to generate documents of different formats.
\seedocumentation

\subsection{Intermediate Code Emitters}

The \ecs{} defines an intermediate code representation that is able to represent arbitrary programs using instructions for an abstract machine.
This intermediate representation can be translated into actual machine code by machine code generators, see Section~\ref{sec:extgenerators}
An intermediate code emitter traverses syntax trees and translates theirs nodes into intermediate code sections and instructions that correspond to the runtime behavior of the original programs.
The \ecs{} provides a generic intermediate code emitter that is a base for all concrete emitters and simplifies the generation of intermediate code.
\seecode

\subsection{Front-Ends}\label{sec:extfrontends}

A front-end combines all of the previous stages required to transform the original source code into an intermediate code representation thereof.
This most often includes the parser, the semantic checker, and the intermediate code emitter.
Since this intermediate representation can be translated later into machine code for a variety of hardware architectures,
programmers only have to provide a single front-end instead of one for each possible combination thereof.
In combination with the intermediate code interpreter, they are useful for automated testing and verification of the generated intermediate code.

\subsection{Compilers}

Compilers translate source code written in a specific programming language into machine code targeting a specific hardware architecture.
They therefore just combine the output of one specific front-end with the input for a specific back-end, see Sections~\ref{sec:extfrontends} and~\ref{sec:extbackends}.
In the end, compilers write the resulting machine code into object files.
\seeobject
If programmers add support for an additional programming language by implementing a front-end accordingly,
or if they support a new hardware architecture by implementing an additional back-end, compilers for all new combinations just get available without further ado.

\section{Hardware Architectures}

This section describes the tools and components typically provided by the \ecs{} for a specific hardware architecture and how they are implemented.
The support for a hardware architecture is generally based on a powerful representation of its processor instructions as shown in Figure~\ref{fig:extinstructions}.
This abstraction must be able to represent any valid combination of mnemonic and operands of the instruction set of the target hardware architecture.
In order to be most useful, this abstract representation must be able to be instantiated in the following three ways:

\begin{figure}
\flowgraph{
\resource{Assemblers} \ar@{-}`d[]+L`[d]+L \ar@{-}`d[]+R`[d]+R & \resource{Generators} \ar@{-}`d[]+L`[d]+L \ar@{-}`d[]+R`[d]+R & \resource{Disassemblers} \ar@{-}`d[]+L`[ddd]+L \ar@{-}`d[]+R`[ddd]+R \\
\resource{textual\\representation} \ar@{-->}[dd] \ar[rd] & \resource{program-driven\\instantiation} \ar[d] & \resource{binary\\encoding} \ar[ld] \ar@{-->}[dd] \\
& \converter{Abstract\\Instruction} \ar[ld] \ar@{-->}`d[ld][ld] \ar[rd] \ar@{-->}`d[rd][rd] \ar[rd] \\
\resource{binary\\encoding} & & \resource{textual\\representation} \\
}\caption{Abstract representation of instructions}
\label{fig:extinstructions}
\end{figure}

\begin{itemize}

\item Translation from Text\nopagebreak

A concrete instruction and its operands have to be recognized by their textual representation.
Usually, the documentation of the target hardware architecture defines how its instructions are textually represented.
This functionality is needed by assemblers, see Section~\ref{sec:extassemblers}.

\item Instantiation by Code\nopagebreak

A concrete instruction shall be instantiatable by providing its mnemonic and an abstract representation of its operands.
The operands themselves are composed of immediate values, registers, or special-purpose data addressing.
This functionality is needed for the translation from text as well as by machine code generators, see Section~\ref{sec:extgenerators}.

\item Decoding from Machine Code\nopagebreak

A representation of a concrete instruction with its operands must also be creatable by decoding one or more binary octets.
The actual encoding and decoding of instructions is defined by the instruction set of the target hardware architecture.
This functionality is needed by disassemblers, see Section~\ref{sec:extdisassemblers}.

\end{itemize}

Additionally, instances of the abstract instruction representation must also be able to emit themselves in the following two ways:

\begin{itemize}

\item Encoding into Machine Code\nopagebreak

Any representation of a valid instruction shall be encoded into one or more binary octets.
This functionality is needed by assemblers and machine code generators, see Sections~\ref{sec:extassemblers} and~\ref{sec:extgenerators}.

\item Translation into Text\nopagebreak

Any representation of a valid instruction and its operands shall be translated into its textual representation.
This functionality is needed by disassemblers and machine code generators, see Sections~\ref{sec:extdisassemblers} and~\ref{sec:extgenerators}.

\end{itemize}

Usually, all combinations of instruction mnemonics and operand types that are valid according to the instruction set of the target hardware architecture are stored in a table.
This table often also contains information about how a concrete instruction with its mnemonic and operands is represented using machine code.
This information is needed for encoding and decoding instructions as discussed above.

\subsection{Assemblers}\label{sec:extassemblers}

All assemblers featured by the \ecs{} implement the generic assembly language.
It provides a generic abstraction for any textual representation of processor instructions.
It also supports common functionality like directives or the evaluation of arithmetic, bitwise, and logical operations.
\seeassembly
The \ecs{} provides a generic assembler that is a base for all concrete assemblers and implements the generic assembly language.
It performs tasks like managing sections, evaluating expressions, and writing the resulting object files.
Concrete assemblers only have to apply the translation of a simple textual representation of a single instruction into its machine code encoding.

\subsection{Disassemblers}\label{sec:extdisassemblers}

The \ecs{} provides a generic disassembler that is a base for all concrete disassemblers.
It is able to process the sections stored in an object file and to print their textual representation.
Data sections are represented using the generic data directives supported by the generic assembly language,
while the machine code stored in code sections are passed to the concrete disassemblers.
They only have to provide the encoding of some binary octets representing a single instruction and to print its textual representation.

\subsection{Machine Code Generators}\label{sec:extgenerators}

Machine code generators translate intermediate code into equivalent machine code for the target hardware architecture.
The \ecs{} provides a generic machine code generator that is a base for all concrete generators.
It is able to manage intermediate code sections and process the instructions that have the same binary encoding across all concrete generators.
This includes all instructions contained in data sections or instructions that are not natively supported by the concrete machine code generator.
The generators translate one intermediate section at a time and create a corresponding object file section and a debugging information entry.
Additionally, generators are able to create an assembly code listing of the generated machine code for verification and debugging purposes.
Since the resulting assembly code listing is by design based on the generic assembly language, it can also be processed by the corresponding assembler yielding an exact copy of the original machine code.

\subsection{Back-Ends}\label{sec:extbackends}

Back-ends combine the assembler and the machine code generator for a specific hardware architecture and provide an abstract description of some of the architectural characteristics that are necessary for the generation of intermediate code.
This includes platform-specific information about address and default integer sizes as well as data alignment constraints for example.
The abstract description is designed to be passed to front-ends for programming languages that need this information, see Section~\ref{sec:extfrontends}.

\section{Runtime Environments}

The \ecs{} supports several different runtime environments.
A runtime environment is characterized by the actual hardware architecture it is running on, as well as the underlying operating system.

\subsection{Linkers}

In order to run programs created by compilers and assemblers of the \ecs{} on a specific operating system or machine,
there must be a linker that generates the output files that are executable on the target platform.
The \ecs{} provides a generic linker that is able to map all sections of several object files into one or two binary arrangements and do the necessary linking therein.
Two binary arrangements are needed, when data and code have to be arranged in separate address spaces.
A concrete linker has just to write the binary data into a file using the target file format.
Oftentimes however, binary file formats can also be represented using only the features already provided by the generic assembly language.
In these cases, the executable output files can also be created using the plain binary file linker provided by the \ecs{} and no specific linker is necessary.

\subsection{Runtime Support}

Runtime support is needed for the initialization at the beginning of the program execution as well as the implementation of some standard functions that interface the environment.
This support depends on the actual hardware architecture, the file format of the executable file, as well as the underlying operating system.
Additional runtime support is needed for intermediate code instructions that are not natively supported by the machine code generator for a specific hardware architecture.
This support depends only on the actual hardware architecture and is the same across different operating systems.
Some programming languages may also require runtime support for their language features.
The runtime support for a language is bundled into so-called \emph{library files}\index{Library files}.
A library file is a collection of object files targeting a single hardware architecture.

\section{Development Tools}

The source code of the \ecs{} is accompanied by a makefile and some utility tools for developers.
These facilities are not required to build, execute, test, or modify the \ecs{} but are intended to simplify these activities.
The remainder of this section describes all utilities provided by the \ecs{} and their individual command-line interfaces.
The capabilities of the makefile on the other hand are described in a readme file also supplied with the source code.

\toolsection{depwalk} is a utility tool for dependency walking.
It accepts a list of \cpp{} source files and writes the set of all directly and indirectly included header files of each source file to the standard output stream.
The output is designed to be included in the makefile of the \ecs{} in order to consistently track all dependencies when changing the include directives of one or more of its source files.

\flowgraph{\resource{\cpp{}\\source file} \ar[r] & \converter{depwalk} \ar[r] & \resource{dependency\\list}}

\toolsection{ecsd} is a driver utility tool which conveniently invokes the tools of the \ecs{} with appropriate command-line arguments and environment variables.
Its functionality and command-line interface are described in \Documentation{}~\documentationref{interface}{User Interface}.

\flowgraph{\resource{input\\files} \ar[r] & \converter{ecsd} \ar[r] & \resource{executable\\file}}

For developers of the \ecs{} it additionally provides the \texttt{-m}~command-line flag which automatically builds all dependencies of a tool before invoking it.
A special target environment called \environment{code} allows exposing and executing the intermediate code representation of a program passed between front-ends and back-ends.
For the purpose of creating and debugging runtime environments on the other hand, the driver allows targeting freestanding environments for all supported hardware architectures using the \texttt{-t} option.
\ifbook The available names correspond to the common architecture prefixes and suffixes of the \ref*{tools:compilers}~compiler and assembler tools listed in Table~\ref{tab:tools} on page~\pageref{tab:tools}. \fi

\toolsection{hexdump} is a utility tool for viewing binary files.
It accepts the name of a binary file and writes its contents in hexadecimal form to the standard output stream.
If additionally provided with a corresponding map file as generated by linkers, it lists all sections contained therein and colors their contents.

\flowgraph{\resource{binary file/\\map file} \ar[r] & \converter{hexdump} \ar[r] & \resource{binary contents/\\colored sections}}

\toolsection{linecheck} is a utility tool for checking text lines.
It accepts a list of plain text files and checks their contents for consistent use of white space.
The complete source code of the \ecs{} is stored in plain text files and validated using this tool.

\flowgraph{\resource{text\\files} \ar[r] & \converter{linecheck} \ar[r] & \resource{diagnostic\\messages}}

\toolsection{regtest} is a utility tool for regression testing.
It accepts a quoted command line and the name of a plain text file called a test suite that contains an arbitrary number of tests.
It notes the result of executing the command once for each test and summarizes the differences between consecutive runs if provided with an optional result file for regression testing.
Each test consists of a short description followed by some contrived input that gets stored in a specified temporary file and is intended to cause the command execution to either succeed or fail.

\flowgraph{\resource{command/\\test suite} \ar[r] & \converter{regtest} \ar@{-->}[ld] \ar@{~}[d] \ar@/u/[r] & \resource{test\\results} \ar@/d/[l] \\ \resource{temporary\\input file} \ar[r] & \converter{command} \ar[r] & \resource{success/\\failure} \ar@{-->}[lu]}

Each test is described by a single line of text beginning with the identifier \texttt{positive} or \texttt{negative} indicating the expected result of the command execution, followed by a colon and a unique name.
The actual input of a test consists of all indented text lines following its description ignoring the first horizontal tab character of each line.
Lines beginning with a number sign character denote comments for annotating and structuring test suites and are completely ignored.

\concludechapter


\appendix
\phantomsection\addcontentsline{toc}{part}{Addendum}
\addtocontents{toc}{\protect\setcounter{tocdepth}{0}}
\part*{Addendum}
% Presentation material for the Eigen Compiler Suite
% Copyright (C) Florian Negele

% This file is part of the Eigen Compiler Suite.

% Permission is granted to copy, distribute and/or modify this document
% under the terms of the GNU Free Documentation License, Version 1.3
% or any later version published by the Free Software Foundation.

% You should have received a copy of the GNU Free Documentation License
% along with the ECS.  If not, see <https://www.gnu.org/licenses/>.

% Generic documentation utilities
% Copyright (C) Florian Negele

% This file is part of the Eigen Compiler Suite.

% Permission is granted to copy, distribute and/or modify this document
% under the terms of the GNU Free Documentation License, Version 1.3
% or any later version published by the Free Software Foundation.

% You should have received a copy of the GNU Free Documentation License
% along with the ECS.  If not, see <https://www.gnu.org/licenses/>.

\providecommand{\cpp}{C\texttt{++}}
\providecommand{\opt}{_\mathit{opt}}
\providecommand{\tool}[1]{\texttt{#1}}
\providecommand{\version}{Version 0.0.40}
\providecommand{\resource}[1]{*++\txt{#1}}
\providecommand{\ecs}{Eigen Compiler Suite}
\providecommand{\changed}[1]{\underline{#1}}
\providecommand{\toolbox}[1]{\converter{#1}}
\providecommand{\file}{}\renewcommand{\file}[1]{\texttt{#1}}
\providecommand{\alignright}{\hfill\linebreak[0]\hspace*{\fill}}
\providecommand{\converter}[1]{*++[F][F*:white][F,:gray]\txt{#1}}
\providecommand{\documentation}{\ifbook chapter\else document\fi}
\providecommand{\Documentation}{\ifbook Chapter\else Document\fi}
\providecommand{\variable}[1]{\resource{\texttt{\small#1}\\variable}}
\providecommand{\documentationref}[2]{\ifbook\ref{#1}\else``\href{#1}{#2}''~\cite{#1}\fi}
\providecommand{\objfile}[1]{\texttt{#1}\index[runtime]{#1 object file@\texttt{#1} object file}}
\providecommand{\libfile}[1]{\texttt{#1}\index[runtime]{#1 library file@\texttt{#1} library file}}
\providecommand{\epigraph}[2]{\ifbook\begin{quote}\flushright\textit{#1}\par--- #2\end{quote}\fi}
\providecommand{\environmentvariable}[1]{\texttt{#1}\index{Environment variables!#1@\texttt{#1}}}
\providecommand{\environment}[1]{\texttt{#1}\index[environment]{#1 environment@\texttt{#1} environment}}
\providecommand{\toolsection}{}\renewcommand{\toolsection}[1]{\subsection{#1}\label{\prefix:#1}\tool{#1}}
\providecommand{\instruction}{}\renewcommand{\instruction}[2]{\noindent\qquad\pdftooltip{\texttt{#1}}{#2}\refstepcounter{instruction}\par}
\providecommand{\flowgraph}{}\renewcommand{\flowgraph}[1]{\par\sffamily\begin{displaymath}\xymatrix@=4ex{#1}\end{displaymath}\normalfont\par}
\providecommand{\instructionset}{}\renewcommand{\instructionset}[4]{\setcounter{instruction}{0}\begin{multicols}{\ifbook#3\else#4\fi}[{\captionof{table}[#2]{#2 (\ref*{#1:instructions}~instructions)}\label{tab:#1set}\vspace{-2ex}}]\footnotesize\raggedcolumns\input{#1.set}\label{#1:instructions}\end{multicols}}

\providecommand{\gpl}{GNU General Public License}
\providecommand{\rse}{ECS Runtime Support Exception}
\providecommand{\fdl}{\href{https://www.gnu.org/licenses/fdl.html}{GNU Free Documentation License}}

\providecommand{\docbegin}{}
\providecommand{\docend}{}
\providecommand{\doclabel}[1]{\hypertarget{#1}}
\providecommand{\doclink}[2]{\hyperlink{#1}{#2}}
\providecommand{\docsection}[3]{\hypertarget{#1}{\subsection{#2}}\label{sec:#1}\index[library]{#2@#3}}
\providecommand{\docsectionstar}[1]{}
\providecommand{\docsubbegin}{\begin{description}}
\providecommand{\docsubend}{\end{description}}
\providecommand{\docsubsection}[3]{\item[\hypertarget{#1}{#2}]\index[library]{#2@#3}}
\providecommand{\docsubsectionstar}[1]{\smallskip}
\providecommand{\docsubsubsection}[3]{\docsubsection{#1}{#2}{#3}}
\providecommand{\docsubsubsectionstar}[1]{}
\providecommand{\docsubsubsubsection}[3]{}
\providecommand{\docsubsubsubsectionstar}[1]{}
\providecommand{\doctable}{}

\providecommand{\debuggingtool}{}\renewcommand{\debuggingtool}{This tool is provided for debugging purposes.
It allows exposing and modifying an internal data structure that is usually not accessible.
}

\providecommand{\interface}{All tools accept command-line arguments which are taken as names of plain text files containing the source code.
If no arguments are provided, the standard input stream is used instead.
Output files are generated in the current working directory and have the same name as the input file being processed whereas the filename extension gets replaced by an appropriate suffix.
\seeinterface
}

\providecommand{\license}{\noindent Copyright \copyright{} Florian Negele\par\medskip\noindent
Permission is granted to copy, distribute and/or modify this document under the terms of the
\fdl{}, Version 1.3 or any later version published by the \href{https://fsf.org/}{Free Software Foundation}.
}

\providecommand{\ecslogosurface}{
\fill[darkgray] (0,0,0) -- (0,0,3) -- (0,3,3) -- (0,3,1) -- (0,4,1) -- (0,4,3) -- (0,5,3) -- (0,5,0) -- (0,2,0) -- (0,2,2) -- (0,1,2) -- (0,1,0) -- cycle;
\fill[gray] (0,5,0) -- (0,5,3) -- (1,5,3) -- (1,5,1) -- (2,5,1) -- (2,5,3) -- (3,5,3) -- (3,5,0) -- cycle;
\fill[lightgray] (0,0,0) -- (0,1,0) -- (2,1,0) -- (2,4,0) -- (1,4,0) -- (1,3,0) -- (2,3,0) -- (2,2,0) -- (0,2,0) -- (0,5,0) -- (3,5,0) -- (3,0,0) -- cycle;
\begin{scope}[line width=0.5]
\begin{scope}[gray]
\draw (0,0,0) -- (0,1,0);
\draw (2,1,0) -- (2,2,0);
\draw (0,1,2) -- (0,2,2);
\draw (0,2,0) -- (0,5,0);
\draw (2,3,0) -- (2,4,0);
\end{scope}
\begin{scope}[lightgray]
\draw (0,1,0) -- (0,1,2);
\draw (0,3,1) -- (0,3,3);
\draw (0,5,0) -- (0,5,3);
\draw (2,5,1) -- (2,5,3);
\end{scope}
\begin{scope}[white]
\draw (0,1,0) -- (2,1,0);
\draw (1,3,0) -- (2,3,0);
\draw (0,5,0) -- (3,5,0);
\end{scope}
\end{scope}
}

\providecommand{\ecslogo}[1]{
\begin{tikzpicture}[scale={(#1)/((sin(45)+cos(45))*3cm)},x={({-cos(45)*1cm},{sin(45)*sin(30)*1cm})},y={({0cm},{(cos(30)*1cm})},z={({sin(45)*1cm},{cos(45)*sin(30)*1cm})}]
\begin{scope}[darkgray,line width=1]
\draw (0,0,0) -- (0,0,3) -- (0,3,3) -- (2,3,3) -- (2,5,3) -- (3,5,3) -- (3,5,0) -- (3,0,0) -- cycle;
\draw (0,3,1) -- (0,4,1) -- (0,4,3) -- (0,5,3) -- (1,5,3) -- (1,5,1) -- (2,5,1);
\draw (1,3,0) -- (1,4,0) -- (2,4,0);
\end{scope}
\fill[darkgray] (2,0,0) -- (2,0,3) -- (2,5,3) -- (2,5,1) -- (2,4,1) -- (2,4,0) -- cycle;
\fill[lightgray] (2,0,2) -- (0,0,2) -- (0,2,2) -- (2,2,2) -- cycle;
\fill[gray] (0,1,0) -- (2,1,0) -- (2,1,2) -- (0,1,2) -- cycle;
\fill[gray] (0,3,1) -- (0,3,3) -- (2,3,3) -- (2,3,0) -- (1,3,0) -- (1,3,1) -- cycle;
\ecslogosurface
\end{tikzpicture}
}

\providecommand{\shadowedecslogo}[3]{
\begin{tikzpicture}[scale={(#1)/((sin(#2)+cos(#2))*3cm)},x={({-cos(#2)*1cm},{sin(#2)*sin(#3)*1cm})},y={({0cm},{(cos(#3)*1cm})},z={({sin(#2)*1cm},{cos(#2)*sin(#3)*1cm})}]
\shade[top color=lightgray!50!white,bottom color=white,middle color=lightgray!50!white] (0,0,0) -- (3,0,0) -- (3,{-0.5-3*sin(#2)*sin(#3)/cos(#3)},0) -- (0,-0.5,0) -- cycle;
\shade[top color=darkgray!50!gray,bottom color=white,middle color=darkgray!50!white] (0,0,0) -- (0,0,3) -- (0,{-0.5-3*cos(#2)*sin(#3)/cos(#3)},3) -- (0,-0.5,0) -- cycle;
\begin{scope}[y={({(cos(#2)+sin(#2))*0.5cm},{(cos(#2)*sin(#3)-sin(#2)*sin(#3))*0.5cm})}]
\useasboundingbox (3,0,0) -- (0,0,0) -- (0,0,3);
\shade[left color=darkgray!80!black,right color=lightgray,middle color=gray] (0,0,0) -- (0,1,0) -- (0,1,0.5) -- (0,2,0) -- (0,5,0) -- (0,5,3) -- (1,5,3) -- (1,4,3) -- (1,4,2.5) -- (1,3,3) -- (2,5,3) -- (3,5,3) -- (3,0,3) -- cycle;
\clip (0,0,0) -- (0,0,3) -- ({-3*sin(#2)/cos(#2)},0,0) -- cycle;
\shade[left color=darkgray,right color=lightgray!50!gray] (0,0,0) -- (0,1,0) -- (0,1,0.5) -- (0,2,0) -- (0,5,0) -- (0,5,3) -- (1,5,3) -- (1,4,3) -- (1,4,2.5) -- (1,3,3) -- (2,5,3) -- (3,5,3) -- (3,0,3) -- cycle;
\end{scope}
\shade[left color=darkgray,right color=darkgray!80!black] (2,0,0) -- (2,0,3) -- (2,5,3) -- (2,5,1) -- (2,4,1) -- (2,4,0) -- cycle;
\shade[left color=darkgray!90!black,right color=gray!80!darkgray] (2,0,2) -- (0,0,2) -- (0,2,2) -- (2,2,2) -- cycle;
\shade[top color=darkgray!90!black,bottom color=gray!80!darkgray] (0,1,0) -- (2,1,0) -- (2,1,2) -- (0,1,2) -- cycle;
\shade[top color=darkgray!90!black,bottom color=gray!80!darkgray] (0,3,1) -- (0,3,3) -- (2,3,3) -- (2,3,0) -- (1,3,0) -- (1,3,1) -- cycle;
\fill[gray] (2,1,0) -- (1.5,1,0.5) -- (0,1,0.5) -- (0,1,0) -- cycle;
\fill[gray] (1,3,2) -- (0.5,3,2) -- (0.5,3,3) -- (1,3,3) -- cycle;
\fill[gray] (2,3,0) -- (1.5,3,0.5) -- (1,3,0.5) -- (1,3,0) -- cycle;
\ecslogosurface
\end{tikzpicture}
}

\providecommand{\cpplogo}[1]{
\begin{tikzpicture}[scale=(#1)/512em]
\fill[gray] (435.2794,398.7159) -- (247.1911,507.3075) .. controls (236.3563,513.5642) and (218.6240,513.5642) .. (207.7892,507.3075) -- (19.7009,398.7159) .. controls (8.8646,392.4606) and (0.0000,377.1043) .. (0.0000,364.5924) -- (0.0000,147.4076) .. controls (0.8430,132.8363) and (8.2856,120.7683) .. (19.7009,113.2842) -- (207.7892,4.6926) .. controls (218.6240,-1.5642) and (236.3564,-1.5642) .. (247.1911,4.6926) -- (435.2794,113.2842) .. controls (447.5273,121.4304) and (454.4987,133.6918) .. (454.9803,147.4076) -- (454.9803,364.5924) .. controls (454.5404,377.7571) and (446.6566,391.0351) .. (435.2794,398.7159) -- cycle(75.8301,255.9993) .. controls (74.9389,404.0881) and (273.2892,469.4783) .. (358.8263,331.8769) -- (293.1917,293.8965) .. controls (253.5702,359.4301) and (155.1909,335.9977) .. (151.6601,255.9993) .. controls (152.7204,182.2703) and (249.4137,148.0211) .. (293.1961,218.1065) -- (358.8308,180.1276) .. controls (283.4477,49.2645) and (79.6318,96.3470) .. (75.8301,255.9993) -- cycle(379.1503,247.5747) -- (362.2982,247.5747) -- (362.2982,230.7226) -- (345.4490,230.7226) -- (345.4490,247.5747) -- (328.5969,247.5747) -- (328.5969,264.4254) -- (345.4490,264.4254) -- (345.4490,281.2759) -- (362.2982,281.2759) -- (362.2982,264.4254) -- (379.1503,264.4254) -- cycle(442.3420,247.5747) -- (425.4899,247.5747) -- (425.4899,230.7226) -- (408.6408,230.7226) -- (408.6408,247.5747) -- (391.7886,247.5747) -- (391.7886,264.4254) -- (408.6408,264.4254) -- (408.6408,281.2759) -- (425.4899,281.2759) -- (425.4899,264.4254) -- (442.3420,264.4254) -- cycle;
\end{tikzpicture}
}

\providecommand{\fallogo}[1]{
\begin{tikzpicture}[scale=(#1)/512em]
\fill[gray] (185.7774,0.0000) .. controls (200.4486,15.9798) and (226.8966,8.7148) .. (235.0426,31.5836) .. controls (249.5297,58.0598) and (247.9581,97.9161) .. (280.3335,110.9762) .. controls (309.1690,120.3496) and (337.8406,104.2727) .. (366.5753,103.9379) .. controls (373.4449,111.5171) and (379.2885,128.2574) .. (383.9755,108.9744) .. controls (396.6979,102.5615) and (437.2808,107.6681) .. (426.9652,124.3252) .. controls (408.9822,121.0785) and (412.4742,146.0729) .. (426.5192,131.4996) .. controls (433.8413,120.8489) and (465.1541,126.5522) .. (441.9067,135.7950) .. controls (396.1879,157.7478) and (344.1112,161.5079) .. (298.5528,183.5702) .. controls (277.7471,193.5198) and (284.6941,218.7163) .. (285.2127,236.9640) .. controls (292.3599,316.2826) and (307.3929,394.6311) .. (317.1198,473.6154) .. controls (329.0637,505.4736) and (292.1195,528.5004) .. (265.9183,511.2761) .. controls (237.9284,499.2462) and (237.3684,465.2681) .. (230.9102,439.9421) .. controls (218.6692,374.3397) and (215.6307,306.9662) .. (198.1732,242.3977) .. controls (183.1379,232.7444) and (164.4245,256.0298) .. (149.0430,261.4799) .. controls (116.9328,279.2585) and (87.1822,308.5851) .. (48.2293,307.8914) .. controls (21.3220,306.9037) and (-15.9107,281.8761) .. (7.2921,252.7908) .. controls (29.7799,220.6177) and (67.5177,204.3028) .. (100.9287,185.9449) .. controls (130.8217,170.8906) and (161.1548,156.5903) .. (191.0278,141.5847) .. controls (196.1738,120.0520) and (186.6049,95.2409) .. (186.8382,72.4353) .. controls (185.5234,48.4204) and (183.1700,23.9341) .. (185.7774,0.0000) -- cycle;
\end{tikzpicture}
}

\providecommand{\oblogo}[1]{
\begin{tikzpicture}[scale=(#1)/512em]
\fill[gray] (160.3865,208.9117) .. controls (154.0879,214.6478) and (149.0735,221.2409) .. (145.4125,228.5384) .. controls (184.8790,248.4273) and (234.7122,269.8787) .. (297.5493,291.8782) .. controls (300.3943,281.4769) and (300.9552,268.7619) .. (300.4023,255.2389) .. controls (248.9909,244.7891) and (200.0310,225.9279) .. (160.3865,208.9117) -- cycle(225.7398,392.6996) .. controls (308.0209,392.1716) and (359.3326,345.9277) .. (368.7203,285.2098) .. controls (376.6742,197.1784) and (311.7194,141.3342) .. (205.4287,142.1456) .. controls (139.9485,141.4804) and (88.7155,166.1957) .. (73.5775,228.0086) .. controls (52.0297,320.3408) and (123.4078,391.0103) .. (225.7398,392.6996) -- cycle(216.0739,176.4733) .. controls (268.9183,179.2424) and (315.8292,206.5488) .. (312.7454,265.1139) .. controls (313.2769,315.6384) and (286.5993,353.4946) .. (216.6040,355.7934) .. controls (162.4657,355.7934) and (126.0914,317.5023) .. (126.0914,260.5103) .. controls (126.1733,214.2900) and (163.3363,176.2849) .. (216.0739,176.4733) -- cycle(76.4897,189.1754) .. controls (13.1586,147.5631) and (0.0000,119.4207) .. (0.0000,119.4207) -- (90.6499,170.1632) .. controls (85.3004,175.8497) and (80.5994,182.1633) .. (76.4897,189.1754) -- cycle(353.9486,119.3004) -- (402.9482,119.3004) .. controls (427.0025,137.0797) and (450.9893,162.7034) .. (474.9529,191.0213) .. controls (509.3540,228.5339) and (531.3391,294.2091) .. (487.8149,312.1206) .. controls (462.8165,324.7652) and (394.3874,316.8943) .. (373.8912,313.6651) .. controls (379.9291,297.7449) and (383.2899,278.4204) .. (381.4989,257.7214) .. controls (420.3069,248.0321) and (421.9610,218.3461) .. (407.7867,192.6417) .. controls (391.1113,162.4018) and (370.1114,132.9097) .. (353.9486,119.3004) -- cycle;
\end{tikzpicture}
}

\providecommand{\markuptable}{
\begin{table}
\sffamily\centering
\begin{tabular}{@{}lcl@{}}
\toprule
\texttt{//italics//} & $\rightarrow$ & \textit{italics} \\
\midrule
\texttt{**bold**} & $\rightarrow$ & \textbf{bold} \\
\midrule
\texttt{\# ordered list} & & 1 ordered list \\
\texttt{\# second item} & $\rightarrow$ & 2 second item \\
\texttt{\#\# sub item} & & \hspace{1em} 1 sub item \\
\midrule
\texttt{* unordered list} & & $\bullet$ unordered list \\
\texttt{* second item} & $\rightarrow$ & $\bullet$ second item \\
\texttt{** sub item} & & \hspace{1em} $\bullet$ sub item \\
\midrule
\texttt{link to [[label]]} & $\rightarrow$ & link to \underline{label} \\
\midrule
\texttt{<{}<label>{}> definition } & $\rightarrow$ & definition \\
\midrule
\texttt{[[url|link name]]} & $\rightarrow$ & \underline{link name} \\
\midrule\addlinespace
\texttt{= large heading} & & {\Large large heading} \smallskip \\
\texttt{== medium heading} & $\rightarrow$ & {\large medium heading} \\
\texttt{=== small heading} & & small heading \\
\midrule
\texttt{no line break} & & no line break for paragraphs \\
\texttt{for paragraphs} & $\rightarrow$ \\
& & use empty line \\
\texttt{use empty line} \\
\midrule
\texttt{force\textbackslash\textbackslash line break} & $\rightarrow$ & force \\
& & line break \\
\midrule
\texttt{horizontal line} & $\rightarrow$ & horizontal line \\
\texttt{----} & & \hrulefill \\
\midrule
\texttt{|=a|=table|=header} & & \underline{a \enspace table \enspace header} \\
\texttt{|a|table|row} & $\rightarrow$ & a \enspace table \enspace row \\
\texttt{|b|table|row} & & b \enspace table \enspace row \\
\midrule
\texttt{\{\{\{} \\
\texttt{unformatted} & $\rightarrow$ & \texttt{unformatted} \\
\texttt{code} & & \texttt{code} \\
\texttt{\}\}\}} \\
\midrule\addlinespace
\texttt{@ new article} & & {\Large 1.\ new article} \smallskip \\
\texttt{@ second article} & $\rightarrow$ & {\Large 2.\ second article} \smallskip \\
\texttt{@@ sub article} & & {\large 2.1.\ sub article} \\
\bottomrule
\end{tabular}
\normalfont\caption{Elements of the generic documentation markup language}
\label{tab:docmarkup}
\end{table}
}

\providecommand{\startchapter}[4]{
\documentclass[11pt,a4paper]{article}
\usepackage{booktabs}
\usepackage[format=hang,labelfont=bf]{caption}
\usepackage{changepage}
\usepackage[T1]{fontenc}
\usepackage[margin=2cm]{geometry}
\usepackage{hyperref}
\usepackage[american]{isodate}
\usepackage{lmodern}
\usepackage{longtable}
\usepackage{mathptmx}
\usepackage{microtype}
\usepackage[toc]{multitoc}
\usepackage{multirow}
\usepackage[all]{nowidow}
\usepackage{pdfcomment}
\usepackage{syntax}
\usepackage{tikz}
\usepackage[all]{xy}
\hypersetup{pdfborder={0 0 0},bookmarksnumbered=true,pdftitle={\ecs{}: #2},pdfauthor={Florian Negele},pdfsubject={\ecs{}},pdfkeywords={#1}}
\setlength{\grammarindent}{8em}\setlength{\grammarparsep}{0.2ex}
\setlength{\columnsep}{2em}
\newcommand{\prefix}{}
\newcounter{instruction}
\bibliographystyle{unsrt}
\renewcommand{\index}[2][]{}
\renewcommand{\arraystretch}{1.05}
\renewcommand{\floatpagefraction}{0.7}
\renewcommand{\syntleft}{\itshape}\renewcommand{\syntright}{}
\title{\vspace{-5ex}\Huge{\ecs{}}\medskip\hrule}
\author{\huge{#2}}
\date{\medskip\version}
\newif\ifbook\bookfalse
\pagestyle{headings}
\frenchspacing
\begin{document}
\maketitle\thispagestyle{empty}\noindent#4\setlength{\columnseprule}{0.4pt}\tableofcontents\setlength{\columnseprule}{0pt}\vfill\pagebreak[3]\null\vfill\bigskip\noindent
\parbox{\textwidth-4em}{\license The contents of this \documentation{} are part of the \href{manual}{\ecs{} User Manual}~\cite{manual} and correspond to Chapter ``\href{manual\##3}{#1}''.\alignright\mbox{\today}}
\parbox{4em}{\flushright\ecslogo{3em}}
\clearpage
}

\providecommand{\concludechapter}{
\vfill\pagebreak[3]\null\vfill
\thispagestyle{myheadings}\markright{REFERENCES}
\noindent\begin{minipage}{\textwidth}\begin{multicols}{2}[\section*{References}]
\renewcommand{\section}[2]{}\small\bibliography{references}
\end{multicols}\end{minipage}\end{document}
}

\providecommand{\startpresentation}[2]{
\documentclass[14pt,aspectratio=43,usepdftitle=false]{beamer}
\usepackage{booktabs}
\usepackage{etex}
\usepackage{multicol}
\usepackage{tikz}
\usepackage[all]{xy}
\bibliographystyle{unsrt}
\setlength{\columnsep}{1em}
\setlength{\leftmargini}{1em}
\setbeamercolor{title}{fg=black}
\setbeamercolor{structure}{fg=darkgray}
\setbeamercolor{bibliography item}{fg=darkgray}
\setbeamerfont{title}{series=\bfseries}
\setbeamerfont{subtitle}{series=\normalfont}
\setbeamerfont*{frametitle}{parent=title}
\setbeamerfont{block title}{series=\bfseries}
\setbeamerfont*{framesubtitle}{parent=subtitle}
\setbeamersize{text margin left=1em,text margin right=1em}
\setbeamertemplate{navigation symbols}{}
\setbeamertemplate{itemize item}[circle]{}
\setbeamertemplate{bibliography item}[triangle]{}
\setbeamertemplate{bibliography entry author}{\usebeamercolor[fg]{bibliography item}}
\setbeamertemplate{frametitle}{\medskip\usebeamerfont{frametitle}\color{gray}\raisebox{-2.5ex}[0ex][0ex]{\rule{0.1em}{4.5ex}}}
\addtobeamertemplate{frametitle}{}{\hspace{0.4em}\usebeamercolor[fg]{title}\insertframetitle\par\vspace{0.2ex}\hspace{0.5em}\usebeamerfont{framesubtitle}\insertframesubtitle}
\hypersetup{pdfborder={0 0 0},bookmarksnumbered=true,bookmarksopen=true,bookmarksopenlevel=0,pdftitle={\ecs{}: #1},pdfauthor={Florian Negele},pdfsubject={\ecs{}},pdfkeywords={#1}}
\renewcommand{\flowgraph}[1]{\resizebox{\textwidth}{!}{$$\xymatrix{##1}$$}}
\title{\ecs{}\medskip\hrule\medskip}
\institute{\shadowedecslogo{5em}{30}{15}}
\date{\version}
\subtitle{#1}
\begin{document}
\begin{frame}[plain]\titlepage\nocite{manual}\end{frame}
\begin{frame}{Contents}{#1}\begin{center}\tableofcontents\end{center}\end{frame}
}

\providecommand{\concludepresentation}{
\begin{frame}{References}\begin{footnotesize}\setlength{\columnseprule}{0.4pt}\begin{multicols}{2}\bibliography{references}\end{multicols}\end{footnotesize}\end{frame}
\end{document}
}

\providecommand{\startbook}[1]{
\documentclass[10pt,paper=17cm:24cm,DIV=13,twoside=semi,headings=normal,numbers=noendperiod,cleardoublepage=plain]{scrbook}
\usepackage{atveryend}
\usepackage{booktabs}
\usepackage{caption}
\usepackage{changepage}
\usepackage[T1]{fontenc}
\usepackage{imakeidx}
\usepackage{hyperref}
\usepackage[american]{isodate}
\usepackage{lmodern}
\usepackage{longtable}
\usepackage{mathptmx}
\usepackage[final]{microtype}
\usepackage{multicol}
\usepackage{multirow}
\usepackage[all]{nowidow}
\usepackage{pdfcomment}
\usepackage{scrlayer-scrpage}
\usepackage{setspace}
\usepackage{syntax}
\usepackage[eventxtindent=4pt,oddtxtexdent=4pt]{thumbs}
\usepackage{tikz}
\usepackage[all]{xy}
\hyphenation{Micro-Blaze Open-Cores Open-RISC Power-PC}
\hypersetup{pdfborder={0 0 0},bookmarksnumbered=true,bookmarksopen=true,bookmarksopenlevel=0,pdftitle={\ecs{}: #1},pdfauthor={Florian Negele},pdfsubject={\ecs{}},pdfkeywords={#1}}
\setlength{\grammarindent}{8em}\setlength{\grammarparsep}{0.7ex}
\setkomafont{captionlabel}{\usekomafont{descriptionlabel}}
\renewcommand{\arraystretch}{1.05}\setstretch{1.1}
\renewcommand{\chapterformat}{\thechapter\autodot\enskip\raisebox{-1ex}[0ex][0ex]{\color{gray}\rule{0.1em}{3.5ex}}\enskip}
\renewcommand{\startchapter}[4]{\hypertarget{##3}{\chapter{##1}}\label{##3}##4\addthumb{##1}{\LARGE\sffamily\bfseries\thechapter}{white}{gray}\renewcommand{\prefix}{##3}}
\renewcommand{\concludechapter}{\clearpage{\stopthumb\cleardoublepage}}
\renewcommand{\syntleft}{\itshape}\renewcommand{\syntright}{}
\renewcommand{\floatpagefraction}{0.7}
\renewcommand{\partheademptypage}{}
\DeclareMicrotypeAlias{lmss}{cmr}
\newcommand{\prefix}{}
\newcounter{instruction}
\bibliographystyle{unsrt}
\newif\ifbook\booktrue
\makeindex[intoc,title=Index]
\makeindex[intoc,name=tools,title=Index of Tools,columns=3]
\makeindex[intoc,name=library,title=Index of Library Names]
\makeindex[intoc,name=runtime,title=Index of Runtime Support]
\makeindex[intoc,name=environment,title=Index of Target Environments]
\indexsetup{toclevel=chapter,headers={\indexname}{\indexname}}
\frenchspacing
\begin{document}
\pagenumbering{alph}
\begin{titlepage}\centering
\huge\sffamily\null\vfill\textbf{\ecs{}}\bigskip\hrule\bigskip#1
\normalsize\normalfont\vfill\vfill\shadowedecslogo{10em}{30}{15}
\large\vfill\vfill\version
\end{titlepage}
\null\vfill
\thispagestyle{empty}
\noindent\today\par\medskip
\license A copy of this license is included in Appendix~\ref{fdl} on page~\pageref{fdl}.
All product names used herein are for identification purposes only and may be trademarks of their respective companies.
\concludechapter
\frontmatter
\setcounter{tocdepth}{1}
\tableofcontents
\setcounter{tocdepth}{2}
\concludechapter
\listoffigures
\concludechapter
\listoftables
\concludechapter
}

\providecommand{\concludebook}{
\backmatter
\addtocontents{toc}{\protect\setcounter{tocdepth}{-1}}
\phantomsection\addcontentsline{toc}{part}{Bibliography}
\bibliography{references}
\concludechapter
\phantomsection\addcontentsline{toc}{part}{Indexes}
\printindex
\concludechapter
\indexprologue{\label{idx:tools}}
\printindex[tools]
\concludechapter
\printindex[library]
\concludechapter
\indexprologue{\label{idx:runtime}}
\printindex[runtime]
\concludechapter
\indexprologue{\label{idx:environment}}
\printindex[environment]
\concludechapter
\pagestyle{empty}\pagenumbering{Alph}\null\clearpage
\null\vfill\centering\ecslogo{4em}\par\medskip\license
\end{document}
}

% chapter references

\providecommand{\seedocumentationref}{}\renewcommand{\seedocumentationref}[3]{#1, see \Documentation{}~\documentationref{#2}{#3}. }
\providecommand{\seeinterface}{}\renewcommand{\seeinterface}{\ifbook See \Documentation{}~\documentationref{interface}{User Interface} for more information about the common user interface of all of these tools. \fi}
\providecommand{\seeguide}{}\renewcommand{\seeguide}{\seedocumentationref{For basic examples of using some of these tools in practice}{guide}{User Guide}}
\providecommand{\seecpp}{}\renewcommand{\seecpp}{\seedocumentationref{For more information about the \cpp{} programming language and its implementation by the \ecs{}}{cpp}{User Manual for \cpp{}}}
\providecommand{\seefalse}{}\renewcommand{\seefalse}{\seedocumentationref{For more information about the FALSE programming language and its implementation by the \ecs{}}{false}{User Manual for FALSE}}
\providecommand{\seeoberon}{}\renewcommand{\seeoberon}{\seedocumentationref{For more information about the Oberon programming language and its implementation by the \ecs{}}{oberon}{User Manual for Oberon}}
\providecommand{\seeassembly}{}\renewcommand{\seeassembly}{\seedocumentationref{For more information about the generic assembly language and how to use it}{assembly}{Generic Assembly Language Specification}}
\providecommand{\seeamd}{}\renewcommand{\seeamd}{\seedocumentationref{For more information about how the \ecs{} supports the AMD64 hardware architecture}{amd64}{AMD64 Hardware Architecture Support}}
\providecommand{\seearm}{}\renewcommand{\seearm}{\seedocumentationref{For more information about how the \ecs{} supports the ARM hardware architecture}{arm}{ARM Hardware Architecture Support}}
\providecommand{\seeavr}{}\renewcommand{\seeavr}{\seedocumentationref{For more information about how the \ecs{} supports the AVR hardware architecture}{avr}{AVR Hardware Architecture Support}}
\providecommand{\seeavrtt}{}\renewcommand{\seeavrtt}{\seedocumentationref{For more information about how the \ecs{} supports the AVR32 hardware architecture}{avr32}{AVR32 Hardware Architecture Support}}
\providecommand{\seemabk}{}\renewcommand{\seemabk}{\seedocumentationref{For more information about how the \ecs{} supports the M68000 hardware architecture}{m68k}{M68000 Hardware Architecture Support}}
\providecommand{\seemibl}{}\renewcommand{\seemibl}{\seedocumentationref{For more information about how the \ecs{} supports the MicroBlaze hardware architecture}{mibl}{MicroBlaze Hardware Architecture Support}}
\providecommand{\seemips}{}\renewcommand{\seemips}{\seedocumentationref{For more information about how the \ecs{} supports the MIPS32 and MIPS64 hardware architectures}{mips}{MIPS Hardware Architecture Support}}
\providecommand{\seemmix}{}\renewcommand{\seemmix}{\seedocumentationref{For more information about how the \ecs{} supports the MMIX hardware architecture}{mmix}{MMIX Hardware Architecture Support}}
\providecommand{\seeorok}{}\renewcommand{\seeorok}{\seedocumentationref{For more information about how the \ecs{} supports the OpenRISC 1000 hardware architecture}{or1k}{OpenRISC 1000 Hardware Architecture Support}}
\providecommand{\seeppc}{}\renewcommand{\seeppc}{\seedocumentationref{For more information about how the \ecs{} supports the PowerPC hardware architecture}{ppc}{PowerPC Hardware Architecture Support}}
\providecommand{\seerisc}{}\renewcommand{\seerisc}{\seedocumentationref{For more information about how the \ecs{} supports the RISC hardware architecture}{risc}{RISC Hardware Architecture Support}}
\providecommand{\seewasm}{}\renewcommand{\seewasm}{\seedocumentationref{For more information about how the \ecs{} supports the WebAssembly architecture}{wasm}{WebAssembly Architecture Support}}
\providecommand{\seedocumentation}{}\renewcommand{\seedocumentation}{\seedocumentationref{For more information about generic documentations and their generation by the \ecs{}}{documentation}{Generic Documentation Generation}}
\providecommand{\seedebugging}{}\renewcommand{\seedebugging}{\seedocumentationref{For more information about debugging information and its representation}{debugging}{Debugging Information Representation}}
\providecommand{\seecode}{}\renewcommand{\seecode}{\seedocumentationref{For more information about intermediate code and its purpose}{code}{Intermediate Code Representation}}
\providecommand{\seeobject}{}\renewcommand{\seeobject}{\seedocumentationref{For more information about object files and their purpose}{object}{Object File Representation}}

% generic documentation tools

\providecommand{\docprint}{
\toolsection{docprint} is a pretty printer for generic documentations.
It reformats generic documentations and writes it to the standard output stream.
\debuggingtool
\flowgraph{\resource{generic\\documentation} \ar[r] & \toolbox{docprint} \ar[r] & \resource{generic\\documentation}}
\seedocumentation
}

\providecommand{\doccheck}{
\toolsection{doccheck} is a syntactic and semantic checker for generic documentations.
It just performs syntactic and semantic checks on generic documentations and writes its diagnostic messages to the standard error stream.
\debuggingtool
\flowgraph{\resource{generic\\documentation} \ar[r] & \toolbox{doccheck} \ar[r] & \resource{diagnostic\\messages}}
\seedocumentation
}

\providecommand{\dochtml}{
\toolsection{dochtml} is an HTML documentation generator for generic documentations.
It processes several generic documentations and assembles all information therein into an HTML document.
\debuggingtool
\flowgraph{\resource{generic\\documentation} \ar[r] & \toolbox{dochtml} \ar[r] & \resource{HTML\\document}}
\seedocumentation
}

\providecommand{\doclatex}{
\toolsection{doclatex} is a Latex documentation generator for generic documentations.
It processes several generic documentations and assembles all information therein into a Latex document.
\debuggingtool
\flowgraph{\resource{generic\\documentation} \ar[r] & \toolbox{doclatex} \ar[r] & \resource{Latex\\document}}
\seedocumentation
}

% intermediate code tools

\providecommand{\cdcheck}{
\toolsection{cdcheck} is a syntactic and semantic checker for intermediate code.
It just performs syntactic and semantic checks on programs written in intermediate code and writes its diagnostic messages to the standard error stream.
\debuggingtool
\flowgraph{\resource{intermediate\\code} \ar[r] & \toolbox{cdcheck} \ar[r] & \resource{diagnostic\\messages}}
\seeassembly\seecode
}

\providecommand{\cdopt}{
\toolsection{cdopt} is an optimizer for intermediate code.
It performs various optimizations on programs written in intermediate code and writes the result to the standard output stream.
\debuggingtool
\flowgraph{\resource{intermediate\\code} \ar[r] & \toolbox{cdopt} \ar[r] & \resource{optimized\\code}}
\seeassembly\seecode
}

\providecommand{\cdrun}{
\toolsection{cdrun} is an interpreter for intermediate code.
It processes and executes programs written in intermediate code.
The following code sections are predefined and have the usual semantics:
\texttt{abort}, \texttt{\_Exit}, \texttt{fflush}, \texttt{floor}, \texttt{fputc}, \texttt{free}, \texttt{getchar}, \texttt{malloc}, and \texttt{putchar}.
Diagnostic messages about invalid operations include the name of the executed code section and the index of the erroneous instruction.
\debuggingtool
\flowgraph{\resource{intermediate\\code} \ar[r] & \toolbox{cdrun} \ar@/u/[r] & \resource{input/\\output} \ar@/d/[l]}
\seeassembly\seecode
}

\providecommand{\cdamda}{
\toolsection{cdamd16} is a compiler for intermediate code targeting the AMD64 hardware architecture.
It generates machine code for AMD64 processors from programs written in intermediate code and stores it in corresponding object files.
The compiler generates machine code for the 16-bit operating mode defined by the AMD64 architecture.
It also creates a debugging information file as well as an assembly file containing a listing of the generated machine code.
\debuggingtool
\flowgraph{\resource{intermediate\\code} \ar[r] & \toolbox{cdamd16} \ar[r] \ar[d] \ar[rd] & \resource{object file} \\ & \resource{assembly\\listing} & \resource{debugging\\information}}
\seeassembly\seeamd\seeobject\seecode\seedebugging
}

\providecommand{\cdamdb}{
\toolsection{cdamd32} is a compiler for intermediate code targeting the AMD64 hardware architecture.
It generates machine code for AMD64 processors from programs written in intermediate code and stores it in corresponding object files.
The compiler generates machine code for the 32-bit operating mode defined by the AMD64 architecture.
It also creates a debugging information file as well as an assembly file containing a listing of the generated machine code.
\debuggingtool
\flowgraph{\resource{intermediate\\code} \ar[r] & \toolbox{cdamd32} \ar[r] \ar[d] \ar[rd] & \resource{object file} \\ & \resource{assembly\\listing} & \resource{debugging\\information}}
\seeassembly\seeamd\seeobject\seecode\seedebugging
}

\providecommand{\cdamdc}{
\toolsection{cdamd64} is a compiler for intermediate code targeting the AMD64 hardware architecture.
It generates machine code for AMD64 processors from programs written in intermediate code and stores it in corresponding object files.
The compiler generates machine code for the 64-bit operating mode defined by the AMD64 architecture.
It also creates a debugging information file as well as an assembly file containing a listing of the generated machine code.
\debuggingtool
\flowgraph{\resource{intermediate\\code} \ar[r] & \toolbox{cdamd64} \ar[r] \ar[d] \ar[rd] & \resource{object file} \\ & \resource{assembly\\listing} & \resource{debugging\\information}}
\seeassembly\seeamd\seeobject\seecode\seedebugging
}

\providecommand{\cdarma}{
\toolsection{cdarma32} is a compiler for intermediate code targeting the ARM hardware architecture.
It generates machine code for ARM processors executing A32 instructions from programs written in intermediate code and stores it in corresponding object files.
It also creates a debugging information file as well as an assembly file containing a listing of the generated machine code.
\debuggingtool
\flowgraph{\resource{intermediate\\code} \ar[r] & \toolbox{cdarma32} \ar[r] \ar[d] \ar[rd] & \resource{object file} \\ & \resource{assembly\\listing} & \resource{debugging\\information}}
\seeassembly\seearm\seeobject\seecode\seedebugging
}

\providecommand{\cdarmb}{
\toolsection{cdarma64} is a compiler for intermediate code targeting the ARM hardware architecture.
It generates machine code for ARM processors executing A64 instructions from programs written in intermediate code and stores it in corresponding object files.
It also creates a debugging information file as well as an assembly file containing a listing of the generated machine code.
\debuggingtool
\flowgraph{\resource{intermediate\\code} \ar[r] & \toolbox{cdarma64} \ar[r] \ar[d] \ar[rd] & \resource{object file} \\ & \resource{assembly\\listing} & \resource{debugging\\information}}
\seeassembly\seearm\seeobject\seecode\seedebugging
}

\providecommand{\cdarmc}{
\toolsection{cdarmt32} is a compiler for intermediate code targeting the ARM hardware architecture.
It generates machine code for ARM processors without floating-point extension executing T32 instructions from programs written in intermediate code and stores it in corresponding object files.
It also creates a debugging information file as well as an assembly file containing a listing of the generated machine code.
\debuggingtool
\flowgraph{\resource{intermediate\\code} \ar[r] & \toolbox{cdarmt32} \ar[r] \ar[d] \ar[rd] & \resource{object file} \\ & \resource{assembly\\listing} & \resource{debugging\\information}}
\seeassembly\seearm\seeobject\seecode\seedebugging
}

\providecommand{\cdarmcfpe}{
\toolsection{cdarmt32fpe} is a compiler for intermediate code targeting the ARM hardware architecture.
It generates machine code for ARM processors with floating-point extension executing T32 instructions from programs written in intermediate code and stores it in corresponding object files.
It also creates a debugging information file as well as an assembly file containing a listing of the generated machine code.
\debuggingtool
\flowgraph{\resource{intermediate\\code} \ar[r] & \toolbox{cdarmt32fpe} \ar[r] \ar[d] \ar[rd] & \resource{object file} \\ & \resource{assembly\\listing} & \resource{debugging\\information}}
\seeassembly\seearm\seeobject\seecode\seedebugging
}

\providecommand{\cdavr}{
\toolsection{cdavr} is a compiler for intermediate code targeting the AVR hardware architecture.
It generates machine code for AVR processors from programs written in intermediate code and stores it in corresponding object files.
It also creates a debugging information file as well as an assembly file containing a listing of the generated machine code.
\debuggingtool
\flowgraph{\resource{intermediate\\code} \ar[r] & \toolbox{cdavr} \ar[r] \ar[d] \ar[rd] & \resource{object file} \\ & \resource{assembly\\listing} & \resource{debugging\\information}}
\seeassembly\seeavr\seeobject\seecode\seedebugging
}

\providecommand{\cdavrtt}{
\toolsection{cdavr32} is a compiler for intermediate code targeting the AVR32 hardware architecture.
It generates machine code for AVR32 processors from programs written in intermediate code and stores it in corresponding object files.
It also creates a debugging information file as well as an assembly file containing a listing of the generated machine code.
\debuggingtool
\flowgraph{\resource{intermediate\\code} \ar[r] & \toolbox{cdavr32} \ar[r] \ar[d] \ar[rd] & \resource{object file} \\ & \resource{assembly\\listing} & \resource{debugging\\information}}
\seeassembly\seeavrtt\seeobject\seecode\seedebugging
}

\providecommand{\cdmabk}{
\toolsection{cdm68k} is a compiler for intermediate code targeting the M68000 hardware architecture.
It generates machine code for M68000 processors from programs written in intermediate code and stores it in corresponding object files.
It also creates a debugging information file as well as an assembly file containing a listing of the generated machine code.
\debuggingtool
\flowgraph{\resource{intermediate\\code} \ar[r] & \toolbox{cdm68k} \ar[r] \ar[d] \ar[rd] & \resource{object file} \\ & \resource{assembly\\listing} & \resource{debugging\\information}}
\seeassembly\seemabk\seeobject\seecode\seedebugging
}

\providecommand{\cdmibl}{
\toolsection{cdmibl} is a compiler for intermediate code targeting the MicroBlaze hardware architecture.
It generates machine code for MicroBlaze processors from programs written in intermediate code and stores it in corresponding object files.
It also creates a debugging information file as well as an assembly file containing a listing of the generated machine code.
\debuggingtool
\flowgraph{\resource{intermediate\\code} \ar[r] & \toolbox{cdmibl} \ar[r] \ar[d] \ar[rd] & \resource{object file} \\ & \resource{assembly\\listing} & \resource{debugging\\information}}
\seeassembly\seemibl\seeobject\seecode\seedebugging
}

\providecommand{\cdmipsa}{
\toolsection{cdmips32} is a compiler for intermediate code targeting the MIPS32 hardware architecture.
It generates machine code for MIPS32 processors from programs written in intermediate code and stores it in corresponding object files.
It also creates a debugging information file as well as an assembly file containing a listing of the generated machine code.
\debuggingtool
\flowgraph{\resource{intermediate\\code} \ar[r] & \toolbox{cdmips32} \ar[r] \ar[d] \ar[rd] & \resource{object file} \\ & \resource{assembly\\listing} & \resource{debugging\\information}}
\seeassembly\seemips\seeobject\seecode\seedebugging
}

\providecommand{\cdmipsb}{
\toolsection{cdmips64} is a compiler for intermediate code targeting the MIPS64 hardware architecture.
It generates machine code for MIPS64 processors from programs written in intermediate code and stores it in corresponding object files.
It also creates a debugging information file as well as an assembly file containing a listing of the generated machine code.
\debuggingtool
\flowgraph{\resource{intermediate\\code} \ar[r] & \toolbox{cdmips64} \ar[r] \ar[d] \ar[rd] & \resource{object file} \\ & \resource{assembly\\listing} & \resource{debugging\\information}}
\seeassembly\seemips\seeobject\seecode\seedebugging
}

\providecommand{\cdmmix}{
\toolsection{cdmmix} is a compiler for intermediate code targeting the MMIX hardware architecture.
It generates machine code for MMIX processors from programs written in intermediate code and stores it in corresponding object files.
It also creates a debugging information file as well as an assembly file containing a listing of the generated machine code.
\debuggingtool
\flowgraph{\resource{intermediate\\code} \ar[r] & \toolbox{cdmmix} \ar[r] \ar[d] \ar[rd] & \resource{object file} \\ & \resource{assembly\\listing} & \resource{debugging\\information}}
\seeassembly\seemmix\seeobject\seecode\seedebugging
}

\providecommand{\cdorok}{
\toolsection{cdor1k} is a compiler for intermediate code targeting the OpenRISC 1000 hardware architecture.
It generates machine code for OpenRISC 1000 processors from programs written in intermediate code and stores it in corresponding object files.
It also creates a debugging information file as well as an assembly file containing a listing of the generated machine code.
\debuggingtool
\flowgraph{\resource{intermediate\\code} \ar[r] & \toolbox{cdor1k} \ar[r] \ar[d] \ar[rd] & \resource{object file} \\ & \resource{assembly\\listing} & \resource{debugging\\information}}
\seeassembly\seeorok\seeobject\seecode\seedebugging
}

\providecommand{\cdppca}{
\toolsection{cdppc32} is a compiler for intermediate code targeting the PowerPC hardware architecture.
It generates machine code for PowerPC processors from programs written in intermediate code and stores it in corresponding object files.
The compiler generates machine code for the 32-bit operating mode defined by the PowerPC architecture.
It also creates a debugging information file as well as an assembly file containing a listing of the generated machine code.
\debuggingtool
\flowgraph{\resource{intermediate\\code} \ar[r] & \toolbox{cdppc32} \ar[r] \ar[d] \ar[rd] & \resource{object file} \\ & \resource{assembly\\listing} & \resource{debugging\\information}}
\seeassembly\seeppc\seeobject\seecode\seedebugging
}

\providecommand{\cdppcb}{
\toolsection{cdppc64} is a compiler for intermediate code targeting the PowerPC hardware architecture.
It generates machine code for PowerPC processors from programs written in intermediate code and stores it in corresponding object files.
The compiler generates machine code for the 64-bit operating mode defined by the PowerPC architecture.
It also creates a debugging information file as well as an assembly file containing a listing of the generated machine code.
\debuggingtool
\flowgraph{\resource{intermediate\\code} \ar[r] & \toolbox{cdppc64} \ar[r] \ar[d] \ar[rd] & \resource{object file} \\ & \resource{assembly\\listing} & \resource{debugging\\information}}
\seeassembly\seeppc\seeobject\seecode\seedebugging
}

\providecommand{\cdrisc}{
\toolsection{cdrisc} is a compiler for intermediate code targeting the RISC hardware architecture.
It generates machine code for RISC processors from programs written in intermediate code and stores it in corresponding object files.
It also creates a debugging information file as well as an assembly file containing a listing of the generated machine code.
\debuggingtool
\flowgraph{\resource{intermediate\\code} \ar[r] & \toolbox{cdrisc} \ar[r] \ar[d] \ar[rd] & \resource{object file} \\ & \resource{assembly\\listing} & \resource{debugging\\information}}
\seeassembly\seerisc\seeobject\seecode\seedebugging
}

\providecommand{\cdwasm}{
\toolsection{cdwasm} is a compiler for intermediate code targeting the WebAssembly architecture.
It generates machine code for WebAssembly targets from programs written in intermediate code and stores it in corresponding object files.
It also creates a debugging information file as well as an assembly file containing a listing of the generated machine code.
\debuggingtool
\flowgraph{\resource{intermediate\\code} \ar[r] & \toolbox{cdwasm} \ar[r] \ar[d] \ar[rd] & \resource{object file} \\ & \resource{assembly\\listing} & \resource{debugging\\information}}
\seeassembly\seewasm\seeobject\seecode\seedebugging
}

% C++ tools

\providecommand{\cppprep}{
\toolsection{cppprep} is a preprocessor for the \cpp{} programming language.
It preprocesses source code according to the rules of \cpp{} and writes it to the standard output stream.
Only the macro names \texttt{\_\_DATE\_\_}, \texttt{\_\_FILE\_\_}, \texttt{\_\_LINE\_\_}, and \texttt{\_\_TIME\_\_} are predefined.
\flowgraph{\resource{\cpp{} or other\\source code} \ar[r] & \toolbox{cppprep} \ar[r] & \resource{preprocessed\\source code} \\ & \variable{ECSINCLUDE} \ar[u]}
\seecpp
}

\providecommand{\cppprint}{
\toolsection{cppprint} is a pretty printer for the \cpp{} programming language.
It reformats the source code of \cpp{} programs and writes it to the standard output stream.
\flowgraph{\resource{\cpp{}\\source code} \ar[r] & \toolbox{cppprint} \ar[r] & \resource{reformatted\\source code} \\ & \variable{ECSINCLUDE} \ar[u]}
\seecpp
}

\providecommand{\cppcheck}{
\toolsection{cppcheck} is a syntactic and semantic checker for the \cpp{} programming language.
It just performs syntactic and semantic checks on \cpp{} programs and writes its diagnostic messages to the standard error stream.
\flowgraph{\resource{\cpp{}\\source code} \ar[r] & \toolbox{cppcheck} \ar[r] & \resource{diagnostic\\messages} \\ & \variable{ECSINCLUDE} \ar[u]}
\seecpp
}

\providecommand{\cppdump}{
\toolsection{cppdump} is a serializer for the \cpp{} programming language.
It dumps the complete internal representation of programs written in \cpp{} into an XML document.
\debuggingtool
\flowgraph{\resource{\cpp{}\\source code} \ar[r] & \toolbox{cppdump} \ar[r] & \resource{internal\\representation} \\ & \variable{ECSINCLUDE} \ar[u]}
\seecpp
}

\providecommand{\cpprun}{
\toolsection{cpprun} is an interpreter for the \cpp{} programming language.
It processes and executes programs written in \cpp{}.
The macro \texttt{\_\_run\_\_} is predefined in order to enable programmers to identify this tool while interpreting.
\flowgraph{\resource{\cpp{}\\source code} \ar[r] & \toolbox{cpprun} \ar@/u/[r] & \resource{input/\\output} \ar@/d/[l] \\ & \variable{ECSINCLUDE} \ar[u]}
\seecpp
}

\providecommand{\cppdoc}{
\toolsection{cppdoc} is a generic documentation generator for the \cpp{} programming language.
It processes several \cpp{} source files and assembles all information therein into a generic documentation.
\debuggingtool
\flowgraph{\resource{\cpp{}\\source code} \ar[r] & \toolbox{cppdoc} \ar[r] & \resource{generic\\documentation} \\ & \variable{ECSINCLUDE} \ar[u]}
\seecpp\seedocumentation
}

\providecommand{\cpphtml}{
\toolsection{cpphtml} is an HTML documentation generator for the \cpp{} programming language.
It processes several \cpp{} source files and assembles all information therein into an HTML document.
\flowgraph{\resource{\cpp{}\\source code} \ar[r] & \toolbox{cpphtml} \ar[r] & \resource{HTML\\document} \\ & \variable{ECSINCLUDE} \ar[u]}
\seecpp\seedocumentation
}

\providecommand{\cpplatex}{
\toolsection{cpplatex} is a Latex documentation generator for the \cpp{} programming language.
It processes several \cpp{} source files and assembles all information therein into a Latex document.
\flowgraph{\resource{\cpp{}\\source code} \ar[r] & \toolbox{cpplatex} \ar[r] & \resource{Latex\\document} \\ & \variable{ECSINCLUDE} \ar[u]}
\seecpp\seedocumentation
}

\providecommand{\cppcode}{
\toolsection{cppcode} is an intermediate code generator for the \cpp{} programming language.
It generates intermediate code from programs written in \cpp{} and stores it in corresponding assembly files.
The macro \texttt{\_\_code\_\_} is predefined in order to enable programmers to identify this tool while generating intermediate code.
Programs generated with this tool require additional runtime support that is stored in the \file{cpp\-code\-run} library file.
\debuggingtool
\flowgraph{\resource{\cpp{}\\source code} \ar[r] & \toolbox{cppcode} \ar[r] & \resource{intermediate\\code} \\ & \variable{ECSINCLUDE} \ar[u]}
\seecpp\seeassembly\seecode
}

\providecommand{\cppamda}{
\toolsection{cppamd16} is a compiler for the \cpp{} programming language targeting the AMD64 hardware architecture.
It generates machine code for AMD64 processors from programs written in \cpp{} and stores it in corresponding object files.
The compiler generates machine code for the 16-bit operating mode defined by the AMD64 architecture.
For debugging purposes, it also creates a debugging information file as well as an assembly file containing a listing of the generated machine code.
The macro \texttt{\_\_amd16\_\_} is predefined in order to enable programmers to identify this tool and its target architecture while compiling.
Programs generated with this compiler require additional runtime support that is stored in the \file{cpp\-amd16\-run} library file.
\flowgraph{\resource{\cpp{}\\source code} \ar[r] & \toolbox{cppamd16} \ar[r] \ar[d] \ar[rd] & \resource{object file} \\ \variable{ECSINCLUDE} \ar[ru] & \resource{debugging\\information} & \resource{assembly\\listing}}
\seecpp\seeassembly\seeamd\seeobject\seedebugging
}

\providecommand{\cppamdb}{
\toolsection{cppamd32} is a compiler for the \cpp{} programming language targeting the AMD64 hardware architecture.
It generates machine code for AMD64 processors from programs written in \cpp{} and stores it in corresponding object files.
The compiler generates machine code for the 32-bit operating mode defined by the AMD64 architecture.
For debugging purposes, it also creates a debugging information file as well as an assembly file containing a listing of the generated machine code.
The macro \texttt{\_\_amd32\_\_} is predefined in order to enable programmers to identify this tool and its target architecture while compiling.
Programs generated with this compiler require additional runtime support that is stored in the \file{cpp\-amd32\-run} library file.
\flowgraph{\resource{\cpp{}\\source code} \ar[r] & \toolbox{cppamd32} \ar[r] \ar[d] \ar[rd] & \resource{object file} \\ \variable{ECSINCLUDE} \ar[ru] & \resource{debugging\\information} & \resource{assembly\\listing}}
\seecpp\seeassembly\seeamd\seeobject\seedebugging
}

\providecommand{\cppamdc}{
\toolsection{cppamd64} is a compiler for the \cpp{} programming language targeting the AMD64 hardware architecture.
It generates machine code for AMD64 processors from programs written in \cpp{} and stores it in corresponding object files.
The compiler generates machine code for the 64-bit operating mode defined by the AMD64 architecture.
For debugging purposes, it also creates a debugging information file as well as an assembly file containing a listing of the generated machine code.
The macro \texttt{\_\_amd64\_\_} is predefined in order to enable programmers to identify this tool and its target architecture while compiling.
Programs generated with this compiler require additional runtime support that is stored in the \file{cpp\-amd64\-run} library file.
\flowgraph{\resource{\cpp{}\\source code} \ar[r] & \toolbox{cppamd64} \ar[r] \ar[d] \ar[rd] & \resource{object file} \\ \variable{ECSINCLUDE} \ar[ru] & \resource{debugging\\information} & \resource{assembly\\listing}}
\seecpp\seeassembly\seeamd\seeobject\seedebugging
}

\providecommand{\cpparma}{
\toolsection{cpparma32} is a compiler for the \cpp{} programming language targeting the ARM hardware architecture.
It generates machine code for ARM processors executing A32 instructions from programs written in \cpp{} and stores it in corresponding object files.
For debugging purposes, it also creates a debugging information file as well as an assembly file containing a listing of the generated machine code.
The macro \texttt{\_\_arma32\_\_} is predefined in order to enable programmers to identify this tool and its target architecture while compiling.
Programs generated with this compiler require additional runtime support that is stored in the \file{cpp\-arma32\-run} library file.
\flowgraph{\resource{\cpp{}\\source code} \ar[r] & \toolbox{cpparma32} \ar[r] \ar[d] \ar[rd] & \resource{object file} \\ \variable{ECSINCLUDE} \ar[ru] & \resource{debugging\\information} & \resource{assembly\\listing}}
\seecpp\seeassembly\seearm\seeobject\seedebugging
}

\providecommand{\cpparmb}{
\toolsection{cpparma64} is a compiler for the \cpp{} programming language targeting the ARM hardware architecture.
It generates machine code for ARM processors executing A64 instructions from programs written in \cpp{} and stores it in corresponding object files.
For debugging purposes, it also creates a debugging information file as well as an assembly file containing a listing of the generated machine code.
The macro \texttt{\_\_arma64\_\_} is predefined in order to enable programmers to identify this tool and its target architecture while compiling.
Programs generated with this compiler require additional runtime support that is stored in the \file{cpp\-arma64\-run} library file.
\flowgraph{\resource{\cpp{}\\source code} \ar[r] & \toolbox{cpparma64} \ar[r] \ar[d] \ar[rd] & \resource{object file} \\ \variable{ECSINCLUDE} \ar[ru] & \resource{debugging\\information} & \resource{assembly\\listing}}
\seecpp\seeassembly\seearm\seeobject\seedebugging
}

\providecommand{\cpparmc}{
\toolsection{cpparmt32} is a compiler for the \cpp{} programming language targeting the ARM hardware architecture.
It generates machine code for ARM processors without floating-point extension executing T32 instructions from programs written in \cpp{} and stores it in corresponding object files.
For debugging purposes, it also creates a debugging information file as well as an assembly file containing a listing of the generated machine code.
The macro \texttt{\_\_armt32\_\_} is predefined in order to enable programmers to identify this tool and its target architecture while compiling.
Programs generated with this compiler require additional runtime support that is stored in the \file{cpp\-armt32\-run} library file.
\flowgraph{\resource{\cpp{}\\source code} \ar[r] & \toolbox{cpparmt32} \ar[r] \ar[d] \ar[rd] & \resource{object file} \\ \variable{ECSINCLUDE} \ar[ru] & \resource{debugging\\information} & \resource{assembly\\listing}}
\seecpp\seeassembly\seearm\seeobject\seedebugging
}

\providecommand{\cpparmcfpe}{
\toolsection{cpparmt32fpe} is a compiler for the \cpp{} programming language targeting the ARM hardware architecture.
It generates machine code for ARM processors with floating-point extension executing T32 instructions from programs written in \cpp{} and stores it in corresponding object files.
For debugging purposes, it also creates a debugging information file as well as an assembly file containing a listing of the generated machine code.
The macro \texttt{\_\_armt32fpe\_\_} is predefined in order to enable programmers to identify this tool and its target architecture while compiling.
Programs generated with this compiler require additional runtime support that is stored in the \file{cpp\-armt32\-fpe\-run} library file.
\flowgraph{\resource{\cpp{}\\source code} \ar[r] & \toolbox{cpparmt32fpe} \ar[r] \ar[d] \ar[rd] & \resource{object file} \\ \variable{ECSINCLUDE} \ar[ru] & \resource{debugging\\information} & \resource{assembly\\listing}}
\seecpp\seeassembly\seearm\seeobject\seedebugging
}

\providecommand{\cppavr}{
\toolsection{cppavr} is a compiler for the \cpp{} programming language targeting the AVR hardware architecture.
It generates machine code for AVR processors from programs written in \cpp{} and stores it in corresponding object files.
For debugging purposes, it also creates a debugging information file as well as an assembly file containing a listing of the generated machine code.
The macro \texttt{\_\_avr\_\_} is predefined in order to enable programmers to identify this tool and its target architecture while compiling.
Programs generated with this compiler require additional runtime support that is stored in the \file{cpp\-avr\-run} library file.
\flowgraph{\resource{\cpp{}\\source code} \ar[r] & \toolbox{cppavr} \ar[r] \ar[d] \ar[rd] & \resource{object file} \\ \variable{ECSINCLUDE} \ar[ru] & \resource{debugging\\information} & \resource{assembly\\listing}}
\seecpp\seeassembly\seeavr\seeobject\seedebugging
}

\providecommand{\cppavrtt}{
\toolsection{cppavr32} is a compiler for the \cpp{} programming language targeting the AVR32 hardware architecture.
It generates machine code for AVR32 processors from programs written in \cpp{} and stores it in corresponding object files.
For debugging purposes, it also creates a debugging information file as well as an assembly file containing a listing of the generated machine code.
The macro \texttt{\_\_avr32\_\_} is predefined in order to enable programmers to identify this tool and its target architecture while compiling.
Programs generated with this compiler require additional runtime support that is stored in the \file{cpp\-avr32\-run} library file.
\flowgraph{\resource{\cpp{}\\source code} \ar[r] & \toolbox{cppavr32} \ar[r] \ar[d] \ar[rd] & \resource{object file} \\ \variable{ECSINCLUDE} \ar[ru] & \resource{debugging\\information} & \resource{assembly\\listing}}
\seecpp\seeassembly\seeavrtt\seeobject\seedebugging
}

\providecommand{\cppmabk}{
\toolsection{cppm68k} is a compiler for the \cpp{} programming language targeting the M68000 hardware architecture.
It generates machine code for M68000 processors from programs written in \cpp{} and stores it in corresponding object files.
For debugging purposes, it also creates a debugging information file as well as an assembly file containing a listing of the generated machine code.
The macro \texttt{\_\_m68k\_\_} is predefined in order to enable programmers to identify this tool and its target architecture while compiling.
Programs generated with this compiler require additional runtime support that is stored in the \file{cpp\-m68k\-run} library file.
\flowgraph{\resource{\cpp{}\\source code} \ar[r] & \toolbox{cppm68k} \ar[r] \ar[d] \ar[rd] & \resource{object file} \\ \variable{ECSINCLUDE} \ar[ru] & \resource{debugging\\information} & \resource{assembly\\listing}}
\seecpp\seeassembly\seemabk\seeobject\seedebugging
}

\providecommand{\cppmibl}{
\toolsection{cppmibl} is a compiler for the \cpp{} programming language targeting the MicroBlaze hardware architecture.
It generates machine code for MicroBlaze processors from programs written in \cpp{} and stores it in corresponding object files.
For debugging purposes, it also creates a debugging information file as well as an assembly file containing a listing of the generated machine code.
The macro \texttt{\_\_mibl\_\_} is predefined in order to enable programmers to identify this tool and its target architecture while compiling.
Programs generated with this compiler require additional runtime support that is stored in the \file{cpp\-mibl\-run} library file.
\flowgraph{\resource{\cpp{}\\source code} \ar[r] & \toolbox{cppmibl} \ar[r] \ar[d] \ar[rd] & \resource{object file} \\ \variable{ECSINCLUDE} \ar[ru] & \resource{debugging\\information} & \resource{assembly\\listing}}
\seecpp\seeassembly\seemibl\seeobject\seedebugging
}

\providecommand{\cppmipsa}{
\toolsection{cppmips32} is a compiler for the \cpp{} programming language targeting the MIPS32 hardware architecture.
It generates machine code for MIPS32 processors from programs written in \cpp{} and stores it in corresponding object files.
For debugging purposes, it also creates a debugging information file as well as an assembly file containing a listing of the generated machine code.
The macro \texttt{\_\_mips32\_\_} is predefined in order to enable programmers to identify this tool and its target architecture while compiling.
Programs generated with this compiler require additional runtime support that is stored in the \file{cpp\-mips32\-run} library file.
\flowgraph{\resource{\cpp{}\\source code} \ar[r] & \toolbox{cppmips32} \ar[r] \ar[d] \ar[rd] & \resource{object file} \\ \variable{ECSINCLUDE} \ar[ru] & \resource{debugging\\information} & \resource{assembly\\listing}}
\seecpp\seeassembly\seemips\seeobject\seedebugging
}

\providecommand{\cppmipsb}{
\toolsection{cppmips64} is a compiler for the \cpp{} programming language targeting the MIPS64 hardware architecture.
It generates machine code for MIPS64 processors from programs written in \cpp{} and stores it in corresponding object files.
For debugging purposes, it also creates a debugging information file as well as an assembly file containing a listing of the generated machine code.
The macro \texttt{\_\_mips64\_\_} is predefined in order to enable programmers to identify this tool and its target architecture while compiling.
Programs generated with this compiler require additional runtime support that is stored in the \file{cpp\-mips64\-run} library file.
\flowgraph{\resource{\cpp{}\\source code} \ar[r] & \toolbox{cppmips64} \ar[r] \ar[d] \ar[rd] & \resource{object file} \\ \variable{ECSINCLUDE} \ar[ru] & \resource{debugging\\information} & \resource{assembly\\listing}}
\seecpp\seeassembly\seemips\seeobject\seedebugging
}

\providecommand{\cppmmix}{
\toolsection{cppmmix} is a compiler for the \cpp{} programming language targeting the MMIX hardware architecture.
It generates machine code for MMIX processors from programs written in \cpp{} and stores it in corresponding object files.
For debugging purposes, it also creates a debugging information file as well as an assembly file containing a listing of the generated machine code.
The macro \texttt{\_\_mmix\_\_} is predefined in order to enable programmers to identify this tool and its target architecture while compiling.
Programs generated with this compiler require additional runtime support that is stored in the \file{cpp\-mmix\-run} library file.
\flowgraph{\resource{\cpp{}\\source code} \ar[r] & \toolbox{cppmmix} \ar[r] \ar[d] \ar[rd] & \resource{object file} \\ \variable{ECSINCLUDE} \ar[ru] & \resource{debugging\\information} & \resource{assembly\\listing}}
\seecpp\seeassembly\seemmix\seeobject\seedebugging
}

\providecommand{\cpporok}{
\toolsection{cppor1k} is a compiler for the \cpp{} programming language targeting the OpenRISC 1000 hardware architecture.
It generates machine code for OpenRISC 1000 processors from programs written in \cpp{} and stores it in corresponding object files.
For debugging purposes, it also creates a debugging information file as well as an assembly file containing a listing of the generated machine code.
The macro \texttt{\_\_or1k\_\_} is predefined in order to enable programmers to identify this tool and its target architecture while compiling.
Programs generated with this compiler require additional runtime support that is stored in the \file{cpp\-or1k\-run} library file.
\flowgraph{\resource{\cpp{}\\source code} \ar[r] & \toolbox{cppor1k} \ar[r] \ar[d] \ar[rd] & \resource{object file} \\ \variable{ECSINCLUDE} \ar[ru] & \resource{debugging\\information} & \resource{assembly\\listing}}
\seecpp\seeassembly\seeorok\seeobject\seedebugging
}

\providecommand{\cppppca}{
\toolsection{cppppc32} is a compiler for the \cpp{} programming language targeting the PowerPC hardware architecture.
It generates machine code for PowerPC processors from programs written in \cpp{} and stores it in corresponding object files.
The compiler generates machine code for the 32-bit operating mode defined by the PowerPC architecture.
For debugging purposes, it also creates a debugging information file as well as an assembly file containing a listing of the generated machine code.
The macro \texttt{\_\_ppc32\_\_} is predefined in order to enable programmers to identify this tool and its target architecture while compiling.
Programs generated with this compiler require additional runtime support that is stored in the \file{cpp\-ppc32\-run} library file.
\flowgraph{\resource{\cpp{}\\source code} \ar[r] & \toolbox{cppppc32} \ar[r] \ar[d] \ar[rd] & \resource{object file} \\ \variable{ECSINCLUDE} \ar[ru] & \resource{debugging\\information} & \resource{assembly\\listing}}
\seecpp\seeassembly\seeppc\seeobject\seedebugging
}

\providecommand{\cppppcb}{
\toolsection{cppppc64} is a compiler for the \cpp{} programming language targeting the PowerPC hardware architecture.
It generates machine code for PowerPC processors from programs written in \cpp{} and stores it in corresponding object files.
The compiler generates machine code for the 64-bit operating mode defined by the PowerPC architecture.
For debugging purposes, it also creates a debugging information file as well as an assembly file containing a listing of the generated machine code.
The macro \texttt{\_\_ppc64\_\_} is predefined in order to enable programmers to identify this tool and its target architecture while compiling.
Programs generated with this compiler require additional runtime support that is stored in the \file{cpp\-ppc64\-run} library file.
\flowgraph{\resource{\cpp{}\\source code} \ar[r] & \toolbox{cppppc64} \ar[r] \ar[d] \ar[rd] & \resource{object file} \\ \variable{ECSINCLUDE} \ar[ru] & \resource{debugging\\information} & \resource{assembly\\listing}}
\seecpp\seeassembly\seeppc\seeobject\seedebugging
}

\providecommand{\cpprisc}{
\toolsection{cpprisc} is a compiler for the \cpp{} programming language targeting the RISC hardware architecture.
It generates machine code for RISC processors from programs written in \cpp{} and stores it in corresponding object files.
For debugging purposes, it also creates a debugging information file as well as an assembly file containing a listing of the generated machine code.
The macro \texttt{\_\_risc\_\_} is predefined in order to enable programmers to identify this tool and its target architecture while compiling.
Programs generated with this compiler require additional runtime support that is stored in the \file{cpp\-risc\-run} library file.
\flowgraph{\resource{\cpp{}\\source code} \ar[r] & \toolbox{cpprisc} \ar[r] \ar[d] \ar[rd] & \resource{object file} \\ \variable{ECSINCLUDE} \ar[ru] & \resource{debugging\\information} & \resource{assembly\\listing}}
\seecpp\seeassembly\seerisc\seeobject\seedebugging
}

\providecommand{\cppwasm}{
\toolsection{cppwasm} is a compiler for the \cpp{} programming language targeting the WebAssembly architecture.
It generates machine code for WebAssembly targets from programs written in \cpp{} and stores it in corresponding object files.
For debugging purposes, it also creates a debugging information file as well as an assembly file containing a listing of the generated machine code.
The macro \texttt{\_\_wasm\_\_} is predefined in order to enable programmers to identify this tool and its target architecture while compiling.
Programs generated with this compiler require additional runtime support that is stored in the \file{cpp\-wasm\-run} library file.
\flowgraph{\resource{\cpp{}\\source code} \ar[r] & \toolbox{cppwasm} \ar[r] \ar[d] \ar[rd] & \resource{object file} \\ \variable{ECSINCLUDE} \ar[ru] & \resource{debugging\\information} & \resource{assembly\\listing}}
\seecpp\seeassembly\seewasm\seeobject\seedebugging
}

% FALSE tools

\providecommand{\falprint}{
\toolsection{falprint} is a pretty printer for the FALSE programming language.
It reformats the source code of FALSE programs and writes it to the standard output stream.
\flowgraph{\resource{FALSE\\source code} \ar[r] & \toolbox{falprint} \ar[r] & \resource{reformatted\\source code}}
\seefalse
}

\providecommand{\falcheck}{
\toolsection{falcheck} is a syntactic and semantic checker for the FALSE programming language.
It just performs syntactic and semantic checks on FALSE programs and writes its diagnostic messages to the standard error stream.
\flowgraph{\resource{FALSE\\source code} \ar[r] & \toolbox{falcheck} \ar[r] & \resource{diagnostic\\messages}}
\seefalse
}

\providecommand{\faldump}{
\toolsection{faldump} is a serializer for the FALSE programming language.
It dumps the complete internal representation of programs written in FALSE into an XML document.
\debuggingtool
\flowgraph{\resource{FALSE\\source code} \ar[r] & \toolbox{faldump} \ar[r] & \resource{internal\\representation}}
\seefalse
}

\providecommand{\falrun}{
\toolsection{falrun} is an interpreter for the FALSE programming language.
It processes and executes programs written in FALSE\@.
\flowgraph{\resource{FALSE\\source code} \ar[r] & \toolbox{falrun} \ar@/u/[r] & \resource{input/\\output} \ar@/d/[l]}
\seefalse
}

\providecommand{\falcpp}{
\toolsection{falcpp} is a transpiler for the FALSE programming language.
It translates programs written in FALSE into \cpp{} programs and stores them in corresponding source files.
\flowgraph{\resource{FALSE\\source code} \ar[r] & \toolbox{falcpp} \ar[r] & \resource{\cpp{}\\source file}}
\seefalse\seecpp
}

\providecommand{\falcode}{
\toolsection{falcode} is an intermediate code generator for the FALSE programming language.
It generates intermediate code from programs written in FALSE and stores it in corresponding assembly files.
\debuggingtool
\flowgraph{\resource{FALSE\\source code} \ar[r] & \toolbox{falcode} \ar[r] & \resource{intermediate\\code}}
\seefalse\seeassembly\seecode
}

\providecommand{\falamda}{
\toolsection{falamd16} is a compiler for the FALSE programming language targeting the AMD64 hardware architecture.
It generates machine code for AMD64 processors from programs written in FALSE and stores it in corresponding object files.
The compiler generates machine code for the 16-bit operating mode defined by the AMD64 architecture.
\flowgraph{\resource{FALSE\\source code} \ar[r] & \toolbox{falamd16} \ar[r] & \resource{object file}}
\seefalse\seeamd\seeobject
}

\providecommand{\falamdb}{
\toolsection{falamd32} is a compiler for the FALSE programming language targeting the AMD64 hardware architecture.
It generates machine code for AMD64 processors from programs written in FALSE and stores it in corresponding object files.
The compiler generates machine code for the 32-bit operating mode defined by the AMD64 architecture.
\flowgraph{\resource{FALSE\\source code} \ar[r] & \toolbox{falamd32} \ar[r] & \resource{object file}}
\seefalse\seeamd\seeobject
}

\providecommand{\falamdc}{
\toolsection{falamd64} is a compiler for the FALSE programming language targeting the AMD64 hardware architecture.
It generates machine code for AMD64 processors from programs written in FALSE and stores it in corresponding object files.
The compiler generates machine code for the 64-bit operating mode defined by the AMD64 architecture.
\flowgraph{\resource{FALSE\\source code} \ar[r] & \toolbox{falamd64} \ar[r] & \resource{object file}}
\seefalse\seeamd\seeobject
}

\providecommand{\falarma}{
\toolsection{falarma32} is a compiler for the FALSE programming language targeting the ARM hardware architecture.
It generates machine code for ARM processors executing A32 instructions from programs written in FALSE and stores it in corresponding object files.
\flowgraph{\resource{FALSE\\source code} \ar[r] & \toolbox{falarma32} \ar[r] & \resource{object file}}
\seefalse\seearm\seeobject
}

\providecommand{\falarmb}{
\toolsection{falarma64} is a compiler for the FALSE programming language targeting the ARM hardware architecture.
It generates machine code for ARM processors executing A64 instructions from programs written in FALSE and stores it in corresponding object files.
\flowgraph{\resource{FALSE\\source code} \ar[r] & \toolbox{falarma64} \ar[r] & \resource{object file}}
\seefalse\seearm\seeobject
}

\providecommand{\falarmc}{
\toolsection{falarmt32} is a compiler for the FALSE programming language targeting the ARM hardware architecture.
It generates machine code for ARM processors without floating-point extension executing T32 instructions from programs written in FALSE and stores it in corresponding object files.
\flowgraph{\resource{FALSE\\source code} \ar[r] & \toolbox{falarmt32} \ar[r] & \resource{object file}}
\seefalse\seearm\seeobject
}

\providecommand{\falarmcfpe}{
\toolsection{falarmt32fpe} is a compiler for the FALSE programming language targeting the ARM hardware architecture.
It generates machine code for ARM processors with floating-point extension executing T32 instructions from programs written in FALSE and stores it in corresponding object files.
\flowgraph{\resource{FALSE\\source code} \ar[r] & \toolbox{falarmt32fpe} \ar[r] & \resource{object file}}
\seefalse\seearm\seeobject
}

\providecommand{\falavr}{
\toolsection{falavr} is a compiler for the FALSE programming language targeting the AVR hardware architecture.
It generates machine code for AVR processors from programs written in FALSE and stores it in corresponding object files.
\flowgraph{\resource{FALSE\\source code} \ar[r] & \toolbox{falavr} \ar[r] & \resource{object file}}
\seefalse\seeavr\seeobject
}

\providecommand{\falavrtt}{
\toolsection{falavr32} is a compiler for the FALSE programming language targeting the AVR32 hardware architecture.
It generates machine code for AVR32 processors from programs written in FALSE and stores it in corresponding object files.
\flowgraph{\resource{FALSE\\source code} \ar[r] & \toolbox{falavr32} \ar[r] & \resource{object file}}
\seefalse\seeavrtt\seeobject
}

\providecommand{\falmabk}{
\toolsection{falm68k} is a compiler for the FALSE programming language targeting the M68000 hardware architecture.
It generates machine code for M68000 processors from programs written in FALSE and stores it in corresponding object files.
\flowgraph{\resource{FALSE\\source code} \ar[r] & \toolbox{falm68k} \ar[r] & \resource{object file}}
\seefalse\seemabk\seeobject
}

\providecommand{\falmibl}{
\toolsection{falmibl} is a compiler for the FALSE programming language targeting the MicroBlaze hardware architecture.
It generates machine code for MicroBlaze processors from programs written in FALSE and stores it in corresponding object files.
\flowgraph{\resource{FALSE\\source code} \ar[r] & \toolbox{falmibl} \ar[r] & \resource{object file}}
\seefalse\seemibl\seeobject
}

\providecommand{\falmipsa}{
\toolsection{falmips32} is a compiler for the FALSE programming language targeting the MIPS32 hardware architecture.
It generates machine code for MIPS32 processors from programs written in FALSE and stores it in corresponding object files.
\flowgraph{\resource{FALSE\\source code} \ar[r] & \toolbox{falmips32} \ar[r] & \resource{object file}}
\seefalse\seemips\seeobject
}

\providecommand{\falmipsb}{
\toolsection{falmips64} is a compiler for the FALSE programming language targeting the MIPS64 hardware architecture.
It generates machine code for MIPS64 processors from programs written in FALSE and stores it in corresponding object files.
\flowgraph{\resource{FALSE\\source code} \ar[r] & \toolbox{falmips64} \ar[r] & \resource{object file}}
\seefalse\seemips\seeobject
}

\providecommand{\falmmix}{
\toolsection{falmmix} is a compiler for the FALSE programming language targeting the MMIX hardware architecture.
It generates machine code for MMIX processors from programs written in FALSE and stores it in corresponding object files.
\flowgraph{\resource{FALSE\\source code} \ar[r] & \toolbox{falmmix} \ar[r] & \resource{object file}}
\seefalse\seemmix\seeobject
}

\providecommand{\falorok}{
\toolsection{falor1k} is a compiler for the FALSE programming language targeting the OpenRISC 1000 hardware architecture.
It generates machine code for OpenRISC 1000 processors from programs written in FALSE and stores it in corresponding object files.
\flowgraph{\resource{FALSE\\source code} \ar[r] & \toolbox{falor1k} \ar[r] & \resource{object file}}
\seefalse\seeorok\seeobject
}

\providecommand{\falppca}{
\toolsection{falppc32} is a compiler for the FALSE programming language targeting the PowerPC hardware architecture.
It generates machine code for PowerPC processors from programs written in FALSE and stores it in corresponding object files.
The compiler generates machine code for the 32-bit operating mode defined by the PowerPC architecture.
\flowgraph{\resource{FALSE\\source code} \ar[r] & \toolbox{falppc32} \ar[r] & \resource{object file}}
\seefalse\seeppc\seeobject
}

\providecommand{\falppcb}{
\toolsection{falppc64} is a compiler for the FALSE programming language targeting the PowerPC hardware architecture.
It generates machine code for PowerPC processors from programs written in FALSE and stores it in corresponding object files.
The compiler generates machine code for the 64-bit operating mode defined by the PowerPC architecture.
\flowgraph{\resource{FALSE\\source code} \ar[r] & \toolbox{falppc64} \ar[r] & \resource{object file}}
\seefalse\seeppc\seeobject
}

\providecommand{\falrisc}{
\toolsection{falrisc} is a compiler for the FALSE programming language targeting the RISC hardware architecture.
It generates machine code for RISC processors from programs written in FALSE and stores it in corresponding object files.
\flowgraph{\resource{FALSE\\source code} \ar[r] & \toolbox{falrisc} \ar[r] & \resource{object file}}
\seefalse\seerisc\seeobject
}

\providecommand{\falwasm}{
\toolsection{falwasm} is a compiler for the FALSE programming language targeting the WebAssembly architecture.
It generates machine code for WebAssembly targets from programs written in FALSE and stores it in corresponding object files.
\flowgraph{\resource{FALSE\\source code} \ar[r] & \toolbox{falwasm} \ar[r] & \resource{object file}}
\seefalse\seewasm\seeobject
}

% Oberon tools

\providecommand{\obprint}{
\toolsection{obprint} is a pretty printer for the Oberon programming language.
It reformats the source code of Oberon modules and writes it to the standard output stream.
\flowgraph{\resource{Oberon\\source code} \ar[r] & \toolbox{obprint} \ar[r] & \resource{reformatted\\source code}}
\seeoberon
}

\providecommand{\obcheck}{
\toolsection{obcheck} is a syntactic and semantic checker for the Oberon programming language.
It just performs syntactic and semantic checks on Oberon modules and writes its diagnostic messages to the standard error stream.
In addition, it stores the interface of each module in a symbol file which is required when other modules import the module.
\flowgraph{\resource{Oberon\\source code} \ar[r] & \toolbox{obcheck} \ar[r] \ar@/l/[d] & \resource{diagnostic\\messages} \\ \variable{ECSIMPORT} \ar[ru] & \resource{symbol\\files} \ar@/r/[u]}
\seeoberon
}

\providecommand{\obdump}{
\toolsection{obdump} is a serializer for the Oberon programming language.
It dumps the complete internal representation of modules written in Oberon into an XML document.
\debuggingtool
\flowgraph{\resource{Oberon\\source code} \ar[r] & \toolbox{obdump} \ar[r] \ar@/l/[d] & \resource{internal\\representation} \\ \variable{ECSIMPORT} \ar[ru] & \resource{symbol\\files} \ar@/r/[u]}
\seeoberon
}

\providecommand{\obrun}{
\toolsection{obrun} is an interpreter for the Oberon programming language.
It processes and executes modules written in Oberon.
This tool does neither generate nor process symbol files while interpreting modules.
If a module is imported by another one, its filename has to be named before the other one in the list of command-line arguments.
\flowgraph{\resource{Oberon\\source code} \ar[r] & \toolbox{obrun} \ar@/u/[r] & \resource{input/\\output} \ar@/d/[l]}
\seeoberon
}

\providecommand{\obcpp}{
\toolsection{obcpp} is a transpiler for the Oberon programming language.
It translates programs written in Oberon into \cpp{} programs and stores them in corresponding source and header files.
In addition, it stores the interface of each module in a symbol file which is required when other modules import the module.
The same interface is provided by the generated header file which can be used in other parts of the \cpp{} program.
\flowgraph{\resource{Oberon\\source code} \ar[r] & \toolbox{obcpp} \ar[r] \ar@/l/[d] \ar[rd] & \resource{\cpp{}\\source file} \\ \variable{ECSIMPORT} \ar[ru] & \resource{symbol\\files} \ar@/r/[u] & \resource{\cpp{}\\header file}}
\seeoberon\seecpp
}

\providecommand{\obdoc}{
\toolsection{obdoc} is a generic documentation generator for the Oberon programming language.
It processes several Oberon modules and assembles all information therein into a generic documentation.
In addition, it stores the interface of each module in a symbol file which is required when other modules import the module.
\debuggingtool
\flowgraph{\resource{Oberon\\source code} \ar[r] & \toolbox{obdoc} \ar[r] \ar@/l/[d] & \resource{generic\\documentation} \\ \variable{ECSIMPORT} \ar[ru] & \resource{symbol\\files} \ar@/r/[u]}
\seeoberon\seedocumentation
}

\providecommand{\obhtml}{
\toolsection{obhtml} is an HTML documentation generator for the Oberon programming language.
It processes several Oberon modules and assembles all information therein into an HTML document.
In addition, it stores the interface of each module in a symbol file which is required when other modules import the module.
\flowgraph{\resource{Oberon\\source code} \ar[r] & \toolbox{obhtml} \ar[r] \ar@/l/[d] & \resource{HTML\\document} \\ \variable{ECSIMPORT} \ar[ru] & \resource{symbol\\files} \ar@/r/[u]}
\seeoberon\seedocumentation
}

\providecommand{\oblatex}{
\toolsection{oblatex} is a Latex documentation generator for the Oberon programming language.
It processes several Oberon modules and assembles all information therein into a Latex document.
In addition, it stores the interface of each module in a symbol file which is required when other modules import the module.
\flowgraph{\resource{Oberon\\source code} \ar[r] & \toolbox{oblatex} \ar[r] \ar@/l/[d] & \resource{Latex\\document} \\ \variable{ECSIMPORT} \ar[ru] & \resource{symbol\\files} \ar@/r/[u]}
\seeoberon\seedocumentation
}

\providecommand{\obcode}{
\toolsection{obcode} is an intermediate code generator for the Oberon programming language.
It generates intermediate code from modules written in Oberon and stores it in corresponding assembly files.
In addition, it stores the interface of each module in a symbol file which is required when other modules import the module.
Programs generated with this tool require additional runtime support that is stored in the \file{ob\-code\-run} library file.
\debuggingtool
\flowgraph{\resource{Oberon\\source code} \ar[r] & \toolbox{obcode} \ar[r] \ar@/l/[d] & \resource{intermediate\\code} \\ \variable{ECSIMPORT} \ar[ru] & \resource{symbol\\files} \ar@/r/[u]}
\seeoberon\seeassembly\seecode
}

\providecommand{\obamda}{
\toolsection{obamd16} is a compiler for the Oberon programming language targeting the AMD64 hardware architecture.
It generates machine code for AMD64 processors from modules written in Oberon and stores it in corresponding object files.
The compiler generates machine code for the 16-bit operating mode defined by the AMD64 architecture.
For debugging purposes, it also creates a debugging information file as well as an assembly file containing a listing of the generated machine code.
In addition, it stores the interface of each module in a symbol file which is required when other modules import the module.
Programs generated with this compiler require additional runtime support that is stored in the \file{ob\-amd16\-run} library file.
\flowgraph{\resource{Oberon\\source code} \ar[r] & \toolbox{obamd16} \ar[r] \ar@/l/[d] \ar[rd] & \resource{object file} \\ \variable{ECSIMPORT} \ar[ru] & \resource{symbol\\files} \ar@/r/[u] & \resource{debugging\\information}}
\seeoberon\seeassembly\seeamd\seeobject\seedebugging
}

\providecommand{\obamdb}{
\toolsection{obamd32} is a compiler for the Oberon programming language targeting the AMD64 hardware architecture.
It generates machine code for AMD64 processors from modules written in Oberon and stores it in corresponding object files.
The compiler generates machine code for the 32-bit operating mode defined by the AMD64 architecture.
For debugging purposes, it also creates a debugging information file as well as an assembly file containing a listing of the generated machine code.
In addition, it stores the interface of each module in a symbol file which is required when other modules import the module.
Programs generated with this compiler require additional runtime support that is stored in the \file{ob\-amd32\-run} library file.
\flowgraph{\resource{Oberon\\source code} \ar[r] & \toolbox{obamd32} \ar[r] \ar@/l/[d] \ar[rd] & \resource{object file} \\ \variable{ECSIMPORT} \ar[ru] & \resource{symbol\\files} \ar@/r/[u] & \resource{debugging\\information}}
\seeoberon\seeassembly\seeamd\seeobject\seedebugging
}

\providecommand{\obamdc}{
\toolsection{obamd64} is a compiler for the Oberon programming language targeting the AMD64 hardware architecture.
It generates machine code for AMD64 processors from modules written in Oberon and stores it in corresponding object files.
The compiler generates machine code for the 64-bit operating mode defined by the AMD64 architecture.
For debugging purposes, it also creates a debugging information file as well as an assembly file containing a listing of the generated machine code.
In addition, it stores the interface of each module in a symbol file which is required when other modules import the module.
Programs generated with this compiler require additional runtime support that is stored in the \file{ob\-amd64\-run} library file.
\flowgraph{\resource{Oberon\\source code} \ar[r] & \toolbox{obamd64} \ar[r] \ar@/l/[d] \ar[rd] & \resource{object file} \\ \variable{ECSIMPORT} \ar[ru] & \resource{symbol\\files} \ar@/r/[u] & \resource{debugging\\information}}
\seeoberon\seeassembly\seeamd\seeobject\seedebugging
}

\providecommand{\obarma}{
\toolsection{obarma32} is a compiler for the Oberon programming language targeting the ARM hardware architecture.
It generates machine code for ARM processors executing A32 instructions from modules written in Oberon and stores it in corresponding object files.
For debugging purposes, it also creates a debugging information file as well as an assembly file containing a listing of the generated machine code.
In addition, it stores the interface of each module in a symbol file which is required when other modules import the module.
Programs generated with this compiler require additional runtime support that is stored in the \file{ob\-arma32\-run} library file.
\flowgraph{\resource{Oberon\\source code} \ar[r] & \toolbox{obarma32} \ar[r] \ar@/l/[d] \ar[rd] & \resource{object file} \\ \variable{ECSIMPORT} \ar[ru] & \resource{symbol\\files} \ar@/r/[u] & \resource{debugging\\information}}
\seeoberon\seeassembly\seearm\seeobject\seedebugging
}

\providecommand{\obarmb}{
\toolsection{obarma64} is a compiler for the Oberon programming language targeting the ARM hardware architecture.
It generates machine code for ARM processors executing A64 instructions from modules written in Oberon and stores it in corresponding object files.
For debugging purposes, it also creates a debugging information file as well as an assembly file containing a listing of the generated machine code.
In addition, it stores the interface of each module in a symbol file which is required when other modules import the module.
Programs generated with this compiler require additional runtime support that is stored in the \file{ob\-arma64\-run} library file.
\flowgraph{\resource{Oberon\\source code} \ar[r] & \toolbox{obarma64} \ar[r] \ar@/l/[d] \ar[rd] & \resource{object file} \\ \variable{ECSIMPORT} \ar[ru] & \resource{symbol\\files} \ar@/r/[u] & \resource{debugging\\information}}
\seeoberon\seeassembly\seearm\seeobject\seedebugging
}

\providecommand{\obarmc}{
\toolsection{obarmt32} is a compiler for the Oberon programming language targeting the ARM hardware architecture.
It generates machine code for ARM processors without floating-point extension executing T32 instructions from modules written in Oberon and stores it in corresponding object files.
For debugging purposes, it also creates a debugging information file as well as an assembly file containing a listing of the generated machine code.
In addition, it stores the interface of each module in a symbol file which is required when other modules import the module.
Programs generated with this compiler require additional runtime support that is stored in the \file{ob\-armt32\-run} library file.
\flowgraph{\resource{Oberon\\source code} \ar[r] & \toolbox{obarmt32} \ar[r] \ar@/l/[d] \ar[rd] & \resource{object file} \\ \variable{ECSIMPORT} \ar[ru] & \resource{symbol\\files} \ar@/r/[u] & \resource{debugging\\information}}
\seeoberon\seeassembly\seearm\seeobject\seedebugging
}

\providecommand{\obarmcfpe}{
\toolsection{obarmt32fpe} is a compiler for the Oberon programming language targeting the ARM hardware architecture.
It generates machine code for ARM processors with floating-point extension executing T32 instructions from modules written in Oberon and stores it in corresponding object files.
For debugging purposes, it also creates a debugging information file as well as an assembly file containing a listing of the generated machine code.
In addition, it stores the interface of each module in a symbol file which is required when other modules import the module.
Programs generated with this compiler require additional runtime support that is stored in the \file{ob\-armt32\-fpe\-run} library file.
\flowgraph{\resource{Oberon\\source code} \ar[r] & \toolbox{obarmt32fpe} \ar[r] \ar@/l/[d] \ar[rd] & \resource{object file} \\ \variable{ECSIMPORT} \ar[ru] & \resource{symbol\\files} \ar@/r/[u] & \resource{debugging\\information}}
\seeoberon\seeassembly\seearm\seeobject\seedebugging
}

\providecommand{\obavr}{
\toolsection{obavr} is a compiler for the Oberon programming language targeting the AVR hardware architecture.
It generates machine code for AVR processors from modules written in Oberon and stores it in corresponding object files.
For debugging purposes, it also creates a debugging information file as well as an assembly file containing a listing of the generated machine code.
In addition, it stores the interface of each module in a symbol file which is required when other modules import the module.
Programs generated with this compiler require additional runtime support that is stored in the \file{ob\-avr\-run} library file.
\flowgraph{\resource{Oberon\\source code} \ar[r] & \toolbox{obavr} \ar[r] \ar@/l/[d] \ar[rd] & \resource{object file} \\ \variable{ECSIMPORT} \ar[ru] & \resource{symbol\\files} \ar@/r/[u] & \resource{debugging\\information}}
\seeoberon\seeassembly\seeavr\seeobject\seedebugging
}

\providecommand{\obavrtt}{
\toolsection{obavr32} is a compiler for the Oberon programming language targeting the AVR32 hardware architecture.
It generates machine code for AVR32 processors from modules written in Oberon and stores it in corresponding object files.
For debugging purposes, it also creates a debugging information file as well as an assembly file containing a listing of the generated machine code.
In addition, it stores the interface of each module in a symbol file which is required when other modules import the module.
Programs generated with this compiler require additional runtime support that is stored in the \file{ob\-avr32\-run} library file.
\flowgraph{\resource{Oberon\\source code} \ar[r] & \toolbox{obavr32} \ar[r] \ar@/l/[d] \ar[rd] & \resource{object file} \\ \variable{ECSIMPORT} \ar[ru] & \resource{symbol\\files} \ar@/r/[u] & \resource{debugging\\information}}
\seeoberon\seeassembly\seeavrtt\seeobject\seedebugging
}

\providecommand{\obmabk}{
\toolsection{obm68k} is a compiler for the Oberon programming language targeting the M68000 hardware architecture.
It generates machine code for M68000 processors from modules written in Oberon and stores it in corresponding object files.
For debugging purposes, it also creates a debugging information file as well as an assembly file containing a listing of the generated machine code.
In addition, it stores the interface of each module in a symbol file which is required when other modules import the module.
Programs generated with this compiler require additional runtime support that is stored in the \file{ob\-m68k\-run} library file.
\flowgraph{\resource{Oberon\\source code} \ar[r] & \toolbox{obm68k} \ar[r] \ar@/l/[d] \ar[rd] & \resource{object file} \\ \variable{ECSIMPORT} \ar[ru] & \resource{symbol\\files} \ar@/r/[u] & \resource{debugging\\information}}
\seeoberon\seeassembly\seemabk\seeobject\seedebugging
}

\providecommand{\obmibl}{
\toolsection{obmibl} is a compiler for the Oberon programming language targeting the MicroBlaze hardware architecture.
It generates machine code for MicroBlaze processors from modules written in Oberon and stores it in corresponding object files.
For debugging purposes, it also creates a debugging information file as well as an assembly file containing a listing of the generated machine code.
In addition, it stores the interface of each module in a symbol file which is required when other modules import the module.
Programs generated with this compiler require additional runtime support that is stored in the \file{ob\-mibl\-run} library file.
\flowgraph{\resource{Oberon\\source code} \ar[r] & \toolbox{obmibl} \ar[r] \ar@/l/[d] \ar[rd] & \resource{object file} \\ \variable{ECSIMPORT} \ar[ru] & \resource{symbol\\files} \ar@/r/[u] & \resource{debugging\\information}}
\seeoberon\seeassembly\seemibl\seeobject\seedebugging
}

\providecommand{\obmipsa}{
\toolsection{obmips32} is a compiler for the Oberon programming language targeting the MIPS32 hardware architecture.
It generates machine code for MIPS32 processors from modules written in Oberon and stores it in corresponding object files.
For debugging purposes, it also creates a debugging information file as well as an assembly file containing a listing of the generated machine code.
In addition, it stores the interface of each module in a symbol file which is required when other modules import the module.
Programs generated with this compiler require additional runtime support that is stored in the \file{ob\-mips32\-run} library file.
\flowgraph{\resource{Oberon\\source code} \ar[r] & \toolbox{obmips32} \ar[r] \ar@/l/[d] \ar[rd] & \resource{object file} \\ \variable{ECSIMPORT} \ar[ru] & \resource{symbol\\files} \ar@/r/[u] & \resource{debugging\\information}}
\seeoberon\seeassembly\seemips\seeobject\seedebugging
}

\providecommand{\obmipsb}{
\toolsection{obmips64} is a compiler for the Oberon programming language targeting the MIPS64 hardware architecture.
It generates machine code for MIPS64 processors from modules written in Oberon and stores it in corresponding object files.
For debugging purposes, it also creates a debugging information file as well as an assembly file containing a listing of the generated machine code.
In addition, it stores the interface of each module in a symbol file which is required when other modules import the module.
Programs generated with this compiler require additional runtime support that is stored in the \file{ob\-mips64\-run} library file.
\flowgraph{\resource{Oberon\\source code} \ar[r] & \toolbox{obmips64} \ar[r] \ar@/l/[d] \ar[rd] & \resource{object file} \\ \variable{ECSIMPORT} \ar[ru] & \resource{symbol\\files} \ar@/r/[u] & \resource{debugging\\information}}
\seeoberon\seeassembly\seemips\seeobject\seedebugging
}

\providecommand{\obmmix}{
\toolsection{obmmix} is a compiler for the Oberon programming language targeting the MMIX hardware architecture.
It generates machine code for MMIX processors from modules written in Oberon and stores it in corresponding object files.
For debugging purposes, it also creates a debugging information file as well as an assembly file containing a listing of the generated machine code.
In addition, it stores the interface of each module in a symbol file which is required when other modules import the module.
Programs generated with this compiler require additional runtime support that is stored in the \file{ob\-mmix\-run} library file.
\flowgraph{\resource{Oberon\\source code} \ar[r] & \toolbox{obmmix} \ar[r] \ar@/l/[d] \ar[rd] & \resource{object file} \\ \variable{ECSIMPORT} \ar[ru] & \resource{symbol\\files} \ar@/r/[u] & \resource{debugging\\information}}
\seeoberon\seeassembly\seemmix\seeobject\seedebugging
}

\providecommand{\oborok}{
\toolsection{obor1k} is a compiler for the Oberon programming language targeting the OpenRISC 1000 hardware architecture.
It generates machine code for OpenRISC 1000 processors from modules written in Oberon and stores it in corresponding object files.
For debugging purposes, it also creates a debugging information file as well as an assembly file containing a listing of the generated machine code.
In addition, it stores the interface of each module in a symbol file which is required when other modules import the module.
Programs generated with this compiler require additional runtime support that is stored in the \file{ob\-or1k\-run} library file.
\flowgraph{\resource{Oberon\\source code} \ar[r] & \toolbox{obor1k} \ar[r] \ar@/l/[d] \ar[rd] & \resource{object file} \\ \variable{ECSIMPORT} \ar[ru] & \resource{symbol\\files} \ar@/r/[u] & \resource{debugging\\information}}
\seeoberon\seeassembly\seeorok\seeobject\seedebugging
}

\providecommand{\obppca}{
\toolsection{obppc32} is a compiler for the Oberon programming language targeting the PowerPC hardware architecture.
It generates machine code for PowerPC processors from modules written in Oberon and stores it in corresponding object files.
The compiler generates machine code for the 32-bit operating mode defined by the PowerPC architecture.
For debugging purposes, it also creates a debugging information file as well as an assembly file containing a listing of the generated machine code.
In addition, it stores the interface of each module in a symbol file which is required when other modules import the module.
Programs generated with this compiler require additional runtime support that is stored in the \file{ob\-ppc32\-run} library file.
\flowgraph{\resource{Oberon\\source code} \ar[r] & \toolbox{obppc32} \ar[r] \ar@/l/[d] \ar[rd] & \resource{object file} \\ \variable{ECSIMPORT} \ar[ru] & \resource{symbol\\files} \ar@/r/[u] & \resource{debugging\\information}}
\seeoberon\seeassembly\seeppc\seeobject\seedebugging
}

\providecommand{\obppcb}{
\toolsection{obppc64} is a compiler for the Oberon programming language targeting the PowerPC hardware architecture.
It generates machine code for PowerPC processors from modules written in Oberon and stores it in corresponding object files.
The compiler generates machine code for the 64-bit operating mode defined by the PowerPC architecture.
For debugging purposes, it also creates a debugging information file as well as an assembly file containing a listing of the generated machine code.
In addition, it stores the interface of each module in a symbol file which is required when other modules import the module.
Programs generated with this compiler require additional runtime support that is stored in the \file{ob\-ppc64\-run} library file.
\flowgraph{\resource{Oberon\\source code} \ar[r] & \toolbox{obppc64} \ar[r] \ar@/l/[d] \ar[rd] & \resource{object file} \\ \variable{ECSIMPORT} \ar[ru] & \resource{symbol\\files} \ar@/r/[u] & \resource{debugging\\information}}
\seeoberon\seeassembly\seeppc\seeobject\seedebugging
}

\providecommand{\obrisc}{
\toolsection{obrisc} is a compiler for the Oberon programming language targeting the RISC hardware architecture.
It generates machine code for RISC processors from modules written in Oberon and stores it in corresponding object files.
For debugging purposes, it also creates a debugging information file as well as an assembly file containing a listing of the generated machine code.
In addition, it stores the interface of each module in a symbol file which is required when other modules import the module.
Programs generated with this compiler require additional runtime support that is stored in the \file{ob\-risc\-run} library file.
\flowgraph{\resource{Oberon\\source code} \ar[r] & \toolbox{obrisc} \ar[r] \ar@/l/[d] \ar[rd] & \resource{object file} \\ \variable{ECSIMPORT} \ar[ru] & \resource{symbol\\files} \ar@/r/[u] & \resource{debugging\\information}}
\seeoberon\seeassembly\seerisc\seeobject\seedebugging
}

\providecommand{\obwasm}{
\toolsection{obwasm} is a compiler for the Oberon programming language targeting the WebAssembly architecture.
It generates machine code for WebAssembly targets from modules written in Oberon and stores it in corresponding object files.
For debugging purposes, it also creates a debugging information file as well as an assembly file containing a listing of the generated machine code.
In addition, it stores the interface of each module in a symbol file which is required when other modules import the module.
Programs generated with this compiler require additional runtime support that is stored in the \file{ob\-wasm\-run} library file.
\flowgraph{\resource{Oberon\\source code} \ar[r] & \toolbox{obwasm} \ar[r] \ar@/l/[d] \ar[rd] & \resource{object file} \\ \variable{ECSIMPORT} \ar[ru] & \resource{symbol\\files} \ar@/r/[u] & \resource{debugging\\information}}
\seeoberon\seeassembly\seewasm\seeobject\seedebugging
}

% converter tools

\providecommand{\dbgdwarf}{
\toolsection{dbgdwarf} is a DWARF debugging information converter tool.
It converts debugging information into the DWARF debugging data format and stores it in corresponding object files~\cite{dwarffile}.
The resulting debugging object files can be combined with runtime support that creates Executable and Linking Format (ELF) files~\cite{elffile}.
\flowgraph{\resource{debugging\\information} \ar[r] & \toolbox{dbgdwarf} \ar[r] & \resource{debugging\\object file}}
\seeobject\seedebugging
}

% assembler tools

\providecommand{\asmprint}{
\toolsection{asmprint} is a pretty printer for generic assembly code.
It reformats generic assembly code and writes it to the standard output stream.
\flowgraph{\resource{generic assembly\\source code} \ar[r] & \toolbox{asmprint} \ar[r] & \resource{reformatted\\source code}}
\seeassembly
}

\providecommand{\amdaasm}{
\toolsection{amd16asm} is an assembler for the AMD64 hardware architecture.
It translates assembly code into machine code for AMD64 processors and stores it in corresponding object files.
By default, the assembler generates machine code for the 16-bit operating mode defined by the AMD64 architecture.
\flowgraph{\resource{AMD16 assembly\\source code} \ar[r] & \toolbox{amd16asm} \ar[r] & \resource{object file}}
\seeassembly\seeamd\seeobject
}

\providecommand{\amdadism}{
\toolsection{amd16dism} is a disassembler for the AMD64 hardware architecture.
It translates machine code from object files targeting AMD64 processors into assembly code and writes it to the standard output stream.
It assumes that the machine code was generated for the 16-bit operating mode defined by the AMD64 architecture.
\flowgraph{\resource{object file} \ar[r] & \toolbox{amd16dism} \ar[r] & \resource{disassembly\\listing}}
\seeassembly\seeamd\seeobject
}

\providecommand{\amdbasm}{
\toolsection{amd32asm} is an assembler for the AMD64 hardware architecture.
It translates assembly code into machine code for AMD64 processors and stores it in corresponding object files.
By default, the assembler generates machine code for the 32-bit operating mode defined by the AMD64 architecture.
\flowgraph{\resource{AMD32 assembly\\source code} \ar[r] & \toolbox{amd32asm} \ar[r] & \resource{object file}}
\seeassembly\seeamd\seeobject
}

\providecommand{\amdbdism}{
\toolsection{amd32dism} is a disassembler for the AMD64 hardware architecture.
It translates machine code from object files targeting AMD64 processors into assembly code and writes it to the standard output stream.
It assumes that the machine code was generated for the 32-bit operating mode defined by the AMD64 architecture.
\flowgraph{\resource{object file} \ar[r] & \toolbox{amd32dism} \ar[r] & \resource{disassembly\\listing}}
\seeassembly\seeamd\seeobject
}

\providecommand{\amdcasm}{
\toolsection{amd64asm} is an assembler for the AMD64 hardware architecture.
It translates assembly code into machine code for AMD64 processors and stores it in corresponding object files.
By default, the assembler generates machine code for the 64-bit operating mode defined by the AMD64 architecture.
\flowgraph{\resource{AMD64 assembly\\source code} \ar[r] & \toolbox{amd64asm} \ar[r] & \resource{object file}}
\seeassembly\seeamd\seeobject
}

\providecommand{\amdcdism}{
\toolsection{amd64dism} is a disassembler for the AMD64 hardware architecture.
It translates machine code from object files targeting AMD64 processors into assembly code and writes it to the standard output stream.
It assumes that the machine code was generated for the 64-bit operating mode defined by the AMD64 architecture.
\flowgraph{\resource{object file} \ar[r] & \toolbox{amd64dism} \ar[r] & \resource{disassembly\\listing}}
\seeassembly\seeamd\seeobject
}

\providecommand{\armaasm}{
\toolsection{arma32asm} is an assembler for the ARM hardware architecture.
It translates assembly code into machine code for ARM processors executing A32 instructions and stores it in corresponding object files.
\flowgraph{\resource{ARM A32 assembly\\source code} \ar[r] & \toolbox{arma32asm} \ar[r] & \resource{object file}}
\seeassembly\seearm\seeobject
}

\providecommand{\armadism}{
\toolsection{arma32dism} is a disassembler for the ARM hardware architecture.
It translates machine code from object files targeting ARM processors executing A32 instructions into assembly code and writes it to the standard output stream.
\flowgraph{\resource{object file} \ar[r] & \toolbox{arma32dism} \ar[r] & \resource{disassembly\\listing}}
\seeassembly\seearm\seeobject
}

\providecommand{\armbasm}{
\toolsection{arma64asm} is an assembler for the ARM hardware architecture.
It translates assembly code into machine code for ARM processors executing A64 instructions and stores it in corresponding object files.
\flowgraph{\resource{ARM A64 assembly\\source code} \ar[r] & \toolbox{arma64asm} \ar[r] & \resource{object file}}
\seeassembly\seearm\seeobject
}

\providecommand{\armbdism}{
\toolsection{arma64dism} is a disassembler for the ARM hardware architecture.
It translates machine code from object files targeting ARM processors executing A64 instructions into assembly code and writes it to the standard output stream.
\flowgraph{\resource{object file} \ar[r] & \toolbox{arma64dism} \ar[r] & \resource{disassembly\\listing}}
\seeassembly\seearm\seeobject
}

\providecommand{\armcasm}{
\toolsection{armt32asm} is an assembler for the ARM hardware architecture.
It translates assembly code into machine code for ARM processors executing T32 instructions and stores it in corresponding object files.
\flowgraph{\resource{ARM T32 assembly\\source code} \ar[r] & \toolbox{armt32asm} \ar[r] & \resource{object file}}
\seeassembly\seearm\seeobject
}

\providecommand{\armcdism}{
\toolsection{armt32dism} is a disassembler for the ARM hardware architecture.
It translates machine code from object files targeting ARM processors executing T32 instructions into assembly code and writes it to the standard output stream.
\flowgraph{\resource{object file} \ar[r] & \toolbox{armt32dism} \ar[r] & \resource{disassembly\\listing}}
\seeassembly\seearm\seeobject
}

\providecommand{\avrasm}{
\toolsection{avrasm} is an assembler for the AVR hardware architecture.
It translates assembly code into machine code for AVR processors and stores it in corresponding object files.
The identifiers \texttt{RXL}, \texttt{RXH}, \texttt{RYL}, \texttt{RYH}, \texttt{RZL}, and \texttt{RZH} are predefined and name the corresponding registers.
The identifiers \texttt{SPL} and \texttt{SPH} are also predefined and evaluate to the address of the corresponding registers.
\flowgraph{\resource{AVR assembly\\source code} \ar[r] & \toolbox{avrasm} \ar[r] & \resource{object file}}
\seeassembly\seeavr\seeobject
}

\providecommand{\avrdism}{
\toolsection{avrdism} is a disassembler for the AVR hardware architecture.
It translates machine code from object files targeting AVR processors into assembly code and writes it to the standard output stream.
\flowgraph{\resource{object file} \ar[r] & \toolbox{avrdism} \ar[r] & \resource{disassembly\\listing}}
\seeassembly\seeavr\seeobject
}

\providecommand{\avrttasm}{
\toolsection{avr32asm} is an assembler for the AVR32 hardware architecture.
It translates assembly code into machine code for AVR32 processors and stores it in corresponding object files.
\flowgraph{\resource{AVR32 assembly\\source code} \ar[r] & \toolbox{avr32asm} \ar[r] & \resource{object file}}
\seeassembly\seeavrtt\seeobject
}

\providecommand{\avrttdism}{
\toolsection{avr32dism} is a disassembler for the AVR32 hardware architecture.
It translates machine code from object files targeting AVR32 processors into assembly code and writes it to the standard output stream.
\flowgraph{\resource{object file} \ar[r] & \toolbox{avr32dism} \ar[r] & \resource{disassembly\\listing}}
\seeassembly\seeavrtt\seeobject
}

\providecommand{\mabkasm}{
\toolsection{m68kasm} is an assembler for the M68000 hardware architecture.
It translates assembly code into machine code for M68000 processors and stores it in corresponding object files.
\flowgraph{\resource{68000 assembly\\source code} \ar[r] & \toolbox{m68kasm} \ar[r] & \resource{object file}}
\seeassembly\seemabk\seeobject
}

\providecommand{\mabkdism}{
\toolsection{m68kdism} is a disassembler for the M68000 hardware architecture.
It translates machine code from object files targeting M68000 processors into assembly code and writes it to the standard output stream.
\flowgraph{\resource{object file} \ar[r] & \toolbox{m68kdism} \ar[r] & \resource{disassembly\\listing}}
\seeassembly\seemabk\seeobject
}

\providecommand{\miblasm}{
\toolsection{miblasm} is an assembler for the MicroBlaze hardware architecture.
It translates assembly code into machine code for MicroBlaze processors and stores it in corresponding object files.
\flowgraph{\resource{MicroBlaze assembly\\source code} \ar[r] & \toolbox{miblasm} \ar[r] & \resource{object file}}
\seeassembly\seemibl\seeobject
}

\providecommand{\mibldism}{
\toolsection{mibldism} is a disassembler for the MicroBlaze hardware architecture.
It translates machine code from object files targeting MicroBlaze processors into assembly code and writes it to the standard output stream.
\flowgraph{\resource{object file} \ar[r] & \toolbox{mibldism} \ar[r] & \resource{disassembly\\listing}}
\seeassembly\seemibl\seeobject
}

\providecommand{\mipsaasm}{
\toolsection{mips32asm} is an assembler for the MIPS32 hardware architecture.
It translates assembly code into machine code for MIPS32 processors and stores it in corresponding object files.
\flowgraph{\resource{MIPS32 assembly\\source code} \ar[r] & \toolbox{mips32asm} \ar[r] & \resource{object file}}
\seeassembly\seemips\seeobject
}

\providecommand{\mipsadism}{
\toolsection{mips32dism} is a disassembler for the MIPS32 hardware architecture.
It translates machine code from object files targeting MIPS32 processors into assembly code and writes it to the standard output stream.
\flowgraph{\resource{object file} \ar[r] & \toolbox{mips32dism} \ar[r] & \resource{disassembly\\listing}}
\seeassembly\seemips\seeobject
}

\providecommand{\mipsbasm}{
\toolsection{mips64asm} is an assembler for the MIPS64 hardware architecture.
It translates assembly code into machine code for MIPS64 processors and stores it in corresponding object files.
\flowgraph{\resource{MIPS64 assembly\\source code} \ar[r] & \toolbox{mips64asm} \ar[r] & \resource{object file}}
\seeassembly\seemips\seeobject
}

\providecommand{\mipsbdism}{
\toolsection{mips64dism} is a disassembler for the MIPS64 hardware architecture.
It translates machine code from object files targeting MIPS64 processors into assembly code and writes it to the standard output stream.
\flowgraph{\resource{object file} \ar[r] & \toolbox{mips64dism} \ar[r] & \resource{disassembly\\listing}}
\seeassembly\seemips\seeobject
}

\providecommand{\mmixasm}{
\toolsection{mmixasm} is an assembler for the MMIX hardware architecture.
It translates assembly code into machine code for MMIX processors and stores it in corresponding object files.
The names of all special registers are predefined and evaluate to the corresponding number.
\flowgraph{\resource{MMIX assembly\\source code} \ar[r] & \toolbox{mmixasm} \ar[r] & \resource{object file}}
\seeassembly\seemmix\seeobject
}

\providecommand{\mmixdism}{
\toolsection{mmixdism} is a disassembler for the MMIX hardware architecture.
It translates machine code from object files targeting MMIX processors into assembly code and writes it to the standard output stream.
\flowgraph{\resource{object file} \ar[r] & \toolbox{mmixdism} \ar[r] & \resource{disassembly\\listing}}
\seeassembly\seemmix\seeobject
}

\providecommand{\orokasm}{
\toolsection{or1kasm} is an assembler for the OpenRISC 1000 hardware architecture.
It translates assembly code into machine code for OpenRISC 1000 processors and stores it in corresponding object files.
\flowgraph{\resource{OpenRISC 1000 assembly\\source code} \ar[r] & \toolbox{or1kasm} \ar[r] & \resource{object file}}
\seeassembly\seeorok\seeobject
}

\providecommand{\orokdism}{
\toolsection{or1kdism} is a disassembler for the OpenRISC 1000 hardware architecture.
It translates machine code from object files targeting OpenRISC 1000 processors into assembly code and writes it to the standard output stream.
\flowgraph{\resource{object file} \ar[r] & \toolbox{or1kdism} \ar[r] & \resource{disassembly\\listing}}
\seeassembly\seeorok\seeobject
}

\providecommand{\ppcaasm}{
\toolsection{ppc32asm} is an assembler for the PowerPC hardware architecture.
It translates assembly code into machine code for PowerPC processors and stores it in corresponding object files.
By default, the assembler generates machine code for the 32-bit operating mode defined by the PowerPC architecture.
\flowgraph{\resource{PowerPC assembly\\source code} \ar[r] & \toolbox{ppc32asm} \ar[r] & \resource{object file}}
\seeassembly\seeppc\seeobject
}

\providecommand{\ppcadism}{
\toolsection{ppc32dism} is a disassembler for the PowerPC hardware architecture.
It translates machine code from object files targeting PowerPC processors into assembly code and writes it to the standard output stream.
It assumes that the machine code was generated for the 32-bit operating mode defined by the PowerPC architecture.
\flowgraph{\resource{object file} \ar[r] & \toolbox{ppc32dism} \ar[r] & \resource{disassembly\\listing}}
\seeassembly\seeppc\seeobject
}

\providecommand{\ppcbasm}{
\toolsection{ppc64asm} is an assembler for the PowerPC hardware architecture.
It translates assembly code into machine code for PowerPC processors and stores it in corresponding object files.
By default, the assembler generates machine code for the 64-bit operating mode defined by the PowerPC architecture.
\flowgraph{\resource{PowerPC assembly\\source code} \ar[r] & \toolbox{ppc64asm} \ar[r] & \resource{object file}}
\seeassembly\seeppc\seeobject
}

\providecommand{\ppcbdism}{
\toolsection{ppc64dism} is a disassembler for the PowerPC hardware architecture.
It translates machine code from object files targeting PowerPC processors into assembly code and writes it to the standard output stream.
It assumes that the machine code was generated for the 64-bit operating mode defined by the PowerPC architecture.
\flowgraph{\resource{object file} \ar[r] & \toolbox{ppc64dism} \ar[r] & \resource{disassembly\\listing}}
\seeassembly\seeppc\seeobject
}

\providecommand{\riscasm}{
\toolsection{riscasm} is an assembler for the RISC hardware architecture.
It translates assembly code into machine code for RISC processors and stores it in corresponding object files.
The names of all special registers are predefined and evaluate to the corresponding number.
\flowgraph{\resource{RISC assembly\\source code} \ar[r] & \toolbox{riscasm} \ar[r] & \resource{object file}}
\seeassembly\seerisc\seeobject
}

\providecommand{\riscdism}{
\toolsection{riscdism} is a disassembler for the RISC hardware architecture.
It translates machine code from object files targeting RISC processors into assembly code and writes it to the standard output stream.
\flowgraph{\resource{object file} \ar[r] & \toolbox{riscdism} \ar[r] & \resource{disassembly\\listing}}
\seeassembly\seerisc\seeobject
}

\providecommand{\wasmasm}{
\toolsection{wasmasm} is an assembler for the WebAssembly architecture.
It translates assembly code into machine code for WebAssembly targets and stores it in corresponding object files.
The names of all special registers are predefined and evaluate to the corresponding number.
\flowgraph{\resource{WebAssembly assembly\\source code} \ar[r] & \toolbox{wasmasm} \ar[r] & \resource{object file}}
\seeassembly\seewasm\seeobject
}

\providecommand{\wasmdism}{
\toolsection{wasmdism} is a disassembler for the WebAssembly architecture.
It translates machine code from object files targeting WebAssembly targets into assembly code and writes it to the standard output stream.
\flowgraph{\resource{object file} \ar[r] & \toolbox{wasmdism} \ar[r] & \resource{disassembly\\listing}}
\seeassembly\seewasm\seeobject
}

% linker tools

\providecommand{\linklib}{
\toolsection{linklib} is an object file combiner.
It creates a static library file by combining all object files given to it into a single one.
\flowgraph{\resource{object files} \ar[r] & \toolbox{linklib} \ar[r] & \resource{library file}}
\seeobject
}

\providecommand{\linkbin}{
\toolsection{linkbin} is a linker for plain binary files.
It links all object files given to it into a single image and stores it in a binary file that begins with the first linked section.
It also creates a map file that lists the address, type, name and size of all used sections.
The filename extension of the resulting binary file can be specified by putting it into a constant data section called \texttt{\_extension}.
\flowgraph{\resource{object files} \ar[r] & \toolbox{linkbin} \ar[r] \ar[d] & \resource{binary file} \\ & \resource{map file}}
\seeobject
}

\providecommand{\linkmem}{
\toolsection{linkmem} is a linker for plain binary files partitioned into random-access and read-only memory.
It links all object files given to it into two distinct images, one for data sections and one for code and constant data sections, and stores each image in a binary file that begins with the first linked section of the corresponding type.
It also creates a map file that lists the address, type, name and size of all used sections.
\flowgraph{\resource{object files} \ar[r] & \toolbox{linkmem} \ar[r] \ar[d] & \resource{RAM file/\\ROM file} \\ & \resource{map file}}
\seeobject
}

\providecommand{\linkprg}{
\toolsection{linkprg} is a linker for GEMDOS executable files.
It links all object files given to it into a single image and stores the image in an Atari GEMDOS executable file~\cite{gemdosfile}.
It also creates a map file that lists the address relative to the text segment, type, name and size of all used sections.
The filename extension of the resulting executable file can be specified by putting it into a constant data section called \texttt{\_extension}.
The GEMDOS executable file format requires all patch patterns of absolute link patches to consist of four full bitmasks with descending offsets.
\flowgraph{\resource{object files} \ar[r] & \toolbox{linkprg} \ar[r] \ar[d] & \resource{executable file} \\ & \resource{map file}}
\seeobject
}

\providecommand{\linkhex}{
\toolsection{linkhex} is a linker for Intel HEX files.
It links all code sections of the object files given to it into single image and stores the image in an Intel HEX file~\cite{hexfile} that begins with the first linked section.
It also creates a map file that lists the address, type, name and size of all used sections.
\flowgraph{\resource{object files} \ar[r] & \toolbox{linkhex} \ar[r] \ar[d] & \resource{HEX file} \\ & \resource{map file}}
\seeobject
}

\providecommand{\mapsearch}{
\toolsection{mapsearch} is a debugging tool.
It searches map files generated by linker tools for the name of a binary section that encompasses a memory address read from the standard input stream.
If additionally provided with one or more object files, it also stores an excerpt thereof in a separate object file called map search result which only contains the identified binary section for disassembling purposes.
\flowgraph{& \resource{map files/\\object files} \ar[d] \\ \resource{memory\\address} \ar[r] & \toolbox{mapsearch} \ar[r] \ar[d] & \resource{section name/\\relative offset} \\ & \resource{object file\\excerpt}}
\seeobject
}


\startchapter{Presentation Material}{Presentation Material}{material}
{This \documentation{} is a compilation of presentation material that can be used for presenting the design, functionality, and various implementation details of the \ecs{}.
Each presentation consists of a set of slides and a detailed explanation of their contents.}

\ifbook
\newcommand{\thepresentation}{}\setlength{\marginparwidth}{15em}\setlength{\marginparsep}{-\marginparwidth}
\newcommand{\slide}[1]{\marginpar{\framebox{\pgfimage[width=\dimexpr\marginparwidth-2\fboxrule-2\fboxsep,page=#1]{\thepresentation}}}}
\newenvironment{presentation}[2]{\section{#1}\marginpar{}\renewcommand{\thepresentation}{#2}\begin{adjustwidth}{}{\marginparwidth+1em}}{\end{adjustwidth}}
\else
\newcommand{\thepresentation}{}
\newcommand{\slide}[1]{\begin{center}\framebox{\pgfimage[height=25ex,page=#1]{\thepresentation}}\end{center}}
\newenvironment{presentation}[2]{\bigskip\begin{multicols}{2}[\section{#1}]\renewcommand{\thepresentation}{#2}}{\end{multicols}}
\fi

\begin{presentation}{Overview}{overview}

\emph{\ecs{}} is the name of a free software collection of development tools.
This presentation gives a general overview over the \ecs{} by describing its features, design, functionality and status, and comparing it to related free software.

\slide{2}

\subsection{Features}

The \ecs{} is a toolchain for developing software that contains various tools like compilers, interpreters, assemblers, and linkers targeting a variety of programming languages and hardware architectures.
It implements each of its supported programming languages by providing tools like pretty printers, semantic checkers, interpreters, and front-ends for compilers.
Depending on the programming language, other tools like preprocessors, transpilers, and documentation generators are provided as well.
The \ecs{} additionally defines the generic assembly language that provides a common programming framework for all supported assemblers.
\ifbook\else\nocite{cpp}\nocite{false}\nocite{oberon}\nocite{assembly}\fi

\slide{3}

The \ecs{} supports various hardware architectures by providing tools like assemblers, disassemblers, and machine code generators for compilers.
Some of these architectures like MIPS, AMD64, and PowerPC define different operating modes or bit sizes in which case the \ecs{} provides a set of tools for each variant.
Some of the supported architectures like RISC, MMIX, MicroBlaze, and OpenRISC 1000 may only be available on simulators or FPGA implementations.

\slide{4}

Concerning its supported runtime environments, the \ecs{} provides several different linker tools that generate files like executables, bootloaders, or disk images for simulators.

\slide{5}

\subsection{Design}

The \ecs{} was primarily designed to be a simple but complete and self-contained toolchain.
In particular, all tools featured by the \ecs{} share the same easy-to-use command-line user interface and do not require complex installations or other prerequisites.
Another objective is to produce correct results as well as comprehensive diagnostic messages.
Optimizations and performance are explicitly secondary, as the implementation of the \ecs{} is mainly driven by correctness, reliability, and simplicity.
The \ecs{} is completely written using standard and portable programming language features in order to guarantee the portability of its source code.
Therefore, all compilers featured by the \ecs{} are cross compilers by design.
Furthermore, tools like compilers and assemblers shall enable interoperability in-between all programming languages supported by the \ecs{}.
Since all intermediate data is represented using a human-readable and machine-independent text format, programmers and maintainers are able to view, modify and even manually create all kinds of intermediate data.
Finally, the implementation of the \ecs{} shall provide generic abstractions in order to reuse code wherever possible while implementing these objectives.

\slide{6}

In order to ensure that all programming tools of the \ecs{} are reliable and produce correct results, several test and validation suites are provided which enable automatic regression testing.
Test Suites for programming languages are separated into compilation and execution tests in order to check for code that should or should not compile, as well as complete and valid programs that either generate a runtime error or run successfully to completion.
Regarding the generic assembly language, there exists a test suite for each supported hardware architecture that features at least one compilation test for each element of the corresponding instruction set.

\slide{7}

Each test suite tests one particular tool and is represented using a single text file that contains an arbitrary number of tests.
Each test is identified using a unique description for regression testing and is followed by the corresponding input for the tool under test.
A positive test requires the tool to succeed when given the corresponding input while a negative test expects a failure.
This textual representation is very compact and has proven itself to be very versatile because it allows testing any command-line tool.

\slide{8}

One goal of the \ecs{} is to provide compilers, assemblers, and linkers for all possible combinations of programming languages and target architectures.
For this reason, the \ecs{} uses an intermediate code representation that acts as a generic abstraction in-between \emph{front-ends} implementing programming languages and \emph{back-ends} targeting hardware architectures.
A single data flow of the shown diagram depicts a single compiler for a particular programming language targeting a particular hardware architecture.
\seecode

\slide{9}

Object files on the other hand are the key abstraction that enables interoperability between all compilers and assemblers of the \ecs{}.
All these tools generate the same kind of output called object files which can be processed by all linkers and disassemblers of the \ecs{}.
In addition, the use of object files enables separate compilation which allows compiling only those parts of a program that have actually changed.
\seeobject

\slide{10}

\subsection{Functionality}

Each tool of the \ecs{} provides the same command-line user interface which treats each command-line argument as the name of a textual input file.
Command-line options are not supported in order to make the contents of output files independent from the actual tool invocation.
For compiling programs for example, the compiler as to be invoked using the corresponding name of the source code file.
This results in a new object file for the provided source code which can be used to link the executable program.
Depending on the used compiler, the hardware architecture it targets, as well as the desired runtime environment, the corresponding runtime support files have to be linked as well.
The same result can also be achieved using a utility tool called \tool{ecsd} which automatically infers the required tools from the source file and the specified target environment and conveniently invokes them with the necessary runtime support.

\slide{11}

\subsection{Status}

Except for \cpp{} all programming language implementations provided by the \ecs{} are completed.
The \cpp{} implementation features a preprocessor, parser, and pretty printer but currently lacks proper support for the semantic checker, interpreter, documentation generator, and intermediate code emitter.
Once, \cpp{} is supported sufficiently, its test suite intended to cover all of the ISO \cpp{} Standard ISO/IEC 14882:2023~\cite{iso2023} can be completed as well.
The same holds for the currently still missing software-based floating-point support for all hardware architectures that do not have a native floating-point instruction set.
Some architectures like MIPS or MicroBlaze have not yet been tested on simulators or actual hardware.

\slide{12}

\subsection{Comparison}

In contrast to comparable proprietary compilers, the \ecs{} is free software which allows it to be studied, modified, and used for any purpose.
In comparison to other free software compiler suites like GCC or Clang however, the code base of the \ecs{} is much smaller and allows it to be maintained by a single person.
In addition, the \ecs{} is completely self-contained and requires no other software in order to build programs.
Furthermore, all of its tools are completely portable and therefore executable in any runtime environment.
In order to target different systems, it therefore often suffices to exchange the runtime support for the corresponding runtime environment.

\slide{13}

On the other hand however, the \ecs{} is not highly optimized and does not provide a debugger.
Since it defines its own object file format, the produced object files and binaries are not ABI compatible with existing formats.

\end{presentation}

\begin{presentation}{Implementation Details}{implementation}

The \ecs{} is a completely free and self-contained toolchain written in \cpp{}.
This presentation gives some background information about its implementation and is mainly intended for developers and maintainers of the \ecs{}.

\slide{2}

The \ecs{} provides a makefile which allows building all of its tools in environments like Windows, OS~X, and Linux-based operating systems.
Its source code however is does not depend on any particular operating system, underlying hardware architecture, or any tools other than a conforming \cpp{} compiler.

\subsection{Overall Design}

The tools of the \ecs{} are completely written in standard \cpp{}.
The only issue where its implementation currently depends on the execution environment is the handling of directory path delimiting characters.
Apart from that, the source code of the \ecs{} is completely portable and designed to be maintainable and self-documenting.
By convention, it uniformly uses a consistent naming convention and self-explanatory function predicates which purposely render a lot of comments unnecessary.

\slide{3}

The project itself has a simple structure with meaningful filenames that directly follows its logical layout.
From the beginning, all of the files contained therein have been plain text files and under revision control.
Quality assurance is achieved using an issue tracking system and build automation utilities.

\subsection{Code Structure}

The implementation of each major component of the \ecs{} is physically stored in one source and one header file.
Components of a typical programming language implementation are its syntax representation, the lexer, the parser, the semantic checker, and the intermediate code emitter.
The implementation of a hardware architecture on the other hand is composed of a representation of its instruction set, an assembler, a disassembler, and a machine code generator.
Other examples of major abstractions are the representations of intermediate code, object files, and debugging information.

\slide{4}

Each tool of the \ecs{} combines one or more of these components into an executable file.
For each tool there is a corresponding source file that contains the \texttt{main} function and is called a driver.
Examples of drivers are pretty printers, semantic checkers, interpreters, and compilers for various programming languages.
Furthermore there are one or more assemblers and disassemblers for each hardware architecture, as well as linkers and debugging tools.

\slide{5}

Finally there are common utilities that are used throughout the whole implementation of the \ecs{}.
These utilities are provided by header files and provide frameworks for writing drivers, or generic abstractions like character sets and diagnostic interfaces.

Logically, each programming language, hardware architecture, and major abstraction provides its components as classes in a distinct namespace.
These namespaces and all common utilities reside in a global namespace called \texttt{ECS}.
This allows any component of the \ecs{} to be used as a library in other projects without provoking name clashes.
Drivers are examples of how that library interface is used to combine various components of the \ecs{} into executable files.

\slide{6}

\subsection{Macro Definition Files}

In addition to source and header files, the implementations of almost all major components make use of so-called \emph{macro definition files}.
These are essentially header files that contain only sequential invocations of preprocessor macros typically known as X~macros.
This allows including the same macro definition file in various places using different macro definitions.
The idea is to repeat the sequence of macro invocations in various different contexts while maintaining the actual macro arguments in a single place.

\slide{7}

A typical use case of macro definition files is the definition of enumerations in header files where the macro is defined to provide the name of an enumerator.
The same macro definition file is then typically also included in the corresponding source file where the macro may for example be defined to provide the textual representation of each enumerator in an array of constant strings.
This is useful for maintaining a sorted list of the keywords of a programming language for example.
Since its lexer can compare scanned identifiers with elements of the constant string array, modifying the macro definition file is usually sufficient to add or remove keywords.

\slide{8}

Other components that profit from the same technique are instruction set implementations which can represent complete instruction set tables in a single macro definition file.
Using an appropriate macro definition, the corresponding set of instruction mnemonics can even be included in the documentation of the \ecs{}.
\Documentation{}~\documentationref{assembly}{Generic Assembly Language Specification} for example uses this technique to list all instruction sets supported by the \ecs{}.

\subsection{Context Classes}

Each major component described above is represented using a distinct class that provides an interface for processing or transforming some kind of intermediate representation.
A parser for example is a class that provides a function which takes a character stream as input and transforms it into an abstract syntax tree.
The implementation of that interface however is typically not provided by the component class itself but by a private and nested class.
These classes are conventionally called context classes and have several advantages over more traditional designs which implement the interface directly in the component class.

\slide{9}

Since the implementation of an interface is just forwarded to the context class, the latter can be defined and implemented exclusively in the source file.
All of the processing state that is necessary during an invocation of the interface can therefore be stored by the temporary instance of the context class rather than the component.
As a consequence, the only information that remains necessary to represent directly in the component class is some optional constant configuration.
This renders the actual definition of the component class in the header file light-weight, stable, and by design thread-safe.

\slide{10}

\subsection{Error Handling}

The \ecs{} provides a generic interface for reporting different kinds of diagnostic messages like errors, warnings, and notes.
For drivers, it provides an implementation thereof which prints consistent diagnostic messages to standard output streams.
In contrast to warnings and notes however which have merely informational purposes, errors additionally indicate failures to proceed.
After emitting an arbitrarily detailed diagnostic message, an error is therefore also reported by throwing an exception.
This technique conveniently allows programmers to prematurely abort any processing that results in an error without having to provide any information about the error in the exception itself.
The resulting code can therefore treat error conditions like assertions and does not actually need to handle errors after they have been reported.
As a consequence, most components of the \ecs{} typically abort after the first encounter of an error and do not recover without further precaution.
Usually however, subsequent errors are aftereffects anyway whereas independent exceptions can easily be batched together if necessary.

\slide{11}

\end{presentation}

\concludechapter

% Frequently Asked Questions
% Copyright (C) Florian Negele

% This file is part of the Eigen Compiler Suite.

% Permission is granted to copy, distribute and/or modify this document
% under the terms of the GNU Free Documentation License, Version 1.3
% or any later version published by the Free Software Foundation.

% You should have received a copy of the GNU Free Documentation License
% along with the ECS.  If not, see <https://www.gnu.org/licenses/>.

% Generic documentation utilities
% Copyright (C) Florian Negele

% This file is part of the Eigen Compiler Suite.

% Permission is granted to copy, distribute and/or modify this document
% under the terms of the GNU Free Documentation License, Version 1.3
% or any later version published by the Free Software Foundation.

% You should have received a copy of the GNU Free Documentation License
% along with the ECS.  If not, see <https://www.gnu.org/licenses/>.

\providecommand{\cpp}{C\texttt{++}}
\providecommand{\opt}{_\mathit{opt}}
\providecommand{\tool}[1]{\texttt{#1}}
\providecommand{\version}{Version 0.0.40}
\providecommand{\resource}[1]{*++\txt{#1}}
\providecommand{\ecs}{Eigen Compiler Suite}
\providecommand{\changed}[1]{\underline{#1}}
\providecommand{\toolbox}[1]{\converter{#1}}
\providecommand{\file}{}\renewcommand{\file}[1]{\texttt{#1}}
\providecommand{\alignright}{\hfill\linebreak[0]\hspace*{\fill}}
\providecommand{\converter}[1]{*++[F][F*:white][F,:gray]\txt{#1}}
\providecommand{\documentation}{\ifbook chapter\else document\fi}
\providecommand{\Documentation}{\ifbook Chapter\else Document\fi}
\providecommand{\variable}[1]{\resource{\texttt{\small#1}\\variable}}
\providecommand{\documentationref}[2]{\ifbook\ref{#1}\else``\href{#1}{#2}''~\cite{#1}\fi}
\providecommand{\objfile}[1]{\texttt{#1}\index[runtime]{#1 object file@\texttt{#1} object file}}
\providecommand{\libfile}[1]{\texttt{#1}\index[runtime]{#1 library file@\texttt{#1} library file}}
\providecommand{\epigraph}[2]{\ifbook\begin{quote}\flushright\textit{#1}\par--- #2\end{quote}\fi}
\providecommand{\environmentvariable}[1]{\texttt{#1}\index{Environment variables!#1@\texttt{#1}}}
\providecommand{\environment}[1]{\texttt{#1}\index[environment]{#1 environment@\texttt{#1} environment}}
\providecommand{\toolsection}{}\renewcommand{\toolsection}[1]{\subsection{#1}\label{\prefix:#1}\tool{#1}}
\providecommand{\instruction}{}\renewcommand{\instruction}[2]{\noindent\qquad\pdftooltip{\texttt{#1}}{#2}\refstepcounter{instruction}\par}
\providecommand{\flowgraph}{}\renewcommand{\flowgraph}[1]{\par\sffamily\begin{displaymath}\xymatrix@=4ex{#1}\end{displaymath}\normalfont\par}
\providecommand{\instructionset}{}\renewcommand{\instructionset}[4]{\setcounter{instruction}{0}\begin{multicols}{\ifbook#3\else#4\fi}[{\captionof{table}[#2]{#2 (\ref*{#1:instructions}~instructions)}\label{tab:#1set}\vspace{-2ex}}]\footnotesize\raggedcolumns\input{#1.set}\label{#1:instructions}\end{multicols}}

\providecommand{\gpl}{GNU General Public License}
\providecommand{\rse}{ECS Runtime Support Exception}
\providecommand{\fdl}{\href{https://www.gnu.org/licenses/fdl.html}{GNU Free Documentation License}}

\providecommand{\docbegin}{}
\providecommand{\docend}{}
\providecommand{\doclabel}[1]{\hypertarget{#1}}
\providecommand{\doclink}[2]{\hyperlink{#1}{#2}}
\providecommand{\docsection}[3]{\hypertarget{#1}{\subsection{#2}}\label{sec:#1}\index[library]{#2@#3}}
\providecommand{\docsectionstar}[1]{}
\providecommand{\docsubbegin}{\begin{description}}
\providecommand{\docsubend}{\end{description}}
\providecommand{\docsubsection}[3]{\item[\hypertarget{#1}{#2}]\index[library]{#2@#3}}
\providecommand{\docsubsectionstar}[1]{\smallskip}
\providecommand{\docsubsubsection}[3]{\docsubsection{#1}{#2}{#3}}
\providecommand{\docsubsubsectionstar}[1]{}
\providecommand{\docsubsubsubsection}[3]{}
\providecommand{\docsubsubsubsectionstar}[1]{}
\providecommand{\doctable}{}

\providecommand{\debuggingtool}{}\renewcommand{\debuggingtool}{This tool is provided for debugging purposes.
It allows exposing and modifying an internal data structure that is usually not accessible.
}

\providecommand{\interface}{All tools accept command-line arguments which are taken as names of plain text files containing the source code.
If no arguments are provided, the standard input stream is used instead.
Output files are generated in the current working directory and have the same name as the input file being processed whereas the filename extension gets replaced by an appropriate suffix.
\seeinterface
}

\providecommand{\license}{\noindent Copyright \copyright{} Florian Negele\par\medskip\noindent
Permission is granted to copy, distribute and/or modify this document under the terms of the
\fdl{}, Version 1.3 or any later version published by the \href{https://fsf.org/}{Free Software Foundation}.
}

\providecommand{\ecslogosurface}{
\fill[darkgray] (0,0,0) -- (0,0,3) -- (0,3,3) -- (0,3,1) -- (0,4,1) -- (0,4,3) -- (0,5,3) -- (0,5,0) -- (0,2,0) -- (0,2,2) -- (0,1,2) -- (0,1,0) -- cycle;
\fill[gray] (0,5,0) -- (0,5,3) -- (1,5,3) -- (1,5,1) -- (2,5,1) -- (2,5,3) -- (3,5,3) -- (3,5,0) -- cycle;
\fill[lightgray] (0,0,0) -- (0,1,0) -- (2,1,0) -- (2,4,0) -- (1,4,0) -- (1,3,0) -- (2,3,0) -- (2,2,0) -- (0,2,0) -- (0,5,0) -- (3,5,0) -- (3,0,0) -- cycle;
\begin{scope}[line width=0.5]
\begin{scope}[gray]
\draw (0,0,0) -- (0,1,0);
\draw (2,1,0) -- (2,2,0);
\draw (0,1,2) -- (0,2,2);
\draw (0,2,0) -- (0,5,0);
\draw (2,3,0) -- (2,4,0);
\end{scope}
\begin{scope}[lightgray]
\draw (0,1,0) -- (0,1,2);
\draw (0,3,1) -- (0,3,3);
\draw (0,5,0) -- (0,5,3);
\draw (2,5,1) -- (2,5,3);
\end{scope}
\begin{scope}[white]
\draw (0,1,0) -- (2,1,0);
\draw (1,3,0) -- (2,3,0);
\draw (0,5,0) -- (3,5,0);
\end{scope}
\end{scope}
}

\providecommand{\ecslogo}[1]{
\begin{tikzpicture}[scale={(#1)/((sin(45)+cos(45))*3cm)},x={({-cos(45)*1cm},{sin(45)*sin(30)*1cm})},y={({0cm},{(cos(30)*1cm})},z={({sin(45)*1cm},{cos(45)*sin(30)*1cm})}]
\begin{scope}[darkgray,line width=1]
\draw (0,0,0) -- (0,0,3) -- (0,3,3) -- (2,3,3) -- (2,5,3) -- (3,5,3) -- (3,5,0) -- (3,0,0) -- cycle;
\draw (0,3,1) -- (0,4,1) -- (0,4,3) -- (0,5,3) -- (1,5,3) -- (1,5,1) -- (2,5,1);
\draw (1,3,0) -- (1,4,0) -- (2,4,0);
\end{scope}
\fill[darkgray] (2,0,0) -- (2,0,3) -- (2,5,3) -- (2,5,1) -- (2,4,1) -- (2,4,0) -- cycle;
\fill[lightgray] (2,0,2) -- (0,0,2) -- (0,2,2) -- (2,2,2) -- cycle;
\fill[gray] (0,1,0) -- (2,1,0) -- (2,1,2) -- (0,1,2) -- cycle;
\fill[gray] (0,3,1) -- (0,3,3) -- (2,3,3) -- (2,3,0) -- (1,3,0) -- (1,3,1) -- cycle;
\ecslogosurface
\end{tikzpicture}
}

\providecommand{\shadowedecslogo}[3]{
\begin{tikzpicture}[scale={(#1)/((sin(#2)+cos(#2))*3cm)},x={({-cos(#2)*1cm},{sin(#2)*sin(#3)*1cm})},y={({0cm},{(cos(#3)*1cm})},z={({sin(#2)*1cm},{cos(#2)*sin(#3)*1cm})}]
\shade[top color=lightgray!50!white,bottom color=white,middle color=lightgray!50!white] (0,0,0) -- (3,0,0) -- (3,{-0.5-3*sin(#2)*sin(#3)/cos(#3)},0) -- (0,-0.5,0) -- cycle;
\shade[top color=darkgray!50!gray,bottom color=white,middle color=darkgray!50!white] (0,0,0) -- (0,0,3) -- (0,{-0.5-3*cos(#2)*sin(#3)/cos(#3)},3) -- (0,-0.5,0) -- cycle;
\begin{scope}[y={({(cos(#2)+sin(#2))*0.5cm},{(cos(#2)*sin(#3)-sin(#2)*sin(#3))*0.5cm})}]
\useasboundingbox (3,0,0) -- (0,0,0) -- (0,0,3);
\shade[left color=darkgray!80!black,right color=lightgray,middle color=gray] (0,0,0) -- (0,1,0) -- (0,1,0.5) -- (0,2,0) -- (0,5,0) -- (0,5,3) -- (1,5,3) -- (1,4,3) -- (1,4,2.5) -- (1,3,3) -- (2,5,3) -- (3,5,3) -- (3,0,3) -- cycle;
\clip (0,0,0) -- (0,0,3) -- ({-3*sin(#2)/cos(#2)},0,0) -- cycle;
\shade[left color=darkgray,right color=lightgray!50!gray] (0,0,0) -- (0,1,0) -- (0,1,0.5) -- (0,2,0) -- (0,5,0) -- (0,5,3) -- (1,5,3) -- (1,4,3) -- (1,4,2.5) -- (1,3,3) -- (2,5,3) -- (3,5,3) -- (3,0,3) -- cycle;
\end{scope}
\shade[left color=darkgray,right color=darkgray!80!black] (2,0,0) -- (2,0,3) -- (2,5,3) -- (2,5,1) -- (2,4,1) -- (2,4,0) -- cycle;
\shade[left color=darkgray!90!black,right color=gray!80!darkgray] (2,0,2) -- (0,0,2) -- (0,2,2) -- (2,2,2) -- cycle;
\shade[top color=darkgray!90!black,bottom color=gray!80!darkgray] (0,1,0) -- (2,1,0) -- (2,1,2) -- (0,1,2) -- cycle;
\shade[top color=darkgray!90!black,bottom color=gray!80!darkgray] (0,3,1) -- (0,3,3) -- (2,3,3) -- (2,3,0) -- (1,3,0) -- (1,3,1) -- cycle;
\fill[gray] (2,1,0) -- (1.5,1,0.5) -- (0,1,0.5) -- (0,1,0) -- cycle;
\fill[gray] (1,3,2) -- (0.5,3,2) -- (0.5,3,3) -- (1,3,3) -- cycle;
\fill[gray] (2,3,0) -- (1.5,3,0.5) -- (1,3,0.5) -- (1,3,0) -- cycle;
\ecslogosurface
\end{tikzpicture}
}

\providecommand{\cpplogo}[1]{
\begin{tikzpicture}[scale=(#1)/512em]
\fill[gray] (435.2794,398.7159) -- (247.1911,507.3075) .. controls (236.3563,513.5642) and (218.6240,513.5642) .. (207.7892,507.3075) -- (19.7009,398.7159) .. controls (8.8646,392.4606) and (0.0000,377.1043) .. (0.0000,364.5924) -- (0.0000,147.4076) .. controls (0.8430,132.8363) and (8.2856,120.7683) .. (19.7009,113.2842) -- (207.7892,4.6926) .. controls (218.6240,-1.5642) and (236.3564,-1.5642) .. (247.1911,4.6926) -- (435.2794,113.2842) .. controls (447.5273,121.4304) and (454.4987,133.6918) .. (454.9803,147.4076) -- (454.9803,364.5924) .. controls (454.5404,377.7571) and (446.6566,391.0351) .. (435.2794,398.7159) -- cycle(75.8301,255.9993) .. controls (74.9389,404.0881) and (273.2892,469.4783) .. (358.8263,331.8769) -- (293.1917,293.8965) .. controls (253.5702,359.4301) and (155.1909,335.9977) .. (151.6601,255.9993) .. controls (152.7204,182.2703) and (249.4137,148.0211) .. (293.1961,218.1065) -- (358.8308,180.1276) .. controls (283.4477,49.2645) and (79.6318,96.3470) .. (75.8301,255.9993) -- cycle(379.1503,247.5747) -- (362.2982,247.5747) -- (362.2982,230.7226) -- (345.4490,230.7226) -- (345.4490,247.5747) -- (328.5969,247.5747) -- (328.5969,264.4254) -- (345.4490,264.4254) -- (345.4490,281.2759) -- (362.2982,281.2759) -- (362.2982,264.4254) -- (379.1503,264.4254) -- cycle(442.3420,247.5747) -- (425.4899,247.5747) -- (425.4899,230.7226) -- (408.6408,230.7226) -- (408.6408,247.5747) -- (391.7886,247.5747) -- (391.7886,264.4254) -- (408.6408,264.4254) -- (408.6408,281.2759) -- (425.4899,281.2759) -- (425.4899,264.4254) -- (442.3420,264.4254) -- cycle;
\end{tikzpicture}
}

\providecommand{\fallogo}[1]{
\begin{tikzpicture}[scale=(#1)/512em]
\fill[gray] (185.7774,0.0000) .. controls (200.4486,15.9798) and (226.8966,8.7148) .. (235.0426,31.5836) .. controls (249.5297,58.0598) and (247.9581,97.9161) .. (280.3335,110.9762) .. controls (309.1690,120.3496) and (337.8406,104.2727) .. (366.5753,103.9379) .. controls (373.4449,111.5171) and (379.2885,128.2574) .. (383.9755,108.9744) .. controls (396.6979,102.5615) and (437.2808,107.6681) .. (426.9652,124.3252) .. controls (408.9822,121.0785) and (412.4742,146.0729) .. (426.5192,131.4996) .. controls (433.8413,120.8489) and (465.1541,126.5522) .. (441.9067,135.7950) .. controls (396.1879,157.7478) and (344.1112,161.5079) .. (298.5528,183.5702) .. controls (277.7471,193.5198) and (284.6941,218.7163) .. (285.2127,236.9640) .. controls (292.3599,316.2826) and (307.3929,394.6311) .. (317.1198,473.6154) .. controls (329.0637,505.4736) and (292.1195,528.5004) .. (265.9183,511.2761) .. controls (237.9284,499.2462) and (237.3684,465.2681) .. (230.9102,439.9421) .. controls (218.6692,374.3397) and (215.6307,306.9662) .. (198.1732,242.3977) .. controls (183.1379,232.7444) and (164.4245,256.0298) .. (149.0430,261.4799) .. controls (116.9328,279.2585) and (87.1822,308.5851) .. (48.2293,307.8914) .. controls (21.3220,306.9037) and (-15.9107,281.8761) .. (7.2921,252.7908) .. controls (29.7799,220.6177) and (67.5177,204.3028) .. (100.9287,185.9449) .. controls (130.8217,170.8906) and (161.1548,156.5903) .. (191.0278,141.5847) .. controls (196.1738,120.0520) and (186.6049,95.2409) .. (186.8382,72.4353) .. controls (185.5234,48.4204) and (183.1700,23.9341) .. (185.7774,0.0000) -- cycle;
\end{tikzpicture}
}

\providecommand{\oblogo}[1]{
\begin{tikzpicture}[scale=(#1)/512em]
\fill[gray] (160.3865,208.9117) .. controls (154.0879,214.6478) and (149.0735,221.2409) .. (145.4125,228.5384) .. controls (184.8790,248.4273) and (234.7122,269.8787) .. (297.5493,291.8782) .. controls (300.3943,281.4769) and (300.9552,268.7619) .. (300.4023,255.2389) .. controls (248.9909,244.7891) and (200.0310,225.9279) .. (160.3865,208.9117) -- cycle(225.7398,392.6996) .. controls (308.0209,392.1716) and (359.3326,345.9277) .. (368.7203,285.2098) .. controls (376.6742,197.1784) and (311.7194,141.3342) .. (205.4287,142.1456) .. controls (139.9485,141.4804) and (88.7155,166.1957) .. (73.5775,228.0086) .. controls (52.0297,320.3408) and (123.4078,391.0103) .. (225.7398,392.6996) -- cycle(216.0739,176.4733) .. controls (268.9183,179.2424) and (315.8292,206.5488) .. (312.7454,265.1139) .. controls (313.2769,315.6384) and (286.5993,353.4946) .. (216.6040,355.7934) .. controls (162.4657,355.7934) and (126.0914,317.5023) .. (126.0914,260.5103) .. controls (126.1733,214.2900) and (163.3363,176.2849) .. (216.0739,176.4733) -- cycle(76.4897,189.1754) .. controls (13.1586,147.5631) and (0.0000,119.4207) .. (0.0000,119.4207) -- (90.6499,170.1632) .. controls (85.3004,175.8497) and (80.5994,182.1633) .. (76.4897,189.1754) -- cycle(353.9486,119.3004) -- (402.9482,119.3004) .. controls (427.0025,137.0797) and (450.9893,162.7034) .. (474.9529,191.0213) .. controls (509.3540,228.5339) and (531.3391,294.2091) .. (487.8149,312.1206) .. controls (462.8165,324.7652) and (394.3874,316.8943) .. (373.8912,313.6651) .. controls (379.9291,297.7449) and (383.2899,278.4204) .. (381.4989,257.7214) .. controls (420.3069,248.0321) and (421.9610,218.3461) .. (407.7867,192.6417) .. controls (391.1113,162.4018) and (370.1114,132.9097) .. (353.9486,119.3004) -- cycle;
\end{tikzpicture}
}

\providecommand{\markuptable}{
\begin{table}
\sffamily\centering
\begin{tabular}{@{}lcl@{}}
\toprule
\texttt{//italics//} & $\rightarrow$ & \textit{italics} \\
\midrule
\texttt{**bold**} & $\rightarrow$ & \textbf{bold} \\
\midrule
\texttt{\# ordered list} & & 1 ordered list \\
\texttt{\# second item} & $\rightarrow$ & 2 second item \\
\texttt{\#\# sub item} & & \hspace{1em} 1 sub item \\
\midrule
\texttt{* unordered list} & & $\bullet$ unordered list \\
\texttt{* second item} & $\rightarrow$ & $\bullet$ second item \\
\texttt{** sub item} & & \hspace{1em} $\bullet$ sub item \\
\midrule
\texttt{link to [[label]]} & $\rightarrow$ & link to \underline{label} \\
\midrule
\texttt{<{}<label>{}> definition } & $\rightarrow$ & definition \\
\midrule
\texttt{[[url|link name]]} & $\rightarrow$ & \underline{link name} \\
\midrule\addlinespace
\texttt{= large heading} & & {\Large large heading} \smallskip \\
\texttt{== medium heading} & $\rightarrow$ & {\large medium heading} \\
\texttt{=== small heading} & & small heading \\
\midrule
\texttt{no line break} & & no line break for paragraphs \\
\texttt{for paragraphs} & $\rightarrow$ \\
& & use empty line \\
\texttt{use empty line} \\
\midrule
\texttt{force\textbackslash\textbackslash line break} & $\rightarrow$ & force \\
& & line break \\
\midrule
\texttt{horizontal line} & $\rightarrow$ & horizontal line \\
\texttt{----} & & \hrulefill \\
\midrule
\texttt{|=a|=table|=header} & & \underline{a \enspace table \enspace header} \\
\texttt{|a|table|row} & $\rightarrow$ & a \enspace table \enspace row \\
\texttt{|b|table|row} & & b \enspace table \enspace row \\
\midrule
\texttt{\{\{\{} \\
\texttt{unformatted} & $\rightarrow$ & \texttt{unformatted} \\
\texttt{code} & & \texttt{code} \\
\texttt{\}\}\}} \\
\midrule\addlinespace
\texttt{@ new article} & & {\Large 1.\ new article} \smallskip \\
\texttt{@ second article} & $\rightarrow$ & {\Large 2.\ second article} \smallskip \\
\texttt{@@ sub article} & & {\large 2.1.\ sub article} \\
\bottomrule
\end{tabular}
\normalfont\caption{Elements of the generic documentation markup language}
\label{tab:docmarkup}
\end{table}
}

\providecommand{\startchapter}[4]{
\documentclass[11pt,a4paper]{article}
\usepackage{booktabs}
\usepackage[format=hang,labelfont=bf]{caption}
\usepackage{changepage}
\usepackage[T1]{fontenc}
\usepackage[margin=2cm]{geometry}
\usepackage{hyperref}
\usepackage[american]{isodate}
\usepackage{lmodern}
\usepackage{longtable}
\usepackage{mathptmx}
\usepackage{microtype}
\usepackage[toc]{multitoc}
\usepackage{multirow}
\usepackage[all]{nowidow}
\usepackage{pdfcomment}
\usepackage{syntax}
\usepackage{tikz}
\usepackage[all]{xy}
\hypersetup{pdfborder={0 0 0},bookmarksnumbered=true,pdftitle={\ecs{}: #2},pdfauthor={Florian Negele},pdfsubject={\ecs{}},pdfkeywords={#1}}
\setlength{\grammarindent}{8em}\setlength{\grammarparsep}{0.2ex}
\setlength{\columnsep}{2em}
\newcommand{\prefix}{}
\newcounter{instruction}
\bibliographystyle{unsrt}
\renewcommand{\index}[2][]{}
\renewcommand{\arraystretch}{1.05}
\renewcommand{\floatpagefraction}{0.7}
\renewcommand{\syntleft}{\itshape}\renewcommand{\syntright}{}
\title{\vspace{-5ex}\Huge{\ecs{}}\medskip\hrule}
\author{\huge{#2}}
\date{\medskip\version}
\newif\ifbook\bookfalse
\pagestyle{headings}
\frenchspacing
\begin{document}
\maketitle\thispagestyle{empty}\noindent#4\setlength{\columnseprule}{0.4pt}\tableofcontents\setlength{\columnseprule}{0pt}\vfill\pagebreak[3]\null\vfill\bigskip\noindent
\parbox{\textwidth-4em}{\license The contents of this \documentation{} are part of the \href{manual}{\ecs{} User Manual}~\cite{manual} and correspond to Chapter ``\href{manual\##3}{#1}''.\alignright\mbox{\today}}
\parbox{4em}{\flushright\ecslogo{3em}}
\clearpage
}

\providecommand{\concludechapter}{
\vfill\pagebreak[3]\null\vfill
\thispagestyle{myheadings}\markright{REFERENCES}
\noindent\begin{minipage}{\textwidth}\begin{multicols}{2}[\section*{References}]
\renewcommand{\section}[2]{}\small\bibliography{references}
\end{multicols}\end{minipage}\end{document}
}

\providecommand{\startpresentation}[2]{
\documentclass[14pt,aspectratio=43,usepdftitle=false]{beamer}
\usepackage{booktabs}
\usepackage{etex}
\usepackage{multicol}
\usepackage{tikz}
\usepackage[all]{xy}
\bibliographystyle{unsrt}
\setlength{\columnsep}{1em}
\setlength{\leftmargini}{1em}
\setbeamercolor{title}{fg=black}
\setbeamercolor{structure}{fg=darkgray}
\setbeamercolor{bibliography item}{fg=darkgray}
\setbeamerfont{title}{series=\bfseries}
\setbeamerfont{subtitle}{series=\normalfont}
\setbeamerfont*{frametitle}{parent=title}
\setbeamerfont{block title}{series=\bfseries}
\setbeamerfont*{framesubtitle}{parent=subtitle}
\setbeamersize{text margin left=1em,text margin right=1em}
\setbeamertemplate{navigation symbols}{}
\setbeamertemplate{itemize item}[circle]{}
\setbeamertemplate{bibliography item}[triangle]{}
\setbeamertemplate{bibliography entry author}{\usebeamercolor[fg]{bibliography item}}
\setbeamertemplate{frametitle}{\medskip\usebeamerfont{frametitle}\color{gray}\raisebox{-2.5ex}[0ex][0ex]{\rule{0.1em}{4.5ex}}}
\addtobeamertemplate{frametitle}{}{\hspace{0.4em}\usebeamercolor[fg]{title}\insertframetitle\par\vspace{0.2ex}\hspace{0.5em}\usebeamerfont{framesubtitle}\insertframesubtitle}
\hypersetup{pdfborder={0 0 0},bookmarksnumbered=true,bookmarksopen=true,bookmarksopenlevel=0,pdftitle={\ecs{}: #1},pdfauthor={Florian Negele},pdfsubject={\ecs{}},pdfkeywords={#1}}
\renewcommand{\flowgraph}[1]{\resizebox{\textwidth}{!}{$$\xymatrix{##1}$$}}
\title{\ecs{}\medskip\hrule\medskip}
\institute{\shadowedecslogo{5em}{30}{15}}
\date{\version}
\subtitle{#1}
\begin{document}
\begin{frame}[plain]\titlepage\nocite{manual}\end{frame}
\begin{frame}{Contents}{#1}\begin{center}\tableofcontents\end{center}\end{frame}
}

\providecommand{\concludepresentation}{
\begin{frame}{References}\begin{footnotesize}\setlength{\columnseprule}{0.4pt}\begin{multicols}{2}\bibliography{references}\end{multicols}\end{footnotesize}\end{frame}
\end{document}
}

\providecommand{\startbook}[1]{
\documentclass[10pt,paper=17cm:24cm,DIV=13,twoside=semi,headings=normal,numbers=noendperiod,cleardoublepage=plain]{scrbook}
\usepackage{atveryend}
\usepackage{booktabs}
\usepackage{caption}
\usepackage{changepage}
\usepackage[T1]{fontenc}
\usepackage{imakeidx}
\usepackage{hyperref}
\usepackage[american]{isodate}
\usepackage{lmodern}
\usepackage{longtable}
\usepackage{mathptmx}
\usepackage[final]{microtype}
\usepackage{multicol}
\usepackage{multirow}
\usepackage[all]{nowidow}
\usepackage{pdfcomment}
\usepackage{scrlayer-scrpage}
\usepackage{setspace}
\usepackage{syntax}
\usepackage[eventxtindent=4pt,oddtxtexdent=4pt]{thumbs}
\usepackage{tikz}
\usepackage[all]{xy}
\hyphenation{Micro-Blaze Open-Cores Open-RISC Power-PC}
\hypersetup{pdfborder={0 0 0},bookmarksnumbered=true,bookmarksopen=true,bookmarksopenlevel=0,pdftitle={\ecs{}: #1},pdfauthor={Florian Negele},pdfsubject={\ecs{}},pdfkeywords={#1}}
\setlength{\grammarindent}{8em}\setlength{\grammarparsep}{0.7ex}
\setkomafont{captionlabel}{\usekomafont{descriptionlabel}}
\renewcommand{\arraystretch}{1.05}\setstretch{1.1}
\renewcommand{\chapterformat}{\thechapter\autodot\enskip\raisebox{-1ex}[0ex][0ex]{\color{gray}\rule{0.1em}{3.5ex}}\enskip}
\renewcommand{\startchapter}[4]{\hypertarget{##3}{\chapter{##1}}\label{##3}##4\addthumb{##1}{\LARGE\sffamily\bfseries\thechapter}{white}{gray}\renewcommand{\prefix}{##3}}
\renewcommand{\concludechapter}{\clearpage{\stopthumb\cleardoublepage}}
\renewcommand{\syntleft}{\itshape}\renewcommand{\syntright}{}
\renewcommand{\floatpagefraction}{0.7}
\renewcommand{\partheademptypage}{}
\DeclareMicrotypeAlias{lmss}{cmr}
\newcommand{\prefix}{}
\newcounter{instruction}
\bibliographystyle{unsrt}
\newif\ifbook\booktrue
\makeindex[intoc,title=Index]
\makeindex[intoc,name=tools,title=Index of Tools,columns=3]
\makeindex[intoc,name=library,title=Index of Library Names]
\makeindex[intoc,name=runtime,title=Index of Runtime Support]
\makeindex[intoc,name=environment,title=Index of Target Environments]
\indexsetup{toclevel=chapter,headers={\indexname}{\indexname}}
\frenchspacing
\begin{document}
\pagenumbering{alph}
\begin{titlepage}\centering
\huge\sffamily\null\vfill\textbf{\ecs{}}\bigskip\hrule\bigskip#1
\normalsize\normalfont\vfill\vfill\shadowedecslogo{10em}{30}{15}
\large\vfill\vfill\version
\end{titlepage}
\null\vfill
\thispagestyle{empty}
\noindent\today\par\medskip
\license A copy of this license is included in Appendix~\ref{fdl} on page~\pageref{fdl}.
All product names used herein are for identification purposes only and may be trademarks of their respective companies.
\concludechapter
\frontmatter
\setcounter{tocdepth}{1}
\tableofcontents
\setcounter{tocdepth}{2}
\concludechapter
\listoffigures
\concludechapter
\listoftables
\concludechapter
}

\providecommand{\concludebook}{
\backmatter
\addtocontents{toc}{\protect\setcounter{tocdepth}{-1}}
\phantomsection\addcontentsline{toc}{part}{Bibliography}
\bibliography{references}
\concludechapter
\phantomsection\addcontentsline{toc}{part}{Indexes}
\printindex
\concludechapter
\indexprologue{\label{idx:tools}}
\printindex[tools]
\concludechapter
\printindex[library]
\concludechapter
\indexprologue{\label{idx:runtime}}
\printindex[runtime]
\concludechapter
\indexprologue{\label{idx:environment}}
\printindex[environment]
\concludechapter
\pagestyle{empty}\pagenumbering{Alph}\null\clearpage
\null\vfill\centering\ecslogo{4em}\par\medskip\license
\end{document}
}

% chapter references

\providecommand{\seedocumentationref}{}\renewcommand{\seedocumentationref}[3]{#1, see \Documentation{}~\documentationref{#2}{#3}. }
\providecommand{\seeinterface}{}\renewcommand{\seeinterface}{\ifbook See \Documentation{}~\documentationref{interface}{User Interface} for more information about the common user interface of all of these tools. \fi}
\providecommand{\seeguide}{}\renewcommand{\seeguide}{\seedocumentationref{For basic examples of using some of these tools in practice}{guide}{User Guide}}
\providecommand{\seecpp}{}\renewcommand{\seecpp}{\seedocumentationref{For more information about the \cpp{} programming language and its implementation by the \ecs{}}{cpp}{User Manual for \cpp{}}}
\providecommand{\seefalse}{}\renewcommand{\seefalse}{\seedocumentationref{For more information about the FALSE programming language and its implementation by the \ecs{}}{false}{User Manual for FALSE}}
\providecommand{\seeoberon}{}\renewcommand{\seeoberon}{\seedocumentationref{For more information about the Oberon programming language and its implementation by the \ecs{}}{oberon}{User Manual for Oberon}}
\providecommand{\seeassembly}{}\renewcommand{\seeassembly}{\seedocumentationref{For more information about the generic assembly language and how to use it}{assembly}{Generic Assembly Language Specification}}
\providecommand{\seeamd}{}\renewcommand{\seeamd}{\seedocumentationref{For more information about how the \ecs{} supports the AMD64 hardware architecture}{amd64}{AMD64 Hardware Architecture Support}}
\providecommand{\seearm}{}\renewcommand{\seearm}{\seedocumentationref{For more information about how the \ecs{} supports the ARM hardware architecture}{arm}{ARM Hardware Architecture Support}}
\providecommand{\seeavr}{}\renewcommand{\seeavr}{\seedocumentationref{For more information about how the \ecs{} supports the AVR hardware architecture}{avr}{AVR Hardware Architecture Support}}
\providecommand{\seeavrtt}{}\renewcommand{\seeavrtt}{\seedocumentationref{For more information about how the \ecs{} supports the AVR32 hardware architecture}{avr32}{AVR32 Hardware Architecture Support}}
\providecommand{\seemabk}{}\renewcommand{\seemabk}{\seedocumentationref{For more information about how the \ecs{} supports the M68000 hardware architecture}{m68k}{M68000 Hardware Architecture Support}}
\providecommand{\seemibl}{}\renewcommand{\seemibl}{\seedocumentationref{For more information about how the \ecs{} supports the MicroBlaze hardware architecture}{mibl}{MicroBlaze Hardware Architecture Support}}
\providecommand{\seemips}{}\renewcommand{\seemips}{\seedocumentationref{For more information about how the \ecs{} supports the MIPS32 and MIPS64 hardware architectures}{mips}{MIPS Hardware Architecture Support}}
\providecommand{\seemmix}{}\renewcommand{\seemmix}{\seedocumentationref{For more information about how the \ecs{} supports the MMIX hardware architecture}{mmix}{MMIX Hardware Architecture Support}}
\providecommand{\seeorok}{}\renewcommand{\seeorok}{\seedocumentationref{For more information about how the \ecs{} supports the OpenRISC 1000 hardware architecture}{or1k}{OpenRISC 1000 Hardware Architecture Support}}
\providecommand{\seeppc}{}\renewcommand{\seeppc}{\seedocumentationref{For more information about how the \ecs{} supports the PowerPC hardware architecture}{ppc}{PowerPC Hardware Architecture Support}}
\providecommand{\seerisc}{}\renewcommand{\seerisc}{\seedocumentationref{For more information about how the \ecs{} supports the RISC hardware architecture}{risc}{RISC Hardware Architecture Support}}
\providecommand{\seewasm}{}\renewcommand{\seewasm}{\seedocumentationref{For more information about how the \ecs{} supports the WebAssembly architecture}{wasm}{WebAssembly Architecture Support}}
\providecommand{\seedocumentation}{}\renewcommand{\seedocumentation}{\seedocumentationref{For more information about generic documentations and their generation by the \ecs{}}{documentation}{Generic Documentation Generation}}
\providecommand{\seedebugging}{}\renewcommand{\seedebugging}{\seedocumentationref{For more information about debugging information and its representation}{debugging}{Debugging Information Representation}}
\providecommand{\seecode}{}\renewcommand{\seecode}{\seedocumentationref{For more information about intermediate code and its purpose}{code}{Intermediate Code Representation}}
\providecommand{\seeobject}{}\renewcommand{\seeobject}{\seedocumentationref{For more information about object files and their purpose}{object}{Object File Representation}}

% generic documentation tools

\providecommand{\docprint}{
\toolsection{docprint} is a pretty printer for generic documentations.
It reformats generic documentations and writes it to the standard output stream.
\debuggingtool
\flowgraph{\resource{generic\\documentation} \ar[r] & \toolbox{docprint} \ar[r] & \resource{generic\\documentation}}
\seedocumentation
}

\providecommand{\doccheck}{
\toolsection{doccheck} is a syntactic and semantic checker for generic documentations.
It just performs syntactic and semantic checks on generic documentations and writes its diagnostic messages to the standard error stream.
\debuggingtool
\flowgraph{\resource{generic\\documentation} \ar[r] & \toolbox{doccheck} \ar[r] & \resource{diagnostic\\messages}}
\seedocumentation
}

\providecommand{\dochtml}{
\toolsection{dochtml} is an HTML documentation generator for generic documentations.
It processes several generic documentations and assembles all information therein into an HTML document.
\debuggingtool
\flowgraph{\resource{generic\\documentation} \ar[r] & \toolbox{dochtml} \ar[r] & \resource{HTML\\document}}
\seedocumentation
}

\providecommand{\doclatex}{
\toolsection{doclatex} is a Latex documentation generator for generic documentations.
It processes several generic documentations and assembles all information therein into a Latex document.
\debuggingtool
\flowgraph{\resource{generic\\documentation} \ar[r] & \toolbox{doclatex} \ar[r] & \resource{Latex\\document}}
\seedocumentation
}

% intermediate code tools

\providecommand{\cdcheck}{
\toolsection{cdcheck} is a syntactic and semantic checker for intermediate code.
It just performs syntactic and semantic checks on programs written in intermediate code and writes its diagnostic messages to the standard error stream.
\debuggingtool
\flowgraph{\resource{intermediate\\code} \ar[r] & \toolbox{cdcheck} \ar[r] & \resource{diagnostic\\messages}}
\seeassembly\seecode
}

\providecommand{\cdopt}{
\toolsection{cdopt} is an optimizer for intermediate code.
It performs various optimizations on programs written in intermediate code and writes the result to the standard output stream.
\debuggingtool
\flowgraph{\resource{intermediate\\code} \ar[r] & \toolbox{cdopt} \ar[r] & \resource{optimized\\code}}
\seeassembly\seecode
}

\providecommand{\cdrun}{
\toolsection{cdrun} is an interpreter for intermediate code.
It processes and executes programs written in intermediate code.
The following code sections are predefined and have the usual semantics:
\texttt{abort}, \texttt{\_Exit}, \texttt{fflush}, \texttt{floor}, \texttt{fputc}, \texttt{free}, \texttt{getchar}, \texttt{malloc}, and \texttt{putchar}.
Diagnostic messages about invalid operations include the name of the executed code section and the index of the erroneous instruction.
\debuggingtool
\flowgraph{\resource{intermediate\\code} \ar[r] & \toolbox{cdrun} \ar@/u/[r] & \resource{input/\\output} \ar@/d/[l]}
\seeassembly\seecode
}

\providecommand{\cdamda}{
\toolsection{cdamd16} is a compiler for intermediate code targeting the AMD64 hardware architecture.
It generates machine code for AMD64 processors from programs written in intermediate code and stores it in corresponding object files.
The compiler generates machine code for the 16-bit operating mode defined by the AMD64 architecture.
It also creates a debugging information file as well as an assembly file containing a listing of the generated machine code.
\debuggingtool
\flowgraph{\resource{intermediate\\code} \ar[r] & \toolbox{cdamd16} \ar[r] \ar[d] \ar[rd] & \resource{object file} \\ & \resource{assembly\\listing} & \resource{debugging\\information}}
\seeassembly\seeamd\seeobject\seecode\seedebugging
}

\providecommand{\cdamdb}{
\toolsection{cdamd32} is a compiler for intermediate code targeting the AMD64 hardware architecture.
It generates machine code for AMD64 processors from programs written in intermediate code and stores it in corresponding object files.
The compiler generates machine code for the 32-bit operating mode defined by the AMD64 architecture.
It also creates a debugging information file as well as an assembly file containing a listing of the generated machine code.
\debuggingtool
\flowgraph{\resource{intermediate\\code} \ar[r] & \toolbox{cdamd32} \ar[r] \ar[d] \ar[rd] & \resource{object file} \\ & \resource{assembly\\listing} & \resource{debugging\\information}}
\seeassembly\seeamd\seeobject\seecode\seedebugging
}

\providecommand{\cdamdc}{
\toolsection{cdamd64} is a compiler for intermediate code targeting the AMD64 hardware architecture.
It generates machine code for AMD64 processors from programs written in intermediate code and stores it in corresponding object files.
The compiler generates machine code for the 64-bit operating mode defined by the AMD64 architecture.
It also creates a debugging information file as well as an assembly file containing a listing of the generated machine code.
\debuggingtool
\flowgraph{\resource{intermediate\\code} \ar[r] & \toolbox{cdamd64} \ar[r] \ar[d] \ar[rd] & \resource{object file} \\ & \resource{assembly\\listing} & \resource{debugging\\information}}
\seeassembly\seeamd\seeobject\seecode\seedebugging
}

\providecommand{\cdarma}{
\toolsection{cdarma32} is a compiler for intermediate code targeting the ARM hardware architecture.
It generates machine code for ARM processors executing A32 instructions from programs written in intermediate code and stores it in corresponding object files.
It also creates a debugging information file as well as an assembly file containing a listing of the generated machine code.
\debuggingtool
\flowgraph{\resource{intermediate\\code} \ar[r] & \toolbox{cdarma32} \ar[r] \ar[d] \ar[rd] & \resource{object file} \\ & \resource{assembly\\listing} & \resource{debugging\\information}}
\seeassembly\seearm\seeobject\seecode\seedebugging
}

\providecommand{\cdarmb}{
\toolsection{cdarma64} is a compiler for intermediate code targeting the ARM hardware architecture.
It generates machine code for ARM processors executing A64 instructions from programs written in intermediate code and stores it in corresponding object files.
It also creates a debugging information file as well as an assembly file containing a listing of the generated machine code.
\debuggingtool
\flowgraph{\resource{intermediate\\code} \ar[r] & \toolbox{cdarma64} \ar[r] \ar[d] \ar[rd] & \resource{object file} \\ & \resource{assembly\\listing} & \resource{debugging\\information}}
\seeassembly\seearm\seeobject\seecode\seedebugging
}

\providecommand{\cdarmc}{
\toolsection{cdarmt32} is a compiler for intermediate code targeting the ARM hardware architecture.
It generates machine code for ARM processors without floating-point extension executing T32 instructions from programs written in intermediate code and stores it in corresponding object files.
It also creates a debugging information file as well as an assembly file containing a listing of the generated machine code.
\debuggingtool
\flowgraph{\resource{intermediate\\code} \ar[r] & \toolbox{cdarmt32} \ar[r] \ar[d] \ar[rd] & \resource{object file} \\ & \resource{assembly\\listing} & \resource{debugging\\information}}
\seeassembly\seearm\seeobject\seecode\seedebugging
}

\providecommand{\cdarmcfpe}{
\toolsection{cdarmt32fpe} is a compiler for intermediate code targeting the ARM hardware architecture.
It generates machine code for ARM processors with floating-point extension executing T32 instructions from programs written in intermediate code and stores it in corresponding object files.
It also creates a debugging information file as well as an assembly file containing a listing of the generated machine code.
\debuggingtool
\flowgraph{\resource{intermediate\\code} \ar[r] & \toolbox{cdarmt32fpe} \ar[r] \ar[d] \ar[rd] & \resource{object file} \\ & \resource{assembly\\listing} & \resource{debugging\\information}}
\seeassembly\seearm\seeobject\seecode\seedebugging
}

\providecommand{\cdavr}{
\toolsection{cdavr} is a compiler for intermediate code targeting the AVR hardware architecture.
It generates machine code for AVR processors from programs written in intermediate code and stores it in corresponding object files.
It also creates a debugging information file as well as an assembly file containing a listing of the generated machine code.
\debuggingtool
\flowgraph{\resource{intermediate\\code} \ar[r] & \toolbox{cdavr} \ar[r] \ar[d] \ar[rd] & \resource{object file} \\ & \resource{assembly\\listing} & \resource{debugging\\information}}
\seeassembly\seeavr\seeobject\seecode\seedebugging
}

\providecommand{\cdavrtt}{
\toolsection{cdavr32} is a compiler for intermediate code targeting the AVR32 hardware architecture.
It generates machine code for AVR32 processors from programs written in intermediate code and stores it in corresponding object files.
It also creates a debugging information file as well as an assembly file containing a listing of the generated machine code.
\debuggingtool
\flowgraph{\resource{intermediate\\code} \ar[r] & \toolbox{cdavr32} \ar[r] \ar[d] \ar[rd] & \resource{object file} \\ & \resource{assembly\\listing} & \resource{debugging\\information}}
\seeassembly\seeavrtt\seeobject\seecode\seedebugging
}

\providecommand{\cdmabk}{
\toolsection{cdm68k} is a compiler for intermediate code targeting the M68000 hardware architecture.
It generates machine code for M68000 processors from programs written in intermediate code and stores it in corresponding object files.
It also creates a debugging information file as well as an assembly file containing a listing of the generated machine code.
\debuggingtool
\flowgraph{\resource{intermediate\\code} \ar[r] & \toolbox{cdm68k} \ar[r] \ar[d] \ar[rd] & \resource{object file} \\ & \resource{assembly\\listing} & \resource{debugging\\information}}
\seeassembly\seemabk\seeobject\seecode\seedebugging
}

\providecommand{\cdmibl}{
\toolsection{cdmibl} is a compiler for intermediate code targeting the MicroBlaze hardware architecture.
It generates machine code for MicroBlaze processors from programs written in intermediate code and stores it in corresponding object files.
It also creates a debugging information file as well as an assembly file containing a listing of the generated machine code.
\debuggingtool
\flowgraph{\resource{intermediate\\code} \ar[r] & \toolbox{cdmibl} \ar[r] \ar[d] \ar[rd] & \resource{object file} \\ & \resource{assembly\\listing} & \resource{debugging\\information}}
\seeassembly\seemibl\seeobject\seecode\seedebugging
}

\providecommand{\cdmipsa}{
\toolsection{cdmips32} is a compiler for intermediate code targeting the MIPS32 hardware architecture.
It generates machine code for MIPS32 processors from programs written in intermediate code and stores it in corresponding object files.
It also creates a debugging information file as well as an assembly file containing a listing of the generated machine code.
\debuggingtool
\flowgraph{\resource{intermediate\\code} \ar[r] & \toolbox{cdmips32} \ar[r] \ar[d] \ar[rd] & \resource{object file} \\ & \resource{assembly\\listing} & \resource{debugging\\information}}
\seeassembly\seemips\seeobject\seecode\seedebugging
}

\providecommand{\cdmipsb}{
\toolsection{cdmips64} is a compiler for intermediate code targeting the MIPS64 hardware architecture.
It generates machine code for MIPS64 processors from programs written in intermediate code and stores it in corresponding object files.
It also creates a debugging information file as well as an assembly file containing a listing of the generated machine code.
\debuggingtool
\flowgraph{\resource{intermediate\\code} \ar[r] & \toolbox{cdmips64} \ar[r] \ar[d] \ar[rd] & \resource{object file} \\ & \resource{assembly\\listing} & \resource{debugging\\information}}
\seeassembly\seemips\seeobject\seecode\seedebugging
}

\providecommand{\cdmmix}{
\toolsection{cdmmix} is a compiler for intermediate code targeting the MMIX hardware architecture.
It generates machine code for MMIX processors from programs written in intermediate code and stores it in corresponding object files.
It also creates a debugging information file as well as an assembly file containing a listing of the generated machine code.
\debuggingtool
\flowgraph{\resource{intermediate\\code} \ar[r] & \toolbox{cdmmix} \ar[r] \ar[d] \ar[rd] & \resource{object file} \\ & \resource{assembly\\listing} & \resource{debugging\\information}}
\seeassembly\seemmix\seeobject\seecode\seedebugging
}

\providecommand{\cdorok}{
\toolsection{cdor1k} is a compiler for intermediate code targeting the OpenRISC 1000 hardware architecture.
It generates machine code for OpenRISC 1000 processors from programs written in intermediate code and stores it in corresponding object files.
It also creates a debugging information file as well as an assembly file containing a listing of the generated machine code.
\debuggingtool
\flowgraph{\resource{intermediate\\code} \ar[r] & \toolbox{cdor1k} \ar[r] \ar[d] \ar[rd] & \resource{object file} \\ & \resource{assembly\\listing} & \resource{debugging\\information}}
\seeassembly\seeorok\seeobject\seecode\seedebugging
}

\providecommand{\cdppca}{
\toolsection{cdppc32} is a compiler for intermediate code targeting the PowerPC hardware architecture.
It generates machine code for PowerPC processors from programs written in intermediate code and stores it in corresponding object files.
The compiler generates machine code for the 32-bit operating mode defined by the PowerPC architecture.
It also creates a debugging information file as well as an assembly file containing a listing of the generated machine code.
\debuggingtool
\flowgraph{\resource{intermediate\\code} \ar[r] & \toolbox{cdppc32} \ar[r] \ar[d] \ar[rd] & \resource{object file} \\ & \resource{assembly\\listing} & \resource{debugging\\information}}
\seeassembly\seeppc\seeobject\seecode\seedebugging
}

\providecommand{\cdppcb}{
\toolsection{cdppc64} is a compiler for intermediate code targeting the PowerPC hardware architecture.
It generates machine code for PowerPC processors from programs written in intermediate code and stores it in corresponding object files.
The compiler generates machine code for the 64-bit operating mode defined by the PowerPC architecture.
It also creates a debugging information file as well as an assembly file containing a listing of the generated machine code.
\debuggingtool
\flowgraph{\resource{intermediate\\code} \ar[r] & \toolbox{cdppc64} \ar[r] \ar[d] \ar[rd] & \resource{object file} \\ & \resource{assembly\\listing} & \resource{debugging\\information}}
\seeassembly\seeppc\seeobject\seecode\seedebugging
}

\providecommand{\cdrisc}{
\toolsection{cdrisc} is a compiler for intermediate code targeting the RISC hardware architecture.
It generates machine code for RISC processors from programs written in intermediate code and stores it in corresponding object files.
It also creates a debugging information file as well as an assembly file containing a listing of the generated machine code.
\debuggingtool
\flowgraph{\resource{intermediate\\code} \ar[r] & \toolbox{cdrisc} \ar[r] \ar[d] \ar[rd] & \resource{object file} \\ & \resource{assembly\\listing} & \resource{debugging\\information}}
\seeassembly\seerisc\seeobject\seecode\seedebugging
}

\providecommand{\cdwasm}{
\toolsection{cdwasm} is a compiler for intermediate code targeting the WebAssembly architecture.
It generates machine code for WebAssembly targets from programs written in intermediate code and stores it in corresponding object files.
It also creates a debugging information file as well as an assembly file containing a listing of the generated machine code.
\debuggingtool
\flowgraph{\resource{intermediate\\code} \ar[r] & \toolbox{cdwasm} \ar[r] \ar[d] \ar[rd] & \resource{object file} \\ & \resource{assembly\\listing} & \resource{debugging\\information}}
\seeassembly\seewasm\seeobject\seecode\seedebugging
}

% C++ tools

\providecommand{\cppprep}{
\toolsection{cppprep} is a preprocessor for the \cpp{} programming language.
It preprocesses source code according to the rules of \cpp{} and writes it to the standard output stream.
Only the macro names \texttt{\_\_DATE\_\_}, \texttt{\_\_FILE\_\_}, \texttt{\_\_LINE\_\_}, and \texttt{\_\_TIME\_\_} are predefined.
\flowgraph{\resource{\cpp{} or other\\source code} \ar[r] & \toolbox{cppprep} \ar[r] & \resource{preprocessed\\source code} \\ & \variable{ECSINCLUDE} \ar[u]}
\seecpp
}

\providecommand{\cppprint}{
\toolsection{cppprint} is a pretty printer for the \cpp{} programming language.
It reformats the source code of \cpp{} programs and writes it to the standard output stream.
\flowgraph{\resource{\cpp{}\\source code} \ar[r] & \toolbox{cppprint} \ar[r] & \resource{reformatted\\source code} \\ & \variable{ECSINCLUDE} \ar[u]}
\seecpp
}

\providecommand{\cppcheck}{
\toolsection{cppcheck} is a syntactic and semantic checker for the \cpp{} programming language.
It just performs syntactic and semantic checks on \cpp{} programs and writes its diagnostic messages to the standard error stream.
\flowgraph{\resource{\cpp{}\\source code} \ar[r] & \toolbox{cppcheck} \ar[r] & \resource{diagnostic\\messages} \\ & \variable{ECSINCLUDE} \ar[u]}
\seecpp
}

\providecommand{\cppdump}{
\toolsection{cppdump} is a serializer for the \cpp{} programming language.
It dumps the complete internal representation of programs written in \cpp{} into an XML document.
\debuggingtool
\flowgraph{\resource{\cpp{}\\source code} \ar[r] & \toolbox{cppdump} \ar[r] & \resource{internal\\representation} \\ & \variable{ECSINCLUDE} \ar[u]}
\seecpp
}

\providecommand{\cpprun}{
\toolsection{cpprun} is an interpreter for the \cpp{} programming language.
It processes and executes programs written in \cpp{}.
The macro \texttt{\_\_run\_\_} is predefined in order to enable programmers to identify this tool while interpreting.
\flowgraph{\resource{\cpp{}\\source code} \ar[r] & \toolbox{cpprun} \ar@/u/[r] & \resource{input/\\output} \ar@/d/[l] \\ & \variable{ECSINCLUDE} \ar[u]}
\seecpp
}

\providecommand{\cppdoc}{
\toolsection{cppdoc} is a generic documentation generator for the \cpp{} programming language.
It processes several \cpp{} source files and assembles all information therein into a generic documentation.
\debuggingtool
\flowgraph{\resource{\cpp{}\\source code} \ar[r] & \toolbox{cppdoc} \ar[r] & \resource{generic\\documentation} \\ & \variable{ECSINCLUDE} \ar[u]}
\seecpp\seedocumentation
}

\providecommand{\cpphtml}{
\toolsection{cpphtml} is an HTML documentation generator for the \cpp{} programming language.
It processes several \cpp{} source files and assembles all information therein into an HTML document.
\flowgraph{\resource{\cpp{}\\source code} \ar[r] & \toolbox{cpphtml} \ar[r] & \resource{HTML\\document} \\ & \variable{ECSINCLUDE} \ar[u]}
\seecpp\seedocumentation
}

\providecommand{\cpplatex}{
\toolsection{cpplatex} is a Latex documentation generator for the \cpp{} programming language.
It processes several \cpp{} source files and assembles all information therein into a Latex document.
\flowgraph{\resource{\cpp{}\\source code} \ar[r] & \toolbox{cpplatex} \ar[r] & \resource{Latex\\document} \\ & \variable{ECSINCLUDE} \ar[u]}
\seecpp\seedocumentation
}

\providecommand{\cppcode}{
\toolsection{cppcode} is an intermediate code generator for the \cpp{} programming language.
It generates intermediate code from programs written in \cpp{} and stores it in corresponding assembly files.
The macro \texttt{\_\_code\_\_} is predefined in order to enable programmers to identify this tool while generating intermediate code.
Programs generated with this tool require additional runtime support that is stored in the \file{cpp\-code\-run} library file.
\debuggingtool
\flowgraph{\resource{\cpp{}\\source code} \ar[r] & \toolbox{cppcode} \ar[r] & \resource{intermediate\\code} \\ & \variable{ECSINCLUDE} \ar[u]}
\seecpp\seeassembly\seecode
}

\providecommand{\cppamda}{
\toolsection{cppamd16} is a compiler for the \cpp{} programming language targeting the AMD64 hardware architecture.
It generates machine code for AMD64 processors from programs written in \cpp{} and stores it in corresponding object files.
The compiler generates machine code for the 16-bit operating mode defined by the AMD64 architecture.
For debugging purposes, it also creates a debugging information file as well as an assembly file containing a listing of the generated machine code.
The macro \texttt{\_\_amd16\_\_} is predefined in order to enable programmers to identify this tool and its target architecture while compiling.
Programs generated with this compiler require additional runtime support that is stored in the \file{cpp\-amd16\-run} library file.
\flowgraph{\resource{\cpp{}\\source code} \ar[r] & \toolbox{cppamd16} \ar[r] \ar[d] \ar[rd] & \resource{object file} \\ \variable{ECSINCLUDE} \ar[ru] & \resource{debugging\\information} & \resource{assembly\\listing}}
\seecpp\seeassembly\seeamd\seeobject\seedebugging
}

\providecommand{\cppamdb}{
\toolsection{cppamd32} is a compiler for the \cpp{} programming language targeting the AMD64 hardware architecture.
It generates machine code for AMD64 processors from programs written in \cpp{} and stores it in corresponding object files.
The compiler generates machine code for the 32-bit operating mode defined by the AMD64 architecture.
For debugging purposes, it also creates a debugging information file as well as an assembly file containing a listing of the generated machine code.
The macro \texttt{\_\_amd32\_\_} is predefined in order to enable programmers to identify this tool and its target architecture while compiling.
Programs generated with this compiler require additional runtime support that is stored in the \file{cpp\-amd32\-run} library file.
\flowgraph{\resource{\cpp{}\\source code} \ar[r] & \toolbox{cppamd32} \ar[r] \ar[d] \ar[rd] & \resource{object file} \\ \variable{ECSINCLUDE} \ar[ru] & \resource{debugging\\information} & \resource{assembly\\listing}}
\seecpp\seeassembly\seeamd\seeobject\seedebugging
}

\providecommand{\cppamdc}{
\toolsection{cppamd64} is a compiler for the \cpp{} programming language targeting the AMD64 hardware architecture.
It generates machine code for AMD64 processors from programs written in \cpp{} and stores it in corresponding object files.
The compiler generates machine code for the 64-bit operating mode defined by the AMD64 architecture.
For debugging purposes, it also creates a debugging information file as well as an assembly file containing a listing of the generated machine code.
The macro \texttt{\_\_amd64\_\_} is predefined in order to enable programmers to identify this tool and its target architecture while compiling.
Programs generated with this compiler require additional runtime support that is stored in the \file{cpp\-amd64\-run} library file.
\flowgraph{\resource{\cpp{}\\source code} \ar[r] & \toolbox{cppamd64} \ar[r] \ar[d] \ar[rd] & \resource{object file} \\ \variable{ECSINCLUDE} \ar[ru] & \resource{debugging\\information} & \resource{assembly\\listing}}
\seecpp\seeassembly\seeamd\seeobject\seedebugging
}

\providecommand{\cpparma}{
\toolsection{cpparma32} is a compiler for the \cpp{} programming language targeting the ARM hardware architecture.
It generates machine code for ARM processors executing A32 instructions from programs written in \cpp{} and stores it in corresponding object files.
For debugging purposes, it also creates a debugging information file as well as an assembly file containing a listing of the generated machine code.
The macro \texttt{\_\_arma32\_\_} is predefined in order to enable programmers to identify this tool and its target architecture while compiling.
Programs generated with this compiler require additional runtime support that is stored in the \file{cpp\-arma32\-run} library file.
\flowgraph{\resource{\cpp{}\\source code} \ar[r] & \toolbox{cpparma32} \ar[r] \ar[d] \ar[rd] & \resource{object file} \\ \variable{ECSINCLUDE} \ar[ru] & \resource{debugging\\information} & \resource{assembly\\listing}}
\seecpp\seeassembly\seearm\seeobject\seedebugging
}

\providecommand{\cpparmb}{
\toolsection{cpparma64} is a compiler for the \cpp{} programming language targeting the ARM hardware architecture.
It generates machine code for ARM processors executing A64 instructions from programs written in \cpp{} and stores it in corresponding object files.
For debugging purposes, it also creates a debugging information file as well as an assembly file containing a listing of the generated machine code.
The macro \texttt{\_\_arma64\_\_} is predefined in order to enable programmers to identify this tool and its target architecture while compiling.
Programs generated with this compiler require additional runtime support that is stored in the \file{cpp\-arma64\-run} library file.
\flowgraph{\resource{\cpp{}\\source code} \ar[r] & \toolbox{cpparma64} \ar[r] \ar[d] \ar[rd] & \resource{object file} \\ \variable{ECSINCLUDE} \ar[ru] & \resource{debugging\\information} & \resource{assembly\\listing}}
\seecpp\seeassembly\seearm\seeobject\seedebugging
}

\providecommand{\cpparmc}{
\toolsection{cpparmt32} is a compiler for the \cpp{} programming language targeting the ARM hardware architecture.
It generates machine code for ARM processors without floating-point extension executing T32 instructions from programs written in \cpp{} and stores it in corresponding object files.
For debugging purposes, it also creates a debugging information file as well as an assembly file containing a listing of the generated machine code.
The macro \texttt{\_\_armt32\_\_} is predefined in order to enable programmers to identify this tool and its target architecture while compiling.
Programs generated with this compiler require additional runtime support that is stored in the \file{cpp\-armt32\-run} library file.
\flowgraph{\resource{\cpp{}\\source code} \ar[r] & \toolbox{cpparmt32} \ar[r] \ar[d] \ar[rd] & \resource{object file} \\ \variable{ECSINCLUDE} \ar[ru] & \resource{debugging\\information} & \resource{assembly\\listing}}
\seecpp\seeassembly\seearm\seeobject\seedebugging
}

\providecommand{\cpparmcfpe}{
\toolsection{cpparmt32fpe} is a compiler for the \cpp{} programming language targeting the ARM hardware architecture.
It generates machine code for ARM processors with floating-point extension executing T32 instructions from programs written in \cpp{} and stores it in corresponding object files.
For debugging purposes, it also creates a debugging information file as well as an assembly file containing a listing of the generated machine code.
The macro \texttt{\_\_armt32fpe\_\_} is predefined in order to enable programmers to identify this tool and its target architecture while compiling.
Programs generated with this compiler require additional runtime support that is stored in the \file{cpp\-armt32\-fpe\-run} library file.
\flowgraph{\resource{\cpp{}\\source code} \ar[r] & \toolbox{cpparmt32fpe} \ar[r] \ar[d] \ar[rd] & \resource{object file} \\ \variable{ECSINCLUDE} \ar[ru] & \resource{debugging\\information} & \resource{assembly\\listing}}
\seecpp\seeassembly\seearm\seeobject\seedebugging
}

\providecommand{\cppavr}{
\toolsection{cppavr} is a compiler for the \cpp{} programming language targeting the AVR hardware architecture.
It generates machine code for AVR processors from programs written in \cpp{} and stores it in corresponding object files.
For debugging purposes, it also creates a debugging information file as well as an assembly file containing a listing of the generated machine code.
The macro \texttt{\_\_avr\_\_} is predefined in order to enable programmers to identify this tool and its target architecture while compiling.
Programs generated with this compiler require additional runtime support that is stored in the \file{cpp\-avr\-run} library file.
\flowgraph{\resource{\cpp{}\\source code} \ar[r] & \toolbox{cppavr} \ar[r] \ar[d] \ar[rd] & \resource{object file} \\ \variable{ECSINCLUDE} \ar[ru] & \resource{debugging\\information} & \resource{assembly\\listing}}
\seecpp\seeassembly\seeavr\seeobject\seedebugging
}

\providecommand{\cppavrtt}{
\toolsection{cppavr32} is a compiler for the \cpp{} programming language targeting the AVR32 hardware architecture.
It generates machine code for AVR32 processors from programs written in \cpp{} and stores it in corresponding object files.
For debugging purposes, it also creates a debugging information file as well as an assembly file containing a listing of the generated machine code.
The macro \texttt{\_\_avr32\_\_} is predefined in order to enable programmers to identify this tool and its target architecture while compiling.
Programs generated with this compiler require additional runtime support that is stored in the \file{cpp\-avr32\-run} library file.
\flowgraph{\resource{\cpp{}\\source code} \ar[r] & \toolbox{cppavr32} \ar[r] \ar[d] \ar[rd] & \resource{object file} \\ \variable{ECSINCLUDE} \ar[ru] & \resource{debugging\\information} & \resource{assembly\\listing}}
\seecpp\seeassembly\seeavrtt\seeobject\seedebugging
}

\providecommand{\cppmabk}{
\toolsection{cppm68k} is a compiler for the \cpp{} programming language targeting the M68000 hardware architecture.
It generates machine code for M68000 processors from programs written in \cpp{} and stores it in corresponding object files.
For debugging purposes, it also creates a debugging information file as well as an assembly file containing a listing of the generated machine code.
The macro \texttt{\_\_m68k\_\_} is predefined in order to enable programmers to identify this tool and its target architecture while compiling.
Programs generated with this compiler require additional runtime support that is stored in the \file{cpp\-m68k\-run} library file.
\flowgraph{\resource{\cpp{}\\source code} \ar[r] & \toolbox{cppm68k} \ar[r] \ar[d] \ar[rd] & \resource{object file} \\ \variable{ECSINCLUDE} \ar[ru] & \resource{debugging\\information} & \resource{assembly\\listing}}
\seecpp\seeassembly\seemabk\seeobject\seedebugging
}

\providecommand{\cppmibl}{
\toolsection{cppmibl} is a compiler for the \cpp{} programming language targeting the MicroBlaze hardware architecture.
It generates machine code for MicroBlaze processors from programs written in \cpp{} and stores it in corresponding object files.
For debugging purposes, it also creates a debugging information file as well as an assembly file containing a listing of the generated machine code.
The macro \texttt{\_\_mibl\_\_} is predefined in order to enable programmers to identify this tool and its target architecture while compiling.
Programs generated with this compiler require additional runtime support that is stored in the \file{cpp\-mibl\-run} library file.
\flowgraph{\resource{\cpp{}\\source code} \ar[r] & \toolbox{cppmibl} \ar[r] \ar[d] \ar[rd] & \resource{object file} \\ \variable{ECSINCLUDE} \ar[ru] & \resource{debugging\\information} & \resource{assembly\\listing}}
\seecpp\seeassembly\seemibl\seeobject\seedebugging
}

\providecommand{\cppmipsa}{
\toolsection{cppmips32} is a compiler for the \cpp{} programming language targeting the MIPS32 hardware architecture.
It generates machine code for MIPS32 processors from programs written in \cpp{} and stores it in corresponding object files.
For debugging purposes, it also creates a debugging information file as well as an assembly file containing a listing of the generated machine code.
The macro \texttt{\_\_mips32\_\_} is predefined in order to enable programmers to identify this tool and its target architecture while compiling.
Programs generated with this compiler require additional runtime support that is stored in the \file{cpp\-mips32\-run} library file.
\flowgraph{\resource{\cpp{}\\source code} \ar[r] & \toolbox{cppmips32} \ar[r] \ar[d] \ar[rd] & \resource{object file} \\ \variable{ECSINCLUDE} \ar[ru] & \resource{debugging\\information} & \resource{assembly\\listing}}
\seecpp\seeassembly\seemips\seeobject\seedebugging
}

\providecommand{\cppmipsb}{
\toolsection{cppmips64} is a compiler for the \cpp{} programming language targeting the MIPS64 hardware architecture.
It generates machine code for MIPS64 processors from programs written in \cpp{} and stores it in corresponding object files.
For debugging purposes, it also creates a debugging information file as well as an assembly file containing a listing of the generated machine code.
The macro \texttt{\_\_mips64\_\_} is predefined in order to enable programmers to identify this tool and its target architecture while compiling.
Programs generated with this compiler require additional runtime support that is stored in the \file{cpp\-mips64\-run} library file.
\flowgraph{\resource{\cpp{}\\source code} \ar[r] & \toolbox{cppmips64} \ar[r] \ar[d] \ar[rd] & \resource{object file} \\ \variable{ECSINCLUDE} \ar[ru] & \resource{debugging\\information} & \resource{assembly\\listing}}
\seecpp\seeassembly\seemips\seeobject\seedebugging
}

\providecommand{\cppmmix}{
\toolsection{cppmmix} is a compiler for the \cpp{} programming language targeting the MMIX hardware architecture.
It generates machine code for MMIX processors from programs written in \cpp{} and stores it in corresponding object files.
For debugging purposes, it also creates a debugging information file as well as an assembly file containing a listing of the generated machine code.
The macro \texttt{\_\_mmix\_\_} is predefined in order to enable programmers to identify this tool and its target architecture while compiling.
Programs generated with this compiler require additional runtime support that is stored in the \file{cpp\-mmix\-run} library file.
\flowgraph{\resource{\cpp{}\\source code} \ar[r] & \toolbox{cppmmix} \ar[r] \ar[d] \ar[rd] & \resource{object file} \\ \variable{ECSINCLUDE} \ar[ru] & \resource{debugging\\information} & \resource{assembly\\listing}}
\seecpp\seeassembly\seemmix\seeobject\seedebugging
}

\providecommand{\cpporok}{
\toolsection{cppor1k} is a compiler for the \cpp{} programming language targeting the OpenRISC 1000 hardware architecture.
It generates machine code for OpenRISC 1000 processors from programs written in \cpp{} and stores it in corresponding object files.
For debugging purposes, it also creates a debugging information file as well as an assembly file containing a listing of the generated machine code.
The macro \texttt{\_\_or1k\_\_} is predefined in order to enable programmers to identify this tool and its target architecture while compiling.
Programs generated with this compiler require additional runtime support that is stored in the \file{cpp\-or1k\-run} library file.
\flowgraph{\resource{\cpp{}\\source code} \ar[r] & \toolbox{cppor1k} \ar[r] \ar[d] \ar[rd] & \resource{object file} \\ \variable{ECSINCLUDE} \ar[ru] & \resource{debugging\\information} & \resource{assembly\\listing}}
\seecpp\seeassembly\seeorok\seeobject\seedebugging
}

\providecommand{\cppppca}{
\toolsection{cppppc32} is a compiler for the \cpp{} programming language targeting the PowerPC hardware architecture.
It generates machine code for PowerPC processors from programs written in \cpp{} and stores it in corresponding object files.
The compiler generates machine code for the 32-bit operating mode defined by the PowerPC architecture.
For debugging purposes, it also creates a debugging information file as well as an assembly file containing a listing of the generated machine code.
The macro \texttt{\_\_ppc32\_\_} is predefined in order to enable programmers to identify this tool and its target architecture while compiling.
Programs generated with this compiler require additional runtime support that is stored in the \file{cpp\-ppc32\-run} library file.
\flowgraph{\resource{\cpp{}\\source code} \ar[r] & \toolbox{cppppc32} \ar[r] \ar[d] \ar[rd] & \resource{object file} \\ \variable{ECSINCLUDE} \ar[ru] & \resource{debugging\\information} & \resource{assembly\\listing}}
\seecpp\seeassembly\seeppc\seeobject\seedebugging
}

\providecommand{\cppppcb}{
\toolsection{cppppc64} is a compiler for the \cpp{} programming language targeting the PowerPC hardware architecture.
It generates machine code for PowerPC processors from programs written in \cpp{} and stores it in corresponding object files.
The compiler generates machine code for the 64-bit operating mode defined by the PowerPC architecture.
For debugging purposes, it also creates a debugging information file as well as an assembly file containing a listing of the generated machine code.
The macro \texttt{\_\_ppc64\_\_} is predefined in order to enable programmers to identify this tool and its target architecture while compiling.
Programs generated with this compiler require additional runtime support that is stored in the \file{cpp\-ppc64\-run} library file.
\flowgraph{\resource{\cpp{}\\source code} \ar[r] & \toolbox{cppppc64} \ar[r] \ar[d] \ar[rd] & \resource{object file} \\ \variable{ECSINCLUDE} \ar[ru] & \resource{debugging\\information} & \resource{assembly\\listing}}
\seecpp\seeassembly\seeppc\seeobject\seedebugging
}

\providecommand{\cpprisc}{
\toolsection{cpprisc} is a compiler for the \cpp{} programming language targeting the RISC hardware architecture.
It generates machine code for RISC processors from programs written in \cpp{} and stores it in corresponding object files.
For debugging purposes, it also creates a debugging information file as well as an assembly file containing a listing of the generated machine code.
The macro \texttt{\_\_risc\_\_} is predefined in order to enable programmers to identify this tool and its target architecture while compiling.
Programs generated with this compiler require additional runtime support that is stored in the \file{cpp\-risc\-run} library file.
\flowgraph{\resource{\cpp{}\\source code} \ar[r] & \toolbox{cpprisc} \ar[r] \ar[d] \ar[rd] & \resource{object file} \\ \variable{ECSINCLUDE} \ar[ru] & \resource{debugging\\information} & \resource{assembly\\listing}}
\seecpp\seeassembly\seerisc\seeobject\seedebugging
}

\providecommand{\cppwasm}{
\toolsection{cppwasm} is a compiler for the \cpp{} programming language targeting the WebAssembly architecture.
It generates machine code for WebAssembly targets from programs written in \cpp{} and stores it in corresponding object files.
For debugging purposes, it also creates a debugging information file as well as an assembly file containing a listing of the generated machine code.
The macro \texttt{\_\_wasm\_\_} is predefined in order to enable programmers to identify this tool and its target architecture while compiling.
Programs generated with this compiler require additional runtime support that is stored in the \file{cpp\-wasm\-run} library file.
\flowgraph{\resource{\cpp{}\\source code} \ar[r] & \toolbox{cppwasm} \ar[r] \ar[d] \ar[rd] & \resource{object file} \\ \variable{ECSINCLUDE} \ar[ru] & \resource{debugging\\information} & \resource{assembly\\listing}}
\seecpp\seeassembly\seewasm\seeobject\seedebugging
}

% FALSE tools

\providecommand{\falprint}{
\toolsection{falprint} is a pretty printer for the FALSE programming language.
It reformats the source code of FALSE programs and writes it to the standard output stream.
\flowgraph{\resource{FALSE\\source code} \ar[r] & \toolbox{falprint} \ar[r] & \resource{reformatted\\source code}}
\seefalse
}

\providecommand{\falcheck}{
\toolsection{falcheck} is a syntactic and semantic checker for the FALSE programming language.
It just performs syntactic and semantic checks on FALSE programs and writes its diagnostic messages to the standard error stream.
\flowgraph{\resource{FALSE\\source code} \ar[r] & \toolbox{falcheck} \ar[r] & \resource{diagnostic\\messages}}
\seefalse
}

\providecommand{\faldump}{
\toolsection{faldump} is a serializer for the FALSE programming language.
It dumps the complete internal representation of programs written in FALSE into an XML document.
\debuggingtool
\flowgraph{\resource{FALSE\\source code} \ar[r] & \toolbox{faldump} \ar[r] & \resource{internal\\representation}}
\seefalse
}

\providecommand{\falrun}{
\toolsection{falrun} is an interpreter for the FALSE programming language.
It processes and executes programs written in FALSE\@.
\flowgraph{\resource{FALSE\\source code} \ar[r] & \toolbox{falrun} \ar@/u/[r] & \resource{input/\\output} \ar@/d/[l]}
\seefalse
}

\providecommand{\falcpp}{
\toolsection{falcpp} is a transpiler for the FALSE programming language.
It translates programs written in FALSE into \cpp{} programs and stores them in corresponding source files.
\flowgraph{\resource{FALSE\\source code} \ar[r] & \toolbox{falcpp} \ar[r] & \resource{\cpp{}\\source file}}
\seefalse\seecpp
}

\providecommand{\falcode}{
\toolsection{falcode} is an intermediate code generator for the FALSE programming language.
It generates intermediate code from programs written in FALSE and stores it in corresponding assembly files.
\debuggingtool
\flowgraph{\resource{FALSE\\source code} \ar[r] & \toolbox{falcode} \ar[r] & \resource{intermediate\\code}}
\seefalse\seeassembly\seecode
}

\providecommand{\falamda}{
\toolsection{falamd16} is a compiler for the FALSE programming language targeting the AMD64 hardware architecture.
It generates machine code for AMD64 processors from programs written in FALSE and stores it in corresponding object files.
The compiler generates machine code for the 16-bit operating mode defined by the AMD64 architecture.
\flowgraph{\resource{FALSE\\source code} \ar[r] & \toolbox{falamd16} \ar[r] & \resource{object file}}
\seefalse\seeamd\seeobject
}

\providecommand{\falamdb}{
\toolsection{falamd32} is a compiler for the FALSE programming language targeting the AMD64 hardware architecture.
It generates machine code for AMD64 processors from programs written in FALSE and stores it in corresponding object files.
The compiler generates machine code for the 32-bit operating mode defined by the AMD64 architecture.
\flowgraph{\resource{FALSE\\source code} \ar[r] & \toolbox{falamd32} \ar[r] & \resource{object file}}
\seefalse\seeamd\seeobject
}

\providecommand{\falamdc}{
\toolsection{falamd64} is a compiler for the FALSE programming language targeting the AMD64 hardware architecture.
It generates machine code for AMD64 processors from programs written in FALSE and stores it in corresponding object files.
The compiler generates machine code for the 64-bit operating mode defined by the AMD64 architecture.
\flowgraph{\resource{FALSE\\source code} \ar[r] & \toolbox{falamd64} \ar[r] & \resource{object file}}
\seefalse\seeamd\seeobject
}

\providecommand{\falarma}{
\toolsection{falarma32} is a compiler for the FALSE programming language targeting the ARM hardware architecture.
It generates machine code for ARM processors executing A32 instructions from programs written in FALSE and stores it in corresponding object files.
\flowgraph{\resource{FALSE\\source code} \ar[r] & \toolbox{falarma32} \ar[r] & \resource{object file}}
\seefalse\seearm\seeobject
}

\providecommand{\falarmb}{
\toolsection{falarma64} is a compiler for the FALSE programming language targeting the ARM hardware architecture.
It generates machine code for ARM processors executing A64 instructions from programs written in FALSE and stores it in corresponding object files.
\flowgraph{\resource{FALSE\\source code} \ar[r] & \toolbox{falarma64} \ar[r] & \resource{object file}}
\seefalse\seearm\seeobject
}

\providecommand{\falarmc}{
\toolsection{falarmt32} is a compiler for the FALSE programming language targeting the ARM hardware architecture.
It generates machine code for ARM processors without floating-point extension executing T32 instructions from programs written in FALSE and stores it in corresponding object files.
\flowgraph{\resource{FALSE\\source code} \ar[r] & \toolbox{falarmt32} \ar[r] & \resource{object file}}
\seefalse\seearm\seeobject
}

\providecommand{\falarmcfpe}{
\toolsection{falarmt32fpe} is a compiler for the FALSE programming language targeting the ARM hardware architecture.
It generates machine code for ARM processors with floating-point extension executing T32 instructions from programs written in FALSE and stores it in corresponding object files.
\flowgraph{\resource{FALSE\\source code} \ar[r] & \toolbox{falarmt32fpe} \ar[r] & \resource{object file}}
\seefalse\seearm\seeobject
}

\providecommand{\falavr}{
\toolsection{falavr} is a compiler for the FALSE programming language targeting the AVR hardware architecture.
It generates machine code for AVR processors from programs written in FALSE and stores it in corresponding object files.
\flowgraph{\resource{FALSE\\source code} \ar[r] & \toolbox{falavr} \ar[r] & \resource{object file}}
\seefalse\seeavr\seeobject
}

\providecommand{\falavrtt}{
\toolsection{falavr32} is a compiler for the FALSE programming language targeting the AVR32 hardware architecture.
It generates machine code for AVR32 processors from programs written in FALSE and stores it in corresponding object files.
\flowgraph{\resource{FALSE\\source code} \ar[r] & \toolbox{falavr32} \ar[r] & \resource{object file}}
\seefalse\seeavrtt\seeobject
}

\providecommand{\falmabk}{
\toolsection{falm68k} is a compiler for the FALSE programming language targeting the M68000 hardware architecture.
It generates machine code for M68000 processors from programs written in FALSE and stores it in corresponding object files.
\flowgraph{\resource{FALSE\\source code} \ar[r] & \toolbox{falm68k} \ar[r] & \resource{object file}}
\seefalse\seemabk\seeobject
}

\providecommand{\falmibl}{
\toolsection{falmibl} is a compiler for the FALSE programming language targeting the MicroBlaze hardware architecture.
It generates machine code for MicroBlaze processors from programs written in FALSE and stores it in corresponding object files.
\flowgraph{\resource{FALSE\\source code} \ar[r] & \toolbox{falmibl} \ar[r] & \resource{object file}}
\seefalse\seemibl\seeobject
}

\providecommand{\falmipsa}{
\toolsection{falmips32} is a compiler for the FALSE programming language targeting the MIPS32 hardware architecture.
It generates machine code for MIPS32 processors from programs written in FALSE and stores it in corresponding object files.
\flowgraph{\resource{FALSE\\source code} \ar[r] & \toolbox{falmips32} \ar[r] & \resource{object file}}
\seefalse\seemips\seeobject
}

\providecommand{\falmipsb}{
\toolsection{falmips64} is a compiler for the FALSE programming language targeting the MIPS64 hardware architecture.
It generates machine code for MIPS64 processors from programs written in FALSE and stores it in corresponding object files.
\flowgraph{\resource{FALSE\\source code} \ar[r] & \toolbox{falmips64} \ar[r] & \resource{object file}}
\seefalse\seemips\seeobject
}

\providecommand{\falmmix}{
\toolsection{falmmix} is a compiler for the FALSE programming language targeting the MMIX hardware architecture.
It generates machine code for MMIX processors from programs written in FALSE and stores it in corresponding object files.
\flowgraph{\resource{FALSE\\source code} \ar[r] & \toolbox{falmmix} \ar[r] & \resource{object file}}
\seefalse\seemmix\seeobject
}

\providecommand{\falorok}{
\toolsection{falor1k} is a compiler for the FALSE programming language targeting the OpenRISC 1000 hardware architecture.
It generates machine code for OpenRISC 1000 processors from programs written in FALSE and stores it in corresponding object files.
\flowgraph{\resource{FALSE\\source code} \ar[r] & \toolbox{falor1k} \ar[r] & \resource{object file}}
\seefalse\seeorok\seeobject
}

\providecommand{\falppca}{
\toolsection{falppc32} is a compiler for the FALSE programming language targeting the PowerPC hardware architecture.
It generates machine code for PowerPC processors from programs written in FALSE and stores it in corresponding object files.
The compiler generates machine code for the 32-bit operating mode defined by the PowerPC architecture.
\flowgraph{\resource{FALSE\\source code} \ar[r] & \toolbox{falppc32} \ar[r] & \resource{object file}}
\seefalse\seeppc\seeobject
}

\providecommand{\falppcb}{
\toolsection{falppc64} is a compiler for the FALSE programming language targeting the PowerPC hardware architecture.
It generates machine code for PowerPC processors from programs written in FALSE and stores it in corresponding object files.
The compiler generates machine code for the 64-bit operating mode defined by the PowerPC architecture.
\flowgraph{\resource{FALSE\\source code} \ar[r] & \toolbox{falppc64} \ar[r] & \resource{object file}}
\seefalse\seeppc\seeobject
}

\providecommand{\falrisc}{
\toolsection{falrisc} is a compiler for the FALSE programming language targeting the RISC hardware architecture.
It generates machine code for RISC processors from programs written in FALSE and stores it in corresponding object files.
\flowgraph{\resource{FALSE\\source code} \ar[r] & \toolbox{falrisc} \ar[r] & \resource{object file}}
\seefalse\seerisc\seeobject
}

\providecommand{\falwasm}{
\toolsection{falwasm} is a compiler for the FALSE programming language targeting the WebAssembly architecture.
It generates machine code for WebAssembly targets from programs written in FALSE and stores it in corresponding object files.
\flowgraph{\resource{FALSE\\source code} \ar[r] & \toolbox{falwasm} \ar[r] & \resource{object file}}
\seefalse\seewasm\seeobject
}

% Oberon tools

\providecommand{\obprint}{
\toolsection{obprint} is a pretty printer for the Oberon programming language.
It reformats the source code of Oberon modules and writes it to the standard output stream.
\flowgraph{\resource{Oberon\\source code} \ar[r] & \toolbox{obprint} \ar[r] & \resource{reformatted\\source code}}
\seeoberon
}

\providecommand{\obcheck}{
\toolsection{obcheck} is a syntactic and semantic checker for the Oberon programming language.
It just performs syntactic and semantic checks on Oberon modules and writes its diagnostic messages to the standard error stream.
In addition, it stores the interface of each module in a symbol file which is required when other modules import the module.
\flowgraph{\resource{Oberon\\source code} \ar[r] & \toolbox{obcheck} \ar[r] \ar@/l/[d] & \resource{diagnostic\\messages} \\ \variable{ECSIMPORT} \ar[ru] & \resource{symbol\\files} \ar@/r/[u]}
\seeoberon
}

\providecommand{\obdump}{
\toolsection{obdump} is a serializer for the Oberon programming language.
It dumps the complete internal representation of modules written in Oberon into an XML document.
\debuggingtool
\flowgraph{\resource{Oberon\\source code} \ar[r] & \toolbox{obdump} \ar[r] \ar@/l/[d] & \resource{internal\\representation} \\ \variable{ECSIMPORT} \ar[ru] & \resource{symbol\\files} \ar@/r/[u]}
\seeoberon
}

\providecommand{\obrun}{
\toolsection{obrun} is an interpreter for the Oberon programming language.
It processes and executes modules written in Oberon.
This tool does neither generate nor process symbol files while interpreting modules.
If a module is imported by another one, its filename has to be named before the other one in the list of command-line arguments.
\flowgraph{\resource{Oberon\\source code} \ar[r] & \toolbox{obrun} \ar@/u/[r] & \resource{input/\\output} \ar@/d/[l]}
\seeoberon
}

\providecommand{\obcpp}{
\toolsection{obcpp} is a transpiler for the Oberon programming language.
It translates programs written in Oberon into \cpp{} programs and stores them in corresponding source and header files.
In addition, it stores the interface of each module in a symbol file which is required when other modules import the module.
The same interface is provided by the generated header file which can be used in other parts of the \cpp{} program.
\flowgraph{\resource{Oberon\\source code} \ar[r] & \toolbox{obcpp} \ar[r] \ar@/l/[d] \ar[rd] & \resource{\cpp{}\\source file} \\ \variable{ECSIMPORT} \ar[ru] & \resource{symbol\\files} \ar@/r/[u] & \resource{\cpp{}\\header file}}
\seeoberon\seecpp
}

\providecommand{\obdoc}{
\toolsection{obdoc} is a generic documentation generator for the Oberon programming language.
It processes several Oberon modules and assembles all information therein into a generic documentation.
In addition, it stores the interface of each module in a symbol file which is required when other modules import the module.
\debuggingtool
\flowgraph{\resource{Oberon\\source code} \ar[r] & \toolbox{obdoc} \ar[r] \ar@/l/[d] & \resource{generic\\documentation} \\ \variable{ECSIMPORT} \ar[ru] & \resource{symbol\\files} \ar@/r/[u]}
\seeoberon\seedocumentation
}

\providecommand{\obhtml}{
\toolsection{obhtml} is an HTML documentation generator for the Oberon programming language.
It processes several Oberon modules and assembles all information therein into an HTML document.
In addition, it stores the interface of each module in a symbol file which is required when other modules import the module.
\flowgraph{\resource{Oberon\\source code} \ar[r] & \toolbox{obhtml} \ar[r] \ar@/l/[d] & \resource{HTML\\document} \\ \variable{ECSIMPORT} \ar[ru] & \resource{symbol\\files} \ar@/r/[u]}
\seeoberon\seedocumentation
}

\providecommand{\oblatex}{
\toolsection{oblatex} is a Latex documentation generator for the Oberon programming language.
It processes several Oberon modules and assembles all information therein into a Latex document.
In addition, it stores the interface of each module in a symbol file which is required when other modules import the module.
\flowgraph{\resource{Oberon\\source code} \ar[r] & \toolbox{oblatex} \ar[r] \ar@/l/[d] & \resource{Latex\\document} \\ \variable{ECSIMPORT} \ar[ru] & \resource{symbol\\files} \ar@/r/[u]}
\seeoberon\seedocumentation
}

\providecommand{\obcode}{
\toolsection{obcode} is an intermediate code generator for the Oberon programming language.
It generates intermediate code from modules written in Oberon and stores it in corresponding assembly files.
In addition, it stores the interface of each module in a symbol file which is required when other modules import the module.
Programs generated with this tool require additional runtime support that is stored in the \file{ob\-code\-run} library file.
\debuggingtool
\flowgraph{\resource{Oberon\\source code} \ar[r] & \toolbox{obcode} \ar[r] \ar@/l/[d] & \resource{intermediate\\code} \\ \variable{ECSIMPORT} \ar[ru] & \resource{symbol\\files} \ar@/r/[u]}
\seeoberon\seeassembly\seecode
}

\providecommand{\obamda}{
\toolsection{obamd16} is a compiler for the Oberon programming language targeting the AMD64 hardware architecture.
It generates machine code for AMD64 processors from modules written in Oberon and stores it in corresponding object files.
The compiler generates machine code for the 16-bit operating mode defined by the AMD64 architecture.
For debugging purposes, it also creates a debugging information file as well as an assembly file containing a listing of the generated machine code.
In addition, it stores the interface of each module in a symbol file which is required when other modules import the module.
Programs generated with this compiler require additional runtime support that is stored in the \file{ob\-amd16\-run} library file.
\flowgraph{\resource{Oberon\\source code} \ar[r] & \toolbox{obamd16} \ar[r] \ar@/l/[d] \ar[rd] & \resource{object file} \\ \variable{ECSIMPORT} \ar[ru] & \resource{symbol\\files} \ar@/r/[u] & \resource{debugging\\information}}
\seeoberon\seeassembly\seeamd\seeobject\seedebugging
}

\providecommand{\obamdb}{
\toolsection{obamd32} is a compiler for the Oberon programming language targeting the AMD64 hardware architecture.
It generates machine code for AMD64 processors from modules written in Oberon and stores it in corresponding object files.
The compiler generates machine code for the 32-bit operating mode defined by the AMD64 architecture.
For debugging purposes, it also creates a debugging information file as well as an assembly file containing a listing of the generated machine code.
In addition, it stores the interface of each module in a symbol file which is required when other modules import the module.
Programs generated with this compiler require additional runtime support that is stored in the \file{ob\-amd32\-run} library file.
\flowgraph{\resource{Oberon\\source code} \ar[r] & \toolbox{obamd32} \ar[r] \ar@/l/[d] \ar[rd] & \resource{object file} \\ \variable{ECSIMPORT} \ar[ru] & \resource{symbol\\files} \ar@/r/[u] & \resource{debugging\\information}}
\seeoberon\seeassembly\seeamd\seeobject\seedebugging
}

\providecommand{\obamdc}{
\toolsection{obamd64} is a compiler for the Oberon programming language targeting the AMD64 hardware architecture.
It generates machine code for AMD64 processors from modules written in Oberon and stores it in corresponding object files.
The compiler generates machine code for the 64-bit operating mode defined by the AMD64 architecture.
For debugging purposes, it also creates a debugging information file as well as an assembly file containing a listing of the generated machine code.
In addition, it stores the interface of each module in a symbol file which is required when other modules import the module.
Programs generated with this compiler require additional runtime support that is stored in the \file{ob\-amd64\-run} library file.
\flowgraph{\resource{Oberon\\source code} \ar[r] & \toolbox{obamd64} \ar[r] \ar@/l/[d] \ar[rd] & \resource{object file} \\ \variable{ECSIMPORT} \ar[ru] & \resource{symbol\\files} \ar@/r/[u] & \resource{debugging\\information}}
\seeoberon\seeassembly\seeamd\seeobject\seedebugging
}

\providecommand{\obarma}{
\toolsection{obarma32} is a compiler for the Oberon programming language targeting the ARM hardware architecture.
It generates machine code for ARM processors executing A32 instructions from modules written in Oberon and stores it in corresponding object files.
For debugging purposes, it also creates a debugging information file as well as an assembly file containing a listing of the generated machine code.
In addition, it stores the interface of each module in a symbol file which is required when other modules import the module.
Programs generated with this compiler require additional runtime support that is stored in the \file{ob\-arma32\-run} library file.
\flowgraph{\resource{Oberon\\source code} \ar[r] & \toolbox{obarma32} \ar[r] \ar@/l/[d] \ar[rd] & \resource{object file} \\ \variable{ECSIMPORT} \ar[ru] & \resource{symbol\\files} \ar@/r/[u] & \resource{debugging\\information}}
\seeoberon\seeassembly\seearm\seeobject\seedebugging
}

\providecommand{\obarmb}{
\toolsection{obarma64} is a compiler for the Oberon programming language targeting the ARM hardware architecture.
It generates machine code for ARM processors executing A64 instructions from modules written in Oberon and stores it in corresponding object files.
For debugging purposes, it also creates a debugging information file as well as an assembly file containing a listing of the generated machine code.
In addition, it stores the interface of each module in a symbol file which is required when other modules import the module.
Programs generated with this compiler require additional runtime support that is stored in the \file{ob\-arma64\-run} library file.
\flowgraph{\resource{Oberon\\source code} \ar[r] & \toolbox{obarma64} \ar[r] \ar@/l/[d] \ar[rd] & \resource{object file} \\ \variable{ECSIMPORT} \ar[ru] & \resource{symbol\\files} \ar@/r/[u] & \resource{debugging\\information}}
\seeoberon\seeassembly\seearm\seeobject\seedebugging
}

\providecommand{\obarmc}{
\toolsection{obarmt32} is a compiler for the Oberon programming language targeting the ARM hardware architecture.
It generates machine code for ARM processors without floating-point extension executing T32 instructions from modules written in Oberon and stores it in corresponding object files.
For debugging purposes, it also creates a debugging information file as well as an assembly file containing a listing of the generated machine code.
In addition, it stores the interface of each module in a symbol file which is required when other modules import the module.
Programs generated with this compiler require additional runtime support that is stored in the \file{ob\-armt32\-run} library file.
\flowgraph{\resource{Oberon\\source code} \ar[r] & \toolbox{obarmt32} \ar[r] \ar@/l/[d] \ar[rd] & \resource{object file} \\ \variable{ECSIMPORT} \ar[ru] & \resource{symbol\\files} \ar@/r/[u] & \resource{debugging\\information}}
\seeoberon\seeassembly\seearm\seeobject\seedebugging
}

\providecommand{\obarmcfpe}{
\toolsection{obarmt32fpe} is a compiler for the Oberon programming language targeting the ARM hardware architecture.
It generates machine code for ARM processors with floating-point extension executing T32 instructions from modules written in Oberon and stores it in corresponding object files.
For debugging purposes, it also creates a debugging information file as well as an assembly file containing a listing of the generated machine code.
In addition, it stores the interface of each module in a symbol file which is required when other modules import the module.
Programs generated with this compiler require additional runtime support that is stored in the \file{ob\-armt32\-fpe\-run} library file.
\flowgraph{\resource{Oberon\\source code} \ar[r] & \toolbox{obarmt32fpe} \ar[r] \ar@/l/[d] \ar[rd] & \resource{object file} \\ \variable{ECSIMPORT} \ar[ru] & \resource{symbol\\files} \ar@/r/[u] & \resource{debugging\\information}}
\seeoberon\seeassembly\seearm\seeobject\seedebugging
}

\providecommand{\obavr}{
\toolsection{obavr} is a compiler for the Oberon programming language targeting the AVR hardware architecture.
It generates machine code for AVR processors from modules written in Oberon and stores it in corresponding object files.
For debugging purposes, it also creates a debugging information file as well as an assembly file containing a listing of the generated machine code.
In addition, it stores the interface of each module in a symbol file which is required when other modules import the module.
Programs generated with this compiler require additional runtime support that is stored in the \file{ob\-avr\-run} library file.
\flowgraph{\resource{Oberon\\source code} \ar[r] & \toolbox{obavr} \ar[r] \ar@/l/[d] \ar[rd] & \resource{object file} \\ \variable{ECSIMPORT} \ar[ru] & \resource{symbol\\files} \ar@/r/[u] & \resource{debugging\\information}}
\seeoberon\seeassembly\seeavr\seeobject\seedebugging
}

\providecommand{\obavrtt}{
\toolsection{obavr32} is a compiler for the Oberon programming language targeting the AVR32 hardware architecture.
It generates machine code for AVR32 processors from modules written in Oberon and stores it in corresponding object files.
For debugging purposes, it also creates a debugging information file as well as an assembly file containing a listing of the generated machine code.
In addition, it stores the interface of each module in a symbol file which is required when other modules import the module.
Programs generated with this compiler require additional runtime support that is stored in the \file{ob\-avr32\-run} library file.
\flowgraph{\resource{Oberon\\source code} \ar[r] & \toolbox{obavr32} \ar[r] \ar@/l/[d] \ar[rd] & \resource{object file} \\ \variable{ECSIMPORT} \ar[ru] & \resource{symbol\\files} \ar@/r/[u] & \resource{debugging\\information}}
\seeoberon\seeassembly\seeavrtt\seeobject\seedebugging
}

\providecommand{\obmabk}{
\toolsection{obm68k} is a compiler for the Oberon programming language targeting the M68000 hardware architecture.
It generates machine code for M68000 processors from modules written in Oberon and stores it in corresponding object files.
For debugging purposes, it also creates a debugging information file as well as an assembly file containing a listing of the generated machine code.
In addition, it stores the interface of each module in a symbol file which is required when other modules import the module.
Programs generated with this compiler require additional runtime support that is stored in the \file{ob\-m68k\-run} library file.
\flowgraph{\resource{Oberon\\source code} \ar[r] & \toolbox{obm68k} \ar[r] \ar@/l/[d] \ar[rd] & \resource{object file} \\ \variable{ECSIMPORT} \ar[ru] & \resource{symbol\\files} \ar@/r/[u] & \resource{debugging\\information}}
\seeoberon\seeassembly\seemabk\seeobject\seedebugging
}

\providecommand{\obmibl}{
\toolsection{obmibl} is a compiler for the Oberon programming language targeting the MicroBlaze hardware architecture.
It generates machine code for MicroBlaze processors from modules written in Oberon and stores it in corresponding object files.
For debugging purposes, it also creates a debugging information file as well as an assembly file containing a listing of the generated machine code.
In addition, it stores the interface of each module in a symbol file which is required when other modules import the module.
Programs generated with this compiler require additional runtime support that is stored in the \file{ob\-mibl\-run} library file.
\flowgraph{\resource{Oberon\\source code} \ar[r] & \toolbox{obmibl} \ar[r] \ar@/l/[d] \ar[rd] & \resource{object file} \\ \variable{ECSIMPORT} \ar[ru] & \resource{symbol\\files} \ar@/r/[u] & \resource{debugging\\information}}
\seeoberon\seeassembly\seemibl\seeobject\seedebugging
}

\providecommand{\obmipsa}{
\toolsection{obmips32} is a compiler for the Oberon programming language targeting the MIPS32 hardware architecture.
It generates machine code for MIPS32 processors from modules written in Oberon and stores it in corresponding object files.
For debugging purposes, it also creates a debugging information file as well as an assembly file containing a listing of the generated machine code.
In addition, it stores the interface of each module in a symbol file which is required when other modules import the module.
Programs generated with this compiler require additional runtime support that is stored in the \file{ob\-mips32\-run} library file.
\flowgraph{\resource{Oberon\\source code} \ar[r] & \toolbox{obmips32} \ar[r] \ar@/l/[d] \ar[rd] & \resource{object file} \\ \variable{ECSIMPORT} \ar[ru] & \resource{symbol\\files} \ar@/r/[u] & \resource{debugging\\information}}
\seeoberon\seeassembly\seemips\seeobject\seedebugging
}

\providecommand{\obmipsb}{
\toolsection{obmips64} is a compiler for the Oberon programming language targeting the MIPS64 hardware architecture.
It generates machine code for MIPS64 processors from modules written in Oberon and stores it in corresponding object files.
For debugging purposes, it also creates a debugging information file as well as an assembly file containing a listing of the generated machine code.
In addition, it stores the interface of each module in a symbol file which is required when other modules import the module.
Programs generated with this compiler require additional runtime support that is stored in the \file{ob\-mips64\-run} library file.
\flowgraph{\resource{Oberon\\source code} \ar[r] & \toolbox{obmips64} \ar[r] \ar@/l/[d] \ar[rd] & \resource{object file} \\ \variable{ECSIMPORT} \ar[ru] & \resource{symbol\\files} \ar@/r/[u] & \resource{debugging\\information}}
\seeoberon\seeassembly\seemips\seeobject\seedebugging
}

\providecommand{\obmmix}{
\toolsection{obmmix} is a compiler for the Oberon programming language targeting the MMIX hardware architecture.
It generates machine code for MMIX processors from modules written in Oberon and stores it in corresponding object files.
For debugging purposes, it also creates a debugging information file as well as an assembly file containing a listing of the generated machine code.
In addition, it stores the interface of each module in a symbol file which is required when other modules import the module.
Programs generated with this compiler require additional runtime support that is stored in the \file{ob\-mmix\-run} library file.
\flowgraph{\resource{Oberon\\source code} \ar[r] & \toolbox{obmmix} \ar[r] \ar@/l/[d] \ar[rd] & \resource{object file} \\ \variable{ECSIMPORT} \ar[ru] & \resource{symbol\\files} \ar@/r/[u] & \resource{debugging\\information}}
\seeoberon\seeassembly\seemmix\seeobject\seedebugging
}

\providecommand{\oborok}{
\toolsection{obor1k} is a compiler for the Oberon programming language targeting the OpenRISC 1000 hardware architecture.
It generates machine code for OpenRISC 1000 processors from modules written in Oberon and stores it in corresponding object files.
For debugging purposes, it also creates a debugging information file as well as an assembly file containing a listing of the generated machine code.
In addition, it stores the interface of each module in a symbol file which is required when other modules import the module.
Programs generated with this compiler require additional runtime support that is stored in the \file{ob\-or1k\-run} library file.
\flowgraph{\resource{Oberon\\source code} \ar[r] & \toolbox{obor1k} \ar[r] \ar@/l/[d] \ar[rd] & \resource{object file} \\ \variable{ECSIMPORT} \ar[ru] & \resource{symbol\\files} \ar@/r/[u] & \resource{debugging\\information}}
\seeoberon\seeassembly\seeorok\seeobject\seedebugging
}

\providecommand{\obppca}{
\toolsection{obppc32} is a compiler for the Oberon programming language targeting the PowerPC hardware architecture.
It generates machine code for PowerPC processors from modules written in Oberon and stores it in corresponding object files.
The compiler generates machine code for the 32-bit operating mode defined by the PowerPC architecture.
For debugging purposes, it also creates a debugging information file as well as an assembly file containing a listing of the generated machine code.
In addition, it stores the interface of each module in a symbol file which is required when other modules import the module.
Programs generated with this compiler require additional runtime support that is stored in the \file{ob\-ppc32\-run} library file.
\flowgraph{\resource{Oberon\\source code} \ar[r] & \toolbox{obppc32} \ar[r] \ar@/l/[d] \ar[rd] & \resource{object file} \\ \variable{ECSIMPORT} \ar[ru] & \resource{symbol\\files} \ar@/r/[u] & \resource{debugging\\information}}
\seeoberon\seeassembly\seeppc\seeobject\seedebugging
}

\providecommand{\obppcb}{
\toolsection{obppc64} is a compiler for the Oberon programming language targeting the PowerPC hardware architecture.
It generates machine code for PowerPC processors from modules written in Oberon and stores it in corresponding object files.
The compiler generates machine code for the 64-bit operating mode defined by the PowerPC architecture.
For debugging purposes, it also creates a debugging information file as well as an assembly file containing a listing of the generated machine code.
In addition, it stores the interface of each module in a symbol file which is required when other modules import the module.
Programs generated with this compiler require additional runtime support that is stored in the \file{ob\-ppc64\-run} library file.
\flowgraph{\resource{Oberon\\source code} \ar[r] & \toolbox{obppc64} \ar[r] \ar@/l/[d] \ar[rd] & \resource{object file} \\ \variable{ECSIMPORT} \ar[ru] & \resource{symbol\\files} \ar@/r/[u] & \resource{debugging\\information}}
\seeoberon\seeassembly\seeppc\seeobject\seedebugging
}

\providecommand{\obrisc}{
\toolsection{obrisc} is a compiler for the Oberon programming language targeting the RISC hardware architecture.
It generates machine code for RISC processors from modules written in Oberon and stores it in corresponding object files.
For debugging purposes, it also creates a debugging information file as well as an assembly file containing a listing of the generated machine code.
In addition, it stores the interface of each module in a symbol file which is required when other modules import the module.
Programs generated with this compiler require additional runtime support that is stored in the \file{ob\-risc\-run} library file.
\flowgraph{\resource{Oberon\\source code} \ar[r] & \toolbox{obrisc} \ar[r] \ar@/l/[d] \ar[rd] & \resource{object file} \\ \variable{ECSIMPORT} \ar[ru] & \resource{symbol\\files} \ar@/r/[u] & \resource{debugging\\information}}
\seeoberon\seeassembly\seerisc\seeobject\seedebugging
}

\providecommand{\obwasm}{
\toolsection{obwasm} is a compiler for the Oberon programming language targeting the WebAssembly architecture.
It generates machine code for WebAssembly targets from modules written in Oberon and stores it in corresponding object files.
For debugging purposes, it also creates a debugging information file as well as an assembly file containing a listing of the generated machine code.
In addition, it stores the interface of each module in a symbol file which is required when other modules import the module.
Programs generated with this compiler require additional runtime support that is stored in the \file{ob\-wasm\-run} library file.
\flowgraph{\resource{Oberon\\source code} \ar[r] & \toolbox{obwasm} \ar[r] \ar@/l/[d] \ar[rd] & \resource{object file} \\ \variable{ECSIMPORT} \ar[ru] & \resource{symbol\\files} \ar@/r/[u] & \resource{debugging\\information}}
\seeoberon\seeassembly\seewasm\seeobject\seedebugging
}

% converter tools

\providecommand{\dbgdwarf}{
\toolsection{dbgdwarf} is a DWARF debugging information converter tool.
It converts debugging information into the DWARF debugging data format and stores it in corresponding object files~\cite{dwarffile}.
The resulting debugging object files can be combined with runtime support that creates Executable and Linking Format (ELF) files~\cite{elffile}.
\flowgraph{\resource{debugging\\information} \ar[r] & \toolbox{dbgdwarf} \ar[r] & \resource{debugging\\object file}}
\seeobject\seedebugging
}

% assembler tools

\providecommand{\asmprint}{
\toolsection{asmprint} is a pretty printer for generic assembly code.
It reformats generic assembly code and writes it to the standard output stream.
\flowgraph{\resource{generic assembly\\source code} \ar[r] & \toolbox{asmprint} \ar[r] & \resource{reformatted\\source code}}
\seeassembly
}

\providecommand{\amdaasm}{
\toolsection{amd16asm} is an assembler for the AMD64 hardware architecture.
It translates assembly code into machine code for AMD64 processors and stores it in corresponding object files.
By default, the assembler generates machine code for the 16-bit operating mode defined by the AMD64 architecture.
\flowgraph{\resource{AMD16 assembly\\source code} \ar[r] & \toolbox{amd16asm} \ar[r] & \resource{object file}}
\seeassembly\seeamd\seeobject
}

\providecommand{\amdadism}{
\toolsection{amd16dism} is a disassembler for the AMD64 hardware architecture.
It translates machine code from object files targeting AMD64 processors into assembly code and writes it to the standard output stream.
It assumes that the machine code was generated for the 16-bit operating mode defined by the AMD64 architecture.
\flowgraph{\resource{object file} \ar[r] & \toolbox{amd16dism} \ar[r] & \resource{disassembly\\listing}}
\seeassembly\seeamd\seeobject
}

\providecommand{\amdbasm}{
\toolsection{amd32asm} is an assembler for the AMD64 hardware architecture.
It translates assembly code into machine code for AMD64 processors and stores it in corresponding object files.
By default, the assembler generates machine code for the 32-bit operating mode defined by the AMD64 architecture.
\flowgraph{\resource{AMD32 assembly\\source code} \ar[r] & \toolbox{amd32asm} \ar[r] & \resource{object file}}
\seeassembly\seeamd\seeobject
}

\providecommand{\amdbdism}{
\toolsection{amd32dism} is a disassembler for the AMD64 hardware architecture.
It translates machine code from object files targeting AMD64 processors into assembly code and writes it to the standard output stream.
It assumes that the machine code was generated for the 32-bit operating mode defined by the AMD64 architecture.
\flowgraph{\resource{object file} \ar[r] & \toolbox{amd32dism} \ar[r] & \resource{disassembly\\listing}}
\seeassembly\seeamd\seeobject
}

\providecommand{\amdcasm}{
\toolsection{amd64asm} is an assembler for the AMD64 hardware architecture.
It translates assembly code into machine code for AMD64 processors and stores it in corresponding object files.
By default, the assembler generates machine code for the 64-bit operating mode defined by the AMD64 architecture.
\flowgraph{\resource{AMD64 assembly\\source code} \ar[r] & \toolbox{amd64asm} \ar[r] & \resource{object file}}
\seeassembly\seeamd\seeobject
}

\providecommand{\amdcdism}{
\toolsection{amd64dism} is a disassembler for the AMD64 hardware architecture.
It translates machine code from object files targeting AMD64 processors into assembly code and writes it to the standard output stream.
It assumes that the machine code was generated for the 64-bit operating mode defined by the AMD64 architecture.
\flowgraph{\resource{object file} \ar[r] & \toolbox{amd64dism} \ar[r] & \resource{disassembly\\listing}}
\seeassembly\seeamd\seeobject
}

\providecommand{\armaasm}{
\toolsection{arma32asm} is an assembler for the ARM hardware architecture.
It translates assembly code into machine code for ARM processors executing A32 instructions and stores it in corresponding object files.
\flowgraph{\resource{ARM A32 assembly\\source code} \ar[r] & \toolbox{arma32asm} \ar[r] & \resource{object file}}
\seeassembly\seearm\seeobject
}

\providecommand{\armadism}{
\toolsection{arma32dism} is a disassembler for the ARM hardware architecture.
It translates machine code from object files targeting ARM processors executing A32 instructions into assembly code and writes it to the standard output stream.
\flowgraph{\resource{object file} \ar[r] & \toolbox{arma32dism} \ar[r] & \resource{disassembly\\listing}}
\seeassembly\seearm\seeobject
}

\providecommand{\armbasm}{
\toolsection{arma64asm} is an assembler for the ARM hardware architecture.
It translates assembly code into machine code for ARM processors executing A64 instructions and stores it in corresponding object files.
\flowgraph{\resource{ARM A64 assembly\\source code} \ar[r] & \toolbox{arma64asm} \ar[r] & \resource{object file}}
\seeassembly\seearm\seeobject
}

\providecommand{\armbdism}{
\toolsection{arma64dism} is a disassembler for the ARM hardware architecture.
It translates machine code from object files targeting ARM processors executing A64 instructions into assembly code and writes it to the standard output stream.
\flowgraph{\resource{object file} \ar[r] & \toolbox{arma64dism} \ar[r] & \resource{disassembly\\listing}}
\seeassembly\seearm\seeobject
}

\providecommand{\armcasm}{
\toolsection{armt32asm} is an assembler for the ARM hardware architecture.
It translates assembly code into machine code for ARM processors executing T32 instructions and stores it in corresponding object files.
\flowgraph{\resource{ARM T32 assembly\\source code} \ar[r] & \toolbox{armt32asm} \ar[r] & \resource{object file}}
\seeassembly\seearm\seeobject
}

\providecommand{\armcdism}{
\toolsection{armt32dism} is a disassembler for the ARM hardware architecture.
It translates machine code from object files targeting ARM processors executing T32 instructions into assembly code and writes it to the standard output stream.
\flowgraph{\resource{object file} \ar[r] & \toolbox{armt32dism} \ar[r] & \resource{disassembly\\listing}}
\seeassembly\seearm\seeobject
}

\providecommand{\avrasm}{
\toolsection{avrasm} is an assembler for the AVR hardware architecture.
It translates assembly code into machine code for AVR processors and stores it in corresponding object files.
The identifiers \texttt{RXL}, \texttt{RXH}, \texttt{RYL}, \texttt{RYH}, \texttt{RZL}, and \texttt{RZH} are predefined and name the corresponding registers.
The identifiers \texttt{SPL} and \texttt{SPH} are also predefined and evaluate to the address of the corresponding registers.
\flowgraph{\resource{AVR assembly\\source code} \ar[r] & \toolbox{avrasm} \ar[r] & \resource{object file}}
\seeassembly\seeavr\seeobject
}

\providecommand{\avrdism}{
\toolsection{avrdism} is a disassembler for the AVR hardware architecture.
It translates machine code from object files targeting AVR processors into assembly code and writes it to the standard output stream.
\flowgraph{\resource{object file} \ar[r] & \toolbox{avrdism} \ar[r] & \resource{disassembly\\listing}}
\seeassembly\seeavr\seeobject
}

\providecommand{\avrttasm}{
\toolsection{avr32asm} is an assembler for the AVR32 hardware architecture.
It translates assembly code into machine code for AVR32 processors and stores it in corresponding object files.
\flowgraph{\resource{AVR32 assembly\\source code} \ar[r] & \toolbox{avr32asm} \ar[r] & \resource{object file}}
\seeassembly\seeavrtt\seeobject
}

\providecommand{\avrttdism}{
\toolsection{avr32dism} is a disassembler for the AVR32 hardware architecture.
It translates machine code from object files targeting AVR32 processors into assembly code and writes it to the standard output stream.
\flowgraph{\resource{object file} \ar[r] & \toolbox{avr32dism} \ar[r] & \resource{disassembly\\listing}}
\seeassembly\seeavrtt\seeobject
}

\providecommand{\mabkasm}{
\toolsection{m68kasm} is an assembler for the M68000 hardware architecture.
It translates assembly code into machine code for M68000 processors and stores it in corresponding object files.
\flowgraph{\resource{68000 assembly\\source code} \ar[r] & \toolbox{m68kasm} \ar[r] & \resource{object file}}
\seeassembly\seemabk\seeobject
}

\providecommand{\mabkdism}{
\toolsection{m68kdism} is a disassembler for the M68000 hardware architecture.
It translates machine code from object files targeting M68000 processors into assembly code and writes it to the standard output stream.
\flowgraph{\resource{object file} \ar[r] & \toolbox{m68kdism} \ar[r] & \resource{disassembly\\listing}}
\seeassembly\seemabk\seeobject
}

\providecommand{\miblasm}{
\toolsection{miblasm} is an assembler for the MicroBlaze hardware architecture.
It translates assembly code into machine code for MicroBlaze processors and stores it in corresponding object files.
\flowgraph{\resource{MicroBlaze assembly\\source code} \ar[r] & \toolbox{miblasm} \ar[r] & \resource{object file}}
\seeassembly\seemibl\seeobject
}

\providecommand{\mibldism}{
\toolsection{mibldism} is a disassembler for the MicroBlaze hardware architecture.
It translates machine code from object files targeting MicroBlaze processors into assembly code and writes it to the standard output stream.
\flowgraph{\resource{object file} \ar[r] & \toolbox{mibldism} \ar[r] & \resource{disassembly\\listing}}
\seeassembly\seemibl\seeobject
}

\providecommand{\mipsaasm}{
\toolsection{mips32asm} is an assembler for the MIPS32 hardware architecture.
It translates assembly code into machine code for MIPS32 processors and stores it in corresponding object files.
\flowgraph{\resource{MIPS32 assembly\\source code} \ar[r] & \toolbox{mips32asm} \ar[r] & \resource{object file}}
\seeassembly\seemips\seeobject
}

\providecommand{\mipsadism}{
\toolsection{mips32dism} is a disassembler for the MIPS32 hardware architecture.
It translates machine code from object files targeting MIPS32 processors into assembly code and writes it to the standard output stream.
\flowgraph{\resource{object file} \ar[r] & \toolbox{mips32dism} \ar[r] & \resource{disassembly\\listing}}
\seeassembly\seemips\seeobject
}

\providecommand{\mipsbasm}{
\toolsection{mips64asm} is an assembler for the MIPS64 hardware architecture.
It translates assembly code into machine code for MIPS64 processors and stores it in corresponding object files.
\flowgraph{\resource{MIPS64 assembly\\source code} \ar[r] & \toolbox{mips64asm} \ar[r] & \resource{object file}}
\seeassembly\seemips\seeobject
}

\providecommand{\mipsbdism}{
\toolsection{mips64dism} is a disassembler for the MIPS64 hardware architecture.
It translates machine code from object files targeting MIPS64 processors into assembly code and writes it to the standard output stream.
\flowgraph{\resource{object file} \ar[r] & \toolbox{mips64dism} \ar[r] & \resource{disassembly\\listing}}
\seeassembly\seemips\seeobject
}

\providecommand{\mmixasm}{
\toolsection{mmixasm} is an assembler for the MMIX hardware architecture.
It translates assembly code into machine code for MMIX processors and stores it in corresponding object files.
The names of all special registers are predefined and evaluate to the corresponding number.
\flowgraph{\resource{MMIX assembly\\source code} \ar[r] & \toolbox{mmixasm} \ar[r] & \resource{object file}}
\seeassembly\seemmix\seeobject
}

\providecommand{\mmixdism}{
\toolsection{mmixdism} is a disassembler for the MMIX hardware architecture.
It translates machine code from object files targeting MMIX processors into assembly code and writes it to the standard output stream.
\flowgraph{\resource{object file} \ar[r] & \toolbox{mmixdism} \ar[r] & \resource{disassembly\\listing}}
\seeassembly\seemmix\seeobject
}

\providecommand{\orokasm}{
\toolsection{or1kasm} is an assembler for the OpenRISC 1000 hardware architecture.
It translates assembly code into machine code for OpenRISC 1000 processors and stores it in corresponding object files.
\flowgraph{\resource{OpenRISC 1000 assembly\\source code} \ar[r] & \toolbox{or1kasm} \ar[r] & \resource{object file}}
\seeassembly\seeorok\seeobject
}

\providecommand{\orokdism}{
\toolsection{or1kdism} is a disassembler for the OpenRISC 1000 hardware architecture.
It translates machine code from object files targeting OpenRISC 1000 processors into assembly code and writes it to the standard output stream.
\flowgraph{\resource{object file} \ar[r] & \toolbox{or1kdism} \ar[r] & \resource{disassembly\\listing}}
\seeassembly\seeorok\seeobject
}

\providecommand{\ppcaasm}{
\toolsection{ppc32asm} is an assembler for the PowerPC hardware architecture.
It translates assembly code into machine code for PowerPC processors and stores it in corresponding object files.
By default, the assembler generates machine code for the 32-bit operating mode defined by the PowerPC architecture.
\flowgraph{\resource{PowerPC assembly\\source code} \ar[r] & \toolbox{ppc32asm} \ar[r] & \resource{object file}}
\seeassembly\seeppc\seeobject
}

\providecommand{\ppcadism}{
\toolsection{ppc32dism} is a disassembler for the PowerPC hardware architecture.
It translates machine code from object files targeting PowerPC processors into assembly code and writes it to the standard output stream.
It assumes that the machine code was generated for the 32-bit operating mode defined by the PowerPC architecture.
\flowgraph{\resource{object file} \ar[r] & \toolbox{ppc32dism} \ar[r] & \resource{disassembly\\listing}}
\seeassembly\seeppc\seeobject
}

\providecommand{\ppcbasm}{
\toolsection{ppc64asm} is an assembler for the PowerPC hardware architecture.
It translates assembly code into machine code for PowerPC processors and stores it in corresponding object files.
By default, the assembler generates machine code for the 64-bit operating mode defined by the PowerPC architecture.
\flowgraph{\resource{PowerPC assembly\\source code} \ar[r] & \toolbox{ppc64asm} \ar[r] & \resource{object file}}
\seeassembly\seeppc\seeobject
}

\providecommand{\ppcbdism}{
\toolsection{ppc64dism} is a disassembler for the PowerPC hardware architecture.
It translates machine code from object files targeting PowerPC processors into assembly code and writes it to the standard output stream.
It assumes that the machine code was generated for the 64-bit operating mode defined by the PowerPC architecture.
\flowgraph{\resource{object file} \ar[r] & \toolbox{ppc64dism} \ar[r] & \resource{disassembly\\listing}}
\seeassembly\seeppc\seeobject
}

\providecommand{\riscasm}{
\toolsection{riscasm} is an assembler for the RISC hardware architecture.
It translates assembly code into machine code for RISC processors and stores it in corresponding object files.
The names of all special registers are predefined and evaluate to the corresponding number.
\flowgraph{\resource{RISC assembly\\source code} \ar[r] & \toolbox{riscasm} \ar[r] & \resource{object file}}
\seeassembly\seerisc\seeobject
}

\providecommand{\riscdism}{
\toolsection{riscdism} is a disassembler for the RISC hardware architecture.
It translates machine code from object files targeting RISC processors into assembly code and writes it to the standard output stream.
\flowgraph{\resource{object file} \ar[r] & \toolbox{riscdism} \ar[r] & \resource{disassembly\\listing}}
\seeassembly\seerisc\seeobject
}

\providecommand{\wasmasm}{
\toolsection{wasmasm} is an assembler for the WebAssembly architecture.
It translates assembly code into machine code for WebAssembly targets and stores it in corresponding object files.
The names of all special registers are predefined and evaluate to the corresponding number.
\flowgraph{\resource{WebAssembly assembly\\source code} \ar[r] & \toolbox{wasmasm} \ar[r] & \resource{object file}}
\seeassembly\seewasm\seeobject
}

\providecommand{\wasmdism}{
\toolsection{wasmdism} is a disassembler for the WebAssembly architecture.
It translates machine code from object files targeting WebAssembly targets into assembly code and writes it to the standard output stream.
\flowgraph{\resource{object file} \ar[r] & \toolbox{wasmdism} \ar[r] & \resource{disassembly\\listing}}
\seeassembly\seewasm\seeobject
}

% linker tools

\providecommand{\linklib}{
\toolsection{linklib} is an object file combiner.
It creates a static library file by combining all object files given to it into a single one.
\flowgraph{\resource{object files} \ar[r] & \toolbox{linklib} \ar[r] & \resource{library file}}
\seeobject
}

\providecommand{\linkbin}{
\toolsection{linkbin} is a linker for plain binary files.
It links all object files given to it into a single image and stores it in a binary file that begins with the first linked section.
It also creates a map file that lists the address, type, name and size of all used sections.
The filename extension of the resulting binary file can be specified by putting it into a constant data section called \texttt{\_extension}.
\flowgraph{\resource{object files} \ar[r] & \toolbox{linkbin} \ar[r] \ar[d] & \resource{binary file} \\ & \resource{map file}}
\seeobject
}

\providecommand{\linkmem}{
\toolsection{linkmem} is a linker for plain binary files partitioned into random-access and read-only memory.
It links all object files given to it into two distinct images, one for data sections and one for code and constant data sections, and stores each image in a binary file that begins with the first linked section of the corresponding type.
It also creates a map file that lists the address, type, name and size of all used sections.
\flowgraph{\resource{object files} \ar[r] & \toolbox{linkmem} \ar[r] \ar[d] & \resource{RAM file/\\ROM file} \\ & \resource{map file}}
\seeobject
}

\providecommand{\linkprg}{
\toolsection{linkprg} is a linker for GEMDOS executable files.
It links all object files given to it into a single image and stores the image in an Atari GEMDOS executable file~\cite{gemdosfile}.
It also creates a map file that lists the address relative to the text segment, type, name and size of all used sections.
The filename extension of the resulting executable file can be specified by putting it into a constant data section called \texttt{\_extension}.
The GEMDOS executable file format requires all patch patterns of absolute link patches to consist of four full bitmasks with descending offsets.
\flowgraph{\resource{object files} \ar[r] & \toolbox{linkprg} \ar[r] \ar[d] & \resource{executable file} \\ & \resource{map file}}
\seeobject
}

\providecommand{\linkhex}{
\toolsection{linkhex} is a linker for Intel HEX files.
It links all code sections of the object files given to it into single image and stores the image in an Intel HEX file~\cite{hexfile} that begins with the first linked section.
It also creates a map file that lists the address, type, name and size of all used sections.
\flowgraph{\resource{object files} \ar[r] & \toolbox{linkhex} \ar[r] \ar[d] & \resource{HEX file} \\ & \resource{map file}}
\seeobject
}

\providecommand{\mapsearch}{
\toolsection{mapsearch} is a debugging tool.
It searches map files generated by linker tools for the name of a binary section that encompasses a memory address read from the standard input stream.
If additionally provided with one or more object files, it also stores an excerpt thereof in a separate object file called map search result which only contains the identified binary section for disassembling purposes.
\flowgraph{& \resource{map files/\\object files} \ar[d] \\ \resource{memory\\address} \ar[r] & \toolbox{mapsearch} \ar[r] \ar[d] & \resource{section name/\\relative offset} \\ & \resource{object file\\excerpt}}
\seeobject
}


\startchapter{Questions and Answers}{Frequently Asked Questions}{faq}
{This \documentation{} answers frequently asked questions about the \ecs{} and how to use its tools to accomplish common tasks.}

\newcommand{\question}[1]{\subsection*{#1}}

\section{General Questions}

This section assembles questions frequently asked about the \ecs{} in general.

\question{What is the \ecs{}? Why does it exist?}

The \ecs{} is a completely self-contained collection of software development tools.
It exists to be recognized and adopted as a free development toolchain which is hopefully as useful and easy to use as its source code is intended to be approachable and comprehensible for developers and students wanting to learn, maintain, and customize a complete toolchain.

\question{What does the name \ecs{} mean?}

The \ecs{} provides all essential freedoms guaranteed by free software and is maintainable by a single person.
The term eigen in its name thus means that the \ecs{} is literally your own, regardless of whether you are a user or a developer.

\question{Does the \ecs{} support Unicode source files?}

The \ecs{} currently only supports the ASCII character encoding.
It will however eventually support Unicode source files encoded in UTF-8.

\section{How-Tos}

This section describes how to accomplish commonly requested tasks using the tools of the \ecs{} and therefore assumes some basic knowledge of its features and functionality.

\question{How to disassemble plain binary files?}

Since disassemblers can only process object files which are usually generated by other tools of the \ecs{}, a plain binary file must first be converted into an object file.
The simplest way to achieve this conversion is to assemble an assembly source file consisting of a single line of code which just copies the contents of the binary file using the embed binary file directive.

\concludechapter

% GNU General Public License
% Copyright (C) Florian Negele

% This file is part of the Eigen Compiler Suite.

% Permission is granted to copy, distribute and/or modify this document
% under the terms of the GNU Free Documentation License, Version 1.3
% or any later version published by the Free Software Foundation.

% You should have received a copy of the GNU Free Documentation License
% along with the ECS.  If not, see <https://www.gnu.org/licenses/>.

% Generic documentation utilities
% Copyright (C) Florian Negele

% This file is part of the Eigen Compiler Suite.

% Permission is granted to copy, distribute and/or modify this document
% under the terms of the GNU Free Documentation License, Version 1.3
% or any later version published by the Free Software Foundation.

% You should have received a copy of the GNU Free Documentation License
% along with the ECS.  If not, see <https://www.gnu.org/licenses/>.

\providecommand{\cpp}{C\texttt{++}}
\providecommand{\opt}{_\mathit{opt}}
\providecommand{\tool}[1]{\texttt{#1}}
\providecommand{\version}{Version 0.0.40}
\providecommand{\resource}[1]{*++\txt{#1}}
\providecommand{\ecs}{Eigen Compiler Suite}
\providecommand{\changed}[1]{\underline{#1}}
\providecommand{\toolbox}[1]{\converter{#1}}
\providecommand{\file}{}\renewcommand{\file}[1]{\texttt{#1}}
\providecommand{\alignright}{\hfill\linebreak[0]\hspace*{\fill}}
\providecommand{\converter}[1]{*++[F][F*:white][F,:gray]\txt{#1}}
\providecommand{\documentation}{\ifbook chapter\else document\fi}
\providecommand{\Documentation}{\ifbook Chapter\else Document\fi}
\providecommand{\variable}[1]{\resource{\texttt{\small#1}\\variable}}
\providecommand{\documentationref}[2]{\ifbook\ref{#1}\else``\href{#1}{#2}''~\cite{#1}\fi}
\providecommand{\objfile}[1]{\texttt{#1}\index[runtime]{#1 object file@\texttt{#1} object file}}
\providecommand{\libfile}[1]{\texttt{#1}\index[runtime]{#1 library file@\texttt{#1} library file}}
\providecommand{\epigraph}[2]{\ifbook\begin{quote}\flushright\textit{#1}\par--- #2\end{quote}\fi}
\providecommand{\environmentvariable}[1]{\texttt{#1}\index{Environment variables!#1@\texttt{#1}}}
\providecommand{\environment}[1]{\texttt{#1}\index[environment]{#1 environment@\texttt{#1} environment}}
\providecommand{\toolsection}{}\renewcommand{\toolsection}[1]{\subsection{#1}\label{\prefix:#1}\tool{#1}}
\providecommand{\instruction}{}\renewcommand{\instruction}[2]{\noindent\qquad\pdftooltip{\texttt{#1}}{#2}\refstepcounter{instruction}\par}
\providecommand{\flowgraph}{}\renewcommand{\flowgraph}[1]{\par\sffamily\begin{displaymath}\xymatrix@=4ex{#1}\end{displaymath}\normalfont\par}
\providecommand{\instructionset}{}\renewcommand{\instructionset}[4]{\setcounter{instruction}{0}\begin{multicols}{\ifbook#3\else#4\fi}[{\captionof{table}[#2]{#2 (\ref*{#1:instructions}~instructions)}\label{tab:#1set}\vspace{-2ex}}]\footnotesize\raggedcolumns\input{#1.set}\label{#1:instructions}\end{multicols}}

\providecommand{\gpl}{GNU General Public License}
\providecommand{\rse}{ECS Runtime Support Exception}
\providecommand{\fdl}{\href{https://www.gnu.org/licenses/fdl.html}{GNU Free Documentation License}}

\providecommand{\docbegin}{}
\providecommand{\docend}{}
\providecommand{\doclabel}[1]{\hypertarget{#1}}
\providecommand{\doclink}[2]{\hyperlink{#1}{#2}}
\providecommand{\docsection}[3]{\hypertarget{#1}{\subsection{#2}}\label{sec:#1}\index[library]{#2@#3}}
\providecommand{\docsectionstar}[1]{}
\providecommand{\docsubbegin}{\begin{description}}
\providecommand{\docsubend}{\end{description}}
\providecommand{\docsubsection}[3]{\item[\hypertarget{#1}{#2}]\index[library]{#2@#3}}
\providecommand{\docsubsectionstar}[1]{\smallskip}
\providecommand{\docsubsubsection}[3]{\docsubsection{#1}{#2}{#3}}
\providecommand{\docsubsubsectionstar}[1]{}
\providecommand{\docsubsubsubsection}[3]{}
\providecommand{\docsubsubsubsectionstar}[1]{}
\providecommand{\doctable}{}

\providecommand{\debuggingtool}{}\renewcommand{\debuggingtool}{This tool is provided for debugging purposes.
It allows exposing and modifying an internal data structure that is usually not accessible.
}

\providecommand{\interface}{All tools accept command-line arguments which are taken as names of plain text files containing the source code.
If no arguments are provided, the standard input stream is used instead.
Output files are generated in the current working directory and have the same name as the input file being processed whereas the filename extension gets replaced by an appropriate suffix.
\seeinterface
}

\providecommand{\license}{\noindent Copyright \copyright{} Florian Negele\par\medskip\noindent
Permission is granted to copy, distribute and/or modify this document under the terms of the
\fdl{}, Version 1.3 or any later version published by the \href{https://fsf.org/}{Free Software Foundation}.
}

\providecommand{\ecslogosurface}{
\fill[darkgray] (0,0,0) -- (0,0,3) -- (0,3,3) -- (0,3,1) -- (0,4,1) -- (0,4,3) -- (0,5,3) -- (0,5,0) -- (0,2,0) -- (0,2,2) -- (0,1,2) -- (0,1,0) -- cycle;
\fill[gray] (0,5,0) -- (0,5,3) -- (1,5,3) -- (1,5,1) -- (2,5,1) -- (2,5,3) -- (3,5,3) -- (3,5,0) -- cycle;
\fill[lightgray] (0,0,0) -- (0,1,0) -- (2,1,0) -- (2,4,0) -- (1,4,0) -- (1,3,0) -- (2,3,0) -- (2,2,0) -- (0,2,0) -- (0,5,0) -- (3,5,0) -- (3,0,0) -- cycle;
\begin{scope}[line width=0.5]
\begin{scope}[gray]
\draw (0,0,0) -- (0,1,0);
\draw (2,1,0) -- (2,2,0);
\draw (0,1,2) -- (0,2,2);
\draw (0,2,0) -- (0,5,0);
\draw (2,3,0) -- (2,4,0);
\end{scope}
\begin{scope}[lightgray]
\draw (0,1,0) -- (0,1,2);
\draw (0,3,1) -- (0,3,3);
\draw (0,5,0) -- (0,5,3);
\draw (2,5,1) -- (2,5,3);
\end{scope}
\begin{scope}[white]
\draw (0,1,0) -- (2,1,0);
\draw (1,3,0) -- (2,3,0);
\draw (0,5,0) -- (3,5,0);
\end{scope}
\end{scope}
}

\providecommand{\ecslogo}[1]{
\begin{tikzpicture}[scale={(#1)/((sin(45)+cos(45))*3cm)},x={({-cos(45)*1cm},{sin(45)*sin(30)*1cm})},y={({0cm},{(cos(30)*1cm})},z={({sin(45)*1cm},{cos(45)*sin(30)*1cm})}]
\begin{scope}[darkgray,line width=1]
\draw (0,0,0) -- (0,0,3) -- (0,3,3) -- (2,3,3) -- (2,5,3) -- (3,5,3) -- (3,5,0) -- (3,0,0) -- cycle;
\draw (0,3,1) -- (0,4,1) -- (0,4,3) -- (0,5,3) -- (1,5,3) -- (1,5,1) -- (2,5,1);
\draw (1,3,0) -- (1,4,0) -- (2,4,0);
\end{scope}
\fill[darkgray] (2,0,0) -- (2,0,3) -- (2,5,3) -- (2,5,1) -- (2,4,1) -- (2,4,0) -- cycle;
\fill[lightgray] (2,0,2) -- (0,0,2) -- (0,2,2) -- (2,2,2) -- cycle;
\fill[gray] (0,1,0) -- (2,1,0) -- (2,1,2) -- (0,1,2) -- cycle;
\fill[gray] (0,3,1) -- (0,3,3) -- (2,3,3) -- (2,3,0) -- (1,3,0) -- (1,3,1) -- cycle;
\ecslogosurface
\end{tikzpicture}
}

\providecommand{\shadowedecslogo}[3]{
\begin{tikzpicture}[scale={(#1)/((sin(#2)+cos(#2))*3cm)},x={({-cos(#2)*1cm},{sin(#2)*sin(#3)*1cm})},y={({0cm},{(cos(#3)*1cm})},z={({sin(#2)*1cm},{cos(#2)*sin(#3)*1cm})}]
\shade[top color=lightgray!50!white,bottom color=white,middle color=lightgray!50!white] (0,0,0) -- (3,0,0) -- (3,{-0.5-3*sin(#2)*sin(#3)/cos(#3)},0) -- (0,-0.5,0) -- cycle;
\shade[top color=darkgray!50!gray,bottom color=white,middle color=darkgray!50!white] (0,0,0) -- (0,0,3) -- (0,{-0.5-3*cos(#2)*sin(#3)/cos(#3)},3) -- (0,-0.5,0) -- cycle;
\begin{scope}[y={({(cos(#2)+sin(#2))*0.5cm},{(cos(#2)*sin(#3)-sin(#2)*sin(#3))*0.5cm})}]
\useasboundingbox (3,0,0) -- (0,0,0) -- (0,0,3);
\shade[left color=darkgray!80!black,right color=lightgray,middle color=gray] (0,0,0) -- (0,1,0) -- (0,1,0.5) -- (0,2,0) -- (0,5,0) -- (0,5,3) -- (1,5,3) -- (1,4,3) -- (1,4,2.5) -- (1,3,3) -- (2,5,3) -- (3,5,3) -- (3,0,3) -- cycle;
\clip (0,0,0) -- (0,0,3) -- ({-3*sin(#2)/cos(#2)},0,0) -- cycle;
\shade[left color=darkgray,right color=lightgray!50!gray] (0,0,0) -- (0,1,0) -- (0,1,0.5) -- (0,2,0) -- (0,5,0) -- (0,5,3) -- (1,5,3) -- (1,4,3) -- (1,4,2.5) -- (1,3,3) -- (2,5,3) -- (3,5,3) -- (3,0,3) -- cycle;
\end{scope}
\shade[left color=darkgray,right color=darkgray!80!black] (2,0,0) -- (2,0,3) -- (2,5,3) -- (2,5,1) -- (2,4,1) -- (2,4,0) -- cycle;
\shade[left color=darkgray!90!black,right color=gray!80!darkgray] (2,0,2) -- (0,0,2) -- (0,2,2) -- (2,2,2) -- cycle;
\shade[top color=darkgray!90!black,bottom color=gray!80!darkgray] (0,1,0) -- (2,1,0) -- (2,1,2) -- (0,1,2) -- cycle;
\shade[top color=darkgray!90!black,bottom color=gray!80!darkgray] (0,3,1) -- (0,3,3) -- (2,3,3) -- (2,3,0) -- (1,3,0) -- (1,3,1) -- cycle;
\fill[gray] (2,1,0) -- (1.5,1,0.5) -- (0,1,0.5) -- (0,1,0) -- cycle;
\fill[gray] (1,3,2) -- (0.5,3,2) -- (0.5,3,3) -- (1,3,3) -- cycle;
\fill[gray] (2,3,0) -- (1.5,3,0.5) -- (1,3,0.5) -- (1,3,0) -- cycle;
\ecslogosurface
\end{tikzpicture}
}

\providecommand{\cpplogo}[1]{
\begin{tikzpicture}[scale=(#1)/512em]
\fill[gray] (435.2794,398.7159) -- (247.1911,507.3075) .. controls (236.3563,513.5642) and (218.6240,513.5642) .. (207.7892,507.3075) -- (19.7009,398.7159) .. controls (8.8646,392.4606) and (0.0000,377.1043) .. (0.0000,364.5924) -- (0.0000,147.4076) .. controls (0.8430,132.8363) and (8.2856,120.7683) .. (19.7009,113.2842) -- (207.7892,4.6926) .. controls (218.6240,-1.5642) and (236.3564,-1.5642) .. (247.1911,4.6926) -- (435.2794,113.2842) .. controls (447.5273,121.4304) and (454.4987,133.6918) .. (454.9803,147.4076) -- (454.9803,364.5924) .. controls (454.5404,377.7571) and (446.6566,391.0351) .. (435.2794,398.7159) -- cycle(75.8301,255.9993) .. controls (74.9389,404.0881) and (273.2892,469.4783) .. (358.8263,331.8769) -- (293.1917,293.8965) .. controls (253.5702,359.4301) and (155.1909,335.9977) .. (151.6601,255.9993) .. controls (152.7204,182.2703) and (249.4137,148.0211) .. (293.1961,218.1065) -- (358.8308,180.1276) .. controls (283.4477,49.2645) and (79.6318,96.3470) .. (75.8301,255.9993) -- cycle(379.1503,247.5747) -- (362.2982,247.5747) -- (362.2982,230.7226) -- (345.4490,230.7226) -- (345.4490,247.5747) -- (328.5969,247.5747) -- (328.5969,264.4254) -- (345.4490,264.4254) -- (345.4490,281.2759) -- (362.2982,281.2759) -- (362.2982,264.4254) -- (379.1503,264.4254) -- cycle(442.3420,247.5747) -- (425.4899,247.5747) -- (425.4899,230.7226) -- (408.6408,230.7226) -- (408.6408,247.5747) -- (391.7886,247.5747) -- (391.7886,264.4254) -- (408.6408,264.4254) -- (408.6408,281.2759) -- (425.4899,281.2759) -- (425.4899,264.4254) -- (442.3420,264.4254) -- cycle;
\end{tikzpicture}
}

\providecommand{\fallogo}[1]{
\begin{tikzpicture}[scale=(#1)/512em]
\fill[gray] (185.7774,0.0000) .. controls (200.4486,15.9798) and (226.8966,8.7148) .. (235.0426,31.5836) .. controls (249.5297,58.0598) and (247.9581,97.9161) .. (280.3335,110.9762) .. controls (309.1690,120.3496) and (337.8406,104.2727) .. (366.5753,103.9379) .. controls (373.4449,111.5171) and (379.2885,128.2574) .. (383.9755,108.9744) .. controls (396.6979,102.5615) and (437.2808,107.6681) .. (426.9652,124.3252) .. controls (408.9822,121.0785) and (412.4742,146.0729) .. (426.5192,131.4996) .. controls (433.8413,120.8489) and (465.1541,126.5522) .. (441.9067,135.7950) .. controls (396.1879,157.7478) and (344.1112,161.5079) .. (298.5528,183.5702) .. controls (277.7471,193.5198) and (284.6941,218.7163) .. (285.2127,236.9640) .. controls (292.3599,316.2826) and (307.3929,394.6311) .. (317.1198,473.6154) .. controls (329.0637,505.4736) and (292.1195,528.5004) .. (265.9183,511.2761) .. controls (237.9284,499.2462) and (237.3684,465.2681) .. (230.9102,439.9421) .. controls (218.6692,374.3397) and (215.6307,306.9662) .. (198.1732,242.3977) .. controls (183.1379,232.7444) and (164.4245,256.0298) .. (149.0430,261.4799) .. controls (116.9328,279.2585) and (87.1822,308.5851) .. (48.2293,307.8914) .. controls (21.3220,306.9037) and (-15.9107,281.8761) .. (7.2921,252.7908) .. controls (29.7799,220.6177) and (67.5177,204.3028) .. (100.9287,185.9449) .. controls (130.8217,170.8906) and (161.1548,156.5903) .. (191.0278,141.5847) .. controls (196.1738,120.0520) and (186.6049,95.2409) .. (186.8382,72.4353) .. controls (185.5234,48.4204) and (183.1700,23.9341) .. (185.7774,0.0000) -- cycle;
\end{tikzpicture}
}

\providecommand{\oblogo}[1]{
\begin{tikzpicture}[scale=(#1)/512em]
\fill[gray] (160.3865,208.9117) .. controls (154.0879,214.6478) and (149.0735,221.2409) .. (145.4125,228.5384) .. controls (184.8790,248.4273) and (234.7122,269.8787) .. (297.5493,291.8782) .. controls (300.3943,281.4769) and (300.9552,268.7619) .. (300.4023,255.2389) .. controls (248.9909,244.7891) and (200.0310,225.9279) .. (160.3865,208.9117) -- cycle(225.7398,392.6996) .. controls (308.0209,392.1716) and (359.3326,345.9277) .. (368.7203,285.2098) .. controls (376.6742,197.1784) and (311.7194,141.3342) .. (205.4287,142.1456) .. controls (139.9485,141.4804) and (88.7155,166.1957) .. (73.5775,228.0086) .. controls (52.0297,320.3408) and (123.4078,391.0103) .. (225.7398,392.6996) -- cycle(216.0739,176.4733) .. controls (268.9183,179.2424) and (315.8292,206.5488) .. (312.7454,265.1139) .. controls (313.2769,315.6384) and (286.5993,353.4946) .. (216.6040,355.7934) .. controls (162.4657,355.7934) and (126.0914,317.5023) .. (126.0914,260.5103) .. controls (126.1733,214.2900) and (163.3363,176.2849) .. (216.0739,176.4733) -- cycle(76.4897,189.1754) .. controls (13.1586,147.5631) and (0.0000,119.4207) .. (0.0000,119.4207) -- (90.6499,170.1632) .. controls (85.3004,175.8497) and (80.5994,182.1633) .. (76.4897,189.1754) -- cycle(353.9486,119.3004) -- (402.9482,119.3004) .. controls (427.0025,137.0797) and (450.9893,162.7034) .. (474.9529,191.0213) .. controls (509.3540,228.5339) and (531.3391,294.2091) .. (487.8149,312.1206) .. controls (462.8165,324.7652) and (394.3874,316.8943) .. (373.8912,313.6651) .. controls (379.9291,297.7449) and (383.2899,278.4204) .. (381.4989,257.7214) .. controls (420.3069,248.0321) and (421.9610,218.3461) .. (407.7867,192.6417) .. controls (391.1113,162.4018) and (370.1114,132.9097) .. (353.9486,119.3004) -- cycle;
\end{tikzpicture}
}

\providecommand{\markuptable}{
\begin{table}
\sffamily\centering
\begin{tabular}{@{}lcl@{}}
\toprule
\texttt{//italics//} & $\rightarrow$ & \textit{italics} \\
\midrule
\texttt{**bold**} & $\rightarrow$ & \textbf{bold} \\
\midrule
\texttt{\# ordered list} & & 1 ordered list \\
\texttt{\# second item} & $\rightarrow$ & 2 second item \\
\texttt{\#\# sub item} & & \hspace{1em} 1 sub item \\
\midrule
\texttt{* unordered list} & & $\bullet$ unordered list \\
\texttt{* second item} & $\rightarrow$ & $\bullet$ second item \\
\texttt{** sub item} & & \hspace{1em} $\bullet$ sub item \\
\midrule
\texttt{link to [[label]]} & $\rightarrow$ & link to \underline{label} \\
\midrule
\texttt{<{}<label>{}> definition } & $\rightarrow$ & definition \\
\midrule
\texttt{[[url|link name]]} & $\rightarrow$ & \underline{link name} \\
\midrule\addlinespace
\texttt{= large heading} & & {\Large large heading} \smallskip \\
\texttt{== medium heading} & $\rightarrow$ & {\large medium heading} \\
\texttt{=== small heading} & & small heading \\
\midrule
\texttt{no line break} & & no line break for paragraphs \\
\texttt{for paragraphs} & $\rightarrow$ \\
& & use empty line \\
\texttt{use empty line} \\
\midrule
\texttt{force\textbackslash\textbackslash line break} & $\rightarrow$ & force \\
& & line break \\
\midrule
\texttt{horizontal line} & $\rightarrow$ & horizontal line \\
\texttt{----} & & \hrulefill \\
\midrule
\texttt{|=a|=table|=header} & & \underline{a \enspace table \enspace header} \\
\texttt{|a|table|row} & $\rightarrow$ & a \enspace table \enspace row \\
\texttt{|b|table|row} & & b \enspace table \enspace row \\
\midrule
\texttt{\{\{\{} \\
\texttt{unformatted} & $\rightarrow$ & \texttt{unformatted} \\
\texttt{code} & & \texttt{code} \\
\texttt{\}\}\}} \\
\midrule\addlinespace
\texttt{@ new article} & & {\Large 1.\ new article} \smallskip \\
\texttt{@ second article} & $\rightarrow$ & {\Large 2.\ second article} \smallskip \\
\texttt{@@ sub article} & & {\large 2.1.\ sub article} \\
\bottomrule
\end{tabular}
\normalfont\caption{Elements of the generic documentation markup language}
\label{tab:docmarkup}
\end{table}
}

\providecommand{\startchapter}[4]{
\documentclass[11pt,a4paper]{article}
\usepackage{booktabs}
\usepackage[format=hang,labelfont=bf]{caption}
\usepackage{changepage}
\usepackage[T1]{fontenc}
\usepackage[margin=2cm]{geometry}
\usepackage{hyperref}
\usepackage[american]{isodate}
\usepackage{lmodern}
\usepackage{longtable}
\usepackage{mathptmx}
\usepackage{microtype}
\usepackage[toc]{multitoc}
\usepackage{multirow}
\usepackage[all]{nowidow}
\usepackage{pdfcomment}
\usepackage{syntax}
\usepackage{tikz}
\usepackage[all]{xy}
\hypersetup{pdfborder={0 0 0},bookmarksnumbered=true,pdftitle={\ecs{}: #2},pdfauthor={Florian Negele},pdfsubject={\ecs{}},pdfkeywords={#1}}
\setlength{\grammarindent}{8em}\setlength{\grammarparsep}{0.2ex}
\setlength{\columnsep}{2em}
\newcommand{\prefix}{}
\newcounter{instruction}
\bibliographystyle{unsrt}
\renewcommand{\index}[2][]{}
\renewcommand{\arraystretch}{1.05}
\renewcommand{\floatpagefraction}{0.7}
\renewcommand{\syntleft}{\itshape}\renewcommand{\syntright}{}
\title{\vspace{-5ex}\Huge{\ecs{}}\medskip\hrule}
\author{\huge{#2}}
\date{\medskip\version}
\newif\ifbook\bookfalse
\pagestyle{headings}
\frenchspacing
\begin{document}
\maketitle\thispagestyle{empty}\noindent#4\setlength{\columnseprule}{0.4pt}\tableofcontents\setlength{\columnseprule}{0pt}\vfill\pagebreak[3]\null\vfill\bigskip\noindent
\parbox{\textwidth-4em}{\license The contents of this \documentation{} are part of the \href{manual}{\ecs{} User Manual}~\cite{manual} and correspond to Chapter ``\href{manual\##3}{#1}''.\alignright\mbox{\today}}
\parbox{4em}{\flushright\ecslogo{3em}}
\clearpage
}

\providecommand{\concludechapter}{
\vfill\pagebreak[3]\null\vfill
\thispagestyle{myheadings}\markright{REFERENCES}
\noindent\begin{minipage}{\textwidth}\begin{multicols}{2}[\section*{References}]
\renewcommand{\section}[2]{}\small\bibliography{references}
\end{multicols}\end{minipage}\end{document}
}

\providecommand{\startpresentation}[2]{
\documentclass[14pt,aspectratio=43,usepdftitle=false]{beamer}
\usepackage{booktabs}
\usepackage{etex}
\usepackage{multicol}
\usepackage{tikz}
\usepackage[all]{xy}
\bibliographystyle{unsrt}
\setlength{\columnsep}{1em}
\setlength{\leftmargini}{1em}
\setbeamercolor{title}{fg=black}
\setbeamercolor{structure}{fg=darkgray}
\setbeamercolor{bibliography item}{fg=darkgray}
\setbeamerfont{title}{series=\bfseries}
\setbeamerfont{subtitle}{series=\normalfont}
\setbeamerfont*{frametitle}{parent=title}
\setbeamerfont{block title}{series=\bfseries}
\setbeamerfont*{framesubtitle}{parent=subtitle}
\setbeamersize{text margin left=1em,text margin right=1em}
\setbeamertemplate{navigation symbols}{}
\setbeamertemplate{itemize item}[circle]{}
\setbeamertemplate{bibliography item}[triangle]{}
\setbeamertemplate{bibliography entry author}{\usebeamercolor[fg]{bibliography item}}
\setbeamertemplate{frametitle}{\medskip\usebeamerfont{frametitle}\color{gray}\raisebox{-2.5ex}[0ex][0ex]{\rule{0.1em}{4.5ex}}}
\addtobeamertemplate{frametitle}{}{\hspace{0.4em}\usebeamercolor[fg]{title}\insertframetitle\par\vspace{0.2ex}\hspace{0.5em}\usebeamerfont{framesubtitle}\insertframesubtitle}
\hypersetup{pdfborder={0 0 0},bookmarksnumbered=true,bookmarksopen=true,bookmarksopenlevel=0,pdftitle={\ecs{}: #1},pdfauthor={Florian Negele},pdfsubject={\ecs{}},pdfkeywords={#1}}
\renewcommand{\flowgraph}[1]{\resizebox{\textwidth}{!}{$$\xymatrix{##1}$$}}
\title{\ecs{}\medskip\hrule\medskip}
\institute{\shadowedecslogo{5em}{30}{15}}
\date{\version}
\subtitle{#1}
\begin{document}
\begin{frame}[plain]\titlepage\nocite{manual}\end{frame}
\begin{frame}{Contents}{#1}\begin{center}\tableofcontents\end{center}\end{frame}
}

\providecommand{\concludepresentation}{
\begin{frame}{References}\begin{footnotesize}\setlength{\columnseprule}{0.4pt}\begin{multicols}{2}\bibliography{references}\end{multicols}\end{footnotesize}\end{frame}
\end{document}
}

\providecommand{\startbook}[1]{
\documentclass[10pt,paper=17cm:24cm,DIV=13,twoside=semi,headings=normal,numbers=noendperiod,cleardoublepage=plain]{scrbook}
\usepackage{atveryend}
\usepackage{booktabs}
\usepackage{caption}
\usepackage{changepage}
\usepackage[T1]{fontenc}
\usepackage{imakeidx}
\usepackage{hyperref}
\usepackage[american]{isodate}
\usepackage{lmodern}
\usepackage{longtable}
\usepackage{mathptmx}
\usepackage[final]{microtype}
\usepackage{multicol}
\usepackage{multirow}
\usepackage[all]{nowidow}
\usepackage{pdfcomment}
\usepackage{scrlayer-scrpage}
\usepackage{setspace}
\usepackage{syntax}
\usepackage[eventxtindent=4pt,oddtxtexdent=4pt]{thumbs}
\usepackage{tikz}
\usepackage[all]{xy}
\hyphenation{Micro-Blaze Open-Cores Open-RISC Power-PC}
\hypersetup{pdfborder={0 0 0},bookmarksnumbered=true,bookmarksopen=true,bookmarksopenlevel=0,pdftitle={\ecs{}: #1},pdfauthor={Florian Negele},pdfsubject={\ecs{}},pdfkeywords={#1}}
\setlength{\grammarindent}{8em}\setlength{\grammarparsep}{0.7ex}
\setkomafont{captionlabel}{\usekomafont{descriptionlabel}}
\renewcommand{\arraystretch}{1.05}\setstretch{1.1}
\renewcommand{\chapterformat}{\thechapter\autodot\enskip\raisebox{-1ex}[0ex][0ex]{\color{gray}\rule{0.1em}{3.5ex}}\enskip}
\renewcommand{\startchapter}[4]{\hypertarget{##3}{\chapter{##1}}\label{##3}##4\addthumb{##1}{\LARGE\sffamily\bfseries\thechapter}{white}{gray}\renewcommand{\prefix}{##3}}
\renewcommand{\concludechapter}{\clearpage{\stopthumb\cleardoublepage}}
\renewcommand{\syntleft}{\itshape}\renewcommand{\syntright}{}
\renewcommand{\floatpagefraction}{0.7}
\renewcommand{\partheademptypage}{}
\DeclareMicrotypeAlias{lmss}{cmr}
\newcommand{\prefix}{}
\newcounter{instruction}
\bibliographystyle{unsrt}
\newif\ifbook\booktrue
\makeindex[intoc,title=Index]
\makeindex[intoc,name=tools,title=Index of Tools,columns=3]
\makeindex[intoc,name=library,title=Index of Library Names]
\makeindex[intoc,name=runtime,title=Index of Runtime Support]
\makeindex[intoc,name=environment,title=Index of Target Environments]
\indexsetup{toclevel=chapter,headers={\indexname}{\indexname}}
\frenchspacing
\begin{document}
\pagenumbering{alph}
\begin{titlepage}\centering
\huge\sffamily\null\vfill\textbf{\ecs{}}\bigskip\hrule\bigskip#1
\normalsize\normalfont\vfill\vfill\shadowedecslogo{10em}{30}{15}
\large\vfill\vfill\version
\end{titlepage}
\null\vfill
\thispagestyle{empty}
\noindent\today\par\medskip
\license A copy of this license is included in Appendix~\ref{fdl} on page~\pageref{fdl}.
All product names used herein are for identification purposes only and may be trademarks of their respective companies.
\concludechapter
\frontmatter
\setcounter{tocdepth}{1}
\tableofcontents
\setcounter{tocdepth}{2}
\concludechapter
\listoffigures
\concludechapter
\listoftables
\concludechapter
}

\providecommand{\concludebook}{
\backmatter
\addtocontents{toc}{\protect\setcounter{tocdepth}{-1}}
\phantomsection\addcontentsline{toc}{part}{Bibliography}
\bibliography{references}
\concludechapter
\phantomsection\addcontentsline{toc}{part}{Indexes}
\printindex
\concludechapter
\indexprologue{\label{idx:tools}}
\printindex[tools]
\concludechapter
\printindex[library]
\concludechapter
\indexprologue{\label{idx:runtime}}
\printindex[runtime]
\concludechapter
\indexprologue{\label{idx:environment}}
\printindex[environment]
\concludechapter
\pagestyle{empty}\pagenumbering{Alph}\null\clearpage
\null\vfill\centering\ecslogo{4em}\par\medskip\license
\end{document}
}

% chapter references

\providecommand{\seedocumentationref}{}\renewcommand{\seedocumentationref}[3]{#1, see \Documentation{}~\documentationref{#2}{#3}. }
\providecommand{\seeinterface}{}\renewcommand{\seeinterface}{\ifbook See \Documentation{}~\documentationref{interface}{User Interface} for more information about the common user interface of all of these tools. \fi}
\providecommand{\seeguide}{}\renewcommand{\seeguide}{\seedocumentationref{For basic examples of using some of these tools in practice}{guide}{User Guide}}
\providecommand{\seecpp}{}\renewcommand{\seecpp}{\seedocumentationref{For more information about the \cpp{} programming language and its implementation by the \ecs{}}{cpp}{User Manual for \cpp{}}}
\providecommand{\seefalse}{}\renewcommand{\seefalse}{\seedocumentationref{For more information about the FALSE programming language and its implementation by the \ecs{}}{false}{User Manual for FALSE}}
\providecommand{\seeoberon}{}\renewcommand{\seeoberon}{\seedocumentationref{For more information about the Oberon programming language and its implementation by the \ecs{}}{oberon}{User Manual for Oberon}}
\providecommand{\seeassembly}{}\renewcommand{\seeassembly}{\seedocumentationref{For more information about the generic assembly language and how to use it}{assembly}{Generic Assembly Language Specification}}
\providecommand{\seeamd}{}\renewcommand{\seeamd}{\seedocumentationref{For more information about how the \ecs{} supports the AMD64 hardware architecture}{amd64}{AMD64 Hardware Architecture Support}}
\providecommand{\seearm}{}\renewcommand{\seearm}{\seedocumentationref{For more information about how the \ecs{} supports the ARM hardware architecture}{arm}{ARM Hardware Architecture Support}}
\providecommand{\seeavr}{}\renewcommand{\seeavr}{\seedocumentationref{For more information about how the \ecs{} supports the AVR hardware architecture}{avr}{AVR Hardware Architecture Support}}
\providecommand{\seeavrtt}{}\renewcommand{\seeavrtt}{\seedocumentationref{For more information about how the \ecs{} supports the AVR32 hardware architecture}{avr32}{AVR32 Hardware Architecture Support}}
\providecommand{\seemabk}{}\renewcommand{\seemabk}{\seedocumentationref{For more information about how the \ecs{} supports the M68000 hardware architecture}{m68k}{M68000 Hardware Architecture Support}}
\providecommand{\seemibl}{}\renewcommand{\seemibl}{\seedocumentationref{For more information about how the \ecs{} supports the MicroBlaze hardware architecture}{mibl}{MicroBlaze Hardware Architecture Support}}
\providecommand{\seemips}{}\renewcommand{\seemips}{\seedocumentationref{For more information about how the \ecs{} supports the MIPS32 and MIPS64 hardware architectures}{mips}{MIPS Hardware Architecture Support}}
\providecommand{\seemmix}{}\renewcommand{\seemmix}{\seedocumentationref{For more information about how the \ecs{} supports the MMIX hardware architecture}{mmix}{MMIX Hardware Architecture Support}}
\providecommand{\seeorok}{}\renewcommand{\seeorok}{\seedocumentationref{For more information about how the \ecs{} supports the OpenRISC 1000 hardware architecture}{or1k}{OpenRISC 1000 Hardware Architecture Support}}
\providecommand{\seeppc}{}\renewcommand{\seeppc}{\seedocumentationref{For more information about how the \ecs{} supports the PowerPC hardware architecture}{ppc}{PowerPC Hardware Architecture Support}}
\providecommand{\seerisc}{}\renewcommand{\seerisc}{\seedocumentationref{For more information about how the \ecs{} supports the RISC hardware architecture}{risc}{RISC Hardware Architecture Support}}
\providecommand{\seewasm}{}\renewcommand{\seewasm}{\seedocumentationref{For more information about how the \ecs{} supports the WebAssembly architecture}{wasm}{WebAssembly Architecture Support}}
\providecommand{\seedocumentation}{}\renewcommand{\seedocumentation}{\seedocumentationref{For more information about generic documentations and their generation by the \ecs{}}{documentation}{Generic Documentation Generation}}
\providecommand{\seedebugging}{}\renewcommand{\seedebugging}{\seedocumentationref{For more information about debugging information and its representation}{debugging}{Debugging Information Representation}}
\providecommand{\seecode}{}\renewcommand{\seecode}{\seedocumentationref{For more information about intermediate code and its purpose}{code}{Intermediate Code Representation}}
\providecommand{\seeobject}{}\renewcommand{\seeobject}{\seedocumentationref{For more information about object files and their purpose}{object}{Object File Representation}}

% generic documentation tools

\providecommand{\docprint}{
\toolsection{docprint} is a pretty printer for generic documentations.
It reformats generic documentations and writes it to the standard output stream.
\debuggingtool
\flowgraph{\resource{generic\\documentation} \ar[r] & \toolbox{docprint} \ar[r] & \resource{generic\\documentation}}
\seedocumentation
}

\providecommand{\doccheck}{
\toolsection{doccheck} is a syntactic and semantic checker for generic documentations.
It just performs syntactic and semantic checks on generic documentations and writes its diagnostic messages to the standard error stream.
\debuggingtool
\flowgraph{\resource{generic\\documentation} \ar[r] & \toolbox{doccheck} \ar[r] & \resource{diagnostic\\messages}}
\seedocumentation
}

\providecommand{\dochtml}{
\toolsection{dochtml} is an HTML documentation generator for generic documentations.
It processes several generic documentations and assembles all information therein into an HTML document.
\debuggingtool
\flowgraph{\resource{generic\\documentation} \ar[r] & \toolbox{dochtml} \ar[r] & \resource{HTML\\document}}
\seedocumentation
}

\providecommand{\doclatex}{
\toolsection{doclatex} is a Latex documentation generator for generic documentations.
It processes several generic documentations and assembles all information therein into a Latex document.
\debuggingtool
\flowgraph{\resource{generic\\documentation} \ar[r] & \toolbox{doclatex} \ar[r] & \resource{Latex\\document}}
\seedocumentation
}

% intermediate code tools

\providecommand{\cdcheck}{
\toolsection{cdcheck} is a syntactic and semantic checker for intermediate code.
It just performs syntactic and semantic checks on programs written in intermediate code and writes its diagnostic messages to the standard error stream.
\debuggingtool
\flowgraph{\resource{intermediate\\code} \ar[r] & \toolbox{cdcheck} \ar[r] & \resource{diagnostic\\messages}}
\seeassembly\seecode
}

\providecommand{\cdopt}{
\toolsection{cdopt} is an optimizer for intermediate code.
It performs various optimizations on programs written in intermediate code and writes the result to the standard output stream.
\debuggingtool
\flowgraph{\resource{intermediate\\code} \ar[r] & \toolbox{cdopt} \ar[r] & \resource{optimized\\code}}
\seeassembly\seecode
}

\providecommand{\cdrun}{
\toolsection{cdrun} is an interpreter for intermediate code.
It processes and executes programs written in intermediate code.
The following code sections are predefined and have the usual semantics:
\texttt{abort}, \texttt{\_Exit}, \texttt{fflush}, \texttt{floor}, \texttt{fputc}, \texttt{free}, \texttt{getchar}, \texttt{malloc}, and \texttt{putchar}.
Diagnostic messages about invalid operations include the name of the executed code section and the index of the erroneous instruction.
\debuggingtool
\flowgraph{\resource{intermediate\\code} \ar[r] & \toolbox{cdrun} \ar@/u/[r] & \resource{input/\\output} \ar@/d/[l]}
\seeassembly\seecode
}

\providecommand{\cdamda}{
\toolsection{cdamd16} is a compiler for intermediate code targeting the AMD64 hardware architecture.
It generates machine code for AMD64 processors from programs written in intermediate code and stores it in corresponding object files.
The compiler generates machine code for the 16-bit operating mode defined by the AMD64 architecture.
It also creates a debugging information file as well as an assembly file containing a listing of the generated machine code.
\debuggingtool
\flowgraph{\resource{intermediate\\code} \ar[r] & \toolbox{cdamd16} \ar[r] \ar[d] \ar[rd] & \resource{object file} \\ & \resource{assembly\\listing} & \resource{debugging\\information}}
\seeassembly\seeamd\seeobject\seecode\seedebugging
}

\providecommand{\cdamdb}{
\toolsection{cdamd32} is a compiler for intermediate code targeting the AMD64 hardware architecture.
It generates machine code for AMD64 processors from programs written in intermediate code and stores it in corresponding object files.
The compiler generates machine code for the 32-bit operating mode defined by the AMD64 architecture.
It also creates a debugging information file as well as an assembly file containing a listing of the generated machine code.
\debuggingtool
\flowgraph{\resource{intermediate\\code} \ar[r] & \toolbox{cdamd32} \ar[r] \ar[d] \ar[rd] & \resource{object file} \\ & \resource{assembly\\listing} & \resource{debugging\\information}}
\seeassembly\seeamd\seeobject\seecode\seedebugging
}

\providecommand{\cdamdc}{
\toolsection{cdamd64} is a compiler for intermediate code targeting the AMD64 hardware architecture.
It generates machine code for AMD64 processors from programs written in intermediate code and stores it in corresponding object files.
The compiler generates machine code for the 64-bit operating mode defined by the AMD64 architecture.
It also creates a debugging information file as well as an assembly file containing a listing of the generated machine code.
\debuggingtool
\flowgraph{\resource{intermediate\\code} \ar[r] & \toolbox{cdamd64} \ar[r] \ar[d] \ar[rd] & \resource{object file} \\ & \resource{assembly\\listing} & \resource{debugging\\information}}
\seeassembly\seeamd\seeobject\seecode\seedebugging
}

\providecommand{\cdarma}{
\toolsection{cdarma32} is a compiler for intermediate code targeting the ARM hardware architecture.
It generates machine code for ARM processors executing A32 instructions from programs written in intermediate code and stores it in corresponding object files.
It also creates a debugging information file as well as an assembly file containing a listing of the generated machine code.
\debuggingtool
\flowgraph{\resource{intermediate\\code} \ar[r] & \toolbox{cdarma32} \ar[r] \ar[d] \ar[rd] & \resource{object file} \\ & \resource{assembly\\listing} & \resource{debugging\\information}}
\seeassembly\seearm\seeobject\seecode\seedebugging
}

\providecommand{\cdarmb}{
\toolsection{cdarma64} is a compiler for intermediate code targeting the ARM hardware architecture.
It generates machine code for ARM processors executing A64 instructions from programs written in intermediate code and stores it in corresponding object files.
It also creates a debugging information file as well as an assembly file containing a listing of the generated machine code.
\debuggingtool
\flowgraph{\resource{intermediate\\code} \ar[r] & \toolbox{cdarma64} \ar[r] \ar[d] \ar[rd] & \resource{object file} \\ & \resource{assembly\\listing} & \resource{debugging\\information}}
\seeassembly\seearm\seeobject\seecode\seedebugging
}

\providecommand{\cdarmc}{
\toolsection{cdarmt32} is a compiler for intermediate code targeting the ARM hardware architecture.
It generates machine code for ARM processors without floating-point extension executing T32 instructions from programs written in intermediate code and stores it in corresponding object files.
It also creates a debugging information file as well as an assembly file containing a listing of the generated machine code.
\debuggingtool
\flowgraph{\resource{intermediate\\code} \ar[r] & \toolbox{cdarmt32} \ar[r] \ar[d] \ar[rd] & \resource{object file} \\ & \resource{assembly\\listing} & \resource{debugging\\information}}
\seeassembly\seearm\seeobject\seecode\seedebugging
}

\providecommand{\cdarmcfpe}{
\toolsection{cdarmt32fpe} is a compiler for intermediate code targeting the ARM hardware architecture.
It generates machine code for ARM processors with floating-point extension executing T32 instructions from programs written in intermediate code and stores it in corresponding object files.
It also creates a debugging information file as well as an assembly file containing a listing of the generated machine code.
\debuggingtool
\flowgraph{\resource{intermediate\\code} \ar[r] & \toolbox{cdarmt32fpe} \ar[r] \ar[d] \ar[rd] & \resource{object file} \\ & \resource{assembly\\listing} & \resource{debugging\\information}}
\seeassembly\seearm\seeobject\seecode\seedebugging
}

\providecommand{\cdavr}{
\toolsection{cdavr} is a compiler for intermediate code targeting the AVR hardware architecture.
It generates machine code for AVR processors from programs written in intermediate code and stores it in corresponding object files.
It also creates a debugging information file as well as an assembly file containing a listing of the generated machine code.
\debuggingtool
\flowgraph{\resource{intermediate\\code} \ar[r] & \toolbox{cdavr} \ar[r] \ar[d] \ar[rd] & \resource{object file} \\ & \resource{assembly\\listing} & \resource{debugging\\information}}
\seeassembly\seeavr\seeobject\seecode\seedebugging
}

\providecommand{\cdavrtt}{
\toolsection{cdavr32} is a compiler for intermediate code targeting the AVR32 hardware architecture.
It generates machine code for AVR32 processors from programs written in intermediate code and stores it in corresponding object files.
It also creates a debugging information file as well as an assembly file containing a listing of the generated machine code.
\debuggingtool
\flowgraph{\resource{intermediate\\code} \ar[r] & \toolbox{cdavr32} \ar[r] \ar[d] \ar[rd] & \resource{object file} \\ & \resource{assembly\\listing} & \resource{debugging\\information}}
\seeassembly\seeavrtt\seeobject\seecode\seedebugging
}

\providecommand{\cdmabk}{
\toolsection{cdm68k} is a compiler for intermediate code targeting the M68000 hardware architecture.
It generates machine code for M68000 processors from programs written in intermediate code and stores it in corresponding object files.
It also creates a debugging information file as well as an assembly file containing a listing of the generated machine code.
\debuggingtool
\flowgraph{\resource{intermediate\\code} \ar[r] & \toolbox{cdm68k} \ar[r] \ar[d] \ar[rd] & \resource{object file} \\ & \resource{assembly\\listing} & \resource{debugging\\information}}
\seeassembly\seemabk\seeobject\seecode\seedebugging
}

\providecommand{\cdmibl}{
\toolsection{cdmibl} is a compiler for intermediate code targeting the MicroBlaze hardware architecture.
It generates machine code for MicroBlaze processors from programs written in intermediate code and stores it in corresponding object files.
It also creates a debugging information file as well as an assembly file containing a listing of the generated machine code.
\debuggingtool
\flowgraph{\resource{intermediate\\code} \ar[r] & \toolbox{cdmibl} \ar[r] \ar[d] \ar[rd] & \resource{object file} \\ & \resource{assembly\\listing} & \resource{debugging\\information}}
\seeassembly\seemibl\seeobject\seecode\seedebugging
}

\providecommand{\cdmipsa}{
\toolsection{cdmips32} is a compiler for intermediate code targeting the MIPS32 hardware architecture.
It generates machine code for MIPS32 processors from programs written in intermediate code and stores it in corresponding object files.
It also creates a debugging information file as well as an assembly file containing a listing of the generated machine code.
\debuggingtool
\flowgraph{\resource{intermediate\\code} \ar[r] & \toolbox{cdmips32} \ar[r] \ar[d] \ar[rd] & \resource{object file} \\ & \resource{assembly\\listing} & \resource{debugging\\information}}
\seeassembly\seemips\seeobject\seecode\seedebugging
}

\providecommand{\cdmipsb}{
\toolsection{cdmips64} is a compiler for intermediate code targeting the MIPS64 hardware architecture.
It generates machine code for MIPS64 processors from programs written in intermediate code and stores it in corresponding object files.
It also creates a debugging information file as well as an assembly file containing a listing of the generated machine code.
\debuggingtool
\flowgraph{\resource{intermediate\\code} \ar[r] & \toolbox{cdmips64} \ar[r] \ar[d] \ar[rd] & \resource{object file} \\ & \resource{assembly\\listing} & \resource{debugging\\information}}
\seeassembly\seemips\seeobject\seecode\seedebugging
}

\providecommand{\cdmmix}{
\toolsection{cdmmix} is a compiler for intermediate code targeting the MMIX hardware architecture.
It generates machine code for MMIX processors from programs written in intermediate code and stores it in corresponding object files.
It also creates a debugging information file as well as an assembly file containing a listing of the generated machine code.
\debuggingtool
\flowgraph{\resource{intermediate\\code} \ar[r] & \toolbox{cdmmix} \ar[r] \ar[d] \ar[rd] & \resource{object file} \\ & \resource{assembly\\listing} & \resource{debugging\\information}}
\seeassembly\seemmix\seeobject\seecode\seedebugging
}

\providecommand{\cdorok}{
\toolsection{cdor1k} is a compiler for intermediate code targeting the OpenRISC 1000 hardware architecture.
It generates machine code for OpenRISC 1000 processors from programs written in intermediate code and stores it in corresponding object files.
It also creates a debugging information file as well as an assembly file containing a listing of the generated machine code.
\debuggingtool
\flowgraph{\resource{intermediate\\code} \ar[r] & \toolbox{cdor1k} \ar[r] \ar[d] \ar[rd] & \resource{object file} \\ & \resource{assembly\\listing} & \resource{debugging\\information}}
\seeassembly\seeorok\seeobject\seecode\seedebugging
}

\providecommand{\cdppca}{
\toolsection{cdppc32} is a compiler for intermediate code targeting the PowerPC hardware architecture.
It generates machine code for PowerPC processors from programs written in intermediate code and stores it in corresponding object files.
The compiler generates machine code for the 32-bit operating mode defined by the PowerPC architecture.
It also creates a debugging information file as well as an assembly file containing a listing of the generated machine code.
\debuggingtool
\flowgraph{\resource{intermediate\\code} \ar[r] & \toolbox{cdppc32} \ar[r] \ar[d] \ar[rd] & \resource{object file} \\ & \resource{assembly\\listing} & \resource{debugging\\information}}
\seeassembly\seeppc\seeobject\seecode\seedebugging
}

\providecommand{\cdppcb}{
\toolsection{cdppc64} is a compiler for intermediate code targeting the PowerPC hardware architecture.
It generates machine code for PowerPC processors from programs written in intermediate code and stores it in corresponding object files.
The compiler generates machine code for the 64-bit operating mode defined by the PowerPC architecture.
It also creates a debugging information file as well as an assembly file containing a listing of the generated machine code.
\debuggingtool
\flowgraph{\resource{intermediate\\code} \ar[r] & \toolbox{cdppc64} \ar[r] \ar[d] \ar[rd] & \resource{object file} \\ & \resource{assembly\\listing} & \resource{debugging\\information}}
\seeassembly\seeppc\seeobject\seecode\seedebugging
}

\providecommand{\cdrisc}{
\toolsection{cdrisc} is a compiler for intermediate code targeting the RISC hardware architecture.
It generates machine code for RISC processors from programs written in intermediate code and stores it in corresponding object files.
It also creates a debugging information file as well as an assembly file containing a listing of the generated machine code.
\debuggingtool
\flowgraph{\resource{intermediate\\code} \ar[r] & \toolbox{cdrisc} \ar[r] \ar[d] \ar[rd] & \resource{object file} \\ & \resource{assembly\\listing} & \resource{debugging\\information}}
\seeassembly\seerisc\seeobject\seecode\seedebugging
}

\providecommand{\cdwasm}{
\toolsection{cdwasm} is a compiler for intermediate code targeting the WebAssembly architecture.
It generates machine code for WebAssembly targets from programs written in intermediate code and stores it in corresponding object files.
It also creates a debugging information file as well as an assembly file containing a listing of the generated machine code.
\debuggingtool
\flowgraph{\resource{intermediate\\code} \ar[r] & \toolbox{cdwasm} \ar[r] \ar[d] \ar[rd] & \resource{object file} \\ & \resource{assembly\\listing} & \resource{debugging\\information}}
\seeassembly\seewasm\seeobject\seecode\seedebugging
}

% C++ tools

\providecommand{\cppprep}{
\toolsection{cppprep} is a preprocessor for the \cpp{} programming language.
It preprocesses source code according to the rules of \cpp{} and writes it to the standard output stream.
Only the macro names \texttt{\_\_DATE\_\_}, \texttt{\_\_FILE\_\_}, \texttt{\_\_LINE\_\_}, and \texttt{\_\_TIME\_\_} are predefined.
\flowgraph{\resource{\cpp{} or other\\source code} \ar[r] & \toolbox{cppprep} \ar[r] & \resource{preprocessed\\source code} \\ & \variable{ECSINCLUDE} \ar[u]}
\seecpp
}

\providecommand{\cppprint}{
\toolsection{cppprint} is a pretty printer for the \cpp{} programming language.
It reformats the source code of \cpp{} programs and writes it to the standard output stream.
\flowgraph{\resource{\cpp{}\\source code} \ar[r] & \toolbox{cppprint} \ar[r] & \resource{reformatted\\source code} \\ & \variable{ECSINCLUDE} \ar[u]}
\seecpp
}

\providecommand{\cppcheck}{
\toolsection{cppcheck} is a syntactic and semantic checker for the \cpp{} programming language.
It just performs syntactic and semantic checks on \cpp{} programs and writes its diagnostic messages to the standard error stream.
\flowgraph{\resource{\cpp{}\\source code} \ar[r] & \toolbox{cppcheck} \ar[r] & \resource{diagnostic\\messages} \\ & \variable{ECSINCLUDE} \ar[u]}
\seecpp
}

\providecommand{\cppdump}{
\toolsection{cppdump} is a serializer for the \cpp{} programming language.
It dumps the complete internal representation of programs written in \cpp{} into an XML document.
\debuggingtool
\flowgraph{\resource{\cpp{}\\source code} \ar[r] & \toolbox{cppdump} \ar[r] & \resource{internal\\representation} \\ & \variable{ECSINCLUDE} \ar[u]}
\seecpp
}

\providecommand{\cpprun}{
\toolsection{cpprun} is an interpreter for the \cpp{} programming language.
It processes and executes programs written in \cpp{}.
The macro \texttt{\_\_run\_\_} is predefined in order to enable programmers to identify this tool while interpreting.
\flowgraph{\resource{\cpp{}\\source code} \ar[r] & \toolbox{cpprun} \ar@/u/[r] & \resource{input/\\output} \ar@/d/[l] \\ & \variable{ECSINCLUDE} \ar[u]}
\seecpp
}

\providecommand{\cppdoc}{
\toolsection{cppdoc} is a generic documentation generator for the \cpp{} programming language.
It processes several \cpp{} source files and assembles all information therein into a generic documentation.
\debuggingtool
\flowgraph{\resource{\cpp{}\\source code} \ar[r] & \toolbox{cppdoc} \ar[r] & \resource{generic\\documentation} \\ & \variable{ECSINCLUDE} \ar[u]}
\seecpp\seedocumentation
}

\providecommand{\cpphtml}{
\toolsection{cpphtml} is an HTML documentation generator for the \cpp{} programming language.
It processes several \cpp{} source files and assembles all information therein into an HTML document.
\flowgraph{\resource{\cpp{}\\source code} \ar[r] & \toolbox{cpphtml} \ar[r] & \resource{HTML\\document} \\ & \variable{ECSINCLUDE} \ar[u]}
\seecpp\seedocumentation
}

\providecommand{\cpplatex}{
\toolsection{cpplatex} is a Latex documentation generator for the \cpp{} programming language.
It processes several \cpp{} source files and assembles all information therein into a Latex document.
\flowgraph{\resource{\cpp{}\\source code} \ar[r] & \toolbox{cpplatex} \ar[r] & \resource{Latex\\document} \\ & \variable{ECSINCLUDE} \ar[u]}
\seecpp\seedocumentation
}

\providecommand{\cppcode}{
\toolsection{cppcode} is an intermediate code generator for the \cpp{} programming language.
It generates intermediate code from programs written in \cpp{} and stores it in corresponding assembly files.
The macro \texttt{\_\_code\_\_} is predefined in order to enable programmers to identify this tool while generating intermediate code.
Programs generated with this tool require additional runtime support that is stored in the \file{cpp\-code\-run} library file.
\debuggingtool
\flowgraph{\resource{\cpp{}\\source code} \ar[r] & \toolbox{cppcode} \ar[r] & \resource{intermediate\\code} \\ & \variable{ECSINCLUDE} \ar[u]}
\seecpp\seeassembly\seecode
}

\providecommand{\cppamda}{
\toolsection{cppamd16} is a compiler for the \cpp{} programming language targeting the AMD64 hardware architecture.
It generates machine code for AMD64 processors from programs written in \cpp{} and stores it in corresponding object files.
The compiler generates machine code for the 16-bit operating mode defined by the AMD64 architecture.
For debugging purposes, it also creates a debugging information file as well as an assembly file containing a listing of the generated machine code.
The macro \texttt{\_\_amd16\_\_} is predefined in order to enable programmers to identify this tool and its target architecture while compiling.
Programs generated with this compiler require additional runtime support that is stored in the \file{cpp\-amd16\-run} library file.
\flowgraph{\resource{\cpp{}\\source code} \ar[r] & \toolbox{cppamd16} \ar[r] \ar[d] \ar[rd] & \resource{object file} \\ \variable{ECSINCLUDE} \ar[ru] & \resource{debugging\\information} & \resource{assembly\\listing}}
\seecpp\seeassembly\seeamd\seeobject\seedebugging
}

\providecommand{\cppamdb}{
\toolsection{cppamd32} is a compiler for the \cpp{} programming language targeting the AMD64 hardware architecture.
It generates machine code for AMD64 processors from programs written in \cpp{} and stores it in corresponding object files.
The compiler generates machine code for the 32-bit operating mode defined by the AMD64 architecture.
For debugging purposes, it also creates a debugging information file as well as an assembly file containing a listing of the generated machine code.
The macro \texttt{\_\_amd32\_\_} is predefined in order to enable programmers to identify this tool and its target architecture while compiling.
Programs generated with this compiler require additional runtime support that is stored in the \file{cpp\-amd32\-run} library file.
\flowgraph{\resource{\cpp{}\\source code} \ar[r] & \toolbox{cppamd32} \ar[r] \ar[d] \ar[rd] & \resource{object file} \\ \variable{ECSINCLUDE} \ar[ru] & \resource{debugging\\information} & \resource{assembly\\listing}}
\seecpp\seeassembly\seeamd\seeobject\seedebugging
}

\providecommand{\cppamdc}{
\toolsection{cppamd64} is a compiler for the \cpp{} programming language targeting the AMD64 hardware architecture.
It generates machine code for AMD64 processors from programs written in \cpp{} and stores it in corresponding object files.
The compiler generates machine code for the 64-bit operating mode defined by the AMD64 architecture.
For debugging purposes, it also creates a debugging information file as well as an assembly file containing a listing of the generated machine code.
The macro \texttt{\_\_amd64\_\_} is predefined in order to enable programmers to identify this tool and its target architecture while compiling.
Programs generated with this compiler require additional runtime support that is stored in the \file{cpp\-amd64\-run} library file.
\flowgraph{\resource{\cpp{}\\source code} \ar[r] & \toolbox{cppamd64} \ar[r] \ar[d] \ar[rd] & \resource{object file} \\ \variable{ECSINCLUDE} \ar[ru] & \resource{debugging\\information} & \resource{assembly\\listing}}
\seecpp\seeassembly\seeamd\seeobject\seedebugging
}

\providecommand{\cpparma}{
\toolsection{cpparma32} is a compiler for the \cpp{} programming language targeting the ARM hardware architecture.
It generates machine code for ARM processors executing A32 instructions from programs written in \cpp{} and stores it in corresponding object files.
For debugging purposes, it also creates a debugging information file as well as an assembly file containing a listing of the generated machine code.
The macro \texttt{\_\_arma32\_\_} is predefined in order to enable programmers to identify this tool and its target architecture while compiling.
Programs generated with this compiler require additional runtime support that is stored in the \file{cpp\-arma32\-run} library file.
\flowgraph{\resource{\cpp{}\\source code} \ar[r] & \toolbox{cpparma32} \ar[r] \ar[d] \ar[rd] & \resource{object file} \\ \variable{ECSINCLUDE} \ar[ru] & \resource{debugging\\information} & \resource{assembly\\listing}}
\seecpp\seeassembly\seearm\seeobject\seedebugging
}

\providecommand{\cpparmb}{
\toolsection{cpparma64} is a compiler for the \cpp{} programming language targeting the ARM hardware architecture.
It generates machine code for ARM processors executing A64 instructions from programs written in \cpp{} and stores it in corresponding object files.
For debugging purposes, it also creates a debugging information file as well as an assembly file containing a listing of the generated machine code.
The macro \texttt{\_\_arma64\_\_} is predefined in order to enable programmers to identify this tool and its target architecture while compiling.
Programs generated with this compiler require additional runtime support that is stored in the \file{cpp\-arma64\-run} library file.
\flowgraph{\resource{\cpp{}\\source code} \ar[r] & \toolbox{cpparma64} \ar[r] \ar[d] \ar[rd] & \resource{object file} \\ \variable{ECSINCLUDE} \ar[ru] & \resource{debugging\\information} & \resource{assembly\\listing}}
\seecpp\seeassembly\seearm\seeobject\seedebugging
}

\providecommand{\cpparmc}{
\toolsection{cpparmt32} is a compiler for the \cpp{} programming language targeting the ARM hardware architecture.
It generates machine code for ARM processors without floating-point extension executing T32 instructions from programs written in \cpp{} and stores it in corresponding object files.
For debugging purposes, it also creates a debugging information file as well as an assembly file containing a listing of the generated machine code.
The macro \texttt{\_\_armt32\_\_} is predefined in order to enable programmers to identify this tool and its target architecture while compiling.
Programs generated with this compiler require additional runtime support that is stored in the \file{cpp\-armt32\-run} library file.
\flowgraph{\resource{\cpp{}\\source code} \ar[r] & \toolbox{cpparmt32} \ar[r] \ar[d] \ar[rd] & \resource{object file} \\ \variable{ECSINCLUDE} \ar[ru] & \resource{debugging\\information} & \resource{assembly\\listing}}
\seecpp\seeassembly\seearm\seeobject\seedebugging
}

\providecommand{\cpparmcfpe}{
\toolsection{cpparmt32fpe} is a compiler for the \cpp{} programming language targeting the ARM hardware architecture.
It generates machine code for ARM processors with floating-point extension executing T32 instructions from programs written in \cpp{} and stores it in corresponding object files.
For debugging purposes, it also creates a debugging information file as well as an assembly file containing a listing of the generated machine code.
The macro \texttt{\_\_armt32fpe\_\_} is predefined in order to enable programmers to identify this tool and its target architecture while compiling.
Programs generated with this compiler require additional runtime support that is stored in the \file{cpp\-armt32\-fpe\-run} library file.
\flowgraph{\resource{\cpp{}\\source code} \ar[r] & \toolbox{cpparmt32fpe} \ar[r] \ar[d] \ar[rd] & \resource{object file} \\ \variable{ECSINCLUDE} \ar[ru] & \resource{debugging\\information} & \resource{assembly\\listing}}
\seecpp\seeassembly\seearm\seeobject\seedebugging
}

\providecommand{\cppavr}{
\toolsection{cppavr} is a compiler for the \cpp{} programming language targeting the AVR hardware architecture.
It generates machine code for AVR processors from programs written in \cpp{} and stores it in corresponding object files.
For debugging purposes, it also creates a debugging information file as well as an assembly file containing a listing of the generated machine code.
The macro \texttt{\_\_avr\_\_} is predefined in order to enable programmers to identify this tool and its target architecture while compiling.
Programs generated with this compiler require additional runtime support that is stored in the \file{cpp\-avr\-run} library file.
\flowgraph{\resource{\cpp{}\\source code} \ar[r] & \toolbox{cppavr} \ar[r] \ar[d] \ar[rd] & \resource{object file} \\ \variable{ECSINCLUDE} \ar[ru] & \resource{debugging\\information} & \resource{assembly\\listing}}
\seecpp\seeassembly\seeavr\seeobject\seedebugging
}

\providecommand{\cppavrtt}{
\toolsection{cppavr32} is a compiler for the \cpp{} programming language targeting the AVR32 hardware architecture.
It generates machine code for AVR32 processors from programs written in \cpp{} and stores it in corresponding object files.
For debugging purposes, it also creates a debugging information file as well as an assembly file containing a listing of the generated machine code.
The macro \texttt{\_\_avr32\_\_} is predefined in order to enable programmers to identify this tool and its target architecture while compiling.
Programs generated with this compiler require additional runtime support that is stored in the \file{cpp\-avr32\-run} library file.
\flowgraph{\resource{\cpp{}\\source code} \ar[r] & \toolbox{cppavr32} \ar[r] \ar[d] \ar[rd] & \resource{object file} \\ \variable{ECSINCLUDE} \ar[ru] & \resource{debugging\\information} & \resource{assembly\\listing}}
\seecpp\seeassembly\seeavrtt\seeobject\seedebugging
}

\providecommand{\cppmabk}{
\toolsection{cppm68k} is a compiler for the \cpp{} programming language targeting the M68000 hardware architecture.
It generates machine code for M68000 processors from programs written in \cpp{} and stores it in corresponding object files.
For debugging purposes, it also creates a debugging information file as well as an assembly file containing a listing of the generated machine code.
The macro \texttt{\_\_m68k\_\_} is predefined in order to enable programmers to identify this tool and its target architecture while compiling.
Programs generated with this compiler require additional runtime support that is stored in the \file{cpp\-m68k\-run} library file.
\flowgraph{\resource{\cpp{}\\source code} \ar[r] & \toolbox{cppm68k} \ar[r] \ar[d] \ar[rd] & \resource{object file} \\ \variable{ECSINCLUDE} \ar[ru] & \resource{debugging\\information} & \resource{assembly\\listing}}
\seecpp\seeassembly\seemabk\seeobject\seedebugging
}

\providecommand{\cppmibl}{
\toolsection{cppmibl} is a compiler for the \cpp{} programming language targeting the MicroBlaze hardware architecture.
It generates machine code for MicroBlaze processors from programs written in \cpp{} and stores it in corresponding object files.
For debugging purposes, it also creates a debugging information file as well as an assembly file containing a listing of the generated machine code.
The macro \texttt{\_\_mibl\_\_} is predefined in order to enable programmers to identify this tool and its target architecture while compiling.
Programs generated with this compiler require additional runtime support that is stored in the \file{cpp\-mibl\-run} library file.
\flowgraph{\resource{\cpp{}\\source code} \ar[r] & \toolbox{cppmibl} \ar[r] \ar[d] \ar[rd] & \resource{object file} \\ \variable{ECSINCLUDE} \ar[ru] & \resource{debugging\\information} & \resource{assembly\\listing}}
\seecpp\seeassembly\seemibl\seeobject\seedebugging
}

\providecommand{\cppmipsa}{
\toolsection{cppmips32} is a compiler for the \cpp{} programming language targeting the MIPS32 hardware architecture.
It generates machine code for MIPS32 processors from programs written in \cpp{} and stores it in corresponding object files.
For debugging purposes, it also creates a debugging information file as well as an assembly file containing a listing of the generated machine code.
The macro \texttt{\_\_mips32\_\_} is predefined in order to enable programmers to identify this tool and its target architecture while compiling.
Programs generated with this compiler require additional runtime support that is stored in the \file{cpp\-mips32\-run} library file.
\flowgraph{\resource{\cpp{}\\source code} \ar[r] & \toolbox{cppmips32} \ar[r] \ar[d] \ar[rd] & \resource{object file} \\ \variable{ECSINCLUDE} \ar[ru] & \resource{debugging\\information} & \resource{assembly\\listing}}
\seecpp\seeassembly\seemips\seeobject\seedebugging
}

\providecommand{\cppmipsb}{
\toolsection{cppmips64} is a compiler for the \cpp{} programming language targeting the MIPS64 hardware architecture.
It generates machine code for MIPS64 processors from programs written in \cpp{} and stores it in corresponding object files.
For debugging purposes, it also creates a debugging information file as well as an assembly file containing a listing of the generated machine code.
The macro \texttt{\_\_mips64\_\_} is predefined in order to enable programmers to identify this tool and its target architecture while compiling.
Programs generated with this compiler require additional runtime support that is stored in the \file{cpp\-mips64\-run} library file.
\flowgraph{\resource{\cpp{}\\source code} \ar[r] & \toolbox{cppmips64} \ar[r] \ar[d] \ar[rd] & \resource{object file} \\ \variable{ECSINCLUDE} \ar[ru] & \resource{debugging\\information} & \resource{assembly\\listing}}
\seecpp\seeassembly\seemips\seeobject\seedebugging
}

\providecommand{\cppmmix}{
\toolsection{cppmmix} is a compiler for the \cpp{} programming language targeting the MMIX hardware architecture.
It generates machine code for MMIX processors from programs written in \cpp{} and stores it in corresponding object files.
For debugging purposes, it also creates a debugging information file as well as an assembly file containing a listing of the generated machine code.
The macro \texttt{\_\_mmix\_\_} is predefined in order to enable programmers to identify this tool and its target architecture while compiling.
Programs generated with this compiler require additional runtime support that is stored in the \file{cpp\-mmix\-run} library file.
\flowgraph{\resource{\cpp{}\\source code} \ar[r] & \toolbox{cppmmix} \ar[r] \ar[d] \ar[rd] & \resource{object file} \\ \variable{ECSINCLUDE} \ar[ru] & \resource{debugging\\information} & \resource{assembly\\listing}}
\seecpp\seeassembly\seemmix\seeobject\seedebugging
}

\providecommand{\cpporok}{
\toolsection{cppor1k} is a compiler for the \cpp{} programming language targeting the OpenRISC 1000 hardware architecture.
It generates machine code for OpenRISC 1000 processors from programs written in \cpp{} and stores it in corresponding object files.
For debugging purposes, it also creates a debugging information file as well as an assembly file containing a listing of the generated machine code.
The macro \texttt{\_\_or1k\_\_} is predefined in order to enable programmers to identify this tool and its target architecture while compiling.
Programs generated with this compiler require additional runtime support that is stored in the \file{cpp\-or1k\-run} library file.
\flowgraph{\resource{\cpp{}\\source code} \ar[r] & \toolbox{cppor1k} \ar[r] \ar[d] \ar[rd] & \resource{object file} \\ \variable{ECSINCLUDE} \ar[ru] & \resource{debugging\\information} & \resource{assembly\\listing}}
\seecpp\seeassembly\seeorok\seeobject\seedebugging
}

\providecommand{\cppppca}{
\toolsection{cppppc32} is a compiler for the \cpp{} programming language targeting the PowerPC hardware architecture.
It generates machine code for PowerPC processors from programs written in \cpp{} and stores it in corresponding object files.
The compiler generates machine code for the 32-bit operating mode defined by the PowerPC architecture.
For debugging purposes, it also creates a debugging information file as well as an assembly file containing a listing of the generated machine code.
The macro \texttt{\_\_ppc32\_\_} is predefined in order to enable programmers to identify this tool and its target architecture while compiling.
Programs generated with this compiler require additional runtime support that is stored in the \file{cpp\-ppc32\-run} library file.
\flowgraph{\resource{\cpp{}\\source code} \ar[r] & \toolbox{cppppc32} \ar[r] \ar[d] \ar[rd] & \resource{object file} \\ \variable{ECSINCLUDE} \ar[ru] & \resource{debugging\\information} & \resource{assembly\\listing}}
\seecpp\seeassembly\seeppc\seeobject\seedebugging
}

\providecommand{\cppppcb}{
\toolsection{cppppc64} is a compiler for the \cpp{} programming language targeting the PowerPC hardware architecture.
It generates machine code for PowerPC processors from programs written in \cpp{} and stores it in corresponding object files.
The compiler generates machine code for the 64-bit operating mode defined by the PowerPC architecture.
For debugging purposes, it also creates a debugging information file as well as an assembly file containing a listing of the generated machine code.
The macro \texttt{\_\_ppc64\_\_} is predefined in order to enable programmers to identify this tool and its target architecture while compiling.
Programs generated with this compiler require additional runtime support that is stored in the \file{cpp\-ppc64\-run} library file.
\flowgraph{\resource{\cpp{}\\source code} \ar[r] & \toolbox{cppppc64} \ar[r] \ar[d] \ar[rd] & \resource{object file} \\ \variable{ECSINCLUDE} \ar[ru] & \resource{debugging\\information} & \resource{assembly\\listing}}
\seecpp\seeassembly\seeppc\seeobject\seedebugging
}

\providecommand{\cpprisc}{
\toolsection{cpprisc} is a compiler for the \cpp{} programming language targeting the RISC hardware architecture.
It generates machine code for RISC processors from programs written in \cpp{} and stores it in corresponding object files.
For debugging purposes, it also creates a debugging information file as well as an assembly file containing a listing of the generated machine code.
The macro \texttt{\_\_risc\_\_} is predefined in order to enable programmers to identify this tool and its target architecture while compiling.
Programs generated with this compiler require additional runtime support that is stored in the \file{cpp\-risc\-run} library file.
\flowgraph{\resource{\cpp{}\\source code} \ar[r] & \toolbox{cpprisc} \ar[r] \ar[d] \ar[rd] & \resource{object file} \\ \variable{ECSINCLUDE} \ar[ru] & \resource{debugging\\information} & \resource{assembly\\listing}}
\seecpp\seeassembly\seerisc\seeobject\seedebugging
}

\providecommand{\cppwasm}{
\toolsection{cppwasm} is a compiler for the \cpp{} programming language targeting the WebAssembly architecture.
It generates machine code for WebAssembly targets from programs written in \cpp{} and stores it in corresponding object files.
For debugging purposes, it also creates a debugging information file as well as an assembly file containing a listing of the generated machine code.
The macro \texttt{\_\_wasm\_\_} is predefined in order to enable programmers to identify this tool and its target architecture while compiling.
Programs generated with this compiler require additional runtime support that is stored in the \file{cpp\-wasm\-run} library file.
\flowgraph{\resource{\cpp{}\\source code} \ar[r] & \toolbox{cppwasm} \ar[r] \ar[d] \ar[rd] & \resource{object file} \\ \variable{ECSINCLUDE} \ar[ru] & \resource{debugging\\information} & \resource{assembly\\listing}}
\seecpp\seeassembly\seewasm\seeobject\seedebugging
}

% FALSE tools

\providecommand{\falprint}{
\toolsection{falprint} is a pretty printer for the FALSE programming language.
It reformats the source code of FALSE programs and writes it to the standard output stream.
\flowgraph{\resource{FALSE\\source code} \ar[r] & \toolbox{falprint} \ar[r] & \resource{reformatted\\source code}}
\seefalse
}

\providecommand{\falcheck}{
\toolsection{falcheck} is a syntactic and semantic checker for the FALSE programming language.
It just performs syntactic and semantic checks on FALSE programs and writes its diagnostic messages to the standard error stream.
\flowgraph{\resource{FALSE\\source code} \ar[r] & \toolbox{falcheck} \ar[r] & \resource{diagnostic\\messages}}
\seefalse
}

\providecommand{\faldump}{
\toolsection{faldump} is a serializer for the FALSE programming language.
It dumps the complete internal representation of programs written in FALSE into an XML document.
\debuggingtool
\flowgraph{\resource{FALSE\\source code} \ar[r] & \toolbox{faldump} \ar[r] & \resource{internal\\representation}}
\seefalse
}

\providecommand{\falrun}{
\toolsection{falrun} is an interpreter for the FALSE programming language.
It processes and executes programs written in FALSE\@.
\flowgraph{\resource{FALSE\\source code} \ar[r] & \toolbox{falrun} \ar@/u/[r] & \resource{input/\\output} \ar@/d/[l]}
\seefalse
}

\providecommand{\falcpp}{
\toolsection{falcpp} is a transpiler for the FALSE programming language.
It translates programs written in FALSE into \cpp{} programs and stores them in corresponding source files.
\flowgraph{\resource{FALSE\\source code} \ar[r] & \toolbox{falcpp} \ar[r] & \resource{\cpp{}\\source file}}
\seefalse\seecpp
}

\providecommand{\falcode}{
\toolsection{falcode} is an intermediate code generator for the FALSE programming language.
It generates intermediate code from programs written in FALSE and stores it in corresponding assembly files.
\debuggingtool
\flowgraph{\resource{FALSE\\source code} \ar[r] & \toolbox{falcode} \ar[r] & \resource{intermediate\\code}}
\seefalse\seeassembly\seecode
}

\providecommand{\falamda}{
\toolsection{falamd16} is a compiler for the FALSE programming language targeting the AMD64 hardware architecture.
It generates machine code for AMD64 processors from programs written in FALSE and stores it in corresponding object files.
The compiler generates machine code for the 16-bit operating mode defined by the AMD64 architecture.
\flowgraph{\resource{FALSE\\source code} \ar[r] & \toolbox{falamd16} \ar[r] & \resource{object file}}
\seefalse\seeamd\seeobject
}

\providecommand{\falamdb}{
\toolsection{falamd32} is a compiler for the FALSE programming language targeting the AMD64 hardware architecture.
It generates machine code for AMD64 processors from programs written in FALSE and stores it in corresponding object files.
The compiler generates machine code for the 32-bit operating mode defined by the AMD64 architecture.
\flowgraph{\resource{FALSE\\source code} \ar[r] & \toolbox{falamd32} \ar[r] & \resource{object file}}
\seefalse\seeamd\seeobject
}

\providecommand{\falamdc}{
\toolsection{falamd64} is a compiler for the FALSE programming language targeting the AMD64 hardware architecture.
It generates machine code for AMD64 processors from programs written in FALSE and stores it in corresponding object files.
The compiler generates machine code for the 64-bit operating mode defined by the AMD64 architecture.
\flowgraph{\resource{FALSE\\source code} \ar[r] & \toolbox{falamd64} \ar[r] & \resource{object file}}
\seefalse\seeamd\seeobject
}

\providecommand{\falarma}{
\toolsection{falarma32} is a compiler for the FALSE programming language targeting the ARM hardware architecture.
It generates machine code for ARM processors executing A32 instructions from programs written in FALSE and stores it in corresponding object files.
\flowgraph{\resource{FALSE\\source code} \ar[r] & \toolbox{falarma32} \ar[r] & \resource{object file}}
\seefalse\seearm\seeobject
}

\providecommand{\falarmb}{
\toolsection{falarma64} is a compiler for the FALSE programming language targeting the ARM hardware architecture.
It generates machine code for ARM processors executing A64 instructions from programs written in FALSE and stores it in corresponding object files.
\flowgraph{\resource{FALSE\\source code} \ar[r] & \toolbox{falarma64} \ar[r] & \resource{object file}}
\seefalse\seearm\seeobject
}

\providecommand{\falarmc}{
\toolsection{falarmt32} is a compiler for the FALSE programming language targeting the ARM hardware architecture.
It generates machine code for ARM processors without floating-point extension executing T32 instructions from programs written in FALSE and stores it in corresponding object files.
\flowgraph{\resource{FALSE\\source code} \ar[r] & \toolbox{falarmt32} \ar[r] & \resource{object file}}
\seefalse\seearm\seeobject
}

\providecommand{\falarmcfpe}{
\toolsection{falarmt32fpe} is a compiler for the FALSE programming language targeting the ARM hardware architecture.
It generates machine code for ARM processors with floating-point extension executing T32 instructions from programs written in FALSE and stores it in corresponding object files.
\flowgraph{\resource{FALSE\\source code} \ar[r] & \toolbox{falarmt32fpe} \ar[r] & \resource{object file}}
\seefalse\seearm\seeobject
}

\providecommand{\falavr}{
\toolsection{falavr} is a compiler for the FALSE programming language targeting the AVR hardware architecture.
It generates machine code for AVR processors from programs written in FALSE and stores it in corresponding object files.
\flowgraph{\resource{FALSE\\source code} \ar[r] & \toolbox{falavr} \ar[r] & \resource{object file}}
\seefalse\seeavr\seeobject
}

\providecommand{\falavrtt}{
\toolsection{falavr32} is a compiler for the FALSE programming language targeting the AVR32 hardware architecture.
It generates machine code for AVR32 processors from programs written in FALSE and stores it in corresponding object files.
\flowgraph{\resource{FALSE\\source code} \ar[r] & \toolbox{falavr32} \ar[r] & \resource{object file}}
\seefalse\seeavrtt\seeobject
}

\providecommand{\falmabk}{
\toolsection{falm68k} is a compiler for the FALSE programming language targeting the M68000 hardware architecture.
It generates machine code for M68000 processors from programs written in FALSE and stores it in corresponding object files.
\flowgraph{\resource{FALSE\\source code} \ar[r] & \toolbox{falm68k} \ar[r] & \resource{object file}}
\seefalse\seemabk\seeobject
}

\providecommand{\falmibl}{
\toolsection{falmibl} is a compiler for the FALSE programming language targeting the MicroBlaze hardware architecture.
It generates machine code for MicroBlaze processors from programs written in FALSE and stores it in corresponding object files.
\flowgraph{\resource{FALSE\\source code} \ar[r] & \toolbox{falmibl} \ar[r] & \resource{object file}}
\seefalse\seemibl\seeobject
}

\providecommand{\falmipsa}{
\toolsection{falmips32} is a compiler for the FALSE programming language targeting the MIPS32 hardware architecture.
It generates machine code for MIPS32 processors from programs written in FALSE and stores it in corresponding object files.
\flowgraph{\resource{FALSE\\source code} \ar[r] & \toolbox{falmips32} \ar[r] & \resource{object file}}
\seefalse\seemips\seeobject
}

\providecommand{\falmipsb}{
\toolsection{falmips64} is a compiler for the FALSE programming language targeting the MIPS64 hardware architecture.
It generates machine code for MIPS64 processors from programs written in FALSE and stores it in corresponding object files.
\flowgraph{\resource{FALSE\\source code} \ar[r] & \toolbox{falmips64} \ar[r] & \resource{object file}}
\seefalse\seemips\seeobject
}

\providecommand{\falmmix}{
\toolsection{falmmix} is a compiler for the FALSE programming language targeting the MMIX hardware architecture.
It generates machine code for MMIX processors from programs written in FALSE and stores it in corresponding object files.
\flowgraph{\resource{FALSE\\source code} \ar[r] & \toolbox{falmmix} \ar[r] & \resource{object file}}
\seefalse\seemmix\seeobject
}

\providecommand{\falorok}{
\toolsection{falor1k} is a compiler for the FALSE programming language targeting the OpenRISC 1000 hardware architecture.
It generates machine code for OpenRISC 1000 processors from programs written in FALSE and stores it in corresponding object files.
\flowgraph{\resource{FALSE\\source code} \ar[r] & \toolbox{falor1k} \ar[r] & \resource{object file}}
\seefalse\seeorok\seeobject
}

\providecommand{\falppca}{
\toolsection{falppc32} is a compiler for the FALSE programming language targeting the PowerPC hardware architecture.
It generates machine code for PowerPC processors from programs written in FALSE and stores it in corresponding object files.
The compiler generates machine code for the 32-bit operating mode defined by the PowerPC architecture.
\flowgraph{\resource{FALSE\\source code} \ar[r] & \toolbox{falppc32} \ar[r] & \resource{object file}}
\seefalse\seeppc\seeobject
}

\providecommand{\falppcb}{
\toolsection{falppc64} is a compiler for the FALSE programming language targeting the PowerPC hardware architecture.
It generates machine code for PowerPC processors from programs written in FALSE and stores it in corresponding object files.
The compiler generates machine code for the 64-bit operating mode defined by the PowerPC architecture.
\flowgraph{\resource{FALSE\\source code} \ar[r] & \toolbox{falppc64} \ar[r] & \resource{object file}}
\seefalse\seeppc\seeobject
}

\providecommand{\falrisc}{
\toolsection{falrisc} is a compiler for the FALSE programming language targeting the RISC hardware architecture.
It generates machine code for RISC processors from programs written in FALSE and stores it in corresponding object files.
\flowgraph{\resource{FALSE\\source code} \ar[r] & \toolbox{falrisc} \ar[r] & \resource{object file}}
\seefalse\seerisc\seeobject
}

\providecommand{\falwasm}{
\toolsection{falwasm} is a compiler for the FALSE programming language targeting the WebAssembly architecture.
It generates machine code for WebAssembly targets from programs written in FALSE and stores it in corresponding object files.
\flowgraph{\resource{FALSE\\source code} \ar[r] & \toolbox{falwasm} \ar[r] & \resource{object file}}
\seefalse\seewasm\seeobject
}

% Oberon tools

\providecommand{\obprint}{
\toolsection{obprint} is a pretty printer for the Oberon programming language.
It reformats the source code of Oberon modules and writes it to the standard output stream.
\flowgraph{\resource{Oberon\\source code} \ar[r] & \toolbox{obprint} \ar[r] & \resource{reformatted\\source code}}
\seeoberon
}

\providecommand{\obcheck}{
\toolsection{obcheck} is a syntactic and semantic checker for the Oberon programming language.
It just performs syntactic and semantic checks on Oberon modules and writes its diagnostic messages to the standard error stream.
In addition, it stores the interface of each module in a symbol file which is required when other modules import the module.
\flowgraph{\resource{Oberon\\source code} \ar[r] & \toolbox{obcheck} \ar[r] \ar@/l/[d] & \resource{diagnostic\\messages} \\ \variable{ECSIMPORT} \ar[ru] & \resource{symbol\\files} \ar@/r/[u]}
\seeoberon
}

\providecommand{\obdump}{
\toolsection{obdump} is a serializer for the Oberon programming language.
It dumps the complete internal representation of modules written in Oberon into an XML document.
\debuggingtool
\flowgraph{\resource{Oberon\\source code} \ar[r] & \toolbox{obdump} \ar[r] \ar@/l/[d] & \resource{internal\\representation} \\ \variable{ECSIMPORT} \ar[ru] & \resource{symbol\\files} \ar@/r/[u]}
\seeoberon
}

\providecommand{\obrun}{
\toolsection{obrun} is an interpreter for the Oberon programming language.
It processes and executes modules written in Oberon.
This tool does neither generate nor process symbol files while interpreting modules.
If a module is imported by another one, its filename has to be named before the other one in the list of command-line arguments.
\flowgraph{\resource{Oberon\\source code} \ar[r] & \toolbox{obrun} \ar@/u/[r] & \resource{input/\\output} \ar@/d/[l]}
\seeoberon
}

\providecommand{\obcpp}{
\toolsection{obcpp} is a transpiler for the Oberon programming language.
It translates programs written in Oberon into \cpp{} programs and stores them in corresponding source and header files.
In addition, it stores the interface of each module in a symbol file which is required when other modules import the module.
The same interface is provided by the generated header file which can be used in other parts of the \cpp{} program.
\flowgraph{\resource{Oberon\\source code} \ar[r] & \toolbox{obcpp} \ar[r] \ar@/l/[d] \ar[rd] & \resource{\cpp{}\\source file} \\ \variable{ECSIMPORT} \ar[ru] & \resource{symbol\\files} \ar@/r/[u] & \resource{\cpp{}\\header file}}
\seeoberon\seecpp
}

\providecommand{\obdoc}{
\toolsection{obdoc} is a generic documentation generator for the Oberon programming language.
It processes several Oberon modules and assembles all information therein into a generic documentation.
In addition, it stores the interface of each module in a symbol file which is required when other modules import the module.
\debuggingtool
\flowgraph{\resource{Oberon\\source code} \ar[r] & \toolbox{obdoc} \ar[r] \ar@/l/[d] & \resource{generic\\documentation} \\ \variable{ECSIMPORT} \ar[ru] & \resource{symbol\\files} \ar@/r/[u]}
\seeoberon\seedocumentation
}

\providecommand{\obhtml}{
\toolsection{obhtml} is an HTML documentation generator for the Oberon programming language.
It processes several Oberon modules and assembles all information therein into an HTML document.
In addition, it stores the interface of each module in a symbol file which is required when other modules import the module.
\flowgraph{\resource{Oberon\\source code} \ar[r] & \toolbox{obhtml} \ar[r] \ar@/l/[d] & \resource{HTML\\document} \\ \variable{ECSIMPORT} \ar[ru] & \resource{symbol\\files} \ar@/r/[u]}
\seeoberon\seedocumentation
}

\providecommand{\oblatex}{
\toolsection{oblatex} is a Latex documentation generator for the Oberon programming language.
It processes several Oberon modules and assembles all information therein into a Latex document.
In addition, it stores the interface of each module in a symbol file which is required when other modules import the module.
\flowgraph{\resource{Oberon\\source code} \ar[r] & \toolbox{oblatex} \ar[r] \ar@/l/[d] & \resource{Latex\\document} \\ \variable{ECSIMPORT} \ar[ru] & \resource{symbol\\files} \ar@/r/[u]}
\seeoberon\seedocumentation
}

\providecommand{\obcode}{
\toolsection{obcode} is an intermediate code generator for the Oberon programming language.
It generates intermediate code from modules written in Oberon and stores it in corresponding assembly files.
In addition, it stores the interface of each module in a symbol file which is required when other modules import the module.
Programs generated with this tool require additional runtime support that is stored in the \file{ob\-code\-run} library file.
\debuggingtool
\flowgraph{\resource{Oberon\\source code} \ar[r] & \toolbox{obcode} \ar[r] \ar@/l/[d] & \resource{intermediate\\code} \\ \variable{ECSIMPORT} \ar[ru] & \resource{symbol\\files} \ar@/r/[u]}
\seeoberon\seeassembly\seecode
}

\providecommand{\obamda}{
\toolsection{obamd16} is a compiler for the Oberon programming language targeting the AMD64 hardware architecture.
It generates machine code for AMD64 processors from modules written in Oberon and stores it in corresponding object files.
The compiler generates machine code for the 16-bit operating mode defined by the AMD64 architecture.
For debugging purposes, it also creates a debugging information file as well as an assembly file containing a listing of the generated machine code.
In addition, it stores the interface of each module in a symbol file which is required when other modules import the module.
Programs generated with this compiler require additional runtime support that is stored in the \file{ob\-amd16\-run} library file.
\flowgraph{\resource{Oberon\\source code} \ar[r] & \toolbox{obamd16} \ar[r] \ar@/l/[d] \ar[rd] & \resource{object file} \\ \variable{ECSIMPORT} \ar[ru] & \resource{symbol\\files} \ar@/r/[u] & \resource{debugging\\information}}
\seeoberon\seeassembly\seeamd\seeobject\seedebugging
}

\providecommand{\obamdb}{
\toolsection{obamd32} is a compiler for the Oberon programming language targeting the AMD64 hardware architecture.
It generates machine code for AMD64 processors from modules written in Oberon and stores it in corresponding object files.
The compiler generates machine code for the 32-bit operating mode defined by the AMD64 architecture.
For debugging purposes, it also creates a debugging information file as well as an assembly file containing a listing of the generated machine code.
In addition, it stores the interface of each module in a symbol file which is required when other modules import the module.
Programs generated with this compiler require additional runtime support that is stored in the \file{ob\-amd32\-run} library file.
\flowgraph{\resource{Oberon\\source code} \ar[r] & \toolbox{obamd32} \ar[r] \ar@/l/[d] \ar[rd] & \resource{object file} \\ \variable{ECSIMPORT} \ar[ru] & \resource{symbol\\files} \ar@/r/[u] & \resource{debugging\\information}}
\seeoberon\seeassembly\seeamd\seeobject\seedebugging
}

\providecommand{\obamdc}{
\toolsection{obamd64} is a compiler for the Oberon programming language targeting the AMD64 hardware architecture.
It generates machine code for AMD64 processors from modules written in Oberon and stores it in corresponding object files.
The compiler generates machine code for the 64-bit operating mode defined by the AMD64 architecture.
For debugging purposes, it also creates a debugging information file as well as an assembly file containing a listing of the generated machine code.
In addition, it stores the interface of each module in a symbol file which is required when other modules import the module.
Programs generated with this compiler require additional runtime support that is stored in the \file{ob\-amd64\-run} library file.
\flowgraph{\resource{Oberon\\source code} \ar[r] & \toolbox{obamd64} \ar[r] \ar@/l/[d] \ar[rd] & \resource{object file} \\ \variable{ECSIMPORT} \ar[ru] & \resource{symbol\\files} \ar@/r/[u] & \resource{debugging\\information}}
\seeoberon\seeassembly\seeamd\seeobject\seedebugging
}

\providecommand{\obarma}{
\toolsection{obarma32} is a compiler for the Oberon programming language targeting the ARM hardware architecture.
It generates machine code for ARM processors executing A32 instructions from modules written in Oberon and stores it in corresponding object files.
For debugging purposes, it also creates a debugging information file as well as an assembly file containing a listing of the generated machine code.
In addition, it stores the interface of each module in a symbol file which is required when other modules import the module.
Programs generated with this compiler require additional runtime support that is stored in the \file{ob\-arma32\-run} library file.
\flowgraph{\resource{Oberon\\source code} \ar[r] & \toolbox{obarma32} \ar[r] \ar@/l/[d] \ar[rd] & \resource{object file} \\ \variable{ECSIMPORT} \ar[ru] & \resource{symbol\\files} \ar@/r/[u] & \resource{debugging\\information}}
\seeoberon\seeassembly\seearm\seeobject\seedebugging
}

\providecommand{\obarmb}{
\toolsection{obarma64} is a compiler for the Oberon programming language targeting the ARM hardware architecture.
It generates machine code for ARM processors executing A64 instructions from modules written in Oberon and stores it in corresponding object files.
For debugging purposes, it also creates a debugging information file as well as an assembly file containing a listing of the generated machine code.
In addition, it stores the interface of each module in a symbol file which is required when other modules import the module.
Programs generated with this compiler require additional runtime support that is stored in the \file{ob\-arma64\-run} library file.
\flowgraph{\resource{Oberon\\source code} \ar[r] & \toolbox{obarma64} \ar[r] \ar@/l/[d] \ar[rd] & \resource{object file} \\ \variable{ECSIMPORT} \ar[ru] & \resource{symbol\\files} \ar@/r/[u] & \resource{debugging\\information}}
\seeoberon\seeassembly\seearm\seeobject\seedebugging
}

\providecommand{\obarmc}{
\toolsection{obarmt32} is a compiler for the Oberon programming language targeting the ARM hardware architecture.
It generates machine code for ARM processors without floating-point extension executing T32 instructions from modules written in Oberon and stores it in corresponding object files.
For debugging purposes, it also creates a debugging information file as well as an assembly file containing a listing of the generated machine code.
In addition, it stores the interface of each module in a symbol file which is required when other modules import the module.
Programs generated with this compiler require additional runtime support that is stored in the \file{ob\-armt32\-run} library file.
\flowgraph{\resource{Oberon\\source code} \ar[r] & \toolbox{obarmt32} \ar[r] \ar@/l/[d] \ar[rd] & \resource{object file} \\ \variable{ECSIMPORT} \ar[ru] & \resource{symbol\\files} \ar@/r/[u] & \resource{debugging\\information}}
\seeoberon\seeassembly\seearm\seeobject\seedebugging
}

\providecommand{\obarmcfpe}{
\toolsection{obarmt32fpe} is a compiler for the Oberon programming language targeting the ARM hardware architecture.
It generates machine code for ARM processors with floating-point extension executing T32 instructions from modules written in Oberon and stores it in corresponding object files.
For debugging purposes, it also creates a debugging information file as well as an assembly file containing a listing of the generated machine code.
In addition, it stores the interface of each module in a symbol file which is required when other modules import the module.
Programs generated with this compiler require additional runtime support that is stored in the \file{ob\-armt32\-fpe\-run} library file.
\flowgraph{\resource{Oberon\\source code} \ar[r] & \toolbox{obarmt32fpe} \ar[r] \ar@/l/[d] \ar[rd] & \resource{object file} \\ \variable{ECSIMPORT} \ar[ru] & \resource{symbol\\files} \ar@/r/[u] & \resource{debugging\\information}}
\seeoberon\seeassembly\seearm\seeobject\seedebugging
}

\providecommand{\obavr}{
\toolsection{obavr} is a compiler for the Oberon programming language targeting the AVR hardware architecture.
It generates machine code for AVR processors from modules written in Oberon and stores it in corresponding object files.
For debugging purposes, it also creates a debugging information file as well as an assembly file containing a listing of the generated machine code.
In addition, it stores the interface of each module in a symbol file which is required when other modules import the module.
Programs generated with this compiler require additional runtime support that is stored in the \file{ob\-avr\-run} library file.
\flowgraph{\resource{Oberon\\source code} \ar[r] & \toolbox{obavr} \ar[r] \ar@/l/[d] \ar[rd] & \resource{object file} \\ \variable{ECSIMPORT} \ar[ru] & \resource{symbol\\files} \ar@/r/[u] & \resource{debugging\\information}}
\seeoberon\seeassembly\seeavr\seeobject\seedebugging
}

\providecommand{\obavrtt}{
\toolsection{obavr32} is a compiler for the Oberon programming language targeting the AVR32 hardware architecture.
It generates machine code for AVR32 processors from modules written in Oberon and stores it in corresponding object files.
For debugging purposes, it also creates a debugging information file as well as an assembly file containing a listing of the generated machine code.
In addition, it stores the interface of each module in a symbol file which is required when other modules import the module.
Programs generated with this compiler require additional runtime support that is stored in the \file{ob\-avr32\-run} library file.
\flowgraph{\resource{Oberon\\source code} \ar[r] & \toolbox{obavr32} \ar[r] \ar@/l/[d] \ar[rd] & \resource{object file} \\ \variable{ECSIMPORT} \ar[ru] & \resource{symbol\\files} \ar@/r/[u] & \resource{debugging\\information}}
\seeoberon\seeassembly\seeavrtt\seeobject\seedebugging
}

\providecommand{\obmabk}{
\toolsection{obm68k} is a compiler for the Oberon programming language targeting the M68000 hardware architecture.
It generates machine code for M68000 processors from modules written in Oberon and stores it in corresponding object files.
For debugging purposes, it also creates a debugging information file as well as an assembly file containing a listing of the generated machine code.
In addition, it stores the interface of each module in a symbol file which is required when other modules import the module.
Programs generated with this compiler require additional runtime support that is stored in the \file{ob\-m68k\-run} library file.
\flowgraph{\resource{Oberon\\source code} \ar[r] & \toolbox{obm68k} \ar[r] \ar@/l/[d] \ar[rd] & \resource{object file} \\ \variable{ECSIMPORT} \ar[ru] & \resource{symbol\\files} \ar@/r/[u] & \resource{debugging\\information}}
\seeoberon\seeassembly\seemabk\seeobject\seedebugging
}

\providecommand{\obmibl}{
\toolsection{obmibl} is a compiler for the Oberon programming language targeting the MicroBlaze hardware architecture.
It generates machine code for MicroBlaze processors from modules written in Oberon and stores it in corresponding object files.
For debugging purposes, it also creates a debugging information file as well as an assembly file containing a listing of the generated machine code.
In addition, it stores the interface of each module in a symbol file which is required when other modules import the module.
Programs generated with this compiler require additional runtime support that is stored in the \file{ob\-mibl\-run} library file.
\flowgraph{\resource{Oberon\\source code} \ar[r] & \toolbox{obmibl} \ar[r] \ar@/l/[d] \ar[rd] & \resource{object file} \\ \variable{ECSIMPORT} \ar[ru] & \resource{symbol\\files} \ar@/r/[u] & \resource{debugging\\information}}
\seeoberon\seeassembly\seemibl\seeobject\seedebugging
}

\providecommand{\obmipsa}{
\toolsection{obmips32} is a compiler for the Oberon programming language targeting the MIPS32 hardware architecture.
It generates machine code for MIPS32 processors from modules written in Oberon and stores it in corresponding object files.
For debugging purposes, it also creates a debugging information file as well as an assembly file containing a listing of the generated machine code.
In addition, it stores the interface of each module in a symbol file which is required when other modules import the module.
Programs generated with this compiler require additional runtime support that is stored in the \file{ob\-mips32\-run} library file.
\flowgraph{\resource{Oberon\\source code} \ar[r] & \toolbox{obmips32} \ar[r] \ar@/l/[d] \ar[rd] & \resource{object file} \\ \variable{ECSIMPORT} \ar[ru] & \resource{symbol\\files} \ar@/r/[u] & \resource{debugging\\information}}
\seeoberon\seeassembly\seemips\seeobject\seedebugging
}

\providecommand{\obmipsb}{
\toolsection{obmips64} is a compiler for the Oberon programming language targeting the MIPS64 hardware architecture.
It generates machine code for MIPS64 processors from modules written in Oberon and stores it in corresponding object files.
For debugging purposes, it also creates a debugging information file as well as an assembly file containing a listing of the generated machine code.
In addition, it stores the interface of each module in a symbol file which is required when other modules import the module.
Programs generated with this compiler require additional runtime support that is stored in the \file{ob\-mips64\-run} library file.
\flowgraph{\resource{Oberon\\source code} \ar[r] & \toolbox{obmips64} \ar[r] \ar@/l/[d] \ar[rd] & \resource{object file} \\ \variable{ECSIMPORT} \ar[ru] & \resource{symbol\\files} \ar@/r/[u] & \resource{debugging\\information}}
\seeoberon\seeassembly\seemips\seeobject\seedebugging
}

\providecommand{\obmmix}{
\toolsection{obmmix} is a compiler for the Oberon programming language targeting the MMIX hardware architecture.
It generates machine code for MMIX processors from modules written in Oberon and stores it in corresponding object files.
For debugging purposes, it also creates a debugging information file as well as an assembly file containing a listing of the generated machine code.
In addition, it stores the interface of each module in a symbol file which is required when other modules import the module.
Programs generated with this compiler require additional runtime support that is stored in the \file{ob\-mmix\-run} library file.
\flowgraph{\resource{Oberon\\source code} \ar[r] & \toolbox{obmmix} \ar[r] \ar@/l/[d] \ar[rd] & \resource{object file} \\ \variable{ECSIMPORT} \ar[ru] & \resource{symbol\\files} \ar@/r/[u] & \resource{debugging\\information}}
\seeoberon\seeassembly\seemmix\seeobject\seedebugging
}

\providecommand{\oborok}{
\toolsection{obor1k} is a compiler for the Oberon programming language targeting the OpenRISC 1000 hardware architecture.
It generates machine code for OpenRISC 1000 processors from modules written in Oberon and stores it in corresponding object files.
For debugging purposes, it also creates a debugging information file as well as an assembly file containing a listing of the generated machine code.
In addition, it stores the interface of each module in a symbol file which is required when other modules import the module.
Programs generated with this compiler require additional runtime support that is stored in the \file{ob\-or1k\-run} library file.
\flowgraph{\resource{Oberon\\source code} \ar[r] & \toolbox{obor1k} \ar[r] \ar@/l/[d] \ar[rd] & \resource{object file} \\ \variable{ECSIMPORT} \ar[ru] & \resource{symbol\\files} \ar@/r/[u] & \resource{debugging\\information}}
\seeoberon\seeassembly\seeorok\seeobject\seedebugging
}

\providecommand{\obppca}{
\toolsection{obppc32} is a compiler for the Oberon programming language targeting the PowerPC hardware architecture.
It generates machine code for PowerPC processors from modules written in Oberon and stores it in corresponding object files.
The compiler generates machine code for the 32-bit operating mode defined by the PowerPC architecture.
For debugging purposes, it also creates a debugging information file as well as an assembly file containing a listing of the generated machine code.
In addition, it stores the interface of each module in a symbol file which is required when other modules import the module.
Programs generated with this compiler require additional runtime support that is stored in the \file{ob\-ppc32\-run} library file.
\flowgraph{\resource{Oberon\\source code} \ar[r] & \toolbox{obppc32} \ar[r] \ar@/l/[d] \ar[rd] & \resource{object file} \\ \variable{ECSIMPORT} \ar[ru] & \resource{symbol\\files} \ar@/r/[u] & \resource{debugging\\information}}
\seeoberon\seeassembly\seeppc\seeobject\seedebugging
}

\providecommand{\obppcb}{
\toolsection{obppc64} is a compiler for the Oberon programming language targeting the PowerPC hardware architecture.
It generates machine code for PowerPC processors from modules written in Oberon and stores it in corresponding object files.
The compiler generates machine code for the 64-bit operating mode defined by the PowerPC architecture.
For debugging purposes, it also creates a debugging information file as well as an assembly file containing a listing of the generated machine code.
In addition, it stores the interface of each module in a symbol file which is required when other modules import the module.
Programs generated with this compiler require additional runtime support that is stored in the \file{ob\-ppc64\-run} library file.
\flowgraph{\resource{Oberon\\source code} \ar[r] & \toolbox{obppc64} \ar[r] \ar@/l/[d] \ar[rd] & \resource{object file} \\ \variable{ECSIMPORT} \ar[ru] & \resource{symbol\\files} \ar@/r/[u] & \resource{debugging\\information}}
\seeoberon\seeassembly\seeppc\seeobject\seedebugging
}

\providecommand{\obrisc}{
\toolsection{obrisc} is a compiler for the Oberon programming language targeting the RISC hardware architecture.
It generates machine code for RISC processors from modules written in Oberon and stores it in corresponding object files.
For debugging purposes, it also creates a debugging information file as well as an assembly file containing a listing of the generated machine code.
In addition, it stores the interface of each module in a symbol file which is required when other modules import the module.
Programs generated with this compiler require additional runtime support that is stored in the \file{ob\-risc\-run} library file.
\flowgraph{\resource{Oberon\\source code} \ar[r] & \toolbox{obrisc} \ar[r] \ar@/l/[d] \ar[rd] & \resource{object file} \\ \variable{ECSIMPORT} \ar[ru] & \resource{symbol\\files} \ar@/r/[u] & \resource{debugging\\information}}
\seeoberon\seeassembly\seerisc\seeobject\seedebugging
}

\providecommand{\obwasm}{
\toolsection{obwasm} is a compiler for the Oberon programming language targeting the WebAssembly architecture.
It generates machine code for WebAssembly targets from modules written in Oberon and stores it in corresponding object files.
For debugging purposes, it also creates a debugging information file as well as an assembly file containing a listing of the generated machine code.
In addition, it stores the interface of each module in a symbol file which is required when other modules import the module.
Programs generated with this compiler require additional runtime support that is stored in the \file{ob\-wasm\-run} library file.
\flowgraph{\resource{Oberon\\source code} \ar[r] & \toolbox{obwasm} \ar[r] \ar@/l/[d] \ar[rd] & \resource{object file} \\ \variable{ECSIMPORT} \ar[ru] & \resource{symbol\\files} \ar@/r/[u] & \resource{debugging\\information}}
\seeoberon\seeassembly\seewasm\seeobject\seedebugging
}

% converter tools

\providecommand{\dbgdwarf}{
\toolsection{dbgdwarf} is a DWARF debugging information converter tool.
It converts debugging information into the DWARF debugging data format and stores it in corresponding object files~\cite{dwarffile}.
The resulting debugging object files can be combined with runtime support that creates Executable and Linking Format (ELF) files~\cite{elffile}.
\flowgraph{\resource{debugging\\information} \ar[r] & \toolbox{dbgdwarf} \ar[r] & \resource{debugging\\object file}}
\seeobject\seedebugging
}

% assembler tools

\providecommand{\asmprint}{
\toolsection{asmprint} is a pretty printer for generic assembly code.
It reformats generic assembly code and writes it to the standard output stream.
\flowgraph{\resource{generic assembly\\source code} \ar[r] & \toolbox{asmprint} \ar[r] & \resource{reformatted\\source code}}
\seeassembly
}

\providecommand{\amdaasm}{
\toolsection{amd16asm} is an assembler for the AMD64 hardware architecture.
It translates assembly code into machine code for AMD64 processors and stores it in corresponding object files.
By default, the assembler generates machine code for the 16-bit operating mode defined by the AMD64 architecture.
\flowgraph{\resource{AMD16 assembly\\source code} \ar[r] & \toolbox{amd16asm} \ar[r] & \resource{object file}}
\seeassembly\seeamd\seeobject
}

\providecommand{\amdadism}{
\toolsection{amd16dism} is a disassembler for the AMD64 hardware architecture.
It translates machine code from object files targeting AMD64 processors into assembly code and writes it to the standard output stream.
It assumes that the machine code was generated for the 16-bit operating mode defined by the AMD64 architecture.
\flowgraph{\resource{object file} \ar[r] & \toolbox{amd16dism} \ar[r] & \resource{disassembly\\listing}}
\seeassembly\seeamd\seeobject
}

\providecommand{\amdbasm}{
\toolsection{amd32asm} is an assembler for the AMD64 hardware architecture.
It translates assembly code into machine code for AMD64 processors and stores it in corresponding object files.
By default, the assembler generates machine code for the 32-bit operating mode defined by the AMD64 architecture.
\flowgraph{\resource{AMD32 assembly\\source code} \ar[r] & \toolbox{amd32asm} \ar[r] & \resource{object file}}
\seeassembly\seeamd\seeobject
}

\providecommand{\amdbdism}{
\toolsection{amd32dism} is a disassembler for the AMD64 hardware architecture.
It translates machine code from object files targeting AMD64 processors into assembly code and writes it to the standard output stream.
It assumes that the machine code was generated for the 32-bit operating mode defined by the AMD64 architecture.
\flowgraph{\resource{object file} \ar[r] & \toolbox{amd32dism} \ar[r] & \resource{disassembly\\listing}}
\seeassembly\seeamd\seeobject
}

\providecommand{\amdcasm}{
\toolsection{amd64asm} is an assembler for the AMD64 hardware architecture.
It translates assembly code into machine code for AMD64 processors and stores it in corresponding object files.
By default, the assembler generates machine code for the 64-bit operating mode defined by the AMD64 architecture.
\flowgraph{\resource{AMD64 assembly\\source code} \ar[r] & \toolbox{amd64asm} \ar[r] & \resource{object file}}
\seeassembly\seeamd\seeobject
}

\providecommand{\amdcdism}{
\toolsection{amd64dism} is a disassembler for the AMD64 hardware architecture.
It translates machine code from object files targeting AMD64 processors into assembly code and writes it to the standard output stream.
It assumes that the machine code was generated for the 64-bit operating mode defined by the AMD64 architecture.
\flowgraph{\resource{object file} \ar[r] & \toolbox{amd64dism} \ar[r] & \resource{disassembly\\listing}}
\seeassembly\seeamd\seeobject
}

\providecommand{\armaasm}{
\toolsection{arma32asm} is an assembler for the ARM hardware architecture.
It translates assembly code into machine code for ARM processors executing A32 instructions and stores it in corresponding object files.
\flowgraph{\resource{ARM A32 assembly\\source code} \ar[r] & \toolbox{arma32asm} \ar[r] & \resource{object file}}
\seeassembly\seearm\seeobject
}

\providecommand{\armadism}{
\toolsection{arma32dism} is a disassembler for the ARM hardware architecture.
It translates machine code from object files targeting ARM processors executing A32 instructions into assembly code and writes it to the standard output stream.
\flowgraph{\resource{object file} \ar[r] & \toolbox{arma32dism} \ar[r] & \resource{disassembly\\listing}}
\seeassembly\seearm\seeobject
}

\providecommand{\armbasm}{
\toolsection{arma64asm} is an assembler for the ARM hardware architecture.
It translates assembly code into machine code for ARM processors executing A64 instructions and stores it in corresponding object files.
\flowgraph{\resource{ARM A64 assembly\\source code} \ar[r] & \toolbox{arma64asm} \ar[r] & \resource{object file}}
\seeassembly\seearm\seeobject
}

\providecommand{\armbdism}{
\toolsection{arma64dism} is a disassembler for the ARM hardware architecture.
It translates machine code from object files targeting ARM processors executing A64 instructions into assembly code and writes it to the standard output stream.
\flowgraph{\resource{object file} \ar[r] & \toolbox{arma64dism} \ar[r] & \resource{disassembly\\listing}}
\seeassembly\seearm\seeobject
}

\providecommand{\armcasm}{
\toolsection{armt32asm} is an assembler for the ARM hardware architecture.
It translates assembly code into machine code for ARM processors executing T32 instructions and stores it in corresponding object files.
\flowgraph{\resource{ARM T32 assembly\\source code} \ar[r] & \toolbox{armt32asm} \ar[r] & \resource{object file}}
\seeassembly\seearm\seeobject
}

\providecommand{\armcdism}{
\toolsection{armt32dism} is a disassembler for the ARM hardware architecture.
It translates machine code from object files targeting ARM processors executing T32 instructions into assembly code and writes it to the standard output stream.
\flowgraph{\resource{object file} \ar[r] & \toolbox{armt32dism} \ar[r] & \resource{disassembly\\listing}}
\seeassembly\seearm\seeobject
}

\providecommand{\avrasm}{
\toolsection{avrasm} is an assembler for the AVR hardware architecture.
It translates assembly code into machine code for AVR processors and stores it in corresponding object files.
The identifiers \texttt{RXL}, \texttt{RXH}, \texttt{RYL}, \texttt{RYH}, \texttt{RZL}, and \texttt{RZH} are predefined and name the corresponding registers.
The identifiers \texttt{SPL} and \texttt{SPH} are also predefined and evaluate to the address of the corresponding registers.
\flowgraph{\resource{AVR assembly\\source code} \ar[r] & \toolbox{avrasm} \ar[r] & \resource{object file}}
\seeassembly\seeavr\seeobject
}

\providecommand{\avrdism}{
\toolsection{avrdism} is a disassembler for the AVR hardware architecture.
It translates machine code from object files targeting AVR processors into assembly code and writes it to the standard output stream.
\flowgraph{\resource{object file} \ar[r] & \toolbox{avrdism} \ar[r] & \resource{disassembly\\listing}}
\seeassembly\seeavr\seeobject
}

\providecommand{\avrttasm}{
\toolsection{avr32asm} is an assembler for the AVR32 hardware architecture.
It translates assembly code into machine code for AVR32 processors and stores it in corresponding object files.
\flowgraph{\resource{AVR32 assembly\\source code} \ar[r] & \toolbox{avr32asm} \ar[r] & \resource{object file}}
\seeassembly\seeavrtt\seeobject
}

\providecommand{\avrttdism}{
\toolsection{avr32dism} is a disassembler for the AVR32 hardware architecture.
It translates machine code from object files targeting AVR32 processors into assembly code and writes it to the standard output stream.
\flowgraph{\resource{object file} \ar[r] & \toolbox{avr32dism} \ar[r] & \resource{disassembly\\listing}}
\seeassembly\seeavrtt\seeobject
}

\providecommand{\mabkasm}{
\toolsection{m68kasm} is an assembler for the M68000 hardware architecture.
It translates assembly code into machine code for M68000 processors and stores it in corresponding object files.
\flowgraph{\resource{68000 assembly\\source code} \ar[r] & \toolbox{m68kasm} \ar[r] & \resource{object file}}
\seeassembly\seemabk\seeobject
}

\providecommand{\mabkdism}{
\toolsection{m68kdism} is a disassembler for the M68000 hardware architecture.
It translates machine code from object files targeting M68000 processors into assembly code and writes it to the standard output stream.
\flowgraph{\resource{object file} \ar[r] & \toolbox{m68kdism} \ar[r] & \resource{disassembly\\listing}}
\seeassembly\seemabk\seeobject
}

\providecommand{\miblasm}{
\toolsection{miblasm} is an assembler for the MicroBlaze hardware architecture.
It translates assembly code into machine code for MicroBlaze processors and stores it in corresponding object files.
\flowgraph{\resource{MicroBlaze assembly\\source code} \ar[r] & \toolbox{miblasm} \ar[r] & \resource{object file}}
\seeassembly\seemibl\seeobject
}

\providecommand{\mibldism}{
\toolsection{mibldism} is a disassembler for the MicroBlaze hardware architecture.
It translates machine code from object files targeting MicroBlaze processors into assembly code and writes it to the standard output stream.
\flowgraph{\resource{object file} \ar[r] & \toolbox{mibldism} \ar[r] & \resource{disassembly\\listing}}
\seeassembly\seemibl\seeobject
}

\providecommand{\mipsaasm}{
\toolsection{mips32asm} is an assembler for the MIPS32 hardware architecture.
It translates assembly code into machine code for MIPS32 processors and stores it in corresponding object files.
\flowgraph{\resource{MIPS32 assembly\\source code} \ar[r] & \toolbox{mips32asm} \ar[r] & \resource{object file}}
\seeassembly\seemips\seeobject
}

\providecommand{\mipsadism}{
\toolsection{mips32dism} is a disassembler for the MIPS32 hardware architecture.
It translates machine code from object files targeting MIPS32 processors into assembly code and writes it to the standard output stream.
\flowgraph{\resource{object file} \ar[r] & \toolbox{mips32dism} \ar[r] & \resource{disassembly\\listing}}
\seeassembly\seemips\seeobject
}

\providecommand{\mipsbasm}{
\toolsection{mips64asm} is an assembler for the MIPS64 hardware architecture.
It translates assembly code into machine code for MIPS64 processors and stores it in corresponding object files.
\flowgraph{\resource{MIPS64 assembly\\source code} \ar[r] & \toolbox{mips64asm} \ar[r] & \resource{object file}}
\seeassembly\seemips\seeobject
}

\providecommand{\mipsbdism}{
\toolsection{mips64dism} is a disassembler for the MIPS64 hardware architecture.
It translates machine code from object files targeting MIPS64 processors into assembly code and writes it to the standard output stream.
\flowgraph{\resource{object file} \ar[r] & \toolbox{mips64dism} \ar[r] & \resource{disassembly\\listing}}
\seeassembly\seemips\seeobject
}

\providecommand{\mmixasm}{
\toolsection{mmixasm} is an assembler for the MMIX hardware architecture.
It translates assembly code into machine code for MMIX processors and stores it in corresponding object files.
The names of all special registers are predefined and evaluate to the corresponding number.
\flowgraph{\resource{MMIX assembly\\source code} \ar[r] & \toolbox{mmixasm} \ar[r] & \resource{object file}}
\seeassembly\seemmix\seeobject
}

\providecommand{\mmixdism}{
\toolsection{mmixdism} is a disassembler for the MMIX hardware architecture.
It translates machine code from object files targeting MMIX processors into assembly code and writes it to the standard output stream.
\flowgraph{\resource{object file} \ar[r] & \toolbox{mmixdism} \ar[r] & \resource{disassembly\\listing}}
\seeassembly\seemmix\seeobject
}

\providecommand{\orokasm}{
\toolsection{or1kasm} is an assembler for the OpenRISC 1000 hardware architecture.
It translates assembly code into machine code for OpenRISC 1000 processors and stores it in corresponding object files.
\flowgraph{\resource{OpenRISC 1000 assembly\\source code} \ar[r] & \toolbox{or1kasm} \ar[r] & \resource{object file}}
\seeassembly\seeorok\seeobject
}

\providecommand{\orokdism}{
\toolsection{or1kdism} is a disassembler for the OpenRISC 1000 hardware architecture.
It translates machine code from object files targeting OpenRISC 1000 processors into assembly code and writes it to the standard output stream.
\flowgraph{\resource{object file} \ar[r] & \toolbox{or1kdism} \ar[r] & \resource{disassembly\\listing}}
\seeassembly\seeorok\seeobject
}

\providecommand{\ppcaasm}{
\toolsection{ppc32asm} is an assembler for the PowerPC hardware architecture.
It translates assembly code into machine code for PowerPC processors and stores it in corresponding object files.
By default, the assembler generates machine code for the 32-bit operating mode defined by the PowerPC architecture.
\flowgraph{\resource{PowerPC assembly\\source code} \ar[r] & \toolbox{ppc32asm} \ar[r] & \resource{object file}}
\seeassembly\seeppc\seeobject
}

\providecommand{\ppcadism}{
\toolsection{ppc32dism} is a disassembler for the PowerPC hardware architecture.
It translates machine code from object files targeting PowerPC processors into assembly code and writes it to the standard output stream.
It assumes that the machine code was generated for the 32-bit operating mode defined by the PowerPC architecture.
\flowgraph{\resource{object file} \ar[r] & \toolbox{ppc32dism} \ar[r] & \resource{disassembly\\listing}}
\seeassembly\seeppc\seeobject
}

\providecommand{\ppcbasm}{
\toolsection{ppc64asm} is an assembler for the PowerPC hardware architecture.
It translates assembly code into machine code for PowerPC processors and stores it in corresponding object files.
By default, the assembler generates machine code for the 64-bit operating mode defined by the PowerPC architecture.
\flowgraph{\resource{PowerPC assembly\\source code} \ar[r] & \toolbox{ppc64asm} \ar[r] & \resource{object file}}
\seeassembly\seeppc\seeobject
}

\providecommand{\ppcbdism}{
\toolsection{ppc64dism} is a disassembler for the PowerPC hardware architecture.
It translates machine code from object files targeting PowerPC processors into assembly code and writes it to the standard output stream.
It assumes that the machine code was generated for the 64-bit operating mode defined by the PowerPC architecture.
\flowgraph{\resource{object file} \ar[r] & \toolbox{ppc64dism} \ar[r] & \resource{disassembly\\listing}}
\seeassembly\seeppc\seeobject
}

\providecommand{\riscasm}{
\toolsection{riscasm} is an assembler for the RISC hardware architecture.
It translates assembly code into machine code for RISC processors and stores it in corresponding object files.
The names of all special registers are predefined and evaluate to the corresponding number.
\flowgraph{\resource{RISC assembly\\source code} \ar[r] & \toolbox{riscasm} \ar[r] & \resource{object file}}
\seeassembly\seerisc\seeobject
}

\providecommand{\riscdism}{
\toolsection{riscdism} is a disassembler for the RISC hardware architecture.
It translates machine code from object files targeting RISC processors into assembly code and writes it to the standard output stream.
\flowgraph{\resource{object file} \ar[r] & \toolbox{riscdism} \ar[r] & \resource{disassembly\\listing}}
\seeassembly\seerisc\seeobject
}

\providecommand{\wasmasm}{
\toolsection{wasmasm} is an assembler for the WebAssembly architecture.
It translates assembly code into machine code for WebAssembly targets and stores it in corresponding object files.
The names of all special registers are predefined and evaluate to the corresponding number.
\flowgraph{\resource{WebAssembly assembly\\source code} \ar[r] & \toolbox{wasmasm} \ar[r] & \resource{object file}}
\seeassembly\seewasm\seeobject
}

\providecommand{\wasmdism}{
\toolsection{wasmdism} is a disassembler for the WebAssembly architecture.
It translates machine code from object files targeting WebAssembly targets into assembly code and writes it to the standard output stream.
\flowgraph{\resource{object file} \ar[r] & \toolbox{wasmdism} \ar[r] & \resource{disassembly\\listing}}
\seeassembly\seewasm\seeobject
}

% linker tools

\providecommand{\linklib}{
\toolsection{linklib} is an object file combiner.
It creates a static library file by combining all object files given to it into a single one.
\flowgraph{\resource{object files} \ar[r] & \toolbox{linklib} \ar[r] & \resource{library file}}
\seeobject
}

\providecommand{\linkbin}{
\toolsection{linkbin} is a linker for plain binary files.
It links all object files given to it into a single image and stores it in a binary file that begins with the first linked section.
It also creates a map file that lists the address, type, name and size of all used sections.
The filename extension of the resulting binary file can be specified by putting it into a constant data section called \texttt{\_extension}.
\flowgraph{\resource{object files} \ar[r] & \toolbox{linkbin} \ar[r] \ar[d] & \resource{binary file} \\ & \resource{map file}}
\seeobject
}

\providecommand{\linkmem}{
\toolsection{linkmem} is a linker for plain binary files partitioned into random-access and read-only memory.
It links all object files given to it into two distinct images, one for data sections and one for code and constant data sections, and stores each image in a binary file that begins with the first linked section of the corresponding type.
It also creates a map file that lists the address, type, name and size of all used sections.
\flowgraph{\resource{object files} \ar[r] & \toolbox{linkmem} \ar[r] \ar[d] & \resource{RAM file/\\ROM file} \\ & \resource{map file}}
\seeobject
}

\providecommand{\linkprg}{
\toolsection{linkprg} is a linker for GEMDOS executable files.
It links all object files given to it into a single image and stores the image in an Atari GEMDOS executable file~\cite{gemdosfile}.
It also creates a map file that lists the address relative to the text segment, type, name and size of all used sections.
The filename extension of the resulting executable file can be specified by putting it into a constant data section called \texttt{\_extension}.
The GEMDOS executable file format requires all patch patterns of absolute link patches to consist of four full bitmasks with descending offsets.
\flowgraph{\resource{object files} \ar[r] & \toolbox{linkprg} \ar[r] \ar[d] & \resource{executable file} \\ & \resource{map file}}
\seeobject
}

\providecommand{\linkhex}{
\toolsection{linkhex} is a linker for Intel HEX files.
It links all code sections of the object files given to it into single image and stores the image in an Intel HEX file~\cite{hexfile} that begins with the first linked section.
It also creates a map file that lists the address, type, name and size of all used sections.
\flowgraph{\resource{object files} \ar[r] & \toolbox{linkhex} \ar[r] \ar[d] & \resource{HEX file} \\ & \resource{map file}}
\seeobject
}

\providecommand{\mapsearch}{
\toolsection{mapsearch} is a debugging tool.
It searches map files generated by linker tools for the name of a binary section that encompasses a memory address read from the standard input stream.
If additionally provided with one or more object files, it also stores an excerpt thereof in a separate object file called map search result which only contains the identified binary section for disassembling purposes.
\flowgraph{& \resource{map files/\\object files} \ar[d] \\ \resource{memory\\address} \ar[r] & \toolbox{mapsearch} \ar[r] \ar[d] & \resource{section name/\\relative offset} \\ & \resource{object file\\excerpt}}
\seeobject
}


\startchapter{GNU General~Public~License}{GNU General Public License}{gpl}{}

\begin{center}

Version 3, 29 June 2007

\medskip
Copyright \copyright{} 2007 Free Software Foundation, Inc. \url{https://fsf.org/}

\end{center}

Everyone is permitted to copy and distribute verbatim copies
of this license document, but changing it is not allowed.

\setcounter{section}{-1}
\renewcommand{\thesection}{\arabic{section}.}

\section*{\centering Preamble}

The GNU General Public License is a free, copyleft license for
software and other kinds of works.

The licenses for most software and other practical works are designed
to take away your freedom to share and change the works. By contrast,
the GNU General Public License is intended to guarantee your freedom to
share and change all versions of a program--to make sure it remains free
software for all its users. We, the Free Software Foundation, use the
GNU General Public License for most of our software; it applies also to
any other work released this way by its authors. You can apply it to
your programs, too.

When we speak of free software, we are referring to freedom, not
price. Our General Public Licenses are designed to make sure that you
have the freedom to distribute copies of free software (and charge for
them if you wish), that you receive source code or can get it if you
want it, that you can change the software or use pieces of it in new
free programs, and that you know you can do these things.

To protect your rights, we need to prevent others from denying you
these rights or asking you to surrender the rights. Therefore, you have
certain responsibilities if you distribute copies of the software, or if
you modify it: responsibilities to respect the freedom of others.

For example, if you distribute copies of such a program, whether
gratis or for a fee, you must pass on to the recipients the same
freedoms that you received. You must make sure that they, too, receive
or can get the source code. And you must show them these terms so they
know their rights.

Developers that use the GNU GPL protect your rights with two steps:
(1) assert copyright on the software, and (2) offer you this License
giving you legal permission to copy, distribute and/or modify it.

For the developers' and authors' protection, the GPL clearly explains
that there is no warranty for this free software. For both users' and
authors' sake, the GPL requires that modified versions be marked as
changed, so that their problems will not be attributed erroneously to
authors of previous versions.

Some devices are designed to deny users access to install or run
modified versions of the software inside them, although the manufacturer
can do so. This is fundamentally incompatible with the aim of
protecting users' freedom to change the software. The systematic
pattern of such abuse occurs in the area of products for individuals to
use, which is precisely where it is most unacceptable. Therefore, we
have designed this version of the GPL to prohibit the practice for those
products. If such problems arise substantially in other domains, we
stand ready to extend this provision to those domains in future versions
of the GPL, as needed to protect the freedom of users.

Finally, every program is threatened constantly by software patents.
States should not allow patents to restrict development and use of
software on general-purpose computers, but in those that do, we wish to
avoid the special danger that patents applied to a free program could
make it effectively proprietary. To prevent this, the GPL assures that
patents cannot be used to render the program non-free.

The precise terms and conditions for copying, distribution and
modification follow.

\section*{\centering TERMS AND CONDITIONS}

\section{Definitions}

``This License'' refers to version 3 of the GNU General Public License.

``Copyright'' also means copyright-like laws that apply to other kinds of
works, such as semiconductor masks.

``The Program'' refers to any copyrightable work licensed under this
License. Each licensee is addressed as ``you''. ``Licensees'' and
``recipients'' may be individuals or organizations.

To ``modify'' a work means to copy from or adapt all or part of the work
in a fashion requiring copyright permission, other than the making of an
exact copy. The resulting work is called a ``modified version'' of the
earlier work or a work ``based on'' the earlier work.

A ``covered work'' means either the unmodified Program or a work based
on the Program.

To ``propagate'' a work means to do anything with it that, without
permission, would make you directly or secondarily liable for
infringement under applicable copyright law, except executing it on a
computer or modifying a private copy. Propagation includes copying,
distribution (with or without modification), making available to the
public, and in some countries other activities as well.

To ``convey'' a work means any kind of propagation that enables other
parties to make or receive copies. Mere interaction with a user through
a computer network, with no transfer of a copy, is not conveying.

An interactive user interface displays ``Appropriate Legal Notices''
to the extent that it includes a convenient and prominently visible
feature that (1) displays an appropriate copyright notice, and (2)
tells the user that there is no warranty for the work (except to the
extent that warranties are provided), that licensees may convey the
work under this License, and how to view a copy of this License. If
the interface presents a list of user commands or options, such as a
menu, a prominent item in the list meets this criterion.

\section{Source Code}

The ``source code'' for a work means the preferred form of the work
for making modifications to it. ``Object code'' means any non-source
form of a work.

A ``Standard Interface'' means an interface that either is an official
standard defined by a recognized standards body, or, in the case of
interfaces specified for a particular programming language, one that
is widely used among developers working in that language.

The ``System Libraries'' of an executable work include anything, other
than the work as a whole, that (a) is included in the normal form of
packaging a Major Component, but which is not part of that Major
Component, and (b) serves only to enable use of the work with that
Major Component, or to implement a Standard Interface for which an
implementation is available to the public in source code form. A
``Major Component'', in this context, means a major essential component
(kernel, window system, and so on) of the specific operating system
(if any) on which the executable work runs, or a compiler used to
produce the work, or an object code interpreter used to run it.

The ``Corresponding Source'' for a work in object code form means all
the source code needed to generate, install, and (for an executable
work) run the object code and to modify the work, including scripts to
control those activities. However, it does not include the work's
System Libraries, or general-purpose tools or generally available free
programs which are used unmodified in performing those activities but
which are not part of the work. For example, Corresponding Source
includes interface definition files associated with source files for
the work, and the source code for shared libraries and dynamically
linked subprograms that the work is specifically designed to require,
such as by intimate data communication or control flow between those
subprograms and other parts of the work.

The Corresponding Source need not include anything that users
can regenerate automatically from other parts of the Corresponding
Source.

The Corresponding Source for a work in source code form is that
same work.

\section{Basic Permissions}

All rights granted under this License are granted for the term of
copyright on the Program, and are irrevocable provided the stated
conditions are met. This License explicitly affirms your unlimited
permission to run the unmodified Program. The output from running a
covered work is covered by this License only if the output, given its
content, constitutes a covered work. This License acknowledges your
rights of fair use or other equivalent, as provided by copyright law.

You may make, run and propagate covered works that you do not
convey, without conditions so long as your license otherwise remains
in force. You may convey covered works to others for the sole purpose
of having them make modifications exclusively for you, or provide you
with facilities for running those works, provided that you comply with
the terms of this License in conveying all material for which you do
not control copyright. Those thus making or running the covered works
for you must do so exclusively on your behalf, under your direction
and control, on terms that prohibit them from making any copies of
your copyrighted material outside their relationship with you.

Conveying under any other circumstances is permitted solely under
the conditions stated below. Sublicensing is not allowed; section 10
makes it unnecessary.

\section{Protecting Users' Legal Rights From Anti-Circumvention Law}

No covered work shall be deemed part of an effective technological
measure under any applicable law fulfilling obligations under article
11 of the WIPO copyright treaty adopted on 20 December 1996, or
similar laws prohibiting or restricting circumvention of such
measures.

When you convey a covered work, you waive any legal power to forbid
circumvention of technological measures to the extent such circumvention
is effected by exercising rights under this License with respect to
the covered work, and you disclaim any intention to limit operation or
modification of the work as a means of enforcing, against the work's
users, your or third parties' legal rights to forbid circumvention of
technological measures.

\section{Conveying Verbatim Copies}

You may convey verbatim copies of the Program's source code as you
receive it, in any medium, provided that you conspicuously and
appropriately publish on each copy an appropriate copyright notice;
keep intact all notices stating that this License and any
non-permissive terms added in accord with section 7 apply to the code;
keep intact all notices of the absence of any warranty; and give all
recipients a copy of this License along with the Program.

You may charge any price or no price for each copy that you convey,
and you may offer support or warranty protection for a fee.

\section{Conveying Modified Source Versions}

You may convey a work based on the Program, or the modifications to
produce it from the Program, in the form of source code under the
terms of section 4, provided that you also meet all of these conditions:

\begin{itemize}

\item[a)]
The work must carry prominent notices stating that you modified
it, and giving a relevant date.

\item[b)]
The work must carry prominent notices stating that it is
released under this License and any conditions added under section
7. This requirement modifies the requirement in section 4 to
``keep intact all notices''.

\item[c)]
You must license the entire work, as a whole, under this
License to anyone who comes into possession of a copy. This
License will therefore apply, along with any applicable section 7
additional terms, to the whole of the work, and all its parts,
regardless of how they are packaged. This License gives no
permission to license the work in any other way, but it does not
invalidate such permission if you have separately received it.

\item[d)]
If the work has interactive user interfaces, each must display
Appropriate Legal Notices; however, if the Program has interactive
interfaces that do not display Appropriate Legal Notices, your
work need not make them do so.

\end{itemize}

A compilation of a covered work with other separate and independent
works, which are not by their nature extensions of the covered work,
and which are not combined with it such as to form a larger program,
in or on a volume of a storage or distribution medium, is called an
``aggregate'' if the compilation and its resulting copyright are not
used to limit the access or legal rights of the compilation's users
beyond what the individual works permit. Inclusion of a covered work
in an aggregate does not cause this License to apply to the other
parts of the aggregate.

\section{Conveying Non-Source Forms}

You may convey a covered work in object code form under the terms
of sections 4 and 5, provided that you also convey the
machine-readable Corresponding Source under the terms of this License,
in one of these ways:

\begin{itemize}

\item[a)]
Convey the object code in, or embodied in, a physical product
(including a physical distribution medium), accompanied by the
Corresponding Source fixed on a durable physical medium
customarily used for software interchange.

\item[b)]
Convey the object code in, or embodied in, a physical product
(including a physical distribution medium), accompanied by a
written offer, valid for at least three years and valid for as
long as you offer spare parts or customer support for that product
model, to give anyone who possesses the object code either (1) a
copy of the Corresponding Source for all the software in the
product that is covered by this License, on a durable physical
medium customarily used for software interchange, for a price no
more than your reasonable cost of physically performing this
conveying of source, or (2) access to copy the
Corresponding Source from a network server at no charge.

\item[c)]
Convey individual copies of the object code with a copy of the
written offer to provide the Corresponding Source. This
alternative is allowed only occasionally and noncommercially, and
only if you received the object code with such an offer, in accord
with subsection 6b.

\item[d)]
Convey the object code by offering access from a designated
place (gratis or for a charge), and offer equivalent access to the
Corresponding Source in the same way through the same place at no
further charge. You need not require recipients to copy the
Corresponding Source along with the object code. If the place to
copy the object code is a network server, the Corresponding Source
may be on a different server (operated by you or a third party)
that supports equivalent copying facilities, provided you maintain
clear directions next to the object code saying where to find the
Corresponding Source. Regardless of what server hosts the
Corresponding Source, you remain obligated to ensure that it is
available for as long as needed to satisfy these requirements.

\item[e)]
Convey the object code using peer-to-peer transmission, provided
you inform other peers where the object code and Corresponding
Source of the work are being offered to the general public at no
charge under subsection 6d.

\end{itemize}

A separable portion of the object code, whose source code is excluded
from the Corresponding Source as a System Library, need not be
included in conveying the object code work.

A ``User Product'' is either (1) a ``consumer product'', which means any
tangible personal property which is normally used for personal, family,
or household purposes, or (2) anything designed or sold for incorporation
into a dwelling. In determining whether a product is a consumer product,
doubtful cases shall be resolved in favor of coverage. For a particular
product received by a particular user, ``normally used'' refers to a
typical or common use of that class of product, regardless of the status
of the particular user or of the way in which the particular user
actually uses, or expects or is expected to use, the product. A product
is a consumer product regardless of whether the product has substantial
commercial, industrial or non-consumer uses, unless such uses represent
the only significant mode of use of the product.

``Installation Information'' for a User Product means any methods,
procedures, authorization keys, or other information required to install
and execute modified versions of a covered work in that User Product from
a modified version of its Corresponding Source. The information must
suffice to ensure that the continued functioning of the modified object
code is in no case prevented or interfered with solely because
modification has been made.

If you convey an object code work under this section in, or with, or
specifically for use in, a User Product, and the conveying occurs as
part of a transaction in which the right of possession and use of the
User Product is transferred to the recipient in perpetuity or for a
fixed term (regardless of how the transaction is characterized), the
Corresponding Source conveyed under this section must be accompanied
by the Installation Information. But this requirement does not apply
if neither you nor any third party retains the ability to install
modified object code on the User Product (for example, the work has
been installed in ROM).

The requirement to provide Installation Information does not include a
requirement to continue to provide support service, warranty, or updates
for a work that has been modified or installed by the recipient, or for
the User Product in which it has been modified or installed. Access to a
network may be denied when the modification itself materially and
adversely affects the operation of the network or violates the rules and
protocols for communication across the network.

Corresponding Source conveyed, and Installation Information provided,
in accord with this section must be in a format that is publicly
documented (and with an implementation available to the public in
source code form), and must require no special password or key for
unpacking, reading or copying.

\section{Additional Terms}

``Additional permissions'' are terms that supplement the terms of this
License by making exceptions from one or more of its conditions.
Additional permissions that are applicable to the entire Program shall
be treated as though they were included in this License, to the extent
that they are valid under applicable law. If additional permissions
apply only to part of the Program, that part may be used separately
under those permissions, but the entire Program remains governed by
this License without regard to the additional permissions.

When you convey a copy of a covered work, you may at your option
remove any additional permissions from that copy, or from any part of
it. (Additional permissions may be written to require their own
removal in certain cases when you modify the work.) You may place
additional permissions on material, added by you to a covered work,
for which you have or can give appropriate copyright permission.

Notwithstanding any other provision of this License, for material you
add to a covered work, you may (if authorized by the copyright holders of
that material) supplement the terms of this License with terms:

\begin{itemize}

\item[a)]
Disclaiming warranty or limiting liability differently from the
terms of sections 15 and 16 of this License; or

\item[b)]
Requiring preservation of specified reasonable legal notices or
author attributions in that material or in the Appropriate Legal
Notices displayed by works containing it; or

\item[c)]
Prohibiting misrepresentation of the origin of that material, or
requiring that modified versions of such material be marked in
reasonable ways as different from the original version; or

\item[d)]
Limiting the use for publicity purposes of names of licensors or
authors of the material; or

\item[e)]
Declining to grant rights under trademark law for use of some
trade names, trademarks, or service marks; or

\item[f)]
Requiring indemnification of licensors and authors of that
material by anyone who conveys the material (or modified versions of
it) with contractual assumptions of liability to the recipient, for
any liability that these contractual assumptions directly impose on
those licensors and authors.

\end{itemize}

All other non-permissive additional terms are considered ``further
restrictions'' within the meaning of section 10. If the Program as you
received it, or any part of it, contains a notice stating that it is
governed by this License along with a term that is a further
restriction, you may remove that term. If a license document contains
a further restriction but permits relicensing or conveying under this
License, you may add to a covered work material governed by the terms
of that license document, provided that the further restriction does
not survive such relicensing or conveying.

If you add terms to a covered work in accord with this section, you
must place, in the relevant source files, a statement of the
additional terms that apply to those files, or a notice indicating
where to find the applicable terms.

Additional terms, permissive or non-permissive, may be stated in the
form of a separately written license, or stated as exceptions;
the above requirements apply either way.

\section{Termination}

You may not propagate or modify a covered work except as expressly
provided under this License. Any attempt otherwise to propagate or
modify it is void, and will automatically terminate your rights under
this License (including any patent licenses granted under the third
paragraph of section 11).

However, if you cease all violation of this License, then your
license from a particular copyright holder is reinstated (a)
provisionally, unless and until the copyright holder explicitly and
finally terminates your license, and (b) permanently, if the copyright
holder fails to notify you of the violation by some reasonable means
prior to 60 days after the cessation.

Moreover, your license from a particular copyright holder is
reinstated permanently if the copyright holder notifies you of the
violation by some reasonable means, this is the first time you have
received notice of violation of this License (for any work) from that
copyright holder, and you cure the violation prior to 30 days after
your receipt of the notice.

Termination of your rights under this section does not terminate the
licenses of parties who have received copies or rights from you under
this License. If your rights have been terminated and not permanently
reinstated, you do not qualify to receive new licenses for the same
material under section 10.

\section{Acceptance Not Required for Having Copies}

You are not required to accept this License in order to receive or
run a copy of the Program. Ancillary propagation of a covered work
occurring solely as a consequence of using peer-to-peer transmission
to receive a copy likewise does not require acceptance. However,
nothing other than this License grants you permission to propagate or
modify any covered work. These actions infringe copyright if you do
not accept this License. Therefore, by modifying or propagating a
covered work, you indicate your acceptance of this License to do so.

\section{Automatic Licensing of Downstream Recipients}

Each time you convey a covered work, the recipient automatically
receives a license from the original licensors, to run, modify and
propagate that work, subject to this License. You are not responsible
for enforcing compliance by third parties with this License.

An ``entity transaction'' is a transaction transferring control of an
organization, or substantially all assets of one, or subdividing an
organization, or merging organizations. If propagation of a covered
work results from an entity transaction, each party to that
transaction who receives a copy of the work also receives whatever
licenses to the work the party's predecessor in interest had or could
give under the previous paragraph, plus a right to possession of the
Corresponding Source of the work from the predecessor in interest, if
the predecessor has it or can get it with reasonable efforts.

You may not impose any further restrictions on the exercise of the
rights granted or affirmed under this License. For example, you may
not impose a license fee, royalty, or other charge for exercise of
rights granted under this License, and you may not initiate litigation
(including a cross-claim or counterclaim in a lawsuit) alleging that
any patent claim is infringed by making, using, selling, offering for
sale, or importing the Program or any portion of it.

\section{Patents}

A ``contributor'' is a copyright holder who authorizes use under this
License of the Program or a work on which the Program is based. The
work thus licensed is called the contributor's ``contributor version''.

A contributor's ``essential patent claims'' are all patent claims
owned or controlled by the contributor, whether already acquired or
hereafter acquired, that would be infringed by some manner, permitted
by this License, of making, using, or selling its contributor version,
but do not include claims that would be infringed only as a
consequence of further modification of the contributor version. For
purposes of this definition, ``control'' includes the right to grant
patent sublicenses in a manner consistent with the requirements of
this License.

Each contributor grants you a non-exclusive, worldwide, royalty-free
patent license under the contributor's essential patent claims, to
make, use, sell, offer for sale, import and otherwise run, modify and
propagate the contents of its contributor version.

In the following three paragraphs, a ``patent license'' is any express
agreement or commitment, however denominated, not to enforce a patent
(such as an express permission to practice a patent or covenant not to
sue for patent infringement). To ``grant'' such a patent license to a
party means to make such an agreement or commitment not to enforce a
patent against the party.

If you convey a covered work, knowingly relying on a patent license,
and the Corresponding Source of the work is not available for anyone
to copy, free of charge and under the terms of this License, through a
publicly available network server or other readily accessible means,
then you must either (1) cause the Corresponding Source to be so
available, or (2) arrange to deprive yourself of the benefit of the
patent license for this particular work, or (3) arrange, in a manner
consistent with the requirements of this License, to extend the patent
license to downstream recipients. ``Knowingly relying'' means you have
actual knowledge that, but for the patent license, your conveying the
covered work in a country, or your recipient's use of the covered work
in a country, would infringe one or more identifiable patents in that
country that you have reason to believe are valid.

If, pursuant to or in connection with a single transaction or
arrangement, you convey, or propagate by procuring conveyance of, a
covered work, and grant a patent license to some of the parties
receiving the covered work authorizing them to use, propagate, modify
or convey a specific copy of the covered work, then the patent license
you grant is automatically extended to all recipients of the covered
work and works based on it.

A patent license is ``discriminatory'' if it does not include within
the scope of its coverage, prohibits the exercise of, or is
conditioned on the non-exercise of one or more of the rights that are
specifically granted under this License. You may not convey a covered
work if you are a party to an arrangement with a third party that is
in the business of distributing software, under which you make payment
to the third party based on the extent of your activity of conveying
the work, and under which the third party grants, to any of the
parties who would receive the covered work from you, a discriminatory
patent license (a) in connection with copies of the covered work
conveyed by you (or copies made from those copies), or (b) primarily
for and in connection with specific products or compilations that
contain the covered work, unless you entered into that arrangement,
or that patent license was granted, prior to 28 March 2007.

Nothing in this License shall be construed as excluding or limiting
any implied license or other defenses to infringement that may
otherwise be available to you under applicable patent law.

\section{No Surrender of Others' Freedom}

If conditions are imposed on you (whether by court order, agreement or
otherwise) that contradict the conditions of this License, they do not
excuse you from the conditions of this License. If you cannot convey a
covered work so as to satisfy simultaneously your obligations under this
License and any other pertinent obligations, then as a consequence you may
not convey it at all. For example, if you agree to terms that obligate you
to collect a royalty for further conveying from those to whom you convey
the Program, the only way you could satisfy both those terms and this
License would be to refrain entirely from conveying the Program.

\section{Use with the GNU Affero General Public License}

Notwithstanding any other provision of this License, you have
permission to link or combine any covered work with a work licensed
under version 3 of the GNU Affero General Public License into a single
combined work, and to convey the resulting work. The terms of this
License will continue to apply to the part which is the covered work,
but the special requirements of the GNU Affero General Public License,
section 13, concerning interaction through a network will apply to the
combination as such.

\section{Revised Versions of this License}

The Free Software Foundation may publish revised and/or new versions of
the GNU General Public License from time to time. Such new versions will
be similar in spirit to the present version, but may differ in detail to
address new problems or concerns.

Each version is given a distinguishing version number. If the
Program specifies that a certain numbered version of the GNU General
Public License ``or any later version'' applies to it, you have the
option of following the terms and conditions either of that numbered
version or of any later version published by the Free Software
Foundation. If the Program does not specify a version number of the
GNU General Public License, you may choose any version ever published
by the Free Software Foundation.

If the Program specifies that a proxy can decide which future
versions of the GNU General Public License can be used, that proxy's
public statement of acceptance of a version permanently authorizes you
to choose that version for the Program.

Later license versions may give you additional or different
permissions. However, no additional obligations are imposed on any
author or copyright holder as a result of your choosing to follow a
later version.

\section{Disclaimer of Warranty}

\begin{sloppypar}
THERE IS NO WARRANTY FOR THE PROGRAM, TO THE EXTENT PERMITTED BY
APPLICABLE LAW. EXCEPT WHEN OTHERWISE STATED IN WRITING THE COPYRIGHT
HOLDERS AND/OR OTHER PARTIES PROVIDE THE PROGRAM ``AS IS'' WITHOUT WARRANTY
OF ANY KIND, EITHER EXPRESSED OR IMPLIED, INCLUDING, BUT NOT LIMITED TO,
THE IMPLIED WARRANTIES OF MERCHANTABILITY AND FITNESS FOR A PARTICULAR
PURPOSE. THE ENTIRE RISK AS TO THE QUALITY AND PERFORMANCE OF THE PROGRAM
IS WITH YOU. SHOULD THE PROGRAM PROVE DEFECTIVE, YOU ASSUME THE COST OF
ALL NECESSARY SERVICING, REPAIR OR CORRECTION.
\end{sloppypar}

\section{Limitation of Liability}

\begin{sloppypar}
IN NO EVENT UNLESS REQUIRED BY APPLICABLE LAW OR AGREED TO IN WRITING
WILL ANY COPYRIGHT HOLDER, OR ANY OTHER PARTY WHO MODIFIES AND/OR CONVEYS
THE PROGRAM AS PERMITTED ABOVE, BE LIABLE TO YOU FOR DAMAGES, INCLUDING ANY
GENERAL, SPECIAL, INCIDENTAL OR CONSEQUENTIAL DAMAGES ARISING OUT OF THE
USE OR INABILITY TO USE THE PROGRAM (INCLUDING BUT NOT LIMITED TO LOSS OF
DATA OR DATA BEING RENDERED INACCURATE OR LOSSES SUSTAINED BY YOU OR THIRD
PARTIES OR A FAILURE OF THE PROGRAM TO OPERATE WITH ANY OTHER PROGRAMS),
EVEN IF SUCH HOLDER OR OTHER PARTY HAS BEEN ADVISED OF THE POSSIBILITY OF
SUCH DAMAGES.
\end{sloppypar}

\section{Interpretation of Sections 15 and 16}

If the disclaimer of warranty and limitation of liability provided
above cannot be given local legal effect according to their terms,
reviewing courts shall apply local law that most closely approximates
an absolute waiver of all civil liability in connection with the
Program, unless a warranty or assumption of liability accompanies a
copy of the Program in return for a fee.

\section*{\centering END OF TERMS AND CONDITIONS}

\section*{\centering How to Apply These Terms to Your New Programs}

If you develop a new program, and you want it to be of the greatest
possible use to the public, the best way to achieve this is to make it
free software which everyone can redistribute and change under these terms.

To do so, attach the following notices to the program. It is safest
to attach them to the start of each source file to most effectively
state the exclusion of warranty; and each file should have at least
the ``copyright'' line and a pointer to where the full notice is found.

\footnotesize
\begin{verbatim}
    <one line to give the program's name and a brief idea of what it does.>
    Copyright (C) <year>  <name of author>

    This program is free software: you can redistribute it and/or modify
    it under the terms of the GNU General Public License as published by
    the Free Software Foundation, either version 3 of the License, or
    (at your option) any later version.

    This program is distributed in the hope that it will be useful,
    but WITHOUT ANY WARRANTY; without even the implied warranty of
    MERCHANTABILITY or FITNESS FOR A PARTICULAR PURPOSE.  See the
    GNU General Public License for more details.

    You should have received a copy of the GNU General Public License
    along with this program.  If not, see <https://www.gnu.org/licenses/>.
\end{verbatim}
\normalsize

Also add information on how to contact you by electronic and paper mail.
If the program does terminal interaction, make it output a short
notice like this when it starts in an interactive mode:

\footnotesize
\begin{verbatim}
    <program>  Copyright (C) <year>  <name of author>
    This program comes with ABSOLUTELY NO WARRANTY; for details type `show w'.
    This is free software, and you are welcome to redistribute it
    under certain conditions; type `show c' for details.
\end{verbatim}
\normalsize

The hypothetical commands \texttt{show w} and \texttt{show c} should show the appropriate
parts of the General Public License. Of course, your program's commands
might be different; for a GUI interface, you would use an ``about box''.

You should also get your employer (if you work as a programmer) or school,
if any, to sign a ``copyright disclaimer'' for the program, if necessary.
For more information on this, and how to apply and follow the GNU GPL, see
\url{https://www.gnu.org/licenses/}.

The GNU General Public License does not permit incorporating your program
into proprietary programs. If your program is a subroutine library, you
may consider it more useful to permit linking proprietary applications with
the library. If this is what you want to do, use the GNU Lesser General
Public License instead of this License. But first, please read
\url{https://www.gnu.org/licenses/why-not-lgpl.html}.

\concludechapter

% ECS Runtime Support Exception
% Copyright (C) Florian Negele

% This file is part of the Eigen Compiler Suite.

% Permission is granted to copy, distribute and/or modify this document
% under the terms of the GNU Free Documentation License, Version 1.3
% or any later version published by the Free Software Foundation.

% You should have received a copy of the GNU Free Documentation License
% along with the ECS.  If not, see <https://www.gnu.org/licenses/>.

% Generic documentation utilities
% Copyright (C) Florian Negele

% This file is part of the Eigen Compiler Suite.

% Permission is granted to copy, distribute and/or modify this document
% under the terms of the GNU Free Documentation License, Version 1.3
% or any later version published by the Free Software Foundation.

% You should have received a copy of the GNU Free Documentation License
% along with the ECS.  If not, see <https://www.gnu.org/licenses/>.

\providecommand{\cpp}{C\texttt{++}}
\providecommand{\opt}{_\mathit{opt}}
\providecommand{\tool}[1]{\texttt{#1}}
\providecommand{\version}{Version 0.0.40}
\providecommand{\resource}[1]{*++\txt{#1}}
\providecommand{\ecs}{Eigen Compiler Suite}
\providecommand{\changed}[1]{\underline{#1}}
\providecommand{\toolbox}[1]{\converter{#1}}
\providecommand{\file}{}\renewcommand{\file}[1]{\texttt{#1}}
\providecommand{\alignright}{\hfill\linebreak[0]\hspace*{\fill}}
\providecommand{\converter}[1]{*++[F][F*:white][F,:gray]\txt{#1}}
\providecommand{\documentation}{\ifbook chapter\else document\fi}
\providecommand{\Documentation}{\ifbook Chapter\else Document\fi}
\providecommand{\variable}[1]{\resource{\texttt{\small#1}\\variable}}
\providecommand{\documentationref}[2]{\ifbook\ref{#1}\else``\href{#1}{#2}''~\cite{#1}\fi}
\providecommand{\objfile}[1]{\texttt{#1}\index[runtime]{#1 object file@\texttt{#1} object file}}
\providecommand{\libfile}[1]{\texttt{#1}\index[runtime]{#1 library file@\texttt{#1} library file}}
\providecommand{\epigraph}[2]{\ifbook\begin{quote}\flushright\textit{#1}\par--- #2\end{quote}\fi}
\providecommand{\environmentvariable}[1]{\texttt{#1}\index{Environment variables!#1@\texttt{#1}}}
\providecommand{\environment}[1]{\texttt{#1}\index[environment]{#1 environment@\texttt{#1} environment}}
\providecommand{\toolsection}{}\renewcommand{\toolsection}[1]{\subsection{#1}\label{\prefix:#1}\tool{#1}}
\providecommand{\instruction}{}\renewcommand{\instruction}[2]{\noindent\qquad\pdftooltip{\texttt{#1}}{#2}\refstepcounter{instruction}\par}
\providecommand{\flowgraph}{}\renewcommand{\flowgraph}[1]{\par\sffamily\begin{displaymath}\xymatrix@=4ex{#1}\end{displaymath}\normalfont\par}
\providecommand{\instructionset}{}\renewcommand{\instructionset}[4]{\setcounter{instruction}{0}\begin{multicols}{\ifbook#3\else#4\fi}[{\captionof{table}[#2]{#2 (\ref*{#1:instructions}~instructions)}\label{tab:#1set}\vspace{-2ex}}]\footnotesize\raggedcolumns\input{#1.set}\label{#1:instructions}\end{multicols}}

\providecommand{\gpl}{GNU General Public License}
\providecommand{\rse}{ECS Runtime Support Exception}
\providecommand{\fdl}{\href{https://www.gnu.org/licenses/fdl.html}{GNU Free Documentation License}}

\providecommand{\docbegin}{}
\providecommand{\docend}{}
\providecommand{\doclabel}[1]{\hypertarget{#1}}
\providecommand{\doclink}[2]{\hyperlink{#1}{#2}}
\providecommand{\docsection}[3]{\hypertarget{#1}{\subsection{#2}}\label{sec:#1}\index[library]{#2@#3}}
\providecommand{\docsectionstar}[1]{}
\providecommand{\docsubbegin}{\begin{description}}
\providecommand{\docsubend}{\end{description}}
\providecommand{\docsubsection}[3]{\item[\hypertarget{#1}{#2}]\index[library]{#2@#3}}
\providecommand{\docsubsectionstar}[1]{\smallskip}
\providecommand{\docsubsubsection}[3]{\docsubsection{#1}{#2}{#3}}
\providecommand{\docsubsubsectionstar}[1]{}
\providecommand{\docsubsubsubsection}[3]{}
\providecommand{\docsubsubsubsectionstar}[1]{}
\providecommand{\doctable}{}

\providecommand{\debuggingtool}{}\renewcommand{\debuggingtool}{This tool is provided for debugging purposes.
It allows exposing and modifying an internal data structure that is usually not accessible.
}

\providecommand{\interface}{All tools accept command-line arguments which are taken as names of plain text files containing the source code.
If no arguments are provided, the standard input stream is used instead.
Output files are generated in the current working directory and have the same name as the input file being processed whereas the filename extension gets replaced by an appropriate suffix.
\seeinterface
}

\providecommand{\license}{\noindent Copyright \copyright{} Florian Negele\par\medskip\noindent
Permission is granted to copy, distribute and/or modify this document under the terms of the
\fdl{}, Version 1.3 or any later version published by the \href{https://fsf.org/}{Free Software Foundation}.
}

\providecommand{\ecslogosurface}{
\fill[darkgray] (0,0,0) -- (0,0,3) -- (0,3,3) -- (0,3,1) -- (0,4,1) -- (0,4,3) -- (0,5,3) -- (0,5,0) -- (0,2,0) -- (0,2,2) -- (0,1,2) -- (0,1,0) -- cycle;
\fill[gray] (0,5,0) -- (0,5,3) -- (1,5,3) -- (1,5,1) -- (2,5,1) -- (2,5,3) -- (3,5,3) -- (3,5,0) -- cycle;
\fill[lightgray] (0,0,0) -- (0,1,0) -- (2,1,0) -- (2,4,0) -- (1,4,0) -- (1,3,0) -- (2,3,0) -- (2,2,0) -- (0,2,0) -- (0,5,0) -- (3,5,0) -- (3,0,0) -- cycle;
\begin{scope}[line width=0.5]
\begin{scope}[gray]
\draw (0,0,0) -- (0,1,0);
\draw (2,1,0) -- (2,2,0);
\draw (0,1,2) -- (0,2,2);
\draw (0,2,0) -- (0,5,0);
\draw (2,3,0) -- (2,4,0);
\end{scope}
\begin{scope}[lightgray]
\draw (0,1,0) -- (0,1,2);
\draw (0,3,1) -- (0,3,3);
\draw (0,5,0) -- (0,5,3);
\draw (2,5,1) -- (2,5,3);
\end{scope}
\begin{scope}[white]
\draw (0,1,0) -- (2,1,0);
\draw (1,3,0) -- (2,3,0);
\draw (0,5,0) -- (3,5,0);
\end{scope}
\end{scope}
}

\providecommand{\ecslogo}[1]{
\begin{tikzpicture}[scale={(#1)/((sin(45)+cos(45))*3cm)},x={({-cos(45)*1cm},{sin(45)*sin(30)*1cm})},y={({0cm},{(cos(30)*1cm})},z={({sin(45)*1cm},{cos(45)*sin(30)*1cm})}]
\begin{scope}[darkgray,line width=1]
\draw (0,0,0) -- (0,0,3) -- (0,3,3) -- (2,3,3) -- (2,5,3) -- (3,5,3) -- (3,5,0) -- (3,0,0) -- cycle;
\draw (0,3,1) -- (0,4,1) -- (0,4,3) -- (0,5,3) -- (1,5,3) -- (1,5,1) -- (2,5,1);
\draw (1,3,0) -- (1,4,0) -- (2,4,0);
\end{scope}
\fill[darkgray] (2,0,0) -- (2,0,3) -- (2,5,3) -- (2,5,1) -- (2,4,1) -- (2,4,0) -- cycle;
\fill[lightgray] (2,0,2) -- (0,0,2) -- (0,2,2) -- (2,2,2) -- cycle;
\fill[gray] (0,1,0) -- (2,1,0) -- (2,1,2) -- (0,1,2) -- cycle;
\fill[gray] (0,3,1) -- (0,3,3) -- (2,3,3) -- (2,3,0) -- (1,3,0) -- (1,3,1) -- cycle;
\ecslogosurface
\end{tikzpicture}
}

\providecommand{\shadowedecslogo}[3]{
\begin{tikzpicture}[scale={(#1)/((sin(#2)+cos(#2))*3cm)},x={({-cos(#2)*1cm},{sin(#2)*sin(#3)*1cm})},y={({0cm},{(cos(#3)*1cm})},z={({sin(#2)*1cm},{cos(#2)*sin(#3)*1cm})}]
\shade[top color=lightgray!50!white,bottom color=white,middle color=lightgray!50!white] (0,0,0) -- (3,0,0) -- (3,{-0.5-3*sin(#2)*sin(#3)/cos(#3)},0) -- (0,-0.5,0) -- cycle;
\shade[top color=darkgray!50!gray,bottom color=white,middle color=darkgray!50!white] (0,0,0) -- (0,0,3) -- (0,{-0.5-3*cos(#2)*sin(#3)/cos(#3)},3) -- (0,-0.5,0) -- cycle;
\begin{scope}[y={({(cos(#2)+sin(#2))*0.5cm},{(cos(#2)*sin(#3)-sin(#2)*sin(#3))*0.5cm})}]
\useasboundingbox (3,0,0) -- (0,0,0) -- (0,0,3);
\shade[left color=darkgray!80!black,right color=lightgray,middle color=gray] (0,0,0) -- (0,1,0) -- (0,1,0.5) -- (0,2,0) -- (0,5,0) -- (0,5,3) -- (1,5,3) -- (1,4,3) -- (1,4,2.5) -- (1,3,3) -- (2,5,3) -- (3,5,3) -- (3,0,3) -- cycle;
\clip (0,0,0) -- (0,0,3) -- ({-3*sin(#2)/cos(#2)},0,0) -- cycle;
\shade[left color=darkgray,right color=lightgray!50!gray] (0,0,0) -- (0,1,0) -- (0,1,0.5) -- (0,2,0) -- (0,5,0) -- (0,5,3) -- (1,5,3) -- (1,4,3) -- (1,4,2.5) -- (1,3,3) -- (2,5,3) -- (3,5,3) -- (3,0,3) -- cycle;
\end{scope}
\shade[left color=darkgray,right color=darkgray!80!black] (2,0,0) -- (2,0,3) -- (2,5,3) -- (2,5,1) -- (2,4,1) -- (2,4,0) -- cycle;
\shade[left color=darkgray!90!black,right color=gray!80!darkgray] (2,0,2) -- (0,0,2) -- (0,2,2) -- (2,2,2) -- cycle;
\shade[top color=darkgray!90!black,bottom color=gray!80!darkgray] (0,1,0) -- (2,1,0) -- (2,1,2) -- (0,1,2) -- cycle;
\shade[top color=darkgray!90!black,bottom color=gray!80!darkgray] (0,3,1) -- (0,3,3) -- (2,3,3) -- (2,3,0) -- (1,3,0) -- (1,3,1) -- cycle;
\fill[gray] (2,1,0) -- (1.5,1,0.5) -- (0,1,0.5) -- (0,1,0) -- cycle;
\fill[gray] (1,3,2) -- (0.5,3,2) -- (0.5,3,3) -- (1,3,3) -- cycle;
\fill[gray] (2,3,0) -- (1.5,3,0.5) -- (1,3,0.5) -- (1,3,0) -- cycle;
\ecslogosurface
\end{tikzpicture}
}

\providecommand{\cpplogo}[1]{
\begin{tikzpicture}[scale=(#1)/512em]
\fill[gray] (435.2794,398.7159) -- (247.1911,507.3075) .. controls (236.3563,513.5642) and (218.6240,513.5642) .. (207.7892,507.3075) -- (19.7009,398.7159) .. controls (8.8646,392.4606) and (0.0000,377.1043) .. (0.0000,364.5924) -- (0.0000,147.4076) .. controls (0.8430,132.8363) and (8.2856,120.7683) .. (19.7009,113.2842) -- (207.7892,4.6926) .. controls (218.6240,-1.5642) and (236.3564,-1.5642) .. (247.1911,4.6926) -- (435.2794,113.2842) .. controls (447.5273,121.4304) and (454.4987,133.6918) .. (454.9803,147.4076) -- (454.9803,364.5924) .. controls (454.5404,377.7571) and (446.6566,391.0351) .. (435.2794,398.7159) -- cycle(75.8301,255.9993) .. controls (74.9389,404.0881) and (273.2892,469.4783) .. (358.8263,331.8769) -- (293.1917,293.8965) .. controls (253.5702,359.4301) and (155.1909,335.9977) .. (151.6601,255.9993) .. controls (152.7204,182.2703) and (249.4137,148.0211) .. (293.1961,218.1065) -- (358.8308,180.1276) .. controls (283.4477,49.2645) and (79.6318,96.3470) .. (75.8301,255.9993) -- cycle(379.1503,247.5747) -- (362.2982,247.5747) -- (362.2982,230.7226) -- (345.4490,230.7226) -- (345.4490,247.5747) -- (328.5969,247.5747) -- (328.5969,264.4254) -- (345.4490,264.4254) -- (345.4490,281.2759) -- (362.2982,281.2759) -- (362.2982,264.4254) -- (379.1503,264.4254) -- cycle(442.3420,247.5747) -- (425.4899,247.5747) -- (425.4899,230.7226) -- (408.6408,230.7226) -- (408.6408,247.5747) -- (391.7886,247.5747) -- (391.7886,264.4254) -- (408.6408,264.4254) -- (408.6408,281.2759) -- (425.4899,281.2759) -- (425.4899,264.4254) -- (442.3420,264.4254) -- cycle;
\end{tikzpicture}
}

\providecommand{\fallogo}[1]{
\begin{tikzpicture}[scale=(#1)/512em]
\fill[gray] (185.7774,0.0000) .. controls (200.4486,15.9798) and (226.8966,8.7148) .. (235.0426,31.5836) .. controls (249.5297,58.0598) and (247.9581,97.9161) .. (280.3335,110.9762) .. controls (309.1690,120.3496) and (337.8406,104.2727) .. (366.5753,103.9379) .. controls (373.4449,111.5171) and (379.2885,128.2574) .. (383.9755,108.9744) .. controls (396.6979,102.5615) and (437.2808,107.6681) .. (426.9652,124.3252) .. controls (408.9822,121.0785) and (412.4742,146.0729) .. (426.5192,131.4996) .. controls (433.8413,120.8489) and (465.1541,126.5522) .. (441.9067,135.7950) .. controls (396.1879,157.7478) and (344.1112,161.5079) .. (298.5528,183.5702) .. controls (277.7471,193.5198) and (284.6941,218.7163) .. (285.2127,236.9640) .. controls (292.3599,316.2826) and (307.3929,394.6311) .. (317.1198,473.6154) .. controls (329.0637,505.4736) and (292.1195,528.5004) .. (265.9183,511.2761) .. controls (237.9284,499.2462) and (237.3684,465.2681) .. (230.9102,439.9421) .. controls (218.6692,374.3397) and (215.6307,306.9662) .. (198.1732,242.3977) .. controls (183.1379,232.7444) and (164.4245,256.0298) .. (149.0430,261.4799) .. controls (116.9328,279.2585) and (87.1822,308.5851) .. (48.2293,307.8914) .. controls (21.3220,306.9037) and (-15.9107,281.8761) .. (7.2921,252.7908) .. controls (29.7799,220.6177) and (67.5177,204.3028) .. (100.9287,185.9449) .. controls (130.8217,170.8906) and (161.1548,156.5903) .. (191.0278,141.5847) .. controls (196.1738,120.0520) and (186.6049,95.2409) .. (186.8382,72.4353) .. controls (185.5234,48.4204) and (183.1700,23.9341) .. (185.7774,0.0000) -- cycle;
\end{tikzpicture}
}

\providecommand{\oblogo}[1]{
\begin{tikzpicture}[scale=(#1)/512em]
\fill[gray] (160.3865,208.9117) .. controls (154.0879,214.6478) and (149.0735,221.2409) .. (145.4125,228.5384) .. controls (184.8790,248.4273) and (234.7122,269.8787) .. (297.5493,291.8782) .. controls (300.3943,281.4769) and (300.9552,268.7619) .. (300.4023,255.2389) .. controls (248.9909,244.7891) and (200.0310,225.9279) .. (160.3865,208.9117) -- cycle(225.7398,392.6996) .. controls (308.0209,392.1716) and (359.3326,345.9277) .. (368.7203,285.2098) .. controls (376.6742,197.1784) and (311.7194,141.3342) .. (205.4287,142.1456) .. controls (139.9485,141.4804) and (88.7155,166.1957) .. (73.5775,228.0086) .. controls (52.0297,320.3408) and (123.4078,391.0103) .. (225.7398,392.6996) -- cycle(216.0739,176.4733) .. controls (268.9183,179.2424) and (315.8292,206.5488) .. (312.7454,265.1139) .. controls (313.2769,315.6384) and (286.5993,353.4946) .. (216.6040,355.7934) .. controls (162.4657,355.7934) and (126.0914,317.5023) .. (126.0914,260.5103) .. controls (126.1733,214.2900) and (163.3363,176.2849) .. (216.0739,176.4733) -- cycle(76.4897,189.1754) .. controls (13.1586,147.5631) and (0.0000,119.4207) .. (0.0000,119.4207) -- (90.6499,170.1632) .. controls (85.3004,175.8497) and (80.5994,182.1633) .. (76.4897,189.1754) -- cycle(353.9486,119.3004) -- (402.9482,119.3004) .. controls (427.0025,137.0797) and (450.9893,162.7034) .. (474.9529,191.0213) .. controls (509.3540,228.5339) and (531.3391,294.2091) .. (487.8149,312.1206) .. controls (462.8165,324.7652) and (394.3874,316.8943) .. (373.8912,313.6651) .. controls (379.9291,297.7449) and (383.2899,278.4204) .. (381.4989,257.7214) .. controls (420.3069,248.0321) and (421.9610,218.3461) .. (407.7867,192.6417) .. controls (391.1113,162.4018) and (370.1114,132.9097) .. (353.9486,119.3004) -- cycle;
\end{tikzpicture}
}

\providecommand{\markuptable}{
\begin{table}
\sffamily\centering
\begin{tabular}{@{}lcl@{}}
\toprule
\texttt{//italics//} & $\rightarrow$ & \textit{italics} \\
\midrule
\texttt{**bold**} & $\rightarrow$ & \textbf{bold} \\
\midrule
\texttt{\# ordered list} & & 1 ordered list \\
\texttt{\# second item} & $\rightarrow$ & 2 second item \\
\texttt{\#\# sub item} & & \hspace{1em} 1 sub item \\
\midrule
\texttt{* unordered list} & & $\bullet$ unordered list \\
\texttt{* second item} & $\rightarrow$ & $\bullet$ second item \\
\texttt{** sub item} & & \hspace{1em} $\bullet$ sub item \\
\midrule
\texttt{link to [[label]]} & $\rightarrow$ & link to \underline{label} \\
\midrule
\texttt{<{}<label>{}> definition } & $\rightarrow$ & definition \\
\midrule
\texttt{[[url|link name]]} & $\rightarrow$ & \underline{link name} \\
\midrule\addlinespace
\texttt{= large heading} & & {\Large large heading} \smallskip \\
\texttt{== medium heading} & $\rightarrow$ & {\large medium heading} \\
\texttt{=== small heading} & & small heading \\
\midrule
\texttt{no line break} & & no line break for paragraphs \\
\texttt{for paragraphs} & $\rightarrow$ \\
& & use empty line \\
\texttt{use empty line} \\
\midrule
\texttt{force\textbackslash\textbackslash line break} & $\rightarrow$ & force \\
& & line break \\
\midrule
\texttt{horizontal line} & $\rightarrow$ & horizontal line \\
\texttt{----} & & \hrulefill \\
\midrule
\texttt{|=a|=table|=header} & & \underline{a \enspace table \enspace header} \\
\texttt{|a|table|row} & $\rightarrow$ & a \enspace table \enspace row \\
\texttt{|b|table|row} & & b \enspace table \enspace row \\
\midrule
\texttt{\{\{\{} \\
\texttt{unformatted} & $\rightarrow$ & \texttt{unformatted} \\
\texttt{code} & & \texttt{code} \\
\texttt{\}\}\}} \\
\midrule\addlinespace
\texttt{@ new article} & & {\Large 1.\ new article} \smallskip \\
\texttt{@ second article} & $\rightarrow$ & {\Large 2.\ second article} \smallskip \\
\texttt{@@ sub article} & & {\large 2.1.\ sub article} \\
\bottomrule
\end{tabular}
\normalfont\caption{Elements of the generic documentation markup language}
\label{tab:docmarkup}
\end{table}
}

\providecommand{\startchapter}[4]{
\documentclass[11pt,a4paper]{article}
\usepackage{booktabs}
\usepackage[format=hang,labelfont=bf]{caption}
\usepackage{changepage}
\usepackage[T1]{fontenc}
\usepackage[margin=2cm]{geometry}
\usepackage{hyperref}
\usepackage[american]{isodate}
\usepackage{lmodern}
\usepackage{longtable}
\usepackage{mathptmx}
\usepackage{microtype}
\usepackage[toc]{multitoc}
\usepackage{multirow}
\usepackage[all]{nowidow}
\usepackage{pdfcomment}
\usepackage{syntax}
\usepackage{tikz}
\usepackage[all]{xy}
\hypersetup{pdfborder={0 0 0},bookmarksnumbered=true,pdftitle={\ecs{}: #2},pdfauthor={Florian Negele},pdfsubject={\ecs{}},pdfkeywords={#1}}
\setlength{\grammarindent}{8em}\setlength{\grammarparsep}{0.2ex}
\setlength{\columnsep}{2em}
\newcommand{\prefix}{}
\newcounter{instruction}
\bibliographystyle{unsrt}
\renewcommand{\index}[2][]{}
\renewcommand{\arraystretch}{1.05}
\renewcommand{\floatpagefraction}{0.7}
\renewcommand{\syntleft}{\itshape}\renewcommand{\syntright}{}
\title{\vspace{-5ex}\Huge{\ecs{}}\medskip\hrule}
\author{\huge{#2}}
\date{\medskip\version}
\newif\ifbook\bookfalse
\pagestyle{headings}
\frenchspacing
\begin{document}
\maketitle\thispagestyle{empty}\noindent#4\setlength{\columnseprule}{0.4pt}\tableofcontents\setlength{\columnseprule}{0pt}\vfill\pagebreak[3]\null\vfill\bigskip\noindent
\parbox{\textwidth-4em}{\license The contents of this \documentation{} are part of the \href{manual}{\ecs{} User Manual}~\cite{manual} and correspond to Chapter ``\href{manual\##3}{#1}''.\alignright\mbox{\today}}
\parbox{4em}{\flushright\ecslogo{3em}}
\clearpage
}

\providecommand{\concludechapter}{
\vfill\pagebreak[3]\null\vfill
\thispagestyle{myheadings}\markright{REFERENCES}
\noindent\begin{minipage}{\textwidth}\begin{multicols}{2}[\section*{References}]
\renewcommand{\section}[2]{}\small\bibliography{references}
\end{multicols}\end{minipage}\end{document}
}

\providecommand{\startpresentation}[2]{
\documentclass[14pt,aspectratio=43,usepdftitle=false]{beamer}
\usepackage{booktabs}
\usepackage{etex}
\usepackage{multicol}
\usepackage{tikz}
\usepackage[all]{xy}
\bibliographystyle{unsrt}
\setlength{\columnsep}{1em}
\setlength{\leftmargini}{1em}
\setbeamercolor{title}{fg=black}
\setbeamercolor{structure}{fg=darkgray}
\setbeamercolor{bibliography item}{fg=darkgray}
\setbeamerfont{title}{series=\bfseries}
\setbeamerfont{subtitle}{series=\normalfont}
\setbeamerfont*{frametitle}{parent=title}
\setbeamerfont{block title}{series=\bfseries}
\setbeamerfont*{framesubtitle}{parent=subtitle}
\setbeamersize{text margin left=1em,text margin right=1em}
\setbeamertemplate{navigation symbols}{}
\setbeamertemplate{itemize item}[circle]{}
\setbeamertemplate{bibliography item}[triangle]{}
\setbeamertemplate{bibliography entry author}{\usebeamercolor[fg]{bibliography item}}
\setbeamertemplate{frametitle}{\medskip\usebeamerfont{frametitle}\color{gray}\raisebox{-2.5ex}[0ex][0ex]{\rule{0.1em}{4.5ex}}}
\addtobeamertemplate{frametitle}{}{\hspace{0.4em}\usebeamercolor[fg]{title}\insertframetitle\par\vspace{0.2ex}\hspace{0.5em}\usebeamerfont{framesubtitle}\insertframesubtitle}
\hypersetup{pdfborder={0 0 0},bookmarksnumbered=true,bookmarksopen=true,bookmarksopenlevel=0,pdftitle={\ecs{}: #1},pdfauthor={Florian Negele},pdfsubject={\ecs{}},pdfkeywords={#1}}
\renewcommand{\flowgraph}[1]{\resizebox{\textwidth}{!}{$$\xymatrix{##1}$$}}
\title{\ecs{}\medskip\hrule\medskip}
\institute{\shadowedecslogo{5em}{30}{15}}
\date{\version}
\subtitle{#1}
\begin{document}
\begin{frame}[plain]\titlepage\nocite{manual}\end{frame}
\begin{frame}{Contents}{#1}\begin{center}\tableofcontents\end{center}\end{frame}
}

\providecommand{\concludepresentation}{
\begin{frame}{References}\begin{footnotesize}\setlength{\columnseprule}{0.4pt}\begin{multicols}{2}\bibliography{references}\end{multicols}\end{footnotesize}\end{frame}
\end{document}
}

\providecommand{\startbook}[1]{
\documentclass[10pt,paper=17cm:24cm,DIV=13,twoside=semi,headings=normal,numbers=noendperiod,cleardoublepage=plain]{scrbook}
\usepackage{atveryend}
\usepackage{booktabs}
\usepackage{caption}
\usepackage{changepage}
\usepackage[T1]{fontenc}
\usepackage{imakeidx}
\usepackage{hyperref}
\usepackage[american]{isodate}
\usepackage{lmodern}
\usepackage{longtable}
\usepackage{mathptmx}
\usepackage[final]{microtype}
\usepackage{multicol}
\usepackage{multirow}
\usepackage[all]{nowidow}
\usepackage{pdfcomment}
\usepackage{scrlayer-scrpage}
\usepackage{setspace}
\usepackage{syntax}
\usepackage[eventxtindent=4pt,oddtxtexdent=4pt]{thumbs}
\usepackage{tikz}
\usepackage[all]{xy}
\hyphenation{Micro-Blaze Open-Cores Open-RISC Power-PC}
\hypersetup{pdfborder={0 0 0},bookmarksnumbered=true,bookmarksopen=true,bookmarksopenlevel=0,pdftitle={\ecs{}: #1},pdfauthor={Florian Negele},pdfsubject={\ecs{}},pdfkeywords={#1}}
\setlength{\grammarindent}{8em}\setlength{\grammarparsep}{0.7ex}
\setkomafont{captionlabel}{\usekomafont{descriptionlabel}}
\renewcommand{\arraystretch}{1.05}\setstretch{1.1}
\renewcommand{\chapterformat}{\thechapter\autodot\enskip\raisebox{-1ex}[0ex][0ex]{\color{gray}\rule{0.1em}{3.5ex}}\enskip}
\renewcommand{\startchapter}[4]{\hypertarget{##3}{\chapter{##1}}\label{##3}##4\addthumb{##1}{\LARGE\sffamily\bfseries\thechapter}{white}{gray}\renewcommand{\prefix}{##3}}
\renewcommand{\concludechapter}{\clearpage{\stopthumb\cleardoublepage}}
\renewcommand{\syntleft}{\itshape}\renewcommand{\syntright}{}
\renewcommand{\floatpagefraction}{0.7}
\renewcommand{\partheademptypage}{}
\DeclareMicrotypeAlias{lmss}{cmr}
\newcommand{\prefix}{}
\newcounter{instruction}
\bibliographystyle{unsrt}
\newif\ifbook\booktrue
\makeindex[intoc,title=Index]
\makeindex[intoc,name=tools,title=Index of Tools,columns=3]
\makeindex[intoc,name=library,title=Index of Library Names]
\makeindex[intoc,name=runtime,title=Index of Runtime Support]
\makeindex[intoc,name=environment,title=Index of Target Environments]
\indexsetup{toclevel=chapter,headers={\indexname}{\indexname}}
\frenchspacing
\begin{document}
\pagenumbering{alph}
\begin{titlepage}\centering
\huge\sffamily\null\vfill\textbf{\ecs{}}\bigskip\hrule\bigskip#1
\normalsize\normalfont\vfill\vfill\shadowedecslogo{10em}{30}{15}
\large\vfill\vfill\version
\end{titlepage}
\null\vfill
\thispagestyle{empty}
\noindent\today\par\medskip
\license A copy of this license is included in Appendix~\ref{fdl} on page~\pageref{fdl}.
All product names used herein are for identification purposes only and may be trademarks of their respective companies.
\concludechapter
\frontmatter
\setcounter{tocdepth}{1}
\tableofcontents
\setcounter{tocdepth}{2}
\concludechapter
\listoffigures
\concludechapter
\listoftables
\concludechapter
}

\providecommand{\concludebook}{
\backmatter
\addtocontents{toc}{\protect\setcounter{tocdepth}{-1}}
\phantomsection\addcontentsline{toc}{part}{Bibliography}
\bibliography{references}
\concludechapter
\phantomsection\addcontentsline{toc}{part}{Indexes}
\printindex
\concludechapter
\indexprologue{\label{idx:tools}}
\printindex[tools]
\concludechapter
\printindex[library]
\concludechapter
\indexprologue{\label{idx:runtime}}
\printindex[runtime]
\concludechapter
\indexprologue{\label{idx:environment}}
\printindex[environment]
\concludechapter
\pagestyle{empty}\pagenumbering{Alph}\null\clearpage
\null\vfill\centering\ecslogo{4em}\par\medskip\license
\end{document}
}

% chapter references

\providecommand{\seedocumentationref}{}\renewcommand{\seedocumentationref}[3]{#1, see \Documentation{}~\documentationref{#2}{#3}. }
\providecommand{\seeinterface}{}\renewcommand{\seeinterface}{\ifbook See \Documentation{}~\documentationref{interface}{User Interface} for more information about the common user interface of all of these tools. \fi}
\providecommand{\seeguide}{}\renewcommand{\seeguide}{\seedocumentationref{For basic examples of using some of these tools in practice}{guide}{User Guide}}
\providecommand{\seecpp}{}\renewcommand{\seecpp}{\seedocumentationref{For more information about the \cpp{} programming language and its implementation by the \ecs{}}{cpp}{User Manual for \cpp{}}}
\providecommand{\seefalse}{}\renewcommand{\seefalse}{\seedocumentationref{For more information about the FALSE programming language and its implementation by the \ecs{}}{false}{User Manual for FALSE}}
\providecommand{\seeoberon}{}\renewcommand{\seeoberon}{\seedocumentationref{For more information about the Oberon programming language and its implementation by the \ecs{}}{oberon}{User Manual for Oberon}}
\providecommand{\seeassembly}{}\renewcommand{\seeassembly}{\seedocumentationref{For more information about the generic assembly language and how to use it}{assembly}{Generic Assembly Language Specification}}
\providecommand{\seeamd}{}\renewcommand{\seeamd}{\seedocumentationref{For more information about how the \ecs{} supports the AMD64 hardware architecture}{amd64}{AMD64 Hardware Architecture Support}}
\providecommand{\seearm}{}\renewcommand{\seearm}{\seedocumentationref{For more information about how the \ecs{} supports the ARM hardware architecture}{arm}{ARM Hardware Architecture Support}}
\providecommand{\seeavr}{}\renewcommand{\seeavr}{\seedocumentationref{For more information about how the \ecs{} supports the AVR hardware architecture}{avr}{AVR Hardware Architecture Support}}
\providecommand{\seeavrtt}{}\renewcommand{\seeavrtt}{\seedocumentationref{For more information about how the \ecs{} supports the AVR32 hardware architecture}{avr32}{AVR32 Hardware Architecture Support}}
\providecommand{\seemabk}{}\renewcommand{\seemabk}{\seedocumentationref{For more information about how the \ecs{} supports the M68000 hardware architecture}{m68k}{M68000 Hardware Architecture Support}}
\providecommand{\seemibl}{}\renewcommand{\seemibl}{\seedocumentationref{For more information about how the \ecs{} supports the MicroBlaze hardware architecture}{mibl}{MicroBlaze Hardware Architecture Support}}
\providecommand{\seemips}{}\renewcommand{\seemips}{\seedocumentationref{For more information about how the \ecs{} supports the MIPS32 and MIPS64 hardware architectures}{mips}{MIPS Hardware Architecture Support}}
\providecommand{\seemmix}{}\renewcommand{\seemmix}{\seedocumentationref{For more information about how the \ecs{} supports the MMIX hardware architecture}{mmix}{MMIX Hardware Architecture Support}}
\providecommand{\seeorok}{}\renewcommand{\seeorok}{\seedocumentationref{For more information about how the \ecs{} supports the OpenRISC 1000 hardware architecture}{or1k}{OpenRISC 1000 Hardware Architecture Support}}
\providecommand{\seeppc}{}\renewcommand{\seeppc}{\seedocumentationref{For more information about how the \ecs{} supports the PowerPC hardware architecture}{ppc}{PowerPC Hardware Architecture Support}}
\providecommand{\seerisc}{}\renewcommand{\seerisc}{\seedocumentationref{For more information about how the \ecs{} supports the RISC hardware architecture}{risc}{RISC Hardware Architecture Support}}
\providecommand{\seewasm}{}\renewcommand{\seewasm}{\seedocumentationref{For more information about how the \ecs{} supports the WebAssembly architecture}{wasm}{WebAssembly Architecture Support}}
\providecommand{\seedocumentation}{}\renewcommand{\seedocumentation}{\seedocumentationref{For more information about generic documentations and their generation by the \ecs{}}{documentation}{Generic Documentation Generation}}
\providecommand{\seedebugging}{}\renewcommand{\seedebugging}{\seedocumentationref{For more information about debugging information and its representation}{debugging}{Debugging Information Representation}}
\providecommand{\seecode}{}\renewcommand{\seecode}{\seedocumentationref{For more information about intermediate code and its purpose}{code}{Intermediate Code Representation}}
\providecommand{\seeobject}{}\renewcommand{\seeobject}{\seedocumentationref{For more information about object files and their purpose}{object}{Object File Representation}}

% generic documentation tools

\providecommand{\docprint}{
\toolsection{docprint} is a pretty printer for generic documentations.
It reformats generic documentations and writes it to the standard output stream.
\debuggingtool
\flowgraph{\resource{generic\\documentation} \ar[r] & \toolbox{docprint} \ar[r] & \resource{generic\\documentation}}
\seedocumentation
}

\providecommand{\doccheck}{
\toolsection{doccheck} is a syntactic and semantic checker for generic documentations.
It just performs syntactic and semantic checks on generic documentations and writes its diagnostic messages to the standard error stream.
\debuggingtool
\flowgraph{\resource{generic\\documentation} \ar[r] & \toolbox{doccheck} \ar[r] & \resource{diagnostic\\messages}}
\seedocumentation
}

\providecommand{\dochtml}{
\toolsection{dochtml} is an HTML documentation generator for generic documentations.
It processes several generic documentations and assembles all information therein into an HTML document.
\debuggingtool
\flowgraph{\resource{generic\\documentation} \ar[r] & \toolbox{dochtml} \ar[r] & \resource{HTML\\document}}
\seedocumentation
}

\providecommand{\doclatex}{
\toolsection{doclatex} is a Latex documentation generator for generic documentations.
It processes several generic documentations and assembles all information therein into a Latex document.
\debuggingtool
\flowgraph{\resource{generic\\documentation} \ar[r] & \toolbox{doclatex} \ar[r] & \resource{Latex\\document}}
\seedocumentation
}

% intermediate code tools

\providecommand{\cdcheck}{
\toolsection{cdcheck} is a syntactic and semantic checker for intermediate code.
It just performs syntactic and semantic checks on programs written in intermediate code and writes its diagnostic messages to the standard error stream.
\debuggingtool
\flowgraph{\resource{intermediate\\code} \ar[r] & \toolbox{cdcheck} \ar[r] & \resource{diagnostic\\messages}}
\seeassembly\seecode
}

\providecommand{\cdopt}{
\toolsection{cdopt} is an optimizer for intermediate code.
It performs various optimizations on programs written in intermediate code and writes the result to the standard output stream.
\debuggingtool
\flowgraph{\resource{intermediate\\code} \ar[r] & \toolbox{cdopt} \ar[r] & \resource{optimized\\code}}
\seeassembly\seecode
}

\providecommand{\cdrun}{
\toolsection{cdrun} is an interpreter for intermediate code.
It processes and executes programs written in intermediate code.
The following code sections are predefined and have the usual semantics:
\texttt{abort}, \texttt{\_Exit}, \texttt{fflush}, \texttt{floor}, \texttt{fputc}, \texttt{free}, \texttt{getchar}, \texttt{malloc}, and \texttt{putchar}.
Diagnostic messages about invalid operations include the name of the executed code section and the index of the erroneous instruction.
\debuggingtool
\flowgraph{\resource{intermediate\\code} \ar[r] & \toolbox{cdrun} \ar@/u/[r] & \resource{input/\\output} \ar@/d/[l]}
\seeassembly\seecode
}

\providecommand{\cdamda}{
\toolsection{cdamd16} is a compiler for intermediate code targeting the AMD64 hardware architecture.
It generates machine code for AMD64 processors from programs written in intermediate code and stores it in corresponding object files.
The compiler generates machine code for the 16-bit operating mode defined by the AMD64 architecture.
It also creates a debugging information file as well as an assembly file containing a listing of the generated machine code.
\debuggingtool
\flowgraph{\resource{intermediate\\code} \ar[r] & \toolbox{cdamd16} \ar[r] \ar[d] \ar[rd] & \resource{object file} \\ & \resource{assembly\\listing} & \resource{debugging\\information}}
\seeassembly\seeamd\seeobject\seecode\seedebugging
}

\providecommand{\cdamdb}{
\toolsection{cdamd32} is a compiler for intermediate code targeting the AMD64 hardware architecture.
It generates machine code for AMD64 processors from programs written in intermediate code and stores it in corresponding object files.
The compiler generates machine code for the 32-bit operating mode defined by the AMD64 architecture.
It also creates a debugging information file as well as an assembly file containing a listing of the generated machine code.
\debuggingtool
\flowgraph{\resource{intermediate\\code} \ar[r] & \toolbox{cdamd32} \ar[r] \ar[d] \ar[rd] & \resource{object file} \\ & \resource{assembly\\listing} & \resource{debugging\\information}}
\seeassembly\seeamd\seeobject\seecode\seedebugging
}

\providecommand{\cdamdc}{
\toolsection{cdamd64} is a compiler for intermediate code targeting the AMD64 hardware architecture.
It generates machine code for AMD64 processors from programs written in intermediate code and stores it in corresponding object files.
The compiler generates machine code for the 64-bit operating mode defined by the AMD64 architecture.
It also creates a debugging information file as well as an assembly file containing a listing of the generated machine code.
\debuggingtool
\flowgraph{\resource{intermediate\\code} \ar[r] & \toolbox{cdamd64} \ar[r] \ar[d] \ar[rd] & \resource{object file} \\ & \resource{assembly\\listing} & \resource{debugging\\information}}
\seeassembly\seeamd\seeobject\seecode\seedebugging
}

\providecommand{\cdarma}{
\toolsection{cdarma32} is a compiler for intermediate code targeting the ARM hardware architecture.
It generates machine code for ARM processors executing A32 instructions from programs written in intermediate code and stores it in corresponding object files.
It also creates a debugging information file as well as an assembly file containing a listing of the generated machine code.
\debuggingtool
\flowgraph{\resource{intermediate\\code} \ar[r] & \toolbox{cdarma32} \ar[r] \ar[d] \ar[rd] & \resource{object file} \\ & \resource{assembly\\listing} & \resource{debugging\\information}}
\seeassembly\seearm\seeobject\seecode\seedebugging
}

\providecommand{\cdarmb}{
\toolsection{cdarma64} is a compiler for intermediate code targeting the ARM hardware architecture.
It generates machine code for ARM processors executing A64 instructions from programs written in intermediate code and stores it in corresponding object files.
It also creates a debugging information file as well as an assembly file containing a listing of the generated machine code.
\debuggingtool
\flowgraph{\resource{intermediate\\code} \ar[r] & \toolbox{cdarma64} \ar[r] \ar[d] \ar[rd] & \resource{object file} \\ & \resource{assembly\\listing} & \resource{debugging\\information}}
\seeassembly\seearm\seeobject\seecode\seedebugging
}

\providecommand{\cdarmc}{
\toolsection{cdarmt32} is a compiler for intermediate code targeting the ARM hardware architecture.
It generates machine code for ARM processors without floating-point extension executing T32 instructions from programs written in intermediate code and stores it in corresponding object files.
It also creates a debugging information file as well as an assembly file containing a listing of the generated machine code.
\debuggingtool
\flowgraph{\resource{intermediate\\code} \ar[r] & \toolbox{cdarmt32} \ar[r] \ar[d] \ar[rd] & \resource{object file} \\ & \resource{assembly\\listing} & \resource{debugging\\information}}
\seeassembly\seearm\seeobject\seecode\seedebugging
}

\providecommand{\cdarmcfpe}{
\toolsection{cdarmt32fpe} is a compiler for intermediate code targeting the ARM hardware architecture.
It generates machine code for ARM processors with floating-point extension executing T32 instructions from programs written in intermediate code and stores it in corresponding object files.
It also creates a debugging information file as well as an assembly file containing a listing of the generated machine code.
\debuggingtool
\flowgraph{\resource{intermediate\\code} \ar[r] & \toolbox{cdarmt32fpe} \ar[r] \ar[d] \ar[rd] & \resource{object file} \\ & \resource{assembly\\listing} & \resource{debugging\\information}}
\seeassembly\seearm\seeobject\seecode\seedebugging
}

\providecommand{\cdavr}{
\toolsection{cdavr} is a compiler for intermediate code targeting the AVR hardware architecture.
It generates machine code for AVR processors from programs written in intermediate code and stores it in corresponding object files.
It also creates a debugging information file as well as an assembly file containing a listing of the generated machine code.
\debuggingtool
\flowgraph{\resource{intermediate\\code} \ar[r] & \toolbox{cdavr} \ar[r] \ar[d] \ar[rd] & \resource{object file} \\ & \resource{assembly\\listing} & \resource{debugging\\information}}
\seeassembly\seeavr\seeobject\seecode\seedebugging
}

\providecommand{\cdavrtt}{
\toolsection{cdavr32} is a compiler for intermediate code targeting the AVR32 hardware architecture.
It generates machine code for AVR32 processors from programs written in intermediate code and stores it in corresponding object files.
It also creates a debugging information file as well as an assembly file containing a listing of the generated machine code.
\debuggingtool
\flowgraph{\resource{intermediate\\code} \ar[r] & \toolbox{cdavr32} \ar[r] \ar[d] \ar[rd] & \resource{object file} \\ & \resource{assembly\\listing} & \resource{debugging\\information}}
\seeassembly\seeavrtt\seeobject\seecode\seedebugging
}

\providecommand{\cdmabk}{
\toolsection{cdm68k} is a compiler for intermediate code targeting the M68000 hardware architecture.
It generates machine code for M68000 processors from programs written in intermediate code and stores it in corresponding object files.
It also creates a debugging information file as well as an assembly file containing a listing of the generated machine code.
\debuggingtool
\flowgraph{\resource{intermediate\\code} \ar[r] & \toolbox{cdm68k} \ar[r] \ar[d] \ar[rd] & \resource{object file} \\ & \resource{assembly\\listing} & \resource{debugging\\information}}
\seeassembly\seemabk\seeobject\seecode\seedebugging
}

\providecommand{\cdmibl}{
\toolsection{cdmibl} is a compiler for intermediate code targeting the MicroBlaze hardware architecture.
It generates machine code for MicroBlaze processors from programs written in intermediate code and stores it in corresponding object files.
It also creates a debugging information file as well as an assembly file containing a listing of the generated machine code.
\debuggingtool
\flowgraph{\resource{intermediate\\code} \ar[r] & \toolbox{cdmibl} \ar[r] \ar[d] \ar[rd] & \resource{object file} \\ & \resource{assembly\\listing} & \resource{debugging\\information}}
\seeassembly\seemibl\seeobject\seecode\seedebugging
}

\providecommand{\cdmipsa}{
\toolsection{cdmips32} is a compiler for intermediate code targeting the MIPS32 hardware architecture.
It generates machine code for MIPS32 processors from programs written in intermediate code and stores it in corresponding object files.
It also creates a debugging information file as well as an assembly file containing a listing of the generated machine code.
\debuggingtool
\flowgraph{\resource{intermediate\\code} \ar[r] & \toolbox{cdmips32} \ar[r] \ar[d] \ar[rd] & \resource{object file} \\ & \resource{assembly\\listing} & \resource{debugging\\information}}
\seeassembly\seemips\seeobject\seecode\seedebugging
}

\providecommand{\cdmipsb}{
\toolsection{cdmips64} is a compiler for intermediate code targeting the MIPS64 hardware architecture.
It generates machine code for MIPS64 processors from programs written in intermediate code and stores it in corresponding object files.
It also creates a debugging information file as well as an assembly file containing a listing of the generated machine code.
\debuggingtool
\flowgraph{\resource{intermediate\\code} \ar[r] & \toolbox{cdmips64} \ar[r] \ar[d] \ar[rd] & \resource{object file} \\ & \resource{assembly\\listing} & \resource{debugging\\information}}
\seeassembly\seemips\seeobject\seecode\seedebugging
}

\providecommand{\cdmmix}{
\toolsection{cdmmix} is a compiler for intermediate code targeting the MMIX hardware architecture.
It generates machine code for MMIX processors from programs written in intermediate code and stores it in corresponding object files.
It also creates a debugging information file as well as an assembly file containing a listing of the generated machine code.
\debuggingtool
\flowgraph{\resource{intermediate\\code} \ar[r] & \toolbox{cdmmix} \ar[r] \ar[d] \ar[rd] & \resource{object file} \\ & \resource{assembly\\listing} & \resource{debugging\\information}}
\seeassembly\seemmix\seeobject\seecode\seedebugging
}

\providecommand{\cdorok}{
\toolsection{cdor1k} is a compiler for intermediate code targeting the OpenRISC 1000 hardware architecture.
It generates machine code for OpenRISC 1000 processors from programs written in intermediate code and stores it in corresponding object files.
It also creates a debugging information file as well as an assembly file containing a listing of the generated machine code.
\debuggingtool
\flowgraph{\resource{intermediate\\code} \ar[r] & \toolbox{cdor1k} \ar[r] \ar[d] \ar[rd] & \resource{object file} \\ & \resource{assembly\\listing} & \resource{debugging\\information}}
\seeassembly\seeorok\seeobject\seecode\seedebugging
}

\providecommand{\cdppca}{
\toolsection{cdppc32} is a compiler for intermediate code targeting the PowerPC hardware architecture.
It generates machine code for PowerPC processors from programs written in intermediate code and stores it in corresponding object files.
The compiler generates machine code for the 32-bit operating mode defined by the PowerPC architecture.
It also creates a debugging information file as well as an assembly file containing a listing of the generated machine code.
\debuggingtool
\flowgraph{\resource{intermediate\\code} \ar[r] & \toolbox{cdppc32} \ar[r] \ar[d] \ar[rd] & \resource{object file} \\ & \resource{assembly\\listing} & \resource{debugging\\information}}
\seeassembly\seeppc\seeobject\seecode\seedebugging
}

\providecommand{\cdppcb}{
\toolsection{cdppc64} is a compiler for intermediate code targeting the PowerPC hardware architecture.
It generates machine code for PowerPC processors from programs written in intermediate code and stores it in corresponding object files.
The compiler generates machine code for the 64-bit operating mode defined by the PowerPC architecture.
It also creates a debugging information file as well as an assembly file containing a listing of the generated machine code.
\debuggingtool
\flowgraph{\resource{intermediate\\code} \ar[r] & \toolbox{cdppc64} \ar[r] \ar[d] \ar[rd] & \resource{object file} \\ & \resource{assembly\\listing} & \resource{debugging\\information}}
\seeassembly\seeppc\seeobject\seecode\seedebugging
}

\providecommand{\cdrisc}{
\toolsection{cdrisc} is a compiler for intermediate code targeting the RISC hardware architecture.
It generates machine code for RISC processors from programs written in intermediate code and stores it in corresponding object files.
It also creates a debugging information file as well as an assembly file containing a listing of the generated machine code.
\debuggingtool
\flowgraph{\resource{intermediate\\code} \ar[r] & \toolbox{cdrisc} \ar[r] \ar[d] \ar[rd] & \resource{object file} \\ & \resource{assembly\\listing} & \resource{debugging\\information}}
\seeassembly\seerisc\seeobject\seecode\seedebugging
}

\providecommand{\cdwasm}{
\toolsection{cdwasm} is a compiler for intermediate code targeting the WebAssembly architecture.
It generates machine code for WebAssembly targets from programs written in intermediate code and stores it in corresponding object files.
It also creates a debugging information file as well as an assembly file containing a listing of the generated machine code.
\debuggingtool
\flowgraph{\resource{intermediate\\code} \ar[r] & \toolbox{cdwasm} \ar[r] \ar[d] \ar[rd] & \resource{object file} \\ & \resource{assembly\\listing} & \resource{debugging\\information}}
\seeassembly\seewasm\seeobject\seecode\seedebugging
}

% C++ tools

\providecommand{\cppprep}{
\toolsection{cppprep} is a preprocessor for the \cpp{} programming language.
It preprocesses source code according to the rules of \cpp{} and writes it to the standard output stream.
Only the macro names \texttt{\_\_DATE\_\_}, \texttt{\_\_FILE\_\_}, \texttt{\_\_LINE\_\_}, and \texttt{\_\_TIME\_\_} are predefined.
\flowgraph{\resource{\cpp{} or other\\source code} \ar[r] & \toolbox{cppprep} \ar[r] & \resource{preprocessed\\source code} \\ & \variable{ECSINCLUDE} \ar[u]}
\seecpp
}

\providecommand{\cppprint}{
\toolsection{cppprint} is a pretty printer for the \cpp{} programming language.
It reformats the source code of \cpp{} programs and writes it to the standard output stream.
\flowgraph{\resource{\cpp{}\\source code} \ar[r] & \toolbox{cppprint} \ar[r] & \resource{reformatted\\source code} \\ & \variable{ECSINCLUDE} \ar[u]}
\seecpp
}

\providecommand{\cppcheck}{
\toolsection{cppcheck} is a syntactic and semantic checker for the \cpp{} programming language.
It just performs syntactic and semantic checks on \cpp{} programs and writes its diagnostic messages to the standard error stream.
\flowgraph{\resource{\cpp{}\\source code} \ar[r] & \toolbox{cppcheck} \ar[r] & \resource{diagnostic\\messages} \\ & \variable{ECSINCLUDE} \ar[u]}
\seecpp
}

\providecommand{\cppdump}{
\toolsection{cppdump} is a serializer for the \cpp{} programming language.
It dumps the complete internal representation of programs written in \cpp{} into an XML document.
\debuggingtool
\flowgraph{\resource{\cpp{}\\source code} \ar[r] & \toolbox{cppdump} \ar[r] & \resource{internal\\representation} \\ & \variable{ECSINCLUDE} \ar[u]}
\seecpp
}

\providecommand{\cpprun}{
\toolsection{cpprun} is an interpreter for the \cpp{} programming language.
It processes and executes programs written in \cpp{}.
The macro \texttt{\_\_run\_\_} is predefined in order to enable programmers to identify this tool while interpreting.
\flowgraph{\resource{\cpp{}\\source code} \ar[r] & \toolbox{cpprun} \ar@/u/[r] & \resource{input/\\output} \ar@/d/[l] \\ & \variable{ECSINCLUDE} \ar[u]}
\seecpp
}

\providecommand{\cppdoc}{
\toolsection{cppdoc} is a generic documentation generator for the \cpp{} programming language.
It processes several \cpp{} source files and assembles all information therein into a generic documentation.
\debuggingtool
\flowgraph{\resource{\cpp{}\\source code} \ar[r] & \toolbox{cppdoc} \ar[r] & \resource{generic\\documentation} \\ & \variable{ECSINCLUDE} \ar[u]}
\seecpp\seedocumentation
}

\providecommand{\cpphtml}{
\toolsection{cpphtml} is an HTML documentation generator for the \cpp{} programming language.
It processes several \cpp{} source files and assembles all information therein into an HTML document.
\flowgraph{\resource{\cpp{}\\source code} \ar[r] & \toolbox{cpphtml} \ar[r] & \resource{HTML\\document} \\ & \variable{ECSINCLUDE} \ar[u]}
\seecpp\seedocumentation
}

\providecommand{\cpplatex}{
\toolsection{cpplatex} is a Latex documentation generator for the \cpp{} programming language.
It processes several \cpp{} source files and assembles all information therein into a Latex document.
\flowgraph{\resource{\cpp{}\\source code} \ar[r] & \toolbox{cpplatex} \ar[r] & \resource{Latex\\document} \\ & \variable{ECSINCLUDE} \ar[u]}
\seecpp\seedocumentation
}

\providecommand{\cppcode}{
\toolsection{cppcode} is an intermediate code generator for the \cpp{} programming language.
It generates intermediate code from programs written in \cpp{} and stores it in corresponding assembly files.
The macro \texttt{\_\_code\_\_} is predefined in order to enable programmers to identify this tool while generating intermediate code.
Programs generated with this tool require additional runtime support that is stored in the \file{cpp\-code\-run} library file.
\debuggingtool
\flowgraph{\resource{\cpp{}\\source code} \ar[r] & \toolbox{cppcode} \ar[r] & \resource{intermediate\\code} \\ & \variable{ECSINCLUDE} \ar[u]}
\seecpp\seeassembly\seecode
}

\providecommand{\cppamda}{
\toolsection{cppamd16} is a compiler for the \cpp{} programming language targeting the AMD64 hardware architecture.
It generates machine code for AMD64 processors from programs written in \cpp{} and stores it in corresponding object files.
The compiler generates machine code for the 16-bit operating mode defined by the AMD64 architecture.
For debugging purposes, it also creates a debugging information file as well as an assembly file containing a listing of the generated machine code.
The macro \texttt{\_\_amd16\_\_} is predefined in order to enable programmers to identify this tool and its target architecture while compiling.
Programs generated with this compiler require additional runtime support that is stored in the \file{cpp\-amd16\-run} library file.
\flowgraph{\resource{\cpp{}\\source code} \ar[r] & \toolbox{cppamd16} \ar[r] \ar[d] \ar[rd] & \resource{object file} \\ \variable{ECSINCLUDE} \ar[ru] & \resource{debugging\\information} & \resource{assembly\\listing}}
\seecpp\seeassembly\seeamd\seeobject\seedebugging
}

\providecommand{\cppamdb}{
\toolsection{cppamd32} is a compiler for the \cpp{} programming language targeting the AMD64 hardware architecture.
It generates machine code for AMD64 processors from programs written in \cpp{} and stores it in corresponding object files.
The compiler generates machine code for the 32-bit operating mode defined by the AMD64 architecture.
For debugging purposes, it also creates a debugging information file as well as an assembly file containing a listing of the generated machine code.
The macro \texttt{\_\_amd32\_\_} is predefined in order to enable programmers to identify this tool and its target architecture while compiling.
Programs generated with this compiler require additional runtime support that is stored in the \file{cpp\-amd32\-run} library file.
\flowgraph{\resource{\cpp{}\\source code} \ar[r] & \toolbox{cppamd32} \ar[r] \ar[d] \ar[rd] & \resource{object file} \\ \variable{ECSINCLUDE} \ar[ru] & \resource{debugging\\information} & \resource{assembly\\listing}}
\seecpp\seeassembly\seeamd\seeobject\seedebugging
}

\providecommand{\cppamdc}{
\toolsection{cppamd64} is a compiler for the \cpp{} programming language targeting the AMD64 hardware architecture.
It generates machine code for AMD64 processors from programs written in \cpp{} and stores it in corresponding object files.
The compiler generates machine code for the 64-bit operating mode defined by the AMD64 architecture.
For debugging purposes, it also creates a debugging information file as well as an assembly file containing a listing of the generated machine code.
The macro \texttt{\_\_amd64\_\_} is predefined in order to enable programmers to identify this tool and its target architecture while compiling.
Programs generated with this compiler require additional runtime support that is stored in the \file{cpp\-amd64\-run} library file.
\flowgraph{\resource{\cpp{}\\source code} \ar[r] & \toolbox{cppamd64} \ar[r] \ar[d] \ar[rd] & \resource{object file} \\ \variable{ECSINCLUDE} \ar[ru] & \resource{debugging\\information} & \resource{assembly\\listing}}
\seecpp\seeassembly\seeamd\seeobject\seedebugging
}

\providecommand{\cpparma}{
\toolsection{cpparma32} is a compiler for the \cpp{} programming language targeting the ARM hardware architecture.
It generates machine code for ARM processors executing A32 instructions from programs written in \cpp{} and stores it in corresponding object files.
For debugging purposes, it also creates a debugging information file as well as an assembly file containing a listing of the generated machine code.
The macro \texttt{\_\_arma32\_\_} is predefined in order to enable programmers to identify this tool and its target architecture while compiling.
Programs generated with this compiler require additional runtime support that is stored in the \file{cpp\-arma32\-run} library file.
\flowgraph{\resource{\cpp{}\\source code} \ar[r] & \toolbox{cpparma32} \ar[r] \ar[d] \ar[rd] & \resource{object file} \\ \variable{ECSINCLUDE} \ar[ru] & \resource{debugging\\information} & \resource{assembly\\listing}}
\seecpp\seeassembly\seearm\seeobject\seedebugging
}

\providecommand{\cpparmb}{
\toolsection{cpparma64} is a compiler for the \cpp{} programming language targeting the ARM hardware architecture.
It generates machine code for ARM processors executing A64 instructions from programs written in \cpp{} and stores it in corresponding object files.
For debugging purposes, it also creates a debugging information file as well as an assembly file containing a listing of the generated machine code.
The macro \texttt{\_\_arma64\_\_} is predefined in order to enable programmers to identify this tool and its target architecture while compiling.
Programs generated with this compiler require additional runtime support that is stored in the \file{cpp\-arma64\-run} library file.
\flowgraph{\resource{\cpp{}\\source code} \ar[r] & \toolbox{cpparma64} \ar[r] \ar[d] \ar[rd] & \resource{object file} \\ \variable{ECSINCLUDE} \ar[ru] & \resource{debugging\\information} & \resource{assembly\\listing}}
\seecpp\seeassembly\seearm\seeobject\seedebugging
}

\providecommand{\cpparmc}{
\toolsection{cpparmt32} is a compiler for the \cpp{} programming language targeting the ARM hardware architecture.
It generates machine code for ARM processors without floating-point extension executing T32 instructions from programs written in \cpp{} and stores it in corresponding object files.
For debugging purposes, it also creates a debugging information file as well as an assembly file containing a listing of the generated machine code.
The macro \texttt{\_\_armt32\_\_} is predefined in order to enable programmers to identify this tool and its target architecture while compiling.
Programs generated with this compiler require additional runtime support that is stored in the \file{cpp\-armt32\-run} library file.
\flowgraph{\resource{\cpp{}\\source code} \ar[r] & \toolbox{cpparmt32} \ar[r] \ar[d] \ar[rd] & \resource{object file} \\ \variable{ECSINCLUDE} \ar[ru] & \resource{debugging\\information} & \resource{assembly\\listing}}
\seecpp\seeassembly\seearm\seeobject\seedebugging
}

\providecommand{\cpparmcfpe}{
\toolsection{cpparmt32fpe} is a compiler for the \cpp{} programming language targeting the ARM hardware architecture.
It generates machine code for ARM processors with floating-point extension executing T32 instructions from programs written in \cpp{} and stores it in corresponding object files.
For debugging purposes, it also creates a debugging information file as well as an assembly file containing a listing of the generated machine code.
The macro \texttt{\_\_armt32fpe\_\_} is predefined in order to enable programmers to identify this tool and its target architecture while compiling.
Programs generated with this compiler require additional runtime support that is stored in the \file{cpp\-armt32\-fpe\-run} library file.
\flowgraph{\resource{\cpp{}\\source code} \ar[r] & \toolbox{cpparmt32fpe} \ar[r] \ar[d] \ar[rd] & \resource{object file} \\ \variable{ECSINCLUDE} \ar[ru] & \resource{debugging\\information} & \resource{assembly\\listing}}
\seecpp\seeassembly\seearm\seeobject\seedebugging
}

\providecommand{\cppavr}{
\toolsection{cppavr} is a compiler for the \cpp{} programming language targeting the AVR hardware architecture.
It generates machine code for AVR processors from programs written in \cpp{} and stores it in corresponding object files.
For debugging purposes, it also creates a debugging information file as well as an assembly file containing a listing of the generated machine code.
The macro \texttt{\_\_avr\_\_} is predefined in order to enable programmers to identify this tool and its target architecture while compiling.
Programs generated with this compiler require additional runtime support that is stored in the \file{cpp\-avr\-run} library file.
\flowgraph{\resource{\cpp{}\\source code} \ar[r] & \toolbox{cppavr} \ar[r] \ar[d] \ar[rd] & \resource{object file} \\ \variable{ECSINCLUDE} \ar[ru] & \resource{debugging\\information} & \resource{assembly\\listing}}
\seecpp\seeassembly\seeavr\seeobject\seedebugging
}

\providecommand{\cppavrtt}{
\toolsection{cppavr32} is a compiler for the \cpp{} programming language targeting the AVR32 hardware architecture.
It generates machine code for AVR32 processors from programs written in \cpp{} and stores it in corresponding object files.
For debugging purposes, it also creates a debugging information file as well as an assembly file containing a listing of the generated machine code.
The macro \texttt{\_\_avr32\_\_} is predefined in order to enable programmers to identify this tool and its target architecture while compiling.
Programs generated with this compiler require additional runtime support that is stored in the \file{cpp\-avr32\-run} library file.
\flowgraph{\resource{\cpp{}\\source code} \ar[r] & \toolbox{cppavr32} \ar[r] \ar[d] \ar[rd] & \resource{object file} \\ \variable{ECSINCLUDE} \ar[ru] & \resource{debugging\\information} & \resource{assembly\\listing}}
\seecpp\seeassembly\seeavrtt\seeobject\seedebugging
}

\providecommand{\cppmabk}{
\toolsection{cppm68k} is a compiler for the \cpp{} programming language targeting the M68000 hardware architecture.
It generates machine code for M68000 processors from programs written in \cpp{} and stores it in corresponding object files.
For debugging purposes, it also creates a debugging information file as well as an assembly file containing a listing of the generated machine code.
The macro \texttt{\_\_m68k\_\_} is predefined in order to enable programmers to identify this tool and its target architecture while compiling.
Programs generated with this compiler require additional runtime support that is stored in the \file{cpp\-m68k\-run} library file.
\flowgraph{\resource{\cpp{}\\source code} \ar[r] & \toolbox{cppm68k} \ar[r] \ar[d] \ar[rd] & \resource{object file} \\ \variable{ECSINCLUDE} \ar[ru] & \resource{debugging\\information} & \resource{assembly\\listing}}
\seecpp\seeassembly\seemabk\seeobject\seedebugging
}

\providecommand{\cppmibl}{
\toolsection{cppmibl} is a compiler for the \cpp{} programming language targeting the MicroBlaze hardware architecture.
It generates machine code for MicroBlaze processors from programs written in \cpp{} and stores it in corresponding object files.
For debugging purposes, it also creates a debugging information file as well as an assembly file containing a listing of the generated machine code.
The macro \texttt{\_\_mibl\_\_} is predefined in order to enable programmers to identify this tool and its target architecture while compiling.
Programs generated with this compiler require additional runtime support that is stored in the \file{cpp\-mibl\-run} library file.
\flowgraph{\resource{\cpp{}\\source code} \ar[r] & \toolbox{cppmibl} \ar[r] \ar[d] \ar[rd] & \resource{object file} \\ \variable{ECSINCLUDE} \ar[ru] & \resource{debugging\\information} & \resource{assembly\\listing}}
\seecpp\seeassembly\seemibl\seeobject\seedebugging
}

\providecommand{\cppmipsa}{
\toolsection{cppmips32} is a compiler for the \cpp{} programming language targeting the MIPS32 hardware architecture.
It generates machine code for MIPS32 processors from programs written in \cpp{} and stores it in corresponding object files.
For debugging purposes, it also creates a debugging information file as well as an assembly file containing a listing of the generated machine code.
The macro \texttt{\_\_mips32\_\_} is predefined in order to enable programmers to identify this tool and its target architecture while compiling.
Programs generated with this compiler require additional runtime support that is stored in the \file{cpp\-mips32\-run} library file.
\flowgraph{\resource{\cpp{}\\source code} \ar[r] & \toolbox{cppmips32} \ar[r] \ar[d] \ar[rd] & \resource{object file} \\ \variable{ECSINCLUDE} \ar[ru] & \resource{debugging\\information} & \resource{assembly\\listing}}
\seecpp\seeassembly\seemips\seeobject\seedebugging
}

\providecommand{\cppmipsb}{
\toolsection{cppmips64} is a compiler for the \cpp{} programming language targeting the MIPS64 hardware architecture.
It generates machine code for MIPS64 processors from programs written in \cpp{} and stores it in corresponding object files.
For debugging purposes, it also creates a debugging information file as well as an assembly file containing a listing of the generated machine code.
The macro \texttt{\_\_mips64\_\_} is predefined in order to enable programmers to identify this tool and its target architecture while compiling.
Programs generated with this compiler require additional runtime support that is stored in the \file{cpp\-mips64\-run} library file.
\flowgraph{\resource{\cpp{}\\source code} \ar[r] & \toolbox{cppmips64} \ar[r] \ar[d] \ar[rd] & \resource{object file} \\ \variable{ECSINCLUDE} \ar[ru] & \resource{debugging\\information} & \resource{assembly\\listing}}
\seecpp\seeassembly\seemips\seeobject\seedebugging
}

\providecommand{\cppmmix}{
\toolsection{cppmmix} is a compiler for the \cpp{} programming language targeting the MMIX hardware architecture.
It generates machine code for MMIX processors from programs written in \cpp{} and stores it in corresponding object files.
For debugging purposes, it also creates a debugging information file as well as an assembly file containing a listing of the generated machine code.
The macro \texttt{\_\_mmix\_\_} is predefined in order to enable programmers to identify this tool and its target architecture while compiling.
Programs generated with this compiler require additional runtime support that is stored in the \file{cpp\-mmix\-run} library file.
\flowgraph{\resource{\cpp{}\\source code} \ar[r] & \toolbox{cppmmix} \ar[r] \ar[d] \ar[rd] & \resource{object file} \\ \variable{ECSINCLUDE} \ar[ru] & \resource{debugging\\information} & \resource{assembly\\listing}}
\seecpp\seeassembly\seemmix\seeobject\seedebugging
}

\providecommand{\cpporok}{
\toolsection{cppor1k} is a compiler for the \cpp{} programming language targeting the OpenRISC 1000 hardware architecture.
It generates machine code for OpenRISC 1000 processors from programs written in \cpp{} and stores it in corresponding object files.
For debugging purposes, it also creates a debugging information file as well as an assembly file containing a listing of the generated machine code.
The macro \texttt{\_\_or1k\_\_} is predefined in order to enable programmers to identify this tool and its target architecture while compiling.
Programs generated with this compiler require additional runtime support that is stored in the \file{cpp\-or1k\-run} library file.
\flowgraph{\resource{\cpp{}\\source code} \ar[r] & \toolbox{cppor1k} \ar[r] \ar[d] \ar[rd] & \resource{object file} \\ \variable{ECSINCLUDE} \ar[ru] & \resource{debugging\\information} & \resource{assembly\\listing}}
\seecpp\seeassembly\seeorok\seeobject\seedebugging
}

\providecommand{\cppppca}{
\toolsection{cppppc32} is a compiler for the \cpp{} programming language targeting the PowerPC hardware architecture.
It generates machine code for PowerPC processors from programs written in \cpp{} and stores it in corresponding object files.
The compiler generates machine code for the 32-bit operating mode defined by the PowerPC architecture.
For debugging purposes, it also creates a debugging information file as well as an assembly file containing a listing of the generated machine code.
The macro \texttt{\_\_ppc32\_\_} is predefined in order to enable programmers to identify this tool and its target architecture while compiling.
Programs generated with this compiler require additional runtime support that is stored in the \file{cpp\-ppc32\-run} library file.
\flowgraph{\resource{\cpp{}\\source code} \ar[r] & \toolbox{cppppc32} \ar[r] \ar[d] \ar[rd] & \resource{object file} \\ \variable{ECSINCLUDE} \ar[ru] & \resource{debugging\\information} & \resource{assembly\\listing}}
\seecpp\seeassembly\seeppc\seeobject\seedebugging
}

\providecommand{\cppppcb}{
\toolsection{cppppc64} is a compiler for the \cpp{} programming language targeting the PowerPC hardware architecture.
It generates machine code for PowerPC processors from programs written in \cpp{} and stores it in corresponding object files.
The compiler generates machine code for the 64-bit operating mode defined by the PowerPC architecture.
For debugging purposes, it also creates a debugging information file as well as an assembly file containing a listing of the generated machine code.
The macro \texttt{\_\_ppc64\_\_} is predefined in order to enable programmers to identify this tool and its target architecture while compiling.
Programs generated with this compiler require additional runtime support that is stored in the \file{cpp\-ppc64\-run} library file.
\flowgraph{\resource{\cpp{}\\source code} \ar[r] & \toolbox{cppppc64} \ar[r] \ar[d] \ar[rd] & \resource{object file} \\ \variable{ECSINCLUDE} \ar[ru] & \resource{debugging\\information} & \resource{assembly\\listing}}
\seecpp\seeassembly\seeppc\seeobject\seedebugging
}

\providecommand{\cpprisc}{
\toolsection{cpprisc} is a compiler for the \cpp{} programming language targeting the RISC hardware architecture.
It generates machine code for RISC processors from programs written in \cpp{} and stores it in corresponding object files.
For debugging purposes, it also creates a debugging information file as well as an assembly file containing a listing of the generated machine code.
The macro \texttt{\_\_risc\_\_} is predefined in order to enable programmers to identify this tool and its target architecture while compiling.
Programs generated with this compiler require additional runtime support that is stored in the \file{cpp\-risc\-run} library file.
\flowgraph{\resource{\cpp{}\\source code} \ar[r] & \toolbox{cpprisc} \ar[r] \ar[d] \ar[rd] & \resource{object file} \\ \variable{ECSINCLUDE} \ar[ru] & \resource{debugging\\information} & \resource{assembly\\listing}}
\seecpp\seeassembly\seerisc\seeobject\seedebugging
}

\providecommand{\cppwasm}{
\toolsection{cppwasm} is a compiler for the \cpp{} programming language targeting the WebAssembly architecture.
It generates machine code for WebAssembly targets from programs written in \cpp{} and stores it in corresponding object files.
For debugging purposes, it also creates a debugging information file as well as an assembly file containing a listing of the generated machine code.
The macro \texttt{\_\_wasm\_\_} is predefined in order to enable programmers to identify this tool and its target architecture while compiling.
Programs generated with this compiler require additional runtime support that is stored in the \file{cpp\-wasm\-run} library file.
\flowgraph{\resource{\cpp{}\\source code} \ar[r] & \toolbox{cppwasm} \ar[r] \ar[d] \ar[rd] & \resource{object file} \\ \variable{ECSINCLUDE} \ar[ru] & \resource{debugging\\information} & \resource{assembly\\listing}}
\seecpp\seeassembly\seewasm\seeobject\seedebugging
}

% FALSE tools

\providecommand{\falprint}{
\toolsection{falprint} is a pretty printer for the FALSE programming language.
It reformats the source code of FALSE programs and writes it to the standard output stream.
\flowgraph{\resource{FALSE\\source code} \ar[r] & \toolbox{falprint} \ar[r] & \resource{reformatted\\source code}}
\seefalse
}

\providecommand{\falcheck}{
\toolsection{falcheck} is a syntactic and semantic checker for the FALSE programming language.
It just performs syntactic and semantic checks on FALSE programs and writes its diagnostic messages to the standard error stream.
\flowgraph{\resource{FALSE\\source code} \ar[r] & \toolbox{falcheck} \ar[r] & \resource{diagnostic\\messages}}
\seefalse
}

\providecommand{\faldump}{
\toolsection{faldump} is a serializer for the FALSE programming language.
It dumps the complete internal representation of programs written in FALSE into an XML document.
\debuggingtool
\flowgraph{\resource{FALSE\\source code} \ar[r] & \toolbox{faldump} \ar[r] & \resource{internal\\representation}}
\seefalse
}

\providecommand{\falrun}{
\toolsection{falrun} is an interpreter for the FALSE programming language.
It processes and executes programs written in FALSE\@.
\flowgraph{\resource{FALSE\\source code} \ar[r] & \toolbox{falrun} \ar@/u/[r] & \resource{input/\\output} \ar@/d/[l]}
\seefalse
}

\providecommand{\falcpp}{
\toolsection{falcpp} is a transpiler for the FALSE programming language.
It translates programs written in FALSE into \cpp{} programs and stores them in corresponding source files.
\flowgraph{\resource{FALSE\\source code} \ar[r] & \toolbox{falcpp} \ar[r] & \resource{\cpp{}\\source file}}
\seefalse\seecpp
}

\providecommand{\falcode}{
\toolsection{falcode} is an intermediate code generator for the FALSE programming language.
It generates intermediate code from programs written in FALSE and stores it in corresponding assembly files.
\debuggingtool
\flowgraph{\resource{FALSE\\source code} \ar[r] & \toolbox{falcode} \ar[r] & \resource{intermediate\\code}}
\seefalse\seeassembly\seecode
}

\providecommand{\falamda}{
\toolsection{falamd16} is a compiler for the FALSE programming language targeting the AMD64 hardware architecture.
It generates machine code for AMD64 processors from programs written in FALSE and stores it in corresponding object files.
The compiler generates machine code for the 16-bit operating mode defined by the AMD64 architecture.
\flowgraph{\resource{FALSE\\source code} \ar[r] & \toolbox{falamd16} \ar[r] & \resource{object file}}
\seefalse\seeamd\seeobject
}

\providecommand{\falamdb}{
\toolsection{falamd32} is a compiler for the FALSE programming language targeting the AMD64 hardware architecture.
It generates machine code for AMD64 processors from programs written in FALSE and stores it in corresponding object files.
The compiler generates machine code for the 32-bit operating mode defined by the AMD64 architecture.
\flowgraph{\resource{FALSE\\source code} \ar[r] & \toolbox{falamd32} \ar[r] & \resource{object file}}
\seefalse\seeamd\seeobject
}

\providecommand{\falamdc}{
\toolsection{falamd64} is a compiler for the FALSE programming language targeting the AMD64 hardware architecture.
It generates machine code for AMD64 processors from programs written in FALSE and stores it in corresponding object files.
The compiler generates machine code for the 64-bit operating mode defined by the AMD64 architecture.
\flowgraph{\resource{FALSE\\source code} \ar[r] & \toolbox{falamd64} \ar[r] & \resource{object file}}
\seefalse\seeamd\seeobject
}

\providecommand{\falarma}{
\toolsection{falarma32} is a compiler for the FALSE programming language targeting the ARM hardware architecture.
It generates machine code for ARM processors executing A32 instructions from programs written in FALSE and stores it in corresponding object files.
\flowgraph{\resource{FALSE\\source code} \ar[r] & \toolbox{falarma32} \ar[r] & \resource{object file}}
\seefalse\seearm\seeobject
}

\providecommand{\falarmb}{
\toolsection{falarma64} is a compiler for the FALSE programming language targeting the ARM hardware architecture.
It generates machine code for ARM processors executing A64 instructions from programs written in FALSE and stores it in corresponding object files.
\flowgraph{\resource{FALSE\\source code} \ar[r] & \toolbox{falarma64} \ar[r] & \resource{object file}}
\seefalse\seearm\seeobject
}

\providecommand{\falarmc}{
\toolsection{falarmt32} is a compiler for the FALSE programming language targeting the ARM hardware architecture.
It generates machine code for ARM processors without floating-point extension executing T32 instructions from programs written in FALSE and stores it in corresponding object files.
\flowgraph{\resource{FALSE\\source code} \ar[r] & \toolbox{falarmt32} \ar[r] & \resource{object file}}
\seefalse\seearm\seeobject
}

\providecommand{\falarmcfpe}{
\toolsection{falarmt32fpe} is a compiler for the FALSE programming language targeting the ARM hardware architecture.
It generates machine code for ARM processors with floating-point extension executing T32 instructions from programs written in FALSE and stores it in corresponding object files.
\flowgraph{\resource{FALSE\\source code} \ar[r] & \toolbox{falarmt32fpe} \ar[r] & \resource{object file}}
\seefalse\seearm\seeobject
}

\providecommand{\falavr}{
\toolsection{falavr} is a compiler for the FALSE programming language targeting the AVR hardware architecture.
It generates machine code for AVR processors from programs written in FALSE and stores it in corresponding object files.
\flowgraph{\resource{FALSE\\source code} \ar[r] & \toolbox{falavr} \ar[r] & \resource{object file}}
\seefalse\seeavr\seeobject
}

\providecommand{\falavrtt}{
\toolsection{falavr32} is a compiler for the FALSE programming language targeting the AVR32 hardware architecture.
It generates machine code for AVR32 processors from programs written in FALSE and stores it in corresponding object files.
\flowgraph{\resource{FALSE\\source code} \ar[r] & \toolbox{falavr32} \ar[r] & \resource{object file}}
\seefalse\seeavrtt\seeobject
}

\providecommand{\falmabk}{
\toolsection{falm68k} is a compiler for the FALSE programming language targeting the M68000 hardware architecture.
It generates machine code for M68000 processors from programs written in FALSE and stores it in corresponding object files.
\flowgraph{\resource{FALSE\\source code} \ar[r] & \toolbox{falm68k} \ar[r] & \resource{object file}}
\seefalse\seemabk\seeobject
}

\providecommand{\falmibl}{
\toolsection{falmibl} is a compiler for the FALSE programming language targeting the MicroBlaze hardware architecture.
It generates machine code for MicroBlaze processors from programs written in FALSE and stores it in corresponding object files.
\flowgraph{\resource{FALSE\\source code} \ar[r] & \toolbox{falmibl} \ar[r] & \resource{object file}}
\seefalse\seemibl\seeobject
}

\providecommand{\falmipsa}{
\toolsection{falmips32} is a compiler for the FALSE programming language targeting the MIPS32 hardware architecture.
It generates machine code for MIPS32 processors from programs written in FALSE and stores it in corresponding object files.
\flowgraph{\resource{FALSE\\source code} \ar[r] & \toolbox{falmips32} \ar[r] & \resource{object file}}
\seefalse\seemips\seeobject
}

\providecommand{\falmipsb}{
\toolsection{falmips64} is a compiler for the FALSE programming language targeting the MIPS64 hardware architecture.
It generates machine code for MIPS64 processors from programs written in FALSE and stores it in corresponding object files.
\flowgraph{\resource{FALSE\\source code} \ar[r] & \toolbox{falmips64} \ar[r] & \resource{object file}}
\seefalse\seemips\seeobject
}

\providecommand{\falmmix}{
\toolsection{falmmix} is a compiler for the FALSE programming language targeting the MMIX hardware architecture.
It generates machine code for MMIX processors from programs written in FALSE and stores it in corresponding object files.
\flowgraph{\resource{FALSE\\source code} \ar[r] & \toolbox{falmmix} \ar[r] & \resource{object file}}
\seefalse\seemmix\seeobject
}

\providecommand{\falorok}{
\toolsection{falor1k} is a compiler for the FALSE programming language targeting the OpenRISC 1000 hardware architecture.
It generates machine code for OpenRISC 1000 processors from programs written in FALSE and stores it in corresponding object files.
\flowgraph{\resource{FALSE\\source code} \ar[r] & \toolbox{falor1k} \ar[r] & \resource{object file}}
\seefalse\seeorok\seeobject
}

\providecommand{\falppca}{
\toolsection{falppc32} is a compiler for the FALSE programming language targeting the PowerPC hardware architecture.
It generates machine code for PowerPC processors from programs written in FALSE and stores it in corresponding object files.
The compiler generates machine code for the 32-bit operating mode defined by the PowerPC architecture.
\flowgraph{\resource{FALSE\\source code} \ar[r] & \toolbox{falppc32} \ar[r] & \resource{object file}}
\seefalse\seeppc\seeobject
}

\providecommand{\falppcb}{
\toolsection{falppc64} is a compiler for the FALSE programming language targeting the PowerPC hardware architecture.
It generates machine code for PowerPC processors from programs written in FALSE and stores it in corresponding object files.
The compiler generates machine code for the 64-bit operating mode defined by the PowerPC architecture.
\flowgraph{\resource{FALSE\\source code} \ar[r] & \toolbox{falppc64} \ar[r] & \resource{object file}}
\seefalse\seeppc\seeobject
}

\providecommand{\falrisc}{
\toolsection{falrisc} is a compiler for the FALSE programming language targeting the RISC hardware architecture.
It generates machine code for RISC processors from programs written in FALSE and stores it in corresponding object files.
\flowgraph{\resource{FALSE\\source code} \ar[r] & \toolbox{falrisc} \ar[r] & \resource{object file}}
\seefalse\seerisc\seeobject
}

\providecommand{\falwasm}{
\toolsection{falwasm} is a compiler for the FALSE programming language targeting the WebAssembly architecture.
It generates machine code for WebAssembly targets from programs written in FALSE and stores it in corresponding object files.
\flowgraph{\resource{FALSE\\source code} \ar[r] & \toolbox{falwasm} \ar[r] & \resource{object file}}
\seefalse\seewasm\seeobject
}

% Oberon tools

\providecommand{\obprint}{
\toolsection{obprint} is a pretty printer for the Oberon programming language.
It reformats the source code of Oberon modules and writes it to the standard output stream.
\flowgraph{\resource{Oberon\\source code} \ar[r] & \toolbox{obprint} \ar[r] & \resource{reformatted\\source code}}
\seeoberon
}

\providecommand{\obcheck}{
\toolsection{obcheck} is a syntactic and semantic checker for the Oberon programming language.
It just performs syntactic and semantic checks on Oberon modules and writes its diagnostic messages to the standard error stream.
In addition, it stores the interface of each module in a symbol file which is required when other modules import the module.
\flowgraph{\resource{Oberon\\source code} \ar[r] & \toolbox{obcheck} \ar[r] \ar@/l/[d] & \resource{diagnostic\\messages} \\ \variable{ECSIMPORT} \ar[ru] & \resource{symbol\\files} \ar@/r/[u]}
\seeoberon
}

\providecommand{\obdump}{
\toolsection{obdump} is a serializer for the Oberon programming language.
It dumps the complete internal representation of modules written in Oberon into an XML document.
\debuggingtool
\flowgraph{\resource{Oberon\\source code} \ar[r] & \toolbox{obdump} \ar[r] \ar@/l/[d] & \resource{internal\\representation} \\ \variable{ECSIMPORT} \ar[ru] & \resource{symbol\\files} \ar@/r/[u]}
\seeoberon
}

\providecommand{\obrun}{
\toolsection{obrun} is an interpreter for the Oberon programming language.
It processes and executes modules written in Oberon.
This tool does neither generate nor process symbol files while interpreting modules.
If a module is imported by another one, its filename has to be named before the other one in the list of command-line arguments.
\flowgraph{\resource{Oberon\\source code} \ar[r] & \toolbox{obrun} \ar@/u/[r] & \resource{input/\\output} \ar@/d/[l]}
\seeoberon
}

\providecommand{\obcpp}{
\toolsection{obcpp} is a transpiler for the Oberon programming language.
It translates programs written in Oberon into \cpp{} programs and stores them in corresponding source and header files.
In addition, it stores the interface of each module in a symbol file which is required when other modules import the module.
The same interface is provided by the generated header file which can be used in other parts of the \cpp{} program.
\flowgraph{\resource{Oberon\\source code} \ar[r] & \toolbox{obcpp} \ar[r] \ar@/l/[d] \ar[rd] & \resource{\cpp{}\\source file} \\ \variable{ECSIMPORT} \ar[ru] & \resource{symbol\\files} \ar@/r/[u] & \resource{\cpp{}\\header file}}
\seeoberon\seecpp
}

\providecommand{\obdoc}{
\toolsection{obdoc} is a generic documentation generator for the Oberon programming language.
It processes several Oberon modules and assembles all information therein into a generic documentation.
In addition, it stores the interface of each module in a symbol file which is required when other modules import the module.
\debuggingtool
\flowgraph{\resource{Oberon\\source code} \ar[r] & \toolbox{obdoc} \ar[r] \ar@/l/[d] & \resource{generic\\documentation} \\ \variable{ECSIMPORT} \ar[ru] & \resource{symbol\\files} \ar@/r/[u]}
\seeoberon\seedocumentation
}

\providecommand{\obhtml}{
\toolsection{obhtml} is an HTML documentation generator for the Oberon programming language.
It processes several Oberon modules and assembles all information therein into an HTML document.
In addition, it stores the interface of each module in a symbol file which is required when other modules import the module.
\flowgraph{\resource{Oberon\\source code} \ar[r] & \toolbox{obhtml} \ar[r] \ar@/l/[d] & \resource{HTML\\document} \\ \variable{ECSIMPORT} \ar[ru] & \resource{symbol\\files} \ar@/r/[u]}
\seeoberon\seedocumentation
}

\providecommand{\oblatex}{
\toolsection{oblatex} is a Latex documentation generator for the Oberon programming language.
It processes several Oberon modules and assembles all information therein into a Latex document.
In addition, it stores the interface of each module in a symbol file which is required when other modules import the module.
\flowgraph{\resource{Oberon\\source code} \ar[r] & \toolbox{oblatex} \ar[r] \ar@/l/[d] & \resource{Latex\\document} \\ \variable{ECSIMPORT} \ar[ru] & \resource{symbol\\files} \ar@/r/[u]}
\seeoberon\seedocumentation
}

\providecommand{\obcode}{
\toolsection{obcode} is an intermediate code generator for the Oberon programming language.
It generates intermediate code from modules written in Oberon and stores it in corresponding assembly files.
In addition, it stores the interface of each module in a symbol file which is required when other modules import the module.
Programs generated with this tool require additional runtime support that is stored in the \file{ob\-code\-run} library file.
\debuggingtool
\flowgraph{\resource{Oberon\\source code} \ar[r] & \toolbox{obcode} \ar[r] \ar@/l/[d] & \resource{intermediate\\code} \\ \variable{ECSIMPORT} \ar[ru] & \resource{symbol\\files} \ar@/r/[u]}
\seeoberon\seeassembly\seecode
}

\providecommand{\obamda}{
\toolsection{obamd16} is a compiler for the Oberon programming language targeting the AMD64 hardware architecture.
It generates machine code for AMD64 processors from modules written in Oberon and stores it in corresponding object files.
The compiler generates machine code for the 16-bit operating mode defined by the AMD64 architecture.
For debugging purposes, it also creates a debugging information file as well as an assembly file containing a listing of the generated machine code.
In addition, it stores the interface of each module in a symbol file which is required when other modules import the module.
Programs generated with this compiler require additional runtime support that is stored in the \file{ob\-amd16\-run} library file.
\flowgraph{\resource{Oberon\\source code} \ar[r] & \toolbox{obamd16} \ar[r] \ar@/l/[d] \ar[rd] & \resource{object file} \\ \variable{ECSIMPORT} \ar[ru] & \resource{symbol\\files} \ar@/r/[u] & \resource{debugging\\information}}
\seeoberon\seeassembly\seeamd\seeobject\seedebugging
}

\providecommand{\obamdb}{
\toolsection{obamd32} is a compiler for the Oberon programming language targeting the AMD64 hardware architecture.
It generates machine code for AMD64 processors from modules written in Oberon and stores it in corresponding object files.
The compiler generates machine code for the 32-bit operating mode defined by the AMD64 architecture.
For debugging purposes, it also creates a debugging information file as well as an assembly file containing a listing of the generated machine code.
In addition, it stores the interface of each module in a symbol file which is required when other modules import the module.
Programs generated with this compiler require additional runtime support that is stored in the \file{ob\-amd32\-run} library file.
\flowgraph{\resource{Oberon\\source code} \ar[r] & \toolbox{obamd32} \ar[r] \ar@/l/[d] \ar[rd] & \resource{object file} \\ \variable{ECSIMPORT} \ar[ru] & \resource{symbol\\files} \ar@/r/[u] & \resource{debugging\\information}}
\seeoberon\seeassembly\seeamd\seeobject\seedebugging
}

\providecommand{\obamdc}{
\toolsection{obamd64} is a compiler for the Oberon programming language targeting the AMD64 hardware architecture.
It generates machine code for AMD64 processors from modules written in Oberon and stores it in corresponding object files.
The compiler generates machine code for the 64-bit operating mode defined by the AMD64 architecture.
For debugging purposes, it also creates a debugging information file as well as an assembly file containing a listing of the generated machine code.
In addition, it stores the interface of each module in a symbol file which is required when other modules import the module.
Programs generated with this compiler require additional runtime support that is stored in the \file{ob\-amd64\-run} library file.
\flowgraph{\resource{Oberon\\source code} \ar[r] & \toolbox{obamd64} \ar[r] \ar@/l/[d] \ar[rd] & \resource{object file} \\ \variable{ECSIMPORT} \ar[ru] & \resource{symbol\\files} \ar@/r/[u] & \resource{debugging\\information}}
\seeoberon\seeassembly\seeamd\seeobject\seedebugging
}

\providecommand{\obarma}{
\toolsection{obarma32} is a compiler for the Oberon programming language targeting the ARM hardware architecture.
It generates machine code for ARM processors executing A32 instructions from modules written in Oberon and stores it in corresponding object files.
For debugging purposes, it also creates a debugging information file as well as an assembly file containing a listing of the generated machine code.
In addition, it stores the interface of each module in a symbol file which is required when other modules import the module.
Programs generated with this compiler require additional runtime support that is stored in the \file{ob\-arma32\-run} library file.
\flowgraph{\resource{Oberon\\source code} \ar[r] & \toolbox{obarma32} \ar[r] \ar@/l/[d] \ar[rd] & \resource{object file} \\ \variable{ECSIMPORT} \ar[ru] & \resource{symbol\\files} \ar@/r/[u] & \resource{debugging\\information}}
\seeoberon\seeassembly\seearm\seeobject\seedebugging
}

\providecommand{\obarmb}{
\toolsection{obarma64} is a compiler for the Oberon programming language targeting the ARM hardware architecture.
It generates machine code for ARM processors executing A64 instructions from modules written in Oberon and stores it in corresponding object files.
For debugging purposes, it also creates a debugging information file as well as an assembly file containing a listing of the generated machine code.
In addition, it stores the interface of each module in a symbol file which is required when other modules import the module.
Programs generated with this compiler require additional runtime support that is stored in the \file{ob\-arma64\-run} library file.
\flowgraph{\resource{Oberon\\source code} \ar[r] & \toolbox{obarma64} \ar[r] \ar@/l/[d] \ar[rd] & \resource{object file} \\ \variable{ECSIMPORT} \ar[ru] & \resource{symbol\\files} \ar@/r/[u] & \resource{debugging\\information}}
\seeoberon\seeassembly\seearm\seeobject\seedebugging
}

\providecommand{\obarmc}{
\toolsection{obarmt32} is a compiler for the Oberon programming language targeting the ARM hardware architecture.
It generates machine code for ARM processors without floating-point extension executing T32 instructions from modules written in Oberon and stores it in corresponding object files.
For debugging purposes, it also creates a debugging information file as well as an assembly file containing a listing of the generated machine code.
In addition, it stores the interface of each module in a symbol file which is required when other modules import the module.
Programs generated with this compiler require additional runtime support that is stored in the \file{ob\-armt32\-run} library file.
\flowgraph{\resource{Oberon\\source code} \ar[r] & \toolbox{obarmt32} \ar[r] \ar@/l/[d] \ar[rd] & \resource{object file} \\ \variable{ECSIMPORT} \ar[ru] & \resource{symbol\\files} \ar@/r/[u] & \resource{debugging\\information}}
\seeoberon\seeassembly\seearm\seeobject\seedebugging
}

\providecommand{\obarmcfpe}{
\toolsection{obarmt32fpe} is a compiler for the Oberon programming language targeting the ARM hardware architecture.
It generates machine code for ARM processors with floating-point extension executing T32 instructions from modules written in Oberon and stores it in corresponding object files.
For debugging purposes, it also creates a debugging information file as well as an assembly file containing a listing of the generated machine code.
In addition, it stores the interface of each module in a symbol file which is required when other modules import the module.
Programs generated with this compiler require additional runtime support that is stored in the \file{ob\-armt32\-fpe\-run} library file.
\flowgraph{\resource{Oberon\\source code} \ar[r] & \toolbox{obarmt32fpe} \ar[r] \ar@/l/[d] \ar[rd] & \resource{object file} \\ \variable{ECSIMPORT} \ar[ru] & \resource{symbol\\files} \ar@/r/[u] & \resource{debugging\\information}}
\seeoberon\seeassembly\seearm\seeobject\seedebugging
}

\providecommand{\obavr}{
\toolsection{obavr} is a compiler for the Oberon programming language targeting the AVR hardware architecture.
It generates machine code for AVR processors from modules written in Oberon and stores it in corresponding object files.
For debugging purposes, it also creates a debugging information file as well as an assembly file containing a listing of the generated machine code.
In addition, it stores the interface of each module in a symbol file which is required when other modules import the module.
Programs generated with this compiler require additional runtime support that is stored in the \file{ob\-avr\-run} library file.
\flowgraph{\resource{Oberon\\source code} \ar[r] & \toolbox{obavr} \ar[r] \ar@/l/[d] \ar[rd] & \resource{object file} \\ \variable{ECSIMPORT} \ar[ru] & \resource{symbol\\files} \ar@/r/[u] & \resource{debugging\\information}}
\seeoberon\seeassembly\seeavr\seeobject\seedebugging
}

\providecommand{\obavrtt}{
\toolsection{obavr32} is a compiler for the Oberon programming language targeting the AVR32 hardware architecture.
It generates machine code for AVR32 processors from modules written in Oberon and stores it in corresponding object files.
For debugging purposes, it also creates a debugging information file as well as an assembly file containing a listing of the generated machine code.
In addition, it stores the interface of each module in a symbol file which is required when other modules import the module.
Programs generated with this compiler require additional runtime support that is stored in the \file{ob\-avr32\-run} library file.
\flowgraph{\resource{Oberon\\source code} \ar[r] & \toolbox{obavr32} \ar[r] \ar@/l/[d] \ar[rd] & \resource{object file} \\ \variable{ECSIMPORT} \ar[ru] & \resource{symbol\\files} \ar@/r/[u] & \resource{debugging\\information}}
\seeoberon\seeassembly\seeavrtt\seeobject\seedebugging
}

\providecommand{\obmabk}{
\toolsection{obm68k} is a compiler for the Oberon programming language targeting the M68000 hardware architecture.
It generates machine code for M68000 processors from modules written in Oberon and stores it in corresponding object files.
For debugging purposes, it also creates a debugging information file as well as an assembly file containing a listing of the generated machine code.
In addition, it stores the interface of each module in a symbol file which is required when other modules import the module.
Programs generated with this compiler require additional runtime support that is stored in the \file{ob\-m68k\-run} library file.
\flowgraph{\resource{Oberon\\source code} \ar[r] & \toolbox{obm68k} \ar[r] \ar@/l/[d] \ar[rd] & \resource{object file} \\ \variable{ECSIMPORT} \ar[ru] & \resource{symbol\\files} \ar@/r/[u] & \resource{debugging\\information}}
\seeoberon\seeassembly\seemabk\seeobject\seedebugging
}

\providecommand{\obmibl}{
\toolsection{obmibl} is a compiler for the Oberon programming language targeting the MicroBlaze hardware architecture.
It generates machine code for MicroBlaze processors from modules written in Oberon and stores it in corresponding object files.
For debugging purposes, it also creates a debugging information file as well as an assembly file containing a listing of the generated machine code.
In addition, it stores the interface of each module in a symbol file which is required when other modules import the module.
Programs generated with this compiler require additional runtime support that is stored in the \file{ob\-mibl\-run} library file.
\flowgraph{\resource{Oberon\\source code} \ar[r] & \toolbox{obmibl} \ar[r] \ar@/l/[d] \ar[rd] & \resource{object file} \\ \variable{ECSIMPORT} \ar[ru] & \resource{symbol\\files} \ar@/r/[u] & \resource{debugging\\information}}
\seeoberon\seeassembly\seemibl\seeobject\seedebugging
}

\providecommand{\obmipsa}{
\toolsection{obmips32} is a compiler for the Oberon programming language targeting the MIPS32 hardware architecture.
It generates machine code for MIPS32 processors from modules written in Oberon and stores it in corresponding object files.
For debugging purposes, it also creates a debugging information file as well as an assembly file containing a listing of the generated machine code.
In addition, it stores the interface of each module in a symbol file which is required when other modules import the module.
Programs generated with this compiler require additional runtime support that is stored in the \file{ob\-mips32\-run} library file.
\flowgraph{\resource{Oberon\\source code} \ar[r] & \toolbox{obmips32} \ar[r] \ar@/l/[d] \ar[rd] & \resource{object file} \\ \variable{ECSIMPORT} \ar[ru] & \resource{symbol\\files} \ar@/r/[u] & \resource{debugging\\information}}
\seeoberon\seeassembly\seemips\seeobject\seedebugging
}

\providecommand{\obmipsb}{
\toolsection{obmips64} is a compiler for the Oberon programming language targeting the MIPS64 hardware architecture.
It generates machine code for MIPS64 processors from modules written in Oberon and stores it in corresponding object files.
For debugging purposes, it also creates a debugging information file as well as an assembly file containing a listing of the generated machine code.
In addition, it stores the interface of each module in a symbol file which is required when other modules import the module.
Programs generated with this compiler require additional runtime support that is stored in the \file{ob\-mips64\-run} library file.
\flowgraph{\resource{Oberon\\source code} \ar[r] & \toolbox{obmips64} \ar[r] \ar@/l/[d] \ar[rd] & \resource{object file} \\ \variable{ECSIMPORT} \ar[ru] & \resource{symbol\\files} \ar@/r/[u] & \resource{debugging\\information}}
\seeoberon\seeassembly\seemips\seeobject\seedebugging
}

\providecommand{\obmmix}{
\toolsection{obmmix} is a compiler for the Oberon programming language targeting the MMIX hardware architecture.
It generates machine code for MMIX processors from modules written in Oberon and stores it in corresponding object files.
For debugging purposes, it also creates a debugging information file as well as an assembly file containing a listing of the generated machine code.
In addition, it stores the interface of each module in a symbol file which is required when other modules import the module.
Programs generated with this compiler require additional runtime support that is stored in the \file{ob\-mmix\-run} library file.
\flowgraph{\resource{Oberon\\source code} \ar[r] & \toolbox{obmmix} \ar[r] \ar@/l/[d] \ar[rd] & \resource{object file} \\ \variable{ECSIMPORT} \ar[ru] & \resource{symbol\\files} \ar@/r/[u] & \resource{debugging\\information}}
\seeoberon\seeassembly\seemmix\seeobject\seedebugging
}

\providecommand{\oborok}{
\toolsection{obor1k} is a compiler for the Oberon programming language targeting the OpenRISC 1000 hardware architecture.
It generates machine code for OpenRISC 1000 processors from modules written in Oberon and stores it in corresponding object files.
For debugging purposes, it also creates a debugging information file as well as an assembly file containing a listing of the generated machine code.
In addition, it stores the interface of each module in a symbol file which is required when other modules import the module.
Programs generated with this compiler require additional runtime support that is stored in the \file{ob\-or1k\-run} library file.
\flowgraph{\resource{Oberon\\source code} \ar[r] & \toolbox{obor1k} \ar[r] \ar@/l/[d] \ar[rd] & \resource{object file} \\ \variable{ECSIMPORT} \ar[ru] & \resource{symbol\\files} \ar@/r/[u] & \resource{debugging\\information}}
\seeoberon\seeassembly\seeorok\seeobject\seedebugging
}

\providecommand{\obppca}{
\toolsection{obppc32} is a compiler for the Oberon programming language targeting the PowerPC hardware architecture.
It generates machine code for PowerPC processors from modules written in Oberon and stores it in corresponding object files.
The compiler generates machine code for the 32-bit operating mode defined by the PowerPC architecture.
For debugging purposes, it also creates a debugging information file as well as an assembly file containing a listing of the generated machine code.
In addition, it stores the interface of each module in a symbol file which is required when other modules import the module.
Programs generated with this compiler require additional runtime support that is stored in the \file{ob\-ppc32\-run} library file.
\flowgraph{\resource{Oberon\\source code} \ar[r] & \toolbox{obppc32} \ar[r] \ar@/l/[d] \ar[rd] & \resource{object file} \\ \variable{ECSIMPORT} \ar[ru] & \resource{symbol\\files} \ar@/r/[u] & \resource{debugging\\information}}
\seeoberon\seeassembly\seeppc\seeobject\seedebugging
}

\providecommand{\obppcb}{
\toolsection{obppc64} is a compiler for the Oberon programming language targeting the PowerPC hardware architecture.
It generates machine code for PowerPC processors from modules written in Oberon and stores it in corresponding object files.
The compiler generates machine code for the 64-bit operating mode defined by the PowerPC architecture.
For debugging purposes, it also creates a debugging information file as well as an assembly file containing a listing of the generated machine code.
In addition, it stores the interface of each module in a symbol file which is required when other modules import the module.
Programs generated with this compiler require additional runtime support that is stored in the \file{ob\-ppc64\-run} library file.
\flowgraph{\resource{Oberon\\source code} \ar[r] & \toolbox{obppc64} \ar[r] \ar@/l/[d] \ar[rd] & \resource{object file} \\ \variable{ECSIMPORT} \ar[ru] & \resource{symbol\\files} \ar@/r/[u] & \resource{debugging\\information}}
\seeoberon\seeassembly\seeppc\seeobject\seedebugging
}

\providecommand{\obrisc}{
\toolsection{obrisc} is a compiler for the Oberon programming language targeting the RISC hardware architecture.
It generates machine code for RISC processors from modules written in Oberon and stores it in corresponding object files.
For debugging purposes, it also creates a debugging information file as well as an assembly file containing a listing of the generated machine code.
In addition, it stores the interface of each module in a symbol file which is required when other modules import the module.
Programs generated with this compiler require additional runtime support that is stored in the \file{ob\-risc\-run} library file.
\flowgraph{\resource{Oberon\\source code} \ar[r] & \toolbox{obrisc} \ar[r] \ar@/l/[d] \ar[rd] & \resource{object file} \\ \variable{ECSIMPORT} \ar[ru] & \resource{symbol\\files} \ar@/r/[u] & \resource{debugging\\information}}
\seeoberon\seeassembly\seerisc\seeobject\seedebugging
}

\providecommand{\obwasm}{
\toolsection{obwasm} is a compiler for the Oberon programming language targeting the WebAssembly architecture.
It generates machine code for WebAssembly targets from modules written in Oberon and stores it in corresponding object files.
For debugging purposes, it also creates a debugging information file as well as an assembly file containing a listing of the generated machine code.
In addition, it stores the interface of each module in a symbol file which is required when other modules import the module.
Programs generated with this compiler require additional runtime support that is stored in the \file{ob\-wasm\-run} library file.
\flowgraph{\resource{Oberon\\source code} \ar[r] & \toolbox{obwasm} \ar[r] \ar@/l/[d] \ar[rd] & \resource{object file} \\ \variable{ECSIMPORT} \ar[ru] & \resource{symbol\\files} \ar@/r/[u] & \resource{debugging\\information}}
\seeoberon\seeassembly\seewasm\seeobject\seedebugging
}

% converter tools

\providecommand{\dbgdwarf}{
\toolsection{dbgdwarf} is a DWARF debugging information converter tool.
It converts debugging information into the DWARF debugging data format and stores it in corresponding object files~\cite{dwarffile}.
The resulting debugging object files can be combined with runtime support that creates Executable and Linking Format (ELF) files~\cite{elffile}.
\flowgraph{\resource{debugging\\information} \ar[r] & \toolbox{dbgdwarf} \ar[r] & \resource{debugging\\object file}}
\seeobject\seedebugging
}

% assembler tools

\providecommand{\asmprint}{
\toolsection{asmprint} is a pretty printer for generic assembly code.
It reformats generic assembly code and writes it to the standard output stream.
\flowgraph{\resource{generic assembly\\source code} \ar[r] & \toolbox{asmprint} \ar[r] & \resource{reformatted\\source code}}
\seeassembly
}

\providecommand{\amdaasm}{
\toolsection{amd16asm} is an assembler for the AMD64 hardware architecture.
It translates assembly code into machine code for AMD64 processors and stores it in corresponding object files.
By default, the assembler generates machine code for the 16-bit operating mode defined by the AMD64 architecture.
\flowgraph{\resource{AMD16 assembly\\source code} \ar[r] & \toolbox{amd16asm} \ar[r] & \resource{object file}}
\seeassembly\seeamd\seeobject
}

\providecommand{\amdadism}{
\toolsection{amd16dism} is a disassembler for the AMD64 hardware architecture.
It translates machine code from object files targeting AMD64 processors into assembly code and writes it to the standard output stream.
It assumes that the machine code was generated for the 16-bit operating mode defined by the AMD64 architecture.
\flowgraph{\resource{object file} \ar[r] & \toolbox{amd16dism} \ar[r] & \resource{disassembly\\listing}}
\seeassembly\seeamd\seeobject
}

\providecommand{\amdbasm}{
\toolsection{amd32asm} is an assembler for the AMD64 hardware architecture.
It translates assembly code into machine code for AMD64 processors and stores it in corresponding object files.
By default, the assembler generates machine code for the 32-bit operating mode defined by the AMD64 architecture.
\flowgraph{\resource{AMD32 assembly\\source code} \ar[r] & \toolbox{amd32asm} \ar[r] & \resource{object file}}
\seeassembly\seeamd\seeobject
}

\providecommand{\amdbdism}{
\toolsection{amd32dism} is a disassembler for the AMD64 hardware architecture.
It translates machine code from object files targeting AMD64 processors into assembly code and writes it to the standard output stream.
It assumes that the machine code was generated for the 32-bit operating mode defined by the AMD64 architecture.
\flowgraph{\resource{object file} \ar[r] & \toolbox{amd32dism} \ar[r] & \resource{disassembly\\listing}}
\seeassembly\seeamd\seeobject
}

\providecommand{\amdcasm}{
\toolsection{amd64asm} is an assembler for the AMD64 hardware architecture.
It translates assembly code into machine code for AMD64 processors and stores it in corresponding object files.
By default, the assembler generates machine code for the 64-bit operating mode defined by the AMD64 architecture.
\flowgraph{\resource{AMD64 assembly\\source code} \ar[r] & \toolbox{amd64asm} \ar[r] & \resource{object file}}
\seeassembly\seeamd\seeobject
}

\providecommand{\amdcdism}{
\toolsection{amd64dism} is a disassembler for the AMD64 hardware architecture.
It translates machine code from object files targeting AMD64 processors into assembly code and writes it to the standard output stream.
It assumes that the machine code was generated for the 64-bit operating mode defined by the AMD64 architecture.
\flowgraph{\resource{object file} \ar[r] & \toolbox{amd64dism} \ar[r] & \resource{disassembly\\listing}}
\seeassembly\seeamd\seeobject
}

\providecommand{\armaasm}{
\toolsection{arma32asm} is an assembler for the ARM hardware architecture.
It translates assembly code into machine code for ARM processors executing A32 instructions and stores it in corresponding object files.
\flowgraph{\resource{ARM A32 assembly\\source code} \ar[r] & \toolbox{arma32asm} \ar[r] & \resource{object file}}
\seeassembly\seearm\seeobject
}

\providecommand{\armadism}{
\toolsection{arma32dism} is a disassembler for the ARM hardware architecture.
It translates machine code from object files targeting ARM processors executing A32 instructions into assembly code and writes it to the standard output stream.
\flowgraph{\resource{object file} \ar[r] & \toolbox{arma32dism} \ar[r] & \resource{disassembly\\listing}}
\seeassembly\seearm\seeobject
}

\providecommand{\armbasm}{
\toolsection{arma64asm} is an assembler for the ARM hardware architecture.
It translates assembly code into machine code for ARM processors executing A64 instructions and stores it in corresponding object files.
\flowgraph{\resource{ARM A64 assembly\\source code} \ar[r] & \toolbox{arma64asm} \ar[r] & \resource{object file}}
\seeassembly\seearm\seeobject
}

\providecommand{\armbdism}{
\toolsection{arma64dism} is a disassembler for the ARM hardware architecture.
It translates machine code from object files targeting ARM processors executing A64 instructions into assembly code and writes it to the standard output stream.
\flowgraph{\resource{object file} \ar[r] & \toolbox{arma64dism} \ar[r] & \resource{disassembly\\listing}}
\seeassembly\seearm\seeobject
}

\providecommand{\armcasm}{
\toolsection{armt32asm} is an assembler for the ARM hardware architecture.
It translates assembly code into machine code for ARM processors executing T32 instructions and stores it in corresponding object files.
\flowgraph{\resource{ARM T32 assembly\\source code} \ar[r] & \toolbox{armt32asm} \ar[r] & \resource{object file}}
\seeassembly\seearm\seeobject
}

\providecommand{\armcdism}{
\toolsection{armt32dism} is a disassembler for the ARM hardware architecture.
It translates machine code from object files targeting ARM processors executing T32 instructions into assembly code and writes it to the standard output stream.
\flowgraph{\resource{object file} \ar[r] & \toolbox{armt32dism} \ar[r] & \resource{disassembly\\listing}}
\seeassembly\seearm\seeobject
}

\providecommand{\avrasm}{
\toolsection{avrasm} is an assembler for the AVR hardware architecture.
It translates assembly code into machine code for AVR processors and stores it in corresponding object files.
The identifiers \texttt{RXL}, \texttt{RXH}, \texttt{RYL}, \texttt{RYH}, \texttt{RZL}, and \texttt{RZH} are predefined and name the corresponding registers.
The identifiers \texttt{SPL} and \texttt{SPH} are also predefined and evaluate to the address of the corresponding registers.
\flowgraph{\resource{AVR assembly\\source code} \ar[r] & \toolbox{avrasm} \ar[r] & \resource{object file}}
\seeassembly\seeavr\seeobject
}

\providecommand{\avrdism}{
\toolsection{avrdism} is a disassembler for the AVR hardware architecture.
It translates machine code from object files targeting AVR processors into assembly code and writes it to the standard output stream.
\flowgraph{\resource{object file} \ar[r] & \toolbox{avrdism} \ar[r] & \resource{disassembly\\listing}}
\seeassembly\seeavr\seeobject
}

\providecommand{\avrttasm}{
\toolsection{avr32asm} is an assembler for the AVR32 hardware architecture.
It translates assembly code into machine code for AVR32 processors and stores it in corresponding object files.
\flowgraph{\resource{AVR32 assembly\\source code} \ar[r] & \toolbox{avr32asm} \ar[r] & \resource{object file}}
\seeassembly\seeavrtt\seeobject
}

\providecommand{\avrttdism}{
\toolsection{avr32dism} is a disassembler for the AVR32 hardware architecture.
It translates machine code from object files targeting AVR32 processors into assembly code and writes it to the standard output stream.
\flowgraph{\resource{object file} \ar[r] & \toolbox{avr32dism} \ar[r] & \resource{disassembly\\listing}}
\seeassembly\seeavrtt\seeobject
}

\providecommand{\mabkasm}{
\toolsection{m68kasm} is an assembler for the M68000 hardware architecture.
It translates assembly code into machine code for M68000 processors and stores it in corresponding object files.
\flowgraph{\resource{68000 assembly\\source code} \ar[r] & \toolbox{m68kasm} \ar[r] & \resource{object file}}
\seeassembly\seemabk\seeobject
}

\providecommand{\mabkdism}{
\toolsection{m68kdism} is a disassembler for the M68000 hardware architecture.
It translates machine code from object files targeting M68000 processors into assembly code and writes it to the standard output stream.
\flowgraph{\resource{object file} \ar[r] & \toolbox{m68kdism} \ar[r] & \resource{disassembly\\listing}}
\seeassembly\seemabk\seeobject
}

\providecommand{\miblasm}{
\toolsection{miblasm} is an assembler for the MicroBlaze hardware architecture.
It translates assembly code into machine code for MicroBlaze processors and stores it in corresponding object files.
\flowgraph{\resource{MicroBlaze assembly\\source code} \ar[r] & \toolbox{miblasm} \ar[r] & \resource{object file}}
\seeassembly\seemibl\seeobject
}

\providecommand{\mibldism}{
\toolsection{mibldism} is a disassembler for the MicroBlaze hardware architecture.
It translates machine code from object files targeting MicroBlaze processors into assembly code and writes it to the standard output stream.
\flowgraph{\resource{object file} \ar[r] & \toolbox{mibldism} \ar[r] & \resource{disassembly\\listing}}
\seeassembly\seemibl\seeobject
}

\providecommand{\mipsaasm}{
\toolsection{mips32asm} is an assembler for the MIPS32 hardware architecture.
It translates assembly code into machine code for MIPS32 processors and stores it in corresponding object files.
\flowgraph{\resource{MIPS32 assembly\\source code} \ar[r] & \toolbox{mips32asm} \ar[r] & \resource{object file}}
\seeassembly\seemips\seeobject
}

\providecommand{\mipsadism}{
\toolsection{mips32dism} is a disassembler for the MIPS32 hardware architecture.
It translates machine code from object files targeting MIPS32 processors into assembly code and writes it to the standard output stream.
\flowgraph{\resource{object file} \ar[r] & \toolbox{mips32dism} \ar[r] & \resource{disassembly\\listing}}
\seeassembly\seemips\seeobject
}

\providecommand{\mipsbasm}{
\toolsection{mips64asm} is an assembler for the MIPS64 hardware architecture.
It translates assembly code into machine code for MIPS64 processors and stores it in corresponding object files.
\flowgraph{\resource{MIPS64 assembly\\source code} \ar[r] & \toolbox{mips64asm} \ar[r] & \resource{object file}}
\seeassembly\seemips\seeobject
}

\providecommand{\mipsbdism}{
\toolsection{mips64dism} is a disassembler for the MIPS64 hardware architecture.
It translates machine code from object files targeting MIPS64 processors into assembly code and writes it to the standard output stream.
\flowgraph{\resource{object file} \ar[r] & \toolbox{mips64dism} \ar[r] & \resource{disassembly\\listing}}
\seeassembly\seemips\seeobject
}

\providecommand{\mmixasm}{
\toolsection{mmixasm} is an assembler for the MMIX hardware architecture.
It translates assembly code into machine code for MMIX processors and stores it in corresponding object files.
The names of all special registers are predefined and evaluate to the corresponding number.
\flowgraph{\resource{MMIX assembly\\source code} \ar[r] & \toolbox{mmixasm} \ar[r] & \resource{object file}}
\seeassembly\seemmix\seeobject
}

\providecommand{\mmixdism}{
\toolsection{mmixdism} is a disassembler for the MMIX hardware architecture.
It translates machine code from object files targeting MMIX processors into assembly code and writes it to the standard output stream.
\flowgraph{\resource{object file} \ar[r] & \toolbox{mmixdism} \ar[r] & \resource{disassembly\\listing}}
\seeassembly\seemmix\seeobject
}

\providecommand{\orokasm}{
\toolsection{or1kasm} is an assembler for the OpenRISC 1000 hardware architecture.
It translates assembly code into machine code for OpenRISC 1000 processors and stores it in corresponding object files.
\flowgraph{\resource{OpenRISC 1000 assembly\\source code} \ar[r] & \toolbox{or1kasm} \ar[r] & \resource{object file}}
\seeassembly\seeorok\seeobject
}

\providecommand{\orokdism}{
\toolsection{or1kdism} is a disassembler for the OpenRISC 1000 hardware architecture.
It translates machine code from object files targeting OpenRISC 1000 processors into assembly code and writes it to the standard output stream.
\flowgraph{\resource{object file} \ar[r] & \toolbox{or1kdism} \ar[r] & \resource{disassembly\\listing}}
\seeassembly\seeorok\seeobject
}

\providecommand{\ppcaasm}{
\toolsection{ppc32asm} is an assembler for the PowerPC hardware architecture.
It translates assembly code into machine code for PowerPC processors and stores it in corresponding object files.
By default, the assembler generates machine code for the 32-bit operating mode defined by the PowerPC architecture.
\flowgraph{\resource{PowerPC assembly\\source code} \ar[r] & \toolbox{ppc32asm} \ar[r] & \resource{object file}}
\seeassembly\seeppc\seeobject
}

\providecommand{\ppcadism}{
\toolsection{ppc32dism} is a disassembler for the PowerPC hardware architecture.
It translates machine code from object files targeting PowerPC processors into assembly code and writes it to the standard output stream.
It assumes that the machine code was generated for the 32-bit operating mode defined by the PowerPC architecture.
\flowgraph{\resource{object file} \ar[r] & \toolbox{ppc32dism} \ar[r] & \resource{disassembly\\listing}}
\seeassembly\seeppc\seeobject
}

\providecommand{\ppcbasm}{
\toolsection{ppc64asm} is an assembler for the PowerPC hardware architecture.
It translates assembly code into machine code for PowerPC processors and stores it in corresponding object files.
By default, the assembler generates machine code for the 64-bit operating mode defined by the PowerPC architecture.
\flowgraph{\resource{PowerPC assembly\\source code} \ar[r] & \toolbox{ppc64asm} \ar[r] & \resource{object file}}
\seeassembly\seeppc\seeobject
}

\providecommand{\ppcbdism}{
\toolsection{ppc64dism} is a disassembler for the PowerPC hardware architecture.
It translates machine code from object files targeting PowerPC processors into assembly code and writes it to the standard output stream.
It assumes that the machine code was generated for the 64-bit operating mode defined by the PowerPC architecture.
\flowgraph{\resource{object file} \ar[r] & \toolbox{ppc64dism} \ar[r] & \resource{disassembly\\listing}}
\seeassembly\seeppc\seeobject
}

\providecommand{\riscasm}{
\toolsection{riscasm} is an assembler for the RISC hardware architecture.
It translates assembly code into machine code for RISC processors and stores it in corresponding object files.
The names of all special registers are predefined and evaluate to the corresponding number.
\flowgraph{\resource{RISC assembly\\source code} \ar[r] & \toolbox{riscasm} \ar[r] & \resource{object file}}
\seeassembly\seerisc\seeobject
}

\providecommand{\riscdism}{
\toolsection{riscdism} is a disassembler for the RISC hardware architecture.
It translates machine code from object files targeting RISC processors into assembly code and writes it to the standard output stream.
\flowgraph{\resource{object file} \ar[r] & \toolbox{riscdism} \ar[r] & \resource{disassembly\\listing}}
\seeassembly\seerisc\seeobject
}

\providecommand{\wasmasm}{
\toolsection{wasmasm} is an assembler for the WebAssembly architecture.
It translates assembly code into machine code for WebAssembly targets and stores it in corresponding object files.
The names of all special registers are predefined and evaluate to the corresponding number.
\flowgraph{\resource{WebAssembly assembly\\source code} \ar[r] & \toolbox{wasmasm} \ar[r] & \resource{object file}}
\seeassembly\seewasm\seeobject
}

\providecommand{\wasmdism}{
\toolsection{wasmdism} is a disassembler for the WebAssembly architecture.
It translates machine code from object files targeting WebAssembly targets into assembly code and writes it to the standard output stream.
\flowgraph{\resource{object file} \ar[r] & \toolbox{wasmdism} \ar[r] & \resource{disassembly\\listing}}
\seeassembly\seewasm\seeobject
}

% linker tools

\providecommand{\linklib}{
\toolsection{linklib} is an object file combiner.
It creates a static library file by combining all object files given to it into a single one.
\flowgraph{\resource{object files} \ar[r] & \toolbox{linklib} \ar[r] & \resource{library file}}
\seeobject
}

\providecommand{\linkbin}{
\toolsection{linkbin} is a linker for plain binary files.
It links all object files given to it into a single image and stores it in a binary file that begins with the first linked section.
It also creates a map file that lists the address, type, name and size of all used sections.
The filename extension of the resulting binary file can be specified by putting it into a constant data section called \texttt{\_extension}.
\flowgraph{\resource{object files} \ar[r] & \toolbox{linkbin} \ar[r] \ar[d] & \resource{binary file} \\ & \resource{map file}}
\seeobject
}

\providecommand{\linkmem}{
\toolsection{linkmem} is a linker for plain binary files partitioned into random-access and read-only memory.
It links all object files given to it into two distinct images, one for data sections and one for code and constant data sections, and stores each image in a binary file that begins with the first linked section of the corresponding type.
It also creates a map file that lists the address, type, name and size of all used sections.
\flowgraph{\resource{object files} \ar[r] & \toolbox{linkmem} \ar[r] \ar[d] & \resource{RAM file/\\ROM file} \\ & \resource{map file}}
\seeobject
}

\providecommand{\linkprg}{
\toolsection{linkprg} is a linker for GEMDOS executable files.
It links all object files given to it into a single image and stores the image in an Atari GEMDOS executable file~\cite{gemdosfile}.
It also creates a map file that lists the address relative to the text segment, type, name and size of all used sections.
The filename extension of the resulting executable file can be specified by putting it into a constant data section called \texttt{\_extension}.
The GEMDOS executable file format requires all patch patterns of absolute link patches to consist of four full bitmasks with descending offsets.
\flowgraph{\resource{object files} \ar[r] & \toolbox{linkprg} \ar[r] \ar[d] & \resource{executable file} \\ & \resource{map file}}
\seeobject
}

\providecommand{\linkhex}{
\toolsection{linkhex} is a linker for Intel HEX files.
It links all code sections of the object files given to it into single image and stores the image in an Intel HEX file~\cite{hexfile} that begins with the first linked section.
It also creates a map file that lists the address, type, name and size of all used sections.
\flowgraph{\resource{object files} \ar[r] & \toolbox{linkhex} \ar[r] \ar[d] & \resource{HEX file} \\ & \resource{map file}}
\seeobject
}

\providecommand{\mapsearch}{
\toolsection{mapsearch} is a debugging tool.
It searches map files generated by linker tools for the name of a binary section that encompasses a memory address read from the standard input stream.
If additionally provided with one or more object files, it also stores an excerpt thereof in a separate object file called map search result which only contains the identified binary section for disassembling purposes.
\flowgraph{& \resource{map files/\\object files} \ar[d] \\ \resource{memory\\address} \ar[r] & \toolbox{mapsearch} \ar[r] \ar[d] & \resource{section name/\\relative offset} \\ & \resource{object file\\excerpt}}
\seeobject
}


\startchapter{ECS Runtime Support~Exception}{ECS Runtime Support Exception}{rse}{}

\begin{center}

Copyright \copyright{} Florian Negele

\end{center}

Everyone is permitted to copy and distribute verbatim copies of this
license document, but changing it is not allowed.

This ECS Runtime Support Exception (``Exception'') is an additional
permission under section 7 of the \gpl{}, version
3 (``GPLv3''). It applies to a given file (the ``Runtime Support'') that
bears a notice placed by the copyright holder of the file stating that
the file is governed by GPLv3 along with this Exception.

When you use the ECS to compile a program, it may combine portions of
certain ECS header and runtime support files with the compiled
program. The purpose of this Exception is to allow compilation of
non-GPL (including proprietary) programs to use, in this way, the
header and runtime support files covered by this Exception.

\setcounter{section}{-1}
\renewcommand{\thesection}{\arabic{section}.}

\section{Definitions}

A file is an ``Independent Module'' if it either requires the Runtime
Support for execution after a Compilation Process, or makes use of an
interface provided by the Runtime Support, but is not otherwise based
on the Runtime Support.

``ECS'' means a version of the \ecs{}, with or without
modifications, governed by version 3 (or a specified later version) of
the \gpl{} (GPL) with the option of using any
subsequent versions published by the FSF.

``GPL-compatible Software'' is software whose conditions of propagation,
modification and use would permit combination with the ECS in accord
with the license of the ECS.

``Target Code'' refers to output from any compiler for a real or virtual
target processor architecture, in executable form or suitable for
input to an assembler, loader, linker and/or execution
phase. Notwithstanding that, Target Code does not include data in any
format that is used as a compiler intermediate representation, or used
for producing a compiler intermediate representation.

The ``Compilation Process'' transforms code entirely represented in
non-intermediate languages designed for human-written code, and/or in
Java Virtual Machine byte code, into Target Code. Thus, for example,
use of source code generators and preprocessors need not be considered
part of the Compilation Process, since the Compilation Process can be
understood as starting with the output of the generators or
preprocessors.

A Compilation Process is ``Eligible'' if it is done using the ECS, alone
or with other GPL-compatible software, or if it is done without using
any work based on the ECS. For example, using non-GPL-compatible
Software to optimize any ECS intermediate representations would not
qualify as an Eligible Compilation Process.

\section{Grant of Additional Permission}

You have permission to propagate a work of Target Code formed by
combining the Runtime Support with Independent Modules, even if such
propagation would otherwise violate the terms of GPLv3, provided that
all Target Code was generated by Eligible Compilation Processes. You
may then convey such a combination under terms of your choice,
consistent with the licensing of the Independent Modules.

\section{No Weakening of ECS Copyleft}

The availability of this Exception does not imply any general
presumption that third-party software is unaffected by the copyleft
requirements of the license of the ECS.

\concludechapter

% GNU Free Documentation License
% Copyright (C) Florian Negele

% This file is part of the Eigen Compiler Suite.

% Permission is granted to copy, distribute and/or modify this document
% under the terms of the GNU Free Documentation License, Version 1.3
% or any later version published by the Free Software Foundation.

% You should have received a copy of the GNU Free Documentation License
% along with the ECS.  If not, see <https://www.gnu.org/licenses/>.

% Generic documentation utilities
% Copyright (C) Florian Negele

% This file is part of the Eigen Compiler Suite.

% Permission is granted to copy, distribute and/or modify this document
% under the terms of the GNU Free Documentation License, Version 1.3
% or any later version published by the Free Software Foundation.

% You should have received a copy of the GNU Free Documentation License
% along with the ECS.  If not, see <https://www.gnu.org/licenses/>.

\providecommand{\cpp}{C\texttt{++}}
\providecommand{\opt}{_\mathit{opt}}
\providecommand{\tool}[1]{\texttt{#1}}
\providecommand{\version}{Version 0.0.40}
\providecommand{\resource}[1]{*++\txt{#1}}
\providecommand{\ecs}{Eigen Compiler Suite}
\providecommand{\changed}[1]{\underline{#1}}
\providecommand{\toolbox}[1]{\converter{#1}}
\providecommand{\file}{}\renewcommand{\file}[1]{\texttt{#1}}
\providecommand{\alignright}{\hfill\linebreak[0]\hspace*{\fill}}
\providecommand{\converter}[1]{*++[F][F*:white][F,:gray]\txt{#1}}
\providecommand{\documentation}{\ifbook chapter\else document\fi}
\providecommand{\Documentation}{\ifbook Chapter\else Document\fi}
\providecommand{\variable}[1]{\resource{\texttt{\small#1}\\variable}}
\providecommand{\documentationref}[2]{\ifbook\ref{#1}\else``\href{#1}{#2}''~\cite{#1}\fi}
\providecommand{\objfile}[1]{\texttt{#1}\index[runtime]{#1 object file@\texttt{#1} object file}}
\providecommand{\libfile}[1]{\texttt{#1}\index[runtime]{#1 library file@\texttt{#1} library file}}
\providecommand{\epigraph}[2]{\ifbook\begin{quote}\flushright\textit{#1}\par--- #2\end{quote}\fi}
\providecommand{\environmentvariable}[1]{\texttt{#1}\index{Environment variables!#1@\texttt{#1}}}
\providecommand{\environment}[1]{\texttt{#1}\index[environment]{#1 environment@\texttt{#1} environment}}
\providecommand{\toolsection}{}\renewcommand{\toolsection}[1]{\subsection{#1}\label{\prefix:#1}\tool{#1}}
\providecommand{\instruction}{}\renewcommand{\instruction}[2]{\noindent\qquad\pdftooltip{\texttt{#1}}{#2}\refstepcounter{instruction}\par}
\providecommand{\flowgraph}{}\renewcommand{\flowgraph}[1]{\par\sffamily\begin{displaymath}\xymatrix@=4ex{#1}\end{displaymath}\normalfont\par}
\providecommand{\instructionset}{}\renewcommand{\instructionset}[4]{\setcounter{instruction}{0}\begin{multicols}{\ifbook#3\else#4\fi}[{\captionof{table}[#2]{#2 (\ref*{#1:instructions}~instructions)}\label{tab:#1set}\vspace{-2ex}}]\footnotesize\raggedcolumns\input{#1.set}\label{#1:instructions}\end{multicols}}

\providecommand{\gpl}{GNU General Public License}
\providecommand{\rse}{ECS Runtime Support Exception}
\providecommand{\fdl}{\href{https://www.gnu.org/licenses/fdl.html}{GNU Free Documentation License}}

\providecommand{\docbegin}{}
\providecommand{\docend}{}
\providecommand{\doclabel}[1]{\hypertarget{#1}}
\providecommand{\doclink}[2]{\hyperlink{#1}{#2}}
\providecommand{\docsection}[3]{\hypertarget{#1}{\subsection{#2}}\label{sec:#1}\index[library]{#2@#3}}
\providecommand{\docsectionstar}[1]{}
\providecommand{\docsubbegin}{\begin{description}}
\providecommand{\docsubend}{\end{description}}
\providecommand{\docsubsection}[3]{\item[\hypertarget{#1}{#2}]\index[library]{#2@#3}}
\providecommand{\docsubsectionstar}[1]{\smallskip}
\providecommand{\docsubsubsection}[3]{\docsubsection{#1}{#2}{#3}}
\providecommand{\docsubsubsectionstar}[1]{}
\providecommand{\docsubsubsubsection}[3]{}
\providecommand{\docsubsubsubsectionstar}[1]{}
\providecommand{\doctable}{}

\providecommand{\debuggingtool}{}\renewcommand{\debuggingtool}{This tool is provided for debugging purposes.
It allows exposing and modifying an internal data structure that is usually not accessible.
}

\providecommand{\interface}{All tools accept command-line arguments which are taken as names of plain text files containing the source code.
If no arguments are provided, the standard input stream is used instead.
Output files are generated in the current working directory and have the same name as the input file being processed whereas the filename extension gets replaced by an appropriate suffix.
\seeinterface
}

\providecommand{\license}{\noindent Copyright \copyright{} Florian Negele\par\medskip\noindent
Permission is granted to copy, distribute and/or modify this document under the terms of the
\fdl{}, Version 1.3 or any later version published by the \href{https://fsf.org/}{Free Software Foundation}.
}

\providecommand{\ecslogosurface}{
\fill[darkgray] (0,0,0) -- (0,0,3) -- (0,3,3) -- (0,3,1) -- (0,4,1) -- (0,4,3) -- (0,5,3) -- (0,5,0) -- (0,2,0) -- (0,2,2) -- (0,1,2) -- (0,1,0) -- cycle;
\fill[gray] (0,5,0) -- (0,5,3) -- (1,5,3) -- (1,5,1) -- (2,5,1) -- (2,5,3) -- (3,5,3) -- (3,5,0) -- cycle;
\fill[lightgray] (0,0,0) -- (0,1,0) -- (2,1,0) -- (2,4,0) -- (1,4,0) -- (1,3,0) -- (2,3,0) -- (2,2,0) -- (0,2,0) -- (0,5,0) -- (3,5,0) -- (3,0,0) -- cycle;
\begin{scope}[line width=0.5]
\begin{scope}[gray]
\draw (0,0,0) -- (0,1,0);
\draw (2,1,0) -- (2,2,0);
\draw (0,1,2) -- (0,2,2);
\draw (0,2,0) -- (0,5,0);
\draw (2,3,0) -- (2,4,0);
\end{scope}
\begin{scope}[lightgray]
\draw (0,1,0) -- (0,1,2);
\draw (0,3,1) -- (0,3,3);
\draw (0,5,0) -- (0,5,3);
\draw (2,5,1) -- (2,5,3);
\end{scope}
\begin{scope}[white]
\draw (0,1,0) -- (2,1,0);
\draw (1,3,0) -- (2,3,0);
\draw (0,5,0) -- (3,5,0);
\end{scope}
\end{scope}
}

\providecommand{\ecslogo}[1]{
\begin{tikzpicture}[scale={(#1)/((sin(45)+cos(45))*3cm)},x={({-cos(45)*1cm},{sin(45)*sin(30)*1cm})},y={({0cm},{(cos(30)*1cm})},z={({sin(45)*1cm},{cos(45)*sin(30)*1cm})}]
\begin{scope}[darkgray,line width=1]
\draw (0,0,0) -- (0,0,3) -- (0,3,3) -- (2,3,3) -- (2,5,3) -- (3,5,3) -- (3,5,0) -- (3,0,0) -- cycle;
\draw (0,3,1) -- (0,4,1) -- (0,4,3) -- (0,5,3) -- (1,5,3) -- (1,5,1) -- (2,5,1);
\draw (1,3,0) -- (1,4,0) -- (2,4,0);
\end{scope}
\fill[darkgray] (2,0,0) -- (2,0,3) -- (2,5,3) -- (2,5,1) -- (2,4,1) -- (2,4,0) -- cycle;
\fill[lightgray] (2,0,2) -- (0,0,2) -- (0,2,2) -- (2,2,2) -- cycle;
\fill[gray] (0,1,0) -- (2,1,0) -- (2,1,2) -- (0,1,2) -- cycle;
\fill[gray] (0,3,1) -- (0,3,3) -- (2,3,3) -- (2,3,0) -- (1,3,0) -- (1,3,1) -- cycle;
\ecslogosurface
\end{tikzpicture}
}

\providecommand{\shadowedecslogo}[3]{
\begin{tikzpicture}[scale={(#1)/((sin(#2)+cos(#2))*3cm)},x={({-cos(#2)*1cm},{sin(#2)*sin(#3)*1cm})},y={({0cm},{(cos(#3)*1cm})},z={({sin(#2)*1cm},{cos(#2)*sin(#3)*1cm})}]
\shade[top color=lightgray!50!white,bottom color=white,middle color=lightgray!50!white] (0,0,0) -- (3,0,0) -- (3,{-0.5-3*sin(#2)*sin(#3)/cos(#3)},0) -- (0,-0.5,0) -- cycle;
\shade[top color=darkgray!50!gray,bottom color=white,middle color=darkgray!50!white] (0,0,0) -- (0,0,3) -- (0,{-0.5-3*cos(#2)*sin(#3)/cos(#3)},3) -- (0,-0.5,0) -- cycle;
\begin{scope}[y={({(cos(#2)+sin(#2))*0.5cm},{(cos(#2)*sin(#3)-sin(#2)*sin(#3))*0.5cm})}]
\useasboundingbox (3,0,0) -- (0,0,0) -- (0,0,3);
\shade[left color=darkgray!80!black,right color=lightgray,middle color=gray] (0,0,0) -- (0,1,0) -- (0,1,0.5) -- (0,2,0) -- (0,5,0) -- (0,5,3) -- (1,5,3) -- (1,4,3) -- (1,4,2.5) -- (1,3,3) -- (2,5,3) -- (3,5,3) -- (3,0,3) -- cycle;
\clip (0,0,0) -- (0,0,3) -- ({-3*sin(#2)/cos(#2)},0,0) -- cycle;
\shade[left color=darkgray,right color=lightgray!50!gray] (0,0,0) -- (0,1,0) -- (0,1,0.5) -- (0,2,0) -- (0,5,0) -- (0,5,3) -- (1,5,3) -- (1,4,3) -- (1,4,2.5) -- (1,3,3) -- (2,5,3) -- (3,5,3) -- (3,0,3) -- cycle;
\end{scope}
\shade[left color=darkgray,right color=darkgray!80!black] (2,0,0) -- (2,0,3) -- (2,5,3) -- (2,5,1) -- (2,4,1) -- (2,4,0) -- cycle;
\shade[left color=darkgray!90!black,right color=gray!80!darkgray] (2,0,2) -- (0,0,2) -- (0,2,2) -- (2,2,2) -- cycle;
\shade[top color=darkgray!90!black,bottom color=gray!80!darkgray] (0,1,0) -- (2,1,0) -- (2,1,2) -- (0,1,2) -- cycle;
\shade[top color=darkgray!90!black,bottom color=gray!80!darkgray] (0,3,1) -- (0,3,3) -- (2,3,3) -- (2,3,0) -- (1,3,0) -- (1,3,1) -- cycle;
\fill[gray] (2,1,0) -- (1.5,1,0.5) -- (0,1,0.5) -- (0,1,0) -- cycle;
\fill[gray] (1,3,2) -- (0.5,3,2) -- (0.5,3,3) -- (1,3,3) -- cycle;
\fill[gray] (2,3,0) -- (1.5,3,0.5) -- (1,3,0.5) -- (1,3,0) -- cycle;
\ecslogosurface
\end{tikzpicture}
}

\providecommand{\cpplogo}[1]{
\begin{tikzpicture}[scale=(#1)/512em]
\fill[gray] (435.2794,398.7159) -- (247.1911,507.3075) .. controls (236.3563,513.5642) and (218.6240,513.5642) .. (207.7892,507.3075) -- (19.7009,398.7159) .. controls (8.8646,392.4606) and (0.0000,377.1043) .. (0.0000,364.5924) -- (0.0000,147.4076) .. controls (0.8430,132.8363) and (8.2856,120.7683) .. (19.7009,113.2842) -- (207.7892,4.6926) .. controls (218.6240,-1.5642) and (236.3564,-1.5642) .. (247.1911,4.6926) -- (435.2794,113.2842) .. controls (447.5273,121.4304) and (454.4987,133.6918) .. (454.9803,147.4076) -- (454.9803,364.5924) .. controls (454.5404,377.7571) and (446.6566,391.0351) .. (435.2794,398.7159) -- cycle(75.8301,255.9993) .. controls (74.9389,404.0881) and (273.2892,469.4783) .. (358.8263,331.8769) -- (293.1917,293.8965) .. controls (253.5702,359.4301) and (155.1909,335.9977) .. (151.6601,255.9993) .. controls (152.7204,182.2703) and (249.4137,148.0211) .. (293.1961,218.1065) -- (358.8308,180.1276) .. controls (283.4477,49.2645) and (79.6318,96.3470) .. (75.8301,255.9993) -- cycle(379.1503,247.5747) -- (362.2982,247.5747) -- (362.2982,230.7226) -- (345.4490,230.7226) -- (345.4490,247.5747) -- (328.5969,247.5747) -- (328.5969,264.4254) -- (345.4490,264.4254) -- (345.4490,281.2759) -- (362.2982,281.2759) -- (362.2982,264.4254) -- (379.1503,264.4254) -- cycle(442.3420,247.5747) -- (425.4899,247.5747) -- (425.4899,230.7226) -- (408.6408,230.7226) -- (408.6408,247.5747) -- (391.7886,247.5747) -- (391.7886,264.4254) -- (408.6408,264.4254) -- (408.6408,281.2759) -- (425.4899,281.2759) -- (425.4899,264.4254) -- (442.3420,264.4254) -- cycle;
\end{tikzpicture}
}

\providecommand{\fallogo}[1]{
\begin{tikzpicture}[scale=(#1)/512em]
\fill[gray] (185.7774,0.0000) .. controls (200.4486,15.9798) and (226.8966,8.7148) .. (235.0426,31.5836) .. controls (249.5297,58.0598) and (247.9581,97.9161) .. (280.3335,110.9762) .. controls (309.1690,120.3496) and (337.8406,104.2727) .. (366.5753,103.9379) .. controls (373.4449,111.5171) and (379.2885,128.2574) .. (383.9755,108.9744) .. controls (396.6979,102.5615) and (437.2808,107.6681) .. (426.9652,124.3252) .. controls (408.9822,121.0785) and (412.4742,146.0729) .. (426.5192,131.4996) .. controls (433.8413,120.8489) and (465.1541,126.5522) .. (441.9067,135.7950) .. controls (396.1879,157.7478) and (344.1112,161.5079) .. (298.5528,183.5702) .. controls (277.7471,193.5198) and (284.6941,218.7163) .. (285.2127,236.9640) .. controls (292.3599,316.2826) and (307.3929,394.6311) .. (317.1198,473.6154) .. controls (329.0637,505.4736) and (292.1195,528.5004) .. (265.9183,511.2761) .. controls (237.9284,499.2462) and (237.3684,465.2681) .. (230.9102,439.9421) .. controls (218.6692,374.3397) and (215.6307,306.9662) .. (198.1732,242.3977) .. controls (183.1379,232.7444) and (164.4245,256.0298) .. (149.0430,261.4799) .. controls (116.9328,279.2585) and (87.1822,308.5851) .. (48.2293,307.8914) .. controls (21.3220,306.9037) and (-15.9107,281.8761) .. (7.2921,252.7908) .. controls (29.7799,220.6177) and (67.5177,204.3028) .. (100.9287,185.9449) .. controls (130.8217,170.8906) and (161.1548,156.5903) .. (191.0278,141.5847) .. controls (196.1738,120.0520) and (186.6049,95.2409) .. (186.8382,72.4353) .. controls (185.5234,48.4204) and (183.1700,23.9341) .. (185.7774,0.0000) -- cycle;
\end{tikzpicture}
}

\providecommand{\oblogo}[1]{
\begin{tikzpicture}[scale=(#1)/512em]
\fill[gray] (160.3865,208.9117) .. controls (154.0879,214.6478) and (149.0735,221.2409) .. (145.4125,228.5384) .. controls (184.8790,248.4273) and (234.7122,269.8787) .. (297.5493,291.8782) .. controls (300.3943,281.4769) and (300.9552,268.7619) .. (300.4023,255.2389) .. controls (248.9909,244.7891) and (200.0310,225.9279) .. (160.3865,208.9117) -- cycle(225.7398,392.6996) .. controls (308.0209,392.1716) and (359.3326,345.9277) .. (368.7203,285.2098) .. controls (376.6742,197.1784) and (311.7194,141.3342) .. (205.4287,142.1456) .. controls (139.9485,141.4804) and (88.7155,166.1957) .. (73.5775,228.0086) .. controls (52.0297,320.3408) and (123.4078,391.0103) .. (225.7398,392.6996) -- cycle(216.0739,176.4733) .. controls (268.9183,179.2424) and (315.8292,206.5488) .. (312.7454,265.1139) .. controls (313.2769,315.6384) and (286.5993,353.4946) .. (216.6040,355.7934) .. controls (162.4657,355.7934) and (126.0914,317.5023) .. (126.0914,260.5103) .. controls (126.1733,214.2900) and (163.3363,176.2849) .. (216.0739,176.4733) -- cycle(76.4897,189.1754) .. controls (13.1586,147.5631) and (0.0000,119.4207) .. (0.0000,119.4207) -- (90.6499,170.1632) .. controls (85.3004,175.8497) and (80.5994,182.1633) .. (76.4897,189.1754) -- cycle(353.9486,119.3004) -- (402.9482,119.3004) .. controls (427.0025,137.0797) and (450.9893,162.7034) .. (474.9529,191.0213) .. controls (509.3540,228.5339) and (531.3391,294.2091) .. (487.8149,312.1206) .. controls (462.8165,324.7652) and (394.3874,316.8943) .. (373.8912,313.6651) .. controls (379.9291,297.7449) and (383.2899,278.4204) .. (381.4989,257.7214) .. controls (420.3069,248.0321) and (421.9610,218.3461) .. (407.7867,192.6417) .. controls (391.1113,162.4018) and (370.1114,132.9097) .. (353.9486,119.3004) -- cycle;
\end{tikzpicture}
}

\providecommand{\markuptable}{
\begin{table}
\sffamily\centering
\begin{tabular}{@{}lcl@{}}
\toprule
\texttt{//italics//} & $\rightarrow$ & \textit{italics} \\
\midrule
\texttt{**bold**} & $\rightarrow$ & \textbf{bold} \\
\midrule
\texttt{\# ordered list} & & 1 ordered list \\
\texttt{\# second item} & $\rightarrow$ & 2 second item \\
\texttt{\#\# sub item} & & \hspace{1em} 1 sub item \\
\midrule
\texttt{* unordered list} & & $\bullet$ unordered list \\
\texttt{* second item} & $\rightarrow$ & $\bullet$ second item \\
\texttt{** sub item} & & \hspace{1em} $\bullet$ sub item \\
\midrule
\texttt{link to [[label]]} & $\rightarrow$ & link to \underline{label} \\
\midrule
\texttt{<{}<label>{}> definition } & $\rightarrow$ & definition \\
\midrule
\texttt{[[url|link name]]} & $\rightarrow$ & \underline{link name} \\
\midrule\addlinespace
\texttt{= large heading} & & {\Large large heading} \smallskip \\
\texttt{== medium heading} & $\rightarrow$ & {\large medium heading} \\
\texttt{=== small heading} & & small heading \\
\midrule
\texttt{no line break} & & no line break for paragraphs \\
\texttt{for paragraphs} & $\rightarrow$ \\
& & use empty line \\
\texttt{use empty line} \\
\midrule
\texttt{force\textbackslash\textbackslash line break} & $\rightarrow$ & force \\
& & line break \\
\midrule
\texttt{horizontal line} & $\rightarrow$ & horizontal line \\
\texttt{----} & & \hrulefill \\
\midrule
\texttt{|=a|=table|=header} & & \underline{a \enspace table \enspace header} \\
\texttt{|a|table|row} & $\rightarrow$ & a \enspace table \enspace row \\
\texttt{|b|table|row} & & b \enspace table \enspace row \\
\midrule
\texttt{\{\{\{} \\
\texttt{unformatted} & $\rightarrow$ & \texttt{unformatted} \\
\texttt{code} & & \texttt{code} \\
\texttt{\}\}\}} \\
\midrule\addlinespace
\texttt{@ new article} & & {\Large 1.\ new article} \smallskip \\
\texttt{@ second article} & $\rightarrow$ & {\Large 2.\ second article} \smallskip \\
\texttt{@@ sub article} & & {\large 2.1.\ sub article} \\
\bottomrule
\end{tabular}
\normalfont\caption{Elements of the generic documentation markup language}
\label{tab:docmarkup}
\end{table}
}

\providecommand{\startchapter}[4]{
\documentclass[11pt,a4paper]{article}
\usepackage{booktabs}
\usepackage[format=hang,labelfont=bf]{caption}
\usepackage{changepage}
\usepackage[T1]{fontenc}
\usepackage[margin=2cm]{geometry}
\usepackage{hyperref}
\usepackage[american]{isodate}
\usepackage{lmodern}
\usepackage{longtable}
\usepackage{mathptmx}
\usepackage{microtype}
\usepackage[toc]{multitoc}
\usepackage{multirow}
\usepackage[all]{nowidow}
\usepackage{pdfcomment}
\usepackage{syntax}
\usepackage{tikz}
\usepackage[all]{xy}
\hypersetup{pdfborder={0 0 0},bookmarksnumbered=true,pdftitle={\ecs{}: #2},pdfauthor={Florian Negele},pdfsubject={\ecs{}},pdfkeywords={#1}}
\setlength{\grammarindent}{8em}\setlength{\grammarparsep}{0.2ex}
\setlength{\columnsep}{2em}
\newcommand{\prefix}{}
\newcounter{instruction}
\bibliographystyle{unsrt}
\renewcommand{\index}[2][]{}
\renewcommand{\arraystretch}{1.05}
\renewcommand{\floatpagefraction}{0.7}
\renewcommand{\syntleft}{\itshape}\renewcommand{\syntright}{}
\title{\vspace{-5ex}\Huge{\ecs{}}\medskip\hrule}
\author{\huge{#2}}
\date{\medskip\version}
\newif\ifbook\bookfalse
\pagestyle{headings}
\frenchspacing
\begin{document}
\maketitle\thispagestyle{empty}\noindent#4\setlength{\columnseprule}{0.4pt}\tableofcontents\setlength{\columnseprule}{0pt}\vfill\pagebreak[3]\null\vfill\bigskip\noindent
\parbox{\textwidth-4em}{\license The contents of this \documentation{} are part of the \href{manual}{\ecs{} User Manual}~\cite{manual} and correspond to Chapter ``\href{manual\##3}{#1}''.\alignright\mbox{\today}}
\parbox{4em}{\flushright\ecslogo{3em}}
\clearpage
}

\providecommand{\concludechapter}{
\vfill\pagebreak[3]\null\vfill
\thispagestyle{myheadings}\markright{REFERENCES}
\noindent\begin{minipage}{\textwidth}\begin{multicols}{2}[\section*{References}]
\renewcommand{\section}[2]{}\small\bibliography{references}
\end{multicols}\end{minipage}\end{document}
}

\providecommand{\startpresentation}[2]{
\documentclass[14pt,aspectratio=43,usepdftitle=false]{beamer}
\usepackage{booktabs}
\usepackage{etex}
\usepackage{multicol}
\usepackage{tikz}
\usepackage[all]{xy}
\bibliographystyle{unsrt}
\setlength{\columnsep}{1em}
\setlength{\leftmargini}{1em}
\setbeamercolor{title}{fg=black}
\setbeamercolor{structure}{fg=darkgray}
\setbeamercolor{bibliography item}{fg=darkgray}
\setbeamerfont{title}{series=\bfseries}
\setbeamerfont{subtitle}{series=\normalfont}
\setbeamerfont*{frametitle}{parent=title}
\setbeamerfont{block title}{series=\bfseries}
\setbeamerfont*{framesubtitle}{parent=subtitle}
\setbeamersize{text margin left=1em,text margin right=1em}
\setbeamertemplate{navigation symbols}{}
\setbeamertemplate{itemize item}[circle]{}
\setbeamertemplate{bibliography item}[triangle]{}
\setbeamertemplate{bibliography entry author}{\usebeamercolor[fg]{bibliography item}}
\setbeamertemplate{frametitle}{\medskip\usebeamerfont{frametitle}\color{gray}\raisebox{-2.5ex}[0ex][0ex]{\rule{0.1em}{4.5ex}}}
\addtobeamertemplate{frametitle}{}{\hspace{0.4em}\usebeamercolor[fg]{title}\insertframetitle\par\vspace{0.2ex}\hspace{0.5em}\usebeamerfont{framesubtitle}\insertframesubtitle}
\hypersetup{pdfborder={0 0 0},bookmarksnumbered=true,bookmarksopen=true,bookmarksopenlevel=0,pdftitle={\ecs{}: #1},pdfauthor={Florian Negele},pdfsubject={\ecs{}},pdfkeywords={#1}}
\renewcommand{\flowgraph}[1]{\resizebox{\textwidth}{!}{$$\xymatrix{##1}$$}}
\title{\ecs{}\medskip\hrule\medskip}
\institute{\shadowedecslogo{5em}{30}{15}}
\date{\version}
\subtitle{#1}
\begin{document}
\begin{frame}[plain]\titlepage\nocite{manual}\end{frame}
\begin{frame}{Contents}{#1}\begin{center}\tableofcontents\end{center}\end{frame}
}

\providecommand{\concludepresentation}{
\begin{frame}{References}\begin{footnotesize}\setlength{\columnseprule}{0.4pt}\begin{multicols}{2}\bibliography{references}\end{multicols}\end{footnotesize}\end{frame}
\end{document}
}

\providecommand{\startbook}[1]{
\documentclass[10pt,paper=17cm:24cm,DIV=13,twoside=semi,headings=normal,numbers=noendperiod,cleardoublepage=plain]{scrbook}
\usepackage{atveryend}
\usepackage{booktabs}
\usepackage{caption}
\usepackage{changepage}
\usepackage[T1]{fontenc}
\usepackage{imakeidx}
\usepackage{hyperref}
\usepackage[american]{isodate}
\usepackage{lmodern}
\usepackage{longtable}
\usepackage{mathptmx}
\usepackage[final]{microtype}
\usepackage{multicol}
\usepackage{multirow}
\usepackage[all]{nowidow}
\usepackage{pdfcomment}
\usepackage{scrlayer-scrpage}
\usepackage{setspace}
\usepackage{syntax}
\usepackage[eventxtindent=4pt,oddtxtexdent=4pt]{thumbs}
\usepackage{tikz}
\usepackage[all]{xy}
\hyphenation{Micro-Blaze Open-Cores Open-RISC Power-PC}
\hypersetup{pdfborder={0 0 0},bookmarksnumbered=true,bookmarksopen=true,bookmarksopenlevel=0,pdftitle={\ecs{}: #1},pdfauthor={Florian Negele},pdfsubject={\ecs{}},pdfkeywords={#1}}
\setlength{\grammarindent}{8em}\setlength{\grammarparsep}{0.7ex}
\setkomafont{captionlabel}{\usekomafont{descriptionlabel}}
\renewcommand{\arraystretch}{1.05}\setstretch{1.1}
\renewcommand{\chapterformat}{\thechapter\autodot\enskip\raisebox{-1ex}[0ex][0ex]{\color{gray}\rule{0.1em}{3.5ex}}\enskip}
\renewcommand{\startchapter}[4]{\hypertarget{##3}{\chapter{##1}}\label{##3}##4\addthumb{##1}{\LARGE\sffamily\bfseries\thechapter}{white}{gray}\renewcommand{\prefix}{##3}}
\renewcommand{\concludechapter}{\clearpage{\stopthumb\cleardoublepage}}
\renewcommand{\syntleft}{\itshape}\renewcommand{\syntright}{}
\renewcommand{\floatpagefraction}{0.7}
\renewcommand{\partheademptypage}{}
\DeclareMicrotypeAlias{lmss}{cmr}
\newcommand{\prefix}{}
\newcounter{instruction}
\bibliographystyle{unsrt}
\newif\ifbook\booktrue
\makeindex[intoc,title=Index]
\makeindex[intoc,name=tools,title=Index of Tools,columns=3]
\makeindex[intoc,name=library,title=Index of Library Names]
\makeindex[intoc,name=runtime,title=Index of Runtime Support]
\makeindex[intoc,name=environment,title=Index of Target Environments]
\indexsetup{toclevel=chapter,headers={\indexname}{\indexname}}
\frenchspacing
\begin{document}
\pagenumbering{alph}
\begin{titlepage}\centering
\huge\sffamily\null\vfill\textbf{\ecs{}}\bigskip\hrule\bigskip#1
\normalsize\normalfont\vfill\vfill\shadowedecslogo{10em}{30}{15}
\large\vfill\vfill\version
\end{titlepage}
\null\vfill
\thispagestyle{empty}
\noindent\today\par\medskip
\license A copy of this license is included in Appendix~\ref{fdl} on page~\pageref{fdl}.
All product names used herein are for identification purposes only and may be trademarks of their respective companies.
\concludechapter
\frontmatter
\setcounter{tocdepth}{1}
\tableofcontents
\setcounter{tocdepth}{2}
\concludechapter
\listoffigures
\concludechapter
\listoftables
\concludechapter
}

\providecommand{\concludebook}{
\backmatter
\addtocontents{toc}{\protect\setcounter{tocdepth}{-1}}
\phantomsection\addcontentsline{toc}{part}{Bibliography}
\bibliography{references}
\concludechapter
\phantomsection\addcontentsline{toc}{part}{Indexes}
\printindex
\concludechapter
\indexprologue{\label{idx:tools}}
\printindex[tools]
\concludechapter
\printindex[library]
\concludechapter
\indexprologue{\label{idx:runtime}}
\printindex[runtime]
\concludechapter
\indexprologue{\label{idx:environment}}
\printindex[environment]
\concludechapter
\pagestyle{empty}\pagenumbering{Alph}\null\clearpage
\null\vfill\centering\ecslogo{4em}\par\medskip\license
\end{document}
}

% chapter references

\providecommand{\seedocumentationref}{}\renewcommand{\seedocumentationref}[3]{#1, see \Documentation{}~\documentationref{#2}{#3}. }
\providecommand{\seeinterface}{}\renewcommand{\seeinterface}{\ifbook See \Documentation{}~\documentationref{interface}{User Interface} for more information about the common user interface of all of these tools. \fi}
\providecommand{\seeguide}{}\renewcommand{\seeguide}{\seedocumentationref{For basic examples of using some of these tools in practice}{guide}{User Guide}}
\providecommand{\seecpp}{}\renewcommand{\seecpp}{\seedocumentationref{For more information about the \cpp{} programming language and its implementation by the \ecs{}}{cpp}{User Manual for \cpp{}}}
\providecommand{\seefalse}{}\renewcommand{\seefalse}{\seedocumentationref{For more information about the FALSE programming language and its implementation by the \ecs{}}{false}{User Manual for FALSE}}
\providecommand{\seeoberon}{}\renewcommand{\seeoberon}{\seedocumentationref{For more information about the Oberon programming language and its implementation by the \ecs{}}{oberon}{User Manual for Oberon}}
\providecommand{\seeassembly}{}\renewcommand{\seeassembly}{\seedocumentationref{For more information about the generic assembly language and how to use it}{assembly}{Generic Assembly Language Specification}}
\providecommand{\seeamd}{}\renewcommand{\seeamd}{\seedocumentationref{For more information about how the \ecs{} supports the AMD64 hardware architecture}{amd64}{AMD64 Hardware Architecture Support}}
\providecommand{\seearm}{}\renewcommand{\seearm}{\seedocumentationref{For more information about how the \ecs{} supports the ARM hardware architecture}{arm}{ARM Hardware Architecture Support}}
\providecommand{\seeavr}{}\renewcommand{\seeavr}{\seedocumentationref{For more information about how the \ecs{} supports the AVR hardware architecture}{avr}{AVR Hardware Architecture Support}}
\providecommand{\seeavrtt}{}\renewcommand{\seeavrtt}{\seedocumentationref{For more information about how the \ecs{} supports the AVR32 hardware architecture}{avr32}{AVR32 Hardware Architecture Support}}
\providecommand{\seemabk}{}\renewcommand{\seemabk}{\seedocumentationref{For more information about how the \ecs{} supports the M68000 hardware architecture}{m68k}{M68000 Hardware Architecture Support}}
\providecommand{\seemibl}{}\renewcommand{\seemibl}{\seedocumentationref{For more information about how the \ecs{} supports the MicroBlaze hardware architecture}{mibl}{MicroBlaze Hardware Architecture Support}}
\providecommand{\seemips}{}\renewcommand{\seemips}{\seedocumentationref{For more information about how the \ecs{} supports the MIPS32 and MIPS64 hardware architectures}{mips}{MIPS Hardware Architecture Support}}
\providecommand{\seemmix}{}\renewcommand{\seemmix}{\seedocumentationref{For more information about how the \ecs{} supports the MMIX hardware architecture}{mmix}{MMIX Hardware Architecture Support}}
\providecommand{\seeorok}{}\renewcommand{\seeorok}{\seedocumentationref{For more information about how the \ecs{} supports the OpenRISC 1000 hardware architecture}{or1k}{OpenRISC 1000 Hardware Architecture Support}}
\providecommand{\seeppc}{}\renewcommand{\seeppc}{\seedocumentationref{For more information about how the \ecs{} supports the PowerPC hardware architecture}{ppc}{PowerPC Hardware Architecture Support}}
\providecommand{\seerisc}{}\renewcommand{\seerisc}{\seedocumentationref{For more information about how the \ecs{} supports the RISC hardware architecture}{risc}{RISC Hardware Architecture Support}}
\providecommand{\seewasm}{}\renewcommand{\seewasm}{\seedocumentationref{For more information about how the \ecs{} supports the WebAssembly architecture}{wasm}{WebAssembly Architecture Support}}
\providecommand{\seedocumentation}{}\renewcommand{\seedocumentation}{\seedocumentationref{For more information about generic documentations and their generation by the \ecs{}}{documentation}{Generic Documentation Generation}}
\providecommand{\seedebugging}{}\renewcommand{\seedebugging}{\seedocumentationref{For more information about debugging information and its representation}{debugging}{Debugging Information Representation}}
\providecommand{\seecode}{}\renewcommand{\seecode}{\seedocumentationref{For more information about intermediate code and its purpose}{code}{Intermediate Code Representation}}
\providecommand{\seeobject}{}\renewcommand{\seeobject}{\seedocumentationref{For more information about object files and their purpose}{object}{Object File Representation}}

% generic documentation tools

\providecommand{\docprint}{
\toolsection{docprint} is a pretty printer for generic documentations.
It reformats generic documentations and writes it to the standard output stream.
\debuggingtool
\flowgraph{\resource{generic\\documentation} \ar[r] & \toolbox{docprint} \ar[r] & \resource{generic\\documentation}}
\seedocumentation
}

\providecommand{\doccheck}{
\toolsection{doccheck} is a syntactic and semantic checker for generic documentations.
It just performs syntactic and semantic checks on generic documentations and writes its diagnostic messages to the standard error stream.
\debuggingtool
\flowgraph{\resource{generic\\documentation} \ar[r] & \toolbox{doccheck} \ar[r] & \resource{diagnostic\\messages}}
\seedocumentation
}

\providecommand{\dochtml}{
\toolsection{dochtml} is an HTML documentation generator for generic documentations.
It processes several generic documentations and assembles all information therein into an HTML document.
\debuggingtool
\flowgraph{\resource{generic\\documentation} \ar[r] & \toolbox{dochtml} \ar[r] & \resource{HTML\\document}}
\seedocumentation
}

\providecommand{\doclatex}{
\toolsection{doclatex} is a Latex documentation generator for generic documentations.
It processes several generic documentations and assembles all information therein into a Latex document.
\debuggingtool
\flowgraph{\resource{generic\\documentation} \ar[r] & \toolbox{doclatex} \ar[r] & \resource{Latex\\document}}
\seedocumentation
}

% intermediate code tools

\providecommand{\cdcheck}{
\toolsection{cdcheck} is a syntactic and semantic checker for intermediate code.
It just performs syntactic and semantic checks on programs written in intermediate code and writes its diagnostic messages to the standard error stream.
\debuggingtool
\flowgraph{\resource{intermediate\\code} \ar[r] & \toolbox{cdcheck} \ar[r] & \resource{diagnostic\\messages}}
\seeassembly\seecode
}

\providecommand{\cdopt}{
\toolsection{cdopt} is an optimizer for intermediate code.
It performs various optimizations on programs written in intermediate code and writes the result to the standard output stream.
\debuggingtool
\flowgraph{\resource{intermediate\\code} \ar[r] & \toolbox{cdopt} \ar[r] & \resource{optimized\\code}}
\seeassembly\seecode
}

\providecommand{\cdrun}{
\toolsection{cdrun} is an interpreter for intermediate code.
It processes and executes programs written in intermediate code.
The following code sections are predefined and have the usual semantics:
\texttt{abort}, \texttt{\_Exit}, \texttt{fflush}, \texttt{floor}, \texttt{fputc}, \texttt{free}, \texttt{getchar}, \texttt{malloc}, and \texttt{putchar}.
Diagnostic messages about invalid operations include the name of the executed code section and the index of the erroneous instruction.
\debuggingtool
\flowgraph{\resource{intermediate\\code} \ar[r] & \toolbox{cdrun} \ar@/u/[r] & \resource{input/\\output} \ar@/d/[l]}
\seeassembly\seecode
}

\providecommand{\cdamda}{
\toolsection{cdamd16} is a compiler for intermediate code targeting the AMD64 hardware architecture.
It generates machine code for AMD64 processors from programs written in intermediate code and stores it in corresponding object files.
The compiler generates machine code for the 16-bit operating mode defined by the AMD64 architecture.
It also creates a debugging information file as well as an assembly file containing a listing of the generated machine code.
\debuggingtool
\flowgraph{\resource{intermediate\\code} \ar[r] & \toolbox{cdamd16} \ar[r] \ar[d] \ar[rd] & \resource{object file} \\ & \resource{assembly\\listing} & \resource{debugging\\information}}
\seeassembly\seeamd\seeobject\seecode\seedebugging
}

\providecommand{\cdamdb}{
\toolsection{cdamd32} is a compiler for intermediate code targeting the AMD64 hardware architecture.
It generates machine code for AMD64 processors from programs written in intermediate code and stores it in corresponding object files.
The compiler generates machine code for the 32-bit operating mode defined by the AMD64 architecture.
It also creates a debugging information file as well as an assembly file containing a listing of the generated machine code.
\debuggingtool
\flowgraph{\resource{intermediate\\code} \ar[r] & \toolbox{cdamd32} \ar[r] \ar[d] \ar[rd] & \resource{object file} \\ & \resource{assembly\\listing} & \resource{debugging\\information}}
\seeassembly\seeamd\seeobject\seecode\seedebugging
}

\providecommand{\cdamdc}{
\toolsection{cdamd64} is a compiler for intermediate code targeting the AMD64 hardware architecture.
It generates machine code for AMD64 processors from programs written in intermediate code and stores it in corresponding object files.
The compiler generates machine code for the 64-bit operating mode defined by the AMD64 architecture.
It also creates a debugging information file as well as an assembly file containing a listing of the generated machine code.
\debuggingtool
\flowgraph{\resource{intermediate\\code} \ar[r] & \toolbox{cdamd64} \ar[r] \ar[d] \ar[rd] & \resource{object file} \\ & \resource{assembly\\listing} & \resource{debugging\\information}}
\seeassembly\seeamd\seeobject\seecode\seedebugging
}

\providecommand{\cdarma}{
\toolsection{cdarma32} is a compiler for intermediate code targeting the ARM hardware architecture.
It generates machine code for ARM processors executing A32 instructions from programs written in intermediate code and stores it in corresponding object files.
It also creates a debugging information file as well as an assembly file containing a listing of the generated machine code.
\debuggingtool
\flowgraph{\resource{intermediate\\code} \ar[r] & \toolbox{cdarma32} \ar[r] \ar[d] \ar[rd] & \resource{object file} \\ & \resource{assembly\\listing} & \resource{debugging\\information}}
\seeassembly\seearm\seeobject\seecode\seedebugging
}

\providecommand{\cdarmb}{
\toolsection{cdarma64} is a compiler for intermediate code targeting the ARM hardware architecture.
It generates machine code for ARM processors executing A64 instructions from programs written in intermediate code and stores it in corresponding object files.
It also creates a debugging information file as well as an assembly file containing a listing of the generated machine code.
\debuggingtool
\flowgraph{\resource{intermediate\\code} \ar[r] & \toolbox{cdarma64} \ar[r] \ar[d] \ar[rd] & \resource{object file} \\ & \resource{assembly\\listing} & \resource{debugging\\information}}
\seeassembly\seearm\seeobject\seecode\seedebugging
}

\providecommand{\cdarmc}{
\toolsection{cdarmt32} is a compiler for intermediate code targeting the ARM hardware architecture.
It generates machine code for ARM processors without floating-point extension executing T32 instructions from programs written in intermediate code and stores it in corresponding object files.
It also creates a debugging information file as well as an assembly file containing a listing of the generated machine code.
\debuggingtool
\flowgraph{\resource{intermediate\\code} \ar[r] & \toolbox{cdarmt32} \ar[r] \ar[d] \ar[rd] & \resource{object file} \\ & \resource{assembly\\listing} & \resource{debugging\\information}}
\seeassembly\seearm\seeobject\seecode\seedebugging
}

\providecommand{\cdarmcfpe}{
\toolsection{cdarmt32fpe} is a compiler for intermediate code targeting the ARM hardware architecture.
It generates machine code for ARM processors with floating-point extension executing T32 instructions from programs written in intermediate code and stores it in corresponding object files.
It also creates a debugging information file as well as an assembly file containing a listing of the generated machine code.
\debuggingtool
\flowgraph{\resource{intermediate\\code} \ar[r] & \toolbox{cdarmt32fpe} \ar[r] \ar[d] \ar[rd] & \resource{object file} \\ & \resource{assembly\\listing} & \resource{debugging\\information}}
\seeassembly\seearm\seeobject\seecode\seedebugging
}

\providecommand{\cdavr}{
\toolsection{cdavr} is a compiler for intermediate code targeting the AVR hardware architecture.
It generates machine code for AVR processors from programs written in intermediate code and stores it in corresponding object files.
It also creates a debugging information file as well as an assembly file containing a listing of the generated machine code.
\debuggingtool
\flowgraph{\resource{intermediate\\code} \ar[r] & \toolbox{cdavr} \ar[r] \ar[d] \ar[rd] & \resource{object file} \\ & \resource{assembly\\listing} & \resource{debugging\\information}}
\seeassembly\seeavr\seeobject\seecode\seedebugging
}

\providecommand{\cdavrtt}{
\toolsection{cdavr32} is a compiler for intermediate code targeting the AVR32 hardware architecture.
It generates machine code for AVR32 processors from programs written in intermediate code and stores it in corresponding object files.
It also creates a debugging information file as well as an assembly file containing a listing of the generated machine code.
\debuggingtool
\flowgraph{\resource{intermediate\\code} \ar[r] & \toolbox{cdavr32} \ar[r] \ar[d] \ar[rd] & \resource{object file} \\ & \resource{assembly\\listing} & \resource{debugging\\information}}
\seeassembly\seeavrtt\seeobject\seecode\seedebugging
}

\providecommand{\cdmabk}{
\toolsection{cdm68k} is a compiler for intermediate code targeting the M68000 hardware architecture.
It generates machine code for M68000 processors from programs written in intermediate code and stores it in corresponding object files.
It also creates a debugging information file as well as an assembly file containing a listing of the generated machine code.
\debuggingtool
\flowgraph{\resource{intermediate\\code} \ar[r] & \toolbox{cdm68k} \ar[r] \ar[d] \ar[rd] & \resource{object file} \\ & \resource{assembly\\listing} & \resource{debugging\\information}}
\seeassembly\seemabk\seeobject\seecode\seedebugging
}

\providecommand{\cdmibl}{
\toolsection{cdmibl} is a compiler for intermediate code targeting the MicroBlaze hardware architecture.
It generates machine code for MicroBlaze processors from programs written in intermediate code and stores it in corresponding object files.
It also creates a debugging information file as well as an assembly file containing a listing of the generated machine code.
\debuggingtool
\flowgraph{\resource{intermediate\\code} \ar[r] & \toolbox{cdmibl} \ar[r] \ar[d] \ar[rd] & \resource{object file} \\ & \resource{assembly\\listing} & \resource{debugging\\information}}
\seeassembly\seemibl\seeobject\seecode\seedebugging
}

\providecommand{\cdmipsa}{
\toolsection{cdmips32} is a compiler for intermediate code targeting the MIPS32 hardware architecture.
It generates machine code for MIPS32 processors from programs written in intermediate code and stores it in corresponding object files.
It also creates a debugging information file as well as an assembly file containing a listing of the generated machine code.
\debuggingtool
\flowgraph{\resource{intermediate\\code} \ar[r] & \toolbox{cdmips32} \ar[r] \ar[d] \ar[rd] & \resource{object file} \\ & \resource{assembly\\listing} & \resource{debugging\\information}}
\seeassembly\seemips\seeobject\seecode\seedebugging
}

\providecommand{\cdmipsb}{
\toolsection{cdmips64} is a compiler for intermediate code targeting the MIPS64 hardware architecture.
It generates machine code for MIPS64 processors from programs written in intermediate code and stores it in corresponding object files.
It also creates a debugging information file as well as an assembly file containing a listing of the generated machine code.
\debuggingtool
\flowgraph{\resource{intermediate\\code} \ar[r] & \toolbox{cdmips64} \ar[r] \ar[d] \ar[rd] & \resource{object file} \\ & \resource{assembly\\listing} & \resource{debugging\\information}}
\seeassembly\seemips\seeobject\seecode\seedebugging
}

\providecommand{\cdmmix}{
\toolsection{cdmmix} is a compiler for intermediate code targeting the MMIX hardware architecture.
It generates machine code for MMIX processors from programs written in intermediate code and stores it in corresponding object files.
It also creates a debugging information file as well as an assembly file containing a listing of the generated machine code.
\debuggingtool
\flowgraph{\resource{intermediate\\code} \ar[r] & \toolbox{cdmmix} \ar[r] \ar[d] \ar[rd] & \resource{object file} \\ & \resource{assembly\\listing} & \resource{debugging\\information}}
\seeassembly\seemmix\seeobject\seecode\seedebugging
}

\providecommand{\cdorok}{
\toolsection{cdor1k} is a compiler for intermediate code targeting the OpenRISC 1000 hardware architecture.
It generates machine code for OpenRISC 1000 processors from programs written in intermediate code and stores it in corresponding object files.
It also creates a debugging information file as well as an assembly file containing a listing of the generated machine code.
\debuggingtool
\flowgraph{\resource{intermediate\\code} \ar[r] & \toolbox{cdor1k} \ar[r] \ar[d] \ar[rd] & \resource{object file} \\ & \resource{assembly\\listing} & \resource{debugging\\information}}
\seeassembly\seeorok\seeobject\seecode\seedebugging
}

\providecommand{\cdppca}{
\toolsection{cdppc32} is a compiler for intermediate code targeting the PowerPC hardware architecture.
It generates machine code for PowerPC processors from programs written in intermediate code and stores it in corresponding object files.
The compiler generates machine code for the 32-bit operating mode defined by the PowerPC architecture.
It also creates a debugging information file as well as an assembly file containing a listing of the generated machine code.
\debuggingtool
\flowgraph{\resource{intermediate\\code} \ar[r] & \toolbox{cdppc32} \ar[r] \ar[d] \ar[rd] & \resource{object file} \\ & \resource{assembly\\listing} & \resource{debugging\\information}}
\seeassembly\seeppc\seeobject\seecode\seedebugging
}

\providecommand{\cdppcb}{
\toolsection{cdppc64} is a compiler for intermediate code targeting the PowerPC hardware architecture.
It generates machine code for PowerPC processors from programs written in intermediate code and stores it in corresponding object files.
The compiler generates machine code for the 64-bit operating mode defined by the PowerPC architecture.
It also creates a debugging information file as well as an assembly file containing a listing of the generated machine code.
\debuggingtool
\flowgraph{\resource{intermediate\\code} \ar[r] & \toolbox{cdppc64} \ar[r] \ar[d] \ar[rd] & \resource{object file} \\ & \resource{assembly\\listing} & \resource{debugging\\information}}
\seeassembly\seeppc\seeobject\seecode\seedebugging
}

\providecommand{\cdrisc}{
\toolsection{cdrisc} is a compiler for intermediate code targeting the RISC hardware architecture.
It generates machine code for RISC processors from programs written in intermediate code and stores it in corresponding object files.
It also creates a debugging information file as well as an assembly file containing a listing of the generated machine code.
\debuggingtool
\flowgraph{\resource{intermediate\\code} \ar[r] & \toolbox{cdrisc} \ar[r] \ar[d] \ar[rd] & \resource{object file} \\ & \resource{assembly\\listing} & \resource{debugging\\information}}
\seeassembly\seerisc\seeobject\seecode\seedebugging
}

\providecommand{\cdwasm}{
\toolsection{cdwasm} is a compiler for intermediate code targeting the WebAssembly architecture.
It generates machine code for WebAssembly targets from programs written in intermediate code and stores it in corresponding object files.
It also creates a debugging information file as well as an assembly file containing a listing of the generated machine code.
\debuggingtool
\flowgraph{\resource{intermediate\\code} \ar[r] & \toolbox{cdwasm} \ar[r] \ar[d] \ar[rd] & \resource{object file} \\ & \resource{assembly\\listing} & \resource{debugging\\information}}
\seeassembly\seewasm\seeobject\seecode\seedebugging
}

% C++ tools

\providecommand{\cppprep}{
\toolsection{cppprep} is a preprocessor for the \cpp{} programming language.
It preprocesses source code according to the rules of \cpp{} and writes it to the standard output stream.
Only the macro names \texttt{\_\_DATE\_\_}, \texttt{\_\_FILE\_\_}, \texttt{\_\_LINE\_\_}, and \texttt{\_\_TIME\_\_} are predefined.
\flowgraph{\resource{\cpp{} or other\\source code} \ar[r] & \toolbox{cppprep} \ar[r] & \resource{preprocessed\\source code} \\ & \variable{ECSINCLUDE} \ar[u]}
\seecpp
}

\providecommand{\cppprint}{
\toolsection{cppprint} is a pretty printer for the \cpp{} programming language.
It reformats the source code of \cpp{} programs and writes it to the standard output stream.
\flowgraph{\resource{\cpp{}\\source code} \ar[r] & \toolbox{cppprint} \ar[r] & \resource{reformatted\\source code} \\ & \variable{ECSINCLUDE} \ar[u]}
\seecpp
}

\providecommand{\cppcheck}{
\toolsection{cppcheck} is a syntactic and semantic checker for the \cpp{} programming language.
It just performs syntactic and semantic checks on \cpp{} programs and writes its diagnostic messages to the standard error stream.
\flowgraph{\resource{\cpp{}\\source code} \ar[r] & \toolbox{cppcheck} \ar[r] & \resource{diagnostic\\messages} \\ & \variable{ECSINCLUDE} \ar[u]}
\seecpp
}

\providecommand{\cppdump}{
\toolsection{cppdump} is a serializer for the \cpp{} programming language.
It dumps the complete internal representation of programs written in \cpp{} into an XML document.
\debuggingtool
\flowgraph{\resource{\cpp{}\\source code} \ar[r] & \toolbox{cppdump} \ar[r] & \resource{internal\\representation} \\ & \variable{ECSINCLUDE} \ar[u]}
\seecpp
}

\providecommand{\cpprun}{
\toolsection{cpprun} is an interpreter for the \cpp{} programming language.
It processes and executes programs written in \cpp{}.
The macro \texttt{\_\_run\_\_} is predefined in order to enable programmers to identify this tool while interpreting.
\flowgraph{\resource{\cpp{}\\source code} \ar[r] & \toolbox{cpprun} \ar@/u/[r] & \resource{input/\\output} \ar@/d/[l] \\ & \variable{ECSINCLUDE} \ar[u]}
\seecpp
}

\providecommand{\cppdoc}{
\toolsection{cppdoc} is a generic documentation generator for the \cpp{} programming language.
It processes several \cpp{} source files and assembles all information therein into a generic documentation.
\debuggingtool
\flowgraph{\resource{\cpp{}\\source code} \ar[r] & \toolbox{cppdoc} \ar[r] & \resource{generic\\documentation} \\ & \variable{ECSINCLUDE} \ar[u]}
\seecpp\seedocumentation
}

\providecommand{\cpphtml}{
\toolsection{cpphtml} is an HTML documentation generator for the \cpp{} programming language.
It processes several \cpp{} source files and assembles all information therein into an HTML document.
\flowgraph{\resource{\cpp{}\\source code} \ar[r] & \toolbox{cpphtml} \ar[r] & \resource{HTML\\document} \\ & \variable{ECSINCLUDE} \ar[u]}
\seecpp\seedocumentation
}

\providecommand{\cpplatex}{
\toolsection{cpplatex} is a Latex documentation generator for the \cpp{} programming language.
It processes several \cpp{} source files and assembles all information therein into a Latex document.
\flowgraph{\resource{\cpp{}\\source code} \ar[r] & \toolbox{cpplatex} \ar[r] & \resource{Latex\\document} \\ & \variable{ECSINCLUDE} \ar[u]}
\seecpp\seedocumentation
}

\providecommand{\cppcode}{
\toolsection{cppcode} is an intermediate code generator for the \cpp{} programming language.
It generates intermediate code from programs written in \cpp{} and stores it in corresponding assembly files.
The macro \texttt{\_\_code\_\_} is predefined in order to enable programmers to identify this tool while generating intermediate code.
Programs generated with this tool require additional runtime support that is stored in the \file{cpp\-code\-run} library file.
\debuggingtool
\flowgraph{\resource{\cpp{}\\source code} \ar[r] & \toolbox{cppcode} \ar[r] & \resource{intermediate\\code} \\ & \variable{ECSINCLUDE} \ar[u]}
\seecpp\seeassembly\seecode
}

\providecommand{\cppamda}{
\toolsection{cppamd16} is a compiler for the \cpp{} programming language targeting the AMD64 hardware architecture.
It generates machine code for AMD64 processors from programs written in \cpp{} and stores it in corresponding object files.
The compiler generates machine code for the 16-bit operating mode defined by the AMD64 architecture.
For debugging purposes, it also creates a debugging information file as well as an assembly file containing a listing of the generated machine code.
The macro \texttt{\_\_amd16\_\_} is predefined in order to enable programmers to identify this tool and its target architecture while compiling.
Programs generated with this compiler require additional runtime support that is stored in the \file{cpp\-amd16\-run} library file.
\flowgraph{\resource{\cpp{}\\source code} \ar[r] & \toolbox{cppamd16} \ar[r] \ar[d] \ar[rd] & \resource{object file} \\ \variable{ECSINCLUDE} \ar[ru] & \resource{debugging\\information} & \resource{assembly\\listing}}
\seecpp\seeassembly\seeamd\seeobject\seedebugging
}

\providecommand{\cppamdb}{
\toolsection{cppamd32} is a compiler for the \cpp{} programming language targeting the AMD64 hardware architecture.
It generates machine code for AMD64 processors from programs written in \cpp{} and stores it in corresponding object files.
The compiler generates machine code for the 32-bit operating mode defined by the AMD64 architecture.
For debugging purposes, it also creates a debugging information file as well as an assembly file containing a listing of the generated machine code.
The macro \texttt{\_\_amd32\_\_} is predefined in order to enable programmers to identify this tool and its target architecture while compiling.
Programs generated with this compiler require additional runtime support that is stored in the \file{cpp\-amd32\-run} library file.
\flowgraph{\resource{\cpp{}\\source code} \ar[r] & \toolbox{cppamd32} \ar[r] \ar[d] \ar[rd] & \resource{object file} \\ \variable{ECSINCLUDE} \ar[ru] & \resource{debugging\\information} & \resource{assembly\\listing}}
\seecpp\seeassembly\seeamd\seeobject\seedebugging
}

\providecommand{\cppamdc}{
\toolsection{cppamd64} is a compiler for the \cpp{} programming language targeting the AMD64 hardware architecture.
It generates machine code for AMD64 processors from programs written in \cpp{} and stores it in corresponding object files.
The compiler generates machine code for the 64-bit operating mode defined by the AMD64 architecture.
For debugging purposes, it also creates a debugging information file as well as an assembly file containing a listing of the generated machine code.
The macro \texttt{\_\_amd64\_\_} is predefined in order to enable programmers to identify this tool and its target architecture while compiling.
Programs generated with this compiler require additional runtime support that is stored in the \file{cpp\-amd64\-run} library file.
\flowgraph{\resource{\cpp{}\\source code} \ar[r] & \toolbox{cppamd64} \ar[r] \ar[d] \ar[rd] & \resource{object file} \\ \variable{ECSINCLUDE} \ar[ru] & \resource{debugging\\information} & \resource{assembly\\listing}}
\seecpp\seeassembly\seeamd\seeobject\seedebugging
}

\providecommand{\cpparma}{
\toolsection{cpparma32} is a compiler for the \cpp{} programming language targeting the ARM hardware architecture.
It generates machine code for ARM processors executing A32 instructions from programs written in \cpp{} and stores it in corresponding object files.
For debugging purposes, it also creates a debugging information file as well as an assembly file containing a listing of the generated machine code.
The macro \texttt{\_\_arma32\_\_} is predefined in order to enable programmers to identify this tool and its target architecture while compiling.
Programs generated with this compiler require additional runtime support that is stored in the \file{cpp\-arma32\-run} library file.
\flowgraph{\resource{\cpp{}\\source code} \ar[r] & \toolbox{cpparma32} \ar[r] \ar[d] \ar[rd] & \resource{object file} \\ \variable{ECSINCLUDE} \ar[ru] & \resource{debugging\\information} & \resource{assembly\\listing}}
\seecpp\seeassembly\seearm\seeobject\seedebugging
}

\providecommand{\cpparmb}{
\toolsection{cpparma64} is a compiler for the \cpp{} programming language targeting the ARM hardware architecture.
It generates machine code for ARM processors executing A64 instructions from programs written in \cpp{} and stores it in corresponding object files.
For debugging purposes, it also creates a debugging information file as well as an assembly file containing a listing of the generated machine code.
The macro \texttt{\_\_arma64\_\_} is predefined in order to enable programmers to identify this tool and its target architecture while compiling.
Programs generated with this compiler require additional runtime support that is stored in the \file{cpp\-arma64\-run} library file.
\flowgraph{\resource{\cpp{}\\source code} \ar[r] & \toolbox{cpparma64} \ar[r] \ar[d] \ar[rd] & \resource{object file} \\ \variable{ECSINCLUDE} \ar[ru] & \resource{debugging\\information} & \resource{assembly\\listing}}
\seecpp\seeassembly\seearm\seeobject\seedebugging
}

\providecommand{\cpparmc}{
\toolsection{cpparmt32} is a compiler for the \cpp{} programming language targeting the ARM hardware architecture.
It generates machine code for ARM processors without floating-point extension executing T32 instructions from programs written in \cpp{} and stores it in corresponding object files.
For debugging purposes, it also creates a debugging information file as well as an assembly file containing a listing of the generated machine code.
The macro \texttt{\_\_armt32\_\_} is predefined in order to enable programmers to identify this tool and its target architecture while compiling.
Programs generated with this compiler require additional runtime support that is stored in the \file{cpp\-armt32\-run} library file.
\flowgraph{\resource{\cpp{}\\source code} \ar[r] & \toolbox{cpparmt32} \ar[r] \ar[d] \ar[rd] & \resource{object file} \\ \variable{ECSINCLUDE} \ar[ru] & \resource{debugging\\information} & \resource{assembly\\listing}}
\seecpp\seeassembly\seearm\seeobject\seedebugging
}

\providecommand{\cpparmcfpe}{
\toolsection{cpparmt32fpe} is a compiler for the \cpp{} programming language targeting the ARM hardware architecture.
It generates machine code for ARM processors with floating-point extension executing T32 instructions from programs written in \cpp{} and stores it in corresponding object files.
For debugging purposes, it also creates a debugging information file as well as an assembly file containing a listing of the generated machine code.
The macro \texttt{\_\_armt32fpe\_\_} is predefined in order to enable programmers to identify this tool and its target architecture while compiling.
Programs generated with this compiler require additional runtime support that is stored in the \file{cpp\-armt32\-fpe\-run} library file.
\flowgraph{\resource{\cpp{}\\source code} \ar[r] & \toolbox{cpparmt32fpe} \ar[r] \ar[d] \ar[rd] & \resource{object file} \\ \variable{ECSINCLUDE} \ar[ru] & \resource{debugging\\information} & \resource{assembly\\listing}}
\seecpp\seeassembly\seearm\seeobject\seedebugging
}

\providecommand{\cppavr}{
\toolsection{cppavr} is a compiler for the \cpp{} programming language targeting the AVR hardware architecture.
It generates machine code for AVR processors from programs written in \cpp{} and stores it in corresponding object files.
For debugging purposes, it also creates a debugging information file as well as an assembly file containing a listing of the generated machine code.
The macro \texttt{\_\_avr\_\_} is predefined in order to enable programmers to identify this tool and its target architecture while compiling.
Programs generated with this compiler require additional runtime support that is stored in the \file{cpp\-avr\-run} library file.
\flowgraph{\resource{\cpp{}\\source code} \ar[r] & \toolbox{cppavr} \ar[r] \ar[d] \ar[rd] & \resource{object file} \\ \variable{ECSINCLUDE} \ar[ru] & \resource{debugging\\information} & \resource{assembly\\listing}}
\seecpp\seeassembly\seeavr\seeobject\seedebugging
}

\providecommand{\cppavrtt}{
\toolsection{cppavr32} is a compiler for the \cpp{} programming language targeting the AVR32 hardware architecture.
It generates machine code for AVR32 processors from programs written in \cpp{} and stores it in corresponding object files.
For debugging purposes, it also creates a debugging information file as well as an assembly file containing a listing of the generated machine code.
The macro \texttt{\_\_avr32\_\_} is predefined in order to enable programmers to identify this tool and its target architecture while compiling.
Programs generated with this compiler require additional runtime support that is stored in the \file{cpp\-avr32\-run} library file.
\flowgraph{\resource{\cpp{}\\source code} \ar[r] & \toolbox{cppavr32} \ar[r] \ar[d] \ar[rd] & \resource{object file} \\ \variable{ECSINCLUDE} \ar[ru] & \resource{debugging\\information} & \resource{assembly\\listing}}
\seecpp\seeassembly\seeavrtt\seeobject\seedebugging
}

\providecommand{\cppmabk}{
\toolsection{cppm68k} is a compiler for the \cpp{} programming language targeting the M68000 hardware architecture.
It generates machine code for M68000 processors from programs written in \cpp{} and stores it in corresponding object files.
For debugging purposes, it also creates a debugging information file as well as an assembly file containing a listing of the generated machine code.
The macro \texttt{\_\_m68k\_\_} is predefined in order to enable programmers to identify this tool and its target architecture while compiling.
Programs generated with this compiler require additional runtime support that is stored in the \file{cpp\-m68k\-run} library file.
\flowgraph{\resource{\cpp{}\\source code} \ar[r] & \toolbox{cppm68k} \ar[r] \ar[d] \ar[rd] & \resource{object file} \\ \variable{ECSINCLUDE} \ar[ru] & \resource{debugging\\information} & \resource{assembly\\listing}}
\seecpp\seeassembly\seemabk\seeobject\seedebugging
}

\providecommand{\cppmibl}{
\toolsection{cppmibl} is a compiler for the \cpp{} programming language targeting the MicroBlaze hardware architecture.
It generates machine code for MicroBlaze processors from programs written in \cpp{} and stores it in corresponding object files.
For debugging purposes, it also creates a debugging information file as well as an assembly file containing a listing of the generated machine code.
The macro \texttt{\_\_mibl\_\_} is predefined in order to enable programmers to identify this tool and its target architecture while compiling.
Programs generated with this compiler require additional runtime support that is stored in the \file{cpp\-mibl\-run} library file.
\flowgraph{\resource{\cpp{}\\source code} \ar[r] & \toolbox{cppmibl} \ar[r] \ar[d] \ar[rd] & \resource{object file} \\ \variable{ECSINCLUDE} \ar[ru] & \resource{debugging\\information} & \resource{assembly\\listing}}
\seecpp\seeassembly\seemibl\seeobject\seedebugging
}

\providecommand{\cppmipsa}{
\toolsection{cppmips32} is a compiler for the \cpp{} programming language targeting the MIPS32 hardware architecture.
It generates machine code for MIPS32 processors from programs written in \cpp{} and stores it in corresponding object files.
For debugging purposes, it also creates a debugging information file as well as an assembly file containing a listing of the generated machine code.
The macro \texttt{\_\_mips32\_\_} is predefined in order to enable programmers to identify this tool and its target architecture while compiling.
Programs generated with this compiler require additional runtime support that is stored in the \file{cpp\-mips32\-run} library file.
\flowgraph{\resource{\cpp{}\\source code} \ar[r] & \toolbox{cppmips32} \ar[r] \ar[d] \ar[rd] & \resource{object file} \\ \variable{ECSINCLUDE} \ar[ru] & \resource{debugging\\information} & \resource{assembly\\listing}}
\seecpp\seeassembly\seemips\seeobject\seedebugging
}

\providecommand{\cppmipsb}{
\toolsection{cppmips64} is a compiler for the \cpp{} programming language targeting the MIPS64 hardware architecture.
It generates machine code for MIPS64 processors from programs written in \cpp{} and stores it in corresponding object files.
For debugging purposes, it also creates a debugging information file as well as an assembly file containing a listing of the generated machine code.
The macro \texttt{\_\_mips64\_\_} is predefined in order to enable programmers to identify this tool and its target architecture while compiling.
Programs generated with this compiler require additional runtime support that is stored in the \file{cpp\-mips64\-run} library file.
\flowgraph{\resource{\cpp{}\\source code} \ar[r] & \toolbox{cppmips64} \ar[r] \ar[d] \ar[rd] & \resource{object file} \\ \variable{ECSINCLUDE} \ar[ru] & \resource{debugging\\information} & \resource{assembly\\listing}}
\seecpp\seeassembly\seemips\seeobject\seedebugging
}

\providecommand{\cppmmix}{
\toolsection{cppmmix} is a compiler for the \cpp{} programming language targeting the MMIX hardware architecture.
It generates machine code for MMIX processors from programs written in \cpp{} and stores it in corresponding object files.
For debugging purposes, it also creates a debugging information file as well as an assembly file containing a listing of the generated machine code.
The macro \texttt{\_\_mmix\_\_} is predefined in order to enable programmers to identify this tool and its target architecture while compiling.
Programs generated with this compiler require additional runtime support that is stored in the \file{cpp\-mmix\-run} library file.
\flowgraph{\resource{\cpp{}\\source code} \ar[r] & \toolbox{cppmmix} \ar[r] \ar[d] \ar[rd] & \resource{object file} \\ \variable{ECSINCLUDE} \ar[ru] & \resource{debugging\\information} & \resource{assembly\\listing}}
\seecpp\seeassembly\seemmix\seeobject\seedebugging
}

\providecommand{\cpporok}{
\toolsection{cppor1k} is a compiler for the \cpp{} programming language targeting the OpenRISC 1000 hardware architecture.
It generates machine code for OpenRISC 1000 processors from programs written in \cpp{} and stores it in corresponding object files.
For debugging purposes, it also creates a debugging information file as well as an assembly file containing a listing of the generated machine code.
The macro \texttt{\_\_or1k\_\_} is predefined in order to enable programmers to identify this tool and its target architecture while compiling.
Programs generated with this compiler require additional runtime support that is stored in the \file{cpp\-or1k\-run} library file.
\flowgraph{\resource{\cpp{}\\source code} \ar[r] & \toolbox{cppor1k} \ar[r] \ar[d] \ar[rd] & \resource{object file} \\ \variable{ECSINCLUDE} \ar[ru] & \resource{debugging\\information} & \resource{assembly\\listing}}
\seecpp\seeassembly\seeorok\seeobject\seedebugging
}

\providecommand{\cppppca}{
\toolsection{cppppc32} is a compiler for the \cpp{} programming language targeting the PowerPC hardware architecture.
It generates machine code for PowerPC processors from programs written in \cpp{} and stores it in corresponding object files.
The compiler generates machine code for the 32-bit operating mode defined by the PowerPC architecture.
For debugging purposes, it also creates a debugging information file as well as an assembly file containing a listing of the generated machine code.
The macro \texttt{\_\_ppc32\_\_} is predefined in order to enable programmers to identify this tool and its target architecture while compiling.
Programs generated with this compiler require additional runtime support that is stored in the \file{cpp\-ppc32\-run} library file.
\flowgraph{\resource{\cpp{}\\source code} \ar[r] & \toolbox{cppppc32} \ar[r] \ar[d] \ar[rd] & \resource{object file} \\ \variable{ECSINCLUDE} \ar[ru] & \resource{debugging\\information} & \resource{assembly\\listing}}
\seecpp\seeassembly\seeppc\seeobject\seedebugging
}

\providecommand{\cppppcb}{
\toolsection{cppppc64} is a compiler for the \cpp{} programming language targeting the PowerPC hardware architecture.
It generates machine code for PowerPC processors from programs written in \cpp{} and stores it in corresponding object files.
The compiler generates machine code for the 64-bit operating mode defined by the PowerPC architecture.
For debugging purposes, it also creates a debugging information file as well as an assembly file containing a listing of the generated machine code.
The macro \texttt{\_\_ppc64\_\_} is predefined in order to enable programmers to identify this tool and its target architecture while compiling.
Programs generated with this compiler require additional runtime support that is stored in the \file{cpp\-ppc64\-run} library file.
\flowgraph{\resource{\cpp{}\\source code} \ar[r] & \toolbox{cppppc64} \ar[r] \ar[d] \ar[rd] & \resource{object file} \\ \variable{ECSINCLUDE} \ar[ru] & \resource{debugging\\information} & \resource{assembly\\listing}}
\seecpp\seeassembly\seeppc\seeobject\seedebugging
}

\providecommand{\cpprisc}{
\toolsection{cpprisc} is a compiler for the \cpp{} programming language targeting the RISC hardware architecture.
It generates machine code for RISC processors from programs written in \cpp{} and stores it in corresponding object files.
For debugging purposes, it also creates a debugging information file as well as an assembly file containing a listing of the generated machine code.
The macro \texttt{\_\_risc\_\_} is predefined in order to enable programmers to identify this tool and its target architecture while compiling.
Programs generated with this compiler require additional runtime support that is stored in the \file{cpp\-risc\-run} library file.
\flowgraph{\resource{\cpp{}\\source code} \ar[r] & \toolbox{cpprisc} \ar[r] \ar[d] \ar[rd] & \resource{object file} \\ \variable{ECSINCLUDE} \ar[ru] & \resource{debugging\\information} & \resource{assembly\\listing}}
\seecpp\seeassembly\seerisc\seeobject\seedebugging
}

\providecommand{\cppwasm}{
\toolsection{cppwasm} is a compiler for the \cpp{} programming language targeting the WebAssembly architecture.
It generates machine code for WebAssembly targets from programs written in \cpp{} and stores it in corresponding object files.
For debugging purposes, it also creates a debugging information file as well as an assembly file containing a listing of the generated machine code.
The macro \texttt{\_\_wasm\_\_} is predefined in order to enable programmers to identify this tool and its target architecture while compiling.
Programs generated with this compiler require additional runtime support that is stored in the \file{cpp\-wasm\-run} library file.
\flowgraph{\resource{\cpp{}\\source code} \ar[r] & \toolbox{cppwasm} \ar[r] \ar[d] \ar[rd] & \resource{object file} \\ \variable{ECSINCLUDE} \ar[ru] & \resource{debugging\\information} & \resource{assembly\\listing}}
\seecpp\seeassembly\seewasm\seeobject\seedebugging
}

% FALSE tools

\providecommand{\falprint}{
\toolsection{falprint} is a pretty printer for the FALSE programming language.
It reformats the source code of FALSE programs and writes it to the standard output stream.
\flowgraph{\resource{FALSE\\source code} \ar[r] & \toolbox{falprint} \ar[r] & \resource{reformatted\\source code}}
\seefalse
}

\providecommand{\falcheck}{
\toolsection{falcheck} is a syntactic and semantic checker for the FALSE programming language.
It just performs syntactic and semantic checks on FALSE programs and writes its diagnostic messages to the standard error stream.
\flowgraph{\resource{FALSE\\source code} \ar[r] & \toolbox{falcheck} \ar[r] & \resource{diagnostic\\messages}}
\seefalse
}

\providecommand{\faldump}{
\toolsection{faldump} is a serializer for the FALSE programming language.
It dumps the complete internal representation of programs written in FALSE into an XML document.
\debuggingtool
\flowgraph{\resource{FALSE\\source code} \ar[r] & \toolbox{faldump} \ar[r] & \resource{internal\\representation}}
\seefalse
}

\providecommand{\falrun}{
\toolsection{falrun} is an interpreter for the FALSE programming language.
It processes and executes programs written in FALSE\@.
\flowgraph{\resource{FALSE\\source code} \ar[r] & \toolbox{falrun} \ar@/u/[r] & \resource{input/\\output} \ar@/d/[l]}
\seefalse
}

\providecommand{\falcpp}{
\toolsection{falcpp} is a transpiler for the FALSE programming language.
It translates programs written in FALSE into \cpp{} programs and stores them in corresponding source files.
\flowgraph{\resource{FALSE\\source code} \ar[r] & \toolbox{falcpp} \ar[r] & \resource{\cpp{}\\source file}}
\seefalse\seecpp
}

\providecommand{\falcode}{
\toolsection{falcode} is an intermediate code generator for the FALSE programming language.
It generates intermediate code from programs written in FALSE and stores it in corresponding assembly files.
\debuggingtool
\flowgraph{\resource{FALSE\\source code} \ar[r] & \toolbox{falcode} \ar[r] & \resource{intermediate\\code}}
\seefalse\seeassembly\seecode
}

\providecommand{\falamda}{
\toolsection{falamd16} is a compiler for the FALSE programming language targeting the AMD64 hardware architecture.
It generates machine code for AMD64 processors from programs written in FALSE and stores it in corresponding object files.
The compiler generates machine code for the 16-bit operating mode defined by the AMD64 architecture.
\flowgraph{\resource{FALSE\\source code} \ar[r] & \toolbox{falamd16} \ar[r] & \resource{object file}}
\seefalse\seeamd\seeobject
}

\providecommand{\falamdb}{
\toolsection{falamd32} is a compiler for the FALSE programming language targeting the AMD64 hardware architecture.
It generates machine code for AMD64 processors from programs written in FALSE and stores it in corresponding object files.
The compiler generates machine code for the 32-bit operating mode defined by the AMD64 architecture.
\flowgraph{\resource{FALSE\\source code} \ar[r] & \toolbox{falamd32} \ar[r] & \resource{object file}}
\seefalse\seeamd\seeobject
}

\providecommand{\falamdc}{
\toolsection{falamd64} is a compiler for the FALSE programming language targeting the AMD64 hardware architecture.
It generates machine code for AMD64 processors from programs written in FALSE and stores it in corresponding object files.
The compiler generates machine code for the 64-bit operating mode defined by the AMD64 architecture.
\flowgraph{\resource{FALSE\\source code} \ar[r] & \toolbox{falamd64} \ar[r] & \resource{object file}}
\seefalse\seeamd\seeobject
}

\providecommand{\falarma}{
\toolsection{falarma32} is a compiler for the FALSE programming language targeting the ARM hardware architecture.
It generates machine code for ARM processors executing A32 instructions from programs written in FALSE and stores it in corresponding object files.
\flowgraph{\resource{FALSE\\source code} \ar[r] & \toolbox{falarma32} \ar[r] & \resource{object file}}
\seefalse\seearm\seeobject
}

\providecommand{\falarmb}{
\toolsection{falarma64} is a compiler for the FALSE programming language targeting the ARM hardware architecture.
It generates machine code for ARM processors executing A64 instructions from programs written in FALSE and stores it in corresponding object files.
\flowgraph{\resource{FALSE\\source code} \ar[r] & \toolbox{falarma64} \ar[r] & \resource{object file}}
\seefalse\seearm\seeobject
}

\providecommand{\falarmc}{
\toolsection{falarmt32} is a compiler for the FALSE programming language targeting the ARM hardware architecture.
It generates machine code for ARM processors without floating-point extension executing T32 instructions from programs written in FALSE and stores it in corresponding object files.
\flowgraph{\resource{FALSE\\source code} \ar[r] & \toolbox{falarmt32} \ar[r] & \resource{object file}}
\seefalse\seearm\seeobject
}

\providecommand{\falarmcfpe}{
\toolsection{falarmt32fpe} is a compiler for the FALSE programming language targeting the ARM hardware architecture.
It generates machine code for ARM processors with floating-point extension executing T32 instructions from programs written in FALSE and stores it in corresponding object files.
\flowgraph{\resource{FALSE\\source code} \ar[r] & \toolbox{falarmt32fpe} \ar[r] & \resource{object file}}
\seefalse\seearm\seeobject
}

\providecommand{\falavr}{
\toolsection{falavr} is a compiler for the FALSE programming language targeting the AVR hardware architecture.
It generates machine code for AVR processors from programs written in FALSE and stores it in corresponding object files.
\flowgraph{\resource{FALSE\\source code} \ar[r] & \toolbox{falavr} \ar[r] & \resource{object file}}
\seefalse\seeavr\seeobject
}

\providecommand{\falavrtt}{
\toolsection{falavr32} is a compiler for the FALSE programming language targeting the AVR32 hardware architecture.
It generates machine code for AVR32 processors from programs written in FALSE and stores it in corresponding object files.
\flowgraph{\resource{FALSE\\source code} \ar[r] & \toolbox{falavr32} \ar[r] & \resource{object file}}
\seefalse\seeavrtt\seeobject
}

\providecommand{\falmabk}{
\toolsection{falm68k} is a compiler for the FALSE programming language targeting the M68000 hardware architecture.
It generates machine code for M68000 processors from programs written in FALSE and stores it in corresponding object files.
\flowgraph{\resource{FALSE\\source code} \ar[r] & \toolbox{falm68k} \ar[r] & \resource{object file}}
\seefalse\seemabk\seeobject
}

\providecommand{\falmibl}{
\toolsection{falmibl} is a compiler for the FALSE programming language targeting the MicroBlaze hardware architecture.
It generates machine code for MicroBlaze processors from programs written in FALSE and stores it in corresponding object files.
\flowgraph{\resource{FALSE\\source code} \ar[r] & \toolbox{falmibl} \ar[r] & \resource{object file}}
\seefalse\seemibl\seeobject
}

\providecommand{\falmipsa}{
\toolsection{falmips32} is a compiler for the FALSE programming language targeting the MIPS32 hardware architecture.
It generates machine code for MIPS32 processors from programs written in FALSE and stores it in corresponding object files.
\flowgraph{\resource{FALSE\\source code} \ar[r] & \toolbox{falmips32} \ar[r] & \resource{object file}}
\seefalse\seemips\seeobject
}

\providecommand{\falmipsb}{
\toolsection{falmips64} is a compiler for the FALSE programming language targeting the MIPS64 hardware architecture.
It generates machine code for MIPS64 processors from programs written in FALSE and stores it in corresponding object files.
\flowgraph{\resource{FALSE\\source code} \ar[r] & \toolbox{falmips64} \ar[r] & \resource{object file}}
\seefalse\seemips\seeobject
}

\providecommand{\falmmix}{
\toolsection{falmmix} is a compiler for the FALSE programming language targeting the MMIX hardware architecture.
It generates machine code for MMIX processors from programs written in FALSE and stores it in corresponding object files.
\flowgraph{\resource{FALSE\\source code} \ar[r] & \toolbox{falmmix} \ar[r] & \resource{object file}}
\seefalse\seemmix\seeobject
}

\providecommand{\falorok}{
\toolsection{falor1k} is a compiler for the FALSE programming language targeting the OpenRISC 1000 hardware architecture.
It generates machine code for OpenRISC 1000 processors from programs written in FALSE and stores it in corresponding object files.
\flowgraph{\resource{FALSE\\source code} \ar[r] & \toolbox{falor1k} \ar[r] & \resource{object file}}
\seefalse\seeorok\seeobject
}

\providecommand{\falppca}{
\toolsection{falppc32} is a compiler for the FALSE programming language targeting the PowerPC hardware architecture.
It generates machine code for PowerPC processors from programs written in FALSE and stores it in corresponding object files.
The compiler generates machine code for the 32-bit operating mode defined by the PowerPC architecture.
\flowgraph{\resource{FALSE\\source code} \ar[r] & \toolbox{falppc32} \ar[r] & \resource{object file}}
\seefalse\seeppc\seeobject
}

\providecommand{\falppcb}{
\toolsection{falppc64} is a compiler for the FALSE programming language targeting the PowerPC hardware architecture.
It generates machine code for PowerPC processors from programs written in FALSE and stores it in corresponding object files.
The compiler generates machine code for the 64-bit operating mode defined by the PowerPC architecture.
\flowgraph{\resource{FALSE\\source code} \ar[r] & \toolbox{falppc64} \ar[r] & \resource{object file}}
\seefalse\seeppc\seeobject
}

\providecommand{\falrisc}{
\toolsection{falrisc} is a compiler for the FALSE programming language targeting the RISC hardware architecture.
It generates machine code for RISC processors from programs written in FALSE and stores it in corresponding object files.
\flowgraph{\resource{FALSE\\source code} \ar[r] & \toolbox{falrisc} \ar[r] & \resource{object file}}
\seefalse\seerisc\seeobject
}

\providecommand{\falwasm}{
\toolsection{falwasm} is a compiler for the FALSE programming language targeting the WebAssembly architecture.
It generates machine code for WebAssembly targets from programs written in FALSE and stores it in corresponding object files.
\flowgraph{\resource{FALSE\\source code} \ar[r] & \toolbox{falwasm} \ar[r] & \resource{object file}}
\seefalse\seewasm\seeobject
}

% Oberon tools

\providecommand{\obprint}{
\toolsection{obprint} is a pretty printer for the Oberon programming language.
It reformats the source code of Oberon modules and writes it to the standard output stream.
\flowgraph{\resource{Oberon\\source code} \ar[r] & \toolbox{obprint} \ar[r] & \resource{reformatted\\source code}}
\seeoberon
}

\providecommand{\obcheck}{
\toolsection{obcheck} is a syntactic and semantic checker for the Oberon programming language.
It just performs syntactic and semantic checks on Oberon modules and writes its diagnostic messages to the standard error stream.
In addition, it stores the interface of each module in a symbol file which is required when other modules import the module.
\flowgraph{\resource{Oberon\\source code} \ar[r] & \toolbox{obcheck} \ar[r] \ar@/l/[d] & \resource{diagnostic\\messages} \\ \variable{ECSIMPORT} \ar[ru] & \resource{symbol\\files} \ar@/r/[u]}
\seeoberon
}

\providecommand{\obdump}{
\toolsection{obdump} is a serializer for the Oberon programming language.
It dumps the complete internal representation of modules written in Oberon into an XML document.
\debuggingtool
\flowgraph{\resource{Oberon\\source code} \ar[r] & \toolbox{obdump} \ar[r] \ar@/l/[d] & \resource{internal\\representation} \\ \variable{ECSIMPORT} \ar[ru] & \resource{symbol\\files} \ar@/r/[u]}
\seeoberon
}

\providecommand{\obrun}{
\toolsection{obrun} is an interpreter for the Oberon programming language.
It processes and executes modules written in Oberon.
This tool does neither generate nor process symbol files while interpreting modules.
If a module is imported by another one, its filename has to be named before the other one in the list of command-line arguments.
\flowgraph{\resource{Oberon\\source code} \ar[r] & \toolbox{obrun} \ar@/u/[r] & \resource{input/\\output} \ar@/d/[l]}
\seeoberon
}

\providecommand{\obcpp}{
\toolsection{obcpp} is a transpiler for the Oberon programming language.
It translates programs written in Oberon into \cpp{} programs and stores them in corresponding source and header files.
In addition, it stores the interface of each module in a symbol file which is required when other modules import the module.
The same interface is provided by the generated header file which can be used in other parts of the \cpp{} program.
\flowgraph{\resource{Oberon\\source code} \ar[r] & \toolbox{obcpp} \ar[r] \ar@/l/[d] \ar[rd] & \resource{\cpp{}\\source file} \\ \variable{ECSIMPORT} \ar[ru] & \resource{symbol\\files} \ar@/r/[u] & \resource{\cpp{}\\header file}}
\seeoberon\seecpp
}

\providecommand{\obdoc}{
\toolsection{obdoc} is a generic documentation generator for the Oberon programming language.
It processes several Oberon modules and assembles all information therein into a generic documentation.
In addition, it stores the interface of each module in a symbol file which is required when other modules import the module.
\debuggingtool
\flowgraph{\resource{Oberon\\source code} \ar[r] & \toolbox{obdoc} \ar[r] \ar@/l/[d] & \resource{generic\\documentation} \\ \variable{ECSIMPORT} \ar[ru] & \resource{symbol\\files} \ar@/r/[u]}
\seeoberon\seedocumentation
}

\providecommand{\obhtml}{
\toolsection{obhtml} is an HTML documentation generator for the Oberon programming language.
It processes several Oberon modules and assembles all information therein into an HTML document.
In addition, it stores the interface of each module in a symbol file which is required when other modules import the module.
\flowgraph{\resource{Oberon\\source code} \ar[r] & \toolbox{obhtml} \ar[r] \ar@/l/[d] & \resource{HTML\\document} \\ \variable{ECSIMPORT} \ar[ru] & \resource{symbol\\files} \ar@/r/[u]}
\seeoberon\seedocumentation
}

\providecommand{\oblatex}{
\toolsection{oblatex} is a Latex documentation generator for the Oberon programming language.
It processes several Oberon modules and assembles all information therein into a Latex document.
In addition, it stores the interface of each module in a symbol file which is required when other modules import the module.
\flowgraph{\resource{Oberon\\source code} \ar[r] & \toolbox{oblatex} \ar[r] \ar@/l/[d] & \resource{Latex\\document} \\ \variable{ECSIMPORT} \ar[ru] & \resource{symbol\\files} \ar@/r/[u]}
\seeoberon\seedocumentation
}

\providecommand{\obcode}{
\toolsection{obcode} is an intermediate code generator for the Oberon programming language.
It generates intermediate code from modules written in Oberon and stores it in corresponding assembly files.
In addition, it stores the interface of each module in a symbol file which is required when other modules import the module.
Programs generated with this tool require additional runtime support that is stored in the \file{ob\-code\-run} library file.
\debuggingtool
\flowgraph{\resource{Oberon\\source code} \ar[r] & \toolbox{obcode} \ar[r] \ar@/l/[d] & \resource{intermediate\\code} \\ \variable{ECSIMPORT} \ar[ru] & \resource{symbol\\files} \ar@/r/[u]}
\seeoberon\seeassembly\seecode
}

\providecommand{\obamda}{
\toolsection{obamd16} is a compiler for the Oberon programming language targeting the AMD64 hardware architecture.
It generates machine code for AMD64 processors from modules written in Oberon and stores it in corresponding object files.
The compiler generates machine code for the 16-bit operating mode defined by the AMD64 architecture.
For debugging purposes, it also creates a debugging information file as well as an assembly file containing a listing of the generated machine code.
In addition, it stores the interface of each module in a symbol file which is required when other modules import the module.
Programs generated with this compiler require additional runtime support that is stored in the \file{ob\-amd16\-run} library file.
\flowgraph{\resource{Oberon\\source code} \ar[r] & \toolbox{obamd16} \ar[r] \ar@/l/[d] \ar[rd] & \resource{object file} \\ \variable{ECSIMPORT} \ar[ru] & \resource{symbol\\files} \ar@/r/[u] & \resource{debugging\\information}}
\seeoberon\seeassembly\seeamd\seeobject\seedebugging
}

\providecommand{\obamdb}{
\toolsection{obamd32} is a compiler for the Oberon programming language targeting the AMD64 hardware architecture.
It generates machine code for AMD64 processors from modules written in Oberon and stores it in corresponding object files.
The compiler generates machine code for the 32-bit operating mode defined by the AMD64 architecture.
For debugging purposes, it also creates a debugging information file as well as an assembly file containing a listing of the generated machine code.
In addition, it stores the interface of each module in a symbol file which is required when other modules import the module.
Programs generated with this compiler require additional runtime support that is stored in the \file{ob\-amd32\-run} library file.
\flowgraph{\resource{Oberon\\source code} \ar[r] & \toolbox{obamd32} \ar[r] \ar@/l/[d] \ar[rd] & \resource{object file} \\ \variable{ECSIMPORT} \ar[ru] & \resource{symbol\\files} \ar@/r/[u] & \resource{debugging\\information}}
\seeoberon\seeassembly\seeamd\seeobject\seedebugging
}

\providecommand{\obamdc}{
\toolsection{obamd64} is a compiler for the Oberon programming language targeting the AMD64 hardware architecture.
It generates machine code for AMD64 processors from modules written in Oberon and stores it in corresponding object files.
The compiler generates machine code for the 64-bit operating mode defined by the AMD64 architecture.
For debugging purposes, it also creates a debugging information file as well as an assembly file containing a listing of the generated machine code.
In addition, it stores the interface of each module in a symbol file which is required when other modules import the module.
Programs generated with this compiler require additional runtime support that is stored in the \file{ob\-amd64\-run} library file.
\flowgraph{\resource{Oberon\\source code} \ar[r] & \toolbox{obamd64} \ar[r] \ar@/l/[d] \ar[rd] & \resource{object file} \\ \variable{ECSIMPORT} \ar[ru] & \resource{symbol\\files} \ar@/r/[u] & \resource{debugging\\information}}
\seeoberon\seeassembly\seeamd\seeobject\seedebugging
}

\providecommand{\obarma}{
\toolsection{obarma32} is a compiler for the Oberon programming language targeting the ARM hardware architecture.
It generates machine code for ARM processors executing A32 instructions from modules written in Oberon and stores it in corresponding object files.
For debugging purposes, it also creates a debugging information file as well as an assembly file containing a listing of the generated machine code.
In addition, it stores the interface of each module in a symbol file which is required when other modules import the module.
Programs generated with this compiler require additional runtime support that is stored in the \file{ob\-arma32\-run} library file.
\flowgraph{\resource{Oberon\\source code} \ar[r] & \toolbox{obarma32} \ar[r] \ar@/l/[d] \ar[rd] & \resource{object file} \\ \variable{ECSIMPORT} \ar[ru] & \resource{symbol\\files} \ar@/r/[u] & \resource{debugging\\information}}
\seeoberon\seeassembly\seearm\seeobject\seedebugging
}

\providecommand{\obarmb}{
\toolsection{obarma64} is a compiler for the Oberon programming language targeting the ARM hardware architecture.
It generates machine code for ARM processors executing A64 instructions from modules written in Oberon and stores it in corresponding object files.
For debugging purposes, it also creates a debugging information file as well as an assembly file containing a listing of the generated machine code.
In addition, it stores the interface of each module in a symbol file which is required when other modules import the module.
Programs generated with this compiler require additional runtime support that is stored in the \file{ob\-arma64\-run} library file.
\flowgraph{\resource{Oberon\\source code} \ar[r] & \toolbox{obarma64} \ar[r] \ar@/l/[d] \ar[rd] & \resource{object file} \\ \variable{ECSIMPORT} \ar[ru] & \resource{symbol\\files} \ar@/r/[u] & \resource{debugging\\information}}
\seeoberon\seeassembly\seearm\seeobject\seedebugging
}

\providecommand{\obarmc}{
\toolsection{obarmt32} is a compiler for the Oberon programming language targeting the ARM hardware architecture.
It generates machine code for ARM processors without floating-point extension executing T32 instructions from modules written in Oberon and stores it in corresponding object files.
For debugging purposes, it also creates a debugging information file as well as an assembly file containing a listing of the generated machine code.
In addition, it stores the interface of each module in a symbol file which is required when other modules import the module.
Programs generated with this compiler require additional runtime support that is stored in the \file{ob\-armt32\-run} library file.
\flowgraph{\resource{Oberon\\source code} \ar[r] & \toolbox{obarmt32} \ar[r] \ar@/l/[d] \ar[rd] & \resource{object file} \\ \variable{ECSIMPORT} \ar[ru] & \resource{symbol\\files} \ar@/r/[u] & \resource{debugging\\information}}
\seeoberon\seeassembly\seearm\seeobject\seedebugging
}

\providecommand{\obarmcfpe}{
\toolsection{obarmt32fpe} is a compiler for the Oberon programming language targeting the ARM hardware architecture.
It generates machine code for ARM processors with floating-point extension executing T32 instructions from modules written in Oberon and stores it in corresponding object files.
For debugging purposes, it also creates a debugging information file as well as an assembly file containing a listing of the generated machine code.
In addition, it stores the interface of each module in a symbol file which is required when other modules import the module.
Programs generated with this compiler require additional runtime support that is stored in the \file{ob\-armt32\-fpe\-run} library file.
\flowgraph{\resource{Oberon\\source code} \ar[r] & \toolbox{obarmt32fpe} \ar[r] \ar@/l/[d] \ar[rd] & \resource{object file} \\ \variable{ECSIMPORT} \ar[ru] & \resource{symbol\\files} \ar@/r/[u] & \resource{debugging\\information}}
\seeoberon\seeassembly\seearm\seeobject\seedebugging
}

\providecommand{\obavr}{
\toolsection{obavr} is a compiler for the Oberon programming language targeting the AVR hardware architecture.
It generates machine code for AVR processors from modules written in Oberon and stores it in corresponding object files.
For debugging purposes, it also creates a debugging information file as well as an assembly file containing a listing of the generated machine code.
In addition, it stores the interface of each module in a symbol file which is required when other modules import the module.
Programs generated with this compiler require additional runtime support that is stored in the \file{ob\-avr\-run} library file.
\flowgraph{\resource{Oberon\\source code} \ar[r] & \toolbox{obavr} \ar[r] \ar@/l/[d] \ar[rd] & \resource{object file} \\ \variable{ECSIMPORT} \ar[ru] & \resource{symbol\\files} \ar@/r/[u] & \resource{debugging\\information}}
\seeoberon\seeassembly\seeavr\seeobject\seedebugging
}

\providecommand{\obavrtt}{
\toolsection{obavr32} is a compiler for the Oberon programming language targeting the AVR32 hardware architecture.
It generates machine code for AVR32 processors from modules written in Oberon and stores it in corresponding object files.
For debugging purposes, it also creates a debugging information file as well as an assembly file containing a listing of the generated machine code.
In addition, it stores the interface of each module in a symbol file which is required when other modules import the module.
Programs generated with this compiler require additional runtime support that is stored in the \file{ob\-avr32\-run} library file.
\flowgraph{\resource{Oberon\\source code} \ar[r] & \toolbox{obavr32} \ar[r] \ar@/l/[d] \ar[rd] & \resource{object file} \\ \variable{ECSIMPORT} \ar[ru] & \resource{symbol\\files} \ar@/r/[u] & \resource{debugging\\information}}
\seeoberon\seeassembly\seeavrtt\seeobject\seedebugging
}

\providecommand{\obmabk}{
\toolsection{obm68k} is a compiler for the Oberon programming language targeting the M68000 hardware architecture.
It generates machine code for M68000 processors from modules written in Oberon and stores it in corresponding object files.
For debugging purposes, it also creates a debugging information file as well as an assembly file containing a listing of the generated machine code.
In addition, it stores the interface of each module in a symbol file which is required when other modules import the module.
Programs generated with this compiler require additional runtime support that is stored in the \file{ob\-m68k\-run} library file.
\flowgraph{\resource{Oberon\\source code} \ar[r] & \toolbox{obm68k} \ar[r] \ar@/l/[d] \ar[rd] & \resource{object file} \\ \variable{ECSIMPORT} \ar[ru] & \resource{symbol\\files} \ar@/r/[u] & \resource{debugging\\information}}
\seeoberon\seeassembly\seemabk\seeobject\seedebugging
}

\providecommand{\obmibl}{
\toolsection{obmibl} is a compiler for the Oberon programming language targeting the MicroBlaze hardware architecture.
It generates machine code for MicroBlaze processors from modules written in Oberon and stores it in corresponding object files.
For debugging purposes, it also creates a debugging information file as well as an assembly file containing a listing of the generated machine code.
In addition, it stores the interface of each module in a symbol file which is required when other modules import the module.
Programs generated with this compiler require additional runtime support that is stored in the \file{ob\-mibl\-run} library file.
\flowgraph{\resource{Oberon\\source code} \ar[r] & \toolbox{obmibl} \ar[r] \ar@/l/[d] \ar[rd] & \resource{object file} \\ \variable{ECSIMPORT} \ar[ru] & \resource{symbol\\files} \ar@/r/[u] & \resource{debugging\\information}}
\seeoberon\seeassembly\seemibl\seeobject\seedebugging
}

\providecommand{\obmipsa}{
\toolsection{obmips32} is a compiler for the Oberon programming language targeting the MIPS32 hardware architecture.
It generates machine code for MIPS32 processors from modules written in Oberon and stores it in corresponding object files.
For debugging purposes, it also creates a debugging information file as well as an assembly file containing a listing of the generated machine code.
In addition, it stores the interface of each module in a symbol file which is required when other modules import the module.
Programs generated with this compiler require additional runtime support that is stored in the \file{ob\-mips32\-run} library file.
\flowgraph{\resource{Oberon\\source code} \ar[r] & \toolbox{obmips32} \ar[r] \ar@/l/[d] \ar[rd] & \resource{object file} \\ \variable{ECSIMPORT} \ar[ru] & \resource{symbol\\files} \ar@/r/[u] & \resource{debugging\\information}}
\seeoberon\seeassembly\seemips\seeobject\seedebugging
}

\providecommand{\obmipsb}{
\toolsection{obmips64} is a compiler for the Oberon programming language targeting the MIPS64 hardware architecture.
It generates machine code for MIPS64 processors from modules written in Oberon and stores it in corresponding object files.
For debugging purposes, it also creates a debugging information file as well as an assembly file containing a listing of the generated machine code.
In addition, it stores the interface of each module in a symbol file which is required when other modules import the module.
Programs generated with this compiler require additional runtime support that is stored in the \file{ob\-mips64\-run} library file.
\flowgraph{\resource{Oberon\\source code} \ar[r] & \toolbox{obmips64} \ar[r] \ar@/l/[d] \ar[rd] & \resource{object file} \\ \variable{ECSIMPORT} \ar[ru] & \resource{symbol\\files} \ar@/r/[u] & \resource{debugging\\information}}
\seeoberon\seeassembly\seemips\seeobject\seedebugging
}

\providecommand{\obmmix}{
\toolsection{obmmix} is a compiler for the Oberon programming language targeting the MMIX hardware architecture.
It generates machine code for MMIX processors from modules written in Oberon and stores it in corresponding object files.
For debugging purposes, it also creates a debugging information file as well as an assembly file containing a listing of the generated machine code.
In addition, it stores the interface of each module in a symbol file which is required when other modules import the module.
Programs generated with this compiler require additional runtime support that is stored in the \file{ob\-mmix\-run} library file.
\flowgraph{\resource{Oberon\\source code} \ar[r] & \toolbox{obmmix} \ar[r] \ar@/l/[d] \ar[rd] & \resource{object file} \\ \variable{ECSIMPORT} \ar[ru] & \resource{symbol\\files} \ar@/r/[u] & \resource{debugging\\information}}
\seeoberon\seeassembly\seemmix\seeobject\seedebugging
}

\providecommand{\oborok}{
\toolsection{obor1k} is a compiler for the Oberon programming language targeting the OpenRISC 1000 hardware architecture.
It generates machine code for OpenRISC 1000 processors from modules written in Oberon and stores it in corresponding object files.
For debugging purposes, it also creates a debugging information file as well as an assembly file containing a listing of the generated machine code.
In addition, it stores the interface of each module in a symbol file which is required when other modules import the module.
Programs generated with this compiler require additional runtime support that is stored in the \file{ob\-or1k\-run} library file.
\flowgraph{\resource{Oberon\\source code} \ar[r] & \toolbox{obor1k} \ar[r] \ar@/l/[d] \ar[rd] & \resource{object file} \\ \variable{ECSIMPORT} \ar[ru] & \resource{symbol\\files} \ar@/r/[u] & \resource{debugging\\information}}
\seeoberon\seeassembly\seeorok\seeobject\seedebugging
}

\providecommand{\obppca}{
\toolsection{obppc32} is a compiler for the Oberon programming language targeting the PowerPC hardware architecture.
It generates machine code for PowerPC processors from modules written in Oberon and stores it in corresponding object files.
The compiler generates machine code for the 32-bit operating mode defined by the PowerPC architecture.
For debugging purposes, it also creates a debugging information file as well as an assembly file containing a listing of the generated machine code.
In addition, it stores the interface of each module in a symbol file which is required when other modules import the module.
Programs generated with this compiler require additional runtime support that is stored in the \file{ob\-ppc32\-run} library file.
\flowgraph{\resource{Oberon\\source code} \ar[r] & \toolbox{obppc32} \ar[r] \ar@/l/[d] \ar[rd] & \resource{object file} \\ \variable{ECSIMPORT} \ar[ru] & \resource{symbol\\files} \ar@/r/[u] & \resource{debugging\\information}}
\seeoberon\seeassembly\seeppc\seeobject\seedebugging
}

\providecommand{\obppcb}{
\toolsection{obppc64} is a compiler for the Oberon programming language targeting the PowerPC hardware architecture.
It generates machine code for PowerPC processors from modules written in Oberon and stores it in corresponding object files.
The compiler generates machine code for the 64-bit operating mode defined by the PowerPC architecture.
For debugging purposes, it also creates a debugging information file as well as an assembly file containing a listing of the generated machine code.
In addition, it stores the interface of each module in a symbol file which is required when other modules import the module.
Programs generated with this compiler require additional runtime support that is stored in the \file{ob\-ppc64\-run} library file.
\flowgraph{\resource{Oberon\\source code} \ar[r] & \toolbox{obppc64} \ar[r] \ar@/l/[d] \ar[rd] & \resource{object file} \\ \variable{ECSIMPORT} \ar[ru] & \resource{symbol\\files} \ar@/r/[u] & \resource{debugging\\information}}
\seeoberon\seeassembly\seeppc\seeobject\seedebugging
}

\providecommand{\obrisc}{
\toolsection{obrisc} is a compiler for the Oberon programming language targeting the RISC hardware architecture.
It generates machine code for RISC processors from modules written in Oberon and stores it in corresponding object files.
For debugging purposes, it also creates a debugging information file as well as an assembly file containing a listing of the generated machine code.
In addition, it stores the interface of each module in a symbol file which is required when other modules import the module.
Programs generated with this compiler require additional runtime support that is stored in the \file{ob\-risc\-run} library file.
\flowgraph{\resource{Oberon\\source code} \ar[r] & \toolbox{obrisc} \ar[r] \ar@/l/[d] \ar[rd] & \resource{object file} \\ \variable{ECSIMPORT} \ar[ru] & \resource{symbol\\files} \ar@/r/[u] & \resource{debugging\\information}}
\seeoberon\seeassembly\seerisc\seeobject\seedebugging
}

\providecommand{\obwasm}{
\toolsection{obwasm} is a compiler for the Oberon programming language targeting the WebAssembly architecture.
It generates machine code for WebAssembly targets from modules written in Oberon and stores it in corresponding object files.
For debugging purposes, it also creates a debugging information file as well as an assembly file containing a listing of the generated machine code.
In addition, it stores the interface of each module in a symbol file which is required when other modules import the module.
Programs generated with this compiler require additional runtime support that is stored in the \file{ob\-wasm\-run} library file.
\flowgraph{\resource{Oberon\\source code} \ar[r] & \toolbox{obwasm} \ar[r] \ar@/l/[d] \ar[rd] & \resource{object file} \\ \variable{ECSIMPORT} \ar[ru] & \resource{symbol\\files} \ar@/r/[u] & \resource{debugging\\information}}
\seeoberon\seeassembly\seewasm\seeobject\seedebugging
}

% converter tools

\providecommand{\dbgdwarf}{
\toolsection{dbgdwarf} is a DWARF debugging information converter tool.
It converts debugging information into the DWARF debugging data format and stores it in corresponding object files~\cite{dwarffile}.
The resulting debugging object files can be combined with runtime support that creates Executable and Linking Format (ELF) files~\cite{elffile}.
\flowgraph{\resource{debugging\\information} \ar[r] & \toolbox{dbgdwarf} \ar[r] & \resource{debugging\\object file}}
\seeobject\seedebugging
}

% assembler tools

\providecommand{\asmprint}{
\toolsection{asmprint} is a pretty printer for generic assembly code.
It reformats generic assembly code and writes it to the standard output stream.
\flowgraph{\resource{generic assembly\\source code} \ar[r] & \toolbox{asmprint} \ar[r] & \resource{reformatted\\source code}}
\seeassembly
}

\providecommand{\amdaasm}{
\toolsection{amd16asm} is an assembler for the AMD64 hardware architecture.
It translates assembly code into machine code for AMD64 processors and stores it in corresponding object files.
By default, the assembler generates machine code for the 16-bit operating mode defined by the AMD64 architecture.
\flowgraph{\resource{AMD16 assembly\\source code} \ar[r] & \toolbox{amd16asm} \ar[r] & \resource{object file}}
\seeassembly\seeamd\seeobject
}

\providecommand{\amdadism}{
\toolsection{amd16dism} is a disassembler for the AMD64 hardware architecture.
It translates machine code from object files targeting AMD64 processors into assembly code and writes it to the standard output stream.
It assumes that the machine code was generated for the 16-bit operating mode defined by the AMD64 architecture.
\flowgraph{\resource{object file} \ar[r] & \toolbox{amd16dism} \ar[r] & \resource{disassembly\\listing}}
\seeassembly\seeamd\seeobject
}

\providecommand{\amdbasm}{
\toolsection{amd32asm} is an assembler for the AMD64 hardware architecture.
It translates assembly code into machine code for AMD64 processors and stores it in corresponding object files.
By default, the assembler generates machine code for the 32-bit operating mode defined by the AMD64 architecture.
\flowgraph{\resource{AMD32 assembly\\source code} \ar[r] & \toolbox{amd32asm} \ar[r] & \resource{object file}}
\seeassembly\seeamd\seeobject
}

\providecommand{\amdbdism}{
\toolsection{amd32dism} is a disassembler for the AMD64 hardware architecture.
It translates machine code from object files targeting AMD64 processors into assembly code and writes it to the standard output stream.
It assumes that the machine code was generated for the 32-bit operating mode defined by the AMD64 architecture.
\flowgraph{\resource{object file} \ar[r] & \toolbox{amd32dism} \ar[r] & \resource{disassembly\\listing}}
\seeassembly\seeamd\seeobject
}

\providecommand{\amdcasm}{
\toolsection{amd64asm} is an assembler for the AMD64 hardware architecture.
It translates assembly code into machine code for AMD64 processors and stores it in corresponding object files.
By default, the assembler generates machine code for the 64-bit operating mode defined by the AMD64 architecture.
\flowgraph{\resource{AMD64 assembly\\source code} \ar[r] & \toolbox{amd64asm} \ar[r] & \resource{object file}}
\seeassembly\seeamd\seeobject
}

\providecommand{\amdcdism}{
\toolsection{amd64dism} is a disassembler for the AMD64 hardware architecture.
It translates machine code from object files targeting AMD64 processors into assembly code and writes it to the standard output stream.
It assumes that the machine code was generated for the 64-bit operating mode defined by the AMD64 architecture.
\flowgraph{\resource{object file} \ar[r] & \toolbox{amd64dism} \ar[r] & \resource{disassembly\\listing}}
\seeassembly\seeamd\seeobject
}

\providecommand{\armaasm}{
\toolsection{arma32asm} is an assembler for the ARM hardware architecture.
It translates assembly code into machine code for ARM processors executing A32 instructions and stores it in corresponding object files.
\flowgraph{\resource{ARM A32 assembly\\source code} \ar[r] & \toolbox{arma32asm} \ar[r] & \resource{object file}}
\seeassembly\seearm\seeobject
}

\providecommand{\armadism}{
\toolsection{arma32dism} is a disassembler for the ARM hardware architecture.
It translates machine code from object files targeting ARM processors executing A32 instructions into assembly code and writes it to the standard output stream.
\flowgraph{\resource{object file} \ar[r] & \toolbox{arma32dism} \ar[r] & \resource{disassembly\\listing}}
\seeassembly\seearm\seeobject
}

\providecommand{\armbasm}{
\toolsection{arma64asm} is an assembler for the ARM hardware architecture.
It translates assembly code into machine code for ARM processors executing A64 instructions and stores it in corresponding object files.
\flowgraph{\resource{ARM A64 assembly\\source code} \ar[r] & \toolbox{arma64asm} \ar[r] & \resource{object file}}
\seeassembly\seearm\seeobject
}

\providecommand{\armbdism}{
\toolsection{arma64dism} is a disassembler for the ARM hardware architecture.
It translates machine code from object files targeting ARM processors executing A64 instructions into assembly code and writes it to the standard output stream.
\flowgraph{\resource{object file} \ar[r] & \toolbox{arma64dism} \ar[r] & \resource{disassembly\\listing}}
\seeassembly\seearm\seeobject
}

\providecommand{\armcasm}{
\toolsection{armt32asm} is an assembler for the ARM hardware architecture.
It translates assembly code into machine code for ARM processors executing T32 instructions and stores it in corresponding object files.
\flowgraph{\resource{ARM T32 assembly\\source code} \ar[r] & \toolbox{armt32asm} \ar[r] & \resource{object file}}
\seeassembly\seearm\seeobject
}

\providecommand{\armcdism}{
\toolsection{armt32dism} is a disassembler for the ARM hardware architecture.
It translates machine code from object files targeting ARM processors executing T32 instructions into assembly code and writes it to the standard output stream.
\flowgraph{\resource{object file} \ar[r] & \toolbox{armt32dism} \ar[r] & \resource{disassembly\\listing}}
\seeassembly\seearm\seeobject
}

\providecommand{\avrasm}{
\toolsection{avrasm} is an assembler for the AVR hardware architecture.
It translates assembly code into machine code for AVR processors and stores it in corresponding object files.
The identifiers \texttt{RXL}, \texttt{RXH}, \texttt{RYL}, \texttt{RYH}, \texttt{RZL}, and \texttt{RZH} are predefined and name the corresponding registers.
The identifiers \texttt{SPL} and \texttt{SPH} are also predefined and evaluate to the address of the corresponding registers.
\flowgraph{\resource{AVR assembly\\source code} \ar[r] & \toolbox{avrasm} \ar[r] & \resource{object file}}
\seeassembly\seeavr\seeobject
}

\providecommand{\avrdism}{
\toolsection{avrdism} is a disassembler for the AVR hardware architecture.
It translates machine code from object files targeting AVR processors into assembly code and writes it to the standard output stream.
\flowgraph{\resource{object file} \ar[r] & \toolbox{avrdism} \ar[r] & \resource{disassembly\\listing}}
\seeassembly\seeavr\seeobject
}

\providecommand{\avrttasm}{
\toolsection{avr32asm} is an assembler for the AVR32 hardware architecture.
It translates assembly code into machine code for AVR32 processors and stores it in corresponding object files.
\flowgraph{\resource{AVR32 assembly\\source code} \ar[r] & \toolbox{avr32asm} \ar[r] & \resource{object file}}
\seeassembly\seeavrtt\seeobject
}

\providecommand{\avrttdism}{
\toolsection{avr32dism} is a disassembler for the AVR32 hardware architecture.
It translates machine code from object files targeting AVR32 processors into assembly code and writes it to the standard output stream.
\flowgraph{\resource{object file} \ar[r] & \toolbox{avr32dism} \ar[r] & \resource{disassembly\\listing}}
\seeassembly\seeavrtt\seeobject
}

\providecommand{\mabkasm}{
\toolsection{m68kasm} is an assembler for the M68000 hardware architecture.
It translates assembly code into machine code for M68000 processors and stores it in corresponding object files.
\flowgraph{\resource{68000 assembly\\source code} \ar[r] & \toolbox{m68kasm} \ar[r] & \resource{object file}}
\seeassembly\seemabk\seeobject
}

\providecommand{\mabkdism}{
\toolsection{m68kdism} is a disassembler for the M68000 hardware architecture.
It translates machine code from object files targeting M68000 processors into assembly code and writes it to the standard output stream.
\flowgraph{\resource{object file} \ar[r] & \toolbox{m68kdism} \ar[r] & \resource{disassembly\\listing}}
\seeassembly\seemabk\seeobject
}

\providecommand{\miblasm}{
\toolsection{miblasm} is an assembler for the MicroBlaze hardware architecture.
It translates assembly code into machine code for MicroBlaze processors and stores it in corresponding object files.
\flowgraph{\resource{MicroBlaze assembly\\source code} \ar[r] & \toolbox{miblasm} \ar[r] & \resource{object file}}
\seeassembly\seemibl\seeobject
}

\providecommand{\mibldism}{
\toolsection{mibldism} is a disassembler for the MicroBlaze hardware architecture.
It translates machine code from object files targeting MicroBlaze processors into assembly code and writes it to the standard output stream.
\flowgraph{\resource{object file} \ar[r] & \toolbox{mibldism} \ar[r] & \resource{disassembly\\listing}}
\seeassembly\seemibl\seeobject
}

\providecommand{\mipsaasm}{
\toolsection{mips32asm} is an assembler for the MIPS32 hardware architecture.
It translates assembly code into machine code for MIPS32 processors and stores it in corresponding object files.
\flowgraph{\resource{MIPS32 assembly\\source code} \ar[r] & \toolbox{mips32asm} \ar[r] & \resource{object file}}
\seeassembly\seemips\seeobject
}

\providecommand{\mipsadism}{
\toolsection{mips32dism} is a disassembler for the MIPS32 hardware architecture.
It translates machine code from object files targeting MIPS32 processors into assembly code and writes it to the standard output stream.
\flowgraph{\resource{object file} \ar[r] & \toolbox{mips32dism} \ar[r] & \resource{disassembly\\listing}}
\seeassembly\seemips\seeobject
}

\providecommand{\mipsbasm}{
\toolsection{mips64asm} is an assembler for the MIPS64 hardware architecture.
It translates assembly code into machine code for MIPS64 processors and stores it in corresponding object files.
\flowgraph{\resource{MIPS64 assembly\\source code} \ar[r] & \toolbox{mips64asm} \ar[r] & \resource{object file}}
\seeassembly\seemips\seeobject
}

\providecommand{\mipsbdism}{
\toolsection{mips64dism} is a disassembler for the MIPS64 hardware architecture.
It translates machine code from object files targeting MIPS64 processors into assembly code and writes it to the standard output stream.
\flowgraph{\resource{object file} \ar[r] & \toolbox{mips64dism} \ar[r] & \resource{disassembly\\listing}}
\seeassembly\seemips\seeobject
}

\providecommand{\mmixasm}{
\toolsection{mmixasm} is an assembler for the MMIX hardware architecture.
It translates assembly code into machine code for MMIX processors and stores it in corresponding object files.
The names of all special registers are predefined and evaluate to the corresponding number.
\flowgraph{\resource{MMIX assembly\\source code} \ar[r] & \toolbox{mmixasm} \ar[r] & \resource{object file}}
\seeassembly\seemmix\seeobject
}

\providecommand{\mmixdism}{
\toolsection{mmixdism} is a disassembler for the MMIX hardware architecture.
It translates machine code from object files targeting MMIX processors into assembly code and writes it to the standard output stream.
\flowgraph{\resource{object file} \ar[r] & \toolbox{mmixdism} \ar[r] & \resource{disassembly\\listing}}
\seeassembly\seemmix\seeobject
}

\providecommand{\orokasm}{
\toolsection{or1kasm} is an assembler for the OpenRISC 1000 hardware architecture.
It translates assembly code into machine code for OpenRISC 1000 processors and stores it in corresponding object files.
\flowgraph{\resource{OpenRISC 1000 assembly\\source code} \ar[r] & \toolbox{or1kasm} \ar[r] & \resource{object file}}
\seeassembly\seeorok\seeobject
}

\providecommand{\orokdism}{
\toolsection{or1kdism} is a disassembler for the OpenRISC 1000 hardware architecture.
It translates machine code from object files targeting OpenRISC 1000 processors into assembly code and writes it to the standard output stream.
\flowgraph{\resource{object file} \ar[r] & \toolbox{or1kdism} \ar[r] & \resource{disassembly\\listing}}
\seeassembly\seeorok\seeobject
}

\providecommand{\ppcaasm}{
\toolsection{ppc32asm} is an assembler for the PowerPC hardware architecture.
It translates assembly code into machine code for PowerPC processors and stores it in corresponding object files.
By default, the assembler generates machine code for the 32-bit operating mode defined by the PowerPC architecture.
\flowgraph{\resource{PowerPC assembly\\source code} \ar[r] & \toolbox{ppc32asm} \ar[r] & \resource{object file}}
\seeassembly\seeppc\seeobject
}

\providecommand{\ppcadism}{
\toolsection{ppc32dism} is a disassembler for the PowerPC hardware architecture.
It translates machine code from object files targeting PowerPC processors into assembly code and writes it to the standard output stream.
It assumes that the machine code was generated for the 32-bit operating mode defined by the PowerPC architecture.
\flowgraph{\resource{object file} \ar[r] & \toolbox{ppc32dism} \ar[r] & \resource{disassembly\\listing}}
\seeassembly\seeppc\seeobject
}

\providecommand{\ppcbasm}{
\toolsection{ppc64asm} is an assembler for the PowerPC hardware architecture.
It translates assembly code into machine code for PowerPC processors and stores it in corresponding object files.
By default, the assembler generates machine code for the 64-bit operating mode defined by the PowerPC architecture.
\flowgraph{\resource{PowerPC assembly\\source code} \ar[r] & \toolbox{ppc64asm} \ar[r] & \resource{object file}}
\seeassembly\seeppc\seeobject
}

\providecommand{\ppcbdism}{
\toolsection{ppc64dism} is a disassembler for the PowerPC hardware architecture.
It translates machine code from object files targeting PowerPC processors into assembly code and writes it to the standard output stream.
It assumes that the machine code was generated for the 64-bit operating mode defined by the PowerPC architecture.
\flowgraph{\resource{object file} \ar[r] & \toolbox{ppc64dism} \ar[r] & \resource{disassembly\\listing}}
\seeassembly\seeppc\seeobject
}

\providecommand{\riscasm}{
\toolsection{riscasm} is an assembler for the RISC hardware architecture.
It translates assembly code into machine code for RISC processors and stores it in corresponding object files.
The names of all special registers are predefined and evaluate to the corresponding number.
\flowgraph{\resource{RISC assembly\\source code} \ar[r] & \toolbox{riscasm} \ar[r] & \resource{object file}}
\seeassembly\seerisc\seeobject
}

\providecommand{\riscdism}{
\toolsection{riscdism} is a disassembler for the RISC hardware architecture.
It translates machine code from object files targeting RISC processors into assembly code and writes it to the standard output stream.
\flowgraph{\resource{object file} \ar[r] & \toolbox{riscdism} \ar[r] & \resource{disassembly\\listing}}
\seeassembly\seerisc\seeobject
}

\providecommand{\wasmasm}{
\toolsection{wasmasm} is an assembler for the WebAssembly architecture.
It translates assembly code into machine code for WebAssembly targets and stores it in corresponding object files.
The names of all special registers are predefined and evaluate to the corresponding number.
\flowgraph{\resource{WebAssembly assembly\\source code} \ar[r] & \toolbox{wasmasm} \ar[r] & \resource{object file}}
\seeassembly\seewasm\seeobject
}

\providecommand{\wasmdism}{
\toolsection{wasmdism} is a disassembler for the WebAssembly architecture.
It translates machine code from object files targeting WebAssembly targets into assembly code and writes it to the standard output stream.
\flowgraph{\resource{object file} \ar[r] & \toolbox{wasmdism} \ar[r] & \resource{disassembly\\listing}}
\seeassembly\seewasm\seeobject
}

% linker tools

\providecommand{\linklib}{
\toolsection{linklib} is an object file combiner.
It creates a static library file by combining all object files given to it into a single one.
\flowgraph{\resource{object files} \ar[r] & \toolbox{linklib} \ar[r] & \resource{library file}}
\seeobject
}

\providecommand{\linkbin}{
\toolsection{linkbin} is a linker for plain binary files.
It links all object files given to it into a single image and stores it in a binary file that begins with the first linked section.
It also creates a map file that lists the address, type, name and size of all used sections.
The filename extension of the resulting binary file can be specified by putting it into a constant data section called \texttt{\_extension}.
\flowgraph{\resource{object files} \ar[r] & \toolbox{linkbin} \ar[r] \ar[d] & \resource{binary file} \\ & \resource{map file}}
\seeobject
}

\providecommand{\linkmem}{
\toolsection{linkmem} is a linker for plain binary files partitioned into random-access and read-only memory.
It links all object files given to it into two distinct images, one for data sections and one for code and constant data sections, and stores each image in a binary file that begins with the first linked section of the corresponding type.
It also creates a map file that lists the address, type, name and size of all used sections.
\flowgraph{\resource{object files} \ar[r] & \toolbox{linkmem} \ar[r] \ar[d] & \resource{RAM file/\\ROM file} \\ & \resource{map file}}
\seeobject
}

\providecommand{\linkprg}{
\toolsection{linkprg} is a linker for GEMDOS executable files.
It links all object files given to it into a single image and stores the image in an Atari GEMDOS executable file~\cite{gemdosfile}.
It also creates a map file that lists the address relative to the text segment, type, name and size of all used sections.
The filename extension of the resulting executable file can be specified by putting it into a constant data section called \texttt{\_extension}.
The GEMDOS executable file format requires all patch patterns of absolute link patches to consist of four full bitmasks with descending offsets.
\flowgraph{\resource{object files} \ar[r] & \toolbox{linkprg} \ar[r] \ar[d] & \resource{executable file} \\ & \resource{map file}}
\seeobject
}

\providecommand{\linkhex}{
\toolsection{linkhex} is a linker for Intel HEX files.
It links all code sections of the object files given to it into single image and stores the image in an Intel HEX file~\cite{hexfile} that begins with the first linked section.
It also creates a map file that lists the address, type, name and size of all used sections.
\flowgraph{\resource{object files} \ar[r] & \toolbox{linkhex} \ar[r] \ar[d] & \resource{HEX file} \\ & \resource{map file}}
\seeobject
}

\providecommand{\mapsearch}{
\toolsection{mapsearch} is a debugging tool.
It searches map files generated by linker tools for the name of a binary section that encompasses a memory address read from the standard input stream.
If additionally provided with one or more object files, it also stores an excerpt thereof in a separate object file called map search result which only contains the identified binary section for disassembling purposes.
\flowgraph{& \resource{map files/\\object files} \ar[d] \\ \resource{memory\\address} \ar[r] & \toolbox{mapsearch} \ar[r] \ar[d] & \resource{section name/\\relative offset} \\ & \resource{object file\\excerpt}}
\seeobject
}


\startchapter{GNU Free Documentation~License}{GNU Free Documentation License}{fdl}{}

\begin{center}

Version 1.3, 3 November 2008

\medskip
Copyright \copyright{} 2000, 2001, 2002, 2007, 2008 Free Software Foundation, Inc. \url{https://fsf.org/}

\end{center}

Everyone is permitted to copy and distribute verbatim copies
of this license document, but changing it is not allowed.

\setcounter{section}{-1}
\renewcommand{\thesection}{\arabic{section}.}

\section{Preamble}

The purpose of this License is to make a manual, textbook, or other
functional and useful document ``free'' in the sense of freedom: to
assure everyone the effective freedom to copy and redistribute it,
with or without modifying it, either commercially or noncommercially.
Secondarily, this License preserves for the author and publisher a way
to get credit for their work, while not being considered responsible
for modifications made by others.

This License is a kind of ``copyleft'', which means that derivative
works of the document must themselves be free in the same sense. It
complements the \gpl{}, which is a copyleft
license designed for free software.

We have designed this License in order to use it for manuals for free
software, because free software needs free documentation: a free
program should come with manuals providing the same freedoms that the
software does. But this License is not limited to software manuals;
it can be used for any textual work, regardless of subject matter or
whether it is published as a printed book. We recommend this License
principally for works whose purpose is instruction or reference.

\section{Applicability and Definitions}

This License applies to any manual or other work, in any medium, that
contains a notice placed by the copyright holder saying it can be
distributed under the terms of this License. Such a notice grants a
world-wide, royalty-free license, unlimited in duration, to use that
work under the conditions stated herein. The ``Document'', below,
refers to any such manual or work. Any member of the public is a
licensee, and is addressed as ``you''. You accept the license if you
copy, modify or distribute the work in a way requiring permission
under copyright law.

A ``Modified Version'' of the Document means any work containing the
Document or a portion of it, either copied verbatim, or with
modifications and/or translated into another language.

A ``Secondary Section'' is a named appendix or a front-matter section of
the Document that deals exclusively with the relationship of the
publishers or authors of the Document to the Document's overall
subject (or to related matters) and contains nothing that could fall
directly within that overall subject. (Thus, if the Document is in
part a textbook of mathematics, a Secondary Section may not explain
any mathematics.) The relationship could be a matter of historical
connection with the subject or with related matters, or of legal,
commercial, philosophical, ethical or political position regarding
them.

The ``Invariant Sections'' are certain Secondary Sections whose titles
are designated, as being those of Invariant Sections, in the notice
that says that the Document is released under this License. If a
section does not fit the above definition of Secondary then it is not
allowed to be designated as Invariant. The Document may contain zero
Invariant Sections. If the Document does not identify any Invariant
Sections then there are none.

The ``Cover Texts'' are certain short passages of text that are listed,
as Front-Cover Texts or Back-Cover Texts, in the notice that says that
the Document is released under this License. A Front-Cover Text may
be at most 5 words, and a Back-Cover Text may be at most 25 words.

A ``Transparent'' copy of the Document means a machine-readable copy,
represented in a format whose specification is available to the
general public, that is suitable for revising the document
straightforwardly with generic text editors or (for images composed of
pixels) generic paint programs or (for drawings) some widely available
drawing editor, and that is suitable for input to text formatters or
for automatic translation to a variety of formats suitable for input
to text formatters. A copy made in an otherwise Transparent file
format whose markup, or absence of markup, has been arranged to thwart
or discourage subsequent modification by readers is not Transparent.
An image format is not Transparent if used for any substantial amount
of text. A copy that is not ``Transparent'' is called ``Opaque''.

Examples of suitable formats for Transparent copies include plain
ASCII without markup, Texinfo input format, LaTeX input format, SGML
or XML using a publicly available DTD, and standard-conforming simple
HTML, PostScript or PDF designed for human modification. Examples of
transparent image formats include PNG, XCF and JPG. Opaque formats
include proprietary formats that can be read and edited only by
proprietary word processors, SGML or XML for which the DTD and/or
processing tools are not generally available, and the
machine-generated HTML, PostScript or PDF produced by some word
processors for output purposes only.

The ``Title Page'' means, for a printed book, the title page itself,
plus such following pages as are needed to hold, legibly, the material
this License requires to appear in the title page. For works in
formats which do not have any title page as such, ``Title Page'' means
the text near the most prominent appearance of the work's title,
preceding the beginning of the body of the text.

The ``publisher'' means any person or entity that distributes copies of
the Document to the public.

A section ``Entitled XYZ'' means a named subunit of the Document whose
title either is precisely XYZ or contains XYZ in parentheses following
text that translates XYZ in another language. (Here XYZ stands for a
specific section name mentioned below, such as ``Acknowledgements'',
``Dedications'', ``Endorsements'', or ``History''.) To ``Preserve the Title''
of such a section when you modify the Document means that it remains a
section ``Entitled XYZ'' according to this definition.

The Document may include Warranty Disclaimers next to the notice which
states that this License applies to the Document. These Warranty
Disclaimers are considered to be included by reference in this
License, but only as regards disclaiming warranties: any other
implication that these Warranty Disclaimers may have is void and has
no effect on the meaning of this License.

\section{Verbatim Copying}

You may copy and distribute the Document in any medium, either
commercially or noncommercially, provided that this License, the
copyright notices, and the license notice saying this License applies
to the Document are reproduced in all copies, and that you add no
other conditions whatsoever to those of this License. You may not use
technical measures to obstruct or control the reading or further
copying of the copies you make or distribute. However, you may accept
compensation in exchange for copies. If you distribute a large enough
number of copies you must also follow the conditions in section 3.

You may also lend copies, under the same conditions stated above, and
you may publicly display copies.

\section{Copying in Quantity}

If you publish printed copies (or copies in media that commonly have
printed covers) of the Document, numbering more than 100, and the
Document's license notice requires Cover Texts, you must enclose the
copies in covers that carry, clearly and legibly, all these Cover
Texts: Front-Cover Texts on the front cover, and Back-Cover Texts on
the back cover. Both covers must also clearly and legibly identify
you as the publisher of these copies. The front cover must present
the full title with all words of the title equally prominent and
visible. You may add other material on the covers in addition.
Copying with changes limited to the covers, as long as they preserve
the title of the Document and satisfy these conditions, can be treated
as verbatim copying in other respects.

If the required texts for either cover are too voluminous to fit
legibly, you should put the first ones listed (as many as fit
reasonably) on the actual cover, and continue the rest onto adjacent
pages.

If you publish or distribute Opaque copies of the Document numbering
more than 100, you must either include a machine-readable Transparent
copy along with each Opaque copy, or state in or with each Opaque copy
a computer-network location from which the general network-using
public has access to download using public-standard network protocols
a complete Transparent copy of the Document, free of added material.
If you use the latter option, you must take reasonably prudent steps,
when you begin distribution of Opaque copies in quantity, to ensure
that this Transparent copy will remain thus accessible at the stated
location until at least one year after the last time you distribute an
Opaque copy (directly or through your agents or retailers) of that
edition to the public.

It is requested, but not required, that you contact the authors of the
Document well before redistributing any large number of copies, to
give them a chance to provide you with an updated version of the
Document.

\section{Modifications}

You may copy and distribute a Modified Version of the Document under
the conditions of sections 2 and 3 above, provided that you release
the Modified Version under precisely this License, with the Modified
Version filling the role of the Document, thus licensing distribution
and modification of the Modified Version to whoever possesses a copy
of it. In addition, you must do these things in the Modified Version:

\begin{itemize}

\item[A.]
Use in the Title Page (and on the covers, if any) a title distinct
from that of the Document, and from those of previous versions
(which should, if there were any, be listed in the History section
of the Document). You may use the same title as a previous version
if the original publisher of that version gives permission.

\item[B.]
List on the Title Page, as authors, one or more persons or entities
responsible for authorship of the modifications in the Modified
Version, together with at least five of the principal authors of the
Document (all of its principal authors, if it has fewer than five),
unless they release you from this requirement.

\item[C.]
State on the Title page the name of the publisher of the
Modified Version, as the publisher.

\item[D.]
Preserve all the copyright notices of the Document.

\item[E.]
Add an appropriate copyright notice for your modifications
adjacent to the other copyright notices.

\item[F.]
Include, immediately after the copyright notices, a license notice
giving the public permission to use the Modified Version under the
terms of this License, in the form shown in the Addendum below.

\item[G.]
Preserve in that license notice the full lists of Invariant Sections
and required Cover Texts given in the Document's license notice.

\item[H.]
Include an unaltered copy of this License.

\item[I.]
Preserve the section Entitled ``History'', Preserve its Title, and add
to it an item stating at least the title, year, new authors, and
publisher of the Modified Version as given on the Title Page. If
there is no section Entitled ``History'' in the Document, create one
stating the title, year, authors, and publisher of the Document as
given on its Title Page, then add an item describing the Modified
Version as stated in the previous sentence.

\item[J.]
Preserve the network location, if any, given in the Document for
public access to a Transparent copy of the Document, and likewise
the network locations given in the Document for previous versions
it was based on. These may be placed in the ``History'' section.
You may omit a network location for a work that was published at
least four years before the Document itself, or if the original
publisher of the version it refers to gives permission.

\item[K.]
For any section Entitled ``Acknowledgements'' or ``Dedications'',
Preserve the Title of the section, and preserve in the section all
the substance and tone of each of the contributor acknowledgements
and/or dedications given therein.

\item[L.]
Preserve all the Invariant Sections of the Document,
unaltered in their text and in their titles. Section numbers
or the equivalent are not considered part of the section titles.

\item[M.]
Delete any section Entitled ``Endorsements''. Such a section
may not be included in the Modified Version.

\item[N.]
Do not retitle any existing section to be Entitled ``Endorsements''
or to conflict in title with any Invariant Section.

\item[O.]
Preserve any Warranty Disclaimers.

\end{itemize}

If the Modified Version includes new front-matter sections or
appendices that qualify as Secondary Sections and contain no material
copied from the Document, you may at your option designate some or all
of these sections as invariant. To do this, add their titles to the
list of Invariant Sections in the Modified Version's license notice.
These titles must be distinct from any other section titles.

You may add a section Entitled ``Endorsements'', provided it contains
nothing but endorsements of your Modified Version by various
parties--for example, statements of peer review or that the text has
been approved by an organization as the authoritative definition of a
standard.

You may add a passage of up to five words as a Front-Cover Text, and a
passage of up to 25 words as a Back-Cover Text, to the end of the list
of Cover Texts in the Modified Version. Only one passage of
Front-Cover Text and one of Back-Cover Text may be added by (or
through arrangements made by) any one entity. If the Document already
includes a cover text for the same cover, previously added by you or
by arrangement made by the same entity you are acting on behalf of,
you may not add another; but you may replace the old one, on explicit
permission from the previous publisher that added the old one.

The author(s) and publisher(s) of the Document do not by this License
give permission to use their names for publicity for or to assert or
imply endorsement of any Modified Version.

\section{Combining Documents}

You may combine the Document with other documents released under this
License, under the terms defined in section 4 above for modified
versions, provided that you include in the combination all of the
Invariant Sections of all of the original documents, unmodified, and
list them all as Invariant Sections of your combined work in its
license notice, and that you preserve all their Warranty Disclaimers.

The combined work need only contain one copy of this License, and
multiple identical Invariant Sections may be replaced with a single
copy. If there are multiple Invariant Sections with the same name but
different contents, make the title of each such section unique by
adding at the end of it, in parentheses, the name of the original
author or publisher of that section if known, or else a unique number.
Make the same adjustment to the section titles in the list of
Invariant Sections in the license notice of the combined work.

In the combination, you must combine any sections Entitled ``History''
in the various original documents, forming one section Entitled
``History''; likewise combine any sections Entitled ``Acknowledgements'',
and any sections Entitled ``Dedications''. You must delete all sections
Entitled ``Endorsements''.

\section{Collections of Documents}

You may make a collection consisting of the Document and other
documents released under this License, and replace the individual
copies of this License in the various documents with a single copy
that is included in the collection, provided that you follow the rules
of this License for verbatim copying of each of the documents in all
other respects.

You may extract a single document from such a collection, and
distribute it individually under this License, provided you insert a
copy of this License into the extracted document, and follow this
License in all other respects regarding verbatim copying of that
document.

\section{Aggregation of Independent Works}

A compilation of the Document or its derivatives with other separate
and independent documents or works, in or on a volume of a storage or
distribution medium, is called an ``aggregate'' if the copyright
resulting from the compilation is not used to limit the legal rights
of the compilation's users beyond what the individual works permit.
When the Document is included in an aggregate, this License does not
apply to the other works in the aggregate which are not themselves
derivative works of the Document.

If the Cover Text requirement of section 3 is applicable to these
copies of the Document, then if the Document is less than one half of
the entire aggregate, the Document's Cover Texts may be placed on
covers that bracket the Document within the aggregate, or the
electronic equivalent of covers if the Document is in electronic form.
Otherwise they must appear on printed covers that bracket the whole
aggregate.

\section{Translation}

Translation is considered a kind of modification, so you may
distribute translations of the Document under the terms of section 4.
Replacing Invariant Sections with translations requires special
permission from their copyright holders, but you may include
translations of some or all Invariant Sections in addition to the
original versions of these Invariant Sections. You may include a
translation of this License, and all the license notices in the
Document, and any Warranty Disclaimers, provided that you also include
the original English version of this License and the original versions
of those notices and disclaimers. In case of a disagreement between
the translation and the original version of this License or a notice
or disclaimer, the original version will prevail.

If a section in the Document is Entitled ``Acknowledgements'',
``Dedications'', or ``History'', the requirement (section 4) to Preserve
its Title (section 1) will typically require changing the actual
title.

\section{Termination}

You may not copy, modify, sublicense, or distribute the Document
except as expressly provided under this License. Any attempt
otherwise to copy, modify, sublicense, or distribute it is void, and
will automatically terminate your rights under this License.

However, if you cease all violation of this License, then your license
from a particular copyright holder is reinstated (a) provisionally,
unless and until the copyright holder explicitly and finally
terminates your license, and (b) permanently, if the copyright holder
fails to notify you of the violation by some reasonable means prior to
60 days after the cessation.

Moreover, your license from a particular copyright holder is
reinstated permanently if the copyright holder notifies you of the
violation by some reasonable means, this is the first time you have
received notice of violation of this License (for any work) from that
copyright holder, and you cure the violation prior to 30 days after
your receipt of the notice.

Termination of your rights under this section does not terminate the
licenses of parties who have received copies or rights from you under
this License. If your rights have been terminated and not permanently
reinstated, receipt of a copy of some or all of the same material does
not give you any rights to use it.

\section{Future Revisions of this License}

The Free Software Foundation may publish new, revised versions of the
GNU Free Documentation License from time to time. Such new versions
will be similar in spirit to the present version, but may differ in
detail to address new problems or concerns. See
\url{https://www.gnu.org/licenses/}.

Each version of the License is given a distinguishing version number.
If the Document specifies that a particular numbered version of this
License ``or any later version'' applies to it, you have the option of
following the terms and conditions either of that specified version or
of any later version that has been published (not as a draft) by the
Free Software Foundation. If the Document does not specify a version
number of this License, you may choose any version ever published (not
as a draft) by the Free Software Foundation. If the Document
specifies that a proxy can decide which future versions of this
License can be used, that proxy's public statement of acceptance of a
version permanently authorizes you to choose that version for the
Document.

\section{Relicensing}

``Massive Multiauthor Collaboration Site'' (or ``MMC Site'') means any
World Wide Web server that publishes copyrightable works and also
provides prominent facilities for anybody to edit those works. A
public wiki that anybody can edit is an example of such a server. A
``Massive Multiauthor Collaboration'' (or ``MMC'') contained in the site
means any set of copyrightable works thus published on the MMC site.

``CC-BY-SA'' means the Creative Commons Attribution-Share Alike 3.0
license published by Creative Commons Corporation, a not-for-profit
corporation with a principal place of business in San Francisco,
California, as well as future copyleft versions of that license
published by that same organization.

``Incorporate'' means to publish or republish a Document, in whole or in
part, as part of another Document.

An MMC is ``eligible for relicensing'' if it is licensed under this
License, and if all works that were first published under this License
somewhere other than this MMC, and subsequently incorporated in whole or
in part into the MMC, (1) had no cover texts or invariant sections, and
(2) were thus incorporated prior to November 1, 2008.

The operator of an MMC Site may republish an MMC contained in the site
under CC-BY-SA on the same site at any time before August 1, 2009,
provided the MMC is eligible for relicensing.

\section*{\centering Addendum: \\* How to use this License for your documents}

To use this License in a document you have written, include a copy of
the License in the document and put the following copyright and
license notices just after the title page:

\begin{quote}
Copyright (c) YEAR YOUR NAME.
Permission is granted to copy, distribute and/or modify this document
under the terms of the GNU Free Documentation License, Version 1.3
or any later version published by the Free Software Foundation;
with no Invariant Sections, no Front-Cover Texts, and no Back-Cover Texts.
A copy of the license is included in the section entitled ``GNU
Free Documentation License''.
\end{quote}

If you have Invariant Sections, Front-Cover Texts and Back-Cover Texts,
replace the ``with \ldots Texts.'' line with this:

\begin{quote}
with the Invariant Sections being LIST THEIR TITLES, with the
Front-Cover Texts being LIST, and with the Back-Cover Texts being LIST.
\end{quote}

If you have Invariant Sections without Cover Texts, or some other
combination of the three, merge those two alternatives to suit the
situation.

If your document contains nontrivial examples of program code, we
recommend releasing these examples in parallel under your choice of
free software license, such as the \gpl{},
to permit their use in free software.

\concludechapter


\concludebook
